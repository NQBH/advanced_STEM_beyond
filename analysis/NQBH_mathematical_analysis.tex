\documentclass{article}
\usepackage[backend=biber,natbib=true,style=alphabetic,maxbibnames=50]{biblatex}
\addbibresource{/home/nqbh/reference/bib.bib}
\usepackage[utf8]{vietnam}
\DeclareUnicodeCharacter{00A0}{~}
\usepackage{tocloft}
\renewcommand{\cftsecleader}{\cftdotfill{\cftdotsep}}
\usepackage[colorlinks=true,linkcolor=blue,urlcolor=red,citecolor=magenta]{hyperref}
\usepackage{amsmath,amssymb,amsthm,enumitem,float,graphicx,mathtools,tikz}
\usetikzlibrary{angles,calc,intersections,matrix,patterns,quotes,shadings}
\allowdisplaybreaks
\newtheorem{assumption}{Assumption}
\newtheorem{baitoan}{}
\newtheorem{cauhoi}{Câu hỏi}
\newtheorem{conjecture}{Conjecture}
\newtheorem{corollary}{Corollary}
\newtheorem{dangtoan}{Dạng toán}
\newtheorem{definition}{Definition}
\newtheorem{dinhly}{Định lý}
\newtheorem{dinhnghia}{Định nghĩa}
\newtheorem{example}{Example}
\newtheorem{ghichu}{Ghi chú}
\newtheorem{goal}{Goal}
\newtheorem{hequa}{Hệ quả}
\newtheorem{hypothesis}{Hypothesis}
\newtheorem{lemma}{Lemma}
\newtheorem{luuy}{Lưu ý}
\newtheorem{nhanxet}{Nhận xét}
\newtheorem{notation}{Notation}
\newtheorem{note}{Note}
\newtheorem{principle}{Principle}
\newtheorem{problem}{Problem}
\newtheorem{proposition}{Proposition}
\newtheorem{question}{Question}
\newtheorem{remark}{Remark}
\newtheorem{theorem}{Theorem}
\newtheorem{vidu}{Ví dụ}
\usepackage[left=1cm,right=1cm,top=5mm,bottom=5mm,footskip=4mm]{geometry}
\def\labelitemii{$\circ$}
\DeclareRobustCommand{\divby}{%
	\mathrel{\vbox{\baselineskip.65ex\lineskiplimit0pt\hbox{.}\hbox{.}\hbox{.}}}%
}
\setlist[itemize]{leftmargin=*}
\setlist[enumerate]{leftmargin=*}

\title{Mathematical Analysis \& Numerical Analysis\\Giải Tích Toán Học \& Giải Tích Số}
\author{Nguyễn Quản Bá Hồng\footnote{A Scientist {\it\&} Creative Artist Wannabe. E-mail: {\tt nguyenquanbahong@gmail.com}. Bến Tre City, Việt Nam.}}
\date{\today}

\begin{document}
\maketitle
\begin{abstract}
	This text is a part of the series {\it Some Topics in Advanced STEM \& Beyond}:
	
	{\sc url}: \url{https://nqbh.github.io/advanced_STEM/}.
	
	Latest version:
	\begin{enumerate}
		\item {\it Mathematical Analysis -- Giải Tích Toán Học}.
		
		PDF: {\sc url}: \url{https://github.com/NQBH/advanced_STEM_beyond/blob/main/analysis/NQBH_mathematical_analysis.pdf}.
		
		\TeX: {\sc url}: \url{https://github.com/NQBH/advanced_STEM_beyond/blob/main/analysis/NQBH_mathematical_analysis.tex}.
	\end{enumerate}
\end{abstract}
\tableofcontents

%------------------------------------------------------------------------------%

Tôi được giải Nhì Giải tích Olympic Toán Sinh viên 2014 (VMC2014) khi còn học năm nhất Đại học \& được giải Nhất Giải tích Olympic Toán Sinh viên 2015 (VMC2015) khi học năm 2 Đại học. Nhưng điều đó không có nghĩa là tôi giỏi Giải tích. Bằng chứng là 10 năm sau khi nhận các giải đó, tôi đang tự học lại Giải tích Toán học với hy vọng có 1 hay nhiều cách nhìn mới mẻ hơn \& mang tính ứng dụng hơn cho các đề tài cá nhân của tôi.

\section{Wikipedia}
\textbf{\textsf{Resources -- Tài nguyên.}}
\begin{enumerate}
	\item \cite{Lieb_Loss2001}. {\sc Elliott Lieb, Michael Loss}. {\it Analysis}.
	\item \cite{Rudin1976}. {\sc Walter Rudin}. {\it Principles Principles of Mathematical Analysis}.
	\item \cite{Rudin1973,Rudin1987}. {\sc Walter Rudin}. {\it Real \& Complex Analysis}.
	\item \cite{Tao_analysis_1}. {\sc Terence Tao}. {\it Analysis I}.
	\item \cite{Tao_analysis_2}. {\sc Terence Tao}. {\it Analysis II}.
\end{enumerate}
``Analysis is the art of taking limits, \& the constraint of having to deal with an integration theory that does not allow taking limits is much like having to do mathematics only with rational numbers \& excluding the irrational ones.'' -- \cite[Chap. 1, p. 1]{Lieb_Loss2001}

\subsection{Wikipedia{\tt/}physics-informed neural networks PINNs}
``{\it Physics-informed neural networks} (PINNs), also referred as as {\it Theory-Trained Neural Networks} (TTNs), are a type of universal function approximators that can be embed the knowledge of any physical laws that govern a given data-set in the learning process, \& can be described by PDEs. They overcome the low data availability of some biological \& engineering systems that makes most state-of-the-art machine learning techniques lack robustness, rendering them ineffective in these scenarios. The prior knowledge of general physical laws acts in the training of \href{https://en.wikipedia.org/wiki/Neural_network}{neural networks} (NNs) as a \href{https://en.wikipedia.org/wiki/Regularization_(mathematics)}{regularization} agent that limits the space of admissible solutions, increasing the correctness of the function approximation. This way, embedding this prior information into a neural network results in enhancing the information content of the available data, facilitating the learning algorithm to capture the right solution \& to generalize well even with a low amount of training examples. {\sf Physics-informed neural networks for solving NSEs.}

\subsubsection{Function approximation}
Most of the physical laws that govern the dynamics of a system can be described by PDEs. E.g., the NSEs are a set of PDEs derived from the \href{https://en.wikipedia.org/wiki/Conservation_law}{conservation laws} (i.e., conservation of mass, momentum, \& energy) that govern \href{https://en.wikipedia.org/wiki/Fluid_mechanics}{fluid mechanics}. The solution of the NSEs with appropriate initial \& boundary conditions allows the quantification of flow dynamics in a precisely defined geometry. However, these equations cannot be solved exactly \& therefore numerical methods must be used (e.g. FDs, FEs, \& FVs). In this setting, these governing equations must be solved while accounting for prior assumptions, linearization, \& adequate time \& space discretization.

Recently, solving the governing PDEs of physical phenomena using \href{https://en.wikipedia.org/wiki/Deep_learning}{deep learning} has emerged as a new field of scientific machine learning (SciML), leveraging the \href{https://en.wikipedia.org/wiki/Universal_approximation_theorem}{universal approximation theorem} \& high expressivity of neural networks. In general, deep neural networks could approximate any high-dimensional function given that sufficient training data are supplied. However, such networks do not consider the physical characteristics underlying the problem, \& the level of approximation accuracy provided by them is still heavily dependent on careful specifications of the problem geometry as well as the initial \& boundary conditions. Without this preliminary information, the solution is not unique \& may lose physical correctness. On the other hand, physics-informed neural networks (PINNs) leverage governing physical equations in neural network training. Namely, PINNs are designed to be trained to satisfy the given training data as well as the imposed governing equations. In this fashion, a neural network can be guided with training data that do not necessarily need to be large \& complete. Potentially, an accurate solution of PDEs can be found without knowing the boundary conditions. Therefore, with some knowledge about the physical characteristics of the problem \& some form of training data (even sparse \& incomplete), PINN may be used for finding an optimal solution with high fidelity.

PINNs allow for addressing a wide range of problems in computational science \& represent a pioneering technology leading to the development of new classes of numerical solvers for PDEs. PINNs can be thought of as a meshfree alternative to traditional approaches (e.g., \href{https://en.wikipedia.org/wiki/Computational_Fluid_Dynamics}{CFD} for fluid dynamics), \& new data-driven approaches for model inversion \& system identification. Notably, the trained PINN network can be used for predicting the values on simulation grids of different resolutions without the need to be retrained. In addition, they allow for exploiting \href{https://en.wikipedia.org/wiki/Automatic_differentiation}{automatic differentiation} (AD) to compute the required derivatives in the PDEs, a new class of differentiation techniques widely used to derive neural networks assessed to be superior to \href{https://en.wikipedia.org/wiki/Numerical_differentiation}{numerical differentiation} or \href{https://en.wikipedia.org/wiki/Symbolic_differentiation}{symbolic differentiation}.

\subsubsection{Modeling \& computation}
A general nonlinear PDE can be:
\begin{equation*}
	u_t + N[u;\lambda] = 0,\ {\bf x}\in\Omega,\ t\in[0,T],
\end{equation*}
where $u(t,{\bf x})$ denotes the solution, $N[\cdot;\lambda]$: a nonlinear operator parametrized by $\lambda$, \& $\Omega\subset\mathbb{R}^d$. This general form of governing equations summarizes a wide range of problems in mathematical physics, e.g. conservative laws, diffusion process, advection-diffusion systems, \& kinetic equations. Given noisy measurements of a generic dynamic system described by the equation above, PINNs can be designed to solve 2 classes of problems:
\begin{itemize}
	\item data-driven solution
	\item data-driven discovery of PDEs.
\end{itemize}

\paragraph{Data-driven solution of PDEs.} The {\it data-driven solution of PDE} computes the hidden state $u(t,{\bf x})$ of the system given boundary data \&{\tt/}or measurements $z$, \& fixed model parameters $\lambda$. Solve:
\begin{equation*}
	u_t + N[u] = 0,\ {\bf x}\in\Omega,\ t\in[0,T].
\end{equation*}
By defining the residual $f(t,{\bf x})$ as $f\coloneqq u_t + N[u] = 0$, \& approximating $u(t,{\bf x})$ by a deep neural network. This network can be differentiated using automatic differentiation. The parameters of $u(t,{\bf x})$ \& $f(t,{\bf x})$ can be then learned by minimizing the following loss function $L_{\rm tot}\coloneqq L_u + L_f$ where $L_u\coloneqq\|u - z\|_\Gamma$ is the error between the PINN $u(t,{\bf x})$ \& the set of boundary conditions \& measured data on the set of points $\Gamma$ where the boundary conditions \& data are defined, \& $L_f\coloneqq\|f\|_\Gamma$ is the mean-squared error of the residual function. This 2nd term encourages the PINN to learn the structural information expressed by the PDE during the training process.

This approach has been used to yield computationally efficient physics-informed surrogate models with applications in the forecasting of physical processes, model predictive control, multi-physics \& multi-scale modeling, \& simulation. It has been shown to converge to the solution of the PDE.

\paragraph{Data-driven discovery of PDEs.} Given noisy \& incomplete measurements $z$ of the state of the system, the {\it data-driven discovery of PDE} results in computing the unknown state $u(t,{\bf x})$ \& learning model parameters $\lambda$ that best describe the observed data \& it reads as follows:
\begin{equation}
	u_t + N[u;\lambda] = 0,\ {\bf x}\in\Omega,\ t\in[0,T].
\end{equation}
By defining $f(t,{\bf x})\coloneqq u_t + N[u;\lambda] = 0$, \& approximating $u(t,{\bf x})$ by a deep neural network, $f(t,{\bf x})$ results in a PINN. This network can be derived using automatic differentiation. The parameters of $u(t,{\bf x}),f(t,{\bf x})$, together with the parameter $\lambda$ of the differential operator can be then learned by minimizing the following loss function $L_{\rm tot}\coloneqq L_u + L_f$ where $L_u\coloneqq\|u - z\|_\Gamma$, with $u,z$: state solutions \& measurements at sparse location $\Gamma$, respectively \& $L_f\coloneqq\|f\|_\Gamma$ residual function. This 2nd term requires the structured information represented by the PDEs to be satisfied in the training process.

This strategy allows for discovering dynamic models described by nonlinear PDEs assembling computationally efficient \& fully differentiable surrogate models that may find application in predictive forecasting, control, \& \href{https://en.wikipedia.org/wiki/Data_assimilation}{data assimilation}.

\subsubsection{Physics-informed neural networks for piecewise function approximation}
PINN is unable to approximate PDEs that have strong nonlinearity or sharp gradients that commonly occur in practical fluid flow problems. Piecewise approximation has been an old practice in the field of numerical approximation. With the capability of approximating strong nonlinearity extremely light weight PINNs are used to solve PDEs in much larger discrete subdomains that increases accuracy substantially \& decreases computational load as well. DPINN (Distributed physics-informed neural networks) \& DPIELM (Distributed physics-informed extreme learning machines) are generalizable space-time domain discretization for better approximation. DPIELM is an extremely fast \& lightweight approximator with competitive accuracy. Domain scaling on the top has a special effect. Another school of thought is discretization for parallel computation to leverage usage of available computational resources.

XPINNs is a generalized space-time domain decomposition approach for the physics-informed neural networks (PINNs) to solve nonlinear PDEs on arbitrary complex-geometry domains. The XPINNs further pushes the boundaries of both PINNs as well as Conservative PINNs (cPINNs), which is a spatial domain decomposition approach in the PINN framework tailored to conservation laws. Compared to PINN, the XPINN method has large representation \& parallelization capacity due to the inherent property of deployment of multiple neural networks in the smaller subdomains. Unlike cPINN, XPINN can be extended to any type of PDEs. Moreover, the domain can be decomposed in any arbitrary way (in space \& time), which is not possible in cPINN. Thus, XPINN offers both space \& time parallelization, thereby reducing the training cost more effectively. The XPINN is particularly effective for the large-scale problems (involving large data set) as well as for the high-dimensional problems where single network based PINN is not adequate. The rigorous bounds on the errors resulting from the approximation of the nonlinear PDEs (incompressible NSEs) with PINNS \& XPINNs are proved. However, DPINN debunks the use of residual (flux) matching at the domain interfaces as they hardly seem to improve the optimization.

\subsubsection{Physics-informed neural networks \& functional interpolation}
{\sf X-TFC framework scheme for PDE solution learning.} In the PINN framework, initial \& boundary conditions are not analytically satisfied, thus they need to be included in the loss function of the network to be simultaneously learned with the differential equation (DE) unknown functions. Having competing objectives during the network's training can lead to unbalanced gradients while using gradient-based techniques, which causes PINNs to often struggle to accurately learn the underlying DE solution. This drawback is overcome by using functional interpolation techniques such as the Theory of Functional Connections (TFC)'s constrained expression, in the Deep-TFC framework, which reduces the solution search space of constrained problems to the subspace of neural network that analytically satisfies the constraints. A further improvement of PINN \& functional interpolation approach is given by the Extreme Theory of Functional Connections (X-TFC) framework, where a single-layer Neural Network \& the \href{https://en.wikipedia.org/wiki/Extreme_learning_machine}{extreme learning machine} training algorithm are employed. X-TFC allows to improve the accuracy \& performance of regular PINNs, \& its robustness \& reliability are proved for stiff problems, optimal control, aerospace, \& rarefied gas dynamics applications.

\subsubsection{Physics-informed PointNet (PIPN) for multiple sets of irregular geometries}
Regular PINNs are only able to obtain the solution of a forward or inverse problem on a single geometry. It means that for any new geometry (computational domain), one must retrain a PINN. This limitation of regular PINNs imposes high computational costs, specifically for a comprehensive investigation of geometric parameters in industrial designs. Physics-informed PointNet (PIPN) is fundamentally the result of a combination of PINN's loss function with PointNet. In fact, instead of using a simple fully connected neural network, PIPN uses Pointnet as the core of its neural network. PointNet has been primarily designed for \href{https://en.wikipedia.org/wiki/Deep_learning}{deep learning} of 3D object classification \& segmentation by the research group of \href{https://en.wikipedia.org/wiki/Leonidas_J._Guibas}{Leonidas J. Guibas}. PointNet extracts geometric features of input computational domains in PIPN. Thus, PIPN is able to solve governing equations on multiple computational domains (rather than only a single domain) with irregular geometries, simultaneously. The effectiveness of PIPN has been shown for \href{https://en.wikipedia.org/wiki/Incompressible_flow}{incompressible flow}, \href{https://en.wikipedia.org/wiki/Heat_transfer}{heat transfer}, \& \href{https://en.wikipedia.org/wiki/Linear_elasticity}{linear elasticity}.

\subsubsection{Physics-informed neural networks (PINNs) for inverse computations}
Physics-informed neural networks (PINNs) have proven particularly effective in solving inverse problems within differential equations, demonstrating their applicability across science, engineering, \& economics. They have shown useful for solving inverse problems in a variety of fields, including nano-optics, topology optimization{\tt/}characterization, multiphase flow in porous media, \& high-speed fluid flow. PINNs have demonstrated flexibility when dealing with noisy \& uncertain observation datasets. They also demonstrated clear advantages in the inverse calculation of parameters for multi-fidelity datasets, meaning datasets with different quality, quantity, \& types of observations. Uncertainties in calculations can be evaluated using ensemble-based or Bayesian-based calculations.

\subsubsection{Physics-informed neural networks (PINNs) with backward stochastic differential equation}
\href{https://en.wikipedia.org/wiki/Deep_backward_stochastic_differential_equation_method}{Deep backward stochastic differential equation method} is a numerical method that combines deep learning with \href{https://en.wikipedia.org/wiki/Backward_stochastic_differential_equation}{Backward stochastic differential equation} (BSDE) to solve high-dimensional problems in financial mathematics. By leveraging the powerful function approximation capabilities of \href{https://en.wikipedia.org/wiki/Deep_neural_networks}{deep neural networks}, deep BSDE addresses the computational challenges faced by traditional numerical methods like FDMs or Monte Carlo simulations, which struggle with the curse of dimensionality. Deep BSDE methods use neural networks to approximate solutions of high-dimensional PDEs, effectively reducing the computational burden. Additionally, integrating Physics-informed neural networks (PINNs) into the deep BSDE framework enhances its capability by embedding the underlying physical laws into the neural network architecture, ensuring solutions adhere to governing stochastic differential equations, resulting in more accurate \& reliable solutions.

\subsubsection{Physics-informed neural networks for biology}
An extension for adaptation of PINNs are Biologically-informed neural networks (BINNs). BINNs introduce 2 key adaptations to the typical PINN framework:
\begin{itemize}
	\item[(i)] the mechanistic terms of the governing PDE are replaced by neural networks, \&
	\item[(ii)] the loss function $L_{\rm tot}$ is modified to include $L_{\rm constr}$, a term used to incorporate domain-specific knowledge that helps enforce biological applicability.
\end{itemize}
For (i), this adaptation has the advantage of relaxing the need to specify the governing differential equation a priori, either explicitly or by using a library of candidate terms. Additionally, this approach circumvents the potential issue of misspecifying regularization terms in stricter theory-informed cases.

A natural example of BINNs can be found in cell dynamics, where the cell density $u(t,{\bf x})$ is governed by a reaction-diffusion equation with diffusion \& growth functions $D(u),G(u)$, respectively:
\begin{equation*}
	u_t = \nabla\cdot(D(u)\nabla u) + G(u)u,\ {\bf x}\in\Omega,\ t\in[0,T].
\end{equation*}
In this case, a component of $L_{\rm constr}$ could be $\|D\|_\Gamma$ for $D < D_{\min}$, $D > D_{\max}$, which penalizes values of $D$ that fall outside a biologically relevant diffusion range defined by $D_{\min}\le D\le D_{\max}$. Furthermore, the BINN architecture, when utilizing \href{https://en.wikipedia.org/wiki/Multilayer_perceptron}{multiplayer-perceptrons} (MLPs), would function as follows: an MLP is used to construct $u_{\rm MLP}(t,{\bf x})$ from model inputs $(t,{\bf x})$, serving as a surrogate model for the cell density $u(t,{\bf x})$. This surrogate is then fed into the 2 additional MLPs, $D_{\rm MLP}(u_{\rm MLP}),G_{\rm MLP}(u_{\rm MLP})$, which model the diffusion \& growth functions. Automatic differentiation can then be applied to compute the necessary derivatives of $u_{\rm MLP},D_{\rm MLP},G_{\rm MLP}$ to form the governing reaction-diffusion equation.

Note that since $u_{\rm MLP}$ is a surrogate for the cell density, it may contain errors, particularly in regions where the PDE is not fully satisfied. Therefore, the reaction-diffusion equation may be solved numerically, e.g. using a \href{https://en.wikipedia.org/wiki/Method_of_lines}{method-of-lines approach}.

\subsubsection{Limitations}
Translation \& discontinuous behavior are hard to approximate using PINNs. They fail when solving differential equations with slight advective dominance \& hence asymptotic behavior causes the method to fail. Such PDEs could be solved by scaling variables. This difficulty in training of PINNs in advection-dominated PDEs can be explained by the Kolmogorov $n$-width of the solution. They also fail to solve a system of dynamical systems \& hence have not been a success in solving chaotic equations. 1 of the reasons behind the failure of regular PINNs is soft-constraining of Dirichlet \& Neumann boundary conditions which pose a multi-objective optimization problem which requires manually weighing the loss terms to be able to optimize. More generally, posing the solution of a PDE as an optimization problem brings with it all the problems that are faced in the world of optimization, the major one being getting stuck in local optima.'' -- \href{https://en.wikipedia.org/wiki/Physics-informed_neural_networks}{Wikipedia{\tt/}physics-informed neural networks PINNs}

%------------------------------------------------------------------------------%

\subsection{Wikipedia{\tt/}Helmholtz decomposition}
``In physics \& mathematics, the {\it Helmholtz decomposition theorem} or the {\it fundamental theorem of vector calculus} states that certain differentiable \href{https://en.wikipedia.org/wiki/Vector_field}{vector fields} can be resolved into the sum of an \href{https://en.wikipedia.org/wiki/Irrotational_vector_field}{irrotational} (\href{https://en.wikipedia.org/wiki/Curl_(mathematics)}{curl}-free) vector field \& a \href{https://en.wikipedia.org/wiki/Solenoidal}{solenoidal} (\href{https://en.wikipedia.org/wiki/Divergence}{divergence}-free) vector field. In physics, often only the decomposition of sufficiently smooth, rapidly decaying vector fields in 3D is discussed. It is named after \href{https://en.wikipedia.org/wiki/Hermann_von_Helmholtz}{\sc Hermann von Helmholtz}.

\begin{definition}
	For a vector field ${\bf F}\in C^1(V,\mathbb{R}^n)$ defined on a domain $V\subseteq\mathbb{R}^n$, a \emph{Helmholtz decomposition} is a pair of vector fields ${\bf G}\in C^1(V,\mathbb{R}^n),{\bf R}\in C^1(V,\mathbb{R}^n)$ s.t. ${\bf F}({\bf r}) = {\bf G}({\bf r}) + {\bf R}({\bf r})$, ${\bf G}({\bf r}) = -\nabla\Phi({\bf r})$, $\nabla\cdot{\bf R}({\bf r}) = 0$. Here, $\Phi\in C^2(V,\mathbb{R})$ is a \href{https://en.wikipedia.org/wiki/Scalar_potential}{scalar potential}, $\nabla\Phi$ is its gradient, \& $\nabla\cdot{\bf R}$ is the divergence of the vector field ${\bf R}$. The irrotational vector field ${\bf G}$ is called a \emph{gradient field} \& $\mathbb{R}$ is called a \emph{solenoidal field} or \emph{rotation field}. This decomposition does not exist for all vector fields \& is not unique.
\end{definition}

\subsubsection{History}
The Helmholtz decomposition in 3D was 1st described in 1849 by \href{https://en.wikipedia.org/wiki/George_Gabriel_Stokes}{\sc George Gabriel Stokes} for a theory of \href{https://en.wikipedia.org/wiki/Diffraction}{diffraction}. \href{https://en.wikipedia.org/wiki/Hermann_von_Helmholtz}{\sc Hermann von Helmholtz} published his paper on some \href{https://en.wikipedia.org/wiki/Hydrodynamics}{hydrodynamic} basic equations in 1858, which was part of his research on the \href{https://en.wikipedia.org/wiki/Helmholtz%27s_theorems}{Helmholtz's theorems} describing the motion of fluid in the vicinity of vortex lines. Their derivation required the vector fields to decay sufficiently fast at $\infty$. Later, this condition could be relaxed, \& the Helmholtz decomposition could be extended to higher dimensions. For \href{https://en.wikipedia.org/wiki/Riemannian_manifold}{Riemannian manifolds}, the Helmholtz-Hodge decomposition using \href{https://en.wikipedia.org/wiki/Differential_geometry}{differential geometry} \& \href{https://en.wikipedia.org/wiki/Tensor_calculus}{tensor calculus} was derived.

The decomposition has become an important tool for many problems in \href{https://en.wikipedia.org/wiki/Theoretical_physics}{theoretical physics}, but has also found applications in \href{https://en.wikipedia.org/wiki/Animation}{animation}, \href{https://en.wikipedia.org/wiki/Computer_vision}{computer vision} as well as \href{https://en.wikipedia.org/wiki/Robotics}{robotics}

\subsubsection{3D space}
Many physics textbooks restrict the Helmholtz decomposition to 3D space \& limit its application to vector fields that decay sufficiently fast at $\infty$ or to \href{https://en.wikipedia.org/wiki/Bump_function}{bump functions} that are defined on a \href{https://en.wikipedia.org/wiki/Bounded_domain}{bounded domain}. Then, a \href{https://en.wikipedia.org/wiki/Vector_potential}{vector potential} $A$ can be defined, s.t. the rotation field is given by ${\bf R} = \nabla\times{\bf A}$, using the curl of a vector field.

Let ${\bf F}$ be a vector field on a bounded domain $V\subseteq\mathbb{R}^3$, which is twice continuously differentiable inside $V$, \& let $S$ be the surface that encloses the domain $V$. Then ${\bf F}$ can be decomposed into a curl-free component \& a divergence-free component as \fbox{${\bf F} = -\nabla\Phi + \nabla\times{\bf A}$}. [$\ldots$]

\subsubsection{Generalization to higher dimensions}

\subsubsection{Differential forms}
The \href{https://en.wikipedia.org/wiki/Hodge_decomposition#Hodge_decomposition}{Hodge decomposition} is closely related to the Helmholtz decomposition, generalizing from vector fields on $\mathbb{R}^3$ to \href{https://en.wikipedia.org/wiki/Differential_forms}{differential forms} on a \href{https://en.wikipedia.org/wiki/Riemannian_manifold}{Riemannian manifold} $M$. Most formulations of the Hodge decomposition require $M$ to be \href{https://en.wikipedia.org/wiki/Compact_space}{compact}. Since this is not true of $\mathbb{R}^3$, the Hodge decomposition theorem is not strictly a generalization of the Helmholtz theorem. However, the compactness restriction in the usual formulation of the Hodge decomposition can be replaced by suitable decay assumptions at infinity on the differential forms involved, giving a proper generalization of the Helmholtz theorem.

\subsubsection{Extensions to fields not decaying at infinity}

\subsubsection{Uniqueness of the solution}

\subsubsection{Applications}

\begin{itemize}
	\item {\bf Electrodynamics.} The Helmholtz theorem is of particular interest in \href{https://en.wikipedia.org/wiki/Electrodynamics}{electrodynamics}, since it can be used to write \href{https://en.wikipedia.org/wiki/Maxwell%27s_equations}{Maxwell's equations} in the potential image \& solve them more easily. The Helmholtz decomposition can be used to prove that, given \href{https://en.wikipedia.org/wiki/Electric_current_density}{electric current density} \& \href{https://en.wikipedia.org/wiki/Charge_density}{charge density}, the \href{https://en.wikipedia.org/wiki/Electric_field}{electric field} \& the \href{https://en.wikipedia.org/wiki/Magnetic_flux_density}{magnetic flux density} can be determined. They are unique if the densities vanish at infinity \& one assumes the same for the potentials.
	\item {\bf Fluid dynamics.} In \href{https://en.wikipedia.org/wiki/Fluid_dynamics}{fluid dynamics}, the Helmholtz projection plays an important role, especially for the solvability theory of NSEs. If the Helmholtz projection is applied to the linearized incompressible NSEs, the \href{https://en.wikipedia.org/wiki/Stokes_flow}{Stokes equation} is obtained. This depends only on the velocity of the particles in the flow, but no longer on the static pressure, allowing the equation to be reduced to 1 unknown. However, both equations, the Stokes \& linearization equations, are equivalent. The operator $P\Delta$ is called the \href{https://en.wikipedia.org/wiki/Stokes_operator}{Stokes operator}.
	\item {\bf Dynamical systems theory.} In the theory of \href{https://en.wikipedia.org/wiki/Dynamical_system}{dynamical systems}, Helmholtz decomposition can be used to determine ``quasipotentials'' as well as to compute \href{https://en.wikipedia.org/wiki/Lyapunov_function}{Lyapunov functions} in some cases. 
	\item {\bf Medical Imaging.} In \href{https://en.wikipedia.org/wiki/Magnetic_resonance_elastography}{magnetic resonance elastography}, a variant of MR imaging where mechanical waves are used to probe the viscoelasticity of organs, the Helmholtz decomposition is sometimes used to separate the measured displacement fields into its shear component (divergence-free) \& its compression component (curl-free). In this way, the complex shear modulus can be calculated without contributions from compression waves.
	\item {\bf Computer animation \& robotics.} The Helmholtz decomposition is also used in the field of computer engineering. This includes robotics, image reconstruction but also computer animation, where the decomposition is used for realistic visualization of fluids or vector fields.
\end{itemize}

'' -- \href{https://en.wikipedia.org/wiki/Helmholtz_decomposition}{Wikipedia{\tt/}Helmholtz decomposition}

%------------------------------------------------------------------------------%

\section{$C_0$ Semigroup -- Nửa Nhóm $C_0$}
\textbf{\textsf{Resources -- Tài nguyên.}}
\begin{enumerate}
	\item \cite{Anh_Ke_semigroup}. {\sc Cung Thế Anh, Trần Đình Kế}. {\it Nửa Nhóm Các Toán Tử Tuyến Tính \& Ứng Dụng}.
\end{enumerate}
``In mathematical analysis, a {\it $C_0$-semigroup}, also known as a {\it strongly continuous 1-parameter semigroup}, is a generalization of the \href{https://en.wikipedia.org/wiki/Exponential_function}{exponential function}. Just as exponential functions provide solutions of scalar linear constant ODEs, strongly continuous semigroups provide solutions of linear constant coefficient ODEs in \href{https://en.wikipedia.org/wiki/Banach_space}{Banach spaces}. Such differential equations in Banach spaces arise from e.g. \href{https://en.wikipedia.org/wiki/Delay_differential_equation}{delay differential equations} \& PDEs. Formally, a strongly continuous semigroup is a representation of the \href{https://en.wikipedia.org/wiki/Semigroup}{semigroup} $(\mathbb{R}_+,+)$ on some Banach space $X$ that is continuous in the \href{https://en.wikipedia.org/wiki/Strong_operator_topology}{strong toperator topology}.'' -\href{https://en.wikipedia.org/wiki/C0-semigroup}{Wikipedia{\tt/}$C_0$-semigroup}

%------------------------------------------------------------------------------%

\section{Differential Geometry -- Hình Học Vi Phân}
\textbf{\textsf{Resources -- Tài nguyên.}}
\begin{enumerate}
	\item \cite{Carmo2016}. {\sc Manfredo P. do Carmo}. {\it Differential Geometry of Curves \& Surfaces}.
	\item \cite{Delfour_Zolesio2001,Delfour_Zolesio2011}. {\sc Michael C. Delfour, Jean-Paul Zol\'{e}sio}. {\it Shapes \& Geometries}.
	\item \cite{Kuhnel2015}. {\sc Wolfgang K\"uhnel}. {\it Differential Geometry}.
	\item \cite{Walker2015}. {\sc Shawn W. Walker}. {\it The Shapes of Things}.
	
	``Differential geometry is the detailed study of the {\it shape} of a surface (manifold), including {\it local} \& {\it global} properties. A plane in $\mathbb{R}^3$ is a very simple surface \& does not require many tools to characterize. An ``arbitrarily'' shaped surface, e.g., hood of a car, has many distinguished geometric features (e.g., highly curved regions, regions of near flatness, etc.). Characterizing these features quantitatively \& qualitatively requires the tools of differential geometry. Geometric details are important in many physical \& biological processes, e.g., surface tension, biomembranes.
	
	The framework of differential geometry is built by 1st defining a local map (i.e., surface parameterization) which defines the surface. Then, a calculus framework is built up on the surface analogous to the standard ``Euclidean calculus''. Other approaches are also possible, e.g., those with implicit surfaces defined by level sets \& distance functions. But parameterizations, though arbitrary, are quite useful in a variety of settings $\Rightarrow$ stick mostly with those. Emphasize: The geometry of a surface does not depend on a particular parameterization. Otherwise, we will emphasize the distinction between {\tt object 1} \& {\tt object 2}.
	
	We will use this ``abuse'' of notation when there is no possibility of ambiguity.
	
	{\bf Open set.} The concept of open set is critical in multivariate calculus to properly define differentiability. The notation for referencing boundaries of sets, as well as the closure of sets, is practical for referencing geometric details of solid objects \& their surfaces.
	
	{\bf Compactness.} Compact support is useful for ignoring boundary effects. This concept is needed to keep the ``action of a function'' away from the boundary of a set, or to localize the function in a region of interest. 1 reason is to avoid potential difficulties with differentiating a function at its boundary of definition. Or, more commonly, we wish to ignore a quantity depending on the value of a function at a boundary point, e.g., $\int_{\partial S} f = 0$ if $f$ has compact support in $S$.
	
	{\bf Topological mapping{\tt/}homeomorphism.} A bijective, continuous mapping $\boldsymbol{\Phi}$ whose inverse $\boldsymbol{\Phi}^{-1}$ is also continuous is called a {\it topological mapping} or {\it homeomorphism}. Point sets that can be topologically mapped onto each other are said to be {\it homeomorphic}. Sets that are homeomorphic have the ``same topology'', i.e., their connectedness is the same; they have the same kinds of ``holes''. See \cite[Sect. 2.3.1]{Walker2015} for what can happen when a mapping is not a homeomorphism.
	
	{\bf Rigid motion mapping.} A mapping $\boldsymbol{\Phi}$ is called a {\it rigid motion} if any pair of points ${\bf a},{\bf b}$ are the same distance apart as the corresponding pair $\boldsymbol{\Phi}({\bf a}),\boldsymbol{\Phi}({\bf b})$.
	
	{\bf Orthogonal Transformations.} Define the (affine) linear map $\boldsymbol{\Phi}$ (transformation)
	\begin{equation}
		\label{transformation}
		\widetilde{\bf x} = \boldsymbol{\Phi}({\bf x}) = A{\bf x} + {\bf b}.
	\end{equation}
	If $A$ satisfies the properties $A^{-1} = A^\top$, $\det A = 1$ then $\boldsymbol{\Phi}$ represents a rigid motion. Basically, $\boldsymbol{\Phi}$ consists of a rotation represented by $A$ followed by a translation represented by ${\bf b}$. A rigid motion can be used to transition from 1 Cartesian coordinate system to another. If ${\bf b} = {\bf 0}$ \& $A^{-1} = A^\top$, $\det A = 1$, then $\boldsymbol{\Phi}({\bf x}) = A{\bf x}$ is a linear map known as a {\it direct orthogonal transformation}, which is nothing more than a rotation of the coordinate system with the origin as the center. If $A^{-1} = A^\top$, $\det A = 1$ is replaced by $A^{-1} = A^\top$, $\det A = -1$, then $\boldsymbol{\Phi}({\bf x}) = A{\bf x}$ is called an {\it opposite orthogonal transformation}, which consists of a rotation about the origin \& a reflection in a plane. Both $A^{-1} = A^\top$, $\det A = \pm1$ are examples of {\it orthogonal matrices}.
	
	{\bf Interpretation of transformations.} Can interpret $\widetilde{\bf x} = \boldsymbol{\Phi}({\bf x}) = A{\bf x} + {\bf b}$ in 2 different ways. Consider a point $P\in\mathbb{R}^3$ with coordinates ${\bf x}$:
	\begin{itemize}
		\item {\it Alias} (Euler perspective). Viewing \eqref{transformation} as a transformation of coordinates, it appears that ${\bf x},\widetilde{\bf x}$ are the coordinates of the same point w.r.t. 2 different coordinate systems, equivalently, the point is referenced by 2 different ``names'' (sets of coordinates).
		\item {\it Alibi} (Lagrange perspective). Viewing \eqref{transformation} as a mapping of sets, it appears that ${\bf x},\widetilde{\bf x}$ are the coordinates of 2 different points w.r.t. the same coordinate system, equivalently, the point at $\widetilde{\bf x}$ ``was previously'' at ${\bf x}$ before applying the map.
	\end{itemize}
	The concept of material point is directly related to the alibi viewpoint. One can think of a ``particle'' of material, i.e., {\it material point}, initially located at ${\bf x}$, that then moves to $\widetilde{\bf x}$ because of some physical process. The transformation \eqref{transformation} simply represents the kinematic outcome of that physical process, which is a standard concept in deformable continuum mechanics, especially nonlinear elasticity.
	
	{\bf General transformations.} In general, transformation may not be linear. The alias viewpoint yields a {\it curvilinear} coordinate system. The alibi viewpoint implies that the set $S$ is {\it deformed} into the set $S' = \boldsymbol{\Phi}(S)$. 
	
	{\bf Parametric approach -- what is a surface?} A {\it surface} is a set of points in space that is ``regular enough''. A random scattering of points in space does not match our intuitive notion of what a surface is, i.e., it is not regular enough. The boundary of a sphere does match our notion of a surface, i.e., regular enough to be a surface because a sphere is ``smooth''. {\it Intuition}: Can think of creating a surface as deforming a flat rubber sheet into a curved sheet. Let $U\subset\mathbb{R}^2$ be a ``flat'' domain \& let ${\bf X}:U\to\mathbb{R}^3$ be this deforming transformation, i.e., for each point $(s_1,s_2)^\top\in U$ there is a corresponding point ${\bf x} = (x_1,x_2,x_3)^\top\in\mathbb{R}^3$ s.t. ${\bf x} = {\bf X}(s_1,s_2)$. Let $\Gamma = {\bf X}(U)$ denote the surface obtained from ``deforming'' $U$. Call ${\bf x} = {\bf X}(s_1,s_2)$ a {\it parametric representation} of the surface $\Gamma$, where $s_1,s_2$ are called the {\it parameters} of the representation. Refer to $U$ as a {\it reference} domain.
	
	{\bf Allowable parameterization{\tt/}immersion.} If use ${\bf x} = {\bf X}(s_1,s_2)$ to define surfaces, then we must place assumptions on ${\bf X}$ to guarantee that $\Gamma = {\bf X}(U)$ is a valid surface. At the bare minimum, ${\bf X}$ must be continuous to avoid ``tearing'' the rubber sheet. But if want to perform calculus on $\Gamma$, need more:
	\begin{assumption}[Regularity assumptions on ${\bf X}$]
		An allowable parameterization{\tt/}immersion is a parameterization of the form ${\bf x} = {\bf X}(s_1,s_2)$ satisfying:
		\item[(A1)] The function ${\bf X}(s_1,s_2)\in C^\infty(U)$ \& each point ${\bf x} = {\bf X}(s_1,s_2)\in\Gamma$ corresponds to just 1 point $(s_1,s_2)\in U$, i.e., ${\bf X}$ is injective.
		\item[(A2)] The Jacobian matrix $J = [\partial_{s_1}{\bf X},\partial_{s_2}{\bf X}]$ is of rank $2$ on $U$, i.e., the columns of $J$ are linearly independent.
	\end{assumption}
	{\bf Regular surface.} The fundamental property that makes a set of points in $\mathbb{R}^3$ a surface is that it {\it locally looks like a plane} at every point. If you ``zoom into'' a surface, it should look flat. Definition defining a surface in terms of a parameterization is inadequate. Want to define a set in $\mathbb{R}^3$ that is ``intrinsically'' 2D \& is smooth enough so we can perform calculus on it, without regard to a specific parameterization.
	
	\begin{definition}[Regular surface]
		
	\end{definition}
	
	\begin{remark}[Local chart]
		
	\end{remark}
\end{enumerate}
1 trong những ứng dụng của Hình Học Vi Phân là {\it Shape Calculus \& Tangential Calculus -- Phép Tính Vi Tích Phân cho Tối Ưu Hình Dáng \& Phép Tính Vi Tích Phân Trên Mặt Phẳng Tiếp Tuyến}.

\subsection{Calculus on Surfaces}
{\bf Goal.} Define \& develop the fundamental tools of calculus on a regular surface. Start with the notion of differentiability of functions defined only on a surface. Define the concept of vector fields in a surface. Then proceed to develop the gradient \& Laplacian operators w.r.t. a surface. These operators allow for alternative expressions of the summed \& Gaussian curvatures. Derive integration by parts on surfaces, i.e., the domain of integration is a surface. Conclude with some useful identities \& inequalities. Always take $\Gamma$: a regular surface, either with or without a boundary.

%------------------------------------------------------------------------------%

\section{Functional Analysis -- Giải Tích Hàm}
\textbf{\textsf{Resources -- Tài nguyên.}}
\begin{enumerate}
	\item \cite{Alt2016}. {\sc Hans Wilhelm Alt}. {\it Linear Functional Analysis}.
	\item \cite{Brezis2011}. {\sc Ha\"\i m Brezis}. {\it Functional Analysis, Sobolev Spaces \& PDEs}.
	\item \cite{Evans2010}. {\sc Lawrence C. Evans}. {\it PDEs}.
	\item \cite{Rudin1991}. {\sc Walter Rudin}. {\it Functional Analysis}.
	
	``Functional analysis is the study of certain topological-algebraic structures \& of the methods by which knowledge of these structures can be applied to analytic problems.''
	
	The material of a theory should be fully adequate for almost all applications to concrete problems. \& this is what ought to be stressed in such a course: The close interplay between the abstract \& the concrete is not only the most useful aspect of the whole subject but also the most fascinating one.
	
	Many problems that analysts study are not primarily concerned with a single object such as a function, a measure, or an operator, but they deal instead with large classes of such objects. Most of the interesting classes that occur in this way turn out to be vector spaces, either with real scalars or with complex ones. Since limit processes play a role in every analytic problem (explicitly or implicitly), it should be no surprise that these vector spaces are supplied with metrics, or at least with topologies, that bear some natural relation to the objects of which the spaces are made up. The simplest \& most important way of doing this is to introduce a {\it norm}. The resulting structure is called a {\it normed vector space}, or a normed linear space, or simply a {\it normed space}.
	
	``The theory of distributions frees differential calculus from certain difficulties that arise because nondifferentiable functions exist. This is done by extending it to a class of objects (called {\it distributions} or {\it generalized functions}) which is much larger than the class of differential functions to which calculus applies in its original form. Here are some features that any such extension ought to have in order to be useful; our setting is some open subset of $\mathbb{R}^d$:
	\begin{enumerate}
		\item Every continuous function should be a distribution.
		\item Every distribution should have partial derivatives which are again distributions. For differentiable functions, the new motion of derivative should coincide with the old one. (Every distribution should therefore be infinitely differentiable $C^\infty$.)
		\item The usual formal rules of calculus should hold.
		\item There should be a supply of convergence theorems that is adequate for handling the usual limit processes.'' -- \cite[Chap. 6, pp. 149--150]{Rudin1991}
	\end{enumerate}
	\item \cite{Simon1987}. {\sc Jacques Simon}. {\it Compact sets in the space $L^p(0,T;B)$}.
	\item \cite{Simon2022}. {\sc Jacques Simon}. {\it Distributions}.
	\item \cite{Thanh_Thanh_Vu_gth}. {\sc Đinh Ngọc Thanh, Bùi Lê Trọng Thanh, Huỳnh Quang Vũ}. {\it Bài Giảng Giải Tích Hàm}.
	\item {\sc Yosida}.
\end{enumerate}

\subsection{Discrete Functional Analysis -- Giải Tích Hàm Rời Rạc}
Giải Tích Hàm Rời Rạc cung cấp các tools \& theorems để chứng minh các kết quả bên Numerical Analysis.

\noindent\textbf{\textsf{Resources -- Tài nguyên.}}
\begin{enumerate}
	\item \cite[Sect. 5: Appendix: Discrete Functional Analysis]{Gallouet_Herbin_Latche_Mallem2018}. {\sc Gallou\"{e}t, Herbin, Latch\'{e}, J.-C., Mallem, K.} {\it Convergence of the marker-and-cell scheme for the incompressible NSEs on non-uniform grids}.
\end{enumerate}

%------------------------------------------------------------------------------%

\section{Inverse Problems -- Bài Toán Ngược}
\textbf{\textsf{Resources -- Tài nguyên.}}
\begin{enumerate}
	\item \cite{Aster_Borchers_Thurber2018}. {\sc Richard Aster, Brian Borchers, Clifford H. Thurber}. {\it Parameter Estimation \& Inverse Problems}.
	\item \cite{Kirsch2021}. {\sc Andreas Kirsch}. {\it An Introduction to The Mathematical Theory of Inverse Problems}.
	\item \cite{Ito_Jin2015}. {\sc Kazufumi Ito, Bangti Jin}. {\it Inverse Problems}.
\end{enumerate}

%------------------------------------------------------------------------------%

\section{Measure \& Integration -- Độ Đo \& Tích Phân}
\textbf{\textsf{Resources -- Tài nguyên.}}
\begin{enumerate}
	\item \cite{Evans_Gariepy2015}. {\sc Lawrence C. Evans, Ronald F. Gariepy}. {\it Measure Theory \& Fine Properties of Functions}.
\end{enumerate}
The point of view of integration defined as a Riemann integral may be historically grounded \& useful in many areas of mathematics but is far from being adequate for the requirements of modern analysis since Riemann integral can be defined only for a special class of functions \& this class is not closed under the process of taking pointwise limits of sequence (not even monotonic sequences) of functions in this class.

``The useful \& far-reaching idea of Lebesgue \& others was to compute the $(n + 1)$-dimensional volume `in the other direction' by 1st computing the $n$-dimensional volume of the set where the function $> y$. This volume is a well-behaved, monotone nonincreasing function of $y$, which then can be integrated in the manner of Riemann. This method of integration not only works for a large class of functions (which is closed under taking pointwise limits), but it also greatly simplifies a problem that used to plague analysts: {\it Is it permissible to exchange limits \& integration?}'' -- \cite[Chap. 1, pp. 1--2]{Lieb_Loss2001}

Lebesgue integration theory is 1 of the great triumphs of 20th century mathematics \& is the culmination of a long struggle to find the right perspective from which to view integration theory.

%------------------------------------------------------------------------------%

\section{Mean-Field Game Theory -- Lý Thuyết Trò Chơi Trường Trung Bình}
\textbf{\textsf{Community -- Cộng đồng.}} {\sc Nicholetta Tchou (French), Đào Mạnh Khang (Vietnamese), Michael Hinterm\"uller (Austrian), Steven-Marian Stengl (German)}.

\subsection{Wikipedia{\tt/}mean-field game theory}
``{\it Mean-field game theory} is the study of strategic decision making by small interacting \href{https://en.wikipedia.org/wiki/Agent_(economics)}{agents} in very large populations. It lies at the intersection of \href{https://en.wikipedia.org/wiki/Game_theory}{game theory} with stochastic analysis \& \href{https://en.wikipedia.org/wiki/Control_theory}{control theory}. The use of the term ``mean field'' is inspired by \href{https://en.wikipedia.org/wiki/Mean-field_theory}{mean-field theory} in physics, which considers the behavior of systems of large numbers of particles where individual particles have negligible impacts upon the system. In other words, each agent acts according to his minimization or maximization problem taking into account other agents' decisions \& because their population is large we can assume the number of agents goes to infinity \& a representative agent exists.

In traditional \href{https://en.wikipedia.org/wiki/Game_theory}{game theory}, the subject of study is usually a game with 2 players \& discrete time space, \& extends the results to more complex situations by induction. However, for games in continuous time with continuous states (differential games or stochastic differential games) this strategy cannot be used because of the complexity that the dynamic interactions generate. On the other hand with MFGs we can handle large numbers of players through the mean representative agent \& at the same time describe complex state dynamics.

This class of problems was considered in the economics literature by \href{https://en.wikipedia.org/wiki/Boyan_Jovanovic}{Boyan Jovanovic} \& \href{https://en.wikipedia.org/wiki/Robert_W._Rosenthal}{Robert W. Rosenthal}, in the engineering literature by Minyi Huang, Roland Malhame, \& \href{https://en.wikipedia.org/wiki/Peter_E._Caines}{Peter E. Caines} \& independently \& around the same time by mathematicians Jean-Michel Lasry \& \href{https://en.wikipedia.org/wiki/Pierre-Louis_Lions}{Pierre-Louis Lions}.

In continuous time a mean-field game is typically composed of a \href{https://en.wikipedia.org/wiki/Hamilton%E2%80%93Jacobi%E2%80%93Bellman_equation}{Hamilton-Jacobi-Bellman equation} that describes the \href{https://en.wikipedia.org/wiki/Optimal_control}{optimal control} problem of an individual \& a \href{https://en.wikipedia.org/wiki/Fokker%E2%80%93Planck_equation}{Fokker--Planck equation} that describes the dynamics of the aggregate distribution of agents. Under fairly general assumptions it can be proved that a class of mean-field games is the limit as $N\to\infty$ of an $N$-player \href{https://en.wikipedia.org/wiki/Nash_equilibrium}{Nash equilibrium}.

A related concept to that of mean-field games is ``mean-field-type control''. In this case, a \href{https://en.wikipedia.org/wiki/Social_planner}{social planner} controls the distribution of states \& chooses a control strategy. The solution to a mean-field-type control problem can typically be expressed as a dual adjoint Hamilton-Jacobi-Bellman equation coupled with \href{https://en.wikipedia.org/wiki/Fokker%E2%80%93Planck_equation}{Kolmogorov equation}. Mean-field-type game theory is the multi-agent generalization of the single-agent mean-field-type control.

\subsubsection{General Form of a Mean-field Game}
The system of equations
\begin{equation*}
	\left\{\begin{split}
		-\partial_tu - \nu\Delta u + H(x,m,Du) &= 0,\\
		\partial_tm - \nu\Delta m - \nabla\cdot(D_pH(x,m,Du)m) &= 0,\\
		m(0) &= m_0,\\
		u(T,x) &= G(x,m(T)),
	\end{split}\right.
\end{equation*}
can be used to model a typical Mean-field game. The basic dynamics of this set of equations can be explained by an average agent's optimal control problem. In a mean-field game, an average agent can control their movement $\alpha$ to influence the population's overall location by
\begin{equation*}
	dX_t = \alpha_tdt + \sqrt{2\nu}dB_t,
\end{equation*}
where $\nu$: a parameter, $B_t$: a standard Brownian motion. By controlling their movement, the agent aims to minimize their overall expected cost $C$ throughout the time period $[0,T]$:
\begin{equation*}
	C = \mathbb{E}\left[\int_0^T L(X_s,\alpha_s,m(s))\,{\rm d}s + G(X_T,m(T))\right],
\end{equation*}
where $L(X_s,\alpha_s,m(s))$ is the running cost at time $s$ \& $G(X_T,m(T))$ is the terminal cost at time $T$. By this definition, at time $t$ \& position $x$, the value function $u(t,x)$ can be determined as
\begin{equation*}
	u(t,x) = \inf_\alpha \mathbb{E}\left[\int_t^T L(X_s,\alpha_s,m(s))\,{\rm d}s + G(X_T,m(T))\right].
\end{equation*}
Given the definition of the value function $u(t,x)$, it can be tracked by the Hamilton-Jacobi equation. The optimal action of the average players $\alpha^*(t,x)$ can be determined as $\alpha^*(t,x) = D_pH(x,m,Du)$. As all agents are relatively small \& cannot single-handedly change the dynamics of the population, they will individually adapt the optimal control \& the population would move in that way. This is similar to a Nash Equilibrium, in which all agents act in response to a specific set of others' strategies. The optimal control solution then leads to the Kolmogorov-Fokker-Planck equation $\partial_tm - \nu\Delta m - \nabla\cdot(D_pH(x,m,Du)m) = 0$.

\subsubsection{Finite State Games}
A prominent category of mean field is games with a finite number of states \& a finite number of actions per player. For those games, the analog of the Hamilton-Jacobi-Bellman equation is the Bellman equation, \& the discrete version of the Fokker-Planck equation is the Kolmogorov equation. Specifically, for discrete-time models, the players' strategy is the Kolmogorv equation's probability matrix. In continuous time models, players have the ability to control the transition rate matrix.

A discrete mean field game can be defined by a tuple $\mathcal{G} = (\mathcal{E},\mathcal{A},\{Q_a\},{\bf m}_0,\{c_a\},\beta)$ where $\mathcal{E}$ is the state space, $\mathcal{A}$ the action set, $Q_a$ the transition rate matrices, ${\bf m}_0$ the initial state, $\{c_a\}$ the cost functions \& $\beta\in\mathbb{R}$ a discount factor. Furthermore, a mixed strategy is a measurable function $\pi:\mathbb{R}^+\times\mathbb{E}\to\mathcal{P}(\mathcal{A})$, that associates to each state $i\in\mathcal{E}$ \& each time $t\ge0$ a probability measure $\pi_i(t)\in\mathcal{P}(\mathcal{A})$ on the set of possible actions. Thus $\pi_{i,a}(t)$ is the probability that, at time $t$ a player in state $i$ takes action $a$, under strategy $\pi$. Additionally, rate matrices $\{Q_a({\bf m}^\pi(t))\}_{a\in\mathcal{A}}$ define the evolution over the time of population distribution, where ${\bf m}^\pi(t)\in\mathcal{P}(\mathcal{E})$ is the population distribution at time $t$.

\subsubsection{Linear-quadratic Gaussian game problem}
From Caines (2009), a relatively simple model of large-scale games is the \href{https://en.wikipedia.org/wiki/Linear%E2%80%93quadratic%E2%80%93Gaussian_control}{linear-quadratic Gaussian} model. The individual agent's dynamics are modeled as a \href{https://en.wikipedia.org/wiki/Stochastic_differential_equation}{stochastic differential equation}
\begin{equation*}
	dX_i = (a_iX_i + b_iu_i)dt + \sigma_idW_i,\ i = 1,\ldots,N,
\end{equation*}
where $X_i$: the state of the $i$th agent, $u_i$: control of the $i$th agent, $W_i$: independent \href{https://en.wikipedia.org/wiki/Wiener_process}{Wiener processes} $\forall i = 1,\ldots,N$. The individual agent's cost is
\begin{equation*}
	J_i(u_i,\nu) = \mathbb{E}\left[\int_0^\infty e^{-\rho t}[(X_i - \nu)^2 + ru_i^2]\,{\rm d}t\right],\ \nu = \Phi\left(\frac{1}{N}\sum_{k\ne i}^N X_k + \eta\right).
\end{equation*}
The coupling between agents occurs in the cost function.

\subsubsection{General \& Applied Use}
The paradigm of Mean Field Games has become a major connection between distributed decision-making \& stochastic modeling. Starting out tin the stochastic control literature, it is gaining rapid adoption across a range of applications, including:
\begin{enumerate}
	\item {\bf Financial market.} Carmona reviews applications in financial engineering \& economics that can be cast \& tackled within the framework of the MFG paradigm. Carmona argues that models in macroeconomics, contract theory, finance, $\ldots$, greatly benefit from the switch to continuous time from the more traditional discrete-time models. He considers only continuous time models in his review chapter, including systemic risk, price impact, optimal execution, models for bank runs, high-frequency trading, \& cryptocurrencies.
	\item {\bf Crowd motions.} MFG assumes that individuals are smart players which try to optimize their strategy \& path w.r.t. certain costs (equilibrium with rational expectations approach). MFG models are useful to describe the anticipation phenomenon: the forward part describes the crowd evolution while the backward gives the process of how the anticipations are built. Additionally, compared to multi-agent microscopic model computations, MFG only requires lower computational costs for the macroscopic simulations. Some researchers have turned to MFG in order to model the interaction between populations \& study the decision-making process of intelligent agents, including aversion \& congestion behavior between 2 groups of pedestrians, departure time choice of morning commuters, \& decision-making processes for autonomous vehicle.
	\item {\bf Control \& mitigation of Epidemics.} Since the epidemic has affected society \& individuals significantly, MFG \& mean-field controls (MFCs) provide a perspective to study \& understand the underlying population dynamics, especially in the context of the Covid-19 pandemic response. MFG has been used to extend the SIR-type dynamics with spatial effects or allowing for individuals to choose their behaviors \& control their contributions to the spread of the disease. MFC is applied to design the optimal strategy to control the virus spreading within a spatial domain, control individuals' decisions to limit their social interactions, \& support the government's nonpharmaceutical interventions.'' -- \href{https://en.wikipedia.org/wiki/Mean-field_game_theory}{Wikipedia{\tt/}mean-field game theory}
\end{enumerate}

%------------------------------------------------------------------------------%

\section{Partial Differential Equations (PDEs) -- Phương Trình Vi Phân Đạo Hàm Riêng}
\textbf{\textsf{Resources -- Tài nguyên.}}
\begin{enumerate}
	\item \cite{Brezis2011}. {\sc Ha\"im Brezis}. {\it Functional Analysis, Sobolev Spaces, \& Partial Differential Equations}.
	\item \cite{Evans2010}. {\sc Lawrence C. Evans}. {\it Partial Differential Equations}.
	\item \cite{Gilbarg_Trudinger2001}. {\sc David Gilbarg, Neil S. Trudinger}. {\it Elliptic Partial Differential Equations of 2nd Order}.
\end{enumerate}

%------------------------------------------------------------------------------%

\subsection{Weak solution -- Nghiệm yếu}

\begin{definition}[Weak solution -- Nghiệm yếu]
	``In mathematics, a \emph{weak solution} (also called a \emph{generalized solution}) to an ODE or PDE is a function for which the derivatives may not all exist but which is nonetheless deemed to satisfy the equation in some precisely defined sense. There are many different definitions of weak solution, appropriate for different classes of equations. 1 of the most important is based on the notion of \href{https://en.wikipedia.org/wiki/Distribution_(mathematics)}{distributions}.'' -- \href{https://en.wikipedia.org/wiki/Weak_solution}{Wikipedia{\tt/}weak solution}
\end{definition}
``Avoiding the language of distributions, one starts with a differential equation \& rewrites it in such a way that no derivatives of the solution of the equation show up (the new form is called the \href{https://en.wikipedia.org/wiki/Weak_formulation}{weak formulation}, \& the solutions to it are called {\it weak solutions}). Somewhat surprisingly, a differential equation may have solutions which are not differentiable; \& the weak formulation allows one to find such solutions.

Weak solutions are important because many differential equations encountered in modeling real-world phenomena do not admit of sufficiently smooth solutions, \& the only way of solving such equations is using the weak formulation. Even in situations where an equation does have differentiable solutions, it is often convenient to 1st prove the existence of weak solutions \& only alter show that those solutions are in fact smooth enough.'' -- \href{https://en.wikipedia.org/wiki/Weak_solution}{Wikipedia{\tt/}weak solution}

\begin{example}[1st-order wave equation]
	The 1st-order \href{https://en.wikipedia.org/wiki/Wave_equation}{wave equation} $\partial_tu + \partial_xu = 0$ in $\mathbb{R}^2$ with $u = u(t,x)$ has the weak form $\int_{\mathbb{R}^2} u\partial_t\varphi + u\partial_x\varphi\,{\rm d}t\,{\rm d}x = 0$ has a solution $u(t,x) = |t - x|$ which may be checked by splitting the integrals over region $\{x\ge t\}$ \& $\{x\le t\}$ where $u$ is smooth.
\end{example}
``The notion of weak solution based on distribution is sometimes inadequate. In the case of \href{https://en.wikipedia.org/wiki/Hyperbolic_system}{hyperbolic systems}, the notion of weak solution based on distributions does not guarantee uniqueness, \& it is necessary to supplement it with {\it entropy conditions} or some other selection criterion. In fully nonlinear PDE e.g. \href{https://en.wikipedia.org/wiki/Hamilton%E2%80%93Jacobi_equation}{Hamilton-Jacobi equation}, there is a very different definition of weak solution called \href{https://en.wikipedia.org/wiki/Viscosity_solution}{\it viscosity solution}.'' -- \href{https://en.wikipedia.org/wiki/Weak_solution}{Wikipedia{\tt/}weak solution}

\subsubsection{General idea}
When solving a differential equation in $u$, one can rewrite it using a \href{https://en.wikipedia.org/wiki/Test_function}{test function} $\varphi$ s.t. whatever derivatives in $u$ show up in the equation, they are ``transferred'' via integration by parts to $\varphi$, resulting in an equation without derivatives of $u$. This new equation generalizes the original equation to include solutions which are not necessarily differentiable. The approach illustrated above works in great generality. Consider a linear differential operator in an open set $W\subset\mathbb{R}^d$:
\begin{equation*}
	P({\bf x},\partial)u({\bf x}) = \sum a_\alpha({\bf x})\partial^\alpha u({\bf x}),
\end{equation*}
where the multi-index $\alpha = (\alpha_1,\ldots,\alpha_d)$ varies over some finite set in $\mathbb{N}^d$ \& the coefficients $a_\alpha$ are smooth enough functions of ${\bf x}\in\mathbb{R}^d$. The differential equation $P({\bf x},\partial)u({\bf x} = 0$ can, after being multiplied by a smooth test function $\varphi\in C_c^\infty(W)$ \& integrated by parts, be written as
\begin{equation*}
	\int_W u({\bf x})Q({\bf x},\partial)\varphi({\bf x})\,{\rm d}{\bf x} = 0,
\end{equation*}
where the differential operator $Q({\bf x},\partial)$ is given by the formula
\begin{equation*}
	Q({\bf x},\partial)\varphi({\bf x}) = \sum (-1)^{|\alpha|}\partial^\alpha[a_\alpha({\bf x})\varphi({\bf x})],
\end{equation*}
which is the \href{https://en.wikipedia.org/wiki/Formal_adjoint}{formal adjoint} of $P({\bf x},\partial)$.

In summary, if the original (strong) problem was to find a $|\alpha|$-times differentiable function $u$ defined on the open set $W$ s.t. $P({\bf x},\partial)u({\bf x}) = 0$, $\forall{\bf x}\in W$ (a so-called {\it strong solution}), then an integrable function $u$ would be said to be a {\it weak solution} if $\int_W u({\bf x})Q({\bf x},\partial)\varphi({\bf x})\,{\rm d}{\bf x} = 0$, $\forall\varphi\in C_c^\infty(W)$.

%------------------------------------------------------------------------------%

\subsection{Viscosity solution -- Nghiệm trơn{\tt/}nhớt}

\begin{example}[Viscosity solution for Hamilton--Jacobi equation]
	Hamilton--Jacobi equation.
\end{example}

%------------------------------------------------------------------------------%

\subsection{Very weak solution -- Nghiệm rất yếu}

\begin{example}[Very weak solution of porous medium equation (PME) \cite{Vazquez2007}]
	.
\end{example}

\begin{example}[Very weak solution of multi-dimensional slow diffusion equations with a singular quenching term \cite{Dao_Diaz_Nguyen2020}]
	Given $f\in L_\delta^1(\Omega),\lambda\ge0$, a function $u\in L_\delta^1(\Omega)$ is called a \emph{very weak solution} of
	\begin{equation*}
		\left\{\begin{split}
			-\Delta(|u|^{m-1}u) + \lambda u &= f&&\mbox{in }\Omega,\\
			|u|^{m-1}u &= 1&&\mbox{on }\Gamma,
		\end{split}\right.
	\end{equation*}
	if $|u|^{m-1}u\in L^1(\Omega)$ and
	\begin{equation*}
		\int_\Omega u^m\Delta\varphi + \lambda u\varphi\,{\rm d}{\bf x} = \int_\Omega f\varphi\,{\rm d}{\bf x} - \int_\Gamma \partial_{\bf n}\varphi\,{\rm d}{\bf x}.
	\end{equation*}
\end{example}

\begin{example}[Very weak solution of NSEs \cite{Tsai2018}]
	.
\end{example}

\subsection{Navier--Stokes Equations [NSEs]}
\textbf{\textsf{Resources -- Tài nguyên.}}
\begin{enumerate}
	\item \cite{Amirat_Bresch_Lemoine_Simon2001}. {\sc Youcef Amirat, Didier Bresch, J\'{e}r\^{o}me Lemoine, Jacques Simon}. {\it Effect of rugosity on a flow governed by stationary NSEs}.
	\item \cite{Amrouche_Penel_Seloula2013}. {\sc Ch\'{e}rif Amrouche, Patrick Penel, Nour Seloula}. {\it Some remarks on the boundary conditions in the theory of NSEs}.
	\item \cite{Amrouche_Rodriguez-Bellido2011}. {\sc Ch\'{e}rif Amrouche, M. \'{A}ngeles Rodr\'{\i}guez-Bellido}. {\it Stationary Stokes, Oseen, \& NSEs with singular data}.
	\item \cite{An_Li2009}. {\sc Rong An, Kai Tai Li}. {\it Existence of weak solution to nonhomogeneous steady NSEs with mixed boundary conditions}.
	\item \cite{An_Li_Li2009}. {\sc Rong An, Yuan Li, Kai Tai Li}. {\it Regularity of solutions to stationary NSEs with mixed boundary conditions}.
	\item \cite{Bansch2001}. {\sc Eberhard B\"{a}nsch}. {\it Finite element discretization of the NSEs with a free capillary surface}.
	\item \cite{Bellout_Neustupa_Penel2004}. {\sc Hamid Bellout, Ji\v{r}\'i Neustupa, Patrick Penel}. {\it On the NSE with boundary conditions based on vorticity}.
	\item \cite{Benes2011}. {\sc Michal Bene\v{s}}. {\it Mixed IBVP for 3D NSEs in polyhedral domains}.
	\item \cite{Benes_Kucera2007}. {\sc Michal Bene\v{s}, Petr Ku\v{c}era}. {\it Non-steady NSEs with homogeneous mixed boundary conditions and arbitrarily large initial condition}.
	\item \cite{Benes_Kucera2012}. {\sc Michal Bene\v{s}, Petr Ku\v{c}era}. {\it On the Navier--Stokes flows for heat-conducting fluids with mixed boundary conditions}.
	\item \cite{Benes_Kucera2016}. {\sc Michal Bene\v{s}, Petr Ku\v{c}era}. {\it Solutions to the NSEs with mixed boundary conditions in 2D bounded domains}.	
	\item \cite{Bernardi_Hecht_Verfurth2009}. {\sc Christine Bernardi, Fr\'{e}d\'{e}ric Hecht, R\"{u}diger Verf\"{u}rth}. {\it A finite element discretization of 3D NSEs with mixed boundary conditions}.
	\item \cite{Bernardi_Rebollo_Yakoubi2015}. {\sc Christine Bernardi, Tom\'{a}s Chac\'{o}n Rebollo, Driss Yakoubi}. {\it Finite element discretization of the Stokes \& NSEs with boundary conditions on the pressure}.
	\item \cite{Bernardi_Rebollo_Yakoubi2015}. {\sc Christine Bernardi, Toni Sayah}. {\it A posteriori error analysis of the time dependent NSEs with mixed boundary conditions}.
	\item \cite{Berselli2009}. {\sc Luigi C. Berselli}. {\it Some criteria concerning the vorticity and the problem of global regularity for the 3D NSEs}.
	\item \cite{Boukrouche_Boussetouan_Paoli2014}. {\sc Mahdi Boukrouche, Imane Boussetouan, Laetitia Paoli}. {\it Non-isothermal Navier-Stokes system with mixed boundary conditions and friction law: uniqueness and regularity properties}.
	\item \cite{Braack_Mucha2014}. {\sc Malte Braack, Piotr Boguslaw Mucha}. {\it Directional do-nothing condition for the NSEs}.
	\item \cite{Brizitskii2009}. {\sc R. V. Brizitski\u{\i}}. {\it Investigation of a class of control problems for stationary NSEs with mixed boundary conditions}.
	\item \cite{Brown_Mitrea_Mitrea_Wright2010}. {\sc R. Brown, I. Mitrea, M. Mitrea, M. Wright}. {\it Mixed BVPs for the Stokes system}.
	\item \cite{Bucur_Feireisl_Necasova_Wolf2008}. {\sc Dorin Bucur, Eduard Feireisl, \v{S}\'{a}rka Ne\v{c}asov\'{a}, Joerg Wolf}. {\it On the asymptotic limit of the Navier-Stokes system on domains with rough boundaries}.
	\item \cite{Bulicek_Malek_Rajagopal2007}. {\sc M. Bul\'{i}\v{c}ek, J. M\'{a}lek, K. R. Rajagopal}. {\it Navier's slip and evolutionary {N}avier-{S}tokes-like systems with pressure and shear-rate dependent viscosity}.
	\item \cite{Cahouet_Chabard1988}. {\sc J. Cahouet, J.-P. Chabard}. {\it Some fast 3D finite element solvers for the generalized Stokes problem}.
	\item \cite{Camano_Oyarzua_Ruiz-Baier_Tierra2018}. {\sc Jessika Cama\~{n}o, Ricardo Oyarz\'{u}a,  Ricardo Ruiz-Baier, Giordano Tierra}. {\it Error analysis of an augmented mixed method for the Navier-Stokes problem with mixed boundary conditions}.
	\item \cite{Cattabriga1961}. {\sc Lamberto Cattabriga}. {\it Su un problema al contorno relativo al sistema di equazioni di Stokes}.
	\item \cite{Chen_Osborne_Qian2009}. {\sc Gui-Qiang Chen, Dan Osborne, Zhongmin Qian}. {\it The NSEs with the kinematic and vorticity boundary conditions on non-flat boundaries}.
	\item \cite{Daikh_Yakoubi2017}. {\sc Yasmina Daikh, Driss Yakoubi}. {\it Spectral discretization of the Navier-Stokes problem with mixed boundary conditions}.
	\item \cite{Deuring_von-Walh1995}. {\sc Paul Deuring, Wolf von Wahl}. {\it Strong solutions of the Navier-Stokes system in Lipschitz bounded domains}.
	\item \cite{Ebmeyer_Frehse2001}. {\sc Carsten Ebmeyer, Jens Frehse}. {\it Steady NSEs with mixed boundary value conditions in 3D Lipschitzian domains}.
	\item \cite{Elghaoui_Pasquetti1999}. {\sc M.  Elghaoui, R. Pasquetti}. {\it Mixed spectral-boundary element embedding algorithms for the NSEs in the vorticity-stream function formulation}.
	\item \cite{Farhloul_Nicaise_Paquet2008}. {\sc Mohamed Farhloul, Serge Nicaise, Luc Paquet}. {\it A refined mixed FEM for the stationary NSEs with mixed boundary conditions}.
	\item \cite{Fefferman2006}. {\sc Charles L. Fefferman}. {\it Existence and smoothness of the NSE}: {\bf The millennium prize problems}.
	\item \cite{Foias_Manley_Rosa_Temam2001}. {\sc C. Foias, O. Manley, R. Rosa, R. Temam}. {\it NSEs \& Turbulence}.
	\item \cite{Foias_Manley_Temam1987}. {\sc C. Foias, O. Manley,  R. Temam}. {\it Attractors for the B\'{e}nard problem: existence and physical bounds on their fractal dimension}.
	\item \cite{Fujita_Kato1964}. {\sc Hiroshi Fujita, Tosio Kato}. {\it On the Navier-Stokes IVP. I}.
	\item \cite{Fursikov1980}. {\sc A. V. Fursikov}. {\it Some control problems and results related to the unique solvability of the mixed BVP for the NSEs and Euler 3D systems}.
	\item \cite{Fursikov1981}. {\sc A. V. Fursikov}. {\it Control problems and theorems concerning unique solvability of a mixed BVP for 3D NSEs and Euler equations}.
	\item \cite{Fursikov1982}. {\sc A. V. Fursikov}. {\it Properties of solutions of control problems that are connected with the Navier-Stokes system}.
	\item \cite{Galdi2000}. {\sc Giovanni P. Galdi}. {\it An introduction to the Navier-Stokes IBVP}.
	\item \cite{Galdi2011}. {\sc Giovanni P. Galdi}. {\it An introduction to the mathematical theory of NSEs}.
	\item \cite{Gie_Kelliher2012}. {\sc Gung-Min Gie, James P. Kelliher}. {\it Boundary layer analysis of the NSEs with generalized Navier boundary conditions}.
	\item \cite{Guerra_Sequeira2015}. {\sc Telma Guerra,  Ad\'{e}lia Sequeira, Jorge Tiago}. {\it Existence of optimal boundary control for NSEs with mixed boundary conditions}.
	\item \cite{Heywood_Rannacher_Turek1996}. {\sc John G. Heywood, Rolf Rannacher, Stefan Turek}. {\it Artificial boundaries and flux and pressure conditions for the incompressible NSEs}.
	\item \cite{Hoang2010}. {\sc Luan Thach Hoang}. {\it Incompressible fluids in thin domains with Navier friction boundary conditions (I)}.
	\item \cite{Hoang2013}. {\sc Luan Thach Hoang}. {\it Incompressible fluids in thin domains with Navier friction boundary conditions (II)}.
	\item \cite{Hoang_Sell2010}. {\sc Luan Thach Hoang, George R. Sell}. {\it NSEs with Navier boundary conditions for an oceanic model}.
	\item \cite{Hou_Pei2019}. {\sc Yanren Hou, Shuaichao Pei}. {\it On the weak solutions to steady NSEs with mixed boundary conditions}.
	\item \cite{Iftimie_Raugel2001}. {\sc Drago\c{s} Iftimie, Genevi\`eve Raugel}. {\it Some results on NSEs in thin 3D domains}.
	\item \cite{Iftimie_Raugel_Sell2007}. {\sc Drago\c{s} Iftimie, Genevi\`eve Raugel, George R. Sell}. {\it NSEs in thin 3D domains with Navier boundary conditions}.
	\item \cite{Iftimie_Sueur2011}. {\sc Drago\c{s} Iftimie, Franck Sueur}. {\it Viscous boundary layers for the NSEs with the Navier slip conditions}.
	\item \cite{Iliescu_John_Layton2002}. {\sc Traian Iliescu, Volker John, William J. Layton}. {\it Convergence of finite element approximations of large eddy motion}.
	\item \cite{Illarionov_Chebotarev2001}. {\sc A. A. Illarionov, A. Yu. Chebotarev}. {\it On the solvability of a mixed BVP for stationary NSEs}.
	\item \cite{Jager_Mikelic2000}. {\sc Willi J\"{a}ger, Andro Mikeli\'{c}}. {\it On the interface boundary condition of Beavers, Joseph, \& Saffman}.
	\item \cite{Jager_Mikelic2001}. {\sc Willi J\"{a}ger, Andro Mikeli\'{c}}. {\it On the roughness-induced effective boundary conditions for an incompressible viscous flow}.
	\item \cite{John2002}. {\sc Volker John}. {\it Slip with friction and penetration with resistance boundary conditions for the NSEs---numerical tests and aspects of the implementation}.
	\item \cite{John_Layton_Sahin2004}. {\sc Volker John, W. Layton, N. Sahin}. {\it Derivation and analysis of near wall models for channel and recirculating flows}.
	\item \cite{Kaladhar_Komuraiah_Madhusudhan-Reddy2019}. {\sc K. Kaladhar, E. Komuraiah, K. Madhusudhan Reddy}. {\it Soret and Dufour effects on chemically reacting mixed convection flow in an annulus with {N}avier slip and convective boundary conditions}.
	\item \cite{Kelliher2006}. {\sc James P. Kelliher}. {\it NSEs with Navier boundary conditions for a bounded domain in the plane}.
	\item \cite{Kim_Cao2015}. {\sc Tujin Kim, Daomin Cao}. {\it Some properties on the surfaces of vector fields and its application to the Stokes and Navier-Stokes problems with mixed boundary conditions}.
	\item \cite{Kim2016}. {\sc Tujin Kim}. {\it Erratum to ``Some properties on the surfaces of vector fields and its application to the Stokes and Navier-Stokes problems with mixed boundary conditions'' [{N}onlinear {A}nal. 113 (2015) 94--114] [ {MR}3281848]}
	\item \cite{Kim_Cao2016}. {\sc Tujin Kim, Daomin Cao}. {\it The steady {N}avier-{S}tokes and Stokes systems with mixed boundary conditions including one-sided leaks and pressure}.
	\item \cite{Kim_Cao2017}. {\sc Tujin Kim, Daomin Cao}. {\it Non-stationary NSEs with mixed boundary conditions}.
	\item \cite{Kim_Huang2018}. {\sc Tujin Kim, Feimin Huang}. {\it The non-steady Navier-Stokes systems with mixed boundary conditions including friction conditions}.
	\item \cite{Korobkov_Pileckas_Russo2015}. {\sc Mikhail V. Korobkov, Konstantin Pileckas, Remigio Russo}. {\it Solution of Leray's problem for stationary NSEs in plane and axially symmetric spatial domains}.
	\item \cite{Kracmar_Neustupa2001}. {\sc S. Kra\v{c}mar, J. Neustupa}. {\it A weak solvability of a steady variational inequality of the Navier-Stokes type with mixed boundary conditions}.
	\item \cite{Kucera1998a}. {\sc Petr Ku\v{c}era}. {\it A structure of the set of critical points to the NSEs with mixed boundary conditions}.
	\item \cite{Kucera1998b}. {\sc Petr Ku\v{c}era}. {\it Solutions of the stationary Navier-Stokes equations with mixed boundary conditions in bounded domain}.
	\item \cite{Kucera2009}. {\sc Petr Ku\v{c}era}. {\it Basic properties of solution of the non-steady NSEs with mixed boundary conditions in a bounded domain}.
	\item \cite{Kucera2010}. {\sc Petr Ku\v{c}era}. {\it The time-periodic solutions of the NSEs with mixed boundary conditions}.
	\item \cite{Kucera_Skalak1998}. {\sc Petr Ku\v{c}era, Zden\v{e}k Skal\'{a}k}. {\it Local solutions to the NSEs with mixed boundary conditions}.
	\item \cite{Ladyzhenskaya1969}. {\sc O. A. Ladyzhenskaya}. {\it The Mathematical Theory of Viscous Incompressible Flow}.
	\item \cite{Lasiecka_Szulc_Zochowski2018}. {\sc Irena Lasiecka, Katarzyna Szulc, Antoni \.{Z}ochowski}. {\it Boundary control of small solutions to fluid-structure interactions arising in coupling of elasticity with NSE under mixed boundary conditions}.
	\item \cite{Lemarie-Rieusset2002}. {\sc Pierre Gilles Lemari\'{e}-Rieusset}. {\it Recent developments in the Navier-Stokes problem}.
	\item \cite{le-Roux_Reddy1993}. {\sc C. le Roux, B. D. Reddy}. {\it The steady NSEs with mixed boundary conditions: application to free boundary flows}.
	\item \cite{Lemarie-Rieusset2016}. {\sc Pierre Gilles Lemari\'{e}-Rieusset}. {\it The Navier-Stokes problem in the 21st century}.
	\item \cite{Leray1933}. {\sc Jean Lerray}. {\it \'{E}tude de diverses \'{e}quations int\'{e}grales non lin\'{e}aires et de quelques probl\`emes que pose l'hydrodynamique}.
	\item \cite{Leray1934a}. {\sc Jean Lerray}. {\it Essai sur les mouvements plans d'un liquide visqueaux que limitent des parois}.
	\item \cite{Leray1934b}. {\sc Jean Lerray}. {\it Sur le mouvement d'un liquide visqueux emplissant l'espace}.
	\item \cite{Li_An2008}. {\sc Kai Tai Li, Rong An}. {\it On the rotating NSEs with mixed boundary conditions}.
	\item \cite{Liakos2001}. {\sc Anastasios Liakos}. {\it Discretization of the NSEs with slip boundary condition}.
	\item \cite{Liakos2004}. {\sc Anastasios Liakos}. {\it Discretization of the Navier-Stokes equations with slip boundary condition. II}.
	\item \cite{Liu_Yu2008}. {\sc Dongjie Liu, Dehao Yu}. {\it The coupling method of natural boundary element and mixed finite element for stationary NSE in unbounded domains}.
	\item \cite{Lyashko_Prokhur1985}. {\sc I. I. Lyashko, N. Z. Prokhur}. {\it Construction of stable schemes for solving a mixed BVP for evolution generalization of 3D NSEs}.
	\item \cite{Mazya_Rossmann2007}. {\sc Vladimir Maz'ya, J. Rossmann}. {\it$L_p$ estimates of solutions to mixed BVPs for the Stokes system in polyhedral domains}.
	\item \cite{Mazya_Rossmann2009}. {\sc Vladimir Maz'ya, J. Rossmann}. {\it Mixed boundary value problems for the stationary Navier-Stokes system in polyhedral domains}.
	\item \cite{Mihailov1968}. {\sc V. P. Miha\u{\i}lov}. {\it The mixed boundary value problem for the Navier-Stokes system of equations}.
	\item \cite{Mitrea_Monniaux2008}. {\sc Marius Mitrea, Sylvie Monniaux}. {\it On the analyticity of the semigroup generated by the Stokes operator with Neumann-type boundary conditions on Lipschitz subdomains of Riemannian manifolds}.
	\item \cite{Mitrea_Monniaux2009}. {\sc Marius Mitrea, Sylvie Monniaux}. {\it The nonlinear Hodge-Navier-Stokes equations in Lipschitz domains}.
	\item \cite{Monniaux2006}. {\sc Sylvie Monniaux}. {\it NSEs in arbitrary domains: the Fujita-Kato scheme}.
	\item \cite{Mukminov1992a}. {\sc F. Kh. Mukminov}. {\it On the decay of the solution of the first mixed problem for a linearized system of NSEs in a domain with a noncompact boundary}.
	\item \cite{Mukminov1992b}. {\sc F. Kh. Mukminov}. {\it On the rate of decay of the solution of a mixed problem for a system of NSEs in a domain with a noncompact boundary}.
	\item \cite{Nguyen_Raymond2015}. {\sc Nguyen Phuong Anh, Jean-Pierre Raymond}. {\it Boundary stabilization of the NSEs in the case of mixed boundary conditions}.
	\item \cite{Nicaise_Paquet_Rafilipojaona2007}. {\sc S. Nicaise, L. Paquet, Rafilipojaona}. {\it A refined mixed finite element method for stationary NSEs with mixed boundary conditions using Lagrange multipliers}.
	\item \cite{Ovsienko1978}. {\sc V. G. Ovsienko}. {\it A mixed boundary value problem for nonstationary NSEs on the exterior of a circular cylinder}.
	\item \cite{Papoutsis-Kiachagias_Magoulas_Mueller_Othmer_Giannakoglou2015}. {\sc E. M. Papoutsis-Kiachagias, N. Magoulas, J. Mueller, C. Othmer, K. C. Giannakoglou}. {\it Noise reduction in car aerodynamics using a surrogate objective function and the continuous adjoint method with wall functions}
	\item \cite{Phan_Sergio2017}. {\sc Phan Đức Duy, S\'{e}rgio S. Rodrigues}. {\it Gevrey regularity for NSEs under Lions boundary conditions}.
	\item \cite{Plotnikov_Sokolowski2008}. {\sc P. I. Plotnikov, J. Sokolowski}. {\it Stationary BVPs for compressible NSEs}.
	\item \cite{Plotnikov_Sokolowski2012}. {\sc Pavel Plotnikov, Jan Soko\l owski}. {\it Compressible NSEs: Theory and Shape Optimization}.
	\item \cite{Rossmann2009}. {\sc J\"{u}rgen Rossmann}. {\it Mixed BVPs for Stokes and Navier-Stokes systems in polyhedral domains}.
	\item \cite{Sene_Ngom_Ngom2019}. {\sc Abdou S\`ene, Timack Ngom, Evrad M. D. Ngom}. {\it Global stabilization of the NSEs around an unstable steady state with mixed boundary kinetic energy controller}.
	\item \cite{Seregin2015}. {\sc Gregory Seregin}. {\it Lecture notes on regularity theory for NSEs}.	 
	\item \cite{Sohr2001,Sohr2013}. {\sc Hermann Sohr}. {\it The NSEs: An Elementary Functional Analytic Approach}.
	
	{\sf Primary objective.} To develop an elementary \& self-contained approach to the mathematical theory of a viscous incompressible fluid in a domain $\Omega\subset\mathbb{R}^d$, described by NSEs. Formulate the theory for a completely general domain $\Omega$.
	\item \cite{Su_Li2008}. {\sc Jian Su, Kai Tai Li}. {\it A FEM for 3D stationary rotating NSEs in primitive variables with mixed boundary conditions}.
	\item \cite{Tao2013}. {\sc Terence Tao}. {\it Localisation and compactness properties of the Navier-Stokes global regularity problem}.
	\item \cite{Tartar2006}. {\sc Luc Tartar}. {\it An introduction to {N}avier-{S}tokes equation and oceanography}.
	\item \cite{Temam1977,Temam2000}. {\sc Roger Temam}. {\it NSES: Theory \& Numerical Analysis}.
	\item \cite{Temam1983,Temam1995}. {\sc Roger Temam}. {\it NSEs \& nonlinear functional analysis}.
	\item \cite{Tsai2018}. {\sc Tai-Peng Tsai}. {\it Lectures on NSEs}.
	\item \cite{Verfurth1987}. {\sc R\"{u}diger Verf\"{u}rth}. {\it Finite element approximation of incompressible NSEs with slip boundary condition}.
	\item \cite{Verfurth1991}. {\sc R\"{u}diger Verf\"{u}rth}. {\it Finite element approximation of incompressible NSEs with slip boundary condition. II}.
	\item \cite{Yudovich1963}. {\sc V. I. Yudovich}. {\it Non-stationary flow of an ideal incompressible liquid}.
	\item \cite{Yudovich1967}. {\sc V. I. Yudovich}. {\it An example of the loss of stability and the generation of a secondary flow of a fluid in a closed container}.
\end{enumerate}

\subsection{Schr\"odinger equations}
\textbf{\textsf{Resources -- Tài nguyên.}}
\begin{enumerate}
	\item \cite{Weinstein1983}. {\sc Michael I. Weinstein}. {\it Nonlinear {S}chr\"{o}dinger equations and sharp interpolation estimates}.
	\begin{itemize}
		\item NQBH. Master 2 Seminar: {\it On the smallest constant for a Gagliardo--Nirenberg functional inequality}. [\href{https://github.com/NQBH/advanced_STEM_beyond/blob/main/Master_of_Science/Master_Seminar/NQBH_Master_seminar.pdf}{report}][\href{https://github.com/NQBH/advanced_STEM_beyond/blob/main/Master_of_Science/Master_Seminar/NQBH_Master_seminar_summary.pdf}{summary}][\href{https://github.com/NQBH/advanced_STEM_beyond/blob/main/Master_of_Science/Master_Seminar/NQBH_Master_seminar_slide.pdf}{slide}]
	\end{itemize}
	{\bf Abstract.} Obtain a sharp sufficient condition for global existence for nonlinear Schr\"odinger equation $2i\phi_t + \Delta\phi + |\phi|^{2\sigma}\phi = 0$, in $\mathbb{R}^+\times\mathbb{R}^N$ in case $\sigma = \frac{2}{N}$, in terms of an exact stationary solution (nonlinear ground state) of NLS, derived by solving a variational problem to obtain the ``best constant'' for classical interpolation estimates of Nirenberg \& Gagliardo.
	\begin{itemize}
		\item {\sf Sect. 1: Introduction.} The ``best constant'' of an interpolation estimate among various norms often has an analytical or geometrical significance.
		\begin{goal}
			Present a relationship between the best constant for a classical interpolation inequality due to Nirenberg \& Gagliardo, \& a sharp criterion for existence of global solutions to nonlinear Schr\"odinger equation:
			\begin{equation}
				\label{nonlinear Schrodinger}
				\tag{nSch}
				2i\partial_t\phi + \Delta\phi + |\phi|^{2\sigma}\phi = 0\mbox{ in }\mathbb{R}^+\times\mathbb{R}^N,\ \phi(0,{\bf x}) = \phi_0({\bf x})
			\end{equation}
			in critical case $\sigma = \frac{2}{N}$.
		\end{goal}
		\item {\sf Sect. 2: Solution of a Variational Problem.}
		\item {\sf Sect. 3: Global Existence for IVP in Critical Case $\sigma = \frac{2}{N}$.}
		\item {\sf Sect. 4: Blowing Up Solutions.}
		\item {\sf Sect. 5: Numerical Observations \& Open Questions.}
	\end{itemize}
\end{enumerate}

\subsection{Water Waves Systems}
\textbf{\textsf{Community -- Cộng đồng.}} {\sc Vincent Duchene, David Lannes, Michael I. Weinstein}.

\noindent\textbf{\textsf{Resources -- Tài nguyên.}}
\begin{enumerate}
	\item {\sc Vincent Duch\^ene}. {\it Many Models for Water Waves}.
	\item \cite{Lannes2013}. {\sc David Lannes}. {\it The Water Waves Problem}.
\end{enumerate}

\subsection{Elliptic PDEs}
\textbf{\textsf{Resources -- Tài nguyên.}}
\begin{enumerate}
	\item \cite{Agmon1965,Agmon2010}. {\sc Shmuel Agmon}. {\it Lectures on Elliptic BVPs}.
	\item \cite{Gilbarg_Trudinger2001}. {\sc David Gilbarg, Neil S. Trudinger}. {\it Elliptic PDEs of 2nd Order}.
	\item \cite{Grisvard1980}. {\sc Pierre Grisvard}. {\it BVPs in Nonsmooth Domains}.
	\item \cite{Grisvard1985,Grisvard2011}. {\sc Pierre Grisvard}. {\it Elliptic Problems in Nonsmooth Domains}.
	\item \cite{Han_Lin2011}. {\sc Qing Han, Fanghua Lin}. {\sc Elliptic PDEs}.
	\item \cite{Hung_Phuc2020}. {\sc Nguyễn Quốc Hưng, Nguyễn Công Phúc}. {\it Pointwise gradient estimates for a class of singular quasilinear equations with measure data}.
	
	{\bf Keywords.} Riesz's potential, Wolff's potential, pointwise gradient estimate, Reifenberg flat domain.
	
	{\bf Abstract.} Local \& global pointwise gradient estimates are obtained for solutions to quasilinear elliptic equation with measure data $-\nabla\cdot(A({\bf x},\nabla u)) = \mu$ in a bounded \& possibly nonsmooth domain $\Omega\subset\mathbb{R}^n$ where $\nabla\cdot(A({\bf x},\nabla u))$ is modeled after the $p$-Laplacian. Extend earlier known results to the singular case in which $\frac{3n - 2}{2n - 1} < p\le2 - \frac{1}{n}$.
	
	\begin{itemize}
		\item {\sf Sect. 1: Introduction \& main results.} Consider quasilinear elliptic equation with measure data $-\nabla\cdot(A({\bf x},\nabla u)) = \mu$ in a bounded open subset $\Omega$ of $\mathbb{R}^n$, $n\ge2$, $\mu$: a finite signed measure in $\Omega$, nonlinearity $A = (A_1,\ldots,A_n):\mathbb{R}^n\times\mathbb{R}^n\to\mathbb{R}^n$ is vector-valued function.
		
		\begin{goal}
			Obtain pointwise estimates for gradients of solutions to $-\nabla\cdot(A({\bf x},\nabla u)) = \mu$ by means of nonlinear potentials of Wolff type.
		\end{goal}
		Assume $A = A({\bf x},\xi)$ satisfies growth, ellipticity, \& continuity assumptions. Dini's condition $\int_0^1 \omega(r)^{\gamma_0}\frac{dr}{r} < \infty$.
		
		A typical model for main PDE is given by $p$-Laplace equation with measure data $-\Delta_p u\coloneqq-\nabla\cdot(|\nabla u|^{p - 2}\nabla u) = \mu$ in $\Omega$, or its nondegenerate version ($s > 0$): $-\nabla\cdot((|\nabla u| + s^2)^{\frac{p - 2}{2}}\nabla u) = \mu$ in $\Omega$.
		\item {\sf Sect. 2: Sharp quantitative $C^{1,\sigma}$ regularity estimates.}
		\item {\sf Sect. 3: Interior pointwise gradient estimates.}
		\item {\sf Sect. 4: Global pointwise gradient estimates.}
	\end{itemize}
	\item \cite{Ladyzhenskaya_Uraltseva1968}. {\sc Olga A. Ladyzhenskaya, Nina N. Ural'tseva}. {\it Linear \& Quasilinear Elliptic Equations}.
	\item \cite{Mazya_Rossmann2010}. {\sc Vladimir Maz'ya, J\"{u}rgen Rossmann}. {\it Elliptic Equations in Polyhedral Domains}.
	\item \cite{Necas1967,Necas2012}. {\sc Jind\v{r}ich Ne\v{c}as}. {\it Les m\'{e}thodes directes en th\'{e}orie des \'{e}quations elliptiques -- Direct Methods in The Theory of Elliptic Equations}.
\end{enumerate}

\subsection{Parabolic PDEs}
\textbf{\textsf{Resources -- Tài nguyên.}}
\begin{enumerate}
	\item \cite{Friedman1964}. {\sc Avner Friedman}. {\it PDEs of Parabolic Type}.
	\item \cite{Holcman_Schuss2018}. {\sc David Holcman, Zeev Schuss}. {\it Asymptotics of Elliptic \& Parabolic PDEs -- \& Their Applications in Statistical Physics, Computational Neuroscience, \& Biophysics}.
	\item \cite{Knabner_Angermann2003}. {\sc Peter Knabner, Lutz Angermann}. {\it Numerical Methods for Elliptic \& Parabolic PDEs}.
	\item \cite{Krylov2008}. {\sc N. V. Krylov}. {\it Lectures on Elliptic \& Parabolic Equations in Sobolev Spaces}.
	\item \cite{Ladyzhenskaja_Solonnikov_Uralceva1968}. {\sc O. A. Lady\v{z}enskaja, V. A. Solonnikov, N. N. Ural'ceva}. {\it Linear \& Quasi-linear Equations of Parabolic Type}.
	\item \cite{Lunardi1995}. {\sc Alessandra Lunardi}. {\it Analytic Semigroup \& Optimal Regularity in Parabolic Problems}.
\end{enumerate}

\subsection{Hyperbolic PDEs}
\textbf{\textsf{Resources -- Tài nguyên.}}
\begin{enumerate}
	\item \cite{Alinhac2009}. {\sc Serge Alinhac}. {\it Hyperbolic PDEs}.
	\item \cite{Benzoni-Gavage_Serre2007}. {\sc Sylvie Benzoni-Gavage, Denis Serre}. {\it Multidimensional Hyperbolic PDEs}.
	\item \cite{Ikawa2000}. {\sc Mitsuru Ikawa}. {\it Hyperbolic PDEs \& Wave Phenomena}.
	\item \cite{Lasiecka_Triggiani2000}. {\sc Irena Lasiecka, Roberto Triggiani}. {\it Control Theory for PDEs: Continuous \& Approximation Theories. II: Abstract Hyperbolic-Like Systems Over a Finite Time Horizon}.
	\item \cite{Lax1987}. {\sc Peter D. Lax}. {\it Hyperbolic Systems of Conversation Laws \& The Mathematical Theory of Shock Waves}.
	\item \cite{Lax2006} . {\sc Peter D. Lax}. {\it Hyperbolic PDEs}.
	\item \cite{Rauch2012}. {\sc Jeffrey Rauch}. {\it Hyperbolic PDEs \& Geometric Optics}.
	\item \cite{Tartar2008}. {\sc Luc Tartar}. {\it From Hyperbolic Systems to Kinetic Theory}.
\end{enumerate}

\subsection{Porous Medium Equations [PMEs]}
\textbf{\textsf{Resources -- Tài nguyên.}}
\begin{enumerate}
	\item \cite{Acker_Kawohl1989}. {\sc Andrew F. Acker, Bernhard Kawohl}. {\it Remarks on quenching}.\hfill{\sf[95 citations]}
	\item \cite{Bonforte_Figalli_Vazquez2018}. {\sc Matteo Bonforte, Alessio Figalli, Juan Luis V\'{a}zquez}. {\it Sharp global estimates for local \& nonlocal porous medium-type equations in bounded domains}.\hfill{\sf[55 citations]}
	
	{\bf Keywords.} nonlocal diffusion, nonlinear equations, bounded domains, a priori estimates, positivity, boundary behavior, regularity, Harnack inequalities.
	
	{\bf Abstract.} Provide a quantitative study of nonnegative solutions to nonlinear diffusion equations of porous medium-type of the form $\partial_tu + \mathcal{L}u^m = 0$, $m > 1$, where the operator $\mathcal{L}$ belongs to a general class of linear operators, \& eqn is posed in a bounded domain $\Omega\subset\mathbb{R}^N$. As possible operators: include 3 most common definitions of the fractional Laplacian in a bounded domain with zero Dirichlet conditions, \& also a number of other nonlocal versions. In particular, $\mathcal{L}$ can be a fractional power of a uniformly elliptic operator with $C^1$ coefficients. Since nonlinearity is given by $u^m$ with $m > 1$, eqn is degenerate parabolic.
	
	Basic well-posed theory for this class of equations was recently developed by Bonforte \& V\'azquez. Address regularity theory: decay \& positivity, boundary behavior, Harnack inequalities, interior \& boundary regularity, \& asymptotic behavior. All this is done in a quantitative way, based on sharp a priori estimates. Although focusing on fractional models, results cover also local case when $\mathcal{L}$ is a uniformly elliptic operator, \& provide new estimates even in this setting.
	
	A surprising aspect discovered: possible presence of nonmatching powers for long-time boundary behavior, i.e., when $\mathcal{L} = (-\Delta)^s$ is a spectral power of Dirichlet Laplacian inside a smooth domain, can prove that: (i) when $s > 2\left(1 - \frac{1}{m}\right)$, for large times all solutions behave as ${\rm dist}^{\frac{1}{m}}$ near the boundary; (ii) when $s\le2\left(1 - \frac{1}{m}\right)$, different solutions may exhibit different boundary behavior. This unexpected phenomenon is a completely new feature of nonlocal nonlinear structure of this model, \& not present in semilinear elliptic equation $\mathcal{L}u^m = u$.
	\begin{itemize}
		\item {\sf Sect. 1: Introduction.}
		\begin{goal}
			Address question of obtaining a priori estimates, positivity, boundary behavior, Harnack inequalities, \& regularity for a suitable class of weak solutions of nonlinear nonlocal diffusion equations of form $\partial_tu + \mathcal{L}F(u) = 0$ in $Q_\infty = (0,\infty)\times\Omega$, where $\Omega\subset\mathbb{R}^N$ is a bounded domain with $C^{1,1}$ boundary, $N\ge2$ (results work also in 1D if fractional exponent $0 < s <\frac{1}{2}$. The interval $\frac{1}{2}\le s < 1$ requires some minor modifications preferred to avoid.), $\mathcal{L}$: a linear operator representing diffusion of local or nonlocal type, the prototype example being fractional Laplacian (class of admissible operators).
		\end{goal}
		Although arguments hold for a rather general class of nonlinearities $F:\mathbb{R}\to\mathbb{R}$, for simplicity, focus on model case $F(u)\coloneqq u^m$ with $m > 1$.
		
		Use of nonlocal operators in diffusion equations reflects the need to model presence of long-distance effects not included in evolution driven by Laplace operator: well documented in literature. Physical motivation \& relevance of nonlinear diffusion models with nonlocal operators. Because $u$ usually represents a density, all data \& solutions are supposed to be nonnegative. Since the problem is posed on a bounded domain, need boundary or external conditions assumed to be of Dirichlet type.
		
		Extensively studied when $\mathcal{L} = -\Delta,F(u) = u^m$, $m > 1$, eqn becomes classical PME. Here interested in treating nonlocal diffusion operators, in particular fractional Laplacian operators. Since working on a bounded domain, the concept of fractional Laplacian operator admits several nonequivalent versions, the best known being the restricted fractional Laplacian (RFL), the spectral fractional Laplacian (SFL), \& the censored fractional Laplacian (CFL). RFL is usually known as the {\it standard fractional Laplacian}, or plainly fractional Laplacian, \& the CFL is often called the {\it regional fractional Laplacian}.
		
		The case of SFL operator with $F(u) = u^m$, $m > 1$, was already studied in [Bonforte \& Vázquez 2015; 2016]. In [Bonforte \& Vázquez 2016] presented a rather abstract setting where they were able to treat not only usual fractional Laplacians but also a large number of variants listed. Rather general increasing nonlinearities $F$ were allowed. Basic questions of existence \& uniqueness of suitable solutions for this problem were solved in [Bonforte \& Vázquez 2016] in the class of ``weak dual solutions'', an extension of the concept of solution introduced in [Bonforte \& Vázquez 2015] having proved to be quite flexible \& efficient. Derived a number of a priori estimates (absolute bounds \& smoothing effects) in that generality.
		
		Since these basic facts are settled, here focus on finer aspects of theory, mainly sharp boundary estimates \& decay estimates. Such upper \& lower bounds will be formulated in terms of 1st eigenfunction $\Phi_1$ of $\mathcal{L}$, which under our assumptions will satisfy $\Phi_1\asymp{\rm dist}(\cdot,\Gamma)^\gamma$ for a certain characteristics power $\gamma\in(0,1]$ depending on particular operator being considered. Typical values: $\gamma = s$ (SFL), $\gamma = 1$ (RFL), $\gamma = s - \frac{1}{2}$ for $s > \frac{1}{2}$ (CFL) $\Rightarrow$ get various kinds of local \& global Harnack-type inequalities.
		
		Some of the boundary estimates obtained for parabolic case are essentially elliptic in nature. Study of this issue for stationary problems is done in a companion paper [Bonforte et al. 2017b]. Advantage: many arguments are clearer, since parabolic problem is more complicated than elliptic one. Clarifying such differences is 1 of main contributions. Prove both interior \& boundary regularity, \& to find large-time asymptotic behavior of solutions.
		
		{\sf Notation.} Some notation of general use. Notation $a\asymp b$ whenever there exist universal constants $c_0,c_1 > 0$ s.t. $c_0b\le a\le c_1b$. $a\lor b = \max\{a,b\},a\land b = \min\{a,b\}$. Always consider bounded domains $\Omega$ with boundary of class $C^2$. Use short form ``solution'' to mean ``weak dual solution'', unless differently stated.
		
		{\bf Presentation of results on sharp boundary behavior.} A basic principle: sharp boundary estimates depend not only on $\mathcal{L}$ but also on behavior of nonlinearity $F(u)$ near $u = 0$, i.e., on exponent $m > 1$. [$\ldots$]
		
		{\bf Asymptotic behavior \& regularity.}
		\item {\sf Sect. 2: General class of operators \& their kernels.} The interest of theory developed here lies both in the sharpness of results \& in wide range of applicability. Mentioned most relevant examples appearing in literature. Theory applies to a general class of operators with definite assumptions. Properties having to be assumed on class of admissible operators, which some of them already appeared in [Bonforte \& Vázquez 2016]. To further develop theory, need to introduce more hypotheses. [Bonforte \& Vázquez 2016] only uses properties of Green function, here make some assumptions also on kernel of $\mathcal{L}$ whenever it exists. Assumptions on the kernel $K$ of $\mathcal{L}$ are needed for positivity results, because need to distinguish between local \& nonlocal cases. Perform study of kernel $K$.
		\item {\sf Sect. 3: Reminders about weak dual solutions.}
		\item {\sf Sect. 4: Upper boundary estimates.}
		\item {\sf Sect. 5: Lower bounds.}
		\item {\sf Sect. 6: Summary of general decay \& boundary results.}
		\item {\sf Sect. 7: Asymptotic behavior.}
		\item {\sf Sect. 8: Regularity results.}
		\item {\sf Sect. 9: Numerical evidence.}
		\item {\sf Sect. 10: Complements, extensions, \& further examples.}
	\end{itemize}
	\item {\sc Matteo Bonforte, Maria Pia Gualdani, Peio Ibarrondo}. {\it Time-Fractional Porous Medium Type Equations: Sharp Time Decay \& Regularization.}.
	
	{\bf Keywords.} PME, fast diffusion equation, nonlocal operators, Caputo fractional time derivative, subdiffusion comparison principle, regularity estimates, long time behavior.
	\begin{itemize}
		\item {\sf Sect 1. Intro.} Several phenomena, from physics to biology to finance, exhibit events during which fractional behavior \& memory effects become predominant; e.g., viscoelastic materials (whose response depends on their current \& past states), certain geographical processes including movement of groundwater or transportation through porous media, neuronal \& gene regulation networks, but also control theory \& more recent modeling of financial market. Fractional calculus provides a reliable tools to describe memory effects. In 1967 M. Caputo, in the context of modeling heterogeneous elastic fields, introduced the {\it Caputo time derivative} of order $\alpha$:
		\begin{equation}
			\label{Caputo time fractional derivative}
			D_t^\alpha f(t)\coloneqq\frac{1}{\Gamma(1 - \alpha)}\frac{\rm d}{{\rm d}t}\int_0^t \frac{f(\tau) - f(0)}{(t - \tau)^\alpha}\,{\rm d}\tau,\ \alpha\in(0,1).
		\end{equation}
		Caputo modeled certain type of fluid diffusing in porous media using this novel nonlocal operator. In these geothermal studies, Darcy's Law is adapted to describe fluids that may carry solid particles obstructing the pores, thus diminishing their size \& creating a pattern of mineralization. This phenomenon has recently been observed in various other types of porous materials, including building materials, \& zeolite.
		
		Systems where particles exhibit anomalous diffusion (sub- or super-diffusion behavior) often involve memory effects, e.g., diffusion in porous media, turbulent flows, \& biological transport processes. These applications require mathematical models allowing particles to do macroscopical long jumps (L\'evy flights), leading to the use of nonlocal operators in the spatial \& time variables. Many different mathematical models describing anomalous diffusion in a porous medium in literature. Main PDE:
		\begin{equation}
			\label{CPME}
			\tag{CPME}
			D_t^\alpha = -\mathcal{L}u^m,\ m > 0,
		\end{equation}
		\& includes a general class of densely defined operators $\mathcal{L}$ both of local \& nonlocal type. Eqn \eqref{CPME} is a {\it density dependent diffusion} resulting in a characteristic scaling $\frac{x}{t^{\frac{\alpha}{2 + d(m - 1)}}}$, whenever diffusion operator is $\mathcal{L} = -\Delta$.
		
		From a mathematical point of view, characteristic scaling of the wetting front variable in anomalous diffusion differs from 1 of classical Heat Eqn $\frac{x}{\sqrt{t}}$. Originally, Caputo derivative arose in linear setting to achieve a subdiffusive characteristic scaling of form $\frac{x}{t^{\frac{\alpha}{2}}}$ with $\alpha\in(0,1)$. Non-locality in time, or memory effect, represents a ``waiting time'' phenomenon typically derived within stochastic framework of Continuous Time Random Walk. \eqref{CPME} encompasses a wide variety of anomalous diffusion models combining local \& nonlocal spatial operators, Caputo fractional time derivative, \& $m$-power like nonlinearities for any $m > 0$. Provide a comprehensive qualitative \& quantitative study of \eqref{CPME}. Beside global well-posedness, main contributions are 3fold: (i) study of comparison principle \& time monotonicity formula, (ii) $L^p$-$L^\infty$ smoothing effects, (iii) optimal long time behavior. {\sf Surprising facts}: {\it regularity effects} \& {\it non-extinction in finite time} for all solutions of \eqref{CPME} when $m\in(0,1)$. Notably: memory effect slows down diffusion, minimizing relevance of nonlinearity parameter $m$ in ranges $m\in(0,1)$ \& $m > 1$. Diffusive nature of eqn still provides a regularization of solution, a feature previously unknown for nonlinear equations involving Caputo derivatives. All these results are new even for classical Laplacian $\mathcal{L} = -\Delta$ \& $\alpha\in(0,1)$.
		
		Within mathematical framework, prototype subdiffusive PDE is ``Heat Eqn with memory'':
		\begin{equation}
			\label{heat with memory}
			D_t^\alpha u = \Delta u.
		\end{equation}
		{\sf Vast literature.} well-posedness \& regularity in $\mathbb{R}^d$, optimal asymptotic decay for solutions of \eqref{heat with memory} for Cauchy problem in $\mathbb{R}^d$. For Dirichlet problem on bounded domains, long time decay estimates, also allowing for general operators with variable coefficients in space \& time. Global well-posedness of \eqref{CPME} with $m = 1$ for singular solutions on bounded domains. From a nonlinear perspective, consider several fractional nonlinear models \& obtain sharp decay estimates of $L^q$ norms using fractional ODEs techniques. Porous Medium with fractional pressure, associated to Caffarelli--Vázquez model studied in its Caputo derivative version. Develop a theory of fractional gradient flows in Hilbert spaces, analogous of Brezis-Komura theory with fractional time derivative. This theory provides well-posedness for both linear \& for nonlinear problems.
		
		Memory effects complicate analysis quite a lot. Memory effects somehow destroy semigroup structure, essential in the De Giorgi-Nash-Moser theory. A nontrivial adaptation of Green function method, achieved by employing novel time monotonicity estimates, a feature coming as a surprise in context of Caputo setting.
		
		To provide further insights about results, some numerology\footnote{the use of numbers to try to tell somebody what will happen in the future. số học.} is in order: asymptotic estimates correspond to known estimates in formal limit $m\to1$, which does not happen in the --formal-- limit $\alpha\to1^-$. Neither exponents of smoothing effects nor the ones in long time behavior estimates, correspond to known ones when $\alpha = 1$.
		
		Principal findings:
		\begin{itemize}
			\item {\bf Comparison principle \& time monotonicity.}
			\item {\bf Smoothing effects \& boundary estimates.}
			\item {\bf Case of unbounded domains.}
			\item {\bf Optimal long time behavior.}
		\end{itemize}		
		\item {\sf Discretized problem.}
		\item {\sf Continuous problem.}
		\item {\sf Smoothing effects.}
		\item {\sf Sharp time decay of $L^p$-norms.}
		\item {\sf Open questions.}
		\item {\sf Appendix A: Fractional ODEs.}
	\end{itemize}
	
	{\bf Abstract.} Consider a class of porous medium type of equations with Caputo time derivative. Prototype problem: $D_t^\alpha u = -\mathcal{L}u^m$ posed on a bounded Euclidean domain $\Omega\subset\mathbb{R}N$ with zero Dirichlet boundary conditions. The operator $\mathcal{L}$ falls within a wide class of either local or nonlocal operators, \& nonlinearity is allowed to be of degenerate or singular type, namely, $0 < m < 1$ \& $m > 1$. Most general form of a variety of models used to describe anomalous\footnote{different from what is normal or expected. dị thường.} diffusion processes with memory effects, \& finds application in various fields, e.g., visco-elastic materials, signal processing, biological systems, \& geophysical science. Prove existence of unique solution \& new $L^p$-$L^\infty$ smoothing effects. The comparison principle, provided in the most general setting, serves as a crucial tool in the proof \& provides a novel monotonicity formula. Consequently, establish: regularizing effects from diffusion are stronger than memory effects introduced by fractional time derivative. Solution attains boundary conditions pointwise. Prove: solution does not vanish in finite time if $0 < m < 1$, unlike case with classical time derivative. Provide a sharp rate of decay for any $L^p$-norm of solution for any $m > 0$. Memory effects weaken the spatial diffusion \& mitigate the difference between slow \& fast diffusion.
	
	
	\item \cite{Dao_Diaz_Nguyen2020}. {\sc Đào Nguyên Anh, Jesus Ildefonso D\'{i}az, Nguyễn Quản Bá Hồng}. {\it Pointwise gradient estimates in multi-dimensional slow diffusion equations with a singular quenching term}. {\sf[4 citations]}
	
	{\bf Keywords.} singular absorption, nonlinear diffusion equations, pointwise gradient estimates, quenching phenomenon, free boundary.
	
	{\bf Abstract.} Consider high-dimensional equation $\partial_tu - \Delta u^m + u^{-\beta}\chi_{\{u > 0\}}$, extend \cite{Kawohl_Kersner1992} 1D case. Prove existence of a very weak solution (VWS) $u\in C([0,T];L_\delta^1(\Omega))$ with $u^{-\beta}\chi_{\{u > 0\}}\in L^1((0,T)\times\Omega)$, $\delta({\bf x})\coloneqq d({\bf x},\partial\Omega)$. Prove some pointwise gradient estimates for a certain range of the dimension $N$, $m\ge1$, $\beta\in(0,m)$, mainly when the absorption dominates over diffusion $1\le m < 2 + \beta$. Prove a new kind of universal gradient estimate when $m + \beta\le2$. Consider several qualitative properties (e.g. finite time quenching phenomena \& finite speed of propagation) \& study of Cauchy problem.
	
	\begin{goal}
		Extend to high-dimensional case \cite{Kawohl_Kersner1992} for a 1D degenerate diffusion equation with a singular absorption term. Study nonnegative solutions of possibly degenerate reaction-diffusion multi-dimensional problem $\partial_tu - \Delta u^m + u^{-\beta}\chi_{\{u > 0\}}$ in $(0,\infty)\times\Omega$, $u^m = 0$ on $(0,\infty)\times\Gamma$, $u(0,{\bf x}) = u_0({\bf x})$ in $\Omega$.
	\end{goal}
	$\Omega$: an open regular bounded domain of $\mathbb{R}^N$, e.g., with $\Gamma$ of calss $C^{1,\alpha}$ for some $\alpha\in(0,1]$, $N\ge1$, $m\ge 1$ ($m > 1$ corresponds to a typical slow diffusion) \& mainly $\beta\in(0,m)$ with some remarks for case $\beta\ge m$. Treat separately the case of whole space $\Omega = \mathbb{R}^N$. The absorption term $u^{-\beta}\chi_{\{u > 0\}}$ becomes singular (\& the diffusion becomes degenerate if $m > 1$) when $u = 0$, \& by this normalization, have $u(t,{\bf x}) = 0\Rightarrow u^{-\beta}\chi_{\{u > 0\}}(t,{\bf x}) = 0$. Boundary condition implies an automatic permanent singularity on $\Gamma$, in contrast to other related problems in which the singularity is permanently excluded of the boundary $u^m = 1$ on $(0,\infty)\times\Gamma$.  Change of unknown $v\coloneqq1 - u^m$ in semilinear case $m = 1$ leads to the formulation $\partial_tv - \Delta v = \frac{\chi_{\{v < 1\}}}{(1 - v)^\beta}$ on $(0,\infty)\times\Omega$. Study the associated Cauchy problem in $(0,\infty)\times\mathbb{R}^N$ can be regarded from 2 different points of view according to the assumptions made on the asymptotic behavior of the initial datum when $|{\bf x}|\to\infty$.	
	\begin{goal}
		Analyze problem of type (P) \& (CP) when $u_0({\bf x})\searrow0$ as $|{\bf x}|\to\infty$.
	\end{goal}
	{\sf Motivation.} Problem (P) was regarded as the limit case of regularized Langmuir--Hinshelwood model in chemical catalyst kinetics for elliptic- \& parabolic equations.
	
	{\sf Interesting point.} Solutions may raise to a free boundary defined as $\partial\{(t,{\bf x});u(t,{\bf x}) > 0\}$. (P1) denoted as a {\it quenching problem}. Appearance of a blow-up time for $\partial_tu$ at the 1st time $T_{\rm c} > 0$ in which $u(T_{\rm c},{\bf x}) = 0$ at some point ${\bf x}\in\Omega$.
	
	Case $\beta\ge m$ presents special difficulties when the free boundary $\partial\{(t,{\bf x});u(t,{\bf x}) > 0\}$ is a nonempty hypersurface, this set corresponds to the set of {\it rupture points} in study of thin films. This case $\beta\ge m$ also arises in the modeling of micro-electromechanical systems (MEMS), in which mainly $m = 1,\beta = 2$.
	
	\item \cite{Kawohl1992}. {\sc Bernhard Kawohl}. {\it Remarks on quenching, blow up, \& dead cores}.
	\item \cite{Kawohl1996}. {\sc Bernhard Kawohl}. {\it Remarks on quenching}.
	\item \cite{Kawohl_Kersner1992}. {\sc Bernhard Kawohl, Robert Kersner}. {\it On degenerate diffusion with very strong absorption}.
	\item \cite{Phillips1987}. {\sc Daniel Phillips}. {\it Existence of solutions of quenching problems}.\hfill{\sf[100 citations]}
	
	Prove existence of weak solutions to PME when $N\ge1$, $m = 1$, $\beta\in(0,1)$.
	\item \cite{Vazquez2007}. {\sc Juan Luis V\'{a}zquez}. {\it The Porous Medium Equation}.
	
	{\sf Note.} Có nhiều bộ ký hiệu xung đột nhau do tác giả chắp vá quyển sách từ nhiều bài báo, công trình khác nhau. Nên cẩn thận khi thống nhất bộ ký hiệu.
\end{enumerate}

%------------------------------------------------------------------------------%

\section{Sobolev Spaces -- Không Gian Sobolev}
\textbf{\textsf{Resources -- Tài nguyên.}}
\begin{enumerate}
	\item \cite{Adams_Fournier2003}. {\sc Robert A. Adams, John J. F. Fournier}. {\it Sobolev Spaces}.
	\item \cite{Gagliardo1957}. {\sc Emilio Gagliardo}. {\it Caratterizzazioni delle tracce sulla frontiera relative ad alcune classi di funzioni in {$n$} variabili}.
	\item {\sc Nec\v{a}s}.
	\item \cite{Tartar2006}. {\sc Luc Tartar}. {\it An Introduction to Sobolev Spaces \& Interpolation Spaces}.
\end{enumerate}

%------------------------------------------------------------------------------%

\section{Finite Difference Methods FDMs -- Phương Pháp Sai Phân Hữu Hạn}
\textbf{\textsf{Resources -- Tài nguyên.}}
\begin{enumerate}
	\item \cite{LeVeque2007}. {\sc Randall J. LeVeque}. {\it FDMs for ODE \& PDEs: Steady-State \& Time-Dependent Problems}.
\end{enumerate}

%------------------------------------------------------------------------------%

\section{Finite Element Methods FEMs -- Phương Pháp Phần Tử Hữu Hạn}
\textbf{\textsf{Resources -- Tài nguyên.}}
\begin{enumerate}
	\item \cite{Brenner_Scott2008}. {\sc Susanne C. Brenner, L. Ridgway Scott}. {\it The Mathematical Theory of FEMs}.
	\item \cite{Ern_Guermond2004}. {\sc Alexandre Ern, Jean-Luc Guermond}. {\it Theory \& Practice of Finite Elements}.
	\item \cite{Girault_Raviart1986}. {\sc Vivette Girault, Pierre-Arnaud Raviart}. {\it FEMs for NSEs}.
	\item \cite{Gunzburger1989}. {\sc Max D. Gunzburger}. {\it FEMs for Viscous Incompressible Flows}.
	\item \cite{John2016}. {\sc Volker John}. {\it FEMs for Incompressible Flow Problems}.
\end{enumerate}
I met {\sc Volker John}, lead of Research Group 3 in WIAS in 2020 to discuss on turbulence models, e.g., Smagonrinsky, $k$-$\epsilon$ \& their simulations.

%------------------------------------------------------------------------------%

\section{Finite Volume Methods FVMs -- Phương Pháp Thể Tích Hữu Hạn}
\textbf{\textsf{Resources -- Tài nguyên.}}
\begin{enumerate}
	\item \cite{Barth_Jespersen1989}. {\sc Timothy J. Barth, Dennis C. Jespersen}. {\it The Design and Application of Upwind Schemes on Unstructured Meshes}.
	\item \cite{Darwish_Asmar_Moukalled2004}. {\sc M. Darwish, D. Asmar, F. Moukalled}. {\it A comparative assessment within a multigrid environment of segregated pressure-based algorithms for fluid flow at all speeds}.
	\item \cite{Darwish_Moukalled1996}. {\sc M. Darwish, F. Moukalled}. {\it A new approach for building bounded skew-upwind schemes}.
	\item \cite{Darwish_Moukalled1996a}. {\sc M. Darwish, F. Moukalled}. {\it The normalized weighting factor method: a novel technique for accelerating the convergence of high-resolution convective schemes}.
	\item \cite{Darwish_Moukalled2006}. {\sc M. Darwish, F. Moukalled}. {\it Convective Schemes for Capturing Interfaces of Free-Surface Flows on Unstructured Grids}.
	\item \cite{Darwish_Moukalled_Sekar2001}. {\sc M. Darwish, F. Moukalled, B. Sekar}. {\it A unified formulation of the segregated class of algorithms for multifluid flow at all speeds}.
	\item \cite{Demirdzic2015}. {\sc I. Demird\v{z}i\'c}. {\it On the Discretization of the Diffusion Term in Finite-Volume Continuum Mechanics}.
	\item \cite{Demirdzic_Muzaferija1995}. {\sc I. Demird\v{z}i\'c, S. Muzaferija}. {\it Numerical method for coupled fluid flow, heat transfer and stress analysis using unstructured moving meshes with cells of arbitrary topology},
	\item \cite{Denner_vanWachem2014}. {\sc Fabian Denner, Berend G. M. van Wachem}. {\it Compressive VOF method with skewness correction to capture sharp interfaces on arbitrary meshes}.
	\item \cite{Eymard_Gallouet_Herbin2019}. {\sc Robert Eymard, Thierry Gallou\"et, Rapha\`ele Herbin}. {\it Finite Volume Methods}.
	\item \cite{Gallouet_Herbin_Latche_Mallem2018}. {\sc T. Gallou\"{e}t, R. Herbin, J.-C. Latch\'{e},  K. Mallem}. {\it Convergence of the marker-and-cell scheme for the incompressible NSEs on non-uniform grids}.
	\item \cite{Gallouet_Herbin_Maltese_Novotny2017}. {\sc T. Gallou\"{e}t, R. Herbin, D. Maltese, A.Novotny}. {\it onvergence of the marker-and-cell scheme for the semi-stationary compressible Stokes problem}.
	\item \cite{Gaskell_Lau1988}. {\sc P. H. Gaskell, A. K. C. Lau}. {\it Curvature-compensated convective transport: SMART, a new boundedness-preserving transport algorithm}.
	\item \cite{Harten1983}. {\sc Ami Harten}. {\it High resolution schemes for hyperbolic conservation laws}.
	\item \cite{Harten1997}. {\sc Ami Harten}. {\it High resolution schemes for hyperbolic conservation laws [{MR}0701178 (84g:65115)]}.
	\item \cite{Issa1986}. {\sc R. I. Issa}. {\it Solution of the Implicit Discretized Fluid Flow Equations by Operator Splitting}.
	\item \cite{Jang_Jetli_Acharya1986}. {\sc D. S. Jang, R. Jetli, S. Acharya}. {\it Comparison of the PISO, SIMPLER, and SIMPLEC algorithms for the treatment of the pressure-velocity coupling in steady flow problems}.
	\item \cite{Jasak1996}. {\sc H. Jasak}. {\it Error Analysis and Estimation for the Finite Volume Method with Applications for Fluid Flow}.
	\item \cite{Jasak_Gosman2000}. {\sc H. Jasak,  A. D. Gosman}. {\it Automatic Resolution Control for the Finite Volume  Method, Part 1}.
	\item \cite{Leonard1979}. {\sc B. P. Leonard}. {\it A stable and accurate convective modelling procedure based on quadratic upstream interpolation}.
	\item \cite{Leonard1987}. {\sc B. P. Leonard}. {\it SHARP Simulation of Discontinuities in Highly Convective Steady Flow}.
	\item \cite{Leonard1988}. {\sc B. P. Leonard}. {\it Universal Limiter for Transient Interpolation Modeling of the Advective Transport Equations: The ULTIMATE Conservative Difference Scheme}.
	\item \cite{LeVeque2002}. {\sc Randall J. LeVeque}. {\it FEMs for Hyperbolic Problems}.
	\item \cite{Mathur_Murthy1997}. {\sc S. R. Mathur, J. Y. Murthy}. {\it A Pressure-Based Method for Unstructured Meshes}.
	\item \cite{Moukalled_Darwish2000}. {\sc F. Moukalled, M. Darwish}. {\it A unified formulation of the segregated class of algorithms for fluid flow at all speeds}.
	\item \cite{Moukalled_Darwish2012}. {\sc F. Moukalled, M. Darwish}. {\it Transient Schemes for Capturing Interfaces of Free-Surface Flows}.
	\item \cite{Moukalled_Mangani_Darwish2016}. {\sc F. Moukalled, L. Mangani, M. Darwish}. {\it The FVM in CFD}.
	\item \cite{Muzaferija1994}. {\sc S. Muzaferija}. {\it Adaptive Finite Volume Method for Flow Predictions Using Unstructured Meshes and Multigrid Approach}.
	\item \cite{Muzaferija_Gosman1997}. {\sc S. Muzaferija, A. D. Gosman}. {\it Finite Volume CFD Procedure and Adaptive Error Control Strategy for Grids of Arbitrary Topology}.
	\item \cite{Nicolaides1992}. {\sc R. A. Nicolaides}. {\it Analysis and convergence of the {MAC} scheme. I. The linear problem}.
	\item \cite{Nicolaides_Wu1996}. {\sc R. A. Nicolaides, X. Wu}. {\it Analysis and convergence of the MAC scheme. II. NSEs}.
	\item \cite{Ollivier-Gooch_Altena2002}. {\sc Carl Ollivier-Gooch, Michael Van Altena}. {\it A High-Order-Accurate Unstructured Mesh Finite-Volume Scheme for the Advection--Diffusion Equation}.
	\item \cite{Perron_Boivin_Herard2004}. {\sc S\'{e}bastien Perron, Sylvain Boivin, Jean-Marc H\'{e}rard}. {\it A FVM to solve the 3D NSEs on unstructured collocated meshes}.
	\item \cite{Rhie_Chow1983}. {\sc C. M. Rhie, W. L. Chow}. {\it Numerical Study of the Turbulent Flow Past an Airfoil with Trailing Edge Separation}.
	\item \cite{Sweby1984}. {\sc Peter K. Sweby}. {\it High resolution schemes using flux limiters for hyperbolic conservation laws}.
	\item \cite{Sweby1985}. {\sc Peter K. Sweby}. {\it High resolution TVD schemes using flux limiters}.
	\item \cite{Versteeg_Malalasekera2007}. {\sc H. K. Versteeg, W. Malalasekera}. {\it An Introduction to CFD: The FVM}.
	\item \cite{Yen_Liu1993}. {\sc Ruey-Hor Yen, Chen-Hua Liu}. {\it Enhancement of the SIMPLE algorithm by an additional explicit corrector step}.
\end{enumerate}

%------------------------------------------------------------------------------%

\section{Mathematicians \& Their Legacies -- Các Nhà Toán Học \& Các Di Sản}

\subsection{\href{https://en.wikipedia.org/wiki/Mathematician}{Wikipedia\texttt{/}Mathematician}}
\textbf{Mathematician.} \href{https://en.wikipedia.org/wiki/Euclid}{Euclid} (holding \href{https://en.wikipedia.org/wiki/Calipers}{calipers}), Greek mathematician, known as the ``Father of Geometry''

\noindent
\textbf{Occupation.}
\begin{itemize}
	\item \textbf{Occupation type.} \href{https://en.wikipedia.org/wiki/Academic}{Academic}
\end{itemize}
\textbf{Description.}
\begin{itemize}
	\item \textbf{Competencies.} \href{https://en.wikipedia.org/wiki/Mathematics}{Mathematics}, \href{https://en.wikipedia.org/wiki/Analytical_skill}{analytical skills} \& \href{https://en.wikipedia.org/wiki/Critical_thinking}{critical thinking} skills.
	\item \textbf{Education required.} \href{https://en.wikipedia.org/wiki/Doctoral_degree}{Doctoral degree}, occasionally \href{https://en.wikipedia.org/wiki/Master's_degree}{master's degree}.
	\item \textbf{Fields of employment.}
	\begin{itemize}
		\item universities,
		\item private corporations,
		\item financial industry,
		\item government
	\end{itemize}
	\item \textbf{Related jobs.} \href{https://en.wikipedia.org/wiki/Statistician}{statistician}, \href{https://en.wikipedia.org/wiki/Actuary}{actuary}.
\end{itemize}
A \textit{mathematician} is someone who uses an extensive knowledge of \href{https://en.wikipedia.org/wiki/Mathematics}{mathematics} in their work, typically to solve \href{https://en.wikipedia.org/wiki/Mathematical_problem}{mathematical problems}.

Mathematicians are concerned with \href{https://en.wikipedia.org/wiki/Number}{numbers}, \href{https://en.wikipedia.org/wiki/Data}{data}, \href{https://en.wikipedia.org/wiki/Quantity}{quantity}, \href{https://en.wikipedia.org/wiki/Mathematical_structure}{structure}, \href{https://en.wikipedia.org/wiki/Space}{space}, \href{https://en.wikipedia.org/wiki/Mathematical_model}{models}, \& \href{https://en.wikipedia.org/wiki/Mathematics#Change}{change}.

\subsubsection{History}
For broader coverage of this topic, see \href{https://en.wikipedia.org/wiki/History_of_mathematics}{History of mathematics}.

%
1 of the earliest known mathematicians was \href{https://en.wikipedia.org/wiki/Thales_of_Miletus}{Thales of Miletus} (c. 624--c.546 BC); he has been hailed as the 1st true mathematician \& the 1st known individual to whom a mathematical discovery has been attributed.[Boyer (1991), \textit{A History of Mathematics}, p. 43]

He is credited with the 1st use of deductive reasoning applied to geometry, by deriving 4 corollaries to \href{https://en.wikipedia.org/wiki/Thales%27_Theorem}{Thales' Theorem}.

%
The number of known mathematicians grew when \href{https://en.wikipedia.org/wiki/Pythagoras_of_Samos}{Pythagoras of Samos} (c. 582--c. 507 BC) established the \href{https://en.wikipedia.org/wiki/Pythagoreans}{Pythagorean School}, whose doctrine it was that mathematics ruled the universe \& whose motto was ``All is number''.[Boyer 1991, ``\textit{Ionia \& the Pythagoreans}'', p. 49]

It was the Pythagoreans who coined the term ``mathematics'', \& with whom the study of mathematics for its own sake begins.

%
The 1st woman mathematician recorded by history was \href{https://en.wikipedia.org/wiki/Hypatia}{Hypatia} of Alexandria (AD 350--415).

She succeeded her father as Librarian at the Great Library \& wrote many works on applied mathematics.

Because of a political dispute, the Christian community in Alexandria punished her, presuming she was involved, by stripping her naked \& scraping off her skin with clamshells (some say roofing tiles).[``\textit{Ecclesiastical History}, Bk VI: Chap. 15''. Archived from the original on 2014-08-14. Retrieved 2014-11-19.]

%
Science \& mathematics in the Islamic world during the Middle Ages followed various models \& modes of funding varied based primarily on scholars.

It was extensive patronage \& strong intellectual policies implemented by specific rulers that allowed scientific knowledge to develop in many areas.

Funding for translation of scientific texts in other languages was ongoing throughout the reign of certain caliphs,[Abattouy, M., Renn, J. \& Weinig, P., 2001. \textit{Transmission as Transformation: The Translation Movements in the Medieval East \& West in a Comparative Perspective}. Science in Context, 14(1-2), 1-12.] \& it turned out that certain scholars became experts in the works they translated \& in turn received further support for continuing to develop certain sciences.

As these sciences received wider attention from the elite, more scholars were invited \& funded to study particular sciences.

An example of a translator \& mathematician who benefited from this type of support was \href{https://en.wikipedia.org/wiki/Al-Khawarizmi}{al-Khawarizmi}.

A notable feature of many scholars working under Muslim rule in medieval times is that they were often polymaths.

Examples include the work on \href{https://en.wikipedia.org/wiki/Optics}{optics}, maths \& \href{https://en.wikipedia.org/wiki/Astronomy}{astronomy} of \href{https://en.wikipedia.org/wiki/Ibn_al-Haytham}{Ibn al-Haytham}.

%
The \href{https://en.wikipedia.org/wiki/Renaissance}{Renaissance} brought an increased emphasis on mathematics \& science to Europe.

During this period of transition from a mainly feudal \& ecclesiastical culture to a predominantly secular one, many notable mathematicians had other occupations: \href{https://en.wikipedia.org/wiki/Luca_Pacioli}{Luca Pacioli} (founder of \href{https://en.wikipedia.org/wiki/Accounting}{accounting}); \href{https://en.wikipedia.org/wiki/Niccol%C3%B2_Fontana_Tartaglia}{Niccolò Fontana Tartaglia} (notable engineer \& bookkeeper); \href{https://en.wikipedia.org/wiki/Gerolamo_Cardano}{Gerolamo Cardano} (earliest founder of probability \& binomial expansion); \href{https://en.wikipedia.org/wiki/Robert_Recorde}{Robert Recorde} (physician) \& \href{https://en.wikipedia.org/wiki/Fran%C3%A7ois_Vi%C3%A8te}{François Viète}s (lawyer). 

%
As time passed, many mathematicians gravitated towards universities.

An emphasis on free thinking \& experimentation had begun in Britain's oldest universities beginning in the 17th century at Oxford with the scientists \href{https://en.wikipedia.org/wiki/Robert_Hooke}{Robert Hooke} \& \href{https://en.wikipedia.org/wiki/Robert_Boyle}{Robert Boyle}, \& at Cambridge where \href{https://en.wikipedia.org/wiki/Isaac_Newton}{Isaac Newton} was \href{https://en.wikipedia.org/wiki/Lucasian_Professor_of_Mathematics}{Lucasian Professor of Mathematics \& Physics}.

Moving into the 19th century, the objective of universities all across Europe evolved from teaching the ``\textit{regurgitation of knowledge}'' to ``\textit{encourag[ing] productive thinking}.''[Röhrs, ``\textit{The Classical Idea of the University},'' Tradition \& Reform of the University under an International Perspective p. 20]

In 1810, Humboldt convinced the King of Prussia to build a university in Berlin based on \href{https://en.wikipedia.org/wiki/Friedrich_Schleiermacher}{Friedrich Schleiermacher}'s liberal ideas; the goal was to demonstrate the process of the discovery of knowledge \& to teach students to ``\textit{take account of fundamental laws of science in all their thinking}.''

Thus, seminars \& laboratories started to evolve.[Rüegg, ``Themes'', \textit{A History of the University in Europe, Vol. III}, p. 5--6]

%
British universities of this period adopted some approaches familiar to the Italian \& German universities, but as they already enjoyed substantial freedoms \& \href{https://en.wikipedia.org/wiki/Autonomy}{autonomy} the changes there had begun with the \href{https://en.wikipedia.org/wiki/Age_of_Enlightenment}{Age of Enlightenment}, the same influences that inspired Humboldt.

The Universities of \href{https://en.wikipedia.org/wiki/University_of_Oxford}{Oxford} \& \href{https://en.wikipedia.org/wiki/University_of_Cambridge}{Cambridge} emphasized the importance of \href{https://en.wikipedia.org/wiki/Research}{research}, arguably more authentically implementing Humboldt's idea of a university than even German universities, which were subject to state authority.[Rüegg, ``Themes'', \textit{A History of the University in Europe, Vol. III}, p. 12]

Overall, science (including mathematics) became the focus of universities in the 19th \& 20th centuries.

Students could conduct research in \href{https://en.wikipedia.org/wiki/Seminars}{seminars} or \href{https://en.wikipedia.org/wiki/Laboratories}{laboratories} \& began to produce doctoral theses with more scientific content.[Rüegg, ``Themes'', \textit{A History of the University in Europe, Vol. III}, p. 13]

According to Humboldt, the mission of the \href{https://en.wikipedia.org/wiki/University_of_Berlin}{University of Berlin} was to pursue scientific knowledge.[Rüegg, ``Themes'', \textit{A History of the University in Europe, Vol. III}, p. 16]

\textit{The German university system fostered professional, bureaucratically regulated scientific research performed in well-equipped laboratories, instead of the kind of research done by private \& individual scholars in Great Britain \& France}.[Rüegg, ``Themes'', \textit{A History of the University in Europe, Vol. III}, p. 17--18]

In fact, Rüegg asserts that the \textit{German system is responsible for the development of the modern research university because it focused on the idea of ``freedom of scientific research, teaching \& study.''}[Rüegg, ``Themes'', \textit{A History of the University in Europe, Vol. III}, p. 31]

\subsubsection{Required education}
Mathematicians usually cover a breadth of topics within mathematics in their \href{https://en.wikipedia.org/wiki/Undergraduate_education}{undergraduate education}, \& then proceed to specialize in topics of their own choice at the \href{https://en.wikipedia.org/wiki/Graduate-level}{graduate level}.

In some universities, a \href{https://en.wikipedia.org/wiki/Qualifying_exam}{qualifying exam} serves to test both the breadth \& depth of a student's understanding of mathematics; the students, who pass, are permitted to work on a \href{https://en.wikipedia.org/wiki/Doctoral_dissertation}{doctoral dissertation}.

\subsubsection{Activities}
\textsf{\href{https://en.wikipedia.org/wiki/Emmy_Noether}{Emmy Noether}, mathematical theorist \& teacher.}

\paragraph{Applied mathematics.} Main article: \href{https://en.wikipedia.org/wiki/Applied_mathematics}{Applied mathematics}. Mathematicians involved with solving problems with applications in real life are called \href{https://en.wikipedia.org/wiki/Applied_mathematician}{applied mathematicians}.

Applied mathematicians are mathematical scientists who, with their specialized knowledge \& \href{https://en.wikipedia.org/wiki/Professional}{professional} methodology, approach many of the imposing problems presented in related scientific fields.

With professional focus on a wide variety of problems, theoretical systems, \& localized constructs, applied mathematicians work regularly in the study \& formulation of \href{https://en.wikipedia.org/wiki/Mathematical_models}{mathematical models}.

Mathematicians \& applied mathematicians are considered to be 2 of the STEM (science, technology, engineering, \& mathematics) careers.

%
The discipline of \href{https://en.wikipedia.org/wiki/Applied_mathematics}{applied mathematics} concerns itself with mathematical methods that are typically used in science, engineering, business, \& industry; thus, ``applied mathematics'' is a \href{https://en.wikipedia.org/wiki/Mathematical_science}{mathematical science} with specialized knowledge.

The term ``applied mathematics'' also describes the professional specialty in which mathematicians work on problems, often concrete but sometimes abstract.

As professionals focused on problem solving, \textit{applied mathematicians} look into the \textit{formulation, study}, \& \textit{use of mathematical models} in \href{https://en.wikipedia.org/wiki/Science}{science}, \href{https://en.wikipedia.org/wiki/Engineering}{engineering}, \href{https://en.wikipedia.org/wiki/Business}{business}, \& other areas of mathematical practice.

\paragraph{Abstract mathematics.} Main article: \href{https://en.wikipedia.org/wiki/Pure_mathematics}{Pure mathematics}. \href{https://en.wikipedia.org/wiki/Pure_mathematics}{Pure mathematics} is mathematics that studies entirely abstract \href{https://en.wikipedia.org/wiki/Concept}{concepts}.

From the 18th century onwards, this was a recognized category of mathematical activity, sometimes characterized as \textit{speculative mathematics},[See for example titles of works by Thomas Simpson from the mid-18th century: \textit{Essays on Several Curious \& Useful Subjects in Speculative \& Mixed Mathematics, Miscellaneous Tracts on Some Curious \& Very Interesting Subjects in Mechanics, Physical Astronomy \& Speculative Mathematics. Chisholm}, Hugh, ed. (1911). ``Simpson, Thomas''. Encyclop\ae dia Britannica. 25 (11th ed.). Cambridge University Press. p. 135.] \& at variance with the trend towards meeting the needs of \href{https://en.wikipedia.org/wiki/Navigation}{navigation}, \href{https://en.wikipedia.org/wiki/Astronomy}{astronomy}, \href{https://en.wikipedia.org/wiki/Physics}{physics}, \href{https://en.wikipedia.org/wiki/Economics}{economics}, \href{https://en.wikipedia.org/wiki/Engineering}{engineering}, \& other applications.

%
Another insightful view put forth is that \textit{pure mathematics is not necessarily \href{https://en.wikipedia.org/wiki/Applied_mathematics}{applied mathematics}}: it is possible to study abstract entities w.r.t. their intrinsic nature, \& not be concerned with how they manifest in the real world.[Andy Magid, \textit{Letter from the Editor, in Notices of the AMS}, Nov 2005, American Mathematical Society, p. 1173. [1] Archived 2016-03-03 at the Wayback Machine]

\textit{Even though the pure \& applied viewpoints are distinct philosophical positions, in practice there is much overlap in the activity of pure \& applied mathematicians}.

%
\textit{To develop accurate models for describing the real world, many applied mathematicians draw on tools \& techniques that are often considered to be ``pure'' mathematics}.

\textit{On the other hand, many pure mathematicians draw on natural \& social phenomena as inspiration for their abstract research}.

\paragraph{Mathematics teaching.} Many professional mathematicians also engage in the teaching of mathematics.

Duties may include:
\begin{itemize}
	\item teaching university mathematics courses;
	\item supervising undergraduate \& graduate research; and
	\item serving on academic committees.
\end{itemize}

\paragraph{Consulting.} Many careers in mathematics outside of universities involve consulting.

E.g., actuaries assemble \& analyze data to estimate the probability \& likely cost of the occurrence of an event such as death, sickness, injury, disability, or loss of property.

Actuaries also address financial questions, including those involving the level of pension contributions required to produce a certain retirement income \& the way in which a company should invest resources to maximize its return on investments in light of potential risk.

Using their broad knowledge, actuaries help design \& price insurance policies, pension plans, \& other financial strategies in a manner which will help ensure that the plans are maintained on a sound financial basis.

%
As another example, mathematical finance will derive \& extend the \href{https://en.wikipedia.org/wiki/Mathematical_model}{mathematical} or \href{https://en.wikipedia.org/wiki/Numerical_analysis}{numerical} models without necessarily establishing a link to \textit{financial theory}, taking observed market prices as input.

Mathematical consistency is required, not compatibility with \textit{economic theory}.

Thus, e.g., while a financial economist might study the structural reasons why a company may have a certain \href{https://en.wikipedia.org/wiki/Share_price}{share price}, a financial mathematician may take the share price as a given, \& attempt to use \href{https://en.wikipedia.org/wiki/Stochastic_calculus}{stochastic calculus} to obtain the corresponding value of \href{https://en.wikipedia.org/wiki/Derivative_(finance)}{derivatives} of the \href{https://en.wikipedia.org/wiki/Stock}{stock} (see: \href{https://en.wikipedia.org/wiki/Valuation_of_options}{Valuation of options}; \href{https://en.wikipedia.org/wiki/Financial_modeling#Quantitative_finance}{Financial modeling}).

\subsubsection{Occupations}
\textsf{In 1938 in the United States, mathematicians were desired as teachers, calculating machine operators, mechanical engineers, accounting auditor bookkeepers, \& actuary statisticians.}

According to the \href{https://en.wikipedia.org/wiki/Dictionary_of_Occupational_Titles}{Dictionary of Occupational Titles} occupations in mathematics include the following.[``020 OCCUPATIONS IN MATHEMATICS''. \textit{Dictionary Of Occupational Titles}. Archived from the original on 2012-11-02. Retrieved 2013-01-20.]
\begin{itemize}
	\item Mathematician
	\item Operations-Research Analyst
	\item Mathematical Statistician
	\item Mathematical Technician
	\item \href{https://en.wikipedia.org/wiki/Actuary}{Actuary}
	\item Applied Statistician
	\item Weight Analyst
\end{itemize}

\subsubsection{Quotations about mathematicians}
The following are quotations about mathematicians, or by mathematicians.
\begin{quotation}
	``\textit{A mathematician is a device for turning coffee into theorems}.'' - Attributed to both \href{https://en.wikipedia.org/wiki/Alfr%C3%A9d_R%C3%A9nyi}{Alfréd Rényi}[15] \& \href{https://en.wikipedia.org/wiki/Paul_Erd%C5%91s}{Paul Erd\"os}
\end{quotation}

\begin{quotation}
	``\textit{Die Mathematiker sind eine Art Franzosen; redet man mit ihnen, so übersetzen sie es in ihre Sprache, und dann ist es alsobald ganz etwas anderes}.''
	
	(Mathematicians are [like] a sort of Frenchmen; if you talk to them, they translate it into their own language, \& then it is immediately something quite different.) - \href{https://en.wikipedia.org/wiki/Johann_Wolfgang_von_Goethe}{Johann Wolfgang von Goethe}[16]
\end{quotation}

\begin{quotation}
	``\textit{Each generation has its few great mathematicians$\ldots$ \& [the others'] research harms no one}.'' - Alfred W. Adler ($\sim$1930), ``\textit{Mathematics \& Creativity}''[17]
\end{quotation}

\begin{quotation}
	``In short, I never yet encountered the mere mathematician who could be trusted out of equal roots, or one who did not clandestinely hold it as a point of his faith that x squared + px was absolutely \& unconditionally equal to q. Say to one of these gentlemen, by way of experiment, if you please, that you believe occasions may occur where x squared + px is not altogether equal to q, and, having made him understand what you mean, get out of his reach as speedily as convenient, for, beyond doubt, he will endeavor to knock you down.'' - \href{https://en.wikipedia.org/wiki/Edgar_Allan_Poe}{Edgar Allan Poe}, \textit{The purloined letter}
\end{quotation}

\begin{quotation}
	``A mathematician, like a painter or poet, is a maker of patterns. If his patterns are more permanent than theirs, it is because they are made with ideas.'' - \href{https://en.wikipedia.org/wiki/G._H._Hardy}{G. H. Hardy}, \textit{A Mathematician's Apology}
\end{quotation}

\begin{quotation}
	``\textit{Some of you may have met mathematicians \& wondered how they got that way}.'' - \href{https://en.wikipedia.org/wiki/Tom_Lehrer}{Tom Lehrer}
\end{quotation}

\begin{quotation}
	``\textit{It is impossible to be a mathematician without being a poet in soul}.'' - \href{https://en.wikipedia.org/wiki/Sofia_Kovalevskaya}{Sofia Kovalevskaya}
\end{quotation}

\begin{quotation}
	``\textit{There are 2 ways to do great mathematics. The first is to be smarter than everybody else. The second way is to be stupider than everybody else - but persistent.}'' - \href{https://en.wikipedia.org/wiki/Raoul_Bott}{Raoul Bott}
\end{quotation}

\begin{quotation}
	``\textit{Mathematics is the queen of the sciences \& arithmetic the queen of mathematics}.'' - \href{https://en.wikipedia.org/wiki/Carl_Friedrich_Gauss}{Carl Friedrich Gauss} [18]
\end{quotation}

\subsubsection{Prizes in mathematics}
There is no Nobel Prize in mathematics, though sometimes mathematicians have won the Nobel Prize in a different field, such as economics.

Prominent prizes in mathematics include the \href{https://en.wikipedia.org/wiki/Abel_Prize}{Abel Prize}, the \href{https://en.wikipedia.org/wiki/Chern_Medal}{Chern Medal}, the \href{https://en.wikipedia.org/wiki/Fields_Medal}{Fields Medal}, the \href{https://en.wikipedia.org/wiki/Gauss_Prize}{Gauss Prize}, the \href{https://en.wikipedia.org/wiki/Frederic_Esser_Nemmers_Prize}{Nemmers Prize}, the \href{https://en.wikipedia.org/wiki/Balzan_Prize}{Balzan Prize}, the \href{https://en.wikipedia.org/wiki/Crafoord_Prize}{Crafoord Prize}, the \href{https://en.wikipedia.org/wiki/Shaw_Prize}{Shaw Prize}, the \href{https://en.wikipedia.org/wiki/Steele_Prize}{Steele Prize}, the \href{https://en.wikipedia.org/wiki/Wolf_Prize}{Wolf Prize}, the \href{https://en.wikipedia.org/wiki/Schock_Prize}{Schock Prize}, \& the \href{https://en.wikipedia.org/wiki/Nevanlinna_Prize}{Nevanlinna Prize}.

%
The \href{https://en.wikipedia.org/wiki/American_Mathematical_Society}{American Mathematical Society}, \href{https://en.wikipedia.org/wiki/Association_for_Women_in_Mathematics}{Association for Women in Mathematics}, \& other mathematical societies offer several prizes aimed at increasing the representation of women \& minorities in the future of mathematics.

\subsubsection{Mathematical autobiographies}
Several well known mathematicians have written autobiographies in part to explain to a general audience what it is about mathematics that has made them want to devote their lives to its study.

These provide some of the best glimpses into what it means to be a mathematician.

The following list contains some works that are not autobiographies, but rather essays on mathematics \& mathematicians with strong autobiographical elements.
\begin{itemize}
	\item \textit{The Book of My Life} - \href{https://en.wikipedia.org/wiki/Girolamo_Cardano}{Girolamo Cardano}[19]
	\item \href{https://en.wikipedia.org/wiki/A_Mathematician's_Apology}{A Mathematician's Apology} - \href{https://en.wikipedia.org/wiki/G.H._Hardy}{G.H. Hardy}[20]
	\item \href{https://en.wikipedia.org/wiki/A_Mathematician's_Miscellany}{A Mathematician's Miscellany} (republished as Littlewood's miscellany) - \href{https://en.wikipedia.org/wiki/J._E._Littlewood}{J. E. Littlewood}[Littlewood, J. E. (1990) [Originally \textit{A Mathematician's Miscellany} published in 1953], Béla Bollobás (ed.), \textit{Littlewood's miscellany}, Cambridge University Press, ISBN 0-521-33702 X]
	\item \textit{I Am a Mathematician} - \href{https://en.wikipedia.org/wiki/Norbert_Wiener}{Norbert Wiener}[Wiener, Norbert (1956), \textit{I Am a Mathematician / The Later Life of a Prodigy}, The M.I.T. Press, ISBN 0-262-73007-3]
	\item \textit{I Want to be a Mathematician} - \href{https://en.wikipedia.org/wiki/Paul_R._Halmos}{Paul R. Halmos}
	\item \textit{Adventures of a Mathematician} - \href{https://en.wikipedia.org/wiki/Stanislaw_Ulam}{Stanislaw Ulam}[Ulam, S. M. (1976), \textit{Adventures of a Mathematician}, Charles Scribner's Sons, ISBN 0-684-14391-7]
	\item \textit{Enigmas of Chance} - \href{https://en.wikipedia.org/wiki/Mark_Kac}{Mark Kac}[Kac, Mark (1987), \textit{Enigmas of Chance/An Autobiography}, University of California Press, ISBN 0-520-05986-7]
	\item \textit{Random Curves} - \href{https://en.wikipedia.org/wiki/Neal_Koblitz}{Neal Koblitz}
	\item \href{https://en.wikipedia.org/wiki/Edward_Frenkel#Love_and_Math}{\textit{Love \& Math}} - \href{https://en.wikipedia.org/wiki/Edward_Frenkel}{Edward Frenkel}
	\item \textit{Mathematics Without Apologies} - \href{https://en.wikipedia.org/wiki/Michael_Harris_(mathematician)}{Michael Harris}[Harris, Michael (2015), \textit{Mathematics without apologies/portrait of a problematic vocation}, Princeton University Press, ISBN 978-0-691-15423-7]
\end{itemize}

\subsubsection{See also}
\begin{itemize}
	\item \href{https://en.wikipedia.org/wiki/Lists_of_mathematicians}{Lists of mathematicians}
	\item \href{https://en.wikipedia.org/wiki/Human_computer}{Human computer}
	\item \href{https://en.wikipedia.org/wiki/Mathematical_joke}{Mathematical joke}
	\item \href{https://en.wikipedia.org/wiki/A_Mathematician's_Apology}{A Mathematician's Apology}
	\item \href{https://en.wikipedia.org/wiki/Men_of_Mathematics}{Men of Mathematics} (book)
	\item \href{https://en.wikipedia.org/wiki/Mental_calculator}{Mental calculator}\hfill$\square$
\end{itemize}

%------------------------------------------------------------------------------%

\subsection{\href{https://en.wikipedia.org/wiki/Henri_Berestycki}{Wikipedia{\tt/}Henri Berestycki}}
\textit{Henri Berestycki} (born Mar 25, 1951, in Paris)[1] is a French mathematician who obtained his PhD from Université Paris VI - \href{https://en.wikipedia.org/wiki/Universit\%C3\%A9_Pierre_et_Marie_Curie}{Université Pierre et Marie Curie} in 1975.

His Dissertation was titled \textit{Contributions à l'étude des problèmes elliptiques non linéaires}, \& his doctoral advisor was \href{https://en.wikipedia.org/wiki/Haim_Brezis}{Haim Brezis}.[2]

He was an \href{https://en.wikipedia.org/wiki/Leonard_Eugene_Dickson}{L.E. Dickson} Instructor in Mathematics at the \href{https://en.wikipedia.org/wiki/University_of_Chicago}{University of Chicago} from 1975--77, after which he returned to France to continue his research.

He has made many contributions in \textit{nonlinear analysis}, ranging from \textit{nonlinear elliptic equations, hamiltonian systems, spectral theory of elliptic operators}, \& with applications to the description of \textit{mathematical modelling of fluid mechanics \& combustion}.

His current research interests include the mathematical modelling of financial markets, mathematical models in biology \& especially in ecology, \& modelling in social sciences (in particular, urban planning \& criminology).

For these latter topics, he obtained an \href{http://erc.europa.eu/advanced-grants}{ERC Advanced grant} in 2012.

%
He worked at the French National Center of Scientific Research (\href{https://en.wikipedia.org/wiki/CNRS}{CNRS}), then moved to an appointment as Professor at Univ. Paris XIII (1983--1985).

He became a Professor of Mathematics in 1988 at Université Pierre et Marie Curie, Paris VI (1988--2001 of ``exceptional class'' since 1993), \& became Professor at \href{https://en.wikipedia.org/wiki/Ecole_normale_sup%C3%A9rieure}{Ecole normale supérieure}, Paris (1994--1999), \& part-time professor \href{https://en.wikipedia.org/wiki/Ecole_Polytechnique}{Ecole Polytechnique} (1987--1999).

He is also a visiting Professor in the Department of Mathematics at the University of Chicago, \& was also co-director of the Stevanovich Center of Financial Mathematics in Chicago.

He is currently the Directeur d'études (Research Professor) at \href{https://en.wikipedia.org/wiki/School_for_Advanced_Studies_in_the_Social_Sciences}{École des hautes études en sciences sociales} (\href{https://en.wikipedia.org/wiki/EHESS}{EHESS}), since 2001.

\subsubsection{Services}
\begin{itemize}
	\item National Committee of French universities (1992--1995).
	\item Since 2002 director of Centre d'analyse et mathématique sociales (CAMS), CNRS -EHESS.
	\item Vice-president, EHESS (2004--2006).
	\item Member of the thesis prize committee of the universities of Paris (since 2006).
\end{itemize}

\subsubsection{Awards}
\begin{itemize}
	\item Carrière Prize(1988)
	\item \href{https://en.wikipedia.org/wiki/Sophie_Germain#Sophie_Germain_Prize}{Prix Sophie Germain} of the \href{https://en.wikipedia.org/wiki/French_Academy_of_Sciences}{French Academy of Sciences} (2004),
	\item \href{https://en.wikipedia.org/wiki/Humboldt_Prize}{Humboldt Prize} in Germany (2004)
	\item \href{https://en.wikipedia.org/wiki/French_Legion_of_Honor}{French Legion of Honor} in 2010.
	\item \href{https://en.wikipedia.org/wiki/American_Mathematical_Society}{American Mathematical Society} Fellowship (2012).[3]
	\item Foreign honorary member of the \href{https://en.wikipedia.org/wiki/American_Academy_of_Arts_and_Sciences}{American Academy of Arts \& Sciences}, 2013.[4]
\end{itemize}

\subsubsection{Articles}
\begin{itemize}
	\item Berestycki, Henri; Roquejoffre, Jean-Michel; Rossi, Luca; The influence of a line with fast diffusion on Fisher-KPP propagation. \textit{J. Math. Biol.} 66 (2013), no. 4-5, 743--766.
	\item Barthélemy, Marc; Nadal, Jean-Pierre; Berestycki, Henri Disentangling collective trends from local dynamics. \textit{Proc. Natl. Acad. Sci. USA} 107 (2010), no. 17, 7629--7634.
	\item Berestycki, Henri; Hamel, François; Nadirashvili, Nikolai Elliptic eigenvalue problems with large drift \& applications to nonlinear propagation phenomena. \textit{Comm. Math. Phys.} 253 (2005), no. 2, 451--480.
	\item Berestycki, Henri; Hamel, François Front propagation in periodic excitable media. \textit{Comm. Pure Appl. Math.} 55 (2002), no. 8, 949--1032.
	\item Berestycki, H.; Caffarelli, L. A.; Nirenberg, L. Inequalities for second-order elliptic equations with applications to unbounded domains. I. A celebration of John F. Nash, Jr. \textit{Duke Math. J.} 81 (1996), no. 2, 467--494.
	\item Berestycki, H.; Nirenberg, L.; Varadhan, S. R. S. The principal eigenvalue \& maximum principle for 2nd-order elliptic operators in general domains. \textit{Comm. Pure Appl. Math.} 47 (1994), no. 1, 47--92.
	\item Berestycki, H.; Lions, P.-L. Nonlinear scalar field equations. I. Existence of a ground state. \textit{Arch. Rational Mech. Anal.} 82 (1983), no. 4, 313--345; II. Existence of infinitely many solutions, \textit{Arch. Rational Mech. Anal.} 82 (1983), no. 4, 347--375.
	\item Bahri, Abbas; Berestycki, Henri A perturbation method in critical point theory \& applications. \textit{Trans. Amer. Math. Soc.} 267 (1981), no. 1, 1--32.\hfill$\square$
\end{itemize}

%------------------------------------------------------------------------------%

\subsection{\href{https://en.wikipedia.org/wiki/Haim_Brezis}{Wikipedia{\tt/}Ha\"im Brezis}}
\textbf{Haïm Brezis.}
\begin{itemize}
	\item \textbf{Born.} Jun 1m 1944 (age 76). Riom-ès-Montagnes, Cantal, France.
	\item \textbf{Nationality.} French.
	\item \textbf{Alma mater.} \href{https://en.wikipedia.org/wiki/University_of_Paris}{University of Paris}.
	\item \textbf{Known for.}
	\begin{itemize}
		\item \href{https://en.wikipedia.org/wiki/Brezis-Gallouet_inequality}{Brezis-Gallouet inequality}
		\item \href{https://en.wikipedia.org/wiki/Bony-Brezis_theorem}{Bony-Brezis theorem}
		\item \href{https://en.wikipedia.org/wiki/Brezis-Lieb_lemma}{Brezis-Lieb lemma}
	\end{itemize}
\end{itemize}
\textbf{Scientific career.}
\begin{itemize}
	\item \textbf{Fields.} Mathematics.
	\item \textbf{Institutions.} \href{https://en.wikipedia.org/wiki/Pierre_and_Marie_Curie_University}{Pierre \& Marie Curie University}.
	\item \textbf{Doctoral advisor.}
	\begin{itemize}
		\item \href{https://en.wikipedia.org/wiki/Gustave_Choquet}{Gustave Choquet}
		\item \href{https://en.wikipedia.org/wiki/Jacques-Louis_Lions}{Jacques-Louis Lions}
	\end{itemize}
	\item \textbf{Doctoral students.}
	\begin{itemize}
		\item \href{https://en.wikipedia.org/wiki/Abbas_Bahri}{Abbas Bahri}
		\item \href{https://en.wikipedia.org/wiki/Jean-Michel_Coron}{Henri Berestycki}
		\item \href{https://en.wikipedia.org/wiki/Jean-Michel_Coron}{Jean-Michel Coron}
		\item \href{https://en.wikipedia.org/wiki/Jes%C3%BAs_Ildefonso_D%C3%ADaz}{Jes\'us Ildefonso D\'iaz}
		\item \href{https://en.wikipedia.org/wiki/Pierre-Louis_Lions}{Pierre-Louis Lions}
		\item \href{https://en.wikipedia.org/wiki/Juan_Luis_V%C3%A1zquez_Su%C3%A1rez}{Juan Luis V\'azquez Su\'arez}
	\end{itemize}
\end{itemize}
\textit{Haïm Brezis} (born Jun 1, 1944) is a French mathematician who works in \href{https://en.wikipedia.org/wiki/Functional_analysis}{functional analysis} \& \href{https://en.wikipedia.org/wiki/Partial_differential_equation}{partial differential equations}.

\subsubsection{Biography}
Born in \href{https://en.wikipedia.org/wiki/Riom-%C3%A8s-Montagnes}{Riom-ès-Montagnes}, \href{https://en.wikipedia.org/wiki/Cantal}{Cantal}, France.

Brezis is the son of a Romanian immigrant father, who came to France in the 1930s, \& a Jewish mother who fled from the Netherlands.

His wife, \href{https://en.wikipedia.org/wiki/Michal_Govrin}{Michal Govrin}, a native Israeli, works as a novelist, poet, \& theater director.[1]

Brezis received his Ph.D. from the \href{https://en.wikipedia.org/wiki/University_of_Paris}{University of Paris} in 1972 under the supervision of \href{https://en.wikipedia.org/wiki/Gustave_Choquet}{Gustave Choquet}.

He is currently a Professor at the \href{https://en.wikipedia.org/wiki/Pierre_and_Marie_Curie_University}{Pierre \& Marie Curie University} \& a Visiting Distinguished Professor at \href{https://en.wikipedia.org/wiki/Rutgers_University}{Rutgers University}.

He is a member of the \href{https://en.wikipedia.org/wiki/Academia_Europaea}{Academia Europaea} (1988) \& a foreign associate of the \href{https://en.wikipedia.org/wiki/United_States_National_Academy_of_Sciences}{United States National Academy of Sciences} (2003).

In 2012 he became a fellow of the \href{https://en.wikipedia.org/wiki/American_Mathematical_Society}{American Mathematical Society}.[2]

He holds honorary doctorates from several universities including \href{https://en.wikipedia.org/wiki/National_Technical_University_of_Athens}{National Technical University of Athens}.[3]

Brezis is listed as an \href{https://en.wikipedia.org/wiki/ISI_highly_cited_researcher}{ISI highly cited researcher}.[4]

\subsubsection{Works}
\begin{itemize}
	\item \textit{Opérateurs maximaux monotones et semi-groupes de contractions dans les espaces de Hilbert} (1973)
	\item \textit{Analyse Fonctionnelle}. Théorie et Applications (1983)
	\item \textit{Haïm Brezis. Un mathématicien juif}. Entretien Avec Jacques Vauthier. Collection Scientifiques \& Croyants. Editions Beauchesne, 1999. ISBN 978-2-7010-1335-0, ISBN 2-7010-1335-6
	\item \textit{Functional Analysis, Sobolev Spaces \& Partial Differential Equations}, Springer; 1st Edition. edition (November 10, 2010), ISBN 978-0-387-70913-0, ISBN 0-387-70913-4
\end{itemize}

\subsubsection{See also}
\begin{itemize}
	\item \href{https://en.wikipedia.org/wiki/Bony%E2%80%93Brezis_theorem}{Bony-Brezis theorem}
	\item \href{https://en.wikipedia.org/wiki/Brezis%E2%80%93Gallouet_inequality}{Brezis-Gallouet inequality}
	\item \href{https://en.wikipedia.org/wiki/Brezis%E2%80%93Lieb_lemma}{Brezis-Lieb lemma}\hfill$\square$
\end{itemize}

%------------------------------------------------------------------------------%

%------------------------------------------------------------------------------%

\subsection{Lawrence Chris Evans}

%------------------------------------------------------------------------------%

\subsection{\href{https://en.wikipedia.org/wiki/Herbert_Federer}{Wikipedia{\tt/}Herbert Federer}}
\textit{Herbert Federer} (Jul 23, 1920 -- Apr 21, 2010)[``NAS Membership Directory: Federer, Herbert''. National Academy of Sciences. Retrieved Jun 15, 2010.] was an American mathematician.

He is 1 of the creators of \href{https://en.wikipedia.org/wiki/Geometric_measure_theory}{geometric measure theory}, at the meeting point of \href{https://en.wikipedia.org/wiki/Differential_geometry}{differential geometry} \& \href{https://en.wikipedia.org/wiki/Mathematical_analysis}{mathematical analysis}.[Parks, H. (2012) \href{http://www.ams.org/notices/201205/rtx120500622p.pdf}{\textit{Remembering Herbert Federer (1920--2010)}}, NAMS 59(5), 622--631.]

\subsubsection{Career}
Federer was born Jul 23, 1920, in \href{https://en.wikipedia.org/wiki/Vienna}{Vienna}, \href{https://en.wikipedia.org/wiki/Austria}{Austria}.

After emigrating to the US in 1938, he studied mathematics \& physics at the \href{https://en.wikipedia.org/wiki/University_of_California,_Berkeley}{University of California, Berkeley}, earning the Ph.D. as a student of \href{https://en.wikipedia.org/wiki/Anthony_Morse}{Anthony Morse} in 1944.

He then spent virtually his entire career as a member of the \href{https://en.wikipedia.org/wiki/Brown_University}{Brown University} Mathematics Department, where he eventually retired with the title of Professor Emeritus.

%
Federer wrote more than 30 research papers in addition to his book \textit{Geometric measure theory}.

The \href{https://en.wikipedia.org/wiki/Mathematics_Genealogy_Project}{Mathematics Genealogy Project} assigns him 9 Ph.D. students \& well over a hundred subsequent descendants.

His most productive students include the late \href{https://en.wikipedia.org/wiki/Frederick_J._Almgren,_Jr.}{Frederick J. Almgren, Jr.} (1933--1997) a professor at Princeton for 35 years, \& his last student, \href{https://en.wikipedia.org/wiki/Robert_Miller_Hardt}{Robert Hardt}, now at Rice University.

%
Federer was a member of the \href{https://en.wikipedia.org/wiki/United_States_National_Academy_of_Sciences}{National Academy of Sciences}.

In 1987, he \& his Brown colleague \href{https://en.wikipedia.org/wiki/Wendell_Fleming}{Wendell Fleming} won the American Mathematical Society's \href{https://en.wikipedia.org/wiki/Steele_Prize}{Steele Prize} ``for their pioneering work in \textit{Normal \& Integral currents}.''

\subsubsection{Normal \& integral currents}
Federer's mathematical work separates thematically into the periods before \& after his watershed 1960 paper \textit{Normal \& integral currents}, co-authored with Fleming.

That paper provided the \textit{1st satisfactory general solution to \href{https://en.wikipedia.org/wiki/Plateau's_problem}{Plateau's problem}} - the problem of finding a $(k + 1)$-dimensional least-area surface spanning a given $k$-dimensional boundary cycle in $n$-dimensional Euclidean space.

Their solution inaugurated a new \& fruitful period of research on a large class of \textit{geometric variational problems} - especially \textit{minimal surfaces} - via what came to be known as \textit{Geometric Measure Theory}.

\subsubsection{Earlier work}
During the 15 or so years prior to that paper, Federer worked at the technical interface of geometry \& measure theory.

He focused particularly on surface area, rectifiability of sets, \& the extent to which one could substitute rectifiability for smoothness in the analysis of surfaces.

His 1947 paper on the \textit{rectifiable subsets of $n$-space} characterized purely unrectifiable sets by their ``invisibility'' under almost all projections.

\href{https://en.wikipedia.org/wiki/Abram_Samoilovitch_Besicovitch}{A. S. Besicovitch} had proven this for 1-dimensional sets in the plane, but Federer's generalization, valid for subsets of arbitrary dimension in any Euclidean space, was a major technical accomplishment, \& later played a key role in \textit{Normal \& Integral Currents}.

%
In 1958, Federer wrote \textit{Curvature Measures}, a paper that took some early steps toward understanding 2nd-order properties of surfaces lacking the differentiability properties typically assumed in order to discuss curvature.

He also developed \& named what he called the \href{https://en.wikipedia.org/wiki/Coarea_formula}{coarea formula} in that paper.

That formula has become a standard analytical tool.

\subsubsection{Geometric measure theory}
Federer is perhaps best known for his treatise \textit{Geometric Measure Theory}, published in 1969.[Goffman, Casper (1971). ``\href{http://www.ams.org/journals/bull/1971-77-01/S0002-9904-1971-12603-4/S0002-9904-1971-12603-4.pdf}{Review: Geometric measure theory, by Herbert Federer}'' (PDF). Bull. Amer. Math. Soc. 77 (1): 27--35. doi:10.1090/s0002-9904-1971-12603-4.]

Intended as both a text \& a reference work, the book is unusually complete, general \& authoritative: its nearly 600 pages cover a substantial amount of linear \& multilinear algebra, give a profound treatment of measure theory, integration \& differentiation, \& then move on to rectifiability, theory of currents, \& finally, variational applications.

Nevertheless, the book's unique style exhibits a rare \& artistic economy that still inspires admiration, respect - \& exasperation.

A more accessible introduction may be found in F. Morgan's book listed below.

\subsubsection{See also}
\begin{itemize}
	\item \href{https://en.wikipedia.org/wiki/Integral_current}{Integral current}
	\item \href{https://en.wikipedia.org/wiki/Federer-Morse_theorem}{Federer-Morse theorem}
\end{itemize}

\subsubsection{External links}
\begin{itemize}
	\item \href{https://web.archive.org/web/20070426204453/http://www.math.brown.edu/faculty/federer.html}{Federer's page a Brown}\hfill$\square$
\end{itemize}

%------------------------------------------------------------------------------%

%------------------------------------------------------------------------------%

\subsection{Peter Lax}

%------------------------------------------------------------------------------%

\subsection{Jacques-Louis Lions}
\begin{itemize}
	\item \textbf{Born.} May 3, 1928. \href{https://en.wikipedia.org/wiki/Grasse}{Grasse}, \href{https://en.wikipedia.org/wiki/Alpes-Maritimes}{Alpes-Maritimes}, \href{https://en.wikipedia.org/wiki/France}{France}.
	\item \textbf{Died.} May 17, 2001 (aged 73).
	\item \textbf{Nationality.} French.
	\item \textbf{Alma mater.} \href{https://en.wikipedia.org/wiki/University_of_Nancy}{University of Nancy}.
	\item \textbf{Known for.} PDEs.
	\item \textbf{Awards.} \href{https://en.wikipedia.org/wiki/Japan_Prize}{Japan Prize} (1991).
\end{itemize}
\textbf{Scientific career.}
\begin{itemize}
	\item \textbf{Fields.} Mathematics.
	\item \textbf{Institutions.}
	\begin{itemize}
		\item \href{https://en.wikipedia.org/wiki/%C3%89cole_Polytechnique}{\'Ecole Polytechnique}
		\item \href{https://en.wikipedia.org/wiki/Coll%C3%A8ge_de_France}{Coll\`ege de France}
	\end{itemize}
	\item \textbf{\href{https://en.wikipedia.org/wiki/Doctoral_advisor}{Doctoral advisor}.} \href{https://en.wikipedia.org/wiki/Laurent_Schwartz}{Laurent Schwartz}.
	\item \textbf{Doctoral students.}
	\begin{itemize}
		\item \href{https://en.wikipedia.org/wiki/Alain_Bensoussan}{Alain Bensoussan}
		\item \href{https://en.wikipedia.org/wiki/Jean-Michel_Bismut}{Jean-Michel Bismut}
		\item \href{https://en.wikipedia.org/wiki/Ha%C3%AFm_Brezis}{Ha\"im Brezis}
		\item \href{https://en.wikipedia.org/wiki/Erol_Gelenbe}{Erol Gelenbe}
		\item \href{https://en.wikipedia.org/wiki/Roland_Glowinski}{Roland Glowinski}
		\item \href{https://en.wikipedia.org/wiki/Roger_Temam}{Roger Temam}
	\end{itemize}
\end{itemize}
\textit{Jacques-Louis Lions} ([1] 3 May 1928 - May 17, 2001) was a French mathematician who made contributions to the theory of \href{https://en.wikipedia.org/wiki/Partial_differential_equation}{partial differential equations} \& to \href{https://en.wikipedia.org/wiki/Stochastic_processes}{stochastic control}, among other areas.

He received the \href{https://en.wikipedia.org/wiki/Society_for_Industrial_and_Applied_Mathematics}{SIAM}'s \href{https://en.wikipedia.org/wiki/John_von_Neumann_Lecture}{John von Neumann Lecture} prize in 1986 \& numerous other distinctions.[2][3]

Lions is listed as an \href{https://en.wikipedia.org/wiki/ISI_highly_cited_researcher}{ISI highly cited researcher}.[4] 

\subsubsection{Biography}
After being part of the French Résistance in 1943 \& 1944, J.-L. Lions entered the \href{https://en.wikipedia.org/wiki/%C3%89cole_Normale_Sup%C3%A9rieure}{École Normale Supérieure} in 1947.

He was a professor of mathematics at the Université of Nancy, the Faculty of Sciences of Paris, \& the \href{https://en.wikipedia.org/wiki/%C3%89cole_polytechnique}{École polytechnique}.

%
In 1966 he sent an invitation to \href{https://en.wikipedia.org/wiki/Gury_Marchuk}{Gury Marchuk}, the soviet mathematician to visit Paris.

This was hand delivered by \href{https://en.wikipedia.org/wiki/General_De_Gaulle}{General De Gaulle} during his visit to \href{https://en.wikipedia.org/wiki/Akademgorodok}{Akademgorodok} in June of that year.[5]

%
He joined the prestigious \href{https://en.wikipedia.org/wiki/Coll%C3%A8ge_de_France}{Collège de France} as well as the French Academy of Sciences in 1973.

In 1979, he was appointed director of the Institut National de la Recherche en Informatique et Automatique (\href{https://en.wikipedia.org/wiki/INRIA}{INRIA}), where he taught \& promoted the use of numerical simulations using finite elements integration.

Throughout his career, Lions insisted on the \textit{use of mathematics in industry}, with a particular involvement in the French space program, as well as in domains such as energy \& the environment.

This eventually led him to be appointed director of the Centre National d'Etudes Spatiales (\href{https://en.wikipedia.org/wiki/CNES}{CNES}) from 1984 to 1992.

%
Lions was elected President of the \href{https://en.wikipedia.org/wiki/International_Mathematical_Union}{International Mathematical Union} in 1991 \& also received the \href{https://en.wikipedia.org/wiki/Japan_Prize}{Japan Prize} \& the \href{https://en.wikipedia.org/wiki/Harvey_Prize}{Harvey Prize} that same year.[3]

In 1992, the \href{https://en.wikipedia.org/wiki/University_of_Houston}{University of Houston} awarded him an honorary doctoral degree.

He was elected president of the \href{https://en.wikipedia.org/wiki/French_Academy_of_Sciences}{French Academy of Sciences} in 1996 \& was also a Foreign Member of the \href{https://en.wikipedia.org/wiki/Royal_Society}{Royal Society} (ForMemRS)[6] \& numerous other foreign academies.[2][3]

%
He has left a considerable body of work, among this more than 400 scientific articles, 20 volumes of mathematics that were translated into English \& Russian, \& major contributions to several collective works, including the 4000 pages of the monumental \textit{Mathematical analysis \& numerical methods for science \& technology} (in collaboration with Robert Dautray), as well as the \textit{Handbook of numerical analysis} in 7 volumes (with \href{https://en.wikipedia.org/wiki/Philippe_G._Ciarlet}{Philippe G. Ciarlet}).

%
His son \href{https://en.wikipedia.org/wiki/Pierre-Louis_Lions}{Pierre-Louis Lions} is also a well-known mathematician who was awarded a \href{https://en.wikipedia.org/wiki/Fields_Medal}{Fields Medal} in 1994.[7]

Both father \& son have received honorary doctorates from \href{https://en.wikipedia.org/wiki/Heriot-Watt_University}{Heriot-Watt University} in 1986 \& 1995 respectively.[8]

\subsubsection{Books}
\begin{itemize}
	\item with Enrico Magenes: \textit{Problèmes aux limites non homogènes et applications}. 3 vols., 1968, 1970
	\item \textit{Contrôle optimal de systèmes gouvernés par des équations aux dérivées partielles}. 1968
	\item with L. Cesari: \textit{Quelques méthodes de résolution des problèmes aux limites non linéaires}. 1969
	\item with Roger Dautray: \textit{Mathematical analysis \& numerical methods for science \& technology}. 9 vols., 1984/5
	\item with Philippe Ciarlet: \textit{Handbook of numerical analysis}. 7 vols.
	\item with Alain Bensoussan, Papanicolaou: \textit{Asymptotic analysis of periodic structures}. North Holland 1978
	\item \textit{Controlabilité exacte, perturbations et stabilisation de systèmes distribués}[9]
	\item with John E. Lagnese: \textit{Modelling Analysis \& Control of Thin Plates}.\hfill$\square$
\end{itemize}

%------------------------------------------------------------------------------%

\subsection{Andrew Joseph Majda}
\textbf{\textsf{Resources -- Tài nguyên.}}
\begin{enumerate}
	\item \cite{memory_Andrew_Joseph_Majda}. {\sc Bjorn Engquist, Panagiotis Souganidis, Samuel N. Stechmann, Vlad Vicol}. {\it In memory of Andrew J. Majda}.
	
	``He was hard working until the end even though he suffered from serious health issues for quite some time.''
	
	``He advocated a philosophy for applied mathematics research that involves the interaction of math theory, asymptotic modeling, numerical modeling, \& observed \& experimental data $\ldots$ Andy Majda's modus operandi of modern applied mathematics, as a symbiotic relationship between (i) rigorous mathematical theory, (ii) numerical analysis \& numerical modeling, (iii) observed phenomena \& experimental data, \& (iv) qualitative and/or asymptotic modeling [Maj00].''
	
	``Andy's legacy lives on in the mathematical science he created, but also in the many students \& postdocs he so enthusiastically taught \& mentored.''
	
	``The period at UCLA was followed by 5 years at Berkeley, 1979--1984. During this productive time, he developed ``Majda's model'' for combustion in reactive flows, \& together with Tosio Kato \& Tom Beale derived ``Beale-Kato-Majda criterion,'' which characterizes a putative incompressible Euler singularity in terms of the accumulation of vorticity [BKM84].''
	
	``At Courant, Andy shifted his research efforts to cross-disciplinary research in modern applied mathematics with climate--atmosphere--ocean science.''
\end{enumerate}

%------------------------------------------------------------------------------%

\subsection{Vladimir Mazya}


%------------------------------------------------------------------------------%

\subsection{Phan Thành Nam}
``2008, sang Đan Mạch \& làm TS tại ĐH Copenhagen dưới sự hướng dẫn của Prof. {\sc Jan Philip Solovej}. Bảo vệ luận văn về ngành Vật Lý Toán năm 2011 với nhan đề {\it``Contributions to the Rigorous Study of the Structure of Atoms''}. 2011--2013: làm nghiên cứu viên sau tiến sĩ tại Đại học Cergy-Pontoise (Pháp) dưới sự hướng dẫn của Prof. {\sc Mathieu Lewin}. 2013--2016,  làm nghiên cứu viên sau tiến sĩ tại Viện khoa học \& công nghệ Áo (IST) dưới sự hướng dẫn của Prof. {\sc Robert Seiringer}. 2016: sang Cộng hòa Séc làm giáo sư trợ giảng tại Đại học Masaryk. 2017--now: GS tại Đại học Ludwig Maximilian Munich, CHLB Đức. 2020: đạt giải thưởng Hội Toán học Châu Âu EMS Prize.
\begin{quotation}
	``{\sc Phan Thành Nam} (1985--?) đã có những công trình đáng chú ý về toán học của hệ đa vật lượng tử (quantum many-body system) bao gồm hệ nguyên tử, phân tử cũng như các khí Bose \& Fermi. Kết quả của anh anh liên quan đến sự cân bằng \& tính chất động lực học của những hệ như vậy. Nhiều kết quả nổi tiếng trong lãnh vực này là do công của Nam. Chúng bao gồm những chặn tốt nhất cho ion hóa cực đại của các nguyên tử \& những hằng số nổi tiếng của của bất đẳng thức Lieb--Thirring lừng danh. Hơn nữa, Nam \& các cộng sự đã phát triển một cách tiếp cận tổng quát để thiết lập giới hạn trường trung bình (mean-field limit) của các hệ boson dựa trên định lý de Finetti lượng tử. Đó là thứ mà bây giờ trở tiêu chuẩn vàng trong lãnh vực này.'' -- Prof. {\sc Jan Philip Solovej}
\end{quotation}
Các lĩnh vực nghiên cứu của GS Phan Thành Nam là giải tích \& vật lý toán, đặc biệt là cơ học lượng tử nhiều hạt, lý thuyết phổ, phép tính biến phân \& phương trình đạo hàm riêng, giải tích số.'' -- \href{https://vi.wikipedia.org/wiki/Phan_Th%C3%A0nh_Nam}{Wikipedia{\tt/}Phan Thành Nam}

%------------------------------------------------------------------------------%

\subsection{Jind\v{r}ich Ne\v{c}as}

%------------------------------------------------------------------------------%

\subsection{Louis Nirenberg}
\textbf{\textsf{Resources -- Tài nguyên.}}
\begin{enumerate}
	\item \cite{Vazquez_remember_Nirenberg}. {\sc Juan Luis V\'{a}zquez}. {\it Remembering Louis Nirenberg \& His Mathematics}.
\end{enumerate}

\subsubsection*{Abstract}
The article is dedicated to recalling the life \& mathematics of Louis Nirenberg, a distinguished Canadian mathematician who recently died in New York, where he lived.

An emblematic figure of analysis \& PDEs in the last century, he was awarded the Abel Prize in 2015.

From this watchtower at the Courant Institute in New York, he was for many years a global teacher \& master.

He was a good friend of Spain.
\begin{quotation}\it
	1 of the wonders of mathematics is you go somewhere in the world \& you meet other mathematicians, \& it is like 1 big family.
	
	This large family is a wonderful joy.\footnote{From an interview with Louis Nirenberg appeared in \textit{Notices of the AMS}, 2002, [43]}
\end{quotation}

\subsubsection{Introduction}
This article is dedicated to remembering the life \& work of the prestigious Canadian mathematician Louis Nirenberg, born in Hamilton, Ontario, in 1925, who died in New York on Jan 26, 2020, at the age of 94.

Professor for much of his life at the mythical Courant Institute of New York University, he was considered 1 of the best mathematical analysts of the 20th century, a specialist in the analysis of PDEs.

%
When the news of his death was received, it was a very sad moment for many mathematicians, but it was also the opportunity of reviewing an exemplary life \& underlining some of its landmarks.

His work unites diverse fields between what is considered Pure Mathematics \& Applied Mathematics, \& in particular he was cult figure in the discipline of PDEs, a key theory \& tool in the mathematical formulation of many processes in science, in engineering, \& in other branches of mathematics.

\textit{His work is a prodigy of sharpness \& logical perfection, \& at the same time its applications span today multiple scientific areas.}

%
In recognition of his work, in 2015 he received the Abel Prize along with the another great mathematician, John Nash.

The Abel Prize is 1 of the greatest awards in Mathematics, comparable to the Nobel prizes in other sciences.

At that time, the Courant Institute, where he was for so many decades a renowned professor, published an article called \textit{Beautiful Minds}\footnote{\href{https://www.nyu.edu/about/news-publications/news/2015/march/beautiful-minds-courantsnirenberg-princetons-john-nash-win-abel-prize-in-mathematics-.html}{Beautiful Minds: Courant's Nirenberg, Princeton's John Nash Win Abel Prize in Mathematics }.} which is quite enjoyable reading.

%
He was a distinguished member of the AMS (American Mathematical Society).

Throughout his life, he received many other honors \& awards, e.g. the AMS B\^ocher Memorial Prize (1959), the Jeffery-Williams Prize (1987), the Steele Prize for Lifetime Achievement (1994 \& 2014), the National Medal of Science (1995), the inaugural Crafoord Prize from the Royal Swedish Academy (1982), \& the 1st Chern medal at the 2010 International Congress of Mathematicians, awarded by the International Mathematical Union \& the Chern Foundation.

He was a plenary speaker at the International Congress of Mathematicians held in Stockholm in Aug 1962; the title of the conference was \textit{``Some Aspects of Linear \& Nonlinear PDE''}.

In 1969 he was elected Member of the U.S. National Academy of Sciences.

%
It was not honors that concerned him most, but rather his profession \& the mathematical community that surrounded him.

In his long career at the Courant Institute he discovered many mathematical talents \& collaborated in numerous relevant works with distinguished colleagues.

A wise man in science \& life, he was 1 of the most influential \& beloved mathematicians of the last century, \& the current century too.

His teaching extended 1st to the international centers that he loved to visit, \& then to the entire world.

Indeed, we live at this height of time in a world-wide scientific society whose close connection brings so many benefits to the pursuit of knowledge.

Many of his articles are among the most cited in the world.\footnote{Topic 35, PDEs, from the mathematical database Mathscinet, includes 3 articles by L. Nirenberg among the 10 most cited ever.}

\subsubsection{Starting}
In order to start the tour of his mathematics, nothing better than to quote a few paragraphs from the mention of the Abel Prize Committee in 2015:\footnote{\href{https://www.abelprize.no/nyheter/vis.html?tid=63589}{John F. Nash, Jr. \& Louis Nirenberg share the Abel Prize}.}

\textsf{Fig. Louis Nirenberg receiving the Abel Prize from King Harald V of Norway in the presence of John Nash (photo: Berit Roald{\tt/}NTB scanpix).}

\textbf{Mathematical giants:}
\begin{quotation}\it
	Nash \& Nirenberg are 2 mathematical giants of the 20th century.
	
	They are being recognized for their contributions to the field of PDEs, which are equations involving rates of change that originally arose to describe physical phenomena but, as they showed, are also helpful in analyzing abstract geometrical objects.
\end{quotation}
The Abel committee writes:
\begin{quotation}\it
	``Their breakthroughs have developed into versatile \& robust techniques that have become essential tools for the study of nonlinear PDEs.
	
	Their impact can be felt in all branches of the theory.''
\end{quotation}
About Louis they say:
\begin{quotation}\it
	``Nirenberg has had 1 of the longest \& most f\^eted careers in mathematics, having produced important results right up until his 70s.
	
	Unlike Nash, who wrote papers alone, Nirenberg preferred to work in collaboration with others, with more than 90\% of his papers written jointly.
	
	Many results in the world of elliptic PDEs are named after him \& his collaborators, e.g. the Gagliardo--Nirenberg inequalities, the John--Nirenberg inequality \& the Kohn--Nirenberg theory of pseudo-differential operators.''
\end{quotation}
They conclude:
\begin{quotation}\it
	``Far from being confined to the solutions of the problems for which they were devised, the results proven by Nash \& Nirenberg have become very useful tools \& have found tremendous applications in further contexts.''
\end{quotation}
To be precise, Nirenberg made fundamental contributions to both linear \& nonlinear PDEs, functional analysis, \& their application in geometry \& complex analysis.

Among the most famous contributions we will discuss are the Gagliardo-Nirenberg interpolation inequality, which is important in solving the elliptic PDEs that arise within many areas of mathematics; the formalization of the BMO spaces of bounded mean oscillation, \& others that we will be seeing.

%
A work of utmost relevance was the work with Luis Caffarelli \& Robert Kohn aimed at solving the big open problem of existence \& smoothness of the solutions of the Navier-Stokes system of fluid mechanics.

This work was described by the AMS in 2002 as ``1 of the best ever done.''

The problem is on the Millennium Problems List (the list compiled by the Clay Foundation), \& is 1 of the most appealing open problems of mathematical physics, raised nearly 2 centuries ago.

Fermat's Last Theorem \& the Poincar\'e Conjecture have been defeated at the turn of the century, but the Navier-Stokes enigma (and in some sense its companion about the Euler's system) keep defying us.

We will deal with the issue in detail in Section 4.

\paragraph{The beginnings. From Canada to New York.} Louis Nirenberg grew up in Montr\'eal, where his fatehr was a Hebrew teacher.

After graduating\footnote{With a degree in mathematics \& physics, also in mathematics being bilingual counts.} in 1945 at McGill University, Montréal, Louis found a summer job at the National Research Council of Canada, where he met the physicist Ernest Courant, the son of Richard Courant, a famous professor at New York University.

Ernest mentioned to Nirenberg that he was going to New York to see his father \& Louis begged him for advice on a good place to apply for a master study in physics.

He returned with Richard Courant's invitation for Louis to go to New York University (NYU) for a master's degree in mathematics, after which he would be prepared for a physics program.

%
But once Louis began studying Mathematics at NYU, he never changed.

He defended his doctoral thesis under James Stoker in 1949, solving a problem in differential geometry.

The dice were cast.

We reach a crucial moment in Louis's life.

Breaking with the golden rule\footnote{which is an essential part of the American professional practice.} according to which ``a recent doctor should move to a different environment'', Richard Courant kept his best students around him, including Louis Nirenberg, \& he thus created the NYU Mathematical Institute, the famous Courant Institute, which has become a world benchmark for high mathematics, comparable only to the Princeton Institute for Advanced Study on the East Coast of the USA.

Louis was 1st a postdoc \& then a permanent member of the faculty.

There he thrived \& spent his life.

\paragraph{Equations \& Geometry.} The problem Stoker gave to Louis for his thesis, entitled \textit{``The Determination of a Closed Convex Surface Having Given Line Elements''}, is called ``the embedding problem'' or ``Weyl Problem''.

It can be stated as follows: Given a 2D sphere with a Riemannian metric s.t. the Gaussian curvature is positive everywhere, the question is whether a surface can be constructed in 3D space so that the Riemannian distance function coincides with the distance inherited from the usual Euclidean distance in the 3D space (in other words, whether there is an isometric embedding as a convex surface in $\mathbb{R}^3$).

The great German mathematician Hermann Weyl had taken a significant 1st step to solve the problem in 1916, \& Nirenberg, as a student, completed Weyl's construction.

The work to do was to solve a system of nonlinear PDEs of the so-called ``elliptic type''.

It is the kind of equation \& application that Louis Nirenberg has been working on ever since.

Progress has been slow but continued over time \& is impressive at this moment.\footnote{Isometrically embedding low dimensional manifolds into higher dimensional Euclidean spaces is the contents of a famous paper by J. Nash in 1956.}

\subsubsection{The power \& beauty of inequalities}
Focus on 1 of the most relevant topics in Louis Nirenberg's broad legacy, at the same time closest to our mathematical interest.

(Almost) every career in PDEs begin with the study of linear elliptic equations.

These form nowadays a well-established theory which combines Functional Analysis, Calculus of Variations, \& explicit representations to produce solutions in suitable functional spaces.

For the classical equilibrium equations in the mechanics of continuous media, known as Laplace's \& Poisson's equations, in symbols $\Delta u = f$, there is a classical ``maximum principle'' that provides the necessary estimates that guarantee the existence \& uniqueness of solutions.

When combined with skillful tricks of the trade, it makes possible to obtain finer estimates, e.g. regularity \& other properties.

Let us mention the estimates known under the names Harnack \& Schauder, cf. \cite{Evans2010, Gilbarg_Trudinger2001}.

In this regard, Nirenberg is quoted as saying, either jokingly or seriously,
\begin{quotation}\it
	``I made a living off the maximum principle.''\footnote{Curiously, it applies to V\'azquez too.
		
		V\'azquez's most read article deals with the ``Strong Maximum Principle'', [74].}
\end{quotation}
Many of the interesting problems that are proposed in Physics \& other sciences \& involve PDEs are \textbf{nonlinear}, e.g. the fluid equations or the curvature problems in geometry.

These nonlinear problems can seldom be solved by explicit formulas.

Because of that difficulty, the mathematical study of these problems has attracted increasing attention from the best mathematical minds of the past century, with remarkable success stories.

The usual approach goes as follows: the solution has to be obtained by some kind of approximation, \& an essential technical point is usually to show that the proposed approximation procedure (or procedures) converge to a solution\footnote{Taken in some sense acceptable to physics, e.g., the solution in the weak sense or the solution in the distributional sense.}.

A complicated topology \& functional analysis machinery has been developed over time \& is available to test such convergence, provided certain estimates are fulfilled; their role is to allow for the approximation to be controlled.

See in this sense the book that many of us have studied as young people \cite{Brezis2011}.

%
Much of the work of an ``EDP Analyst''\footnote{\textit{Analysis of PDEs} is an area of Mathematics in the US that perfectly describes our specialty which is neither pure nor applied, \& does not need to declare itself in either direction.
	
	Such a denomination is not much used in Spain \& other countries; i.e., in V\'azquez's opinion, the source of some persistent malfunctions.} consists in finding estimates that control the passage to the limit that has to be applied, or to find a convenient fixed point theorem.

A common saying in our trade goes as follows: \textit{Existence theorems come from a priori estimates \& suitable functional analysis}.

Estimate, this is the key word in the world that Louis Nirenberg \& his colleagues bequeathed us.

``Estimate'' means the same thing as ``inequality'', \& here V\'azquez refers of course to a functional or numerical inequality.

%
It may look surprising to the reader, even weird, to find it so clearly stated: Inequalities, \& not equalities (or identities), are the technical core of such a central theory of mathematics as PDEs.

However, this is precisely the mathematical revolution that was in the making when Louis was young.

Indeed, when he arrived at NYU, the most active \& renowned researcher was probably Kurt Otto Friedrichs, who decisively influenced Nirenberg's future research career.

Friedrichs loved inequalities, as Louis put it:
\begin{quotation}\it
	``Friedrichs was a great lover of inequalities \& that affected me very much.
	
	The point of view was that the inequalities are more interesting than the equalities.''
\end{quotation}
Carrying forward on that idea, Nirenberg has been unanimously recognized as a world master of inequalities''.

Here is another saying by Louis:
\begin{quotation}\it
	``I love inequalities.
	
	So if somebody shows me a new inequality, I say: ``Oh, that's beautiful, let me think about it,'' \& I may have some ideas connected to it.''
\end{quotation}
For many years, mathematicians from all over the world came to the Courant Institute to seek his advice on issues involving inequalities.

%
And there we are.

We do not reject or despise the beauty of the exact solution if there is one, but functional inequalities are our firm support in an uncertain world that is yet to be discovered \& described.

The key technical point of modern PDE theory is to establish the most needed \& appropriate estimates in the strongest possible way.

\paragraph{Sobolev, Gagliardo \& Nirenberg.} There are many types of estimates the researcher needs in the study of nonlinear PDEs, but some have turned out to be much more relevant than others.

V\'azquez will talk here about a type that has become particularly famous \& useful.

They are often collectively called ``Sobolev estimates'' in honor of the great Russian mathematician Sergei Lvovich Sobolev because of his seminal work [68], 1938.

Briefly stated, they estimate the norms of functions belonging to the Lebesgue spaces $L^p(\Omega)$, $1\le p\le\infty$, in terms of their (weak) derivatives of various orders.

In 1959 Emilio Gagliardo [35] \& Louis Nirenberg [59] gave an independent \& very simple proof of the following inequality:

\textsf{Fig. The Talenti profile for different values of the parameters.}

\begin{theorem}[Gagliardo-Nirenberg-Sobolev Inequality]
	Let $1\le p < n$. There exists a constant $C > 0$ s.t. the following inequality
	\begin{align*}
		\|u\|_{L^{p^*}(\mathbb{R}^n)}\le C\|Du\|_{L^p(\mathbb{R}^n)},\ p^*\coloneqq\frac{np}{n - p},
	\end{align*}
	holds true for all functions $u\in C_c^1(\mathbb{R}^n)$. The constant $C$ depends only on $p$ \& $n$. The exponent $p^*$ is called the \emph{Sobolev conjugate} of $p$. $Du$ denotes the gradient vector.
\end{theorem}
Gagliardo \& Nirenberg included as their starting point the important case of exponent $p = 1$, left out by Sobolev.

The inequality implies the continuous inclusion of the Banach space called $W^{1,p}(\mathbb{R}^n)$ into $L^{p^*}(\mathbb{R}^n)$ (immersion theorem).

Versions for functions defined in bounded open sets $\mathbb{R}^n$ followed naturally.

This inequality soon attracted multiple applications \& a wide array of variants \& improvements.

Very interesting versions deal with functions defined on Riemannian manifolds.

V\'azquez comments below 4 additional aspects that he finds appropriate for the curious reader.
\begin{itemize}
	\item[(i)] Thierry Aubin [3] \& Giorgio Talenti [72] obtained in 1976 the \textit{best constant} in this inequality, finding the functions that exhibit the \textit{worst behavior}\footnote{This is an apparent grammatical contradiction that gives rise to beautiful functions.}
	
	Indeed, when $1 < p < n$ the maximum quotient $\frac{\|u\|_{L^{p^*}(\mathbb{R}^n)}}{\|Du\|_{L^p(\mathbb{R}^n)}}$ is optimally realized by the function
	\begin{align*}
		U({\bf x}) = \left(a + b|{\bf x}|^{\frac{p}{p - 1}}\right)^{-\frac{n - p}{p}},
	\end{align*}
	where $a,b > 0$ are arbitrary constants.\footnote{Consider the simple case $a = b = 1$, $p = 2$ in dimension $n = 4$.
		
		The function looks a bit like Gaussian but it is not at all.}
	
	It is the famous \textit{Talenti profile}.
	
	Note that $\frac{n - p}{p} = \frac{n}{p^*}$.
	
	It happens that $U$ is a probability density (integrable) if $\frac{n - p}{p - 1} > n$, i.e., if $1 < p < p_c = \frac{2n}{n + 1}$.
	
	The $U$ profile \& its power appear recurrently in PDEs.
	
	Thus, in nonlinear diffusion we find it as a power of the Barenblatt profile in fast diffusion, see Chap. 11 of [75], \& the curiously critical exponent $p_c$ also appears, but with consequences that go in the converse direction.
	\item Gagliardo \& Nirenberg's work extends to the famous \textit{Gagliardo-Nirenberg interpolation inequality}, a result in Sobolev's theory of spaces that estimates a certain norm of a function in terms of a product of norms of functions \& derivatives thereof.
	
	We enter here a realm of higher complexity.\footnote{V\'azquez will avoid further details on these inequalities that can be found in the cited references.}
	
	See details in [10].
	\item[(iii)] In 1984 Luis Caffarelli, Bob Kohn \& Louis Nirenberg needed inequalities of the previous type in functional Lebesgue spaces but with the novelty of including so-called \textit{weights}, \& this motivated the article [18], on the famous \textit{CKN estimates originated} for spaces with power weights.
	
	This was the beginning of an extensive literature.
	
	A very striking effect arose in those studies: unlike GNS inequalities, there exists a phenomenon of symmetry breaking in the CKN inequalities, i.e., minimizers of such inequalities need not be symmetric functions, even when posed in the whole space or in balls.
	
	The exact range of parameters for the symmetry breaking was found by J. Dolbeault, M. J. Esteban \& M. Loss in [29].
	\item[(iv)] In 2004 D. Cordero-Erausquin, B. Nazareth \& C. Villani [24] used mass transport methods to obtain sharp versions of the Sobolev-Gagliardo-Nirenberg inequalities.
	
	Mass transport is 1 of the most powerful new instruments used in PDE research.
	
	This topic is related to the isoperimetric inequalities of ancient fame.\footnote{See \href{https://en.wikipedia.org/wiki/Isoperimetric-inequality}{Wikipedia{\tt/}Isoperimetric Inequality}.} that now live moments of fruitful coincidence with Sobolev theory.
	
	The survey [15] talks about this relationship.
\end{itemize}
The world of estimates that we have outlined has came to be an enormous space presided over by quite distinguished names, like H. Poincar\'e, J. Nash, G. H. Hardy, C. Morrey, J. Moser, N. Trudinger \& other remarkable figures.

Hardy-Littlewood-P\'olya's book [41] had a great influence on generations of analysts.

A commendable book on the importance of inequalities in Physics is the 2nd volume of Elliott Lieb's selected works, [53].

%
As a representative example chosen from among the numerous recent works, V\'azquez mentions the arcile by M. del Pino \& Jean Dolbeault [25].

It establishes a new optimal version of the Euclidean Gagliardo-Nirenberg inequalities.

This allows the authors to obtain the convergence rates to the equilibrium profiles of some nonlinear diffusion equations, e.g. those of the ``porous media'' type, 1 of the leitmotifs of V\'azquez's research.

The authors completed the study \& application with 2 new articles in 2003.

New functional inequalities based on entropy, maximum principles, \& symmetrization processes allowed a group of V\'azquez to find convergence rates for very fast diffusion equations in [7], thus solving in 2009 a much studied open problem.

It was almost 3 years of work by a team of 5 people.

Plus the work of previous authors.

%
Finally, there is a great deal of activity in the world of Sobolev spaces of fractional order (also called \textit{Slobodeckii spaces}), \& the associated fractional diffusions, cf. [21, 27].

It is a topic in full swing, a part of V\'azquez's current mathematical efforts.\footnote{There is a wide representation of Spanish mathematicians active in these subjects with remarkable results that would be well worth a review.}

\paragraph{New Spaces. John-Nirenberg space.} Go back for a moment to the origins.

The limiting case of the Gagliardo-Nirenberg-Sobolev inequality happens for $p = n$.

Thanks to new inequalities due to C. Morrey, we know that for $p > n$ the resulting functions are H\"older continuous functions, \cite{Evans2010}.

But the $p = n$ case was bizarre \& it was left to Fritz John \& Louis Nirenberg to solve the puzzle in 1961 by introducing the new BMO space of functions of \textit{bounded mean oscillation}, see [44].

Actually, BMO is not a function space but rather a space of function classes modulo constants.

For this space there is the appropriate inequality.

\begin{theorem}[John-Nirenberg]
	If $u\in W^{1,n}(\mathbb{R}^n)$ then $u$ belongs to $BMO$ and
	\begin{align*}
		\|u\|_{BMO}\le C\|Du\|_{L^n(\mathbb{R}^n)},
	\end{align*}
	for a constant $C > 0$ depending only on $n$.\footnote{The curious reader will wonder which function optimizes the constant. So?}
\end{theorem}
The BMO spaces are once a very popular new object in functional \& harmonic analysis, they replace $L^\infty$ when it turns out so.

They were characterized by Charles Fefferman in [32].

The BMO spaces are slightly larger than $L^\infty$.

The possible inequality (and functional immersion) of John-Nirenberg type using $L^\infty$ instead of BMO as image space may seem reasonable but it is false.\footnote{Find an elementary counterexample.}

We ought to be very careful then with the critical cases, that Louis treated with utmost attention.

The John-Nirenberg spaces are used in analysis, in PDEs, in stochastic processes, \& in multiple applications.

The reader may use the references [49] \& [8] for some updates to recent work.

\subsubsection{Navier-Stokes Equations}
The Navier-Stokes system of equations describes the dynamics of an incompressible viscous fluid.

It was proposed in the 19th century to correct Euler's equations of ideal fluids, \& adapt them to the more realistic viscous real world, [4].

The system reads \textbf{(1)}
\begin{equation*}
	\left\{\begin{split}
		\partial_t{\bf u} + ({\bf u}\cdot\nabla){\bf u} + \frac{1}{\rho}\nabla p &= \nu\Delta{\bf u} + \frac{1}{\rho}{\bf f},\\
		\nabla\cdot{\bf u} = 0,
	\end{split}\right.
\end{equation*}
where ${\bf u}$ is the \textit{velocity vector}, $p$ is the \textit{pressure}, both variable, while $\rho$ (the \textit{density}) \& $\nu$ (the \textit{viscosity}) can be taken as positive constants.

It has had a spectacular success in practical science \& engineering, but its essential mathematical aspects (existence, uniqueness, \& regularity) have offered a stubborn resistance in the physical case of 3 space dimensions (3 or $> 3$ for the mathematician).

\textsf{Nirenberg on the blackboard (photo: Courant Institute, NYU).}

%
Fundamental works to cast the theory in a modern functional framework are due to Jean Leray [50, 51], who already in 1934 speaks of weak derivatives in spaces of integrable functions.

Using the new methods of functional analysis, authors soon obtained estimates that proved to be good enough to establish the existence \& uniqueness of Leray solutions in 2 space dimensions, $n = 2$.

Furthermore, for regular initial data the solution is classical.

But the advance stopped sharply in higher dimensions, $n\ge 3$.

V\'azquez gives the word to Charles Fefferman, of Princeton University, in his description of the open problem as the Clay Foundation Millennium Problem.

It is about proving or refuting the following Conjecture:
\begin{quotation}\it
	(A) Existence \& smoothness of Navier-Stokes solutions on $\mathbb{R}^3$.
	
	Take viscosity $\nu > 0$ \& $n = 3$.
	
	Let ${\bf u}_0({\bf x})$ be any smooth, divergence-free vector field satisfying the regularity \& decay conditions (\emph{specified}).
	
	Take external force $f(t,{\bf x})$ to be identically zero.
	
	Then there exists smooth functions $p(t,{\bf x})$, $u_i(t,{\bf x})$ on $[0,\infty)\times\mathbb{R}^3$ that satisfy the Navier-Stokes system with initial conditions in the whole space.\footnote{See full details of the presentation in \url{https://www.claymath.org/sites/default/files/navierstokes.pdf}.}
\end{quotation}
The most significant advance in this field is in V\'azquez's opinion the article [17] in which L. Caffarelli, R. Kohn \& L. Nirenberg attached the problem of regularity \& showed that if a solution with classical data develops singularities in a finite time, the set of such singularities must be in any case quite small in size.

More specifically, ``the 1D measure, in the Hausdorff sense, of the set of possible singularities (located in space-time) is zero.''

This implies that if the singular set is not empty, it cannot contain any line or filament.

In 1998 F. H. Lin [54] gave an interesting new proof of this result.

%
V\'azquez is talking about 1 of the milestones of the authors' career; it happened during the stay of a young Luis Caffarelli at the Courant Institute at Louis's invitation, \& was published in 1982.

The topic Fluids is completely different from the previous sections, but the functional estimates in Sobolev spaces play an essential role, along with the machinery of geometric measure theory.

%
The possible presence of these singularities was conjectured by Leray as a possible explanation for the phenomenon of \textit{turbulence}.

According to this hypothesis, even for regular data, solutions in 3 or more dimensions can develop singularities in finite time in the form of points where the so-called \textit{vorticity} becomes infinite.

%
In the elapsed time, it has not been possible to prove or refute Conjecture (A).

Many efforts have been invested \& V\'azquez believes that will bear fruit 1 day.

An account of the state of affairs in the Euler \& NSEs around 2008 is due to P. Constantin [23].

At the present moment V\'azquez is entertained by a number of trials \& false proofs (some of them quite well published).

There are excellent general texts on Navier-Stokes, e.g. [36] \& [73].

2 very recent texts are [66] \& [67].

\subsubsection{Elliptic Equations \& the Calculus of Variations}
For reasons of selection \& space, V\'azquez will be quite brief on a subject in which Louis made so many contributions.

V\'azquez mentions 1st of all the article [11] by Haim Brezis \& Louis Nirenberg, which figures among the most widely read among the works of both authors.

It deals with the existence of solutions of semilinear elliptic equations with critical exponent (once again!)
\begin{align*}
	\Delta u + f({\bf x},u) + u^{\frac{n + 2}{n - 2}} = 0.
\end{align*}
2 further articles that had great impact are work in collaboration with Shmuel Agmon \& Avron Douglis [1], year 1959, \& [2], year 1964.

They are near-the-boundary estimates for solutions of elliptic equations that satisfy general boundary conditions.

Behavior near the boundary of nonlinear or degenerate PDE solutions, or in domains with nonsmooth boundaries, is a really delicate issue.

Indeed, it is a topic of permanent interest in our community, in theory \& also because of its practical interest\footnote{Think about the behavior of fluids in domains with corners.}.

%
The article [6]with Henri Berestycki \& S. R. S. Varadhan links the study of the 1st eigenvalue with the maximum principle, a subject that Louis enjoyed so much.

In this context V\'azquez finds the famous article on the method of the ``moving planes'' of 1991 [5] in collaboration with Henri Berestycki, which V\'azquez consider a gem.

%
In the Calculus of Variations, V\'azquez quotes the article [11] with Haim Brezis, about the difference between local minimizers in the spaces $H^1$ \& $C^1$. See also [12].

%
A topic of great interest for Louis was the study of geometric properties e.g. symmetry.

The articles [37, 38] with Vasilis Gidas \& Wei-Ming Ni deal with the radial symmetry of certain positive solutions of nonlinear elliptic equations that is imposed by the equation \& the shape of the domain.

\subsubsection{Other contributions}
V\'azquez collects here brief comments on important results obtained by Louis \& his collaborators on various topics that would deserve a more extensive treatment.

\paragraph{Operator theory.} Nirenberg \& Joseph J. Kohn\footnote{J. J. Kohn is a brilliant Princeton analyst, not to be confused with R. Kohn from Courant.
	
	J. J. Kohn speaks perfect Spanish with an Ecuadorian accent.} introduced of a \textit{pseudo-differential operator} that helped generate a huge amount of later work in the brilliant school of harmonic analysis.

In a 1965 article, [48], they dealt with pseudo-differential operators with a complete \& algebraic view.

The operators in question act on the space of tempered distributions at $\mathbb{R}^n$, \& are estimated in terms of Fourier transform norms.

The importance of these results is that they take into account all the ``lower order terms'', difficult to deal with in previous articles.

See also the volume [61] edited by Louis.

\paragraph{Free boundary problems.} This is 1 of the favorite topics of this reviewer.

In 1977 Louis published with David Kinderlehrer the article [45] on the regularity of free boundary problems for elliptic equations, at the beginning of an era that was to witness great progress.

To put it clearly, let us assume that $u$ is a solution to the problem
\begin{align*}
	\Delta u\le f,\ u\ge 0,\ (\Delta u - f)u = 0,
\end{align*}
defined in a domain $D\subset\mathbb{R}^n$.

Boundary data are also given at the fixed boundary $\partial D$.

These data are intended to determine not only $u$ but also the positivity domain $\Omega = \{{\bf x}\in D;u({\bf x}) > 0\}$, or still better the boundary of $\Omega$ that lies within $D$, called the \textit{free boundary}:
\begin{align*}
	\Gamma(u) = \partial\Omega\cap D.
\end{align*}
This is properly called an \textit{obstacle problem}.

To get a physical idea, we can imagine a membrane in space $\mathbb{R}^3$ of height $z = U(x,y)$ that is subject to boundary conditions $U = h\ge 0$ in $\partial D$ \& must lie above a stable (obstacle) of height $U_{\rm obst}(x,y) = 0$.

\textsf{Fig. Free boundaries \& obstacles (pictures: X. Ros-Oton).}

%
Often, we want to consider a nontrivial obstacle $\varphi$, usually a concave function as in the figure.

This leads to an interesting equivalent formulation.

If we put $u = U + \varphi$, we arrive at the problem
\begin{align*}
	\Delta u\le g,\ u\ge\varphi,\ (\Delta u - g)(U - \varphi) = 0,
\end{align*}
with \textit{driving term} $g = f + \Delta\varphi$, \& then we usually take $g = 0$.

In this formulation, $u$ is constrained to stay above the obstacle $u_{\rm obst}({\bf x}) = var\phi$.

%
In any case, in the ``free part'', $\{{\bf x}\in\mathbb{R}^n;U({\bf x}) > 0\} = \{{\bf x}\in\mathbb{R}^n;u({\bf x}) > \varphi\}$, an elastic equation $\Delta U = f$ is satisfied, but a priori we do not know where that part could be located.

It is therefore a problem that combines PDEs \& Geometry (again!).

%
This problem was known to have a unique \textit{solution pair}, $(u,\Gamma)$.

The attentive reader will have observed that once $\Gamma$ is known, \& with it $\Omega$, the PDE problem to find $u$ is rather elementary.

Therefore, the difficulty lies in principle in the geometry.

However, the solution to the puzzle was rather found in nonlinear analysis, [47], which also produces efficient numerical methods.

%
We then encounter a big theoretical problem: determining how regular is the set $\Gamma$, that we have found by abstract methods, \& also determining how regular is $u$ near $\Gamma$.

Even the simplest question: ``is $\Gamma$ a surface?'' has to be answered.

D. Kinderlehrer \& L. Nirenberg gave local conditions on $f$ \& assumed a certain initial regularity of $u$ to conclude that then $\Gamma$ is a very regular, even analytical, hyper-surface.

The study of free boundaries extends to problems evolving in time, e.g. the very famous Stefan problem discussed by Louis in [46].

The 1980s were years of great progress in the mathematical understanding of free boundaries, with reference books e.g. [28, 34].

%
This is a field of very intense activity, both theoretical \& applied, in which V\'azquez has worked with great delight for decades.

A required reference for in-depth study of the regularity of the free boundaries is the book [20] by L. Caffarelli \& S. Salsa, see also A. Petrosyan et al. [64].

A study of tumor growth modeling, seen as a free boundary problem, was done by B. Perthame et al. in [63], it is just an example from a vast literature.

\paragraph{Geometric Equations.} The article [55] with Charles Loewner in 1974 deals with PDEs that are invariant under conformal or projective transformations.

The reader will recall in this context the current relevance of PDEs linked to problems of Riemannian geometry, e.g. the Yamabe problem.

V\'azquez refers to the lengthy overview [52] due to Yan Yan Li, Louis's doctoral student that has been for many years professor at Rutgers.

\paragraph{Complex geometry.} The topic interested Louis a lot in his beginnings.

A Newlander \& L. Nirenberg wrote in 1965 an article published in Annals of Mathematics [56] on analytical coordinates in quasi-complex manifolds.

The Newlander-Nirenberg Theorem states that any integrable quasi-complex structure is induced by a complex structure.

Integrability is expressed through a list of differential conditions.

%
V\'azquez puts an end here to the mathematical journey, unfortunately unfair in many aspects due to the brevity of space \& his ignorance in so many subjects.

V\'azquez hopes that the extensive cited literature will serve as an indication of the profound influence of Louis Nirenberg \& his world on the mathematicians \& mathematics that have followed him.

For the curious reader, there are excellent articles dealing with the work \& life of Louis Nirenberg: a congress in his honor on the occasion of the 75th anniversary was organized by Alice Chang et al. \& is collected in [22].

He was interviewed by Allyn Jackson for the \textit{AMS Notices} in 2002, [43], \& Simon Donaldson, Fields Medal, wrote about him in the same journal in 2011, [30].

Yan-Yan Li's [52] 2010 article focuses on the analysis of geometric problems.

On the occasion of the Abel Prize, Xavier Cabré wrote a review in Catalan in [14] \& Tristan Rivi\`ere reviews his work in PDEs in [65].

A mathematical description of the influence of his ideas appeared in 2016 in [69] with contributions of a number of experts: X. Cabr\'e (symmetries of solutions), A. Chang (Gauss curvature problem), G. Seregin (Navier-Stokes problem), E. Carlen \& A. Figalli (stability of the GNS inequality), M. T. Wang \& S. T. Yau (Weyl problem \& general relativity).

Finally, the book [42] presents the laureates of the Abel Prize in the period 2013--2017.

In it Robert V. Kohn devotes to L. Nirenberg the article \textit{``A few of Louis Nirenberg's many contributions to the theory of PDEs''}.

By the way, there is a beautiful quotation from Abel as motto for the book:
\begin{quotation}
	``Au reste il me para\^{\i}t que si l'on veut faire des progr\`es dans les math\'ematiques il faut \'etudier les maîtres et non pas les \'ecoliers.''\footnote{In English: \textit{``Finally, it appears to me that if one wants to make progress in mathematics, one should study the masters, not the students.''} Taken from the book.}
\end{quotation}
\textbf{Update.} the article \textit{``A personal tribute to Louis Nirenberg''}, posted by Prof. Joel Spruck in the Arxiv repository in May 2021, [70].

As a person who met Louis Nirenberg in 1972 \& became a Courant Instructor, his detailed report on a selection of Louis's works is a very commendable reading.

He concentrates on the work inspired by geometric problems beginning around 1974, especially the method of moving planes, \& implicit fully nonlinear elliptic equations, \& makes comments on Louis' personality.

\textsf{Fig. Nirenberg in Barcelona in 2017 (photo: Jordi Play).}

\subsubsection{The quiet wise man \& Spain}
V\'azquez's 1st memory of Louis Nirenberg sets them in Lisbon in the spring of 1982.\footnote{At the International Symposium in Homage to Prof. J. Sebati\~ao e Silva.}

Louis was already famous \& V\'azquez was a novice in the art.

In Lisbon V\'azquez listened to 1 of his talks, which brought together the depth of the mathematics, the simplicity of the exposition \& a grace to add some comment as timely as it was nice, characteristic features of Louis that delighted the public.

%
In the fall of that same year V\'azquez set foot in the US, headed for the University of Minnesota,\footnote{This American university was very popular with young Spanish graduates \& doctors for the excellence of its studies in Mathematics \& Economics.} to work on free boundary problems with Don Aronson \& with Luis Caffarelli, who was back from his visit to Courant Institute.

Then V\'azquez saw, through the group of great professor V\'azquez had access to, that mathematical research offered a much better way of life.

Among that group of friends V\'azquez counts Haim Brezis \& Luis Caffarelli who have been V\'azquez's masters, Louis Nirenberg, Constantine Dafermos, Donald Aronson, Mike Crandall, Hans Weinberger,$\ldots$ V\'azquez will never cease from thanking them for that vision.

%
A few years later, V\'azquez had the honor of participating in the organization of a summer course at the UIMP\footnote{Men\'endez Pelayo International University, the course took place in 1987 at the Palacio de la Magdalena in Santander.} which included Louis as lecturer along with Don G. Aronson (Minnesota), Philippe Bénilan (Besançon), Luis A. Caffarelli (IAS Princeton) \& Constantine Dafermos (Brown Univ.).

These courses were inspired by Luis Caffarelli, close collaborator \& friend of Louis, with the support of the Rector of the UIMP, Prof. Ernest Lluch,\footnote{Scholar of indelible memory, great protector of science \& great conversationalist, he died tragically for being a good person at a very turbulent time.} \& somehow they transmitted a certain spirit of mathematics that was being done around the Courant Institute.

The course had a remarkable consequence.

A young mathematician from Barcelona, Xavier Cabr\'e, a student in the course, went to the Courant Institute with Louis Nirenberg \& thus began an international mathematical career, like the ones that so many young people crave today.

His thesis, directed by Louis, dealt with ``Estimates for Solutions of Elliptic \& Parabolic Equations'' (NYU, 1994).

Following his stay in New York, he published with Luis Caffarelli the beautiful book [16] on the so-called \textit{completely nonlinear elliptic equations}.

Xavier Cabr\'e is now an ICREA Professor at the UPC in Barcelona.

Louis Nirenberg visited Spain several times, specially Barcelona, \& had many Spanish friends \& admirers.

%
Although V\'azquez did not become a collaborator of Louis, V\'azquez had the opportunity of seeing him \& talking to him on several occasions.

V\'azquez highlights a stay at the Courant Institute in the winter of 1996 where V\'azquez could appreciate the day-to-day life of the ``quiet wise man'', or a congress in Argentina in 2009 when Louis was already very senior but loved life as the 1st day.

The last event in which V\'azquez saw him took place at Columbia University, New York, in May of last year (2019), in a congress in honor of Luis Caffarelli.

He went to some talks in his wheelchair at 94 years old, and, with his proverbial good humor he told them that it was a bit difficult for him to follow the lectures!

%
Impressed by his personality, the young mathematician David Fern\'andez \& V\'azquez wrote a portrait of him in 2 entries in the blog \textit{``The Republic of Mathematics''} that they edit in ``Investigaci\'on y Ciencia'' (Spanish partner of ``Sciencific American'').

They called the essays ``Louis Nirenberg, the quiet wise man'' (I) \& (II).\footnote{\url{https://www.investigacionyciencia.es/blogs/matematicas/75/posts}.}

He was a teacher \& master of science as those described by George Steiner in [71], where the relationship between teacher \& pupil, master \& disciple, is what matters.

Louis had 46 doctoral students, many of them well-known mathematicians.\footnote{The 1st was Walter Littman (in 1956), whom V\'azquez treated so much in Minnesota.}

It was not his style to write long textbooks, he was the author of [60] \& the recently published [62].

%
We will miss the teacher, master \& senior friend who always looked gentle \& kind, who loved Italy (\textit{il bel paese}), culture, good food \& talking about movies \& friends, \& with whom mathematics was easy \& exciting.

Nirenberg lived in New York since 1949, in the Upper West Side, he was a perfect New Yorker \& at the same time a citizen of the wide world.

He worked until the end of his life, frequently visiting ``his'' Institute.

Lucky soul, how V\'azquez envies him, now \& here the ``elders'' seem expendable for public utility.

%
V\'azquez is proud to bear his name Louis $=$ Luis, like Luis Caffarelli or Jacques Louis Lions or Luigi Ambrosio.

He is already a great name in mathematics \& it is an honor that carries the burden of working as Louis Nirenberg, only for the best \& always in a good mood, \& that is not easy.

Rest in eternal peace, beloved Master.

In the Elysian fields you will have time to think about new functional inequalities, the beautiful functions that optimize them, \& their surprising fruits.

In our own small way, we also follow them, as in [26].\hfill$\square$

%------------------------------------------------------------------------------%

%------------------------------------------------------------------------------%

\subsection{Stanley Osher}

%------------------------------------------------------------------------------%

%------------------------------------------------------------------------------%

\subsection{Laurent Schwartz}
\textbf{Laurent Schwartz.}
\begin{itemize}
	\item \textbf{Born.} Mar 5, 1915. \href{https://en.wikipedia.org/wiki/Paris}{Paris}, France.
	\item \textbf{Died.} Jul 4, 2002 (aged 87). Paris, France.
	\item \textbf{Nationality.} French.
	\item \textbf{Alma mater.} \href{https://en.wikipedia.org/wiki/%C3%89cole_Normale_Sup%C3%A9rieure}{Ecole Normale Sup\'erieure}.
	\item \textbf{Known for.}
	\begin{itemize}
		\item \href{https://en.wikipedia.org/wiki/Distribution_(mathematics)}{Theory of Distributions}
		\item \href{https://en.wikipedia.org/wiki/Schwartz_kernel_theorem}{Schwartz kernel theorem}
		\item \href{https://en.wikipedia.org/wiki/Schwartz_space}{Schwartz space}
		\item \href{https://en.wikipedia.org/wiki/Schwartz-Bruhat_function}{Schwartz-Bruhat function}
		\item \href{https://en.wikipedia.org/wiki/Radonifying_function}{Radonifying operator}
		\item \href{https://en.wikipedia.org/wiki/Cylinder_set_measure}{Cylinder set measure}
	\end{itemize}
	\item \textbf{Awards.} \href{https://en.wikipedia.org/wiki/Fields_Medal}{Fields Medal} (1950).
\end{itemize}
\textbf{Scientific career.}
\begin{itemize}
	\item \textbf{Fields.} Mathematics.
	\item \textbf{Institutions.}
	\begin{itemize}
		\item \href{https://en.wikipedia.org/wiki/University_of_Strasbourg}{University of Strashbourg}
		\item \href{https://en.wikipedia.org/wiki/University_of_Nancy}{University of Nancy}
		\item \href{https://en.wikipedia.org/wiki/University_of_Grenoble}{University of Grenoble}
		\item \href{https://en.wikipedia.org/wiki/%C3%89cole_Polytechnique}{\'Ecole Polytechnique}
		\item \href{https://en.wikipedia.org/wiki/Universit%C3%A9_de_Paris_VII}{Universit\'e de Paris VII}
	\end{itemize}
	\item \textbf{Doctoral advisor.} \href{https://en.wikipedia.org/wiki/Georges_Valiron}{Georges Valiron}.
	\item \textbf{Doctoral students.}
	\begin{itemize}
		\item \href{https://en.wikipedia.org/wiki/Maurice_Audin}{Maurice Audin}
		\item \href{https://en.wikipedia.org/wiki/Georges_Glaeser}{Georges Glaeser}
		\item \href{https://en.wikipedia.org/wiki/Alexander_Grothendieck}{Alexander Grothendieck}
		\item \href{https://en.wikipedia.org/wiki/Jacques-Louis_Lions}{Jacques-Louis Lions}
		\item \href{https://en.wikipedia.org/wiki/Bernard_Malgrange}{Bernard Malgrange}
		\item \href{https://en.wikipedia.org/wiki/Andr%C3%A9_Martineau}{Andr\'e Martineau}
		\item \href{https://en.wikipedia.org/wiki/Bernard_Maurey}{Bernard Maurey}
		\item \href{https://en.wikipedia.org/wiki/Leopoldo_Nachbin}{Leopoldo Nachbin}
		\item \href{https://en.wikipedia.org/wiki/Henri_Hogbe_Nlend}{Henri Hogbe Nlend}
		\item \href{https://en.wikipedia.org/wiki/Gilles_Pisier}{Gilles Pisier}
		\item \href{https://en.wikipedia.org/wiki/Fran%C3%A7ois_Treves}{Fran\c{c}ois Treves}
	\end{itemize}
	\item \textbf{Influenced.} \href{https://en.wikipedia.org/wiki/Per_Enflo}{Per Enflo}.
\end{itemize}
\textit{Laurent-Moïse Schwartz} (Mar 5, 1915 - Jul 4, 2002) was a French mathematician.

He pioneered the \href{https://en.wikipedia.org/wiki/Theory}{theory} of \href{https://en.wikipedia.org/wiki/Distribution_(mathematics)}{distributions}, which gives a well-defined meaning to objects such as the \href{https://en.wikipedia.org/wiki/Dirac_delta_function}{Dirac delta function}.

He was awarded the \href{https://en.wikipedia.org/wiki/Fields_Medal}{Fields Medal} in 1950 for his work on the \href{https://en.wikipedia.org/wiki/Distribution_(mathematics)}{theory of distributions}.

For several years he taught at the \href{https://en.wikipedia.org/wiki/%C3%89cole_polytechnique}{École polytechnique}.

\subsubsection{Biography}

\paragraph{Family}
Laurent Schwartz came from a Jewish family of \href{https://en.wikipedia.org/wiki/Alsace}{Alsatian} origin, with a strong scientific background: his father was a well-known \href{https://en.wikipedia.org/wiki/Surgeon}{surgeon}, his uncle \href{https://en.wikipedia.org/wiki/Robert_Debr%C3%A9}{Robert Debré} (who contributed to the creation of \href{https://en.wikipedia.org/wiki/UNICEF}{UNICEF}) was a famous \href{https://en.wikipedia.org/wiki/Pediatrics}{pediatrician}, \& his great-uncle-in-law, \href{https://en.wikipedia.org/wiki/Jacques_Hadamard}{Jacques Hadamard}, was a famous mathematician.

%
During his training at \href{https://en.wikipedia.org/wiki/Lyc%C3%A9e_Louis-le-Grand}{Lycée Louis-le-Grand} to enter the \href{https://en.wikipedia.org/wiki/%C3%89cole_Normale_Sup%C3%A9rieure}{École Normale Supérieure}, he fell in love with \href{https://en.wikipedia.org/wiki/Marie-H%C3%A9l%C3%A8ne_Schwartz}{Marie-Hélène Lévy}, daughter of the probabilist \href{https://en.wikipedia.org/wiki/Paul_L%C3%A9vy_(mathematician)}{Paul Lévy} who was then teaching at the \href{https://en.wikipedia.org/wiki/%C3%89cole_polytechnique}{École polytechnique}.

Later they would have 2 children, Marc-André \& Claudine.

Marie-Hélène was gifted in mathematics as well, as she contributed to the geometry of singular analytic spaces \& taught at the \href{https://en.wikipedia.org/wiki/Universit%C3%A9_Lille_Nord_de_France}{University of Lille}.

%
Angelo Guerraggio describes ``Mathematics, politics \& butterflies'' as his ``3 great loves''.[1]

\paragraph{Education}
According to his teachers, Schwartz was an exceptional student.

He was particularly gifted in Latin, Greek \& mathematics.

1 of his teachers told his parents: ``\textit{Beware, some will say your son has a gift for languages, but he is only interested in the scientific \& mathematical aspect of languages: he should become a mathematician}.''

%
In 1934, he was admitted at the École Normale Supérieure, \& in 1937 he obtained the \href{https://en.wikipedia.org/wiki/Agr%C3%A9gation}{agrégation} (with rank 2).

\paragraph{World War II}
As a man of \href{https://en.wikipedia.org/wiki/Trotskyism}{Trotskyist} affinities \& \href{https://en.wikipedia.org/wiki/Jew}{Jewish} descent, life was difficult for Schwartz during \href{https://en.wikipedia.org/wiki/World_War_II}{World War II}.

He had to hide \& change his identity to avoid being \href{https://en.wikipedia.org/wiki/Deportation}{deported} after Nazi Germany overran France.

He worked for the \href{https://en.wikipedia.org/wiki/University_of_Strasbourg}{University of Strasbourg} (which had been relocated in \href{https://en.wikipedia.org/wiki/Clermont-Ferrand}{Clermont-Ferrand} because of the war) under the name of Laurent-Marie Sélimartin, while Marie-Hélène used the name Lengé instead of Lévy.

Unlike other mathematicians at Clermont-Ferrand such as \href{https://en.wikipedia.org/wiki/Jacques_Feldbau}{Feldbau}, the couple managed to escape the Nazis.

\paragraph{Later career}
Schwartz taught mainly at \href{https://en.wikipedia.org/wiki/%C3%89cole_Polytechnique}{École Polytechnique}, from 1958 to 1980.

At the end of the war, he spent one year in \href{https://en.wikipedia.org/wiki/Grenoble}{Grenoble} (1944), then in 1945 joined the University of \href{https://en.wikipedia.org/wiki/Nancy,_France}{Nancy} on the advice of \href{https://en.wikipedia.org/wiki/Jean_Delsarte}{Jean Delsarte} \& \href{https://en.wikipedia.org/wiki/Jean_Dieudonn%C3%A9}{Jean Dieudonné}, where he spent 7 years.

He was both an influential researcher \& teacher, with students such as \href{https://en.wikipedia.org/wiki/Bernard_Malgrange}{Bernard Malgrange}, \href{https://en.wikipedia.org/wiki/Jacques-Louis_Lions}{Jacques-Louis Lions}, \href{https://en.wikipedia.org/wiki/Fran%C3%A7ois_Bruhat}{François Bruhat} \& \href{https://en.wikipedia.org/wiki/Alexander_Grothendieck}{Alexander Grothendieck}.

He joined the science faculty of the \href{https://en.wikipedia.org/wiki/University_of_Paris}{University of Paris} in 1952.

In 1958 he became a teacher at the \href{https://en.wikipedia.org/wiki/%C3%89cole_polytechnique}{École polytechnique} after having at 1st refused this position.

From 1961 to 1963 the École polytechnique suspended his right to teach, because of his having signed the \href{https://en.wikipedia.org/wiki/Manifesto_of_the_121}{Manifesto of the 121} about the \href{https://en.wikipedia.org/wiki/Algerian_war}{Algerian war}, a gesture not appreciated by Polytechnique's military administration.

However, Schwartz had a lasting influence on mathematics at the École polytechnique, having reorganized both teaching \& research there.

In 1965 he established the \href{https://en.wikipedia.org/wiki/Centre_de_math%C3%A9matiques_Laurent-Schwartz}{Centre de mathématiques Laurent-Schwartz} (CMLS) as its 1st director.

%
In 1973 he was elected corresponding member of the \href{https://en.wikipedia.org/wiki/French_Academy_of_Sciences}{French Academy of Sciences}, \& was promoted to full membership in 1975.

\subsubsection{Mathematical legacy}
In 1950 at the \href{https://en.wikipedia.org/wiki/International_Congress_of_Mathematicians}{International Congress of Mathematicians}, Schwartz was a plenary speaker[Schwartz, Laurent (1950). ``\textit{Théorie des noyaux}'' (PDF). In: Proceedings of the International Congress of Mathematicians, Cambridge, Massachusetts, U.S.A., Aug 30--Sep 6, 1950. vol. 1. pp. 220--230.] \& was awarded the \href{https://en.wikipedia.org/wiki/Fields_Medal}{Fields Medal} for his work on \href{https://en.wikipedia.org/wiki/Distribution_(mathematics)}{distributions}.

He was the 1st French mathematician to receive the Fields medal.

Because of his sympathy for \href{https://en.wikipedia.org/wiki/Trotskyism}{Trotskyism}, Schwartz encountered serious problems trying to enter the United States to receive the medal; however, he was ultimately successful.

%
The theory of distributions clarified the (then) mysteries of the \href{https://en.wikipedia.org/wiki/Dirac_delta_function}{Dirac delta function} \& \href{https://en.wikipedia.org/wiki/Heaviside_step_function}{Heaviside step function}.

It helps to extend the theory of \href{https://en.wikipedia.org/wiki/Fourier_transform}{Fourier transforms} \& is now of critical importance to the theory of \href{https://en.wikipedia.org/wiki/Partial_differential_equation}{partial differential equations}.

\subsubsection{Popular science}
Throughout his life, Schwartz actively worked to promote science \& bring it closer to the general audience.

Schwartz said:
\begin{quotation}\it
	``What are mathematics helpful for? Mathematics are helpful for physics.
	
	Physics helps us make fridges.
	
	Fridges are made to contain spiny lobsters, \& spiny lobsters help mathematicians who eat them \& have hence better abilities to do mathematics, which are helpful for physics, which helps us make fridges which$\ldots$''[3]
\end{quotation}

\subsubsection{Entomology}
\textsf{\href{https://en.wikipedia.org/wiki/Clanis_schwartzi}{Clanis schwartzi} Paratype \href{https://en.wikipedia.org/wiki/MHNT}{MHNT}.}

His mother, who was passionate about natural science, passed on her taste for \href{https://en.wikipedia.org/wiki/Entomology}{entomology} to Laurent.

His personal collection of 20,000 \href{https://en.wikipedia.org/wiki/Lepidoptera}{Lepidoptera} specimens, collected during his various travels was bequeathed to the \href{https://en.wikipedia.org/wiki/National_Museum_of_Natural_History_(France)}{Muséum national d'histoire naturelle}), the \href{https://en.wikipedia.org/wiki/Mus%C3%A9e_des_Confluences}{Science Museum of Lyon}, the \href{https://en.wikipedia.org/wiki/Mus%C3%A9um_de_Toulouse}{Museum of Toulouse} \& the Museo de Historia Natural Alcide d'Orbigny in \href{https://en.wikipedia.org/wiki/Cochabamba}{Cochabamba} (Bolivia).

Several species discovered by Schwartz bear his name.

\subsubsection{Personal ideology}
Apart from his scientific work, Schwartz was a well-known outspoken \href{https://en.wikipedia.org/wiki/Intellectual}{intellectual}.

As a young socialist influenced by \href{https://en.wikipedia.org/wiki/Leon_Trotsky}{Leon Trotsky}, Schwartz opposed the \href{https://en.wikipedia.org/wiki/Totalitarianism}{}{totalitarianism} of the \href{https://en.wikipedia.org/wiki/Soviet_Union}{Soviet Union}, particularly under \href{https://en.wikipedia.org/wiki/Joseph_Stalin}{Joseph Stalin}.

Schwartz ultimately rejected \href{https://en.wikipedia.org/wiki/Trotskyism}{Trotskyism} for \href{https://en.wikipedia.org/wiki/Democratic_socialism}{democratic socialism}.

%
On his religious views, Schwartz called himself an atheist.[4]

\subsubsection{Books}

\paragraph{Research articles}
\begin{itemize}
	\item \textit{Œuvres scientifiques. I}.
	
	With a general introduction to the works of Schwartz by Claude Viterbo \& an appreciation of Schwartz by Bernard Malgrange.
	
	With 1 DVD.
	
	Documents Mathématiques (Paris), 9. Société Mathématique de France, Paris, 2011. x+523 pp. ISBN 978-2-85629-317-1
	\begin{quotation}
		the 1st half of his works in analysis \& partial differential equations.
		
		After a preface by Claude Viterbo, which includes a few photos, one will find a note by Schwartz himself about his works, followed by a few original documents (letters, course notes), a presentation by Bernard Malgrange of the theory of distributions for which Schwartz received the Fields Medal in 1950, \& a selection of articles covering the period 1944--1954.
	\end{quotation}
	\item \textit{Œuvres scientifiques. II}.
	
	With an appreciation of Schwartz by Alain Guichardet.
	
	With 1 DVD.
	
	Documents Mathématiques (Paris), 10.
	
	Société Mathématique de France, Paris, 2011. x+507 pp. ISBN 978-2-85629-318-8
	\begin{quotation}
		the 2nd half of his works in analysis \& partial differential equations.
		
		After a note by Alain Guichardet on Schwartz \& his seminars, one will find a selection of articles covering the period 1954--1966.
	\end{quotation}
	\item \textit{Œuvres scientifiques. III}.
	
	With appreciations of Schwartz by Gilles Godefroy \& Michel Émery.
	
	With 1 DVD.
	
	Documents Mathématiques (Paris), 11. Société Mathématique de France, Paris, 2011. x+619 pp. ISBN 978-2-85629-319-5
	\begin{quotation}
		his works on Banach space theory (1968--1987), introduced by Gilles Godefroy, \& on probability theory (1970--1996), presented by Michel Émery, as well as some articles of a historical nature (1955--1994).
	\end{quotation}
\end{itemize}

\paragraph{Technical books}
\begin{itemize}
	\item \textit{Analyse hilbertienne}. Collection Méthodes. Hermann, Paris, 1979. ii+297 pp. ISBN 2-7056-5897-1
	\item \textit{Application of distributions to the theory of elementary particles in quantum mechanics}. Gordon \& Breach, New York, NY, 1968. 144pp. ISBN 9780677300900
	\item \textit{Cours d'analyse. 1}. 2nd edition. Hermann, Paris, 1981. xxix+830 pp. ISBN 2-7056-5764-9
	\item \textit{Cours d'analyse. 2}. 2nd edition. Hermann, Paris, 1981. xxiii+475+21+75 pp. ISBN 2-7056-5765-7
	\item [5] \textit{Étude des sommes d'exponentielles. 2ième éd}. Publications de l'Institut de Mathématique de l'Université de Strasbourg, V. Actualités Sci. Ind., Hermann, Paris 1959 151 pp.
	\item \textit{Geometry \& probability in Banach spaces}. Based on notes taken by Paul R. Chernoff. Lecture Notes in Mathematics, 852. Springer-Verlag, Berlin-New York, 1981. x+101 pp. ISBN 3-540-10691-X
	\item \textit{Lectures on complex analytic manifolds}. With notes by M. S. Narasimhan. Reprint of the 1955 edition. Tata Institute of Fundamental Research Lectures on Mathematics \& Physics, 4. Published for the Tata Institute of Fundamental Research, Bombay; by Springer-Verlag, Berlin, 1986. iv+182 pp. ISBN 3-540-12877-8
	\item \textit{Mathematics for the physical sciences}. Hermann, Paris; Addison-Wesley Publishing Co., Reading, Mass.-London-Don Mills, Ont. 1966 358 pp.
	\item \textit{Radon measures on arbitrary topological spaces \& cylindrical measures}. Tata Institute of Fundamental Research Studies in Mathematics, No. 6. Published for the Tata Institute of Fundamental Research, Bombay by Oxford University Press, London, 1973. xii+393 pp.
	\item \textit{Semimartingales \& their stochastic calculus on manifolds}. Edited \& with a preface by Ian Iscoe. Collection de la Chaire Aisenstadt. Presses de l'Université de Montréal, Montreal, QC, 1984. 187 pp. ISBN 2-7606-0660-0
	\item \textit{Semi-martingales sur des variétés, et martingales conformes sur des variétés analytiques complexes}. Lecture Notes in Mathematics, 780. Springer, Berlin, 1980. xv+132 pp. ISBN 3-540-09749-X
	\item Les tenseurs. \textit{Suivi de ``Torseurs sur un espace affine'' by Y. Bamberger \& J.-P. Bourguignon}. 2nd edition. Hermann, Paris, 1981. i+203 pp. ISBN 2-7056-1376-5
	\item [6] \textit{Théorie des distributions}. Publications de l'Institut de Mathématique de l'Université de Strasbourg, No. IX-X. Nouvelle édition, entiérement corrigée, refondue et augmentée. Hermann, Paris 1966 xiii+420 pp.
\end{itemize}

\paragraph{Seminar notes}
\begin{itemize}
	\item \textit{Séminaire Schwartz in Paris 1953 bis 1961}. Online edition: [1]
\end{itemize}

\paragraph{Popular books}
\begin{itemize}
	\item \textit{Pour sauver l'université.} Editions du Seuil, 1983. 122 pp. ISBN 2020065878
	\item \textit{A mathematician grappling with his century}. Translated from the 1997 French original by Leila Schneps. Birkhäuser Verlag, Basel, 2001. viii+490 pp. ISBN 3-7643-6052-6
\end{itemize}

\subsubsection{See also}
\begin{itemize}
	\item \href{https://en.wikipedia.org/wiki/Schwartz_distribution}{Schwartz distribution}
	\item \href{https://en.wikipedia.org/wiki/Schwartz_kernel_theorem}{Schwartz kernel theorem}
	\item \href{https://en.wikipedia.org/wiki/Schwartz_space}{Schwartz space}
	\item \href{https://en.wikipedia.org/wiki/Schwartz-Bruhat_function}{Schwartz-Bruhat function}
	\item \href{https://en.wikipedia.org/wiki/Nicolas_Bourbaki}{Nicolas Bourbaki}'' -- \href{https://en.wikipedia.org/wiki/Laurent_Schwartz}{Wikipedia{\tt/}Laurent Schwartz}
\end{itemize}

%------------------------------------------------------------------------------%

\subsection{Roger Temam}
\textbf{Roger Meyer Temam.}
\begin{itemize}
	\item \textbf{Born.} May 19, 1940 (age 80).
	\item \textbf{Nationality.} French.
	\item \textbf{Alma mater.} \href{https://en.wikipedia.org/wiki/University_of_Paris}{University of Paris}.
	\item \textbf{Known for.} \href{https://en.wikipedia.org/wiki/Navier-Stokes_equations}{Navier-Stokes equations}.
\end{itemize}
\textbf{Scientific career.}
\begin{itemize}
	\item \textbf{Fields.} \href{https://en.wikipedia.org/wiki/Applied_mathematics}{Applied mathematics.}
	\item \textbf{Institutions.}
	\begin{itemize}
		\item \href{https://en.wikipedia.org/wiki/Paris-Sud_University}{Paris-Sud University} (\href{https://en.wikipedia.org/wiki/Orsay}{Orsay})
		\item \href{https://en.wikipedia.org/wiki/Indiana_University_Bloomington}{Indiana University}
	\end{itemize}
	\item \textbf{Doctoral advisor.} \href{https://en.wikipedia.org/wiki/Jacques-Louis_Lions}{Jacques-Louis Lions}.
	\item \textbf{Doctoral students.}
	\begin{itemize}
		\item \href{https://en.wikipedia.org/wiki/Etienne_Pardoux}{Etienne Pardoux}
		\item \href{https://en.wikipedia.org/wiki/Denis_Serre}{Denis Serre}
	\end{itemize}
\end{itemize}
\textit{Roger Meyer Temam} (born May 19, 1940) is a French applied mathematician working in \href{https://en.wikipedia.org/wiki/Numerical_analysis}{numerical analysis}, \href{https://en.wikipedia.org/wiki/Nonlinear_partial_differential_equations}{nonlinear partial differential equations} \& \href{https://en.wikipedia.org/wiki/Fluid_mechanics}{fluid mechanics}.

He graduated from the \href{https://en.wikipedia.org/wiki/University_of_Paris}{University of Paris} - the \href{https://en.wikipedia.org/wiki/Sorbonne}{Sorbonne} in 1967, completing a doctorate (\textit{thèse d'Etat}) under the direction of \href{https://en.wikipedia.org/wiki/Jacques-Louis_Lions}{Jacques-Louis Lions}.

He has published over 400 articles, as well as 12 (authored or co-authored) books.

\subsubsection{Scientific work}
The 1st work of Temam in his thesis dealt with the \textit{fractional steps method}.

Thereafter, ``he has continually explored \& developed new directions \& techniques'':[2]
\begin{itemize}
	\item \href{https://en.wikipedia.org/wiki/Calculus_of_variations}{calculus of variations}, \& the notion of \textit{duality} (book \#7), developing the mathematical framework for \textit{discontinuous} (in displacement) \textit{solutions}; a concept later used for his works on the \textit{mathematical theory of plasticity} (book \#5);
	\item mathematical formulation of the equilibrium of a plasma in a cavity, expressed as a nonlinear \href{https://en.wikipedia.org/wiki/Free_boundary_problem}{free boundary problem};[R. Temam, A nonlinear eigenvalue problem: the shape at equilibrium of a confined plasma, \textit{Arch. Rational Mech. Anal.}, 60, 1975, 51--73.]
	\item \href{https://en.wikipedia.org/wiki/Korteweg-de_Vries_equation}{Korteweg-de Vries equation};[R. Temam, Sur un problème non linéaire, \textit{J. Math. Pures Appl.}, 48, 1969, 159--172.]
	\item \href{https://en.wikipedia.org/wiki/Kuramoto%E2%80%93Sivashinsky_equation}{Kuramoto-Sivashinsky equation};[5]
	\item \href{https://en.wikipedia.org/wiki/Euler_equations}{Euler equations} in a bounded domain;[R. Temam, On the Euler equations of incompressible perfect fluids, \textit{J. Funct. Anal.}, 20, 1975, 32--43.]
	\item infinite-dimensional \href{https://en.wikipedia.org/wiki/Dynamical_systems}{dynamical systems} theory.
	
	In particular, he studied the existence of the finite-dimensional global \href{https://en.wikipedia.org/wiki/Attractor}{attractor} for many dissipative equations of mathematical physics, including the incompressible \href{https://en.wikipedia.org/wiki/Navier-Stokes_equations}{Navier-Stokes equations}.[P. Constantin, C. Foias, O. Manley \& R. Temam, Determining modes \& fractal dimension of turbulent flows, \textit{J. Fluid Mech.}, 150, 1985, 427--440.][C. Foias, O.P. Manley \& R. Temam, Physical estimates of the number of degrees of freedom in free convection, \textit{Phys. Fluids}, 29, 1986, 3101--3103.]
	
	He was also the co-founder of the notion of \textit{inertial manifolds}[C. Foias, G.R. Sell \& R. Temam, Inertial manifolds for nonlinear evolutionary equations, \textit{J. Diff. Equ.}, 73, 1988, 309--353.] together with Ciprian Foias \& \href{https://en.wikipedia.org/wiki/George_R._Sell}{George R. Sell} \& of exponential attractors[A. Eden, C. Foias, B. Nicolaenko \& R. Temam, \textit{Exponential attractors for dissipative evolution equations}, Collection Recherches en Mathématiques Appliquées, Masson, Paris, \& John Wiley, England, 1994.] together with Alp Eden, Ciprian Foias \& Basil Nicolaenko;[2]
	\item \href{https://en.wikipedia.org/wiki/Optimal_control}{optimal control} of the incompressible Navier-Stokes equations as a tool for the \textit{control of turbulence};[F. Abergel \& R. Temam, On some control problems in fluid mechanics, \textit{Theoret. Comput. Fluid Dynamics}, 1, 1990, 303--325.]
	\item \href{https://en.wikipedia.org/wiki/Boundary_layer}{boundary layer} phenomena for incompressible flows.[12]
\end{itemize}
Temam's main activities concern the study of \href{https://en.wikipedia.org/wiki/Geophysical_fluid_dynamics}{geophysical flows}, the atmosphere \& oceans.[2]

This started in the 1990s by collaboration with Jacques-Louis Lions \& Shouhong Wang.[J.L. Lions, R. Temam \& S. Wang, New formulations of the primitive equations of the atmosphere \& applications, \textit{Nonlinearity}, 5, 1992, 237--288.][J.L. Lions, R. Temam \& S. Wang, On the equations of the large-scale ocean, \textit{Nonlinearity}, 5, 1992, 1007--1053.][M. Coti Zelati, M. Frémond, R. Temam \& J. Tribbia, Uniqueness, regularity \& maximum principles for the equations of the atmosphere with humidity \& saturation, \textit{Physica D}, 264, 2013, 49-65, https://doi.org/10.1016/j.physd.2013.08.007][Y. Cao, M. Hamouda, R. Temam, J. Tribbia \& X. Wang, The equations of the multi-phase humid atmosphere expressed as a quasi variational inequality, \textit{Nonlinearity}, 31, 2018, 4692-4723, https://doi.org/10.1088/1361-6544/aad525.]

%
According to the \href{https://en.wikipedia.org/wiki/Mathematical_Genealogy_Project}{Mathematical Genealogy Project} database,[17][18] he holds the first position in the top 50 advisors.

More than 30 of his students are now full professors all over the world, \& have themselves many descendants.[19]

\subsubsection{Administrative activities}
Temam became a professor at the \href{https://en.wikipedia.org/wiki/Paris-Sud_University}{Paris-Sud University} at Orsay in 1968.

There, he co-founded the Laboratory of Numerical \& Functional Analysis which he directed from 1972 to 1988.

He was also a \textit{Maître de Conférences} at the \href{https://en.wikipedia.org/wiki/Ecole_Polytechnique}{Ecole Polytechnique} in Paris from 1968 to 1986.[20]

%
In 1983, Temam co-founded the French \href{https://en.wikipedia.org/wiki/Soci%C3%A9t%C3%A9_de_Math%C3%A9matiques_Appliqu%C3%A9es_et_Industrielles}{Société de Mathématiques Appliquées et Industrielles} (SMAI), analogous to the \href{https://en.wikipedia.org/wiki/Society_for_Industrial_and_Applied_Mathematics}{Society for Industrial \& Applied Mathematics} (SIAM), \& served as its 1st president.[21]

He was also 1 of the founders of the \href{https://en.wikipedia.org/wiki/International_Congress_on_Industrial_and_Applied_Mathematics}{International Congress on Industrial \& Applied Mathematics} (ICIAM) series \& was the chair of the steering committee of the 1st ICIAM meeting held in Paris in 1987; \& the chair of the standing committee of the 2nd ICIAM meeting held in Washington, D.C., in 1991.[22]

He was the Editor-in-Chief of the mathematical journal M2AN[23] from 1986 to 1997.

%
Temam has been the Director of the Institute for Scientific Computing \& Applied Mathematics (ISCAM)[24] at \href{https://en.wikipedia.org/wiki/Indiana_University_Bloomington}{Indiana University} since 1986 (co-director with Ciprian Foias from 1986 to 1992).

He is also a College Professor (part-time till 2003) \& he has been a Distinguished Professor since 2014.[25]

\subsubsection{Books}
\begin{enumerate}
	\item (with G.-M. Gie, M. Hamouda \& C.-Y. Jung): \textit{Singular perturbations \& boundary layers}, Springer-Verlag, New-York, 2018.
	\item (with A. Miranville): \textit{Mathematical Modelling in Continuum Mechanics}, Cambridge University Press, 2001. French Translation, Springer-Verlag France, 2002. Chinese Translation, Tsinghua University Press, 2004. 2nd English Edition 2005. Russian translation, Moskva Linom, 2013.
	\item (with T. Dubois \& F. Jauberteau): \textit{Dynamic, multilevel methods \& the numerical simulation of turbulence}; Cambridge University Press, 1999.
	\item \textit{Infinite Dimensional Dynamical Systems in Mechanics \& Physics}, Springer-Verlag, New-York, Applied Mathematical Sciences Series, vol. 68, 1988. 2nd augmented edition, 1997. Reprinted in China by Beijing World Publishing Corp., 2000.
	\item \textit{Mathematical Problems in Plasticity}, Gauthier-Villars, Paris, 1983 (in French). English Transl., Gauthier-Villars, New-York, 1985. Russian Transl., Nauk, Moscow, 1991. ``Republished by Dover books in Physics, 2018.''
	\item \textit{Navier-Stokes Equations}, North-Holland Pub. Company, in English, 1977, 500 pages. Revised editions 1979, 1984 \& 1985. Russian Translation, Mir, Moscow, 1981. ``Republished in the AMS-Chelsea Series, AMS, Providence, 2001.''
	\item (with I. Ekeland): \textit{Convex Analysis \& Variational Problems}. Dunod, Paris, 1974, 350 pages (in French). English Translation, North-Holland, Amsterdam, 1976. Russian Translation, Mir, Moscow, 1979. ``English version republished in the Series 'Classics in Applied Mathematics', SIAM, Philadelphia, 1999.''
\end{enumerate}

\subsubsection{Awards \& honors}
\begin{itemize}
	\item Fellow of the \href{https://en.wikipedia.org/wiki/American_Academy_of_Arts_and_Sciences}{American Academy of Arts \& Sciences} (2015),[26] of the \href{https://en.wikipedia.org/wiki/American_Mathematical_Society}{American Mathematical Society} (2013),[27] of the American Association for the Advancement of Science (2011),[28] of the Society for Industrial \& Applied Mathematics (2009).[29]
	\item Knight of the \href{https://en.wikipedia.org/wiki/Legion_of_Honour}{Legion of Honor}, France, 2012.[30]
	\item Member of the \href{https://en.wikipedia.org/wiki/French_Academy_of_Sciences}{French Academy of Sciences} since 2007.[31]'' -- \href{https://en.wikipedia.org/wiki/Roger_Temam}{Wikipedia{\tt/}Roger Temam}
\end{itemize}

%------------------------------------------------------------------------------%

\subsection{Karl Weierstrass}
\textbf{Karl Weierstrass/Karl Weierstraß.}
\begin{itemize}
	\item \textbf{Born.} Oct 31, 1815. \href{https://en.wikipedia.org/wiki/Ennigerloh}{Ostenfelde}, \href{https://en.wikipedia.org/wiki/Province_of_Westphalia}{Province of Westphalia}, \href{https://en.wikipedia.org/wiki/Kingdom_of_Prussia}{Kingdom of Prussia}.
	\item \textbf{Died.} Feb 19, 1897 (aged 81). Berlin, \href{https://en.wikipedia.org/wiki/Province_of_Brandenburg}{Province of Brandenburg}, \href{https://en.wikipedia.org/wiki/Kingdom_of_Prussia}{Kingdom of Prussia}.
	\item \textbf{Nationality.} German.
	\item \textbf{Alma mater.}
	\begin{itemize}
		\item \href{https://en.wikipedia.org/wiki/University_of_Bonn}{University of Bonn}
		\item \href{https://en.wikipedia.org/wiki/University_of_M%C3%BCnster}{Münster Academy}
	\end{itemize}
	\item \textbf{Known for.}
	\begin{itemize}
		\item \href{https://en.wikipedia.org/wiki/Weierstrass_function}{Weierstrass function}
		\item \href{https://en.wikipedia.org/wiki/Weierstrass_product_inequality}{Weierstrass product inequality}
		\item \href{https://en.wikipedia.org/wiki/(%CE%B5,_%CE%B4)-definition_of_limit}{$(\varepsilon,\delta)$-definition of limit}
		\item \href{https://en.wikipedia.org/wiki/Weierstrass%E2%80%93Erdmann_condition}{Weierstrass-Erdmann condition}
		\item \href{https://en.wikipedia.org/wiki/Weierstrass_theorem_(disambiguation)}{Weierstrass theorems}
		\item \href{https://en.wikipedia.org/wiki/Bolzano-Weierstrass_theorem}{Bolzano-Weierstrass theorem}
	\end{itemize}
	\item \textbf{Awards.}
	\begin{itemize}
		\item \href{https://en.wikipedia.org/wiki/PhD_(Hon)}{PhD (Hon)}: \href{https://en.wikipedia.org/wiki/University_of_K%C3%B6nigsberg}{University of Königsberg} (1854)
		\item \href{https://en.wikipedia.org/wiki/Copley_Medal}{Copley Medal} (1895)
	\end{itemize}
\end{itemize}
\textbf{Scientific career.}
\begin{itemize}
	\item \textbf{Fields.} Mathematics.
	\item \textbf{Institutions.}
	\begin{itemize}
		\item \href{https://en.wikipedia.org/wiki/Technical_University_of_Berlin}{Gewerbeinstitut}
		\item \href{https://en.wikipedia.org/wiki/Humboldt_University_of_Berlin}{Friedrich Wilhelm University}
	\end{itemize}
	\item \textbf{Academic advisors.} \href{https://en.wikipedia.org/wiki/Christoph_Gudermann}{Christoph Gudermann}.
	\item \textbf{Doctoral students.}
	\begin{itemize}
		\item \href{https://en.wikipedia.org/wiki/Nikolai_Bugaev}{Nikolai Bugaev}
		\item \href{https://en.wikipedia.org/wiki/Georg_Cantor}{Georg Cantor}
		\item \href{https://en.wikipedia.org/wiki/Georg_Frobenius}{Georg Frobenius}
		\item \href{https://en.wikipedia.org/wiki/Lazarus_Fuchs}{Lazarus Fuchs}
		\item \href{https://en.wikipedia.org/wiki/Wilhelm_Killing}{Wilhelm Killing}
		\item \href{https://en.wikipedia.org/wiki/Leo_K%C3%B6nigsberger}{Leo Königsberger}
		\item \href{https://en.wikipedia.org/wiki/Sofia_Kovalevskaya}{Sofia Kovalevskaya}
		\item \href{https://en.wikipedia.org/wiki/Mathias_Lerch}{Mathias Lerch}
		\item \href{https://en.wikipedia.org/wiki/Hans_von_Mangoldt}{Hans von Mangoldt}
		\item \href{https://en.wikipedia.org/wiki/Eugen_Netto}{Eugen Netto}
		\item \href{https://en.wikipedia.org/wiki/Adolf_Piltz}{Adolf Piltz}
		\item \href{https://en.wikipedia.org/wiki/Carl_Runge}{Carl Runge}
		\item \href{https://en.wikipedia.org/wiki/Arthur_Schoenflies}{Arthur Schoenflies}
		\item \href{https://en.wikipedia.org/wiki/Friedrich_Schottky}{Friedrich Schottky}
		\item \href{https://en.wikipedia.org/wiki/Hermann_Schwarz}{Hermann Schwarz}
		\item \href{https://en.wikipedia.org/wiki/Ludwig_Stickelberger}{Ludwig Stickelberger}
		\item \href{https://en.wikipedia.org/wiki/Ernst_K%C3%B6tter}{Ernst Kötter}
	\end{itemize}
\end{itemize}
\textit{Karl Theodor Wilhelm Weierstrass} (German: \textit{Weierstraß};[Duden. \textit{Das Aussprachewörterbuch}. 7. Auflage. Bibliographisches Institut, Berlin 2015, ISBN 978-3-411-04067-4] Oct 31, 1815 - Feb 19, 1897) was a German mathematician often cited as the \fbox{``\textit{father of modern \href{https://en.wikipedia.org/wiki/Mathematical_analysis}{analysis}}''}.

Despite leaving university without a degree, he studied mathematics \& trained as a school teacher, eventually teaching mathematics, physics, \href{https://en.wikipedia.org/wiki/Botany}{botany} \& gymnastics.[Weierstrass, Karl Theodor Wilhelm. (2018). In Helicon (Ed.), \textit{The Hutchinson unabridged encyclopedia with atlas \& weather guide}. [Online]. Abington: Helicon. Available from: \href{http://libezproxy.open.ac.uk/login?url=https://search.credoreference.com/content/entry/heliconhe/weierstrass_karl_theodor_wilhelm/0?institutionId=292}{link} [Accessed Jul 8, 2018].]

He later received an honorary doctorate \& became professor of mathematics in Berlin.

%
Among many other contributions, Weierstrass formalized the definition of the \href{https://en.wikipedia.org/wiki/Continuous_function}{continuity of a function}, proved the \href{https://en.wikipedia.org/wiki/Intermediate_value_theorem}{intermediate value theorem} \& the \href{https://en.wikipedia.org/wiki/Bolzano-Weierstrass_theorem}{Bolzano-Weierstrass theorem}, \& used the latter to study the properties of continuous functions on closed bounded intervals.

\subsubsection{Biography}
Weierstrass was born in Ostenfelde, part of \href{https://en.wikipedia.org/wiki/Ennigerloh}{Ennigerloh}, \href{https://en.wikipedia.org/wiki/Province_of_Westphalia}{Province of Westphalia}.[O'Connor, J. J.; Robertson, E. F. (October 1998). ``\textit{Karl Theodor Wilhelm Weierstrass}''. School of Mathematics \& Statistics, University of St Andrews, Scotland. Retrieved Sep 7, 2014.]

%
Weierstrass was the son of Wilhelm Weierstrass, a government official, \& Theodora Vonderforst.

His interest in mathematics began while he was a \href{https://en.wikipedia.org/wiki/Gymnasium_(Germany)}{gymnasium} student at the \href{https://en.wikipedia.org/wiki/Gymnasium_Theodorianum}{Theodorianum} in \href{https://en.wikipedia.org/wiki/Paderborn}{Paderborn}.

He was sent to the \href{https://en.wikipedia.org/wiki/University_of_Bonn}{University of Bonn} upon graduation to prepare for a government position.

Because his studies were to be in the fields of law, economics, \& finance, he was immediately in conflict with his hopes to study mathematics.

He resolved the conflict by paying little heed to his planned course of study but continuing private study in mathematics.

The outcome was that he left the university without a degree.

He then studied mathematics at the \href{https://en.wikipedia.org/wiki/University_of_M%C3%BCnster}{Münster Academy} (which was even then famous for mathematics) \& his father was able to obtain a place for him in a teacher training school in \href{https://en.wikipedia.org/wiki/M%C3%BCnster}{Münster}.

Later he was certified as a teacher in that city.

During this period of study, Weierstrass attended the lectures of \href{https://en.wikipedia.org/wiki/Christoph_Gudermann}{Christoph Gudermann} \& became interested in \href{https://en.wikipedia.org/wiki/Elliptic_function}{elliptic functions}.

%
In 1843 he taught in \href{https://en.wikipedia.org/wiki/Wa%C5%82cz}{Deutsch Krone} in \href{https://en.wikipedia.org/wiki/West_Prussia}{West Prussia} \& since 1848 he taught at the \href{https://en.wikipedia.org/wiki/Collegium_Hosianum}{Lyceum Hosianum} in \href{https://en.wikipedia.org/wiki/Braunsberg}{Braunsberg}.

Besides mathematics he also taught physics, botany, \& gymnastics.[3]

%
Weierstrass may have had an illegitimate child named Franz with the widow of his friend \href{https://en.wikipedia.org/wiki/Carl_Wilhelm_Borchardt}{Carl Wilhelm Borchardt}.[Biermann, Kurt-R.; Schubring, Gert (1996). ``Einige Nachträge zur Biographie von Karl Weierstraß. (German) [Some postscripts to the biography of Karl Weierstrass]''. \textit{History of mathematics}. San Diego, CA: Academic Press. pp. 65--91.]

%
After 1850 Weierstrass suffered from a long period of illness, but was able to publish mathematical articles that brought him fame \& distinction.

The \href{https://en.wikipedia.org/wiki/University_of_K%C3%B6nigsberg}{University of Königsberg} conferred an \href{https://en.wikipedia.org/wiki/Honorary_doctor%27s_degree}{honorary doctor's degree} on him on Mar 31, 1854.

In 1856 he took a chair at the \textit{Gewerbeinstitut} in Berlin (an institute to educate technical workers which would later merge with the \textit{Bauakademie} to form the \href{https://en.wikipedia.org/wiki/Technical_University_of_Berlin}{Technical University of Berlin}).

In 1864 he became professor at the Friedrich-Wilhelms-Universität Berlin, which later became the \href{https://en.wikipedia.org/wiki/Humboldt_University_of_Berlin}{Humboldt Universität zu Berlin}.

%
In 1870, at the age of 55, Weierstrass met \href{https://en.wikipedia.org/wiki/Sofia_Kovalevskaya}{Sofia Kovalevsky} whom he tutored privately after failing to secure her admission to the University.
They had a fruitful intellectual, but troubled personal, relationship that ``far transcended the usual teacher-student relationship''.

The misinterpretation of this relationship \& Kovalevsky's early death in 1891 was said to have contributed to Weierstrass' later ill-health.

He was immobile for the last 3 years of his life, \& died in Berlin from \href{https://en.wikipedia.org/wiki/Pneumonia}{pneumonia}.[\textit{Dictionary of scientific biography}. Gillispie, Charles Coulston,, American Council of Learned Societies. New York. p. 223. ISBN 978-0-684-12926-6. OCLC 89822.]

\subsubsection{Mathematical contributions}

\paragraph{Soundness of calculus}
Weierstrass was interested in the \href{https://en.wikipedia.org/wiki/Soundness}{soundness} of calculus, \& at the time there were somewhat ambiguous definitions of the foundations of calculus so that important theorems could not be proven with sufficient rigor.

Although \href{https://en.wikipedia.org/wiki/Bernard_Bolzano}{Bolzano} had developed a reasonably rigorous definition of a \href{https://en.wikipedia.org/wiki/Limit_of_a_function}{limit} as early as 1817 (and possibly even earlier) his work remained unknown to most of the mathematical community until years later, \& many mathematicians had only vague definitions of \href{https://en.wikipedia.org/wiki/Limit_of_a_function}{limits} \& \href{https://en.wikipedia.org/wiki/Continuous_function}{continuity} of functions.

%
The basic idea behind \href{https://en.wikipedia.org/wiki/(%CE%B5,_%CE%B4)-definition_of_limit}{Delta-epsilon} proofs is, arguably, 1st found in the works of \href{https://en.wikipedia.org/wiki/Augustin-Louis_Cauchy}{Cauchy} in the 1820s.
\begin{itemize}
	\item Grabiner, Judith V. (March 1983), ``Who Gave You the Epsilon? Cauchy \& the Origins of Rigorous Calculus'' (PDF), \textit{The American Mathematical Monthly}, 90 (3): 185--194, doi:10.2307/2975545, JSTOR 2975545
	\item Cauchy, A.-L. (1823), ``Septième Leçon – Valeurs de quelques expressions qui se présentent sous les formes indéterminées $\frac{\infty}{\infty},\infty^0,\ldots$ Relation qui existe entre le rapport aux différences finies et la fonction dérivée'', \textit{Résumé des leçons données à l'école royale polytechnique sur le calcul infinitésimal}, Paris, archived from the original on 2009-05-04, retrieved 2009-05-01, p. 44.
\end{itemize}
Cauchy did not clearly distinguish between continuity \& uniform continuity on an interval.

Notably, in his 1821 \textit{Cours d'analyse}, Cauchy argued that the (pointwise) limit of (pointwise) continuous functions was itself (pointwise) continuous, a statement interpreted as being incorrect by many scholars.

The correct statement is rather that the \href{https://en.wikipedia.org/wiki/Uniform_limit}{uniform limit} of continuous functions is continuous (also, the uniform limit of uniformly continuous functions is uniformly continuous).

This required the concept of \href{https://en.wikipedia.org/wiki/Uniform_convergence}{uniform convergence}, which was 1st observed by Weierstrass's advisor, \href{https://en.wikipedia.org/wiki/Christoph_Gudermann}{Christoph Gudermann}, in an 1838 paper, where Gudermann noted the phenomenon but did not define it or elaborate on it.

Weierstrass saw the importance of the concept, \& both formalized it \& applied it widely throughout the foundations of calculus.

%
The formal definition of continuity of a function, as formulated by Weierstrass, is as follows:
\begin{quotation}
	$f(x)$ is continuous at $x = x_0$ if $\forall\varepsilon > 0$, $\exists\delta > 0$ s.t. for every $x$ in the domain of $f$, $|x - x_0| < \delta\Rightarrow|f(x) - f(x_0)| < \varepsilon$.
	
	In simple English, $f(x)$ is continuous at a point $x = x_0$ if for each $x$ close enough to $x_0$, the function value $f(x)$ is very close to $f(x_0)$, where the ``close enough'' restriction typically depends on the desired closeness of $f(x_0)$ to $f(x)$.
\end{quotation}
Using this definition, he proved the \href{https://en.wikipedia.org/wiki/Intermediate_value_theorem}{Intermediate Value Theorem}.

He also proved the \href{https://en.wikipedia.org/wiki/Bolzano-Weierstrass_theorem}{Bolzano-Weierstrass theorem} \& used it to study the properties of continuous functions on closed \& bounded intervals.

\paragraph{Calculus of variations}
Weierstrass also made advances in the field of \href{https://en.wikipedia.org/wiki/Calculus_of_variations}{calculus of variations}.

Using the apparatus of analysis that he helped to develop, Weierstrass was able to give a complete reformulation of the theory that paved the way for the modern study of the calculus of variations.

Among several axioms, Weierstrass established a necessary condition for the existence of \href{https://en.wikipedia.org/wiki/Strong_extrema}{strong extrema} of variational problems.

He also helped devise the \href{https://en.wikipedia.org/wiki/Weierstrass-Erdmann_condition}{Weierstrass-Erdmann condition}, which gives sufficient conditions for an extremal to have a corner along a given extremum \& allows one to find a minimizing curve for a given integral.

\paragraph{Other analytical theorems}
See also: \href{https://en.wikipedia.org/wiki/List_of_things_named_after_Karl_Weierstrass}{List of things named after Karl Weierstrass}.
\begin{itemize}
	\item \href{https://en.wikipedia.org/wiki/Stone-Weierstrass_theorem}{Stone-Weierstrass theorem}
	\item Casorati-Weierstrass-Sokhotski theorem
	\item \href{https://en.wikipedia.org/wiki/Weierstrass's_elliptic_functions}{Weierstrass's elliptic functions}
	\item \href{https://en.wikipedia.org/wiki/Weierstrass\_function}{Weierstrass function}
	\item \href{https://en.wikipedia.org/wiki/Weierstrass_M-test}{Weierstrass M-test}
	\item \href{https://en.wikipedia.org/wiki/Weierstrass_preparation_theorem}{Weierstrass preparation theorem}
	\item \href{https://en.wikipedia.org/wiki/Lindemann-Weierstrass_theorem}{Lindemann-Weierstrass theorem}
	\item \href{https://en.wikipedia.org/wiki/Weierstrass_factorization_theorem}{Weierstrass factorization theorem}
	\item \href{https://en.wikipedia.org/wiki/Enneper-Weierstrass_parameterization}{Enneper-Weierstrass parameterization}
\end{itemize}

\subsubsection{Students}
\begin{itemize}
	\item \href{https://en.wikipedia.org/wiki/Edmund_Husserl}{Edmund Husserl}
	\item \href{https://en.wikipedia.org/wiki/Sofia_Kovalevskaya}{Sofia Kovalevskaya}
	\item \href{https://en.wikipedia.org/wiki/G%C3%B6sta_Mittag-Leffler}{Gösta Mittag-Leffler}
	\item \href{https://en.wikipedia.org/wiki/Hermann_Schwarz}{Hermann Schwarz}
	\item \href{https://en.wikipedia.org/wiki/Carl_Johannes_Thomae}{Carl Johannes Thomae}
	\item \href{https://en.wikipedia.org/wiki/Georg_Cantor}{Georg Cantor}
\end{itemize}

\subsubsection{Honors \& awards}
The lunar \href{https://en.wikipedia.org/wiki/Impact_crater}{crater} \href{https://en.wikipedia.org/wiki/Weierstrass_(crater)}{Weierstrass} \& the \href{https://en.wikipedia.org/wiki/Asteroid}{asteroid} \href{https://en.wikipedia.org/wiki/14100_Weierstrass}{14100 Weierstrass} are named after him.

Also, there is the \href{https://en.wikipedia.org/wiki/Weierstrass_Institute_for_Applied_Analysis_and_Stochastics}{Weierstrass Institute for Applied Analysis \& Stochastics} in Berlin.

\subsubsection{Selected works}
\begin{itemize}
	\item \textit{Zur Theorie der Abelschen Funktionen} (1854)
	\item \textit{Theorie der Abelschen Funktionen} (1856)
	\item \href{http://name.umdl.umich.edu/AAN8481.0001.001}{\textit{Abhandlungen-1}}, Math. Werke. Bd. 1. Berlin, 1894
	\item \href{http://name.umdl.umich.edu/AAN8481.0002.001}{\textit{Abhandlungen-2}}, Math. Werke. Bd. 2. Berlin, 1895
	\item \href{http://name.umdl.umich.edu/AAN8481.0003.001}{\textit{Abhandlungen-3}}, Math. Werke. Bd. 3. Berlin, 1903
	\item \href{http://name.umdl.umich.edu/AAN8481.0004.001}{\textit{Vorl. ueber die Theorie der Abelschen Transcendenten}}, Math. Werke. Bd. 4. Berlin, 1902
	\item \href{http://name.umdl.umich.edu/AAN8481.0007.001}{\textit{Vorl. ueber Variationsrechnung}}, Math. Werke. Bd. 7. Leipzig, 1927
\end{itemize}

\subsubsection{External links}
\begin{itemize}
	\item O'Connor, John J.; Robertson, Edmund F., ``Karl Weierstrass'', \textit{MacTutor History of Mathematics archive}, University of St Andrews.
	\item \href{http://bibliothek.bbaw.de/bibliothek-digital/digitalequellen/schriften/autoren/weierstr/}{Digitalized versions of Weierstrass's original publications} are freely available online from the library of the \textit{Berlin Brandenburgische Akademie der Wissenschaften}.
	\item \href{https://www.gutenberg.org/author/Weierstrass,+Karl}{Works by Karl Weierstrass} at Project Gutenberg
	\item Works by or about Karl Weierstrass at Internet Archive'' -- \href{https://en.wikipedia.org/wiki/Karl_Weierstrass}{Wikipedia{\tt/}Karl Weierstrass}
\end{itemize}

%------------------------------------------------------------------------------%

\section{Miscellaneous}

%------------------------------------------------------------------------------%

\printbibliography[heading=bibintoc]
	
\end{document}