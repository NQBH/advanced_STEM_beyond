\documentclass{article}
\usepackage[backend=biber,natbib=true,style=alphabetic,maxbibnames=50]{biblatex}
\addbibresource{/home/nqbh/reference/bib.bib}
\usepackage[utf8]{vietnam}
\usepackage{tocloft}
\renewcommand{\cftsecleader}{\cftdotfill{\cftdotsep}}
\usepackage[colorlinks=true,linkcolor=blue,urlcolor=red,citecolor=magenta]{hyperref}
\usepackage{amsmath,amssymb,amsthm,enumitem,float,graphicx,mathtools,tikz}
\usetikzlibrary{angles,calc,intersections,matrix,patterns,quotes,shadings}
\allowdisplaybreaks
\newtheorem{assumption}{Assumption}
\newtheorem{baitoan}{}
\newtheorem{cauhoi}{Câu hỏi}
\newtheorem{conjecture}{Conjecture}
\newtheorem{corollary}{Corollary}
\newtheorem{dangtoan}{Dạng toán}
\newtheorem{definition}{Definition}
\newtheorem{dinhly}{Định lý}
\newtheorem{dinhnghia}{Định nghĩa}
\newtheorem{example}{Example}
\newtheorem{ghichu}{Ghi chú}
\newtheorem{hequa}{Hệ quả}
\newtheorem{hypothesis}{Hypothesis}
\newtheorem{lemma}{Lemma}
\newtheorem{luuy}{Lưu ý}
\newtheorem{nhanxet}{Nhận xét}
\newtheorem{notation}{Notation}
\newtheorem{note}{Note}
\newtheorem{principle}{Principle}
\newtheorem{problem}{Problem}
\newtheorem{proposition}{Proposition}
\newtheorem{question}{Question}
\newtheorem{remark}{Remark}
\newtheorem{theorem}{Theorem}
\newtheorem{vidu}{Ví dụ}
\usepackage[left=1cm,right=1cm,top=5mm,bottom=5mm,footskip=4mm]{geometry}
\def\labelitemii{$\circ$}
\DeclareRobustCommand{\divby}{%
	\mathrel{\vbox{\baselineskip.65ex\lineskiplimit0pt\hbox{.}\hbox{.}\hbox{.}}}%
}
\setlist[itemize]{leftmargin=*}
\setlist[enumerate]{leftmargin=*}

\title{Mathematics -- Toán Học}
\author{Nguyễn Quản Bá Hồng\footnote{A Scientist {\it\&} Creative Artist Wannabe. E-mail: {\tt nguyenquanbahong@gmail.com}. Bến Tre City, Việt Nam.}}
\date{\today}

\begin{document}
\maketitle
\begin{abstract}
	This text is a part of the series {\it Some Topics in Advanced STEM \& Beyond}:
	
	{\sc url}: \url{https://nqbh.github.io/advanced_STEM/}.
	
	Latest version:
	\begin{itemize}
		\item {\it Mathematics -- Toán Học}.
		
		PDF: {\sc url}: \url{.pdf}.
		
		\TeX: {\sc url}: \url{.tex}.
	\end{itemize}
\end{abstract}
\tableofcontents

%------------------------------------------------------------------------------%

\section{Wikipedia}

\subsection{Wikipedia{\tt/}pure mathematics}
``{\sf Pure mathematics studies the properties \& structure of abstract objects, e.g., the \href{https://en.wikipedia.org/wiki/E8_(mathematics)}{E8 group}, in \href{https://en.wikipedia.org/wiki/Group_theory}{group theory}. This may be done without focusing on concrete applications of the concepts in the physical world.} {\it Pure mathematics} is the study of mathematical concepts independently of any application outside mathematics. These concepts may originate in real-word concerns, \& the results obtained may later turn out to be useful for practical applications, but pure mathematicians are not primarily motivated by such applications. Instead, the appeal is attributed to the intellectual challenge \& \href{https://en.wikipedia.org/wiki/Mathematical_beauty}{aesthetic beauty} of working out the logical consequences of basic principles.

While pure mathematics has existed as an activity since at least \href{https://en.wikipedia.org/wiki/Ancient_Greece}{ancient Greece}, the concept was elaborated upon around the year 1900, after the introduction of theries with counter-intuitive properties (e.g., \href{https://en.wikipedia.org/wiki/Non-Euclidean_geometries}{non-Euclidean geometries} \& \href{https://en.wikipedia.org/wiki/Georg_Cantor}{\sc Cantor}'s theory of infinite sets), \& the discovery of apparent paradoxes (e.g., \href{https://en.wikipedia.org/wiki/Continuous_function}{continuous functions} that are nowhere differentiable, \& \href{https://en.wikipedia.org/wiki/Russell%27s_paradox}{Russell's paradox}). This introduced the need to renew the concept of \href{https://en.wikipedia.org/wiki/Mathematical_rigor}{mathematical rigor} \& rewrite all mathematics accordingly, with a systematic use of \href{https://en.wikipedia.org/wiki/Axiomatic_method}{axiomatic methods}. This led many mathematicians to focus on mathematics for its own sake, i.e., pure mathematics.

Nevertheless, almost all mathematical theories remained motivated by problems coming from the real world or form less abstract mathematical theories. Also, many mathematical theories, which had seemed to be totally pure mathematics, were eventually used in applied areas, mainly physics \& computer science. A famous early example is \href{https://en.wikipedia.org/wiki/Isaac_Newton}{\sc Isaac Newton}'s demonstration that his \href{https://en.wikipedia.org/wiki/Law_of_universal_gravitation}{law of universal gravitation} implied that \href{https://en.wikipedia.org/wiki/Planet}{planets} move in orbits that are \href{https://en.wikipedia.org/wiki/Conic_section}{conic sections}, geometrical curves that had been studied in antiquity by \href{https://en.wikipedia.org/wiki/Apollonius_of_Perga}{\sc Apollonius}. Another example is the problem of \href{https://en.wikipedia.org/wiki/Factorization}{Factoring} large integers, which is the basis of the \href{https://en.wikipedia.org/wiki/RSA_cryptosystem}{RSA cryptosystem}, widely used to secure \href{https://en.wikipedia.org/wiki/Internet}{internet} communications.

It follows that, presently, the distinction between pure \& \href{https://en.wikipedia.org/wiki/Applied_mathematics}{applied mathematics} is more a philosophical point of view or a mathematician's preference rather than a rigid subdivision of mathematics.

\subsubsection{History}

\begin{enumerate}
	\item {\bf Ancient Greece.} Ancient Greek mathematicians were among the earliest to make a distinction between pure \& applied mathematics. \href{https://en.wikipedia.org/wiki/Plato}{\sc Plato} helped to create the gap between ``arithmetic'', now called \href{https://en.wikipedia.org/wiki/Number_theory}{number theory}, \& ``logistic'', now called \href{https://en.wikipedia.org/wiki/Arithmetic}{arithmetic}. {\sc Plato} regarded logistic (arithmetic) as appropriate for businessmen \& men of war who ``must learn the art of numbers or [they] will not know how to array [their] troops'' \& arithmetic (number theory) as appropriate for philosophers ``because [they have] to arise out of the sea of change \& lay hold of true being.'' \href{https://en.wikipedia.org/wiki/Euclid_of_Alexandria}{{\sc Euclid} of Alexandria}, when asked by 1 of his students of what use was the study of geometry, asked his slave to give the student threepence, ``since he must make gain of what he learns.'' The Greek mathematician \href{https://en.wikipedia.org/wiki/Apollonius_of_Perga}{{\sc Apollonius} of Pega} was asked about the usefulness of some of his theorems in Book IV of {\it Conics} to which he proudly asserted,
	\begin{quote}
		They are worthy of acceptance for the sake of the demonstrations themselves, in the same way as we accept many other things in mathematics for this \& for no other reason.
	\end{quote}
	\& since many of his results were not applicable to the science or engineering of his day, {\sc Apollonius} further argued in the preface of the 5th book of {\it Conics} that the subject is 1 of those that ``$\ldots$ seem worthy of study for their own sake.''
	\item {\bf19th century.} The term itself is enshrined in the full title of the \href{https://en.wikipedia.org/wiki/Sadleirian_Professor_of_Pure_Mathematics}{Sadleirian Chair}, ``Sadleirian Professor of Pure Mathematics'', founded (as a professorship) in the mid-19th century. The idea of a separate discipline of {\it pure} mathematics may have emerged at that time. The generation of \href{https://en.wikipedia.org/wiki/Carl_Friedrich_Gauss}{\sc Gauss} made no sweeping distinction of the kind between {\it pure} \& {\it applied}. In the following years, specialization \& professionalization (particularly in the \href{https://en.wikipedia.org/wiki/Weierstrass}{\sc Weierstrass} approach to \href{https://en.wikipedia.org/wiki/Mathematical_analysis}{mathematical analysis}) started to make a rift more apparent.
	\item {\bf20th century.} At the start of the 20th century mathematicians took up the \href{https://en.wikipedia.org/wiki/Axiomatic_method}{axiomatic method}, strongly influenced by \href{https://en.wikipedia.org/wiki/David_Hilbert}{\sc David Hilbert}'s example. The logical formulation of pure mathematics suggested by \href{https://en.wikipedia.org/wiki/Bertrand_Russell}{\sc Bertrand Russell} in terms of a \href{https://en.wikipedia.org/wiki/Quantifier_(logic)}{quantifier} structure of \href{https://en.wikipedia.org/wiki/Proposition_(mathematics)}{propositions} seemed more \& more plausible, as large parts of mathematics became axiomatized \& thus subject to the simple criteria of \href{https://en.wikipedia.org/wiki/Rigorous_proof}{\it rigorous proof}.
	
	Pure mathematics, according to a view that can be ascribed to the \href{https://en.wikipedia.org/wiki/Bourbaki_group}{Bourbaki group}, is what is proved. ``Pure mathematician'' became a recognized vocation, achievable through training.
	
	The case was made that pure mathematics is useful in \href{https://en.wikipedia.org/wiki/Engineering_education}{engineering education}:
	\begin{quote}
		There is a training in habits of thought, points of view, \& intellectual comprehension of ordinary engineering problems, which only the study of higher mathematics can give.
	\end{quote}
\end{enumerate}

\subsubsection{Generality \& abstraction}
{\sf An illustration of the \href{https://en.wikipedia.org/wiki/Banach%E2%80%93Tarski_paradox}{Banach--Tarski paradox}, a famous result in pure mathematics. Although it is proven that is possible to convert 1 sphere into 2 using nothing but cuts \& rotations, the transformation involves objects that cannot exist in the physical world.} 1 central concept in pure mathematics is the idea of generality; pure mathematics often exhibits a trend towards increased generality. Uses \& advantages of generality include:
\begin{itemize}
	\item Generalizing theorems or mathematical structures can lead to deeper understanding of the original theorems or structures
	\item Generality can simplify the presentation of material, resulting in shorter proofs or arguments that are easier to follow.
	\item One can use generality to avoid duplication of effort, proving a general result instead of having to prove separate cases independently, or using results from other areas of mathematics.
	\item Generality can facilitate connections between different branches of mathematics. \href{https://en.wikipedia.org/wiki/Category_theory}{Category theory} is 1 area of mathematics dedicated to exploring this commonality of structure as it plays out in some areas of math.
\end{itemize}
Generality's impact on \href{https://en.wikipedia.org/wiki/Intuition_(knowledge)}{intuition} is both dependent on the subject \& a matter of personal preference or learning style. Often generality is seen as a hindrance to intuition, although it can certainly function as an aid to it, especially when it provides analogies to material for which one already has good intuition.

As a prime example of generality, the \href{https://en.wikipedia.org/wiki/Erlangen_program}{Erlangen program} involved an expansion of \href{https://en.wikipedia.org/wiki/Geometry}{geometry} to accommodate \href{https://en.wikipedia.org/wiki/Non-Euclidean_geometries}{non-Euclidean geometries} as well as the field of \href{https://en.wikipedia.org/wiki/Topology}{topology}, \& other forms of geometry, by viewing geometry as the study of a space together with a \href{https://en.wikipedia.org/wiki/Group_(mathematics)}{group} of transformations. The study of numbers, called \href{https://en.wikipedia.org/wiki/Algebra}{algebra} at the beginning undergraduate level, extends to \href{https://en.wikipedia.org/wiki/Abstract_algebra}{abstract algebra} at a more advanced level; \& the study of functions, called \href{https://en.wikipedia.org/wiki/Calculus}{calculus} at the college freshman level becomes mathematical analysis \& functional analysis at a more advanced level. Each of these branches of more {\it abstract} mathematics have many sub-specialties, \& there are in fact many connections between pure mathematics \& applied mathematics disciplines. A steep rise in \href{https://en.wikipedia.org/wiki/Abstraction}{abstraction} was seen mid 20th century.

In practice, however, these developments led to a sharp divergence from physics, particularly from 1950--1983. Later this was criticized, e.g., by \href{https://en.wikipedia.org/wiki/Vladimir_Arnold}{\sc Vladimir Arnold}, as too much Hilbert, not enough Poincar\'e. The point does not yet seem to be settled, in that \href{https://en.wikipedia.org/wiki/String_theory}{string theory} pulls 1 way, while \href{https://en.wikipedia.org/wiki/Discrete_mathematics}{discrete mathematics} pulls back towards proof as central.

\subsubsection{Pure vs. applied mathematics}
Mathematicians have always had differing opinions regarding the distinction between pure \& applied mathematics. 1 of the most famous (but perhaps misunderstood) modern examples of this debate can b found in \href{https://en.wikipedia.org/wiki/G.H._Hardy}{\sc G. H. Hardy}'s 1940 essay \href{https://en.wikipedia.org/wiki/A_Mathematician%27s_Apology}{\it A Mathematician's Apology}.

It is widely believed that {\sc Hardy} considered applied mathematics to be ugly \& dull. Although it is true that {\sc Hardy} preferred pure mathematics, which he often compared to \href{https://en.wikipedia.org/wiki/Painting}{painting} \& \href{https://en.wikipedia.org/wiki/Poetry}{poetry}, {\sc Hardy} saw the distinction between pure \& applied mathematics to be simply that applied mathematics sought to express {\it physical} truth in a mathematical framework, whereas pure mathematics expressed truths that were independent of the physical world. {\sc Hardy} made a separate distinction in mathematics between what he called ``real'' mathematics, ``which has permanent aesthetic value'', \& ``the dull \& elementary parts of mathematics'' that have practical use.

{\sc Hardy} considered some physicists, e.g., \href{https://en.wikipedia.org/wiki/Albert_Einstein}{\sc Einstein} \& \href{https://en.wikipedia.org/wiki/Paul_Dirac}{\sc Dirac}, to be among the ``real'' mathematicians, but at the time that he was writing his {\it Apology}, he considered \href{https://en.wikipedia.org/wiki/General_relativity}{general relativity} \& \href{https://en.wikipedia.org/wiki/Quantum_mechanics}{quantum mechanics} to be ``useless'', which allowed him to hold the opinion that only ``dull'' mathematics was useful. Moreover, {\sc Hardy} briefly admitted that -- just as the application of \href{https://en.wikipedia.org/wiki/Matrix_(mathematics)}{matrix theory} \& \href{https://en.wikipedia.org/wiki/Group_theory}{group theory} to physics had come unexpectedly -- the time may come where some kind of beautiful, ``real'' mathematics may be useful as well.

Another insightful view is offered by American mathematician \href{https://en.wikipedia.org/wiki/Andy_Magid}{\sc Andy Magid}:
\begin{quote}
	I've always thought that a good model here could be drawn from ring theory. In that subject, one has the subareas of \href{https://en.wikipedia.org/wiki/Commutative_ring}{commutative ring theory} \& \href{https://en.wikipedia.org/wiki/Non-commutative_ring}{non-commutatitve ring theory}. A uninformed observer might think that these represent a dichotomy, but in fact the latter subsumes the former: a non-commutative ring is a not-necessarily-commutative ring. If we use similar conventions, then we could refer to applied mathematics \& nonapplied mathematics, where by the latter we {\it mean not necessarily-applied mathematics} $\ldots$
\end{quote}
\href{https://en.wikipedia.org/wiki/Friedrich_Engels}{\sc Friedrich Engels} argued in this 1878 book \href{https://en.wikipedia.org/wiki/Anti-D%C3%BChring}{\it Anti-D\"uhring} that ``it is not at all true that in pure mathematics the mind deals only with its own creations \& imaginations. The concepts of number \& figure have not been invented from any source other than the world of reality''. He further argued that ``Before one came upon the idea of deducing the form of a cylinder from the rotation of a rectangle about 1 of its sides, a number of real rectangles \& cylinders, however imperfect in form, must have been examined. Like all other sciences, mathematics arose out of the needs of men $\ldots$ But, as in every department of thought, at a certain stage of development the independent, as laws coming from outside, to which the world has to conform.'''' -- \href{https://en.wikipedia.org/wiki/Pure_mathematics}{Wikipedia{\tt/}pure mathematics}

%------------------------------------------------------------------------------%

\section{Miscellaneous}

%------------------------------------------------------------------------------%

\printbibliography[heading=bibintoc]
	
\end{document}