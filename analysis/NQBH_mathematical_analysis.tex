\documentclass{article}
\usepackage[backend=biber,natbib=true,style=alphabetic,maxbibnames=50]{biblatex}
\addbibresource{/home/nqbh/reference/bib.bib}
\usepackage[utf8]{vietnam}
\usepackage{tocloft}
\renewcommand{\cftsecleader}{\cftdotfill{\cftdotsep}}
\usepackage[colorlinks=true,linkcolor=blue,urlcolor=red,citecolor=magenta]{hyperref}
\usepackage{amsmath,amssymb,amsthm,enumitem,float,graphicx,mathtools,tikz}
\usetikzlibrary{angles,calc,intersections,matrix,patterns,quotes,shadings}
\allowdisplaybreaks
\newtheorem{assumption}{Assumption}
\newtheorem{baitoan}{}
\newtheorem{cauhoi}{Câu hỏi}
\newtheorem{conjecture}{Conjecture}
\newtheorem{corollary}{Corollary}
\newtheorem{dangtoan}{Dạng toán}
\newtheorem{definition}{Definition}
\newtheorem{dinhly}{Định lý}
\newtheorem{dinhnghia}{Định nghĩa}
\newtheorem{example}{Example}
\newtheorem{ghichu}{Ghi chú}
\newtheorem{hequa}{Hệ quả}
\newtheorem{hypothesis}{Hypothesis}
\newtheorem{lemma}{Lemma}
\newtheorem{luuy}{Lưu ý}
\newtheorem{nhanxet}{Nhận xét}
\newtheorem{notation}{Notation}
\newtheorem{note}{Note}
\newtheorem{principle}{Principle}
\newtheorem{problem}{Problem}
\newtheorem{proposition}{Proposition}
\newtheorem{question}{Question}
\newtheorem{remark}{Remark}
\newtheorem{theorem}{Theorem}
\newtheorem{vidu}{Ví dụ}
\usepackage[left=1cm,right=1cm,top=5mm,bottom=5mm,footskip=4mm]{geometry}
\def\labelitemii{$\circ$}
\DeclareRobustCommand{\divby}{%
	\mathrel{\vbox{\baselineskip.65ex\lineskiplimit0pt\hbox{.}\hbox{.}\hbox{.}}}%
}
\setlist[itemize]{leftmargin=*}
\setlist[enumerate]{leftmargin=*}

\title{Mathematical Analysis \& Numerical Analysis\\Giải Tích Toán Học \& Giải Tích Số}
\author{Nguyễn Quản Bá Hồng\footnote{A Scientist {\it\&} Creative Artist Wannabe. E-mail: {\tt nguyenquanbahong@gmail.com}. Bến Tre City, Việt Nam.}}
\date{\today}

\begin{document}
\maketitle
\begin{abstract}
	This text is a part of the series {\it Some Topics in Advanced STEM \& Beyond}:
	
	{\sc url}: \url{https://nqbh.github.io/advanced_STEM/}.
	
	Latest version:
	\begin{enumerate}
		\item {\it Mathematical Analysis -- Giải Tích Toán Học}.
		
		PDF: {\sc url}: \url{https://github.com/NQBH/advanced_STEM_beyond/blob/main/analysis/NQBH_mathematical_analysis.pdf}.
		
		\TeX: {\sc url}: \url{https://github.com/NQBH/advanced_STEM_beyond/blob/main/analysis/NQBH_mathematical_analysis.tex}.
	\end{enumerate}
\end{abstract}
\tableofcontents

%------------------------------------------------------------------------------%

Tôi được giải Nhì Giải tích Olympic Toán Sinh viên 2014 (VMC2014) khi còn học năm nhất Đại học \& được giải Nhất Giải tích Olympic Toán Sinh viên 2015 (VMC2015) khi học năm 2 Đại học. Nhưng điều đó không có nghĩa là tôi giỏi Giải tích. Bằng chứng là 10 năm sau khi nhận các giải đó, tôi đang tự học lại Giải tích Toán học với hy vọng có 1 hay nhiều cách nhìn mới mẻ hơn \& mang tính ứng dụng hơn cho các đề tài cá nhân của tôi.

\section{Basic}
\textbf{\textsf{Resources -- Tài nguyên.}}
\begin{enumerate}
	\item \cite{Lieb_Loss2001}. {\sc Elliott Lieb, Michael Loss}. {\it Analysis}.
	\item \cite{Rudin1976}. {\sc Walter Rudin}. {\it Principles Principles of Mathematical Analysis}.
	\item \cite{Rudin1973,Rudin1987}. {\sc Walter Rudin}. {\it Real \& Complex Analysis}.
	\item \cite{Tao_analysis_1}. {\sc Terence Tao}. {\it Analysis I}.
	\item \cite{Tao_analysis_2}. {\sc Terence Tao}. {\it Analysis II}.
\end{enumerate}
``Analysis is the art of taking limits, \& the constraint of having to deal with an integration theory that does not allow taking limits is much like having to do mathematics only with rational numbers \& excluding the irrational ones.'' -- \cite[Chap. 1, p. 1]{Lieb_Loss2001}

%------------------------------------------------------------------------------%

\section{$C_0$ Semigroup -- Nửa Nhóm $C_0$}
\textbf{\textsf{Resources -- Tài nguyên.}}
\begin{enumerate}
	\item \cite{Anh_Ke_semigroup}. {\sc Cung Thế Anh, Trần Đình Kế}. {\it Nửa Nhóm Các Toán Tử Tuyến Tính \& Ứng Dụng}.
\end{enumerate}
``In mathematical analysis, a {\it $C_0$-semigroup}, also known as a {\it strongly continuous 1-parameter semigroup}, is a generalization of the \href{https://en.wikipedia.org/wiki/Exponential_function}{exponential function}. Just as exponential functions provide solutions of scalar linear constant ODEs, strongly continuous semigroups provide solutions of linear constant coefficient ODEs in \href{https://en.wikipedia.org/wiki/Banach_space}{Banach spaces}. Such differential equations in Banach spaces arise from e.g. \href{https://en.wikipedia.org/wiki/Delay_differential_equation}{delay differential equations} \& PDEs. Formally, a strongly continuous semigroup is a representation of the \href{https://en.wikipedia.org/wiki/Semigroup}{semigroup} $(\mathbb{R}_+,+)$ on some Banach space $X$ that is continuous in the \href{https://en.wikipedia.org/wiki/Strong_operator_topology}{strong toperator topology}.'' -\href{https://en.wikipedia.org/wiki/C0-semigroup}{Wikipedia{\tt/}$C_0$-semigroup}

%------------------------------------------------------------------------------%

\section{Differential Geometry -- Hình Học Vi Phân}
\textbf{\textsf{Resources -- Tài nguyên.}}
\begin{enumerate}
	\item \cite{Carmo2016}. {\sc Manfredo P. do Carmo}. {\it Differential Geometry of Curves \& Surfaces}.
	\item \cite{Delfour_Zolesio2001,Delfour_Zolesio2011}. {\sc Michael C. Delfour, Jean-Paul Zol\'{e}sio}. {\it Shapes \& Geometries}.
	\item \cite{Kuhnel2015}. {\sc Wolfgang K\"uhnel}. {\it Differential Geometry}.
	\item \cite{Walker2015}. {\sc Shawn W. Walker}. {\it The Shapes of Things}.
	
	``Differential geometry is the detailed study of the {\it shape} of a surface (manifold), including {\it local} \& {\it global} properties. A plane in $\mathbb{R}^3$ is a very simple surface \& does not require many tools to characterize. An ``arbitrarily'' shaped surface, e.g., hood of a car, has many distinguished geometric features (e.g., highly curved regions, regions of near flatness, etc.). Characterizing these features quantitatively \& qualitatively requires the tools of differential geometry. Geometric details are important in many physical \& biological processes, e.g., surface tension, biomembranes.
	
	The framework of differential geometry is built by 1st defining a local map (i.e., surface parameterization) which defines the surface. Then, a calculus framework is built up on the surface analogous to the standard ``Euclidean calculus''. Other approaches are also possible, e.g., those with implicit surfaces defined by level sets \& distance functions. But parameterizations, though arbitrary, are quite useful in a variety of settings $\Rightarrow$ stick mostly with those. Emphasize: The geometry of a surface does not depend on a particular parameterization. Otherwise, we will emphasize the distinction between {\tt object 1} \& {\tt object 2}.
	
	We will use this ``abuse'' of notation when there is no possibility of ambiguity.
	
	{\bf Open set.} The concept of open set is critical in multivariate calculus to properly define differentiability. The notation for referencing boundaries of sets, as well as the closure of sets, is practical for referencing geometric details of solid objects \& their surfaces.
	
	{\bf Compactness.} Compact support is useful for ignoring boundary effects. This concept is needed to keep the ``action of a function'' away from the boundary of a set, or to localize the function in a region of interest. 1 reason is to avoid potential difficulties with differentiating a function at its boundary of definition. Or, more commonly, we wish to ignore a quantity depending on the value of a function at a boundary point, e.g., $\int_{\partial S} f = 0$ if $f$ has compact support in $S$.
	
	{\bf Topological mapping{\tt/}homeomorphism.} A bijective, continuous mapping $\boldsymbol{\Phi}$ whose inverse $\boldsymbol{\Phi}^{-1}$ is also continuous is called a {\it topological mapping} or {\it homeomorphism}. Point sets that can be topologically mapped onto each other are said to be {\it homeomorphic}. Sets that are homeomorphic have the ``same topology'', i.e., their connectedness is the same; they have the same kinds of ``holes''. See \cite[Sect. 2.3.1]{Walker2015} for what can happen when a mapping is not a homeomorphism.
	
	{\bf Rigid motion mapping.} A mapping $\boldsymbol{\Phi}$ is called a {\it rigid motion} if any pair of points ${\bf a},{\bf b}$ are the same distance apart as the corresponding pair $\boldsymbol{\Phi}({\bf a}),\boldsymbol{\Phi}({\bf b})$.
	
	{\bf Orthogonal Transformations.} Define the (affine) linear map $\boldsymbol{\Phi}$ (transformation)
	\begin{equation}
		\label{transformation}
		\widetilde{\bf x} = \boldsymbol{\Phi}({\bf x}) = A{\bf x} + {\bf b}.
	\end{equation}
	If $A$ satisfies the properties $A^{-1} = A^\top$, $\det A = 1$ then $\boldsymbol{\Phi}$ represents a rigid motion. Basically, $\boldsymbol{\Phi}$ consists of a rotation represented by $A$ followed by a translation represented by ${\bf b}$. A rigid motion can be used to transition from 1 Cartesian coordinate system to another. If ${\bf b} = {\bf 0}$ \& $A^{-1} = A^\top$, $\det A = 1$, then $\boldsymbol{\Phi}({\bf x}) = A{\bf x}$ is a linear map known as a {\it direct orthogonal transformation}, which is nothing more than a rotation of the coordinate system with the origin as the center. If $A^{-1} = A^\top$, $\det A = 1$ is replaced by $A^{-1} = A^\top$, $\det A = -1$, then $\boldsymbol{\Phi}({\bf x}) = A{\bf x}$ is called an {\it opposite orthogonal transformation}, which consists of a rotation about the origin \& a reflection in a plane. Both $A^{-1} = A^\top$, $\det A = \pm1$ are examples of {\it orthogonal matrices}.
	
	{\bf Interpretation of transformations.} Can interpret $\widetilde{\bf x} = \boldsymbol{\Phi}({\bf x}) = A{\bf x} + {\bf b}$ in 2 different ways. Consider a point $P\in\mathbb{R}^3$ with coordinates ${\bf x}$:
	\begin{itemize}
		\item {\it Alias} (Euler perspective). Viewing \eqref{transformation} as a transformation of coordinates, it appears that ${\bf x},\widetilde{\bf x}$ are the coordinates of the same point w.r.t. 2 different coordinate systems, equivalently, the point is referenced by 2 different ``names'' (sets of coordinates).
		\item {\it Alibi} (Lagrange perspective). Viewing \eqref{transformation} as a mapping of sets, it appears that ${\bf x},\widetilde{\bf x}$ are the coordinates of 2 different points w.r.t. the same coordinate system, equivalently, the point at $\widetilde{\bf x}$ ``was previously'' at ${\bf x}$ before applying the map.
	\end{itemize}
	The concept of material point is directly related to the alibi viewpoint. One can think of a ``particle'' of material, i.e., {\it material point}, initially located at ${\bf x}$, that then moves to $\widetilde{\bf x}$ because of some physical process. The transformation \eqref{transformation} simply represents the kinematic outcome of that physical process, which is a standard concept in deformable continuum mechanics, especially nonlinear elasticity.
	
	{\bf General transformations.} In general, transformation may not be linear. The alias viewpoint yields a {\it curvilinear} coordinate system. The alibi viewpoint implies that the set $S$ is {\it deformed} into the set $S' = \boldsymbol{\Phi}(S)$. 
	
	{\bf Parametric approach -- what is a surface?} A {\it surface} is a set of points in space that is ``regular enough''. A random scattering of points in space does not match our intuitive notion of what a surface is, i.e., it is not regular enough. The boundary of a sphere does match our notion of a surface, i.e., regular enough to be a surface because a sphere is ``smooth''. {\it Intuition}: Can think of creating a surface as deforming a flat rubber sheet into a curved sheet. Let $U\subset\mathbb{R}^2$ be a ``flat'' domain \& let ${\bf X}:U\to\mathbb{R}^3$ be this deforming transformation, i.e., for each point $(s_1,s_2)^\top\in U$ there is a corresponding point ${\bf x} = (x_1,x_2,x_3)^\top\in\mathbb{R}^3$ s.t. ${\bf x} = {\bf X}(s_1,s_2)$. Let $\Gamma = {\bf X}(U)$ denote the surface obtained from ``deforming'' $U$. Call ${\bf x} = {\bf X}(s_1,s_2)$ a {\it parametric representation} of the surface $\Gamma$, where $s_1,s_2$ are called the {\it parameters} of the representation. Refer to $U$ as a {\it reference} domain.
	
	{\bf Allowable parameterization{\tt/}immersion.} If use ${\bf x} = {\bf X}(s_1,s_2)$ to define surfaces, then we must place assumptions on ${\bf X}$ to guarantee that $\Gamma = {\bf X}(U)$ is a valid surface. At the bare minimum, ${\bf X}$ must be continuous to avoid ``tearing'' the rubber sheet. But if want to perform calculus on $\Gamma$, need more:
	\begin{assumption}[Regularity assumptions on ${\bf X}$]
		An allowable parameterization{\tt/}immersion is a parameterization of the form ${\bf x} = {\bf X}(s_1,s_2)$ satisfying:
		\item[(A1)] The function ${\bf X}(s_1,s_2)\in C^\infty(U)$ \& each point ${\bf x} = {\bf X}(s_1,s_2)\in\Gamma$ corresponds to just 1 point $(s_1,s_2)\in U$, i.e., ${\bf X}$ is injective.
		\item[(A2)] The Jacobian matrix $J = [\partial_{s_1}{\bf X},\partial_{s_2}{\bf X}]$ is of rank $2$ on $U$, i.e., the columns of $J$ are linearly independent.
	\end{assumption}
	{\bf Regular surface.} The fundamental property that makes a set of points in $\mathbb{R}^3$ a surface is that it {\it locally looks like a plane} at every point. If you ``zoom into'' a surface, it should look flat. Definition defining a surface in terms of a parameterization is inadequate. Want to define a set in $\mathbb{R}^3$ that is ``intrinsically'' 2D \& is smooth enough so we can perform calculus on it, without regard to a specific parameterization.
	
	\begin{definition}[Regular surface]
		
	\end{definition}
	
	\begin{remark}[Local chart]
		
	\end{remark}
\end{enumerate}
1 trong những ứng dụng của Hình Học Vi Phân là {\it Shape Calculus \& Tangential Calculus -- Phép Tính Vi Tích Phân cho Tối Ưu Hình Dáng \& Phép Tính Vi Tích Phân Trên Mặt Phẳng Tiếp Tuyến}.

\subsection{Calculus on Surfaces}
{\bf Goal.} Define \& develop the fundamental tools of calculus on a regular surface. Start with the notion of differentiability of functions defined only on a surface. Define the concept of vector fields in a surface. Then proceed to develop the gradient \& Laplacian operators w.r.t. a surface. These operators allow for alternative expressions of the summed \& Gaussian curvatures. Derive integration by parts on surfaces, i.e., the domain of integration is a surface. Conclude with some useful identities \& inequalities. Always take $\Gamma$: a regular surface, either with or without a boundary.

%------------------------------------------------------------------------------%

\section{Functional Analysis -- Giải Tích Hàm}
\textbf{\textsf{Resources -- Tài nguyên.}}
\begin{enumerate}
	\item \cite{Rudin1991}. {\sc Walter Rudin}. {\it Functional Analysis}.
	\item {\sc Yosida}.
\end{enumerate}

%------------------------------------------------------------------------------%

\section{Inverse Problems -- Bài Toán Ngược}
\textbf{\textsf{Resources -- Tài nguyên.}}
\begin{enumerate}
	\item \cite{Aster_Borchers_Thurber2018}. {\sc Richard Aster, Brian Borchers, Clifford H. Thurber}. {\it Parameter Estimation \& Inverse Problems}.
	\item \cite{Kirsch2021}. {\sc Andreas Kirsch}. {\it An Introduction to The Mathematical Theory of Inverse Problems}.
	\item \cite{Ito_Jin2015}. {\sc Kazufumi Ito, Bangti Jin}. {\it Inverse Problems}.
\end{enumerate}

%------------------------------------------------------------------------------%

\section{Measure \& Integration -- Độ Đo \& Tích Phân}
\textbf{\textsf{Resources -- Tài nguyên.}}
\begin{enumerate}
	\item \cite{Evans_Gariepy2015}. {\sc Lawrence C. Evans, Ronald F. Gariepy}. {\it Measure Theory \& Fine Properties of Functions}.
\end{enumerate}
The point of view of integration defined as a Riemann integral may be historically grounded \& useful in many areas of mathematics but is far from being adequate for the requirements of modern analysis since Riemann integral can be defined only for a special class of functions \& this class is not closed under the process of taking pointwise limits of sequence (not even monotonic sequences) of functions in this class.

``The useful \& far-reaching idea of Lebesgue \& others was to compute the $(n + 1)$-dimensional volume `in the other direction' by 1st computing the $n$-dimensional volume of the set where the function $> y$. This volume is a well-behaved, monotone nonincreasing function of $y$, which then can be integrated in the manner of Riemann. This method of integration not only works for a large class of functions (which is closed under taking pointwise limits), but it also greatly simplifies a problem that used to plague analysts: {\it Is it permissible to exchange limits \& integration?}'' -- \cite[Chap. 1, pp. 1--2]{Lieb_Loss2001}

Lebesgue integration theory is 1 of the great triumphs of 20th century mathematics \& is the culmination of a long struggle to find the right perspective from which to view integration theory.

%------------------------------------------------------------------------------%

\section{Mean-Field Game Theory -- Lý Thuyết Trò Chơi Trường Trung Bình}
\textbf{\textsf{Community -- Cộng đồng.}} {\sc Nicholetta Tchou (French), Đào Mạnh Khang (Vietnamese), Michael Hinterm\"uller (Austrian), Steven-Marian Stengl (German)}.

\subsection{Wikipedia{\tt/}mean-field game theory}
``{\it Mean-field game theory} is the study of strategic decision making by small interacting \href{https://en.wikipedia.org/wiki/Agent_(economics)}{agents} in very large populations. It lies at the intersection of \href{https://en.wikipedia.org/wiki/Game_theory}{game theory} with stochastic analysis \& \href{https://en.wikipedia.org/wiki/Control_theory}{control theory}. The use of the term ``mean field'' is inspired by \href{https://en.wikipedia.org/wiki/Mean-field_theory}{mean-field theory} in physics, which considers the behavior of systems of large numbers of particles where individual particles have negligible impacts upon the system. In other words, each agent acts according to his minimization or maximization problem taking into account other agents' decisions \& because their population is large we can assume the number of agents goes to infinity \& a representative agent exists.

In traditional \href{https://en.wikipedia.org/wiki/Game_theory}{game theory}, the subject of study is usually a game with 2 players \& discrete time space, \& extends the results to more complex situations by induction. However, for games in continuous time with continuous states (differential games or stochastic differential games) this strategy cannot be used because of the complexity that the dynamic interactions generate. On the other hand with MFGs we can handle large numbers of players through the mean representative agent \& at the same time describe complex state dynamics.

This class of problems was considered in the economics literature by \href{https://en.wikipedia.org/wiki/Boyan_Jovanovic}{Boyan Jovanovic} \& \href{https://en.wikipedia.org/wiki/Robert_W._Rosenthal}{Robert W. Rosenthal}, in the engineering literature by Minyi Huang, Roland Malhame, \& \href{https://en.wikipedia.org/wiki/Peter_E._Caines}{Peter E. Caines} \& independently \& around the same time by mathematicians Jean-Michel Lasry \& \href{https://en.wikipedia.org/wiki/Pierre-Louis_Lions}{Pierre-Louis Lions}.

In continuous time a mean-field game is typically composed of a \href{https://en.wikipedia.org/wiki/Hamilton%E2%80%93Jacobi%E2%80%93Bellman_equation}{Hamilton-Jacobi-Bellman equation} that describes the \href{https://en.wikipedia.org/wiki/Optimal_control}{optimal control} problem of an individual \& a \href{https://en.wikipedia.org/wiki/Fokker%E2%80%93Planck_equation}{Fokker--Planck equation} that describes the dynamics of the aggregate distribution of agents. Under fairly general assumptions it can be proved that a class of mean-field games is the limit as $N\to\infty$ of an $N$-player \href{https://en.wikipedia.org/wiki/Nash_equilibrium}{Nash equilibrium}.

A related concept to that of mean-field games is ``mean-field-type control''. In this case, a \href{https://en.wikipedia.org/wiki/Social_planner}{social planner} controls the distribution of states \& chooses a control strategy. The solution to a mean-field-type control problem can typically be expressed as a dual adjoint Hamilton-Jacobi-Bellman equation coupled with \href{https://en.wikipedia.org/wiki/Fokker%E2%80%93Planck_equation}{Kolmogorov equation}. Mean-field-type game theory is the multi-agent generalization of the single-agent mean-field-type control.

\subsubsection{General Form of a Mean-field Game}
The system of equations
\begin{equation*}
	\left\{\begin{split}
		-\partial_tu - \nu\Delta u + H(x,m,Du) &= 0,\\
		\partial_tm - \nu\Delta m - \nabla\cdot(D_pH(x,m,Du)m) &= 0,\\
		m(0) &= m_0,\\
		u(T,x) &= G(x,m(T)),
	\end{split}\right.
\end{equation*}
can be used to model a typical Mean-field game. The basic dynamics of this set of equations can be explained by an average agent's optimal control problem. In a mean-field game, an average agent can control their movement $\alpha$ to influence the population's overall location by
\begin{equation*}
	dX_t = \alpha_tdt + \sqrt{2\nu}dB_t,
\end{equation*}
where $\nu$: a parameter, $B_t$: a standard Brownian motion. By controlling their movement, the agent aims to minimize their overall expected cost $C$ throughout the time period $[0,T]$:
\begin{equation*}
	C = \mathbb{E}\left[\int_0^T L(X_s,\alpha_s,m(s))\,{\rm d}s + G(X_T,m(T))\right],
\end{equation*}
where $L(X_s,\alpha_s,m(s))$ is the running cost at time $s$ \& $G(X_T,m(T))$ is the terminal cost at time $T$. By this definition, at time $t$ \& position $x$, the value function $u(t,x)$ can be determined as
\begin{equation*}
	u(t,x) = \inf_\alpha \mathbb{E}\left[\int_t^T L(X_s,\alpha_s,m(s))\,{\rm d}s + G(X_T,m(T))\right].
\end{equation*}
Given the definition of the value function $u(t,x)$, it can be tracked by the Hamilton-Jacobi equation. The optimal action of the average players $\alpha^*(t,x)$ can be determined as $\alpha^*(t,x) = D_pH(x,m,Du)$. As all agents are relatively small \& cannot single-handedly change the dynamics of the population, they will individually adapt the optimal control \& the population would move in that way. This is similar to a Nash Equilibrium, in which all agents act in response to a specific set of others' strategies. The optimal control solution then leads to the Kolmogorov-Fokker-Planck equation $\partial_tm - \nu\Delta m - \nabla\cdot(D_pH(x,m,Du)m) = 0$.

\subsubsection{Finite State Games}
A prominent category of mean field is games with a finite number of states \& a finite number of actions per player. For those games, the analog of the Hamilton-Jacobi-Bellman equation is the Bellman equation, \& the discrete version of the Fokker-Planck equation is the Kolmogorov equation. Specifically, for discrete-time models, the players' strategy is the Kolmogorv equation's probability matrix. In continuous time models, players have the ability to control the transition rate matrix.

A discrete mean field game can be defined by a tuple $\mathcal{G} = (\mathcal{E},\mathcal{A},\{Q_a\},{\bf m}_0,\{c_a\},\beta)$ where $\mathcal{E}$ is the state space, $\mathcal{A}$ the action set, $Q_a$ the transition rate matrices, ${\bf m}_0$ the initial state, $\{c_a\}$ the cost functions \& $\beta\in\mathbb{R}$ a discount factor. Furthermore, a mixed strategy is a measurable function $\pi:\mathbb{R}^+\times\mathbb{E}\to\mathcal{P}(\mathcal{A})$, that associates to each state $i\in\mathcal{E}$ \& each time $t\ge0$ a probability measure $\pi_i(t)\in\mathcal{P}(\mathcal{A})$ on the set of possible actions. Thus $\pi_{i,a}(t)$ is the probability that, at time $t$ a player in state $i$ takes action $a$, under strategy $\pi$. Additionally, rate matrices $\{Q_a({\bf m}^\pi(t))\}_{a\in\mathcal{A}}$ define the evolution over the time of population distribution, where ${\bf m}^\pi(t)\in\mathcal{P}(\mathcal{E})$ is the population distribution at time $t$.

\subsubsection{Linear-quadratic Gaussian game problem}
From Caines (2009), a relatively simple model of large-scale games is the \href{https://en.wikipedia.org/wiki/Linear%E2%80%93quadratic%E2%80%93Gaussian_control}{linear-quadratic Gaussian} model. The individual agent's dynamics are modeled as a \href{https://en.wikipedia.org/wiki/Stochastic_differential_equation}{stochastic differential equation}
\begin{equation*}
	dX_i = (a_iX_i + b_iu_i)dt + \sigma_idW_i,\ i = 1,\ldots,N,
\end{equation*}
where $X_i$: the state of the $i$th agent, $u_i$: control of the $i$th agent, $W_i$: independent \href{https://en.wikipedia.org/wiki/Wiener_process}{Wiener processes} $\forall i = 1,\ldots,N$. The individual agent's cost is
\begin{equation*}
	J_i(u_i,\nu) = \mathbb{E}\left[\int_0^\infty e^{-\rho t}[(X_i - \nu)^2 + ru_i^2]\,{\rm d}t\right],\ \nu = \Phi\left(\frac{1}{N}\sum_{k\ne i}^N X_k + \eta\right).
\end{equation*}
The coupling between agents occurs in the cost function.

\subsubsection{General \& Applied Use}
The paradigm of Mean Field Games has become a major connection between distributed decision-making \& stochastic modeling. Starting out tin the stochastic control literature, it is gaining rapid adoption across a range of applications, including:
\begin{enumerate}
	\item {\bf Financial market.} Carmona reviews applications in financial engineering \& economics that can be cast \& tackled within the framework of the MFG paradigm. Carmona argues that models in macroeconomics, contract theory, finance, $\ldots$, greatly benefit from the switch to continuous time from the more traditional discrete-time models. He considers only continuous time models in his review chapter, including systemic risk, price impact, optimal execution, models for bank runs, high-frequency trading, \& cryptocurrencies.
	\item {\bf Crowd motions.} MFG assumes that individuals are smart players which try to optimize their strategy \& path w.r.t. certain costs (equilibrium with rational expectations approach). MFG models are useful to describe the anticipation phenomenon: the forward part describes the crowd evolution while the backward gives the process of how the anticipations are built. Additionally, compared to multi-agent microscopic model computations, MFG only requires lower computational costs for the macroscopic simulations. Some researchers have turned to MFG in order to model the interaction between populations \& study the decision-making process of intelligent agents, including aversion \& congestion behavior between 2 groups of pedestrians, departure time choice of morning commuters, \& decision-making processes for autonomous vehicle.
	\item {\bf Control \& mitigation of Epidemics.} Since the epidemic has affected society \& individuals significantly, MFG \& mean-field controls (MFCs) provide a perspective to study \& understand the underlying population dynamics, especially in the context of the Covid-19 pandemic response. MFG has been used to extend the SIR-type dynamics with spatial effects or allowing for individuals to choose their behaviors \& control their contributions to the spread of the disease. MFC is applied to design the optimal strategy to control the virus spreading within a spatial domain, control individuals' decisions to limit their social interactions, \& support the government's nonpharmaceutical interventions.'' -- \href{https://en.wikipedia.org/wiki/Mean-field_game_theory}{Wikipedia{\tt/}mean-field game theory}
\end{enumerate}

%------------------------------------------------------------------------------%

\section{Partial Differential Equations (PDEs) -- Phương Trình Vi Phân Đạo Hàm Riêng}
\textbf{\textsf{Resources -- Tài nguyên.}}
\begin{enumerate}
	\item \cite{Brezis2011}. {\sc Ha\"im Brezis}. {\it Functional Analysis, Sobolev Spaces, \& Partial Differential Equations}.
	\item \cite{Evans2010}. {\sc Lawrence C. Evans}. {\it Partial Differential Equations}.
	\item \cite{Gilbarg_Trudinger2001}. {\sc David Gilbarg, Neil S. Trudinger}. {\it Elliptic Partial Differential Equations of 2nd Order}.
\end{enumerate}

%------------------------------------------------------------------------------%

\subsection{Weak solution -- Nghiệm yếu}

\begin{definition}[Weak solution -- Nghiệm yếu]
	``In mathematics, a \emph{weak solution} (also called a \emph{generalized solution}) to an ODE or PDE is a function for which the derivatives may not all exist but which is nonetheless deemed to satisfy the equation in some precisely defined sense. There are many different definitions of weak solution, appropriate for different classes of equations. 1 of the most important is based on the notion of \href{https://en.wikipedia.org/wiki/Distribution_(mathematics)}{distributions}.'' -- \href{https://en.wikipedia.org/wiki/Weak_solution}{Wikipedia{\tt/}weak solution}
\end{definition}
``Avoiding the language of distributions, one starts with a differential equation \& rewrites it in such a way that no derivatives of the solution of the equation show up (the new form is called the \href{https://en.wikipedia.org/wiki/Weak_formulation}{weak formulation}, \& the solutions to it are called {\it weak solutions}). Somewhat surprisingly, a differential equation may have solutions which are not differentiable; \& the weak formulation allows one to find such solutions.

Weak solutions are important because many differential equations encountered in modeling real-world phenomena do not admit of sufficiently smooth solutions, \& the only way of solving such equations is using the weak formulation. Even in situations where an equation does have differentiable solutions, it is often convenient to 1st prove the existence of weak solutions \& only alter show that those solutions are in fact smooth enough.'' -- \href{https://en.wikipedia.org/wiki/Weak_solution}{Wikipedia{\tt/}weak solution}

\begin{example}[1st-order wave equation]
	The 1st-order \href{https://en.wikipedia.org/wiki/Wave_equation}{wave equation} $\partial_tu + \partial_xu = 0$ in $\mathbb{R}^2$ with $u = u(t,x)$ has the weak form $\int_{\mathbb{R}^2} u\partial_t\varphi + u\partial_x\varphi\,{\rm d}t\,{\rm d}x = 0$ has a solution $u(t,x) = |t - x|$ which may be checked by splitting the integrals over region $\{x\ge t\}$ \& $\{x\le t\}$ where $u$ is smooth.
\end{example}
``The notion of weak solution based on distribution is sometimes inadequate. In the case of \href{https://en.wikipedia.org/wiki/Hyperbolic_system}{hyperbolic systems}, the notion of weak solution based on distributions does not guarantee uniqueness, \& it is necessary to supplement it with {\it entropy conditions} or some other selection criterion. In fully nonlinear PDE e.g. \href{https://en.wikipedia.org/wiki/Hamilton%E2%80%93Jacobi_equation}{Hamilton-Jacobi equation}, there is a very different definition of weak solution called \href{https://en.wikipedia.org/wiki/Viscosity_solution}{\it viscosity solution}.'' -- \href{https://en.wikipedia.org/wiki/Weak_solution}{Wikipedia{\tt/}weak solution}

\subsubsection{General idea}
When solving a differential equation in $u$, one can rewrite it using a \href{https://en.wikipedia.org/wiki/Test_function}{test function} $\varphi$ s.t. whatever derivatives in $u$ show up in the equation, they are ``transferred'' via integration by parts to $\varphi$, resulting in an equation without derivatives of $u$. This new equation generalizes the original equation to include solutions which are not necessarily differentiable. The approach illustrated above works in great generality. Consider a linear differential operator in an open set $W\subset\mathbb{R}^d$:
\begin{equation*}
	P({\bf x},\partial)u({\bf x}) = \sum a_\alpha({\bf x})\partial^\alpha u({\bf x}),
\end{equation*}
where the multi-index $\alpha = (\alpha_1,\ldots,\alpha_d)$ varies over some finite set in $\mathbb{N}^d$ \& the coefficients $a_\alpha$ are smooth enough functions of ${\bf x}\in\mathbb{R}^d$. The differential equation $P({\bf x},\partial)u({\bf x} = 0$ can, after being multiplied by a smooth test function $\varphi\in C_c^\infty(W)$ \& integrated by parts, be written as
\begin{equation*}
	\int_W u({\bf x})Q({\bf x},\partial)\varphi({\bf x})\,{\rm d}{\bf x} = 0,
\end{equation*}
where the differential operator $Q({\bf x},\partial)$ is given by the formula
\begin{equation*}
	Q({\bf x},\partial)\varphi({\bf x}) = \sum (-1)^{|\alpha|}\partial^\alpha[a_\alpha({\bf x})\varphi({\bf x})],
\end{equation*}
which is the \href{https://en.wikipedia.org/wiki/Formal_adjoint}{formal adjoint} of $P({\bf x},\partial)$.

In summary, if the original (strong) problem was to find a $|\alpha|$-times differentiable function $u$ defined on the open set $W$ s.t. $P({\bf x},\partial)u({\bf x}) = 0$, $\forall{\bf x}\in W$ (a so-called {\it strong solution}), then an integrable function $u$ would be said to be a {\it weak solution} if $\int_W u({\bf x})Q({\bf x},\partial)\varphi({\bf x})\,{\rm d}{\bf x} = 0$, $\forall\varphi\in C_c^\infty(W)$.

%------------------------------------------------------------------------------%

\subsection{Viscosity solution -- Nghiệm trơn{\tt/}nhớt}

\begin{example}[Viscosity solution for Hamilton--Jacobi equation]
	Hamilton--Jacobi equation.
\end{example}

%------------------------------------------------------------------------------%

\subsection{Very weak solution -- Nghiệm rất yếu}

\begin{example}[Very weak solution of porous medium equation (PME) \cite{Vazquez2007}]
	.
\end{example}

\begin{example}[Very weak solution of multi-dimensional slow diffusion equations with a singular quenching term \cite{Dao_Diaz_Nguyen2020}]
	Given $f\in L_\delta^1(\Omega),\lambda\ge0$, a function $u\in L_\delta^1(\Omega)$ is called a \emph{very weak solution} of
	\begin{equation*}
		\left\{\begin{split}
			-\Delta(|u|^{m-1}u) + \lambda u &= f&&\mbox{in }\Omega,\\
			|u|^{m-1}u &= 1&&\mbox{on }\Gamma,
		\end{split}\right.
	\end{equation*}
	if $|u|^{m-1}u\in L^1(\Omega)$ and
	\begin{equation*}
		\int_\Omega u^m\Delta\varphi + \lambda u\varphi\,{\rm d}{\bf x} = \int_\Omega f\varphi\,{\rm d}{\bf x} - \int_\Gamma \partial_{\bf n}\varphi\,{\rm d}{\bf x}.
	\end{equation*}
\end{example}

\begin{example}[Very weak solution of NSEs \cite{Tsai2018}]
	.
\end{example}

\subsection{Navier--Stokes Equations}
\textbf{\textsf{Resources -- Tài nguyên.}}
\begin{enumerate}
	\item \cite{Ladyzhenskaya1969}. {\sc O. A. Ladyzhenskaya}. {\it The Mathematical Theory of Viscous Incompressible Flow}.
	\item \cite{Sohr2001,Sohr2013}. {\sc Hermann Sohr}. {\it The NSEs: An Elementary Functional Analytic Approach}.
	
	{\bf Primary objective.} To develop an elementary \& self-contained approach to the mathematical theory of a viscous incompressible fluid in a domain $\Omega\subset\mathbb{R}^d$, described by NSEs. Formulate the theory for a completely general domain $\Omega$.
	\item \cite{Temam1977,Temam2000}. {\sc Roger Temam}. {\it NSES: Theory \& Numerical Analysis}.
	\item \cite{Tsai2018}. {\sc Tai-Peng Tsai}. {\it Lectures on NSEs}.
\end{enumerate}

%------------------------------------------------------------------------------%

\section{Sobolev Spaces -- Không Gian Sobolev}
\textbf{\textsf{Resources -- Tài nguyên.}}
\begin{enumerate}
	\item \cite{Adams_Fournier2003}. {\sc Robert A. Adams, John J. F. Fournier}. {\it Sobolev Spaces}.
	\item \cite{Gagliardo1957}. {\sc Emilio Gagliardo}. {\it Caratterizzazioni delle tracce sulla frontiera relative ad alcune classi di funzioni in {$n$} variabili}.
	\item {\sc Nec\v{a}s}.
	\item \cite{Tartar2006}. {\sc Luc Tartar}. {\it An Introduction to Sobolev Spaces \& Interpolation Spaces}.
\end{enumerate}

%------------------------------------------------------------------------------%

\section{Finite Difference Methods FDMs -- Phương Pháp Sai Phân Hữu Hạn}
\textbf{\textsf{Resources -- Tài nguyên.}}
\begin{enumerate}
	\item \cite{LeVeque2007}. {\sc Randall J. LeVeque}. {\it FDMs for ODE \& PDEs: Steady-State \& Time-Dependent Problems}.
\end{enumerate}

%------------------------------------------------------------------------------%

\section{Finite Element Methods FEMs -- Phương Pháp Phần Tử Hữu Hạn}
\textbf{\textsf{Resources -- Tài nguyên.}}
\begin{enumerate}
	\item \cite{Brenner_Scott2008}. {\sc Susanne C. Brenner, L. Ridgway Scott}. {\it The Mathematical Theory of FEMs}.
	\item \cite{Ern_Guermond2004}. {\sc Alexandre Ern, Jean-Luc Guermond}. {\it Theory \& Practice of Finite Elements}.
	\item \cite{Girault_Raviart1986}. {\sc Vivette Girault, Pierre-Arnaud Raviart}. {\it FEMs for NSEs}.
	\item \cite{Gunzburger1989}. {\sc Max D. Gunzburger}. {\it FEMs for Viscous Incompressible Flows}.
	\item \cite{John2016}. {\sc Volker John}. {\it FEMs for Incompressible Flow Problems}.
\end{enumerate}
I met {\sc Volker John}, lead of Research Group 3 in WIAS in 2020 to discuss on turbulence models, e.g., Smagonrinsky, $k$-$\epsilon$ \& their simulations.

%------------------------------------------------------------------------------%

\section{Finite Volume Methods FVMs -- Phương Pháp Thể Tích Hữu Hạn}
\textbf{\textsf{Resources -- Tài nguyên.}}
\begin{enumerate}
	\item \cite{Eymard_Gallouet_Herbin2019}. {\sc Robert Eymard, Thierry Gallou\"et, Rapha\`ele Herbin}. {\it Finite Volume Methods}.
	\item \cite{LeVeque2002}. {\sc Randall J. LeVeque}. {\it FEMs for Hyperbolic Problems}.
\end{enumerate}

%------------------------------------------------------------------------------%

\section{Mathematicians \& Their Legacies -- Các Nhà Toán Học \& Các Di Sản}

\subsection{Ha\"\i m Brezis}

%------------------------------------------------------------------------------%

\subsection{Lawrence Chris Evans}

%------------------------------------------------------------------------------%

%------------------------------------------------------------------------------%

\subsection{Peter Lax}

%------------------------------------------------------------------------------%

\subsection{Andrew Joseph Majda}
\textbf{\textsf{Resources -- Tài nguyên.}}
\begin{enumerate}
	\item \cite{memory_Andrew_Joseph_Majda}. {\sc Bjorn Engquist, Panagiotis Souganidis, Samuel N. Stechmann, Vlad Vicol}. {\it In memory of Andrew J. Majda}.
	
	``He was hard working until the end even though he suffered from serious health issues for quite some time.''
	
	``He advocated a philosophy for applied mathematics research that involves the interaction of math theory, asymptotic modeling, numerical modeling, and observed and experimental data $\ldots$ Andy Majda's modus operandi of modern applied mathematics, as a symbiotic relationship between (i) rigorous mathematical theory, (ii) numerical analysis and numerical modeling, (iii) observed phenomena and experimental data, and (iv) qualitative and/or asymptotic modeling [Maj00].''
	
	``Andy's legacy lives on in the mathematical science he created, but also in the many students \& postdocs he so enthusiastically taught \& mentored.''
	
	``The period at UCLA was followed by 5 years at Berkeley, 1979--1984. During this productive time, he developed ``Majda's model'' for combustion in reactive flows, \& together with Tosio Kato \& Tom Beale derived ``Beale-Kato-Majda criterion,'' which characterizes a putative incompressible Euler singularity in terms of the accumulation of vorticity [BKM84].''
	
	``At Courant, Andy shifted his research efforts to cross-disciplinary research in modern applied mathematics with climate--atmosphere--ocean science.''
\end{enumerate}

%------------------------------------------------------------------------------%

\subsection{Vladimir Mazya}

%------------------------------------------------------------------------------%

\subsection{Jind\v{r}ich Ne\v{c}as}

%------------------------------------------------------------------------------%

\subsection{Louis Nirenberg}
\cite{Vazquez2020}

%------------------------------------------------------------------------------%

\subsection{Stanley Osher}

%------------------------------------------------------------------------------%

\section{Miscellaneous}

%------------------------------------------------------------------------------%

\printbibliography[heading=bibintoc]
	
\end{document}