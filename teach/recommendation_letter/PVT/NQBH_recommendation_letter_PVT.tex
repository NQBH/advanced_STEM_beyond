\documentclass[11pt]{article}
\usepackage[backend=biber,natbib=true,style=alphabetic,maxbibnames=50]{biblatex}
\addbibresource{/home/nqbh/reference/bib.bib}
\usepackage[utf8]{vietnam}
\usepackage{tocloft}
\renewcommand{\cftsecleader}{\cftdotfill{\cftdotsep}}
\usepackage[colorlinks=true,linkcolor=blue,urlcolor=red,citecolor=magenta]{hyperref}
\usepackage{amsmath,amssymb,amsthm,enumitem,float,graphicx,mathtools,tikz}
\usetikzlibrary{angles,calc,intersections,matrix,patterns,quotes,shadings}
\allowdisplaybreaks
\newtheorem{assumption}{Assumption}
\newtheorem{baitoan}{}
\newtheorem{cauhoi}{Câu hỏi}
\newtheorem{conjecture}{Conjecture}
\newtheorem{corollary}{Corollary}
\newtheorem{dangtoan}{Dạng toán}
\newtheorem{definition}{Definition}
\newtheorem{dinhly}{Định lý}
\newtheorem{dinhnghia}{Định nghĩa}
\newtheorem{example}{Example}
\newtheorem{ghichu}{Ghi chú}
\newtheorem{hequa}{Hệ quả}
\newtheorem{hypothesis}{Hypothesis}
\newtheorem{lemma}{Lemma}
\newtheorem{luuy}{Lưu ý}
\newtheorem{nhanxet}{Nhận xét}
\newtheorem{notation}{Notation}
\newtheorem{note}{Note}
\newtheorem{principle}{Principle}
\newtheorem{problem}{Problem}
\newtheorem{proposition}{Proposition}
\newtheorem{question}{Question}
\newtheorem{remark}{Remark}
\newtheorem{theorem}{Theorem}
\newtheorem{vidu}{Ví dụ}
\usepackage[margin=2cm]{geometry}
\def\labelitemii{$\circ$}
\DeclareRobustCommand{\divby}{%
    \mathrel{\vbox{\baselineskip.65ex\lineskiplimit0pt\hbox{.}\hbox{.}\hbox{.}}}%
}
\setlist[itemize]{leftmargin=*}
\setlist[enumerate]{leftmargin=*}

\begin{document}

\begin{figure}[H]
    \includegraphics[width=6cm]{UMT_logo}
\end{figure}

\begin{flushleft}
     MSc. {\sc Nguyen Quan Ba Hong}
     
     Department of Artificial Intelligence {\it\&} Data Science (AI-DS),
     
     School of Technology (SOT),
     
     UMT University of Management {\it\&} Technology Ho Chi Minh City.
     
     Address: Street no. 60 -- Cat Lai, Thu Duc, Ho Chi Minh city, Viet Nam.
     
     E-mail: {\sf nguyenquanbahong@gmail.com} {\it\&} {\sf hong.nguyenquanba@umt.edu.vn}.
     
     Website: \url{https://nqbh.github.io/}. GitHub: \url{https://github.com/NQBH}.
\end{flushleft}

\begin{flushright}
    ThS. {\sc Nguyễn Quản Bá Hồng}
    
    Bộ Môn Trí Tuệ Nhân Tạo {\it\&} Khoa Học Dữ Liệu (AI-DS),
    
    Khoa Công Nghệ (SOT),
    
    Trường Đại học Quản lý {\it\&} Công nghệ Thành phố Hồ Chí Minh.
    
    Địa chỉ: Đường 60 -- Phường Cát Lái, Khu đô thị, Thủ Đức, Hồ Chí Minh, Việt Nam.
    
    E-mail: {\sf nguyenquanbahong@gmail.com} {\it\&} {\sf hong.nguyenquanba@umt.edu.vn}.
    
    Website: \url{https://nqbh.github.io/}. GitHub: \url{https://github.com/NQBH}.
\end{flushright}
\vspace{5mm}
\begin{center}
    \LARGE
    \textbf{\textsf{Recommendation Letter for Undergraduate Student}}
\end{center}
As a Mathematics \& Computer Science lecturer at University of Management {\it\&} Technology Ho Chi Minh City, I, {\sc Nguyen Quan Ba Hong}, hereby certify the outstanding academic qualification of Mr. {\sc Phan Vinh Tien}, a 3rd-year undergraduate student in Computer Science \& Information Technology, a Mathematics for Artificial Intelligence major, specialized in Data Science \& AIoT (Artificial Intelligence of Things), especially in developing Mathematics Methods for these modern research fields. Mr. {\sc Phan Vinh Tien} has several outstanding academic achievements, some of these in Mathematics- \& Informatics Olympiads are as follows:
\begin{enumerate}
    \item Special Prize (1st Prize in Analysis \& 1st Prize in Algebra, 1 of 6 students won Special Prize) at the 31st Vietnamese Mathematical Olympiad for College Students in Sai Gon University (SGU), Ho Chi Minh, Apr 2025.
    
    \item 2nd Prize in Analysis \& 3rd Prize in Algebra, at the 30th Vietnamese Mathematical Olympiad for College Students in Duy Tan University (DTU), Da Nang, Apr 2024.
    
    \item 2nd Prize in Analysis \& 3rd Prize in Algebra, at the 29th Vietnamese Mathematical Olympiad for College Students in University of Education (HUEdu), Hue, Apr 2023.
    
    \item 2nd Prize at the UMT Informatics Olympiad for College Students, 2024.
\end{enumerate}
\& other prizes \& achievements presented from when he was in high school, see details in his CV attached.

In research activities, Mr. {\sc Phan Vinh Tien} has been conducting some research topics as follows:
\begin{enumerate}
    \item Research in the field of Computer Vision with topic Emotion Recognition.
    
    \item Conducting Bachelor Thesis in Machine Learning, Deep Learning, \& Optimizations for Combinatorics \& Graph Theory. Advisors: Associate Professor {\sc Tran Dan Thu}, MSc. {\sc Nguyen Quan Ba Hong}, \& MEng. {\sc Huynh Le Phu Trung}.
\end{enumerate}
With these achievements \& positive attitudes in both learning \& researching, I confirm that Mr. {\sc Phan Vinh Tien} has enough academic qualifications to attend VIASM REU 2025 Summer School. If he receives this prestigious opportunity, he will be able to further his academic career \& then contribute to both Mathematics- \& Computer Science communities.

\begin{center}
    \LARGE
    \textbf{\textsf{Thư Giới Thiệu Cho Sinh Viên Đại Học}}
\end{center}
Với vai trò là 1 giảng viên về Toán học \& Khoa học Máy tính tại Trường Đại Học Quản lý {\it\&} Công nghệ Thành phố Hồ Chí Minh, tôi, {\sc Nguyễn Quản Bá Hồng} xác nhận các phẩm chất học tập \& nghiên cứu của sinh viên {\sc Phan Vĩnh Tiến}, 1 sinh viên năm 3 ngành Khoa Học Máy Tính \& Công Nghệ Thông Tin, chuyên về Toán \& Trí Tuệ Nhân Tạo, chuyên ngành Khoa Học Dữ Liệu \& AIoT (Artificial Intelligence of Things), đặc biệt trong việc phát triển Phương pháp Toán học cho các lĩnh vực nghiên cứu hiện đại này. {\sc Phan Vĩnh Tiến} có nhiều thành tích học tập nổi bật, một số trong số đó là trong các kỳ thi Olympic Toán học \& Tin học như sau:
\begin{enumerate}
    \item Giải Đặc Biệt (Giải Nhất môn Giải tích \& Giải Nhất môn Đại số, 1 trong 6 học sinh đạt Giải Đặc Biệt) tại Kỳ thi Olympic Toán học sinh viên toàn quốc lần thứ 31 tại Đại học Sài Gòn (SGU), Hồ Chí Minh, tháng 4 năm 2025.
    
    \item Giải Nhì môn Giải tích \& Giải Ba môn Đại số, tại Kỳ thi Olympic Toán học sinh viên toàn quốc lần thứ 30 tại Đại học Duy Tân (DTU), Đà Nẵng, tháng 4 năm 2024.
    
    \item Giải Nhì môn Giải tích \& Giải Ba môn Đại số, tại Kỳ thi Olympic Toán học sinh viên toàn quốc lần thứ 29 tại Đại học Sư Phạm (HUEdu), Huế, tháng 4 năm 2023.
    
    \item Giải Nhì Olympic Tin học UMT, năm 2024.
\end{enumerate}
\& các giải thưởng khác \& các thành tích đạt được khi còn học trung học, xem chi tiết trong CV đính kèm.

Trong hoạt động nghiên cứu, {\sc Phan Vĩnh Tiến} đã và đang thực hiện một số đề tài nghiên cứu như sau:
\begin{enumerate}
    \item Đang nghiên cứu trong lĩnh vực Thị giác Máy tính với chủ đề Nhận dạng cảm xúc.
    
    \item Đang thực hiện Luận văn Đại học về Học máy, Học sâu, \& Tối ưu hóa cho Tổ hợp \& Lý thuyết Đồ thị. Người hướng dẫn: PGS.TS. {\sc Trần Đan Thư}, ThS. {\sc Nguyễn Quản Bá Hồng}, \& ThS. {\sc Huỳnh Lê Phú Trung}.
\end{enumerate}
Với những thành tích \& thái độ tích cực trong cả học tập \& nghiên cứu, tôi khẳng định sinh viên {\sc Phan Vĩnh Tiến} có đủ phẩm chất về mặt học thuật để tham dự Trường hè VIASM REU 2025. Nếu nhận được cơ hội danh giá này, {\sc Tiến} sẽ có thể tiếp tục sự nghiệp học thuật của mình \& sau đó đóng góp cho cả cộng đồng Toán học \& Khoa học Máy tính.

%------------------------------------------------------------------------------%

\begin{flushright}    
    Mathematics \& Computer Science lecturer\\in Department of Artificial Intelligence {\it\&} Data Science (AI-DS),\\School of Technology (SOT),\\UMT University of Management {\it\&} Technology Ho Chi Minh City\\
    \vspace{2cm}
    ThS. {\sc Nguyễn Quản Bá Hồng}
\end{flushright}

%------------------------------------------------------------------------------%

\printbibliography[heading=bibintoc]

\end{document}