\documentclass{beamer}
\usetheme{Madrid}
\usecolortheme{default}
\usepackage[utf8]{vietnam}
\usepackage[backend=biber,natbib=true,style=alphabetic,maxbibnames=50]{biblatex}
\addbibresource{/home/nqbh/reference/bib.bib}
\usepackage{amsmath,amssymb,amsthm,enumitem,float,graphicx,mathtools,tikz}
\title{Discrete Mathematics for Computer Science\\Toán Rời Rạc Cho Khoa Học Máy Tính}
\author{\sc Nguyễn Quản Bá Hồng}
\institute{Present at UMT -- University of Management \& Technology of HCMC}
\logo{\includegraphics[height=5mm]{UMT_logo}}
\date{\today}

\begin{document}
\frame{\titlepage}
\begin{frame}
	\frametitle{Table of Contents}
	\tableofcontents
\end{frame}

\section{Introduction to Discrete Mathematics}

\begin{frame}
	\frametitle{Def: Discrete Mathematics}
	
	\begin{definition}[Discrete mathematics]
		\emph{Discrete mathematics}: study of countable, distinct, or separate mathematical structures.
	\end{definition}
	Cf. Finite Mathematics vs. Discrete Mathematics vs. ``Continuous Mathematics'', including e.g., Calculus, Mathematical Analysis.
	\vspace{2mm}
	
	\underline{Note}: Beyond the scope of this course: ``Discontinuous Mathematics''.
	
	\begin{example}[Pixel]
		Phones, computer monitors, televisions, modern screens, \& Disney cartoons, animated films for kids \& for adults, e.g., {\it Rick \& Morty} (2013--).
	\end{example}
\end{frame}

\begin{frame}
	\frametitle{Some Critical Thinking Questions}
	{\bf Targets{\tt/}Audiences.} Typical super-lazy mentally lost \& thus unmotivated undergraduate{\tt/}graduate students majored in Natural Science, especially in Mathematics, Information Technology, Computer Science, \& Engineering.
	\begin{block}{Some purpose-driven questions}
		\begin{itemize}\it
			\item[$\bullet$] Why do undergraduate or graduate students need to learn mathematics in general?
			\item[$\bullet$] Which type of mathematics do undergraduate or graduate students need to learn according to their majors?
			\item[$\bullet$] Why do CS-major students need to study Discrete Mathematics in particular?
		\end{itemize}
	\end{block}
\end{frame}

\begin{frame}
	\frametitle{Topics in Discrete Mathematics}
	\begin{itemize}
		\item[$\bullet$] {\bf Theoretical Computer Science}: areas relevant to computing, e.g., study of algorithms \& data structures, computability, complexity theory, automata theory, formal language theory, computational geometry, computer image analysis, etc. 
		\item[$\bullet$] {\bf Information Theory}: quantification of information, coding theory, analog signals{\tt/}coding{\tt/}encryption, etc.
		\item[$\bullet$] {\bf Mathematical Logic}: truth table, mathematical proof, automated theorem proving, formal verification of software, etc.
		\item[$\bullet$] {\bf Set Theory}: (naive set theory, not axiomatic one) main focus: countable (finite, infinitely countable) sets.
		\item[$\bullet$] {\bf Combinatorics}: enumerative combinatorics (counting problems), generating functions, partition theory, etc.
		
		Watch {\it The Man Who Knew Infinity} (2015).
	\end{itemize}
\end{frame}

\begin{frame}
	\frametitle{Topics in Discrete Mathematics}
	\begin{itemize}
		\item[$\bullet$] {\bf Graph Theory}: study of graphs \& networks, e.g., networks of communication, data organization, computational devices, flow of computation, etc.
		\item[$\bullet$] {\bf Number Theory}: study of properties of (integer) numbers, e.g., cryptography, cryptanalysis.
		\item[$\bullet$] {\bf Algebraic Structures}: discrete algebras, e.g., Boolean algebra used in logic gates \& programming; relational algebra used in databases, etc.
		\item[$\bullet$] {\bf Discrete Analogues of Continuous Mathematics}: discrete versions of continuous mathematics, e.g., discrete calculus, discrete probability theory, discrete optimization, discrete dynamical systems, etc.
		\item[$\bullet$] Others topics.
	\end{itemize}
\end{frame}

\begin{frame}
	\frametitle{Motivations}
	\begin{itemize}
		\item[$\bullet$] Learn Discrete Mathematics just for fun, to entertain yourself.
		
		\begin{example}[\href{https://www.imdb.com/title/tt0119217}{\it Good Will Hunting} (1997)]
			{\sc Will Hunting} learned History, Sociology, Psychology $\Psi$, Advanced Mathematics, Combinatorial Discrete Mathematics to flirt hot girls in bars, \& even Advanced Organic Chemistry for fun \& to help her girlfriend.
		\end{example}
		
		\item[$\bullet$] Learn ``just enough'' Discrete Mathematics to understand different branches of Computer Science.
		
		\underline{Main Goal}: Focus strongly on writing programs, developing software, \& building useful applications.
		\item[$\bullet$] If looking for {\color{red} research-oriented} jobs, especially {\color{red} Theoretical Computer Science}, then learn Discrete Mathematics much harder \& deeper.
		
		\underline{Main Goal}: Build some new useful theories, then find their theoretical- or practical real-world applications.
	\end{itemize}
\end{frame}

\begin{frame}
	\frametitle{References on Mathematics \& Computer Science}
	\begin{block}{On choosing Refs in general}\it
		How to choose ``right{\tt/}suitable'' references, e.g., online courses, books, lecture notes, expository notes, other learning materials, etc.?
	\end{block}
		
	[NQBH]'s Lecture Note on Discrete Mathematics \& beyond.
	\vspace{2mm}
	
	[Knu]${}^{\star\star}$ {\sc Donald Erwin Knuth}. {\it\color{red} The Art of Computer Programming}.
	\vspace{2mm}
	
	[GKP89]${}^\star$ {\sc Ronald L. Graham, Donald Erwin Knuth, Oren Patashmik}. {\it\color{blue} Concrete Mathematics: A Foundation for Computer Science}.
	\vspace{2mm}
	
	[Ros19] {\sc Kenneth H. Rosen}. {\it Discrete Mathematics \& Its Applications}.
	\vspace{2mm}
	
	[WR21] {\sc Ryan T. White, Archana Tikayat Ray}. {\it Practical Discrete Mathematics: Discover Math Principles that Fuel Algorithms for Computer Science \& Machine Learning with Python}.
	\vspace{2mm}
	
	[Lib23] {\sc David Liben-Nowell.} {\it Connecting Discrete Mathematics \& Computer Science}.
\end{frame}

\begin{frame}
	\frametitle{References on Pedagogy \& Psychology}
	\begin{block}{On learning \& teaching}\it
		How should we learn \& teach Discrete Mathematics in particular \& other types of Mathematics for Computer Science undergraduate students?
	\end{block}
	[Tru]: {\sc Giản Tư Trung}'s Hat-trick. ($+$ other books of IRED)
	\begin{itemize}
		\item[$\bullet$] {\it Đúng Việc: 1 Góc Nhìn Về Câu Chuyện Khai Minh}.
		\item[$\bullet$] {\it Sư Phạm Khai Phóng: Thế Giới, Việt Nam, \& Tôi}.
		\item[$\bullet$] {\it Quản Trị Bằng Văn Hóa: Cách Thức Kiến Tạo \& Tái Tạo Văn Hóa Tổ Chức}.
	\end{itemize}
	\vspace{2mm}
	
	[Pol14] {\sc George P\'olya}. {\it How To Solve It: A New Aspect of Mathematical Method}.
	\vspace{2mm}
	
	[GA08] {\sc Adam M. Grant, Susan J. Ashford}. {\it The dynamics of proactivity at work}. Research in Organizational Behaviors 28 (2008) 3--34.
\end{frame}

\section{Learning \& Teaching Methodologies}

\begin{frame}
	\frametitle{Teaching \& learning methodologies}
	{\bf Targets{\tt/}Criteria.} precision, robustness, creativity, usefulness, applicability, proactivity, valuable insight, deep comprehension, passion, novelty.
	\begin{block}{Some goal-driven rules in learning, teaching, \& research}
		(will be adjusted according to UMT IT Depart.'s objectives \& visions)
		\begin{itemize}
			\item[$\bullet$] Bonus points for proposing creative problems \&{\tt/}or solutions.
			\item[$\bullet$] Special points for projects combining Math $+$ CS ($+$ Physics, Chemistry, \&{\tt/}or Biology) much harder or more useful than lectures.
		\end{itemize}
	\end{block}
\end{frame}

\section{Applications of Discrete Mathematics}

\begin{frame}
	\frametitle{Combinatorics using {\tt SciPy}: Problems}
	\underline{Important Note}: Obviously, SciPy is not spicy at all like any chicken wings in {\it Hot Ones} show.
	
	Recall from Elementary Mathematics Grade 10{\tt/}combinatorics:
	
	\begin{problem}[Permutation, arrangement, combination]
		Given $n,k\in\mathbb{N}^\star$, $k\le n$. Write {\sf Pascal{\tt/}Python{\tt/}C{\tt/}C++} programs to compute the numbers of permutations $P_n$, of arrangements $A_n^k$, of combinations $C_n^k$.
	\end{problem}
	
	\begin{proof}[Solution]
		$P_n = n!,A_n^k = \dfrac{n!}{(n - k)!},C_n^k = \dfrac{n!}{k!(n - k)!}$. Run {\tt combinatorics.py}.		
	\end{proof}
\end{frame}

\begin{frame}
	\frametitle{Combinatorics using {\tt SciPy}: Problems}
	
	\begin{problem}[Pascal triangle \& Newton binomial expansion]
		Given $m,n\in\mathbb{N}^\star$. Write {\sf Pascal{\tt/}Python{\tt/}C{\tt/}C++} programs to print the 1st $n + 1$ lines of the Pascal triangle \& Newton binomial expansion of $(a + b)^n,(a + b + c)^n,\left(\sum_{i=1}^m a_i\right)^n$,  $\forall a,b,c,a_i\in\mathbb{R}$, $\forall i = 1,\ldots,m$.
	\end{problem}
	Recall from Elementary Mathematics Grade 6{\tt/}plane geometry:
	
	\begin{problem}[Count number of lines formed by some points]
		Write {\sf Pascal{\tt/}Python{\tt/}C{\tt/}C++} programs to count the number of lines formed by $n\in\mathbb{N}^\star$ distinguished points in (2D) plane.
	\end{problem}
	
	\begin{problem}[Count number of intersections formed by some lines]
		Write {\sf Pascal{\tt/}Python{\tt/}C{\tt/}C++} programs to count the number of intersections of $n\in\mathbb{N}^\star$ distinguished lines in (2D) plane.
	\end{problem}
\end{frame}

\begin{frame}
	\frametitle{Combinatorics using {\tt SciPy}: Hints \& Solutions}
	
	\begin{proof}[Solution]
		$C_n^2 - \sum_{i=1}^m C_{a_i}^2 + m = \dfrac{n(n - 1)}{2} - \sum_{i=1}^m \dfrac{a_i(a_i - 1)}{2} + m$ lines, where $n$ given points is partitioned into exactly $m\in\mathbb{N}$ disjoint subsets $A_i$ of collinear points, where $a_i\coloneqq|A_i| = {\rm card}\,A_i$, $\forall i = 1,\ldots,m$.
	\end{proof}
	
	\begin{proof}[Solution]
		Nếu trong $n$ đường thẳng đã cho có đúng $m\in\mathbb{N}$ bộ lần lượt gồm $a_1,\ldots,a_m$ đường thẳng song song đôi một \& $k\in\mathbb{N}$ bộ lần lượt gồm $b_1,\ldots,b_k$ đường thẳng đồng quy thì số giao điểm: $C_n^2 - \sum_{i=1}^m C_{a_i}^2 - \sum_{i=1}^m C_{b_i}^2 + k$ $= \dfrac{n(n - 1)}{2} - \sum_{i=1}^m \dfrac{a_i(a_i - 1)}{2} - \sum_{i=1}^k \dfrac{b_i(b_i - 1)}{2} + k$.
	\end{proof}
\end{frame}

\begin{frame}
	\frametitle{Further \& Beyond}
	More results, problems, \& practical applications of Discrete Mathematics in Number Theory, Graph Theory, Generating Functions, Discrete Probability Theory, Asymptotics, etc.
	
	\begin{block}{Discrete Mathematics vs. DL $\subset$ ML $\subset$ AI}\it
		How can Discrete Mathematics be useful in Artificial Intelligence (AI), Machine Learning (ML), \& Deep Learning (DL), especially Artificial Neural Networks (ANNs)?
	\end{block}	
\end{frame}

\begin{frame}
	\frametitle{Acknowledgment}
	\begin{center}\Large\sf\bf
		All types of feedback \& contributions are welcome.
		\vspace{1cm}
		
		\begin{block}{Thanks}
			Thank You for your valuable time \& attention.
		\end{block}
		\vspace{1cm}
		
		I appreciate all.
	\end{center}
\end{frame}

\end{document}