\subsection{Nguyên lý bù trừ}

\begin{dinhly}[Inclusion--exclusion principle]
	\label{thm: inclusion-exclusion: combinatorial version}
	For any $n\in\mathbb{N}^\star$, and any finite sets $A_1,\,\ldots,\,A_n$, one has the identity
	\begin{equation*}
		\left|\bigcup_{i=1}^n A_i\right| = \sum_{i=1}^n |A_i| - \sum_{1\le i < j\le n} |A_i\cap A_j| + \sum_{1\le i < j < k\le n} |A_i\cap A_j\cap A_k| - \cdots + (-1)^{n+1}|A_1\cap\cdots\cap A_n|,
	\end{equation*}
	which can be compactly written as
	\begin{equation*}
		\left|\bigcup_{i=1}^n A_i\right| = \sum_{k=1}^n (-1)^{k+1}\sum_{1\le i_1 < \cdots < i_k\le n} |A_{i_1}\cap\cdots\cap A_{i_k}|,
	\end{equation*}
	or
	\begin{equation*}
		\left|\bigcup_{i=1}^n A_i\right| = \sum_{\varnothing\ne J\subset\{1,\ldots,n\}} (-1)^{|J|+1} \left|\bigcap_{j\in J} A_j\right|.
	\end{equation*}
\end{dinhly}

\begin{dinhly}[Inclusion--exclusion principle in probability]
	\label{thm: inclusion-exclusion: probabilistic version}
	For any $n\in\mathbb{N}^\star$, and for any events $A_1,\,\ldots,\,A_n$ in a \href{https://en.wikipedia.org/wiki/Probability_space}{probability space} $(\Omega,\,{\cal F},\,\mathbb{P})$, in general
	\begin{align*}
		\mathbb{P}\left(\bigcup_{i=1}^n A_i\right) = \sum_{i=1}^n \mathbb{P}(A_i) - \sum_{1 \leq i < j \leq n} \mathbb{P}(A_i\cap A_j) + \sum_{1 \leq i < j < k \leq n} \mathbb{P}(A_i\cap A_j\cap A_k) + \cdots + (-1)^{n-1}\mathbb{P}\left(\bigcap_{i=1}^n A_i\right),
	\end{align*}
	which can be written in closed form as
	\begin{equation}
		\label{inclusion-exclusion: probabilistic version}
		\mathbb{P}\left(\bigcup_{i=1}^n A_i\right) = \sum_{k=1}^n (-1)^{k-1}\sum_{I\subset[n],\,|I| = k} \mathbb{P}(A_I),
	\end{equation}
	where the last sum runs over all subsets $I$ of the indices $1,\,\ldots,\,n$ which contain exactly $k$ elements, and $\displaystyle A_I\coloneqq\bigcap_{i\in I} A_i$ denotes the intersection of all those $A_i$ with index in $I$.
	
	In particular, if the probability of the intersection $A_I$ only depends on the cardinality of $I$, i.e., for every $k\in[n]$, there is an $a_k$ s.t. $a_k = \mathbb{P}(A_I)$ for every $I\in[n]$ with $|I| = k$, then \eqref{inclusion-exclusion: probabilistic version} simplifies to
	\begin{equation*}
		\mathbb{P}\left(\bigcup_{i=1}^n A_i\right) = \sum_{k=1}^n (-1)^{k-1}\binom{n}{k}a_k.
	\end{equation*}
	In addition, if the events $A_i$ are \href{https://en.wikipedia.org/wiki/Independent_and_identically_distributed}{independent \& identically distributed} (i.i.d.), then $\mathbb{P}(A_i) = p$, $\forall i$, and $a_k = p^k$, hence
	\begin{equation*}
		\mathbb{P}\left(\bigcup_{i=1}^n A_i\right) = 1 - (1 - p)^n.
	\end{equation*}
\end{dinhly}

\begin{tcolorbox}[breakable]
    \begin{baitoan}
        \label{pb:w01:01}
        Với $n \in \mathbb{Z^+}$, xét $A_i\,(i = 1,\,2,\,\ldots,\,n)$ là $n$ tập hợp hữu hạn bất kỳ. 
        
        \begin{enumerate}
            \item[(a)] Chứng minh rằng $$\left|\bigcup\limits_{i = 1}^n A_i\right| = \sum\limits_{\substack{T \subseteq \{1,\,2,\,\ldots,\,n\} \\ T \ne \varnothing}} (-1)^{|T| + 1} \left|\bigcap\limits_{i \in T} A_i\right|.$$
            \item[(b)] Chứng minh rằng khi lấy tổng $m < n$ hạng tử đầu tiên của vế phải, nếu $m$ chẵn thì ta có chặn dưới và nếu $m$ lẻ thì ta có chặn trên của $\displaystyle \left|\bigcup\limits_{i = 1}^n A_i\right|$.
        \end{enumerate}
    \end{baitoan}
\end{tcolorbox}

\textbf{Lời giải. }

\begin{enumerate}
    \item[(a)] {
        Trường hợp $n = 1,\,2$, đẳng thức dễ dàng chứng minh.

        Giả sử đẳng thức đúng đến $n = N$, tức ta đã có $$\left|\bigcup\limits_{i = 1}^N A_i\right| = \sum\limits_{\substack{T \subseteq \{1,\,2,\,\ldots,\,N\} \\ T \ne \varnothing}} (-1)^{|T| + 1} \left|\bigcap\limits_{i \in T} A_i\right|.$$

        Ta cần chứng minh đẳng thức cũng đúng với $n = N + 1$, tức cần chứng minh $$\left|\bigcup\limits_{i = 1}^{N+1} A_i\right| = \sum\limits_{\substack{T \subseteq \{1,\,2,\,\ldots,\,N+1\} \\ T \ne \varnothing}} (-1)^{|T| + 1} \left|\bigcap\limits_{i \in T} A_i\right|.$$

        Thật vậy, áp dụng $|A \cup B| = |A| + |B| - |A \cap B|$ và giả thiết quy nạp, ta được
        \begin{align*}
            \left|\bigcup\limits_{i = 1}^{N+1} A_i\right|
            &= \left|\bigcup\limits_{i = 1}^{N} A_i\right| + \left|A_{N+1}\right| - \left|A_{N+1} \cap \left(\bigcup\limits_{i=1}^N A_i\right)\right| \\
            &= \sum\limits_{\substack{T \subseteq \{1,\,2,\,\ldots,\,N\} \\ T \ne \varnothing}} (-1)^{|T| + 1} \left|\bigcap\limits_{i \in T} A_i\right| + \left|A_{N+1}\right| - \left|\bigcup\limits_{i=1}^N\left(A_{N+1} \cap A_i\right)\right|
        \end{align*}

        Nhận xét rằng tập con khác rỗng của $\{1,\,2,\,\ldots,\,N+1\}$ gồm hai loại: 
        \begin{enumerate}
            \item[$\bullet$] các tập con của $\{1,\,2,\,\ldots,\,N\}$, ký hiệu $S_1,\,S_2,\,\ldots,\,S_m$ (không có phần tử thứ $N+1$);
            \item[$\bullet$] tập $\{N+1\}$ và các tập $\{N+1\} \cup S_i$ (có phần tử thứ $N+1$).
        \end{enumerate}

        Suy ra 
        $$\sum\limits_{\substack{T \subseteq \{1,\,2,\,\ldots,\,N+1\} \\ T \ne \varnothing}} (-1)^{|T| + 1} \left|\bigcap\limits_{i \in T} A_i\right| = \sum\limits_{\substack{T \subseteq \{1,\,2,\,\ldots,\,N\} \\ T \ne \varnothing}} (-1)^{|T| + 1} \left|\bigcap\limits_{i \in T} A_i\right| + \left|A_{N+1}\right| + \sum\limits_{\substack{T \subseteq \{N+1\} \cup S_i \\ T \ne \varnothing}} (-1)^{|T| + 1} \left|\bigcap\limits_{i \in T} A_i\right|.$$

        Như vậy ta chỉ cần chứng minh 
        $$\left|\bigcup\limits_{i=1}^N\left(A_{N+1} \cap A_i\right)\right| = -\left(\sum\limits_{\substack{T \subseteq \{N+1\} \cup S_i \\ T \ne \varnothing}} (-1)^{|T| + 1} \left|\bigcap\limits_{i \in T} A_i\right|\right).$$

        Thật vậy, tiếp tục áp dụng giả thiết quy nạp cho vế trái ta được
        \begin{align*}
            \left|\bigcup\limits_{i=1}^N\left(A_{N+1} \cap A_i\right)\right|
            &= \sum\limits_{\substack{T \subseteq \{1,\,2,\,\ldots,\,N\} \\ T \ne \varnothing}} (-1)^{|T| + 1} \left|\bigcap\limits_{i \in T} \left(A_{N+1} \cap A_i\right)\right| \\
            &= -\left(\sum\limits_{\substack{T \subseteq \{1,\,2,\,\ldots,\,N\} \\ T \ne \varnothing}} (-1)^{|T| + 2} \left|\left(\bigcap\limits_{i \in T} A_i\right) \cap A_{N+1}\right|\right) \\
            &= -\left(\sum\limits_{\substack{T' \subseteq \{N+1\} \cup S_i \\ T' \ne \varnothing}} (-1)^{|T'| + 1} \left|\bigcap\limits_{i \in T'} A_i\right|\right).
        \end{align*}

        Như vậy đẳng thức cũng đúng với $n = N+1$. Theo nguyên lý quy nạp toán học, ta có điều phải chứng minh.
    }
    \item[(b)] {
        Đặt $\displaystyle P(n,\,k) = (-1)^{k+1}\sum\limits_{1 \leq i_1 < \cdots < i_k \leq n} \left|\bigcap\limits_{j=1}^k A_{i_j} \right|$ đẳng thức vừa chứng minh có thể được viết lại 

        $$\left|\bigcup\limits_{i = 1}^n A_i\right| = \sum\limits_{i = 1}^n (-1)^{k+1} \sum\limits_{1 \leq i_1 < \cdots < i_k \leq n} \left|\bigcap\limits_{j=1}^k A_{i_j} \right| = \sum\limits_{k = 1}^n P(n,\,k).$$

        Ta sẽ chứng minh với mọi $m \in \mathbb{Z^+},\,m < n$ thì 

        $$\left|\bigcup\limits_{i = 1}^n A_i\right| \leq \sum\limits_{k = 1}^m P(n,\,k) \text{ nếu $m$ lẻ và }\left|\bigcup\limits_{i = 1}^n A_i\right| \geq \sum\limits_{k = 1}^m P(n,\,k) \text{ nếu $m$ chẵn}$$ bằng phương pháp quy nạp toán học.

        Trường hợp $n = 1$, mệnh đề hiển nhiên đúng. Giả sử mệnh đề đúng đến $n = N$, tức ta đã có 
        $$\left|\bigcup\limits_{i = 1}^N A_i\right| \leq \sum\limits_{k = 1}^m P(N,\,k) \text{ nếu $m$ lẻ và }\left|\bigcup\limits_{i = 1}^N A_i\right| \geq \sum\limits_{k = 1}^m P(N,\,k) \text{ nếu $m$ chẵn}.$$

        Ta cần phải chứng minh mệnh đề cũng đúng với $n = N+1$, tức cần chứng minh 
        $$\left|\bigcup\limits_{i = 1}^{N+1} A_i\right| \leq \sum\limits_{k = 1}^m P(N+1,\,k) \text{ nếu $m$ lẻ và }\left|\bigcup\limits_{i = 1}^{N+1} A_i\right| \geq \sum\limits_{k = 1}^m P(N+1,\,k) \text{ nếu $m$ chẵn}.$$

        Thật vậy, áp dụng $|A \cup B| = |A| + |B| - |A \cap B|$, ta được 
        $$\left|\bigcup\limits_{i = 1}^{N+1} A_i\right| = \left|A_{N+1}\right| + \left|\bigcup\limits_{i = 1}^{N} A_i\right| - \left|\bigcup\limits_{i = 1}^{N} (A_{N+1} \cap A_i)\right|.$$

        Xét trường hợp $m$ chẵn, trường hợp $m$ lẻ việc chứng minh hoàn toàn tương tự. Khi $m$ chẵn thì $m-1$ lẻ, áp dụng giả thiết quy nạp ta có 
        $$\left|\bigcup\limits_{i = 1}^{N} A_i\right| \geq \sum\limits_{k = 1}^m P(N,\,k) \text{ và } \left|\bigcup\limits_{i = 1}^{N} (A_{N+1} \cap A_i)\right| \leq \sum\limits_{k = 1}^{m-1} Q(N,\,k),$$
        với $\displaystyle Q(N,\,k) = (-1)^{k+1}\sum\limits_{1 \leq i_1 < \cdots < i_k \leq N} \left|\bigcap\limits_{j=1}^k \left(A_{N+1} \cap A_{i_j}\right)\right| = \displaystyle (-1)^{k+1}\sum\limits_{1 \leq i_1 < \cdots < i_k \leq N} \left|A_{N+1} \cap \left(\bigcap\limits_{j=1}^k A_{i_j}\right)\right|$.

        Suy ra $$\left|\bigcup\limits_{i = 1}^{N+1} A_i\right| \geq \left|A_{N+1}\right| + \sum\limits_{k = 1}^m P(N,\,k) - \sum\limits_{k = 1}^{m-1} Q(N,\,k) = \left|A_{N+1}\right| + P(N,\,1) + \sum\limits_{k = 2}^m\left(P(N,\,k) - Q(N,\,k-1)\right).$$

        Mặt khác
        \begin{align*}
            P(N+1,\,k) 
            &= (-1)^{k+1}\sum\limits_{1 \leq i_1 < \cdots < i_k \leq N+1} \left|\bigcap\limits_{j=1}^k A_{i_j} \right| \\
            &= (-1)^{k+1} \left(\left(\sum\limits_{1 \leq i_1 < \cdots < i_k \leq N} + \sum\limits_{1 \leq i_1 < \cdots < i_k = N+1}\right)\left|\bigcap\limits_{j=1}^k A_{i_j} \right|\right) \\
            &= (-1)^{k+1}\sum\limits_{1 \leq i_1 < \cdots < i_k \leq N}\left|\bigcap\limits_{j=1}^k A_{i_j} \right| + (-1)^{k+1}\sum\limits_{1 \leq i_1 < \cdots < i_{k-1} \leq N}\left|\left(\bigcap\limits_{j=1}^{k-1} A_{i_j}\right) \cap A_{N+1} \right| \\
            &= P(N,\,k) - Q(N,\,k-1)
        \end{align*} và 
        $$\left|A_{N+1}\right| + \displaystyle P(N,\,1) = \left|A_{N+1}\right| + \sum\limits_{1 \leq i \leq N} \left|A_{i}\right| = \sum\limits_{1 \leq i \leq N+1} \left|A_{i}\right| = P(N+1,\,1).$$

        Do đó $\displaystyle \left|\bigcup\limits_{i = 1}^{N+1} A_i\right| \geq P(N+1,\,1) + \sum\limits_{k = 2}^m P(N+1,\,k) = \sum\limits_{k = 1}^m P(N+1,\,k)$. Như vậy đối với trường hợp $m$ chẵn, mệnh đề cũng đúng với $n = N+1$. Theo nguyên lý quy nạp toán học, mệnh đề đúng với mọi $n$ nguyên dương. Hoàn tất chứng minh.
    }
\end{enumerate}
