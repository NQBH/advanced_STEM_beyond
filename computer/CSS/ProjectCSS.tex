\documentclass[a4paper,oneside]{book}
\usepackage{longtable,float,hyperref,color,amsmath,amsxtra,amssymb,latexsym,amscd,amsthm,amsfonts,graphicx}
\numberwithin{equation}{chapter}
\allowdisplaybreaks
\usepackage{fancyhdr}
\pagestyle{fancy}
\fancyhf{}
\fancyhead[RE,LO]{\footnotesize \textsc \leftmark}
\cfoot{\thepage}
\renewcommand{\headrulewidth}{0.5pt}
\setcounter{tocdepth}{3}
\setcounter{secnumdepth}{3}
\usepackage{imakeidx}
\makeindex[columns=2, title=Alphabetical Index, 
           options= -s index.ist]
\title{\Huge Project CSS}
\author{\textsc{Nguyen Quan Ba Hong}\\
{\small Students at Faculty of Math and Computer Science,}\\ 
{\small Ho Chi Minh University of Science, Vietnam} \\
{\small \texttt{email. nguyenquanbahong@gmail.com}}\\
{\small \texttt{blog. \url{www.nguyenquanbahong.com}} 
\footnote{Copyright \copyright\ 2016 by Nguyen Quan Ba Hong, Student at Ho Chi Minh University of Science, Vietnam. This document may be copied freely for the purposes of education and non-commercial research. Visit my site \texttt{\url{www.nguyenquanbahong.com}} to get more.}}}
\begin{document}
\frontmatter
\maketitle
\tableofcontents
\listoftables


\mainmatter
\chapter{CSS Tutorial}
CSS is used to control the style of a web document in a simple and easy way. CSS is the acronym for ``Cascading Style Sheet''. 
\section{Introduction}
\subsection{What is CSS?}
Cascading Style Sheets, fondly referred to as CSS, is a simple design language intended to simplify the process of making web pages presentable.

CSS handles the look and feel part of a web page. Using CSS, you can control the color of the text, the style of fonts, the spacing between paragraphs, how columns are sized and laid out, what background images or colors are used, layout designs, variations in display for different devices and screen sizes as well as a variety of other effects. 

CSS is easy to learn and understand but it provides powerful control over the presentation of an HTML document. Most commonly, CSS is combined with the markup languages HTML or XHTML.
\subsection{Advantages of CSS}
\begin{enumerate}
\item \textit{CSS saves time.} You can write CSS once and then reuse same sheet in multiple HTML pages. You can define a style for each HTML element and apply it to as many Web pages as you want.
\item \textit{Pages load faster.} If you are using CSS, you do not need to write HTML tag attributes every time. Just write one CSS rule of a tag and apply it to all the occurrences of that tag. SO less code means faster download times.
\item \textit{Easy maintenance.} To make a global change, simply change the style, and all elements in all the web pages will be updated automatically.
\item \textit{Superior styles to HTML.} CSS has a much wider array of attributes than HTML, so you can give a far better look to your HTML page in comparison to HTML attributes.
\item \textit{Multiple Device Compatibility.} Style sheets allow content to be optimized for more than one type of device. By using the same HTML document, different versions of a website can be presented for handheld devices such as PDAs and cell phones for printing.
\item \textit{Global web standards.} Now HTML attributes are being deprecated and it is being recommended to use CSS. So it is a good idea to start using CSS in all the HTML pages to make them compatible to future browsers.
\item \textit{Offline Browsing.} CSS can store web applications locally with the help of an offline cache. Using of this, we can view offline websites. The cache also ensures faster loading and better overall performance of the website.
\item \textit{Platform Independence.} The Script offer consistent platform independence and can support latest browsers as well.
\end{enumerate}
\subsection{Who Creates and Maintains CSS?}
CSS was invited by H\o{a}kon Wium Lie on October 10, 1994 and maintained through a group of people within the W3C called the CSS Working Group. The CSS Working Group creates documents called \textit{specifications}. When a specification has been discussed and officially ratified by W3C members, it becomes a recommendation.

These ratified specifications are called recommendations because the W3C has no control over the actual implementation of the language. Independent companies and organizations create that software.\\
\\
\textbf{Note 1.1.} The World Wide Web Consortium, or W3C is a group that makes recommendations about how the Internet works and how it should evolve.
\subsection{CSS Versions}
Cascading Style Sheets, level 1 (CSS1) was came out of W3C as a recommendation in December 1996. This version describes the CSS language as well as a simple visual formatting model for all the HTML tags.

CSS2 was became a W3C recommendation in May 1998 and builds on CSS1. This version adds support for media-specific style sheets e.g. printers and aural devices, downloadable fonts, element positioning and tables.

CSS3 was became a W3C recommendation in June 1999 and builds on older versions CSS. it has divided into documentations is called as Modules and here each module having new extension features defined in CSS2.
\subsubsection{CSS3 Modules}
CSS3 Modules are having old CSS specifications as well as extension features.
\begin{enumerate}
\item Selectors
\item Box Model
\item Backgrounds and Borders
\item Images Values and Replaced Content
\item Text Effect
\item 2D/3D Transformations
\item Animations
\item Multiple Column Layout
\item User Interface
\end{enumerate}
\section{Syntax}
A CSS comprises of style rules that are interpreted by the browser and then applied to the corresponding elements in your document. A style rule is made of three parts.
\begin{enumerate}
\item \textit{Selector.} A selector is an HTML tag at which a style will be applied. This could be any tag like $<$h1$>$ or $<$table$>$ etc.
\item \textit{Property.} A property is a type of attribute of HTML tag. Put simply, all the HTML attributes are converted into CSS properties. They could be color, border etc.
\item \textit{Value. }V alues are assigned to properties. For example, color property can have value either red or \#F1F1F1 etc.
\end{enumerate}
You can put CSS Style Rule Syntax as follows
\begin{verbatim}
selector { property: value }
\end{verbatim}
\textbf{Example 1.2.} You can define a table border as follows
\begin{verbatim}
table{ border :1px solid #C00; }
\end{verbatim}
Here \texttt{table} is a selector and \texttt{border} is a property and given value \texttt{1px solid \#C00} is the value of that property.

You can define selectors in various simple ways based on your comfort. Let me put these selectors one by one.
\subsection{The Type Selectors}
This is the same selector we have seen above. Again, one more example to give a color to all level 1 headings.
\begin{verbatim}
h1 {
   color: #36CFFF; 
}
\end{verbatim}
\subsection{The Universal Selectors}
Rather than selecting elements of a specific type, the universal selector quite simply matches the name of any element type
\begin{verbatim}
* { 
   color: #000000; 
}
\end{verbatim}
This rule renders the content of every element in our document in black.
\subsection{The Descendant Selectors}
Suppose you want to apply a style rule to a particular element only when it lies inside a particular element. As given in the following example, style rule will apply to $<$em$>$ element only when it lies inside $<$ul$>$ tag.
\begin{verbatim}
ul em {
   color: #000000; 
}
\end{verbatim}
\subsection{The Class Selectors}
You can define style rules based on the class attribute of the elements. All the elements having that class will be formatted according to the defined rule.
\begin{verbatim}
.black {
   color: #000000; 
}
\end{verbatim}
This rule renders the content in black for every element with class attribute set to black in our document. You can make it a bit more particular. For example
\begin{verbatim}
h1.black {
   color: #000000; 
}
\end{verbatim}
This rule renders the content in black for only $<$h1$>$ elements with class attribute set to \texttt{black}.

You can apply more than one class selectors to given element. Consider the following example
\begin{verbatim}
<p class="center bold">
   This para will be styled by the classes center and bold.
</p>
\end{verbatim}
\subsection{The ID Selectors}
You can define style rules based on the \textit{id} attribute of the elements. All the elements having that \textit{id} will be formatted according to the defined rule.
\begin{verbatim}
#black {
   color: #000000; 
}
\end{verbatim}
This rule renders the content in black for every element with \textit{id} attribute set to black in our document. You can make it a bit more particular. For example
\begin{verbatim}
h1#black {
   color: #000000; 
}
\end{verbatim}
This rule renders the content in black for only $<$h1$>$ elements with \textit{id} attribute set to \textit{black}. 

The true power of id selectors is when they are used as the foundation for descendant selectors, For example
\begin{verbatim}
#black h2 {
   color: #000000; 
}
\end{verbatim}
In this example all level 2 headings will be displayed in black color when those headings will lie with in tags having \textit{id} attribute set to \textit{black}.
\subsection{The Child Selectors}
You have seen the descendant selectors. There is one more type of selector, which is very similar to descendants but have different functionality. Consider the following example
\begin{verbatim}
body > p {
   color: #000000; 
}
\end{verbatim}
This rule will render all the paragraphs in black if they are direct child of $<$body$>$ element. Other paragraphs put inside other elements like $<$div$>$ or $<$td$>$ would not have any effect of this rule.
\subsection{The Attribute Selectors}
You can also apply styles to HTML elements with particular attributes. The style rule below will match all the input elements having a type attribute with a value of \textit{text}
\begin{verbatim}
input[type = "text"]{
   color: #000000; 
}
\end{verbatim}
The advantage to this method is that the \verb|<input type = "submit" />| element is unaffected, and the color applied only to the desired text fields.

There are following rules applied to attribute selector.
\begin{enumerate}
\item \verb|p[lang]| Selects all paragraph elements with a \textit{lang} attribute.
\item \verb|p[lang="fr"]| Selects all paragraph elements whose \textit{lang} attribute has a value of exactly \verb|"fr"|.
\item \verb|p[lang~="fr"]| Selects all paragraph elements whose \textit{lang} attribute contains the word \verb|"fr"|.
\item \verb#p[lang|="en"]# Selects all paragraph elements whose \textit{lang} attribute contains values that are exactly \verb|"en"|, or begin with \verb|"en-"|.
\end{enumerate}
\subsection{Multiple Style Rules}
You may need to define multiple style rules for a single element. You can define these rules to combine multiple properties and corresponding values into a single block as defined in the following example
\begin{verbatim}
h1 {
   color: #36C;
   font-weight: normal;
   letter-spacing: .4em;
   margin-bottom: 1em;
   text-transform: lowercase;
}
\end{verbatim}
Here all the property and value pairs are separated by a \textit{semi colon (;)}. You can keep them in a single line or multiple lines. For better readability we keep them into separate lines.
\subsection{Grouping Selectors}
You can apply a style to many selectors if you like. Just separate the selectors with a comma, as given in the following example
\begin{verbatim}
h1, h2, h3 {
   color: #36C;
   font-weight: normal;
   letter-spacing: .4em;
   margin-bottom: 1em;
   text-transform: lowercase;
}
\end{verbatim}
This define style rule will be applicable to \texttt{h1, h2} and \texttt{h3} element as well. The order of the list is irrelevant. All the elements in the selector will have the corresponding declarations applied to them.

You can combine the various id selectors together as shown below
\begin{verbatim}
#content, #footer, #supplement {
   position: absolute;
   left: 510px;
   width: 200px;
}
\end{verbatim}
\section{Inclusion}
There are four ways to associate styles with your HTML document. Most commonly used methods are inline CSS and External CSS.
\subsection{Embedded CSS - The $<$style$>$ Element}
You can put your CSS rules into an HTML document using the <style> element. This tag is placed inside <head>...</head> tags. Rules defined using this syntax will be applied to all the elements available in the document. Here is the generic syntax.

Following is the example of embed CSS based on the above syntax.
\begin{verbatim}
<!DOCTYPE html>
<html>
   <head>
   
      <style type = "text/css" media = "all">
         body {
            background-color: linen;
         }
         h1 {
            color: maroon;
            margin-left: 40px;
         }
      </style>
      
   </head>   
   <body>
      <h1>This is a heading</h1>
      <p>This is a paragraph.</p>
   </body>
</html>
\end{verbatim}
\textbf{Attributes 1.3.} Attributes associated with $<$style$>$ elements are 
\begin{center}
\begin{longtable}{|c|p{3cm}|p{6cm}|}
\hline
\textbf{Attribute} & \textbf{Value} & \textbf{Description}\\
\hline
type & text/css & Specifies the style sheet language as a content-type (MIME type). This is required attribute.\\
\hline
media & screen, tty, tv, projection, handheld, print, braille, aural, all & Specifies the device the document will be displayed on. Default value is \textit{all}. This is an optional attribute.\\
\hline
\caption{Attributes associated with $<$style$>$ elements.}
\end{longtable}
\end{center}
\subsection{Inline CSS - The \textit{style} Attribute}
You can use style attribute of any HTML element to define style rules. These rules will be applied to that element only. Here is the generic syntax
\begin{verbatim}
<element style = "...style rules....">
\end{verbatim}
\textbf{Attributes 1.4.}
\begin{center}
\begin{longtable}{|c|c|p{8cm}|}
\hline
\textbf{Attribute} & \textbf{Value} & \textbf{Description}\\
\hline
style & style rule & The value of \textit{style} attribute is a combination of style declarations separated by semicolon (;).\\
\hline
\caption{Attribute.} 
\end{longtable}
\end{center}
\textbf{Example 1.5.} Following is the example of inline CSS based on the above syntax 
\begin{verbatim}
<html>
   <head>
   </head>
   <body>
      <h1 style = "color:#36C;"> This is inline CSS </h1>
   </body>
</html>
\end{verbatim}
\subsection{External CSS - The $<$link$>$ Element}
The $<$link$>$ element can be used to include an external stylesheet file in your HTML document.

An external style sheet is a separate text file with \texttt{.css} extension. You define all the Style rules within this text file and then you can include this file in any HTML document using $<$link$>$ element.

Here is the generic syntax of including external CSS file.
\begin{verbatim}
<head>
   <link type = "text/css" href = "..." media = "..." />
</head>
\end{verbatim}
\textbf{Attributes 1.6.} Attributes associated with $<$style$>$ elements are
\begin{center}
\begin{longtable}{|c|p{3cm}|p{6cm}|}
\hline
\textbf{Attribute} & \textbf{Value} & \textbf{Description}\\
\hline
type & text/css & Specifies the style sheet language as a content-type (MIME type). This attribute is required.\\
\hline
href & URL & Specifies the style sheet file having Style rules. This attribute is a required.\\
\hline
media & screen, tty, tv, projection, handheld, print, braille, aural, all & Specifies the device the document will be displayed on. Default value is \textit{all}. This is optional attribute.\\
\hline
\caption{Attribute.} 
\end{longtable}
\end{center}
\textbf{Example 1.7.} Consider a simple style sheet file with a name \texttt{mystyle.css} having the following rules.
\begin{verbatim}
h1, h2, h3 {
   color: #36C;
   font-weight: normal;
   letter-spacing: .4em;
   margin-bottom: 1em;
   text-transform: lowercase;
}
\end{verbatim}
Now you can include this file mystyle.css in any HTML document as follows
\begin{verbatim}
<head>
   <link type = "text/css" href = "mystyle.css" media = " all" />
</head>
\end{verbatim}
\subsection{Imported CSS - @import Rule}
@import is used to import an external stylesheet in a manner similar to the $<$link$>$ element. Here is the generic syntax of @import rule.
\begin{verbatim}
<head>
   <@import "URL";
</head>
\end{verbatim}
Here URL is the URL of the style sheet file having style rules. You can use another syntax as well
\begin{verbatim}
<head>
   <@import url("URL");
</head>
\end{verbatim}
\textbf{Example 1.8.} Following is the example showing you how to import a style sheet file into HTML document
\begin{verbatim}
<head>
   @import "mystyle.css";
</head>
\end{verbatim}
\subsection{CSS Rules Overriding}
We have discussed four ways to include style sheet rules in a an HTML document. Here is the rule to override any Style Sheet Rule.
\begin{enumerate}
\item Any inline style sheet takes highest priority. So, it will override any rule defined in $<$style$>$...$<$/style$>$ tags or rules defined in any external style sheet file.
\item Any rule defined in $<$style$>$...$<$/style$>$ tags will override rules defined in any external style sheet file.
\item Any rule defined in external style sheet file takes lowest priority, and rules defined in this file will be applied only when above two rules are not applicable.
\end{enumerate}
\subsection{Handling old Browsers}
There are still many old browsers who do not support CSS. So, we should take care while writing our Embedded CSS in an HTML document. The following snippet shows how you can use comment tags to hide CSS from older browsers.
\begin{verbatim}
<style type="text/css">
   <!--
      body, td {
         color: blue;
      }
   -->
</style>
\end{verbatim}
\subsection{CSS Comments}
Many times, you may need to put additional comments in your style sheet blocks. So, it is very easy to comment any part in style sheet. You can simple put your comments inside /*.....this is a comment in style sheet.....*/.

You can use /* ....*/ to comment multi-line blocks in similar way you do in C and C++ programming languages.\\
\\
\textbf{Example 1.9.} 
\begin{verbatim}
<!DOCTYPE html>
<html>
   <head>
      <style>
         p {
            color: red;
            /* This is a single-line comment */
            text-align: center;
         }
         /* This is a multi-line comment */
      </style>
   </head>
   <body>
      <p>Hello World!</p>
   </body>
</html>
\end{verbatim}
\section{Measurement Units}
CSS supports a number of measurements including absolute units such as inches, centimeters, points, and so on, as well as relative measures such as percentages and em units. You need these values while specifying various measurements in your Style rules e.g \verb|border = "1px solid red"|.

We have listed out all the CSS Measurement Units along with proper Examples
\begin{center}
\begin{longtable}{|c|p{6cm}|p{4cm}|}
\hline
\textbf{Unit} & \textbf{Description} & \textbf{Example}\\
\hline
\% & Defines a measurement as a percentage relative to another value, typically an enclosing element. & p \{font-size: 16pt; line-height: 125\%;\} \\
\hline
cm & Defines a measurement in centimeters. & div \{margin-bottom: 2cm;\}\\
\hline
em & A relative measurement for the height of a font in em spaces. Because an em unit is equivalent to the size of a given font, if you assign a font to 12pt, each "em" unit would be 12pt; thus, 2em would be 24pt. & p \{letter-spacing: 7em;\}\\
\hline
ex & This value defines a measurement relative to a font's x-height. The x-height is determined by the height of the font's lowercase letter \textit{x}.& p \{font-size: 24pt; line-height: 3ex;\}\\
\hline
in & Defines a measurement in inches. & p \{word-spacing: .15in;\}\\
\hline
mm & Defines a measurement in millimeters. & p \{word-spacing: 15mm;\}\\
\hline
pc & Defines a measurement in picas. A pica is equivalent to 12 points; thus, there are 6 picas per inch. & p \{font-size: 20pc;\}\\
\hline
pt & Defines a measurement in points. A point is defined as 1/72nd of an inch. & body \{font-size: 18pt;\}\\
\hline
px & Defines a measurement in screen pixels. & p \{padding: 25px;\}\\
\hline
vh & 1\% of viewport height. & h2 \{ font-size: 3.0vh; \}\\
\hline
vw & 1\% of viewport width. & h1 \{ font-size: 5.9vw; \}\\
\hline
vmin & 1vw or 1vh, whichever is smaller. & p \{ font-size: 2vmin;\}\\
\hline
\caption{All the CSS Measurement Units.} 
\end{longtable}
\end{center}
\section{Colors}
CSS uses color values to specify a color. Typically, these are used to set a color either for the foreground of an element (i.e., its text) or else for the background of the element. They can also be used to affect the color of borders and other decorative effects.

You can specify your color values in various formats. Following table lists all the possible formats.
\begin{center}
\begin{longtable}{|c|l|l|}
\hline
\textbf{Format} & \textbf{Syntax} & \textbf{Example}\\
\hline
Hex Code & \verb|#RRGGBB| & \verb|p{color:#FF0000;}|\\
\hline
Short Hex Code & \verb|#RGB| & \verb|p{color:#6A7;}|\\
\hline
RGB \% & rgb(rrr\%,ggg\%,bbb\%) & p\{color:rgb(50\%,50\%,50\%);\}\\
\hline
RGB Absolute & rgb(rrr,ggg,bbb) & p\{color:rgb(0,0,255);\}\\
\hline
keyword & aqua, black, etc. & p\{color:teal;\}\\
\hline
\caption{Color values in all the possible formats.} 
\end{longtable}
\end{center}
These formats are explained in more detail in the following sections.
\subsection{Hex Codes}
A hexadecimal is a 6 digit representation of a color. The first two digits(RR) represent a red value, the next two are a green value(GG), and the last are the blue value(BB).

A hexadecimal value can be taken from any graphics software like Adobe Photoshop, Jasc Paintshop Pro, or even using Advanced Paint Brush.
\subsection{Short Hex Codes}
This is a shorter form of the six-digit notation. In this format, each digit is replicated to arrive at an equivalent six-digit value. For example: \verb|#6A7| becomes \verb|#66AA77|.

A hexadecimal value can be taken from any graphics software like Adobe Photoshop, Jasc Paintshop Pro, or even using Advanced Paint Brush.

Each hexadecimal code will be preceded by a pound or hash sign \verb|#|.
\subsection{RGB Values}
This color value is specified using the \verb|rgb( )| property. This property takes three values, one each for red, green, and blue. The value can be an integer between 0 and 255 or a percentage.\\
\\
\textbf{Note 1.10.} All the browsers does not support \verb|rgb()| property of color so it is recommended not to use it.
\section{Background}
This chapter teaches you how to set backgrounds of various HTML elements. You can set the following background properties of an element.
\begin{enumerate}
\item The \textit{background-color} property is used to set the background color of an element.
\item The \textit{background-image} property is used to set the background image of an element.
\item The \textit{background-repeat} property is used to control the repetition of an image in the background.
\item The \textit{background-position} property is used to control the position of an image in the background.
\item The \textit{background-attachment} property is used to control the scrolling of an image in the background.
\item The \textit{background property} is used as a shorthand to specify a number of other background properties.
\end{enumerate}
\subsection{Set the Background Color}
Following is the example which demonstrates how to set the background color for an element.
\begin{verbatim}
<html>
   <head>
   <body>
      <p style = "background-color:yellow;">
      This text has a yellow background color.</p>
   </body>
   </head>
<html>
\end{verbatim}
\subsection{Set the Background Image}
We can set the background image by calling local stored images as shown below
\begin{verbatim}
<html>
   <head>
      <style>
         body  {
            background-image: url("/css/images/css.jpg");
            background-color: #cccccc;
         }
      </style>
      <body>
         <h1>Hello World!</h1>
      </body>
   </head>
<html>
\end{verbatim}
\subsection{Repeat the Background Image}
The following example demonstrates how to repeat the background image if an image is small. You can use \texttt{no-repeat} value for \texttt{background-repeat} property if you don't want to repeat an image, in this case image will display only once.

By default \texttt{background-repeat} property will have \texttt{repeat} value.
\begin{verbatim}
<html>
   <head>
      <style>
         body {
            background-image: url("/css/images/css.jpg");
            background-repeat: repeat;
         }
      </style>
   </head>
   <body>
      <p>Tutorials point</p>
   </body>
</html>
\end{verbatim}

The following example which demonstrates how to repeat the background image vertically.
\begin{verbatim}
<html>
   <head>
      <style>
         body {
            background-image: url("/css/images/css.jpg");
            background-repeat: repeat-y;
         }
      </style>
   </head>
   <body>
      <p>Tutorials point</>
   </body>
</html>
\end{verbatim}

The following example demonstrates how to repeat the background image horizontally.
\begin{verbatim}
<html>
   <head>
      <style>
         body {
            background-image: url("/css/images/css.jpg");
            background-repeat: repeat-x;
         }
      </style>
   </head>
   <body>
      <p>Tutorials point</>
   </body>
</html>
\end{verbatim}
\subsection{Set the Background Image Position}
The following example demonstrates how to set the background image position 100 pixels away from the left side.
\begin{verbatim}
<html>
   <head>
      <style>
         body {
            background-image: url("/css/images/css.jpg");
            background-position:100px;
         }
      </style>
   </head>
   <body>
      <p>Tutorials point</>
   </body>
</html>
\end{verbatim}

The following example demonstrates how to set the background image position 100 pixels away from the left side and 200 pixels down from the top.
\begin{verbatim}
<html>
   <head>
      <style>
         body {
            background-image: url("/css/images/css.jpg");
            background-position:100px 200px;
         }
      </style>
   </head>
   <body>
      <p>Tutorials point</>
   </body>
</html>
\end{verbatim}
\subsection{Set the Background Attachment}
Background attachment determines whether a background image is fixed or scrolls with the rest of the page.

The following example demonstrates how to set the fixed background image.
\begin{verbatim}
<!DOCTYPE html>
<html>
   <head>
   
      <style>
         body  {
            background-image: url('/css/images/css.jpg');
            background-repeat: no-repeat;
            background-attachment: fixed;
         }
      </style>
      
   </head>
   <body>
   
      <p>The background-image is fixed. Try to scroll down the page.</p>
      <p>The background-image is fixed. Try to scroll down the page.</p>
      <p>The background-image is fixed. Try to scroll down the page.</p>
      <p>The background-image is fixed. Try to scroll down the page.</p>
      <p>The background-image is fixed. Try to scroll down the page.</p>
      <p>The background-image is fixed. Try to scroll down the page.</p>
      <p>The background-image is fixed. Try to scroll down the page.</p>
      <p>The background-image is fixed. Try to scroll down the page.</p>
      <p>The background-image is fixed. Try to scroll down the page.</p>
      
   </body>
</html>
\end{verbatim}

The following example demonstrates how to set the scrolling background image.
\begin{verbatim}
<!DOCTYPE html>
<html>
   <head>
   
      <style>
         body  {
            background-image: url('/css/images/css.jpg');
            background-repeat: no-repeat;
            background-attachment: fixed;
            background-attachment:scroll;
         }.
      </style>
      
   </head>
   <body>
   
      <p>The background-image is fixed. Try to scroll down the page.</p>
      <p>The background-image is fixed. Try to scroll down the page.</p>
      <p>The background-image is fixed. Try to scroll down the page.</p>
      <p>The background-image is fixed. Try to scroll down the page.</p>
      <p>The background-image is fixed. Try to scroll down the page.</p>
      <p>The background-image is fixed. Try to scroll down the page.</p>
      <p>The background-image is fixed. Try to scroll down the page.</p>
      <p>The background-image is fixed. Try to scroll down the page.</p>
      <p>The background-image is fixed. Try to scroll down the page.</p>
      
   </body>
</html>
\end{verbatim}
\subsection{Shorthand Property}
You can use the background property to set all the background properties at once. For example
\begin{verbatim}
<p style="background:url(/images/pattern1.gif) repeat fixed;">
   This parapgraph has fixed repeated background image.
</p>
\end{verbatim}
\section{Fonts}
This chapter teaches you how to set fonts of a content, available in an HTML element. You can set following font properties of an element.
\begin{enumerate}
\item The \textit{font-family} property is used to change the face of a font.
\item The \textit{font-style} property is used to make a font italic or oblique.
\item The font-variant property is used to create a small-caps effect.
\item The \textit{font-weight} property is used to increase or decrease how bold or light a font appears.
\item The \textit{font-size} property is used to increase or decrease the size of a font.
\item The \textit{font} property is used as shorthand to specify a number of other font properties.
\end{enumerate}
\subsection{Set the Font Family}
Following is the example, which demonstrates how to set the font family of an element. Possible value could be any font family name.
\begin{verbatim}
<html>
   <head>
   </head>
   <body>
      <p style="font-family:georgia,garamond,serif;">
      This text is rendered in either georgia, garamond, or the
      default serif font depending on which font  you have at your 
      system.
      </p>
   </body>
</html>
\end{verbatim}
\subsection{Set the Font Style}
Following is the example, which demonstrates how to set the font style of an element. Possible values are \textit{normal, italic} and \textit{oblique}.
\begin{verbatim}
<html>
   <head>
   </head>
   <body>
      <p style="font-style:italic;">
      This text will be rendered in italic style
      </p>
   </body>
</html>
\end{verbatim}
\subsection{Set the Font Variant}
The following example demonstrates how to set the font variant of an element. Possible values are \textit{normal} and \textit{small-caps}.
\begin{verbatim}
<html>
   <head>
   </head>
   <body>
      <p style="font-variant:small-caps;">
      This text will be rendered as small caps
      </p>
   </body>
</html>
\end{verbatim}
\subsection{Set the Font Weight}
The following example demonstrates how to set the font weight of an element. The font-weight property provides the functionality to specify how bold a font is. Possible values could be \textit{normal, bold, bolder, lighter, 100, 200, 300, 400, 500, 600, 700, 800, 900}.
\begin{verbatim}
<html>
   <head>
   </head>
   <body>
      <p style="font-weight:bold;">This font is bold.</p>
      <p style="font-weight:bolder;">This font is bolder.</p>
      <p style="font-weight:500;">This font is 500 weight.</p>
   </body>
</html>
\end{verbatim}
\subsection{Set the Font Size}
The following example demonstrates how to set the font size of an element. The font-size property is used to control the size of fonts. Possible values could be \textit{xx-small, x-small, small, medium, large, x-large, xx-large, smaller, larger}, size in \textit{pixels} or in \%.
\begin{verbatim}
<html>
   <head>
   </head>
   <body>
      <p style="font-size:20px;">This font size is 20 pixels</p>
      <p style="font-size:small;">This font size is small</p>
      <p style="font-size:large;">This font size is large</p>
   </body>
</html>
\end{verbatim}
\subsection{Set the Font Size Adjust}
The following example demonstrates how to set the font size adjust of an element. This property enables you to adjust the x-height to make fonts more legible. Possible value could be any number.
\begin{verbatim}
<html>
   <head>
   </head>
   <body>
      <p style="font-size-adjust:0.61;">
         This text is using a font-size-adjust value.
      </p>
   </body>
</html>
\end{verbatim}
\subsection{Set the Font Stretch}
The following example demonstrates how to set the font stretch of an element. This property relies on the user's computer to have an expanded or condensed version of the font being used. 

Possible values could be \textit{normal, wider, narrower, ultra-condensed, extra-condensed, condensed, semi-condensed, semi-expanded, expanded, extra-expanded, ultra-expanded}.
\begin{verbatim}
<html>
   <head>
   </head>
   <body>
      <p style="font-stretch:ultra-expanded;">
         If this doesn't appear to work, it is likely that your
         computer doesn't have a condensed or expanded version of
         the font being used.
      </p>
   </body>
</html>
\end{verbatim}
\subsection{Shorthand Property}
You can use the \textit{font} property to set all the font properties at once. For example
\begin{verbatim}
<html>
   <head>
   </head>
   <body>
      <p style="font:italic small-caps bold 15px georgia;">
      Applying all the properties on the text at once.
      </p>
   </body>
</html>
\end{verbatim}
\section{Text}
This chapter teaches you how to manipulate text using CSS properties. You can set following text properties of an element.
\begin{enumerate}
\item The \textit{color} property is used to set the color of a text.
\item The \textit{direction} property is used to set the text direction.
\item The \textit{letter-spacing} property is used to add or subtract space between the letters that make up a word.
\item The \textit{word-spacing} property is used to add or subtract space between the words of a sentence.
\item The \textit{text-indent} property is used to indent the text of a paragraph.
\item The \textit{text-align} property is used to align the text of a document.
\item The \textit{text-decoration} property is used to underline, overline, and strike through text.
\item The \textit{text-transform} property is used to capitalize text or convert text to uppercase or lowercase letters.
\item The \textit{white-space} property is used to control the flow and formatting of text.
\item The \textit{text-shadow} property is used to set the text shadow around a text.
\end{enumerate}
\subsection{Set the Text Color}
The following example demonstrates how to set the text color. Possible value could be any color name in any valid format.
\begin{verbatim}
<html>
   <head>
   </head>
   <body>
      <p style="color:red;">
      This text will be written in red.
      </p>
   </body>
</html>
\end{verbatim}
\subsection{Set the Text Direction}
The following example demonstrates how to set the direction of a text. Possible values are \textit{ltr} or \textit{rtl}.
\begin{verbatim}
<html>
   <head>
   </head>
   <body>
      <p style="direction:rtl;">
      This text will be renedered from right to left
      </p>
   </body>
</html>
\end{verbatim}
\subsection{Set the Space between Characters}
The following example demonstrates how to set the space between characters. Possible values are \textit{normal} or \textit{a number specifying space}.
\begin{verbatim}
<html>
   <head>
   </head>
   <body>
      <p style="letter-spacing:5px;">
      This text is having space between letters.
      </p>
   </body>
</html>
\end{verbatim}
\subsection{Set the Space between Words}
The following example demonstrates how to set the space between words. Possible values are \textit{normal} or \textit{a number specifying space}.
\begin{verbatim}
<html>
   <head>
   </head>
   <body>
      <p style="word-spacing:5px;">
      This text is having space between words.
      </p>
   </body>
</html>
\end{verbatim}
\subsection{Set the Text Indent}
The following example demonstrates how to indent the first line of a paragraph. Possible values are \% or \textit{a number specifying indent space}.
\begin{verbatim}
<html>
   <head>
   </head>
   <body>
      <p style="text-indent:1cm;">
      This text will have first line indented by 1cm and this line
      will remain at its actual position this is done by CSS 
      text-indent property.
      </p>
   </body>
</html>
\end{verbatim}
\subsection{Set the Text Alignment}
The following example demonstrates how to align a text. Possible values are \textit{left, right, center, justify}.
\begin{verbatim}
<html>
   <head>
   </head>
   <body>
      <p style="text-align:right;">
      This will be right aligned.
      </p>
      
      <p style="text-align:center;">
      This will be center aligned.
      </p>
      
      <p style="text-align:left;">
      This will be left aligned.
      </p>
      
   </body>
</html>
\end{verbatim}
\subsection{Decorating the Text}
The following example demonstrates how to decorate a text. Possible values are none, \textit{underline, overline, line-through, blink.}
\begin{verbatim}
<html>
   <head>
   </head>
   <body>
      <p style="text-decoration:underline;">
      This will be underlined
      </p>
      
      <p style="text-decoration:line-through;">
      This will be striked through.
      </p>
      
      <p style="text-decoration:overline;">
      This will have a over line.
      </p>
      
      <p style="text-decoration:blink;">
      This text will have blinking effect
      </p>
   </body>
</html>
\end{verbatim}
\subsection{Set the Text Cases}
The following example demonstrates how to set the cases for a text. Possible values are \textit{none, capitalize, uppercase, lowercase}.
\begin{verbatim}
<html>
   <head>
   </head>
   <body>
      <p style="text-transform:capitalize;">
      This will be capitalized
      </p>
      
      <p style="text-transform:uppercase;">
      This will be in uppercase
      </p>
      
      <p style="text-transform:lowercase;">
      This will be in lowercase
      </p>
      </body>

</html>
\end{verbatim}
\subsection{Set the White Space between Text}
The following example demonstrates how white space inside an element is handled. Possible values are \textit{normal, pre, nowrap}.
\begin{verbatim}
<html>
   <head>
   </head>
   <body>
      <p style="white-space:pre;">
      This text has a line break and the white-space pre setting
      tells the browser to honor it just like the HTML pre tag.</p> 
   </body>
</html> 
\end{verbatim}
\subsection{Set the Text Shadow}
The following example demonstrates how to set the shadow around a text. This may not be supported by all the browsers.
\begin{verbatim}
<html>
   <head>
   </head>
   <body>
      <p style="text-shadow:4px 4px 8px blue;">
      If your browser supports the CSS text-shadow property, this 
      text will have a blue shadow.
      </p>
   </body>
</html> 
\end{verbatim}
\section{Using Images}
Images play an important role in any webpage. Though it is not recommended to include a lot of images, but it is still important to use good images wherever required.

CSS plays a good role to control image display. You can set the following image properties using CSS.
\begin{enumerate}
\item The \textit{border} property is used to set the width of an image border.
\item The \textit{height} property is used to set the height of an image.
\item The \textit{width} property is used to set the width of an image.
\item The \textit{-moz-opacity} property is used to set the opacity of an image.
\end{enumerate}
\subsection{The Image Border Property}
The \textit{border} property of an image is used to set the width of an image border. This property can have a value in length or in \%. A width of zero pixels means no border.

Here is the example.
\begin{verbatim}
<html>
   <head>
   </head>
   <body>
      <img style="border:0px;" src="/css/images/logo.png" />
      <br />
      <img style="border:3px dashed red;" src="/css/images/logo.png" />
   </body>
</html> 
\end{verbatim}
\subsection{The Image Height Property}
The height property of an image is used to set the height of an image. This property can have a value in length or in \%. While giving value in \%, it applies it in respect of the box in which an image is available.

Here is an example.
\begin{verbatim}
<html>
   <head>
   </head>
   <body>
      <img style="border:1px solid red;
       height:100px;" src="/css/images/logo.png" />
      <br />
      <img style="border:1px solid red;
       height:50%;" src="/css/images/logo.png" />
   </body>
</html> 
\end{verbatim}
\subsection{The Image Width Property}
The \textit{width} property of an image is used to set the width of an image. This property can have a value in length or in \%. While giving value in \%, it applies it in respect of the box in which an image is available.

Here is an example.
\begin{verbatim}
<html>
   <head>
   </head>
   <body>
      <img style="border:1px solid red;
       width:150px;" src="/css/images/logo.png" />
      <br />
      <img style="border:1px solid red;
       width:100%;" src="/css/images/logo.png" />
   </body>
</html> 
\end{verbatim}
\subsection{The -moz-opacity Property}
The \textit{-moz-opacity} property of an image is used to set the opacity of an image. This property is used to create a transparent image in Mozilla. IE uses \verb|filter:alpha(opacity=x)| to create transparent images.

In Mozilla \verb|(-moz-opacity:x)| x can be a value from 0.0 - 1.0. A lower value makes the element more transparent (The same things goes for the CSS3-valid syntax opacity:x).

In IE \verb|filter:alpha(opacity=x)| x can be a value from 0 - 100. A lower value makes the element more transparent.

Here is an example.
\begin{verbatim}
<html>
   <head>
   </head>
   <body>
      <img style="border:1px solid red;-moz-opacity:0.4;
      filter:alpha(opacity=40); " src="/css/images/logo.png" />
   </body>
</html> 
\end{verbatim}
\section{Links}
This chapter teaches you how to set different properties of a hyper link using CSS. You can set following properties of a hyper link. 

We will revisit the same properties when we will discuss Pseudo-Classes of CSS.
\begin{enumerate}
\item The \verb|:link| signifies unvisited hyperlinks.
\item The \verb|:visited| signifies visited hyperlinks.
\item The \verb|:hover| signifies an element that currently has the user's mouse pointer hovering over it.
\item The \verb|:active| signifies an element on which the user is currently clicking.
\end{enumerate}
Usually, all these properties are kept in the header part of the HTML document.

Remember a:hover MUST come after a:link and a:visited in the CSS definition in order to be effective. Also, a:active MUST come after a:hover in the CSS definition as follows.
\begin{verbatim}
<style type="text/css">
   a:link {color: #000000}
   a:visited {color: #006600}
   a:hover {color: #FFCC00}
   a:active {color: #FF00CC}
</style>
\end{verbatim}
Now, we will see how to use these properties to give different effects to hyperlinks.
\subsection{Set the Colors of Links}
The following example demonstrates how to set the link color. Possible values could be any color name in any valid format.
\begin{verbatim}
<html>
   <head>
      <style type="text/css">
         a:link {color:#000000}
     </style>
   </head>
   <body>
      <a href="">Link</a>
   </body>
</html> 
\end{verbatim}
\subsection{Set the Color of Visited Links}
The following example demonstrates how to set the color of visited links. Possible values could be any color name in any valid format.
\begin{verbatim}
<html>
   <head>
      <style type="text/css">
         a:visited {color: #006600}
      </style>
   </head>
   <body>
      <a href=""> link</a> 
   </body>
</html> 
\end{verbatim}
\subsection{Change the Color of Links when Mouse is Over}
The following example demonstrates how to change the color of links when we bring a mouse pointer over that link. Possible values could be any color name in any valid format.
\begin{verbatim}
<html>
   <head>
      <style type="text/css">
         a:hover {color: #FFCC00}
      </style>
   </head>
   <body>
      <a href="">Link</a>
   </body>
</html> 
\end{verbatim}
\subsection{Change the Color of Active Links}
The following example demonstrates how to change the color of active links. Possible values could be any color name in any valid format.
\begin{verbatim}
<html>
   <head>
      <style type="text/css">
         a:active {color: #FF00CC}
      </style>
   </head>
   <body>
      <a href="">Link</a>
   </body>
</html> 
\end{verbatim}
\section{Tables}
This tutorial will teach you how to set different properties of an HTML table using CSS. You can set following properties of a table.
\begin{enumerate}
\item The \textit{border-collapse} specifies whether the browser should control the appearance of the adjacent borders that touch each other or whether each cell should maintain its style.
\item The \textit{border-spacing} specifies the width that should appear between table cells.
\item The \textit{caption-side} captions are presented in the $<$caption$>$ element. By default, these are rendered above the table in the document. You use the \textit{caption-side} property to control the placement of the table caption.
\item The \textit{empty-cells} specifies whether the border should be shown if a cell is empty.
\item The \textit{table-layout} allows browsers to speed up layout of a table by using the first width properties it comes across for the rest of a column rather than having to load the whole table before rendering it.
\end{enumerate}
\subsection{The border-collapse Property}
This property can have two values \textit{collapse} and \textit{separate}. The following example uses both the values.
\begin{verbatim}
<html>
   <head>
   
      <style type="text/css">
         table.one {border-collapse:collapse;}
         table.two {border-collapse:separate;}
         td.a {
            border-style:dotted;
            border-width:3px;
            border-color:#000000; 
            padding: 10px;
         }
         td.b {
            border-style:solid;
            border-width:3px;
            border-color:#333333;
            padding:10px;
         }
      </style>
      
   </head>
   <body>
   
      <table class="one">
         <caption>Collapse Border Example</caption>
         <tr><td class="a"> Cell A Collapse Example</td></tr>
         <tr><td class="b"> Cell B Collapse Example</td></tr>
      </table>
      <br />
      
      <table class="two">
         <caption>Separate Border Example</caption>
         <tr><td class="a"> Cell A Separate Example</td></tr>
         <tr><td class="b"> Cell B Separate Example</td></tr>
      </table>
      
   </body>
</html> 
\end{verbatim}
\subsection{The border-spacing Property}
The \textit{border-spacing} property specifies the distance that separates adjacent cells' borders. It can take either one or two values; these should be units of length.

If you provide one value, it will applies to both vertical and horizontal borders. Or you can specify two values, in which case, the first refers to the horizontal spacing and the second to the vertical spacing.\\
\\
\textbf{Note 1.11.} Unfortunately, this property does not work in Netscape 7 or IE 6.
\begin{verbatim}
<style type="text/css">
   /* If you provide one value */
   table.example {border-spacing:10px;}
   /* This is how you can provide two values */
   table.example {border-spacing:10px; 15px;}
</style>
\end{verbatim}
Now let's modify the previous example.
\begin{verbatim}
<html>
   <head>
   
      <style type="text/css">
         table.one {
            border-collapse:separate;
            width:400px;
            border-spacing:10px;
         }
         table.two {
            border-collapse:separate;
            width:400px;
            border-spacing:10px 50px;
         }
      </style>
      
   </head>
   <body>
   
      <table class="one" border="1">
         <caption>Separate Border Example with border-spacing</caption>
         <tr><td> Cell A Collapse Example</td></tr>
         <tr><td> Cell B Collapse Example</td></tr>
      </table>
      <br />
      
      <table class="two" border="1">
         <caption>Separate Border Example with border-spacing</caption>
         <tr><td> Cell A Separate Example</td></tr>
         <tr><td> Cell B Separate Example</td></tr>
      </table>
      
   </body>
</html> 
\end{verbatim}
\subsection{The caption-side Property}
The \textit{caption-side} property allows you to specify where the content of a $<$caption$>$ element should be placed in relationship to the table. The table that follows lists the possible values.

This property can have one of the four values \textit{top, bottom, left} or \textit{right}. The following example uses each value.\\
\\
\textbf{Note.} These properties may not work with your IE Browser.
\begin{verbatim}
<html>
   <head>
   
      <style type="text/css">
         caption.top {caption-side:top}
         caption.bottom {caption-side:bottom}
         caption.left {caption-side:left}
         caption.right {caption-side:right}
      </style>
      
   </head>
   <body>
   
      <table style="width:400px; border:1px solid black;">
         <caption class="top">
         This caption will appear at the top
         </caption>
         <tr><td > Cell A</td></tr>
         <tr><td > Cell B</td></tr>
      </table>
      <br />
      
      <table style="width:400px; border:1px solid black;">
         <caption class="bottom">
         This caption will appear at the bottom
         </caption>
         <tr><td > Cell A</td></tr>
         <tr><td > Cell B</td></tr>
      </table>
      <br />
      
      <table style="width:400px; border:1px solid black;">
         <caption class="left">
         This caption will appear at the left
         </caption>
         <tr><td > Cell A</td></tr>
         <tr><td > Cell B</td></tr>
      </table>
      <br />
      
      <table style="width:400px; border:1px solid black;">
         <caption class="right">
         This caption will appear at the right
         </caption>
         <tr><td > Cell A</td></tr>
         <tr><td > Cell B</td></tr>
      </table>
      
   </body>
</html> 
\end{verbatim}
\subsection{The empty-cells Property}
The \textit{empty-cells} property indicates whether a cell without any content should have a border displayed.

This property can have one of the three values: \textit{show, hide} or \textit{inherit}.

Here is the \textit{empty-cells} property used to hide borders of empty cells in the $<$table$>$ element.
\begin{verbatim}
<html>
   <head>
   
      <style type="text/css">
         table.empty{
            width:350px;
            border-collapse:separate;
            empty-cells:hide;
         }
         td.empty{
            padding:5px;
            border-style:solid;
            border-width:1px;
            border-color:#999999;
         }
      </style>
      
   </head>
   <body>
   
      <table class="empty">
      <tr>
         <th></th>
         <th>Title one</th>
         <th>Title two</th>
      </tr>
      
      <tr>
         <th>Row Title</th>
         <td class="empty">value</td>
         <td class="empty">value</td>
      </tr>
      
      <tr>
         <th>Row Title</th>
         <td class="empty">value</td>
         <td class="empty"></td>
      </tr>
      </table>
      
   </body>
</html> 
\end{verbatim}
\subsection{The table-layout Property}
The \textit{table-layout} property is supposed to help you control how a browser should render or lay out a table.

This property can have one of the three values: \textit{fixed, auto} or \textit{inherit}.

The following example shows the difference between these properties.\\
\\
\textbf{Note.} This property is not supported by many browsers so do not rely on this property.
\begin{verbatim}
<html>
   <head>
   
      <style type="text/css">
         table.auto {
            table-layout: auto
         }
         table.fixed{
            table-layout: fixed
         }
      </style>
      
   </head>
   <body>
   
      <table class="auto" border="1" width="100%">
      <tr>
         <td width="20%">1000000000000000000000000000</td>
         <td width="40%">10000000</td>
         <td width="40%">100</td>
      </tr>
      </table>
      <br />
      
      <table class="fixed" border="1" width="100%">
      <tr>
         <td width="20%">1000000000000000000000000000</td>
         <td width="40%">10000000</td>
         <td width="40%">100</td>
      </tr>
      </table>
      
   </body>
</html> 
\end{verbatim}
\section{Borders}
The \textit{border} properties allow you to specify how the border of the box representing an element should look. There are three properties of a border you can change:
\begin{enumerate}
\item The \textit{border-color} specifies the color of a border.
\item The \textit{border-style} specifies whether a border should be solid, dashed line, double line, or one of the other possible values.
\item The \textit{border-width} specifies the width of a border.
\end{enumerate}
\subsection{The border-color Property}
The \textit{border-color} property allows you to change the color of the border surrounding an element. You can individually change the color of the bottom, left, top and right sides of an element's border using the properties.
\begin{enumerate}
\item border-bottom-color changes the color of bottom border.
\item border-top-color changes the color of top border.
\item border-left-color changes the color of left border.
\item border-right-color changes the color of right border.
\end{enumerate}
The following example shows the effect of all these properties.
\begin{verbatim}
<html>
   <head>
   
      <style type="text/css">
         p.example1{
            border:1px solid;
            border-bottom-color:#009900; /* Green */
            border-top-color:#FF0000;    /* Red */
            border-left-color:#330000;   /* Black */
            border-right-color:#0000CC;  /* Blue */
         }
         p.example2{
            border:1px solid;
            border-color:#009900;        /* Green */
         }
      </style>
      
   </head>
   <body>
   
      <p class="example1">
      This example is showing all borders in different colors.
      </p>
      
      <p class="example2">
      This example is showing all borders in green color only.
      </p>
      
   </body>
</html> 
\end{verbatim}
\subsection{The border-style Property}
The \textit{border-style} property allows you to select one of the following styles of border.
\begin{enumerate}
\item \textit{none}: No border. (Equivalent of border-width:0;)
\item \textit{solid}: Border is a single solid line.
\item \textit{dotted}: Border is a series of dots.
\item \textit{dashed}: Border is a series of short lines.
\item \textit{double}: Border is two solid lines.
\item \textit{groove}: Border looks as though it is carved into the page.
\item \textit{ridge}: Border looks the opposite of groove.
\item \textit{inset}: Border makes the box look like it is embedded in the page.
\item \textit{outset}: Border makes the box look like it is coming out of the canvas.
\item \textit{hidden}: Same as none, except in terms of border-conflict resolution for table elements.
\end{enumerate}

You can individually change the style of the bottom, left, top, and right borders of an element using the following properties.
\begin{enumerate}
\item \textit{border-bottom-style} changes the style of bottom border.
\item \textit{border-top-style} changes the style of top border.
\item \textit{border-left-style} changes the style of left border.
\item \textit{border-right-style} changes the style of right border.
\end{enumerate}
The following example shows all these border styles.
\begin{verbatim}
<html>
   <head>
   </head>
   
   <body>.
      <p style="border-width:4px; border-style:none;">
      This is a border with none width.
      </p>
      
      <p style="border-width:4px; border-style:solid;">
      This is a solid border.
      </p>
      
      <p style="border-width:4px; border-style:dashed;">
      This is a dahsed border.
      </p>
      
      <p style="border-width:4px; border-style:double;">
      This is a double border.
      </p>
      
      <p style="border-width:4px; border-style:groove;">
      This is a groove border.
      </p>
      
      <p style="border-width:4px; border-style:ridge">
      This is aridge  border.
      </p>
      
      <p style="border-width:4px; border-style:inset;">
      This is a inset border.
      </p>
      
      <p style="border-width:4px; border-style:outset;">
      This is a outset border.
      </p>
      
      <p style="border-width:4px; border-style:hidden;">
      This is a hidden border.
      </p>
      
      <p style="border-width:4px;border-top-style:solid;
      border-bottom-style:dashed; 
      border-left-style:groove; border-right-style:double;">
      This is a a border with four different styles.
      </p>
   </body>
   
</html> 
\end{verbatim}
\subsection{The border-width Property}
The \textit{border-width} property allows you to set the width of an element borders. The value of this property could be either a length in px, pt or cm or it should be set to \textit{thin, medium} or \textit{thick}.

You can individually change the width of the bottom, top, left, and right borders of an element using the following properties.
\begin{enumerate}
\item \textit{border-bottom-width} changes the width of bottom border.
\item \textit{border-top-width} changes the width of top border.
\item \textit{border-left-width} changes the width of left border.
\item \textit{border-right-width} changes the width of right border.
\end{enumerate}
The following example shows all these border width.
\begin{verbatim}
<html>
   <head>
   </head>
   <body>
      <p style="border-width:4px; border-style:solid;">
      This is a solid border whose width is 4px.
      </p>
      
      <p style="border-width:4pt; border-style:solid;">
      This is a solid border whose width is 4pt.
      </p>
      
      <p style="border-width:thin; border-style:solid;">
      This is a solid border whose width is thin.
      </p>
      
      <p style="border-width:medium; border-style:solid;">
      This is a solid border whose width is medium;
      </p>
      
      <p style="border-width:thick; border-style:solid;">
      This is a solid border whose width is thick.
      </p>
      
      <p style="border-bottom-width:4px;border-top-width:10px;
      border-left-width: 2px;border-right-width:15px;border-style:solid;">
      This is a a border with four different width.
      </p>
   </body>
</html>
\end{verbatim}
\subsection{Border Properties Using Shorthand}
The \textit{border} property allows you to specify color, style, and width of lines in one property.

The following example shows how to use all the three properties into a single property. This is the most frequently used property to set border around any element.
\begin{verbatim}
<html>
   <head>
   </head>
   <body>
      <p style="border:4px solid red;">
      This example is showing shorthand property for border.
      </p>
   </body>
</html>
\end{verbatim}
\section{Margins}
The \textit{margin} property defines the space around an HTML element. It is possible to use negative values to overlap content.

The values of the margin property are not inherited by the child elements. Remember that the adjacent vertical margins (top and bottom margins) will collapse into each other so that the distance between the blocks is not the sum of the margins, but only the greater of the two margins or the same size as one margin if both are equal.

We have the following properties to set an element margin.
\begin{enumerate}
\item The \textit{margin} specifies a shorthand property for setting the margin properties in one declaration.
\item The \textit{margin-bottom} specifies the bottom margin of an element.
\item The \textit{margin-top} specifies the top margin of an element.
\item The \textit{margin-left} specifies the left margin of an element.
\item The \textit{margin-right} specifies the right margin of an element.
\end{enumerate}
\subsection{The Margin Property}
The margin property allows you set all of the properties for the four margins in one declaration. Here is the syntax to set margin around a paragraph.

Here is an example.
\begin{verbatim}
<html>
   <head>
   </head>
   
   <body>
      <p style="margin: 15px; border:1px solid black;"> 
      all four margins will be 15px 
      </p>
      
      <p style="margin:10px 2%; border:1px solid black;">
      top and bottom margin will be 10px, left and right margin 
      will be 2% of the total width of the document. 
      </p>
      
      <p style="margin: 10px 2% -10px; border:1px solid black;">
      top margin will be 10px, left and right margin will be 2%
      of the total width of the document, bottom margin will be -10px
      </p> 
      
      <p style="margin: 10px 2% -10px auto; border:1px solid black;">
      top margin will be 10px, right margin will be 2% of the total 
      width of the document, bottom margin will be -10px, left margin
      will be set by the browser 
      </p>
   </body>
   
</html>
\end{verbatim}
\subsection{The margin-bottom Property}
The margin-bottom property allows you set bottom margin of an element. It can have a value in length, \% or auto.

Here is an example.
\begin{verbatim}
<html>
   <head>
   </head>
   <body>
      <p style="margin-bottom: 15px; border:1px solid black;">
      This is a paragraph with a specified bottom margin.
      </p> 
   
      <p style="margin-bottom: 5%; border:1px solid black;"> 
      This is another paragraph with a specified bottom margin in percent
      </p>
   </body>
</html> 
\end{verbatim}
\subsection{The margin-top Property}
The margin-top property allows you set top margin of an element. It can have a value in length, \% or auto.

Here is an example.
\begin{verbatim}
<html>
   <head>
   </head>
   <body>
      <p style="margin-top: 15px; border:1px solid black;"> 
      This is a paragraph with a specified top margin 
      </p> 
      
      <p style="margin-top: 5%; border:1px solid black;"> 
      This is another paragraph with a specified top margin 
      in percent 
      </p>
   </body>
</html> 
\end{verbatim}
\subsection{The margin-left Property}
The margin-left property allows you set left margin of an element. It can have a value in length, \% or auto.

Here is an example.
\begin{verbatim}
<html>
   <head>
   </head>
   <body>
      <p style="margin-left: 15px; border:1px solid black;"> 
      This is a paragraph with a specified left margin 
      </p> 
   
      <p style="margin-left: 5%; border:1px solid black;"> 
      This is another paragraph with a specified top margin 
      in percent 
      </p>
   </body>
</html> 
\end{verbatim}
\subsection{The margin-right Property}
The margin-right property allows you set right margin of an element. It can have a value in length, \% or auto.

Here is an example.
\begin{verbatim}
<html>
   <head>
   </head>
   <body>
      <p style="margin-right: 15px; border:1px solid black;"> 
      This is a paragraph with a specified right margin 
      </p> 
      
      <p style="margin-right: 5%; border:1px solid black;"> 
      This is another paragraph with a specified right margin 
      in percent 
      </p>
   </body>
</html> 
\end{verbatim}
\section{Lists}
Lists are very helpful in conveying a set of either numbered or bullet points. This chapter teaches you how to control list type, position, style, etc., using CSS.

We have the following five CSS properties, which can be used to control lists.
\begin{enumerate}
\item The \textit{list-style-type} allows you to control the shape or appearance of the marker.
\item The \textit{list-style-position} specifies whether a long point that wraps to a second line should align with the first line or start underneath the start of the marker.
\item The \textit{list-style-image} specifies an image for the marker rather than a bullet point or number.
\item The \textit{list-style} serves as shorthand for the preceding properties.
\item The \textit{marker-offset} specifies the distance between a marker and the text in the list.
\end{enumerate}
\subsection{The list-style-type Property}
The \textit{list-style-type} property allows you to control the shape or style of bullet point (also known as a marker) in the case of unordered lists and the style of numbering characters in ordered lists.

Here are the values which can be used for an unordered list.
\begin{center}
\begin{longtable}{|c|l|}
\hline
\textbf{Value} & \textbf{Description}\\
\hline
none & NA\\
\hline
disc (default) & A filled-in circle\\
\hline
circle & An empty circle\\
\hline
square & A filled-in square\\
\hline
\caption{Values which can be used for an unordered list.}
\end{longtable}
\end{center}
Here are the values, which can be used for an ordered list.
\begin{center}
\begin{longtable}{|l|p{5cm}|p{3cm}|}
\hline
\textbf{Value} & \textbf{Description} & \textbf{Example}\\
\hline
decimal & Number & 1, 2, 3, 4, 5\\
\hline
decimal-leading-zero & 0 before the number & 01, 02, 03, 04, 05\\
\hline
lower-alpha & Lowercase alphanumeric characters & a, b, c, d, e\\
\hline
upper-alpha & Uppercase alphanumeric characters & A, B, C, D, E\\
\hline
lower-roman &	Lowercase Roman numerals &	i, ii, iii, iv, v\\
\hline
upper-roman &	Uppercase Roman numerals &	I, II, III, IV, V\\
\hline
lower-greek &	The marker is lower-greek &	alpha, beta, gamma\\
\hline
lower-latin &	The marker is lower-latin &	a, b, c, d, e\\
\hline
upper-latin &	The marker is upper-latin &	A, B, C, D, E\\
\hline
hebrew &	The marker is traditional Hebrew numbering &\\
\hline 	 
armenian &	The marker is traditional Armenian numbering & \\
\hline 	 
georgian &	The marker is traditional Georgian numbering & \\
\hline 	 
cjk-ideographic &	The marker is plain ideographic numbers & \\
\hline
hiragana &	The marker is hiragana 	& a, i, u, e, o, ka, ki\\
\hline
katakana & The marker is katakana &	A, I, U, E, O, KA, KI\\
\hline
hiragana-iroha &	The marker is hiragana-iroha &	i, ro, ha, ni, ho, he, to\\
\hline
katakana-iroha &	The marker is katakana-iroha &	I, RO, HA, NI, HO, HE, TO\\
\hline
\caption{Values which can be used for an ordered list.}
\end{longtable}
\end{center}
Here is an example.
\begin{verbatim}
<html>
   <head>
   </head>
   
   <body>
      <ul style="list-style-type:circle;">
         <li>Maths</li>
         <li>Social Science</li>
         <li>Physics</li>
      </ul>
      
      <ul style="list-style-type:square;">
         <li>Maths</li>
         <li>Social Science</li>
         <li>Physics</li>
      </ul>
      
      <ol style="list-style-type:decimal;">
         <li>Maths</li>
         <li>Social Science</li>
         <li>Physics</li>
      </ol>
      
      <ol style="list-style-type:lower-alpha;">
         <li>Maths</li>
         <li>Social Science</li>
         <li>Physics</li>
      </ol>
      
      <ol style="list-style-type:lower-roman;">
         <li>Maths</li>
         <li>Social Science</li>
         <li>Physics</li>
      </ol>
   </body>
   
</html>
\end{verbatim}
\subsection{The list-style-position Property}
The \textit{list-style-position} property indicates whether the marker should appear inside or outside of the box containing the bullet points. It can have one the two values.
\begin{center}
\begin{longtable}{|c|p{10cm}|}
\hline
\textbf{Value} & \textbf{Description} \\
\hline
inside & If the text goes onto a second line, the text will wrap underneath the marker. It will also appear indented to where the text would have started if the list had a value of outside.\\
\hline
outside & If the text goes onto a second line, the text will be aligned with the start of the first line (to the right of the bullet).\\
\hline
\caption{Two values of the \textit{list-style-position} property.}
\end{longtable}
\end{center}
Here is an example.
\begin{verbatim}
<html>
   <head>
   </head>
   
   <body>
      <ul style="list-style-type:circle; list-style-position:outside;">
         <li>Maths</li>
         <li>Social Science</li>
         <li>Physics</li>
      </ul>
      
      <ul style="list-style-type:square;list-style-position:inside;">
         <li>Maths</li>
         <li>Social Science</li>
         <li>Physics</li>
      </ul>
      
      <ol style="list-style-type:decimal;list-style-position:outside;">
         <li>Maths</li>
         <li>Social Science</li>
         <li>Physics</li>
      </ol>
      
      <ol style="list-style-type:lower-alpha;list-style-position:inside;">
         <li>Maths</li>
         <li>Social Science</li>
         <li>Physics</li>
      </ol>
   </body>
   
</html>
\end{verbatim}
\subsection{The list-style-image Property}
The \textit{list-style-image} allows you to specify an image so that you can use your own bullet style. The syntax is similar to the background-image property with the letters url starting the value of the property followed by the URL in brackets. If it does not find the given image then default bullets are used.

Here is an example.
\begin{verbatim}
<html>
   <head>
   </head>
   
   <body>
      <ul>
         <li style="list-style-image: url(/images/bullet.gif);">Maths</li>
         <li>Social Science</li>
         <li>Physics</li>
      </ul>
      
      <ol>
         <li style="list-style-image: url(/images/bullet.gif);">Maths</li>
         <li>Social Science</li>
         <li>Physics</li>
      </ol>
   </body>
   
</html>
\end{verbatim}
\subsection{The list-style Property}
The \textit{list-style} allows you to specify all the list properties into a single expression. These properties can appear in any order.

Here is an example.
\begin{verbatim}
<html>
   <head>
   </head>
   
   <body>
      <ul style="list-style: inside square;">
         <li>Maths</li>
         <li>Social Science</li>
         <li>Physics</li>
      </ul>
      
      <ol style="list-style: outside upper-alpha;">
         <li>Maths</li>
         <li>Social Science</li>
         <li>Physics</li>
      </ol>
   </body>
   
</html> 
\end{verbatim}
\subsection{The marker-offset Property}
The marker-offset property allows you to specify the distance between the marker and the text relating to that marker. Its value should be a length as shown in the following example.

Unfortunately, this property is not supported in IE 6 or Netscape 7.

Here is an example.
\begin{verbatim}
<html>
   <head>
   </head>
   
   <body>
      <ul style="list-style: inside square; marker-offset:2em;">
         <li>Maths</li>
         <li>Social Science</li>
         <li>Physics</li>
      </ul>
      
      <ol style="list-style: outside upper-alpha; marker-offset:2cm;">
         <li>Maths</li>
         <li>Social Science</li>
         <li>Physics</li>
      </ol>
   </body>
   
</html>
\end{verbatim}
\section{Paddings}
The \textit{padding} property allows you to specify how much space should appear between the content of an element and its border.

The value of this attribute should be either a length, a percentage, or the word \textit{inherit}. If the value is \textit{inherit}, it will have the same padding as its parent element. If a percentage is used, the percentage is of the containing box.

The following CSS properties can be used to control lists. You can also set different values for the padding on each side of the box using the following properties.
\begin{enumerate}
\item The padding-bottom specifies the bottom padding of an element.
\item The padding-top specifies the top padding of an element.
\item The padding-left specifies the left padding of an element.
\item The padding-right specifies the right padding of an element.
\item The padding serves as shorthand for the preceding properties.
\end{enumerate}
\subsection{The padding-bottom Property}
The padding-bottom property sets the bottom padding (space) of an element. This can take a value in terms of length of \%.

Here is an example.
\begin{verbatim}
<html>
   <head>
   </head>
   
   <body>
      <p style="padding-bottom: 15px; border:1px solid black;">
      This is a paragraph with a specified bottom padding
      </p>
      
      <p style="padding-bottom: 5%; border:1px solid black;">
      This is another paragraph with a specified bottom padding 
      in percent
      </p>
   </body>
   
</html> 
\end{verbatim}
\subsection{The padding-top Property}
The \textit{padding-top} property sets the top padding (space) of an element. This can take a value in terms of length of \%.

Here is an example.
\begin{verbatim}
<html>
   <head>
   </head>
   
   <body>
      <p style="padding-top: 15px; border:1px solid black;">
      This is a paragraph with a specified top padding
      </p>
      
      <p style="padding-top: 5%; border:1px solid black;">
      This is another paragraph with a specified top padding 
      in percent
      </p>
   </body>
   
</html> 
\end{verbatim}
\subsection{The padding-left Property}
The padding-left property sets the left padding (space) of an element. This can take a value in terms of length of \%.

Here is an example. 
\begin{verbatim}
<html>
   <head>
   </head>
   
   <body>
      <p style="padding-left: 15px; border:1px solid black;">
      This is a paragraph with a specified left padding
      </p>
      
      <p style="padding-left: 15%; border:1px solid black;">
      This is another paragraph with a specified left padding 
      in percent
      </p>
   </body>
   
</html>
\end{verbatim}
\subsection{The padding-right Property}
The padding-right property sets the right padding (space) of an element. This can take a value in terms of length of \%.

Here is an example.
\begin{verbatim}
<html>
   <head>
   </head>
   
   <body>
      <p style="padding-right: 15px; border:1px solid black;">
      This is a paragraph with a specified right padding
      </p>
      
      <p style="padding-right: 5%; border:1px solid black;">
      This is another paragraph with a specified right padding in 
      percent
      </p>
   </body>
   
</html> 
\end{verbatim}
\subsection{The Padding Property}
The padding property sets the left, right, top and bottom padding (space) of an element. This can take a value in terms of length of \%.

Here is an example.
\begin{verbatim}
<html>
   <head>
   </head>
   
   <body>
      <p style="padding: 15px; border:1px solid black;">
      all four padding will be 15px 
      </p> 
      
      <p style="padding:10px 2%; border:1px solid black;"> 
      top and bottom padding will be 10px, left and right
      padding will be 2% of the total width of the document. 
      </p> 
      
      <p style="padding: 10px 2% 10px; border:1px solid black;">
      top padding will be 10px, left and right padding will 
      be 2% of the total width of the document,
      bottom padding will be 10px
      </p> 
      
      <p style="padding: 10px 2% 10px 10px; border:1px solid black;">
      top padding will be 10px, right padding will be 2% of
      the total width of the document,
       bottom padding and top padding will be 10px 
      </p>
   </body>
   
</html> 
\end{verbatim}
\section{Cursors}
The \textit{cursor} property of CSS allows you to specify the type of cursor that should be displayed to the user.

One good usage of this property is in using images for submit buttons on forms. By default, when a cursor hovers over a link, the cursor changes from a pointer to a hand. However, it does not change form for a submit button on a form. Therefore, whenever someone hovers over an image that is a submit button, it provides a visual clue that the image is clickable.

The following table shows the possible values for the cursor property.
\begin{center}
\begin{longtable}{|c|p{10cm}|}
\hline
\textbf{Value} & \textbf{Description}\\
\hline
auto &	Shape of the cursor depends on the context area it is over. For example, an 'I' over text, a 'hand' over a link, and so on.\\
\hline
crosshair &	A crosshair or plus sign\\
\hline
default &	An arrow\\
\hline
pointer &	A pointing hand (in IE 4 this value is hand).\\
\hline
move &	The 'I' bar\\
\hline
e-resize &	The cursor indicates that an edge of a box is to be moved right (east).\\
\hline
ne-resize &	The cursor indicates that an edge of a box is to be moved up and right (north/east).\\
\hline
nw-resize &	The cursor indicates that an edge of a box is to be moved up and left (north/west).\\
\hline
n-resize &	The cursor indicates that an edge of a box is to be moved up (north).\\
\hline
se-resize &	The cursor indicates that an edge of a box is to be moved down and right (south/east).\\
\hline
sw-resize &	The cursor indicates that an edge of a box is to be moved down and left (south/west).\\
\hline
s-resize &	The cursor indicates that an edge of a box is to be moved down (south).\\
\hline
w-resize &	The cursor indicates that an edge of a box is to be moved left (west).\\
\hline
text &	The I bar.\\
\hline
wait &	An hour glass.\\
\hline
help &	A question mark or balloon, ideal for use over help buttons.\\
\hline
<url> &	The source of a cursor image file.\\
\hline
\caption{The possible values for the cursor property.}
\end{longtable}
\end{center}
\textbf{Note 1.12.} You should try to use only these values to add helpful information for users, and in places, they would expect to see that cursor. For example, using the crosshair when someone hovers over a link can confuse visitors.

Here is an example.
\begin{verbatim}
<html>
   <head>
   </head>
   
   <body>
      <p>Move the mouse over the words to see the cursor change:</p>
      
      <div style="cursor:auto">Auto</div>
      <div style="cursor:crosshair">Crosshair</div>
      <div style="cursor:default">Default</div>
      <div style="cursor:pointer">Pointer</div>
      <div style="cursor:move">Move</div>
      <div style="cursor:e-resize">e-resize</div>
      <div style="cursor:ne-resize">ne-resize</div>
      <div style="cursor:nw-resize">nw-resize</div>
      <div style="cursor:n-resize">n-resize</div>
      <div style="cursor:se-resize">se-resize</div>
      <div style="cursor:sw-resize">sw-resize</div>
      <div style="cursor:s-resize">s-resize</div>
      <div style="cursor:w-resize">w-resize</div>
      <div style="cursor:text">text</div>
      <div style="cursor:wait">wait</div>
      <div style="cursor:help">help</div>
   </body>
   
</html> 
\end{verbatim}
\section{Outlines}
Outlines are very similar to borders, but there are few major differences as well.
\begin{enumerate}
\item An outline does not take up space.
\item Outlines do not have to be rectangular.
\item Outline is always the same on all sides; you cannot specify different values for different sides of an element.
\end{enumerate}
\textbf{Note 1.13.} The \textit{outline} properties are not supported by IE 6 or Netscape 7.

You can set the following outline properties using CSS.
\begin{enumerate}
\item The \textit{outline-width} property is used to set the width of the outline.
\item The \textit{outline-style} property is used to set the line style for the outline.
\item The \textit{outline-color} property is used to set the color of the outline.
\item The \textit{outline} property is used to set all the above three properties in a single statement.
\end{enumerate}
\subsection{The outline-width Property}
The \textit{outline-width} property specifies the width of the outline to be added to the box. Its value should be a length or one of the values thin, medium, or thick, just like the border-width attribute.

A width of zero pixels means no outline.

Here is an example.
\begin{verbatim}
<html>
   <head>
   </head>
   
   <body>
      <p style="outline-width:thin; outline-style:solid;">
      This text is having thin outline.
      </p>
      <br />
      
      <p style="outline-width:thick; outline-style:solid;">
      This text is having thick outline.
      </p>
      <br />
      
      <p style="outline-width:5px; outline-style:solid;">
      This text is having 5x outline.
      </p>
   </body>
   
</html> 
\end{verbatim}
\subsection{The outline-style Property}
The \textit{outline-style} property specifies the style for the line (solid, dotted, or dashed) that goes around an element. It can take one of the following values.
\begin{enumerate}
\item \textit{none}: No border. (Equivalent of outline-width:0;)
\item \textit{solid}: Outline is a single solid line.
\item \textit{dotted}: Outline is a series of dots.
\item \textit{dashed}: Outline is a series of short lines.
\item \textit{double}: Outline is two solid lines.
\item \textit{groove}: Outline looks as though it is carved into the page.
\item \textit{ridge}: Outline looks the opposite of groove.
\item \textit{inset}: Outline makes the box look like it is embedded in the page.
\item \textit{outset}: Outline makes the box look like it is coming out of the canvas.
\item \textit{hidden}: Same as none.
\end{enumerate}
Here is an example.
\begin{verbatim}
<html>
   <head>
   </head>
   
   <body>
      <p style="outline-width:thin; outline-style:solid;">
      This text is having thin solid  outline.
      </p>
      <br />
      
      <p style="outline-width:thick; outline-style:dashed;">
      This text is having thick dashed outline.
      </p>
      <br />
      
      <p style="outline-width:5px;outline-style:dotted;">
      This text is having 5x dotted outline.
      </p>
   </body>
   
</html> 
\end{verbatim}
\subsection{The outline-color Property}
The \textit{outline-color} property allows you to specify the color of the outline. Its value should either be a color name, a hex color, or an RGB value, as with the color and border-color properties.

Here is an example.
\begin{verbatim}
<html>
   <head>
   </head>
   
   <body>
      <p style="outline-width:thin; outline-style:solid;
      outline-color:red">
      This text is having thin solid red  outline.
      </p>
      <br />
      
      <p style="outline-width:thick; outline-style:dashed;
      outline-color:#009900">
      This text is having thick dashed green outline.
      </p>
      <br />
      
      <p style="outline-width:5px;outline-style:dotted;
      outline-color:rgb(13,33,232)">
      This text is having 5x dotted blue outline.
      </p>
   </body>
   
</html> 
\end{verbatim}
\subsection{The outline Property}
The \textit{outline} property is a shorthand property that allows you to specify values for any of the three properties discussed previously in any order but in a single statement.

Here is an example.
\begin{verbatim}
<html>
   <head>
   </head>
   
   <body>
      <p style="outline:thin solid red;">
      This text is having thin solid red outline.
      </p>
      <br />
       
      <p style="outline:thick dashed #009900;">
      This text is having thick dashed green outline.
      </p>
      <br />
      
      <p style="outline:5px dotted rgb(13,33,232);">
      This text is having 5x dotted blue outline.
      </p>
   </body>
   
</html> 
\end{verbatim}
\section{Dimension}
You have seen the border that surrounds every box ie. element, the padding that can appear inside each box and the margin that can go around them. In this tutorial we will how we can change the dimensions of boxes.

We have the following properties that allow you to control the dimensions of a box.
\begin{enumerate}
\item The \textit{height} property is used to set the height of a box.
\item The \textit{width} property is used to set the width of a box.
\item The \textit{line-height} property is used to set the height of a line of text.
\item The \textit{max-height} property is used to set a maximum height that a box can be.
\item The \textit{min-height} property is used to set the minimum height that a box can be.
\item The \textit{max-width} property is used to set the maximum width that a box can be.
\item The \textit{min-width} property is used to set the minimum width that a box can be.
\end{enumerate}
\subsection{The Height and Width Properties}
The \textit{height} and \textit{width} properties allow you to set the height and width for boxes. They can take values of a length, a percentage, or the keyword auto.

Here is an example.
\begin{verbatim}
<html>
   <head>
   </head>
   <body>
      <p style="width:400px; height:100px; border:1px solid red; 
      padding:5px; margin:10px;">
      This paragraph is 400pixels wide and 100 pixels high
   </p>
   </body>
</html> 
\end{verbatim}
\subsection{The line-height Property}
The line-height property allows you to increase the space between lines of text. The value of the line-height property can be a number, a length, or a percentage.

Here is an example.
\begin{verbatim}
<html>
   <head>
   </head>
   <body>
      <p style="width:400px; height:100px; border:1px solid red; 
      padding:5px; margin:10px; line-height:30px;">
      This paragraph is 400pixels wide and 100 pixels high and
      here line height is 30pixels.
      This paragraph is 400 pixels wide and 100 pixels high and 
      here line height is 30pixels.
      </p>
   </body>
</html>
\end{verbatim}
\subsection{The max-height Property}
The max-height property allows you to specify maximum height of a box. The value of the max-height property can be a number, a length, or a percentage.\\
\\
\textbf{Note 1.14.} This property does not work in either Netscape 7 or IE 6.

Here is an example.
\begin{verbatim}
<html>
   <head>
   </head>  
   <body>
      <p style="width:400px; max-height:10px; border:1px solid red; 
      padding:5px; margin:10px;">
      This paragraph is 400px wide and max height is 10px
      This paragraph is 400px wide and max height is 10px
      This paragraph is 400px wide and max height is 10px
      This paragraph is 400px wide and max height is 10px
      </p>
      <br>
      <br>
      <br>
      <img alt="logo" src="/css/images/logo.png" width="195" height="84" />
   </body>
</html> 
\end{verbatim}
\subsection{The min-height Property}
The \textit{min-height} property allows you to specify minimum height of a box. The value of the min-height property can be a number, a length, or a percentage.\\
\\
\textbf{Note 1.15.} This property does not work in either Netscape 7 or IE 6.

Here is an example.
\begin{verbatim}
<html>
   <head>
   </head>
   <body>
      <p style="width:400px; min-height:200px; border:1px solid red; 
      padding:5px;  margin:10px;">
      This paragraph is 400px wide and min height is 200px
      This paragraph is 400px wide and min height is 200px
      This paragraph is 400px wide and min height is 200px
      This paragraph is 400px wide and min height is 200px
      </p>
      <img alt="logo" src="/css/images/logo.png" width="95" height="84" />
   </body>
</html> 
\end{verbatim}
\subsection{The max-width Property}
The \textit{max-width} property allows you to specify maximum width of a box. The value of the max-width property can be a number, a length, or a percentage.\\
\\
\textbf{Note 1.16.} This property does not work in either Netscape 7 or IE 6.

Here is an example.
\begin{verbatim}
<html>
   <head>
   </head>
   <body>
      <p style="max-width:100px; height:200px; border:1px solid red; 
      padding:5px;  margin:10px;">
      This paragraph is 200px high and max width is 100px
      This paragraph is 200px high and max width is 100px
      </p>
      <img alt="logo" src="/css/images/logo.png" width="95" height="84" />
   </body>
</html> 
\end{verbatim}
\subsection{The min-width Property}
The \textit{min-width} property allows you to specify minimum width of a box. The value of the min-width property can be a number, a length, or a percentage.\\
\\
\textbf{Note 1.17.} This property does not work in either Netscape 7 or IE 6.

Here is an example.
\begin{verbatim}
<html>
   <head>
   </head>
   <body>
      <p style="min-width:400px; height:100px; border:1px solid red; 
      padding:5px;  margin:10px;">
      This paragraph is 100px high and min width is 400px
      This paragraph is 100px high and min width is 400px
      <img alt="logo" src="/css/images/css.gif" width="95" height="84" />
   </body>
</html> 
\end{verbatim}
\section{Scrollbars}
There may be a case when an element's content might be larger than the amount of space allocated to it. For example, given width and height properties do not allow enough room to accommodate the content of the element.

CSS provides a property called \textit{overflow} which tells the browser what to do if the box's contents is larger than the box itself. This property can take one of the following values.
\begin{center}
\begin{longtable}{|c|p{10cm}|}
\hline
\textbf{Value} & \textbf{Description} \\
\hline
visible & Allows the content to overflow the borders of its containing element.\\
\hline
hidden & The content of the nested element is simply cut off at the border of the containing element and no scrollbars is visible.\\
\hline
scroll & The size of the containing element does not change, but the scrollbars are added to allow the user to scroll to see the content.\\
\hline
auto & The purpose is the same as scroll, but the scrollbar will be shown only if the content does overflow.\\
\hline
\caption{Values of \textit{overflow} property.}
\end{longtable}
\end{center}
Here is an example.
\begin{verbatim}
<html>
   <head>
   </head>
   
      <style type="text/css">
         .scroll{
            display:block;
            border: 1px solid red;
            padding:5px;
            margin-top:5px;
            width:300px;
            height:50px;
            overflow:scroll;
         }
         .auto{
            display:block;
            border: 1px solid red;
            padding:5px;
            margin-top:5px;
            width:300px;
            height:50px;
            overflow:auto;
         }
      </style>
      
   <body>
   
      <p>Example of scroll value:</p>
      <div class="scroll">
      I am going to keep lot of content here just to show you how 
      scrollbars works if there is an overflow in an 
      element box. This provides your horizontal as well as vertical
      scrollbars.
      </div>
      <br />
      
      <p>Example of auto value:</p>
      
      <div class="auto">
      I am going to keep lot of content here just to show you how 
      scrollbars works if there is an overflow in an element box. 
      This provides your horizontal as well as vertical scrollbars.
      </div>
      
   </body>
</html> 
\end{verbatim}
\chapter{CSS Advanced}
\section{Visibility}
A property called \textit{visibility} allows you to hide an element from view. You can use this property along with JavaScript to create very complex menu and very complex webpage layouts.

You may choose to use the visibility property to hide error messages that are only displayed if the user needs to see them, or to hide answers to a quiz until the user selects an option.\\
\\
\textbf{Note 2.1.} Remember that the source code will still contain whatever is in the invisible paragraph, so you should not use this to hide sensitive information such as credit card details or passwords.

The \textit{visibility} property can take the values listed in the table that follows.
\begin{center}
\begin{longtable}{|c|p{10cm}|}
\hline
\textbf{Value} & \textbf{Description} \\
\hline
visible & The box and its contents are shown to the user.\\
\hline
hidden & The box and its content are made invisible, although they still affect the layout of the page.\\
\hline
collapse & This is for use only with dynamic table columns and row effects.\\
\hline
\caption{Values of the \textit{visibility} property.}
\end{longtable}
\end{center}
Here is an example.
\begin{verbatim}
<html>
   <head>
   </head>
   <body>
      <p>
      This paragraph should be visible in normal way.
      </p>
   
      <p style="visibility:hidden;">
      This paragraph should not be visible.
      </p>
   </body>
</html> 
\end{verbatim}
\section{Positioning}
CSS helps you to position your HTML element. You can put any HTML element at whatever location you like. You can specify whether you want the element positioned relative to its natural position in the page or absolute based on its parent element.
\subsection{Relative Positioning}
Relative positioning changes the position of the HTML element relative to where it normally appears. So \verb|left:20| adds 20 pixels to the element's LEFT position.

You can use two values top and left along with the position property to move an HTML element anywhere in the HTML document.
\begin{enumerate}
\item Move Left - Use a negative value for \textit{left}.
\item Move Right - Use a positive value for \textit{left}.
\item Move Up - Use a negative value for \textit{top}.
\item Move Down - Use a positive value for \textit{top}.
\end{enumerate}
\textbf{Note 2.2.} You can use \textit{bottom} or \textit{right} values as well in the same way as \textit{top} and \textit{left}.

Here is an example.
\begin{verbatim}
<html>
   <head>
   </head>
   <body>
      <div style="position:relative;left:80px;top:2px;
      background-color:yellow;">
      This div has relative positioning.
      </div>
   </body>
</html>
\end{verbatim}
\subsection{Absolute Positioning}
An element with \verb|position: absolute| is positioned at the specified coordinates relative to your screen top-left corner.

You can use two values \textit{top} and \textit{left} along with the \textit{position} property to move an HTML element anywhere in the HTML document.
\begin{enumerate}
\item Move Left - Use a negative value for \textit{left}.
\item Move Right - Use a positive value for \textit{left}.
\item Move Up - Use a negative value for \textit{top}.
\item Move Down - Use a positive value for \textit{top}.
\end{enumerate}
\textbf{Note 2.3.} You can use \textit{bottom} or \textit{right} values as well in the same way as \textit{top} and \textit{left}.

Here is an example.
\begin{verbatim}
<html>
   <head>
   </head>
   <body>
      <div style="position:absolute; left:80px; top:20px; 
      background-color:yellow;">
      This div has absolute positioning.
      </div>
   </body>
</html> 
\end{verbatim}
\subsection{Fixed Positioning}
Fixed positioning allows you to fix the position of an element to a particular spot on the page, regardless of scrolling. Specified coordinates will be relative to the browser window.

You can use two values \textit{top} and \textit{left} along with the \textit{position} property to move an HTML element anywhere in the HTML document.
\begin{enumerate}
\item Move Left - Use a negative value for \textit{left}.
\item Move Right - Use a positive value for \textit{left}.
\item Move Up - Use a negative value for \textit{top}.
\item Move Down - Use a positive value for \textit{top}.
\end{enumerate}
\textbf{Note 2.4.} You can use \textit{bottom} or \textit{right} values as well in the same way as \textit{top} and \textit{left}.

Here is an example.
\begin{verbatim}
<html>
   <head>
   </head>
   <body>
      <div style="position:fixed; left:80px; top:20px; 
      background-color:yellow;">
      This div has fixed positioning.
      </div>
   </body>
</html>
\end{verbatim}
\section{Layers}
CSS gives you opportunity to create layers of various divisions. The CSS layers refer to applying the \textit{z-index} property to elements that overlap with each other.

The z-index property is used along with the position property to create an effect of layers. You can specify which element should come on top and which element should come at bottom.

A z-index property can help you to create more complex webpage layouts. Following is the example which shows how to create layers in CSS.
\begin{verbatim}
<html>
   <head>
   </head>
   <body>
      <div style="background-color:red; width:300px; height:100px; 
      position:relative; top:10px; left:80px; z-index:2">
      </div>
      
      <div style="background-color:yellow; width:300px; height:100px; 
      position:relative; top:-60px; left:35px; z-index:1;">
      </div>
      
      <div style="background-color:green; width:300px; height:100px; 
      position:relative; top:-220px; left:120px; z-index:3;">
      </div>
   </body>
</html> 
\end{verbatim}
\subsection{Pseudo Classes}
CSS pseudo-classes are used to add special effects to some selectors. You do not need to use JavaScript or any other script to use those effects. A simple syntax of pseudo-classes is as follows.
\begin{verbatim}
selector:pseudo-class {property: value}
\end{verbatim}
CSS classes can also be used with pseudo-classes.
\begin{verbatim}
selector.class:pseudo-class {property: value}
\end{verbatim}
The most commonly used pseudo-classes are as follows.
\begin{center}
\begin{longtable}{|c|p{10cm}|}
\hline
\textbf{Value} & \textbf{Description}\\
\hline
:link &	Use this class to add special style to an unvisited link.\\
\hline
:visited & Use this class to add special style to a visited link.\\
\hline
:hover & Use this class to add special style to an element when you mouse over it.\\
\hline
:active & Use this class to add special style to an active element.\\
\hline
:focus & Use this class to add special style to an element while the element has focus.\\
\hline
:first-child & Use this class to add special style to an element that is the first child of some other element.\\
\hline
:lang &	Use this class to specify a language to use in a specified element.\\
\hline
\caption{The most commonly used pseudo-classes.}
\end{longtable}
\end{center}
While defining pseudo-classes in a $<$style$>$...$<$/style$>$ block, following points should be noted.
\begin{enumerate}
\item a:hover MUST come after a:link and a:visited in the CSS definition in order to be effective.
\item a:active MUST come after a:hover in the CSS definition in order to be effective.
\item Pseudo-class names are not case-sensitive.
\item Pseudo-class are different from CSS classes but they can be combined.
\end{enumerate}
\subsection{The :link pseudo-class}
The following example demonstrates how to use the \textit{:link} class to set the link color. Possible values could be any color name in any valid format.
\begin{verbatim}
<html>
   <head>
      <style type="text/css">
         a:link {color:#000000}
      </style>
   </head>
   <body>
      <a href="">Black Link</a>
   </body>
</html>
\end{verbatim}
\subsection{The :visited pseudo-class}
The following is the example which demonstrates how to use the \textit{:visited} class to set the color of visited links. Possible values could be any color name in any valid format.
\begin{verbatim}
<html>
   <head>
      <style type="text/css">
         a:visited {color: #006600}
      </style>
   </head>
   <body>
      <a href="">Click this link</a>
   </body>
</html> 
\end{verbatim}
\subsection{The :hover pseudo-class}
The following example demonstrates how to use the \textit{:hover} class to change the color of links when we bring a mouse pointer over that link. Possible values could be any color name in any valid format.
\begin{verbatim}
<html>
   <head>
      <style type="text/css">
         a:hover {color: #FFCC00}
      </style>
   </head>
   <body>
      <a href="">Bring Mouse Here</a>
   </body>
</html> 
\end{verbatim}
\subsection{The :active pseudo-class}
The following example demonstrates how to use the \textit{:active} class to change the color of active links. Possible values could be any color name in any valid format.
\begin{verbatim}
<html>
   <head>
      <style type="text/css">
         a:active {color: #FF00CC}
      </style>
   </head>
   <body>
      <a href="">Click This Link</a>
   </body>
</html> 
\end{verbatim}
\subsection{The :focus pseudo-class}
The following example demonstrates how to use the \textit{:focus} class to change the color of focused links. Possible values could be any color name in any valid format.
\begin{verbatim}
<html>
   <head>
      <style type="text/css">
         a:focus {color: #0000FF}
      </style>
   </head>
   <body>
      <a href="">Click this Link</a>
   </body>
</html> 
\end{verbatim}
\subsection{The :first-child pseudo-class}
The \textit{:first-child} pseudo-class matches a specified element that is the first child of another element and adds special style to that element that is the first child of some other element.

To make :first-child work in IE \verb|<!DOCTYPE>| must be declared at the top of document.

For example, to indent the first paragraph of all <div> elements, you could use this definition.
\begin{verbatim}
<html>
   <head>
   
      <style type="text/css">
         div > p:first-child
         {
            text-indent: 25px;
         }
      </style>
      
   </head>
   <body>
   
      <div>
         <p>First paragraph in div. 
         This paragraph will be indented</p>
         <p>Second paragraph in div. 
         This paragraph will not be indented</p>
      </div>
      <p>But it will not match the paragraph in this HTML:</p>
      
      <div>
         <h3>Heading</h3>
         <p>The first paragraph inside the div. 
         This paragraph will not be effected.</p>
      </div>
      
   </body>
</html>
\end{verbatim}
\subsection{The :lang pseudo-class}
The language pseudo-class \textit{:lang}, allows constructing selectors based on the language setting for specific tags.

This class is useful in documents that must appeal to multiple languages that have different conventions for certain language constructs. For example, the French language typically uses angle brackets ($<$ and $>$) for quoting purposes, while the English language uses quote marks (` and ').

In a document that needs to address this difference, you can use the :lang pseudo-class to change the quote marks appropriately. The following code changes the $<$blockquote$>$ tag appropriately for the language being used.
\begin{verbatim}
<html>
   <head>
      <style type="text/css">
         /* Two levels of quotes for two languages*/
         :lang(en) { quotes: '"' '"'  "'"  "'"; }
         :lang(fr) { quotes: "<<" ">>" "<" ">"; }
      </style>
   </head>
   <body>
      <p>...<q lang="fr">A quote in a paragraph</q>...</p>
   </body>
</html>
\end{verbatim}
The :lang selectors will apply to all the elements in the document. However, not all elements make use of the quotes property, so the effect will be transparent for most elements.
\section{Pseudo Elements}
CSS pseudo-elements are used to add special effects to some selectors. You do not need to use JavaScript or any other script to use those effects. A simple syntax of pseudo-element is as follows.
\begin{verbatim}
selector:pseudo-element {property: value}
\end{verbatim}
CSS classes can also be used with pseudo-elements.
\begin{verbatim}
selector.class:pseudo-element {property: value}
\end{verbatim}
The most commonly used pseudo-elements are as follows.
\begin{center}
\begin{longtable}{|c|p{9cm}|}
\hline
\textbf{Value} & \textbf{Description}\\
\hline
:first-line & Use this element to add special styles to the first line of the text in a selector.\\
\hline
:first-letter & Use this element to add special style to the first letter of the text in a selector.\\
\hline
:before & Use this element to insert some content before an element.\\
\hline
:after & Use this element to insert some content after an element.\\
\hline
\caption{The most commonly used pseudo-elements.}
\end{longtable}
\end{center}
\subsection{The :first-line pseudo-element}
The following example demonstrates how to use the \textit{:first-line} element to add special effects to the first line of elements in the document.
\begin{verbatim}
<html>
   <head>
      <style type="text/css">
         p:first-line { text-decoration: underline; }
         p.noline:first-line { text-decoration: none; }
      </style>
   </head>
   <body>
      <p class="noline"> This line would not have any underline 
      because this belongs to  nline class.</p>
      
      <p>The first line of this paragraph will be underlined as 
      defined in the CSS rule above. Rest of the lines in this 
      paragraph will remain normal. This example shows how to
      use :first-line pseduo element to give effect to the 
      first line of any HTML element.</p>
   </body>
</html>
\end{verbatim}
\subsection{The :first-letter pseudo-element}
The following example demonstrates how to use the \textit{:first-letter} element to add special effects to the first letter of elements in the document.
\begin{verbatim}
<html>
   <head>
      <style type="text/css">
         p:first-letter { font-size: 5em; }
         p.normal:first-letter { font-size: 10px; }
      </style>
   </head>
   <body>
      <p class="normal"> First character of this paragraph will be
       normal and will have font size 10 px;</p>
      
      <p>The first character of this paragraph will be 5em big  as
       defined in the CSS rule above. Rest of the characters
       in this paragraph will remain normal. This example shows how 
       to use :first-letter pseduo element to give effect to the 
       first characters  of any HTML element.</p>
   </body>
</html>
\end{verbatim}
\subsection{The :before pseudo-element}
The following example demonstrates how to use the \textit{:before} element to add some content before any element.
\begin{verbatim}
<html>
   <head>
      <style type="text/css">
         p:before
         {
            content: url(/images/bullet.gif)
         }
      </style>
   </head>
   <body>
      <p> This line will be preceded by a bullet.</p>
      <p> This line will be preceded by a bullet.</p>
      <p> This line will be preceded by a bullet.</p>
   </body>
</html>
\end{verbatim}
\subsection{The :after pseudo-element}
The following example demonstrates how to use the \textit{:after} element to add some content after any element.
\begin{verbatim}
<html>
   <head>
      <style type="text/css">
         p:after
         {
            content: url(/images/bullet.gif)
         }
      </style>
   </head>
   <body>
      <p> This line will be succeeded by a bullet.</p>
      <p> This line will be succeeded by a bullet.</p>
      <p> This line will be succeeded by a bullet.</p>
   </body>
</html>
\end{verbatim}
\section{@Rules}
This chapter will cover the following important @ rules.
\begin{enumerate}
\item The \textit{@import} rule imports another style sheet into the current style sheet.
\item The \textit{@charset} rule indicates the character set the style sheet uses.
\item The \textit{@font-face} rule is used to exhaustively describe a font face for use in a document.
\item The \textit{!important} rule indicates that a user-defined rule should take precedence over the author's style sheets.
\end{enumerate}
\subsection{The @import rule}
The @import rule allows you to import styles from another style sheet. It should appear right at the start of the style sheet before any of the rules, and its value is a URL.

It can be written in one of the two following ways.
\begin{verbatim}
<style tyle="text/css">
   <!--
   @import "mystyle.css";
   or
   @import url("mystyle.css");
   .......other CSS rules .....
   -->
</style>
\end{verbatim}
The significance of the @import rule is that it allows you to develop your style sheets with a modular approach. You can create various style sheets and then include them wherever you need them.
\subsection{The @charset Rule}
If you are writing your document using a character set other than ASCII or ISO-8859-1 you might want to set the @charset rule at the top of your style sheet to indicate what character set the style sheet is written in.

The @charset rule must be written right at the beginning of the style sheet without even a space before it. The value is held in quotes and should be one of the standard character-sets. For example
\begin{verbatim}
<style tyle="text/css">
   <!--
   @charset "iso-8859-1"
   .......other CSS rules .....
   -->
</style>
\end{verbatim}
\subsection{The @font-face Rule}
The @font-face rule is used to exhaustively describe a font face for use in a document. @font-face may also be used to define the location of a font for download, although this may run into implementation-specific limits.

In general, @font-face is extremely complicated, and its use is not recommended for any except those who are expert in font metrics.

Here is an example.
\begin{verbatim}
<style tyle="text/css">
   <!--
   @font-face {
      font-family: "Scarborough Light";
      src: url("http://www.font.site/s/scarbo-lt");
   }
   @font-face {
      font-family: Santiago;
      src: local ("Santiago"),
      url("http://www.font.site/s/santiago.tt")
      format("truetype");
      unicode-range: U+??,U+100-220;
      font-size: all;
      font-family: sans-serif;
   }
   -->
</style>
\end{verbatim}
\subsection{The !important Rule}
Cascading Style Sheets cascade. It means that the styles are applied in the same order as they are read by the browser. The first style is applied and then the second and so on.

The !important rule provides a way to make your CSS cascade. It also includes the rules that are to be applied always. A rule having a !important property will always be applied, no matter where that rule appears in the CSS document.

For example, in the following style sheet, the paragraph text will be black, even though the first style property applied is red.
\begin{verbatim}
<style tyle="text/css">
   <!--
   p { color: #ff0000; }
   p { color: #000000; }
   -->
</style>
\end{verbatim}
So, if you wanted to make sure that a property always applied, you would add the !important property to the tag. So, to make the paragraph text always red, you should write it as follows
\begin{verbatim}
<html>
   <head>
   
      <style tyle="text/css">
         p { color: #ff0000 !important; }
         p { color: #000000; }
      </style>
      
   </head>
   <body>
      <p>Tutorialspoint.com</>
   </body>
</html> 
\end{verbatim}
Here you have made \verb|p { color: #ff0000 !important; }| mandatory, now this rule will always apply even you have defined another rule \verb|p { color: #000000; }|.
\section{CSS Filters - Text and Image Effects}
You can use CSS filters to add special effects to text, images and other aspects of a webpage without using images or other graphics. Filters only work on Internet Explorer 4.0+,. If you are developing your site for multi browsers, then it may not be a good idea to use CSS filters because there is a possibility that it would not give any advantage.

In this chapter, we will see the details of each CSS filter. These filters may not work in your browser.
\subsection{Alpha Channel}
The Alpha Channel filter alters the opacity of the object, which makes it blend into the background. The following parameters can be used in this filter.
\begin{center}
\begin{longtable}{|c|p{9cm}|}
\hline
\textbf{Parameter} & \textbf{Description}\\
\hline
opacity & Level of the opacity. 0 is fully transparent, 100 is fully opaque.\\
\hline
finishopacity & Level of the opacity at the other end of the object.\\
\hline
style & The shape of the opacity gradient. 0 = uniform, 1 = linear, 2 = radial, 3 = rectangular\\
\hline
startX & X coordinate for opacity gradient to begin.\\
\hline
startY & Y coordinate for opacity gradient to begin.\\
\hline
finishX & X coordinate for opacity gradient to end.\\
\hline
finishY & Y coordinate for opacity gradient to end.\\
\hline
\caption{Parameters can be used in the Alpha Channel filters.}
\end{longtable}
\end{center}
\textbf{Example 2.5.} 
\begin{verbatim}
<html>
   <head>
   </head>
   
   <body>
      <p>Image Example:</p>
      
      <img src="/css/images/logo.png" alt="CSS Logo" 
         style="Filter: Alpha(Opacity=100, 
         FinishOpacity=0, 
         Style=2, 
         StartX=20, 
         StartY=40, 
         FinishX=0, 
         FinishY=0)" />
      <p>Text Example:</p>
      
      <div style="width: 357; 
         height: 50; 
         font-size: 30pt; 
         font-family: Arial Black; 
         color: blue;
         Filter: Alpha(Opacity=100, FinishOpacity=0, Style=1, 
         StartX=0, StartY=0, FinishX=580, FinishY=0)">CSS Tutorials</div>
   </body>
   
</html> 
\end{verbatim}
\subsection{Motion Blur}
Motion Blur is used to create blurred pictures or text with the direction and strength. The following parameters can be used in this filter.
\begin{center}
\begin{longtable}{|c|p{9cm}|}
\hline
\textbf{Parameter} & \textbf{Description}\\
\hline
add & True or false. If true, the image is added to the blurred image; and if false, the image is not added to the blurred image.\\
\hline
direction & The direction of the blur, going clockwise, rounded to 45-degree increments. The default value is 270 (left). 0 = Top, 45 = Top right, 90 = Right, 135 = Bottom right, 180 = Bottom, 225 = Bottom left, 270 = Left, 315 = Top left\\
\hline
strength & The number of pixels the blur will extend. The default is 5 pixels.\\
\hline
\caption{Parameters can be used in the Motion Blur.}
\end{longtable}
\end{center}
\textbf{Example 2.6.} 
\begin{verbatim}
<html>
   <head>
   </head>
   
   <body>
      <p>Image Example:</p>
      
      <img src="/css/images/logo.png" alt="CSS Logo" 
         style="Filter: Blur(Add = 0, Direction = 225, Strength = 10)">
      
      <p>Text Example:</p>
      
      <div style="width: 357; 
         height: 50; 
         font-size: 30pt; 
         font-family: Arial Black; 
         color: blue; 
         Filter: Blur(Add = 1, Direction = 225, Strength = 10)
         ">CSS Tutorials</div>
   </body>
   
</html> 
\end{verbatim}
\subsection{Chroma Filter}
Chroma Filter is used to make any particular color transparent and usually it is used with images. You can use it with scrollbars also. The following parameter can be used in this filter.
\begin{center}
\begin{longtable}{|l|l|}
\hline
\textbf{Parameter} & \textbf{Description}\\
\hline
color & The color that you would like to be transparent.\\
\hline
\caption{Parameter can be used in the Chroma Filter.}
\end{longtable}
\end{center}
\textbf{Example 2.7.}
\begin{verbatim}
<html>
   <head>
   </head>
   
   <body>
      <p>Image Example:</p>
      
      <img src="/images/css.gif" 
         alt="CSS Logo" style="Filter: Chroma(Color = #FFFFFF)">
      
      <p>Text Example:</p>
      
      <div style="width: 580; 
         height: 50; 
         font-size: 30pt; 
         font-family: Arial Black; 
         color: #3300FF; 
         Filter: Chroma(Color = #3300FF)">CSS Tutorials</div>
   </body>
   
</html>
\end{verbatim}
\subsection{Drop Shadow Effect}
Drop Shadow is used to create a shadow of your object at the specified X (horizontal) and Y (vertical) offset and color.

The following parameters can be used in this filter.
\begin{center}
\begin{longtable}{|c|p{9cm}|}
\hline
\textbf{Parameter} & \textbf{Description}\\
\hline
color & The color, in \verb|#|RRGGBB format, of the dropshadow.\\
\hline
offX & Number of pixels the drop shadow is offset from the visual object, along the x-axis. Positive integers move the drop shadow to the right, negative integers move the drop shadow to the left.\\
\hline
offY & Number of pixels the drop shadow is offset from the visual object, along the y-axis. Positive integers move the drop shadow down, negative integers move the drop shadow up.\\
\hline
positive & If true, all opaque pixels of the object have a dropshadow. If false, all transparent pixels have a dropshadow. The default is true.\\
\hline
\caption{Parameters can be used in the Drop Shadow.}
\end{longtable}
\end{center}
\textbf{Example 2.8.}
\begin{verbatim}
<html>
   <head>
   </head>
   
   <body>
      <p>Image Example:</p>
      
      <img src="/css/images/logo.png" 
         alt="CSS Logo" 
         style="Filter: Chroma(Color = #000000) 
         DropShadow(Color=#FF0000, 
         OffX=2, 
         OffY=2, Positive=1)">
      
      <p>Text Example:</p>
      
      <div style="width: 357; 
         height: 50; 
         font-size: 30pt; 
         font-family: Arial Black; 
         color: red; 
         Filter: DropShadow(Color=#000000, OffX=2, OffY=2, Positive=1)
         ">CSS Tutorials</div>
   </body>
   
</html>
\end{verbatim}
\subsection{Flip Effect}
Flip effect is used to create a mirror image of the object. The following parameters can be used in this filter.
\begin{center}
\begin{longtable}{|c|l|}
\hline
\textbf{Parameter} & \textbf{Description}\\
\hline
FlipH & Creates a horizontal mirror image.\\
\hline
FlipV & Creates a vertical mirror image.\\
\hline
\caption{Parameters can be used in the Flip effect.}
\end{longtable}
\end{center}
\textbf{Example 2.9.} 
\begin{verbatim}
<html>
   <head>
   </head>
   
   <body>
      <p>Image Example:</p>
      
      <img src="/css/images/logo.png" 
         alt="CSS Logo" 
         style="Filter: FlipH">
      
      <img src="/css/images/logo.png" alt="CSS Logo" style="Filter: FlipV">
      
      <p>Text Example:</p>
      
      <div style="width: 300; 
         height: 50; 
         font-size: 30pt; 
         font-family: Arial Black; 
         color: red; 
         Filter: FlipV">CSS Tutorials</div>
   </body>
   
</html>
\end{verbatim}
\subsection{Glow Effect}
Glow effect is used to create a glow around the object. If it is a transparent image, then glow is created around the opaque pixels of it. The following parameters can be used in this filter.
\begin{center}
\begin{longtable}{|c|l|}
\hline
\textbf{Parameter} & \textbf{Description}\\
\hline
color & The color you want the glow to be.\\
\hline
strength & The intensity of the glow (from 1 to 255).\\
\hline
\caption{Parameters can be used in the Glow effect.}
\end{longtable}
\end{center}
\textbf{Example 2.10.} 
\begin{verbatim}
<html>
   <head>
   </head>
   
   <body>
      <p>Image Example:</p>
      
      <img src="/css/images/logo.png" 
         alt="CSS Logo" 
         style="Filter: Chroma(Color = #000000) 
         Glow(Color=#00FF00, Strength=20)">
      
      <p>Text Example:</p>
      
      <div style="width: 357; 
         height: 50; 
         font-size: 30pt; 
         font-family: Arial Black; 
         color: red; 
         Filter: Glow(Color=#00FF00, Strength=20)">CSS Tutorials</div>
   </body>
   
</html> 
\end{verbatim}
\subsection{Grayscale Effect}
Grayscale effect is used to convert the colors of the object to 256 shades of gray. The following parameter is used in this filter.
\begin{center}
\begin{longtable}{|c|l|}
\hline
\textbf{Parameter} & \textbf{Description}\\
\hline
gray & Converts the colors of the object to 256 shades of gray.\\
\hline
\caption{Parameter can be used in Grayscale effect.}
\end{longtable}
\end{center}
\textbf{Example 2.11.}
\begin{verbatim}
<html>
   <head>
   </head>
   
   <body>
      <p>Image Example:</p>
      
      <img src="/css/images/logo.png" 
         alt="CSS Logo" 
         style="Filter: Gray">
      
      <p>Text Example:</p>
      
      <div style="width: 357; 
         height: 50; 
         font-size: 30pt; 
         font-family: Arial Black; 
         color: red; 
         Filter: Gray">CSS Tutorials</div>
   </body>
   
</html> 
\end{verbatim}
\subsection{Invert Effect}
Invert effect is used to map the colors of the object to their opposite values in the color spectrum, i.e., to create a negative image. The following parameter is used in this filter.
\begin{center}
\begin{longtable}{|c|p{9cm}|}
\hline
\textbf{Parameter} & \textbf{Description}\\
\hline
Invert & Maps the colors of the object to their opposite value in the color spectrum.\\
\hline
\caption{Parameter can be used in Invert effect.}
\end{longtable}
\end{center}
\textbf{Example 2.12.}
\begin{verbatim}
<html>
   <head>
   </head>
   
   <body>
      <p>Image Example:</p>
      
      <img src="/images/css.gif" 
         alt="CSS Logo" 
         style="Filter: invert">
      
      <p>Text Example:</p>
      
      <div style="width: 357; 
         height: 50; 
         font-size: 30pt; 
         font-family: Arial Black; 
         color: red; 
         Filter: invert">CSS Tutorials</div>
   </body>
   
</html> 
\end{verbatim}
\subsection{Mask Effect}
Mask effect is used to turn transparent pixels to a specified color and makes opaque pixels transparent. The following parameter is used in this filter.
\begin{center}
\begin{longtable}{|c|p{9cm}|}
\hline
\textbf{Parameter} & \textbf{Description}\\
\hline
color & The color that the transparent areas will become.\\
\hline
\caption{Parameter can be used in Mask effect.}
\end{longtable}
\end{center}
\textbf{Example 2.13.}
\begin{verbatim}
<html>
   <head>
   </head>
   
   <body>
      <p>Image Example:</p>
      
      <img src="/css/images/logo.png" 
         alt="CSS Logo" 
         style="FILTER: Chroma(Color = #000000) Mask(Color=#00FF00)">
      
      <p>Text Example:</p>
      
      <div style="width: 357; 
         height: 50; 
         font-size: 30pt; 
         font-family: Arial Black; 
         color: red; 
         Filter: Mask(Color=#00FF00)">CSS Tutorials</div>
   </body>
   
</html> 
\end{verbatim}
\subsection{Shadow Filter}
Shadow filter is used to create an attenuated shadow in the direction and color specified. This is a filter that lies in between Dropshadow and Glow. The following parameters can be used in this filter.
\begin{center}
\begin{longtable}{|c|p{9cm}|}
\hline
\textbf{Parameter} & \textbf{Description}\\
\hline
color & The color that you want the shadow to be.\\
\hline
direction & The direction of the blur, going clockwise, rounded to 45-degree increments. The default value is 270 (left). 0 = Top, 45 = Top right, 90 = Right, 135 = Bottom right, 180 = Bottom, 225 = Bottom left, 270 = Left, 315 = Top left\\
\hline
\caption{Parameters can be used in Shadow Filter.}
\end{longtable}
\end{center}
\textbf{Example 2.14.}
\begin{verbatim}
<html>
   <head>
   </head>
   
   <body>
      <p>Image Example:</p>
      
      <img src="/css/images/logo.png" 
         alt="CSS Logo" 
         style="FILTER: Chroma(Color = #000000) 
         Shadow(Color=#00FF00, Direction=225)">
      
      <p>Text Example:</p>
      
      <div style="width: 357; 
         height: 50; 
         font-size: 30pt; 
         font-family: 
         Arial Black; 
         color: red; 
         Filter: Shadow(Color=#0000FF, Direction=225)">CSS Tutorials</div>
   </body>
   
</html> 
\end{verbatim}
\subsection{Wave Effect}
Wave effect is used to give the object a sine wave distortion to make it look wavy. The following parameters can be used in this filter.
\begin{center}
\begin{longtable}{|c|p{9cm}|}
\hline
\textbf{Parameter} & \textbf{Description}\\
\hline
add & A value of 1 adds the original image to the waved image, 0 does not.\\
\hline
freq & The number of waves.\\
\hline
light & The strength of the light on the wave (from 0 to 100).\\
\hline
phase & At what degree the sine wave should start (from 0 to 100).\\
\hline
strength & The intensity of the wave effect.\\
\hline
\caption{Parameters can be used in Wave effect.}
\end{longtable}
\end{center}
\textbf{Example 2.15.} 
\begin{verbatim}
<html>
   <head>
   </head>
   
   <body>
      <p>Image Example:</p>
      
      <img src="/css/images/logo.png" 
         alt="CSS Logo" 
         style="FILTER: Chroma(Color = #000000) 
         Wave(Add=0, Freq=1, LightStrength=10, Phase=220, Strength=10)">
      
      <p>Text Example:</p>
      
      <div style="width: 357; 
         height: 50; 
         font-size: 30pt; 
         font-family: Arial Black; 
         color: red; 
         Filter: Wave(Add=0, Freq=1, LightStrength=10, Phase=20, 
         Strength=20)">CSS Tutorials</div>
   </body>
   
</html>
\end{verbatim}
\subsection{X-Ray Effect}
X-Ray effect grayscales and flattens the color depth. The following parameter is used in this filter.
\begin{center}
\begin{longtable}{|c|p{9cm}|}
\hline
\textbf{Parameter} & \textbf{Description}\\
\hline
xray & Grayscales and flattens the color depth.\\
\hline
\caption{Parameter can be used in X-Ray effect.}
\end{longtable}
\end{center}
\textbf{Example 2.16.} 
\begin{verbatim}
<html>
   <head>
   </head>
   
   <body>
      <p>Image Example:</p>
      
      <img src="/css/images/logo.png" 
         alt="CSS Logo" 
         style="Filter: Xray"">
      
      <p>Text Example:</p>
      
      <div style="width: 357; 
         height: 50; 
         font-size: 30pt; 
         font-family: Arial Black; 
         color: red; 
         style="Filter: Xray">CSS Tutorials</div>
   </body>
   
</html>
\end{verbatim}
\subsection{Media Types}
One of the most important features of style sheets is that they specify how a document is to be presented on different media: on the screen, on paper, with a speech synthesizer, with a braille device, etc.

We have currently two ways to specify media dependencies for style sheets.
\begin{enumerate}
\item Specify the target medium from a style sheet with the @media or @import at-rules.
\item Specify the target medium within the document language.
\end{enumerate}
\subsection{The @media Rule}
An @media rule specifies the target media types (separated by commas) of a set of rules.

Given below is an example.
\begin{verbatim}
<style tyle="text/css">
   <!--
   @media print {
      body { font-size: 10pt }
   }
	
   @media screen {
      body { font-size: 12pt }
   }
   @media screen, print {
      body { line-height: 1.2 }
   }
   -->
</style
\end{verbatim}
\subsection{The Document Language}
In HTML 4.0, the \textit{media} attribute on the LINK element specifies the target media of an external style sheet.

Following is an example.
\begin{verbatim}
<style tyle="text/css">
   <!--
   <!doctype html public "-//w3c//dtd html 4.0//en">
	
   <html>
	
      <head>
         <title>link to a target medium</title>
         <link rel="stylesheet" type="text/css" media="print, 
         handheld" href="foo.css">
      </head>
		
      <body>
         <p>the body...
      </body>
		
   </html>
   -->
</style>
\end{verbatim}
\subsection{Recognized Media Types}
The names chosen for CSS media types reflect target devices for which the relevant properties make sense. They give a sense of what device the media type is meant to refer to. Given below is a list of various media types.
\begin{center}
\begin{longtable}{|c|p{9cm}|}
\hline
\textbf{Value} & \textbf{Description}\\
\hline
all & Suitable for all devices.\\
\hline
aural & Intended for speech synthesizers.\\
\hline
braille & Intended for braille tactile feedback devices.\\
\hline
embossed & Intended for paged braille printers.\\
\hline
handheld & Intended for handheld devices (typically small screen, monochrome, limited bandwidth).\\
\hline
print & Intended for paged, opaque material and for documents viewed on screen in print preview mode. Please consult the section on paged media.\\
\hline
projection & Intended for projected presentations, for example projectors or print to transparencies. Please consult the section on paged media.\\
\hline
screen &	Intended primarily for color computer screens.\\
\hline
tty & Intended for media using a fixed-pitch character grid, such as teletypes, terminals, or portable devices with limited display capabilities.\\
\hline
tv &	Intended for television-type devices.\\
\hline
\caption{A list of various media types.}
\end{longtable}
\end{center}
\textbf{Note 2.17.} Media type names are case-insensitive.
\section{CSS Paged Media, @page Rule}
Paged media differ from continuous media in that the content of the document is split into one or more discrete pages. Paged media includes paper, transparencies, pages that are displayed on computer screens, etc.

The CSS2 standard introduces some basic pagination control features that let authors help the browser figure out how to best print their documents.

The CSS2 page model specifies how a document is formatted within a rectangular area -- the page box -- that has a finite width and height. These features fall into two groups.
\begin{enumerate}
\item CSS2 features that define a particular page layout.
\item CSS2 features that control the pagination of a document.
\end{enumerate}
\subsection{Defining Pages: the @page Rule}
The CSS2 defines a ``page box'', a box of finite dimensions in which content is rendered. The page box is a rectangular region that contains two areas.
\begin{enumerate}
\item \textit{The page area}. The page area includes the boxes laid out on that page. The edges of the page area act as the initial containing block for layout that occurs between page breaks.
\item T\textit{he margin area}. It surrounds the page area.
\end{enumerate}
You can specify the dimensions, orientation, margins, etc., of a page box within an @page rule. The dimensions of the page box are set with the 'size' property. The dimensions of the page area are the dimensions of the page box minus the margin area.

For example, the following @page rule sets the page box size to 8.5 × 11 inches and creates '2cm' margin on all sides between the page box edge and the page area.
\begin{verbatim}
<style type="text/css">
   <!--
   @page { size:8.5in 11in; margin: 2cm }
   -->
</style>
\end{verbatim}
You can use the margin, margin-top, margin-bottom, margin-left, and margin-right properties within the @page rule to set margins for your page.

Finally, the marks property is used within the @page rule to create crop and registration marks outside the page box on the target sheet. By default, no marks are printed. You may use one or both of the crop and cross keywords to create crop marks and registration marks, respectively, on the target print page.
\subsection{Setting Page Size}
The size property specifies the size and orientation of a page box. There are four values which can be used for page size.
\begin{enumerate}
\item \textit{auto}. The page box will be set to the size and orientation of the target sheet.
\item \textit{landscape}. Overrides the target's orientation. The page box is the same size as the target, and the longer sides are horizontal.
\item \textit{portrait}. Overrides the target's orientation. The page box is the same size as the target, and the shorter sides are horizontal.
\item \textit{length}. Length values for the `size' property create an absolute page box. If only one length value is specified, it sets both the width and height of the page box. Percentage values are not allowed for the `size' property.
\end{enumerate}

In the following example, the outer edges of the page box will align with the target. The percentage value on the 'margin' property is relative to the target size so if the target sheet dimensions are 21.0cm $\times$ 29.7cm (i.e., A4), the margins are 2.10cm and 2.97cm.
\begin{verbatim}
<style type="text/css">
   <!--
   @page {
      size: auto;   /* auto is the initial value */
      margin: 10%;
   }
   -->
</style>
\end{verbatim}
The following example sets the width of the page box to be 8.5 inches and the height to be 11 inches. The page box in this example requires a target sheet size of $8.5''\times  11"$ or larger.
\begin{verbatim}
<style type="text/css">
   <!--
   @page {
      size: 8.5in 11in;  /* width height */
   }
   -->
</style>
\end{verbatim}
Once you create a named page layout, you can use it in your document by adding the page property to a style that is later applied to an element in your document. For example, this style renders all the tables in your document on landscape pages.
\begin{verbatim}
<style type="text/css">
   <!--
   @page { size : portrait }
   @page rotated { size : landscape }
   table { page : rotated }
   -->
</style>
\end{verbatim}
Due to the above rule, while printing, if the browser encounters a <table> element in your document and the current page layout is the default portrait layout, it starts a new page and prints the table on a landscape page.
\subsection{Left, Right and First Pages}
When printing double-sided documents, the page boxes on left and right pages should be different. It can be expressed through two CSS pseudo-classes as follows.
\begin{verbatim}
<style type="text/css">
   <!--
   @page :left {
      margin-left: 4cm;
      margin-right: 3cm;
   }

   @page :right {
      margin-left: 3cm;
      margin-right: 4cm;
   }
   -->
</style>
\end{verbatim}
You can specify the style for the first page of a document with the :first pseudo-class.
\begin{verbatim}
<style type="text/css">
   <!--
   @page { margin: 2cm } /* All margins set to 2cm */

   @page :first {
      margin-top: 10cm    /* Top margin on first page 10cm */
   }
   -->
</style>
\end{verbatim}
\subsection{Controlling Pagination}
Unless you specify otherwise, page breaks occur only when the page format changes or when the content overflows the current page box. To otherwise force or suppress page breaks, use the \textit{page-break-before, page-break-after}, and \textit{page-break-inside} properties.

Both the \textit{page-break-before} and \textit{page-break-after} accept the \textit{auto, always, avoid, left} and \textit{right} keywords.

The keyword \textit{auto} is the default, it lets the browser generate page breaks as needed. The keyword \textit{always} forces a page break before or after the element, while \textit{avoid} suppresses a page break immediately before or after the element. The \textit{left} and \textit{right} keywords force one or two page breaks, so that the element is rendered on a left-hand or right-hand page.

Using pagination properties is quite straightforward. Suppose your document has level-1 headers start new chapters with level-2 headers to denote sections. You'd like each chapter to start on a new, right-hand page, but you don't want section headers to be split across a page break from the subsequent content. You can achieve this using following rule.
\begin{verbatim}
<style type="text/css">
   <!--
   h1 { page-break-before : right }
   h2 { page-break-after : avoid }
   -->
</style>
\end{verbatim}
Use only the \textit{auto} and \textit{avoid} values with the\textit{ page-break-inside} property. If you prefer that your tables not be broken across pages if possible, you would write the rule
\begin{verbatim}
<style type="text/css">
   <!--
   table { page-break-inside : avoid }
   -->
</style>
\end{verbatim}
\subsection{Controlling Windows and Orphans}
In typographic lingo, orphans are those lines of a paragraph stranded at the bottom of a page due to a page break, while widows are those lines remaining at the top of a page following a page break. Generally, printed pages do not look attractive with single lines of text stranded at the top or bottom. Most printers try to leave at least two or more lines of text at the top or bottom of each page.
\begin{enumerate}
\item The orphans property specifies the minimum number of lines of a paragraph that must be left at the bottom of a page.
\item The widows property specifies the minimum number of lines of a paragraph that must be left at the top of a page.
\end{enumerate}
Here is the example to create 4 lines at the bottom and 3 lines at the top of each page.
\begin{verbatim}
<style type="text/css">
   <!--
   @page{orphans:4; widows:2;}
   -->
</style>
\end{verbatim}
\section{Aural Media}
A web document can be rendered by a speech synthesizer. CSS2 allows you to attach specific sound style features to specific document elements.

Aural rendering of documents is mainly used by the visually impaired. Some of the situations in which a document can be accessed by means of aural rendering rather than visual rendering are the following.
\begin{enumerate}
\item Learning to read
\item Training
\item Web access in vehicles
\item Home entertainment
\item Industrial documentation
\item Medical documentation
\end{enumerate}
When using aural properties, the canvas consists of a three-dimensional physical space (sound surrounds) and a temporal space (one may specify sounds before, during, and after other sounds).

The CSS properties also allow you to vary the quality of synthesized speech (voice type, frequency, inflection, etc.)

Here is an example.
\begin{verbatim}
html>
   <head>
   
      <style tyle="text/css">
         h1, h2, h3, h4, h5, h6 {
            voice-family: paul;
            stress: 20;
            richness: 90;
            cue-before: url("../audio/pop.au");
         }
         p{
            azimuth:center-right;
         }
      </style>
      
   </head>
   <body>
   
      <h1>Tutorialspoint.com</h1>
      <h2>Tutorialspoint.com</h2>
      <h3>Tutorialspoint.com</h3>
      <h4>Tutorialspoint.com</h4>
      <h5>Tutorialspoint.com</h5>
      <h6>Tutorialspoint.com</h6>
      <p>Tutorialspoint.com</p>
      
   </body>
</html> 
\end{verbatim}
It will direct the speech synthesizer to speak headers in a voice (a kind of audio font) called "paul", on a flat tone, but in a very rich voice. Before speaking the headers, a sound sample will be played from the given URL.

Now, we will see various properties related to aural media.
\begin{enumerate}
\item The \textit{azimuth} property sets, where the sound should come from horizontally.
\item The \textit{elevation} property sets, where the sound should come from vertically.
\item The \textit{cue-after} specifies a sound to be played after speaking an element's content to delimit it from other.
\item The \textit{cue-before} specifies a sound to be played before speaking an element's content to delimit it from other.
\item The \textit{cue} is a shorthand for setting cue-before and cue-after.
\item The \textit{pause-after} specifies a pause to be observed after speaking an element's content.
\item The \textit{pause-before} specifies a pause to be observed before speaking an element's content.
\item The \textit{pause} is a shorthand for setting pause-before and pause-after.
\item The \textit{pitch} specifies the average pitch (a frequency) of the speaking voice.
\item The \textit{pitch-range} specifies variation in average pitch.
\item The \textit{play-during} specifies a sound to be played as a background while an element's content is spoken.
\item The \textit{richness} specifies the richness, or brightness, of the speaking voice.
\item The \textit{speak} specifies whether text will be rendered aurally and if so, in what manner.
\item The \textit{speak-numeral} controls how numerals are spoken.
\item The \textit{speak-punctuation} specifies how punctuation is spoken.
\item The \textit{speech-rate} specifies the speaking rate.
\item The \textit{stress} specifies the height of ``local peaks'' in the intonation contour of a voice.
\item The \textit{voice-family} specifies the prioritized list of voice family names.
\item The \textit{volume} refers to the median volume of the voice.
\end{enumerate}
\subsection{The azimuth Property}
The azimuth property sets where the sound should come from horizontally. The possible values are listed below.
\begin{enumerate}
\item \textit{angle}. Position is described in terms of an angle within the range -360deg to 360deg. The value 0deg means directly ahead in the center of the sound stage. 90deg is to the right, 180deg behind, and 270deg (or, equivalently and more conveniently, -90deg) to the left.
\item \textit{left-side}. Same as `270deg'. With `behind', `270deg'.
\item \textit{far-left}. Same as `300deg'. With `behind', `240deg'.
\item \textit{left}. Same as `320deg'. With `behind', `220deg'.
\item \textit{center-left}. Same as `340deg'. With `behind', `200deg'.
\item \textit{center}. Same as `0deg'. With `behind', `180deg'.
\item \textit{center-right}. Same as `20deg'. With `behind', `160deg'.
\item \textit{right}. Same as `40deg'. With `behind', `140deg'.
\item \textit{far-right}. Same as `60deg'. With `behind', `120deg'.
\item \textit{right-side}. Same as `90deg'. With `behind', `90deg'.
\item \textit{leftwards}. Moves the sound to the left and relative to the current angle. More precisely, subtracts 20 degrees.
\item \textit{rightwards}. Moves the sound to the right, relative to the current angle. More precisely, adds 20 degrees.
\end{enumerate}
Here is an example.
\begin{verbatim}
<style tyle="text/css">
   <!--
   h1   { azimuth: 30deg }
   td.a { azimuth: far-right }          /*  60deg */
   #12  { azimuth: behind far-right }   /* 120deg */
   p.comment { azimuth: behind }        /* 180deg */
   -->
</style>
\end{verbatim}
\subsection{The elevation Property}
The elevation property sets where the sound should come from vertically. The possible values are as follows.
\begin{enumerate}
\item \textit{angle}. Specifies the elevation as an angle, between -90deg and 90deg. 0deg means on the forward horizon, which loosely means level with the listener. 90deg means directly overhead and -90deg means directly below.
\item \textit{below}. Same as '-90deg'.
\item \textit{level}. Same as '0deg'.
\item \textit{above}. Same as '90deg'.
\item \textit{higher}. Adds 10 degrees to the current elevation.
\item \textit{lower}. Subtracts 10 degrees from the current elevation.
\end{enumerate}
Here is an example.
\begin{verbatim}
<style tyle="text/css">
   <!--
   h1   { elevation: above }
   tr.a { elevation: 60deg }
   tr.b { elevation: 30deg }
   tr.c { elevation: level }
   -->
</style>
\end{verbatim}
\subsection{The cue-after Property}
The cue-after property specifies a sound to be played after speaking an element's content to delimit it from other. The possible values include 
\begin{enumerate}
\item \textit{url}. The URL of a sound file to be played.
\item \textit{none}. Nothing has to be played.
\end{enumerate}
Here is an example.
\begin{verbatim}
<style tyle="text/css">
   <!--
   a {cue-after: url("dong.wav");}
   h1 {cue-after: url("pop.au"); }
   -->
</style>
\end{verbatim}
\subsection{The cue Property}
The cue property is a shorthand for setting cue-before and cue-after. If two values are given, the first value is cue-before and the second is cue-after. If only one value is given, it applies to both properties.

For example, the following two rules are equivalent.
\begin{verbatim}
<style tyle="text/css">
   <!--
   h1 {cue-before: url("pop.au"); cue-after: url("pop.au") }
   h1 {cue: url("pop.au") }
   -->
</style>
\end{verbatim}
\subsection{The pause-after Property}
This property specifies a pause to be observed after speaking an element's content. The possible values are
\begin{enumerate}
\item \textit{time}. Expresses the pause in absolute time units (seconds and milliseconds).
\item \textit{percentage}. Refers to the inverse of the value of the \textit{speech-rate} property. For example, if the speech-rate is 120 words per minute (i.e. a word takes half a second, or 500ms), then a \textit{pause-after} of 100\% means a pause of 500 ms and a \textit{pause-after} of 20\% means 100ms.
\end{enumerate}
\subsection{The pause-before Property}
This property specifies a pause to be observed before speaking an element's content. The possible values are
\begin{enumerate}
\item \textit{time}. Expresses the pause in absolute time units (seconds and milliseconds).
\item \textit{percentage}. Refers to the inverse of the value of the \textit{speech-rate} property. For example, if the speech-rate is 120 words per minute (i.e. a word takes half a second, or 500ms), then a \textit{pause-before} of 100\% means a pause of 500 ms and a \textit{pause-before} of 20\% means 100ms.
\end{enumerate}
\subsection{The pause Property}
This property is a shorthand for setting \textit{pause-before} and \textit{pause-after}. If two values are given, the first value is \textit{pause-before} and the second is \textit{pause-after}.

Here is an example.
\begin{verbatim}
<style tyle="text/css">
   <!--
   /* pause-before: 20ms; pause-after: 20ms */
   h1 { pause : 20ms }  
	
   /* pause-before: 30ms; pause-after: 40ms */
   h2{ pause : 30ms 40ms }  
	
   /* pause-before: ?; pause-after: 10ms */
   h3 { pause-after : 10ms }
   -->
</style>
\end{verbatim}
\subsection{The pitch Property}
This property specifies the average pitch (a frequency) of the speaking voice. The average pitch of a voice depends on the voice family. For example, the average pitch for a standard male voice is around 120Hz, but for a female voice, it's around 210Hz. The possible values are
\begin{enumerate}
\item \textit{frequency}. Specifies the average pitch of the speaking voice in hertz (Hz).
\item \textit{x-low, low, medium, high, x-high}. These values do not map to absolute frequencies since these values depend on the voice family.
\end{enumerate}
\subsection{The pitch-range Property}
This property specifies variation in average pitch. The possible values are
\begin{enumerate}
\item \textit{number}. A value between `0' and `100'. A pitch range of `0' produces a flat, monotonic voice. A pitch range of 50 produces normal inflection. Pitch ranges greater than 50 produce animated voices.
\end{enumerate}
\subsection{The play-during Property}
This property specifies a sound to be played as a background while an element's content is spoken. Possible values could be any of the followings.
\begin{enumerate}
\item \textit{URI}. The sound designated by this $<$uri$>$ is played as a background while the element's content is spoken.
\item \textit{mix}. When present, this keyword means that the sound inherited from the parent element's \textit{play-during} property continues to play and the sound designated by the \textit{uri} is mixed with it. If \textit{mix} is not specified, the element's background sound replaces the parent's.
\item \textit{repeat}. When present, this keyword means that the sound will repeat if it is too short to fill the entire duration of the element. Otherwise, the sound plays once and then stops.
\item \textit{auto}. The sound of the parent element continues to play.
\item \textit{none}. This keyword means that there is silence.
\end{enumerate}
Here is an example.
\begin{verbatim}
<style tyle="text/css">
   <!--
   blockquote.sad { play-during: url("violins.aiff") }
   blockquote q   { play-during: url("harp.wav") mix }
   span.quiet     { play-during: none }
   -->
</style>
\end{verbatim}
\subsection{The richness Property}
This property specifies the richness or brightness of the speaking voice. The possible values are
\begin{enumerate}
\item \textit{number}. A value between `0' and `100'. The higher the value, the more the voice will carry. A lower value will produce a soft, mellifluous voice.
\end{enumerate}
\subsection{The speak Property}
This property specifies whether text will be rendered aurally and if so, in what manner. The possible values are 
\begin{enumerate}
\item \textit{none}. Suppresses aural rendering so that the element requires no time to render.
\item \textit{normal}. Uses language-dependent pronunciation rules for rendering an element and its children.
\item \textit{spell-out}. Spells the text one letter at a time.
\end{enumerate}
Note the difference between an element whose `volume' property has a value of `silent' and an element whose `speak' property has the value `none'. The former takes up the same time as if it had been spoken, including any pause before and after the element, but no sound is generated. The latter requires no time and is not rendered.
\subsection{The speak-numerical Property}
This property controls how numerals are spoken. The possible values are
\begin{enumerate}
\item \textit{digits}. Speak the numeral as individual digits. Thus, ``237'' is spoken ``Two Three Seven''.
\item \textit{continuous}. Speak the numeral as a full number. Thus, ``237'' is spoken ``Two hundred thirty seven''. Word representations are language-dependent.
\end{enumerate}
\subsection{The speak-punctuation Property}
This property specifies how punctuation is spoken. The possible values are
\begin{enumerate}
\item \textit{code}. Punctuation such as semicolons, braces, and so on are to be spoken literally.
\item \textit{none}. Punctuation is not to be spoken, but instead rendered naturally as various pauses.
\end{enumerate}
\subsection{The speech-rate Property}
This property specifies the speaking rate. Note that both absolute and relative keyword values are allowed. The possible values are
\begin{enumerate}
\item number − Specifies the speaking rate in words per minute.
\item \textit{x-slow}. Same as 80 words per minute.
\item \textit{slow}. Same as 120 words per minute.
\item \textit{medium}. Same as 180 - 200 words per minute.
\item \textit{fast}. Same as 300 words per minute.
\item \textit{x-fast}. Same as 500 words per minute.
\item \textit{faster}. Adds 40 words per minute to the current speech rate.
\item \textit{slower}. Subtracts 40 words per minutes from the current speech rate.
\end{enumerate}
\subsection{The stress Property}
This property specifies the height of ``local peaks'' in the intonation contour of a voice. English is a stressed language, and different parts of a sentence are assigned primary, secondary, or tertiary stress. The possible values are
\begin{enumerate}
\item \textit{number}. A value between `0' and `100'. The meaning of values depends on the language being spoken. For example, a level of `50' for a standard, English-speaking male voice (average pitch = 122Hz), speaking with normal intonation and emphasis would have a different meaning than `50' for an Italian voice.
\end{enumerate}
\subsection{The voice-family Property}
The value is a comma-separated, prioritized list of voice family names. It can have following values.
\begin{enumerate}
\item \textit{generic-voice}. Values are voice families. Possible values are 'male', 'female', and 'child'.
\item \textit{specific-voice}. Values are specific instances (e.g., comedian, trinoids, carlos, lani).
\end{enumerate}
Here is an example.
\begin{verbatim}
<style tyle="text/css">
   <!--
   h1 { voice-family: announcer, male }
   p.part.romeo  { voice-family: romeo, male }
   p.part.juliet { voice-family: juliet, female }
   -->
</style>
\end{verbatim}
\subsection{The volume Property}
Volume refers to the median volume of the voice. It can have following values
\begin{enumerate}
\item \textit{numbers}. Any number between `0' and `100'. `0' represents the minimum audible volume level and `100' corresponds to the maximum comfortable level.
\item \textit{percentage}. These values are calculated relative to the inherited value, and are then clipped to the range `0' to `100'.
\item \textit{silent}. No sound at all. The value `0' does not mean the same as `silent'.
\item \textit{x-soft}. Same as `0'.
\item \textit{soft}. Same as `25'.
\item \textit{medium}. Same as `50'.
\item \textit{loud}. Same as `75'.
\item \textit{x-loud}. Same as `100'.
\end{enumerate}
Here is an example.
\begin{verbatim}
<style tyle="text/css">
   <!--
   P.goat  { volume: x-soft }
   -->
</style>
\end{verbatim}
Paragraphs with class \textit{goat} will be very soft.
\subsection{@media Rule}
You can use CSS to change the appearance of your web page when it's printed on a paper. You can specify one font for the screen version and another for the print version.

You have seen @media rule in previous chapters. This rule allows you to specify different style for different media. So, you can define different rules for screen and a printer.
\begin{verbatim}
<style tyle="text/css">
   <!--
   @media screen
   {
      p.bodyText {font-family:verdana, arial, sans-serif;}
   }

   @media print
   {
      p.bodyText {font-family:georgia, times, serif;}
   }
   @media screen, print
   {
      p.bodyText {font-size:10pt}
   }
   -->
</style>
\end{verbatim}
If you are defining your style sheet in a separate file, then you can also use the media attribute when linking to an external style sheet.
\begin{verbatim}
<link rel="stylesheet" type="text/css" media="print" href="mystyle.css">
\end{verbatim}
\section{Layouts}
Hope you are very comfortable with HTML tables and you are efficient in designing page layouts using HTML Tables. But you know CSS also provides plenty of controls for positioning elements in a document. Since CSS is the wave of the future, why not learn and use CSS instead of tables for page layout purposes?

The following list collects a few pros and cons of both the technologies
\begin{enumerate}
\item Most browsers support tables, while CSS support is being slowly adopted.
\item Tables are more forgiving when the browser window size changes - morphing their content and wrapping to accommodate the changes accordingly. CSS positioning tends to be exact and fairly inflexible.
\item Tables are much easier to learn and manipulate than CSS rules.
\end{enumerate}
But each of these arguments can be reversed.
\begin{enumerate}
\item CSS is pivotal to the future of Web documents and will be supported by most browsers.
\item CSS is more exact than tables, allowing your document to be viewed as you intended, regardless of the browser window.
\item Keeping track of nested tables can be a real pain. CSS rules tend to be well organized, easily read, and easily changed.
\end{enumerate}

Finally, we would suggest you to use whichever technology makes sense to you and use what you know or what presents your documents in the best way.

CSS also provides \textit{table-layout} property to make your tables load much faster. Following is an example.
\begin{verbatim}
<table style="table-layout:fixed;width:600px;">
   <tr height="30">
      <td width="150">CSS table layout cell 1</td>
      <td width="200">CSS table layout cell 2</td>
      <td width="250">CSS table layout cell 3</td>
   </tr>
</table>
\end{verbatim}
You will notice the benefits more on large tables. With traditional HTML, the browser had to calculate every cell before finally rendering the table. When you set the table-layout algorithm to fixed, however, it only needs to look at the first row before rendering the whole table. It means your table will need to have fixed column widths and row heights.
\subsection{Sample Column Layout}
Here are the steps to create a simple Column Layout using CSS.

Set the margin and padding of the complete document as follows.
\begin{verbatim}
<style style="text/css">
   <!--
   body {
      margin:9px 9px 0 9px;
      padding:0;
      background:#FFF;
   }
   -->
</style>
\end{verbatim}
Now, we will define a column with yellow color and later, we will attach this rule to a $<$div$>$.
\begin{verbatim}
<style style="text/css">
   <!--
   #level0 {
      background:#FC0;
   }
   -->
</style>
\end{verbatim}
Upto this point, we will have a document with yellow body, so let us now define another division inside level0.
\begin{verbatim}
<style style="text/css">
   <!--
   #level1 {
      margin-left:143px;
      padding-left:9px;
      background:#FFF;
   }
   -->
</style>
\end{verbatim}
Now, we will nest one more division inside level1, and we will change just background color.
\begin{verbatim}
<style style="text/css">
   <!--
   #level2 {
      background:#FFF3AC;
   }
   -->
</style>
\end{verbatim}
Finally, we will use the same technique, nest a level3 division inside level2 to get the visual layout for the right column.
\begin{verbatim}
<style style="text/css">
   <!--
   #level3 {
      margin-right:143px;
      padding-right:9px;
      background:#FFF;
   }
   #main {
      background:#CCC;
   }
   -->
</style>
\end{verbatim}
Complete the source code as follows .
\begin{verbatim}
<style style="text/css">
   body {
      margin:9px 9px 0 9px;
      padding:0;
      background:#FFF;
   }
	
   #level0 {background:#FC0;}
	
   #level1 {
      margin-left:143px;
      padding-left:9px;
      background:#FFF;
   }
	
   #level2 {background:#FFF3AC;}
	
   #level3 {
      margin-right:143px;
      padding-right:9px;
      background:#FFF;
   }
	
   #main {background:#CCC;}
</style>
<body>
   <div id="level0">
      <div id="level1">
         <div id="level2">
            <div id="level3">
               <div id="main">
                  Final Content goes here...
               </div>
            </div>
         </div>
      </div>
   </div>
</body>
\end{verbatim}
Similarly, you can add a top navigation bar or an ad bar at the top of the page.
\section{Validations}
Validation is the process of checking something against a rule. When you are a beginner, it is very common that you will commit many mistakes in writing your CSS rules. How you will make sure whatever you have written is 100\% accurate and up to the W3 quality standards?

If you use CSS, your code needs to be correct. Improper code may cause unexpected results in how your page looks or functions.

But if you want to validate your CSS style sheet embedded in an (X)HTML document, you should first check that the (X)HTML you use is valid.

You can use the following tools to check the validity of your CSS.
\begin{enumerate}
\item W3C CSS Validator (World Wide Web Consortium), This validator checks your css by either file upload, direct input, or using URI - one page at a time. This validator helps you to locate all the errors in your CSS.
\item The WDG CSS check validator, lets you validate your css by direct input, file upload, and using URI. Errors will be listed by line and column numbers if you have any. Errors usually come with links to explain the reason of error. 
\end{enumerate}
A CSS validator checks your Cascading Style Sheets to make sure that they comply with the CSS standards set by the W3 Consortium. There are a few validators which will also tell you which CSS features are supported by which browsers (since not all browsers are equal in their CSS implementation).
\subsection{Why Validate Your HTML Code?}
There are a number of reasons why you should validate your code. But major ones are
\begin{enumerate}
\item It Helps Cross-Browser, Cross-Platform, and Future Compatibility.
\item A good quality website increases search engine visibility.
\item Professionalism: As a web developer, your code should not raise errors while seen by the visitors.
\end{enumerate}
\chapter{CSS3 Tutorial}
Cascading Style Sheets (CSS) is a style sheet language used for describing the look and formatting of a document written in a markup language.CSS3 is a latest standard of css earlier versions(CSS2).The main difference between css2 and css3 is follows.
\begin{enumerate}
\item Media Queries
\item Namespaces
\item Selectors Level 3
\item Color
\end{enumerate}
\subsection*{CSS3 Modules}
CSS3 is collaboration of CSS2 specifications and new specifications, we can called this collaboration is module. Some of the modules are shown below.
\begin{enumerate}
\item Selectors
\item Box Model
\item Backgrounds
\item Image Values and Replaced Content
\item Text Effects
\item 2D Transformations
\item 3D Transformations
\item Animations
\item Multiple Column Layout
\item User Interface
\end{enumerate}
\section{Rounded Corners}
CSS3 Rounded corners are used to add special colored corner to body or text by using the border-radius property. A simple syntax of rounded corners is as follows.
\begin{verbatim}
#rcorners7 {
   border-radius: 60px/15px;
   background: #FF0000;
   padding: 20px; 
   width: 200px;
   height: 150px; 
}
\end{verbatim}
The following table shows the possible values for Rounded corners as follows.
\begin{center}
\begin{longtable}{|l|p{7cm}|}
\hline
\textbf{Values} & \textbf{Description}\\
\hline
border-radius &	Use this element for setting four boarder radius property\\
\hline
border-top-left-radius & Use this element for setting the boarder of top left corner\\
\hline
border-top-right-radius & Use this element for setting the boarder of top right corner\\
\hline
border-bottom-right-radius & Use this element for setting the boarder of bottom right corner\\
\hline
border-bottom-left-radius & Use this element for setting the boarder of bottom left corner\\
\hline
\caption{The possible values for Rounded corners.}
\end{longtable}
\end{center}
\textbf{Example 3.1.} This property can have three values. The following example uses both the values.
\begin{verbatim}
<html>
   <head>
   
      <style>
         #rcorners1 {
            border-radius: 25px;
            background: #8AC007;
            padding: 20px;
            width: 200px;
            height: 150px;
         }
         #rcorners2 {
            border-radius: 25px;
            border: 2px solid #8AC007;
            padding: 20px; 
            width: 200px;
            height: 150px;
         }
         #rcorners3 {
            border-radius: 25px;
            background: url(paper.gif);
            background-position: left top;
            background-repeat: repeat;
            padding: 20px; 
            width: 200px;
            height: 150px;
         }
      </style>
      
   </head>
   <body>
      <p id="rcorners1">Rounded corners!</p>
      <p id="rcorners2">Rounded corners!</p>
      <p id="rcorners3">Rounded corners!</p>
   </body>
</html>
\end{verbatim}
\subsection{Each Corner Property}
We can specify the each corner property as shown below example.
\begin{verbatim}
<html>
   <head>
   
      <style>
         #rcorners1 {
            border-radius: 15px 50px 30px 5px;
            background: #a44170;
            padding: 20px; 
            width: 100px;
            height: 100px; 
         }
         #rcorners2 {
            border-radius: 15px 50px 30px;
            background: #a44170;
            padding: 20px;
            width: 100px;
            height: 100px; 
         }
         #rcorners3 {
            border-radius: 15px 50px;
            background: #a44170;
            padding: 20px; 
            width: 100px;
            height: 100px; 
         }
      </style>
      
   </head>
   <body>
      <p id="rcorners1"></p>
      <p id="rcorners2"></p>
      <p id="rcorners3"></p>
   </body>
<body>
\end{verbatim}
\section{Border Image}
CSS Border image property is used to add image boarder to some elements.you don't need to use any HTML code to call boarder image.A sample syntax of boarder image is as follows.
\begin{verbatim}
#borderimg {
   border: 10px solid transparent;
   padding: 15px;
}
\end{verbatim}
The most commonly used values are shown below.
\begin{center}
\begin{longtable}{|l|p{8cm}|}
\hline
\textbf{Values} & \textbf{Description}\\
\hline
border-image-source & Used to set the image path\\
\hline
border-image-slice & Used to slice the boarder image\\
\hline
border-image-width & Used to set the boarder image width\\
\hline
border-image-repeat & Used to set the boarder image as rounded, repeated and stretched\\
\hline
\caption{The most commonly used values in CSS Border image property.}
\end{longtable}
\end{center}
\textbf{Example 3.2.} Following is the example which demonstrates to set image as a border for elements.
\begin{verbatim}
<html>
   <head>
   
      <style>
         #borderimg1 { 
            border: 10px solid transparent;
            padding: 15px;
            border-image-source: url(/css/images/border.png);
            border-image-repeat: round;
            border-image-slice: 30;
            border-image-width: 10px;
         }
         #borderimg2 { 
            border: 10px solid transparent;
            padding: 15px;
            border-image-source: url(/css/images/border.png);
            border-image-repeat: round;
            border-image-slice: 30;
            border-image-width: 20px;
         }
         #borderimg3 { 
            border: 10px solid transparent;
            padding: 15px;
            border-image-source: url(/css/images/border.png);
            border-image-repeat: round;
            border-image-slice: 30;
            border-image-width: 30px;
         }
      </style>
      
   </head>
   <body>
      <p id="borderimg1">This is image boarder example.</p>
      <p id="borderimg2">This is image boarder example.</p>
      <p id="borderimg3">This is image boarder example.</p>
   </body>
</html> 
\end{verbatim}
\section{Multi Background}
CSS Multi background property is used to add one or more images at a time without HTML code, We can add images as per our requirement.A sample syntax of multi background images is as follows.
\begin{verbatim}
#multibackground {
   background-image: url(/css/images/logo.png), 
   url(/css/images/border.png);
   background-position: left top, left top;
   background-repeat: no-repeat, repeat;
   padding: 75px;
}
\end{verbatim}
the most commonly used values are shown below.
\begin{center}
\begin{longtable}{|l|p{8cm}|}
\hline
\textbf{Values} & \textbf{Description}\\
\hline
background & Used to setting all the background image properties in one section\\
\hline
background-clip & Used to declare the painting area of the background\\
\hline
background-image & Used to specify the background image\\
\hline
background-origin & Used to specify position of the background images\\
\hline
background-size & Used to specify size of the background images \\
\hline
\caption{The most commonly used values in CSS Multi background property.}
\end{longtable}
\end{center}
\textbf{Example 3.3.} Following is the example which demonstrate the multi background images.
\begin{verbatim}
<html>
   <head>
   
      <style>
         #multibackground {
            background-image: url(/css/images/logo.png), 
            url(/css/images/border.png);
            background-position: left top, left top;
            background-repeat: no-repeat, repeat;
            padding: 75px;
         }
      </style>
      
   </head>
   <body>
   
      <div id="multibackground">
         <h1>www.tutorialspoint.com</h1>
         <p>Tutorials Point originated from the idea that there 
         exists a class of  readers who respond better to online 
         content and prefer to learn new skills at 
         their own pace from the comforts of their drawing 
         rooms. The journey commenced with a single tutorial 
         on HTML in  2006 and elated by the response it generated, 
         we worked our way to adding fresh tutorials to our 
         repository which now proudly flaunts a wealth of tutorials 
         and allied articles on 
         topics ranging from programming languages to web designing
         to academics and much more..</p>
      </div>
      
   </body>
</html> 
\end{verbatim}
\subsection{Size of Multi Background}
Multi background property is accepted to add different sizes for different images.A sample syntax is as shown below.
\begin{verbatim}
#multibackground {
   background: url(/css/imalges/logo.png) left top no-repeat, 
   url(/css/images/boarder.png) right bottom no-repeat, 
   url(/css/images/css.gif) left top repeat;
   background-size: 50px, 130px, auto;
}
\end{verbatim}
As shown above an example, each image is having specific sizes as 50px, 130px and auto size.
\section{Colors}
CSS3 has Supported additional color properties as follows.
\begin{enumerate}
\item RGBA colors
\item HSL colors
\item HSLA colors
\item Opacity
\end{enumerate}

\textit{RGBA} stands for \textit{Red Green Blue Alpha}.It is an extension of CSS2,Alpha specifies the opacity of a color and parameter number is a numerical between 0.0 to 1.0. A Sample syntax of RGBA as shown below.
\begin{verbatim}
#d1 {background-color: rgba(255, 0, 0, 0.5);} 
#d2 {background-color: rgba(0, 255, 0, 0.5);}  
#d3 {background-color: rgba(0, 0, 255, 0.5);}
\end{verbatim}

\textit{HSL} stands for \textit{hue, saturation, lightness}.Here Huge is a degree on the color wheel, saturation and lightness are percentage values between 0 to 100\%. A Sample syntax of HSL as shown below.
\begin{verbatim}
#g1 {background-color: hsl(120, 100%, 50%);}  
#g2 {background-color: hsl(120, 100%, 75%);}  
#g3 {background-color: hsl(120, 100%, 25%);}
\end{verbatim}

\textit{HSLA} stands for \textit{hue, saturation, lightness} and \textit{alpha}. Alpha value specifies the opacity as shown RGBA. A Sample syntax of HSLA as shown below .
\begin{verbatim}
#g1 {background-color: hsla(120, 100%, 50%, 0.3);}  
#g2 {background-color: hsla(120, 100%, 75%, 0.3);}  
#g3 {background-color: hsla(120, 100%, 25%, 0.3);}  
\end{verbatim}

\textit{opacity} is a thinner paints need black added to increase opacity. A sample syntax of opacity is as shown below.
\begin{verbatim}
#g1 {background-color:rgb(255,0,0);opacity:0.6;}  
#g2 {background-color:rgb(0,255,0);opacity:0.6;}  
#g3 {background-color:rgb(0,0,255);opacity:0.6;} 
\end{verbatim}
The following example shows rgba color property.
\begin{verbatim}
<html>
   <head>
      <style>
         #p1 {background-color:rgba(255,0,0,0.3);}
         #p2 {background-color:rgba(0,255,0,0.3);}
         #p3 {background-color:rgba(0,0,255,0.3);}
      </style>
   </head>
   <body>
      <p>RGBA colors:</p>
      <p id="p1">Red</p>
      <p id="p2">Green</p>
      <p id="p3">Blue</p>
   </body>
</html>
\end{verbatim}
The following example shows HSL color property.
\begin{verbatim}
<html>
   <head>
      <style>
         #g1 {background-color:hsl(120, 100%, 50%);}
         #g2 {background-color:hsl(120,100%,75%);}
         #g3 {background-color:hsl(120,100%,25%);}
      </style>
   </head>
   <body>
      <p>HSL colors:</p>
      <p id="g1">Green</p>
      <p id="g2">Normal Green</p>
      <p id="g3">Dark Green</p>
   </body>
</html>
\end{verbatim}
The following example shows HSLA color property.
\begin{verbatim}
<html>
   <head>
      <style>
         #d1 {background-color:hsla(120,100%,50%,0.3);}
         #d2 {background-color:hsla(120,100%,75%,0.3);}
         #d3 {background-color:hsla(120,100%,25%,0.3);}
      </style>
   </head>
   <body>
      <p>HSLA colors:</p>
      <p id="d1">Less opacity green</p>
      <p id="d2">Green</p>
      <p id="d3">Green</p>
   </body>
</html>
\end{verbatim}
The following example shows Opacity property.
\begin{verbatim}
<html>
   <head>
      <style>
         #m1 {background-color:rgb(255,0,0);opacity:0.6;} 
         #m2 {background-color:rgb(0,255,0);opacity:0.6;} 
         #m3 {background-color:rgb(0,0,255);opacity:0.6;}
      </style>
   </head>
   <body>
      <p>HSLA colors:</p>
      <p id="m1">Red</p>
      <p id="m2">Green</p>
      <p id="m3">Blue</p>
   </body>
</html>
\end{verbatim}
\section{Gradients}
\subsection{What is Gradients?}
Gradients displays the combination of two or more colors.

 Adding a gradient is easy. All gradients are read from a gradients.json file which is available in this project's repo. Simply add your gradient details to it and submit a pull request.\\
\\
\textbf{Types of gradients.}
\begin{enumerate}
\item Linear Gradients(down/up/left/right/diagonally)
\item Radial Gradients
\end{enumerate}
\textbf{Linear gradients.} Linear gradients are used to arrange two or more colors in linear formats like top to bottom.
\subsection{Top to Bottom}
\begin{verbatim}
<html>
   <head>
      <style>
         #grad1 {
            height: 100px;
            background: -webkit-linear-gradient(pink,green);
            background: -o-linear-gradient(pink,green);
            background: -moz-linear-gradient(pink,green); 
            background: linear-gradient(pink,green); 
         }
      </style>
   </head>
   <body>
      <div id="grad1"></div>
   </body>
</html> 
\end{verbatim}
\subsection{Left to Right}
\begin{verbatim}
<html>
   <head>
      <style>
         #grad1 {
            height: 100px;
            background: -webkit-linear-gradient(left, red , blue);
            background: -o-linear-gradient(right, red, blue); 
            background: -moz-linear-gradient(right, red, blue);
            background: linear-gradient(to right, red , blue);
         }
      </style>
   </head>
   <body>
      <div id="grad1"></div>
   </body>
</html> 
\end{verbatim}
\subsection{Diagonal}
Diagonal starts at top left and right button.
\begin{verbatim}
<html>
   <head>
      <style>
         #grad1 {
            height: 100px;
            background: -webkit-linear-gradient(left top, red , blue); 
            background: -o-linear-gradient(bottom right, red, blue); 
            background: -moz-linear-gradient(bottom right, red, blue);
            background: linear-gradient(to bottom right, red , blue); 
         }
      </style>
   </head>
   <body>
      <div id="grad1"></div>
   </body>
</html> 
\end{verbatim}
\subsection{Multi Color}
\begin{verbatim}
<html>
   <head>
      <style>
         #grad2 {
            height: 100px;
            background: -webkit-linear-gradient(red, orange,
             yellow, red, blue, green,pink); 
            background: -o-linear-gradient(red, orange, yellow, 
            red, blue, green,pink); 
            background: -moz-linear-gradient(red, orange, yellow, 
            red, blue, green,pink); 
            background: linear-gradient(red, orange, yellow, 
            red, blue, green,pink); 
         }
      </style>
   </head>
   <body>
      <div id="grad2"></div>
   </body>
</html> 
\end{verbatim}
\subsection{CSS3 Radial Gradients}
Radial gradients appears at center.
\begin{verbatim}
<html>
   <head>
      <style>
         #grad1 {
            height: 100px;
            width: 550px;
            background: -webkit-radial-gradient(red 5%, green 15%, pink 60%); 
            background: -o-radial-gradient(red 5%, green 15%, pink 60%); 
            background: -moz-radial-gradient(red 5%, green 15%, pink 60%); 
            background: radial-gradient(red 5%, green 15%, pink 60%); 
         }
      </style>
   </head>
   <body>
      <div id="grad1"></div>
   </body>
</html> 
\end{verbatim}
\subsection{CSS3 Repeat Radial Gradients}
\begin{verbatim}
<html>
   <head>
      <style>
         #grad1 {
            height: 100px;
            width: 550px;
            background: -webkit-repeating-radial-gradient(blue, yellow 10%, green 15%); 
            background: -o-repeating-radial-gradient(blue, yellow 10%, green 15%);
            background: -moz-repeating-radial-gradient(blue, yellow 10%, green 15%);
            background: repeating-radial-gradient(blue, yellow 10%, green 15%); 
         }
      </style>
   </head>
   <body>
      <div id="grad1"></div>
   </body>
</html> 
\end{verbatim}
\section{Shadow}
CSS3 supported to add shadow to text or elements.Shadow property has divided as follows.
\begin{enumerate}
\item Text shadow.
\item Box shadow.
\end{enumerate}
\subsection{Text Shadow}
CSS3 supported to add shadow effects to text. Following is the example to add shadow effects to text.
\begin{verbatim}
<html>
   <head>
   
      <style>
         h1 {
            text-shadow: 2px 2px;
         }
         h2 {
            text-shadow: 2px 2px red;
         }
         h3 {
            text-shadow: 2px 2px 5px red;
         }
         h4 {
            color: white;
            text-shadow: 2px 2px 4px #000000;
         }
         h5 {
            text-shadow: 0 0 3px #FF0000;
         }
         h6 {
            text-shadow: 0 0 3px #FF0000, 0 0 5px #0000FF;
         }
         p {
            color: white;
            text-shadow: 1px 1px 2px black, 0 0 25px blue, 0 0 5px darkblue;
         }
      </style>
      
   </head>
   <body>
   
      <h1>Tutorialspoint.com</h1>
      <h2>Tutorialspoint.com</h2>
      <h3>Tutorialspoint.com</h3>
      <h4>Tutorialspoint.com</h4>
      <h5>Tutorialspoint.com</h5>
      <h6>Tutorialspoint.com</h6>
      <p>Tutorialspoint.com</p>
      
   </body>
</html>
\end{verbatim}
\subsection{Box Shadow}
Used to add shadow effects to elements. Following is the example to add shadow effects to element.
\begin{verbatim}
<html>
   <head>
      <style>
         div {
            width: 300px;
            height: 100px;
            padding: 15px;
            background-color: red;
            box-shadow: 10px 10px;
         }
      </style>
   </head>
   <body>
      <div>This is a div element with a box-shadow</div>
   </body>
</html>
\end{verbatim}
\subsection{Text}
CSS3 contained several extra features, which is added later on
\begin{enumerate}
\item text-overflow
\item word-wrap
\item word-break
\end{enumerate}
There are following most commonly used property in CSS3.
\begin{center}
\begin{longtable}{|l|p{7cm}|}
\hline
\textbf{Values} & \textbf{Description}\\
\hline
text-align-last & Used to align the last line of the text\\
\hline
text-emphasis & Used to emphasis text and color\\
\hline
text-overflow & used to determines how overflowed content that is not displayed is signaled to users\\
\hline
word-break & Used to break the line based on word\\
\hline
word-wrap & Used to break the line and wrap onto next line\\
\hline
\caption{The most commonly used property in CSS3.}
\end{longtable}
\end{center}
\subsection{Text-overflow}
The text-overflow property determines how overflowed content that is not displayed is signaled to users. The sample example of text overflow is shown as follows.
\begin{verbatim}
<html>
   <head>
   
      <style>
         p.text1 {
            white-space: nowrap; 
            width: 200px; 
            border: 1px solid #000000;
            overflow: hidden;
            text-overflow: clip;
         }
         p.text2 {
            white-space: nowrap; 
            width: 200px; 
            border: 1px solid #000000;
            overflow: hidden;
            text-overflow: ellipsis;
         }
      </style>
      
   </head>
   <body>
   
      <b>Original Text:</b>
   
      <p>Tutorials Point originated from the idea that there 
      exists a class of readers who respond better to online 
      content and prefer to learn new skills at their own pace 
      from the comforts of their drawing rooms.</p>
      
      <b>Text overflow:clip:</b>
   
      <p class="text1">Tutorials Point originated from the idea
      that there exists a class of readers who respond better 
      to online content and prefer to learn  new skills at their 
      own pace from the comforts of their drawing rooms.</p>
      
      <b>Text overflow:ellipsis</b>
   
      <p class="text2">Tutorials Point originated from the idea 
      that there exists a class of readers who respond better to 
      online content and prefer to learn new skills at their own 
      pace from the comforts of their drawing rooms.</p>
      
   </body>
</html>
\end{verbatim}
\subsection{CSS3 Word Breaking}
Used to break the line, following code shows the sample code of word breaking
\begin{verbatim}
<html>
   <head>
   
      <style>
         p.text1 {
            width: 140px; 
            border: 1px solid #000000;
            word-break: keep-all;
         }
         p.text2 {
            width: 140px; 
            border: 1px solid #000000;
            word-break: break-all;
         }
      </style>
      
   </head>
   <body>
   
      <b>line break at hyphens:</b>
      <p class="text1">Tutorials Point originated from the idea 
      that there exists a class of readers who respond better to 
      online content and prefer to learn new skills at their 
      own pace from the comforts of their drawing rooms.</p>
      
      <b>line break at any character</b>
   
      <p class="text2">Tutorials Point originated from the idea 
      that there exists a class of readers who respond better 
      to online content and prefer to learn new skills at their
      own pace from the comforts of their drawing rooms.</p>
      
   </body>
</html>
\end{verbatim}
\subsection{CSS Word Wrapping}
Word wrapping is used to break the line and wrap onto next line. The following code will have sample syntax.
\begin{verbatim}
p {
   word-wrap: break-word;
}
\end{verbatim}
\section{Web Fonts}
Web fonts are used to allows the fonts in CSS, which are not installed on local system.
\subsection{Different Web Fonts Formats}
\begin{center}
\begin{longtable}{|p{3cm}|p{8cm}|}
\hline
\textbf{Fonts} & \textbf{Description}\\
\hline
TrueType Fonts (TTF) & TrueType is an outline font standard developed by Apple and Microsoft in the late 1980s, It became most common fonts for both windows and MAC operating systems.\\
\hline
OpenType Fonts (OTF) & OpenType is a format for scalable computer fonts and developed by Microsoft.\\
\hline
The Web Open Font Format (WOFF) & WOFF is used for develop web page and developed in the year of 2009. Now it is using by W3C recommendation..\\
\hline
SVG Fonts/Shapes & SVG allow SVG fonts within SVG documentation. We can also apply CSS to SVG with font face property.\\
\hline
Embedded OpenType Fonts (EOT) & EOT is used to develop the web pages and it has embedded in webpages so no need to allow 3rd party fonts.\\
\hline
\caption{Different web fonts formats.}
\end{longtable}
\end{center}
The following code shows the sample code of font face.
\begin{verbatim}
<html>
   <head>
      <style>
         @font-face {
            font-family: myFirstFont;
            src: url(/css/font/SansationLight.woff);
         }
         div {
            font-family: myFirstFont;
         }
      </Style>
   </head>
   <body>
      <div>This is the example of font face with CSS3.</div>
      
      <p><b>Original Text :</b>This is the example of font face 
      with CSS3.</p>
   </body>
</html>
\end{verbatim}
\subsection{Fonts Description}
The following list contained all the fonts description which are placed in the @font-face rule.
\begin{center}
\begin{longtable}{|l|l|}
\hline
\textbf{Values} & \textbf{Description}\\
\hline
font-family & Used to defines the name of font.\\
\hline
src & Used to defines the URL.\\
\hline
font-stretch & Used to find, how font should be stretched.\\
\hline
font-style & Used to defines the fonts style.\\
\hline
font-weight & Used to defines the font weight(boldness).\\
\hline
\caption{All the fonts description which are placed in the @font-face rule.}
\end{longtable}
\end{center}
\subsection{2D Transforms}
2D transforms are used to re-change the element structure as translate, rotate, scale, and skew.

The following table has contained common values which are used in 2D transforms.
\begin{center}
\begin{longtable}{|l|p{8cm}|}
\hline
\textbf{Values} & \textbf{Description}\\
\hline
matrix(n,n,n,n,n,n) & Used to defines matrix transforms with six values.\\
\hline
translate(x,y) & Used to transforms the element along with x-axis and y-axis.\\
\hline
translateX(n) & Used to transforms the element along with x-axis.\\
\hline
translateY(n) & Used to transforms the element along with y-axis.\\
\hline
scale(x,y) & Used to change the width and height of element.\\
\hline
scaleX(n) &	Used to change the width of element.\\
\hline
scaleY(n) & Used to change the height of element.\\
\hline
rotate(angle) & Used to rotate the element based on an angle.\\
\hline
skewX(angle) & Used to defines skew transforms along with x axis.\\
\hline
skewY(angle) & Used to defines skew transforms along with y axis.\\
\hline
\caption{Common values which are used in 2D transforms.}
\end{longtable}
\end{center}
The following examples are shown the sample of all above properties.
\subsection{Rotate 20 Degrees}
Box rotation with 20 degrees angle as shown below.
\begin{verbatim}
<html>
   <head>
   
      <style>
         div {
            width: 300px;
            height: 100px;
            background-color: pink;
            border: 1px solid black;
         }
         div#myDiv {
            /* IE 9 */
            -ms-transform: rotate(20deg);
            
            /* Safari */
            -webkit-transform: rotate(20deg);
            
            /* Standard syntax */
            transform: rotate(20deg);
         }
      </style>
      
   </head>
   <body>
      <div>
      Tutorials point.com.
      </div>
      
      <div id="myDiv">
      Tutorials point.com
      </div>
   </body>
</html>
\end{verbatim}
\subsection{Rotate -20 degrees}
Box rotation with -20 degrees angle as shown below.
\begin{verbatim}
<html>
   <head>
   
      <style>
         div {
            width: 300px;
            height: 100px;
            background-color: pink;
            border: 1px solid black;
         }
         div#myDiv {
            /* IE 9 */
            -ms-transform: rotate(-20deg); 
         
            /* Safari */
            -webkit-transform: rotate(-20deg);
         
            /* Standard syntax */	
            transform: rotate(-20deg);
         }
      </style>
      
   </head>
   <body>
      <div>
      Tutorials point.com.
      </div>
      
      <div id="myDiv">
      Tutorials point.com
      </div>
   </body>
</html>
\end{verbatim}
\subsection{Skew X Axis}
Box rotation with skew x-axis as shown below.
\begin{verbatim}
<html>
   <head>
   
      <style>
         div {
            width: 300px;
            height: 100px;
            background-color: pink;
            border: 1px solid black;
         }
         div#skewDiv {
            /* IE 9 */
            -ms-transform: skewX(20deg); 
            
            /* Safari */
            -webkit-transform: skewX(20deg);
            
            /* Standard syntax */	
            transform: skewX(20deg);
         }
      </style>
      
   </head>
   <body>
      <div>
      Tutorials point.com.
      </div>
      
      <div id="skewDiv">
      Tutorials point.com
      </div>
   </body>
</html>
\end{verbatim}
\subsection{Skew Y Axis}
Box rotation with skew y-axis as shown below.
\begin{verbatim}
<html>
   <head>
   
      <style>
         div {
            width: 300px;
            height: 100px;
            background-color: pink;
            border: 1px solid black;
         }
         div#skewDiv {
            /* IE 9 */
            -ms-transform: skewY(20deg); 
            
            /* Safari */
            -webkit-transform: skewY(20deg); 
            
            /* Standard syntax */	
            transform: skewY(20deg);
         }
      </style>
      
   </head>
   <body>
      <div>
      Tutorials point.com.
      </div>
      
      <div id="skewDiv">
      Tutorials point.com
      </div>
   </body>
</html>
\end{verbatim}
\subsection{Matrix Transform}
Box rotation with Matrix transforms as shown below.
\begin{verbatim}
<html>
   <head>
   
      <style>
         div {
            width: 300px;
            height: 100px;
            background-color: pink;
            border: 1px solid black;
         }
         div#myDiv1 {
            /* IE 9 */
            -ms-transform: matrix(1, -0.3, 0, 1, 0, 0);
            
            /* Safari */
            -webkit-transform: matrix(1, -0.3, 0, 1, 0, 0); 
            
            /* Standard syntax */
            transform: matrix(1, -0.3, 0, 1, 0, 0); 
         }
      </style>
      
   </head>
   <body>
      <div>
      Tutorials point.com.
      </div>
      
      <div id="myDiv1">
      Tutorials point.com
      </div>
   </body>
</html>
\end{verbatim}
Matrix transforms with another direction.
\begin{verbatim}
<html>
   <head>
   
      <style>
         div {
            width: 300px;
            height: 100px;
            background-color: pink;
            border: 1px solid black;
         }
         div#myDiv2 {
            /* IE 9 */
            -ms-transform: matrix(1, 0, 0.5, 1, 150, 0);
            
            /* Safari */	
            -webkit-transform: matrix(1, 0, 0.5, 1, 150, 0);
            
            /* Standard syntax */
            transform: matrix(1, 0, 0.5, 1, 150, 0); 
         }
      </style>
      
   </head>
   <body>
      <div>
      Tutorials point.com.
      </div>
      
      <div id="myDiv2">
      Tutorials point.com
      </div>
   </body>
</html>
\end{verbatim}
\section{3D Transforms}
Using with 3d transforms, we can move element to x-axis, y-axis and z-axis, Below example clearly specifies how the element will rotate.
\subsection{Methods of 3D Transforms}
Below methods are used to call 3D transforms.
\begin{center}
\begin{longtable}{|l|p{5cm}|}
\hline
\textbf{Values} & \textbf{Description}\\
\hline
matrix3d(n,n,n,n,n,n,n,n,n,n,n,n,n,n,n,n) &	Used to transforms the element by using 16 values of matrix.\\
\hline
translate3d(x,y,z) & Used to transforms the element by using x-axis,y-axis and z-axis.\\
\hline
translateX(x) &	Used to transforms the element by using x-axis.\\
\hline
translateY(y) &	Used to transforms the element by using y-axis.\\
\hline
translateZ(z) &	Used to transforms the element by using y-axis.\\
\hline
scaleX(x) &	Used to scale transforms the element by using x-axis.\\
\hline
scaleY(y) &	Used to scale transforms the element by using y-axis.\\
\hline
scaleY(y) &	Used to transforms the element by using z-axis.\\
\hline
rotateX(angle) & Used to rotate transforms the element by using x-axis.\\
\hline
rotateY(angle) & Used to rotate transforms the element by using y-axis.\\
\hline
rotateZ(angle) & Used to rotate transforms the element by using z-axis.\\
\hline
\caption{3D transforms.}
\end{longtable}
\end{center}
\subsection{X-Axis 3D Transforms}
The following an example shows the x-axis 3D transforms.
\begin{verbatim}
<html>
   <head>
   
      <style>
         div {
            width: 200px;
            height: 100px;
            background-color: pink;
            border: 1px solid black;
         }
         div#myDiv {
            -webkit-transform: rotateX(150deg); 
            
            /* Safari */
            transform: rotateX(150deg); 
            
            /* Standard syntax */
         }
      </style>
      
   </head>
   <body>
   
      <div>
      tutorials point.com
      </div>
      
      <p>Rotate X-axis</p>
      
      <div id="myDiv">
      tutorials point.com.
      </div>
      
   </body>
</html>
\end{verbatim}
\subsection{Y-Axis 3D Transforms}
The following an example shows the y-axis 3D transforms.
\begin{verbatim}
<html>
   <head>
   
      <style>
         div {
            width: 200px;
            height: 100px;
            background-color: pink;
            border: 1px solid black;
         }
         div#yDiv {
            -webkit-transform: rotateY(150deg); 
            
            /* Safari */
            transform: rotateY(150deg); 
            
            /* Standard syntax */
         }
      </style>
      
   </head>
   <body>
   
      <div>
      tutorials point.com
      </div>
      
      <p>Rotate Y axis</p>
      
      <div id="yDiv">
      tutorials point.com.
      </div>
      
   </body>
</html>
\end{verbatim}
\subsection{Z-Axis 3D Transforms}
The following an example shows the Z-axis 3D transforms.
\begin{verbatim}
<html>
   <head>
   
      <style>
         div {
            width: 200px;
            height: 100px;
            background-color: pink;
            border: 1px solid black;
         }
         div#zDiv {
            -webkit-transform: rotateZ(90deg); 
            
            /* Safari */
            transform: rotateZ(90deg); 
            
            /* Standard syntax */
         }
      </style>
      
   </head>
   <body>
      <p>rotate Z axis</p>
      
      <div id="zDiv">
      tutorials point.com.
      </div>
   </body>
</html> 
\end{verbatim}
\section{Animation}
Animation is process of making shape changes and creating motions with elements.
\subsection{@keyframes}
Keyframes will control the intermediate animation steps in CSS3.\\
\\
\textbf{Example 3.4.} Example of key frames with left animation.
\begin{verbatim}
@keyframes animation {
   from {background-color: pink;}
   to {background-color: green;}
}
div {
   width: 100px;
   height: 100px;
   background-color: red;
   animation-name: animation;
   animation-duration: 5s;
}
\end{verbatim}
The above example shows height, width, color, name and duration of animation with keyframes syntax.
\subsection{Moving Left Animation}
\begin{verbatim}
<html>
   <head>
   
      <style type="text/css">
         h1 {
            -moz-animation-duration: 3s;
            -webkit-animation-duration: 3s;
            -moz-animation-name: slidein;
            -webkit-animation-name: slidein;
         }
         @-moz-keyframes slidein {
            from {
               margin-left:100%;
               width:300%
            }
            to {
               margin-left:0%;
               width:100%;
            }
         }
         @-webkit-keyframes slidein {
            from {
               margin-left:100%;
               width:300%
            }
            to {
               margin-left:0%;
               width:100%;
            }
         }
      </style>
      
   </head>
   <body>
      <h1>Tutorials Point</h1>
      <p>this is an example of moving left animation .</p>
	  <button onclick="myFunction()">Reload page</button>
      <script>
           function myFunction() {
           location.reload();
           }
      </script>
   </body>
</html>
\end{verbatim}
\subsection{Moving Left Animation with keyframes}
\begin{verbatim}
<html>
   <head>
   
      <style type="text/css">
         h1 {
            -moz-animation-duration: 3s;
            -webkit-animation-duration: 3s;
            -moz-animation-name: slidein;
            -webkit-animation-name: slidein;
         }
         @-moz-keyframes slidein {
            from {
               margin-left:100%;
               width:300%
            }
            75% {
               font-size:300%;
               margin-left:25%;
               width:150%;
            }
            to {
               margin-left:0%;
               width:100%;
            }
         }
         @-webkit-keyframes slidein {
            from {
               margin-left:100%;
               width:300%
            }
            75% {
               font-size:300%;
               margin-left:25%;
               width:150%;
            }
            to {
               margin-left:0%;
               width:100%;
            }
         }
      </style>
      
   </head>
   <body>
      <h1>Tutorials Point</h1>
      
      <p>This is an example of animation left with an extra 
      keyframe to make text changes.</p>
	  <button onclick="myFunction()">Reload page</button>
      <script>
           function myFunction() {
           location.reload();
           }
      </script>
   </body>
</html>
\end{verbatim}
\section{Multi Columns}
CSS3 supported multi columns to arrange the text as news paper structure.

Some of most common used multi columns properties as shown below.
\begin{center}
\begin{longtable}{|l|p{9cm}|}
\hline
\textbf{Values} & \textbf{Description}\\
\hline
column-count & Used to count the number of columns that element should be divided.\\
\hline
column-fill & Used to decide, how to fill the columns.\\
\hline
column-gap & Used to decide the gap between the columns.\\
\hline
column-rule & Used to specifies the number of rules.\\
\hline
rule-color & Used to specifies the column rule color.\\
\hline
rule-style & Used to specifies the style rule for column.\\
\hline
rule-width & Used to specifies the width.\\
\hline
column-span & Used to specifies the span between columns.\\
\hline
\caption{Some of most common used multi columns properties.}
\end{longtable}
\end{center}
\textbf{Example 3.5.} Below example shows the arrangement of text as new paper structure.
\begin{verbatim}
<html>
   <head>
   
      <style>
         .multi {
            /* Column count property */
            -webkit-column-count: 4;
            -moz-column-count: 4;
            column-count: 4;
            
            /* Column gap property */
            -webkit-column-gap: 40px; 
            -moz-column-gap: 40px; 
            column-gap: 40px;
            
            /* Column style property */
            -webkit-column-rule-style: solid; 
            -moz-column-rule-style: solid; 
            column-rule-style: solid;
         }
      </style>
      
   </head>
   <body>
   
      <div class="multi">
      Tutorials Point originated from the idea that there exists 
      a class of readers who respond better to online content 
      and prefer to learn new skills at their own pace from the 
      comforts of their drawing rooms. The journey commenced 
      with a single tutorial on HTML in 2006 and elated by the 
      response it generated, we worked our way to adding fresh 
      tutorials to our repository which now proudly flaunts 
      a wealth of tutorials and allied articles on topics ranging 
      from programming languages to web designing to academics 
      and much more.
      </div>
      
   </body>
</html>
\end{verbatim}
For suppose, if user wants to make text as new paper without line, we can do this by removing style syntax as shown below.
\begin{verbatim}
.multi {
   /* Column count property */
   -webkit-column-count: 4;
   -moz-column-count: 4;
   column-count: 4;
   
   /* Column gap property */
   -webkit-column-gap: 40px; 
   -moz-column-gap: 40px; 
   column-gap: 40px;
}
\end{verbatim}
\section{User Interface}
The user interface property allows you to change any element into one of several standard user interface elements.

Some of the common properties which are using in css3 User interface.
\begin{center}
\begin{longtable}{|l|p{9cm}|}
\hline
\textbf{Values} & \textbf{Description}\\
\hline
appearance & Used to allow the user to make elements as user interface elements.\\
\hline
box-sizing & Allows to users to fix elements on area in clear way.\\
\hline
icon & Used to provide the icon on area.\\
\hline
resize & Used to resize elements which are on area.\\
\hline
outline-offset & Used to draw the behind the outline.\\
\hline
nav-down & Used to move down when you have pressed on down arrow button in keypad.\\
\hline
nav-left & Used to move left when you have pressed on left arrow button in keypad.\\
\hline
nav-right & Used to move right when you have pressed on right arrow button in keypad.\\
\hline
nav-up & Used to move up when you have pressed on up arrow button in keypad.\\
\hline
\caption{Some of the common properties which are using in CCS3 User interface.}
\end{longtable}
\end{center}
\subsection{Example of resize Property}
Resize property is having three common values as shown below.
\begin{enumerate}
\item horizontal
\item vertical
\item both
\end{enumerate}

Using of \textit{both} value in resize property in css3 user interface.
\begin{verbatim}
<html>
   <head>
      <style>
         div {
            border: 2px solid;
            padding: 20px; 
            width: 300px;
            resize: both;
            overflow: auto;
         }
      </style>
   </head>
   <body>
      <div>TutorialsPoint.com</div>
   </body>
</html>
\end{verbatim}
\subsection{CSS3 Outline Offset}
Out line means draw a line around the element at outside of boarder.
\begin{verbatim}
<html>
   <head>
      <style>
         div {
            margin: 20px;
            padding: 10px;
            width: 300px; 
            height: 100px;
            border: 5px solid pink;
            outline: 5px solid green;
            outline-offset: 15px;
         }
      </style>
   </head>
   <body>
      <div>TutorialsPoint</div>
   </body>
</html>
\end{verbatim}
\section{Box Sizing}
Box sizing property is using to change the height and width of element.

since css2, the box property has worked like as shown below.
\begin{enumerate}
\item width + padding + border = actual visible/rendered width of an element's box
\item height + padding + border = actual visible/rendered height of an element's box
\end{enumerate}
Means when you set the height and width, it appears little bit bigger then given size cause element boarder and padding added to the element height and width.
\subsection{CSS2 sizing Property}
\begin{verbatim}
<html>
   <head>
   
      <style>
         .div1 {
            width: 200px;
            height: 100px;
            border: 1px solid green;
         }
         .div2 {
            width: 200px;
            height: 100px;    
            padding: 50px;
            border: 1px solid pink;
         }
      </style>
      
   </head>
   <body>
      <div class="div1">TutorialsPoint.com</div></br>
      <div class="div2">TutorialsPoint.com</div>
   </body>
</html>
\end{verbatim}
Above image is having same width and height of two element but giving result is different, cause second one is included padding property.
\subsection{CSS3 box sizing Property}
\begin{verbatim}
<html>
   <head>
   
      <style>
         .div1 {
            width: 300px;
            height: 100px;
            border: 1px solid blue;
            box-sizing: border-box;
         }
         .div2 {
            width: 300px;
            height: 100px;
            padding: 50px;
            border: 1px solid red;
            box-sizing: border-box;
         }
      </style>
      
   </head>
   <body>
      <div class="div1">TutorialsPoint.com</div></br>
      <div class="div2">TutorialsPoint.com</div>
   </body>
</html>
\end{verbatim}
Above sample is having same height and width with \textit{box-sizing:border-box}.
\section{Responsive}
\subsection{CSS3 Responsive Web Design}
Responsive web design provides an optimal experience, easy reading and easy navigation with a minimum of resizing on different devices such as desktops, mobiles and tabs).
\subsection{Flexible Grid demo}
\begin{verbatim}
<html>
   <head>
   </head>
   
      <style>
         body {
            font: 600 14px/24px "Open Sans", "HelveticaNeue-Light", 
            "Helvetica Neue Light", "Helvetica Neue", Helvetica, Arial,
            "Lucida Grande", Sans-Serif;
         }
         h1 {
            color: #9799a7;
            font-size: 14px;
            font-weight: bold;
            margin-bottom: 6px;
         }
         .container:before, .container:after {
            content: "";
            display: table;
         }
         .container:after {
            clear: both;
         }
         .container {
            background: #eaeaed;
            margin-bottom: 24px;
            *zoom: 1;
         }
         .container-75 {
            width: 75%;
         }
         .container-50 {
            margin-bottom: 0;
            width: 50%;
         }
         .container, section, aside {
            border-radius: 6px;
         }
         section, aside {
            background: #2db34a;
            color: #fff;
            margin: 1.858736059%;
            padding: 20px 0;
            text-align: center;
         }
         section {
            float: left;
            width: 63.197026%;
         }
         aside {
            float: right;
            width: 29.3680297%;
         }
      </style>
   <body>
   
      <h1>100% Wide Container</h1>
      
      <div class="container">
         <section>Section</section>
         <aside>Aside</aside>
      </div>
      
      <h1>75% Wide Container</h1>
      
      <div class="container container-75">
         <section>Section</section>
         <aside>Aside</aside>
      </div>
      
      <h1>50% Wide Container</h1>
      
      <div class="container container-50">
         <section>Section</section>
         <aside>Aside</aside>
      </div>
      
   </body>
</html>
\end{verbatim}
\subsection{Media Queries}
Media queries is for different style rules for different size devices such as mobiles, desktops, etc.,
\begin{verbatim}
<html>
   <head>
      <style>
         body {
            background-color: lightpink;
         }
         @media screen and (max-width: 420px) {
            body {
               background-color: lightblue;
            }
         }
      </style>
   </head>
   <body>
      <p>If screen size is less than 420px, then it will show 
      lightblue color, or else it will show   light pink color</p>
   </body>
</html>
\end{verbatim}
\subsection{Bootstrap Responsive Web Design}
Bootstrap is most popular web design framework based on HTML,CSS and Java script and it helps you to design web pages in responsive way for all devices.
\begin{verbatim}
<html>
   <head>
      <meta charset="utf-8">
      <meta name="viewport" content="width=device-width, initial-scale=1">
      <link rel="stylesheet" href=
      "http://maxcdn.bootstrapcdn.com/bootstrap/3.2.0/css/bootstrap.min.css">
      <style>
         body{
            color:green;
         }
      </style>
   </head>
   <body>
   
      <div class="container">
      
         <div class="jumbotron">
            <h1>Tutorials point</h1> 
            <p>Tutorials Point originated from the idea that
             there exists a class of readers who respond better
              to online content and prefer to learn new skills
              at their own pace from the comforts of their 
              drawing rooms.</p> 
         </div>
      
         <div class="row">
            <div class="col-md-4">
               <h2>Android</h2>
               <p>Android is an open source and Linux-based 
               operating system for mobile devices such as 
               smartphones and tablet computers. Android was 
               developed by the Open Handset Alliance, led 
               by Google, and other companies.</p>
         </div>
         
         <div class="col-md-4">
            <h2>CSS</h2>
            <p>Cascading Style Sheets, fondly referred to as CSS, 
            is a simple design language intended to simplify
             the process of making web pages presentable.</p>
         </div>
      
         <div class="col-md-4">
            <h2>Java</h2>
            <p>Java is a high-level programming language originally
             developed by Sun Microsystems and released in 1995.
             Java runs on a variety of platforms, such as Windows,
             Mac OS, and the various versions of UNIX. This 
             tutorial gives a complete understanding of Java.</p>
         </div>
      </div>
      
   </body>
</html>
\end{verbatim}
\vspace{2cm}
\begin{center}
\textsc{The End}
\end{center}
\begin{thebibliography}{999}
\bibitem {1} \url{https://www.tutorialspoint.com/css/index.htm}
\end{thebibliography}
\end{document}