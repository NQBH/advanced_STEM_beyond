\documentclass[a4paper]{article}
\usepackage{longtable,float,hyperref,color,amsmath,amsxtra,amssymb,latexsym,amscd,amsthm,amsfonts,graphicx}
\numberwithin{equation}{section}
\allowdisplaybreaks
\usepackage{fancyhdr}
\pagestyle{fancy}
\fancyhf{}
\fancyhead[RE,LO]{\footnotesize \textsc \leftmark}
\cfoot{\thepage}
\renewcommand{\headrulewidth}{0.5pt}
\setcounter{tocdepth}{3}
\setcounter{secnumdepth}{3}
\usepackage{imakeidx}
\makeindex[columns=2, title=Alphabetical Index, 
           options= -s index.ist]
\title{\Huge Duong Minh Duc, Sobolev Spaces}
\author{\textsc{Nguyen Quan Ba Hong}\\
{\small Students at Faculty of Math and Computer Science,}\\ 
{\small Ho Chi Minh University of Science, Vietnam} \\
{\small \texttt{email. nguyenquanbahong@gmail.com}}\\
{\small \texttt{blog. \url{www.nguyenquanbahong.com}} 
\footnote{Copyright \copyright\ 2016 by Nguyen Quan Ba Hong, Student at Ho Chi Minh University of Science, Vietnam. This document may be copied freely for the purposes of education and non-commercial research. Visit my site \texttt{\url{www.nguyenquanbahong.com}} to get more.}}}
\begin{document}
\maketitle
\begin{abstract}
I retype \cite{1}, which is used to teach the course \textit{Calculus of Variations} in Ho Chi Minh University of Sciences. Share it!
\end{abstract}
\newpage
\tableofcontents
\newpage
\section{Sobolev Spaces}
\textbf{Definition 1.1.} Let $f$ be a real function on an open subset $D$ of $\mathbb{R}^n$, $x = \left( {{x_1}, \ldots ,{x_n}} \right) \in D$ and $i \in \left\{ {1, \ldots ,n} \right\}$. We define 
\begin{align}
&\dfrac{{\partial f}}{{\partial {x_i}}}\left( x \right) \\
&= \mathop {\lim }\limits_{t \to 0} \dfrac{{f\left( {x + t{e_i}} \right) - f\left( x \right)}}{t}\\
& = \mathop {\lim }\limits_{t \to 0} \dfrac{{f\left( {{x_1}, \ldots ,{x_{i - 1}},{x_i} + t,{x_{i + 1}}, \ldots ,{x_n}} \right) - f\left( {{x_1}, \ldots ,{x_{i - 1}},{x_i},{x_{i + 1}}, \ldots ,{x_n}} \right)}}{t}
\end{align}
provided the limit exists, and $\dfrac{{\partial f}}{{\partial {x_i}}}\left( x \right)$ is called the \textit{partial derivative of $f$ at $x$ with respect to the variable $x_i$.}

If $\dfrac{{\partial f}}{{\partial {x_i}}}\left( x \right)$ exists for any $i$ in $\left\{ {1, \ldots ,n} \right\}$, we say $f$ is \textit{differentiable at $x$} and has \textit{derivative}
\begin{align}
Df\left( x \right) &= \nabla f\left( x \right)\\
& = \left( {\dfrac{{\partial f}}{{\partial {x_1}}}\left( x \right),\dfrac{{\partial f}}{{\partial {x_2}}}\left( x \right), \ldots ,\dfrac{{\partial f}}{{\partial {x_n}}}\left( x \right)} \right)
\end{align}
\textbf{Definition 1.2.} Let $f$ be a real function on an open subset $D$ of $\mathbb{R}^n$. We say
\begin{enumerate}
\item $f$ is \textit{differentiable on $D$} if $\nabla f\left( x \right)$ exists for any $x$ in $D$.
\item \textit{$f$ is of class $C^1\left(D\right)$} if $f$ is differentiable on $D$ and $\nabla f$ is a continuous from $D$ into $\mathbb{R}^n$.
\item \textit{$f$ is of class $C_c ^1\left(D\right)$} if $f$ is of class $C^1\left(D\right)$ and $f\left(x\right)=0$ for any $x$ in $D\backslash {K_f}$, where $K_f$ is a compact set contained in $D$.
\item \textit{$f$ is of class ${C^1}\left( {\overline D} \right)$} if $f$ is of class ${C^1}\left( {{D_f}} \right)$, where $D_f$ is a open set containing $D$.
\end{enumerate}
\textbf{Definition 1.3.} Let $f$ be a real differentiable function on an open subset $D$ of $\mathbb{R}^n$ and $x\in D$. Put ${g_j} = \dfrac{{\partial f}}{{\partial {x_j}}}$, then $g_j$ is a real function on $D$ for any $j$ in $\left\{ {1, \ldots ,n} \right\}$. Let $i$ be in $\left\{ {1, \ldots ,n} \right\}$. We say
\begin{enumerate}
\item $f$ has the \textit{second-order partial derivative} $\dfrac{{{\partial ^2}f}}{{\partial {x_i}\partial {x_j}}}\left( x \right)$ at $x$ if $g_j$ has the partial derivative $\dfrac{{\partial g}}{{\partial {x_i}}}\left( x \right)$ at $x$.
\item $f$ has the \textit{second-order partial derivative at $x$} if $\dfrac{{{\partial ^2}f}}{{\partial {x_i}\partial {x_j}}}\left( x \right)$ exists for any $i,j$ in $\left\{ {1, \ldots ,n} \right\}$. In this case the second-order derivative $D^2 f\left(x\right)$ of $f$ at $x$ is the $n\times n$\textit{-matrix}
\end{enumerate}
\begin{align}
{\left[ {\dfrac{{{\partial ^2}f}}{{\partial {x_i}\partial {x_j}}}\left( x \right)} \right]_{i,j = 1,2, \ldots ,n}}
\end{align}
\textbf{Definition 1.4.} Let $f$ be a real function on an open subset $D$ of $\mathbb{R}^n$. We say
\begin{enumerate}
\item $f$ is \textit{differentiable 2-times on $D$} if $D^2 f\left(x\right)$ exists for any $x$ in $D$.
\item \textit{$f$ is of class $C^2\left(D\right)$} if $f$ is differentiable 2-times on $D$ and $D^2f$ is a continuous from $D$ into $\mathbb{R}^{n\times n}$.
\item \textit{$f$ is of class $C_c^2\left( D \right)$} if $f$ is of class $C^2\left(D\right)$ and $f\left(x\right)=0$ for any $x$ in $D\backslash {K_f}$, where $K_f$ is a compact set contained in $D$. 
\item \textit{$f$ is of class ${C^2}\left( {\overline D} \right)$} if $f$ is of class $C^2\left(D_f\right)$, where $D_f$ is a open set containing $D$.
\end{enumerate}

Similarly, we can define the classes ${C^r}\left( D \right),C_c^r\left( D \right)$ and ${C^r}\left( {\overline D} \right)$ for any integer $r>$. We put
\begin{align}
{C^\infty }\left( D \right) &= \bigcap\limits_{n = 1}^\infty  {{C^r}\left( D \right)} \\
C_c^\infty \left( D \right) &= \bigcap\limits_{n = 1}^\infty  {C_c^r\left( D \right)} \\
{C^\infty }\left( {\overline D} \right) &= \bigcap\limits_{n = 1}^\infty  {{C^r}\left( {\overline D} \right)} 
\end{align}
\textbf{Theorem 1.5.} \textit{Let $D$ be an open subset of $\mathbb{R}^n$, $p \in \left[ {1, + \infty } \right)$ and $f$ be in $L^p\left(D\right)$. Assume}
\begin{align}
\int_D {fgdx}  = 0,\hspace{0.2cm}\forall g \in C_c^\infty \left( D \right)
\end{align}
\textit{Then $f=0 \mbox{ a.e. on} D$.}\\
\\
\textbf{Theorem 1.6.} \textit{Let $D$ be an open subset of $\mathbb{R}^n$ with smooth boundary $\partial D$, $i \in \left\{ {1, \ldots ,n} \right\}$ and $f \in {C^1}\left( {\overline D} \right)$. Then}
\begin{align}
\int_D {f\dfrac{{\partial g}}{{\partial {x_i}}}dx}  &= \int_{\partial D} {fgds}  - \int_D {\dfrac{{\partial f}}{{\partial {x_i}}}gdx} ,\hspace{0.2cm}\forall g \in {C^1}\left( {\bar D} \right)\\
\int_D {f\dfrac{{\partial g}}{{\partial {x_i}}}dx}  &=  - \int_D {\dfrac{{\partial f}}{{\partial {x_i}}}gdx} ,\hspace{0.2cm}\forall g \in C_c^1\left( D \right)
\end{align}
\textit{where $ds$ is the measure on the boundary $\partial D$.}\\

Put 
\begin{align}
{\left\| f \right\|_{1,p}} &= {\left( {\int_D {\left( {{{\left| f \right|}^p} + {{\left\| {\nabla f} \right\|}^p}} \right)dx} } \right)^{\dfrac{1}{p}}},\hspace{0.2cm}\forall f \in {C^1}\left( {\bar D} \right)\\
{\left\| f \right\|_{2,p}} &= {\left( {\int_D {\left( {{{\left| f \right|}^p} + {{\left\| {\nabla f} \right\|}^p} + {{\left\| {{D^2}f} \right\|}^p}} \right)dx} } \right)^{\dfrac{1}{p}}},\hspace{0.2cm}\forall f \in {C^2}\left( {\bar D} \right)\\
{\left\| f \right\|_{k,p}} &= {\left( {\int_D {\left( {{{\left| f \right|}^p} + \sum\limits_{r = 1}^k {{{\left\| {{D^r}f} \right\|}^p}} } \right)dx} } \right)^{\dfrac{1}{p}}},\hspace{0.2cm}\forall f \in {C^k}\left( {\bar D} \right)
\end{align}
We see that  $\left( {C_c^k\left( D \right),{{\left\|  \cdot  \right\|}_{1,p}}} \right)$ and $\left( {{C^k}\left( {\overline D } \right),{{\left\|  \cdot  \right\|}_{1,p}}} \right)$ are norm linear spaces. We denote by $W_0^{k,p}\left( D \right)$ and ${W^{k,p}}\left( D \right)$ their completions respectively. These Banach spaces are called \textit{Sobolev spaces.}

We see that
\begin{align}
W_0^{k,p}\left( D \right) &\subset {W^{k,p}}\left( D \right),\hspace{0.2cm}\forall k \ge 1\\
{W^{k,p}}\left( D \right) &\subset {W^{k - 1,p}}\left( D \right) \subset {L^p}\left( D \right),\hspace{0.2cm}\forall k > 1
\end{align}

Let $p \in \left[ {1, + \infty } \right)$ and $u \in {W^{1,p}}\left( D \right)$. There is a Cauchy sequence $\left\{ {{u_m}} \right\}$ ``converges'' to $u$ in following sense: $\left\{ {{u_m}} \right\}$ converges to $u$ in $L^p\left(D\right)$, $\left\{ {\dfrac{{\partial {u_m}}}{{\partial {x_i}}}} \right\}$ is a Cauchy sequence in $L^p\left(D\right)$ for any $i \in \left\{ {1, \ldots ,n} \right\}$.

We can choose $\left\{ {{u_m}} \right\}$ and $v_1,\ldots,v_n$ in $L^p\left(D\right)$ such that
\begin{align}
\mathop {\lim }\limits_{m \to \infty } {\left\| {\dfrac{{\partial {u_m}}}{{\partial {x_i}}} - {v_i}} \right\|_p} &= 0,\hspace{0.2cm}\forall i \in \left\{ {1, \ldots ,n} \right\}\\
u\left( x \right) &= \mathop {\lim }\limits_{m \to \infty } {u_m}\left( x \right)\mbox{ a.e on } D\\
{v_i}\left( x \right) &= \mathop {\lim }\limits_{m \to \infty } \dfrac{{\partial {u_m}}}{{\partial {x_i}}}\left( x \right)\mbox{ a.e on } D,\hspace{0.2cm}\forall i \in \left\{ {1, \ldots ,n} \right\}
\end{align}
We have
\begin{align}
\label{1.21}
\int_D {{u_m}\dfrac{{\partial \varphi }}{{\partial {x_i}}}dx}  &=  - \int_D {\dfrac{{\partial {u_m}}}{{\partial {x_i}}}\varphi dx} ,\hspace{0.2cm}\forall \varphi  \in C_\infty ^1\left( D \right),m \in \mathbb{N}
\end{align}
and
\begin{align}
&\left| {\int_D {{u_m}\dfrac{{\partial \varphi }}{{\partial {x_i}}}dx}  - \int_D {u\dfrac{{\partial \varphi }}{{\partial {x_i}}}dx} } \right| \\
&= \left| {\int_D {\left( {{u_m} - u} \right)\dfrac{{\partial \varphi }}{{\partial {x_i}}}dx} } \right|\\
 &\le \int_D {\left| {\left( {{u_m} - u} \right)\dfrac{{\partial \varphi }}{{\partial {x_i}}}} \right|dx} \\
 &\le {\left( {\int_D {{{\left| {{u_m} - u} \right|}^p}dx} } \right)^{\dfrac{1}{p}}}{\left( {\int_D {{{\left| {\dfrac{{\partial \varphi }}{{\partial {x_i}}}} \right|}^{\dfrac{p}{{p - 1}}}}dx} } \right)^{\dfrac{{p - 1}}{p}}} \to 0 \mbox{ as } m \to \infty \label{1.25}
\end{align}
similarly,
\begin{align}
&\left| {\int_D {\dfrac{{\partial {u_m}}}{{\partial {x_i}}}\varphi dx}  - \int_D {{v_i}\varphi dx} } \right|\\
 &= \left| {\int_D {\left( {\dfrac{{\partial {u_m}}}{{\partial {x_i}}} - {v_i}} \right)\varphi dx} } \right|\\
& \le \int_D {\left| {\left( {\dfrac{{\partial {u_m}}}{{\partial {x_i}}} - {v_i}} \right)\varphi } \right|dx} \\
& \le {\left( {\int_D {{{\left| {\dfrac{{\partial {u_m}}}{{\partial {x_i}}} - {v_i}} \right|}^p}dx} } \right)^{\dfrac{1}{p}}}{\left( {\int_D {{{\left| \varphi  \right|}^{\dfrac{p}{{p - 1}}}}dx} } \right)^{\dfrac{{p - 1}}{p}}} \to 0 \mbox{ as } m \to \infty \label{1.29}
\end{align}

Combining \eqref{1.21}, \eqref{1.25} and \eqref{1.29} yields
\begin{align}
\int_D {u\dfrac{{\partial \varphi }}{{\partial {x_i}}}dx}  =  - \int_D {{v_i}\varphi dx} ,\hspace{0.2cm}\forall \varphi  \in C_\infty ^1\left( D \right),i \in \left\{ {1, \ldots ,n} \right\}
\end{align}

We say $v_i$ is the generalized partial derivative of $u$ with respect to $x_i$ and denote it by $\dfrac{{\partial u}}{{\partial {x_i}}}$.

Thus, let $u$ be in $W^{1,p}\left(D\right)$, then $u$ has its generalized partial derivatives $\dfrac{{\partial u}}{{\partial {x_i}}} \in {L^p}\left( D \right)$ such that
\begin{align}
\int_D {u\dfrac{{\partial \varphi }}{{\partial {x_i}}}dx}  =  - \int_D {\dfrac{{\partial u}}{{\partial {x_i}}}\varphi dx} ,\hspace{0.2cm}\forall \varphi  \in C_\infty ^1\left( D \right),i \in \left\{ {1, \ldots ,n} \right\}
\end{align}

Thus, let $u$ be in $W^{1,p}\left(D\right)$, then $u$ has its generalized partial derivatives $\dfrac{{\partial u}}{{\partial {x_i}}} \in {L^p}\left( D \right)$ such that
\begin{align}
\int_D {u\dfrac{{\partial \varphi }}{{\partial {x_i}}}dx}  =  - \int_D {\dfrac{{\partial u}}{{\partial {x_i}}}\varphi dx} ,\hspace{0.2cm}\forall \varphi  \in C_c ^1\left( D \right),i \in \left\{ {1, \ldots ,n} \right\}
\end{align}
\textbf{Example 1.7.} Let $\eta$ be in $W_0^{1,p}\left(D\right)$. We can choose a sequence $\left\{ {{\varphi _m}} \right\}$ in $C_c^1\left(D\right)$, which converges to $\eta$ in $W_0^{1,p}\left(D\right)$. Arguing as above, we get
\begin{align}
\int_D {u\dfrac{{\partial \eta }}{{\partial {x_i}}}dx}  =  - \int_D {\dfrac{{\partial u}}{{\partial {x_i}}}\eta dx} ,\hspace{0.2cm}\forall \eta  \in W_0^{1,p}\left( D \right),i \in \left\{ {1, \ldots ,n} \right\}
\end{align}

Let $D=\left(-1,1\right)$ and $u\left( x \right) = \left| x \right|$ for any $x$ in $D$. Put
\begin{align}
{u_m}\left( x \right) = \sqrt {{x^2} + \dfrac{1}{m}} ,\hspace{0.2cm}\forall x \in D,m \in {\mathbb{N}^*}
\end{align}
We have
\begin{align}
\left| {{u_m}\left( x \right)} \right| &\le \sqrt 2 \\
\mathop {\lim }\limits_{m \to \infty } {u_m}\left( x \right) &= \sqrt {{x^2}}  = u\left( x \right),\hspace{0.2cm}\forall x \in D\\
\left| {{u_m}'\left( x \right)} \right| &= \left| {\dfrac{x}{{\sqrt {{x^2} + \dfrac{1}{m}} }}} \right| \le 1,\hspace{0.2cm}\forall x \in D\backslash \left\{ 0 \right\}\\
\mathop {\lim }\limits_{m \to \infty } {u_m}'\left( x \right) &= \dfrac{x}{{\sqrt {{x^2}} }} = \mbox{sign}x,\hspace{0.2cm}\forall x \in D\backslash \left\{ 0 \right\}
\end{align}

By the Lebesgue dominated convergence theorem, $u$ is in $W^{1,2}\left(D\right)$ and its generalized derivative is $u'\left(x\right)=\mbox{sign} x$.\hfill $\square$ \\
\\
\textbf{Example 1.8.} Let $D=\left(-1,1\right)$. Put
\begin{align}
u\left( x \right) = \left\{ {\begin{array}{*{20}{c}}
{1,\forall x \in \left( { - 1,0} \right]}\\
{0,\forall x \in \left( {0,1} \right)}
\end{array}} \right.
\end{align}
We see that $u\in L^2\left(D\right)$.

Now assume there is $v\in L^2\left(D\right)$ such that
\begin{align}
\label{1.40}
\int_D {u\varphi 'dx}  =  - \int_D {v\varphi dx} ,\hspace{0.2cm}\forall \varphi  \in C_c^1\left( D \right)
\end{align}
We have
\begin{align}
\int_D {u\varphi 'dx}  &= \int_{ - 1}^0 {\varphi 'dx} \\
& = \varphi \left( 0 \right) - \varphi \left( { - 1} \right)\\
& = \varphi \left( 0 \right),\hspace{0.2cm}\forall \varphi  \in C_c^1\left( D \right) \label{1.43}
\end{align}

By \eqref{1.40} and \eqref{1.43}, we see that
\begin{align}
\int_D {v\varphi dx}  = 0,\hspace{0.2cm}\forall \varphi  \in C_c^1\left( {D\backslash \left\{ 0 \right\}} \right)
\end{align}
which implies $v=0$ a.e. on $D\backslash \left\{ 0\right\}$. Thus $v=0$ a.e. on $D$ or
\begin{align}
\label{1.45}
\int_D {v\varphi dx}  = 0,\forall \varphi  \in C_c^1\left( D \right)
\end{align}

By \eqref{1.43} and \eqref{1.45}, $\varphi \left(0\right) =0$ for any $\varphi \in C_c^1\left(D\right)$.

Therefore, $W^{1,2}\left(D\right) \subset \L^2\left(D\right)$, but $W^{1,2}\left(D\right) \ne L^2\left(D\right)$. \hfill $\square$\\

The following properties of generalized derivatives are proved in Chapter 7 of the book ``D. Gilbarg and N. Trudinger, \textit{Elliptic partial differential equations of second order}''.\\
\\
\textbf{Theorem 1.9.} \textit{Let $D$ be an open subset of $\mathbb{R}^n$, $p$ and $q$ be in $\left(1,+\infty \right)$ such that}
\begin{align}
\dfrac{1}{p} + \dfrac{1}{q} = 1
\end{align}
\textit{Let $u\in W^{1,p}\left(D\right)$ and $v\in W^{1,q}\left(D\right)$. Then $uv \in W^{1,1}\left(D\right)$ and}
\begin{align}
\dfrac{{\partial \left( {uv} \right)}}{{\partial {x_i}}} = \dfrac{{\partial u}}{{\partial {x_i}}}v + u\dfrac{{\partial v}}{{\partial {x_i}}},\forall i \in \left\{ {1, \ldots ,n} \right\}
\end{align}
\textbf{Theorem 1.10.} \textit{Let $a_1<a_2<\ldots<a_k$ be $k$ real numbers, $D$ be an open subset of $\mathbb{R}^n$. Put $B=\left\{a_1,\ldots,a_k\right\}$. Let $f$ be a real function on $\mathbb{R}$ of class $C\left( \mathbb{R} \right) \cap {C^1}\left( {\mathbb{R}\backslash B} \right)$ such that $f'$ is discontinuous at every point of $B$, and $f' \in L^{\infty}\left(\mathbb{R}\backslash B\right)$. Let $u\in W^{1,p}\left(D\right)$ with $p\in \left[1,+\infty\right)$. Then $v = f \circ u$ belongs to $W^{1,p}\left(D\right)$ and}
\begin{align}
\dfrac{{\partial v}}{{\partial {x_i}}}\left( x \right) = \left\{ {\begin{array}{*{20}{c}}
{f'\left( {u\left( x \right)} \right)\dfrac{{\partial u}}{{\partial {x_i}}},\mbox{ if } u\left( x \right) \in R\backslash B}\\
{0,\mbox{ if } u\left( x \right) \in B}
\end{array}} \right.
\end{align}
\textbf{Theorem 1.11.} \textit{Let $D$ be an open subset of $\mathbb{R}^n$ and $u\in W^{1,p}\left(D\right)$ with $p\in \left[1,\infty\right)$. Put}
\begin{align}
{u^ + } &= \max \left\{ {0,u} \right\}\\
{u^ - } &= \max \left\{ {0, - u} \right\}
\end{align}
\textit{Then $u^+,u^-$ and $\left|u\right|$ belong to $W^{1,p}\left(D\right)$ and}
\begin{align}
\frac{{\partial {u^ + }}}{{\partial {x_i}}}\left( x \right) &= \left\{ {\begin{array}{*{20}{c}}
{0,\mbox{ if } u\left( x \right) \le 0}\\
{\frac{{\partial u}}{{\partial {x_i}}}\left( x \right),\mbox{ if } u\left( x \right) > 0}
\end{array}} \right.\\
\frac{{\partial {u^ + }}}{{\partial {x_i}}}\left( x \right) &= \left\{ {\begin{array}{*{20}{c}}
{\frac{{\partial u}}{{\partial {x_i}}}\left( x \right),\mbox{ if } u\left( x \right) < 0}\\
{0,\mbox{ if } u\left( x \right) \ge 0}
\end{array}} \right.\\
\frac{{\partial \left| u \right|}}{{\partial {x_i}}}\left( x \right) &= \left\{ {\begin{array}{*{20}{c}}
{\frac{{\partial u}}{{\partial {x_i}}}\left( x \right),\mbox{ if } u\left( x \right) > 0}\\
\begin{array}{l}
0,\mbox{ if } u\left( x \right) = 0\\
 - \frac{{\partial u}}{{\partial {x_i}}}\left( x \right),\mbox{ if } u\left( x \right) < 0
\end{array}
\end{array}} \right.
\end{align}

We see that
\begin{align}
W_0^{k,p}\left( D \right) &\subset {W^{k,p}}\left( D \right),\forall k \ge 1\\
{W^{k,p}}\left( D \right) &\subset {W^{k - 1,p}}\left( D \right) \subset {L^p}\left( D \right),\forall k > 1\\
W_0^{1,p}\left( D \right) &\subset {W^{1,p}}\left( D \right) \subset {L^p}\left( D \right)
\end{align}
\textbf{Theorem 1.12 (Sobolev embedding).} \textit{Let $D$ be an open subset with smooth boundary in $\mathbb{R}^n$, and $u\in W^{1,p}\left(D\right)$ with $p\in \left[1,+\infty\right)$. Then}
\begin{enumerate}
\item \textit{$u \in L^q\left(D\right)$ where $q = \frac{{np}}{{n - p}}$ if $p < n$.}
\item \textit{$u$ is of class $C^r\left(\overline{D}\right)$ if $0 \le r < 1 - \frac{p}{n}$.}
\end{enumerate}
\textbf{Theorem 1.13 (Sobolev embedding).} \textit{Let $D$ be an open subset with smooth boundary in $\mathbb{R}^n$, and $u\in W^{k,p}\left(D\right)$ with $p\in \left[1,+\infty\right)$. Then}
\begin{enumerate}
\item \textit{$u \in L^q\left(D\right)$ where $q = \frac{{np}}{{n - kp}}$ if $kp < n$.}
\item \textit{$u$ is of class $C^r\left(\overline{D}\right)$ if $0 \le r < k - \frac{p}{n}$.}
\end{enumerate}

The proof of this theorem is in the book of Adams.\\
\\
\textbf{Theorem 1.14 (Sobolev embedding).} \textit{Let $D$ be an open subset with smooth boundary in $\mathbb{R}^n$, and $u\in W^{k,p}\left(D\right)$ with $p\in \left[1,+\infty\right)$. Then $u \in L^q\left(D\right)$ if $q \in \left[ {p,\frac{{np}}{{n - kp}}} \right]$ and $kp < n$.}\\
\\
\textbf{Theorem 1.15 (Sobolev embedding).} \textit{Let $D$ be an open subset with smooth boundary in $\mathbb{R}^n$, and $u\in W^{k,p}\left(D\right)$ with $p\in \left[1,+\infty\right)$. Then $u \in L^q\left(D\right)$ if $q \in \left[ {1,\frac{{np}}{{n - kp}}} \right]$ and $kp < n$.}\\
\\
\textbf{Theorem 1.16 (Sobolev inequality).} \textit{Let $D$ be a bounded open subset with smooth boundary in $\mathbb{R}^n$, $n$ and $k$ be positive integers and $p\in \left[1,+\infty \right)$ such that $kp<n$. Then for any $q \in \left[ {1,\frac{{np}}{{n - kp}}} \right]$ there is a positive real number $C$ such that}
\begin{align}
{\left\| u \right\|_q} \le C{\left\| u \right\|_{k,p}},\hspace{0.2cm} \forall u \in {W^{k,p}}\left( D \right)
\end{align}
\textbf{Theorem 1.17 (Poincare inequality).} \textit{Let $D$ be a bounded open subset with smooth boundary in $\mathbb{R}^n$, $n$ be a positive integer, $p\in \left[1,\infty \right)$ such that $p<n$. Then for any $q \in \left[ {1,\frac{{np}}{{n - kp}}} \right]$ there is a positive real number $C$ such that}
\begin{align}
{\left\| u \right\|_q} \le C{\left\| {\nabla u} \right\|_p},\hspace{0.2cm}\forall u \in W_0^{1,p}\left( D \right)
\end{align}
\textbf{Theorem 1.18.} \textit{Let $D$ be a bounded open subset with smooth boundary in $\mathbb{R}^n$, $n$ be a positive integer, $p\in \left[1,\infty\right)$ such that $p<n$. Put}
\begin{align}
|||u||{|_{1,p}} = {\left( {\int_D {{{\left\| {\nabla u} \right\|}^p}dx} } \right)^{\frac{1}{p}}},\hspace{0.2cm}\forall u \in W_0^{1,p}\left( D \right)
\end{align}
\textit{Then there are a positive real number $c$ such that}
\begin{align}
c{\left\| u \right\|_{1,p}} \le ||u||{|_{1,p}} \le {\left\| u \right\|_{1,p}},\forall u \in W_0^{1,p}\left( D \right)
\end{align}
\textbf{Theorem 1.19.} \textit{$\left( {W_0^{1,2}\left( D \right),||| \cdot |||} \right)$ is a Hilbert space with the following inner product}
\begin{align}
\left\langle {u,v} \right\rangle  = \int_D {\nabla u \cdot \nabla vdx} ,\hspace{0.2cm}\forall u,v \in W_0^{1,2}\left( D \right)
\end{align}
\textbf{Theorem 1.20.} \textit{${W^{1,2}}\left( D \right)$ is a Hilbert space with the following inner product}
\begin{align}
\left\langle {u,v} \right\rangle  = \int_D {\left( {uv + \nabla u \cdot \nabla v} \right)dx} ,\hspace{0.2cm}\forall u,v \in {W^{1,2}}\left( D \right)
\end{align}
\textbf{Theorem 1.21 (Rellich-Kondrachov).} \textit{Let $D$ be a bounded open subset with smooth boundary in $\mathbb{R}^n$, $k$ be positive integer, and $p\in \left[1,+\infty \right)$ such that $kp<n$. Let $q \in \left[ {1,\frac{{np}}{{n - kp}}} \right]$ and put}
\begin{align}
T\left( u \right) = u,\hspace{0.2cm}\forall u \in {W^{k,p}}\left( D \right)
\end{align}
\textit{Then $T$ is a bounded linear mapping from $W^{k,p}\left(D\right)$ into $L^q\left(D\right)$, and the closure $T\left(A\right)$ in $L^q\left(D\right)$ is compact in $L^q\left(D\right)$ for any bounded subset $A$ in $W^{k,p}\left(D\right)$.}\\
\\
\textbf{Theorem 1.22 (Sobolev embedding).} \textit{Let $D$ be a bounded open subset with smooth boundary in $\mathbb{R}$, and $u \in W^{1,p}\left(D\right)$ with $p\in \left(1,+\infty\right)$. Then $u \in L^q\left(D\right)$ for any $q\in \left[1,+\infty \right)$.}\\
\\
\textbf{Theorem 1.23 (Sobolev inequality).} \textit{Let $D$ be a bounded open subset with smooth boundary in $\mathbb{R}$, and $p\in \left(1,+\infty\right)$. Then for any $q\in \left[1,+\infty\right)$, there is a positive real number $C$ such that}
\begin{align}
{\left\| u \right\|_q} \le C{\left\| u \right\|_{1,p}},\hspace{0.2cm} \forall u \in {W^{1,p}}\left( D \right)
\end{align}
\textbf{Theorem 1.24 (Rellich-Kondrachov).} \textit{Let $D$ be a bounded open subset with smooth boundary in $\mathbb{R}$, $p\in \left(1,+\infty\right)$ and $q\in \left[1,+\infty\right)$. Put}
\begin{align}
T\left( u \right) = u,\hspace{0.2cm}\forall u \in {W^{1,p}}\left( D \right)
\end{align}
\textit{Then $T$ is a bounded linear mapping from $W^{1,p}\left(D\right)$ into $L^q\left(D\right)$, and the closure $T\left(A\right)$ in $L^q\left(D\right)$ is compact in $L^q\left(D\right)$ for any bounded subset $A$ in $W^{1,p}\left(D\right)$.}\\
\\
\textbf{Theorem 1.25.} \textit{Let $D$ be a bounded open subset with smooth boundary in $\mathbb{R}^n$, $p\in \left(1,+\infty\right)$, and $T$ be a linear mapping from $W^{1,p}\left(D\right)$ into $\mathbb{R}$. Then $T$ is continuous on $W^{1,p}\left(D\right)$ if and only if there are $g,g_1,\ldots,g_n$ in ${L^{\frac{p}{{p - 1}}}}\left( D \right)$ such that}
\begin{align}
T\left( u \right) = \int_D {\left( {ug + \sum\limits_{i = 1}^n {\frac{{\partial u}}{{\partial {x_i}}}{g_i}} } \right)dx} ,\hspace{0.2cm}\forall u \in {W^{1,p}}\left( D \right)
\end{align}
\textbf{Theorem 1.26.} \textit{Let $D$ be a bounded open subset with smooth boundary in $\mathbb{R}^n$, and $T$ be a linear mapping from $W_0^{1,2}\left(D\right)$ into $\mathbb{R}$. Then $T$ is continuous on $W_0^{1,2}\left(D\right)$ if and only if there is $g$ in $W_0^{1,2}\left(D\right)$ such that}
\begin{align}
T\left( u \right) = \int_D {\left( {\sum\limits_{i = 1}^n {\frac{{\partial u}}{{\partial {x_i}}}\frac{{\partial g}}{{\partial {x_i}}}} } \right)dx} ,\hspace{0.2cm}\forall u \in W_0^{1,2}\left( D \right)
\end{align}
\textbf{Definition 1.27.} Let $D$ be a bounded open subset with smooth boundary in $\mathbb{R}^n$, $p\in \left(1,+\infty\right)$, $v \in W^{1,p}\left(D\right)$ and $\left\{v_m\right\}$ be a sequence in $W^{1,p}\left(D\right)$. Then we say $\left\{v_m\right\}$ weakly converges to $v$ in $W^{1,p}\left(D\right)$ if $\left\{ {T\left( {{v_m}} \right)} \right\}$ converges to $T\left(v\right)$ for any bounded linear mapping $T$ from $W^{1,p}\left(D\right)$ into $\mathbb{R}$.\\
\\
\textbf{Theorem 1.28.} \textit{Let $D$ be a bounded open subset with smooth boundary in $\mathbb{R}^n$, $p\in \left(1,+\infty\right)$ and $\left\{u_m\right\}$ be a bounded sequence in $W^{1,p}\left(D\right)$. Then there are $u \in W^{1,p}\left(D\right)$ and a subsequence $\left\{u_{m_k}\right\}$ such that $\left\{u_{m_k}\right\}$ weakly converges to $u$.}\\
\\
\\
\\
\begin{center}
\textsc{The End}
\end{center}

\newpage
\printindex
\newpage
\begin{thebibliography}{999}
\bibitem {1} Duong Minh Duc, \textit{Sobolev Spaces}, Ho Chi Minh University of Sciences.
\end{thebibliography}
\end{document}