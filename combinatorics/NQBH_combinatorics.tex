\documentclass{article}
\usepackage[backend=biber,natbib=true,style=alphabetic,maxbibnames=50]{biblatex}
\addbibresource{/home/nqbh/reference/bib.bib}
\usepackage[utf8]{vietnam}
\usepackage{tocloft}
\renewcommand{\cftsecleader}{\cftdotfill{\cftdotsep}}
\usepackage[colorlinks=true,linkcolor=blue,urlcolor=red,citecolor=magenta]{hyperref}
\usepackage{amsmath,amssymb,amsthm,enumitem,float,graphicx,mathtools,tikz}
\usetikzlibrary{angles,calc,intersections,matrix,patterns,quotes,shadings}
\allowdisplaybreaks
\newtheorem{assumption}{Assumption}
\newtheorem{baitoan}{}
\newtheorem{cauhoi}{Câu hỏi}
\newtheorem{conjecture}{Conjecture}
\newtheorem{corollary}{Corollary}
\newtheorem{dangtoan}{Dạng toán}
\newtheorem{definition}{Definition}
\newtheorem{dinhly}{Định lý}
\newtheorem{dinhnghia}{Định nghĩa}
\newtheorem{example}{Example}
\newtheorem{ghichu}{Ghi chú}
\newtheorem{hequa}{Hệ quả}
\newtheorem{hypothesis}{Hypothesis}
\newtheorem{lemma}{Lemma}
\newtheorem{luuy}{Lưu ý}
\newtheorem{nhanxet}{Nhận xét}
\newtheorem{notation}{Notation}
\newtheorem{note}{Note}
\newtheorem{principle}{Principle}
\newtheorem{problem}{Problem}
\newtheorem{proposition}{Proposition}
\newtheorem{question}{Question}
\newtheorem{remark}{Remark}
\newtheorem{theorem}{Theorem}
\newtheorem{vidu}{Ví dụ}
\usepackage[left=1cm,right=1cm,top=5mm,bottom=5mm,footskip=4mm]{geometry}
\def\labelitemii{$\circ$}
\DeclareRobustCommand{\divby}{%
	\mathrel{\vbox{\baselineskip.65ex\lineskiplimit0pt\hbox{.}\hbox{.}\hbox{.}}}%
}
\setlist[itemize]{leftmargin=*}
\setlist[enumerate]{leftmargin=*}

\title{Combinatorics -- Tổ Hợp}
\author{Nguyễn Quản Bá Hồng\footnote{A Scientist {\it\&} Creative Artist Wannabe. E-mail: {\tt nguyenquanbahong@gmail.com}. Bến Tre City, Việt Nam.}}
\date{\today}

\begin{document}
\maketitle
\begin{abstract}
	This text is a part of the series {\it Some Topics in Advanced STEM \& Beyond}:
	
	{\sc url}: \url{https://nqbh.github.io/advanced_STEM/}.
	
	Latest version:
	\begin{itemize}
		\item {\it Combinatorics -- Tổ Hợp}.
		
		PDF: {\sc url}: \url{.pdf}.
		
		\TeX: {\sc url}: \url{.tex}.
		\item {\it }.
		
		PDF: {\sc url}: \url{.pdf}.
		
		\TeX: {\sc url}: \url{.tex}.
	\end{itemize}
\end{abstract}
\tableofcontents

%------------------------------------------------------------------------------%

\section{Wikipedia's}

\subsection{Wikipedia{\tt/}extremal combinatorics}
``{\it Extremal combinatorics} is a field of mathematics, which is itself a part of mathematics. Extremal combinatorics studies how large or how small a collection of finite objects (numbers, graphs, vectors, sets, etc.) can be, if it has to satisfy certain restrictions.

Much of extremal combinatorics concerns \href{https://en.wikipedia.org/wiki/Class_(set_theory)}{classes} of sets; this is called {\it extremal set theory}. E.g., in an $n$-element set, what is the largest number of $k$-element subsets that can pairwise intersect one another? What is the largest number of subsets of which more contains any other? The latter question is answered by \href{https://en.wikipedia.org/wiki/Sperner%27s_theorem}{Sperner's theorem}, which gave rise to much of extremal set theory.

Another kind of example: How many people can be invited to a party where among each 3 people there are 2 who know each other \& 2 who don't know each other? \href{https://en.wikipedia.org/wiki/Ramsey_theory}{Ramsey theory} shows: at most 5 persons can attend such a party (see \href{https://en.wikipedia.org/wiki/Theorem_on_Friends_and_Strangers}{Theorem on Friends \& Strangers}). Or, suppose given a finite set of nonzero integers, \& are asked to mark as large a subset as possible of this set under the restriction that the sum of any 2 marked integers cannot be marked. It appears that (independent of what the given integers actually are) we can always mark at least $\frac{1}{3}$ of them.'' -- \href{https://en.wikipedia.org/wiki/Extremal_combinatorics}{Wikipedia{\tt/}extremal combinatorics}

%------------------------------------------------------------------------------%

\subsection{Wikipedia{\tt/}extremal graph theory}
``{\sf\href{https://en.wikipedia.org/wiki/Tur%C3%A1n_graph}{Tur\'an graph} $T(n,r)$ is an example of an extremal graph. It has the maximum possible number of edges for a graph on $n$ vertices without $(r + 1)$-\href{https://en.wikipedia.org/wiki/Clique_(graph_theory)}{cliques}. This is $T(13,4)$.} {\it Extremal graph theory} is a branch of combinatorics, itself an area of mathematics, that lies at the intersection of \href{https://en.wikipedia.org/wiki/Extremal_combinatorics}{extremal combinatorics} \& \href{https://en.wikipedia.org/wiki/Graph_theory}{graph theory}. In essence, extremal graph theory studies how global properties of a graph influence local substructure. Results in extremal graph theory deal with quantitative connections between various \href{https://en.wikipedia.org/wiki/Graph_property}{graph properties}, both global (e.g. number of vertices \& edges) \& local (e.g. existence of specific subgraphs), \& problems in extremal graph theory can often be formulated as optimization problems: how big or small a parameter of a graph can be, given some constraints that the graph has to satisfy? A graph that is an optimal solution to such an optimization problem is called an {\it extremal graph}, \& extremal graphs are important objects of study in extremal graph theory.

Extremal graph theory is closely related to fields e.g. \href{https://en.wikipedia.org/wiki/Ramsey_theory}{Ramsey theory}, \href{https://en.wikipedia.org/wiki/Spectral_graph_theory}{spectral graph theory}, \href{https://en.wikipedia.org/wiki/Computational_complexity_theory}{computational complexity theory}, \& \href{https://en.wikipedia.org/wiki/Additive_combinatorics}{additive combinatorics}, \& frequently employs \href{https://en.wikipedia.org/wiki/Probabilistic_method}{probabilistic method}.

\subsubsection{History}

\begin{quote}
	``Extremal graph theory, in its strictest sense, is a branch of graph theory developed \& loved by Hungarians.'' -- {\sc Bollob\'as} (2004)
\end{quote}
Mantel's Theorem (1907) \& \href{https://en.wikipedia.org/wiki/Tur%C3%A1n%27s_theorem}{Tur\'an's Theorem} (1941) were some of 1st milestones in stud of extremal graph theory. In particular, Tur\'an's theorem would later on become a motivation for the finding of results e.g. \href{https://en.wikipedia.org/wiki/Erd%C5%91s%E2%80%93Stone_theorem}{Erd\H{o}s--Stone theorem} (1946). This result is surprising because it connects chromatic number with maximal number of edges in an $H$-free graph. An alternative proof of Erd\H{o}s--Stone was given in 1975, \& utilized \href{https://en.wikipedia.org/wiki/Szemer%C3%A9di_regularity_lemma}{Szemer\'edi regularity lemma}, an essential technique in resolution of extremal graph theory problems.

\subsubsection{Topics \& concepts}

\begin{itemize}
	\item {\bf Graph coloring.} Main article: \href{https://en.wikipedia.org/wiki/Graph_coloring}{Wikipedia{\tt/}graph coloring}. A {\it proper (vertex) coloring} of a graph $G$ is a coloring of vertices of $G$ s.t. no 2 adjacent vertices have the same color. Minimum number of colors needed to properly color $G$ is called {\it chromatic number} of $G$, denoted $\chi(G)$. Determining chromatic number of specific graphs is a fundamental question in extremal graph theory, because many problems in area \& related areas can be formulated in terms of graph coloring.
	
	2 simple lower bounds to chromatic number of a graph $G$ is given by \href{https://en.wikipedia.org/wiki/Clique_number}{clique number} $\omega(G)$ -- all vertices of a clique must have distinct colors -- \& by $\frac{|V(G)|}{\alpha(G)}$, where $\alpha(G)$ is independence number, because set of vertices with a given color must form an \href{https://en.wikipedia.org/wiki/Independent_set_(graph_theory)}{independent set}.
	
	A \href{https://en.wikipedia.org/wiki/Greedy_coloring}{greedy coloring} gives upper bound $\chi(G)\le\Delta(G) + 1$, where $\Delta(G)$ is maximum degree of $G$. When $G$ is not an odd cycle or a clique, \href{https://en.wikipedia.org/wiki/Brooks%27_theorem}{Brooks' theorem} states: upper bound can be reduced to $\Delta(G)$. When $G$ is a \href{https://en.wikipedia.org/wiki/Planar_graph}{planar graph}, \href{https://en.wikipedia.org/wiki/Four-color_theorem}{4-color theorem} states: $G$ has chromatic number $\le4$.
	
	In general, determining whether a given graph has a coloring with a prescribed number of colors is known to be \href{https://en.wikipedia.org/wiki/NP-hard}{NP-hard}.
	
	In addition to vertex coloring, other types of coloring are also studied, e.g. \href{https://en.wikipedia.org/wiki/Edge_coloring}{edge colorings}. {\it Chromatic index} $\chi'(G)$ of a graph $G$ is minimum number of colors in a proper edge-coloring of a graph, \& \href{https://en.wikipedia.org/wiki/Vizing%27s_theorem}{Vizing's theorem} states: chromatic index of a graph $G$ is either $\Delta(G)$ or $\Delta(G) + 1$.
	\item {\bf Forbidden subgraphs.} Main article: \href{https://en.wikipedia.org/wiki/Forbidden_subgraph_problem}{Wikipedia{\tt/}forbidden subgraph problem}. {\it Forbidden subgraph problem} is 1 of central problems in extremal graph theory. Given a graph $G$, forbidden subgraph problem asks for maximal number of edges ${\rm ex}(n,G)$ in an $n$-vertex graph that does not contain a subgraph isomorphic to $G$.
	
	When $G = K_r$ is a complete graph, \href{https://en.wikipedia.org/wiki/Tur%C3%A1n%27s_theorem}{Tur\'an's theorem} gives an exact value for ${\rm ex}(n,K_r)$ \& characterizes all graphs attaining this maximum; such graphs are known as Tur\'an graphs. For non-bipartite graphs $G$, \href{https://en.wikipedia.org/wiki/Erd%C5%91s%E2%80%93Stone_theorem}{Erd\H{o}s--Stone theorem} gives an asymptotic value of ${\rm ex}(n,G)$ in terms of chromatic number of $G$. Problem of determining asymptotics of ${\rm ex}(n,G)$ when $G$ is a \href{https://en.wikipedia.org/wiki/Bipartite_graph}{bipartite graph} is open; when $G$ is a complete bipartite graph, this is known as \href{https://en.wikipedia.org/wiki/Zarankiewicz_problem}{Zarankiewicz problem}.
	\item {\bf Homomorphism density.} Main article: \href{https://en.wikipedia.org/wiki/Homomorphism_density}{Wikipedia{\tt/}Homomorphism density}. {\it Homomorphism density} $t(H,G)$ of a graph $H$ in a graph $G$ describes probability that a randomly chosen map from vertex set of $H$ to vertex set of $G$ is also a \href{https://en.wikipedia.org/wiki/Graph_homomorphism}{graph homomorphism}. It is closely related to {\it subgraph density}, which describes how often a graph $H$ is found as a subgraph of $G$.
	
	Forbidden subgraph problem can be restated as maximizing edge density of a graph with $G$-density 0, \& this naturally leads to generalization in form of {\it graph homomorphism inequalities}, which are inequalities relating $t(H,G)$ for various graphs $H$. By extending homomorphism density to \href{https://en.wikipedia.org/wiki/Graphon}{graphons}, which are objects that arise as a limit of \href{https://en.wikipedia.org/wiki/Dense_graph}{dense graphs}, graph homomorphism density can be written in form of integrals, \& inequalities e.g. \href{https://en.wikipedia.org/wiki/Cauchy-Schwarz_inequality}{Cauchy--Schwarz inequality} \& \href{https://en.wikipedia.org/wiki/H%C3%B6lder%27s_inequality}{H\"older's inequality} can be used to derive homomorphism inequalities.
	
	A major open problem relating homomorphism densities is \href{https://en.wikipedia.org/wiki/Sidorenko%27s_conjecture}{Sidorenko's conjecture}, which states a tight lower bound on homomorphism density of a bipartite graph in a graph $G$ in terms of edge density of $G$.
	\item {\bf Graph regularity.} Main article: \href{https://en.wikipedia.org/wiki/Szemer%C3%A9di_regularity_lemma}{Wikipedia{\tt/}Szemerédi regularity lemma}. {\sf Edges between parts in a regular partition behave in a ``random-like'' fashion.} {\it Szemerédi's regularity lemma} states: all graphs are `regular' in sense: vertex set of any given graph can be partitioned into a bounded number of parts s.t. bipartite graph between most pairs of parts behave like \href{https://en.wikipedia.org/wiki/Random_graph}{random bipartite graphs}. This partition gives a structural approximation to original graph, which reveals information about properties of original graph.
	
	Regularity lemma is a central result in extremal graph theory, \& also has numerous applications in adjacent fields of \href{https://en.wikipedia.org/wiki/Additive_combinatorics}{additive combinatorics} \& \href{https://en.wikipedia.org/wiki/Computational_complexity_theory}{commputational complexity theory}. In addition to (Szemerédi) regularity, closely related notions of graph regularity e.g. strong regularity \& Frieze-Kannan weak regularity have also been studied, as well as extensions of regularity to \href{https://en.wikipedia.org/wiki/Hypergraphs}{hypergraphs}.
	
	Applications of graph regularity often utilize forms of counting lemmas \& removal lemmas. In simplest forms, \href{https://en.wikipedia.org/wiki/Graph_removal_lemma#graph_counting_lemma}{graph counting lemma} uses regularity between pairs of parts in a regular partition to approximate number of subgraphs, \& \href{https://en.wikipedia.org/wiki/Graph_removal_lemma}{graph removal lemma} states: given a graph with few copies of a given subgraph, can remove a small number of edges to eliminate all copies of subgraph.'' -- \href{https://en.wikipedia.org/wiki/Extremal_graph_theory}{Wikipedia{\tt/}extremal graph theory}
\end{itemize}

%------------------------------------------------------------------------------%

\section{Miscellaneous}

%------------------------------------------------------------------------------%

\printbibliography[heading=bibintoc]
	
\end{document}