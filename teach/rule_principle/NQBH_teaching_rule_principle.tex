\documentclass{article}
\usepackage[backend=biber,natbib=true,style=alphabetic,maxbibnames=50]{biblatex}
\addbibresource{/home/nqbh/reference/bib.bib}
\usepackage[utf8]{vietnam}
\usepackage{tocloft}
\renewcommand{\cftsecleader}{\cftdotfill{\cftdotsep}}
\usepackage[colorlinks=true,linkcolor=blue,urlcolor=red,citecolor=magenta]{hyperref}
\usepackage{amsmath,amssymb,amsthm,enumitem,float,graphicx,mathtools,tikz}
\usetikzlibrary{angles,calc,intersections,matrix,patterns,quotes,shadings}
\allowdisplaybreaks
\newtheorem{assumption}{Assumption}
\newtheorem{baitoan}{}
\newtheorem{cauhoi}{Câu hỏi}
\newtheorem{conjecture}{Conjecture}
\newtheorem{corollary}{Corollary}
\newtheorem{dangtoan}{Dạng toán}
\newtheorem{definition}{Definition}
\newtheorem{dinhly}{Định lý}
\newtheorem{dinhnghia}{Định nghĩa}
\newtheorem{example}{Example}
\newtheorem{ghichu}{Ghi chú}
\newtheorem{hequa}{Hệ quả}
\newtheorem{hypothesis}{Hypothesis}
\newtheorem{lemma}{Lemma}
\newtheorem{luuy}{Lưu ý}
\newtheorem{nhanxet}{Nhận xét}
\newtheorem{notation}{Notation}
\newtheorem{note}{Note}
\newtheorem{principle}{Principle}
\newtheorem{problem}{Problem}
\newtheorem{proposition}{Proposition}
\newtheorem{question}{Question}
\newtheorem{remark}{Remark}
\newtheorem{theorem}{Theorem}
\newtheorem{vidu}{Ví dụ}
\usepackage[left=1cm,right=1cm,top=5mm,bottom=5mm,footskip=4mm]{geometry}
\def\labelitemii{$\circ$}
\DeclareRobustCommand{\divby}{%
	\mathrel{\vbox{\baselineskip.65ex\lineskiplimit0pt\hbox{.}\hbox{.}\hbox{.}}}%
}
\setlist[itemize]{leftmargin=*}
\setlist[enumerate]{leftmargin=*}

\title{On Teaching Rules {\it\&} Principles\\Bàn Về Các Quy Tắc {\it\&} Nguyên Tắc Giảng Dạy}
\author{Nguyễn Quản Bá Hồng\footnote{A scientist- {\it\&} creative artist wannabe, a mathematics {\it\&} computer science lecturer of Department of Artificial Intelligence {\it\&} Data Science (AIDS), School of Technology (SOT), UMT Trường Đại học Quản lý {\it\&} Công nghệ TP.HCM, Hồ Chí Minh City, Việt Nam.\\E-mail: {\sf nguyenquanbahong@gmail.com} {\it\&} {\sf hong.nguyenquanba@umt.edu.vn}. Website: \url{https://nqbh.github.io/}. GitHub: \url{https://github.com/NQBH}.}}
\date{\today}

\begin{document}
\maketitle
\begin{abstract}
	This text is a part of the series {\it Some Topics in Advanced STEM \& Beyond}:
	
	{\sc url}: \url{https://nqbh.github.io/advanced_STEM/}.
	
	Latest version:
	\begin{itemize}
		\item {\it On Teaching Rules \& Principles -- Bàn Về Các Quy Tắc \& Nguyên Tắc Giảng Dạy}.
		
		PDF: {\sc url}: \url{https://github.com/NQBH/advanced_STEM_beyond/blob/main/teach/rule_principle/NQBH_teaching_rule_principle.pdf}.
		
		\TeX: {\sc url}: \url{https://github.com/NQBH/advanced_STEM_beyond/blob/main/teach/rule_principle/NQBH_teaching_rule_principle.tex}.
		\item {\it Advanced STEM students}.
		
		PDF: {\sc url}: \url{https://github.com/NQBH/advanced_STEM_beyond/blob/main/teach/student/NQBH_student.pdf}.
		
		\TeX: {\sc url}: \url{https://github.com/NQBH/advanced_STEM_beyond/blob/main/teach/student/NQBH_student.tex}.
	\end{itemize}
\end{abstract}
\tableofcontents

%------------------------------------------------------------------------------%

\section{Rules -- Các Nguyên Tắc}

\begin{itemize}
	\item Phân loại sinh viên \& loại bài tập:
	\begin{itemize}
		\item Sinh viên thuộc đội tuyển Olympic Toán or Olympic Tin or both.
		\item Sinh viên không thuộc bất cứ đội tuyển Olympic nào.
	\end{itemize}
	\item Bất cứ sinh viên nào đến lớp trước giảng viên \& đúng giờ sẽ được cộng \verb|early_attendence_point| with default value 0.25. Có tất cả 20 buổi học gồm lớp Lý Thuyết \& lớp Thực hành, nên tối đa sẽ được $0.25\cdot20 = 5$.
	\item Bất cứ sinh viên nào không nằm trong đội tuyển Olympic Toán or Olympic Tin or both nhưng làm được bài tập được đánh dấu cho đội tuyển Olympic sẽ được nhân 2 điểm nếu phản biện thành công (câu hỏi phản biện được các bạn khác \& giảng viên đưa ra).
	\item Chỉ khi nào không có ai trong các bạn ngoài đội tuyển giải được các bài được đánh dấu là không Olympic, thì các bạn đội tuyển mới được giải các bài đó \& nhận nửa số điểm.
	\item Ai làm được bài tập ngay trên lớp sẽ được + điểm nhiều hơn cho bài tập đó so với đem về nhà làm rồi nộp. Các bài tập giao về nhà thường khó hơn \& quy mô lớn hơn, phức tạp hơn.
	\item Summary:
	
	Python code: \url{https://github.com/NQBH/advanced_STEM_beyond/blob/main/teach/rule_principle/grade.py}.
	
	Input sample: \url{https://github.com/NQBH/advanced_STEM_beyond/blob/main/teach/rule_principle/grade.inp}.
	\begin{verbatim}
		# default parameters
		absence_point = -1
		late_point = -0.25 # -0.25/30 mins
		
		# parameters depending on each subject
		number_student = int(input())
		number_teaching_day = int(input()) # 20 days for both theory & practical classes
		early_attendence_point = 0.25 # go to class before lecturer & on time
		
		# grading
		while (number_student):
		    number_student -= 1
		    number_early_attendence = int(input()) # max: number_teaching_day
		    number_absence_day = int(input()) # max: number_teaching_day
		    total_late_time = float(input()) # in mins
		    total_plus_point = float(input()) # in 2-decimal accuracy
		    
		    # midterm exam grade 15%
		    midterm_exam_grade = float(input()) # last column in 3 columns 45%
		    total_early_attendence_point = early_attendence_point * number_early_attendence
		    print(total_early_attendence_point)
		
		    # study grade 30%
		    study_grade = total_plus_point + total_early_attendence_point - number_absence_day * absence_point
		                  - total_late_time * late_point / 30
		
		    # final exam grade 55%
		    final_exam_grade = float(input())
		
		    # final grade
		    final_grade = min(study_grade, 10) * 0.3 + midterm_exam_grade * 0.15 + final_exam_grade * 0.55
		    print(final_grade)
	\end{verbatim}
\end{itemize}

%------------------------------------------------------------------------------%

\section{Principles -- Các Nguyên Lý}

\begin{itemize}
	\item Tôn trọng sự tự do \& sáng tạo.
	\item Chỉ nên định hướng nghiên cứu chỉ khi sinh viên tỏ ra quan tâm tới bài toán hoặc bài giảng sâu hơn. Khuyến khích nhưng không thúc ép các sinh viên không có tiềm năng nghiên cứu hay thể hiện rõ được năng lực nghiên cứu. That is a form of torture in teaching \& learning.
\end{itemize}

%------------------------------------------------------------------------------%

\section{Miscellaneous}

\begin{itemize}
	\item Các bài toán mang nhãn [R] có nghĩa là bài toán định hướng nghiên cứu, thường sẽ khó hơn \&{\tt/}hoặc mang tính cấu trúc liên kết hơn các bài toán không có nhãn [R].
	
	Distinguish this [R]-Research label with [R]--Restricted in movie: {\it Under 17 requires accompanying parent or adult guardian}. Contains some adult material. Parents are urged to learn more about the film before taking their young children with them. See, e.g., \href{https://en.wikipedia.org/wiki/Motion_Picture_Association_film_rating_system}{Wikipedia{\tt/} Motion Picture Association film rating system}.
\end{itemize}

%------------------------------------------------------------------------------%

\printbibliography[heading=bibintoc]
	
\end{document}