\documentclass{article}
\usepackage[backend=biber,natbib=true,style=alphabetic,maxbibnames=50]{biblatex}
\addbibresource{/home/nqbh/reference/bib.bib}
\usepackage[utf8]{vietnam}
\usepackage{tocloft}
\renewcommand{\cftsecleader}{\cftdotfill{\cftdotsep}}
\usepackage[colorlinks=true,linkcolor=blue,urlcolor=red,citecolor=magenta]{hyperref}
\usepackage{amsmath,amssymb,amsthm,enumitem,float,graphicx,mathtools,tikz}
\usetikzlibrary{angles,calc,intersections,matrix,patterns,quotes,shadings}
\allowdisplaybreaks
\newtheorem{assumption}{Assumption}
\newtheorem{baitoan}{}
\newtheorem{cauhoi}{Câu hỏi}
\newtheorem{conjecture}{Conjecture}
\newtheorem{corollary}{Corollary}
\newtheorem{dangtoan}{Dạng toán}
\newtheorem{definition}{Definition}
\newtheorem{dinhly}{Định lý}
\newtheorem{dinhnghia}{Định nghĩa}
\newtheorem{example}{Example}
\newtheorem{ghichu}{Ghi chú}
\newtheorem{hequa}{Hệ quả}
\newtheorem{hypothesis}{Hypothesis}
\newtheorem{lemma}{Lemma}
\newtheorem{luuy}{Lưu ý}
\newtheorem{nhanxet}{Nhận xét}
\newtheorem{notation}{Notation}
\newtheorem{note}{Note}
\newtheorem{principle}{Principle}
\newtheorem{problem}{Problem}
\newtheorem{proposition}{Proposition}
\newtheorem{question}{Question}
\newtheorem{remark}{Remark}
\newtheorem{theorem}{Theorem}
\newtheorem{vidu}{Ví dụ}
\usepackage[left=1cm,right=1cm,top=5mm,bottom=5mm,footskip=4mm]{geometry}
\def\labelitemii{$\circ$}
\DeclareRobustCommand{\divby}{%
	\mathrel{\vbox{\baselineskip.65ex\lineskiplimit0pt\hbox{.}\hbox{.}\hbox{.}}}%
}
\setlist[itemize]{leftmargin=*}
\setlist[enumerate]{leftmargin=*}

\title{Mathematics -- Toán Học}
\author{Nguyễn Quản Bá Hồng\footnote{A Scientist {\it\&} Creative Artist Wannabe. E-mail: {\tt nguyenquanbahong@gmail.com}. Bến Tre City, Việt Nam.}}
\date{\today}

\begin{document}
\maketitle
\begin{abstract}
	This text is a part of the series {\it Some Topics in Advanced STEM \& Beyond}:
	
	{\sc url}: \url{https://nqbh.github.io/advanced_STEM/}.
	
	Latest version:
	\begin{itemize}
		\item {\it Mathematics -- Toán Học}.
		
		PDF: {\sc url}: \url{.pdf}.
		
		\TeX: {\sc url}: \url{.tex}.
	\end{itemize}
\end{abstract}
\tableofcontents

%------------------------------------------------------------------------------%

\section{Wikipedia}

\subsection{Wikipedia{\tt/}pure mathematics}
``{\it Pure mathematics} is the study of mathematical concepts independently of any application outside mathematics. These concepts may originate in real-word concerns, \& the results obtained may later turn out to be useful for practical applications, but pure mathematicians are not primarily motivated by such applications. Instead, the appeal is attributed to the intellectual challenge \& \href{https://en.wikipedia.org/wiki/Mathematical_beauty}{aesthetic beauty} of working out the logical consequences of basic principles.

While pure mathematics has existed as an activity since at least \href{https://en.wikipedia.org/wiki/Ancient_Greece}{ancient Greece}, the concept was elaborated upon around the year 1900, after the introduction of theries with counter-intuitive properties (e.g., \href{https://en.wikipedia.org/wiki/Non-Euclidean_geometries}{non-Euclidean geometries} \& \href{https://en.wikipedia.org/wiki/Georg_Cantor}{\sc Cantor}'s theory of infinite sets), \& the discovery of apparent paradoxes (e.g., \href{https://en.wikipedia.org/wiki/Continuous_function}{continuous functions} that are nowhere differentiable, \& \href{https://en.wikipedia.org/wiki/Russell%27s_paradox}{Russell's paradox}). This introduced the need to renew the concept of \href{https://en.wikipedia.org/wiki/Mathematical_rigor}{mathematical rigor} \& rewrite all mathematics accordingly, with a systematic use of \href{https://en.wikipedia.org/wiki/Axiomatic_method}{axiomatic methods}. This led many mathematicians to focus on mathematics for its own sake, i.e., pure mathematics.


'' -- \href{https://en.wikipedia.org/wiki/Pure_mathematics}{Wikipedia{\tt/}pure mathematics}

%------------------------------------------------------------------------------%

\section{Miscellaneous}

%------------------------------------------------------------------------------%

\printbibliography[heading=bibintoc]
	
\end{document}