\documentclass{article}
\usepackage[backend=biber,natbib=true,style=alphabetic,maxbibnames=50]{biblatex}
\addbibresource{/home/nqbh/reference/bib.bib}
\usepackage[utf8]{vietnam}
\usepackage{tocloft}
\renewcommand{\cftsecleader}{\cftdotfill{\cftdotsep}}
\usepackage[colorlinks=true,linkcolor=blue,urlcolor=red,citecolor=magenta]{hyperref}
\usepackage{amsmath,amssymb,amsthm,float,graphicx,mathtools,tikz}
\usetikzlibrary{angles,calc,intersections,matrix,patterns,quotes,shadings}
\allowdisplaybreaks
\newtheorem{assumption}{Assumption}
\newtheorem{baitoan}{}
\newtheorem{cauhoi}{Câu hỏi}
\newtheorem{conjecture}{Conjecture}
\newtheorem{corollary}{Corollary}
\newtheorem{dangtoan}{Dạng toán}
\newtheorem{definition}{Definition}
\newtheorem{dinhly}{Định lý}
\newtheorem{dinhnghia}{Định nghĩa}
\newtheorem{example}{Example}
\newtheorem{ghichu}{Ghi chú}
\newtheorem{hequa}{Hệ quả}
\newtheorem{hypothesis}{Hypothesis}
\newtheorem{lemma}{Lemma}
\newtheorem{luuy}{Lưu ý}
\newtheorem{nhanxet}{Nhận xét}
\newtheorem{notation}{Notation}
\newtheorem{note}{Note}
\newtheorem{principle}{Principle}
\newtheorem{problem}{Problem}
\newtheorem{proposition}{Proposition}
\newtheorem{question}{Question}
\newtheorem{remark}{Remark}
\newtheorem{theorem}{Theorem}
\newtheorem{vidu}{Ví dụ}
\usepackage[left=1cm,right=1cm,top=5mm,bottom=5mm,footskip=4mm]{geometry}
\def\labelitemii{$\circ$}
\DeclareRobustCommand{\divby}{%
	\mathrel{\vbox{\baselineskip.65ex\lineskiplimit0pt\hbox{.}\hbox{.}\hbox{.}}}%
}

\title{Mathematical Analysis \& Numerical Analysis -- Giải Tích Toán Học \& Giải Tích Số}
\author{Nguyễn Quản Bá Hồng\footnote{A Scientist {\it\&} Creative Artist Wannabe. E-mail: {\tt nguyenquanbahong@gmail.com}. Bến Tre City, Việt Nam.}}
\date{\today}

\begin{document}
\maketitle
\begin{abstract}
	This text is a part of the series {\it Some Topics in Advanced STEM \& Beyond}:
	
	{\sc url}: \url{https://nqbh.github.io/advanced_STEM/}.
	
	Latest version:
	\begin{itemize}
		\item {\it Mathematical Analysis -- Giải Tích Toán Học}.
		
		PDF: {\sc url}: \url{https://github.com/NQBH/advanced_STEM_beyond/blob/main/analysis/NQBH_mathematical_analysis.pdf}.
		
		\TeX: {\sc url}: \url{https://github.com/NQBH/advanced_STEM_beyond/blob/main/analysis/NQBH_mathematical_analysis.tex}.
	\end{itemize}
\end{abstract}
\tableofcontents

%------------------------------------------------------------------------------%

\section{Basic}
Tôi được giải Nhì Giải tích Olympic Toán Sinh viên 2014 (VMC2014) khi còn học năm nhất Đại học \& được giải Nhất Giải tích Olympic Toán Sinh viên 2015 (VMC2015) khi học năm 2 Đại học. Nhưng điều đó không có nghĩa là tôi giỏi Giải tích. Bằng chứng là 10 năm sau khi nhận các giải đó, tôi đang tự học lại Giải tích Toán học với hy vọng có 1 hay nhiều cách nhìn mới mẻ hơn \& mang tính ứng dụng hơn cho các đề tài cá nhân của tôi.

\noindent\textbf{\textsf{Resources -- Tài nguyên.}}
\begin{itemize}
	\item \cite{Rudin1976}. {\sc Walter Rudin}. {\it Principles Principles of Mathematical Analysis}.
	\item \cite{Rudin1973,Rudin1987}. {\sc Walter Rudin}. {\it Real \& Complex Analysis}.
	\item \cite{Rudin1991}. {\sc Walter Rudin}. {\it Functional Analysis}.
	\item \cite{Lieb_Loss2001}. {\sc Elliott Lieb, Michael Loss}. {\it Analysis}.
	\item \cite{Evans_Gariepy2015}. {\sc Lawrence C. Evans, Ronald F. Gariepy}. {\it Measure Theory \& Fine Properties of Functions}.
\end{itemize}
``Analysis is the art of taking limits, \& the constraint of having to deal with an integration theory that does not allow taking limits is much like having to do mathematics only with rational numbers \& excluding the irrational ones.'' -- \cite[Chap. 1, p. 1]{Lieb_Loss2001}

\section{Measure \& Integration -- Độ Đo \& Tích Phân}
The point of view of integration defined as a Riemann integral may be historically grounded \& useful in many areas of mathematics but is far from being adequate for the requirements of modern analysis since Riemann integral can be defined only for a special class of functions \& this class is not closed under the process of taking pointwise limits of sequence (not even monotonic sequences) of functions in this class.

``The useful \& far-reaching idea of Lebesgue \& others was to compute the $(n + 1)$-dimensional volume `in the other direction' by 1st computing the $n$-dimensional volume of the set where the function $> y$. This volume is a well-behaved, monotone nonincreasing function of $y$, which then can be integrated in the manner of Riemann. This method of integration not only works for a large class of functions (which is closed under taking pointwise limits), but it also greatly simplifies a problem that used to plague analysts: {\it Is it permissible to exchange limits \& integration?}'' -- \cite[Chap. 1, pp. 1--2]{Lieb_Loss2001}

Lebesgue integration theory is 1 of the great triumphs of 20th century mathematics \& is the culmination of a long struggle to find the right perspective from which to view integration theory.

%------------------------------------------------------------------------------%

\section{Partial Differential Equations (PDEs) -- Phương Trình Vi Phân Đạo Hàm Riêng}
\noindent\textbf{\textsf{Resources -- Tài nguyên.}}
\begin{itemize}
	\item \cite{Brezis2011}. {\sc Ha\"im Brezis}. {\it Functional Analysis, Sobolev Spaces, \& Partial Differential Equations}.
	\item \cite{Evans2010}. {\sc Lawrence C. Evans}. {\it Partial Differential Equations}.
	\item \cite{Gilbarg_Trudinger2001}. {\sc David Gilbarg, Neil S. Trudinger}. {\it Elliptic Partial Differential Equations of 2nd Order}.
\end{itemize}

%------------------------------------------------------------------------------%

\subsection{Weak solution -- Nghiệm yếu}

\begin{definition}[Weak solution -- Nghiệm yếu]
	``In mathematics, a \emph{weak solution} (also called a \emph{generalized solution}) to an ODE or PDE is a function for which the derivatives may not all exist but which is nonetheless deemed to satisfy the equation in some precisely defined sense. There are many different definitions of weak solution, appropriate for different classes of equations. 1 of the most important is based on the notion of \href{https://en.wikipedia.org/wiki/Distribution_(mathematics)}{distributions}.'' -- \href{https://en.wikipedia.org/wiki/Weak_solution}{Wikipedia{\tt/}weak solution}
\end{definition}
``Avoiding the language of distributions, one starts with a differential equation \& rewrites it in such a way that no derivatives of the solution of the equation show up (the new form is called the \href{https://en.wikipedia.org/wiki/Weak_formulation}{weak formulation}, \& the solutions to it are called {\it weak solutions}). Somewhat surprisingly, a differential equation may have solutions which are not differentiable; \& the weak formulation allows one to find such solutions.

Weak solutions are important because many differential equations encountered in modeling real-world phenomena do not admit of sufficiently smooth solutions, \& the only way of solving such equations is using the weak formulation. Even in situations where an equation does have differentiable solutions, it is often convenient to 1st prove the existence of weak solutions \& only alter show that those solutions are in fact smooth enough.'' -- \href{https://en.wikipedia.org/wiki/Weak_solution}{Wikipedia{\tt/}weak solution}

\begin{example}[1st-order wave equation]
	The 1st-order \href{https://en.wikipedia.org/wiki/Wave_equation}{wave equation} $\partial_tu + \partial_xu = 0$ in $\mathbb{R}^2$ with $u = u(t,x)$ has the weak form $\int_{\mathbb{R}^2} u\partial_t\varphi + u\partial_x\varphi\,{\rm d}t\,{\rm d}x = 0$ has a solution $u(t,x) = |t - x|$ which may be checked by splitting the integrals over region $\{x\ge t\}$ \& $\{x\le t\}$ where $u$ is smooth.
\end{example}
``The notion of weak solution based on distribution is sometimes inadequate. In the case of \href{https://en.wikipedia.org/wiki/Hyperbolic_system}{hyperbolic systems}, the notion of weak solution based on distributions does not guarantee uniqueness, \& it is necessary to supplement it with {\it entropy conditions} or some other selection criterion. In fully nonlinear PDE e.g. \href{https://en.wikipedia.org/wiki/Hamilton%E2%80%93Jacobi_equation}{Hamilton-Jacobi equation}, there is a very different definition of weak solution called \href{https://en.wikipedia.org/wiki/Viscosity_solution}{\it viscosity solution}.'' -- \href{https://en.wikipedia.org/wiki/Weak_solution}{Wikipedia{\tt/}weak solution}

\subsubsection{General idea}
When solving a differential equation in $u$, one can rewrite it using a \href{https://en.wikipedia.org/wiki/Test_function}{test function} $\varphi$ s.t. whatever derivatives in $u$ show up in the equation, they are ``transferred'' via integration by parts to $\varphi$, resulting in an equation without derivatives of $u$. This new equation generalizes the original equation to include solutions which are not necessarily differentiable. The approach illustrated above works in great generality. Consider a linear differential operator in an open set $W\subset\mathbb{R}^d$:
\begin{equation}
	P({\bf x},\partial)u({\bf x}) = \sum a_\alpha({\bf x})\partial^\alpha u({\bf x}),
\end{equation}
where the multi-index $\alpha = (\alpha_1,\ldots,\alpha_d)$ varies over some finite set in $\mathbb{N}^d$ and the coefficients $a_\alpha$ are smooth enough functions of ${\bf x}\in\mathbb{R}^d$. The differential equation $P({\bf x},\partial)u({\bf x} = 0$ can, after being multiplied by a smooth test function $\varphi\in C_c^\infty(W)$ \& integrated by parts, be written as
\begin{equation}
	\int_W u({\bf x})Q({\bf x},\partial)\varphi({\bf x})\,{\rm d}{\bf x} = 0,
\end{equation}
where the differential operator $Q({\bf x},\partial)$ is given by the formula
\begin{equation}
	Q({\bf x},\partial)\varphi({\bf x}) = \sum (-1)^{|\alpha|}\partial^\alpha[a_\alpha({\bf x})\varphi({\bf x})],
\end{equation}
which is the \href{https://en.wikipedia.org/wiki/Formal_adjoint}{formal adjoint} of $P({\bf x},\partial)$.

In summary, if the original (strong) problem was to find a $|\alpha|$-times differentiable function $u$ defined on the open set $W$ s.t. $P({\bf x},\partial)u({\bf x}) = 0$, $\forall{\bf x}\in W$ (a so-called {\it strong solution}), then an integrable function $u$ would be said to be a {\it weak solution} if $\int_W u({\bf x})Q({\bf x},\partial)\varphi({\bf x})\,{\rm d}{\bf x} = 0$, $\forall\varphi\in C_c^\infty(W)$.

%------------------------------------------------------------------------------%

\subsection{Viscosity solution -- Nghiệm trơn{\tt/}nhớt}

\begin{example}[Viscosity solution for Hamilton--Jacobi equation]
	Hamilton--Jacobi equation.
\end{example}

%------------------------------------------------------------------------------%

\subsection{Very weak solution -- Nghiệm rất yếu}

\begin{example}[Very weak solution of porous medium equation (PME)]
	\cite{Vazquez2007}.
\end{example}

\begin{example}[Very weak solution of multi-dimensional slow diffusion equations with a singular quenching term]
	\cite{Dao_Diaz_Nguyen2020}. Given $f\in L_\delta^1(\Omega),\lambda\ge0$, a function $u\in L_\delta^1(\Omega)$ is called a \emph{very weak solution} of
	\begin{equation}
		\left\{\begin{split}
			-\Delta(|u|^{m-1}u) + \lambda u &= f&&\mbox{in }\Omega,\\
			|u|^{m-1}u &= 1&&\mbox{on }\Gamma,
		\end{split}\right.
	\end{equation}
	if $|u|^{m-1}u\in L^1(\Omega)$ and
	\begin{equation}
		\int_\Omega u^m\Delta\varphi + \lambda u\varphi\,{\rm d}{\bf x} = \int_\Omega f\varphi\,{\rm d}{\bf x} - \int_\Gamma \partial_{\bf n}\varphi\,{\rm d}{\bf x}.
	\end{equation}
\end{example}

\begin{example}[Very weak solution of NSEs]
	\cite{Tsai2018}.
\end{example}

%------------------------------------------------------------------------------%

\section{Differential Geometry -- Hình Học Vi Phân}
\noindent\textbf{\textsf{Resources -- Tài nguyên.}}
\begin{itemize}
	\item \cite{Walker2015}. {\sc Shawn W. Walker}. {\it The Shapes of Things}.
	\item \cite{Carmo2016}. {\sc Manfredo P. do Carmo}. {\it Differential Geometry of Curves \& Surfaces}.
	\item \cite{Kuhnel2015}. {\sc Wolfgang K\"uhnel}. {\it Differential Geometry}.
\end{itemize}
1 trong những ứng dụng của Hình Học Vi Phân là {\it Shape Calculus \& Tangential Calculus -- Phép Tính Vi Tích Phân cho Tối Ưu Hình Dáng \& Phép Tính Vi Tích Phân Trên Mặt Phẳng Tiếp Tuyến}.

%------------------------------------------------------------------------------%

\section{Miscellaneous}

%------------------------------------------------------------------------------%

\printbibliography[heading=bibintoc]
	
\end{document}