\documentclass{article}
\usepackage[backend=biber,natbib=true,style=alphabetic,maxbibnames=50]{biblatex}
\addbibresource{/home/nqbh/reference/bib.bib}
\usepackage[utf8]{vietnam}
\usepackage{tocloft}
\renewcommand{\cftsecleader}{\cftdotfill{\cftdotsep}}
\usepackage[colorlinks=true,linkcolor=blue,urlcolor=red,citecolor=magenta]{hyperref}
\usepackage{amsmath,amssymb,amsthm,enumitem,float,graphicx,mathtools,tikz}
\usetikzlibrary{angles,calc,intersections,matrix,patterns,quotes,shadings}
\allowdisplaybreaks
\newtheorem{assumption}{Assumption}
\newtheorem{baitoan}{}
\newtheorem{cauhoi}{Câu hỏi}
\newtheorem{conjecture}{Conjecture}
\newtheorem{corollary}{Corollary}
\newtheorem{dangtoan}{Dạng toán}
\newtheorem{definition}{Definition}
\newtheorem{dinhly}{Định lý}
\newtheorem{dinhnghia}{Định nghĩa}
\newtheorem{example}{Example}
\newtheorem{ghichu}{Ghi chú}
\newtheorem{hequa}{Hệ quả}
\newtheorem{hypothesis}{Hypothesis}
\newtheorem{lemma}{Lemma}
\newtheorem{luuy}{Lưu ý}
\newtheorem{nhanxet}{Nhận xét}
\newtheorem{notation}{Notation}
\newtheorem{note}{Note}
\newtheorem{principle}{Principle}
\newtheorem{problem}{Problem}
\newtheorem{proposition}{Proposition}
\newtheorem{question}{Question}
\newtheorem{remark}{Remark}
\newtheorem{theorem}{Theorem}
\newtheorem{vidu}{Ví dụ}
\usepackage[left=1cm,right=1cm,top=5mm,bottom=5mm,footskip=4mm]{geometry}
\def\labelitemii{$\circ$}
\DeclareRobustCommand{\divby}{%
	\mathrel{\vbox{\baselineskip.65ex\lineskiplimit0pt\hbox{.}\hbox{.}\hbox{.}}}%
}
\setlist[itemize]{leftmargin=*}
\setlist[enumerate]{leftmargin=*}

\title{Mathematical Optimization -- Toán Tối Ưu}
\author{Nguyễn Quản Bá Hồng\footnote{A Scientist {\it\&} Creative Artist Wannabe. E-mail: {\tt nguyenquanbahong@gmail.com}. Bến Tre City, Việt Nam.}}
\date{\today}

\begin{document}
\maketitle
\begin{abstract}
	This text is a part of the series {\it Some Topics in Advanced STEM \& Beyond}:
	
	{\sc url}: \url{https://nqbh.github.io/advanced_STEM/}.
	
	Latest version:
	\begin{itemize}
		\item {\it Mathematical Optimization -- Toán Tối Ưu}.
		
		PDF: {\sc url}: \url{https://github.com/NQBH/advanced_STEM_beyond/blob/main/optimization/NQBH_mathematical_optimization.pdf}.
		
		\TeX: {\sc url}: \url{https://github.com/NQBH/advanced_STEM_beyond/blob/main/optimization/NQBH_mathematical_optimization.tex}.
	\end{itemize}
\end{abstract}
\tableofcontents

%------------------------------------------------------------------------------%

\section{Basic}

%------------------------------------------------------------------------------%

\section{Optimal Control -- Điều Khiển Tối Ưu}

%------------------------------------------------------------------------------%

\section{Shape Optimization -- Tối Ưu Hình Dạng}
\textbf{\textsf{Resources -- Tài nguyên.}}
\begin{enumerate}
	\item \cite{Allaire_Henrot2001}. {\sc Gr\'egoire Allaire, Antoine Henrot}. {\it On some recent advances in shape optimization}.
	\item \cite{Azegami2020}. {\sc Hideyuki Azegami}. {\it Shape Optimization Problems}.
	\item \cite{Bandle_Wagner2023}. {\sc Catherine Bandle, Alfred Wagner}. {\it Shape Optimization: Variations of Domains \& Applications}.
	\item \cite{Delfour_Zolesio2001,Delfour_Zolesio2011}. {\sc Michael C. Delfour, Jean-Paul Zol\'{e}sio}. {\it Shapes \& Geometries}.
	\item \cite{Haslinger_Makinen2003}. {\sc J. Haslinger, R. A. E. M\"{a}kinen}. {\it Introduction to Shape Optimization}.
	\item \cite{Mohammadi_Pironneau2010}. {\sc Bijan Mohammadi, Olivier Pironneau}. {\it Applied Shape Optimization for Fluids}.
	\item \cite{Moubachir_Zolesio2006}. {\sc Marwan Moubachir, Jean-Paul Zol\'{e}sio}. {\it Moving Shape Analysis \& Control}.
	\item {\sc Stephan Schmidt}. Master course: {\it Shape \& Geometry}. Humboldt University of Berlin. [written in German, taught in English \& German].
	\item \cite{Sokolowski_Zolesio1992}. {\sc Jan Soko\l owski, Jean-Paul Zol\'{e}sio}. {\it Introduction to Shape Optimization}.
	\item \cite{Walker2015}. {\sc Shawn W. Walker}. {\it The Shapes of Things}.
	
	{\bf Differential equations on surfaces.} Differential geometry is useful for understanding mathematical models containing geometric PDEs, e.g., surface{\tt/}manifold version of the standard Laplace equation, which requires the development of the surface gradient \& surface Laplacian operators -- the usual gradient $\nabla$ \& Laplacian $\Delta = \nabla\cdot\nabla$ operators defined on a surface (manifold) instead of standard Euclidean space $\mathbb{R}^n$. {\it Advantage}: provide alternative formulas for geometric quantities, e.g., the summed (mean) curvature, that are much clearer than the usual presentation of texts on differential geometry.
	
	{\bf Differentiating w.r.t. Shape.} The approach to differential geometry is advantageous for developing the framework of {\it shape differential calculus} -- the study of how quantities change w.r.t. changes of independent ``shape variable''.
	
	``The framework of shape differential calculus provides the tools for developing the equations of mean curvature flow \& Willmore flow, which are geometric flows occurring in many applications such as fluid dynamics \& biology.'' -- \cite[p. 2]{Walker2015}
	
	The shape perturbation $\delta J(\Omega;V)$ is similar to the gradient operator, which is a directional derivative, analogous to $V\cdot\nabla f$ where $V$ is a given direction, providing information about the local slope, or the sensitivity of a quantity w.r.t. some parameters.
	
	It takes only 2 or 3 numbers to specify a point $(x,y)$ in 2D \& a point $(x,y,z)$ in 3D, whereas an ``infinite'' number of coordinate pairs is needed to specify a domain $\Omega$. $V$ is a 2D{\tt/}3D vector in the scalar function setting; for a shape functional, $V$ is a full-blown function requiring definition at every point in $\Omega$. This ``infinite dimensionality'' is the reason for using the notation $\delta J(\Omega;V)$ to denote a shape perturbation. $\delta J(\Omega;V)$ indicates how we should change $\Omega$ to decrease $J$, similarly to how $\nabla f(x,y)$ indicates how the coordinate pair $(x,y)$ should change to decrease $f$, which opens up the world of shape optimization.
	
	{\bf3 schools of shape optimization.} Cf. engineering shape optimization vs. applied shape optimization \cite{Mohammadi_Pironneau2010} vs. theoretical shape optimization \cite{Sokolowski_Zolesio1992,Delfour_Zolesio2011}.
	
	Shape perturbations allow us to ``climb down the hill'' in the infinite dimensional setting of shape, which is a powerful tool for producing sophisticated engineering designs in an automatic way.
	
	{\bf Extrinsic vs. intrinsic point of views.} To make the discussion as clear as possible, we adopt the {\it extrinsic} point of view: curves \& surfaces are assumed to lie in a Euclidean space of higher dimension. The ambient space is 3D Euclidean space. Alternatively, there is the {\it intrinsic} point of view, i.e., the surface is not assumed to lie in an ambient space, i.e., one is not allowed to reference anything ``outside'' of the surface when defining it. Moreover, no mathematical structures ``outside'' of the surface can be utilized. Walker \cite{Walker2015} did not adopt the intrinsic view or consider higher dimensional manifolds, general embedding dimensions, etc. for the reasons: \cite{Walker2015} is meant as a {\it practical guide} to differential geometry \& shape differentiation that can be used by researchers in other fields.
	\begin{itemize}
		\item \cite{Walker2015} is meant to be used as background information for deriving physical models where geometry plays a critical role. Because most physical problems of interest take place in 3D Euclidean space, the extrinsic viewpoint is sufficient.
		\item Many of the proofs \& derivations of differential geometry relations simplify dramatically for 2D surfaces in 3D \& require only basic multivariable calculus \& linear algebra.
		\item The concepts of {\it normal vectors} \& {\it curvature} are harder to motivate with the intrinsic viewpoint. {\it What does it meant for a surface to ``curve through space'' if you cannot talk about the ambient space?}
		\item Walker wants to keep in mind applications of this machinery to geometric PDEs, fluid dynamics, numerical analysis, optimization, etc. An interesting application of this methodology is for the development of numerical methods for mean curvature flow \& surface tension driven fluid flow. Ergo ($=$ therefore), the extrinsic viewpoint is often more convenient for computational purposes.
		\item Walker wants his framework to be useful for analyzing \& solving {\it shape optimization} problems, i.e., optimization problems where geometry{\tt/}shape is the control variable.
	\end{itemize}
\end{enumerate}

\begin{example}[\cite{Walker2015}, Sect. 1.2.1, pp. 1--2]
	Let $f = f(r,\theta)$ be a smooth function defined on the disk $B_{R,2}(0,0)$ of radius $R$ in terms of polar coordinates. The integral of $f$ over $B_{2,R}(0,0)$ $J\coloneqq\int_{B_{2,R}(0,0)} f\,{\rm d}{\bf x} = \int_0^{2\pi}\int_0^R f(r,\theta)\,{\rm d}r\,{\rm d}\theta$ depends on $R$. Assume $f$ also depends on $R$, i.e., $f = f(r,\theta,R)$ with a physical example: $J$ is the \emph{net flow rate} of liquid through a pipe with cross-section $\Omega$, then $f$ is the flow rate per unit area \& could be the solution of a PDE defined on $\Omega$, e.g., a Navier--Stokes fluid flowing in a circular pipe. Advantageous to know the \emph{sensitivity} of $J$ w.r.t. $R$, e.g., for optimization purposes. Differentiate $J$ w.r.t. $R$:
	\begin{equation*}
		\frac{d}{dR}J = \int_0^{2\pi}\left(\frac{d}{dR}\int_0^R f(r,\theta;R)r\,{\rm d}r\right){\rm d}\theta = \int_0^{2\pi}\int_0^R f'(r,\theta;R)r\,{\rm d}r\,{\rm d}\theta + \int_0^{2\pi} f(R,\theta;R)\,{\rm d}\theta.
	\end{equation*}
	The dependence of $f$ on $R$ can more generally be viewed as dependence on $B_{R,2}(0,0)$, i.e., $f(\cdot;R)\equiv f(\cdot;B_{R,2}(0,0))$. Rewriting $d/dR J$ using Cartesian coordinates ${\bf x}$:
	\begin{equation}
		\frac{d}{dR}J = \int_{B_{R,2}(0,0)} f'({\bf x};\Omega)\,{\rm d}{\bf x} + \int_{S_{R,2}(0,0)} f({\bf x};\Omega)\,{\rm d}S({\bf x}),
	\end{equation}
	where ${\rm d}{\bf x}$ is the volume measure, ${\rm d}S({\bf x})$ is the surface area measure.
\end{example}

\begin{example}[Surface height function of a hill]
	Let $f = f(x,y)$ be a function describing the surface height of the hill, where $(x,y)$ are the coordinates of our position. Then, by using basic multivariate calculus, finding a direction that will move us downhill is equivalent to computing the gradient (vector) of $f$ \& moving in the opposite direction to the gradient. In this sense, we do not need to ``see'' the whole function. We just need to \emph{locally} compute the gradient $\nabla f$, analogous to feeling the ground beneath.
\end{example}

\begin{example}[Engineering shape optimization: minimizing drag with Navier--Stokes flow of fluid past a rigid body, \cite{Walker2015}, pp. 3--5]
	A shape functional representing the drag:
	\begin{equation}
		J_{\rm d}(\Omega)\coloneqq-{\bf u}_{\rm out}\cdot\int_{\Gamma_0} \boldsymbol{\sigma}({\bf u},p){\bf n}\,{\rm d}\Gamma = \frac{2}{{\rm Re}}\int_\Omega |\boldsymbol{\varepsilon}({\bf u})|^2\,{\rm d}{\bf x}\ge0,
	\end{equation}
	which physically represents the net force that must be applied to $\Omega_{\rm B}$ to keep it stationary while being acted upon by the imposed flow field \& represents the total amount of viscous dissipation of energy (per unit of time) in the fluid domain $\Omega$. Using the machinery of shape perturbations, $\delta J_{\rm d}(\Omega;V)$ indicates how $J_{\rm d}$ changes when we perturb $\Omega$ in the direction $V$. Hence, we can use this information to change $\Omega$ in small steps so as to slowly deform $\Omega$ into a shape that has better (lower) drag characteristics. A numerical computation: 2 large vortices appear behind the body, which indicate a large amount of viscous dissipation, i.e., large drag. The optimization process then computes $\delta J_{\rm d}(\Omega^0;V)$ for many different choices of $V$ \& chooses the one that drives down $J_{\rm d}$ the most. This choice of $V$ is used to deform $\Gamma_{\rm B}^0$ into a new shape $\Gamma_{\rm B}^1$ at iteration 1, with only a small difference between $\Gamma_{\rm B}^0$ \& $\Gamma_{\rm B}^1$. This process is repeated many times. Note how the vortices are eliminated by the more slender shape.
\end{example}

%------------------------------------------------------------------------------%

\section{Topology Optimization -- Tối Ưu Tôpô}

%------------------------------------------------------------------------------%

\section{Miscellaneous}

%------------------------------------------------------------------------------%

\printbibliography[heading=bibintoc]
	
\end{document}