\documentclass{article}
\usepackage[backend=biber,natbib=true,style=alphabetic,maxbibnames=50]{biblatex}
\addbibresource{/home/nqbh/reference/bib.bib}
\usepackage[utf8]{vietnam}
\usepackage{tocloft}
\renewcommand{\cftsecleader}{\cftdotfill{\cftdotsep}}
\usepackage[colorlinks=true,linkcolor=blue,urlcolor=red,citecolor=magenta]{hyperref}
\usepackage{amsmath,amssymb,amsthm,enumitem,float,graphicx,mathtools,tikz}
\usetikzlibrary{angles,calc,intersections,matrix,patterns,quotes,shadings}
\allowdisplaybreaks
\newtheorem{assumption}{Assumption}
\newtheorem{baitoan}{}
\newtheorem{cauhoi}{Câu hỏi}
\newtheorem{conjecture}{Conjecture}
\newtheorem{corollary}{Corollary}
\newtheorem{dangtoan}{Dạng toán}
\newtheorem{definition}{Definition}
\newtheorem{dinhly}{Định lý}
\newtheorem{dinhnghia}{Định nghĩa}
\newtheorem{example}{Example}
\newtheorem{ghichu}{Ghi chú}
\newtheorem{hequa}{Hệ quả}
\newtheorem{hypothesis}{Hypothesis}
\newtheorem{lemma}{Lemma}
\newtheorem{luuy}{Lưu ý}
\newtheorem{nhanxet}{Nhận xét}
\newtheorem{notation}{Notation}
\newtheorem{note}{Note}
\newtheorem{principle}{Principle}
\newtheorem{problem}{Problem}
\newtheorem{proposition}{Proposition}
\newtheorem{question}{Question}
\newtheorem{remark}{Remark}
\newtheorem{theorem}{Theorem}
\newtheorem{vidu}{Ví dụ}
\usepackage[left=1cm,right=1cm,top=5mm,bottom=5mm,footskip=4mm]{geometry}
\def\labelitemii{$\circ$}
\DeclareRobustCommand{\divby}{%
	\mathrel{\vbox{\baselineskip.65ex\lineskiplimit0pt\hbox{.}\hbox{.}\hbox{.}}}%
}
\setlist[itemize]{leftmargin=*}
\setlist[enumerate]{leftmargin=*}

\title{Olympic Tin Học Sinh Viên OLP \& ACM-ICPC}
\author{Nguyễn Quản Bá Hồng\footnote{A Scientist {\it\&} Creative Artist Wannabe. E-mail: {\tt nguyenquanbahong@gmail.com}. Bến Tre City, Việt Nam.}}
\date{\today}

\begin{document}
\maketitle
\begin{abstract}
	This text is a part of the series {\it Some Topics in Advanced STEM \& Beyond}:
	
	{\sc url}: \url{https://nqbh.github.io/advanced_STEM/}.
	
	Latest version:
	\begin{itemize}
		\item {\it Olympic Tin Học Sinh Viên OLP \& ICPC}.
		
		PDF: {\sc url}: \url{https://github.com/NQBH/advanced_STEM_beyond/blob/main/OLP_ICPC/NQBH_OLP_ICPC.pdf}.
		
		\TeX: {\sc url}: \url{https://github.com/NQBH/advanced_STEM_beyond/blob/main/OLP_ICPC/NQBH_OLP_ICPC.tex}.
		\item Codes:
		\begin{itemize}
			\item C: \url{https://github.com/NQBH/advanced_STEM_beyond/tree/main/OLP_ICPC/C}.
			\item C++: \url{https://github.com/NQBH/advanced_STEM_beyond/tree/main/OLP_ICPC/C++}.
			\item Python: \url{https://github.com/NQBH/advanced_STEM_beyond/tree/main/OLP_ICPC/Python}.
		\end{itemize}
	\end{itemize}
\end{abstract}
\tableofcontents

%------------------------------------------------------------------------------%

\section{Basic Competitive Programming -- Lập Trình Thi Đấu Cơ Bản}
\textbf{\textsf{Resources -- Tài nguyên.}}
\begin{enumerate}
	\item \cite{Laaksonen2020}. {\sc Antti Laaksonen}. {\it Guide to Competitive Programming: Learning \& Improving Algorithms Through Contests}.
	\item \cite{Thu_Phuong_Tien_Triet_NMLT}. {\sc Trần Đan Thư, Nguyễn Thanh Phương, Đinh Bá Tiến, Trần Minh Triết}. {\it Nhập Môn Lập Trình}.
	\item \cite{Thu_Phuong_Tien_Triet_Phuong_KTLT}. {\sc Trần Đan Thư, Nguyễn Thanh Phương, Đinh Bá Tiến, Trần Minh Triết, Đặng Bình Phương}. {\it Kỹ Thuật Lập Trình}.
	\item \cite{Thu_Tien_Khang_PPLTHDT}. {\sc Trần Đan Thư, Đinh Bá Tiến, Nguyễn Tấn Trần Minh Khang}. {\it Phương Pháp Lập Trình Hướng Đối Tượng}.
\end{enumerate}

\subsection{The art of handling inputs \& formatting outputs -- Nghệ thuật xử lý các dạng đầu vào \& định dạng các dạng đầu ra}
To handle various types of inputs \& format various types of outputs, see, e.g.:
\begin{itemize}
	\item \href{http://poj.org/faq.htm}{Peking University Judge Online for ACM{\tt/}ICPC (POJ){\tt/}FAQ}.
	See, e.g., \cite[Chap. 2, Subsect. 2.1.1, pp. 10--11]{Laaksonen2020, Laaksonen2024}.
\end{itemize}
To compile a C++ program in Linux, run in Terminal:
\begin{verbatim}
$ g++ -O2 -Wall program_name.cpp -o program_name
$ ./program_name
\end{verbatim}
or if you want to transfer input file into it \& print output into Terminal screen:
\begin{verbatim}
$ ./program_name < program_name.inp
\end{verbatim}
or if you want to transfer input file into it \& print output into a file:
\begin{verbatim}
$ ./program_name < program_name.inp > program_name.out
\end{verbatim}

\begin{itemize}
	\item Geeks4Geeks{\tt/std::endl} vs. \verb|\n| in C++: \url{https://www.geeksforgeeks.org/endl-vs-n-in-cpp/}. 
	\item {\tt i++} vs. {\tt++i}: \href{https://stackoverflow.com/questions/24886/is-there-a-performance-difference-between-i-and-i-in-c}{StackOverflow{\tt/}Is there a performance difference between i++ \& ++i in C?}
\end{itemize}

\subsection{Repeat{\tt/}Loop -- Lặp}

\subsection{String data -- Kiểu dữ liệu chuỗi}

%------------------------------------------------------------------------------%

\subsection{Array data -- Kiểu dữ liệu mảng}
Về mặt toán học, kiểu dữ liệu mảng là dãy số hữu hạn $(a_i)_{i=1}^n = (a_1,a_2,\ldots,a_n)$. Về mặt Tin học, kiểu dữ liệu mảng được ký hiệu bởi {\tt a[1..n]}.

\begin{baitoan}[\cite{Duc_200_BT_Python}, 141., pp. 140--141: Count digit -- Đếm chữ số]
	Cho dãy số $n$ số nguyên dương $A[1..n]$ \& 1 chữ số $k$. Đếm số lần xuất hiện chữ số $k$ trong dãy $A$ đã cho. E.g., với dãy $A[] = (11,12,13,14,15)$, thì chữ số $k = 1$ xuất hiện $6$ lần trong dãy $A$.
	\item {\sf Input.} Dòng đầu tiên của đầu vào chứa số nguyên $T\in\mathbb{N}^\star$ cho biết số bộ dữ liệu cần kiểm tra. Mỗi bộ dữ liệu gồm: (i) Dòng đầu chứa lần lượt $n,k\in\mathbb{N}$ là số phần tử trong dãy $A[]$ \& chữ số $k$. (ii) Dòng thứ 2 chứa $n$ số nguyên cách nhau 1 dấu cách, mô tả các phần tử của dãy $A$.
	\item {\sf Output.} Ứng với mỗi bộ dữ liệu, in ra 1 dòng chứa kết quả của bài toán tương ứng với bộ dữ liệu đầu vào đó.
	\item {\sf Constraint.} $1\le T\le100,1\le n\le100,0\le k\le9,1\le A[i]\le1000$, $\forall i = 1,\ldots,n$.
\end{baitoan}

\begin{itemize}
	\item Input: \url{https://github.com/NQBH/advanced_STEM_beyond/blob/main/OLP_ICPC/input/count_digit.inp}.
	\item Output: \url{https://github.com/NQBH/advanced_STEM_beyond/blob/main/OLP_ICPC/output/count_digit.out}.
	\item Python: \url{https://github.com/NQBH/advanced_STEM_beyond/blob/main/OLP_ICPC/Python/count_digit.py}.
	\item C++: ?
\end{itemize}

\begin{baitoan}[\cite{Duc_200_BT_Python}, 141., pp. 140--141: Count digit -- Đếm chữ số]
	Cho dãy số nguyên $a[1],a[2],\ldots,a[n]$. Thực hiện nhiệm vụ: Chia dãy thành 2 phần trái \& phải, trong đó phần trái gồm $\frac{n}{2}$ phần tử đầu tiên \& phần phải gồm các phần tử còn lại. Tính tổng các phần tử của mỗi phần, cuối cùng tính \& in ra tích 2 tổng tìm được.
	\item {\sf Input.} Dòng đầu tiên của đầu vào chứa $t\in\mathbb{N}^\star$ cho biết số bộ dữ liệu cần kiểm tra. Mỗi bộ dữ liệu gồm: (i) Dòng đầu chứa $n\in\mathbb{N}^\star$ cho biết số phần tử của dãy. (ii) Dòng 2 chứa $n$ số nguyên cách nhau bởi dấu cách, là các phần tử của dãy.
	\item {\sf Output.} Ứng với mỗi bộ dữ liệu, in ra 1 dong chứa kết quả của bài toán tương ứng với bộ dữ liệu đầu vào đó.
	\item {\sf Constraint.} $1\le t\le100,1\le n\le100,1\le A[i]\le100$, $\forall i = 1,\ldots,n$.
\end{baitoan}

\begin{itemize}
	\item Input: \url{https://github.com/NQBH/advanced_STEM_beyond/blob/main/OLP_ICPC/input/prod_left_right_sums.inp}.
	\item Output: \url{https://github.com/NQBH/advanced_STEM_beyond/blob/main/OLP_ICPC/output/prod_left_right_sums.out}.
	\item Python: \url{https://github.com/NQBH/advanced_STEM_beyond/blob/main/OLP_ICPC/Python/prod_left_right_sums.py}.
	\item C++: ?
\end{itemize}

%------------------------------------------------------------------------------%

\section{Olympic Tin THCS \& THPT}

\begin{baitoan}[\cite{Trung_HSG_THPT_Tin}, HSG12 Tp. Hà Nội 2020--2021, Prob. 1, p. 80: Find mid -- Tìm giữa]
	(a) Cho $l,r\in\mathbb{N}^\star$. Tìm $m\in[l,r)\cap\mathbb{N}^\star$ để chênh lệch giữa tổng các số nguyên liên tiếp từ $l$ đến $m$ \& tổng các số nguyên liên tiếp từ $m + 1$ đến $r$ là nhỏ nhất. (b) Mở rộng cho $l,r\in\mathbb{Z}$. (c$\star$) Thay tổng bởi tổng bình phương, tổng lập phương, tổng lũy thừa bậc $a\in\mathbb{R}$.
	\item {\sf Input.} 2 số $l,r\in\mathbb{N}^\star$, $l < r\le10^9$.
	\item {\sf Output.} Gồm 1 số nguyên duy nhất là $m$ thỏa mãn.
	\item {\sf Limits.} Subtask 1: $60\%$ các test có $l < r\le10^3$. Subtask 2: $40\%$ các test còn lại có $l < r\le10^9$.
\end{baitoan}

\begin{itemize}
	\item Input: \url{https://github.com/NQBH/advanced_STEM_beyond/blob/main/OLP_ICPC/input/find_mid.inp}.
	\item Output: 
	\item C++: \url{https://github.com/NQBH/advanced_STEM_beyond/blob/main/OLP_ICPC/C++/find_mid.cpp}.
	\item Python: \url{https://github.com/NQBH/advanced_STEM_beyond/blob/main/OLP_ICPC/Python/find_mid.py}.
\end{itemize}

%------------------------------------------------------------------------------%

\section{VNOI}

\begin{baitoan}[gcd in Pascal triangle -- ƯCLN trong tam giác Pascal, \url{https://oj.vnoi.info/problem/gpt}]
	Tam giác Pascal là 1 cách sắp xếp hình học của các hệ số nhị thức vào 1 tam giác. Hàng thứ $n\in\mathbb{N}$ của tam giác bao gồm các hệ số trong khai triển của đa thức $f(x,y) = (x + y)^n$. I.e., phần tử tại cột thứ $k$, hàng thứ $n$ của tam giác Pascal là $C_n^k = \binom{n}{k}$, i.e., tổ hợp chập $k$ của $n$ phần tử $0\le k\le n$. Cho $n\in\mathbb{N}$. Tính ${\rm GPT}(n)$ là ƯCLN của các số nằm giữa 2 số 1 trên hàng thứ $n$ của tam giác Pascal.
	\item {\sf Input.} Dòng đầu ghi $T$ là số lượng test. $T$ dòng tiếp theo, mỗi dòng ghi 1 số nguyên $n$.
	\item {\sf Output.} Gồm $T$ dòng, mỗi dòng ghi ${\rm GPT}(n)$ tương ứng.
	\item {\sf Constraint.} $1\le T\le20$, $2\le n\le10^9$.
\end{baitoan}
{\it Phân tích.} Công thức khai triển nhị thức Newton: $(a + b)^n = \sum_{i=0}^n C_n^ia^{n-i}b^i$, $\forall n\in\mathbb{N}$, see, e.g., \href{https://en.wikipedia.org/wiki/Binomial_theorem}{Wikipedia{\tt/}binomial theorem}. Cần tính $\gcd(\{C_n^i;1\le i\le n - 1\}) = \gcd(C_n^1,C_n^2,\ldots,C_n^{n-1})$. Chú ý mỗi hàng của tam giác Pascal có tính chất đối xứng nên chỉ cần xét ``1 nửa'' là đủ. Cụ thể hơn: $C_n^k = C_n^{n-k}$, $\forall k\in\mathbb{N}$, $k\le n$, nên
\begin{equation*}
	\{C_n^1,\ldots,C_n^{n-1}\} = \{C_n^1,\ldots,C_n^{\lfloor\frac{n}{2}\rfloor}\} = \left\{\begin{split}
		&\{C_n^1,\ldots,C_n^{\frac{n-1}{2}}\}&&\mbox{if } n\not{\divby}\ 2,\\
		&\{C_n^1,\ldots,C_n^{\frac{n}{2}}\}&&\mbox{if } n\divby2,
	\end{split}\right.
\end{equation*}
nên thay vì xét $i = 1,\ldots,n-1$, chỉ cần xét $i = 1,\ldots,\lfloor\frac{n}{2}\rfloor$ là đủ.

\begin{theorem}
	\begin{equation*}
		\gcd\{C_n^i\}_{i=1}^{n-1} = \left\{\begin{split}
			&p&&\mbox{if } n = p^k\mbox{ for some prime } p\mbox{ \& some } n\in\mathbb{N}^\star,\\
			&1&&\mbox{if } n\ne p^k\mbox{ for all prime } p\mbox{ \& any } n\in\mathbb{N}^\star.
		\end{split}\right.
	\end{equation*}
\end{theorem}
See also, e.g.:
\begin{itemize}
	\item \href{https://math.stackexchange.com/questions/2067235/gcd-of-binomial-coefficients}{Mathematics StackExchange{\tt/}gcd of binomial coefficients}.
\end{itemize}

%------------------------------------------------------------------------------%

\section{CSES Problem Set}
Link: \url{https://cses.fi/problemset/}.

\subsection{Introductory Problems}

\begin{problem}[CSES]
	There are $n$ concert tickets available, each with a certain price. Then, $m$ customers arrive, one after another. Each customer announces the maximum price they are willing to pay for a ticket, \& after this, they will get a ticket with nearest possible price such that it does not exceed the maximum price.
	\item {\sf Input.} 1st input line contains $n,m\in\mathbb{N}$: number of tickets \& number of customers. The next line contains $n$ integers $h_1,h_2,\ldots,h_n$: the price of each ticket. The last line contains $m$ integers $t_1,t_2,\ldots,t_m$: the maximum price for each customer in the order they arrive.
	\item {\sf Output.} Print, for each customer, the price that they will pay for their ticket. After this, ticket cannot be purchased again. If a customer cannot get any ticket, print $-1$.
	\item {\sf Constraints.} $1\le m,n\le 2\cdot10^5$, $1\le h_i,t_i\le10^9$.
\end{problem}

\subsection{Dynamic Programming}
\textbf{\textsf{Resources -- Tài nguyên.}}
\begin{enumerate}
	\item \cite{Bertsekas2005,Bertsekas2017}. {\sc Dimitri P. Bertsekas}. {\it Dynamic Programming \& Optimal Control. Vol. I}. 3e. 4e (can't download yet).
	\item \cite{Bertsekas2007,Bertsekas2012} {\sc Dimitri P. Bertsekas}. {\it Dynamic Programming \& Optimal Control. Vol. II}. 3e. 4e (can't download yet).
\end{enumerate}

\subsection{Graph Algorithms}

\subsection{Range Queries}

\subsection{Mathematics}

\begin{problem}[CSES{\tt/}Josephus Queries, \url{https://cses.fi/problemset/task/2164}]
	Consider a game where there are $n\in\mathbb{N}^\star$ children, numbered $1,2,\ldots,n$, in a circle. During the game, every 2nd child is removed from circle, until there are no children left. Task: process $q$ queries of the form: ``when there are $n$ children, who is the $k$th child that will be removed?''
	\begin{itemize}
		\item {\sf Input.} The 1st input line has an integer $q$: the number of queries. After this, there are $q$ lines that describe the queries. Each line has 2 integers $n,k$: the number of children \& the position of the child.
		\item {\sf Output.} Print $q$ integers: the answer for each query.
	\end{itemize}
\end{problem}
It seems to me that {\sf Jack97} (nickname: \verb|abortion_grandmaster|) proposed this problem.

Codes:
\begin{itemize}
	\item C++: \url{https://github.com/NQBH/advanced_STEM_beyond/blob/main/OLP_ICPC/C%2B%2B/gcd_Pascal_triangle.cpp}.
\end{itemize}

\begin{problem}[CSES{\tt/}Dice Probability, \url{https://cses.fi/problemset/task/1725}]
	Throw a dice $n\in\mathbb{N}^\star$ times, \& every throw produces an outcome between $1$ \& $6$. What is the probability that the sum of outcomes is between $a,b\in\mathbb{Z}$?
	\begin{itemize}
		\item {\sf Input.} The only input line contains 3 integers $n,a,b\in\mathbb{N}^\star$.
		\item {\sf Output.} Print probability rounded to 6 decimal places (rounding half to even).
		\item {\sf Constraints.} $1\le n\le100,1\le a\le b\le6n$.
		\item {\sf Example.} Input: {\tt2 9 10}. Output: {\tt0.194444}.
	\end{itemize}
\end{problem}
{\it Phân tích.} Gọi $n$ outcomes là $a_1,\ldots,a_n\in\{1,\ldots,6\}$. Sum of outcomes: $S\coloneqq\sum_{i=1}^n a_i\in\{n,\ldots,6n\}$.

\subsection{String Algorithms}

\subsection{Geometry}

\subsection{Advanced Techniques}

\subsection{Additional Problems}

%------------------------------------------------------------------------------%

\section{OLP}

%------------------------------------------------------------------------------%

\section{ICPC}

\begin{enumerate}
	\item \cite{Wu_Wang2016}. {\sc Yonghui Wu, Jiande Wang}. {\it Data Structure Practice for Collegiate Programming Contests \& Education}.
	\item \cite{Wu_Wang2018}. {\sc Yonghui Wu, Jiande Wang}. {\it Algorithm Design Practice for Collegiate Programming Contests \& Education}.
\end{enumerate}

\begin{problem}[\cite{Wu_Wang2016}, p. 4: financial management]
	{\sc Larry} graduated this year \& finally has a job. He's making a lot of money, but somehow never seems to have enough. {\sc Larry} has decided that he needs to get a hold of his financial portfolio \& solve his financial problems. The 1st step is to figure out what's been going on with his money. {\sc Larry} has his bank account statements \& wants to see how much money he has. Help {\sc Larry} by writing a program to take his closing balance from each of the past 12 months \& calculate his average account balance.
	\item {\sf Input.} The input will be 12 lines. Each line will contain the closing balance of his bank account for a particular month. Each number will be positive \& displayed to the penny. No dollar sign will be included.
	\item {\sf Output.} The output will be a single number, the average (mean) of the closing balances for the 12 months. It will be rounded to the nearest penny, preceded immediately by a dollar sign, \& followed by the end of the line. There will be no other spaces or characters in the output.
	\item {\sf Source.} ACM Mid-Atlantic United States 2001.
	\item {\sf IDs for online judges.} POJ 1004, ZOJ 1048, UVA 2362.
\end{problem}
\textbf{\textsf{Math Analysis.}} Let $(a_i)_{i=1}^{12}\subset[0,\infty)$ be monthly incomes of 12 months. Compute their average by the formula $\overline{a} = \frac{1}{12}\sum_{i=1}^{12} a_i$. This can be generalized to $n\in\mathbb{N}^\star$ months with a sequence of monthly incomes $(a_i)_{i=1}^n\subset[0,\infty)$ with its average value given by the formula $\overline{a}\coloneqq\frac{1}{n}\sum_{i=1}^n a_i$.

\textbf{\textsf{CS Analysis.}} The income of 12 months {\tt a[0..11]} is input by a {\tt for} statement \& the total income ${\tt sum}\coloneqq\sum_{i=0}^{11} a[i]$ is calculated. Then the average monthly income {\tt avg = sum/12} is calculated. Finally, {\tt avg} is output in accordance with the problem's output format by utilizing {\tt printf}'s format functionalities via \verb|printf("$%.2f", avg)|.
\begin{itemize}
	\item Input: \url{https://github.com/NQBH/advanced_STEM_beyond/blob/main/OLP_ICPC/input/financial_management.inp}.
	\item Output: \url{https://github.com/NQBH/advanced_STEM_beyond/blob/main/OLP_ICPC/output/financial_management.out}.
	\item C++: \url{https://github.com/NQBH/advanced_STEM_beyond/blob/main/OLP_ICPC/C++/financial_management.cpp}.
	\begin{verbatim}
#include <iostream>
using namespace std;
int main() {
    double avg, sum = 0.0, a[12] = {0};
    for (int i = 0; i < 12; ++i) { // input income of 12 motnhs a[0..11] & summation
        cin >> a[i];
        sum += a[i];
    }
    avg = sum/12; // compute average monthly
    printf("$%.2f", avg); // output average monthly
    return 0;
}		
	\end{verbatim}
	\item Python: \url{https://github.com/NQBH/advanced_STEM_beyond/blob/main/OLP_ICPC/Python/financial_management.py}.
	\begin{verbatim}
sum = 0
for __ in range(12):
    a = float(input())
    sum += a
print("${:.2f}".format(sum / 12))
	\end{verbatim}
\end{itemize}

\begin{baitoan}[Basic statistical data sample -- mẫu dữ liệu thống kê cơ bản]
	Cho 1 mẫu dữ liệu $(a_i)_{i=1}^n$. Tính trung bình, độ lệch chuẩn, phương sai của mẫu.
\end{baitoan}

\begin{problem}[\cite{Wu_Wang2016}, pp. 5--6: doubles]
	As part of an arithmetic competency program, your students will be given randomly generated lists of 2--15 unique positive integers \& asked to determine how many items in each list are twice some other item in same list. You will need a program to help you with the grading. This program should be able to scan the lists \& output the correct answer for each one. E.g., given the list {\tt1 4 3 2 9 7 18 22} your program should answer $3$, as $2$ is twice $1$, $4$ is twice $2$, \& $18$ is twice $9$.
	\item {\sf Input.} The input file will consist of 1 or more lists of numbers. There will be 1 list of numbers per line. Each list will contain from 2--15 unique positive integers. No integer will be $> 99$. Each line will be terminated with the integer $0$, which is not considered part of the list. A line with the single number $-1$ will mark the end of the file. Some lists may not contain any doubles.
	\item {\sf Output.} The output will consist of 1 line per input list, containing a count of the items that are double some other item.
	\item {\sf Source.} ACM Mid-Central United States 2003.
	\item {\sf IDs for online judges.} POJ 1552, ZOJ 1760, UVA 2787.
\end{problem}

\begin{remark}[Multiple test cases -- đa bộ test]
	For any problem with multiple test cases, a loop is used to deal with multiple test cases. The loop enumerates every test case.
	
	-- Đối với bất kỳ vấn đề nào có nhiều trường hợp thử nghiệm, một vòng lặp được sử dụng để xử lý nhiều trường hợp thử nghiệm. Vòng lặp liệt kê mọi trường hợp thử nghiệm.
\end{remark}

\begin{itemize}
	\item Input: \url{https://github.com/NQBH/advanced_STEM_beyond/blob/main/OLP_ICPC/input/double.inp}.
	\item Output: \url{https://github.com/NQBH/advanced_STEM_beyond/blob/main/OLP_ICPC/output/double.out}.
	\item C++:
	\begin{itemize}
		\item \url{https://github.com/NQBH/advanced_STEM_beyond/blob/main/OLP_ICPC/C++/double.cpp}.
		\begin{verbatim}
#include <iostream>
using namespace std;
int main() {
    int i, j, n, count, a[20];
    cin >> a[0]; // input 1st element
    while (a[0] != -1) { // if it is not end of input, input a new test case
        n = 1;
        for ( ; ; ++n) {
            cin >> a[n];
            if (a[n] == 0) break;
        }
        count = 0; // determine how many items in each list are twice some other item
        for (i = 0; i < n - 1; ++i) { // enumerate all pairs
            for (j = i + 1; j < n; ++j) {
                if (a[i]*2 == a[j] || a[j]*2 == a[i]) // accumulation
                    ++count;
            }
        }
        cout << count << endl; // output result
        cin >> a[0]; // input 1st element of next test case
    }
    return 0;
}
		\end{verbatim}
		\item \url{https://github.com/NQBH/advanced_STEM_beyond/blob/main/OLP_ICPC/C++/double_DPAK.cpp}: use map \& vector.
	\end{itemize} 
\end{itemize}




%------------------------------------------------------------------------------%

\section{Miscellaneous}

\subsection{Contributors}

\begin{enumerate}
	\item {\sc Võ Ngọc Trâm Anh}. C++ codes.
	\item {\sc Đặng Phúc An Khang}. C++ codes.
	\begin{itemize}
		\item {\sc Đặng Phúc An Khang}. {\it Combinatorics \& Number Theory in Competitive Programming -- Tổ Hợp \& Lý Thuyết Số trong Lập Trình Thi Đấu}.
		\item {\sc Đặng Phúc An Khang}. {\it Hướng Đến Kỳ Thi Olympic Tin học Sinh Viên Toàn Quốc \& ICPC 2025}.
		
		{\sc url}: \url{https://github.com/GrootTheDeveloper/OLP-ICPC/blob/master/2025/COMPETITIVE_REPORT.pdf}.
	\end{itemize}
	\item {\sc Nguyễn Lê Anh Khoa}. C++ codes.
	\item {\sc Phan Vĩnh Tiến}. C++ codes.
\end{enumerate}

\subsection{Donate or Buy Me Coffee}
Donate (but do not donut) or buy me some coffee via NQBH's bank account information at \url{https://github.com/NQBH/publication/blob/master/bank/NQBH_bank_account_information}.

\subsection{See also}

\begin{enumerate}
	\item {\it Vietnamese Mathematical Olympiad for High School- \& College Students (VMC) -- Olympic Toán Học Học Sinh \& Sinh Viên Toàn Quốc}.
	
	PDF: {\sc url}: \url{https://github.com/NQBH/advanced_STEM_beyond/blob/main/VMC/NQBH_VMC.pdf}.
	
	\TeX: {\sc url}: \url{https://github.com/NQBH/advanced_STEM_beyond/blob/main/VMC/NQBH_VMC.tex}.
	\begin{itemize}
		\item Codes:
		\begin{itemize}
			\item C++ code: \url{https://github.com/NQBH/advanced_STEM_beyond/tree/main/VMC/C++}.
			\item Python code: \url{https://github.com/NQBH/advanced_STEM_beyond/tree/main/VMC/Python}.
		\end{itemize}
		\item Resource: \url{https://github.com/NQBH/advanced_STEM_beyond/tree/main/VMC/resource}.
		\item Figures: \url{https://github.com/NQBH/advanced_STEM_beyond/tree/main/VMC/figure}.
	\end{itemize}
\end{enumerate}

%------------------------------------------------------------------------------%

\printbibliography[heading=bibintoc]
	
\end{document}