\documentclass{article}
\usepackage[backend=biber,natbib=true,style=alphabetic,maxbibnames=50]{biblatex}
\addbibresource{/home/nqbh/reference/bib.bib}
\usepackage[utf8]{vietnam}
\usepackage{tocloft}
\renewcommand{\cftsecleader}{\cftdotfill{\cftdotsep}}
\usepackage[colorlinks=true,linkcolor=blue,urlcolor=red,citecolor=magenta]{hyperref}
\usepackage{amsmath,amssymb,amsthm,enumitem,fancyvrb,float,graphicx,mathtools,soul,tikz}
\usetikzlibrary{angles,calc,intersections,matrix,patterns,quotes,shadings}
\allowdisplaybreaks
\newtheorem{assumption}{Assumption}
\newtheorem{baitoan}{Bài toán}
\newtheorem{cauhoi}{Câu hỏi}
\newtheorem{conjecture}{Conjecture}
\newtheorem{corollary}{Corollary}
\newtheorem{dangtoan}{Dạng toán}
\newtheorem{definition}{Definition}
\newtheorem{dinhly}{Định lý}
\newtheorem{dinhnghia}{Định nghĩa}
\newtheorem{example}{Example}
\newtheorem{ghichu}{Ghi chú}
\newtheorem{hequa}{Hệ quả}
\newtheorem{hypothesis}{Hypothesis}
\newtheorem{lemma}{Lemma}
\newtheorem{luuy}{Lưu ý}
\newtheorem{nhanxet}{Nhận xét}
\newtheorem{notation}{Notation}
\newtheorem{note}{Note}
\newtheorem{principle}{Principle}
\newtheorem{problem}{Problem}
\newtheorem{proposition}{Proposition}
\newtheorem{question}{Question}
\newtheorem{remark}{Remark}
\newtheorem{theorem}{Theorem}
\newtheorem{vidu}{Ví dụ}
\usepackage[left=1cm,right=1cm,top=5mm,bottom=5mm,footskip=4mm]{geometry}
\def\labelitemii{$\circ$}
\DeclareRobustCommand{\divby}{%
    \mathrel{\vbox{\baselineskip.65ex\lineskiplimit0pt\hbox{.}\hbox{.}\hbox{.}}}%
}
\setlist[itemize]{leftmargin=*}
\setlist[enumerate]{leftmargin=*}

\title{Tổng Kết Điểm Lớp Tổ Hợp {\it\&} Lý Thuyết Đồ Thị}
\author{Nguyễn Quản Bá Hồng\footnote{A scientist- {\it\&} creative artist wannabe, a mathematics {\it\&} computer science lecturer of Department of Artificial Intelligence {\it\&} Data Science (AIDS), School of Technology (SOT), UMT Trường Đại học Quản lý {\it\&} Công nghệ TP.HCM, Hồ Chí Minh City, Việt Nam.\\E-mail: {\sf nguyenquanbahong@gmail.com} {\it\&} {\sf hong.nguyenquanba@umt.edu.vn}. Website: \url{https://nqbh.github.io/}. GitHub: \url{https://github.com/NQBH}.}}
\date{\today}

\begin{document}
\maketitle
\tableofcontents

%------------------------------------------------------------------------------%

\section{UMT Summer Semester 2025{\tt/}1387: Combinatorics \& Graph Theory}

%------------------------------------------------------------------------------%

\subsection{Comments on weekly reports \& Final-term projects}

\begin{enumerate}
    \item {\sc Võ Ngọc Trâm Anh.}
    \begin{itemize}
        \item {\bf Weekly reports.}
        \item {\bf Final-term projects.}
        \begin{enumerate}
            \item Project 4, Bài toán 1: In biểu đồ Ferrers \& Ferrers chuyển vị sai định dạng: phải sắp xếp theo thứ tự không tăng chứ không phải không giảm. In dấu khoảng trắng ở bên phải chứ không phải bên trái.
        \end{enumerate}
        \begin{itemize}
            \item BT1 Ferrers:
            \item BT2 so sánh $p_k(n),p_{\max}(n,k)$:
            \item BT3 self-conjugate partition:
            \item BT4 graph \& tree representations:
            \item BT 5:
            \item BT 6:
            \item BT 7:
            \item BT 8--10:
            \item BT 11--13:
            \item BT 14--16:
        \end{itemize}
    \end{itemize}
    \item {\sc Hoàng Anh.}
    \begin{itemize}
        \item {\bf Weekly reports.}
        \item {\bf Final-term projects.} Code lồng trong report khác với file code (rất nặng AIs \& OOP \& chứa nhiều sự phức tạp không cần thiết -- excessively unnecessary complications).
        \begin{itemize}
            \item BT1 Ferrers: Căn trái chứ không phải căn phải. Đánh số biểu đồ Ferrers chuyển vị sai: đánh số bên phải theo từng dòng chứ không phải bên dưới theo từng cột. Điểm mới: Cú pháp Pythonic của Python. Sai chính tả: \st{ferreries} diagram $\to$ Ferrers diagram. Code theo style OOP nặng hình thức, kết quả đúng. \fbox{0.3}.
            \item BT2 so sánh $p_k(n),p_{\max}(n,k)$: $\emptyset$ \fbox{0}.
            \item BT3 self-conjugate partition:$\emptyset$ \fbox{0}.
            \item BT4 graph \& tree representations:
            \item BT 5:
            \item BT 6:
            \item BT 7:
            \item BT 8--10:
            \item BT 11--13:
            \item BT 14--16:
        \end{itemize}
    \end{itemize}
    \item {\sc Võ Huỳnh Thái Bảo.}
    \begin{itemize}
        \item {\bf Weekly reports.}
        \item {\bf Final-term projects.}
        \begin{itemize}
            \item File README.md khá hay \fbox{0.25}.
            \item BT1 Ferrers:
            \item BT2 so sánh $p_k(n),p_{\max}(n,k)$:
            \item BT3 self-conjugate partition:
            \item BT4 graph \& tree representations:
            \item BT 5:
            \item BT 6:
            \item BT 7:
            \item BT 8--10:
            \item BT 11--13:
            \item BT 14--16:
        \end{itemize}
    \end{itemize}
    \item {\sc Trần Mạnh Đức.}
    \begin{itemize}
        \item {\bf Weekly reports.}
        \item {\bf Final-term projects.}
        \begin{itemize}
            \item BT1 Ferrers:
            \item BT2 so sánh $p_k(n),p_{\max}(n,k)$:
            \item BT3 self-conjugate partition:
            \item BT4 graph \& tree representations:
            \item BT 5:
            \item BT 6:
            \item BT 7:
            \item BT 8--10:
            \item BT 11--13:
            \item BT 14--16:
        \end{itemize}
    \end{itemize}
    \item {\sc Nguyễn Trung Hậu.}
    \begin{itemize}
        \item {\bf Weekly reports.}
        \item {\bf Final-term projects.}
        \begin{itemize}
            \item BT1 Ferrers:
            \item BT2 so sánh $p_k(n),p_{\max}(n,k)$:
            \item BT3 self-conjugate partition:
            \item BT4 graph \& tree representations:
            \item BT 5:
            \item BT 6:
            \item BT 7:
            \item BT 8--10:
            \item BT 11--13:
            \item BT 14--16:
        \end{itemize}
    \end{itemize}
    \item {\sc Phạm Phước Minh Hiếu.}
    \begin{itemize}
        \item {\bf Weekly reports.}
        \item {\bf Final-term projects.}
        \begin{itemize}
            \item BT1 Ferrers:
            \item BT2 so sánh $p_k(n),p_{\max}(n,k)$:
            \item BT3 self-conjugate partition:
            \item BT4 graph \& tree representations:
            \item BT 5:
            \item BT 6:
            \item BT 7:
            \item BT 8--10:
            \item BT 11--13:
            \item BT 14--16:
        \end{itemize}
    \end{itemize}
    \item {\sc Hoàng Quang Huy.}
    \begin{itemize}
        \item {\bf Weekly reports.}
        \item {\bf Final-term projects.}
        \begin{itemize}
            \item BT1 Ferrers:
            \item BT2 so sánh $p_k(n),p_{\max}(n,k)$:
            \item BT3 self-conjugate partition:
            \item BT4 graph \& tree representations:
            \item BT 5:
            \item BT 6:
            \item BT 7:
            \item BT 8--10:
            \item BT 11--13:
            \item BT 14--16:
        \end{itemize}
    \end{itemize}
    \item {\sc Phan Nguyễn Duy Kha.}
    \begin{itemize}
        \item {\bf Weekly reports.}
        \item {\bf Final-term projects.}
        \begin{itemize}
            \item BT1 Ferrers:
            \item BT2 so sánh $p_k(n),p_{\max}(n,k)$:
            \item BT3 self-conjugate partition:
            \item BT4 graph \& tree representations:
            \item BT 5:
            \item BT 6:
            \item BT 7:
            \item BT 8--10:
            \item BT 11--13:
            \item BT 14--16:
        \end{itemize}
    \end{itemize}
    \item {\sc Phạm Minh Khoa.}
    \begin{itemize}
        \item {\bf Weekly reports.} Sử dụng AI mà không  edit lại.
        \item {\bf Final-term projects.} Code đậm mùi raw non-edit AIs nhưng bù lại có comment code quá nhiều. Typo: MSVV $\to$ MSSV. Thiếu tên GV.
        \begin{itemize}
            \item Không có code Python, chỉ có code C++ nên chia đôi điểm.
            \item BT1 Ferrers: đúng. \fbox{0.25}.
            \item BT2 so sánh $p_k(n),p_{\max}(n,k)$: Hiểu sai đề. Bài toán yêu cầu tính riêng $p_k(n)$ \& $p_{\max}(n,k)$ rồi so sánh chúng để kiểm tra lại định lý $p_k(n) = p_{\max}(n,k)$ chứ không phải áp dụng định lý để chỉ tính có $p_k(n)$. Phần tính $p_{\max}(n,k)$ mới khó \& là phần chính  của bài toán. \fbox{0.1}.
            \item BT3 self-conjugate partition: Hiểu sai đề. Sai kết quả. Tại sao {\tt problems.cpp, line 21}: $n - i\ge k - 1$ là điều kiện cắt tỉa để tối ưu? Sai vì bài toán chỉ phụ thuộc vào mỗi biến $n$, không phụ thuộc vào biến $k$. \fbox{0.1}.
            \item BT4 graph \& tree representations: chỉ xét simple graph \& multigraph, thiếu general graph, thiếu tree hoàn toàn. Đề bài yêu cầu xử lý tất cả cặp chuyển đổi chứ không phải chỉ nêu ra 1 cặp đại diện. \fbox{0.1}.
            \item BT 5:
            \item BT 6:
            \item BT 7:
            \item BT 8--10:
            \item BT 11--13:
            \item BT 14--16:
        \end{itemize}
    \end{itemize}
    \item {\sc Trần Thành Lợi.}
    \begin{itemize}
        \item {\bf Weekly reports.} $\emptyset$. 0 đ.
        \item {\bf Final-term projects.} $\emptyset$. 0 đ.
    \end{itemize}
    \item {\sc Lê Đức Long.}
    \begin{itemize}
        \item {\bf Weekly reports.}
        \item {\bf Final-term projects.}
        \begin{itemize}
            \item BT1 Ferrers:
            \item BT2 so sánh $p_k(n),p_{\max}(n,k)$:
            \item BT3 self-conjugate partition:
            \item BT4 graph \& tree representations:
            \item BT 5:
            \item BT 6:
            \item BT 7:
            \item BT 8--10: Thiếu đồ thị có hướng.
            \item BT 11--13:
            \item BT 14--16:
        \end{itemize}
    \end{itemize}
    \item {\sc Huỳnh Nhật Quang.}
    \begin{itemize}
        \item {\bf Weekly reports.}
        \item {\bf Final-term projects.}
        \begin{itemize}
            \item BT1 Ferrers:
            \item BT2 so sánh $p_k(n),p_{\max}(n,k)$:
            \item BT3 self-conjugate partition:
            \item BT4 graph \& tree representations:
            \item BT 5:
            \item BT 6:
            \item BT 7:
            \item BT 8--10:
            \item BT 11--13:
            \item BT 14--16:
        \end{itemize}
    \end{itemize}
    \item {\sc Cao Sỹ Siêu.}
    \begin{itemize}
        \item {\bf Weekly reports.}
        \item {\bf Final-term projects.}
        \begin{itemize}
            \item BT1 Ferrers:
            \item BT2 so sánh $p_k(n),p_{\max}(n,k)$:
            \item BT3 self-conjugate partition:
            \item BT4 graph \& tree representations:
            \item BT 5:
            \item BT 6:
            \item BT 7:
            \item BT 8--10:
            \item BT 11--13:
            \item BT 14--16:
        \end{itemize}
    \end{itemize}
    \item {\sc Sơn Tân.}
    \begin{itemize}
        \item {\bf Weekly reports.}
        \item {\bf Final-term projects.}
        \begin{itemize}
            \item BT1 Ferrers:
            \item BT2 so sánh $p_k(n),p_{\max}(n,k)$:
            \item BT3 self-conjugate partition:
            \item BT4 graph \& tree representations:
            \item BT 5:
            \item BT 6:
            \item BT 7:
            \item BT 8--10:
            \item BT 11--13:
            \item BT 14--16:
        \end{itemize}
    \end{itemize}
    \item {\sc Nguyễn Ngọc Thạch.}
    \begin{itemize}
        \item {\bf Weekly reports.}
        \item {\bf Final-term projects.}
        \begin{itemize}
            \item BT1 Ferrers:
            \item BT2 so sánh $p_k(n),p_{\max}(n,k)$:
            \item BT3 self-conjugate partition:
            \item BT4 graph \& tree representations:
            \item BT 5:
            \item BT 6:
            \item BT 7:
            \item BT 8--10:
            \item BT 11--13:
            \item BT 14--16:
        \end{itemize}
    \end{itemize}
    \item {\sc Phan Vĩnh Tiến.}
    \begin{itemize}
        \item {\bf Weekly reports.}
        \item {\bf Final-term projects.}
        \begin{itemize}
            \item BT1 Ferrers:
            \item BT2 so sánh $p_k(n),p_{\max}(n,k)$:
            \item BT3 self-conjugate partition:
            \item BT4 graph \& tree representations:
            \item BT 5:
            \item BT 6:
            \item BT 7:
            \item BT 8--10:
            \item BT 11--13:
            \item BT 14--16:
        \end{itemize}
    \end{itemize}
\end{enumerate}

%------------------------------------------------------------------------------%

\subsection{Final grades}

\begin{table}[H]
    \centering
    \begin{tabular}{|l|r|r|r|r|r|r|}
        \hline
        Student & Attendance & Weekly report & Midterm & Final-term project & Bonus{\tt/}Minus & Final grade \\
        \hline
        {\sc Võ Ngọc Trâm Anh} & 7.5 &  & 11.25 &  &  &  \\
        \hline
        {\sc Hoàng Anh} & 7 &  & 6.5 &  &  &  \\
        \hline
        {\sc Võ Huỳnh Thái Bảo} & 7 &  & 3.75 &  &  &  \\
        \hline
        {\sc Trần Mạnh Đức} & 3 &  & 5.75 &  &  &  \\
        \hline
        {\sc Nguyễn Trung Hậu} & $-11.25$ &  & 0.75 &  &  &  \\
        \hline
        {\sc Phạm Phước Minh Hiếu} & 7.5 &  & 4 &  &  &  \\
        \hline
        {\sc Hoàng Quang Huy} & 3.25 &  & 5.25 &  &  &  \\
        \hline
        {\sc Phan Nguyễn Duy Kha} & -3.25 &  & 7 &  &  &  \\
        \hline
        {\sc Phạm Minh Khoa} & $-3.75$ &  & 0 &  &  &  \\
        \hline
        {\sc Trần Thành Lợi} & $-16$ & 0 & 0 & 0 & 0 & $-16$ \\
        \hline
        {\sc Lê Đức Long} & 4.25 &  & 6 &  &  &  \\
        \hline
        {\sc Lê Công Hoàng Phúc} & 6.25 & & 4.5 &  &  &  \\
        \hline
        {\sc Huỳnh Nhật Quang} & $-10.5$ &  & 2 &  &  &  \\
        \hline
        {\sc Cao Sỹ Siêu} & 6.75 &  & 5.75 &  &  &  \\
        \hline
        {\sc Sơn Tân} & 6.75 &  & 6 &  &  &  \\
        \hline
        {\sc Nguyễn Ngọc Thạch} & 3.25 &  & 8.25 &  &  &  \\
        \hline
        {\sc Phan Vĩnh Tiến} & 3.5 &  & 11 &  &  &  \\
        \hline
    \end{tabular}
\end{table}

%------------------------------------------------------------------------------%

\end{document}