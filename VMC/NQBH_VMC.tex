\documentclass{article}
\usepackage[backend=biber,natbib=true,style=alphabetic,maxbibnames=50]{biblatex}
\addbibresource{/home/nqbh/reference/bib.bib}
\usepackage[utf8]{vietnam}
\usepackage{tocloft}
\renewcommand{\cftsecleader}{\cftdotfill{\cftdotsep}}
\usepackage[colorlinks=true,linkcolor=blue,urlcolor=red,citecolor=magenta]{hyperref}
\usepackage{amsmath,amssymb,amsthm,enumitem,float,graphicx,mathtools,tikz}
\usetikzlibrary{angles,calc,intersections,matrix,patterns,quotes,shadings}
\allowdisplaybreaks
\newtheorem{assumption}{Assumption}
\newtheorem{baitoan}{}
\newtheorem{cauhoi}{Câu hỏi}
\newtheorem{conjecture}{Conjecture}
\newtheorem{corollary}{Corollary}
\newtheorem{dangtoan}{Dạng toán}
\newtheorem{definition}{Definition}
\newtheorem{dinhly}{Định lý}
\newtheorem{dinhnghia}{Định nghĩa}
\newtheorem{example}{Example}
\newtheorem{ghichu}{Ghi chú}
\newtheorem{hequa}{Hệ quả}
\newtheorem{hypothesis}{Hypothesis}
\newtheorem{lemma}{Lemma}
\newtheorem{luuy}{Lưu ý}
\newtheorem{nhanxet}{Nhận xét}
\newtheorem{notation}{Notation}
\newtheorem{note}{Note}
\newtheorem{principle}{Principle}
\newtheorem{problem}{Problem}
\newtheorem{proposition}{Proposition}
\newtheorem{question}{Question}
\newtheorem{remark}{Remark}
\newtheorem{theorem}{Theorem}
\newtheorem{vidu}{Ví dụ}
\usepackage[left=1cm,right=1cm,top=5mm,bottom=5mm,footskip=4mm]{geometry}
\def\labelitemii{$\circ$}
\DeclareRobustCommand{\divby}{%
	\mathrel{\vbox{\baselineskip.65ex\lineskiplimit0pt\hbox{.}\hbox{.}\hbox{.}}}%
}
\setlist[itemize]{leftmargin=*}
\setlist[enumerate]{leftmargin=*}

\title{Vietnamese Mathematical Olympiad for High School- {\it\&} College Students\\Olympic Toán Học Học Sinh {\it\&} Sinh Viên Toàn Quốc (VMC)}
\author{Nguyễn Quản Bá Hồng\footnote{A Scientist {\it\&} Creative Artist Wannabe. E-mail: {\tt nguyenquanbahong@gmail.com}. Bến Tre City, Việt Nam.}}
\date{\today}

\begin{document}
\maketitle
\begin{abstract}
	This text is a part of the series {\it Some Topics in Advanced STEM \& Beyond}:
	
	{\sc url}: \url{https://nqbh.github.io/advanced_STEM/}.
	
	Latest version:
	\begin{itemize}
		\item {\it Vietnamese Mathematical Olympiad for High School- \& College Students (VMC) -- Olympic Toán Học Học Sinh \& Sinh Viên Toàn Quốc}.
		
		PDF: {\sc url}: \url{https://github.com/NQBH/advanced_STEM_beyond/blob/main/VMC/NQBH_VMC.pdf}.
		
		\TeX: {\sc url}: \url{https://github.com/NQBH/advanced_STEM_beyond/blob/main/VMC/NQBH_VMC.tex}.
	\end{itemize}
\end{abstract}
\tableofcontents

%------------------------------------------------------------------------------%

\section{Preliminaries -- Kiến thức chuẩn bị}

\textbf{\textsf{Resources -- Tài nguyên.}}
\begin{enumerate}
	\item \cite{Khai_so_hoc_day_so}. {\sc Phan Huy Khải}. {\it Các Chuyên Đề Số Học Bồi Dưỡng Học Sinh Giỏi Toán Trung Học. Chuyên Đề 2: Số Học \& Dãy Số}.
	\item {\sc VMS -- Hội Toán Học Việt Nam}. {\it Kỷ Yếu Kỳ Thi Olympic Toán Học Sinh Viên--Học Sinh Lần 28}.
	\item {\sc VMS -- Hội Toán Học Việt Nam}. {\it Kỷ Yếu Kỳ Thi Olympic Toán Học Sinh Viên--Học Sinh Lần 29}. Huế, 2--8.4.2023.
\end{enumerate}

\section{Algebra -- Đại Số}
\textbf{\textsf{Resources -- Tài nguyên.}}
\begin{enumerate}
	\item {\sc Lê Tuấn Hoa}. {\it Đại Số Tuyến Tính Qua Các Ví Dụ \& Bài Tập}.
	\item \cite{Hung_linear_algebra}. {\sc Nguyễn Hữu Việt Hưng}. {\it Đại Số Tuyến Tính}.
	\item {\sc Ngô Việt Trung}. {\it Giáo Trình Đại Số Tuyến Tính}.
\end{enumerate}

\subsection{Matrix -- Ma trận}
See, e.g., \cite[Chap. 3, \S9: Cấu trúc nghiệm của hệ phương trình tuyến tính, pp. 163--165]{Hung_linear_algebra}.

Xét các hệ phương trình tuyến tính thuần nhất \& không thuần nhất liên kết với nhau $A{\bf x} = {\bf0}$ \& $A{\bf x} = {\bf b}$, với $A = (a_{ij})_{m\times n}\in M(m\times n,\mathbb{F}),{\bf b}\in\mathbb{F}^m$ (cả 2 hệ phương trình đều gồm $m$ phương trình \& $n$ ẩn).

\begin{dinhly}[\cite{Hung_linear_algebra}, Định lý 9.1, p. 163]
	Tập hợp $L$ tất cả các nghiệm của hệ phương trình tuyến tính thuần nhất $Ax = 0$ là 1 không gian vector con của $\mathbb{F}^n$, có số chiều thỏa mãn hệ thức $\dim L = n - \operatorname{rank}A$.
\end{dinhly}

\begin{dinhly}[\cite{Hung_linear_algebra}, Định lý 9.2, p. 164]
	Giả sử $L$ là không gian vector con gồm các nghiệm của hệ phương trình tuyến tính thuần nhất $A{\bf x} = {\bf0}$, \& ${\bf x}^0$ là 1 nghiệm của hệ $A{\bf x} = {\bf b}$. Khi đó tập hợp các nghiệm của hệ $A{\bf x} = {\bf b}$ là ${\bf x}^0 + L = \{{\bf x}^0 + {\bf a}|{\bf a}\in L\}$.
\end{dinhly}
$L\coloneqq\operatorname{Ker}\tilde{A}$ với $\tilde{A}:\mathbb{F}^n\to\mathbb{F}^m$, ${\bf x}\mapsto A{\bf x}$, i.e., $L$ là hạt nhân{\tt/}hạch của ánh xạ tuyến tính $\tilde{A}$.

\begin{dinhnghia}[Nghiệm riêng \& nghiệm tổng quát của hệ phương trình tuyến tính không thuần nhất]
	Với các giả thiết của định lý trên, ${\bf x}^0$ được gọi là 1 \emph{nghiệm riêng} của hệ phương trình tuyến tính không thuần nhất $A{\bf x} = {\bf b}$. Còn ${\bf x}^0 + {\bf a}$ với ${\bf a}\in L$, được gọi là \emph{nghiệm tổng quát} của hệ phương trình đó.
\end{dinhnghia}

\begin{dinhly}[\cite{Hung_linear_algebra}, Định lý 9.4: Tiêu chuẩn Kronecker--Capelli, p. 164]
	Hệ phương trình tuyến tính $A{\bf x} = {\bf b}$ có nghiệm $\Leftrightarrow\operatorname{rank}A = \operatorname{rank}\overline{A}$ với $\overline{A} = (A|{\bf b})$ là ma trận các hệ số mở rộng của hệ.
\end{dinhly}

\begin{baitoan}[VMC2024A1B1]
	Cho $a\in\mathbb{R}$, $A$ là 1 ma trận phụ thuộc vào $a$:
	\begin{equation}
		A = \begin{pmatrix}
			1 & a + 1 & a + 2 & 0\\a + 3 & 1 & 0 & a + 2\\a + 2 & 0 & 1 & a + 1\\0 & a + 2 & a + 3 & 1
		\end{pmatrix}
	\end{equation}
	(a) Tìm $\operatorname{rank}A$ khi $a = -1$. (b) Tìm tất cả $a\in\mathbb{R}$ để $\det A > 0$. (c) Biện luận số chiều của không gian nghiệm của hệ phương trình tuyến tính $AX = 0 $ theo $a$ với $X = [x,y,z,t]^\top$.
\end{baitoan}

\begin{proof}
	(a) Khi $a = -1$:
	\begin{equation}
		A = \begin{pmatrix}
			1 & 0 & 1 & 0\\2 & 1 & 0 & 1\\1 & 0 & 1 & 0\\0 &1 & 2 & 1
		\end{pmatrix}
	\end{equation}
	Use the following code snippet
	\begin{verbatim}
		import numpy as np
		A = np.matrix([[1,0,1,0], [2,1,0,1], [1,0,1,0], [0,1,2,1]])
		print(np.linalg.matrix_rank(A))
	\end{verbatim}
	to obtain $\operatorname{rank}A = 3$. (b) Dùng công thức tính định thức ma trận để thu được $\det A = -4a^2 - 16a - 12 = -4(a + 1)(a + 3)$, nên $\det A > 0\Leftrightarrow-4(a + 1)(a + 3) > 0\Leftrightarrow a\in(-3,-1)$. (c) Nếu $a = -1$, $\operatorname{rank}A = 3\Rightarrow\dim L = 4 - \operatorname{rank}A = 4 - 3 = 1$. Nếu $a = -3$, tính được $\operatorname{rank}A = 3$, nên $\dim L = 4 - \operatorname{rank}A = 4 - 3 = 1$. Nếu $a\notin\{-1,-3\}$ thì $\det A = -4(a + 1)(a + 3)\ne0\Rightarrow\operatorname{rank}A = 4\Rightarrow\dim L = 4 - \operatorname{rank}A = 4 - 4 = 0$.	
\end{proof}

\begin{baitoan}[Symbolic computation software, libraries]
	Tương tự như phần mềm MATLAB \url{https://www.mathworks.com/products/matlab.html}, tìm các phần mềm, ngôn ngữ, hoặc thư viện của các ngôn ngữ quen thuộc như Python (thư viện SymPy \url{https://www.sympy.org/en/index.html}), C{\tt/}C++ để thực hành symbolic computation.
\end{baitoan}

\subsection{Vector space -- Không gian vector}
Giả sử $V,W$: 2 không gian vector trên trường $\mathbb{F}$ (see, e.g., \cite[Chap. 2, \S2: Ánh xạ tuyến tính, pp. 100--110]{Hung_linear_algebra}).

\begin{dinhnghia}[Ánh xạ tuyến tính]
	Ánh xạ $f:V\to W$ được gọi là 1 \emph{ánh xạ tuyến tính} (hoặc rõ hơn là 1 \emph{ánh xạ $\mathbb{F}$-tuyến tính}), nếu
	\begin{align}
		f(\alpha + \beta) &= f(\alpha) + f(\beta),\ \forall\alpha,\beta\in V,\\
		f(a\alpha) &= af(\alpha),\ \forall a\in\mathbb{F}.
	\end{align}
	Ánh xạ tuyến tính cũng được gọi là \emph{đồng cấu tuyến tính}, hay \emph{đồng cấu} cho đơn giản.
\end{dinhnghia}
2 điều kiện trong định nghĩa ánh xạ tuyến tính $\Leftrightarrow$ điều kiện:
\begin{equation}
	f(\alpha a + \beta b) = af(\alpha) + bf(\beta),\ \forall\alpha,\beta\in V,\ \forall a,b\in\mathbb{R}.
\end{equation}

\begin{dinhly}[Tính chất cơ bản của ánh xạ tuyến tính]
	Giả sử $f:V\to W$ là 1 ánh xạ tuyến tính. Khi đó: (i) $f(0) = 0$. (ii) $f(-\alpha) = -f(\alpha)$, $\forall\alpha\in V$. (iii)
	\begin{equation}
		f\left(\sum_{i=1}^n a_i\alpha_i\right) = \sum_{i=1}^n a_if(\alpha_i),\ \forall a_i\in\mathbb{F},\ \forall\alpha_i\in V,\,\forall i = 1,\ldots,n.
	\end{equation}
\end{dinhly}

\begin{vidu}[Ánh xạ tuyến tính cơ bản]
	\item(i) Ánh xạ không $0:V\to W$, $0(\alpha) = 0$, $\forall\alpha\in V$. Thế còn ánh xạ hằng $C:V\to W$, $C(\alpha) = C$, $\forall\alpha\in V$ với $C\in\mathbb{F}$ cho trước?
	\item(ii) Ánh xạ đồng nhất (identity mapping) ${\rm id}_V:V\to V$, ${\rm id}_V(\alpha) = \alpha$, $\forall\alpha\in V$.
	\item(iii) Đạo hàm hình thức
	\begin{equation}
		\frac{d}{dX}:\mathbb{F}[X]\to\mathbb{F}[X],\ \frac{d}{dX}\sum_{i=0}^n a_iX^i = \sum_{i=1}^n ia_iX^{i-1} = \sum_{i=0}^{n-1} (i + 1)a_{i + 1}X^i.
	\end{equation}
	\item(iv) Tích phân hình thức
	\begin{equation}
		\int {\rm d}X:\mathbb{F}[X]\to\mathbb{F}[X],\ \int \sum_{i=0}^n a_iX^i\,{\rm d}X = \sum_{i=0}^n \frac{a_i}{i + 1}X^{i+1}.
	\end{equation}
	\item(v) Giả sử $A = (a_{ij})\in M(m\times n,\mathbb{F})$,
	\begin{equation}
		\widetilde{A}:\mathbb{F}^n\to\mathbb{F}^m,\begin{pmatrix}
			x_1\\\vdots\\x_n
		\end{pmatrix}\mapsto A\begin{pmatrix}
		x_1\\\vdots\\x_n
		\end{pmatrix}.
	\end{equation}.
	\item(vi) Các phép chiếu
	\begin{equation}
		{\rm pr}_i:V_1\times V_2\to V_i,\ {\rm pr}_i(v_1,v_2) = v_i,\ \forall i = 1,2,
	\end{equation}
	hay tổng quát hơn với $n\in\mathbb{N}$, $n\ge2$:
	\begin{equation}
		{\rm pr}_i:\bigtimes_{i=1}^n V_i = V_1\times V_2\times\cdots\times V_n,\ {\rm pr}_i(v_1,\ldots,v_n) = v_i,\ \forall i = 1,\ldots,n.
	\end{equation}
\end{vidu}
See also, e.g., \href{https://en.wikipedia.org/wiki/Linear_map}{Wikipedia{\tt/}linear map}.

Hạt nhân \& ảnh của 1 đồng cấu là 2 không gian vector đặc biệt quan trọng với việc khảo sát đồng cấu đó, see, e.g., \cite[Chap. 2, \S3: Hạt nhân \& ảnh của đồng cấu, pp. 110--116]{Hung_linear_algebra}.

\begin{dinhnghia}[Hạt nhân{\tt/}hạch \& ảnh của đồng cấu]
	Giả sử $f:V\to W$ là 1 đồng cấu.
	\item(a) $\operatorname{Ker}(f)\coloneqq f^{-1}(0) = \{x\in V|f(x) = 0\}\subset V$ được gọi là \emph{hạt nhân} (hay \emph{hạch}) của $f$. Số chiều của $\operatorname{Ker}(f)$ được gọi là \emph{số khuyết} của $f$.
	\item(b) $\operatorname{Im}(f)\coloneqq f(V) = \{f(x)|x\in V\}\subset W$ được gọi là \emph{ảnh} của $f$. Số chiều của $\operatorname{Im}(f)$ được gọi là \emph{hạng} của $f$ \& được ký hiệu là $\operatorname{rank}(f)$.
\end{dinhnghia}

\begin{dinhly}[Điều kiện cần \& đủ để 1 đồng cấu là 1 toàn cấu]
	Đồng cấu $f:V\to W$ là 1 toàn cấu $\Leftrightarrow$ $\operatorname{rank}(f) = \dim W$.
\end{dinhly}

\begin{dinhly}[Điều kiện cần \& đủ để 1 đồng cấu là 1 đơn cấu]
	Đối với đồng cấu $f:V\to W$ các điều kiện sau là tương đương:
	\item(i) $f$ là 1 đơn cấu.
	\item(ii) $\operatorname{Ker}(f) = \{0\}$.
	\item(iii) Ảnh bởi $f$ của mỗi hệ vector độc lập tuyến tính là 1 hệ vector độc lập tuyến tính.
	\item(iv) Ảnh bởi $f$ của mỗi cơ sở của $V$ là 1 hệ vector độc lập tuyến tính.
	\item(v) Ảnh bởi $f$ của 1 cơ sở nào đó của $V$ là 1 hệ vector độc lập tuyến tính.
	\item(vi) $\operatorname{rank}(f) = \dim V$.
\end{dinhly}

\begin{baitoan}[VMC2023A1]
	Ký hiệu $\mathbb{R}[X]_{2023}$ là $\mathbb{R}$-không gian vector các đa thức 1 biến với bậc $\le2023$. Cho $f$ là ánh xạ đặt tương ứng mỗi đa thức với đạo hàm cấp 2 của nó: $f:\mathbb{R}[X]_{2023}\to\mathbb{R}[X]_{2023}$, $p(X)\mapsto p''(X)$. Đặt $g = f\circ f\circ\cdots\circ f$ (870 lần) là ánh xạ hợp của $870$ lần ánh xạ $f$. (a) Chứng minh $g$ là 1 ánh xạ tuyến tính từ $\mathbb{R}[X]_{2023}$ vào chính nó. (b) Tìm số chiều \& 1 cơ sở của không gian ảnh $\operatorname{Im}g$ \& của không gian hạt nhân $\operatorname{Ker}g$.
\end{baitoan}

\begin{proof}
	(a) Có $f(\alpha p(X) + \beta q(X)) = (\alpha p(X) + \beta q(X))'' = \alpha p''(X) + \beta q''(X) = \alpha f(p(X)) + \beta f(q(X))$, $\forall\alpha,\beta\in\mathbb{R}$, $\forall p(X),q(X)\in\mathbb{R}[X]_{2023}$, nên ánh xạ $f$ là ánh xạ tuyến tính, nên hợp thành của $n\in\mathbb{N}^\star$ lần của ánh xạ $f$, i.e., $f\circ f\circ\cdots\circ f$ ($n$ lần) cũng là 1 ánh xạ tuyến tính từ $\mathbb{R}[X]_{2023}$ vào chính nó. Nói riêng, $g$ là 1 ánh xạ tuyến tính từ $\mathbb{R}[X]_{2023}$ vào chính nó. (b) Ánh của $g$ được sinh bởi các vector $g(1),g(X),\ldots,g(X^{2023})$ (vì $(1,X,X^2,\ldots,X^{2023})$ là 1 cơ sở của khong gian vector $\mathbb{R}[X]_{2023}$ các đa thức $p(X)$ có $\deg p\le2023$. Nhận thấy
	\begin{equation*}
		g(X^k) = \left\{\begin{split}
			&0&&\mbox{if } k < 1740,\\
			&k(k - 1)\cdots(k - 1739)X^{k - 1740}&&\mbox{if } k\ge1740,
		\end{split}\right.
	\end{equation*}
	nên 1 cơ sở của $\operatorname{Im}g$ là $(1,X,X^2,\ldots,X^{283})$, nên $\dim\operatorname{Im}g = 284$.
	
	Với $p(X)\in\mathbb{R}[X]_{2023}$ bất kỳ, $p(X)$ sẽ có dạng $p(X) = \sum_{i=1}^{2023} a_iX^i = a_0 + a_1X + a_2X^2 + \cdots + a_{2023}X^{2023}$, thì $g(p)$ có dạng
	\begin{equation*}
		g(p)(X) = \sum_{i=1}^{283} b_iX^i = b_0 + b_1X + \cdots + b_{283}X^{283}.
	\end{equation*}
	Đa thức $p(X)\in\operatorname{ker}g\Leftrightarrow \sum_{i=1}^{283} b_iX^i = 0\Leftrightarrow a_i = 0,\forall i = 1740,\ldots,2023$, nên 1 cơ sở của $\operatorname{ker}g$ là $(1,X,X^2,\ldots,X^{1739})$ \& $\dim\operatorname{ker}g = 1740$.
\end{proof}

\begin{baitoan}[Mở rộng VMC2023A1]
	Liệu thay các giả thiết trong VMC2023A1 thì bài toán còn đúng{\tt/}giải được không? (a) Thay $2023,870$ bởi $n,m\in\mathbb{N}^\star$. (b) Thay ánh xạ đạo hàm cấp 2 bởi ánh xạ đạo hàm cấp $k\in\mathbb{N}^\star$ hoặc tích phân $\int {\rm d}x$, tích phân bội $k\in\mathbb{N}^\star$ $\int\int\cdots\int {\rm d}x$ ($k$ dấu tích phân).
\end{baitoan}

\begin{baitoan}
	Cho $n\in\mathbb{N}^\star$, $V$ là 1 không gian vector, $f:V\to V$ là 1 ánh xạ tuyến tính. Chứng minh $g_n\coloneqq f\circ f\circ\cdots f$ ($n$ lần) cũng là 1 ánh xạ tuyến tính từ $V$ vào chính nó.
\end{baitoan}

\begin{baitoan}[VMC2023A2]
	(a) 1 thành phố có 2 nhà máy: nhà máy điện (E) \& nhà máy nước (W). Để nhà máy (E) sản xuất điện thì nó cần nguyên liệu đầu vào là điện do chính nó sản xuất trước đó \& nước của nhà máy (W). Tương tự, để nhà máy (W) sản xuất nước thì nó cần đến nước do chính nó sản xuất cũng như điện của nhà máy (E). Cụ thể:
	\begin{itemize}
		\item Để sản xuất được lượng điện tương đương $1$ đồng, nhà máy (E) cần lượng điện tương đương $0.3$ đồng mà nó sản xuất được trước đó \& lượng nước tương đương $0.1$ đồng từ nhà máy (W);
		\item Để sản xuất được lượng nước tương đương $1$ đồng, nhà máy (W) cần lượng điện tương đương $0.2$ đồng từ nhà máy (E) \& lượng nước tương đương $0.4$ đồng do chính nó sản xuất trước đó.
	\end{itemize}
	Chính quyền thành phố yêu cầu 2 nhà máy trên cung cấp đến được với người dân lượng điện tương đương $12$ tỷ đồng \& lương nước tương đương $8$ tỷ đồng. Hỏi thực tế mỗi nhà máy cần sản xuất tổng cộng lượng điện \& lượng nước tương đương với bao nhiêu tỷ đồng để cung cấp đủ nhu cầu của người dân?
	
	(b) Cho $A = (a_{ij})_{2\times2}$ là ma trận thỏa mãn các phần tử đều là số thực không âm \& tổng các phần tử trên mỗi cột của $A$ đều $< 1$. Với ${\bf d} = (d_1,d_2)^\top$ là 1 vector tùy ý, chứng minh tồn tại duy nhất 1 vector cột ${\bf x} = (x_1,x_2)^\top$ sao cho ${\bf x} = A{\bf x} + {\bf d}$.
\end{baitoan}

\begin{baitoan}[VMC2023A3]
	Cho $\alpha,\beta,\gamma,\delta\in\mathbb{C}$ thỏa $x^4 - 2x^3 - 1 = (x - \alpha)(x - \beta)(x - \gamma)(x - \delta)$. (a) Chứng minh $\alpha,\beta,\gamma,\delta$ đôi một khác nhau. (b) Chứng minh $\alpha^3,\beta^3,\gamma^3,\delta^3$ đôi một khác nhau. (c) Tính $\alpha^3 + \beta^3 + \gamma^3 +\delta^3$. (d)${}^\star$ Mở rộng bài toán cho các đa thức khác.
\end{baitoan}

\begin{lemma}[Điều kiện cần \& đủ của nghiệm bội của đa thức]
	Cho $m,n\in\mathbb{R},m\le n$, $P(x)\in\mathbb{R}[x],\deg P = n$. $x = x_0\in\mathbb{R}$ là 1 nghiệm bội $m$ của $P(x)$ khi \& chỉ khi $P(x_0) = P'(x_0) = P''(x_0) = \cdots = P^{(m)}(x_0) = 0$.
\end{lemma}

\begin{proof}
	Giả sử $x = x_0\in\mathbb{R}$ là 1 nghiệm bội $m$ của $P(x)$, thì $P(x)$ sẽ có dạng $P(x) = (x - x_0)^mg(x)$ với $g(x)\in\mathbb{R}[x],\deg g = \deg P - m = n - m\ge0$. Tính các đạo hàm $P'(x),P''(x),\ldots,P^{(m)}(x)$ (có thể sử dụng quy tắc Leibniz tổng quát để tính đạo hàm, see, e.g., \href{https://en.wikipedia.org/wiki/General_Leibniz_rule}{Wikipedia{\tt/}general Leibniz rule}) để suy ra kết luận.
\end{proof}

\begin{proof}[Hint]
	(a) Đặt $P(x) = x^4 - 2x^3 - 1$, có $P'(x) = 4x^3 - 6x^2 = 2x^2(2x - 3$ chỉ có 2 nghiệm $x = 0$ (bội 2) \& $x = \frac{3}{2}$ (bội 1), mà $P(0) = -1\ne0,P(\frac{3}{2}) = -\frac{43}{16}\ne0$ nên $0,\frac{3}{2}$ đều không phải là nghiệm của $P(x)$, suy ra các nghiệm $\alpha,\beta,\gamma,\delta$ của $P(x)$ là phân biệt. (b) 
\end{proof}

\begin{baitoan}[VMC2023A4]
	Với mỗi ma trận vuông $A$ có phần tử là các số phức, định nghĩa
	\begin{equation}
		\sin A = \lim_{k\to\infty} \sum_{n=0}^k \frac{(-1)^n}{(2n + 1)!}A^{2n + 1}.
	\end{equation}
	(Ở đây ma trận giới hạn có phần tử là giới hạn của phần tử tương ứng của các ma trận tổng $S_k = \sum_{n=0}^k \frac{(-1)^n}{(2n + 1)!}A^{2n + 1}$ . Ma trận giới hạn này luôn tồn tại.) (a) Tìm các phần tử của ma trận $\sin A$ với
	\begin{equation}
		A = \begin{pmatrix}
			1 & -1\\0 & 2
		\end{pmatrix}
	\end{equation}
	(b) Cho $x,y\in\mathbb{R}$ bất kỳ, tìm các phần tử của ma trận $\sin A$ với
	\begin{equation}
		A = \begin{pmatrix}
			x & y\\0 & x
		\end{pmatrix}
	\end{equation}
	theo $x,y$. (c) Tồn tại hay không 1 ma trận vuông $A$ cấp 2 với phần tử là các số thực sao cho
	\begin{equation}
		\sin A = \begin{pmatrix}
			1 & 2023\\0 & 1
		\end{pmatrix}?
	\end{equation}
\end{baitoan}

\begin{baitoan}[VMC2023A5]
	Ký hiệu $P_n$ là tập hợp tất cả các ma trận khả nghịch $A$ cấp $n$ sao cho các phần tử của $A$ \& $A^{-1}$ đều bằng $0$ hoặc $1$. (a) Với $n = 3$, tìm tất cả các ma trận thuộc $P_3$. (b) Tính số phần tử của $P_n$ với $n\in\mathbb{N}^\star$ tùy ý.
\end{baitoan}

\begin{proof}
	(a) Đặt $A = (a_{ij})_{3\times3},A^{-1} = (b_{ij})_{3\times3}$, kết hợp với $A,A_{-1}$ đều khả nghịch, có mỗi hàng \& mỗi cột đều có ít nhất 1 số 1. Có $1 = a_{k1}b_{1k} + a_{k2}b_{2k} + a_{k3}b_{3k}$ với $k = 1,2,3$, nên tồn tại duy nhất $m\in\{1,2,3\}$ để $a_{km} = b_{mk} = 1$.
\end{proof}

%------------------------------------------------------------------------------%

\section{Analysis -- Giải Tích}

\subsection{Sequence -- Dãy số}
\textbf{\textsf{Resources -- Tài nguyên.}}
\begin{enumerate}
	\item \cite{Khai_so_hoc_day_so}. {\sc Phan Huy Khải}. {\it Các Chuyên Đề Số Học Bồi Dưỡng Học Sinh Giỏi Toán Trung Học. Chuyên Đề 2: Số Học \& Dãy Số}.
\end{enumerate}

\begin{baitoan}[General recursive sequences -- Dãy truy hồi tổng quát]
	Cho dãy số $(u_n)_{n=1}^\infty$ được xác định bởi công thức truy hồi
	\begin{equation}
		\boxed{u_n = f(u_{n-1},u_{n-2},\ldots,u_{n-m}),\ \forall m,n\in\mathbb{N}^\star,\ m < n.}
	\end{equation}
	Tìm các tính chất tổng quát của dãy theo 1 số dạng đặc biệt của hàm $f$ để lập thành các mệnh đề \& định lý, rồi chứng minh chúng.
\end{baitoan}
{\sf Vài phương pháp phổ biến để giải bài toán dãy số.}
\begin{itemize}
	\item Tìm cách xác định công thức số hạng tổng quát của dãy số: Thử vài trường hợp đầu để dự đoán công thức chính xác rồi chứng minh bằng quy nạp toán học.
	\item Sử dụng phương trình đặc trưng của lý thuyết dãy số.
\end{itemize}

\begin{baitoan}[VMC2023B]
	Cho $(u_n)_{n=1}^\infty$ là dãy số được xác định bởi $u_n = \prod_{k=1}^n \left(1 + \frac{1}{4^k}\right)$, $\forall n\in\mathbb{N}^\star$. (a) Tìm tất cả $n\in\mathbb{N}^\star$ thỏa $u_n > \frac{5}{4}$. (b) Chứng minh $u_n\le2023$, $\forall n\in\mathbb{N}^\star$. (c) Chứng minh dãy số $(u_n)_{n=1}^\infty$ hội tụ.
\end{baitoan}

\begin{proof}
	(a) $u_{n+1} = \left(1 + \frac{1}{4^{n+1}}\right)u_n > u_n$, $\forall n\in\mathbb{N}^\star$, suy ra $(u_n)$ đơn điệu tăng, mà $u_1 = \frac{5}{4}$ nên $u_n > \frac{5}{4}\Leftrightarrow n\ge2$. (b)
\end{proof}

\begin{remark}
	Gặp phải dãy số $(u_n)_{n=1}^\infty$ có công thức mỗi số hạng là 1 tích thì thử tính $\frac{u_{n+1}}{u_n}$ xem có đơn giản hóa được không. Gặp phải dãy số $(u_n)_{n=1}^\infty$ có công thức mỗi số hạng là 1 tổng thì thử tính $u_{n+1} - u_n$ xem có đơn giản hóa được không.
\end{remark}

\begin{baitoan}[Recursive sequence vs. ANN]
	Tìm mối liên hệ giữa các dãy số cho bởi công thức truy hồi (recursive sequences) \& mạng lưới nơ-ron nhân tạo (artificial neural networks, abbr., ANNs).
\end{baitoan}

%------------------------------------------------------------------------------%

\subsection{Integral -- Tích phân}

%------------------------------------------------------------------------------%

\section{Miscellaneous}

\subsection{Contributors}

\begin{enumerate}
	\item {\sc Phan Vĩnh Tiến}: \url{https://github.com/vinhtienlovemath/PublicDocuments/tree/main/MathematicalOlympiad}.
\end{enumerate}

%------------------------------------------------------------------------------%

\printbibliography[heading=bibintoc]
	
\end{document}