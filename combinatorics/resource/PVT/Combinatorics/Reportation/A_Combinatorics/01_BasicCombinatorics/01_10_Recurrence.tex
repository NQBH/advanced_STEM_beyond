\subsection{Quan hệ hồi quy}

\begin{tcolorbox}[breakable]
    \begin{baitoan}[\cite{shahriari2021invitation}, P1.3.3, p. 22]\label{pb:w08:34}
        Giả sử $h_n$ biểu thị số cách phủ mảng $2\times n$ bằng $1\times 2$ domino. Tìm $h_1,h_2,h_3$, \& 1 quan hệ đệ quy cho $h_n$. Sử dụng chúng để tìm $h_8$.
    \end{baitoan}
\end{tcolorbox}

\textbf{Lời giải. }Với $n = 1$ có 1 cách nên $h_1=1$. Với $n=2$ có 2 cách (đặt 2 miếng domino cùng dọc, hoặc cùng ngang). Với $n\geq 2$, nếu ô đầu tiên là domino dọc thì còn lại mảng $2 \times (n-1)$ nên có $h_{n-1}$ cách; nếu ô đầu tiên là domino ngang thì cũng phải có 1 domino ngang đặt song song bên dưới, khi đó còn lại mảng $2\times (n-2)$ nên có $h_{n-2}$ cách. Như vậy $h_n = h_{n-1} + h_{n-2}$ với mọi $n \geq 2$.

\begin{tcolorbox}[breakable]
    \begin{baitoan}[\cite{shahriari2021invitation}, P1.3.8, p. 23]\label{pb:w08:35}
        Bạn làm việc tại 1 đại lý ô tô bán $3$ mẫu xe: Một xe bán tải, 1 xe SUV, \& 1 xe hybrid nhỏ gọn. Công việc của bạn là đỗ những chiếc xe này thành 1 hàng. Xe bán tải \& xe SUV chiếm $2$ chỗ trong khi xe hybrid chiếm $1$ chỗ. Giả sử $n\in\mathbb{N}^\star$ \& giả sử $f(n)$ là số cách sắp xếp xe trong đúng $n$ chỗ. Tìm 1 hệ thức đệ quy cho $f(n)$ \& sử dụng nó để tìm $f(10)$, số cách sắp xếp xe nếu bạn có $10$ chỗ đỗ xe.
    \end{baitoan}
\end{tcolorbox}

\textbf{Lời giải. }Với $n=1$ thì có 1 cách để sắp xếp xe (1 xe hybrid) nên $f(1) = 1$. Với $n=2$ thì có 3 cách sắp xếp xe (2 xe hybrid, 1 xe bán tải, 1 xe SUV). Với $n \ge 2$, nếu xe đầu hàng là xe hybrid thì còn $n-1$ chỗ nên có $f(n-1)$ cách xếp, nếu xe đầu hàng là xe bán tải hoặc xe SUV thì còn $n-2$ chỗ nên có $f(n-2)$ cách xếp. Như vậy $f(n) = f(n-1) + 2f(n-2)$ với mọi $n \geq 2$.

\begin{tcolorbox}[breakable]
    \begin{baitoan}[\cite{shahriari2021invitation}, P1.3.10, pp. 23--24]\label{pb:w08:36}
        Tại 1 bữa tiệc tối trên tàu vũ trụ Enterprise, có $3$ dạng sống hiện diện: Con người, người Klingon, \& Romulan. Bàn ăn là 1 tấm ván dài $1\times n$, \& các dạng sống ngồi ở 1 phía của bàn, cạnh nhau. Từ mỗi dạng sống có $> n$ cá thể hiện diện, \& do đó chỉ có tổng cộng $n$ ngồi vào bàn. Vấn đề duy nhất là không có $2$ con người nào muốn ngồi cạnh nhau. Giả sử $h_n$ biểu thị số cách khác nhau để $n$ cá nhân có thể ngồi vào bàn ăn. Giả sử rằng tất cả con người đều giống nhau, giống như tất cả người Klingon \& tất cả người Romulan. (a) $h_1,h_2$ là gì? (b) Trong các câu sau, câu nào (nếu có) là đúng, \& câu nào là sai? (i) $h_n = 3h_{n-1} - h_{n-2}$. (ii) $h_n = 2h_{n-1} + 2h_{n-2}$. (iii) $h_n = 3h_{n-1} - (n - 1)!$. (iv) $h_n = h_{n-1} + 3h_{n-2} + 2h_{n-3}$. Đưa ra lý lẽ \& đầy đủ cho câu trả lời của bạn.
    \end{baitoan}
\end{tcolorbox}

\textbf{Lời giải. }Ký hiệu $H$ là con người, $K$ là Klingon và $R$ là Romulan. Với $n=1$, có 3 cách xếp (hoặc $H$, hoặc $K$, hoặc $R$) nên $h_1 = 3$. Với $n=2$ có 8 cách xếp ($HK,\,HR;\,KH,\,KK,\,KR;\,RH,\,RK,\,RR$) nên $h_2=8$. Với $n\ge 2$, nếu người đầu tiên không phải là con người (tức là Klingon hoặc Romulan), khi đó trong mỗi trường hợp thì $n-1$ chỗ ngồi còn lại sẽ được sắp xếp theo $h_{n-1}$ cách; nếu người đầu tiên là con người thì người tiếp theo phải là Klingon hoặc Romulan, khi đó trong mỗi trường hợp thì $n-2$ chỗ ngồi còn lại sẽ được sắp xếp theo $h_{n-2}$ cách. Như vậy $h_n = 2h_{n-1}+2h_{n-2}$ với mọi $n\ge 2$.

% \begin{tcolorbox}[breakable]
%     \begin{baitoan}[\cite{shahriari2021invitation}, P1.3.11, p. 24]\label{pb:w08:37}
%         Chỉ sử dụng các chữ số $1,2,3$, bạn có thể tạo ra bao nhiêu số nguyên sao cho tổng các chữ số bằng $9$? Ví dụ: $333,1323,2133,22221$.
%     \end{baitoan}
% \end{tcolorbox}

% \textbf{Lời giải. }

% \begin{tcolorbox}[breakable]
%     \begin{baitoan}[\cite{shahriari2021invitation}, P1.3.12, p. 24]\label{pb:w08:38}
%         Tôi có $12$ cây diên vĩ giống hệt nhau \& $4$ chậu hoa riêng biệt. Tất cả các cây diên vĩ đều được trồng, \& Tôi muốn trồng $2,3$ hoặc $4$ cây diên vĩ vào mỗi chậu. Tôi muốn tìm ra số cách thực hiện điều này. Đầu tiên, chúng ta khái quát hóa câu hỏi. Với $n\in\mathbb{N}^\star$ cây diên vĩ giống hệt nhau \& $k\in\mathbb{N}^\star$ chậu riêng biệt, hãy để $F(n,k)$ là số cách chúng ta có thể phân phối cây diên vĩ vào các chậu nếu mỗi chậu có $2,3$ hoặc $4$ cây diên vĩ. (a) Tìm $F(5,2)$. (b) Tìm 1 hệ thức đệ quy cho $F(n,k)$. Giải thích lý luận của bạn. (c) Tìm $F(n,k)$ cho các giá trị $n,k$ đã cho trong 1 bảng. (d) $F(12,4)$ là gì? Tại sao?
%     \end{baitoan}
% \end{tcolorbox}

% \textbf{Lời giải. }

% \begin{tcolorbox}[breakable]
%     \begin{baitoan}[\cite{shahriari2021invitation}, P1.3.15, pp. 25--26]\label{pb:w08:39}
%         Cho $h(0) = 1$ \& cho $h(n)$ là số dãy số $a_1,a_2,\ldots,a_n$, với các điều kiện là, đối với $i\in[n]$, mỗi $a_i\in\{0,4,7\}$, \& nếu cả $4$ \& $7$ xảy ra trong dãy số thì $4$ đầu tiên xảy ra trước $7$ đầu tiên. (a) Tìm $h(1),h(2)$. (b) Tìm 1 hệ thức đệ quy cho $h(n)$. (c) $h(5)$ là gì?
%     \end{baitoan}
% \end{tcolorbox}

% \textbf{Lời giải. }