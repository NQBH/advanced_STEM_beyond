\documentclass[oneside,11pt]{book}
\usepackage[backend=biber,natbib=true,style=authoryear]{biblatex}
\addbibresource{/home/nguyen/1_NQBH/reference/bib.bib}
\usepackage[utf8]{inputenc}
\usepackage{graphicx}
\usepackage[colorlinks=true,linkcolor=blue,urlcolor=red,citecolor=magenta]{hyperref}
\usepackage{amsmath,amssymb,amsthm,mathtools}
\allowdisplaybreaks
\usepackage{tcolorbox,multicol,enumitem}
\numberwithin{equation}{section}
\newtheorem{assumption}{Assumption}[section]
\newtheorem{lemma}{Lemma}[section]
\newtheorem{corollary}{Corollary}[section]
\newtheorem{definition}{Definition}[section]
\newtheorem{proposition}{Proposition}[section]
\newtheorem{theorem}{Theorem}[section]
\newtheorem{notation}{Notation}[section]
\newtheorem{remark}{Remark}[section]
\newtheorem{example}{Example}[section]
\newtheorem{question}{Question}[section]
\newtheorem{problem}{Problem}[section]
\newtheorem{conjecture}{Conjecture}[section]
\usepackage[margin=1in,left=1in,headheight=0.5\baselineskip]{geometry}
\usepackage{fancyhdr}
\pagestyle{fancy}
\fancyhf{}
\addtolength{\headheight}{0pt}% obsolete
\lhead{\scriptsize \chaptername~\thechapter}
\rhead{\scriptsize \nouppercase{\leftmark}} %\nouppercase !
\renewcommand{\chaptermark}[1]{\markboth{#1}{}}
\cfoot{\thepage}
\def\labelitemii{$\circ$}

\title{PhD Report\\PDE-Constrained Shape Optimization in Fluid Dynamics}
\author{Hong Quan Ba Nguyen}
\date{\today}

\begin{document}
\maketitle
\setcounter{secnumdepth}{6}
\tableofcontents

%------------------------------------------------------------------------------%

\chapter*{Nomenclature}

\section*{General notations}
\begin{itemize}
    \item $\mathbb{R}_{> 0}\coloneqq(0,\infty)$, $\mathbb{R}_{\ge 0}\coloneqq[0,\infty)$, $\mathbb{R}_{< 0}\coloneqq(-\infty,0)$, $\mathbb{R}_{\le 0}\coloneqq(-\infty,0]$.
    \item $\delta_{ij}$: Kronecker tensor, $\delta_{ij} = 1$ if $i = j$, $\delta_{ij} = 0$ if $i\ne j$.
    \item ${\bf u} = (u_1,\ldots,u_N)\in\mathbb{R}^N$: velocity vector field.\footnote{In the literature of fluid dynamics, it is common to denote by ${\bf v}$ the velocity vector field. But in this thesis, the author has to deal primarily with adjoint methods, so ${\bf v}$ is used to denote the \textit{adjoint velocity vector field} instead.}
    \item $p\in\mathbb{R}$: pressure scalar field.
    \item $\boldsymbol{\varepsilon}({\bf u})\coloneqq\frac{1}{2}(\nabla{\bf u} + (\nabla{\bf u})^\top)$: the \textit{symmetrized gradient} of a vector field.
    \item ${\bf u}\otimes{\bf u}\coloneqq(u_iu_j)_{i,j=1}^N$, which is a 2nd-order tensor.
    \item ${\rm Re}\coloneqq\frac{UL}{\nu}$: Reynolds number.
\end{itemize}

\begin{itemize}
    \item $D$: domain in $\mathbb{R}^d$ with piecewise smooth boundary $\partial D$
    \item $\Omega$: measurable set in $\mathbb{R}^d$ or in $D$, or domain of class $C^k$
    \item $\partial\Omega$: boundary of $\Omega$
    \item ${\bf n}$: unit outward normal vector field on $\Gamma$
    \item $\mathcal{N}_0$: unitary extension of ${\bf n}$ to an open neighborhood of $\Gamma$ in $\mathbb{R}^d$
    \item $\chi_\Omega$: characteristic function of $\Omega$
    \item $\Omega^c = D\backslash\overline{\Omega}$ (or $\mathbb{R}^d\backslash\overline{\Omega}$)
    \item $|\Omega|$: $d$-dimensional measure of $\Omega$
    \item $\kappa$: mean curvature of $\Gamma$
    \item $P_D(\Omega)$: perimeter of $\Omega$ in $D$
    \item $T_t$: transformation of $\mathbb{R}^d$ or of $\overline{D}$ into $\mathbb{R}^d$
    \item $DT_t$: Jacobian of $T_t$
    \item $dJ[V](\Omega)$: Eulerian derivative
    \item $\gamma_\Gamma$: trace operator on $\Gamma$, e.g. $\gamma_\Gamma\in\mathcal{L}(H^1(\Omega);H^{1/2}(\Gamma))$
    \item $\nabla_\Gamma$: tangential gradient on $\Gamma$
    \item $\partial_{\bf n}$: normal derivative on $\Gamma$
    \item ${\rm div}_\Gamma$: tangential divergence on $\Gamma$
    \item $\Delta_\Gamma$: Laplace-Beltrami operator on $\Gamma$
    \item $du[V](\Omega)$: material derivative of $u(\Omega)$ at $\Omega$ in the direction of the vector field $V$
    \item $u'[V](\Omega)$: local shape derivative of $u(\Omega)$ at $\Omega$ in the direction of the vector field $V$
\end{itemize}

\section*{Notations for matrix-vector operations}
Given $N\in\mathbb{Z}_{\ge 0}$, let ${\bf x} = (x_i)_{i=1}^N$ and ${\bf y} = (y_i)_{i=1}^N$ be two column vectors and $A = (a_{ij})_{i,j=1}^N$ and $B = (b_{ij})_{i,j=1}^N$ be two $N\times N$ matrices.
\begin{itemize}
    \item ${\bf x}^\top$, $A^\top$: transpose vector and transpose matrix of ${\bf x}$ and $A$, respectively.
    \item The \textit{dot product}\texttt{/}\textit{scalar product}\texttt{/}\textit{inner product} of 2 vectors is
    \begin{align*}
        {\bf x}\cdot{\bf y}\coloneqq\sum_{i=1}^N x_iy_i.
    \end{align*}
    \item The \textit{dyalic product} of 2 vectors of length $N$ is a $N\times N$ matrix:
    \begin{align*}
        {\bf x}\otimes{\bf y}\equiv{\bf x}{\bf y}^\top\coloneqq(x_iy_j)_{i,j=1}^N.
    \end{align*}
    \item Bilinear form:
    \begin{align*}
        {\bf x}^\top A{\bf y}\coloneqq\sum_{i=1}^N\sum_{j=1}^N x_ia_{ij}y_j,\ \forall{\bf x},{\bf y}\in\mathbb{R}^N.
    \end{align*}
    \item The \textit{dot product} of 2 matrices is a scalar
    \begin{align*}
        A:B\coloneqq\sum_{i,j=1}^N a_{ij}b_{ij}.
    \end{align*}
    \item For any $\phi:\mathbb{R}^N\to\mathbb{R}$, define
    \begin{align*}
        \partial_{\bf n}\phi\coloneqq\nabla\phi\cdot{\bf n} = {\bf n}\cdot\nabla\phi = \sum_{i=1}^N n_i\partial_{x_i}\phi.
    \end{align*}
    \item For any vector field $\boldsymbol{\phi}:\mathbb{R}^N\to\mathbb{R}^N$, define
    \begin{align*}
        \operatorname{div}\boldsymbol{\phi}&\coloneqq\nabla\cdot\boldsymbol{\phi}\coloneqq\sum_{i=1}^N \partial_{x_i}\phi_i,\\
        \nabla\boldsymbol{\phi}&\coloneqq(\partial_{x_i}\phi_j)_{i,j=1}^N,\ {\rm D}\boldsymbol{\phi}\coloneqq(\nabla\boldsymbol{\phi})^\top = (\partial_{x_j}\phi_i)_{i,j=1}^N,\\
        \boldsymbol{\phi}\cdot\nabla\boldsymbol{\phi}&\coloneqq(\boldsymbol{\phi}\cdot\nabla)\boldsymbol{\phi}\coloneqq\left(\sum_{j=1}^N \phi_j\partial_{x_j}\phi_i\right)_{i=1}^N\in\mathbb{R}^N,\\
        \partial_{\bf n}\boldsymbol{\phi}&\coloneqq{\bf n}\cdot\nabla\boldsymbol{\phi} = {\rm D}\boldsymbol{\phi}\cdot{\bf n} = \left(\sum_{i=1}^N n_i\partial_{x_i}\phi_j\right)_{j=1}^N\in\mathbb{R}^N,\\
        \boldsymbol{\varepsilon}_{\bf n}(\boldsymbol{\phi})&\coloneqq\boldsymbol{\varepsilon}(\boldsymbol{\phi})\cdot{\bf n} = \boldsymbol{\varepsilon}(\boldsymbol{\phi}){\bf n} = \left(\sum_{i=1}^N n_i\varepsilon_{ij}({\boldsymbol{\phi}})\right)_{j=1}^N = \left(\sum_{i=1}^N \frac{1}{2}n_i(\partial_{x_i}\phi_j + \partial_{x_j}\phi_i)\right)_{j=1}^N\in\mathbb{R}^N.
    \end{align*}
\end{itemize}

\begin{remark}[Convention]
    For clarity instead of brevity, we do not use Einstein summation convention and Levi-Civita tensor, which is fully characterized by $\varepsilon_{123} = 1$ and $\varepsilon_{ijk}$ is antisymmetric against the indices.
\end{remark}

\begin{remark}[Notation]
    In general, scalar quantities are denoted by small letters, vectors and vector-valued functions will be denoted in bold small letters, and matrices or tensors by capital letters, e.g. ${\bf u}(t,{\bf x})$, $p(t,{\bf x})$.
\end{remark}

\begin{remark}[Bilinear form]
    For any $A\in\operatorname{Mat}_N(\mathbb{R})$, the bilinear form
    \begin{align*}
        B_A:\mathbb{R}^N\times\mathbb{R}^N&\to\mathbb{R}\\
        ({\bf x},{\bf y})&\mapsto B_A({\bf x},{\bf y})\coloneqq{\bf x}^\top A{\bf y} = \sum_{i=1}^N\sum_{j=1}^N x_ia_{ij}y_j,
    \end{align*}
    can be rewritten in other representations as follows:
    \begin{align*}
        B_A({\bf x},{\bf y}) = {\bf x}^\top A{\bf y} = (A{\bf y})\cdot{\bf x} = (A^\top{\bf x})^\top{\bf y} = (A^\top{\bf x})\cdot{\bf y} = ({\bf x}\otimes{\bf y}):A = ({\bf x}{\bf y}^\top):A.
    \end{align*}
    As a consequence, when $A = \nabla{\bf u}$ for some vector ${\bf u}$, one has
    \begin{align*}
        {\bf n}^\top\nabla{\bf u}{\bf v} =((\nabla{\bf u})^\top{\bf n})\cdot{\bf v} = ({\rm D}{\bf u}{\bf n})\cdot{\bf v} = \partial_{\bf n}{\bf u}\cdot{\bf v}.
    \end{align*}
\end{remark}

\chapter*{Acronyms}
\begin{multicols}{2}
    \begin{enumerate}
        \item 1D\texttt{/}2D\texttt{/}3D: one-\texttt{/}two-\texttt{/}three-dimensional
        \item a.e.: almost everywhere
        \item a.a.: almost all
        \item BVP(s): boundary value problem(s)
        \item CFD: Computational Fluid Dynamics
        \item DNS: Direct Numerical Simulation
        \item e.g.: for example, for instance
        \item FD(M)(s): finite difference (method)(s)
        \item FE(M)(s): finite element (method)(s)
        \item FV(M)(s): finite volume (method)(s)
        \item LES: Large Eddy Simulation
        \item OSD(s): optimal shape design(s)
        \item ODE(s): ordinary differential equation(s)
        \item PDE(s): partial differential equation(s)
        \item SOP(s): shape optimization problem(s)
        \item s.t.: such that
        \item w.r.t.: with respect to
    \end{enumerate}
\end{multicols}

%------------------------------------------------------------------------------%

\chapter{Introductions}

\section{Short introduction to PDEs}

\subsection{Nonhomogeneous BVPs}
Let $\Omega$ be an open subset of $\mathbb{R}^N$, $N\in\mathbb{Z}_{\ge 0}$, with boundary $\Gamma\coloneqq\partial\Omega$. In $\Omega$ and on $\Gamma$, we introduce, respectively, \textit{differential operators} $P$ and $Q_i$, $0\le i\le n_{\rm bc}$, where $n_{\rm bc}$ is the number of boundary conditions.

In \cite{Lions_Magenes1972}, the authors defined the concept of \textit{nonhomogeneous boundary value problem} as a problem of the following type: let $f$ and $g_i$, $0\le i\le n_{\rm bc}$, be given in function spaces $F$ and $G_i$, $0\le i\le n_{\rm bc}$, respectively, where $F$ is a space defined ``on $\Omega$ and the $G_i$'s spaces are defined ``on $\Gamma$; we seek $u$ in a function space $\mathcal{U}$ ``on $\Omega$'' satisfying
\begin{equation}
    \label{general nonhomogeneous BVP}
    \tag{gnhBVP}
    \left\{\begin{split}
        Pu &= f,&&\mbox{ in }\Omega,\\
        Q_iu &= g_i,&&\mbox{ on }\Gamma,\ \forall i = 1,\ldots,n_{\rm bc}.
    \end{split}\right.
\end{equation}
The $Q_i$'s are called \textit{boundary operators} to distinguish these and the (domain) operator $P$.

Note that it can happen, which is the case of most evolutionary problems, that $Q_i\equiv 0$ on some part of $\Gamma$, so that the number $n_{\rm bc}$ of boundary conditions may depend on the part of $\Gamma$ considered.

\subsubsection{Weak formulations of nonhomogeneous BVPs}
Multiply both side of the first equation in \eqref{general nonhomogeneous BVP} with $v$ and then integrate by parts all the high-order terms to obtain the weak formulation of \eqref{general nonhomogeneous BVP}.

In the study of weak formulations, we recall the theorems of Stampacchia and Lax--Milgram (see, e.g., \cite[Sect. 5.3]{Brezis2011}, \cite[Subsect. I.2.2]{Temam2000}).

\begin{definition}[Continuous\texttt{/}coercive bilinear form]
    A bilinear form $a:H\times H\to\mathbb{R}$ is said to be
    \begin{itemize}
        \item[(i)] \emph{continuous} if there is a constant $C$ such that
        \begin{align*}
            |a(u,v)|\le C\|u\|_H\|v\|_H,\ \forall u,v\in H;
        \end{align*}
        \item[(ii)] \emph{coercive} if there is a constant $\alpha > 0$ such that
        \begin{align*}
            a(v,v)\ge\alpha\|v\|_H^2,\ \forall v\in H.
        \end{align*}
    \end{itemize}
\end{definition}

\begin{theorem}[Stampacchia]
    \label{Stampacchia theorem}
    Assume that $a(u,v)$ is a continuous coercive bilinear form on $H$. Let $K\subset H$ be a nonempty closed and convex subset. Then, given any $\phi\in H^\star$, there exists a unique element $u\in K$ such that
    \begin{align*}
        a(u,v - u)\ge\langle\phi,v - u\rangle_{H^\star,H},\ \forall v\in K.
    \end{align*}
    Moreover, if $a$ is symmetric, then $u$ is characterized by the property
    \begin{align*}
        u\in K,\ \frac{1}{2}a(u,u) - \langle\phi,u\rangle_{H^\star,H} = \min_{v\in K} \frac{1}{2}a(v,v) - \langle\phi,v\rangle_{H^\star,H}.
    \end{align*}
\end{theorem}

\begin{proof}
    See, e.g., \cite[pp. 138--140]{Brezis2011}.
\end{proof}
Applying Theorem \ref{Stampacchia theorem} with $K = H$ yields the following corollary.

\begin{theorem}[Lax-Milgram]
    \label{theorem: Lax-Milgram}
    Assume that $a(u,v)$ is a continuous coercive bilinear form on $H$. Then, given any $\phi\in H^\star$, there exists a unique element $u\in H$ such that
    \begin{align}
        \label{Lax-Milgram: Euler equation}
        a(u,v) = \langle\phi,v\rangle_{H^\star,H},\ \forall v\in H.
    \end{align}
    Moreover, if $a$ is symmetric, then $u$ is characterized by the property
    \begin{align}
        \label{Lax-Milgram: minimization problem}
        u\in H,\ \frac{1}{2}a(u,u) - \langle\phi,u\rangle_{H^\star,H} = \min_{v\in H} \frac{1}{2}a(v,v) - \langle\phi,v\rangle_{H^\star,H}.
    \end{align}
\end{theorem}

\begin{remark}
    In the language of the \emph{calculus of variations}, \eqref{Lax-Milgram: Euler equation} is the \emph{Euler equation} $F'(u) = 0$ associated with the minimization problem \eqref{Lax-Milgram: minimization problem} with $F(v)\coloneqq\frac{1}{2}a(v,v) - \langle\phi,v\rangle$.
\end{remark}

\subsubsection{Unbounded linear operators and adjoint}
We recall the following definitions from \cite[Sect. 2.6]{Brezis2011}.
\begin{definition}[Unbounded linear operator]
    Let $E$ and $F$ be two Banach spaces. An \emph{unbounded linear operator} from $E$ into $F$ is a linear map $A:D(A)\subset E\to F$ defined on a linear subspace $D(A)\subset E$ with values in $F$. The set $D(A)$ is called the \emph{domain} of $A$.
    
    One says that $A$ is \emph{bounded} (or \emph{continuous}) if $D(A) = E$ and if there is a constant $c\ge 0$ such that
    \begin{align*}
        \|Au\|\le c\|u\|,\ \forall u\in E.
    \end{align*}
    The norm of a bounded operator is defined by
    \begin{align*}
        \|A\|_{\mathcal{L}(E,F)}\coloneqq\sup_{u\ne 0} \frac{\|Au\|}{\|u\|}.
    \end{align*}
\end{definition}

\begin{definition}[Adjoint]
    Let $A:D(A)\subset E\to F$ be an unbounded linear operator that is \emph{densely defined}. We define an unbounded operator $A^\star:D(A^\star)\subset F^\star\to E^\star$ as follows. One defines its domain:
    \begin{align*}
        D(A^\star)\coloneqq\{v\in F^\star;\exists c\ge 0\mbox{ such that }|\langle v,Au\rangle|\le c\|u\|,\ \forall u\in D(A)\},
    \end{align*}
    which is clearly is a linear subspace of $F^\star$. Given $v\in D(A^\star)$, consider the map $g:D(A)\to\mathbb{R}$ defined by
    \begin{align*}
        g(u) = \langle v,Au\rangle,\ \forall u\in D(A).
    \end{align*}
    Since $|g(u)|\le c\|u\|$, $\forall u\in D(A)$, applying Hahn--Banach theorem yields that there exists a linear map $f:E\to\mathbb{R}$ that extends $g$ and s.t. $|f(u)|\le c\|u\|$, $\forall u\in E$, and thus $f\in E^\star$. Note that the extension of $g$ is unique, since $D(A)$ is dense in $E$. Set $A^\star v = f$.
    
    The unbounded linear operator $A^\star:D(A^\star)\subset F^\star\to E^\star$ is called the \emph{adjoint} of $A$. Briefly, the fundamental relation between $A$ and $A^\star$ is given by
    \begin{align*}
        \langle v,Au\rangle_{F^\star,F} = \langle A^\star v,u\rangle_{E^\star,E},\ \forall u\in D(A),\ \forall v\in D(A^\star).
    \end{align*}
\end{definition}

\subsection{Abstract linear saddle point problems*}
See \cite[Chap. 3]{John2016}.

\section{Introduction to Shape Optimization}
\begin{align}
    \label{general SOP}
    \tag{sop}
    \min_{\Omega\in\mathcal{O}_{\rm ad}} J(U,\Omega)\mbox{ s.t. } E(U,\Omega) = 0.
\end{align}

\subsection{Cost\texttt{/}Objective functionals}
The problem of how to choose appropriate cost\texttt{/}objective functional to quantify the control objective is an important topic in the field of optimal control of PDEs. This functional, say $J$, depends on the \textit{state variables} $U$ and on the control parameters describing the shape of the domain, i.e., $J(U,\Omega)$.
\begin{itemize}
    \item Tracking-type functionals:
    \begin{align}
        \label{tracking-type functional}
        \tag{trk-func}J_{\rm trk}({\bf u},\tilde{\Omega})\coloneqq\int_{\tilde{\Omega}} |{\bf u} - {\bf u}_{\rm d}|^2{\rm d}{\bf x},\mbox{ for some }\tilde{\Omega}\subset\Omega,
    \end{align}
    where ${\bf u}_{\rm d}:\Omega\to\mathbb{R}^N$ is a given desired flow field which contains some expected features of the controlled flow field.
    \item Minimization of curl of the velocity field:
    \begin{align}
        \label{curl functional}
        \tag{curl-func}
        J_{\rm curl}({\bf u},\tilde{\Omega})\coloneqq\frac{1}{2}\int_{\tilde{\Omega}} |\nabla\times{\bf u}|^2{\rm d}{\bf x},\mbox{ for some }\tilde{\Omega}\subset\Omega.
    \end{align}
    \item See \cite{Hintermueller_Kunisch_Spasov_Volkwein2004}, A Galilean invariant cost functional:
    \begin{align*}
        J({\bf u},\tilde{\Omega})\coloneqq\int_{\tilde{\Omega}} \max\{0,\det\nabla{\bf u}\}{\rm d}{\bf x},\mbox{ for some }\tilde{\Omega}\subset\Omega.
    \end{align*}
    In \cite{Kasumba2010}, the following smoothed cost functional is introduced to treat the non-differentiability of the previous cost functional due to the $\max$ operation:
    \begin{equation*}
        J({\bf u},\tilde{\Omega})\coloneqq\int_{\tilde{\Omega}} g(\det\nabla{\bf u}){\rm d}{\bf x},\mbox{ with } g(t)\coloneqq\left\{\begin{split}
            &0,&&t\le 0,\\
            &\frac{t^3}{1 + t^2},&&t > 0.
        \end{split}\right.
    \end{equation*}
    Note that alternatives for the smoothing function $g$ can be chosen.
    \item In \cite{Kasumba2010}, the author considered the following generalized cost functional:
    \begin{align*}
        J(u,\Omega)\coloneqq\int_\Omega j(C_\gamma u),\mbox{ where } C_\gamma:u\mapsto Cu + \gamma,\ \gamma\in L^2(\Omega),
    \end{align*}
    Note that $C_\gamma$ is an \textit{affine operator}.
    \item In \cite{Ito_Kunisch_Peichl2006}, the authors consider
    \begin{align*}
        J(u,\Gamma)\coloneqq\frac{1}{2}\int_\Gamma u^2{\rm d}\Gamma,
    \end{align*}
    where $u = u(\Gamma)$ is a solution of the mixed BVP
    \begin{equation*}
        \left\{\begin{split}
            -\Delta u &= f,&&\mbox{ in }\Omega,\\
            u &= g_d,&&\mbox{ on }\Gamma_d,\\
            \partial_{\bf n}u &= g,&&\mbox{ on }\Gamma.
        \end{split}\right.
    \end{equation*}
\end{itemize}
In this thesis, we consider the following general cost functional:
\begin{align}
    \label{general cost functional for u, p}
    \tag{cost-func-NS}
    J({\bf u},p,\Omega)\coloneqq\int_\Omega J_\Omega({\bf x},{\bf u},\nabla{\bf u},p){\rm d}{\bf x} + \int_\Gamma J_\Gamma({\bf x},{\bf u},\nabla{\bf u},p,{\bf n},{\bf t}){\rm d}\Gamma.
\end{align}
To calculate the shape derivative of this cost functional, we consider the its perturbed analogue:
\begin{align}
    \label{perturbed general cost functional for u, p}
    \tag{ptb-cost-func-NS}
    J({\bf u}_t,p_t,\Omega_t)\coloneqq\int_{\Omega_t} J_\Omega({\bf x},{\bf u}_t,\nabla{\bf u}_t,p_t){\rm d}{\bf x} + \int_{\Gamma_t} J_\Gamma({\bf x},{\bf u}_t,\nabla{\bf u}_t,p_t,{\bf n}_t,{\bf t}_t){\rm d}\Gamma_t,
\end{align}
where $({\bf u}_t,p_t)$ denotes the strong\texttt{/}classical solution of \eqref{general stationary Stokes} on the perturbed domain $\Omega_t\coloneqq T_t(V)(\Omega)$.

\subsection{Shape sensitivities}

\section{Sensitivity analysis in shape optimization}
We use the material derivative approach. The following presentation will be formal, i.e. it is correct provided that all necessary variables are sufficiently smooth. Let $\Omega\subset\mathbb{R}^d$ be a bounded domain with Lipschitz boundary $\Gamma := \partial\Omega$. We introduce a one-parameter family of mappings $\{T_t\}$, $t\in[0,t_0]$, $T_t:\Omega\to\mathbb{R}^d$ such that $T_0 = {\rm id}$, where ${\rm id}$ is the identity mapping of $\mathbb{R}^d$. Denote
\begin{align}
    x_t := T_t(x),\ x\in\Omega,\ t\in[0,t_0].
\end{align}
The perturbed domain at time $t$ is given by $\Omega_t := T_t(\Omega)$, $t\in[0,t_0]$. We assume additionally that each $T_t$, $t\in[0,t_0]$, has to be a one-to-one transformation of $\Omega$ onto $\Omega_t$ such that
\begin{align}
    \label{interior boundary preserving}
    T_t({\rm int}\,\Omega) = {\rm int}\,\Omega_t,\ T_t(\partial\Omega) = \partial\Omega_t.
\end{align}
Denote also $\Gamma_t := \partial\Omega_t$. In the remaining of this text, we consider $\{T_t\}_t$ as a perturbation of the identity, i.e.
\begin{align}
    \label{perturbation of identity}
    T_t[V] := {\rm id} + tV,\ t > 0,
\end{align} 
where $V\in W^{1,\infty}(\Omega)^d$ is the so-called \textit{velocity field}.

\begin{lemma}
    For $t_0$ small enough, $T_t$ of the form \eqref{perturbation of identity} is a one-to-one mapping of $\Omega$ onto $\Omega_t$ satisfying \eqref{interior boundary preserving} and preserving the Lipschitz continuity of $\partial\Omega_t$.
\end{lemma}

The shape derivative of $J$ at $\Omega$ in the direction of the deformation field $V$ is defined as the limit
\begin{align}
    dJ(\Omega)V = \lim_{t\to 0} \frac{J(\Omega_t) - J(\Omega)}{t},
\end{align}
provided that it exists. If the mapping $V\mapsto dJ(\Omega)V$ is linear and continuous, then we say that $J(\Omega)$ is \emph{shape differentiable}.

Let $DT_t$ denote the Jacobian of $T_t$. The following lemma lists some formula concerning their differentiation.
\begin{lemma}
    The following formulas hold:
    \begin{itemize}
        \item[(i)] $DT_t|_{t=0} = {\rm id}$.
        \item[(ii)] $\left.\frac{d}{dt}T_t\right|_{t=0} = V$.
        \item[(iii)] $\left.\frac{d}{dt}DT_t\right|_{t=0} = DV$.
        \item[(iv)] $\left.\frac{d}{dt}(DT_t)^\top\right|_{t=0} = DV^\top$.
        \item[(v)] $\left.\frac{d}{dt}DT_t^{-1}\right|_{t=0} = -DV$.
        \item[(vi)] $\left.\frac{d}{dt}\det DT_t\right|_{t=0} = \nabla\cdot V$.
    \end{itemize}
\end{lemma}
We consider a state problem, say $(P)$ in a domain $\Omega\subset\mathbb{R}^d$. Let $(P_t)$, $t\in(0,t_0]$, be a family of problems related to $(P)$ but solved in $\Omega_t$ with $T_t$ given by \eqref{perturbation of identity}. Solution of $(P_t)$ will be denoted by $u_t$ defined on $\Omega_t$.

\begin{definition}[Material derivative, local derivative]
    Let $u_t$ solve $(P_t)$ on the perturbed domain $\Omega_t := T_t[V](\Omega)$ and let $x_t := T_t[V](x)$ be the shifted point of a point $x\in\Omega$. Then material derivative is defined by
    \begin{align}
        du[V](x) := \left.\frac{d}{dt}\right|_{t=0}u_t(x_t),
    \end{align}
    and the \emph{local shape derivative} is defined by
    \begin{align}
        u'[V](x) := \left.\frac{d}{dt}\right|_{t=0}u_t(x).
    \end{align}
\end{definition}

\begin{lemma}
    \label{chain rule shape derivative}
    The chain rule allows the material and local shape derivatives to be converted into each other:
    \begin{align}
        \label{Schmidt2020 (4.12)}
        du[V] = u'[V] + \nabla u\cdot V.
    \end{align}
\end{lemma}

\begin{theorem}[Hadamard]
    For $f\in C(\overline{\Omega})$ the directional shape derivative of the domain integral $\int_\Omega f(x){\rm d}x$ is the boundary integral
    \begin{align}
        d\left(\int_\Omega f(x){\rm d}x\right)[V] = \int_\Gamma \langle V,{\bf n}\rangle f(x){\rm d}s.
    \end{align}
\end{theorem}

\section{Shape calculus}
Consider the following shape functional, with $f\in L^2(D)$, $j:D\times\mathbb{R}\to\mathbb{R}$:
\begin{align}
    \label{shape functional}
    J(\Omega) := \int_\Omega j(x,u(x)){\rm d}x,
\end{align}
where $u:\Omega\to\mathbb{R}$ satisfies \eqref{Poisson}.

We now want to calculate the shape derivative of \eqref{shape functional}. For this purpose, consider the \textit{perturbed cost functional}
\begin{align}
    \label{perturbed cost functioanl}
    J(\Omega_t) := \int_{\Omega_t} j(x_t,u_t(x_t)){\rm d}x_t,
\end{align}
where $u_t$ denotes the weak solution of \eqref{Poisson} on the domain $\Omega_t[V]$, i.e., $u_t\in H_0^1(\Omega_t[V])$ solves
\begin{align}
    \label{perturbed variational formulation}
    (\nabla u_t,\nabla v)_{L^2(\Omega_t[V])} = (f,v)_{L^2(\Omega_t[V])},\ \forall v\in H_0^1(\Omega_t[V]).
\end{align}
By Theorem \ref{theorem existence}, there exists a unique weak solution $u_t\in H_0^1(\Omega_t[V])$ of the perturbed BVP:
\begin{equation}
    \label{perturbed Poisson}
    \left\{\begin{split}
        -\Delta u_t &= f &&\mbox{ in } \Omega_t,\\
        u_t &= 0 &&\mbox{ on } \Gamma_t,
    \end{split}\right.
\end{equation}
which satisfies
\begin{align}
    \|u_t\|_{H_0^1(\Omega_t[V])}\le C\|f\|_{L^2(\Omega_t[V])}.
\end{align}
By Theorem \ref{theorem higher regularity}, with the assumption that $\Omega_t$ has $C^2$-boundary, in particular we assume $\partial\Omega$ is $C^2$ and $V\in C^2(\overline{D},\mathbb{R}^d)$, the weak solution $u_t\in H_0^1(\Omega_t[V])$ of the problem \eqref{perturbed Poisson} satisfies $u_t\in H^2(\Omega_t[V])$ and
\begin{align}
    \|u_t\|_{H^2(\Omega_t[V])}\le C\|f\|_{L^2(\Omega_t[V])}.
\end{align}

%\begin{definition}[Local shape derivative]
%    The \emph{local shape derivative} of $u$ in the direction $V\in C^{0,1}(\overline{D},\mathbb{R}^d)$ is defined pointwisely by
%    \begin{align}
%    u'[V](x) := \lim_{t\downarrow 0} \frac{u_t[V](x) - u(x)}{t},\ \forall x\in\left(\bigcap_{t\in[0,t_0]} \Omega_t\right),
%    \end{align}
%    where $t_0 > 0$ is sufficiently small such that the mapping $x\mapsto x + tV(x)$ is invertible for all $t\in[0,t_0]$.
%\end{definition}
%
%\begin{definition}[Material derivative]
%    The \emph{material derivative} of $u$ in the direction $V\in C^{0,1}(\overline{D},\mathbb{R}^d)$ is defined pointwisely by
%    \begin{align}
%    du[V](x) := \lim_{t\downarrow 0} \frac{u_t(T_t[V](x)) - u(x)}{t},\ \forall x\in\Omega.
%    \end{align}
%\end{definition}

By Lemma \ref{chain rule shape derivative}, this yields
\begin{align}
    dJ(\Omega)[V] = \int_\Gamma \langle V,{\bf n}\rangle j(x,g){\rm d}s + \int_\Omega \frac{\partial j}{\partial u}(x,u)du[V]{\rm d}x.
\end{align}
We have
\begin{equation*}
    \left\{\begin{split}
        \Delta u'[V] &= 0 &&\mbox{ in } \Omega,\\
        u'[V] &= -\langle V,{\bf n}\rangle\partial_{\bf n}u &&\mbox{ on } \Gamma.
    \end{split}\right.
\end{equation*}
The 1st equation is deduced from
\begin{align}
    \Delta u'(x)[V] = \lim_{t\to 0} \frac{\Delta u_t[V](x) - \Delta u(x)}{t} = 0,
\end{align}
since both $u_t[V]$ and $u$ satisfy Poisson equation.

For the 2nd boundary condition, apply the following lemma:

\begin{lemma}
    \label{Schmidt2020 Lemma 4.40}
    Let $u$ satisfy
    \begin{align}
        u = u_b \mbox{ on } \partial\Omega,
    \end{align}
    where $u_b$ does not depend on the geometry of $\Omega$, i.e., independent of e.g. the outer normal.
    
    The local shape derivative under a displacement field $V$ is then given by the solution of
    \begin{align}
        u'[V] &= \langle V,{\bf n}\rangle\partial_{\bf n}(u_b - u) \mbox{ on } \Gamma,\\
        u'[V] &= 0 \mbox{ on } \partial\Omega\backslash\Gamma
    \end{align}
    where $\Gamma$ here is only the variable part of the boundary $\partial\Omega$.
\end{lemma}

\begin{proof}
    If the material derivative is taken from both sides of the Dirichlet boundary condition, you get
    \begin{align}
        du[V] = du_b[V] \mbox{ on } \Gamma.
    \end{align}
    By 
    \begin{align}
        u'[V] + \langle\nabla u,V\rangle = du[V] = du_b[V] = \langle\nabla u_b,V\rangle,
    \end{align}
    since $u_b$ does not depend on the geometry and thus $u_b'[V] = 0$. Thus $u'[V] = \langle\nabla(u_b - u),V\rangle$. Plugging in $\widetilde{V} := \langle V,{\bf n}\rangle{\bf n}$, we get
    \begin{align}
        u'[V] = \langle V,{\bf n}\rangle\frac{\partial(u_b - u)}{\partial{\bf n}},
    \end{align}
    which corresponds to the statement.
\end{proof}
Introduce the \textit{adjoint state function} $v$ according to
\begin{equation}
    \left\{\begin{split}
        -\Delta v &= \frac{\partial j}{\partial u}(\cdot,u) &&\mbox{ in } \Omega,\\
        v &= 0 &&\mbox{ on } \Gamma,
    \end{split}\right.
\end{equation}
apply Green's 2nd formula
\begin{align}
    dJ(\Omega)[V] &= \int_\Gamma \langle V,{\bf n}\rangle j(x,g){\rm d}s - \int_\Omega v\Delta u'[V](x){\rm d}x + \int_\Gamma \left(u'[V]\partial_{\bf n}v - v\frac{\partial u'[V]}{\partial{\bf n}}\right){\rm d}s\\
    &= \int_\Gamma \langle V,{\bf n}\rangle j(x,g){\rm d}s + \int_\Gamma u'[V]\partial_{\bf n}v{\rm d}s,
\end{align}
to derive the \textit{boundary integral representation of the shape derivative}
\begin{align}
    dJ(\Omega)[V] = \int_\Gamma \langle V,{\bf n}\rangle\left(j(x,g) - \partial_{\bf n}u\partial_{\bf n}v\right){\rm d}s.
\end{align}
Apply this formula for $j(x,g)  = \frac{1}{2}(u(x) - z(x))^2$, we obtain
\begin{align}
    dJ(\Omega)[V] = \int_\Gamma \langle V,{\bf n}\rangle\left(\frac{1}{2}z^2 - \partial_{\bf n}u\partial_{\bf n}v\right){\rm d}s.
\end{align}

\begin{lemma}[Shape derivative formulas]
    For $\tau_0$ sufficiently small, $f\in C([0,\tau_0],W^{1,1}(D))$, and assume that $f_t(0)$ exists in $L^1(\Omega)$. Then
    \begin{align}
        \label{shape derivative volume integral}
        \frac{d}{dt}\int_{\Omega_t} f(t,x_t){\rm d}x_t|_{t=0} = \int_\Omega f_t(0,x){\rm d}x + \int_\Gamma f(0,x)V\cdot{\bf n}{\rm d}s
    \end{align}
    Moreover, if $f\in C([0,\tau_0],W^{2,1}(\Omega))$, then
    \begin{align}
        \label{shape derivative boundary integral}
        \frac{d}{dt}\int_{\Gamma_t} f(t,s_t){\rm d}s_t|_{t=0} = \int_\Gamma f_t(0,s)ds + \int_\Gamma (\partial_{\bf n}f(0,s) + \kappa f(0,s))V\cdot{\bf n}{\rm d}s,
    \end{align}
    where $\kappa$ stands for the mean curvature of $\Gamma$.
\end{lemma}

\begin{definition}[Tangential divergence]
    \begin{itemize}
        \item[(i)] Let $\Omega$ be a given domain with the boundary $\Gamma$ of class $C^2$, and $V\in C^1(U;\mathbb{R}^d)$ be a vector field; $U$ is an open neighborhood of the manifold $\Gamma\subset\mathbb{R}^d$. We define the \emph{tangential divergence} of $V$ as
        \begin{align}
            \label{tangential div}
            {\rm div}_\Gamma V = (\nabla\cdot V - \langle DV\cdot{\bf n},{\bf n}\rangle_{\mathbb{R}^d})|_\Gamma\in C(U).
        \end{align}
        \item[(ii)] Let $\Omega$ be a bounded domain with the boundary of class $C^2$, and let $V\in C^1(\Gamma;\mathbb{R}^d)$ be a given vector field on $\Gamma$. The \emph{tangential divergence} of $V$ on $\Gamma$ is given by
        \begin{align}
            {\rm div}_\Gamma V = ({\rm div}\widetilde{V} - \langle D\widetilde{V}\cdot{\bf n},{\bf n}\rangle_{\mathbb{R}^d})|_\Gamma\in C(\Gamma),
        \end{align}
        where $\widetilde{V}$ is any $C^1$ extension of $V$ to an open neighborhood of $\Gamma\subset\mathbb{R}^d$.
    \end{itemize}
\end{definition}
See \cite{Sturm2015a, Sturm2015b} for a survey of various methods used to compute shape sensitivities.
\begin{enumerate}
    \item Rearrangement method, see  \cite{Ito_Kunisch_Peichl2008}
    \item An approach using a novel adjoint equation, see Sturm's WIAS preprint.
    \item Material\texttt{/}shape derivative method, also called chain-rule approach, see \cite{Sokolowski_Zolesio1992}.
    \item Min approach for energy cost functional, see \cite{Delfour2012}.
    \item Minimax approach, see \cite{Delfour_Zolesio1988a}.
    \item Penalization method, see \cite{Delfour_Zolesio1988b}.
    \item C\'ea's Lagrange method in \cite{Cea1986} is used to derive the formulas for the shape derivative, however, it does not prove the shape differentiability. See \cite{Pantz2005} for examples where C\'ea's method fails.
\end{enumerate}

\subsection{A $\min$ formulation for variational formulations}
Let $\Omega$ be a bounded open domain in $\mathbb{R}^N$ with a smooth boundary $\Gamma$.

\subsection{A $\min$ formulation for variational inequalities*}

\section{Shape Derivatives for Poisson Equation}

\subsection{Well-posedness of Poisson equation}
We consider the Poisson equation
\begin{equation}
    \label{Poisson}
    \left\{\begin{split}
        -\Delta u &= f &&\mbox{ in } \Omega,\\
        u &= 0 &&\mbox{ on } \Gamma.
    \end{split}\right.
\end{equation}
A function $y\in H_0^1(\Omega)$ is called a \emph{weak solution} of \eqref{Poisson} if it satisfies the \emph{weak formulation}
\begin{align}
    \label{variational formulation}
    (\nabla u,\nabla v)_{L^2(\Omega)} = (f,v)_{L^2(\Omega)},\ \forall v\in H_0^1(\Omega).
\end{align}

\begin{theorem}[Existence]
    \label{theorem existence}
    Assume $f\in L^2(\Omega)$, then there exists a unique weak solution $u\in H_0^1(\Omega)$ of \eqref{Poisson}. Moreover, $u$ satisfies
    \begin{align}
        \|u\|_{H_0^1(\Omega)}\le C\|f\|_{L^2(\Omega)},
    \end{align}
    where $C$ is a constant depending only on $\Omega$.
\end{theorem}

\begin{proof}
    Define for $u$ and $v$ in $H_0^1(\Omega)$ the bilinear and linear forms
    \begin{align}
        a(u,v) := (\nabla u,\nabla v)_{L^2(\Omega)},\ \ F(v) := (f,v)_{L^2(\Omega)},.
    \end{align}
    The boundedness and $H_0^1$-coercive of the bilinear form $a$ follow directly.
    %    \begin{align}
    %    |a(u,v)|&\le\|\nabla u\|_{L^2(\Omega)}\|\nabla v\|_{L^2(\Omega)} = \|u\|_{H_0^1(\Omega)}\|v\|_{H_0^1(\Omega)},\ \forall u,v\in H_0^1(\Omega),\\
    %    a(u,u) &= \|\nabla u\|_{L^2(\Omega)}^2 = \|u\|_{H_0^1(\Omega)}^2,\ \forall u\in H_0^1(\Omega).
    %    \end{align}
    The form $F$ is linear and $F\in (H_0^1(\Omega))^\star = H^{-1}(\Omega)$ due to Poincar\'e inequality:
    %    \begin{align}
    %    |F(v)|\le\|f\|_{L^2(\Omega)}\|v\|_{L^2(\Omega)}\le C\|f\|_{L^2(\Omega)}\|\nabla v\|_{L^2(\Omega)} = C\|f\|_{L^2(\Omega)}\|v\|_{H_0^1(\Omega)},\ \forall v\in H_0^1(\Omega).
    %    \end{align}
    \begin{align}
        |F(v)|\le C\|f\|_{L^2(\Omega)}\|v\|_{H_0^1(\Omega)},\ \forall v\in H_0^1(\Omega).
    \end{align}
    Using Lax-Milgram lemma, the variational equation \eqref{variational formulation} has a unique solution $u\in H_0^1(\Omega)$ which satisfies
    \begin{align}
        \|u\|_{H_0^1(\Omega)}\le\|F\|_{H^{-1}(\Omega)}\le C\|f\|_{L^2(\Omega)}.
    \end{align}
\end{proof}

\begin{theorem}[Higher regularity]
    \label{theorem higher regularity}
    Let $\Omega\subset\mathbb{R}^d$ be open, bounded with $C^2$-boundary. Then for any $f\in L^2(\Omega)$, the weak solution $u\in H_0^1(\Omega)$ of \eqref{Poisson} satisfies $u\in H^2(\Omega)$ and
    \begin{align}
        \|u\|_{H^2(\Omega)}\le C\|f\|_{L^2(\Omega)},
    \end{align}
    with $C > 0$ depending only on $\Omega$.
\end{theorem}

\begin{proof}
    By \cite{Alt2016},
    \begin{align}
        \|u\|_{H^2(\Omega)}\le C\left(\|u\|_{H^1(\Omega)} + \|f\|_{L^2(\Omega)}\right).
    \end{align}
    Applying the estimate from the previous theorem, we get
    \begin{align}
        \|u\|_{H^2(\Omega)}\le C\left(\|u\|_{H_0^1(\Omega)} + \|f\|_{L^2(\Omega)}\right)\le C\|f\|_{L^2(\Omega)}.
    \end{align}
\end{proof}

\subsection{Lagrangian}
Let $z\in H^1(\Omega)$, consider the minimization problem
\begin{equation}
    \label{minimization problem}
    \min_{(u,\Omega)} J(u,\Omega) := \frac{1}{2}\int_\Omega (u(x) - z(x))^2{\rm d}x \mbox{ under } \left\{\begin{split}
        -\Delta u &= f \mbox{ in } \Omega,\\
        u &= 0 \mbox{ on } \partial\Omega.
    \end{split}\right.
\end{equation}
The Lagrange function $\mathcal{L}:H_0^1(\Omega)\times H_0^1(\Omega)\times \{\Omega\subset D;\Omega \mbox{ measurable}\}\to\mathbb{R}$:
\begin{align}
    \label{Lagrangian}
    \mathcal{L}(u,v,\Omega) = J(u,\Omega) + (\nabla u,\nabla v)_{L^2(\Omega)} - (f,v)_{L^2(\Omega)}.
\end{align}

\subsection{Adjoint equation}
We compute the derivative of $\mathcal{L}$ w.r.t. $v$ in the direction $\delta v\in H_0^1(\Omega)$:
\begin{align}
    \mathcal{L}_v'(u,v,\Omega)(\delta v) = (\nabla u,\nabla\delta v)_{L^2(\Omega)} - (f,\delta v)_{L^2(\Omega)} = 0,\ \forall\delta v\in H_0^1(\Omega),
\end{align}
which gives the weak formulation of the state equation $-\Delta u = f$.

We compute the derivative of $\mathcal{L}$ w.r.t. $u$ in the direction $\delta u\in H_0^1(\Omega)$:
\begin{align}
    \mathcal{L}_u'(u,v,\Omega)(\delta u) = (\nabla\delta u,\nabla v)_{L^2(\Omega)} + (u - z,\delta u)_{L^2(\Omega)} = 0,\ \forall\delta u\in H_0^1(\Omega),
\end{align}
which gives the weak formulation of the adjoint equation $-\Delta v = -(u - z)$.
\begin{equation}
    \label{Schmidt2020 (4.23)}
    \left\{\begin{split}
        -\Delta v &= -(u - z) &\mbox{ in } \Omega,\\
        v &= 0 &\mbox{ on } \partial\Omega.
    \end{split}\right.
\end{equation}
Since RHS is in $H^1(\Omega)$, assume additionally $\partial\Omega$ is $C^3$, we get $v\in H^3(\Omega)$.

\subsubsection{Boundary representation with local derivatives.}
\begin{lemma}
    Let $z\in H^1(\Omega)$, $f\in L^2(\Omega)$, $\partial\Omega$ is $C^2$. The shape derivative of $J(\Omega)$ is given by
    \begin{align}
        \label{shape derivative: boundary representation}
        dJ(\Omega)[V] = \int_\Gamma \langle V,{\bf n}\rangle\left(\frac{1}{2}z^2 - \partial_{\bf n}u\partial_{\bf n}v\right){\rm d}s,
    \end{align}
\end{lemma}

\begin{proof}
    By Theorem \ref{theorem higher regularity}, $u,v\in H^2(\Omega)$. Then
    \begin{align}
        dJ(\Omega)[V] &= \int_\Gamma \langle V,{\bf n}\rangle\frac{1}{2}(u - z)^2{\rm d}s + \int_\Omega \left[\frac{1}{2}(u - z)^2\right]'[V]{\rm d}x\\
        &= \int_\Gamma \langle V,{\bf n}\rangle\frac{1}{2}z^2 + \int_\Omega (u - z)u'[V]{\rm d}x\\
        &= \int_\Gamma \langle V,{\bf n}\rangle\frac{1}{2}z^2 - \int_\Omega \Delta vu'[V]{\rm d}x\\
        &= \int_\Gamma \langle V,{\bf n}\rangle\frac{1}{2}z^2 - \int_\Omega v\Delta u'[V]{\rm d}x + \int_\Gamma \left(u'[V]\partial_{\bf n}v - v\frac{\partial u'[V]}{\partial{\bf n}}\right){\rm d}s\\
        &= \int_\Gamma \langle V,{\bf n}\rangle\left(\frac{1}{2}z^2 - \partial_{\bf n}u\partial_{\bf n}v\right){\rm d}s,
    \end{align}
    where we have used Lemma \ref{Schmidt2020 Lemma 4.40} in the last equality.
\end{proof}

\subsubsection{Volume representation with material derivatives}
\begin{lemma}[Material derivative and local derivatives]
    \label{Schmidt2020 Lemma 4.23}
    The following identities hold
    \begin{align}
        d(Du)[V] &= D(du[V]) - DuDV,\\
        d(\nabla\cdot u)[V] &= d({\rm tr}(Du))[V] = \nabla\cdot(du[V]) - {\rm tr}(DuDV),
    \end{align}
    and therefore also immediately if the standard scalar product is used
    \begin{align}
        \label{Schmidt2020 (4.13)}
        d(\nabla u)[V] = \nabla(du[V]) - (DV)^\top\nabla u
    \end{align}
    and with the product rule:
    \begin{align}
        \label{Schmidt2020 (4.14)}
        d\langle\nabla u,\nabla v\rangle[V] = \langle d\nabla u[V],\nabla v\rangle + \langle\nabla u,d\nabla v[V]\rangle = \langle\nabla du[V],\nabla v\rangle + \langle\nabla u,\nabla dv[V]\rangle - \langle\nabla u,(DV + DV^\top)\nabla v\rangle.
    \end{align}
\end{lemma}

\begin{proof}
    By Remark \ref{chain rule shape derivative}, one has
    \begin{align}
        d(Du)[V] &= (Du)'[V] + D(Du)V = D(u'[V]) + D(DuV) - DuDV\\
        &= D(u'[V] + DuV) - DuDV = D(du[V]) - DuDV.
    \end{align}
    The part for the div operator follows immediately, because material derivation and ${\rm tr}(\cdot)$ are interchangeable, because the trace operator contains no local derivatives.
\end{proof}

\begin{lemma}
    Let $z\in L^2(\Omega)$, $f\in L^2(\Omega)$, $\partial\Omega$ is $C^2$. The shape derivative of $J(\Omega)$ is given by
    \begin{align}
        \label{shape derivative: volume representation}
        dJ(u,v,\Omega)[V] = \int_\Omega \frac{1}{2}(u - z)^2\nabla\cdot V - (u - z)dz[V] - \langle\nabla u,(DV + DV^\top)\nabla v\rangle - vdf[V]{\rm d}x.
    \end{align} 
\end{lemma}

\begin{proof}
    We start again with the above Lagrange function again. The shape-directional derivative is given by
    \begin{align}
        \label{3.5}
        d\mathcal{L}(u,v,\Omega)[V] = \int_\Omega {\rm div} V\left(\frac{1}{2}(u - z)^2 + \langle\nabla u,\nabla v\rangle - vf\right){\rm d}x + \int_\Omega d\left(\frac{1}{2}(u - z)^2 + \langle\nabla u,\nabla v\rangle - vf\right){\rm d}x.
    \end{align}
    By Theorem \ref{theorem higher regularity}, $u\in H^2(\Omega)$. We now use Lemma \ref{Schmidt2020 Lemma 4.23} and in particular equation \eqref{Schmidt2020 (4.14)} to draw all material derivatives in the last integral:
    \begin{align}
        &\int_\Omega d\left(\frac{1}{2}(u - z)^2 + \langle\nabla u,\nabla v\rangle - vf\right){\rm d}x \nonumber\\
        &= \int_\Omega (u - z)(du[V] - dz[V]) + \langle\nabla du[V],\nabla v\rangle + \langle\nabla u,\nabla dv[V]\rangle - \langle\nabla u,(DV + DV^\top)\nabla v\rangle - dv[V]f - v df[V]{\rm d}x.\label{Schmidt2020 (4.26)}
    \end{align}
    The following equations hold due to weak formulations of the state and adjoint equations:
    \begin{align}
        \langle\nabla u,\nabla dv[V]\rangle - dv[V]f &= 0,\\
        \langle\nabla du[V],\nabla v\rangle + (u - z)du[V] &= 0,
    \end{align}
    Then \eqref{3.5} becomes the desired formula.
\end{proof}

\begin{remark}
    When comparing \eqref{shape derivative: boundary representation} directly with \eqref{shape derivative: volume representation} it is noticeable that in the formulation \eqref{shape derivative: volume representation} the trace of $u$ and $\xi$ is not needed at the boundary. The normal ${\bf n}$ is not needed either.
    
    The regularity requirements:
    \begin{itemize}
        \item[(i)] Boundary formulation \eqref{shape derivative: boundary representation}: $f\in H^1(\Omega)$, $\partial\Omega$ is $C^3$, $u,v\in H^3(\Omega)$,
        \item[(ii)] Volume formulation \eqref{shape derivative: volume representation}: $f\in L^2(\Omega)$, $\partial\Omega$ is $C^2$, $u,v\in H^2(\Omega)$.
    \end{itemize}
\end{remark}

\section{Introduction to Turbulence Models*}
For mathematical basis of turbulence modeling, see, e.g., \cite[Chap. 3]{Rebollo_Lewandowski2014}.

\subsection{Boundary conditions}
See, e.g., \cite{Gunzburger1989, John2016}. We define $\Gamma_{\rm v}^{\bf u}$ and $\Gamma_{\rm v}^p$ as the ``varying'' components w.r.t. ${\bf u}$ and $p$ of $\Gamma$, respectively, i.e.,
\begin{align*}
    \Gamma_{\rm nv}^{\bf u} &\coloneqq\left\{{\bf x}\in\Gamma;\left({\bf Q}({\bf x},{\bf u} + \tilde{\bf u},\nabla{\bf u} + \nabla\tilde{\bf u},p + \tilde{p},{\bf n},{\bf t}) = {\bf Q}({\bf x},{\bf u},\nabla{\bf u},p,{\bf n},{\bf t})\right)\Rightarrow\tilde{\bf u} = {\bf 0}\right\},\\
    \Gamma_{\rm nv}^p &\coloneqq\left\{{\bf x}\in\Gamma;\left({\bf Q}({\bf x},{\bf u} + \tilde{\bf u},\nabla{\bf u} + \nabla\tilde{\bf u},p + \tilde{p},{\bf n},{\bf t}) = {\bf Q}({\bf x},{\bf u},\nabla{\bf u},p,{\bf n},{\bf t})\right)\Rightarrow\tilde{p} = 0\right\},\\
    \Gamma_{\rm v}^{\bf u} &\coloneqq\Gamma\backslash\Gamma_{\rm nv}^{\bf u},\ \Gamma_{\rm v}^p\coloneqq\Gamma\backslash\Gamma_{\rm nv}^p.
\end{align*}
In the case when ${\bf Q}$ is linear w.r.t. ${\bf u}$ and $p$, i.e.,
\begin{enumerate}
    \item Dirichlet boundary condition:
    \begin{align}
        \label{Dirichlet BC}
        \tag{D-bc}
        {\bf u} = {\bf g},\mbox{ on }\Gamma_{\rm D}^{\bf u}.
    \end{align}
    In particular, ${\bf u } = {\bf 0}$ on solid walls.
    
    Since ${\bf u} = {\bf u} + \tilde{\bf u} = {\bf g}$ implies $\tilde{\bf u} = {\bf 0}$ for all ${\bf x}\in\Gamma_{\rm D}^{\bf u}$, one has $\Gamma_{\rm D}^{\bf u}\subset\Gamma_{\rm nv}^{\bf u}$.
    \item Neumann boundary condition:
    \begin{align}
        \label{Neumann BC}
        \tag{N-bc}
        \nu\partial_{\bf n}{\bf u} - p{\bf n} = {\bf 0},\mbox{ on }\Gamma_{\rm N}.
    \end{align}
    Since $\nu\partial_{\bf n}({\bf u} + \tilde{\bf u}) - (p + \tilde{p}){\bf n} = \nu\partial_{\bf n}{\bf u} - p{\bf n} = {\bf 0}$ implies $\nu\partial_{\bf n}\tilde{\bf u} - \tilde{p}{\bf n} = {\bf 0}$ only and $(\tilde{\bf u},\tilde{p})$ can still vary to satisfy this, one has $\Gamma_{\rm N}\subset\Gamma_{\rm v}^{\bf u}$ and $\Gamma_{\rm N}\subset\Gamma_{\rm v}^p$.
    \item Usually on outflow:
    \begin{align*}
        -p + \alpha\partial_{\bf n}{\bf u}\cdot{\bf n} = 0,\ \alpha\partial_{\bf n}{\bf u}\cdot{\bf t} = 0.
    \end{align*}
    \item Usually on other fixed boundaries of the domain:
    \begin{align*}
        {\bf u}\cdot{\bf n} = 0,\ \alpha\partial_{\bf n}{\bf u}\cdot{\bf t} = 0.
    \end{align*}
    \item On an artificial boundary, e.g., the exit of a canal, or a free surface, a \textit{no-friction condition}:
    \begin{align*}
        2\nu\boldsymbol{\varepsilon}({\bf u}){\bf n} - p{\bf n} = {\bf 0},
    \end{align*}
    may be useful, see, e.g., \cite{Mazya_Rossmann2005, Mazya_Rossmann2007, Mazya_Rossmann2009}.
    \item ${\bf u}_{\boldsymbol{\tau}} = {\bf h}$, $-p + 2\boldsymbol{\varepsilon}_{{\bf n},{\bf n}} = \phi$, where ${\bf u}_{\bf n} = {\bf u}\cdot{\bf n}$ denotes the \textit{normal} and ${\bf u}_{\boldsymbol{\tau}} = {\bf u} - {\bf u}_{\bf n}{\bf n}$ the \textit{tangential component} of ${\bf u}$, $\boldsymbol{\varepsilon}_{\bf n}$ is the vector $\boldsymbol{\varepsilon}({\bf u}){\bf n}$, $\boldsymbol{\varepsilon}_{{\bf n},{\bf n}}({\bf u})$ is the normal component and $\boldsymbol{\varepsilon}_{{\bf n},\boldsymbol{\tau}}({\bf u})$ the tangential component of $\boldsymbol{\varepsilon}_{\bf n}({\bf u})$.
    \item ${\bf u}_{\bf n} = h$, $\boldsymbol{\varepsilon}_{{\bf n},\boldsymbol{\tau}} = \phi$.
    \item $-p{\bf n} + 2\boldsymbol{\varepsilon}_{\bf n}({\bf u}) = \boldsymbol{\phi}$.
\end{enumerate}

%------------------------------------------------------------------------------%

\part{Shape Optimization for Navier-Stokes Equations}

\chapter{Shape Optimization for Stokes Equations}

\section{Stationary Stokes equations}
In this section, we first consider the following general stationary Stokes equations
\begin{equation}
    \label{general stationary Stokes}
    \tag{gS}
    \left\{\begin{split}
        -\operatorname{diff(\nu,{\bf u})} + \nabla p &= {\bf f}({\bf x},{\bf u},\nabla{\bf u},p),&&\mbox{ in }\Omega,\\
        \nabla\cdot{\bf u} &= f_{\rm div}({\bf x},{\bf u},\nabla{\bf u},p),&&\mbox{ in }\Omega,\\
        {\bf Q}({\bf x},{\bf u},\nabla{\bf u},p,{\bf n},{\bf t}) &= {\bf f}_{\rm bc}({\bf x}),&&\mbox{ on }\Gamma,
    \end{split}\right.
\end{equation}
where $\operatorname{diff}:\mathbb{R}_{> 0}\times\mathbb{R}^N\to\mathbb{R}^N$ is a \textit{diffusive operator} and can take one of the following forms: $\nabla\cdot(\nu\nabla{\bf u})$ or $\nabla\cdot(2\nu\boldsymbol{\varepsilon}({\bf u}))$ when the \textit{kinematic viscosity} $\nu$ depends on ${\bf x}$, and $\nu\Delta{\bf u}$ or $2\nu\nabla\cdot\boldsymbol{\varepsilon}({\bf u})$ when $\nu$ is a constant. Hence, we have the following variants:
\begin{equation}
    \label{general stationary Stokes/gradient}
    \tag{$\nabla$-gS}
    \left\{\begin{split}
        -\nabla\cdot(\nu\nabla{\bf u}) + \nabla p &= {\bf f}({\bf x},{\bf u},\nabla{\bf u},p),&&\mbox{ in }\Omega,\\
        \nabla\cdot{\bf u} &= f_{\rm div}({\bf x},{\bf u},\nabla{\bf u},p),&&\mbox{ in }\Omega,\\
        {\bf Q}({\bf x},{\bf u},\nabla{\bf u},p,{\bf n},{\bf t}) &= {\bf f}_{\rm bc}({\bf x}),&&\mbox{ on }\Gamma,
    \end{split}\right.
\end{equation}
and
\begin{equation}
    \label{general stationary Stokes/symmetrized gradient}
    \tag{$\boldsymbol{\varepsilon}$-gS}
    \left\{\begin{split}
        -\nabla\cdot(2\nu\boldsymbol{\varepsilon}({\bf u})) + \nabla p &= {\bf f}({\bf x},{\bf u},\nabla{\bf u},p),&&\mbox{ in }\Omega,\\
        \nabla\cdot{\bf u} &= f_{\rm div}({\bf x},{\bf u},\nabla{\bf u},p),&&\mbox{ in }\Omega,\\
        {\bf Q}({\bf x},{\bf u},\nabla{\bf u},p,{\bf n},{\bf t}) &= {\bf f}_{\rm bc}({\bf x}),&&\mbox{ on }\Gamma,
    \end{split}\right.
\end{equation}

\begin{remark}
    When $\nu$ is a constant, \eqref{general stationary Stokes/gradient} becomes
    \begin{equation}
        \label{stationary Stokes/gradient}
        \tag{$\nabla$-S}
        \left\{\begin{split}
            -\nu\Delta{\bf u} + \nabla p &= {\bf f}({\bf x},{\bf u},\nabla{\bf u},p),&&\mbox{ in }\Omega,\\
            \nabla\cdot{\bf u} &= f_{\rm div}({\bf x},{\bf u},\nabla{\bf u},p),&&\mbox{ in }\Omega,\\
            {\bf Q}({\bf x},{\bf u},\nabla{\bf u},p,{\bf n},{\bf t}) &= {\bf f}_{\rm bc}({\bf x}),&&\mbox{ on }\Gamma,
        \end{split}\right.
    \end{equation}
    which can be found commonly in the literature.
    
    If we set
    \begin{align*}
        P\coloneqq\begin{bmatrix}
            -\nu\Delta & \nabla\\ \operatorname{div} & 0
        \end{bmatrix},\ u\coloneqq\begin{bmatrix}
            {\bf u}\\ p
        \end{bmatrix},\ f\coloneqq\begin{bmatrix}
            {\bf f}\\ f_{\rm div}
        \end{bmatrix},
    \end{align*}
    then \eqref{instationary Stokes} can be rewritten in the form of the first equation in \eqref{general nonhomogeneous BVP}.
\end{remark}
We also need to specify a set of boundary conditions for \eqref{general stationary Stokes} via defining the \textit{boundary operators} $(Q_i)_{i=1}^{n_{\rm bc}}$ s.t.
\begin{align*}
    Q_i\begin{bmatrix}
        {\bf u}\\ p
    \end{bmatrix} = g_i,\mbox{ on }\Gamma,\ \forall i = 1,\ldots,n_{\rm bc}.
\end{align*}

\subsection{Boundary conditions for stationary Stokes equations}
See \cite[Chap. 4]{John2016}.

\subsection{Weak and very weak formulations for stationary Stokes equations}
We test both sides of the first equation of \eqref{general stationary Stokes} with a test function ${\bf v}:\mathbb{R}^N\to\mathbb{R}^N$ and those of the second one with a test function $q:\mathbb{R}^N\to\mathbb{R}$ over $\Omega$:
\begin{equation*}
    \left\{\begin{split}
        \int_\Omega \left(-\operatorname{diff}(\nu,{\bf u}) + \nabla p\right)\cdot{\bf v}{\rm d}{\bf x} &= \int_\Omega {\bf f}({\bf x},{\bf u},\nabla{\bf u},p)\cdot{\bf v}{\rm d}{\bf x},\\
        -\int_\Omega q\nabla\cdot{\bf u}{\rm d}{\bf x} &= \int_\Omega qf_{\rm div}({\bf x},{\bf u},\nabla{\bf u},p){\rm d}{\bf x},
    \end{split}\right.
\end{equation*}
and then integrate by parts:
\begin{itemize}
    \item \textbf{Case $\operatorname{diff}(\nu,{\bf u}) = \nabla\cdot(\nu\nabla{\bf u})$.} Use the corresponding formulas in Appendix \ref{ibp: stationary} to obtain
    \begin{align*}
        \int_\Omega \nu\nabla{\bf u}:\nabla{\bf v} - p\nabla\cdot{\bf v}{\rm d}{\bf x} + \int_\Gamma (p{\bf n} - \nu\partial_{\bf n}{\bf u})\cdot{\bf v}{\rm d}\Gamma = \int_\Omega {\bf f}({\bf x},{\bf u},\nabla{\bf u},p)\cdot{\bf v}{\rm d}{\bf x}.
    \end{align*} 
    \item \textbf{Case $\operatorname{diff}(\nu,{\bf u}) = \nabla\cdot(2\nu\boldsymbol{\varepsilon}({\bf u}))$.} Also use formulas in Appendix \ref{ibp: stationary} to obtain
    \begin{align*}
        \int_\Omega 2\nu\boldsymbol{\varepsilon}({\bf u}):\boldsymbol{\varepsilon}({\bf v}) - p\nabla\cdot{\bf v}{\rm d}{\bf x} + \int_\Gamma (p{\bf n} - 2\nu\boldsymbol{\varepsilon}_{\bf n}({\bf u}))\cdot{\bf v}{\rm d}\Gamma = \int_\Omega {\bf f}({\bf x},{\bf u},\nabla{\bf u},p)\cdot{\bf v}{\rm d}{\bf x}.
    \end{align*}
\end{itemize}

\subsection{Cost functions for \eqref{general stationary Stokes}}
We consider the general cost functional \eqref{general cost functional for u, p} for \eqref{general stationary Stokes}. Here are some examples from the literature: \texttt{[inserting$\ldots$]}

\subsection{Lagrangian \& extended Lagrangian for \eqref{general stationary Stokes}}
To derive the adjoint equations for \eqref{general stationary Stokes}, we first introduce the following \textit{Lagrangian} (see, e.g., \cite{Troltzsch2010}):
\begin{align}
    L({\bf u},p,\Omega,{\bf v},q)&\coloneqq J({\bf u},p,\Omega) - \int_\Omega (-\operatorname{diff}(\nu,{\bf u}) + \nabla p - {\bf f}({\bf x},{\bf u},\nabla{\bf u},p))\cdot{\bf v}\nonumber\\
    &\hspace{3.2cm}+ q(\nabla\cdot{\bf u} - f_{\rm div}({\bf x},{\bf u},\nabla{\bf u},p)){\rm d}{\bf x},
    \label{Lagrangian for general stationary Stokes}
    \tag{$L$-gS}
\end{align}
and the following \textit{extended Lagrangian}:
\begin{align}
    \mathcal{L}({\bf u},p,\Omega,{\bf v},q,{\bf v}_{\rm bc})\coloneqq&\,L({\bf u},p,\Omega,{\bf v},q) - \int_\Gamma ({\bf Q}({\bf x},{\bf u},\nabla{\bf u},p,{\bf n},{\bf t}) - {\bf f}_{\rm bc})\cdot{\bf v}_{\rm bc}{\rm d}\Gamma,
    \label{extended Lagrangian for general stationary Stokes}
    \tag{$\mathcal{L}$-gS}
\end{align}
where ${\bf v},q,{\bf u}_{\rm bc}$ are \textit{Lagrange multipliers}.

We also introduce the following ``mixed'' Lagrangian with a \textit{switching factor} $\delta_{\mathcal{L}}\in\{0,1\}$:
\begin{align}
    \label{mixed Lagrangian for general stationary Stokes}
    \tag{$L_{\mathcal{L}}$-gS}
    L_{\mathcal{L}}({\bf u},p,\Omega,{\bf v},q,{\bf v}_{\rm bc})\coloneqq L({\bf u},p,\Omega,{\bf v},q) - \delta_{\mathcal{L}}\int_\Gamma ({\bf Q}({\bf x},{\bf u},\nabla{\bf u},p,{\bf n},{\bf t}) - {\bf f}_{\rm bc})\cdot{\bf v}_{\rm bc}{\rm d}\Gamma.
\end{align}

\section{Shape optimization problems for \eqref{general stationary Stokes}}
Here are 3 different shape optimization problems associated with \eqref{general cost functional for u, p}, \eqref{Lagrangian for general stationary Stokes}, and \eqref{extended Lagrangian for general stationary Stokes}, respectively:
\begin{align*}
    &\min_{\Omega\in\mathcal{O}_{\rm ad}} J({\bf u},p,\Omega) \mbox{ s.t. } ({\bf u},p) \mbox{ solves \eqref{general stationary Stokes}},\\
    &\min_{\Omega\in\mathcal{O}_{\rm ad}} L({\bf u},p,\Omega,{\bf v},q) \mbox{ s.t. } ({\bf u},p)\mbox{ satisfies }{\bf Q}({\bf x},{\bf u},\nabla{\bf u},p,{\bf n},{\bf t}) = {\bf f}_{\rm bc}({\bf x})\mbox{ on }\Gamma,\\
    &\min_{\Omega\in\mathcal{O}_{\rm ad}} \mathcal{L}({\bf u},p,\Omega,{\bf v},q,{\bf v}_{\rm bc}) \mbox{ with } ({\bf u},p) \mbox{ unconstrained},
\end{align*}
and
\begin{equation*}
    \min_{\Omega\in\mathcal{O}_{\rm ad}} L_{\mathcal{L}}({\bf u},p,\Omega,{\bf v},q,{\bf v}_{\rm bc})\ \left\{\begin{split}
        &\mbox{s.t. } ({\bf u},p) \mbox{ satisfies } {\bf Q}({\bf x},{\bf u},\nabla{\bf u},p,{\bf n},{\bf t}) = {\bf f}_{\rm bc}({\bf x})\mbox{ on }\Gamma &&\mbox{ if } \delta_{\mathcal{L}} = 0,\\
        &\mbox{with } ({\bf u},p) \mbox{ unconstrained} &&\mbox{ if } \delta_{\mathcal{L}} = 1.
    \end{split}\right.
\end{equation*}
Choose the adjoint variables\texttt{/}Lagrangian multipliers $({\bf v},q,{\bf v}_{\rm bc})$ s.t.
\begin{align*}
    D_{\bf u}L_{\mathcal{L}}({\bf u},p,\Omega,{\bf v},q,{\bf v}_{\rm bc})\tilde{\bf u} + D_pL_{\mathcal{L}}({\bf u},p,\Omega,{\bf v},q,{\bf v}_{\rm bc})\tilde{p} = 0,\ \forall({\bf u},p,\Omega,\tilde{\bf u},\tilde{p}).
\end{align*}
Expand this more explicitly for all $({\bf u},p,\Omega,\tilde{\bf u},\tilde{p})$:
\begin{align*}
    &\int_\Omega D_{\bf u}\left(J_\Omega({\bf x},{\bf u},\nabla{\bf u},p)\right)\tilde{\bf u} + D_p\left(J_\Omega({\bf x},{\bf u},\nabla{\bf u},p)\right)\tilde{p}{\rm d}{\bf x}\\
    &+ \int_\Gamma D_{\bf u}\left(J_\Gamma({\bf x},{\bf u},\nabla{\bf u},p,{\bf n},{\bf t})\right)\tilde{\bf u} + D_p\left(J_\Gamma({\bf x},{\bf u},\nabla{\bf u},p,{\bf n},{\bf t})\right)\tilde{p}{\rm d}\Gamma\\
    &-\int_\Omega \left[-D_{\bf u}(\operatorname{diff}(\nu,{\bf u}))\tilde{\bf u} - D_{\bf u}(f({\bf x},{\bf u},\nabla{\bf u},p))\tilde{\bf u} + \nabla\tilde{p} - D_pf({\bf x},{\bf u},\nabla{\bf u},p)\tilde{p}\right]\cdot{\bf v}\\
    &\hspace{1.05cm}+ q\left[\nabla\cdot\tilde{\bf u} - D_{\bf u}(f_{\rm div}({\bf x},{\bf u},\nabla{\bf u},p))\tilde{\bf u} - D_pf_{\rm div}({\bf x},{\bf u},\nabla{\bf u},p)\tilde{p}\right]{\rm d}{\bf x}\\
    &-\delta_{\mathcal{L}}\int_\Gamma \left[D_{\bf u}(Q({\bf x},{\bf u},\nabla{\bf u},p,{\bf n},{\bf t}))\tilde{\bf u} + D_pQ({\bf x},{\bf u},\nabla{\bf u},p,{\bf n},{\bf t})\tilde{p}\right]\cdot{\bf v}_{\rm bc}{\rm d}\Gamma = 0,\ \forall({\bf u},p,\Omega,\tilde{\bf u},\tilde{p}),
\end{align*}
and more explicitly:
\begin{align*}
    &\int_\Omega D_{\bf u}J_\Omega({\bf x},{\bf u},\nabla{\bf u},p)\tilde{\bf u} + D_{\nabla{\bf u}}J_\Omega({\bf x},{\bf u},\nabla{\bf u},p)\nabla\tilde{\bf u} + D_pJ_\Omega({\bf x},{\bf u},\nabla{\bf u},p)\tilde{p}{\rm d}{\bf x}\\
    &+ \int_\Gamma D_{\bf u}J_\Gamma({\bf x},{\bf u},\nabla{\bf u},p,{\bf n},{\bf t})\tilde{\bf u} + D_{\nabla{\bf u}}J_\Gamma({\bf x},{\bf u},\nabla{\bf u},p,{\bf n},{\bf t})\nabla\tilde{\bf u} + D_pJ_\Gamma({\bf x},{\bf u},\nabla{\bf u},p,{\bf n},{\bf t})\tilde{p}{\rm d}\Gamma\\
    &+\int_\Omega D_{\bf u}(\operatorname{diff}(\nu,{\bf u}))\tilde{\bf u}\cdot{\bf v} + D_{\bf u}{\bf f}({\bf x},{\bf u},\nabla{\bf u},p)\tilde{\bf u}\cdot{\bf v} + D_{\nabla{\bf u}}{\bf f}({\bf x},{\bf u},\nabla{\bf u},p)\nabla\tilde{\bf u}\cdot{\bf v} - \nabla\tilde{p}\cdot{\bf v}\\
    &\hspace{1cm}+ D_p{\bf f}({\bf x},{\bf u},\nabla{\bf u},p)\tilde{p}\cdot{\bf v} - q\nabla\cdot\tilde{\bf u} + qD_{\bf u}f_{\rm div}({\bf x},{\bf u},\nabla{\bf u},p)\tilde{\bf u}\\
    &\hspace{1cm}+ qD_{\nabla{\bf u}}f_{\rm div}({\bf x},{\bf u},\nabla{\bf u},p)\nabla\tilde{\bf u} + qD_pf_{\rm div}({\bf x},{\bf u},\nabla{\bf u},p)\tilde{p}{\rm d}{\bf x}\\
    &-\delta_{\mathcal{L}}\int_\Gamma D_{\bf u}{\bf Q}({\bf x},{\bf u},\nabla{\bf u},p,{\bf n},{\bf t})\tilde{\bf u}\cdot{\bf v}_{\rm bc} + D_{\nabla{\bf u}}{\bf Q}({\bf x},{\bf u},\nabla{\bf u},p,{\bf n},{\bf t})\nabla\tilde{\bf u}\cdot{\bf v}_{\rm bc}\\
    &\hspace{1cm}+ D_p{\bf Q}({\bf x},{\bf u},\nabla{\bf u},p,{\bf n},{\bf t})\tilde{p}\cdot{\bf v}_{\rm bc}{\rm d}\Gamma = 0,\ \forall({\bf u},p,\Omega,\tilde{\bf u},\tilde{p}),
\end{align*}
where
\begin{equation*}
    D_{\bf u}(\operatorname{diff}(\nu,{\bf u}))\tilde{\bf u} = \left\{\begin{split}
        &\nabla\cdot(\nu\nabla\tilde{\bf u}),&&\mbox{ if } \operatorname{diff}(\nu,{\bf u}) = \nabla\cdot(\nu\nabla{\bf u}),\\
        &\nabla\cdot(2\nu\boldsymbol{\varepsilon}(\tilde{\bf u})),&&\mbox{ if }\operatorname{diff}(\nu,{\bf u}) = \nabla\cdot(2\nu\boldsymbol{\varepsilon}({\bf u})).
    \end{split}\right.
\end{equation*}
Integrating by parts all the terms whose integrands are: $D_{\nabla{\bf u}}J_\Omega({\bf x},{\bf u},\nabla{\bf u},p)\nabla\tilde{\bf u}$, $D_{\bf u}(\operatorname{diff}(\nu,{\bf u}))\tilde{\bf u}\cdot{\bf v}$, $D_{\nabla{\bf u}}{\bf f}({\bf x},{\bf u},\nabla{\bf u},p)\nabla\tilde{\bf u}\cdot{\bf v}$, $-\nabla\tilde{p}\cdot{\bf v}$, $-q\nabla\cdot\tilde{\bf u}$, $qD_{\nabla{\bf u}}f_{\rm div}({\bf x},{\bf u},\nabla{\bf u},p)\nabla\tilde{\bf u}$\footnote{Look up Appendix \ref{ibp: stationary}.} yields:
\begin{enumerate}[leftmargin=0mm]
    \item \textbf{Case 1: $\operatorname{diff}(\nu,{\bf u}) = \nabla\cdot(\nu\nabla{\bf u})$.}
    \begin{equation}
        \label{Euler-Lagrange/general stationary Stokes/case 1}
        \tag{EL-gS1}
        \left.\begin{split}
            &\int_\Omega \left[\nabla_{\bf u}J_\Omega({\bf x},{\bf u},\nabla{\bf u},p) - \nabla\cdot(\nabla_{\nabla{\bf u}}J_\Omega({\bf x},{\bf u},\nabla{\bf u},p)) + \nabla\cdot(\nu\nabla{\bf v}) + \nabla_{\bf u}{\bf f}({\bf x},{\bf u},\nabla{\bf u},p){\bf v}\right.\\
            &\hspace{1cm}- \nabla\cdot(\nabla_{\nabla{\bf u}}{\bf f}({\bf x},{\bf u},\nabla{\bf u},p))\cdot{\bf v} - \nabla_{\nabla{\bf u}}{\bf f}({\bf x},{\bf u},\nabla{\bf u},p):\nabla{\bf v} + \nabla q + q\nabla_{\bf u}f_{\rm div}({\bf x},{\bf u},\nabla{\bf u},p)\\
            &\hspace{1cm}\left.- D_{\nabla{\bf u}}f_{\rm div}({\bf x},{\bf u},\nabla{\bf u},p)\nabla q - q(\nabla\cdot(\nabla_{\nabla{\bf u}}f_{\rm div}({\bf x},{\bf u},\nabla{\bf u},p)))\right]\cdot\tilde{\bf u}{\rm d}{\bf x}\\
            &+\int_\Omega \tilde{p}\left[D_pJ_\Omega({\bf x},{\bf u},\nabla{\bf u},p) + \nabla\cdot{\bf v} + D_p{\bf f}({\bf x},{\bf u},\nabla{\bf u},p)\cdot{\bf v} + qD_pf_{\rm div}({\bf x},{\bf u},\nabla{\bf u},p)\right]{\rm d}{\bf x}\\
            &+\int_\Gamma \left[D_{\nabla{\bf u}}J_\Omega({\bf x},{\bf u},\nabla{\bf u},p){\bf n} + \nabla_{\bf u}J_\Gamma({\bf x},{\bf u},\nabla{\bf u},p,{\bf n},{\bf t}) - \nu\partial_{\bf n}{\bf v} + ((\nabla_{\nabla{\bf u}}{\bf f}({\bf x},{\bf u},\nabla{\bf u},p)\cdot{\bf n})\cdot{\bf v}) - q{\bf n}\right.\\
            &\hspace{1cm}\left.+ qD_{\nabla{\bf u}}f_{\rm div}({\bf x},{\bf u},\nabla{\bf u},p){\bf n} - \delta_{\mathcal{L}}\nabla_{\bf u}{\bf Q}({\bf x},{\bf u},\nabla{\bf u},p,{\bf n},{\bf t}){\bf v}_{\rm bc}\right]\cdot\tilde{\bf u}{\rm d}\Gamma\\
            &+\int_\Gamma \tilde{p}\left[D_pJ_\Gamma({\bf x},{\bf u},\nabla{\bf u},p,{\bf n},{\bf t}) - {\bf v}\cdot{\bf n} - \delta_{\mathcal{L}}D_p{\bf Q}({\bf x},{\bf u},\nabla{\bf u},p,{\bf n},{\bf t})\cdot{\bf v}_{\rm bc}\right]{\rm d}\Gamma\\
            &+\int_\Gamma D_{\nabla{\bf u}}J_\Gamma({\bf x},{\bf u},\nabla{\bf u},p,{\bf n},{\bf t})\nabla\tilde{\bf u} + \nu\partial_{\bf n}\tilde{\bf u}\cdot{\bf v} - \delta_{\mathcal{L}}D_{\nabla{\bf u}}{\bf Q}({\bf x},{\bf u},\nabla{\bf u},p,{\bf n},{\bf t})\nabla\tilde{\bf u}\cdot{\bf v}_{\rm bc}{\rm d}\Gamma = 0,\ \forall({\bf u},p,\Omega,\tilde{\bf u},\tilde{p}).
        \end{split}\right.        
    \end{equation}
    We consider the following subcases:
    \begin{itemize}[leftmargin=0in]
        \item \textbf{Case 1.1: $\delta_{\mathcal{L}} = 0$.} This means to ``activate'' the boundary-condition constraint ${\bf Q}({\bf x},{\bf u},\nabla{\bf u},p,{\bf n},{\bf t}) = {\bf f}_{\rm bc}({\bf x})$ on $\Gamma$, so it will not be penalized by the Lagrangian $L$.
        
        To see the general structure, we rewrite \eqref{Euler-Lagrange/general stationary Stokes/case 1} as follows:
        \begin{equation}
           \label{brief Euler-Lagrange/general stationary Stokes/case 1.1}
           \tag{brEL-gS1.1}
            \left.\begin{split}
                &\int_\Omega {\bf F}_\Omega^{\tilde{\bf u}}({\bf x},{\bf u},\nabla{\bf u},\Delta{\bf u},p,\nabla p,{\bf v},\nabla{\bf v},\Delta{\bf v},q,\nabla q)\cdot\tilde{\bf u} + F_\Omega^{\tilde{p}}({\bf x},{\bf u},\nabla{\bf u},p,{\bf v},\nabla{\bf v},q)\tilde{p}{\rm d}{\bf x}\\
                &+ \int_\Gamma {\bf F}_\Gamma^{\tilde{\bf u}}({\bf x},{\bf u},\nabla{\bf u},p,{\bf v},\nabla{\bf v},q,{\bf n},{\bf t})\cdot\tilde{\bf u} + F_\Gamma^{\tilde{p}}({\bf x},{\bf u},\nabla{\bf u},p,{\bf v},{\bf n},{\bf t})\tilde{p}\\
                &\hspace{1cm}+ {\bf F}_\Gamma^{\nabla\tilde{\bf u}}({\bf x},{\bf u},\nabla{\bf u},p,{\bf v},{\bf n},{\bf t}):\nabla\tilde{\bf u}{\rm d}\Gamma = 0,\ \forall({\bf u},p,\Omega,\tilde{\bf u},\tilde{p}),
            \end{split}\right.
        \end{equation}
        where
        \begin{align*}
            &{\bf F}_\Omega^{\tilde{\bf u}}({\bf x},{\bf u},\nabla{\bf u},\Delta{\bf u},p,\nabla p,{\bf v},\nabla{\bf v},\Delta{\bf v},q,\nabla q)\\
            &\hspace{5mm}\coloneqq\nabla_{\bf u}J_\Omega({\bf x},{\bf u},\nabla{\bf u},p) - \nabla\cdot(\nabla_{\nabla{\bf u}}J_\Omega({\bf x},{\bf u},\nabla{\bf u},p)) + \nabla\cdot(\nu\nabla{\bf v}) + \nabla_{\bf u}{\bf f}({\bf x},{\bf u},\nabla{\bf u},p){\bf v}\\
            &\hspace{1cm}- \nabla\cdot(\nabla_{\nabla{\bf u}}{\bf f}({\bf x},{\bf u},\nabla{\bf u},p))\cdot{\bf v} - \nabla_{\nabla{\bf u}}{\bf f}({\bf x},{\bf u},\nabla{\bf u},p):\nabla{\bf v} + \nabla q + q\nabla_{\bf u}f_{\rm div}({\bf x},{\bf u},\nabla{\bf u},p)\\
            &\hspace{1cm}- D_{\nabla{\bf u}}f_{\rm div}({\bf x},{\bf u},\nabla{\bf u},p)\nabla q - q(\nabla\cdot(\nabla_{\nabla{\bf u}}f_{\rm div}({\bf x},{\bf u},\nabla{\bf u},p))),\\
            &F_\Omega^{\tilde{p}}({\bf x},{\bf u},\nabla{\bf u},p,{\bf v},\nabla{\bf v},q)\coloneqq D_pJ_\Omega({\bf x},{\bf u},\nabla{\bf u},p) + \nabla\cdot{\bf v} + D_p{\bf f}({\bf x},{\bf u},\nabla{\bf u},p)\cdot{\bf v} + qD_pf_{\rm div}({\bf x},{\bf u},\nabla{\bf u},p),\\
            &{\bf F}_\Gamma^{\tilde{\bf u}}({\bf x},{\bf u},\nabla{\bf u},p,{\bf v},\nabla{\bf v},q,{\bf n},{\bf t})\\
            &\hspace{5mm}\coloneqq D_{\nabla{\bf u}}J_\Omega({\bf x},{\bf u},\nabla{\bf u},p){\bf n} + \nabla_{\bf u}J_\Gamma({\bf x},{\bf u},\nabla{\bf u},p,{\bf n},{\bf t}) - \nu\partial_{\bf n}{\bf v} + ((\nabla_{\nabla{\bf u}}{\bf f}({\bf x},{\bf u},\nabla{\bf u},p)\cdot{\bf n})\cdot{\bf v})\\
            &\hspace{1cm}- q{\bf n} + qD_{\nabla{\bf u}}f_{\rm div}({\bf x},{\bf u},\nabla{\bf u},p){\bf n},\\
            &F_\Gamma^{\tilde{p}}({\bf x},{\bf u},\nabla{\bf u},p,{\bf v},{\bf n},{\bf t})\coloneqq D_pJ_\Gamma({\bf x},{\bf u},\nabla{\bf u},p,{\bf n},{\bf t}) - {\bf v}\cdot{\bf n},\\
            &{\bf F}_\Gamma^{\nabla\tilde{\bf u}}({\bf x},{\bf u},\nabla{\bf u},p,{\bf v},{\bf n},{\bf t},\nabla\tilde{\bf u})\coloneqq\nabla_{\nabla{\bf u}}J_\Gamma({\bf x},{\bf u},\nabla{\bf u},p,{\bf n},{\bf t}) + \nu{\bf n}\otimes{\bf v},
        \end{align*}
        are defined for all $({\bf x},{\bf u},p,{\bf v},q,{\bf n},{\bf t},\tilde{\bf u},\tilde{p})$.
        
        To proceed further, we need the following ``separation'' lemma.
        
        \begin{lemma}[Separation argument]
            \label{lemma: separation argument/general stationary Stokes/case 1.1}
            Suppose that the mappings
            \begin{align*}
                {\bf I}_\Omega^{\tilde{\bf u}}&:\mathbb{R}^N\times\mathbb{R}^N\times\mathbb{R}\times\mathbb{R}^N\times\mathbb{R}\to\mathbb{R}^N,\\
                I_\Omega^{\tilde{p}}&:\mathbb{R}^N\times\mathbb{R}^N\times\mathbb{R}\times\mathbb{R}^N\times\mathbb{R}\to\mathbb{R},\\
                {\bf I}_\Gamma^{\tilde{\bf u}}&:\mathbb{R}^N\times\mathbb{R}^N\times\mathbb{R}\times\mathbb{R}^N\times\mathbb{R}\times\mathbb{R}^N\times\mathbb{R}^{N(N-1)}\to\mathbb{R}^N,\\
                I_\Gamma^{\tilde{p}}&:\mathbb{R}^N\times\mathbb{R}^N\times\mathbb{R}\times\mathbb{R}^N\times\mathbb{R}\times\mathbb{R}^N\times\mathbb{R}^{N(N-1)}\to\mathbb{R},\\
                {\bf I}_\Gamma^{\nabla\tilde{\bf u}}&:\mathbb{R}^N\times\mathbb{R}^N\times\mathbb{R}\times\mathbb{R}^N\times\mathbb{R}^N\times\mathbb{R}^{N(N-1)}\to\mathbb{R}^{N^2},
            \end{align*}
            satisfy the integral equation
            \begin{align}
                \label{separation argument/general stationary Stokes/case 1.1/1}
                &\int_\Omega {\bf I}_\Omega^{\tilde{\bf u}}({\bf x},{\bf u},p,{\bf v},q)\cdot\tilde{\bf u} + I_\Omega^{\tilde{p}}({\bf x},{\bf u},p,{\bf v},q)\tilde{p}{\rm d}{\bf x}\nonumber\\
                &+ \int_\Gamma {\bf I}_\Gamma^{\tilde{\bf u}}({\bf x},{\bf u},p,{\bf v},q,{\bf n},{\bf t})\cdot\tilde{\bf u} + I_\Gamma^{\tilde{p}}({\bf x},{\bf u},p,{\bf v},q,{\bf n},{\bf t})\tilde{p} + {\bf I}_\Gamma^{\nabla\tilde{\bf u}}({\bf x},{\bf u},p,{\bf v},{\bf n},{\bf t}):\nabla\tilde{\bf u}{\rm d}\Gamma = 0,
            \end{align}
            for all $({\bf u},p,\Omega,\tilde{\bf u},\tilde{p})$ satisfying
            \begin{align}
                \label{separation argument/general stationary Stokes/case 1.1/2}
                {\bf Q}({\bf x},{\bf u} + \tilde{\bf u},\nabla{\bf u} + \nabla\tilde{\bf u},p + \tilde{p},{\bf n},{\bf t}) = {\bf Q}({\bf x},{\bf u},\nabla{\bf u},p,{\bf n},{\bf t}) = {\bf f}_{\rm bc}({\bf x})\mbox{ on }\Gamma.
            \end{align}
            Then
            \begin{equation}
                \label{separation argument/general stationary Stokes/case 1.1/3}
                \left\{\begin{split}
                    {\bf I}_\Omega^{\tilde{\bf u}}({\bf x},{\bf u},p,{\bf v},q) &= {\bf 0},&&\mbox{in }\Omega,\\
                    I_\Omega^{\tilde{p}}({\bf x},{\bf u},p,{\bf v},q) &= 0,&&\mbox{in }\Omega,\\
                    {\bf I}_\Gamma^{\tilde{\bf u}}({\bf x},{\bf u},p,{\bf v},q,{\bf n},{\bf t}) &= {\bf 0},&&\mbox{on }\Gamma_{\rm v}^{\bf u},\\
                    I_\Gamma^{\tilde{p}}({\bf x},{\bf u},p,{\bf v},q,{\bf n},{\bf t}) &= 0,&&\mbox{on }\Gamma_{\rm v}^p,\\
                    {\bf I}_\Gamma^{\nabla\tilde{\bf u}}({\bf x},{\bf u},p,{\bf v},{\bf n},{\bf t}) &= {\bf 0}_{N\times N},&&\mbox{on }\Gamma.
                \end{split}\right.
            \end{equation}
        \end{lemma}
    
        \begin{proof}
            Choosing $\tilde{\bf u} = {\bf 0}$ in $\overline{\Omega}$, then $\nabla\tilde{\bf u} = {\bf 0}_{N\times N}$ on $\Gamma$ and \eqref{separation argument/general stationary Stokes/case 1.1/1} becomes
            \begin{align}
                \label{separation argument/general stationary Stokes/case 1.1/4}
                \int_\Omega I_\Omega^{\tilde{p}}({\bf x},{\bf u},p,{\bf v},q)\tilde{p}{\rm d}{\bf x} + \int_\Gamma I_\Gamma^{\tilde{p}}({\bf x},{\bf u},p,{\bf v},q,{\bf n},{\bf t})\tilde{p}{\rm d}\Gamma = 0,\ \forall({\bf u},p,\Omega,\tilde{p})\mbox{ s.t. \eqref{separation argument/general stationary Stokes/case 1.1/2}}.
            \end{align}
            Then choosing $\tilde{p}$ varying such that $\tilde{p}|_\Gamma = 0$, the last equation gives
            \begin{align*}
                \int_\Omega I_\Omega^{\tilde{p}}({\bf x},{\bf u},p,{\bf v},q)\tilde{p}{\rm d}{\bf x} = 0,\ \forall({\bf u},p,\Omega,\tilde{p})\mbox{ s.t. \eqref{separation argument/general stationary Stokes/case 1.1/2}},\,\tilde{p}|_\Gamma = 0.
            \end{align*}
            Hence, the integrand must vanish identically, i.e., $({\bf v},q)$ must satisfy
            \begin{align}
                \label{separation argument/general stationary Stokes/case 1.1/5}
                I_\Omega^{\tilde{p}}({\bf x},{\bf u},p,{\bf v},q) = 0,\mbox{ in }\Omega.
            \end{align}
            Plug \eqref{separation argument/general stationary Stokes/case 1.1/5} back in \eqref{separation argument/general stationary Stokes/case 1.1/4}, we obtain
            \begin{align*}
                \int_\Gamma I_\Gamma^{\tilde{p}}({\bf x},{\bf u},p,{\bf v},q,{\bf n},{\bf t})\tilde{p}{\rm d}\Gamma = 0,\ \forall({\bf u},p,\Omega,\tilde{p})\mbox{ s.t. \eqref{separation argument/general stationary Stokes/case 1.1/2}}.
            \end{align*}
            Since $\tilde{p}|_{\Gamma_{\rm nv}^p} = 0$, the last equation implies that $({\bf v},q)$ must satisfy
            \begin{align}
                \label{separation argument/general stationary Stokes/case 1.1/6}
                I_\Gamma^{\tilde{p}}({\bf x},{\bf u},p,{\bf v},q,{\bf n},{\bf t}) = 0,\mbox{ on }\Gamma_{\rm v}^p.
            \end{align}
            Assume $({\bf v},q)$ satisfies \eqref{separation argument/general stationary Stokes/case 1.1/5} and \eqref{separation argument/general stationary Stokes/case 1.1/6}, then \eqref{separation argument/general stationary Stokes/case 1.1/1} becomes, for all $({\bf u},p,\Omega,\tilde{\bf u})$ satisfying \eqref{separation argument/general stationary Stokes/case 1.1/2},
            \begin{align}
                \label{separation argument/general stationary Stokes/case 1.1/7}
                \int_\Omega {\bf I}_\Omega^{\tilde{\bf u}}({\bf x},{\bf u},p,{\bf v},q)\cdot\tilde{\bf u}{\rm d}{\bf x} + \int_\Gamma {\bf I}_\Gamma^{\tilde{\bf u}}({\bf x},{\bf u},p,{\bf v},q,{\bf n},{\bf t})\cdot\tilde{\bf u} + {\bf I}_\Gamma^{\nabla\tilde{\bf u}}({\bf x},{\bf u},p,{\bf v},{\bf n},{\bf t}):\nabla\tilde{\bf u}{\rm d}\Gamma = 0.
            \end{align}
            Choosing $\tilde{\bf u}$ varying such that $\tilde{\bf u}|_\Gamma = {\bf 0}$ and $\nabla\tilde{\bf u}|_\Gamma = {\bf 0}_{N\times N}$ in \eqref{separation argument/general stationary Stokes/case 1.1/7} yields
            \begin{align*}
                \int_\Omega {\bf I}_\Omega^{\tilde{\bf u}}({\bf x},{\bf u},p,{\bf v},q)\cdot\tilde{\bf u}{\rm d}{\bf x} = 0,\ \forall({\bf u},p,\Omega,\tilde{\bf u})\mbox{ s.t. \eqref{separation argument/general stationary Stokes/case 1.1/2}},\,\tilde{\bf u}|_\Gamma = {\bf 0},\,\nabla\tilde{\bf u}|_\Gamma = {\bf 0}_{N\times N}.
            \end{align*}
            Hence, $({\bf v},q)$ satisfies
            \begin{align}
                \label{separation argument/general stationary Stokes/case 1.1/8}
                {\bf I}_\Omega^{\tilde{\bf u}}({\bf x},{\bf u},p,{\bf v},q) = {\bf 0},\mbox{ in }\Omega.
            \end{align}
            Plug \eqref{separation argument/general stationary Stokes/case 1.1/8} back in \eqref{separation argument/general stationary Stokes/case 1.1/7} to obtain
            \begin{align*}
                \int_\Gamma {\bf I}_\Gamma^{\tilde{\bf u}}({\bf x},{\bf u},p,{\bf v},q,{\bf n},{\bf t})\cdot\tilde{\bf u}{\rm d}\Gamma + \int_\Gamma {\bf I}_\Gamma^{\nabla\tilde{\bf u}}({\bf x},{\bf u},p,{\bf v},{\bf n},{\bf t}):\nabla\tilde{\bf u}{\rm d}\Gamma = 0,\ \forall({\bf u},p,\Omega,\tilde{\bf u})\mbox{ s.t. \eqref{separation argument/general stationary Stokes/case 1.1/2}}.
            \end{align*}
            Since $\tilde{\bf u}|_{\Gamma_{\rm nv}^{\bf u}} = {\bf 0}$, the last equation yields
            \begin{align}
                \label{separation argument/general stationary Stokes/case 1.1/9}
                \int_{\Gamma_{\rm v}^{\bf u}} {\bf I}_\Gamma^{\tilde{\bf u}}({\bf x},{\bf u},p,{\bf v},q,{\bf n},{\bf t})\cdot\tilde{\bf u}{\rm d}\Gamma + \int_\Gamma {\bf I}_\Gamma^{\nabla\tilde{\bf u}}({\bf x},{\bf u},p,{\bf v},{\bf n},{\bf t}):\nabla\tilde{\bf u}{\rm d}\Gamma = 0,\ \forall({\bf u},p,\Omega,\tilde{\bf u})\mbox{ s.t. \eqref{separation argument/general stationary Stokes/case 1.1/2}}.
            \end{align}
            Choosing $\tilde{\bf u}$ varying such that $\tilde{\bf u} = {\bf 0}$ on $\Gamma_{\rm v}^{\bf u}$, then \eqref{separation argument/general stationary Stokes/case 1.1/9} becomes
            \begin{align*}
                \int_\Gamma {\bf I}_\Gamma^{\nabla\tilde{\bf u}}({\bf x},{\bf u},p,{\bf v},{\bf n},{\bf t}):\nabla\tilde{\bf u}{\rm d}\Gamma = 0,\ \forall({\bf u},p,\Omega,\tilde{\bf u})\mbox{ s.t. \eqref{separation argument/general stationary Stokes/case 1.1/2}},\,\tilde{\bf u} = {\bf 0},
            \end{align*}
            which implies that ${\bf v}$ must satisfy
            \begin{align}
                \label{separation argument/general stationary Stokes/case 1.1/10}
                {\bf I}_\Gamma^{\nabla\tilde{\bf u}}({\bf x},{\bf u},p,{\bf v},{\bf n},{\bf t}) = {\bf 0}_{N\times N},\mbox{ on }\Gamma.
            \end{align}
            Plug \eqref{separation argument/general stationary Stokes/case 1.1/10} back in \eqref{separation argument/general stationary Stokes/case 1.1/9} to obtain
            \begin{align*}
                \int_{\Gamma_{\rm v}^{\bf u}} {\bf I}_\Gamma^{\tilde{\bf u}}({\bf x},{\bf u},p,{\bf v},q,{\bf n},{\bf t})\cdot\tilde{\bf u}{\rm d}\Gamma = 0,\ \forall({\bf u},p,\Omega,\tilde{\bf u})\mbox{ s.t. \eqref{separation argument/general stationary Stokes/case 1.1/2}},
            \end{align*}
            which implies that $({\bf v},q)$ must satisfy
            \begin{align}
                \label{separation argument/general stationary Stokes/case 1.1/11}
                {\bf I}_\Gamma^{\tilde{\bf u}}({\bf x},{\bf u},p,{\bf v},q,{\bf n},{\bf t}) = {\bf 0},\mbox{ on }\Gamma_{\rm v}^{\bf u}.
            \end{align}
            Gathering \eqref{separation argument/general stationary Stokes/case 1.1/5}, \eqref{separation argument/general stationary Stokes/case 1.1/6}, \eqref{separation argument/general stationary Stokes/case 1.1/8}, \eqref{separation argument/general stationary Stokes/case 1.1/10}, and \eqref{separation argument/general stationary Stokes/case 1.1/11} yields \eqref{separation argument/general stationary Stokes/case 1.1/3}.
        \end{proof}
        Now applying Lemma \ref{lemma: separation argument/general stationary Stokes/case 1.1} with $({\bf I}_\Omega^{\tilde{\bf u}},I_\Omega^{\tilde{p}},{\bf I}_\Gamma^{\tilde{\bf u}},I_\Gamma^{\tilde{p}},{\bf I}_\Gamma^{\nabla\tilde{\bf u}}) = ({\bf F}_\Omega^{\tilde{\bf u}},F_\Omega^{\tilde{p}},{\bf F}_\Gamma^{\tilde{\bf u}},F_\Gamma^{\tilde{p}},{\bf  F}_\Gamma^{\nabla\tilde{\bf u}})$ yields the following adjoint equation for \eqref{general stationary Stokes}:
        \begin{equation}
            \label{adjoint general stationary Stokes/case 1.1}
            \tag{adj-gS1.1}
            \left\{\begin{split}
                &\nabla\cdot(\nu\nabla{\bf v}) - \nabla_{\nabla{\bf u}}{\bf f}({\bf x},{\bf u},\nabla{\bf u},p):\nabla{\bf v} + \nabla_{\bf u}{\bf f}({\bf x},{\bf u},\nabla{\bf u},p){\bf v} - \nabla\cdot(\nabla_{\nabla{\bf u}}{\bf f}({\bf x},{\bf u},\nabla{\bf u},p))\cdot{\bf v}\\
                &\hspace{5mm}+ (1 - D_{\nabla{\bf u}}f_{\rm div}({\bf x},{\bf u},\nabla{\bf u},p))\nabla q + q[\nabla_{\bf u}f_{\rm div}({\bf x},{\bf u},\nabla{\bf u},p) - (\nabla\cdot(\nabla_{\nabla{\bf u}}f_{\rm div}({\bf x},{\bf u},\nabla{\bf u},p)))]\\
                &\hspace{1cm}= -\nabla_{\bf u}J_\Omega({\bf x},{\bf u},\nabla{\bf u},p) + \nabla\cdot(\nabla_{\nabla{\bf u}}J_\Omega({\bf x},{\bf u},\nabla{\bf u},p)),\mbox{ in }\Omega,\\
                &\nabla\cdot{\bf v} + D_p{\bf f}({\bf x},{\bf u},\nabla{\bf u},p)\cdot{\bf v} + qD_pf_{\rm div}({\bf x},{\bf u},\nabla{\bf u},p) = -D_pJ_\Omega({\bf x},{\bf u},\nabla{\bf u},p),\mbox{ in }\Omega,\\
                &- \nu\partial_{\bf n}{\bf v} + ((\nabla_{\nabla{\bf u}}{\bf f}({\bf x},{\bf u},\nabla{\bf u},p)\cdot{\bf n})\cdot{\bf v}) + q(D_{\nabla{\bf u}}f_{\rm div}({\bf x},{\bf u},\nabla{\bf u},p) - 1){\bf n}\\
                &\hspace{1cm}= -D_{\nabla{\bf u}}J_\Omega({\bf x},{\bf u},\nabla{\bf u},p){\bf n} - \nabla_{\bf u}J_\Gamma({\bf x},{\bf u},\nabla{\bf u},p,{\bf n},{\bf t}),\mbox{ on }\Gamma_{\rm v}^{\bf u},\\
                &{\bf v}\cdot{\bf n} = D_pJ_\Gamma({\bf x},{\bf u},\nabla{\bf u},p,{\bf n},{\bf t}),\mbox{ on }\Gamma_{\rm v}^p,\\
                &\nu{\bf n}\otimes{\bf v} = -\nabla_{\nabla{\bf u}}J_\Gamma({\bf x},{\bf u},\nabla{\bf u},p,{\bf n},{\bf t}),\mbox{ on }\Gamma.
            \end{split}\right.
        \end{equation}
    
        \begin{question}
            Is \eqref{adjoint general stationary Stokes/case 1.1} overdetermined or underdetermined? If yes, in which cases of the cost functional $J$?
        \end{question}
        \item \textbf{Case 1.2: $\delta_{\mathcal{L}} = 1$.} This means to ``deactivate'' the boundary-condition constraint ${\bf Q}({\bf x},{\bf u},\nabla{\bf u},p,{\bf n},{\bf t}) = {\bf f}_{\rm bc}({\bf x})$ on $\Gamma$, so it will be penalized by the extended Lagrangian $\mathcal{L}$.
        
        Again, to see the general structure, we rewrite \eqref{Euler-Lagrange/general stationary Stokes/case 1} as follows:
        \begin{equation}
            \label{brief Euler-Lagrange/general stationary Stokes/case 1.2}
            \tag{brEL-gS1.2}
            \left.\begin{split}
                &\int_\Omega {\bf F}_\Omega^{\tilde{\bf u}}({\bf x},{\bf u},\nabla{\bf u},\Delta{\bf u},p,\nabla p,{\bf v},\nabla{\bf v},\Delta{\bf v},q,\nabla q)\cdot\tilde{\bf u} + F_\Omega^{\tilde{p}}({\bf x},{\bf u},\nabla{\bf u},p,{\bf v},\nabla{\bf v},q)\tilde{p}{\rm d}{\bf x}\\
                &+ \int_\Gamma \boldsymbol{\mathcal{F}}_\Gamma^{\tilde{\bf u}}({\bf x},{\bf u},\nabla{\bf u},p,{\bf v},\nabla{\bf v},q,{\bf v}_{\rm bc},{\bf n},{\bf t})\cdot\tilde{\bf u} + \mathcal{F}_\Gamma^{\tilde{p}}({\bf x},{\bf u},\nabla{\bf u},p,{\bf v},{\bf v}_{\rm bc},{\bf n},{\bf t})\tilde{p}\\
                &\hspace{1cm}+ \boldsymbol{\mathcal{F}}_\Gamma^{\nabla\tilde{\bf u}}({\bf x},{\bf u},\nabla{\bf u},p,{\bf v},{\bf v}_{\rm bc},{\bf n},{\bf t}):\nabla\tilde{\bf u}{\rm d}\Gamma = 0,\ \forall({\bf u},p,\Omega,\tilde{\bf u},\tilde{p}),
            \end{split}\right.
        \end{equation}
        where
        \begin{align*}
            \boldsymbol{\mathcal{F}}_\Gamma^{\tilde{\bf u}}({\bf x},{\bf u},\nabla{\bf u},p,{\bf v},\nabla{\bf v},q,{\bf v}_{\rm bc},{\bf n},{\bf t})&\coloneqq{\bf F}_\Gamma^{\tilde{\bf u}}({\bf x},{\bf u},\nabla{\bf u},p,{\bf v},\nabla{\bf v},q,{\bf n},{\bf t}) - \nabla_{\bf u}{\bf Q}({\bf x},{\bf u},\nabla{\bf u},p,{\bf n},{\bf t}){\bf v}_{\rm bc},\\
            \mathcal{F}_\Gamma^{\tilde{p}}({\bf x},{\bf u},\nabla{\bf u},p,{\bf v},{\bf v}_{\rm bc},{\bf n},{\bf t})&\coloneqq F_\Gamma^{\tilde{p}}({\bf x},{\bf u},\nabla{\bf u},p,{\bf v},{\bf n},{\bf t}) - D_p{\bf Q}({\bf x},{\bf u},\nabla{\bf u},p,{\bf n},{\bf t})\cdot{\bf v}_{\rm bc},\\
            \boldsymbol{\mathcal{F}}_\Gamma^{\nabla\tilde{\bf u}}({\bf x},{\bf u},\nabla{\bf u},p,{\bf v},{\bf v}_{\rm bc},{\bf n},{\bf t})&\coloneqq{\bf F}_\Gamma^{\nabla\tilde{\bf u}}({\bf x},{\bf u},\nabla{\bf u},p,{\bf v},{\bf n},{\bf t}) - \nabla_{\nabla{\bf u}}{\bf Q}({\bf x},{\bf u},\nabla{\bf u},p,{\bf n},{\bf t})\cdot{\bf v}_{\rm bc}.
        \end{align*}
        Note the the domain integrands are the same as the standard-Lagrangian case, only the boundary integrands are modified with the additional Lagrange multiplier ${\bf v}_{\rm bc}$ in this extended-Lagrangian case.
        
        \begin{lemma}[Separation argument]
            \label{lemma: separation argument/general stationary Stokes/case 1.2}
            Suppose that the mappings ${\bf I}_\Omega^{\tilde{\bf u}}$, $I_\Omega^{\tilde{p}}$, $\boldsymbol{\mathcal{I}}_\Gamma^{\tilde{\bf u}}$, $\mathcal{I}_\Gamma^{\tilde{p}}$, and $\mathcal{I}_\Gamma^{\nabla\tilde{\bf u}}$ satisfy
            \begin{align*}
                {\bf I}_\Omega^{\tilde{\bf u}}&:\mathbb{R}^N\times\mathbb{R}^N\times\mathbb{R}\times\mathbb{R}^N\times\mathbb{R}\to\mathbb{R}^N,\\
                I_\Omega^{\tilde{p}}&:\mathbb{R}^N\times\mathbb{R}^N\times\mathbb{R}\times\mathbb{R}^N\times\mathbb{R}\to\mathbb{R},\\
                \boldsymbol{\mathcal{I}}_\Gamma^{\tilde{\bf u}}&:\mathbb{R}^N\times\mathbb{R}^N\times\mathbb{R}\times\mathbb{R}^N\times\mathbb{R}\times\mathbb{R}^N\times\mathbb{R}^N\times\mathbb{R}^{N(N-1)}\to\mathbb{R}^N,\\
                \mathcal{I}_\Gamma^{\tilde{p}}&:\mathbb{R}^N\times\mathbb{R}^N\times\mathbb{R}\times\mathbb{R}^N\times\mathbb{R}\times\mathbb{R}^N\times\mathbb{R}^N\times\mathbb{R}^{N(N-1)}\to\mathbb{R},\\
                \boldsymbol{\mathcal{I}}_\Gamma^{\nabla\tilde{\bf u}}&:\mathbb{R}^N\times\mathbb{R}^N\times\mathbb{R}\times\mathbb{R}^N\times\mathbb{R}^N\times\mathbb{R}^N\times\mathbb{R}^{N(N-1)}\to\mathbb{R}^{N^2},
            \end{align*}
            satisfy the integral equation
            \begin{align}
                \label{separation argument/general stationary Stokes/case 1.2/1}
                &\int_\Omega {\bf I}_\Omega^{\tilde{\bf u}}({\bf x},{\bf u},p,{\bf v},q)\cdot\tilde{\bf u} + I_\Omega^{\tilde{p}}({\bf x},{\bf u},p,{\bf v},q)\tilde{p}{\rm d}{\bf x}\nonumber\\
                &+ \int_\Gamma \boldsymbol{\mathcal{I}}_\Gamma^{\tilde{\bf u}}({\bf x},{\bf u},p,{\bf v},q,{\bf v}_{\rm bc},{\bf n},{\bf t})\cdot\tilde{\bf u} + \mathcal{I}_\Gamma^{\tilde{p}}({\bf x},{\bf u},p,{\bf v},q,{\bf v}_{\rm bc},{\bf n},{\bf t})\tilde{p} + \boldsymbol{\mathcal{I}}_\Gamma^{\nabla\tilde{\bf u}}({\bf x},{\bf u},p,{\bf v},{\bf v}_{\rm bc},{\bf n},{\bf t}):\nabla\tilde{\bf u}{\rm d}\Gamma = 0,
            \end{align}
            for all $({\bf u},p,\Omega,\tilde{\bf u},\tilde{p})$. Then
            \begin{equation}
                \label{separation argument/general stationary Stokes/case 1.2/2}
                \left\{\begin{split}
                    {\bf I}_\Omega^{\tilde{\bf u}}({\bf x},{\bf u},p,{\bf v},q) &= {\bf 0},&&\mbox{in }\Omega,\\
                    I_\Omega^{\tilde{p}}({\bf x},{\bf u},p,{\bf v},q) &= 0,&&\mbox{in }\Omega,\\
                    \boldsymbol{\mathcal{I}}_\Gamma^{\tilde{\bf u}}({\bf x},{\bf u},p,{\bf v},q,{\bf v}_{\rm bc},{\bf n},{\bf t}) &= {\bf 0},&&\mbox{on }\Gamma,\\
                    \mathcal{I}_\Gamma^{\tilde{p}}({\bf x},{\bf u},p,{\bf v},q,{\bf v}_{\rm bc},{\bf n},{\bf t}) &= 0,&&\mbox{on }\Gamma,\\
                    \boldsymbol{\mathcal{I}}_\Gamma^{\nabla\tilde{\bf u}}({\bf x},{\bf u},p,{\bf v},{\bf v}_{\rm bc},{\bf n},{\bf t}) &= {\bf 0}_{N\times N},&&\mbox{ on }\Gamma.
                \end{split}\right.
            \end{equation}
        \end{lemma}
        
        \begin{proof}
            Choosing $\tilde{\bf u} = {\bf 0}$ in $\overline{\Omega}$, then $\nabla\tilde{\bf u} = {\bf 0}_{N\times N}$ on $\Gamma$ and \eqref{separation argument/general stationary Stokes/case 1.2/1} becomes
            \begin{align}
                \label{separation argument/general stationary Stokes/case 1.2/3}
                \int_\Omega I_\Omega^{\tilde{p}}({\bf x},{\bf u},p,{\bf v},q)\tilde{p}{\rm d}{\bf x} + \int_\Gamma \mathcal{I}_\Gamma^{\tilde{p}}({\bf x},{\bf u},p,{\bf v},q,{\bf v}_{\rm bc},{\bf n},{\bf t})\tilde{p}{\rm d}\Gamma = 0,\ \forall({\bf u},p,\Omega,\tilde{p}).
            \end{align}
            Then choosing $\tilde{p}$ varying such that $\tilde{p}|_\Gamma = 0$, the last equation gives
            \begin{align*}
                \int_\Omega I_\Omega^{\tilde{p}}({\bf x},{\bf u},p,{\bf v},q)\tilde{p}{\rm d}{\bf x} = 0,\ \forall({\bf u},p,\Omega,\tilde{p})\mbox{ s.t. }\tilde{p}|_\Gamma = 0.
            \end{align*}
            Hence, the integrand must vanish identically, i.e., $({\bf v},q)$ must satisfy
            \begin{align}
                \label{separation argument/general stationary Stokes/case 1.2/4}
                I_\Omega^{\tilde{p}}({\bf x},{\bf u},p,{\bf v},q) = 0,\mbox{ in }\Omega.
            \end{align}
            Plug \eqref{separation argument/general stationary Stokes/case 1.2/4} back in \eqref{separation argument/general stationary Stokes/case 1.2/3}, we obtain
            \begin{align*}
                \int_\Gamma \mathcal{I}_\Gamma^{\tilde{p}}({\bf x},{\bf u},p,{\bf v},q,{\bf v}_{\rm bc},{\bf n},{\bf t})\tilde{p}{\rm d}\Gamma = 0,\ \forall({\bf u},p,\Omega,\tilde{p}),
            \end{align*}
            which implies that $({\bf v},q)$ must satisfy
            \begin{align}
                \label{separation argument/general stationary Stokes/case 1.2/5}
                \mathcal{I}_\Gamma^{\tilde{p}}({\bf x},{\bf u},p,{\bf v},q,{\bf v}_{\rm bc},{\bf n},{\bf t}) = 0,\mbox{ on }\Gamma.
            \end{align}
            Assume $({\bf v},q)$ satisfies \eqref{separation argument/general stationary Stokes/case 1.2/4} and \eqref{separation argument/general stationary Stokes/case 1.2/5}, then \eqref{separation argument/general stationary Stokes/case 1.2/1} becomes, for all $({\bf u},p,\Omega,\tilde{\bf u})$,
            \begin{align}
                \label{separation argument/general stationary Stokes/case 1.2/6}
                \int_\Omega {\bf I}_\Omega^{\tilde{\bf u}}({\bf x},{\bf u},p,{\bf v},q)\cdot\tilde{\bf u}{\rm d}{\bf x} + \int_\Gamma \boldsymbol{\mathcal{I}}_\Gamma^{\tilde{\bf u}}({\bf x},{\bf u},p,{\bf v},q,{\bf v}_{\rm bc},{\bf n},{\bf t})\cdot\tilde{\bf u} + \boldsymbol{\mathcal{I}}_\Gamma^{\nabla\tilde{\bf u}}({\bf x},{\bf u},p,{\bf v},{\bf v}_{\rm bc},{\bf n},{\bf t}):\nabla\tilde{\bf u}{\rm d}\Gamma = 0.
            \end{align}
            Choosing $\tilde{\bf u}$ varying such that $\tilde{\bf u}|_\Gamma = {\bf 0}$ and $\nabla\tilde{\bf u}|_\Gamma = {\bf 0}_{N\times N}$ in \eqref{separation argument/general stationary Stokes/case 1.2/6} yields
            \begin{align*}
                \int_\Omega {\bf I}_\Omega^{\tilde{\bf u}}({\bf x},{\bf u},p,{\bf v},q)\cdot\tilde{\bf u}{\rm d}{\bf x} = 0,\ \forall({\bf u},p,\Omega,\tilde{\bf u})\mbox{ s.t. }\tilde{\bf u}|_\Gamma = {\bf 0},\,\nabla\tilde{\bf u}|_\Gamma = {\bf 0}_{N\times N},
            \end{align*}
            which implies that $({\bf v},q)$ must satisfy
            \begin{align}
                \label{separation argument/general stationary Stokes/case 1.2/7}
                {\bf I}_\Omega^{\tilde{\bf u}}({\bf x},{\bf u},p,{\bf v},q) = {\bf 0},\mbox{ in }\Omega.
            \end{align}
            Plug \eqref{separation argument/general stationary Stokes/case 1.2/7} back in \eqref{separation argument/general stationary Stokes/case 1.2/6} to obtain
            \begin{align}
                \label{separation argument/general stationary Stokes/case 1.2/8}
                \int_\Gamma \boldsymbol{\mathcal{I}}_\Gamma^{\tilde{\bf u}}({\bf x},{\bf u},p,{\bf v},q,{\bf v}_{\rm bc},{\bf n},{\bf t})\cdot\tilde{\bf u} + \boldsymbol{\mathcal{I}}_\Gamma^{\nabla\tilde{\bf u}}({\bf x},{\bf u},p,{\bf v},{\bf v}_{\rm bc},{\bf n},{\bf t}):\nabla\tilde{\bf u}{\rm d}\Gamma = 0,\ \forall({\bf u},p,\Omega,\tilde{\bf u}).
            \end{align}
            Choosing $\tilde{\bf u}$ varying such that $\tilde{\bf u} = {\bf 0}$ on $\Gamma$, then \eqref{separation argument/general stationary Stokes/case 1.2/8} becomes
            \begin{align*}
                \int_\Gamma \boldsymbol{\mathcal{I}}_\Gamma^{\nabla\tilde{\bf u}}({\bf x},{\bf u},p,{\bf v},{\bf v}_{\rm bc},{\bf n},{\bf t}):\nabla\tilde{\bf u}{\rm d}\Gamma = 0,\ \forall({\bf u},p,\Omega,\tilde{\bf u})\mbox{ s.t. }\tilde{\bf u} = {\bf 0},
            \end{align*}
            which implies that ${\bf v}$ must satisfy
            \begin{align}
                \label{separation argument/general stationary Stokes/case 1.2/9}
                \boldsymbol{\mathcal{I}}_\Gamma^{\nabla\tilde{\bf u}}({\bf x},{\bf u},p,{\bf v},{\bf v}_{\rm bc},{\bf n},{\bf t}) = {\bf 0}_{N\times N},\mbox{ on }\Gamma.
            \end{align}
            Plug \eqref{separation argument/general stationary Stokes/case 1.2/9} back in \eqref{separation argument/general stationary Stokes/case 1.2/8} to obtain
            \begin{align*}
                \int_\Gamma \boldsymbol{\mathcal{I}}_\Gamma^{\tilde{\bf u}}({\bf x},{\bf u},p,{\bf v},q,{\bf v}_{\rm bc},{\bf n},{\bf t})\cdot\tilde{\bf u}{\rm d}\Gamma = 0,\ \forall({\bf u},p,\Omega,\tilde{\bf u}).
            \end{align*}
            which implies that $({\bf v},q)$ must satisfy
            \begin{align}
                \label{separation argument/general stationary Stokes/case 1.2/10}
                \boldsymbol{\mathcal{I}}_\Gamma^{\tilde{\bf u}}({\bf x},{\bf u},p,{\bf v},q,{\bf v}_{\rm bc},{\bf n},{\bf t}) = {\bf 0},\mbox{ on }\Gamma.
            \end{align}
            Gathering \eqref{separation argument/general stationary Stokes/case 1.2/4}, \eqref{separation argument/general stationary Stokes/case 1.2/5}, \eqref{separation argument/general stationary Stokes/case 1.2/7}, \eqref{separation argument/general stationary Stokes/case 1.2/9}, and \eqref{separation argument/general stationary Stokes/case 1.2/10} yields \eqref{separation argument/general stationary Stokes/case 1.2/2}.
        \end{proof}
        Now applying Lemma \ref{lemma: separation argument/general stationary Stokes/case 1.2} with $({\bf I}_\Omega^{\tilde{\bf u}},I_\Omega^{\tilde{p}},\boldsymbol{\mathcal{I}}_\Gamma^{\tilde{\bf u}},\mathcal{I}_\Gamma^{\tilde{p}},\boldsymbol{\mathcal{I}}_\Gamma^{\nabla\tilde{\bf u}}) = ({\bf F}_\Omega^{\tilde{\bf u}},F_\Omega^{\tilde{p}},\boldsymbol{\mathcal{F}}_\Gamma^{\tilde{\bf u}},\mathcal{F}_\Gamma^{\tilde{p}},\boldsymbol{\mathcal{F}}_\Gamma^{\nabla\tilde{\bf u}})$ yields the following adjoint equation for \eqref{general stationary Stokes}:
        \begin{equation}
            \label{adjoint general stationary Stokes/case 1.2}
            \tag{adj-gS1.2}
            \left\{\begin{split}
                &\nabla\cdot(\nu\nabla{\bf v}) - \nabla_{\nabla{\bf u}}{\bf f}({\bf x},{\bf u},\nabla{\bf u},p):\nabla{\bf v} + \nabla_{\bf u}{\bf f}({\bf x},{\bf u},\nabla{\bf u},p){\bf v} - \nabla\cdot(\nabla_{\nabla{\bf u}}{\bf f}({\bf x},{\bf u},\nabla{\bf u},p))\cdot{\bf v}\\
                &\hspace{5mm}+ (1 - D_{\nabla{\bf u}}f_{\rm div}({\bf x},{\bf u},\nabla{\bf u},p))\nabla q + q[\nabla_{\bf u}f_{\rm div}({\bf x},{\bf u},\nabla{\bf u},p) - (\nabla\cdot(\nabla_{\nabla{\bf u}}f_{\rm div}({\bf x},{\bf u},\nabla{\bf u},p)))]\\
                &\hspace{1cm}= -\nabla_{\bf u}J_\Omega({\bf x},{\bf u},\nabla{\bf u},p) + \nabla\cdot(\nabla_{\nabla{\bf u}}J_\Omega({\bf x},{\bf u},\nabla{\bf u},p)),\mbox{ in }\Omega,\\
                &\nabla\cdot{\bf v} + D_p{\bf f}({\bf x},{\bf u},\nabla{\bf u},p)\cdot{\bf v} + qD_pf_{\rm div}({\bf x},{\bf u},\nabla{\bf u},p) = -D_pJ_\Omega({\bf x},{\bf u},\nabla{\bf u},p),\mbox{ in }\Omega,\\
                &- \nu\partial_{\bf n}{\bf v} + ((\nabla_{\nabla{\bf u}}{\bf f}({\bf x},{\bf u},\nabla{\bf u},p)\cdot{\bf n})\cdot{\bf v}) + q(D_{\nabla{\bf u}}f_{\rm div}({\bf x},{\bf u},\nabla{\bf u},p) - 1){\bf n} - \nabla_{\bf u}{\bf Q}({\bf x},{\bf u},\nabla{\bf u},p,{\bf n},{\bf t}){\bf v}_{\rm bc}\\
                &\hspace{1cm}= -D_{\nabla{\bf u}}J_\Omega({\bf x},{\bf u},\nabla{\bf u},p){\bf n} - \nabla_{\bf u}J_\Gamma({\bf x},{\bf u},\nabla{\bf u},p,{\bf n},{\bf t}),\mbox{ on }\Gamma,\\
                &{\bf v}\cdot{\bf n} - D_p{\bf Q}({\bf x},{\bf u},\nabla{\bf u},p,{\bf n},{\bf t})\cdot{\bf v}_{\rm bc} = D_pJ_\Gamma({\bf x},{\bf u},\nabla{\bf u},p,{\bf n},{\bf t}),\mbox{ on }\Gamma,\\
                &\nu{\bf n}\otimes{\bf v} - \nabla_{\nabla{\bf u}}{\bf Q}({\bf x},{\bf u},\nabla{\bf u},p,{\bf n},{\bf t})\cdot{\bf v}_{\rm bc} = -\nabla_{\nabla{\bf u}}J_\Gamma({\bf x},{\bf u},\nabla{\bf u},p,{\bf n},{\bf t}),\mbox{ on }\Gamma.
            \end{split}\right.
        \end{equation}
    \end{itemize}
    \item \textbf{Case 2: $\operatorname{diff}(\nu,{\bf u}) = \nabla\cdot(2\nu\boldsymbol{\varepsilon}({\bf u}))$.}
    \begin{align*}
        &\int_\Omega \left[\nabla_{\bf u}J_\Omega({\bf x},{\bf u},\nabla{\bf u},p) - \nabla\cdot(\nabla_{\nabla{\bf u}}J_\Omega({\bf x},{\bf u},\nabla{\bf u},p)) + \nabla\cdot(2\nu\boldsymbol{\varepsilon}({\bf v})) + \nabla_{\bf u}{\bf f}({\bf x},{\bf u},\nabla{\bf u},p){\bf v}\right.\\
        &\hspace{5mm}- \nabla\cdot(\nabla_{\nabla{\bf u}}{\bf f}({\bf x},{\bf u},\nabla{\bf u},p))\cdot{\bf v} - \nabla_{\nabla{\bf u}}{\bf f}({\bf x},{\bf u},\nabla{\bf u},p):\nabla{\bf v} + \nabla q + q\nabla_{\bf u}f_{\rm div}({\bf x},{\bf u},\nabla{\bf u},p)\\
        &\hspace{5mm}\left.- D_{\nabla{\bf u}}f_{\rm div}({\bf x},{\bf u},\nabla{\bf u},p)\nabla q - q(\nabla\cdot(\nabla_{\nabla{\bf u}}f_{\rm div}({\bf x},{\bf u},\nabla{\bf u},p)))\right]\cdot\tilde{\bf u}{\rm d}{\bf x}\\
        &+\int_\Omega \tilde{p}\left[D_pJ_\Omega({\bf x},{\bf u},\nabla{\bf u},p) + \nabla\cdot{\bf v} + D_p{\bf f}({\bf x},{\bf u},\nabla{\bf u},p)\cdot{\bf v} + qD_pf_{\rm div}({\bf x},{\bf u},\nabla{\bf u},p)\right]{\rm d}{\bf x}\\
        &+\int_\Gamma \left[D_{\nabla{\bf u}}J_\Omega({\bf x},{\bf u},\nabla{\bf u},p){\bf n} + \nabla_{\bf u}J_\Gamma({\bf x},{\bf u},\nabla{\bf u},p,{\bf n},{\bf t}) - 2\nu\boldsymbol{\varepsilon}_{\bf n}({\bf v}) + ((\nabla_{\nabla{\bf u}}{\bf f}({\bf x},{\bf u},\nabla{\bf u},p)\cdot{\bf n})\cdot{\bf v}) - q{\bf n}\right.\\
        &\hspace{1cm}\left.+ qD_{\nabla{\bf u}}f_{\rm div}({\bf x},{\bf u},\nabla{\bf u},p){\bf n} - \delta_{\mathcal{L}}\nabla_{\bf u}{\bf Q}({\bf x},{\bf u},\nabla{\bf u},p,{\bf n},{\bf t}){\bf v}_{\rm bc}\right]\cdot\tilde{\bf u}{\rm d}\Gamma\\
        &+\int_\Gamma \tilde{p}\left[D_pJ_\Gamma({\bf x},{\bf u},\nabla{\bf u},p,{\bf n},{\bf t}) - {\bf v}\cdot{\bf n} - \delta_{\mathcal{L}}D_p{\bf Q}({\bf x},{\bf u},\nabla{\bf u},p,{\bf n},{\bf t})\cdot{\bf v}_{\rm bc}\right]{\rm d}\Gamma\\
        &+\int_\Gamma D_{\nabla{\bf u}}J_\Gamma({\bf x},{\bf u},\nabla{\bf u},p,{\bf n},{\bf t})\nabla\tilde{\bf u} + 2\nu\boldsymbol{\varepsilon}_{\bf n}(\tilde{\bf u})\cdot{\bf v} - \delta_{\mathcal{L}}D_{\nabla{\bf u}}{\bf Q}({\bf x},{\bf u},\nabla{\bf u},p,{\bf n},{\bf t})\nabla\tilde{\bf u}\cdot{\bf v}_{\rm bc}{\rm d}\Gamma = 0,\ \forall({\bf u},p,\Omega,\tilde{\bf u},\tilde{p}).
    \end{align*}
    Note that only the following three terms are changed compared to the previous case:
    \begin{align*}
        &\int_\Omega \nabla\cdot(\nu\nabla{\bf v})\cdot\tilde{\bf u}{\rm d}{\bf x},\,\int_\Gamma -\nu\partial_{\bf n}{\bf v}\cdot\tilde{\bf u}{\rm d}\Gamma,\,\int_\Gamma \nu\partial_{\bf n}\tilde{\bf u}\cdot{\bf v}{\rm d}\Gamma\mbox{ are replaced by }\\
        &\int_\Omega \nabla\cdot(2\nu\boldsymbol{\varepsilon}({\bf v}))\cdot\tilde{\bf u}{\rm d}{\bf x},\,\int_\Gamma -2\nu\boldsymbol{\varepsilon}_{\bf n}({\bf v})\cdot\tilde{\bf u}{\rm d}\Gamma,\,\int_\Gamma 2\nu\boldsymbol{\varepsilon}_{\bf n}(\tilde{\bf u})\cdot{\bf v}{\rm d}\Gamma.
    \end{align*}
    \begin{itemize}[leftmargin=0mm]
        \item \textbf{Case 1.1: $\delta_{\mathcal{L}} = 0$.} Similarly, we obtain the following adjoint equation for \eqref{general stationary Stokes}:
        \begin{equation}
            \label{adjoint general stationary Stokes/case 2.1}
            \tag{adj-gS2.1}
            \left\{\begin{split}
                &\nabla\cdot(2\nu\boldsymbol{\varepsilon}({\bf v})) - \nabla_{\nabla{\bf u}}{\bf f}({\bf x},{\bf u},\nabla{\bf u},p):\nabla{\bf v} + \nabla_{\bf u}{\bf f}({\bf x},{\bf u},\nabla{\bf u},p){\bf v} - \nabla\cdot(\nabla_{\nabla{\bf u}}{\bf f}({\bf x},{\bf u},\nabla{\bf u},p))\cdot{\bf v}\\
                &\hspace{5mm}+ (1 - D_{\nabla{\bf u}}f_{\rm div}({\bf x},{\bf u},\nabla{\bf u},p))\nabla q + q[\nabla_{\bf u}f_{\rm div}({\bf x},{\bf u},\nabla{\bf u},p) - (\nabla\cdot(\nabla_{\nabla{\bf u}}f_{\rm div}({\bf x},{\bf u},\nabla{\bf u},p)))]\\
                &\hspace{1cm}= -\nabla_{\bf u}J_\Omega({\bf x},{\bf u},\nabla{\bf u},p) + \nabla\cdot(\nabla_{\nabla{\bf u}}J_\Omega({\bf x},{\bf u},\nabla{\bf u},p)),\mbox{ in }\Omega,\\
                &\nabla\cdot{\bf v} + D_p{\bf f}({\bf x},{\bf u},\nabla{\bf u},p)\cdot{\bf v} + qD_pf_{\rm div}({\bf x},{\bf u},\nabla{\bf u},p) = -D_pJ_\Omega({\bf x},{\bf u},\nabla{\bf u},p),\mbox{ in }\Omega,\\
                &-2\nu\boldsymbol{\varepsilon}_{\bf n}({\bf v}) + ((\nabla_{\nabla{\bf u}}{\bf f}({\bf x},{\bf u},\nabla{\bf u},p)\cdot{\bf n})\cdot{\bf v}) + q(D_{\nabla{\bf u}}f_{\rm div}({\bf x},{\bf u},\nabla{\bf u},p) - 1){\bf n}\\
                &\hspace{1cm}= -D_{\nabla{\bf u}}J_\Omega({\bf x},{\bf u},\nabla{\bf u},p){\bf n} - \nabla_{\bf u}J_\Gamma({\bf x},{\bf u},\nabla{\bf u},p,{\bf n},{\bf t}),\mbox{ on }\Gamma_{\rm v}^{\bf u},\\
                &{\bf v}\cdot{\bf n} = D_pJ_\Gamma({\bf x},{\bf u},\nabla{\bf u},p,{\bf n},{\bf t}),\mbox{ on }\Gamma_{\rm v}^p,\\
                &\nu({\bf n}\otimes{\bf v} + {\bf v}\otimes{\bf n}) = -\nabla_{\nabla{\bf u}}J_\Gamma({\bf x},{\bf u},\nabla{\bf u},p,{\bf n},{\bf t}),\mbox{ on }\Gamma.
            \end{split}\right.
        \end{equation}
        where in the last equation we have used the following to deduce,
        \begin{align*}
            2\nu\boldsymbol{\varepsilon}_{\bf n}(\tilde{\bf u})\cdot{\bf v} &= 2\nu{\bf n}^\top\boldsymbol{\varepsilon}(\tilde{\bf u}){\bf v} = \nu{\bf n}^\top(\nabla\tilde{\bf u} + D\tilde{\bf u}){\bf v} = \nu({\bf n}^\top\nabla\tilde{\bf u}{\bf v} + {\bf n}^\top D\tilde{\bf u}{\bf v})\\
            &= \nu({\bf n}^\top\nabla\tilde{\bf u}{\bf v} + {\bf v}^\top\nabla\tilde{\bf u}{\bf n}) = \nu({\bf n}\otimes{\bf v} + {\bf v}\otimes{\bf n}):\nabla\tilde{\bf u}.
        \end{align*}
    \end{itemize}
    \textbf{Case 1.2: $\delta_{\mathcal{L}} = 1$.} Similarly, we obtain the following adjoint equation for \eqref{general stationary Stokes}:
    \begin{equation}
        \label{adjoint general stationary Stokes/case 2.2}
        \tag{adj-gS2.2}
        \left\{\begin{split}
            &\nabla\cdot(2\nu\boldsymbol{\varepsilon}({\bf v})) - \nabla_{\nabla{\bf u}}{\bf f}({\bf x},{\bf u},\nabla{\bf u},p):\nabla{\bf v} + \nabla_{\bf u}{\bf f}({\bf x},{\bf u},\nabla{\bf u},p){\bf v} - \nabla\cdot(\nabla_{\nabla{\bf u}}{\bf f}({\bf x},{\bf u},\nabla{\bf u},p))\cdot{\bf v}\\
            &\hspace{5mm}+ (1 - D_{\nabla{\bf u}}f_{\rm div}({\bf x},{\bf u},\nabla{\bf u},p))\nabla q + q[\nabla_{\bf u}f_{\rm div}({\bf x},{\bf u},\nabla{\bf u},p) - (\nabla\cdot(\nabla_{\nabla{\bf u}}f_{\rm div}({\bf x},{\bf u},\nabla{\bf u},p)))]\\
            &\hspace{1cm}= -\nabla_{\bf u}J_\Omega({\bf x},{\bf u},\nabla{\bf u},p) + \nabla\cdot(\nabla_{\nabla{\bf u}}J_\Omega({\bf x},{\bf u},\nabla{\bf u},p)),\mbox{ in }\Omega,\\
            &\nabla\cdot{\bf v} + D_p{\bf f}({\bf x},{\bf u},\nabla{\bf u},p)\cdot{\bf v} + qD_pf_{\rm div}({\bf x},{\bf u},\nabla{\bf u},p) = -D_pJ_\Omega({\bf x},{\bf u},\nabla{\bf u},p),\mbox{ in }\Omega,\\
            &-2\nu\boldsymbol{\varepsilon}_{\bf n}({\bf v}) + ((\nabla_{\nabla{\bf u}}{\bf f}({\bf x},{\bf u},\nabla{\bf u},p)\cdot{\bf n})\cdot{\bf v}) + q(D_{\nabla{\bf u}}f_{\rm div}({\bf x},{\bf u},\nabla{\bf u},p) - 1){\bf n} - \nabla_{\bf u}{\bf Q}({\bf x},{\bf u},\nabla{\bf u},p,{\bf n},{\bf t}){\bf v}_{\rm bc}\\
            &\hspace{1cm}= -D_{\nabla{\bf u}}J_\Omega({\bf x},{\bf u},\nabla{\bf u},p){\bf n} - \nabla_{\bf u}J_\Gamma({\bf x},{\bf u},\nabla{\bf u},p,{\bf n},{\bf t}),\mbox{ on }\Gamma,\\
            &{\bf v}\cdot{\bf n} - D_p{\bf Q}({\bf x},{\bf u},\nabla{\bf u},p,{\bf n},{\bf t})\cdot{\bf v}_{\rm bc} = D_pJ_\Gamma({\bf x},{\bf u},\nabla{\bf u},p,{\bf n},{\bf t}),\mbox{ on }\Gamma,\\
            &\nu({\bf n}\otimes{\bf v} + {\bf v}\otimes{\bf n}) - \nabla_{\nabla{\bf u}}{\bf Q}({\bf x},{\bf u},\nabla{\bf u},p,{\bf n},{\bf t})\cdot{\bf v}_{\rm bc} = -\nabla_{\nabla{\bf u}}J_\Gamma({\bf x},{\bf u},\nabla{\bf u},p,{\bf n},{\bf t}),\mbox{ on }\Gamma.
        \end{split}\right.
    \end{equation}
\end{enumerate}

\subsection{Shape derivatives of \eqref{general stationary Stokes}-constrained \eqref{general cost functional for u, p}}
To calculate the shape derivatives of \eqref{general cost functional for u, p} under the constraint state equation \eqref{general stationary Stokes}, we consider the following \textit{perturbed cost functional}
\begin{align}
    \label{perturbed cost functional for stationary Stokes}
    J({\bf u}_t,p_t,\Omega_t)\coloneqq\int_{\Omega_t} J_\Omega({\bf x},{\bf u}_t,\nabla{\bf u}_t,p_t){\rm d}{\bf x} + \int_{\Gamma_t} J_\Gamma({\bf x},{\bf u}_t,\nabla{\bf u}_t,p_t,{\bf n}_t,{\bf t}_t){\rm d}\Gamma_t,
\end{align}
where $({\bf u}_t,p_t)$ denotes the strong\texttt{/}classical solution of \eqref{general stationary Stokes} on the perturbed domain $\Omega_t\coloneqq T_t(V)(\Omega)$, i.e.:
\begin{equation}
    \label{perturbed stationary Stokes}
    \tag{ptb-S}
    \left\{\begin{split}
        -\nu\Delta{\bf u}_t + \nabla p_t &= {\bf f} &&\mbox{ in } \Omega_t,\\
        \nabla\cdot{\bf u}_t &= f_{\rm div} &&\mbox{ in } \Omega_t,\\
        \gamma_0{\bf u}_t = {\bf u}_\Gamma \mbox{ i.e., } {\bf u}_t &= {\bf u}_\Gamma &&\mbox{ on } \Gamma_t,
    \end{split}\right.
\end{equation}
where $\Gamma_t\coloneqq\partial\Omega_t$.

Now subtracting \eqref{perturbed stationary Stokes} to \eqref{general stationary Stokes} side by side, and taking $\lim_{t\downarrow 0}$, obtain:

\subsection{Existence, uniqueness, and regularity}
We consider the following stationary Stokes equations:
\begin{equation}
    \label{stationary Stokes: Dirichlet BC}
    \tag{S}
    \left\{\begin{split}
        -\nu\Delta{\bf u} + \nabla p &= {\bf f},&&\mbox{ in } \Omega,\\
        \nabla\cdot{\bf u} &= f_{\rm div},&&\mbox{ in } \Omega,\\
        \gamma_0{\bf u} = {\bf u}_\Gamma \mbox{ i.e., } {\bf u} &= {\bf u}_\Gamma,&&\mbox{ on } \Gamma,
    \end{split}\right.
\end{equation}
where $\gamma_0\in\mathcal{L}(H^1(\Omega),L^2(\Gamma))$ is the trace operator s.t. $\gamma_0u =$ the restriction of $u$ to $\Gamma$ for every function $u\in H^1(\Omega)$ (see, e.g., \cite[p. 6]{Temam2000}).

\begin{theorem}[Case $f_{\rm div} = 0$, ${\bf u}_\Gamma = {\bf 0}$]
    \begin{itemize}
        \item[(i)] (Existence) For any open set $\Omega\subset\mathbb{R}^N$ which is bounded in some direction, and for every ${\bf f}\in{\bf H}^{-1}(\Omega)$, the problem
        \begin{align}
            \label{Temam2000 (2.6)}
            {\bf u} \mbox{ belongs to } V \mbox{ and satisfies } \nu(({\bf u},{\bf v})) = \langle{\bf f},{\bf v}\rangle_{V^\star,V},\ \forall{\bf v}\in V
        \end{align}
        has a unique solution ${\bf u}$.
        
        Moreover, there exists a function $p\in L_{\rm loc}^2(\Omega)$ s.t. the following 2 statements are satisfied:
        \begin{itemize}
            \item[(a)] there exists $p\in L^2(\Omega)$ s.t. $-\nu\Delta{\bf u} + \nabla p = {\bf f}$ in the distribution sense in $\Omega$;
            \item[(b)] $\nabla\cdot{\bf u} = 0$ in the distribution sense in $\Omega$.
        \end{itemize}    
        If $\Omega$ is an open bounded set of class $C^2$, then $p\in L^2(\Omega)$ and (a), (b), and $\gamma_0{\bf u} = {\bf 0}$ are satisfied by ${\bf u}$ and $p$.
        \item[(ii)] (A variational property) The solution ${\bf u}$ of \eqref{Temam2000 (2.6)} is also the unique element of $V$ s.t.
        \begin{align*}
            E({\bf u})\le E({\bf v}),\ \forall{\bf v}\in V, \mbox{ where } E({\bf v})\coloneqq\nu\|{\bf v}\|^2 - 2\langle{\bf f},{\bf v}\rangle_{V^\star,V}.
        \end{align*}
    \end{itemize}    
\end{theorem}

\begin{theorem}[Non-homogeneous Stokes: Existence]
    \begin{itemize}
        \item[(i)] Let $\Omega$ be an open bounded set of class $C^2$ in $\mathbb{R}^N$. Let there be given ${\bf f}\in{\bf H}^{-1}(\Omega)$, $f_{\rm div}\in L^2(\Omega)$, ${\bf u}_\Gamma\in{\bf H}^{1/2}(\Gamma)$, s.t.
        \begin{align*}
            \int_\Omega f_{\rm div}{\rm d}{\bf x} = \int_\Gamma {\bf u}_\Gamma\cdot{\bf n}{\rm d}\Gamma.
        \end{align*}
        Then there exists ${\bf u}\in{\bf H}^1(\Omega)$, $p\in L^2(\Omega)$, which are solution of the non-homogeneous Stokes problem \eqref{stationary Stokes: Dirichlet BC}, ${\bf u}$ is unique and $p$ is unique up to the addition of a constant.
        \item[(ii)] Let $\Omega$ be a Lipschitz open bounded set in $\mathbb{R}^N$. Let there be given ${\bf f}\in{\bf H}^{-1}(\Omega)$, $f_{\rm div}\in L^2(\Omega)$, ${\bf u}_\Gamma$ as the trace of a function ${\bf u}_{\Gamma,0}\in{\bf H}^1(\Omega)$, s.t.
        \begin{align*}
            \int_\Omega f_{\rm div}{\rm d}{\bf x} = \int_\Omega \nabla\cdot{\bf u}_{\Gamma,0}{\rm d}{\bf x}.
        \end{align*}
        Then there exists ${\bf u}\in{\bf H}^1(\Omega)$, $p\in L^2(\Omega)$, which are solution of the non-homogeneous Stokes problem
        \begin{equation*}
            \left\{\begin{split}
                -\nu\Delta{\bf u} + \nabla p &= {\bf f} &&\mbox{ in } \Omega,\\
                \nabla\cdot{\bf u} &= f_{\rm div} &&\mbox{ in } \Omega,\\
                {\bf u} - {\bf u}_{\Gamma,0}&\in{\bf H}_0^1(\Omega).
            \end{split}\right.
        \end{equation*}
        ${\bf u}$ is unique and $p$ is unique up to the addition of a constant.
    \end{itemize}
\end{theorem}

\begin{theorem}[Non-homogeneous Stokes: regularity]
    \begin{itemize}
        \item[(i)] Let $\Omega$ be an open bounded set of class $C^r$, $r = \max(m + 2,2)$, $m$ integer $> 0$. Let us suppose that ${\bf u}\in{\bf W}^{2,\alpha}(\Omega)$, $p\in W^{1,\alpha}(\Omega)$, $1 < \alpha < +\infty$, are solutions of the generalized Stokes problem
        \begin{equation*}
            \left\{\begin{split}
                -\nu\Delta{\bf u} + \nabla p &= {\bf f} &&\mbox{ in } \Omega,\\
                \nabla\cdot{\bf u} &= f_{\rm div} &&\mbox{ in } \Omega,\\
                \gamma_0{\bf u} = {\bf u}_\Gamma \mbox{ i.e., } {\bf u} &= {\bf u}_\Gamma &&\mbox{ on } \Gamma,
            \end{split}\right.
        \end{equation*}
        If ${\bf u}\in{\bf W}^{m,\alpha}(\Omega)$, $f_{\rm div}\in W^{m+1,\alpha}(\Omega)$ and ${\bf u}_\Gamma\in{\bf W}^{m + 2 - 1/\alpha,\alpha}(\Gamma)$,\footnote{$W^{m + 2 - 1/\alpha,\alpha}(\Gamma) = \gamma_0W^{m+2,\alpha}(\Omega)$ and is equipped with the image norm
            \begin{align*}
                \|\psi\|_{W^{m + 2 - 1/\alpha,\alpha}(\Gamma)} = \inf_{\gamma_0{\bf u} = \psi} \|{\bf u}\|_{{\bf W}^{m+2,\alpha}(\Omega)}.
        \end{align*}} then ${\bf u}\in{\bf W}^{m+2,\alpha}(\Omega),\ p\in W^{m+1,\alpha}(\Omega)$ and there exists a constant $c_0(\alpha,\nu,m,\Omega)$ s.t.
        \begin{align}
            \label{Temam2000 (2.46)}
            &\|{\bf u}\|_{{\bf W}^{m+2,\alpha}(\Omega)} + \|p\|_{W^{m+1,\alpha}(\Omega)/\mathbb{R}}\nonumber\\
            &\le c_0\left\{\|{\bf f}\|_{{\bf W}^{m,\alpha}(\Omega)} + \|f_{\rm div}\|_{W^{m+1,\alpha}(\Omega)} + \|{\bf u}_\Gamma\|_{{\bf W}^{m + 2 - 1/\alpha,\alpha}(\Gamma)} + d_\alpha\|{\bf u}\|_{{\bf L}^\alpha(\Omega)}\right\},
        \end{align}
        where $d_\alpha = 0$ for $\alpha\ge 2$, $d_\alpha = 1$ for $1 < \alpha < 2$.
    \end{itemize}
\end{theorem}

\begin{proposition}
    Let $\Omega$ be an open set of $\mathbb{R}^N$, $N = 2$ or 3, of class $C^r$, $r = \max(m + 2,2)$, $m$ integer $\ge -1$, and let ${\bf f}\in{\bf W}^{m,\alpha}(\Omega)$, $f_{\rm div}\in W^{m+1,\alpha}(\Omega)$, ${\bf u}_\Gamma\in{\bf W}^{m + 2 - 1/\alpha,\alpha}(\Gamma)$ be given satisfying the compatibility condition
    \begin{align*}
        \int_\Omega f_{\rm div}{\rm d}{\bf x} = \int_\Gamma {\bf u}_\Gamma\cdot{\bf n}{\rm d}\Gamma.
    \end{align*}
    Then there exist unique functions ${\bf u}$ and $p$ ($p$ is unique up to a constant) which are solutions of \eqref{stationary Stokes: Dirichlet BC} and satisfy ${\bf u}\in{\bf W}^{m+2,\alpha}(\Omega)$, $p\in{\bf W}^{m+1,\alpha}(\Omega)$, and \eqref{Temam2000 (2.46)} with $d_\alpha = 0$ for any $\alpha$, $1 < \alpha < \infty$.
\end{proposition}

\section{Instationary Stokes equations*}
In this section, we consider the following instationary Stokes equations
\begin{equation}
    \label{instationary Stokes}
    \tag{iS}
    \left\{\begin{split}
        \partial_t{\bf u} - \nu\Delta{\bf u} + ({\bf u}\cdot\nabla){\bf u} + \nabla p &= {\bf f},&&\mbox{ in }(0,T)\times\Omega,\\
        \nabla\cdot{\bf u} &= f_{\rm div},&&\mbox{ in }(0,T)\times\Omega.
    \end{split}\right.
\end{equation}
If we set
\begin{align*}
    P\coloneqq\begin{bmatrix}
        -\nu\Delta & \nabla\\ \nabla\cdot & 0
    \end{bmatrix}\begin{bmatrix}
        {\bf u}\\ p
    \end{bmatrix} = \begin{bmatrix}
        {\bf f}\\ f_{\rm div}
    \end{bmatrix},
\end{align*}
then \eqref{instationary Stokes} can be rewritten in the form of \eqref{general nonhomogeneous BVP}.

\chapter{Shape Optimization for Incompressible Navier-Stokes Equations}

\section{Incompressible Navier-Stokes equations: Various Variants}
For the derivation of Navier-Stokes equation in general and incompressible Navier-Stokes equation in particular, see, e.g., \cite{Ferziger_Peric_Street2020}, \cite{Moukalled_Mangani_Darwish2016}, \cite{Rebollo_Lewandowski2014}.

We consider the following continuity and momentum equations of a general incompressible Navier-Stokes equations (see, e.g., \cite[Subsubsect. 1.7.1]{Ferziger_Peric_Street2020}):
\begin{equation}
    \left\{\begin{split}
        \partial_t{\bf u} - \operatorname{diff}(\nu,{\bf u}) + \operatorname{conv}({\bf u}) + \nabla p &= f({\bf x},{\bf u},\nabla{\bf u},p),&&\mbox{ in }(0,T)\times\Omega,\\
        \nabla\cdot{\bf u} &= f_{\rm div}(t,{\bf x},{\bf u},\nabla{\bf u},p),&&\mbox{ in }(0,T)\times\Omega.
    \end{split}\right.
\end{equation}

\begin{equation}
    \left\{\begin{split}
        \partial_t(\rho{\bf u}) - \operatorname{diff}(\nu,\rho,{\bf u}) + \operatorname{conv}(\rho,{\bf u}) + \nabla p &= f({\bf x},\rho,{\bf u},\nabla{\bf u},p),&&\mbox{ in }(0,T)\times\Omega,\\
        \partial_t\rho + \nabla\cdot(\rho{\bf u}) &=0,&&\mbox{ in }(0,T)\times\Omega.
    \end{split}\right.
\end{equation}

\begin{equation}
    \label{instationary incompressible NSEs}
    \left\{\begin{split}
        \partial_t{\bf u} + \nabla\cdot({\bf u}\otimes{\bf u}) &= \nabla\cdot(\nu\nabla{\bf u}) - \frac{1}{\rho}\nabla p + {\bf b}, &&\mbox{ in } (0,T)\times\Omega,\\
        \nabla\cdot{\bf u} &= 0, &&\mbox{ in } (0,T)\times\Omega.
    \end{split}\right.
\end{equation}
Newtonian NSEs:
\begin{equation}
    \label{Newtonian NSEs}
    \left\{\begin{split}
        \partial_t\rho + \nabla\cdot(\rho{\bf u}) &= 0, &&\mbox{ in } (0,T)\times\Omega,\\
        \partial_t(\rho{\bf u}) + \nabla\cdot(\rho{\bf u}\otimes{\bf u}) &= -\nabla p + \nabla\cdot\left[\mu(\nabla{\bf u} + (\nabla{\bf u})^\top)\right] + {\bf b}, &&\mbox{ in } (0,T)\times\Omega.
    \end{split}\right.
\end{equation}
General NSEs:
\begin{equation}
    \label{general NSEs}
    \tag{gNSEs}
    \left\{\begin{split}
        \partial_t\rho + \nabla\cdot(\rho{\bf u}) &= 0, &&\mbox{ in } (0,T)\times\Omega,\\
        \partial_t(\rho{\bf u}) + \nabla\cdot(\rho{\bf u}\otimes{\bf u}) &= \nabla\cdot\boldsymbol{\tau} - \nabla p + \rho{\bf b}, &&\mbox{ in } (0,T)\times\Omega.
    \end{split}\right.
\end{equation}

We recall various variants of incompressible Navier-Stokes equations in the literature (see, e.g., \cite[Subsect. 2.6.4]{Rebollo_Lewandowski2014}).
\begin{itemize}
    \item \textbf{Basic form.}
    \begin{equation}
        \label{incompressible instationary NSEs: basic form}
        \left\{\begin{split}
            \partial_t{\bf u} - \nabla\cdot(2\nu\boldsymbol{\varepsilon}({\bf u})) + ({\bf u}\cdot\nabla){\bf u} + \nabla p &= {\bf f},&&\mbox{ in }(0,T)\times\Omega,\\
            \nabla\cdot{\bf u} &= 0,&&\mbox{ in }(0,T)\times\Omega,\\
            {\bf u}_0 &= {\bf u}(0,\cdot),&&\mbox{ in }\Omega,
        \end{split}\right.
    \end{equation}
    where the \textit{external forcing} ${\bf f}:(0,T)\times\Omega\to\mathbb{R}^N$ and the \textit{initial velocity} ${\bf u}_0:\Omega\to\mathbb{R}^N$ are given.
    
    The knowledge of the initial value of the pressure $p$ is not required here since it is not a prognostic variable.
    \item \textbf{Constant viscosity case.} If we consider an \textit{adiabatic flow} whose viscosity $\nu$ remains constant, then \eqref{incompressible instationary NSEs: basic form} becomes
    \begin{equation}
        \label{incompressible instationary NSEs: constant viscosity}
        \left\{\begin{split}
            \partial_t{\bf u} - \nu\Delta{\bf u} + ({\bf u}\cdot\nabla){\bf u} + \nabla p &= {\bf f},&&\mbox{ in }(0,T)\times\Omega,\\
            \nabla\cdot{\bf u} &= 0,&&\mbox{ in }(0,T)\times\Omega,\\
            {\bf u}_0 &= {\bf u}(0,\cdot),&&\mbox{ in }\Omega,
        \end{split}\right.
    \end{equation}
    \item \textbf{The nonlinear term in divergence form.}
    \begin{equation}
        \left\{\begin{split}
            \partial_t{\bf u} - \nabla\cdot(2\nu\boldsymbol{\varepsilon}({\bf u})) + \nabla\cdot({\bf u}\otimes{\bf u}) + \nabla p &= {\bf f},&&\mbox{ in }(0,T)\times\Omega,\\
            \nabla\cdot{\bf u} &= 0,&&\mbox{ on }(0,T)\times\Omega,
        \end{split}\right.
    \end{equation}
    or equivalently,
    \begin{equation}
        \label{incompressible instationary NSEs: conservative form}
        \left\{\begin{split}
            \partial_t{\bf u} + \nabla\cdot(-2\nu\boldsymbol{\varepsilon}({\bf u}) + {\bf u}\otimes{\bf u} + p{\bf I}) &= {\bf f},&&\mbox{ in }(0,T)\times\Omega,\\
            \nabla\cdot{\bf u} &= 0,&&\mbox{ on }(0,T)\times\Omega,
        \end{split}\right.
    \end{equation}
    which may be interesting, especially when ${\bf f}$ is a \textit{restoring force}, i.e., ${\bf f} = \nabla\cdot\boldsymbol{\phi}$ for some vector field $\boldsymbol{\phi}$.
    
    \begin{remark}
        The system \eqref{incompressible instationary NSEs: conservative form} indicates that the general incompressible instationary Navier-Stokes equations can be considered as a \emph{conservative law} of the form
        \begin{equation*}
            \left\{\begin{split}
                \partial_t{\bf u} + \nabla\cdot P({\bf u},p) &= 0,&&\mbox{ in }(0,T)\times\Omega,\\
                \nabla\cdot{\bf u} &= 0,&&\mbox{ in }(0,T)\times\Omega.
            \end{split}\right.
        \end{equation*}
        Thus standard techniques of conservation laws can be then applied to study \eqref{incompressible instationary NSEs: basic form}, see, e.g., \cite{LeVeque2002}.
    \end{remark}
    \item \textbf{Form with the vorticity.}
    \begin{equation}
        \label{incompressible instationary NSEs: vorticity form}
        \left\{\begin{split}
            \partial_t{\bf u} - \nabla\cdot(2\nu\boldsymbol{\varepsilon}({\bf u})) + \boldsymbol{\omega}\times{\bf u} + \nabla\left(p + \frac{|{\bf u}|^2}{2}\right) &= {\bf f},&&\mbox{ in }(0,T)\times\Omega,\\
            \nabla\cdot{\bf u} &= 0,&&\mbox{ in }(0,T)\times\Omega.
        \end{split}\right.
    \end{equation}
    \item \textbf{Rotating fluids.} With $\boldsymbol{\Omega}$ an \textit{angular velocity},
    \begin{equation}
        \label{incompressible instationary NSEs: rotating fluid}
        \left\{\begin{split}
            \partial_t{\bf u} - \nabla\cdot(2\nu\boldsymbol{\varepsilon}({\bf u})) + ({\bf u}\cdot\nabla){\bf u} - 2\boldsymbol{\Omega}\times{\bf u} + \nabla p &= {\bf f},&&\mbox{ in }(0,T)\times\Omega,\\
            \nabla\cdot{\bf u} &= 0,&&\mbox{ in }(0,T)\times\Omega.
        \end{split}\right.
    \end{equation}
\end{itemize}

\section{Boundary conditions}
See \cite[Sect. 2.7]{Rebollo_Lewandowski2014} for a comprehensive list of boundary conditions which are commonly used in the literature of fluid dynamics.

We list here some possibilities of the boundary operators $(Q_i)_{1\le i\le n_{\rm bc}}$ in \eqref{general nonhomogeneous BVP}.
\begin{itemize}
    \item \textbf{Periodic boundary conditions.} Despite being the least physical boundary conditions, periodic boundary conditions still remain very popular due to their great advantage of using Fourier analysis to study \eqref{incompressible instationary NSEs: basic form}, especially in the case of constant viscosity, i.e., \eqref{incompressible instationary NSEs: constant viscosity}.
    \item \textbf{The full space.} In this case, the flow domain $\Omega$ is $\mathbb{R}^3$. The following \textit{integrability condition} is imposed:
    \begin{align*}
        {\bf u}(t,\cdot)\in L^2(\mathbb{R}^3)^3,\mbox{ for almost all } t\in\mathbb{R}_{\ge 0}.
    \end{align*}
    \item \textbf{No-slip condition.} The \textit{no-slip condition} is of the following form
    \begin{align}
        \label{no-slip BC}
        \tag{no-slipBC}
        {\bf u}(t,{\bf x}) = {\bf 0},\ \forall(t,{\bf x})\in\mathbb{R}_{\ge 0}\times\Gamma,
    \end{align}
    or more simply ${\bf u}|_\Gamma = {\bf 0}$, which is also called \textit{homogeneous Dirichlet boundary condition} as commonly used in PDEs.
    \item \textbf{Navier boundary condition.} The Navier boundary condition represents a balance between slip and friction.
    
    Assume that $\Gamma$ is not \textit{porous}, such that no fluid particle crosses $\Gamma$, which means ${\bf u} = {\bf u}_\tau$ a.e. on $\Gamma$, or equivalently the \textit{impermeability condition} (see, e.g., \cite[Sect. 2.4]{Pope2000}) ${\bf u}\cdot{\bf n}|_\Gamma = 0$.
    
    The Navier boundary conditions are
    \begin{align*}
        {\bf u}\cdot{\bf n}|_\Gamma = 0,\ ({\bf u} + \alpha(\boldsymbol{\sigma}\cdot{\bf n}_\tau))|_\Gamma = {\bf 0},\mbox{ for some }\alpha > 0.
    \end{align*}
    \item \textbf{Wall law.}
    \begin{align*}
        {\bf u}\cdot{\bf n}|_\Gamma = 0,\ (\boldsymbol{\sigma}\cdot{\bf n})_\tau|_\Gamma = C({\bf U}_0 - {\bf u}_\tau)|{\bf U}_0 - {\bf u}_\tau|,\mbox{ for some given }{\bf U}_0.
    \end{align*}
\end{itemize}

\section{FVM for \eqref{general NSEs}*}

\subsection{Discretization of convection term*}

\subsection{Discretization of viscous term*}

\section{Domains in $\mathbb{R}^d$}
We denote by $\Omega\subset\mathbb{R}^d$ an open bounded set (hence $\overline{\Omega}$ is compact).  Let $\Gamma := \partial\Omega$ denote the boundary of $\Omega$ and ${\bf n}$ its outer normal vector. Note that if $\Gamma$ is $C^{0,1}$ then ${\bf n}\in L^\infty(\Gamma)$. We denote by $\mathcal{N}_0$ an unitary extension of ${\bf n}$, $\mathcal{N}_0$ is defined in a neighborhood of $\overline{\Omega}$ in $\mathbb{R}^d$.

\section{NSEs}
Let $\Omega\subset\mathbb{R}^d$, $d\in\{2,3\}$, be an open bounded connected set. We consider the BVP for the stationary Navier-Stokes equations with mixed boundary conditions:
\begin{equation}
    \label{NSEs}
    \tag{NSEs}
    \left\{\begin{split}
        ({\bf u}\cdot\nabla){\bf u} - \nu\Delta{\bf u} + \nabla p &= {\bf f} &&\mbox{ in } \Omega,\\
        \nabla\cdot{\bf u} &= 0 &&\mbox{ in } \Omega,\\
        {\bf u} &= {\bf f}_{\rm in} &&\mbox{ on } \Gamma_{\rm in},\\
        {\bf u} &= {\bf 0} &&\mbox{ on } \Gamma_{\rm wall},\\
        -\nu\partial_{\bf n}{\bf u} + p{\bf n} &= {\bf 0} &&\mbox{ on } \Gamma_{\rm out}.
    \end{split}\right.
\end{equation}
The velocity vector is denoted by ${\bf u}:\Omega\to\mathbb{R}^d$ and $p:\Omega\to\mathbb{R}$ denotes the kinematic pressure. We assume that the inflow profile ${\bf f}_{\rm in}\in H^{1/2}(\Gamma_{\rm in})^d$, the kinematic viscosity $\nu > 0$ and the density of external volume force ${\bf f}:\Omega\to\mathbb{R}^d$ are given.

\begin{definition}[Weak solution]
    A vector function $({\bf u},p)\in W^{1,2}(\Omega)^3\times L^2(\Omega)$ is called a \emph{weak solution} of \eqref{NSEs} if it satisfies
    \begin{align}
        \int_\Omega \nu\nabla{\bf u}:\nabla{\bf v} + ({\bf u}\cdot\nabla){\bf u}\cdot{\bf v} - p\nabla\cdot{\bf v}{\rm d}x &= \int_\Omega {\bf f}\cdot{\bf v}{\rm d}x,\ \forall{\bf v}\in V(\Omega),\\
        \nabla\cdot{\bf u} = 0 \mbox{ in } \Omega,\ {\bf u}|_{\Gamma_{\rm in}} &= {\bf f}_{\rm in},\ {\bf u}|_{\Gamma_{\rm wall}} = {\bf 0},
    \end{align}
    where $V(\Omega) = \{{\bf v}\in W^{1,2}(\Omega)^3;{\bf v}|_{\Gamma_{\rm in}\cup\Gamma_{\rm wall}} = {\bf 0}\}$.
\end{definition}

\section{Wellposedness of NSEs}
[See Martin Kanitsar's draft/Maz'ya-Rossmann]

\section{Cost functionals}
Outflow uniformity: The uniformity of the flow upon leaving the outlet plane is an important design criterion of e.g. \textit{automotive air ducts}. Other use: Efficiency of distributing fresh air inside the car.
\begin{align}
    J_1({\bf u},\Omega) := \frac{1}{2}\int_{\Gamma_{\rm out}} ({\bf u}\cdot{\bf n} - \overline{u})^2{\rm d}s \mbox{ with } \overline{u} := -\frac{1}{|\Gamma_{\rm out}|}\int_{\Gamma_{\rm in}} {\bf f}_{\rm in}\cdot{\bf n}{\rm d}s.
\end{align}
Energy dissipation: Compute power dissipated by a fluid dynamic device as the net inward flux of energy.

I.e., total pressure, through the device boundaries for smooth pressure $p$: $\epsilon > 0$
\begin{align}
    J_2^\epsilon({\bf u},p,\Omega) := -\frac{|\Gamma_{\rm in}|}{|\Gamma_{\rm in}^\epsilon|}\int_{\Gamma_{\rm in}^\epsilon} \left(p + \frac{1}{2}|{\bf u}|^2\right){\bf u}\cdot{\bf n}{\rm d}x - \frac{|\Gamma_{\rm out}|}{|\Gamma_{\rm out}^\epsilon|}\int_{\Gamma_{\rm out}^\epsilon} \left(p + \frac{1}{2}|{\bf u}|^2\right){\bf u}\cdot{\bf n}{\rm d}x = \int_\Omega k_\epsilon\left(p + \frac{1}{2}|{\bf u}|^2\right){\bf u}\cdot{\bf n}{\rm d}x,
\end{align}
where
\begin{align}
    k_\epsilon(x):= -\frac{|\Gamma_{\rm in}|}{|\Gamma_{\rm in}^\epsilon|}\chi_{\overline{\Gamma_{\rm in}^\epsilon}}(x) - \frac{|\Gamma_{\rm out}|}{|\Gamma_{\rm out}^\epsilon|}\chi_{\overline{\Gamma_{\rm out}^\epsilon}}(x),\ \forall x\in\Omega,
\end{align}
here $\chi_A$ denotes the characteristic function of at set $A$.

Note that we have used Lebesgue measure in $\mathbb{R}^d$ for $\Gamma_{\rm in}^\epsilon,\Gamma_{\rm out}^\epsilon$ and Lebesgue measure in $\mathbb{R}^{d-1}$ for $\Gamma_{\rm in},\Gamma_{\rm out}$.

We consider the mixed cost functional with a weighting parameter $\gamma\in[0,1]$,
\begin{align}
    J_{12}^{\epsilon,\gamma}({\bf u},p,\Omega) :=&\ (1 - \gamma)J_1({\bf u},\Omega) + \gamma J_2^\epsilon({\bf u},p,\Omega)\\
    =&\ \frac{1 - \gamma}{2}\int_{\Gamma_{\rm out}} ({\bf u}\cdot{\bf n} - \overline{u})^2{\rm d}s + \int_\Omega \gamma k_\epsilon\left(p + \frac{1}{2}|{\bf u}|^2\right){\bf u}\cdot{\bf n}{\rm d}x.
\end{align}
Decomposing $J_{12}^{\epsilon,\gamma}$ into contributions from the boundary $\Gamma = \partial\Omega$ and from the interior of $\Omega$,
\begin{align}
    J_{12}^{\epsilon,\gamma}({\bf u},p,\Omega) = \int_\Gamma J_\Gamma{\rm d}s + \int_\Omega J_\Omega{\rm d}x,
\end{align}
thus
\begin{align}
    \partial_{\bf u}J_{12}^{\epsilon,\gamma}({\bf u},p,\Omega)\cdot\delta{\bf u} = \int_\Gamma \partial_{\bf u}J_\Gamma\cdot\delta{\bf u}{\rm d}s + \int_\Omega \partial_{\bf u}J_\Omega\cdot\delta{\bf u}{\rm d}x,\ \partial_pJ_{12}^{\epsilon,\gamma}\delta p = \int_\Gamma \partial_pJ_\Gamma\delta p{\rm d}s + \int_\Omega \partial_pJ_\Omega\delta p{\rm d}x.
\end{align}
We have
\begin{align}
    J_\Omega({\bf u},p) &= \gamma k_\epsilon\left(p + \frac{1}{2}|{\bf u}|^2\right){\bf u}\cdot{\bf n},\\
    \partial_{\bf u}J_\Omega({\bf u},p) &= \gamma k_\epsilon\left(\left(p + \frac{1}{2}|{\bf u}|^2\right){\bf n} + ({\bf u}\cdot{\bf n}){\bf u}\right),\ \partial_pJ_\Omega({\bf u},p) = \gamma k_\epsilon{\bf u}\cdot{\bf n},
\end{align}
and
\begin{align}
    J_\Gamma({\bf u}) &= \frac{1 - \gamma}{2}({\bf u}\cdot{\bf n} - \overline{u})^2,\\
    \partial_{\bf u}J_\Gamma({\bf u}) &= (1 - \gamma)({\bf u}\cdot{\bf n} - \overline{u}){\bf n},\ \partial_pJ_\Gamma({\bf u}) = 0.
\end{align}

\section{Optimization problem}
The optimization problems can be formulated as follows: Find $\Omega$ over a class of admissible domain $\mathcal{O}_{\rm ad}$ such that the cost functional $J_{12}^{\epsilon,\gamma}$ is minimized subject to the NSEs \eqref{NSEs},
\begin{align}
    \label{SOP}
    \min_{\Omega\in\mathcal{O}_{\rm ad}} J_{12}^{\epsilon,\gamma}({\bf u},p,\Omega) \mbox{ such that } ({\bf u},p) \mbox{ solves \eqref{NSEs}}.
\end{align}

\section{Derive adjoint equation of NSEs - Kasumba, Kunisch version}
For each weak solution $({\bf u},p)$ of \eqref{NSEs}, we introduce the associated adjoint equation which is given by
\begin{equation}
    \left\{\begin{split}
        -\nu\Delta{\bf v} + \nabla{\bf u}{\bf v} - ({\bf u}\cdot\nabla){\bf v} + \nabla q &= \gamma k_\epsilon\left(\left(p + \frac{1}{2}|{\bf u}|^2\right){\bf n} + ({\bf u}\cdot{\bf n}){\bf u}\right) &&\mbox{ in } \Omega,\\
        \nabla\cdot{\bf v} &= -\gamma k_\epsilon{\bf u}\cdot{\bf n} &&\mbox{ in } \Omega,\\
        {\bf v} &= {\bf 0} &&\mbox{ on } \Gamma_{\rm in}\cup\Gamma_{\rm wall},\\
        -({\bf u}\cdot{\bf n}){\bf v} - \nu\partial_{\bf n}{\bf v} + q{\bf n} &= -(1 - \gamma)({\bf u}\cdot{\bf n} - \overline{u}){\bf n} &&\mbox{ on } \Gamma_{\rm out}.
    \end{split}\right.
\end{equation}


\begin{proof}[Demonstration]
    Choose the Lagrange multiplier $({\bf v},q)$ such that the variation with respect to the state variables vanishes identically, $\partial_{\bf u}\mathcal{L}\cdot\delta{\bf u} + \partial_p\mathcal{L}\delta p = 0$, which reads as
    \begin{align}
        \partial_{\bf u}J_{12}^{\epsilon,\gamma}\cdot\delta{\bf u} + \partial_pJ_{12}^{\epsilon,\gamma}\delta p &- \int_\Omega {\bf v}\cdot\left(- \nu\Delta\delta{\bf u} + (\delta{\bf u}\cdot\nabla){\bf u} + ({\bf u}\cdot\nabla)\delta{\bf u}\right) - q\nabla\cdot\delta{\bf u}{\rm d}x - \int_\Omega {\bf v}\cdot\nabla\delta p{\rm d}x\nonumber\\
        &- \int_{\Gamma_{\rm in}} {\bf v}_{\rm in}\cdot\delta{\bf u}{\rm d}s - \int_{\Gamma_{\rm wall}} {\bf v}_{\rm wall}\cdot\delta{\bf u}{\rm d}s - \int_{\Gamma_{\rm out}} {\bf v}_{\rm out}\cdot(\delta p{\bf n} - \nu\partial_{\bf n}\delta{\bf u}){\rm d}s = 0.\label{1.4.2}
    \end{align}
    We have
    \begin{align*}
        \int_\Omega {\bf v}\cdot((\delta{\bf u}\cdot\nabla){\bf u}){\rm d}x = \int_\Omega {\bf v}^\top D{\bf u}\delta{\bf u}{\rm d}x = \int_\Omega (\nabla{\bf u}{\bf v})\cdot\delta{\bf u}{\rm d}x.
    \end{align*}
    We integrate by parts term by term: the second term produced by the nonlinear term $({\bf u}\cdot\nabla){\bf u}$:
    \begin{align}
        \int_\Omega {\bf v}\cdot(({\bf u}\cdot\nabla)\delta{\bf u}){\rm d}x &= \int_\Omega \sum_{i=1}^d\sum_{j=1}^d v_iu_j\partial_{x_j}\delta u_i{\rm d}x = \int_\Gamma \sum_{i=1}^d\sum_{j=1}^d v_i\delta u_iu_jn_j{\rm d}s - \int_\Omega \sum_{i=1}^d\sum_{j=1}^d (\delta u_i\partial_{x_j}v_iu_j + \delta u_iv_i\partial_{x_j}u_j){\rm d}x\\
        &= \int_\Gamma ({\bf u}\cdot{\bf n})({\bf v}\cdot\delta{\bf u}){\rm d}s - \int_\Omega \left[({\bf u}\cdot\nabla){\bf v}\cdot\delta{\bf u} + \nabla\cdot{\bf u}({\bf v}\cdot\delta{\bf u})\right]{\rm d}x = \int_\Gamma ({\bf u}\cdot{\bf n})({\bf v}\cdot\delta{\bf u}){\rm d}s - \int_\Omega ({\bf u}\cdot\nabla){\bf v}\cdot\delta{\bf u}{\rm d}x,
    \end{align}
    Laplacian term:
    \begin{align}
        -\nu\int_\Omega {\bf v}\cdot\Delta\delta{\bf u}{\rm d}x &= -\nu\int_\Omega \sum_{i=1}^d\sum_{j=1}^d v_i\partial_{x_j}^2\delta u_i{\rm d}x = -\nu\int_\Gamma \sum_{i=1}^d\sum_{j=1}^d v_i\partial_{x_j}\delta u_in_j{\rm d}s + \nu\int_\Omega \sum_{i=1}^d\sum_{j=1}^d \partial_{x_j}v_i\partial_{x_j}\delta u_i{\rm d}x\\
        &= -\nu\int_\Gamma {\bf n}\cdot\nabla\delta{\bf u}\cdot{\bf v}{\rm d}s + \nu\int_\Gamma \sum_{i=1}^d\sum_{j=1}^d \partial_{x_j}v_i\delta u_in_j{\rm d}s - \nu \int_\Omega \sum_{i=1}^d\sum_{j=1}^d \partial_{x_j}^2v_i\delta u_i{\rm d}x\\
        &= -\nu\int_\Gamma {\bf n}\cdot\nabla\delta{\bf u}\cdot{\bf v}{\rm d}s + \nu\int_\Gamma {\bf n}\cdot\nabla{\bf v}\cdot\delta{\bf u}{\rm d}s - \nu\int_\Omega \Delta{\bf v}\cdot\delta{\bf u}{\rm d}x,
    \end{align}
    divergence term:
    \begin{align}
        -\int_\Omega q\nabla\cdot\delta{\bf u}{\rm d}x = \int_\Omega \delta{\bf u}\cdot\nabla q{\rm d}x - \int_\Gamma q\delta{\bf u}\cdot{\bf n}{\rm d}s,
    \end{align}
    and the term produced by $\nabla p$:
    \begin{align}
        \int_\Omega {\bf v}\cdot\nabla\delta p{\rm d}x = -\int_\Omega \delta p\nabla\cdot{\bf v}{\rm d}x + \int_\Gamma \delta p{\bf v}\cdot{\bf n}{\rm d}s.
    \end{align}
    We can reformulate \eqref{1.4.2} as
    \begin{align}
        &\int_\Gamma (-{\bf v}\cdot{\bf n} + \partial_pJ_\Gamma)\delta p{\rm d}s + \int_\Omega (\nabla\cdot{\bf v} + \partial_pJ_\Omega)\delta p{\rm d}x + \int_\Gamma \left(-({\bf u}\cdot{\bf n}){\bf v} - \nu\partial_{\bf n}{\bf v} + q{\bf n} + \partial_{\bf u}J_\Gamma\right)\cdot\delta{\bf u}{\rm d}s \nonumber\\
        &+\nu\int_\Gamma \partial_{\bf n}\delta{\bf u}\cdot{\bf v}{\rm d}s + \int_\Omega \left(-\nabla{\bf u}{\bf v} + ({\bf u}\cdot\nabla){\bf v} + \nu\Delta{\bf v} - \nabla q + \partial_{\bf u}J_\Omega\right)\cdot\delta{\bf u}{\rm d}x\nonumber\\
        &-\int_{\Gamma_{\rm in}} {\bf v}_{\rm in}\cdot\delta{\bf u}{\rm d}s - \int_{\Gamma_{\rm wall}} {\bf v}_{\rm wall}\cdot\delta{\bf u}{\rm d}s - \int_{\Gamma_{\rm out}} {\bf v}_{\rm out}\cdot(\delta p{\bf n} - \nu\partial_{\bf n}\delta{\bf u}){\rm d}s = 0.\label{1.4.3}
    \end{align}
    Since this holds for any $\delta{\bf u}$ and $\delta p$ satisfying the primal NSEs, the integrals vanish individually. The vanishing of the integrals over the domain yields the adjoint NSEs
    \begin{equation}
        \left\{\begin{split}
            -\nu\Delta{\bf v} + \nabla{\bf u}{\bf v} - ({\bf u}\cdot\nabla){\bf v} + \nabla q &= \partial_{\bf u}J_\Omega = \gamma k_\epsilon\left(\left(p + \frac{1}{2}|{\bf u}|^2\right){\bf n} + ({\bf u}\cdot{\bf n}){\bf u}\right) &&\mbox{ in } \Omega,\\
            \nabla\cdot{\bf v} &= -\partial_pJ_\Omega = -\gamma k_\epsilon{\bf u}\cdot{\bf n} &&\mbox{ in } \Omega.
        \end{split}\right.    
    \end{equation}
    As boundary conditions for adjoint velocity and pressure, deduce from \eqref{1.4.3} that
    \begin{align}
        &\int_\Gamma \left(-({\bf u}\cdot{\bf n}){\bf v} - \nu\partial_{\bf n}{\bf v} + q{\bf n} + \partial_{\bf u}J_\Gamma\right)\cdot\delta{\bf u} + \nu\partial_{\bf n}\delta{\bf u}\cdot{\bf v}{\rm d}s - \int_{\Gamma_{\rm in}} {\bf v}_{\rm in}\cdot\delta{\bf u}{\rm d}s - \int_{\Gamma_{\rm wall}} {\bf v}_{\rm wall}\cdot\delta{\bf u}{\rm d}s + \int_{\Gamma_{\rm out}} \nu{\bf v}_{\rm out}\cdot\partial_{\bf n}\delta{\bf u}{\rm d}s = 0,\label{BC velocity}\\
        &\int_\Gamma (-{\bf v}\cdot{\bf n} + \partial_pJ_\Gamma)\delta p{\rm d}s - \int_{\Gamma_{\rm out}} {\bf v}_{\rm out}\cdot\delta p{\bf n}{\rm d}s = 0.\label{BC pressure}
    \end{align}
    Since $\partial_pJ_\Gamma = 0$ and ${\bf v} = \delta{\bf u} = {\bf 0}$ on $\Gamma_{\rm in}\cup\Gamma_{\rm wall}$, \eqref{BC pressure} reduces to
    \begin{align}
        \int_{\Gamma_{\rm out}} (-{\bf v}\cdot{\bf n} - {\bf v}_{\rm out}\cdot{\bf n})\delta p{\rm d}s = 0,
    \end{align}
    and thus we can set ${\bf v}_{\rm out} = -{\bf v}$ on $\Gamma_{\rm out}$.
    
    Similarly, \eqref{BC velocity} reduces to
    \begin{align}
        \int_{\Gamma_{\rm out}} \left(-({\bf u}\cdot{\bf n}){\bf v} - \nu\partial_{\bf n}{\bf v} + q{\bf n} + (1 - \gamma)({\bf u}\cdot{\bf n} - \overline{u}){\bf n}\right)\cdot\delta{\bf u} + \nu\partial_{\bf n}\delta{\bf u}\cdot{\bf v} + \nu{\bf v}_{\rm out}\cdot\partial_{\bf n}\delta{\bf u}{\rm d}s = 0.
    \end{align}
    Note that the sum of the last 2 terms vanishes, the last equation reduces to
    \begin{align}
        \int_{\Gamma_{\rm out}} \left(-({\bf u}\cdot{\bf n}){\bf v} - \nu\partial_{\bf n}{\bf v} + q{\bf n} + (1 - \gamma)({\bf u}\cdot{\bf n} - \overline{u}){\bf n}\right)\cdot\delta{\bf u}{\rm d}s = 0.
    \end{align}
    Thus,
    \begin{align}
        -({\bf u}\cdot{\bf n}){\bf v} - \nu\partial_{\bf n}{\bf v} + q{\bf n} = -(1 - \gamma)({\bf u}\cdot{\bf n} - \overline{u}){\bf n} \mbox{ on } \Gamma_{\rm out}.
    \end{align}
    We obtain the desired adjoint equation.
\end{proof}

\section{Adjoint equation of NSEs}
For each weak solution $({\bf u},p)$ of \eqref{NSEs}, we introduce the associated adjoint equation which is given by
\begin{equation}
    \label{adjoint NSEs}
    \tag{adjNSEs}
    \left\{\begin{split}
        -(\nabla{\bf v})^\top\cdot{\bf u} - \nabla{\bf v}\cdot{\bf u} - \nu\Delta{\bf v} + \nabla q &= -\gamma k_\epsilon\left(\left(p + \frac{1}{2}|{\bf u}|^2\right){\bf n} + ({\bf u}\cdot{\bf n}){\bf u}\right) &&\mbox{ in } \Omega,\\
        \nabla\cdot{\bf v} &= \gamma k_\epsilon{\bf u}\cdot{\bf n} &&\mbox{ in } \Omega,\\
        {\bf v} &= {\bf 0} &&\mbox{ on } \Gamma_{\rm in}\cup\Gamma_{\rm wall},\\
        ({\bf v}\cdot{\bf u}){\bf n} + ({\bf u}\cdot{\bf n}){\bf v} + \nu\partial_{\bf n}{\bf v} - q{\bf n} &= -(1 - \gamma)({\bf u}\cdot{\bf n} - \bar{u}){\bf n} &&\mbox{ on } \Gamma_{\rm out},
    \end{split}\right.    
\end{equation}

\begin{definition}[Weak solution]
    For a given weak solution $({\bf u},p)\in W^{1,2}(\Omega)^3\times L^2(\Omega)$ of \eqref{NSEs}, a vector function $({\bf v},q)\in V(\Omega)\times L^2(\Omega)$ is called a \emph{weak solution} of \eqref{adjoint NSEs} if it satisfies\footnote{Integrate by parts
        \begin{align}
            -\nu\int_\Omega \Delta{\bf v}\cdot{\bf w}{\rm d}x &= -\nu\int_\Gamma {\bf n}\cdot\nabla{\bf v}\cdot{\bf w}{\rm d}s + \nu\int_\Omega \nabla{\bf v}:\nabla{\bf w}{\rm d}x,\\
            \int_\Omega \nabla q\cdot{\bf w}{\rm d}x &= -\int_\Omega q\nabla\cdot{\bf w}{\rm d}x + \int_\Gamma q{\bf w}\cdot{\bf n}{\rm d}s.
    \end{align}}
    \begin{align}
        \label{weak formulation adjoint NSEs}
        \int_\Omega \nu\nabla{\bf v}:\nabla{\bf w} - ((\nabla{\bf v})^\top\cdot{\bf u} + \nabla{\bf v}\cdot{\bf u})\cdot{\bf w}  - q\nabla\cdot{\bf w}{\rm d}x &+ \int_{\Gamma_{\rm out}} (({\bf v}\cdot{\bf u}){\bf n} + ({\bf u}\cdot{\bf n}){\bf v} + (1 - \gamma)({\bf u}\cdot{\bf n} - \overline{u}){\bf n})\cdot{\bf w}{\rm d}s\\
        &= -\int_\Omega \gamma k_\epsilon\left(\left(p + \frac{1}{2}|{\bf u}|^2\right){\bf n} + ({\bf u}\cdot{\bf n}){\bf u}\right)\cdot{\bf w}{\rm d}x,\ \forall{\bf w}\in V(\Omega),\\
        \nabla\cdot{\bf v} &= \gamma k_\epsilon{\bf u}\cdot{\bf n} \mbox{ in } \Omega,\ {\bf v}|_{\Gamma_{\rm in}\cup\Gamma_{\rm wall}} = {\bf 0}.
    \end{align}
\end{definition}

\section{Direct computation of shape derivatives without Lagrangian}
From now on, let the perturbation field $V$ belong to the following space
\begin{align}
    \mathcal{F}_\epsilon := \{V\in C^{1,1}(\overline{D});V|_{\Gamma_{\rm in}^\epsilon\cup\Gamma_{\rm out}^\epsilon} = {\bf 0}\}.
\end{align}
There exists a unique $\hat{\bf f}_{\rm in}$ satisfying
\begin{equation*}
    \left\{\begin{split}
        \nabla\cdot\hat{\bf f}_{\rm in} &= 0 &&\mbox{ in } \Omega,\\
        \hat{\bf f}_{\rm in} &= {\bf f}_{\rm in} &&\mbox{ on } \Gamma_{\rm in},\\
        \hat{\bf f}_{\rm in} &= {\bf 0} &&\mbox{ on } \Gamma_{\rm wall}\cup\Gamma_{\rm out}.
    \end{split}\right.
\end{equation*}
Let $\hat{\bf u} = {\bf u} - \hat{\bf f}_{\rm in}$. Substituting $\hat{\bf u}$ into \eqref{NSEs} yields
\begin{equation}
    \left\{\begin{split}
        -\nu\Delta\hat{\bf u} + D\hat{\bf u}\cdot\hat{\bf u} + D\hat{\bf u}\cdot\hat{\bf f}_{\rm in} + D\hat{\bf f}_{\rm in}\cdot\hat{\bf u} + \nabla p &= {\bf F} &&\mbox{ in } \Omega,\\
        \nabla\cdot\hat{\bf u} &= 0 &&\mbox{ in } \Omega,\\
        \hat{\bf u} &= {\bf 0} &&\mbox{ on } \Gamma_{\rm in}\cup\Gamma_{\rm wall},\\
        -\nu\partial_{\bf n}\hat{\bf u} + p{\bf n} &= \nu\partial_{\bf n}\hat{\bf f}_{\rm in} &&\mbox{ on } \Gamma_{\rm out},
    \end{split}\right.
\end{equation}
where ${\bf F} := {\bf f} + \nu\Delta\hat{\bf f}_{\rm in} - D\hat{\bf f}_{\rm in}\cdot\hat{\bf f}_{\rm in}$.

\begin{proposition}
    We assume that the material derivative $d\hat{\bf u}[V]$ and $dp[V]$ exist. Then the shape derivatives $\hat{\bf u}'[V] = d\hat{\bf u}[V] - D\hat{\bf u}\cdot V$ and $p'[V] = dp[V] - \nabla p\cdot V$ exist by formal arguments, and they are characterized as the solution of the system
    \begin{equation*}
        \left\{\begin{split}
            -\nu\Delta\hat{\bf u}'[V] + D\hat{\bf u}\cdot(\hat{\bf u}'[V] + \hat{\bf g}'[V]) + D\hat{\bf u}'[V]\cdot{\bf u} + D\hat{\bf g}\cdot\hat{\bf u}'[V] + D\hat{\bf g}'[V]\cdot\hat{\bf u} + \nabla p'[V] &= {\bf F}'[V] &&\mbox{ in } \Omega,\\
            \nabla\cdot\hat{\bf u}'[V] &= 0 &&\mbox{ in } \Omega,\\
            \hat{\bf u}'[V] &= -V\cdot{\bf n}\partial_{\bf n}\hat{\bf u} &&\mbox{ on } \Gamma_{\rm in}\cup\Gamma_{\rm wall},\\
            -\nu\partial_{\bf n}\hat{\bf u}'[V] + p'[V]{\bf n} &= \nu\partial_{\bf n}\hat{\bf f}_{\rm in}'[V] &&\mbox{ on } \Gamma_{\rm out},
        \end{split}\right.
    \end{equation*}
    where ${\bf F}'[V] = \nu\Delta\hat{\bf g}'[V] - D\hat{\bf g}\cdot\hat{\bf g}'[V] - D\hat{\bf g}'[V]\cdot\hat{\bf g}$. Here $V$ denotes a fixed deformation field.
\end{proposition}


\begin{lemma}
    Formally the local shape derivative ${\bf u}'[V]$ of \eqref{NSEs} satisfies the following system
    \begin{equation}
        \label{PDE for u', p'}
        \left\{\begin{split}
            -\nu\Delta{\bf u}'[V] + D{\bf u}\cdot{\bf u}'[V] + D{\bf u}'[V]\cdot{\bf u} + \nabla p'[V] &= {\bf 0} &&\mbox{ in } \Omega,\\
            \nabla\cdot{\bf u}'[V] &= 0 &&\mbox{ in } \Omega,\\
            {\bf u}'[V] &= -V\cdot{\bf n}\partial_{\bf n}({\bf u} - {\bf f}_{\rm in}) &&\mbox{ on } \Gamma_{\rm in},\\
            {\bf u}'[V] &= -V\cdot{\bf n}\partial_{\bf n}{\bf u} &&\mbox{ on } \Gamma_{\rm wall},\\
            -\nu\partial_{\bf n}{\bf u}'[V] + p'[V]{\bf n} &= 0 &&\mbox{ on } \Gamma_{\rm out}.
        \end{split}\right.
    \end{equation}
\end{lemma}

\begin{proof}
    *
\end{proof}

\begin{lemma}
    \label{Kasumba_Kunisch2012 Lemma 3.2}
    Formally the local shape derivative ${\bf u}'[V]$ satisfies
    \begin{align}
        {\bf u}'[V]\cdot{\bf n} = 0 \mbox{ on } \Gamma_{\rm wall}.
    \end{align}
\end{lemma}

\begin{proof}
    Using ${\bf u}'[V] = -(V\cdot{\bf n})\partial_{\bf n}{\bf u}$ on $\Gamma_{\rm wall}$, using the tangential divergence formula \eqref{tangential div}, we have on $\Gamma_{\rm wall}$ that
    \begin{align}
        {\bf u}'[V]\cdot{\bf n} = -(V\cdot{\bf n})\partial_{\bf n}{\bf u}\cdot{\bf n} = (V\cdot{\bf n}){\rm div}_\Gamma{\bf u} - (V\cdot{\bf n})\nabla\cdot{\bf u}.
    \end{align}
    Since ${\bf u}|_{\Gamma_{\rm wall}} = {\bf 0}$, ${\rm div}_\Gamma{\bf u} = 0$. Combining this with $\nabla\cdot{\bf u} = 0$ yields the desired formula.
\end{proof}

\begin{lemma}
    The shape derivative $dJ_{12}^{\epsilon\gamma}({\bf u},p,\Omega)[V]$ can be represented as
    \begin{itemize}
        \item[(i)] (Boundary representation)
        \begin{align}
            dJ_{12}^{\epsilon,\gamma}({\bf u},p,\Omega)[V] = -\int_{\Gamma_{\rm wall}} \nu(V\cdot{\bf n})\partial_{\bf n}{\bf u}\cdot\partial_{\bf n}{\bf v}{\rm d}s.
        \end{align}
        \item[(ii)] (Volume representation)
    \end{itemize}
\end{lemma}

\begin{proof}
    (i) Applying the boundary formula for shape derivatives yields
    \begin{align}
        dJ_{12}^{\epsilon,\gamma}({\bf u},p,\Omega)[V] =&\ \int_\Gamma (V\cdot{\bf n})\gamma k_\epsilon\left(p + \frac{1}{2}|{\bf u}|^2\right){\bf u}\cdot{\bf n}{\rm d}s + \int_\Omega \left(\gamma k_\epsilon\left(p + \frac{1}{2}|{\bf u}|^2\right){\bf u}\cdot{\bf n}\right)'[V]{\rm d}x\\
        &+\frac{1 - \gamma}{2}\int_{\Gamma_{\rm out}} (V\cdot{\bf n})\left[\partial_{\bf n}(({\bf u}\cdot{\bf n} - \overline{u})^2) + \kappa({\bf u}\cdot{\bf n} - \overline{u})^2\right] + (({\bf u}\cdot{\bf n} - \overline{u})^2)'[V]{\rm d}s\\
        =&\ \int_\Omega \gamma k_\epsilon\left(\left(p + \frac{1}{2}|{\bf u}|^2\right){\bf n} + ({\bf u}\cdot{\bf n}){\bf u}\right){\bf u}'[V] + \gamma k_\epsilon{\bf u}\cdot{\bf n}p'[V]{\rm d}x + (1 - \gamma)\int_{\Gamma_{\rm out}} ({\bf u}\cdot{\bf n} - \overline{u}){\bf n}\cdot{\bf u}'[V]{\rm d}s.\label{3.7.6}
    \end{align}
    %    =&\ \int_\Omega \left((\nabla{\bf v})^\top\cdot{\bf u} + \nabla{\bf v}\cdot{\bf u} + \nu\Delta{\bf v} - \nabla q\right){\bf u}'[V] + \nabla\cdot{\bf v}p'[V]{\rm d}x\\    
    %    &+ \int_{\Gamma_{\rm out}} (-({\bf v}\cdot{\bf u}){\bf n} - ({\bf u}\cdot{\bf n}){\bf v} - \nu\partial_{\bf n}{\bf v} + q{\bf n})\cdot{\bf u}'[V]{\rm d}s.
    Testing \eqref{PDE for u', p'} with the adjoint variable $({\bf v},q)$ yields
    \begin{align}
        \label{3.7.3}
        \int_\Omega {\bf v}\cdot(-\nu\Delta{\bf u}'[V] + D{\bf u}\cdot{\bf u}'[V] + D{\bf u}'[V]\cdot{\bf u} + \nabla p'[V]) - q\nabla\cdot{\bf u}'[V] {\rm d}x = 0.
    \end{align}
    Integrating by parts \eqref{3.7.3} gives
    \begin{align}
        &\int_\Omega {\bf u}'[V]\cdot\left[-(\nabla{\bf v})^\top\cdot{\bf u} - \nabla{\bf v}\cdot{\bf u} -\nu\Delta{\bf v} + \nabla q\right] - p'[V]\nabla\cdot{\bf v}{\rm d}x\\
        &+ \int_\Gamma {\bf u}'[V]\cdot(({\bf v}\cdot{\bf u}){\bf n} + ({\bf u}\cdot{\bf n}){\bf v} + \nu\partial_{\bf n}{\bf v} - q{\bf n}) - \nu{\bf n}\cdot\nabla{\bf u}'[V]\cdot{\bf v} + p'[V]{\bf v}\cdot{\bf n}{\rm d}s = 0.
    \end{align}
    Since $({\bf v},q)$ satisfies \eqref{adjoint NSEs}, the last equality becomes
    \begin{align}
        &\int_\Omega -{\bf u}'[V]\cdot\gamma k_\epsilon\left(\left(p + \frac{1}{2}|{\bf u}|^2\right){\bf n} + ({\bf u}\cdot{\bf n}){\bf u}\right) - \gamma k_\epsilon p'[V]{\bf u}\cdot{\bf n}{\rm d}x + \int_{\Gamma_{\rm wall}} {\bf u}'[V]\cdot(\nu\partial_{\bf n}{\bf v} - q{\bf n}){\rm d}s\\
        &+ \int_{\Gamma_{\rm out}} -{\bf u}'[V]\cdot(1 - \gamma)({\bf u}\cdot{\bf n} - \overline{u}){\bf n}{\rm d}s = 0.
    \end{align}
    Then \eqref{3.7.6} becomes
    \begin{align}
        \label{3.7.14}
        dJ_{12}^{\epsilon,\gamma}({\bf u},p,\Omega)[V] = \int_{\Gamma_{\rm wall}} {\bf u}'[V]\cdot(\nu\partial_{\bf n}{\bf v} - q{\bf n}){\rm d}s.
    \end{align}
    The term ${\bf u}'[V]\cdot q{\bf n}$ vanishes in $\Gamma_{\rm wall}$ due to Lemma \ref{Kasumba_Kunisch2012 Lemma 3.2}. Using \eqref{PDE for u', p'}, we obtain from \eqref{3.7.14}
    \begin{align}
        dJ_{12}^{\epsilon,\gamma}({\bf u},p,\Omega)[V] = -\int_{\Gamma_{\rm wall}} \nu(V\cdot{\bf n})\partial_{\bf n}{\bf u}\cdot\partial_{\bf n}{\bf v}{\rm d}s.
    \end{align}
    Since the mapping $V\mapsto dJ_{12}^{\epsilon\gamma}({\bf u},p,\Omega)[V]V$ is linear and continuous, the shape gradient of $J_{12}^{\epsilon,\gamma}({\bf u},p,\Omega)$ is given by $\nabla J_{12}^{\epsilon,\gamma}({\bf u},p,\Omega){\bf n} = -\nu(\partial_{\bf n}{\bf u}\cdot\partial_{\bf n}{\bf v}){\bf n}|_{\Gamma_{\rm wall}}$.
    
    (ii) Applying the volume formula for shape derivatives yields
    \begin{align}
        dJ_{12}^{\epsilon\gamma}({\bf u},p,\Omega)[V] =&\ \int_\Omega \gamma k_\epsilon\left(p + \frac{1}{2}|{\bf u}|^2\right){\bf u}\cdot{\bf n}\nabla\cdot V + \gamma k_\epsilon d\left(\left(p + \frac{1}{2}|{\bf u}|^2\right){\bf u}\cdot{\bf n}\right)[V]{\rm d}x\\
        &+ \frac{1 - \gamma}{2}\int_{\Gamma_{\rm out}} (V\cdot{\bf n})\left(\partial_{\bf n}(({\bf u}\cdot{\bf n} - \overline{u})^2) + \kappa({\bf u}\cdot{\bf n} - \overline{u})^2\right) + (({\bf u}\cdot{\bf n} - \overline{u})^2)'[V]{\rm d}s\\
        =&\ \int_\Omega \gamma k_\epsilon\left(p + \frac{1}{2}|{\bf u}|^2\right){\bf u}\cdot{\bf n}\nabla\cdot V + \gamma k_\epsilon d\left(\left(p + \frac{1}{2}|{\bf u}|^2\right){\bf u}\cdot{\bf n}\right)[V]{\rm d}x + (1 - \gamma)\int_{\Gamma_{\rm out}} ({\bf u}\cdot{\bf n} - \overline{u}){\bf n}\cdot{\bf u}'[V]{\rm d}s.
    \end{align}
    We compute the material derivative $d\left(\left(p + \frac{1}{2}|{\bf u}|^2\right){\bf u}\cdot{\bf n}\right)[V]$ in $\Gamma_{\rm in}^\epsilon\cup\Gamma_{\rm wall}^\epsilon$:
    \begin{align}
        d\left(\left(p + \frac{1}{2}|{\bf u}|^2\right){\bf u}\cdot{\bf n}\right)[V] =&\ + \left(\left(p + \frac{1}{2}|{\bf u}|^2\right){\bf u}\cdot{\bf n}\right)'[V] + V\cdot\nabla\left(\left(p + \frac{1}{2}|{\bf u}|^2\right){\bf u}\cdot{\bf n}\right)\\
        =&\ \left(\left(p + \frac{1}{2}|{\bf u}|^2\right){\bf n} + ({\bf u}\cdot{\bf n}){\bf u}\right){\bf u}'[V] + {\bf u}\cdot{\bf n}p'[V] + V\cdot\left[\left(\nabla p + \nabla{\bf u}\cdot{\bf u}\right){\bf u}\cdot{\bf n} + \left(p + \frac{1}{2}|{\bf u}|^2\right)\nabla{\bf u}\cdot{\bf n}\right].
    \end{align}
    Then
    \begin{align}
        dJ_{12}^{\epsilon,\gamma}({\bf u},p,\Omega)[V] =&\ \int_\Omega \gamma k_\epsilon\left(p + \frac{1}{2}|{\bf u}|^2\right){\bf u}\cdot{\bf n}\nabla\cdot V\\
        &+ \gamma k_\epsilon \left\{V\cdot\left[\left(\nabla p + \nabla{\bf u}\cdot{\bf u}\right){\bf u}\cdot{\bf n} + \left(p + \frac{1}{2}|{\bf u}|^2\right)\nabla{\bf u}\cdot{\bf n}\right] + \left(\left(p + \frac{1}{2}|{\bf u}|^2\right){\bf n} + ({\bf u}\cdot{\bf n}){\bf u}\right){\bf u}'[V] + {\bf u}\cdot{\bf n}p'[V]\right\}{\rm d}x\\
        &+ (1 - \gamma)\int_{\Gamma_{\rm out}} ({\bf u}\cdot{\bf n} - \overline{u}){\bf n}\cdot{\bf u}'[V]{\rm d}s.
    \end{align}
    We recall from (i) that
    \begin{align*}
        \int_\Omega {\bf u}'[V]\cdot\gamma k_\epsilon\left(\left(p + \frac{1}{2}|{\bf u}|^2\right){\bf n} + ({\bf u}\cdot{\bf n}){\bf u}\right) + \gamma k_\epsilon p'[V]{\bf u}\cdot{\bf n}{\rm d}x = \int_{\Gamma_{\rm wall}} {\bf u}'[V]\cdot(\nu\partial_{\bf n}{\bf v} - q{\bf n}){\rm d}s + \int_{\Gamma_{\rm out}} -{\bf u}'[V]\cdot(1 - \gamma)({\bf u}\cdot{\bf n} - \overline{u}){\bf n}{\rm d}s.
    \end{align*}
    Then
    \begin{align*}
        dJ_{12}^{\epsilon,\gamma}(\Omega)[V] =&\ \int_\Omega \gamma k_\epsilon\left(p + \frac{1}{2}|{\bf u}|^2\right){\bf u}\cdot{\bf n}\nabla\cdot V + \gamma k_\epsilon V\cdot\left((\nabla p + \nabla{\bf u}\cdot{\bf u}){\bf u}\cdot{\bf n} + \left(p + \frac{1}{2}|{\bf u}|^2\right)\nabla{\bf u}\cdot{\bf n}\right){\rm d}x\\
        &+ \int_{\Gamma_{\rm wall}} {\bf u}'[V]\cdot(\nu\partial_{\bf n}{\bf v} - q{\bf n}){\rm d}s\\
        =&\ \int_\Omega \gamma k_\epsilon\left(p + \frac{1}{2}|{\bf u}|^2\right){\bf u}\cdot{\bf n}\nabla\cdot V + \gamma k_\epsilon V\cdot\left((\nabla p + \nabla{\bf u}\cdot{\bf u}){\bf u}\cdot{\bf n} + \left(p + \frac{1}{2}|{\bf u}|^2\right)\nabla{\bf u}\cdot{\bf n}\right){\rm d}x\\
        &- \int_{\Gamma_{\rm wall}} \nu(V\cdot{\bf n})\partial_{\bf n}{\bf u}\cdot\partial_{\bf n}{\bf v}{\rm d}s.
    \end{align*}
\end{proof}

\begin{remark}
    \textbf{Open}: Why there are still some terms except $\partial_{\bf n}{\bf u}\cdot\partial_{\bf n}{\bf v}$ left?
\end{remark}

\section{Formal Lagrangian}
To set up the optimality system for the shape optimization problem, we consider the following Lagrange function:
\begin{align}
    \label{full Lagrangian}
    \mathcal{L}({\bf u},p,{\bf v},q,\Omega,{\bf v}_{\rm in},{\bf v}_{\rm wall},{\bf v}_{\rm out}) :=& J_{12}^{\epsilon,\gamma}({\bf u},p,\Omega) - \int_\Omega \left({\bf v}\cdot(-\nu\Delta{\bf u} + ({\bf u}\cdot\nabla){\bf u} + \nabla p - {\bf f}) - q\nabla\cdot{\bf u}\right){\rm d}x \nonumber\\
    &- \int_{\Gamma_{\rm in}} {\bf v}_{\rm in}\cdot({\bf u} - {\bf f}_{\rm in}){\rm d}s - \int_{\Gamma_{\rm wall}} {\bf v}_{\rm wall}\cdot{\bf u}{\rm d}s - \int_{\Gamma_{\rm out}} {\bf v}_{\rm out}\cdot(p{\bf n} - \nu\partial_{\bf n}{\bf u}){\rm d}s,
\end{align}
where ${\bf v},q,{\bf v}_{\rm in},{\bf v}_{\rm wall},{\bf v}_{\rm out}$ are Lagrange multipliers.

\begin{proof}[Demonstration]
    Choose the Lagrange multiplier $({\bf v},q)$ such that the variation with respect to the state variables vanishes identically, $\partial_{\bf u}\mathcal{L}\cdot\delta{\bf u} + \partial_p\mathcal{L}\delta p = 0$, which reads as
    \begin{align}
        \partial_{\bf u}J_{12}^{\epsilon,\gamma}\delta{\bf u} + \partial_pJ_{12}^{\epsilon,\gamma}\delta p &- \int_\Omega {\bf v}\cdot((\delta{\bf u}\cdot\nabla){\bf u} + ({\bf u}\cdot\nabla)\delta{\bf u} - \nu\Delta\delta{\bf u}){\rm d}x + \int_\Omega q\nabla\cdot\delta{\bf u}dx - \int_\Omega {\bf v}\cdot\nabla\delta pdx \nonumber\\
        &- \int_{\Gamma_{\rm in}} {\bf v}_{\rm in}\cdot\delta{\bf u}ds - \int_{\Gamma_{\rm wall}} {\bf v}_{\rm wall}\cdot\delta{\bf u}ds - \int_{\Gamma_{\rm out}} {\bf v}_{\rm out}\cdot(\delta p{\bf n} - \nu\partial_{\bf n}\delta{\bf u})ds = 0.\label{1.4.2}
    \end{align}
    We integrate by parts term by term: the 2 terms produced by the nonlinear term $({\bf u}\cdot\nabla){\bf u}$:
    \begin{align}
        \int_\Omega {\bf v}\cdot((\delta{\bf u}\cdot\nabla){\bf u}){\rm d}x &= \int_\Omega \sum_{i=1}^d\sum_{j=1}^d v_i\partial_{x_j}u_i\delta u_j{\rm d}x = \int_\Gamma \sum_{i=1}^d\sum_{j=1}^d v_iu_i\delta u_jn_j{\rm d}s - \int_\Omega \sum_{i=1}^d\sum_{j=1}^d (u_i\partial_{x_j}v_i\delta u_j + u_iv_i\partial_{x_j}\delta u_j){\rm d}x\\
        &= \int_\Gamma ({\bf v}\cdot{\bf u})(\delta{\bf u}\cdot{\bf n}){\rm d}s - \int_\Omega [{\bf u}\cdot\nabla{\bf v}^\top\cdot\delta{\bf u} + ({\bf u}\cdot{\bf v})\nabla\cdot\delta{\bf u}]{\rm d}x = \int_\Gamma ({\bf v}\cdot{\bf u})(\delta{\bf u}\cdot{\bf n}){\rm d}s - \int_\Omega {\bf u}\cdot\nabla{\bf v}^\top\cdot\delta{\bf u}{\rm d}x,\\
        \int_\Omega {\bf v}\cdot(({\bf u}\cdot\nabla)\delta{\bf u}){\rm d}x &= \int_\Omega \sum_{i=1}^d\sum_{j=1}^d v_iu_j\partial_{x_j}\delta u_i{\rm d}x = \int_\Gamma \sum_{i=1}^d\sum_{j=1}^d v_i\delta u_iu_jn_j{\rm d}s - \int_\Omega \sum_{i=1}^d\sum_{j=1}^d (\delta u_i\partial_{x_j}v_iu_j + \delta u_iv_i\partial_{x_j}u_j){\rm d}x\\
        &= \int_\Gamma ({\bf u}\cdot{\bf n})({\bf v}\cdot\delta{\bf u}){\rm d}s - \int_\Omega \left[({\bf u}\cdot\nabla){\bf v}\cdot\delta{\bf u} + \nabla\cdot{\bf u}({\bf v}\cdot\delta{\bf u})\right]{\rm d}x = \int_\Gamma ({\bf u}\cdot{\bf n})({\bf v}\cdot\delta{\bf u}){\rm d}s - \int_\Omega ({\bf u}\cdot\nabla){\bf v}\cdot\delta{\bf u}{\rm d}x,
    \end{align}
    Laplacian term:
    \begin{align}
        -\nu\int_\Omega {\bf v}\cdot\Delta\delta{\bf u}{\rm d}x &= -\nu\int_\Omega \sum_{i=1}^d\sum_{j=1}^d v_i\partial_{x_j}^2\delta u_i{\rm d}x = -\nu\int_\Gamma \sum_{i=1}^d\sum_{j=1}^d v_i\partial_{x_j}\delta u_in_j{\rm d}s + \nu\int_\Omega \sum_{i=1}^d\sum_{j=1}^d \partial_{x_j}v_i\partial_{x_j}\delta u_i{\rm d}x\\
        &= -\nu\int_\Gamma {\bf n}\cdot\nabla\delta{\bf u}\cdot{\bf v}{\rm d}s + \nu\int_\Gamma \sum_{i=1}^d\sum_{j=1}^d \partial_{x_j}v_i\delta u_in_j{\rm d}s - \nu \int_\Omega \sum_{i=1}^d\sum_{j=1}^d \partial_{x_j}^2v_i\delta u_i{\rm d}x\\
        &= -\nu\int_\Gamma {\bf n}\cdot\nabla\delta{\bf u}\cdot{\bf v}{\rm d}s + \nu\int_\Gamma {\bf n}\cdot\nabla{\bf v}\cdot\delta{\bf u}{\rm d}s - \nu\int_\Omega \Delta{\bf v}\cdot\delta{\bf u}{\rm d}x,
    \end{align}
    divergence term:
    \begin{align}
        -\int_\Omega q\nabla\cdot\delta{\bf u}{\rm d}x = \int_\Omega \delta{\bf u}\cdot\nabla q{\rm d}x - \int_\Gamma q\delta{\bf u}\cdot{\bf n}{\rm d}s,
    \end{align}
    and the term produced by $\nabla p$:
    \begin{align}
        \int_\Omega {\bf v}\cdot\nabla\delta p{\rm d}x = -\int_\Omega \delta p\nabla\cdot{\bf v}{\rm d}x + \int_\Gamma \delta p{\bf v}\cdot{\bf n}{\rm d}s.
    \end{align}
    Decomposing $J_{12}^{\epsilon,\gamma}$ into contributions from the boundary $\Gamma = \partial\Omega$ and from the interior of $\Omega$,
    \begin{align}
        J_{12}^{\epsilon,\gamma} = \int_\Gamma J_\Gamma{\rm d}s + \int_\Omega J_\Omega{\rm d}x,
    \end{align}
    thus
    \begin{align}
        \partial_{\bf u}J_{12}^{\epsilon,\gamma}\delta{\bf u} &= \int_\Gamma \partial_{\bf u}J_\Gamma\delta{\bf u}{\rm d}s + \int_\Omega \partial_{\bf u}J_\Omega\delta{\bf u}{\rm d}x,\\
        \partial_pJ_{12}^{\epsilon,\gamma}\delta p &= \int_\Gamma \partial_pJ_\Gamma\delta p{\rm d}s + \int_\Omega \partial_pJ_\Omega\delta p{\rm d}x,
    \end{align}
    we can reformulate \eqref{1.4.2} as
    \begin{align}
        &\int_\Gamma ({\bf v}\cdot{\bf n} - \partial_pJ_\Gamma)\delta p{\rm d}s + \int_\Omega (-\nabla\cdot{\bf v} - \partial_pJ_\Omega)\delta p{\rm d}x + \int_\Gamma \left[({\bf v}\cdot{\bf u}){\bf n} + ({\bf u}\cdot{\bf n}){\bf v} + \nu{\bf n}\cdot\nabla{\bf v} - q{\bf n} - \partial_{\bf u}J_\Gamma\right]\cdot\delta{\bf u}{\rm d}s \nonumber\\
        &- \nu\int_\Gamma {\bf n}\cdot\nabla\delta{\bf u}\cdot{\bf v}{\rm d}s + \int_\Omega [-\nabla{\bf v}\cdot{\bf u} - ({\bf u}\cdot\nabla){\bf v} - \nu\Delta{\bf v} + \nabla q - \partial_{\bf u}J_\Omega]\cdot\delta{\bf u} \nonumber\\
        &+ \int_{\Gamma_{\rm in}} {\bf v}_{\rm in}\cdot\delta{\bf u}ds + \int_{\Gamma_{\rm wall}} {\bf v}_{\rm wall}\cdot\delta{\bf u}ds + \int_{\Gamma_{\rm out}} {\bf v}_{\rm out}\cdot(\delta p{\bf n} - \nu\partial_{\bf n}\delta{\bf u})ds = 0.\label{1.4.3}
    \end{align}
    Since this holds for any $\delta{\bf u}$ and $\delta p$ satisfying the primal NSEs, the integrals vanish individually. The vanishing of the integrals over the domain yields the adjoint NSEs
    \begin{equation}
        \label{1.4.4}
        \left\{\begin{split}
            -2\varepsilon({\bf v}){\bf u} &= -\nabla q + \nu\Delta{\bf v} + \partial_{\bf u}J_\Omega &&\mbox{ in } \Omega,\\
            \nabla\cdot{\bf v} &= -\partial_pJ_\Omega &&\mbox{ in } \Omega.
        \end{split}\right.    
    \end{equation}
    where we have expressed $-\nabla{\bf v}\cdot{\bf u} - ({\bf u}\cdot\nabla){\bf v}$ as $-2\varepsilon({\bf v}){\bf u}$. Then \eqref{1.4.4} becomes
    \begin{equation*}
        \left\{\begin{split}
            -2\varepsilon({\bf v}){\bf u} - \nu\Delta{\bf v} + \nabla q &= \gamma k_\epsilon\left(\left(p + \frac{1}{2}|{\bf u}|^2\right){\bf n} + ({\bf u}\cdot{\bf n}){\bf u}\right) &&\mbox{ in } \Omega,\\
            \nabla\cdot{\bf v} &= -\gamma k_\epsilon{\bf u}\cdot{\bf n} &&\mbox{ in } \Omega.
        \end{split}\right.    
    \end{equation*}
    As boundary conditions for adjoint velocity and pressure, deduce from \eqref{1.4.3} that
    \begin{align}
        &\int_\Gamma \left[({\bf v}\cdot{\bf u}){\bf n} + ({\bf u}\cdot{\bf n}){\bf v} + \nu{\bf n}\cdot\nabla{\bf v} - q{\bf n} - \partial_{\bf u}J_\Gamma\right]\cdot\delta{\bf u}ds - \nu\int_\Gamma {\bf n}\cdot\nabla\delta{\bf u}\cdot{\bf v}ds \nonumber\\
        &\hspace{1cm}+ \int_{\Gamma_{\rm in}} {\bf v}_{\rm in}\cdot\delta{\bf u}ds + \int_{\Gamma_{\rm wall}} {\bf v}_{\rm wall}\cdot\delta{\bf u}ds - \int_{\Gamma_{\rm out}} \nu{\bf v}_{\rm out}\cdot\partial_{\bf n}\delta{\bf u}ds = 0,\label{BC velocity}\\
        &\int_\Gamma ({\bf v}\cdot{\bf n} - \partial_pJ_\Gamma)\delta pds + \int_{\Gamma_{\rm out}} {\bf v}_{\rm out}\cdot\delta p{\bf n}ds = 0.\label{BC pressure}
    \end{align}
    Use ${\bf v} = \delta{\bf u} = {\bf 0}$ on $\Gamma_{\rm in}\cup\Gamma_{\rm wall}$, \eqref{BC pressure} reduces to
    \begin{align}
        \int_{\Gamma_{\rm out}} ({\bf v}\cdot{\bf n} + {\bf v}_{\rm out}\cdot{\bf n})\delta p{\rm d}s = 0,
    \end{align}
    and thus we can set ${\bf v}_{\rm out} = -{\bf v}$ on $\Gamma_{\rm out}$.
    
    Similarly, \eqref{BC velocity} reduces to
    \begin{align}
        \int_{\Gamma_{\rm out}} \left[({\bf v}\cdot{\bf u}){\bf n} + ({\bf u}\cdot{\bf n}){\bf v} + \nu{\bf n}\cdot\nabla{\bf v} - q{\bf n} + (1 - \gamma)({\bf u}\cdot{\bf n} - \bar{u}){\bf n}\right]\cdot\delta{\bf u}ds - \nu\int_{\Gamma_{\rm out}} {\bf n}\cdot\nabla\delta{\bf u}\cdot{\bf v}ds - \int_{\Gamma_{\rm out}} \nu{\bf v}_{\rm out}\cdot\partial_{\bf n}\delta{\bf u}ds = 0.
    \end{align}
    Note that the sum of the last 2 terms vanishes, the last equation reduces to
    \begin{align}
        \int_{\Gamma_{\rm out}} \left[({\bf v}\cdot{\bf u}){\bf n} + ({\bf u}\cdot{\bf n}){\bf v} + \nu{\bf n}\cdot\nabla{\bf v} - q{\bf n} + (1 - \gamma)({\bf u}\cdot{\bf n} - \bar{u}){\bf n}\right]\cdot\delta{\bf u}{\rm d}s = 0.
    \end{align}
    Thus,
    \begin{align}
        ({\bf v}\cdot{\bf u}){\bf n} + ({\bf u}\cdot{\bf n}){\bf v} + \nu{\bf n}\cdot\nabla{\bf v} - q{\bf n} = -(1 - \gamma)({\bf u}\cdot{\bf n} - \bar{u}){\bf n} \mbox{ on } \Gamma_{\rm out}.
    \end{align}
    We obtain the desired adjoint equation.
\end{proof}

\begin{lemma}[Navier-Stokes shape derivative]
    The formal shape derivative for \eqref{NSEs} is given by
    \begin{align}
        dJ_{12}^{\epsilon,\gamma}(\Omega)[V] = \int_{\Gamma_{\rm wall}} \nu(V\cdot{\bf n})\partial_{\bf n}{\bf u}\cdot\partial_{\bf n}{\bf v}{\rm d}s.
    \end{align}
    where ${\bf v}$ is the solution of the adjoint equation.
\end{lemma}

\begin{proof}
    A formal shape differentiation of the Lagrangian \eqref{full Lagrangian} yields
    \begin{align}
        d\mathcal{L}({\bf u},p,{\bf v},q,\Omega,{\bf v}_{\rm in},{\bf v}_{\rm wall},{\bf v}_{\rm out})[V] =& \int_\Gamma \langle V,{\bf n}\rangle\left[\gamma k_\epsilon\left(p + \frac{1}{2}|{\bf u}|^2\right){\bf u}\cdot{\bf n} + {\bf v}\cdot({\color{red} -\nu\Delta{\bf u} + ({\bf u}\cdot\nabla){\bf u} + \nabla p - {\bf f}}) - q\nabla\cdot{\bf u}\right]{\rm d}s\\
        &+ \int_\Omega \left[\gamma k_\epsilon\left(p + \frac{1}{2}|{\bf u}|^2\right){\bf u}\cdot{\bf n} + {\bf v}\cdot({\color{red} -\nu\Delta{\bf u} + ({\bf u}\cdot\nabla){\bf u} + \nabla p - {\bf f}}) - q\nabla\cdot{\bf u}\right]'[V]{\rm d}x\\
        &+ \int_{\Gamma_{\rm out}} \left(\frac{1 - \gamma}{2}({\bf u}\cdot{\bf n} - \overline{u})^2 - {\bf v}\cdot(p{\bf n} - \nu\partial_{\bf n}{\bf u})\right)'[V]{\rm d}s.
    \end{align}
    Since $\Gamma_{\rm in}$ and $\Gamma_{\rm out}$ are fixed, ${\bf u}_{\Gamma_{\rm wall}} = {\bf 0}$, $V|_{\Gamma_{\rm in}\cup\Gamma_{\rm out}} = {\bf 0}$ and then
    \begin{align}
        d\mathcal{L}({\bf u},p,{\bf v},q,\Omega,{\bf v}_{\rm in},{\bf v}_{\rm wall},{\bf v}_{\rm out})[V] =& \int_\Omega \left[\gamma k_\epsilon\left(p + \frac{1}{2}|{\bf u}|^2\right){\bf u}\cdot{\bf n} - {\bf v}\cdot({\color{red} -\nu\Delta{\bf u} + ({\bf u}\cdot\nabla){\bf u} + \nabla p - {\bf f}}) + q\nabla\cdot{\bf u}\right]'[V]{\rm d}x\\
        &+ \int_{\Gamma_{\rm out}} \left(\frac{1 - \gamma}{2}({\bf u}\cdot{\bf n} - \overline{u})^2 - {\bf v}\cdot(p{\bf n} - \nu\partial_{\bf n}{\bf u})\right)'[V]{\rm d}s = \int_\Omega I_1{\rm d}x + \int_{\Gamma_{\rm out}} I_2{\rm d}s. 
    \end{align}
    We have
    \begin{align}
        I_1 =&\ \left[\gamma k_\epsilon\left(p + \frac{1}{2}|{\bf u}|^2\right){\bf u}\cdot{\bf n} + {\bf v}\cdot({\color{red} -\nu\Delta{\bf u} + ({\bf u}\cdot\nabla){\bf u} + \nabla p - {\bf f}}) - q\nabla\cdot{\bf u}\right]'[V]\\
        =&\ \gamma k_\epsilon\left(\left(p + \frac{1}{2}|{\bf u}|^2\right){\bf n} + ({\bf u}\cdot{\bf n}){\bf u}\right)\cdot{\bf u}'[V] + \gamma k_\epsilon p'[V]{\bf u}\cdot{\bf n} + \gamma k_\epsilon\left(p + \frac{1}{2}|{\bf u}|^2\right){\bf u}\cdot{\bf n}'[V] + {\bf v}'[V]\cdot({\color{red} -\nu\Delta{\bf u} + ({\bf u}\cdot\nabla){\bf u} + \nabla p - {\bf f}}) \\
        &+ {\bf v}\cdot(-\nu\Delta{\bf u}'[V] + ({\bf u}'[V]\cdot\nabla){\bf u} + ({\bf u}\cdot\nabla){\bf u}'[V] + \nabla p'[V]) - q'[V]{\color{red} \nabla\cdot{\bf u}} - q\nabla\cdot{\bf u}'[V]\\
        =&\ \gamma k_\epsilon\left(\left(p + \frac{1}{2}|{\bf u}|^2\right){\bf n} + ({\bf u}\cdot{\bf n}){\bf u}\right)\cdot{\bf u}'[V] + \gamma k_\epsilon p'[V]{\bf u}\cdot{\bf n} + \gamma k_\epsilon\left(p + \frac{1}{2}|{\bf u}|^2\right){\bf u}\cdot{\bf n}'[V] \\
        &+ {\bf v}\cdot(-\nu\Delta{\bf u}'[V] + ({\bf u}'[V]\cdot\nabla){\bf u} + ({\bf u}\cdot\nabla){\bf u}'[V] + \nabla p'[V]) - q\nabla\cdot{\bf u}'[V].
    \end{align}
    Integrating by parts yields
    \begin{align}
        &\int_\Omega {\bf v}\cdot(-\nu\Delta{\bf u}'[V] + ({\bf u}'[V]\cdot\nabla){\bf u} + ({\bf u}\cdot\nabla){\bf u}'[V] + \nabla p'[V]) - q\nabla\cdot{\bf u}'[V]{\rm d}x\\
        =&\ \int_\Omega {\bf u}'[V]\cdot({\color{blue} -(\nabla{\bf v})^\top\cdot{\bf u} - \nabla{\bf v}\cdot{\bf u} -\nu\Delta{\bf v} + \nabla q) - p'[V]\nabla\cdot{\bf v}} {\rm d}x\\
        &+ \int_\Gamma {\bf u}'[V]\cdot({\color{blue} ({\bf v}\cdot{\bf u}){\bf n} + ({\bf u}\cdot{\bf n}){\bf v} + \nu\partial_{\bf n}{\bf v} - q{\bf n}}) - \nu{\bf n}\cdot\nabla{\bf u}'[V]\cdot{\bf v} + p'[V]{\bf v}\cdot{\bf n}{\rm d}s\\
        =&\ \int_\Omega -{\bf u}'[V]\cdot\gamma k_\epsilon\left(\left(p + \frac{1}{2}|{\bf u}|^2\right){\bf n} + ({\bf u}\cdot{\bf n}){\bf u}\right) - p'[V]\gamma k_\epsilon{\bf u}\cdot{\bf n}{\rm d}x\\
        &+ \int_\Gamma {\bf u}'[V]\cdot({\color{blue} ({\bf v}\cdot{\bf u}){\bf n} + ({\bf u}\cdot{\bf n}){\bf v} + \nu\partial_{\bf n}{\bf v} - q{\bf n}}) - \nu{\bf n}\cdot\nabla{\bf u}'[V]\cdot{\bf v} + p'[V]{\bf v}\cdot{\bf n}{\rm d}s.
    \end{align}
    Since ${\bf u}'[V] = 0$ in $\Gamma_{\rm in}$ and ${\bf v}|_{\Gamma_{\rm in}\cup\Gamma_{\rm wall}} = {\bf 0}$, we have
    \begin{align}
        &\int_\Gamma {\bf u}'[V]\cdot({\color{blue} ({\bf v}\cdot{\bf u}){\bf n} + ({\bf u}\cdot{\bf n}){\bf v} + \nu\partial_{\bf n}{\bf v} - q{\bf n}}){\rm d}s\\
        =&\ \int_{\Gamma_{\rm wall}} {\bf u}'[V]\cdot(\nu\partial_{\bf n}{\bf v} - q{\bf n}){\rm d}s + \int_{\Gamma_{\rm out}} {\bf u}'[V]\cdot({\color{blue} ({\bf v}\cdot{\bf u}){\bf n} + ({\bf u}\cdot{\bf n}){\bf v} + \nu\partial_{\bf n}{\bf v} - q{\bf n}}){\rm d}s\\
        =&\ -\int_{\Gamma_{\rm wall}} V\cdot{\bf n}(\nu\partial_{\bf n}{\bf u}\cdot\partial_{\bf n}{\bf v} - q{\bf n}){\rm d}s + \int_{\Gamma_{\rm out}} {\bf u}'[V]\cdot(\gamma - 1)({\bf u}\cdot{\bf n} - \overline{u}){\bf n}{\rm d}s.
    \end{align}
    Hence
    \begin{align}
        &\int_\Omega {\bf v}\cdot(-\nu\Delta{\bf u}'[V] + ({\bf u}'[V]\cdot\nabla){\bf u} + ({\bf u}\cdot\nabla){\bf u}'[V] + \nabla p'[V]) - q\nabla\cdot{\bf u}'[V]{\rm d}x\\
        =&\ \int_\Omega -{\bf u}'[V]\cdot\gamma k_\epsilon\left(\left(p + \frac{1}{2}|{\bf u}|^2\right){\bf n} + ({\bf u}\cdot{\bf n}){\bf u}\right) - p'[V]\gamma k_\epsilon{\bf u}\cdot{\bf n}{\rm d}x\\
        &+ \int_{\Gamma_{\rm wall}} V\cdot{\bf n}(\nu\partial_{\bf n}{\bf u}\cdot\partial_{\bf n}{\bf v} - q{\bf n}){\rm d}s + \int_{\Gamma_{\rm out}} {\bf u}'[V]\cdot(\gamma - 1)({\bf u}\cdot{\bf n} - \overline{u}){\bf n}{\rm d}s + \int_{\Gamma_{\rm out}} -\nu{\bf n}\cdot\nabla{\bf u}'[V]\cdot{\bf v} + p'[V]{\bf v}\cdot{\bf n}{\rm d}s.
    \end{align}
    Moreover,
    \begin{align}
        I_2 =&\ \left(\frac{1 - \gamma}{2}({\bf u}\cdot{\bf n} - \overline{u})^2 - {\bf v}\cdot(p{\bf n} - \nu\partial_{\bf n}{\bf u})\right)'[V]\\
        =&\ (1 - \gamma)({\bf u}\cdot{\bf n} - \overline{u}){\bf n}\cdot{\bf u}'[V] + (1 - \gamma)({\bf u}\cdot{\bf n} - \overline{u}){\bf u}\cdot{\bf n}'[V] \\
        &- {\bf v}'[V]\cdot(p{\bf n} - \nu\partial_{\bf n}{\bf u}) - {\bf v}\cdot(p'[V]{\bf n} + p{\bf n}'[V] - \nu\nabla{\bf u}'[V]\cdot{\bf n} - \nu\nabla{\bf u}\cdot{\bf n}'[V])\\
        =&\ (1 - \gamma)({\bf u}\cdot{\bf n} - \overline{u}){\bf n}\cdot{\bf u}'[V] - {\bf v}\cdot(p'[V]{\bf n} - \nu\nabla{\bf u}'[V]\cdot{\bf n}),
    \end{align}
    since ${\bf n}'[V] = {\bf 0}$ on $\Gamma_{\rm out}$.
    
    Then the shape derivative $d\mathcal{L}$ becomes
    \begin{align}
        &d\mathcal{L}({\bf u},p,{\bf v},q,\Omega,{\bf v}_{\rm in},{\bf v}_{\rm wall},{\bf v}_{\rm out})[V]\\
        =&\ \int_\Omega \gamma k_\epsilon\left(\left(p + \frac{1}{2}|{\bf u}|^2\right){\bf n} + ({\bf u}\cdot{\bf n}){\bf u}\right)\cdot{\bf u}'[V] + \gamma k_\epsilon p'[V]{\bf u}\cdot{\bf n} + \gamma k_\epsilon\left(p + \frac{1}{2}|{\bf u}|^2\right){\bf u}\cdot{\bf n}'[V]\\
        &\hspace{1cm}-{\bf u}'[V]\cdot\gamma k_\epsilon\left(\left(p + \frac{1}{2}|{\bf u}|^2\right){\bf n} + ({\bf u}\cdot{\bf n}){\bf u}\right) - p'[V]\gamma k_\epsilon{\bf u}\cdot{\bf n}{\rm d}x\\
        &+ \int_{\Gamma_{\rm wall}} V\cdot{\bf n}(\nu\partial_{\bf n}{\bf u}\cdot\partial_{\bf n}{\bf v} - q{\bf n}){\rm d}s + \int_{\Gamma_{\rm out}} {\bf u}'[V]\cdot(\gamma - 1)({\bf u}\cdot{\bf n} - \overline{u}){\bf n}{\rm d}s + \int_{\Gamma_{\rm out}} -\nu{\bf n}\cdot\nabla{\bf u}'[V]\cdot{\bf v} + p'[V]{\bf v}\cdot{\bf n}{\rm d}s\\
        &+ \int_{\Gamma_{\rm out}} (1 - \gamma)({\bf u}\cdot{\bf n} - \overline{u}){\bf n}\cdot{\bf u}'[V] + (1 - \gamma)({\bf u}\cdot{\bf n} - \overline{u}){\bf u}\cdot{\bf n}'[V]\\
        &\hspace{1cm} -{\bf v}\cdot(p'[V]{\bf n} + p{\bf n}'[V] - \nu\nabla{\bf u}'[V]\cdot{\bf n} - \nu\nabla{\bf u}\cdot{\bf n}'[V]){\rm d}s\\
        =&\ \int_\Omega \gamma k_\epsilon\left(p + \frac{1}{2}|{\bf u}|^2\right){\bf u}\cdot{\bf n}'[V]{\rm d}x + \int_{\Gamma_{\rm wall}} V\cdot{\bf n}\partial_{\bf n}{\bf u}\cdot(\nu\partial_{\bf n}{\bf v} - q{\bf n}){\rm d}s\\
        =&\ \int_{\Gamma_{\rm wall}} \nu(V\cdot{\bf n})\partial_{\bf n}{\bf u}\cdot\partial_{\bf n}{\bf v}{\rm d}s.
    \end{align}
\end{proof}

\section{Lagrangian in weak sense}
To set up the optimality system for the shape optimization problem, we consider the following Lagrange function $\mathcal{L}:W^{1,2}(\Omega)^3\times L^2(\Omega)\times V(\Omega)\times L^2(\Omega)\times\mathcal{O}_{\rm ad}\times L^2(\Gamma_{\rm in})^d\times L^2(\Gamma_{\rm wall})^d\to\mathbb{R}$:
\begin{align}
    \label{weak Lagrangian}
    \mathcal{L}({\bf u},p,{\bf v},q,\Omega,{\bf v}_{\rm in},{\bf v}_{\rm wall}) :=& J_{12}^{\epsilon,\gamma}({\bf u},p,\Omega) + \int_\Omega \nu\nabla{\bf u}:\nabla{\bf v} + ({\bf u}\cdot\nabla){\bf u}\cdot{\bf v} - p\nabla\cdot{\bf v} - {\bf f}\cdot{\bf v} - q\nabla\cdot{\bf u}{\rm d}x \nonumber\\
    &+ \int_{\Gamma_{\rm in}} {\bf v}_{\rm in}\cdot({\bf u} - {\bf f}_{\rm in}){\rm d}s + \int_{\Gamma_{\rm wall}} {\bf v}_{\rm wall}\cdot{\bf u}{\rm d}s,
\end{align}
where ${\bf v},q,{\bf v}_{\rm in},{\bf v}_{\rm wall}$ are Lagrange multipliers.

\begin{lemma}
    Let $({\bf u},p)$ and $({\bf v},q)$ be weak solutions of \eqref{NSEs} and \eqref{adjoint NSEs}, respectively. Then
    \begin{itemize}
        \item[(i)] The first variation of $\mathcal{L}$ with respect to Lagrange multipliers is zero
        \begin{align}
            \mathcal{L}_{({\bf v},q,{\bf v}_{\rm in},{\bf v}_{\rm wall})}'({\bf u},p,{\bf v},q,\Omega,{\bf v}_{\rm in},{\bf v}_{\rm wall},{\bf v}_{\rm out})(\delta{\bf v},\delta q,\delta{\bf v}_{\rm in},\delta{\bf v}_{\rm wall}) = 0,
        \end{align}
        for all $(\delta{\bf v},\delta q,\delta{\bf v}_{\rm in},\delta{\bf v}_{\rm wall})\in V(\Omega)\times L^2(\Omega)\times L^2(\Gamma_{\rm in})^d\times L^2(\Gamma_{\rm wall})^d$.
        \item[(ii)] The first variation of $\mathcal{L}$ with respect to state variables is zero
        \begin{align}
            \mathcal{L}_{({\bf u},p)}'({\bf u},p,{\bf v},q,\Omega,{\bf v}_{\rm in},{\bf v}_{\rm wall})(\delta{\bf u},\delta p) = 0,
        \end{align}
        for all $(\delta{\bf u},\delta p)\in V(\Omega)\times L^2(\Omega)$.
        \item[(iii)] Then the first total variation of $\mathcal{L}$ satisfies
        \begin{align}
            \mathcal{L}'({\bf u},p,{\bf v},q,\Omega,{\bf v}_{\rm in},{\bf v}_{\rm wall})(\delta{\bf u},\delta p,\delta{\bf v},\delta q,\delta\Omega,\delta{\bf v}_{\rm in},\delta{\bf v}_{\rm wall}) = \mathcal{L}_{\Omega}({\bf u},p,{\bf v},q,\Omega,{\bf v}_{\rm in},{\bf v}_{\rm wall})(\delta\Omega),
        \end{align}
        for all $(\delta{\bf u},\delta p,\delta{\bf v},\delta q,\delta\Omega,\delta{\bf v}_{\rm in},\delta{\bf v}_{\rm wall})\in V(\Omega)\times L^2(\Omega)\times V(\Omega)\times C^{1,1}(\overline{D})\times L^2(\Omega)\times L^2(\Gamma_{\rm in})^d\times L^2(\Gamma_{\rm wall})^d$.
    \end{itemize}
\end{lemma}

\begin{proof}
    (i) We have
    \begin{align}
        \mathcal{L}_{{\bf v},q,{\bf v}_{\rm in},{\bf v}_{\rm wall}}'({\bf u},p,{\bf v},q,\Omega,{\bf v}_{\rm in},{\bf v}_{\rm wall},{\bf v}_{\rm out})(\delta{\bf v},\delta q,\delta{\bf v}_{\rm in},\delta{\bf v}_{\rm wall}) =&\ \int_\Omega \nu\nabla{\bf u}:\nabla\delta{\bf v} + ({\bf u}\cdot\nabla){\bf u}\cdot\delta{\bf v} - p\nabla\cdot\delta{\bf v} - {\bf f}\cdot\delta{\bf v} - \delta q\nabla\cdot{\bf u}\\
        &+ \int_{\Gamma_{\rm in}} \delta{\bf v}_{\rm in}\cdot({\bf u} - {\bf f}_{\rm in}){\rm d}s + \int_{\Gamma_{\rm wall}} \delta{\bf v}_{\rm wall}\cdot{\bf u}{\rm d}s,
    \end{align}
    which is zero since $({\bf u},p)$ is a weak solution of \eqref{NSEs}.
    
    (ii) Similarly,
    \begin{align}
        \mathcal{L}_{{\bf u},p}'({\bf u},p,{\bf v},q,\Omega,{\bf v}_{\rm in},{\bf v}_{\rm wall})(\delta{\bf u},\delta p) =&\ \int_\Omega \nu\nabla\delta{\bf u}:\nabla{\bf v} + (\delta{\bf u}\cdot\nabla){\bf u}\cdot{\bf v} + ({\bf u}\cdot\nabla)\delta{\bf u}\cdot{\bf v} - \delta p\nabla\cdot{\bf v} - q\nabla\cdot\delta{\bf u} \nonumber\\
        &\hspace{1cm} + \gamma k_\epsilon\left(\left(p + \frac{1}{2}|{\bf u}|^2\right){\bf n} + ({\bf u}\cdot{\bf n}){\bf u}\right)\cdot\delta{\bf u} + \gamma k_\epsilon{\bf u}\cdot{\bf n}\delta p{\rm d}x \nonumber\\
        &+ \int_{\Gamma_{\rm in}} {\bf v}_{\rm in}\cdot\delta{\bf u}{\rm d}s + \int_{\Gamma_{\rm wall}} {\bf v}_{\rm wall}\cdot\delta{\bf u}{\rm d}s + \int_{\Gamma_{\rm out}} (1 - \gamma)({\bf u}\cdot{\bf n} - \overline{u}){\bf n}\cdot\delta{\bf u}{\rm d}s.\label{1.7.2}
    \end{align}
    Integrating by parts
    \begin{align}
        \int_\Omega (\delta{\bf u}\cdot\nabla){\bf u}\cdot{\bf v}{\rm d}x &= \int_\Omega \sum_{i=1}^d\sum_{j=1}^d \delta u_j\partial_{x_j}u_iv_i{\rm d}x = \int_\Gamma \sum_{i=1}^d\sum_{j=1}^d \delta u_ju_iv_in_j{\rm d}s - \int_\Omega \sum_{i=1}^d\sum_{j=1}^d (u_i\partial_{x_j}\delta u_jv_i + u_i\delta u_j\partial_{x_j}v_i){\rm d}x\\
        &= \int_\Gamma ({\bf u}\cdot{\bf v})(\delta{\bf u}\cdot{\bf n}){\rm d}s - \int_\Omega ({\bf u}\cdot{\bf v})\nabla\cdot\delta{\bf u} + {\bf u}\cdot\nabla{\bf v}^\top\cdot\delta{\bf u}{\rm d}x,\\
        \int_\Omega ({\bf u}\cdot\nabla)\delta{\bf u}\cdot{\bf v}{\rm d}x &= \int_\Omega \sum_{i=1}^d\sum_{j=1}^d u_j\partial_{x_j}\delta u_iv_i{\rm d}x = \int_\Gamma \sum_{i=1}^d\sum_{j=1}^d u_j\delta u_iv_in_j{\rm d}s - \int_\Omega \sum_{i=1}^d\sum_{j=1}^d \delta u_i\partial_{x_j}u_jv_i + \delta u_iu_j\partial_{x_j}v_i{\rm d}x\\
        &= \int_\Gamma (\delta{\bf u}\cdot{\bf v})({\bf u}\cdot{\bf n}){\rm d}s - \int_\Omega (\delta{\bf u}\cdot{\bf v})\nabla\cdot{\bf u} + {\bf u}\cdot\nabla{\bf v}\cdot\delta{\bf u}{\rm d}x,
    \end{align}
    and then \eqref{1.7.2} becomes
    \begin{align}
        &\mathcal{L}_{{\bf u},p}'({\bf u},p,{\bf v},q,\Omega,{\bf v}_{\rm in},{\bf v}_{\rm wall})(\delta{\bf u},\delta p)\\
        =&\ \int_\Omega \nu\nabla\delta{\bf u}:\nabla{\bf v} - ((\nabla{\bf v})^\top\cdot{\bf u} + \nabla{\bf v}\cdot{\bf u})\cdot\delta{\bf u} - q\nabla\cdot\delta{\bf u} \\
        &\hspace{1cm}+ \gamma k_\epsilon\left(\left(p + \frac{1}{2}|{\bf u}|^2\right){\bf n} + ({\bf u}\cdot{\bf n}){\bf u}\right)\cdot\delta{\bf u} - \delta p\nabla\cdot{\bf v} + \gamma k_\epsilon{\bf u}\cdot{\bf n}\delta p - ({\bf u}\cdot{\bf v})\nabla\cdot\delta{\bf u} - (\delta{\bf u}\cdot{\bf v})\nabla\cdot{\bf u}{\rm d}x\\
        &+ \int_{\Gamma_{\rm out}} ({\bf u}\cdot{\bf v})(\delta{\bf u}\cdot{\bf n}) + (\delta{\bf u}\cdot{\bf v})({\bf u}\cdot{\bf n}) + (1 - \gamma)({\bf u}\cdot{\bf n} - \overline{u}){\bf n}\cdot\delta{\bf u}{\rm d}s = 0,
    \end{align}
    due to \eqref{weak formulation adjoint NSEs} and $\nabla\cdot{\bf u} = \nabla\cdot\delta{\bf u} = 0$.
    
    (iii) This follows directly from (i) and (ii).
\end{proof}

\begin{lemma}
    The shape derivative $\mathcal{L}_{\Omega}'(V)$ can be represented as
    \begin{itemize}
        \item[(i)] (Boundary representation)
        \item[(ii)] (Volume representation)
    \end{itemize}
\end{lemma}

\begin{proof}
    (i) Using \eqref{shape derivative boundary integral} and \eqref{shape derivative volume integral}, we obtain
    \begin{align}
        \mathcal{L}_{\Omega}'[V] =&\ \int_{\Gamma_{\rm wall}} V\cdot{\bf n}\left[\gamma k_\epsilon\left(p + \frac{1}{2}|{\bf u}|^2\right){\bf u}\cdot{\bf n} + \nu\nabla{\bf u}:\nabla{\bf v} + ({\bf u}\cdot\nabla){\bf u}\cdot{\bf v} - p\nabla\cdot{\bf v} - {\bf f}\cdot{\bf v} - q\nabla\cdot{\bf u}\right]{\rm d}s\\
        &+ \int_\Omega \left[\gamma k_\epsilon\left(p + \frac{1}{2}|{\bf u}|^2\right){\bf u}\cdot{\bf n} + \nu\nabla{\bf u}:\nabla{\bf v} + ({\bf u}\cdot\nabla){\bf u}\cdot{\bf v} - p\nabla\cdot{\bf v} - {\bf f}\cdot{\bf v} - q\nabla\cdot{\bf u}\right]'[V]{\rm d}x\\
        &+ \frac{1 - \gamma}{2}\int_{\Gamma_{\rm out}} (({\bf u}\cdot{\bf n} - \overline{u})^2)'[V]{\rm d}s.
    \end{align}
    
    (ii) 
\end{proof}

\section{Instationary NSEs}
We consider the following boundary value problem for the instationary Navier-Stokes equations with mixed boundary conditions:
\begin{equation}
    \label{iNSEs}
    \tag{iNSEs}
    \left\{\begin{split}
        {\bf u}_t - \nu\Delta{\bf u} + ({\bf u}\cdot\nabla){\bf u} + \nabla p &= {\bf f} &&\mbox{ in } (0,T)\times\Omega,\\
        \nabla\cdot{\bf u} &= 0 &&\mbox{ in } (0,T)\times\Omega,\\
        {\bf u}(0,\cdot) &= {\bf u}_0 &&\mbox{ in } \Omega,\\
        {\bf u} &= {\bf f}_{\rm in} &&\mbox{ on } (0,T)\times\Gamma_{\rm in},\\
        {\bf u} &= {\bf 0} &&\mbox{ on } (0,T)\times\Gamma_{\rm wall},\\
        -\nu\partial_{\bf n}{\bf u} + p{\bf n} &= {\bf 0} &&\mbox{ on } (0,T)\times\Gamma_{\rm out},
    \end{split}\right.
\end{equation}
with a given finite time time $T < \infty$. Here ${\bf u}:[0,T]\times\Omega\to\mathbb{R}^d$ and $p:[0,T]\times\Omega\to\mathbb{R}$ denote the velocity vector and the kinematic pressure, respectively. We assume that the kinematic viscosity $\nu > 0$ and the density of the external volume force ${\bf f}\in L^2(0,T;L^2(\Omega))$, the initial condition ${\bf u}_0\in L^2(\Omega)$, the inflow profile ${\bf f}_{\rm in}\in L^2(0,T;H^{1/2}(\Gamma_{\rm in})^d)$ are given.

Let
\begin{align*}
    V(0,T;\Omega) := H^1(0,T;L^2(\Omega)^d)\cap L^2(0,T;W^{1,2}(\Omega)^d)
\end{align*}
equipped with
\begin{align*}
    \|{\bf v}\|_V := \|\nabla{\bf v}\|_{L^2(0,T;L^2(\Omega))} + \|{\bf v}_t\|_{L^2(0,T;L^2(\Omega))}.
\end{align*}

\begin{definition}[Weak solution]
    A vector function $({\bf u},p)\in V(0,T;\Omega)\times L^2(0,T;L^2(\Omega))$ is called a \emph{weak solution} of \eqref{iNSEs} if it satisfies
    \begin{align}
        \int_0^T\int_\Omega {\bf u}_t\cdot{\bf v} + \nu\nabla{\bf u}:\nabla{\bf v} + ({\bf u}\cdot\nabla){\bf u}\cdot{\bf v} - p\nabla\cdot{\bf v} - {\bf f}\cdot{\bf v}{\rm d}x = 0,\ \forall{\bf v}\in V_{\rm D}(0,T;\Omega),\\
        \nabla\cdot{\bf u} = 0 \mbox{ in } [0,T]\times\Omega,\ {\bf u}|_{[0,T]\times\Gamma_{\rm in}} = {\bf f}_{\rm in},\ {\bf u}|_{[0,T]\times\Gamma_{\rm wall}} = {\bf 0},
    \end{align}
    where $V_{\rm D}(0,T;\Omega) := \{{\bf v}\in V(0,T;\Omega);{\bf v}|_{[0,T]\times(\Gamma_{\rm in}\cup\Gamma_{\rm wall})} = {\bf 0}\}$.
\end{definition}

\section{Well-posedness of instationary NSEs}
[Do not know yet]

\section{Cost functionals}
We consider the following two criteria.

\textbf{Outflow uniformity}: The uniformity of the flow upon leaving the outlet plane is an important design criterion of e.g. \textit{automotive air ducts}. Other use: Efficiency of distributing fresh air inside the car.
\begin{align*}
    J_1({\bf u},\Omega) := \frac{1}{2}\int_0^T\int_{\Gamma_{\rm out}} ({\bf u}\cdot{\bf n} - \overline{u})^2{\rm d}s{\rm d}t \mbox{ with } \overline{u} := -\frac{1}{T|\Gamma_{\rm out}|}\int_0^T\int_{\Gamma_{\rm in}} {\bf f}_{\rm in}\cdot{\bf n}{\rm d}s{\rm d}t.
\end{align*}

\begin{remark}
    Other choices of $J_1$: 
    \begin{align}
        J_1({\bf u}(T),\Omega) := \frac{1}{2}\int_{\Gamma_{\rm out}} ({\bf u}(T)\cdot{\bf n} - \overline{u})^2{\rm d}s \mbox{ with } \overline{u}(T) := -\frac{1}{|\Gamma_{\rm out}|}\int_{\Gamma_{\rm in}} {\bf f}_{\rm in}(T)\cdot{\bf n}{\rm d}s.
    \end{align}
    or if ${\bf f}_{\rm in}\in C([0,T];L^2(\Gamma_{\rm in})^d)$, we consider
    \begin{align}
        J_1({\bf u},\Omega) := \frac{1}{2}\int_0^T\int_{\Gamma_{\rm out}} ({\bf u}\cdot{\bf n} - \overline{u})^2{\rm d}s{\rm d}t \mbox{ with } \overline{u}(t) := -\frac{1}{|\Gamma_{\rm out}|}\int_{\Gamma_{\rm in}} {\bf f}_{\rm in}(t)\cdot{\bf n}{\rm d}s,\ \forall t\in[0,T].
    \end{align}
\end{remark}
\textbf{Energy dissipation}: Compute power dissipated by a fluid dynamic device as the net inward flux of energy.

I.e., total pressure, through the device boundaries for smooth pressure $p$: for a fixed $\epsilon > 0$, we consider
\begin{align}
    J_2^\epsilon({\bf u},p,\Omega) :=&\ -\frac{|\Gamma_{\rm in}|}{|\Gamma_{\rm in}^\epsilon|}\int_0^T\int_{\Gamma_{\rm in}^\epsilon} \left(p + \frac{1}{2}|{\bf u}|^2\right){\bf u}\cdot{\bf n}{\rm d}x{\rm d}t - \frac{|\Gamma_{\rm out}|}{|\Gamma_{\rm out}^\epsilon|}\int_0^T\int_{\Gamma_{\rm out}^\epsilon} \left(p + \frac{1}{2}|{\bf u}|^2\right){\bf u}\cdot{\bf n}{\rm d}x{\rm d}t\\
    =&\ \int_0^T\int_\Omega k_\epsilon\left(p + \frac{1}{2}|{\bf u}|^2\right){\bf u}\cdot{\bf n}{\rm d}x{\rm d}t,
\end{align}
where
\begin{align}
    k_\epsilon(x):= -\frac{|\Gamma_{\rm in}|}{|\Gamma_{\rm in}^\epsilon|}\chi_{\overline{\Gamma_{\rm in}^\epsilon}}(x) - \frac{|\Gamma_{\rm out}|}{|\Gamma_{\rm out}^\epsilon|}\chi_{\overline{\Gamma_{\rm out}^\epsilon}}(x),\ \forall x\in\Omega,
\end{align}
here $\chi_A$ denotes the characteristic function of at set $A$.

Note that we have used Lebesgue measure in $\mathbb{R}^d$ for $\Gamma_{\rm in}^\epsilon,\Gamma_{\rm out}^\epsilon$ and Lebesgue measure in $\mathbb{R}^{d-1}$ for $\Gamma_{\rm in},\Gamma_{\rm out}$.

We consider the mixed cost functional with a weighting parameter $\gamma\in[0,1]$,
\begin{align}
    J_{12}^{\epsilon,\gamma}({\bf u},p,\Omega) :=&\ (1 - \gamma)J_1({\bf u},\Omega) + \gamma J_2^\epsilon({\bf u},p,\Omega)\\
    =&\ \frac{1 - \gamma}{2}\int_0^T\int_{\Gamma_{\rm out}} ({\bf u}\cdot{\bf n} - \overline{u})^2{\rm d}s{\rm d}t + \int_0^T\int_\Omega \gamma k_\epsilon\left(p + \frac{1}{2}|{\bf u}|^2\right){\bf u}\cdot{\bf n}{\rm d}x{\rm d}t.
\end{align}

\section{Optimization problem}
The optimization problems can be formulated as follows: Find $\Omega$ over a class of admissible domain $\mathcal{O}_{\rm ad}$ such that the cost functional $J_{12}^{\epsilon,\gamma}({\bf u},p,\Omega)$ is minimized subject to the NSEs \eqref{iNSEs},
\begin{align}
    \label{SOP}
    \min_{\Omega\in\mathcal{O}_{\rm ad}} J_{12}^{\epsilon,\gamma}({\bf u},p,\Omega) \mbox{ such that } ({\bf u},p) \mbox{ solves \eqref{iNSEs}}.
\end{align}

\section{Adjoint equation of instationary NSEs}
For each weak solution $({\bf u},p)\in V(0,T;\Omega)\times L^2(0,T;L^2(\Omega))$ of \eqref{iNSEs}, we introduce the associated adjoint equation which is given by
\begin{equation}
    \label{adjoint insNSEs}
    \tag{adjInsNSEs}
    \left\{\begin{split}
        -{\bf v}_t - \nu\Delta{\bf v} - \nabla{\bf v}\cdot{\bf u} - ({\bf u}\cdot\nabla){\bf v} + \nabla q &= -\gamma k_\epsilon\left(\left(p + \frac{1}{2}|{\bf u}|^2\right){\bf n} + ({\bf u}\cdot{\bf n}){\bf u}\right) &&\mbox{ in } (0,T)\times\Omega,\\
        \nabla\cdot{\bf v} &= \gamma k_\epsilon{\bf u}\cdot{\bf n} &&\mbox{ in } (0,T)\times\Omega,\\
        {\bf v}(T) &= {\bf 0} &&\mbox{ in } \Omega,\\
        {\bf v} &= {\bf 0} &&\mbox{ on } (0,T)\times(\Gamma_{\rm in}\cup\Gamma_{\rm wall}),\\
        ({\bf v}\cdot{\bf u}){\bf n} + ({\bf u}\cdot{\bf n}){\bf v} + \nu\partial_{\bf n}{\bf v} - q{\bf n} &= -(1 - \gamma)({\bf u}\cdot{\bf n} - \bar{u}){\bf n} &&\mbox{ on } (0,T)\times\Gamma_{\rm out},
    \end{split}\right.    
\end{equation}

\begin{definition}[Weak solution]
    For a given weak solution $({\bf u},p)\in V(0,T;\Omega)\times L^2(0,T;L^2(\Omega))$ of \eqref{iNSEs}, a vector function $({\bf v},q)\in V_{\rm D}(0,T;\Omega)\times L^2(0,T;L^2(\Omega))$ is called a \emph{weak solution} of \eqref{adjoint NSEs} if it satisfies\footnote{Integrate by parts
        \begin{align}
            -\nu\int_\Omega \Delta{\bf v}\cdot{\bf w}{\rm d}x &= -\nu\int_\Gamma {\bf n}\cdot\nabla{\bf v}\cdot{\bf w}{\rm d}s + \nu\int_\Omega \nabla{\bf v}:\nabla{\bf w}{\rm d}x,\\
            \int_\Omega \nabla q\cdot{\bf w}{\rm d}x &= -\int_\Omega q\nabla\cdot{\bf w}{\rm d}x + \int_\Gamma q{\bf w}\cdot{\bf n}{\rm d}s.
    \end{align}}
    \begin{align}
        \label{weak formulation adjoint insNSEs}
        &\int_0^T\int_\Omega {\bf v}_t\cdot{\bf w} + \nu\nabla{\bf v}:\nabla{\bf w} - (({\bf u}\cdot\nabla){\bf v} + \nabla{\bf v}\cdot{\bf u})\cdot{\bf w}  - q\nabla\cdot{\bf w}{\rm d}x{\rm d}t\\
        &+ \int_0^T\int_{\Gamma_{\rm out}} (({\bf v}\cdot{\bf u}){\bf n} + ({\bf u}\cdot{\bf n}){\bf v} + (1 - \gamma)({\bf u}\cdot{\bf n} - \overline{u}){\bf n})\cdot{\bf w}{\rm d}s{\rm d}t\\
        &= -\int_0^T\int_\Omega \gamma k_\epsilon\left(\left(p + \frac{1}{2}|{\bf u}|^2\right){\bf n} + ({\bf u}\cdot{\bf n}){\bf u}\right)\cdot{\bf w}{\rm d}x{\rm d}t,\ \forall{\bf w}\in V(\Omega),\\
        \nabla\cdot{\bf v} &= \gamma k_\epsilon{\bf u}\cdot{\bf n} \mbox{ in } (0,T)\times\Omega,\ {\bf v}|_{(0,T)\times(\Gamma_{\rm in}\cup\Gamma_{\rm wall})} = {\bf 0}.
    \end{align}
\end{definition}

\section{Direct computation of shape derivatives without Lagrangian}
From now on, let the perturbation field $V$ belong to the following space
\begin{align}
    \mathcal{F}_\epsilon := \{V\in C^{1,1}(\overline{D});V|_{\Gamma_{\rm in}^\epsilon\cup\Gamma_{\rm out}^\epsilon} = {\bf 0}\}.
\end{align}

\begin{lemma}
    Formally the local shape derivative ${\bf u}'[V]$ of \eqref{iNSEs} satisfies the following system
    \begin{equation}
        \label{PDE for ins u', p'}
        \left\{\begin{split}
            {\bf u}_t'[V] - \nu\Delta{\bf u}'[V] + D{\bf u}\cdot{\bf u}'[V] + D{\bf u}'[V]\cdot{\bf u} + \nabla p'[V] &= {\bf 0} &&\mbox{ in } (0,T)\times\Omega,\\
            \nabla\cdot{\bf u}'[V] &= 0 &&\mbox{ in } \Omega,\\
            {\bf u}'(0)[V] &= {\bf 0} &&\mbox{ in } \Omega,\\
            {\bf u}'[V] &= -V\cdot{\bf n}\partial_{\bf n}({\bf u} - {\bf f}_{\rm in}) &&\mbox{ on } (0,T)\times\Gamma_{\rm in},\\
            {\bf u}'[V] &= -V\cdot{\bf n}\partial_{\bf n}{\bf u} &&\mbox{ on } (0,T)\times\Gamma_{\rm wall},\\
            -\nu\partial_{\bf n}{\bf u}'[V] + p'[V]{\bf n} &= 0 &&\mbox{ on } (0,T)\times\Gamma_{\rm out}.
        \end{split}\right.
    \end{equation}
\end{lemma}

\begin{proof}
    *
\end{proof}

\begin{lemma}
    \label{Kasumba_Kunisch2012 Lemma 3.2}
    Formally the local shape derivative ${\bf u}'[V]$ satisfies
    \begin{align}
        {\bf u}'[V]\cdot{\bf n} = 0 \mbox{ on } (0,T)\times\Gamma_{\rm wall}.
    \end{align}
\end{lemma}

\begin{proof}
    Using ${\bf u}'[V] = -(V\cdot{\bf n})\partial_{\bf n}{\bf u}$ on $(0,T)\times\Gamma_{\rm wall}$, using the tangential divergence formula \eqref{tangential div}, we have on $(0,T)\times\Gamma_{\rm wall}$ that
    \begin{align}
        {\bf u}'[V]\cdot{\bf n} = -(V\cdot{\bf n})\partial_{\bf n}{\bf u}\cdot{\bf n} = (V\cdot{\bf n}){\rm div}_\Gamma{\bf u} - (V\cdot{\bf n})\nabla\cdot{\bf u}.
    \end{align}
    Since ${\bf u}|_{(0,T)\times\Gamma_{\rm wall}} = {\bf 0}$, ${\rm div}_\Gamma{\bf u} = 0$. Combining this with $\nabla\cdot{\bf u} = 0$ yields the desired formula.
\end{proof}

\begin{lemma}
    The shape derivative $dJ_{12}^{\epsilon\gamma}({\bf u},p,\Omega)[V]$ can be represented as
    \begin{itemize}
        \item[(i)] (Boundary representation)
        \begin{align}
            dJ_{12}^{\epsilon,\gamma}({\bf u},p,\Omega)[V] = -\int_0^T\int_{\Gamma_{\rm wall}} \nu(V\cdot{\bf n})\partial_{\bf n}{\bf u}\cdot\partial_{\bf n}{\bf v}{\rm d}s{\rm d}t.
        \end{align}
        \item[(ii)] (Volume representation)
    \end{itemize}
\end{lemma}

\begin{proof}
    (i) Applying the boundary formula for shape derivatives yields
    \begin{align}
        dJ_{12}^{\epsilon,\gamma}({\bf u},p,\Omega)[V] =&\ \int_0^T\int_\Gamma (V\cdot{\bf n})\gamma k_\epsilon\left(p + \frac{1}{2}|{\bf u}|^2\right){\bf u}\cdot{\bf n}{\rm d}s{\rm d}t + \int_0^T\int_\Omega \left(\gamma k_\epsilon\left(p + \frac{1}{2}|{\bf u}|^2\right){\bf u}\cdot{\bf n}\right)'[V]{\rm d}x{\rm d}t\\
        &+\frac{1 - \gamma}{2}\int_0^T\int_{\Gamma_{\rm out}} (V\cdot{\bf n})\left[\partial_{\bf n}(({\bf u}\cdot{\bf n} - \overline{u})^2) + \kappa({\bf u}\cdot{\bf n} - \overline{u})^2\right] + (({\bf u}\cdot{\bf n} - \overline{u})^2)'[V]{\rm d}s{\rm d}t\\
        =&\ \int_0^T\int_\Omega \gamma k_\epsilon\left(\left(p + \frac{1}{2}|{\bf u}|^2\right){\bf n} + ({\bf u}\cdot{\bf n}){\bf u}\right)\cdot{\bf u}'[V] + \gamma k_\epsilon{\bf u}\cdot{\bf n}p'[V]{\rm d}x{\rm d}t + (1 - \gamma)\int_0^T\int_{\Gamma_{\rm out}} ({\bf u}\cdot{\bf n} - \overline{u}){\bf n}\cdot{\bf u}'[V]{\rm d}s{\rm d}t.\label{3.7.6}
    \end{align}
    %    =&\ \int_\Omega \left((\nabla{\bf v})^\top\cdot{\bf u} + \nabla{\bf v}\cdot{\bf u} + \nu\Delta{\bf v} - \nabla q\right){\bf u}'[V] + \nabla\cdot{\bf v}p'[V]{\rm d}x\\    
    %    &+ \int_{\Gamma_{\rm out}} (-({\bf v}\cdot{\bf u}){\bf n} - ({\bf u}\cdot{\bf n}){\bf v} - \nu\partial_{\bf n}{\bf v} + q{\bf n})\cdot{\bf u}'[V]{\rm d}s.
    Testing \eqref{PDE for u', p'} with the adjoint variable $({\bf v},q)$ yields
    \begin{align}
        \label{3.7.3}
        \int_0^T\int_\Omega {\bf v}\cdot({\bf u}'_t[V] - \nu\Delta{\bf u}'[V] + D{\bf u}\cdot{\bf u}'[V] + D{\bf u}'[V]\cdot{\bf u} + \nabla p'[V]) - q\nabla\cdot{\bf u}'[V] {\rm d}x{\rm d}t = 0.
    \end{align}
    Integrating by parts \eqref{3.7.3} gives
    \begin{align}
        &\int_0^T\int_\Omega {\bf u}'[V]\cdot\left[-{\bf v}_t -\nu\Delta{\bf v} - (\nabla{\bf v})^\top\cdot{\bf u} - \nabla{\bf v}\cdot{\bf u} + \nabla q\right] - p'[V]\nabla\cdot{\bf v}{\rm d}x{\rm d}t + \int_\Omega {\bf v}(T)\cdot{\bf u}'[V](T) - {\bf v}(0)\cdot{\bf u}'[V](0){\rm d}x\\
        &+ \int_0^T\int_\Gamma {\bf u}'[V]\cdot(({\bf v}\cdot{\bf u}){\bf n} + ({\bf u}\cdot{\bf n}){\bf v} + \nu\partial_{\bf n}{\bf v} - q{\bf n}) - \nu{\bf n}\cdot\nabla{\bf u}'[V]\cdot{\bf v} + p'[V]{\bf v}\cdot{\bf n}{\rm d}s{\rm d}t = 0.
    \end{align}
    Since $({\bf v},q)$ satisfies \eqref{adjoint NSEs}, the last equality becomes
    \begin{align}
        &\int_0^T\int_\Omega -{\bf u}'[V]\cdot\gamma k_\epsilon\left(\left(p + \frac{1}{2}|{\bf u}|^2\right){\bf n} + ({\bf u}\cdot{\bf n}){\bf u}\right) - \gamma k_\epsilon p'[V]{\bf u}\cdot{\bf n}{\rm d}x{\rm d}t + \int_0^T\int_{\Gamma_{\rm wall}} {\bf u}'[V]\cdot(\nu\partial_{\bf n}{\bf v} - q{\bf n}){\rm d}s{\rm d}t\\
        &+ \int_0^T\int_{\Gamma_{\rm out}} -{\bf u}'[V]\cdot(1 - \gamma)({\bf u}\cdot{\bf n} - \overline{u}){\bf n}{\rm d}s{\rm d}t = 0.
    \end{align}
    Then \eqref{3.7.6} becomes
    \begin{align}
        \label{3.7.14}
        dJ_{12}^{\epsilon,\gamma}({\bf u},p,\Omega)[V] = \int_0^T\int_{\Gamma_{\rm wall}} {\bf u}'[V]\cdot(\nu\partial_{\bf n}{\bf v} - q{\bf n}){\rm d}s{\rm d}t.
    \end{align}
    The term ${\bf u}'[V]\cdot q{\bf n}$ vanishes in $\Gamma_{\rm wall}$ due to Lemma \ref{Kasumba_Kunisch2012 Lemma 3.2}. Using \eqref{PDE for u', p'}, we obtain from \eqref{3.7.14}
    \begin{align}
        dJ_{12}^{\epsilon,\gamma}({\bf u},p,\Omega)[V] = -\int_0^T\int_{\Gamma_{\rm wall}} \nu(V\cdot{\bf n})\partial_{\bf n}{\bf u}\cdot\partial_{\bf n}{\bf v}{\rm d}s{\rm d}t.
    \end{align}
    Since the mapping $V\mapsto dJ_{12}^{\epsilon\gamma}({\bf u},p,\Omega)[V]V$ is linear and continuous, the shape gradient of $J_{12}^{\epsilon,\gamma}({\bf u},p,\Omega)$ is given by $\nabla J_{12}^{\epsilon,\gamma}({\bf u},p,\Omega){\bf n} = -\int_0^T \nu(\partial_{\bf n}{\bf u}\cdot\partial_{\bf n}{\bf v}){\bf n}|_{\Gamma_{\rm wall}}{\rm d}t$.
    
    (ii) Applying the volume formula for shape derivatives yields
    \begin{align}
        dJ_{12}^{\epsilon\gamma}({\bf u},p,\Omega)[V] =&\ \int_\Omega \gamma k_\epsilon\left(p + \frac{1}{2}|{\bf u}|^2\right){\bf u}\cdot{\bf n}\nabla\cdot V + \gamma k_\epsilon d\left(\left(p + \frac{1}{2}|{\bf u}|^2\right){\bf u}\cdot{\bf n}\right)[V]{\rm d}x\\
        &+ \frac{1 - \gamma}{2}\int_{\Gamma_{\rm out}} (V\cdot{\bf n})\left(\partial_{\bf n}(({\bf u}\cdot{\bf n} - \overline{u})^2) + \kappa({\bf u}\cdot{\bf n} - \overline{u})^2\right) + (({\bf u}\cdot{\bf n} - \overline{u})^2)'[V]{\rm d}s\\
        =&\ \int_\Omega \gamma k_\epsilon\left(p + \frac{1}{2}|{\bf u}|^2\right){\bf u}\cdot{\bf n}\nabla\cdot V + \gamma k_\epsilon d\left(\left(p + \frac{1}{2}|{\bf u}|^2\right){\bf u}\cdot{\bf n}\right)[V]{\rm d}x + (1 - \gamma)\int_{\Gamma_{\rm out}} ({\bf u}\cdot{\bf n} - \overline{u}){\bf n}\cdot{\bf u}'[V]{\rm d}s.
    \end{align}
    We compute the material derivative $d\left(\left(p + \frac{1}{2}|{\bf u}|^2\right){\bf u}\cdot{\bf n}\right)[V]$ in $\Gamma_{\rm in}^\epsilon\cup\Gamma_{\rm wall}^\epsilon$:
    \begin{align}
        d\left(\left(p + \frac{1}{2}|{\bf u}|^2\right){\bf u}\cdot{\bf n}\right)[V] =&\ + \left(\left(p + \frac{1}{2}|{\bf u}|^2\right){\bf u}\cdot{\bf n}\right)'[V] + V\cdot\nabla\left(\left(p + \frac{1}{2}|{\bf u}|^2\right){\bf u}\cdot{\bf n}\right)\\
        =&\ \left(\left(p + \frac{1}{2}|{\bf u}|^2\right){\bf n} + ({\bf u}\cdot{\bf n}){\bf u}\right){\bf u}'[V] + {\bf u}\cdot{\bf n}p'[V] + V\cdot\left[\left(\nabla p + \nabla{\bf u}\cdot{\bf u}\right){\bf u}\cdot{\bf n} + \left(p + \frac{1}{2}|{\bf u}|^2\right)\nabla{\bf u}\cdot{\bf n}\right].
    \end{align}
    Then
    \begin{align}
        dJ_{12}^{\epsilon,\gamma}({\bf u},p,\Omega)[V] =&\ \int_\Omega \gamma k_\epsilon\left(p + \frac{1}{2}|{\bf u}|^2\right){\bf u}\cdot{\bf n}\nabla\cdot V\\
        &+ \gamma k_\epsilon \left\{V\cdot\left[\left(\nabla p + \nabla{\bf u}\cdot{\bf u}\right){\bf u}\cdot{\bf n} + \left(p + \frac{1}{2}|{\bf u}|^2\right)\nabla{\bf u}\cdot{\bf n}\right] + \left(\left(p + \frac{1}{2}|{\bf u}|^2\right){\bf n} + ({\bf u}\cdot{\bf n}){\bf u}\right){\bf u}'[V] + {\bf u}\cdot{\bf n}p'[V]\right\}{\rm d}x\\
        &+ (1 - \gamma)\int_{\Gamma_{\rm out}} ({\bf u}\cdot{\bf n} - \overline{u}){\bf n}\cdot{\bf u}'[V]{\rm d}s.
    \end{align}
    We recall from (i) that
    \begin{align*}
        \int_\Omega {\bf u}'[V]\cdot\gamma k_\epsilon\left(\left(p + \frac{1}{2}|{\bf u}|^2\right){\bf n} + ({\bf u}\cdot{\bf n}){\bf u}\right) + \gamma k_\epsilon p'[V]{\bf u}\cdot{\bf n}{\rm d}x = \int_{\Gamma_{\rm wall}} {\bf u}'[V]\cdot(\nu\partial_{\bf n}{\bf v} - q{\bf n}){\rm d}s + \int_{\Gamma_{\rm out}} -{\bf u}'[V]\cdot(1 - \gamma)({\bf u}\cdot{\bf n} - \overline{u}){\bf n}{\rm d}s.
    \end{align*}
    Then
    \begin{align*}
        dJ_{12}^{\epsilon,\gamma}(\Omega)[V] =&\ \int_\Omega \gamma k_\epsilon\left(p + \frac{1}{2}|{\bf u}|^2\right){\bf u}\cdot{\bf n}\nabla\cdot V + \gamma k_\epsilon V\cdot\left((\nabla p + \nabla{\bf u}\cdot{\bf u}){\bf u}\cdot{\bf n} + \left(p + \frac{1}{2}|{\bf u}|^2\right)\nabla{\bf u}\cdot{\bf n}\right){\rm d}x\\
        &+ \int_{\Gamma_{\rm wall}} {\bf u}'[V]\cdot(\nu\partial_{\bf n}{\bf v} - q{\bf n}){\rm d}s\\
        =&\ \int_\Omega \gamma k_\epsilon\left(p + \frac{1}{2}|{\bf u}|^2\right){\bf u}\cdot{\bf n}\nabla\cdot V + \gamma k_\epsilon V\cdot\left((\nabla p + \nabla{\bf u}\cdot{\bf u}){\bf u}\cdot{\bf n} + \left(p + \frac{1}{2}|{\bf u}|^2\right)\nabla{\bf u}\cdot{\bf n}\right){\rm d}x\\
        &- \int_{\Gamma_{\rm wall}} \nu(V\cdot{\bf n})\partial_{\bf n}{\bf u}\cdot\partial_{\bf n}{\bf v}{\rm d}s.
    \end{align*}
\end{proof}

%\begin{remark}
%    \textbf{Open}: Why there are still some terms except $\partial_{\bf n}{\bf u}\cdot\partial_{\bf n}{\bf v}$ left?
%\end{remark}

\section{Formal Lagrangian}
To set up the optimality system for the shape optimization problem, we consider the following Lagrange function:
\begin{align}
    \label{full instationary Lagrangian}
    &\mathcal{L}({\bf u},p,{\bf v},q,\Omega,{\bf v}_0,{\bf v}_{\rm in},{\bf v}_{\rm wall},{\bf v}_{\rm out}) := J_{12}^{\epsilon,\gamma}({\bf u},p,\Omega) + \int_0^T\int_\Omega {\bf v}\cdot({\bf u}_t - \nu\Delta{\bf u} + ({\bf u}\cdot\nabla){\bf u} + \nabla p - {\bf f}) - q\nabla\cdot{\bf u}{\rm d}x{\rm d}t \nonumber\\
    &+ \int_\Omega {\bf v}_0\cdot({\bf u}(0) - {\bf u}_0){\rm d}x + \int_0^T\int_{\Gamma_{\rm in}} {\bf v}_{\rm in}\cdot({\bf u} - {\bf f}_{\rm in}){\rm d}s{\rm d}t + \int_0^T\int_{\Gamma_{\rm wall}} {\bf v}_{\rm wall}\cdot{\bf u}{\rm d}s{\rm d}t + \int_0^T\int_{\Gamma_{\rm out}} {\bf v}_{\rm out}\cdot(p{\bf n} - \nu\partial_{\bf n}{\bf u}){\rm d}s{\rm d}t,
\end{align}
where ${\bf v},q,{\bf v}_0,{\bf v}_{\rm in},{\bf v}_{\rm wall},{\bf v}_{\rm out}$ are Lagrange multipliers.

\begin{proof}[Demonstration]
    Choose the Lagrange multiplier $({\bf v},q)$ such that the variation with respect to the state variables vanishes identically, $\partial_{\bf u}\mathcal{L}\cdot\delta{\bf u} + \partial_p\mathcal{L}\delta p = 0$, which reads as
    \begin{align}
        &\partial_{\bf u}J_{12}^{\epsilon,\gamma}({\bf u},p,\Omega)\cdot\delta{\bf u} + \partial_pJ_{12}^{\epsilon,\gamma}({\bf u},p,\Omega)\delta p + \int_0^T\int_\Omega {\bf v}\cdot[\delta{\bf u}_t - \nu\Delta\delta{\bf u} + (\delta{\bf u}\cdot\nabla){\bf u} + ({\bf u}\cdot\nabla)\delta{\bf u}] - q\nabla\cdot\delta{\bf u} + {\bf v}\cdot\nabla\delta p{\rm d}x \nonumber\\
        &+ \int_\Omega {\bf v}_0\cdot\delta{\bf u}(0){\rm d}x + \int_0^T\int_{\Gamma_{\rm in}} {\bf v}_{\rm in}\cdot\delta{\bf u}{\rm d}s{\rm d}t + \int_0^T\int_{\Gamma_{\rm wall}} {\bf v}_{\rm wall}\cdot\delta{\bf u}{\rm d}s{\rm d}t + \int_0^T\int_{\Gamma_{\rm out}} {\bf v}_{\rm out}\cdot(\delta p{\bf n} - \nu\partial_{\bf n}\delta{\bf u}){\rm d}s{\rm d}t = 0.\label{1.4.2}
    \end{align}
    We integrate by parts term by term: the term involving time derivative:
    \begin{align}
        \int_0^T\int_\Omega {\bf v}\cdot\delta{\bf u}_t{\rm d}x{\rm d}t = \int_\Omega {\bf v}(T)\cdot\delta{\bf u}(T) - {\bf v}(0)\cdot\delta{\bf u}(0){\rm d}x - \int_0^T\int_\Omega {\bf v}_t\cdot\delta{\bf u}{\rm d}x{\rm d}t,
    \end{align}
    Laplacian term:
    \begin{align}
        -\nu\int_\Omega {\bf v}\cdot\Delta\delta{\bf u}{\rm d}x &= -\nu\int_\Omega \sum_{i=1}^d\sum_{j=1}^d v_i\partial_{x_j}^2\delta u_i{\rm d}x = -\nu\int_\Gamma \sum_{i=1}^d\sum_{j=1}^d v_i\partial_{x_j}\delta u_in_j{\rm d}s + \nu\int_\Omega \sum_{i=1}^d\sum_{j=1}^d \partial_{x_j}v_i\partial_{x_j}\delta u_i{\rm d}x\\
        &= -\nu\int_\Gamma {\bf n}\cdot\nabla\delta{\bf u}\cdot{\bf v}{\rm d}s + \nu\int_\Gamma \sum_{i=1}^d\sum_{j=1}^d \partial_{x_j}v_i\delta u_in_j{\rm d}s - \nu \int_\Omega \sum_{i=1}^d\sum_{j=1}^d \partial_{x_j}^2v_i\delta u_i{\rm d}x\\
        &= -\nu\int_\Gamma {\bf n}\cdot\nabla\delta{\bf u}\cdot{\bf v}{\rm d}s + \nu\int_\Gamma {\bf n}\cdot\nabla{\bf v}\cdot\delta{\bf u}{\rm d}s - \nu\int_\Omega \Delta{\bf v}\cdot\delta{\bf u}{\rm d}x,
    \end{align}
    the 2 terms produced by the nonlinear term $({\bf u}\cdot\nabla){\bf u}$:
    \begin{align}
        \int_\Omega {\bf v}\cdot((\delta{\bf u}\cdot\nabla){\bf u}){\rm d}x &= \int_\Omega \sum_{i=1}^d\sum_{j=1}^d v_i\partial_{x_j}u_i\delta u_j{\rm d}x = \int_\Gamma \sum_{i=1}^d\sum_{j=1}^d v_iu_i\delta u_jn_j{\rm d}s - \int_\Omega \sum_{i=1}^d\sum_{j=1}^d (u_i\partial_{x_j}v_i\delta u_j + u_iv_i\partial_{x_j}\delta u_j){\rm d}x\\
        &= \int_\Gamma ({\bf v}\cdot{\bf u})(\delta{\bf u}\cdot{\bf n}){\rm d}s - \int_\Omega [{\bf u}\cdot\nabla{\bf v}^\top\cdot\delta{\bf u} + ({\bf u}\cdot{\bf v})\nabla\cdot\delta{\bf u}]{\rm d}x = \int_\Gamma ({\bf v}\cdot{\bf u})(\delta{\bf u}\cdot{\bf n}){\rm d}s - \int_\Omega {\bf u}\cdot\nabla{\bf v}^\top\cdot\delta{\bf u}{\rm d}x,\\
        \int_\Omega {\bf v}\cdot(({\bf u}\cdot\nabla)\delta{\bf u}){\rm d}x &= \int_\Omega \sum_{i=1}^d\sum_{j=1}^d v_iu_j\partial_{x_j}\delta u_i{\rm d}x = \int_\Gamma \sum_{i=1}^d\sum_{j=1}^d v_i\delta u_iu_jn_j{\rm d}s - \int_\Omega \sum_{i=1}^d\sum_{j=1}^d (\delta u_i\partial_{x_j}v_iu_j + \delta u_iv_i\partial_{x_j}u_j){\rm d}x\\
        &= \int_\Gamma ({\bf u}\cdot{\bf n})({\bf v}\cdot\delta{\bf u}){\rm d}s - \int_\Omega \left[({\bf u}\cdot\nabla){\bf v}\cdot\delta{\bf u} + \nabla\cdot{\bf u}({\bf v}\cdot\delta{\bf u})\right]{\rm d}x = \int_\Gamma ({\bf u}\cdot{\bf n})({\bf v}\cdot\delta{\bf u}){\rm d}s - \int_\Omega ({\bf u}\cdot\nabla){\bf v}\cdot\delta{\bf u}{\rm d}x,
    \end{align}
    divergence term:
    \begin{align}
        -\int_\Omega q\nabla\cdot\delta{\bf u}{\rm d}x = \int_\Omega \delta{\bf u}\cdot\nabla q{\rm d}x - \int_\Gamma q\delta{\bf u}\cdot{\bf n}{\rm d}s,
    \end{align}
    and the term produced by $\nabla p$:
    \begin{align}
        \int_\Omega {\bf v}\cdot\nabla\delta p{\rm d}x = -\int_\Omega \delta p\nabla\cdot{\bf v}{\rm d}x + \int_\Gamma \delta p{\bf v}\cdot{\bf n}{\rm d}s.
    \end{align}
    Decomposing $J_{12}^{\epsilon,\gamma}$ into contributions from the boundary $\Gamma = \partial\Omega$ and from the interior of $\Omega$,
    \begin{align}
        J_{12}^{\epsilon,\gamma}({\bf u},p,\Omega) = \int_0^T\int_\Gamma J_\Gamma{\rm d}s{\rm d}t + \int_0^T\int_\Omega J_\Omega{\rm d}x{\rm d}t,
    \end{align}
    thus
    \begin{align}
        \partial_{\bf u}J_{12}^{\epsilon,\gamma}({\bf u},p,\Omega)\cdot\delta{\bf u} &= \int_0^T\int_\Gamma \partial_{\bf u}J_\Gamma\cdot\delta{\bf u}{\rm d}s{\rm d}t + \int_0^T\int_\Omega \partial_{\bf u}J_\Omega\cdot\delta{\bf u}{\rm d}x{\rm d}t,\\
        \partial_pJ_{12}^{\epsilon,\gamma}({\bf u},p,\Omega)\cdot\delta p &= \int_0^T\int_\Gamma \partial_pJ_\Gamma\cdot\delta p{\rm d}s{\rm d}t + \int_0^T\int_\Omega \partial_pJ_\Omega\cdot\delta p{\rm d}x{\rm d}t,
    \end{align}
    we can reformulate \eqref{1.4.2} as
    \begin{align}
        &\int_0^T\int_\Gamma ({\bf v}\cdot{\bf n} + \partial_pJ_\Gamma)\delta p{\rm d}s{\rm d}t + \int_0^T\int_\Omega (-\nabla\cdot{\bf v} + \partial_pJ_\Omega)\delta p{\rm d}x{\rm d}t\\
        &+ \int_0^T\int_\Gamma \left[({\bf v}\cdot{\bf u}){\bf n} + ({\bf u}\cdot{\bf n}){\bf v} + \nu{\bf n}\cdot\nabla{\bf v} - q{\bf n} + \partial_{\bf u}J_\Gamma\right]\cdot\delta{\bf u}{\rm d}s{\rm d}t\\
        & + \int_0^T\int_\Omega [-{\bf v}_t - \nu\Delta{\bf v} -\nabla{\bf v}\cdot{\bf u} - ({\bf u}\cdot\nabla){\bf v} + \nabla q + \partial_{\bf u}J_\Omega]\cdot\delta{\bf u}{\rm d}x{\rm d}t\\
        &- \int_0^T\int_\Gamma \nu{\bf n}\cdot\nabla\delta{\bf u}\cdot{\bf v}{\rm d}s{\rm d}t + \int_\Omega {\bf v}(T)\cdot\delta{\bf u}(T) - {\bf v}(0)\cdot\delta{\bf u}(0) + {\bf v}_0\cdot\delta{\bf u}(0){\rm d}x\\
        &+ \int_0^T\int_{\Gamma_{\rm in}} {\bf v}_{\rm in}\cdot\delta{\bf u}{\rm d}s{\rm d}t + \int_0^T\int_{\Gamma_{\rm wall}} {\bf v}_{\rm wall}\cdot\delta{\bf u}{\rm d}s{\rm d}t + \int_0^T\int_{\Gamma_{\rm out}} {\bf v}_{\rm out}\cdot(\delta p{\bf n} - \nu\partial_{\bf n}\delta{\bf u}){\rm d}s{\rm d}t = 0.\label{1.4.3}
    \end{align}
    Since this holds for any $\delta{\bf u}$ and $\delta p$ satisfying the primal NSEs, the integrals vanish individually. The vanishing of the integrals over the domain yields the adjoint NSEs (choose $\delta{\bf u}$ s.t. $\delta{\bf u}(0) = \delta{\bf u}(T) = {\bf 0}$)
    \begin{equation}
        \label{1.4.4}
        \left\{\begin{split}
            -{\bf v}_t - \nu\Delta{\bf v} - \nabla{\bf v}\cdot{\bf u} - ({\bf u}\cdot\nabla){\bf v} + \nabla q &=  - \partial_{\bf u}J_\Omega &&\mbox{ in } \Omega,\\
            \nabla\cdot{\bf v} &= \partial_pJ_\Omega &&\mbox{ in } \Omega.
        \end{split}\right.    
    \end{equation}
    Plugging explicit formulas of $J_\Omega$ yields
    \begin{align}
        J_\Omega({\bf u},p) &= \gamma k_\epsilon\left(p + \frac{1}{2}|{\bf u}|^2\right){\bf u}\cdot{\bf n},\\
        \partial_{\bf u}J_\Omega({\bf u},p) &= \gamma k_\epsilon\left(\left(p + \frac{1}{2}|{\bf u}|^2\right){\bf n} + ({\bf u}\cdot{\bf n}){\bf u}\right),\\
        \partial_pJ_\Omega({\bf u},p) &= \gamma k_\epsilon{\bf u}\cdot{\bf n}.
    \end{align}
    Then \eqref{1.4.4} becomes
    \begin{equation*}
        \left\{\begin{split}
            -{\bf v}_t - \nu\Delta{\bf v} - \nabla{\bf v}\cdot{\bf u} - ({\bf u}\cdot\nabla){\bf v} + \nabla q &= -\gamma k_\epsilon\left(\left(p + \frac{1}{2}|{\bf u}|^2\right){\bf n} + ({\bf u}\cdot{\bf n}){\bf u}\right) &&\mbox{ in } [0,T]\times\Omega,\\
            \nabla\cdot{\bf v} &= \gamma k_\epsilon{\bf u}\cdot{\bf n} &&\mbox{ in } [0,T]\times\Omega.
        \end{split}\right.    
    \end{equation*}
    The remaining terms yields
    \begin{align}
        &\int_0^T\int_\Gamma \left[({\bf v}\cdot{\bf u}){\bf n} + ({\bf u}\cdot{\bf n}){\bf v} + \nu{\bf n}\cdot\nabla{\bf v} - q{\bf n} + \partial_{\bf u}J_\Gamma\right]\cdot\delta{\bf u}{\rm d}s{\rm d}t - \nu\int_0^T\int_\Gamma {\bf n}\cdot\nabla\delta{\bf u}\cdot{\bf v}{\rm d}s{\rm d}t \nonumber\\
        &\hspace{1cm}+ \int_0^T\int_{\Gamma_{\rm in}} {\bf v}_{\rm in}\cdot\delta{\bf u}{\rm d}s{\rm d}t + \int_0^T\int_{\Gamma_{\rm wall}} {\bf v}_{\rm wall}\cdot\delta{\bf u}{\rm d}s{\rm d}t - \int_0^T\int_{\Gamma_{\rm out}} \nu{\bf v}_{\rm out}\cdot\partial_{\bf n}\delta{\bf u}{\rm d}s{\rm d}t = 0,\label{BC velocity}\\
        &\int_0^T\int_\Gamma ({\bf v}\cdot{\bf n} + \partial_pJ_\Gamma)\delta p{\rm d}s{\rm d}t + \int_0^T\int_{\Gamma_{\rm out}} {\bf v}_{\rm out}\cdot\delta p{\bf n}{\rm d}s{\rm d}t = 0,\label{BC pressure}\\
        &\int_\Omega {\bf v}(T)\cdot\delta{\bf u}(T) - {\bf v}(0)\cdot\delta{\bf u}(0) + {\bf v}_0\cdot\delta{\bf u}(0){\rm d}x = 0.\label{IC}
    \end{align}
    In \eqref{IC}, choosing $\delta{\bf u}$ s.t. $\delta{\bf u}(0) = {\bf 0}$ yields ${\bf v}(T) = 0$, and ${\bf v}_0 = {\bf v}(0)$. Plugging the explicit formula for $J_\Gamma({\bf u})$ gives
    \begin{align}
        J_\Gamma({\bf u}) &= \frac{1 - \gamma}{2}({\bf u}\cdot{\bf n} - \overline{u})^2,\\
        \partial_{\bf u}J_\Gamma({\bf u}) &= (1 - \gamma)({\bf u}\cdot{\bf n} - \overline{u}){\bf n},\\
        \partial_pJ_\Gamma({\bf u}) &= 0.
    \end{align}
    Use ${\bf v} = \delta{\bf u} = {\bf 0}$ on $\Gamma_{\rm in}\cup\Gamma_{\rm wall}$, \eqref{BC pressure} reduces to
    \begin{align}
        \int_0^T\int_{\Gamma_{\rm out}} ({\bf v}\cdot{\bf n} + {\bf v}_{\rm out}\cdot{\bf n})\delta p{\rm d}s{\rm d}t = 0,
    \end{align}
    and thus we can set ${\bf v}_{\rm out} = -{\bf v}$ on $\Gamma_{\rm out}$.
    
    Similarly, \eqref{BC velocity} reduces to
    \begin{align}
        \int_0^T\int_{\Gamma_{\rm out}} \left[({\bf v}\cdot{\bf u}){\bf n} + ({\bf u}\cdot{\bf n}){\bf v} + \nu{\bf n}\cdot\nabla{\bf v} - q{\bf n} + (1 - \gamma)({\bf u}\cdot{\bf n} - \bar{u}){\bf n}\right]\cdot\delta{\bf u} - \nu{\bf n}\cdot\nabla\delta{\bf u}\cdot{\bf v} - \nu{\bf v}_{\rm out}\cdot\partial_{\bf n}\delta{\bf u}{\rm d}s{\rm d}t = 0.
    \end{align}
    Note that the sum of the last 2 terms vanishes, the last equation reduces to
    \begin{align}
        \int_0^T\int_{\Gamma_{\rm out}} \left[({\bf v}\cdot{\bf u}){\bf n} + ({\bf u}\cdot{\bf n}){\bf v} + \nu{\bf n}\cdot\nabla{\bf v} - q{\bf n} + (1 - \gamma)({\bf u}\cdot{\bf n} - \bar{u}){\bf n}\right]\cdot\delta{\bf u}{\rm d}s{\rm d}t = 0.
    \end{align}
    Thus,
    \begin{align}
        ({\bf v}\cdot{\bf u}){\bf n} + ({\bf u}\cdot{\bf n}){\bf v} + \nu{\bf n}\cdot\nabla{\bf v} - q{\bf n} = -(1 - \gamma)({\bf u}\cdot{\bf n} - \bar{u}){\bf n} \mbox{ on } [0,T]\times\Gamma_{\rm out}.
    \end{align}
    We obtain the desired adjoint equation.
\end{proof}

\section{A general framework}
\textbf{General Outline.} In this part: consider a PDEs/mathematical models (e.g., Navier-Stokes equations, Smagorinsky turbulence models, $k$-$\epsilon$ turbulence models, etc.) with the following outline:
\begin{enumerate}
    \item A PDEs\texttt{/}mathematical models
    \item Weak formulation of the considered PDEs\texttt{/}mathematical models
    \item Adjoint equations of the considered PDEs\texttt{/}mathematical models
    \item Weak formulation of the adjoint equations of the considered PDEs/mathematical models
    \item Shape derivatives for the considered PDEs\texttt{/}mathematical models (with boundary conditions plugged in and other add-ons, e.g., wall laws)
    \begin{enumerate}
        \item 1st-order shape derivatives for the considered PDEs\texttt{/}mathematical models
        \item 2nd-order shape derivatives for the considered PDEs\texttt{/}mathematical models [hard, added later]
    \end{enumerate}
    \item Shape optimization for the considered PDEs\texttt{/}mathematical models
\end{enumerate}

In this chapter, we will design a framework for shape optimization for the PDEs of the following form (with assumption that the density $\rho = \mbox{const}$):
\begin{equation}
    \label{general stationary fluid dynamics PDEs}
    \tag{gfld}
    \left\{\begin{split}
        {\bf P}({\bf x},{\bf u},\nabla{\bf u},\Delta{\bf u},p,\nabla p) &= {\bf f}({\bf x},{\bf u},\nabla{\bf u},p) &&\mbox{ in } \Omega,\\
        -\nabla\cdot{\bf u} &= f_{\rm div}({\bf x},{\bf u},\nabla{\bf u},p) &&\mbox{ in } \Omega,\\
        {\bf Q}({\bf x},{\bf u},\nabla{\bf u},p,{\bf n},{\bf t}) &= {\bf f}_{\rm bc}({\bf x}) &&\mbox{ on } \Gamma,
    \end{split}\right.
\end{equation}
where ${\bf P}(\cdot,\ldots,\cdot) = (P_1,\ldots,P_N)(\cdot,\ldots,\cdot)$, ${\bf f}(\cdot,\ldots,\cdot) = (f_1,\ldots,f_N)(\cdot,\ldots,\cdot)$ and ${\bf Q}(\cdot,\ldots,\cdot)$ denote the main PDE (e.g., here, NSEs), the source terms, and the set of boundary conditions, respectively.

\subsection{Weak formulation of \eqref{general stationary fluid dynamics PDEs}}
Test both sides of the 1st equation of \eqref{general stationary fluid dynamics PDEs} with a test function ${\bf v}$ and those of the 2nd one with a test function $q$ over $\Omega$:
\begin{equation}
    \label{weak formulation of general stationary fluid dynamics PDEs}
    \tag{wf-gfld}
    \left\{\begin{split}
        \int_\Omega {\bf P}({\bf x},{\bf u},\nabla{\bf u},\Delta{\bf u},p,\nabla p)\cdot{\bf v}{\rm d}{\bf x} &= \int_\Omega {\bf f}({\bf x},{\bf u},\nabla{\bf u},p)\cdot{\bf v}{\rm d}{\bf x},\\
        -\int_\Omega q\nabla\cdot{\bf u}{\rm d}{\bf x} &= \int_\Omega qf_{\rm div}({\bf x},{\bf u},\nabla{\bf u},p){\rm d}{\bf x},
    \end{split}\right.
\end{equation}
and then integrate by parts all the 2nd-order terms w.r.t. ${\bf u}$ and all the 1st-order terms w.r.t. $p$ in the 1st equation of \eqref{weak formulation of general stationary fluid dynamics PDEs}.

Plugging the boundary conditions (i.e., the 3rd equation of \eqref{general stationary fluid dynamics PDEs}) into the equations just obtained to embed them into the weak formulation.

\subsection{Cost functionals}
A general cost functional for \eqref{general stationary fluid dynamics PDEs} is given by
\begin{align}
    \label{cost functional of general stationary fluid dynamics PDEs}
    \tag{cost-gfld}
    J({\bf u},p,\Omega)\coloneqq\int_\Omega J_\Omega({\bf x},{\bf u},\nabla{\bf u},p){\rm d}{\bf x} + \int_\Gamma J_\Gamma({\bf x},{\bf u},\nabla{\bf u},p,{\bf n},{\bf t}){\rm d}\Gamma.
\end{align}

\subsection{Lagrangian \& extended Lagrangian}
To derive the adjoint equations for \eqref{general stationary fluid dynamics PDEs}, 1st introduce the following Lagrangian (see, e.g., \cite{Troltzsch2010}):
\begin{align}
    \label{Lagrangian for general stationary fluid dynamics PDEs}
    \tag{$L$-gfld}
    L({\bf u},p,\Omega,{\bf v},q)\coloneqq J({\bf u},p,\Omega) + \int_\Omega -\left({\bf P}({\bf x},{\bf u},\nabla{\bf u},\Delta{\bf u},p,\nabla p) - {\bf f}({\bf x},{\bf u},\nabla{\bf u},p)\right)\cdot{\bf v} + q\left(\nabla\cdot{\bf u} + f_{\rm div}({\bf x},{\bf u},\nabla{\bf u},p)\right){\rm d}{\bf x},
\end{align}
and the following extended Lagrangian:
\begin{align}
    \label{extended Lagrangian for general stationary fluid dynamics PDEs}
    \tag{$\mathcal{L}$-gfld}
    \mathcal{L}({\bf u},p,\Omega,{\bf v},q,{\bf v}_{\rm bc})\coloneqq&\, L({\bf u},p,\Omega,{\bf v},q)
    - \int_\Gamma \left({\bf Q}({\bf x},{\bf u},\nabla{\bf u},p,{\bf n},{\bf t}) - {\bf f}_{\rm bc}({\bf x})\right)\cdot{\bf v}_{\rm bc}{\rm d}\Gamma\\
    =&\, J({\bf u},p,\Omega) + \int_\Omega -\left({\bf P}({\bf x},{\bf u},\nabla{\bf u},\Delta{\bf u},p,\nabla p) - {\bf f}({\bf x},{\bf u},\nabla{\bf u},p)\right)\cdot{\bf v} + q\left(\nabla\cdot{\bf u} + f_{\rm div}({\bf x},{\bf u},\nabla{\bf u},p)\right){\rm d}{\bf x}\nonumber\\
    &- \int_\Gamma \left({\bf Q}({\bf x},{\bf u},\nabla{\bf u},p,{\bf n},{\bf t}) - {\bf f}_{\rm bc}({\bf x})\right)\cdot{\bf v}_{\rm bc}{\rm d}\Gamma,\nonumber
\end{align}
where ${\bf v}$, $q$, ${\bf v}_{\rm bc}$ are Lagrange multipliers.

\subsection{Shape optimization problems}
Here are 3 different shape optimization problems associated with \eqref{cost functional of general stationary fluid dynamics PDEs}, \eqref{Lagrangian for general stationary fluid dynamics PDEs}, and \eqref{extended Lagrangian for general stationary fluid dynamics PDEs}, respectively:
\begin{align*}
    &\min_{\Omega\in\mathcal{O}_{\rm ad}} J({\bf u},p,\Omega) \mbox{ s.t. } ({\bf u},p) \mbox{ solves \eqref{general stationary fluid dynamics PDEs}},\\
    &\min_{\Omega\in\mathcal{O}_{\rm ad}} L({\bf u},p,\Omega,{\bf v},q) \mbox{ s.t. } ({\bf u},p) \mbox{ satisfies } {\bf Q}({\bf x},{\bf u},\nabla{\bf u},p,{\bf n},{\bf t}) = {\bf f}_{\rm bc}({\bf x}) \mbox{ on } \Gamma,\\
    &\min_{\Omega\in\mathcal{O}_{\rm ad}} \mathcal{L}({\bf u},p,\Omega,{\bf v},q,{\bf v}_{\rm bc}) \mbox{ with } ({\bf u},p) \mbox{ unconstrained}.
\end{align*}
In the 2nd optimization problem, the main PDEs (i.e., equations in $\Omega$) are already penalized by implicitly embedded into the Lagrangian \eqref{Lagrangian for general stationary fluid dynamics PDEs}, meanwhile in the 3rd one, both main PDEs and boundary conditions are penalized by implicitly embedded into the extended Lagrangian \eqref{extended Lagrangian for general stationary fluid dynamics PDEs}.
\begin{remark}
    Remind that this is just a formal framework. For a rigorous one, the correct function spaces of both state and adjoint variables need inserting into these optimization problems.
\end{remark}
Here $({\bf u},p)$, $({\bf v},q)$ (also ${\bf v}_{\rm bc}$ for the last shape optimization problem with the extended Lagrangian), and $\Omega$ are the \textit{state variables, adjoint variables}, and \textit{control variable} for these shape optimization problems.

\subsection{Adjoint equations of \eqref{general stationary fluid dynamics PDEs}}
A natural question arises:
\begin{question}
    Should the Lagrangian or the extended Lagrangian be used to derive the adjoint equations?
\end{question}
To answer this question, introduce the following ``mixed'' Lagrangian:
\begin{align}
    \label{mixed Lagrangian for general stationary fluid dynamics PDEs}
    \tag{$L_{\mathcal{L}}$-gfld}
    L_{\mathcal{L}}({\bf u},p,\Omega,{\bf v},q,{\bf v}_{\rm bc})\coloneqq&\, L({\bf u},p,\Omega,{\bf v},q)
    - \delta_{\mathcal{L}}\int_\Gamma \left({\bf Q}({\bf x},{\bf u},\nabla{\bf u},p,{\bf n},{\bf t}) - {\bf f}_{\rm bc}({\bf x})\right)\cdot{\bf v}_{\rm bc}{\rm d}\Gamma\\
    =&\, J({\bf u},p,\Omega) + \int_\Omega -\left({\bf P}({\bf x},{\bf u},\nabla{\bf u},\Delta{\bf u},p,\nabla p) - {\bf f}({\bf x},{\bf u},\nabla{\bf u},p)\right)\cdot{\bf v} + q\left(\nabla\cdot{\bf u} + f_{\rm div}({\bf x},{\bf u},\nabla{\bf u},p)\right){\rm d}{\bf x}\nonumber\\
    &- \delta_{\mathcal{L}}\int_\Gamma \left({\bf Q}({\bf x},{\bf u},\nabla{\bf u},p,{\bf n},{\bf t}) - {\bf f}_{\rm bc}({\bf x})\right)\cdot{\bf v}_{\rm bc}{\rm d}\Gamma, \mbox{ where } \delta_{\mathcal{L}}\in\{0,1\}\nonumber.
\end{align}

\begin{remark}
    No need to assume the ``switch'' between Lagrangian and extended Lagrangian is a real number, i.e. $\delta_{\mathcal{L}}\in\mathbb{R}$, since the scaling is already embedded in the Lagrange multiplier ${\bf v}_{\rm bc}$.
\end{remark}
Hence,
\begin{equation*}
    L_{\mathcal{L}}({\bf u},p,\Omega,{\bf v},q,{\bf v}_{\rm bc}) = \left\{\begin{split}
        &L({\bf u},p,\Omega,{\bf v},q) &&\mbox{ if } \delta_{\mathcal{L}} = 0,\\
        &\mathcal{L}({\bf u},p,\Omega,{\bf v},q,{\bf v}_{\rm bc}) &&\mbox{ if } \delta_{\mathcal{L}} = 1.
    \end{split}\right.
\end{equation*}
Then the shape optimization problem reads
\begin{equation*}
    \min_{\Omega\in\mathcal{O}_{\rm ad}} L_{\mathcal{L}}({\bf u},p,\Omega,{\bf v},q,{\bf v}_{\rm bc})\ \left\{\begin{split}
        &\mbox{s.t. } ({\bf u},p) \mbox{ satisfies } {\bf Q}({\bf x},{\bf u},\nabla{\bf u},p,{\bf n},{\bf t}) = {\bf f}_{\rm bc}({\bf x}) \mbox{ on } \Gamma &&\mbox{ if } \delta_{\mathcal{L}} = 0,\\
        &\mbox{with } ({\bf u},p) \mbox{ unconstrained} &&\mbox{ if } \delta_{\mathcal{L}} = 1.
    \end{split}\right.
\end{equation*}
Next, choose the Lagrangian multiplier $({\bf v},q,{\bf v}_{\rm bc})$ s.t. the variation of the chosen Lagrangian (simple/extended/mixed) w.r.t. state variables vanishes, i.e.:
\begin{align*}
    \delta_{({\bf u},p)} L_{\mathcal{L}}({\bf u},p,\Omega,{\bf v},q,{\bf v}_{\rm bc};\tilde{\bf u},\tilde{p}) = 0,\ \forall(\tilde{\bf u},\tilde{p}),
\end{align*}
where
\begin{align*}
    &\delta_{({\bf u},p)} L_{\mathcal{L}}({\bf u},p,\Omega,{\bf v},q,{\bf v}_{\rm bc};\tilde{\bf u},\tilde{p})\\
    \coloneqq&\, \lim_{t\downarrow 0} \frac{1}{t}\left(L_{\mathcal{L}}({\bf u} + t\tilde{\bf u},p + t\tilde{p},\Omega,{\bf v},q,{\bf v}_{\rm bc}) - L_{\mathcal{L}}({\bf u},p,\Omega,{\bf v},q,{\bf v}_{\rm bc})\right)\\
    =&\, \lim_{t\downarrow 0} \frac{1}{t}\left(L_{\mathcal{L}}({\bf u} + t\tilde{\bf u},p + t\tilde{p},\Omega,{\bf v},q,{\bf v}_{\rm bc}) - L_{\mathcal{L}}({\bf u},p + t\tilde{p},\Omega,{\bf v},q,{\bf v}_{\rm bc})\right) + \lim_{t\downarrow 0} \frac{1}{t}\left(L_{\mathcal{L}}({\bf u},p + t\tilde{p},\Omega,{\bf v},q,{\bf v}_{\rm bc}) - L_{\mathcal{L}}({\bf u},p,\Omega,{\bf v},q,{\bf v}_{\rm bc})\right)\\
    =&\, \lim_{t\downarrow 0} \delta_{\bf u} L_{\mathcal{L}}({\bf u},p + t\tilde{p},\Omega,{\bf v},q,{\bf v}_{\rm bc};\tilde{\bf u}) + \delta_p L_{\mathcal{L}}({\bf u},p,\Omega,{\bf v},q,{\bf v}_{\rm bc};\tilde{p}) = \delta_{\bf u} L_{\mathcal{L}}({\bf u},p,\Omega,{\bf v},q,{\bf v}_{\rm bc};\tilde{\bf u}) + \delta_p L_{\mathcal{L}}({\bf u},p,\Omega,{\bf v},q,{\bf v}_{\rm bc};\tilde{p}).
\end{align*}
Provided Gateaux/Fr\'echet differentiability of $L_{\mathcal{L}}$ is guaranteed, in the case $\delta_{\mathcal{L}} = 1$, since the state variables $({\bf u},p)$ is now formally unconstrained \texttt{[added corrected function spaces for its rigorous counterpart]}, the derivative of the Lagrangian w.r.t. $({\bf u},p)$ has to vanish at any optimal point, say e.g. $({\bf u}^\star,p^\star,\Omega^\star)$, i.e.,
\begin{align*}
    D_{({\bf u},p)} L_{\mathcal{L}}({\bf u}^\star,p^\star,\Omega^\star,{\bf v},q,{\bf v}_{\rm bc})(\tilde{\bf u},\tilde{p}) = 0,\ \forall(\tilde{\bf u},\tilde{p}).
\end{align*}
Combine this with
\begin{align*}
    D_{({\bf u},p)} L_{\mathcal{L}}({\bf u},p,\Omega,{\bf v},q,{\bf v}_{\rm bc})(\tilde{\bf u},\tilde{p}) = D_{\bf u}L_{\mathcal{L}} ({\bf u},p,\Omega,{\bf v},q,{\bf v}_{\rm bc})\tilde{\bf u} + D_pL_{\mathcal{L}} ({\bf u},p,\Omega,{\bf v},q,{\bf v}_{\rm bc})\tilde{p},\ \forall(\tilde{\bf u},\tilde{p}),
\end{align*}
then obtain
\begin{align*}
    D_{\bf u}L_{\mathcal{L}} ({\bf u}^\star,p^\star,\Omega^\star,{\bf v},q,{\bf v}_{\rm bc})\tilde{\bf u} + D_pL_{\mathcal{L}} ({\bf u}^\star,p^\star,\Omega^\star,{\bf v},q,{\bf v}_{\rm bc})\tilde{p} = 0,\ \forall(\tilde{\bf u},\tilde{p}).
\end{align*}
Motivated by this stationary equation, choose the adjoint variables/Lagrange multipliers $({\bf v},q,{\bf v}_{\rm bc})$ s.t. 
\begin{align*}
    \boxed{D_{\bf u} L_{\mathcal{L}}({\bf u},p,\Omega,{\bf v},q,{\bf v}_{\rm bc})\tilde{\bf u} + D_pL_{\mathcal{L}}({\bf u},p,\Omega,{\bf v},q,{\bf v}_{\rm bc})\tilde{p} = 0,\ \forall({\bf u},p,\Omega,\tilde{\bf u},\tilde{p}).}
\end{align*}
Expand this more explicitly for all $({\bf u},p,\Omega,\tilde{\bf u},\tilde{p})$:
\begin{align*}
    &\int_\Omega D_{\bf u}\left(J_\Omega({\bf x},{\bf u},\nabla{\bf u},p)\right)\tilde{\bf u} + D_p\left(J_\Omega({\bf x},{\bf u},\nabla{\bf u},p)\right)\tilde{p}{\rm d}{\bf x} + \int_\Gamma D_{\bf u}\left(J_\Gamma({\bf x},{\bf u},\nabla{\bf u},p,{\bf n},{\bf t})\right)\tilde{\bf u} + D_p\left(J_\Gamma({\bf x},{\bf u},\nabla{\bf u},p,{\bf n},{\bf t})\right)\tilde{p}{\rm d}\Gamma\\
    &+ \int_\Omega -D_{\bf u}\left({\bf P}({\bf x},{\bf u},\nabla{\bf u},\Delta{\bf u},p,\nabla p) - {\bf f}({\bf x},{\bf u},\nabla{\bf u},p)\right)\tilde{\bf u}\cdot{\bf v} - D_p\left({\bf P}({\bf x},{\bf u},\nabla{\bf u},\Delta{\bf u},p,\nabla p) - {\bf f}({\bf x},{\bf u},\nabla{\bf u},p)\right)\tilde{p}\cdot{\bf v}\\
    &\hspace{1cm} + qD_{\bf u}\left(\nabla\cdot{\bf u} + f_{\rm div}({\bf x},{\bf u},\nabla{\bf u},p)\right)\tilde{\bf u} + qD_p\left(\nabla\cdot{\bf u} + f_{\rm div}({\bf x},{\bf u},\nabla{\bf u},p)\right)\tilde{p}{\rm d}{\bf x}\\
    &- \delta_{\mathcal{L}}\int_\Gamma D_{\bf u}\left({\bf Q}({\bf x},{\bf u},\nabla{\bf u},p,{\bf n},{\bf t}) - {\bf f}_{\rm bc}({\bf x})\right)\tilde{\bf u}\cdot{\bf v}_{\rm bc} + D_p\left({\bf Q}({\bf x},{\bf u},\nabla{\bf u},p,{\bf n},{\bf t}) - {\bf f}_{\rm bc}({\bf x})\right)\tilde{p}\cdot{\bf v}_{\rm bc}{\rm d}\Gamma,
\end{align*}
and more explicitly:
\begin{align*}
    &\int_\Omega D_{\bf u}J_\Omega({\bf x},{\bf u},\nabla{\bf u},p)\tilde{\bf u} + D_{\nabla{\bf u}}J_\Omega({\bf x},{\bf u},\nabla{\bf u},p)\nabla\tilde{\bf u} + D_pJ_\Omega({\bf x},{\bf u},\nabla{\bf u},p)\tilde{p}{\rm d}{\bf x}\\
    &+ \int_\Gamma D_{\bf u}J_\Gamma({\bf x},{\bf u},\nabla{\bf u},p,{\bf n},{\bf t})\tilde{\bf u} + D_{\nabla{\bf u}}J_\Gamma({\bf x},{\bf u},\nabla{\bf u},p,{\bf n},{\bf t})\nabla\tilde{\bf u} + D_pJ_\Gamma({\bf x},{\bf u},\nabla{\bf u},p,{\bf n},{\bf t})\tilde{p}{\rm d}\Gamma\\
    &+ \int_\Omega -D_{\bf u}{\bf P}({\bf x},{\bf u},\nabla{\bf u},\Delta{\bf u},p,\nabla p)\tilde{\bf u}\cdot{\bf v} - D_{\nabla{\bf u}}{\bf P}({\bf x},{\bf u},\nabla{\bf u},\Delta{\bf u},p,\nabla p)\nabla\tilde{\bf u}\cdot{\bf v} - D_{\Delta{\bf u}}{\bf P}({\bf x},{\bf u},\nabla{\bf u},\Delta{\bf u},p,\nabla p)\Delta\tilde{\bf u}\cdot{\bf v}\\
    &\hspace{1cm} + D_{\bf u}{\bf f}({\bf x},{\bf u},\nabla{\bf u},p)\tilde{\bf u}\cdot{\bf v} + D_{\nabla{\bf u}}{\bf f}({\bf x},{\bf u},\nabla{\bf u},p)\nabla\tilde{\bf u}\cdot{\bf v}\\
    &\hspace{1cm} - D_p{\bf P}({\bf x},{\bf u},\nabla{\bf u},\Delta{\bf u},p,\nabla p)\tilde{p}\cdot{\bf v} - D_{\nabla p}{\bf P}({\bf x},{\bf u},\nabla{\bf u},\Delta{\bf u},p,\nabla p)\nabla\tilde{p}\cdot{\bf v} + D_p{\bf f}({\bf x},{\bf u},\nabla{\bf u},p)\tilde{p}\cdot{\bf v}\\
    &\hspace{1cm} + q\nabla\cdot\tilde{\bf u} + qD_{\bf u}f_{\rm div}({\bf x},{\bf u},\nabla{\bf u},p)\tilde{\bf u} + qD_{\nabla{\bf u}}f_{\rm div}({\bf x},{\bf u},\nabla{\bf u},p)\nabla\tilde{\bf u} + qD_pf_{\rm div}({\bf x},{\bf u},\nabla{\bf u},p)\tilde{p}{\rm d}{\bf x}\\
    &- \delta_{\mathcal{L}}\int_\Gamma D_{\bf u}{\bf Q}({\bf x},{\bf u},\nabla{\bf u},p,{\bf n},{\bf t})\tilde{\bf u}\cdot{\bf v}_{\rm bc} + D_{\nabla{\bf u}}{\bf Q}({\bf x},{\bf u},\nabla{\bf u},p,{\bf n},{\bf t})\nabla\tilde{\bf u}\cdot{\bf v}_{\rm bc} + D_p{\bf Q}({\bf x},{\bf u},\nabla{\bf u},p,{\bf n},{\bf t})\tilde{p}\cdot{\bf v}_{\rm bc}{\rm d}\Gamma = 0,\ \forall({\bf u},p,\Omega,\tilde{\bf u},\tilde{p}).
\end{align*}

\begin{question}
    Integrate by parts which terms?
    
    \textsc{Answer.} All the terms which are the domain integrals containing derivatives of the variations of state variables to transfer their derivatives to the adjoint variables. Roughly speaking:
    \begin{align*}
        \int_\Omega \left\{\nabla\tilde{\bf u},\Delta\tilde{\bf u},\nabla\tilde{p}\right\}\cdot\left\{{\bf v},q\right\}{\rm d}{\bf x}\xrightarrow{\rm i.b.p.}\int_\Omega \left\{\tilde{\bf u},\tilde{p}\right\}\cdot\left\{\nabla{\bf v},\Delta{\bf v},\nabla q\right\}{\rm d}{\bf x} + \mbox{ boundary integrals } \int_\Gamma \cdots{\rm d}\Gamma.
    \end{align*}
    The domain integrals on the r.h.s., after gathered appropriately, will yield the adjoint PDEs, meanwhile the boundary integrals, also after gathered appropriately, will yield the adjoint boundary conditions.
\end{question}
\begin{align*}
    &\int_\Omega D_{\bf u}J_\Omega({\bf x},{\bf u},\nabla{\bf u},p)\tilde{\bf u} + {\color{red}D_{\nabla{\bf u}}J_\Omega({\bf x},{\bf u},\nabla{\bf u},p)\nabla\tilde{\bf u}} + D_pJ_\Omega({\bf x},{\bf u},\nabla{\bf u},p)\tilde{p}{\rm d}{\bf x}\\
    &+ \int_\Gamma D_{\bf u}J_\Gamma({\bf x},{\bf u},\nabla{\bf u},p,{\bf n},{\bf t})\tilde{\bf u} + D_{\nabla{\bf u}}J_\Gamma({\bf x},{\bf u},\nabla{\bf u},p,{\bf n},{\bf t})\nabla\tilde{\bf u} + D_pJ_\Gamma({\bf x},{\bf u},\nabla{\bf u},p,{\bf n},{\bf t})\tilde{p}{\rm d}\Gamma\\
    &+ \int_\Omega -D_{\bf u}{\bf P}({\bf x},{\bf u},\nabla{\bf u},\Delta{\bf u},p,\nabla p)\tilde{\bf u}\cdot{\bf v} - {\color{red}D_{\nabla{\bf u}}{\bf P}({\bf x},{\bf u},\nabla{\bf u},\Delta{\bf u},p,\nabla p)\nabla\tilde{\bf u}\cdot{\bf v}} -  {\color{red}D_{\Delta{\bf u}}{\bf P}({\bf x},{\bf u},\nabla{\bf u},\Delta{\bf u},p,\nabla p)\Delta\tilde{\bf u}\cdot{\bf v}}\\
    &\hspace{1cm} + D_{\bf u}{\bf f}({\bf x},{\bf u},\nabla{\bf u},p)\tilde{\bf u}\cdot{\bf v} +  {\color{red}D_{\nabla{\bf u}}{\bf f}({\bf x},{\bf u},\nabla{\bf u},p)\nabla\tilde{\bf u}\cdot{\bf v}}\\
    &\hspace{1cm} - D_p{\bf P}({\bf x},{\bf u},\nabla{\bf u},\Delta{\bf u},p,\nabla p)\tilde{p}\cdot{\bf v} -  {\color{red}D_{\nabla p}{\bf P}({\bf x},{\bf u},\nabla{\bf u},\Delta{\bf u},p,\nabla p)\nabla\tilde{p}\cdot{\bf v}} + D_p{\bf f}({\bf x},{\bf u},\nabla{\bf u},p)\tilde{p}\cdot{\bf v}\\
    &\hspace{1cm} +  {\color{red}q\nabla\cdot\tilde{\bf u}} + qD_{\bf u}f_{\rm div}({\bf x},{\bf u},\nabla{\bf u},p)\tilde{\bf u} +  {\color{red}qD_{\nabla{\bf u}}f_{\rm div}({\bf x},{\bf u},\nabla{\bf u},p)\nabla\tilde{\bf u}} + qD_pf_{\rm div}({\bf x},{\bf u},\nabla{\bf u},p)\tilde{p}{\rm d}{\bf x}\\
    &- \delta_{\mathcal{L}}\int_\Gamma D_{\bf u}{\bf Q}({\bf x},{\bf u},\nabla{\bf u},p,{\bf n},{\bf t})\tilde{\bf u}\cdot{\bf v}_{\rm bc} + D_{\nabla{\bf u}}{\bf Q}({\bf x},{\bf u},\nabla{\bf u},p,{\bf n},{\bf t})\nabla\tilde{\bf u}\cdot{\bf v}_{\rm bc} + D_p{\bf Q}({\bf x},{\bf u},\nabla{\bf u},p,{\bf n},{\bf t})\tilde{p}\cdot{\bf v}_{\rm bc}{\rm d}\Gamma = 0,\ \forall({\bf u},p,\Omega,\tilde{\bf u},\tilde{p}).
\end{align*}
Integrate by parts:
\begin{enumerate}[leftmargin=0in]
    \item Term $D_{\nabla{\bf u}}J_\Omega({\bf x},{\bf u},\nabla{\bf u},p)\nabla\tilde{\bf u}$:
    \begin{align*}
        &\int_\Omega D_{\nabla{\bf u}} J_\Omega({\bf x},{\bf u},\nabla{\bf u},p)\nabla\tilde{\bf u}{\rm d}{\bf x} = \int_\Omega \nabla_{\nabla{\bf u}}J_\Omega({\bf x},{\bf u},\nabla{\bf u},p):\nabla\tilde{\bf u}{\rm d}{\bf x} = \int_\Omega \sum_{i=1}^N\sum_{j=1}^N \partial_{\partial_{x_i}u_j}J_\Omega({\bf x},{\bf u},\nabla{\bf u},p)\partial_{x_i}\tilde{u}_j{\rm d}{\bf x}\\
        =&\, \sum_{i=1}^N\sum_{j=1}^N \int_\Omega \partial_{\partial_{x_i}u_j}J_\Omega({\bf x},{\bf u},\nabla{\bf u},p)\partial_{x_i}\tilde{u}_j{\rm d}{\bf x} = \sum_{i=1}^N\sum_{j=1}^N -\int_\Omega \partial_{x_i}\partial_{\partial_{x_i}u_j}J_\Omega({\bf x},{\bf u},\nabla{\bf u},p)\tilde{u}_j{\rm d}{\bf x} + \int_\Gamma n_i\partial_{\partial_{x_i}u_j}J_\Omega({\bf x},{\bf u},\nabla{\bf u},p)\tilde{u}_j{\rm d}\Gamma\\
        =&\, -\int_\Omega \sum_{j=1}^N \tilde{u}_j\sum_{i=1}^N \partial_{x_i}\partial_{\partial_{x_i}u_j}J_\Omega({\bf x},{\bf u},\nabla{\bf u},p){\rm d}{\bf x} + \int_\Gamma \sum_{i=1}^N\sum_{j=1}^N n_i\partial_{\partial_{x_i}u_j}J_\Omega({\bf x},{\bf u},\nabla{\bf u},p)\tilde{u}_j{\rm d}\Gamma\\
        =&\, -\int_\Omega \sum_{j=1}^N \tilde{u}_j\nabla\cdot\left(\nabla_{\nabla u_j}J_\Omega({\bf x},{\bf u},\nabla{\bf u},p)\right){\rm d}{\bf x} + \int_\Gamma {\bf n}^\top\nabla_{\nabla{\bf u}}J_\Omega({\bf x},{\bf u},\nabla{\bf u},p)\tilde{\bf u}{\rm d}\Gamma\\
        =&\, -\int_\Omega \nabla\cdot\left(\nabla_{\nabla{\bf u}}J_\Omega({\bf x},{\bf u},\nabla{\bf u},p)\right)\cdot\tilde{\bf u}{\rm d}{\bf x} + \int_\Gamma {\bf n}^\top\nabla_{\nabla{\bf u}}J_\Omega({\bf x},{\bf u},\nabla{\bf u},p)\tilde{\bf u}{\rm d}\Gamma,
    \end{align*}
    where $\nabla_{\nabla{\bf u}} f(\nabla{\bf u})\coloneqq\left(\partial_{\partial_{x_i}u_j}f(\nabla{\bf u})\right)_{i,j=1}^N$ for any scalar function $f$.
    \item Term $- D_{\nabla{\bf u}}{\bf P}({\bf x},{\bf u},\nabla{\bf u},\Delta{\bf u},p,\nabla p)\nabla\tilde{\bf u}\cdot{\bf v}$:
    \begin{align*}
        &-\int_\Omega D_{\nabla{\bf u}}{\bf P}({\bf x},{\bf u},\nabla{\bf u},\Delta{\bf u},p,\nabla p)\nabla\tilde{\bf u}\cdot{\bf v}{\rm d}{\bf x} = -\int_\Omega \left(\nabla_{\nabla{\bf u}}P_k({\bf x},{\bf u},\nabla{\bf u},\Delta{\bf u},p,\nabla p):\nabla\tilde{\bf u}\right)_{k=1}^N\cdot{\bf v}{\rm d}{\bf x}\\
        =&\, -\int_\Omega \sum_{k=1}^N \nabla_{\nabla{\bf u}}P_k({\bf x},{\bf u},\nabla{\bf u},\Delta{\bf u},p,\nabla p):\nabla\tilde{\bf u}v_k{\rm d}{\bf x} = -\int_\Omega \sum_{k=1}^N\sum_{i=1}^N\sum_{j=1}^N \partial_{\partial_{x_i}u_j}P_k({\bf x},{\bf u},\nabla{\bf u},\Delta{\bf u},p,\nabla p)\partial_{x_i}\tilde{u}_jv_k{\rm d}{\bf x}\\
        =&\, -\sum_{i=1}^N\sum_{j=1}^N\sum_{k=1}^N \int_\Omega \partial_{\partial_{x_i}u_j}P_k({\bf x},{\bf u},\nabla{\bf u},\Delta{\bf u},p,\nabla p)\partial_{x_i}\tilde{u}_jv_k{\rm d}{\bf x}\\
        =&\, \sum_{i=1}^N\sum_{j=1}^N\sum_{k=1}^N \int_\Omega \partial_{x_i}\partial_{\partial_{x_i}u_j}P_k({\bf x},{\bf u},\nabla{\bf u},\Delta{\bf u},p,\nabla p)\tilde{u}_jv_k + \partial_{\partial_{x_i}u_j}P_k({\bf x},{\bf u},\nabla{\bf u},\Delta{\bf u},p,\nabla p)\tilde{u}_j\partial_{x_i}v_k{\rm d}{\bf x}\\
        &\hspace{15mm} -\int_\Gamma n_i\partial_{\partial_{x_i}u_j}P_k({\bf x},{\bf u},\nabla{\bf u},\Delta{\bf u},p,\nabla p)\tilde{u}_jv_k{\rm d}\Gamma\\
        =&\, \int_\Omega \sum_{j=1}^N \tilde{u}_j\sum_{k=1}^N v_k\sum_{i=1}^N \partial_{x_i}\partial_{\partial_{x_i}u_j}P_k({\bf x},{\bf u},\nabla{\bf u},\Delta{\bf u},p,\nabla p) + \sum_{j=1}^N \tilde{u}_j\sum_{k=1}^N\sum_{i=1}^N \partial_{\partial_{x_i}u_j}P_k({\bf x},{\bf u},\nabla{\bf u},\Delta{\bf u},p,\nabla p)\partial_{x_i}v_k{\rm d}{\bf x}\\
        & -\int_\Gamma \sum_{j=1}^N \tilde{u}_j\sum_{k=1}^N v_k\sum_{i=1}^N n_i\partial_{\partial_{x_i}u_j}P_k({\bf x},{\bf u},\nabla{\bf u},\Delta{\bf u},p,\nabla p){\rm d}\Gamma\\
        =&\, \int_\Omega \sum_{j=1}^N \tilde{u}_j\sum_{k=1}^N v_k\nabla\cdot\left(\nabla_{\nabla u_j}P_k({\bf x},{\bf u},\nabla{\bf u},\Delta{\bf u},p,\nabla p)\right) + \sum_{j=1}^N \tilde{u}_j\sum_{k=1}^N \nabla_{\nabla u_j}P_k({\bf x},{\bf u},\nabla{\bf u},\Delta{\bf u},p,\nabla p)\cdot\nabla v_k{\rm d}{\bf x}\\
        & -\int_\Gamma \sum_{j=1}^N \tilde{u}_j\sum_{k=1}^N v_k\nabla_{\nabla u_j}P_k({\bf x},{\bf u},\nabla{\bf u},\Delta{\bf u},p,\nabla p)\cdot{\bf n}{\rm d}\Gamma\\
        =&\, \int_\Omega \sum_{j=1}^N \tilde{u}_j\nabla\cdot\left(\nabla_{\nabla u_j}{\bf P}({\bf x},{\bf u},\nabla{\bf u},\Delta{\bf u},p,\nabla p)\right)\cdot{\bf v} + \sum_{j=1}^N \tilde{u}_j\nabla_{\nabla u_j}{\bf P}({\bf x},{\bf u},\nabla{\bf u},\Delta{\bf u},p,\nabla p):\nabla{\bf v}{\rm d}{\bf x}\\
        &- \int_\Gamma \sum_{j=1}^N \tilde{u}_j\left(\nabla_{\nabla u_j}{\bf P}({\bf x},{\bf u},\nabla{\bf u},\Delta{\bf u},p,\nabla p)\cdot{\bf n}\right)\cdot{\bf v}{\rm d}\Gamma\\
        =&\, \int_\Omega \left(\nabla\cdot\left(\nabla_{\nabla{\bf u}}{\bf P}({\bf x},{\bf u},\nabla{\bf u},\Delta{\bf u},p,\nabla p)\right)\cdot{\bf v}\right)\cdot\tilde{\bf u} + \left(\nabla_{\nabla{\bf u}}{\bf P}({\bf x},{\bf u},\nabla{\bf u},\Delta{\bf u},p,\nabla p):\nabla{\bf v}\right)\cdot\tilde{\bf u}{\rm d}{\bf x}\\
        &- \int_\Gamma \left(\left(\nabla_{\nabla{\bf u}}{\bf P}({\bf x},{\bf u},\nabla{\bf u},\Delta{\bf u},p,\nabla p)\cdot{\bf n}\right)\cdot{\bf v}\right)\cdot\tilde{\bf u}{\rm d}\Gamma.
    \end{align*}
    \item Term $D_{\Delta{\bf u}}{\bf P}({\bf x},{\bf u},\nabla{\bf u},\Delta{\bf u},p,\nabla p)\Delta\tilde{\bf u}\cdot{\bf v}$:
    \begin{align*}
        &-\int_\Omega D_{\Delta{\bf u}}{\bf P}({\bf x},{\bf u},\nabla{\bf u},\Delta{\bf u},p,\nabla p)\Delta\tilde{\bf u}\cdot{\bf v}{\rm d}{\bf x} = -\int_\Omega \left(\partial_{\Delta u_j}P_i({\bf x},{\bf u},\nabla{\bf u},\Delta{\bf u},p,\nabla p)\right)_{i,j=1}^N\Delta\tilde{\bf u}\cdot{\bf v}{\rm d}{\bf x}\\
        =&\, -\int_\Omega \left(\sum_{j=1}^N \partial_{\Delta u_j}P_i({\bf x},{\bf u},\nabla{\bf u},\Delta{\bf u},p,\nabla p)\Delta\tilde{u}_j\right)_{j=1}^N\cdot{\bf v}{\rm d}{\bf x} = -\int_\Omega \sum_{i=1}^N\sum_{j=1}^N \partial_{\Delta u_j}P_i({\bf x},{\bf u},\nabla{\bf u},\Delta{\bf u},p,\nabla p)\Delta\tilde{u}_jv_i{\rm d}{\bf x}\\
        =&\, -\sum_{i=1}^N\sum_{j=1}^N \int_\Omega \partial_{\Delta u_j}P_i({\bf x},{\bf u},\nabla{\bf u},\Delta{\bf u},p,\nabla p)\Delta\tilde{u}_jv_i{\rm d}{\bf x}\\
        =&\, -\sum_{i=1}^N\sum_{j=1}^N \int_\Omega \tilde{u}_j\Delta\left(\partial_{\Delta u_j}P_i({\bf x},{\bf u},\nabla{\bf u},\Delta{\bf u},p,\nabla p)v_i\right){\rm d}{\bf x}\\
        &+ \int_\Gamma \partial_{\bf n}\tilde{u}_j\partial_{\Delta u_j}P_i({\bf x},{\bf u},\nabla{\bf u},\Delta{\bf u},p,\nabla p)v_i - \tilde{u}_j\partial_{\bf n}\left(\partial_{\Delta u_j}P_i({\bf x},{\bf u},\nabla{\bf u},\Delta{\bf u},p,\nabla p)v_i\right){\rm d}\Gamma\\
        =&\, -\sum_{i=1}^N\sum_{j=1}^N \int_\Omega \tilde{u}_j\Delta\partial_{\Delta u_j}P_i({\bf x},{\bf u},\nabla{\bf u},\Delta{\bf u},p,\nabla p)v_i + \tilde{u}_j\partial_{\Delta u_j}P_i({\bf x},{\bf u},\nabla{\bf u},\Delta{\bf u},p,\nabla p)\Delta v_i{\rm d}{\bf x}\\
        &+ \int_\Gamma \partial_{\bf n}\tilde{u}_j\partial_{\Delta u_j}P_i({\bf x},{\bf u},\nabla{\bf u},\Delta{\bf u},p,\nabla p)v_i - \tilde{u}_j\partial_{\bf n}\partial_{\Delta u_j}P_i({\bf x},{\bf u},\nabla{\bf u},\Delta{\bf u},p,\nabla p)v_i - \tilde{u}_j\partial_{\Delta u_j}P_i({\bf x},{\bf u},\nabla{\bf u},\Delta{\bf u},p,\nabla p)\partial_{\bf n}v_i{\rm d}\Gamma\\
        =&\, -\int_\Omega \sum_{i=1}^N\sum_{j=1}^N v_i\Delta\partial_{\Delta u_j}P_i({\bf x},{\bf u},\nabla{\bf u},\Delta{\bf u},p,\nabla p)\tilde{u}_j + \Delta v_i\partial_{\Delta u_j}P_i({\bf x},{\bf u},\nabla{\bf u},\Delta{\bf u},p,\nabla p)\tilde{u}_j{\rm d}{\bf x}\\
        &+ \int_\Gamma \sum_{i=1}^N\sum_{j=1}^N v_i\partial_{\Delta u_j}P_i({\bf x},{\bf u},\nabla{\bf u},\Delta{\bf u},p,\nabla p)\partial_{\bf n}\tilde{u}_j - \sum_{i=1}^N\sum_{j=1}^N v_i\partial_{\bf n}\partial_{\Delta u_j}P_i({\bf x},{\bf u},\nabla{\bf u},\Delta{\bf u},p,\nabla p)\tilde{u}_j\\
        &\hspace{1cm} - \sum_{i=1}^N\sum_{j=1}^N \partial_{\bf n}v_i\partial_{\Delta u_j}P_i({\bf x},{\bf u},\nabla{\bf u},\Delta{\bf u},p,\nabla p)\tilde{u}_j{\rm d}\Gamma\\
        =&\, -\int_\Omega {\bf v}^\top\Delta D_{\Delta{\bf u}}{\bf P}({\bf x},{\bf u},\nabla{\bf u},\Delta{\bf u},p,\nabla p)\tilde{\bf u} + \Delta{\bf v}^\top D_{\Delta{\bf u}}{\bf P}({\bf x},{\bf u},\nabla{\bf u},\Delta{\bf u},p,\nabla p)\tilde{\bf u}{\rm d}{\bf x}\\
        &+ \int_\Gamma {\bf v}^\top D_{\Delta{\bf u}}{\bf P}({\bf x},{\bf u},\nabla{\bf u},\Delta{\bf u},p,\nabla p)\partial_{\bf n}\tilde{\bf u} - {\bf v}^\top\partial_{\bf n}D_{\Delta{\bf u}}{\bf P}({\bf x},{\bf u},\nabla{\bf u},\Delta{\bf u},p,\nabla p)\tilde{\bf u} - \partial_{\bf n}{\bf v}^\top D_{\Delta{\bf u}}{\bf P}({\bf x},{\bf u},\nabla{\bf u},\Delta{\bf u},p,\nabla p)\tilde{\bf u}{\rm d}\Gamma.
    \end{align*}
    \item Term $D_{\nabla{\bf u}}{\bf f}({\bf x},{\bf u},\nabla{\bf u},p)\nabla\tilde{\bf u}\cdot{\bf v}$:
    \begin{align*}
        &\int_\Omega D_{\nabla{\bf u}}{\bf f}({\bf x},{\bf u},\nabla{\bf u},p)\nabla\tilde{\bf u}\cdot{\bf v}{\rm d}{\bf x} = \int_\Omega \left(\nabla_{\nabla{\bf u}}f_k({\bf x},{\bf u},\nabla{\bf u},p):\nabla\tilde{\bf u}\right)_{k=1}^N\cdot{\bf v}{\rm d}{\bf x} = \int_\Omega \sum_{k=1}^N \nabla_{\nabla{\bf u}}f_k({\bf x},{\bf u},\nabla{\bf u},p):\nabla\tilde{\bf u}v_k{\rm d}{\bf x}\\
        =&\, \int_\Omega \sum_{k=1}^N\sum_{i=1}^N\sum_{j=1}^N \partial_{\partial_{x_i}u_j}f_k({\bf x},{\bf u},\nabla{\bf u},p)\partial_{x_i}\tilde{u}_jv_k{\rm d}{\bf x} = \sum_{i=1}^N\sum_{j=1}^N\sum_{k=1}^N \int_\Omega \partial_{\partial_{x_i}u_j}f_k({\bf x},{\bf u},\nabla{\bf u},p)\partial_{x_i}\tilde{u}_jv_k{\rm d}{\bf x}\\
        =&\, \sum_{i=1}^N\sum_{j=1}^N\sum_{k=1}^N -\int_\Omega \partial_{x_i}\partial_{\partial_{x_i}u_j}f_k({\bf x},{\bf u},\nabla{\bf u},p)\tilde{u}_jv_k + \partial_{\partial_{x_i}u_j}f_k({\bf x},{\bf u},\nabla{\bf u},p)\tilde{u}_j\partial_{x_i}v_k{\rm d}{\bf x} + \int_\Gamma n_i\partial_{\partial_{x_i}u_j}f_k({\bf x},{\bf u},\nabla{\bf u},p)\tilde{u}_jv_k{\rm d}\Gamma\\
        =&\, -\int_\Omega \sum_{j=1}^N \tilde{u}_j\sum_{k=1}^N v_k\sum_{i=1}^N \partial_{x_i}\partial_{\partial_{x_i}u_j}f_k({\bf x},{\bf u},\nabla{\bf u},p) + \sum_{j=1}^N \tilde{u}_j\sum_{k=1}^N\sum_{i=1}^N \partial_{\partial_{x_i}u_j}f_k({\bf x},{\bf u},\nabla{\bf u},p)\partial_{x_i}v_k{\rm d}{\bf x}\\
        &+ \int_\Gamma \sum_{j=1}^N \tilde{u}_j\sum_{k=1}^N v_k\sum_{i=1}^N n_i\partial_{\partial_{x_i}u_j}f_k({\bf x},{\bf u},\nabla{\bf u},p){\rm d}\Gamma\\
        =&\, - \int_\Omega \sum_{j=1}^N \tilde{u}_j\sum_{k=1}^N v_k\nabla\cdot\left(\nabla_{\nabla u_j}f_k({\bf x},{\bf u},\nabla{\bf u},p)\right) + \sum_{j=1}^N \tilde{u}_j\sum_{k=1}^N \nabla_{\nabla u_j}f_k({\bf x},{\bf u},\nabla{\bf u},p)\cdot\nabla v_k{\rm d}{\bf x}\\
        &+ \int_\Gamma \sum_{j=1}^N \tilde{u}_j\sum_{k=1}^N v_k\nabla_{\nabla u_j}f_k({\bf x},{\bf u},\nabla{\bf u},p)\cdot{\bf n}{\rm d}\Gamma\\
        =&\, -\int_\Omega \sum_{j=1}^N \tilde{u}_j\nabla\cdot\left(\nabla_{\nabla u_j}{\bf f}({\bf x},{\bf u},\nabla{\bf u},p)\right)\cdot{\bf v} + \sum_{j=1}^N \tilde{u}_j\nabla_{\nabla u_j}{\bf f}({\bf x},{\bf u},\nabla{\bf u},p):\nabla{\bf v}{\rm d}{\bf x} + \int_\Gamma \sum_{j=1}^N \tilde{u}_j\left(\nabla_{\nabla u_j}{\bf f}({\bf x},{\bf u},\nabla{\bf u},p)\cdot{\bf n}\right)\cdot{\bf v}{\rm d}\Gamma\\
        =&\, -\int_\Omega \left(\nabla\cdot\left(\nabla_{\nabla{\bf u}}{\bf f}({\bf x},{\bf u},\nabla{\bf u},p)\right)\cdot{\bf v}\right)\cdot\tilde{\bf u} + \left(\nabla_{\nabla{\bf u}}{\bf f}({\bf x},{\bf u},\nabla{\bf u},p):\nabla{\bf v}\right)\cdot\tilde{\bf u}{\rm d}{\bf x} + \int_\Gamma \left(\left(\nabla_{\nabla{\bf u}}{\bf f}({\bf x},{\bf u},\nabla{\bf u},p)\cdot{\bf n}\right)\cdot{\bf v}\right)\cdot\tilde{\bf u}{\rm d}\Gamma.
    \end{align*}
    \item Term $-D_{\nabla p}{\bf P}({\bf x},{\bf u},\nabla{\bf u},\Delta{\bf u},p,\nabla p)\nabla\tilde{p}\cdot{\bf v}$:
    \begin{align*}
        &-\int_\Omega D_{\nabla p}{\bf P}({\bf x},{\bf u},\nabla{\bf u},\Delta{\bf u},p,\nabla p)\nabla\tilde{p}\cdot{\bf v}{\rm d}{\bf x} = -\int_\Omega \left(\sum_{j=1}^N \partial_{\partial_{x_j}p}P_i({\bf x},{\bf u},\nabla{\bf u},\Delta{\bf u},p,\nabla p)\partial_{x_j}\tilde{p}\right)_{i=1}^N\cdot{\bf v}{\rm d}{\bf x}\\
        =&\, - \int_\Omega \sum_{i=1}^N\sum_{j=1}^N v_i\partial_{\partial_{x_j}p}P_i({\bf x},{\bf u},\nabla{\bf u},\Delta{\bf u},p,\nabla p)\partial_{x_j}\tilde{p}{\rm d}{\bf x} = - \sum_{i=1}^N\sum_{j=1}^N \int_\Omega v_i\partial_{\partial_{x_j}p}P_i({\bf x},{\bf u},\nabla{\bf u},\Delta{\bf u},p,\nabla p)\partial_{x_j}\tilde{p}{\rm d}{\bf x}\\
        =&\, \sum_{i=1}^N\sum_{j=1}^N \int_\Omega \partial_{x_j}v_i\partial_{\partial_{x_j}p}P_i({\bf x},{\bf u},\nabla{\bf u},\Delta{\bf u},p,\nabla p)\tilde{p} + v_i\partial_{x_j}\partial_{\partial_{x_j}p}P_i({\bf x},{\bf u},\nabla{\bf u},\Delta{\bf u},p,\nabla p)\tilde{p}{\rm d}{\bf x} - \int_\Gamma v_i\partial_{\partial_{x_j}p}P_i({\bf x},{\bf u},\nabla{\bf u},\Delta{\bf u},p,\nabla p)\tilde{p}n_j{\rm d}\Gamma\\
        =&\, \int_\Omega \tilde{p}\sum_{i=1}^N\sum_{j=1}^N \partial_{x_i}v_j\partial_{\partial_{x_i}p}P_j({\bf x},{\bf u},\nabla{\bf u},\Delta{\bf u},p,\nabla p) + \tilde{p}\sum_{i=1}^N v_i\sum_{j=1}^N \partial_{x_j}\partial_{\partial_{x_j}p}P_i({\bf x},{\bf u},\nabla{\bf u},\Delta{\bf u},p,\nabla p){\rm d}{\bf x}\\
        &- \int_\Gamma \tilde{p}\sum_{i=1}^N\sum_{j=1}^N n_i\partial_{\partial_{x_j}p}P_j({\bf x},{\bf u},\nabla{\bf u},\Delta{\bf u},p,\nabla p)v_j{\rm d}\Gamma\\
        =&\, \int_\Omega \tilde{p}\nabla_{\nabla p}{\bf P}({\bf x},{\bf u},\nabla{\bf u},\Delta{\bf u},p,\nabla p):\nabla{\bf v} + \tilde{p}\sum_{i=1}^N v_i\nabla\cdot\left(\nabla_{\nabla p}P_i({\bf x},{\bf u},\nabla{\bf u},\Delta{\bf u},p,\nabla p)\right){\rm d}{\bf x} - \int_\Gamma \tilde{p}{\bf n}^\top\nabla_{\nabla p}{\bf P}({\bf x},{\bf u},\nabla{\bf u},\Delta{\bf u},p,\nabla p){\bf v}{\rm d}\Gamma\\
        =&\, \int_\Omega \tilde{p}\nabla_{\nabla p}{\bf P}({\bf x},{\bf u},\nabla{\bf u},\Delta{\bf u},p,\nabla p):\nabla{\bf v} + \tilde{p}\nabla\cdot\left(\nabla_{\nabla p}{\bf P}({\bf x},{\bf u},\nabla{\bf u},\Delta{\bf u},p,\nabla p)\right)\cdot{\bf v}{\rm d}{\bf x} - \int_\Gamma \tilde{p}{\bf n}^\top\nabla_{\nabla p}{\bf P}({\bf x},{\bf u},\nabla{\bf u},\Delta{\bf u},p,\nabla p){\bf v}{\rm d}\Gamma.
    \end{align*}
    \item Term $q\nabla\cdot\tilde{\bf u}$: Use \eqref{ibp},
    \begin{align*}
        \int_\Omega q\nabla\cdot\tilde{\bf u}{\rm d}{\bf x} = -\int_\Omega \nabla q\cdot\tilde{\bf u}{\rm d}{\bf x} + \int_\Gamma q\tilde{\bf u}\cdot{\bf n}{\rm d}\Gamma.
    \end{align*}
    \item Term $qD_{\nabla{\bf u}}f_{\rm div}({\bf x},{\bf u},\nabla{\bf u},p)\nabla\tilde{\bf u}$:
    \begin{align*}
        &\int_\Omega qD_{\nabla{\bf u}}f_{\rm div}({\bf x},{\bf u},\nabla{\bf u},p)\nabla\tilde{\bf u}{\rm d}{\bf x} = \int_\Omega q\nabla_{\nabla{\bf u}}f_{\rm div}({\bf x},{\bf u},\nabla{\bf u},p):\nabla\tilde{\bf u}{\rm d}{\bf x}\\
        =&\, \int_\Omega q\sum_{i=1}^N\sum_{j=1}^N \partial_{\partial_{x_i}u_j}f_{\rm div}({\bf x},{\bf u},\nabla{\bf u},p)\partial_{x_i}\tilde{u}_j{\rm d}{\bf x} = \sum_{i=1}^N\sum_{j=1}^N \int_\Omega q\partial_{\partial_{x_i}u_j}f_{\rm div}({\bf x},{\bf u},\nabla{\bf u},p)\partial_{x_i}\tilde{u}_j{\rm d}{\bf x}\\
        =&\, \sum_{i=1}^N\sum_{j=1}^N -\int_\Omega \partial_{x_i}q\partial_{\partial_{x_i}u_j}f_{\rm div}({\bf x},{\bf u},\nabla{\bf u},p)\tilde{u}_j + q\partial_{x_i}\partial_{\partial_{x_i}u_j}f_{\rm div}({\bf x},{\bf u},\nabla{\bf u},p)\tilde{u}_j{\rm d}{\bf x} + \int_\Gamma qn_i\partial_{\partial_{x_i}u_j}f_{\rm div}({\bf x},{\bf u},\nabla{\bf u},p)\tilde{u}_j{\rm d}\Gamma\\
        =&\, -\int_\Omega \sum_{i=1}^N\sum_{j=1}^N \partial_{x_i}q\partial_{\partial_{x_i}u_j}f_{\rm div}({\bf x},{\bf u},\nabla{\bf u},p)\tilde{u}_j + q\sum_{j=1}^N \tilde{u}_j\sum_{i=1}^N \partial_{x_i}\partial_{\partial_{x_i}u_j}f_{\rm div}({\bf x},{\bf u},\nabla{\bf u},p){\rm d}{\bf x}\\
        &+ \int_\Gamma q\sum_{i=1}^N\sum_{j=1}^N n_i\partial_{\partial_{x_i}u_j}f_{\rm div}({\bf x},{\bf u},\nabla{\bf u},p)\tilde{u}_j{\rm d}\Gamma\\
        =&\, -\int_\Omega \nabla^\top q\nabla_{\nabla{\bf u}}f_{\rm div}({\bf x},{\bf u},\nabla{\bf u},p)\tilde{\bf u} + q\sum_{j=1}^N \tilde{u}_j\nabla\cdot\left(\nabla_{\nabla u_j}f_{\rm div}({\bf x},{\bf u},\nabla{\bf u},p)\right){\rm d}{\bf x} + \int_\Gamma q{\bf n}^\top\nabla_{\nabla{\bf u}}f_{\rm div}({\bf x},{\bf u},\nabla{\bf u},p)\tilde{\bf u}{\rm d}\Gamma\\
        =&\, -\int_\Omega \nabla^\top q\nabla_{\nabla{\bf u}}f_{\rm div}({\bf x},{\bf u},\nabla{\bf u},p)\tilde{\bf u} + q\left(\nabla\cdot\left(\nabla_{\nabla{\bf u}}f_{\rm div}({\bf x},{\bf u},\nabla{\bf u},p)\right)\right)\cdot\tilde{\bf u}{\rm d}{\bf x} + \int_\Gamma q{\bf n}^\top\nabla_{\nabla{\bf u}}f_{\rm div}({\bf x},{\bf u},\nabla{\bf u},p)\tilde{\bf u}{\rm d}\Gamma.
    \end{align*}
\end{enumerate}
Gather terms:
\begin{equation}
    \label{Euler-Lagrange for general stationary fluid dynamics PDEs}
    \tag{EuLa-gfld}
    \left.\begin{split}
        &\int_\Omega \left[\nabla_{\bf u}J_\Omega({\bf x},{\bf u},\nabla{\bf u},p) - \nabla\cdot\left(\nabla_{\nabla{\bf u}}J_\Omega({\bf x},{\bf u},\nabla{\bf u},p)\right) - \nabla_{\bf u}{\bf P}({\bf x},{\bf u},\nabla{\bf u},\Delta{\bf u},p,\nabla p){\bf v} + \nabla\cdot\left(\nabla_{\nabla{\bf u}}{\bf P}({\bf x},{\bf u},\nabla{\bf u},\Delta{\bf u},p,\nabla p)\right)\cdot{\bf v}\right.\\
        &\hspace{1cm} + \nabla_{\nabla{\bf u}}{\bf P}({\bf x},{\bf u},\nabla{\bf u},\Delta{\bf u},p,\nabla p):\nabla{\bf v} - \Delta\nabla_{\Delta{\bf u}}{\bf P}({\bf x},{\bf u},\nabla{\bf u},\Delta{\bf u},p,\nabla p){\bf v} - \nabla_{\Delta{\bf u}}{\bf P}({\bf x},{\bf u},\nabla{\bf u},\Delta{\bf u},p,\nabla p)\Delta{\bf v}\\
        &\hspace{1cm} + \nabla_{\bf u}{\bf f}({\bf x},{\bf u},\nabla{\bf u},p){\bf v} - \nabla\cdot\left(\nabla_{\nabla{\bf u}}{\bf f}({\bf x},{\bf u},\nabla{\bf u},p)\right)\cdot{\bf v} - \nabla_{\nabla{\bf u}}{\bf f}({\bf x},{\bf u},\nabla{\bf u},p):\nabla{\bf v} - \nabla q + q\nabla_{\bf u}f_{\rm div}({\bf x},{\bf u},\nabla{\bf u},p)\\
        &\hspace{1cm} \left.- \nabla_{\nabla{\bf u}}f_{\rm div}({\bf x},{\bf u},\nabla{\bf u},p)\cdot\nabla q - q\nabla\cdot\left(\nabla_{\nabla{\bf u}}f_{\rm div}({\bf x},{\bf u},\nabla{\bf u},p)\right)\right]\cdot\tilde{\bf u}{\rm d}{\bf x}\\
        &+ \int_\Omega \tilde{p}\left[D_pJ_\Omega({\bf x},{\bf u},\nabla{\bf u},p) - D_p{\bf P}({\bf x},{\bf u},\nabla{\bf u},\Delta{\bf u},p,\nabla p)\cdot{\bf v} + \nabla_{\nabla p}{\bf P}({\bf x},{\bf u},\nabla{\bf u},\Delta{\bf u},p,\nabla p):\nabla{\bf v}\right.\\
        &\hspace{1cm} \left.+ \nabla\cdot\left(\nabla_{\nabla p}{\bf P}({\bf x},{\bf u},\nabla{\bf u},\Delta{\bf u},p,\nabla p)\right)\cdot{\bf v} + D_p{\bf f}({\bf x},{\bf u},\nabla{\bf u},p)\cdot{\bf v} + qD_pf_{\rm div}({\bf x},{\bf u},\nabla{\bf u},p)\right]{\rm d}{\bf x}\\
        &+ \int_\Gamma \left[\nabla_{\nabla{\bf u}}J_\Omega({\bf x},{\bf u},\nabla{\bf u},p)\cdot{\bf n} + \nabla_{\bf u}J_\Gamma({\bf x},{\bf u},\nabla{\bf u},p,{\bf n},{\bf t}) - \left(\nabla_{\nabla{\bf u}}{\bf P}({\bf x},{\bf u},\nabla{\bf u},\Delta{\bf u},p,\nabla p)\cdot{\bf n}\right)\cdot{\bf v}\right.\\
        &\hspace{1cm} - \partial_{\bf n}\nabla_{\Delta{\bf u}}{\bf P}({\bf x},{\bf u},\nabla{\bf u},\Delta{\bf u},p,\nabla p){\bf v} - \nabla_{\Delta{\bf u}}{\bf P}({\bf x},{\bf u},\nabla{\bf u},\Delta{\bf u},p,\nabla p)\partial_{\bf n}{\bf v} + \left(\nabla_{\nabla{\bf u}}{\bf f}({\bf x},{\bf u},\nabla{\bf u},p)\cdot{\bf n}\right)\cdot{\bf v} + q{\bf n}\\
        &\hspace{1cm} \left.+ q\nabla_{\nabla{\bf u}}f_{\rm div}({\bf x},{\bf u},\nabla{\bf u},p)\cdot{\bf n} - \delta_{\mathcal{L}}\nabla_{\bf u}{\bf Q}({\bf x},{\bf u},\nabla{\bf u},p,{\bf n},{\bf t}){\bf v}_{\rm bc}\right]\cdot\tilde{\bf u}{\rm d}\Gamma\\
        &+ \int_\Gamma \tilde{p}\left[D_pJ_\Gamma({\bf x},{\bf u},\nabla{\bf u},p,{\bf n},{\bf t}) - {\bf n}^\top\nabla_{\nabla p}{\bf P}({\bf x},{\bf u},\nabla{\bf u},\Delta{\bf u},p,\nabla p){\bf v} - \delta_{\mathcal{L}}D_p{\bf Q}({\bf x},{\bf u},\nabla{\bf u},p,{\bf n},{\bf t})\cdot{\bf v}_{\rm bc}\right]{\rm d}\Gamma\\
        &+ \int_\Gamma \nabla_{\nabla{\bf u}}J_\Gamma({\bf x},{\bf u},\nabla{\bf u},p,{\bf n},{\bf t}):\nabla\tilde{\bf u} + {\bf v}^\top D_{\Delta{\bf u}}{\bf P}({\bf x},{\bf u},\nabla{\bf u},\Delta{\bf u},p,\nabla p)\partial_{\bf n}\tilde{\bf u}\\
        &\hspace{1cm} - \delta_{\mathcal{L}}D_{\nabla{\bf u}}{\bf Q}({\bf x},{\bf u},\nabla{\bf u},p,{\bf n},{\bf t})\nabla\tilde{\bf u}\cdot{\bf v}_{\rm bc}{\rm d}\Gamma = 0,\ \forall({\bf u},p,\Omega,\tilde{\bf u},\tilde{p}).
    \end{split}\right.
\end{equation}
Since this equation holds for all variations $(\tilde{\bf u},\tilde{p})$, consider the following 2 cases:
\begin{itemize}[leftmargin=0in]
    \item \textbf{Case $\delta_{\mathcal{L}} = 0$.} This means to ``activate'' the boundary-condition constraint ${\bf Q}({\bf x},{\bf u},\nabla{\bf u},p,{\bf n},{\bf t}) = {\bf f}_{\rm bc}({\bf x})$ on $\Gamma$, so it will not be penalized by the Lagrangian $L$. To simplify further the last equation, we define $\Gamma_{\rm v}^{\bf u}$ and $\Gamma_{\rm v}^p$ as the ``varying'' components w.r.t. ${\bf u}$ and $p$ of $\Gamma$, respectively, i.e.,
    \begin{align*}
        \Gamma_{\rm nv}^{\bf u} &\coloneqq\left\{{\bf x}\in\Gamma;\left({\bf Q}({\bf x},{\bf u} + \tilde{\bf u},\nabla{\bf u} + \nabla\tilde{\bf u},p + \tilde{p},{\bf n},{\bf t}) = {\bf Q}({\bf x},{\bf u},\nabla{\bf u},p,{\bf n},{\bf t})\right)\Rightarrow\tilde{\bf u} = {\bf 0}\right\},\\
        \Gamma_{\rm nv}^p &\coloneqq\left\{{\bf x}\in\Gamma;\left({\bf Q}({\bf x},{\bf u} + \tilde{\bf u},\nabla{\bf u} + \nabla\tilde{\bf u},p + \tilde{p},{\bf n},{\bf t}) = {\bf Q}({\bf x},{\bf u},\nabla{\bf u},p,{\bf n},{\bf t})\right)\Rightarrow\tilde{p} = 0\right\},\\
        \Gamma_{\rm v}^{\bf u} &\coloneqq\Gamma\backslash\Gamma_{\rm nv}^{\bf u},\ \Gamma_{\rm v}^p\coloneqq\Gamma\backslash\Gamma_{\rm nv}^p.
    \end{align*}
    E.g., the Dirichlet components of $\Gamma$ w.r.t. ${\bf u}$ and $p$, denoted by $\Gamma_{\rm D}^{\bf u}$ and $\Gamma_{\rm D}^p$, respectively:
    \begin{equation*}
        \left\{\begin{split}
            {\bf u} &= {\bf u}_{\rm D},&&\mbox{ on } \Gamma_{\rm D}^{\bf u},\\
            p &= p_{\rm D},&&\mbox{ on } \Gamma_{\rm D}^p.
        \end{split}\right.
    \end{equation*}
    belong to the ``non-variation'' components just defined of $\Gamma$: $\Gamma_{\rm D}^{\bf u}\subset\Gamma_{\rm nv}^{\bf u}$ and $\Gamma_{\rm D}^p\subset\Gamma_{\rm nv}^p$.
    
    To see the general structure, we rewrite \eqref{Euler-Lagrange for general stationary fluid dynamics PDEs} as follows:
    \begin{equation}
        \label{brief Euler-Lagrange for general stationary fluid dynamics PDEs}
        \tag{brief-EuLa-gfld}
        \left.\begin{split}
            &\int_\Omega {\bf F}_\Omega^{\tilde{\bf u}}({\bf x},{\bf u},\nabla{\bf u},\Delta{\bf u},p,\nabla p,{\bf v},\nabla{\bf v},\Delta{\bf v},q,\nabla q)\cdot\tilde{\bf u}{\rm d}{\bf x} + \int_\Omega F_\Omega^{\tilde{p}}({\bf x},{\bf u},\nabla{\bf u},\Delta{\bf u},p,\nabla p,{\bf v},\nabla{\bf v},q)\tilde{p}{\rm d}{\bf x}\\
            &\hspace{5mm}+ \int_\Gamma {\bf F}_\Gamma^{\tilde{\bf u}}({\bf x},{\bf u},\nabla{\bf u},\Delta{\bf u},p,\nabla p,{\bf v},\nabla{\bf v},q,{\bf n},{\bf t})\cdot\tilde{\bf u}{\rm d}\Gamma + \int_\Gamma F_\Gamma^{\tilde{p}}({\bf x},{\bf u},\nabla{\bf u},\Delta{\bf u},p,\nabla p,{\bf v},{\bf n},{\bf t})\tilde{p}{\rm d}\Gamma\\
            &\hspace{5mm}+ \int_\Gamma F_\Gamma^{\nabla\tilde{\bf u}}({\bf x},{\bf u},\nabla{\bf u},\Delta{\bf u},p,\nabla p,{\bf v},{\bf n},{\bf t},\nabla\tilde{\bf u}){\rm d}\Gamma = 0,\ \forall({\bf u},p,\Omega,\tilde{\bf u},\tilde{p}),
        \end{split}\right.
    \end{equation}
    where
    \begin{align*}
        &{\bf F}_\Omega^{\tilde{\bf u}}({\bf x},{\bf u},\nabla{\bf u},\Delta{\bf u},p,\nabla p,{\bf v},\nabla{\bf v},\Delta{\bf v},q,\nabla q)\\
        &\hspace{5mm}\coloneqq\nabla_{\bf u}J_\Omega({\bf x},{\bf u},\nabla{\bf u},p) - \nabla\cdot\left(\nabla_{\nabla{\bf u}}J_\Omega({\bf x},{\bf u},\nabla{\bf u},p)\right) - \nabla_{\bf u}{\bf P}({\bf x},{\bf u},\nabla{\bf u},\Delta{\bf u},p,\nabla p){\bf v} + \nabla\cdot\left(\nabla_{\nabla{\bf u}}{\bf P}({\bf x},{\bf u},\nabla{\bf u},\Delta{\bf u},p,\nabla p)\right)\cdot{\bf v}\\
        &\hspace{1cm}+ \nabla_{\nabla{\bf u}}{\bf P}({\bf x},{\bf u},\nabla{\bf u},\Delta{\bf u},p,\nabla p):\nabla{\bf v} - \Delta\nabla_{\Delta{\bf u}}{\bf P}({\bf x},{\bf u},\nabla{\bf u},\Delta{\bf u},p,\nabla p){\bf v} - \nabla_{\Delta{\bf u}}{\bf P}({\bf x},{\bf u},\nabla{\bf u},\Delta{\bf u},p,\nabla p)\Delta{\bf v}\\
        &\hspace{1cm}+ \nabla_{\bf u}{\bf f}({\bf x},{\bf u},\nabla{\bf u},p){\bf v} - \nabla\cdot\left(\nabla_{\nabla{\bf u}}{\bf f}({\bf x},{\bf u},\nabla{\bf u},p)\right)\cdot{\bf v} - \nabla_{\nabla{\bf u}}{\bf f}({\bf x},{\bf u},\nabla{\bf u},p):\nabla{\bf v} - \nabla q + q\nabla_{\bf u}f_{\rm div}({\bf x},{\bf u},\nabla{\bf u},p)\\
        &\hspace{1cm}- \nabla_{\nabla{\bf u}}f_{\rm div}({\bf x},{\bf u},\nabla{\bf u},p)\cdot\nabla q - q\nabla\cdot\left(\nabla_{\nabla{\bf u}}f_{\rm div}({\bf x},{\bf u},\nabla{\bf u},p)\right),\\
        &F_\Omega^{\tilde{p}}({\bf x},{\bf u},\nabla{\bf u},\Delta{\bf u},p,\nabla p,{\bf v},\nabla{\bf v},q)\\
        &\hspace{5mm}\coloneqq D_pJ_\Omega({\bf x},{\bf u},\nabla{\bf u},p) - D_p{\bf P}({\bf x},{\bf u},\nabla{\bf u},\Delta{\bf u},p,\nabla p)\cdot{\bf v} + \nabla_{\nabla p}{\bf P}({\bf x},{\bf u},\nabla{\bf u},\Delta{\bf u},p,\nabla p):\nabla{\bf v}\\
        &\hspace{1cm} + \nabla\cdot\left(\nabla_{\nabla p}{\bf P}({\bf x},{\bf u},\nabla{\bf u},\Delta{\bf u},p,\nabla p)\right)\cdot{\bf v} + D_p{\bf f}({\bf x},{\bf u},\nabla{\bf u},p)\cdot{\bf v} + qD_pf_{\rm div}({\bf x},{\bf u},\nabla{\bf u},p),\\
        &{\bf F}_\Gamma^{\tilde{\bf u}}({\bf x},{\bf u},\nabla{\bf u},\Delta{\bf u},p,\nabla p,{\bf v},\nabla{\bf v},q,{\bf n},{\bf t})\\
        &\hspace{5mm}\coloneqq\nabla_{\nabla{\bf u}}J_\Omega({\bf x},{\bf u},\nabla{\bf u},p)\cdot{\bf n} + \nabla_{\bf u}J_\Gamma({\bf x},{\bf u},\nabla{\bf u},p,{\bf n},{\bf t}) - \left(\nabla_{\nabla{\bf u}}{\bf P}({\bf x},{\bf u},\nabla{\bf u},\Delta{\bf u},p,\nabla p)\cdot{\bf n}\right)\cdot{\bf v}\\
        &\hspace{1cm} - \partial_{\bf n}\nabla_{\Delta{\bf u}}{\bf P}({\bf x},{\bf u},\nabla{\bf u},\Delta{\bf u},p,\nabla p){\bf v} - \nabla_{\Delta{\bf u}}{\bf P}({\bf x},{\bf u},\nabla{\bf u},\Delta{\bf u},p,\nabla p)\partial_{\bf n}{\bf v} + \left(\nabla_{\nabla{\bf u}}{\bf f}({\bf x},{\bf u},\nabla{\bf u},p)\cdot{\bf n}\right)\cdot{\bf v} + q{\bf n}\\
        &\hspace{1cm} + q\nabla_{\nabla{\bf u}}f_{\rm div}({\bf x},{\bf u},\nabla{\bf u},p)\cdot{\bf n},\\
        &F_\Gamma^{\tilde{p}}({\bf x},{\bf u},\nabla{\bf u},\Delta{\bf u},p,\nabla p,{\bf v},{\bf n},{\bf t})\coloneqq D_pJ_\Gamma({\bf x},{\bf u},\nabla{\bf u},p,{\bf n},{\bf t}) - {\bf n}^\top\nabla_{\nabla p}{\bf P}({\bf x},{\bf u},\nabla{\bf u},\Delta{\bf u},p,\nabla p){\bf v},\\
        &F_\Gamma^{\nabla\tilde{\bf u}}({\bf x},{\bf u},\nabla{\bf u},\Delta{\bf u},p,\nabla p,{\bf v},{\bf n},{\bf t},\nabla\tilde{\bf u})\coloneqq\nabla_{\nabla{\bf u}}J_\Gamma({\bf x},{\bf u},\nabla{\bf u},p,{\bf n},{\bf t}):\nabla\tilde{\bf u} + {\bf v}^\top D_{\Delta{\bf u}}{\bf P}({\bf x},{\bf u},\nabla{\bf u},\Delta{\bf u},p,\nabla p)\partial_{\bf n}\tilde{\bf u},
    \end{align*}
    for all $({\bf x},{\bf u},p,{\bf v},q,{\bf n},{\bf t},\tilde{\bf u},\tilde{p})$ s.t.
    \begin{align*}
        {\bf Q}({\bf x},{\bf u},\nabla{\bf u},p,{\bf n},{\bf t}) = {\bf Q}({\bf x},{\bf u} + \tilde{\bf u},\nabla{\bf u} + \nabla\tilde{\bf u},p + \tilde{p},{\bf n},{\bf t}) = {\bf f}_{\rm bc}({\bf x}) \mbox{ on } \Gamma.
    \end{align*}    
    We now deduce the adjoint equations of \eqref{general stationary fluid dynamics PDEs} from \eqref{brief Euler-Lagrange for general stationary fluid dynamics PDEs} as follows:
    \begin{itemize}
        \item Choose $\tilde{\bf u} = {\bf 0}$ in $\overline{\Omega}$, \eqref{brief Euler-Lagrange for general stationary fluid dynamics PDEs} then becomes
        \begin{align*}
            \int_\Omega F_\Omega^{\tilde{p}}({\bf x},{\bf u},\nabla{\bf u},\Delta{\bf u},p,\nabla p,{\bf v},\nabla{\bf v},q)\tilde{p}{\rm d}{\bf x} + \int_\Gamma F_\Gamma^{\tilde{p}}({\bf x},{\bf u},\nabla{\bf u},\Delta{\bf u},p,\nabla p,{\bf v},{\bf n},{\bf t})\tilde{p}{\rm d}\Gamma = 0,\ \forall({\bf u},p,\Omega,\tilde{p}).
        \end{align*}
        Then choose $\tilde{p}$ varying s.t. $\tilde{p}|_\Gamma = 0$, then the last equality yields
        \begin{align*}
            \int_\Omega F_\Omega^{\tilde{p}}({\bf x},{\bf u},\nabla{\bf u},\Delta{\bf u},p,\nabla p,{\bf v},\nabla{\bf v},q)\tilde{p}{\rm d}{\bf x} = 0,\ \forall({\bf u},p,\Omega,\tilde{p}) \mbox{ s.t. } \tilde{p}|_\Gamma = 0.
        \end{align*}
        Hence, $({\bf v},q)$ satisfies
        \begin{align}
            \label{domain integrand variation p}
            \boxed{F_\Omega^{\tilde{p}}({\bf x},{\bf u},\nabla{\bf u},\Delta{\bf u},p,\nabla p,{\bf v},\nabla{\bf v},q) = 0 \mbox{ in } \Omega.}
        \end{align}
        Plug it back in, obtain
        \begin{align*}
            \int_\Gamma F_\Gamma^{\tilde{p}}({\bf x},{\bf u},\nabla{\bf u},\Delta{\bf u},p,\nabla p,{\bf v},{\bf n},{\bf t})\tilde{p}{\rm d}\Gamma = 0,\ \forall({\bf u},p,\Omega,\tilde{p}).
        \end{align*}
        Note that $\tilde{p}|_{\Gamma_{\rm nv}^p} = 0$, the last equality yields
        \begin{align*}
            \int_{\Gamma_{\rm v}^p} F_\Gamma^{\tilde{p}}({\bf x},{\bf u},\nabla{\bf u},\Delta{\bf u},p,\nabla p,{\bf v},{\bf n},{\bf t})\tilde{p}{\rm d}\Gamma = 0,\ \forall({\bf u},p,\Omega,\tilde{p}).
        \end{align*}
        Thus, $({\bf v},q)$ satisfies
        \begin{align}
            \label{boundary integrand variation p}
            \boxed{F_\Gamma^{\tilde{p}}({\bf x},{\bf u},\nabla{\bf u},\Delta{\bf u},p,\nabla p,{\bf v},{\bf n},{\bf t}) = 0 \mbox{ on } \Gamma_{\rm v}^p.}
        \end{align}
        \item Assume that $({\bf v},q)$ satisfies \eqref{domain integrand variation p} and \eqref{boundary integrand variation p}, \eqref{brief Euler-Lagrange for general stationary fluid dynamics PDEs} then becomes
        \begin{align*}
            &\int_\Omega {\bf F}_\Omega^{\tilde{\bf u}}({\bf x},{\bf u},\nabla{\bf u},\Delta{\bf u},p,\nabla p,{\bf v},\nabla{\bf v},\Delta{\bf v},q,\nabla q)\cdot\tilde{\bf u}{\rm d}{\bf x} + \int_\Gamma {\bf F}_\Gamma^{\tilde{\bf u}}({\bf x},{\bf u},\nabla{\bf u},\Delta{\bf u},p,\nabla p,{\bf v},\nabla{\bf v},q,{\bf n},{\bf t})\cdot\tilde{\bf u}{\rm d}\Gamma\\
            &\hspace{5mm}+ \int_\Gamma F_\Gamma^{\nabla\tilde{\bf u}}({\bf x},{\bf u},\nabla{\bf u},\Delta{\bf u},p,\nabla p,{\bf v},{\bf n},{\bf t},\nabla\tilde{\bf u}){\rm d}\Gamma = 0,\ \forall({\bf u},p,\Omega,\tilde{\bf u}),
        \end{align*}
        Choose $\tilde{\bf u}$ varying s.t. $\tilde{\bf u}|_\Gamma = {\bf 0}$ and $\nabla\tilde{\bf u}|_\Gamma = {\bf 0}_{N\times N}$, the last equality yields
        \begin{align*}
            \int_\Omega {\bf F}_\Omega^{\tilde{\bf u}}({\bf x},{\bf u},\nabla{\bf u},\Delta{\bf u},p,\nabla p,{\bf v},\nabla{\bf v},\Delta{\bf v},q,\nabla q)\cdot\tilde{\bf u}{\rm d}{\bf x} = 0,\ \forall({\bf u},p,\Omega,\tilde{\bf u}) \mbox{ s.t. } \tilde{\bf u}|_\Gamma = {\bf 0} \mbox{ and } \nabla\tilde{\bf u}|_\Gamma = {\bf 0}_{N\times N}.
        \end{align*}
        Hence, $({\bf v},q)$ satisfies
        \begin{align}
            \label{domain integrand variation u}
            \boxed{{\bf F}_\Omega^{\tilde{\bf u}}({\bf x},{\bf u},\nabla{\bf u},\Delta{\bf u},p,\nabla p,{\bf v},\nabla{\bf v},\Delta{\bf v},q,\nabla q) = {\bf 0} \mbox{ in } \Omega.}
        \end{align}
        Plug it back in, obtain
        \begin{align*}
            \int_\Gamma {\bf F}_\Gamma^{\tilde{\bf u}}({\bf x},{\bf u},\nabla{\bf u},\Delta{\bf u},p,\nabla p,{\bf v},\nabla{\bf v},q,{\bf n},{\bf t})\cdot\tilde{\bf u}{\rm d}\Gamma + \int_\Gamma F_\Gamma^{\nabla\tilde{\bf u}}({\bf x},{\bf u},\nabla{\bf u},\Delta{\bf u},p,\nabla p,{\bf v},{\bf n},{\bf t},\nabla\tilde{\bf u}){\rm d}\Gamma = 0,\ \forall({\bf u},p,\Omega,\tilde{\bf u}),
        \end{align*}
        Note that $\tilde{\bf u}|_{\Gamma_{\rm nv}^{\bf u}} = {\bf 0}$, the last equality yields
        \begin{align*}
            \int_{\Gamma_{\rm v}^{\bf u}} {\bf F}_\Gamma^{\tilde{\bf u}}({\bf x},{\bf u},\nabla{\bf u},\Delta{\bf u},p,\nabla p,{\bf v},\nabla{\bf v},q,{\bf n},{\bf t})\cdot\tilde{\bf u}{\rm d}\Gamma + \int_\Gamma F_\Gamma^{\nabla\tilde{\bf u}}({\bf x},{\bf u},\nabla{\bf u},\Delta{\bf u},p,\nabla p,{\bf v},{\bf n},{\bf t},\nabla\tilde{\bf u}){\rm d}\Gamma = 0,\ \forall({\bf u},p,\Omega,\tilde{\bf u}).
        \end{align*}
        To simplify the last equality further, we need the explicit formula of ${\bf P}$ and ${\bf Q}$.
    \end{itemize}
    Conclude:
    \begin{equation}
        \label{adjoint general stationary fluid dynamics PDEs}
        \tag{adj-gfld}
        \boxed{\left\{\begin{split}
                &-\nabla_{\Delta{\bf u}}{\bf P}({\bf x},{\bf u},\nabla{\bf u},\Delta{\bf u},p,\nabla p)\Delta{\bf v} + \left(\nabla_{\nabla{\bf u}}{\bf P}({\bf x},{\bf u},\nabla{\bf u},\Delta{\bf u},p,\nabla p) - \nabla_{\nabla{\bf u}}{\bf f}({\bf x},{\bf u},\nabla{\bf u},p)\right):\nabla{\bf v}\\
                &\hspace{1cm}- \left[\nabla_{\bf u}{\bf P}({\bf x},{\bf u},\nabla{\bf u},\Delta{\bf u},p,\nabla p) + \Delta\nabla_{\Delta{\bf u}}{\bf P}({\bf x},{\bf u},\nabla{\bf u},\Delta{\bf u},p,\nabla p) - \nabla_{\bf u}{\bf f}({\bf x},{\bf u},\nabla{\bf u},p)\right]{\bf v}\\
                &\hspace{1cm}+ \left[\nabla\cdot\left(\nabla_{\nabla{\bf u}}{\bf P}({\bf x},{\bf u},\nabla{\bf u},\Delta{\bf u},p,\nabla p)\right) - \nabla\cdot\left(\nabla_{\nabla{\bf u}}{\bf f}({\bf x},{\bf u},\nabla{\bf u},p)\right)\right]\cdot{\bf v} - \nabla q - \nabla_{\nabla{\bf u}}f_{\rm div}({\bf x},{\bf u},\nabla{\bf u},p)\cdot\nabla q\\
                &\hspace{1cm}+ q\left[-\nabla\cdot\left(\nabla_{\nabla{\bf u}}f_{\rm div}({\bf x},{\bf u},\nabla{\bf u},p)\right) + \nabla_{\bf u}f_{\rm div}({\bf x},{\bf u},\nabla{\bf u},p)\right] = \nabla\cdot\left(\nabla_{\nabla{\bf u}}J_\Omega({\bf x},{\bf u},\nabla{\bf u},p)\right) - \nabla_{\bf u}J_\Omega({\bf x},{\bf u},\nabla{\bf u},p) \mbox{ in } \Omega,\\
                &\nabla_{\nabla p}{\bf P}({\bf x},{\bf u},\nabla{\bf u},\Delta{\bf u},p,\nabla p):\nabla{\bf v} + \left[-D_p{\bf P}({\bf x},{\bf u},\nabla{\bf u},\Delta{\bf u},p,\nabla p) + \nabla\cdot\left(\nabla_{\nabla p}{\bf P}({\bf x},{\bf u},\nabla{\bf u},\Delta{\bf u},p,\nabla p)\right) + D_p{\bf f}({\bf x},{\bf u},\nabla{\bf u},p)\right]\cdot{\bf v}\\
                &\hspace{1cm} = - D_pJ_\Omega ({\bf x},{\bf u},\nabla{\bf u},p) - qD_pf_{\rm div}({\bf x},{\bf u},\nabla{\bf u},p) \mbox{ in } \Omega,\\
                &{\bf Q}({\bf x},{\bf u},\nabla{\bf u},p,{\bf n},{\bf t}) = {\bf f}_{\rm bc}({\bf x}) \mbox{ on } \Gamma,\\
                &{\bf n}^\top\nabla_{\nabla p}{\bf P}({\bf x},{\bf u},\nabla{\bf u},\Delta{\bf u},p,\nabla p){\bf v} = D_pJ_\Gamma({\bf x},{\bf u},\nabla{\bf u},p,{\bf n},{\bf t}) \mbox{ on } \Gamma_{\rm v}^p,\\
                &\int_{\Gamma_{\rm v}^{\bf u}} \left[\nabla_{\nabla{\bf u}}J_\Omega({\bf x},{\bf u},\nabla{\bf u},p)\cdot{\bf n} + \nabla_{\bf u}J_\Gamma({\bf x},{\bf u},\nabla{\bf u},p,{\bf n},{\bf t}) - \left(\nabla_{\nabla{\bf u}}{\bf P}({\bf x},{\bf u},\nabla{\bf u},\Delta{\bf u},p,\nabla p)\cdot{\bf n}\right)\cdot{\bf v}\right.\\
                &\hspace{1cm} - \partial_{\bf n}\nabla_{\Delta{\bf u}}{\bf P}({\bf x},{\bf u},\nabla{\bf u},\Delta{\bf u},p,\nabla p){\bf v} - \nabla_{\Delta{\bf u}}{\bf P}({\bf x},{\bf u},\nabla{\bf u},\Delta{\bf u},p,\nabla p)\partial_{\bf n}{\bf v} + \left(\nabla_{\nabla{\bf u}}{\bf f}({\bf x},{\bf u},\nabla{\bf u},p)\cdot{\bf n}\right)\cdot{\bf v} + q{\bf n}\\
                &\hspace{1cm} \left.+ q\nabla_{\nabla{\bf u}}f_{\rm div}({\bf x},{\bf u},\nabla{\bf u},p)\cdot{\bf n}\right]\cdot\tilde{\bf u}{\rm d}\Gamma\\
                &\hspace{5mm}+ \int_\Gamma \nabla_{\nabla{\bf u}}J_\Gamma({\bf x},{\bf u},\nabla{\bf u},p,{\bf n},{\bf t}):\nabla\tilde{\bf u} + {\bf v}^\top D_{\Delta{\bf u}}{\bf P}({\bf x},{\bf u},\nabla{\bf u},\Delta{\bf u},p,\nabla p)\partial_{\bf n}\tilde{\bf u}{\rm d}\Gamma = 0,\ \forall({\bf u},p,\Omega,\tilde{\bf u}).
            \end{split}\right.}
    \end{equation}
    
    \begin{remark}
        The last integral equation in \eqref{adjoint general stationary fluid dynamics PDEs} seems ``overdetermined''. Indeed, choose $\tilde{\bf u}$ varying s.t. $\tilde{\bf u}|_\Gamma = {\bf 0}$, then $({\bf v},q)$ satisfies
        \begin{align*}
            \int_\Gamma \nabla_{\nabla{\bf u}}J_\Gamma({\bf x},{\bf u},\nabla{\bf u},p,{\bf n},{\bf t}):\nabla\tilde{\bf u} + {\bf v}^\top D_{\Delta{\bf u}}{\bf P}({\bf x},{\bf u},\nabla{\bf u},\Delta{\bf u},p,\nabla p)\partial_{\bf n}\tilde{\bf u}{\rm d}\Gamma = 0,\ \forall({\bf u},p,\Omega,\tilde{\bf u}) \mbox{ s.t. } \tilde{\bf u}|_\Gamma = {\bf 0}.
        \end{align*}
        The l.h.s. of this equals
        \begin{align*}
            &\int_\Gamma \sum_{i=1}^N\sum_{j=1}^N \partial_{\partial_{x_i}u_j}J_\Gamma({\bf x},{\bf u},\nabla{\bf u},p,{\bf n},{\bf t})\partial_{x_i}\tilde{u}_j + \sum_{i=1}^N\sum_{j=1}^N v_i\partial_{\Delta u_j}P_i({\bf x},{\bf u},\nabla{\bf u},\Delta{\bf u},p,\nabla p)\partial_{\bf n}\tilde{u}_j{\rm d}\Gamma\\
            =&\, \int_\Gamma \sum_{i=1}^N\sum_{j=1}^N \partial_{\partial_{x_i}u_j}J_\Gamma({\bf x},{\bf u},\nabla{\bf u},p,{\bf n},{\bf t})\partial_{x_i}\tilde{u}_j + \sum_{i=1}^N\sum_{j=1}^N v_i\partial_{\Delta u_j}P_i({\bf x},{\bf u},\nabla{\bf u},\Delta{\bf u},p,\nabla p)\sum_{k=1}^N n_k\partial_{x_k}\tilde{u}_j{\rm d}\Gamma\\
            =&\, \int_\Gamma \sum_{i=1}^N\sum_{j=1}^N \partial_{\partial_{x_i}u_j}J_\Gamma({\bf x},{\bf u},\nabla{\bf u},p,{\bf n},{\bf t})\partial_{x_i}\tilde{u}_j + \sum_{k=1}^N\sum_{j=1}^N v_k\partial_{\Delta u_j}P_k({\bf x},{\bf u},\nabla{\bf u},\Delta{\bf u},p,\nabla p)\sum_{i=1}^N n_i\partial_{x_i}\tilde{u}_j{\rm d}\Gamma\\
            =&\, \int_\Gamma \sum_{i=1}^N\sum_{j=1}^N \left[\partial_{\partial_{x_i}u_j}J_\Gamma({\bf x},{\bf u},\nabla{\bf u},p,{\bf n},{\bf t}) + \sum_{k=1}^N v_k\partial_{\Delta u_j}P_k({\bf x},{\bf u},\nabla{\bf u},\Delta{\bf u},p,\nabla p)n_i\right]\partial_{x_i}\tilde{u}_j{\rm d}\Gamma,
        \end{align*}
        hence
        \begin{align*}
            \int_\Gamma \sum_{i=1}^N\sum_{j=1}^N \left[\partial_{\partial_{x_i}u_j}J_\Gamma({\bf x},{\bf u},\nabla{\bf u},p,{\bf n},{\bf t}) + \sum_{k=1}^N v_k\partial_{\Delta u_j}P_k({\bf x},{\bf u},\nabla{\bf u},\Delta{\bf u},p,\nabla p)n_i\right]\partial_{x_i}\tilde{u}_j{\rm d}\Gamma = 0,\ \forall({\bf u},p,\Omega,\tilde{\bf u}) \mbox{ s.t. } \tilde{\bf u}|_\Gamma = {\bf 0}.
        \end{align*}
        This implies that
        \begin{align*}
            \partial_{\partial_{x_i}u_j}J_\Gamma({\bf x},{\bf u},\nabla{\bf u},p,{\bf n},{\bf t}) + \sum_{k=1}^N v_k\partial_{\Delta u_j}P_k({\bf x},{\bf u},\nabla{\bf u},\Delta{\bf u},p,\nabla p)n_i = 0,\ \forall i,j = 1,\ldots,N.
        \end{align*}
        To determine ${\bf v}$, we only need to solve one of the following $N$ linear equations:
        \begin{align*}
            \forall j = 1,\ldots,N,\ j^{\rm th}\mbox{ linear system}:\ 
            \sum_{k=1}^N v_k\partial_{\Delta u_j}P_k({\bf x},{\bf u},\nabla{\bf u},\Delta{\bf u},p,\nabla p)n_i = -\partial_{\partial_{x_i}u_j}J_\Gamma({\bf x},{\bf u},\nabla{\bf u},p,{\bf n},{\bf t}),\ \forall i = 1,\ldots,N.
        \end{align*}
    \end{remark}    
    \item \textbf{Case $\delta_{\mathcal{L}} = 1$.} This means to ``deactivate'' the boundary-condition constraint ${\bf Q}({\bf x},{\bf u},\nabla{\bf u},p,{\bf n},{\bf t}) = {\bf f}_{\rm bc}({\bf x})$ on $\Gamma$, so it will be penalized by the extended Lagrangian $\mathcal{L}$.
    
    Again, to see the general structure, we rewrite \eqref{Euler-Lagrange for general stationary fluid dynamics PDEs} as follows:
    \begin{equation}
        \label{brief extended Euler-Lagrange for general stationary fluid dynamics PDEs}
        \tag{brief-exEuLa-gfld}
        \left.\begin{split}
            &\int_\Omega \boldsymbol{\mathcal{F}}_\Omega^{\tilde{\bf u}}({\bf x},{\bf u},\nabla{\bf u},\Delta{\bf u},p,\nabla p,{\bf v},\nabla{\bf v},\Delta{\bf v},q,\nabla q)\cdot\tilde{\bf u}{\rm d}{\bf x} + \int_\Omega \mathcal{F}_\Omega^{\tilde{p}}({\bf x},{\bf u},\nabla{\bf u},\Delta{\bf u},p,\nabla p,{\bf v},\nabla{\bf v},q)\tilde{p}{\rm d}{\bf x}\\
            &\hspace{5mm}+ \int_\Gamma \boldsymbol{\mathcal{F}}_\Gamma^{\tilde{\bf u}}({\bf x},{\bf u},\nabla{\bf u},\Delta{\bf u},p,\nabla p,{\bf v},\nabla{\bf v},q,{\bf v}_{\rm bc},{\bf n},{\bf t})\cdot\tilde{\bf u}{\rm d}\Gamma + \int_\Gamma \mathcal{F}_\Gamma^{\tilde{p}}({\bf x},{\bf u},\nabla{\bf u},\Delta{\bf u},p,\nabla p,{\bf v},{\bf v}_{\rm bc},{\bf n},{\bf t})\tilde{p}{\rm d}\Gamma\\
            &\hspace{5mm}+ \int_\Gamma \mathcal{F}_\Gamma^{\nabla\tilde{\bf u}}({\bf x},{\bf u},\nabla{\bf u},\Delta{\bf u},p,\nabla p,{\bf v},{\bf v}_{\rm bc},{\bf n},{\bf t},\nabla\tilde{\bf u}){\rm d}\Gamma = 0,\ \forall({\bf u},p,\Omega,\tilde{\bf u},\tilde{p}),
        \end{split}\right.
    \end{equation}
    where
    \begin{align*}
        &\boldsymbol{\mathcal{F}}_\Omega^{\tilde{\bf u}}({\bf x},{\bf u},\nabla{\bf u},\Delta{\bf u},p,\nabla p,{\bf v},\nabla{\bf v},\Delta{\bf v},q,\nabla q)\\
        &\hspace{5mm}\coloneqq\nabla_{\bf u}J_\Omega({\bf x},{\bf u},\nabla{\bf u},p) - \nabla\cdot\left(\nabla_{\nabla{\bf u}}J_\Omega({\bf x},{\bf u},\nabla{\bf u},p)\right) - \nabla_{\bf u}{\bf P}({\bf x},{\bf u},\nabla{\bf u},\Delta{\bf u},p,\nabla p){\bf v} + \nabla\cdot\left(\nabla_{\nabla{\bf u}}{\bf P}({\bf x},{\bf u},\nabla{\bf u},\Delta{\bf u},p,\nabla p)\right)\cdot{\bf v}\\
        &\hspace{1cm}+ \nabla_{\nabla{\bf u}}{\bf P}({\bf x},{\bf u},\nabla{\bf u},\Delta{\bf u},p,\nabla p):\nabla{\bf v} - \Delta\nabla_{\Delta{\bf u}}{\bf P}({\bf x},{\bf u},\nabla{\bf u},\Delta{\bf u},p,\nabla p){\bf v} - \nabla_{\Delta{\bf u}}{\bf P}({\bf x},{\bf u},\nabla{\bf u},\Delta{\bf u},p,\nabla p)\Delta{\bf v}\\
        &\hspace{1cm}+ \nabla_{\bf u}{\bf f}({\bf x},{\bf u},\nabla{\bf u},p){\bf v} - \nabla\cdot\left(\nabla_{\nabla{\bf u}}{\bf f}({\bf x},{\bf u},\nabla{\bf u},p)\right)\cdot{\bf v} - \nabla_{\nabla{\bf u}}{\bf f}({\bf x},{\bf u},\nabla{\bf u},p):\nabla{\bf v} - \nabla q + q\nabla_{\bf u}f_{\rm div}({\bf x},{\bf u},\nabla{\bf u},p)\\
        &\hspace{1cm}- \nabla_{\nabla{\bf u}}f_{\rm div}({\bf x},{\bf u},\nabla{\bf u},p)\cdot\nabla q - q\nabla\cdot\left(\nabla_{\nabla{\bf u}}f_{\rm div}({\bf x},{\bf u},\nabla{\bf u},p)\right),\\
        &\mathcal{F}_\Omega^{\tilde{p}}({\bf x},{\bf u},\nabla{\bf u},\Delta{\bf u},p,\nabla p,{\bf v},\nabla{\bf v},q)\\
        &\hspace{5mm}\coloneqq D_pJ_\Omega({\bf x},{\bf u},\nabla{\bf u},p) - D_p{\bf P}({\bf x},{\bf u},\nabla{\bf u},\Delta{\bf u},p,\nabla p)\cdot{\bf v} + \nabla_{\nabla p}{\bf P}({\bf x},{\bf u},\nabla{\bf u},\Delta{\bf u},p,\nabla p):\nabla{\bf v}\\
        &\hspace{1cm} + \nabla\cdot\left(\nabla_{\nabla p}{\bf P}({\bf x},{\bf u},\nabla{\bf u},\Delta{\bf u},p,\nabla p)\right)\cdot{\bf v} + D_p{\bf f}({\bf x},{\bf u},\nabla{\bf u},p)\cdot{\bf v} + qD_pf_{\rm div}({\bf x},{\bf u},\nabla{\bf u},p),\\
        &\boldsymbol{\mathcal{F}}_\Gamma^{\tilde{\bf u}}({\bf x},{\bf u},\nabla{\bf u},\Delta{\bf u},p,\nabla p,{\bf v},\nabla{\bf v},q,{\bf v}_{\rm bc},{\bf n},{\bf t})\\
        &\hspace{5mm}\coloneqq\nabla_{\nabla{\bf u}}J_\Omega({\bf x},{\bf u},\nabla{\bf u},p)\cdot{\bf n} + \nabla_{\bf u}J_\Gamma({\bf x},{\bf u},\nabla{\bf u},p,{\bf n},{\bf t}) - \left(\nabla_{\nabla{\bf u}}{\bf P}({\bf x},{\bf u},\nabla{\bf u},\Delta{\bf u},p,\nabla p)\cdot{\bf n}\right)\cdot{\bf v}\\
        &\hspace{1cm} - \partial_{\bf n}\nabla_{\Delta{\bf u}}{\bf P}({\bf x},{\bf u},\nabla{\bf u},\Delta{\bf u},p,\nabla p){\bf v} - \nabla_{\Delta{\bf u}}{\bf P}({\bf x},{\bf u},\nabla{\bf u},\Delta{\bf u},p,\nabla p)\partial_{\bf n}{\bf v} + \left(\nabla_{\nabla{\bf u}}{\bf f}({\bf x},{\bf u},\nabla{\bf u},p)\cdot{\bf n}\right)\cdot{\bf v} + q{\bf n}\\
        &\hspace{1cm} + q\nabla_{\nabla{\bf u}}f_{\rm div}({\bf x},{\bf u},\nabla{\bf u},p)\cdot{\bf n} - \nabla_{\bf u}{\bf Q}({\bf x},{\bf u},\nabla{\bf u},p,{\bf n},{\bf t}){\bf v}_{\rm bc},\\
        &\mathcal{F}_\Gamma^{\tilde{p}}({\bf x},{\bf u},\nabla{\bf u},\Delta{\bf u},p,\nabla p,{\bf v},{\bf v}_{\rm bc},{\bf n},{\bf t})\coloneqq D_pJ_\Gamma({\bf x},{\bf u},\nabla{\bf u},p,{\bf n},{\bf t}) - {\bf n}^\top\nabla_{\nabla p}{\bf P}({\bf x},{\bf u},\nabla{\bf u},\Delta{\bf u},p,\nabla p){\bf v} - D_p{\bf Q}({\bf x},{\bf u},\nabla{\bf u},p,{\bf n},{\bf t})\cdot{\bf v}_{\rm bc},\\
        &\mathcal{F}_\Gamma^{\nabla\tilde{\bf u}}({\bf x},{\bf u},\nabla{\bf u},\Delta{\bf u},p,\nabla p,{\bf v},{\bf v}_{\rm bc},{\bf n},{\bf t},\nabla\tilde{\bf u})\\
        &\hspace{5mm}\coloneqq\nabla_{\nabla{\bf u}}J_\Gamma({\bf x},{\bf u},\nabla{\bf u},p,{\bf n},{\bf t}):\nabla\tilde{\bf u} + {\bf v}^\top D_{\Delta{\bf u}}{\bf P}({\bf x},{\bf u},\nabla{\bf u},\Delta{\bf u},p,\nabla p)\partial_{\bf n}\tilde{\bf u} - D_{\nabla{\bf u}}{\bf Q}({\bf x},{\bf u},\nabla{\bf u},p,{\bf n},{\bf t})\nabla\tilde{\bf u}\cdot{\bf v}_{\rm bc},
    \end{align*}
    for all $({\bf x},{\bf u},p,{\bf v},q,{\bf n},{\bf t},\tilde{\bf u},\tilde{p})$.
    
    We now deduce the adjoint equations of \eqref{general stationary fluid dynamics PDEs} from \eqref{brief extended Euler-Lagrange for general stationary fluid dynamics PDEs} as follows:
    \begin{itemize}
        \item Choose $\tilde{\bf u} = {\bf 0}$ in $\overline{\Omega}$, \eqref{brief extended Euler-Lagrange for general stationary fluid dynamics PDEs} then becomes
        \begin{align*}
            \int_\Omega \mathcal{F}_\Omega^{\tilde{p}}({\bf x},{\bf u},\nabla{\bf u},\Delta{\bf u},p,\nabla p,{\bf v},\nabla{\bf v},q)\tilde{p}{\rm d}{\bf x} + \int_\Gamma \mathcal{F}_\Gamma^{\tilde{p}}({\bf x},{\bf u},\nabla{\bf u},\Delta{\bf u},p,\nabla p,{\bf v},{\bf v}_{\rm bc},{\bf n},{\bf t})\tilde{p}{\rm d}\Gamma = 0,\ \forall({\bf u},p,\Omega,\tilde{p}).
        \end{align*}
        Then choose $\tilde{p}$ varying s.t. $\tilde{p}|_\Gamma = 0$, then the last equality yields
        \begin{align*}
            \int_\Omega \mathcal{F}_\Omega^{\tilde{p}}({\bf x},{\bf u},\nabla{\bf u},\Delta{\bf u},p,\nabla p,{\bf v},\nabla{\bf v},q)\tilde{p}{\rm d}{\bf x} = 0,\ \forall({\bf u},p,\Omega,\tilde{p}) \mbox{ s.t. } \tilde{p}|_\Gamma = 0.
        \end{align*}
        Hence, $({\bf v},q)$ satisfies
        \begin{align}
            \label{extended domain integrand variation p}
            \boxed{\mathcal{F}_\Omega^{\tilde{p}}({\bf x},{\bf u},\nabla{\bf u},\Delta{\bf u},p,\nabla p,{\bf v},\nabla{\bf v},q) = 0 \mbox{ in } \Omega.}
        \end{align}
        Plug it back in, obtain
        \begin{align*}
            \int_\Gamma \mathcal{F}_\Gamma^{\tilde{p}}({\bf x},{\bf u},\nabla{\bf u},\Delta{\bf u},p,\nabla p,{\bf v},{\bf v}_{\rm bc},{\bf n},{\bf t})\tilde{p}{\rm d}\Gamma = 0,\ \forall({\bf u},p,\Omega,\tilde{p}).
        \end{align*}
        Unlike the previous case, $\tilde{p}$ can vary on the whole of $\Gamma$, hence the last equality implies that $({\bf v},q)$ satisfies
        \begin{align}
            \label{extended boundary integrand variation p}
            \boxed{\mathcal{F}_\Gamma^{\tilde{p}}({\bf x},{\bf u},\nabla{\bf u},\Delta{\bf u},p,\nabla p,{\bf v},{\bf v}_{\rm bc},{\bf n},{\bf t}) = 0 \mbox{ on } \Gamma.}
        \end{align}
        \item Assume that $({\bf v},q)$ satisfies \eqref{extended domain integrand variation p} and \eqref{extended boundary integrand variation p}, \eqref{brief extended Euler-Lagrange for general stationary fluid dynamics PDEs} then becomes
        \begin{align*}
            &\int_\Omega \boldsymbol{\mathcal{F}}_\Omega^{\tilde{\bf u}}({\bf x},{\bf u},\nabla{\bf u},\Delta{\bf u},p,\nabla p,{\bf v},\nabla{\bf v},\Delta{\bf v},q,\nabla q)\cdot\tilde{\bf u}{\rm d}{\bf x} + \int_\Gamma \boldsymbol{\mathcal{F}}_\Gamma^{\tilde{\bf u}}({\bf x},{\bf u},\nabla{\bf u},\Delta{\bf u},p,\nabla p,{\bf v},\nabla{\bf v},q,{\bf v}_{\rm bc},{\bf n},{\bf t})\cdot\tilde{\bf u}{\rm d}\Gamma\\
            &\hspace{5mm}+ \int_\Gamma \mathcal{F}_\Gamma^{\nabla\tilde{\bf u}}({\bf x},{\bf u},\nabla{\bf u},\Delta{\bf u},p,\nabla p,{\bf v},{\bf v}_{\rm bc},{\bf n},{\bf t},\nabla\tilde{\bf u}){\rm d}\Gamma = 0,\ \forall({\bf u},p,\Omega,\tilde{\bf u}).
        \end{align*}
        Choose $\tilde{\bf u}$ varying s.t. $\tilde{\bf u}|_\Gamma = {\bf 0}$ and $\nabla\tilde{\bf u}|_\Gamma = {\bf 0}_{N\times N}$, the last equality yields
        \begin{align*}
            \int_\Omega \boldsymbol{\mathcal{F}}_\Omega^{\tilde{\bf u}}({\bf x},{\bf u},\nabla{\bf u},\Delta{\bf u},p,\nabla p,{\bf v},\nabla{\bf v},\Delta{\bf v},q,\nabla q)\cdot\tilde{\bf u}{\rm d}{\bf x} = 0,\ \forall({\bf u},p,\Omega,\tilde{\bf u}) \mbox{ s.t. } \tilde{\bf u}|_\Gamma = {\bf 0} \mbox{ and } \nabla\tilde{\bf u}|_\Gamma = {\bf 0}_{N\times N}.
        \end{align*}
        Hence, $({\bf v},q)$ satisfies
        \begin{align*}
            \boxed{\boldsymbol{\mathcal{F}}_\Omega^{\tilde{\bf u}}({\bf x},{\bf u},\nabla{\bf u},\Delta{\bf u},p,\nabla p,{\bf v},\nabla{\bf v},\Delta{\bf v},q,\nabla q) = {\bf 0} \mbox{ in } \Omega.}
        \end{align*}
        Plug it back in, obtain
        \begin{align*}
            \int_\Gamma \boldsymbol{\mathcal{F}}_\Gamma^{\tilde{\bf u}}({\bf x},{\bf u},\nabla{\bf u},\Delta{\bf u},p,\nabla p,{\bf v},\nabla{\bf v},q,{\bf v}_{\rm bc},{\bf n},{\bf t})\cdot\tilde{\bf u}{\rm d}\Gamma + \int_\Gamma \mathcal{F}_\Gamma^{\nabla\tilde{\bf u}}({\bf x},{\bf u},\nabla{\bf u},\Delta{\bf u},p,\nabla p,{\bf v},{\bf v}_{\rm bc},{\bf n},{\bf t},\nabla\tilde{\bf u}){\rm d}\Gamma = 0,\ \forall({\bf u},p,\Omega,\tilde{\bf u}),
        \end{align*}
        Choose $\tilde{\bf u}$ varying s.t. $\tilde{\bf u}|_\Gamma = {\bf 0}$, the last equality then becomes
        \begin{align*}
            \int_\Gamma \mathcal{F}_\Gamma^{\nabla\tilde{\bf u}}({\bf x},{\bf u},\nabla{\bf u},\Delta{\bf u},p,\nabla p,{\bf v},{\bf v}_{\rm bc},{\bf n},{\bf t},\nabla\tilde{\bf u}){\rm d}\Gamma = 0,\ \forall({\bf u},p,\Omega,\tilde{\bf u}) \mbox{ s.t. } \tilde{\bf u}|_\Gamma = {\bf 0}.
        \end{align*}    
        The l.h.s. equals
        \begin{align*}
            &\int_\Gamma \mathcal{F}_\Gamma^{\nabla\tilde{\bf u}}({\bf x},{\bf u},\nabla{\bf u},\Delta{\bf u},p,\nabla p,{\bf v},{\bf v}_{\rm bc},{\bf n},{\bf t},\nabla\tilde{\bf u}){\rm d}\Gamma\\
            =&\, \int_\Gamma \nabla_{\nabla{\bf u}}J_\Gamma({\bf x},{\bf u},\nabla{\bf u},p,{\bf n},{\bf t}):\nabla\tilde{\bf u} + {\bf v}^\top D_{\Delta{\bf u}}{\bf P}({\bf x},{\bf u},\nabla{\bf u},\Delta{\bf u},p,\nabla p)\partial_{\bf n}\tilde{\bf u} - D_{\nabla{\bf u}}{\bf Q}({\bf x},{\bf u},\nabla{\bf u},p,{\bf n},{\bf t})\nabla\tilde{\bf u}\cdot{\bf v}_{\rm bc}{\rm d}\Gamma\\
            =&\, \int_\Gamma \sum_{i=1}^N\sum_{j=1}^N \partial_{\partial_{x_i}u_j}J_\Gamma({\bf x},{\bf u},\nabla{\bf u},p,{\bf n},{\bf t})\partial_{x_i}\tilde{u}_j + \sum_{i=1}^N\sum_{j=1}^N v_i\partial_{\Delta u_j}P_i({\bf x},{\bf u},\nabla{\bf u},\Delta{\bf u},p,\nabla p)\partial_{\bf n}\tilde{u}_j\\
            &\hspace{1cm}- \sum_{k=1}^N\sum_{i=1}^N\sum_{j=1}^N \partial_{\partial_{x_i}u_j}Q_k({\bf x},{\bf u},\nabla{\bf u},p,{\bf n},{\bf t})\partial_{x_i}\tilde{u}_jv_{{\rm bc},k}{\rm d}\Gamma\\
            =&\, \int_\Gamma \sum_{i=1}^N\sum_{j=1}^N \partial_{\partial_{x_i}u_j}J_\Gamma({\bf x},{\bf u},\nabla{\bf u},p,{\bf n},{\bf t})\partial_{x_i}\tilde{u}_j + \sum_{i=1}^N\sum_{j=1}^N v_i\partial_{\Delta u_j}P_i({\bf x},{\bf u},\nabla{\bf u},\Delta{\bf u},p,\nabla p)\sum_{k=1}^N n_k\partial_{x_k}\tilde{u}_j\\
            &\hspace{1cm}- \sum_{i=1}^N\sum_{j=1}^N \partial_{x_i}\tilde{u}_j\sum_{k=1}^N\partial_{\partial_{x_i}u_j}Q_k({\bf x},{\bf u},\nabla{\bf u},p,{\bf n},{\bf t})v_{{\rm bc},k}{\rm d}\Gamma\\
            =&\, \int_\Gamma \sum_{i=1}^N\sum_{j=1}^N \partial_{\partial_{x_i}u_j}J_\Gamma({\bf x},{\bf u},\nabla{\bf u},p,{\bf n},{\bf t})\partial_{x_i}\tilde{u}_j + \sum_{k=1}^N\sum_{j=1}^N v_k\partial_{\Delta u_j}P_k({\bf x},{\bf u},\nabla{\bf u},\Delta{\bf u},p,\nabla p)\sum_{i=1}^N n_i\partial_{x_i}\tilde{u}_j\\
            &\hspace{1cm}- \sum_{i=1}^N\sum_{j=1}^N \partial_{x_i}\tilde{u}_j\sum_{k=1}^N\partial_{\partial_{x_i}u_j}Q_k({\bf x},{\bf u},\nabla{\bf u},p,{\bf n},{\bf t})v_{{\rm bc},k}{\rm d}\Gamma\\
            =&\, \int_\Gamma \sum_{i=1}^N\sum_{j=1}^N \left[\partial_{\partial_{x_i}u_j}J_\Gamma({\bf x},{\bf u},\nabla{\bf u},p,{\bf n},{\bf t}) + \sum_{k=1}^N v_k\partial_{\Delta u_j}P_k({\bf x},{\bf u},\nabla{\bf u},\Delta{\bf u},p,\nabla p)n_i\right.\\
            &\hspace{2cm}\left. - \sum_{k=1}^N \partial_{\partial_{x_i}u_j}Q_k({\bf x},{\bf u},\nabla{\bf u},p,{\bf n},{\bf t})v_{{\rm bc},k}\right]\partial_{x_i}\tilde{u}_j{\rm d}\Gamma,
        \end{align*}
        hence
        \begin{align*}
            &\int_\Gamma \sum_{i=1}^N\sum_{j=1}^N \left[\partial_{\partial_{x_i}u_j}J_\Gamma({\bf x},{\bf u},\nabla{\bf u},p,{\bf n},{\bf t}) + \sum_{k=1}^N v_k\partial_{\Delta u_j}P_k({\bf x},{\bf u},\nabla{\bf u},\Delta{\bf u},p,\nabla p)n_i\right.\\
            &\hspace{2cm}\left. - \sum_{k=1}^N \partial_{\partial_{x_i}u_j}Q_k({\bf x},{\bf u},\nabla{\bf u},p,{\bf n},{\bf t})v_{{\rm bc},k}\right]\partial_{x_i}\tilde{u}_j{\rm d}\Gamma = 0,\ \forall({\bf u},p,\Omega,\tilde{u}) \mbox{ s.t. } \tilde{\bf u}|_\Gamma = {\bf 0}.
        \end{align*}
        This implies that
        \begin{align*}
            \partial_{\partial_{x_i}u_j}J_\Gamma({\bf x},{\bf u},\nabla{\bf u},p,{\bf n},{\bf t}) + \sum_{k=1}^N v_k\partial_{\Delta u_j}P_k({\bf x},{\bf u},\nabla{\bf u},\Delta{\bf u},p,\nabla p)n_i - \sum_{k=1}^N \partial_{\partial_{x_i}u_j}Q_k({\bf x},{\bf u},\nabla{\bf u},p,{\bf n},{\bf t})v_{{\rm bc},k} = 0,\ \forall i,j = 1,\ldots,N,
        \end{align*}
        or equivalently,
        \begin{align*}
            \boxed{\sum_{k=1}^N v_k\partial_{\Delta u_j}P_k({\bf x},{\bf u},\nabla{\bf u},\Delta{\bf u},p,\nabla p)n_i - \partial_{\partial_{x_i}u_j}Q_k({\bf x},{\bf u},\nabla{\bf u},p,{\bf n},{\bf t})v_{{\rm bc},k} = -\partial_{\partial_{x_i}u_j}J_\Gamma({\bf x},{\bf u},\nabla{\bf u},p,{\bf n},{\bf t}),\ \forall i,j = 1,\ldots,N.}
        \end{align*}
        Assume that the last equality holds, then plug it back in to obtain:
        \begin{align*}
            \int_\Gamma \boldsymbol{\mathcal{F}}_\Gamma^{\tilde{\bf u}}({\bf x},{\bf u},\nabla{\bf u},\Delta{\bf u},p,\nabla p,{\bf v},\nabla{\bf v},q,{\bf v}_{\rm bc},{\bf n},{\bf t})\cdot\tilde{\bf u}{\rm d}\Gamma = 0,\ \forall({\bf u},p,\Omega,\tilde{\bf u}),
        \end{align*}
        hence $({\bf v},q,{\bf v}_{\rm bc})$ satisfies
        \begin{align*}
            \boxed{\boldsymbol{\mathcal{F}}_\Gamma^{\tilde{\bf u}}({\bf x},{\bf u},\nabla{\bf u},\Delta{\bf u},p,\nabla p,{\bf v},\nabla{\bf v},q,{\bf v}_{\rm bc},{\bf n},{\bf t}) = {\bf 0} \mbox{ on } \Gamma.}
        \end{align*}
    \end{itemize}
    Conclude:
    \begin{equation}
        \label{extended adjoint general stationary fluid dynamics PDEs}
        \tag{ex-adj-gfld}
        \boxed{\left\{\begin{split}
                &-\nabla_{\Delta{\bf u}}{\bf P}({\bf x},{\bf u},\nabla{\bf u},\Delta{\bf u},p,\nabla p)\Delta{\bf v} + \left(\nabla_{\nabla{\bf u}}{\bf P}({\bf x},{\bf u},\nabla{\bf u},\Delta{\bf u},p,\nabla p) - \nabla_{\nabla{\bf u}}{\bf f}({\bf x},{\bf u},\nabla{\bf u},p)\right):\nabla{\bf v}\\
                &\hspace{1cm}- \left[\nabla_{\bf u}{\bf P}({\bf x},{\bf u},\nabla{\bf u},\Delta{\bf u},p,\nabla p) + \Delta\nabla_{\Delta{\bf u}}{\bf P}({\bf x},{\bf u},\nabla{\bf u},\Delta{\bf u},p,\nabla p) - \nabla_{\bf u}{\bf f}({\bf x},{\bf u},\nabla{\bf u},p)\right]{\bf v}\\
                &\hspace{1cm}+ \left[\nabla\cdot\left(\nabla_{\nabla{\bf u}}{\bf P}({\bf x},{\bf u},\nabla{\bf u},\Delta{\bf u},p,\nabla p)\right) - \nabla\cdot\left(\nabla_{\nabla{\bf u}}{\bf f}({\bf x},{\bf u},\nabla{\bf u},p)\right)\right]\cdot{\bf v} - \nabla q - \nabla_{\nabla{\bf u}}f_{\rm div}({\bf x},{\bf u},\nabla{\bf u},p)\cdot\nabla q\\
                &\hspace{1cm}+ q\left[-\nabla\cdot\left(\nabla_{\nabla{\bf u}}f_{\rm div}({\bf x},{\bf u},\nabla{\bf u},p)\right) + \nabla_{\bf u}f_{\rm div}({\bf x},{\bf u},\nabla{\bf u},p)\right] = \nabla\cdot\left(\nabla_{\nabla{\bf u}}J_\Omega({\bf x},{\bf u},\nabla{\bf u},p)\right) - \nabla_{\bf u}J_\Omega({\bf x},{\bf u},\nabla{\bf u},p) \mbox{ in } \Omega,\\
                &\nabla_{\nabla p}{\bf P}({\bf x},{\bf u},\nabla{\bf u},\Delta{\bf u},p,\nabla p):\nabla{\bf v} + \left[-D_p{\bf P}({\bf x},{\bf u},\nabla{\bf u},\Delta{\bf u},p,\nabla p) + \nabla\cdot\left(\nabla_{\nabla p}{\bf P}({\bf x},{\bf u},\nabla{\bf u},\Delta{\bf u},p,\nabla p)\right) + D_p{\bf f}({\bf x},{\bf u},\nabla{\bf u},p)\right]\cdot{\bf v}\\
                &\hspace{1cm} = - D_pJ_\Omega ({\bf x},{\bf u},\nabla{\bf u},p) - qD_pf_{\rm div}({\bf x},{\bf u},\nabla{\bf u},p) \mbox{ in } \Omega,\\
                &- \nabla_{\Delta{\bf u}}{\bf P}({\bf x},{\bf u},\nabla{\bf u},\Delta{\bf u},p,\nabla p)\partial_{\bf n}{\bf v} + \left[\left(-\nabla_{\nabla{\bf u}}{\bf P}({\bf x},{\bf u},\nabla{\bf u},\Delta{\bf u},p,\nabla p) + \nabla_{\nabla{\bf u}}{\bf f}({\bf x},{\bf u},\nabla{\bf u},p)\right)\cdot{\bf n}\right]\cdot{\bf v}\\
                &\hspace{1cm}- \partial_{\bf n}\nabla_{\Delta{\bf u}}{\bf P}({\bf x},{\bf u},\nabla{\bf u},\Delta{\bf u},p,\nabla p){\bf v} + q{\bf n} - \nabla_{\bf u}{\bf Q}({\bf x},{\bf u},\nabla{\bf u},p,{\bf n},{\bf t}){\bf v}_{\rm bc}\\
                &\hspace{1cm}= - \nabla_{\nabla{\bf u}}J_\Omega({\bf x},{\bf u},\nabla{\bf u},p)\cdot{\bf n} - \nabla_{\bf u}J_\Gamma({\bf x},{\bf u},\nabla{\bf u},p,{\bf n},{\bf t}) - q\nabla_{\nabla{\bf u}}f_{\rm div}({\bf x},{\bf u},\nabla{\bf u},p)\cdot{\bf n} \mbox{ on } \Gamma,\\
                &{\bf n}^\top\nabla_{\nabla p}{\bf P}({\bf x},{\bf u},\nabla{\bf u},\Delta{\bf u},p,\nabla p){\bf v} + D_p{\bf Q}({\bf x},{\bf u},\nabla{\bf u},p,{\bf n},{\bf t})\cdot{\bf v}_{\rm bc} = D_pJ_\Gamma({\bf x},{\bf u},\nabla{\bf u},p,{\bf n},{\bf t}) \mbox{ on } \Gamma,\\
                &\sum_{k=1}^N v_k\partial_{\Delta u_j}P_k({\bf x},{\bf u},\nabla{\bf u},\Delta{\bf u},p,\nabla p)n_i - \partial_{\partial_{x_i}u_j}Q_k({\bf x},{\bf u},\nabla{\bf u},p,{\bf n},{\bf t})v_{{\rm bc},k} = -\partial_{\partial_{x_i}u_j}J_\Gamma({\bf x},{\bf u},\nabla{\bf u},p,{\bf n},{\bf t}),\ \forall i,j = 1,\ldots,N.
            \end{split}\right.}
    \end{equation}
\end{itemize}

\begin{remark}
    The last equation in \eqref{extended adjoint general stationary fluid dynamics PDEs} can be simplified further when the explicit form of ${\bf P}$ and ${\bf Q}$ are given.
\end{remark}

\subsection{Weak formulation of adjoint equations of \eqref{general stationary fluid dynamics PDEs}*}
Test the 1st 2 equations of \eqref{adjoint general stationary fluid dynamics PDEs} with ${\bf w}$ and $r$, respectively, over $\Omega$:
\begin{equation*}
    \left\{\begin{split}
        &\int_\Omega {\color{red}\nabla_{\Delta{\bf u}}{\bf P}({\bf x},{\bf u},\nabla{\bf u},\Delta{\bf u},p,\nabla p)\Delta{\bf v}\cdot{\bf w}} + \left(\nabla_{\nabla{\bf u}}{\bf P}({\bf x},{\bf u},\nabla{\bf u},\Delta{\bf u},p,\nabla p) - \nabla_{\nabla{\bf u}}{\bf f}({\bf x},{\bf u},\nabla{\bf u},p)\right):\nabla{\bf v}\cdot{\bf w}\\
        &\hspace{1cm}- \left[\nabla_{\bf u}{\bf P}({\bf x},{\bf u},\nabla{\bf u},\Delta{\bf u},p,\nabla p) + \Delta\nabla_{\Delta{\bf u}}{\bf P}({\bf x},{\bf u},\nabla{\bf u},\Delta{\bf u},p,\nabla p) - \nabla_{\bf u}{\bf f}({\bf x},{\bf u},\nabla{\bf u},p)\right]{\bf v}\cdot{\bf w}\\
        &\hspace{1cm}+ \left[\left[\nabla\cdot\left(\nabla_{\nabla{\bf u}}{\bf P}({\bf x},{\bf u},\nabla{\bf u},\Delta{\bf u},p,\nabla p)\right) - \nabla\cdot\left(\nabla_{\nabla{\bf u}}{\bf f}({\bf x},{\bf u},\nabla{\bf u},p)\right)\right]\cdot{\bf v}\right]\cdot{\bf w} - {\color{red}\nabla q\cdot{\bf w}} + {\color{red}\left(\nabla_{\nabla{\bf u}}f_{\rm div}({\bf x},{\bf u},\nabla{\bf u},p)\cdot\nabla q\right)\cdot{\bf w}}\\
        &\hspace{1cm}+ q\left[\nabla\cdot\left(\nabla_{\nabla{\bf u}}f_{\rm div}({\bf x},{\bf u},\nabla{\bf u},p)\right) - \nabla_{\bf u}f_{\rm div}({\bf x},{\bf u},\nabla{\bf u},p)\right]\cdot{\bf w}{\rm d}{\bf x}\\
        &\hspace{5mm} = \int_\Omega \nabla\cdot\left(\nabla_{\nabla{\bf u}}J_\Omega({\bf x},{\bf u},\nabla{\bf u},p)\right)\cdot{\bf w} - \nabla_{\bf u}J_\Omega({\bf x},{\bf u},\nabla{\bf u},p)\cdot{\bf w}{\rm d}{\bf x},\\
        &\int_\Omega r\nabla_{\nabla p}{\bf P}({\bf x},{\bf u},\nabla{\bf u},\Delta{\bf u},p,\nabla p):\nabla{\bf v}\\
        &\hspace{1cm} + r\left[-D_p{\bf P}({\bf x},{\bf u},\nabla{\bf u},\Delta{\bf u},p,\nabla p) + \nabla\cdot\left(\nabla_{\nabla p}{\bf P}({\bf x},{\bf u},\nabla{\bf u},\Delta{\bf u},p,\nabla p)\right) + D_p{\bf f}({\bf x},{\bf u},\nabla{\bf u},p)\right]\cdot{\bf v}{\rm d}{\bf x}\\
        &\hspace{5mm} = \int_\Omega - rD_pJ_\Omega ({\bf x},{\bf u},\nabla{\bf u},p) + qrD_pf_{\rm div}({\bf x},{\bf u},\nabla{\bf u},p){\rm d}{\bf x}.
    \end{split}\right.
\end{equation*}
Integrate by parts:
\begin{enumerate}[leftmargin=0in]
    \item Term $\nabla_{\Delta{\bf u}}{\bf P}({\bf x},{\bf u},\nabla{\bf u},\Delta{\bf u},p,\nabla p)\Delta{\bf v}\cdot{\bf w}$:
    \begin{align*}
        &\int_\Omega -\nabla_{\Delta{\bf u}}{\bf P}({\bf x},{\bf u},\nabla{\bf u},\Delta{\bf u},p,\nabla p)\Delta{\bf v}\cdot{\bf w}{\rm d}{\bf x} = \int_\Omega \left(\sum_{j=1}^N \partial_{\Delta u_i}P_j({\bf x},{\bf u},\nabla{\bf u},\Delta{\bf u},p,\nabla p)\Delta v_j\right)_{i=1}^N\cdot{\bf w}{\rm d}{\bf x}\\
        =&\, \int_\Omega \sum_{i=1}^N\sum_{j=1}^N w_i\partial_{\Delta u_i}P_j({\bf x},{\bf u},\nabla{\bf u},\Delta{\bf u},p,\nabla p)\Delta v_j{\rm d}{\bf x} = \sum_{i=1}^N\sum_{j=1}^N \int_\Omega w_i\partial_{\Delta u_i}P_j({\bf x},{\bf u},\nabla{\bf u},\Delta{\bf u},p,\nabla p)\Delta v_j{\rm d}{\bf x}\\
        =&\, \sum_{i=1}^N\sum_{j=1}^N -\int_\Omega \nabla v_j\cdot\nabla w_i\partial_{\Delta u_i}P_j({\bf x},{\bf u},\nabla{\bf u},\Delta{\bf u},p,\nabla p) + \nabla v_j\cdot\nabla\partial_{\Delta u_i}P_j({\bf x},{\bf u},\nabla{\bf u},\Delta{\bf u},p,\nabla p)w_i{\rm d}{\bf x}\\
        &\hspace{15mm}+ \int_\Gamma \partial_{\bf n}v_jw_i\partial_{\Delta u_i}P_j({\bf x},{\bf u},\nabla{\bf u},\Delta{\bf u},p,\nabla p){\rm d}\Gamma\\
        =&\, -\int_\Omega \sum_{j=1}^N \nabla v_j\cdot\sum_{i=1}^N \nabla w_i\partial_{\Delta u_i}P_j({\bf x},{\bf u},\nabla{\bf u},\Delta{\bf u},p,\nabla p) + \sum_{j=1}^N \nabla v_j\cdot\sum_{i=1}^N w_i\nabla\partial_{\Delta u_i}P_j({\bf x},{\bf u},\nabla{\bf u},\Delta{\bf u},p,\nabla p){\rm d}{\bf x}\\
        &+ \int_\Gamma \sum_{i=1}^N\sum_{j=1}^N w_i\partial_{\Delta u_i}P_j({\bf x},{\bf u},\nabla{\bf u},\Delta{\bf u},p,\nabla p)\partial_{\bf n}v_j{\rm d}\Gamma\\
        =&\, -\int_\Omega \sum_{j=1}^N \nabla v_j\cdot\left(\nabla_{\Delta{\bf u}}P_j({\bf x},{\bf u},\nabla{\bf u},\Delta{\bf u},p,\nabla p)\cdot\nabla{\bf w}\right) + \sum_{j=1}^N \nabla v_j\cdot\left(\nabla\nabla_{\Delta{\bf u}}P_j({\bf x},{\bf u},\nabla{\bf u},\Delta{\bf u},p,\nabla p)\cdot{\bf w}\right){\rm d}{\bf x}\\
        &+ \int_\Gamma {\bf w}^\top\nabla_{\Delta{\bf u}}{\bf P}({\bf x},{\bf u},\nabla{\bf u},\Delta{\bf u},p,\nabla p)\partial_{\bf n}{\bf v}{\rm d}\Gamma\\
        =&\, -\int_\Omega \nabla{\bf v}:\left(\nabla_{\Delta{\bf u}}{\bf P}({\bf x},{\bf u},\nabla{\bf u},\Delta{\bf u},p,\nabla p)\cdot\nabla{\bf w}\right) + \nabla{\bf v}:\left(\nabla\nabla_{\Delta{\bf u}}{\bf P}({\bf x},{\bf u},\nabla{\bf u},\Delta{\bf u},p,\nabla p)\cdot{\bf w}\right){\rm d}{\bf x}\\
        &+ \int_\Gamma {\bf w}^\top\nabla_{\Delta{\bf u}}{\bf P}({\bf x},{\bf u},\nabla{\bf u},\Delta{\bf u},p,\nabla p)\partial_{\bf n}{\bf v}{\rm d}\Gamma.
    \end{align*}
    \item Term $-\nabla q\cdot{\bf w}$:
    \begin{align*}
        -\int_\Omega \nabla q\cdot{\bf w}{\rm d}{\bf x} = \int_\Omega q\nabla\cdot{\bf w}{\rm d}{\bf x} - \int_\Gamma q{\bf w}\cdot{\bf n}{\rm d}\Gamma.
    \end{align*}
    \item Term $\left(\nabla_{\nabla{\bf u}}f_{\rm div}({\bf x},{\bf u},\nabla{\bf u},p)\cdot\nabla q\right)\cdot{\bf w}$:
    \begin{align*}
        &\int_\Omega \left(\nabla_{\nabla{\bf u}}f_{\rm div}({\bf x},{\bf u},\nabla{\bf u},p)\cdot\nabla q\right)\cdot{\bf w}{\rm d}{\bf x} = \int_\Omega \nabla^\top q\nabla_{\nabla{\bf u}}f_{\rm div}({\bf x},{\bf u},\nabla{\bf u},p){\bf w}{\rm d}{\bf x}\\
        =&\, \int_\Omega \sum_{i=1}^N\sum_{j=1}^N \partial_{x_i}q\partial_{\partial_{x_i}u_j}f_{\rm div}({\bf x},{\bf u},\nabla{\bf u},p)w_j{\rm d}{\bf x} = \sum_{i=1}^N\sum_{j=1}^N \int_\Omega \partial_{x_i}q\partial_{\partial_{x_i}u_j}f_{\rm div}({\bf x},{\bf u},\nabla{\bf u},p)w_j{\rm d}{\bf x}\\
        =&\, \sum_{i=1}^N\sum_{j=1}^N -\int_\Omega q\partial_{x_i}\partial_{\partial_{x_i}u_j}f_{\rm div}({\bf x},{\bf u},\nabla{\bf u},p)w_j + q\partial_{\partial_{x_i}u_j}f_{\rm div}({\bf x},{\bf u},\nabla{\bf u},p)\partial_{x_i}w_j{\rm d}{\bf x} + \int_\Gamma qn_i\partial_{\partial_{x_i}u_j}f_{\rm div}({\bf x},{\bf u},\nabla{\bf u},p)w_j{\rm d}\Gamma\\
        =&\, -\int_\Omega q\sum_{j=1}^N w_j\sum_{i=1}^N \partial_{x_i}\partial_{\partial_{x_i}u_j}f_{\rm div}({\bf x},{\bf u},\nabla{\bf u},p) + q\sum_{i=1}^N\sum_{j=1}^N \partial_{\partial_{x_i}u_j}f_{\rm div}({\bf x},{\bf u},\nabla{\bf u},p)\partial_{x_i}w_j{\rm d}{\bf x}\\
        &+ \int_\Gamma q\sum_{i=1}^N\sum_{j=1}^N n_i\partial_{\partial_{x_i}u_j}f_{\rm div}({\bf x},{\bf u},\nabla{\bf u},p)w_j{\rm d}\Gamma\\
        =&\, -\int_\Omega q\sum_{j=1}^N w_j\nabla\cdot\left(\nabla_{\nabla u_j}f_{\rm div}({\bf x},{\bf u},\nabla{\bf u},p)\right) + q\nabla_{\nabla{\bf u}}f_{\rm div}({\bf x},{\bf u},\nabla{\bf u},p):\nabla{\bf w}{\rm d}{\bf x} + \int_\Gamma q{\bf n}^\top\nabla_{\nabla{\bf u}}f_{\rm div}({\bf x},{\bf u},\nabla{\bf u},p){\bf w}{\rm d}\Gamma\\
        =&\, -\int_\Omega q\nabla\cdot\left(\nabla_{\nabla{\bf u}}f_{\rm div}({\bf x},{\bf u},\nabla{\bf u},p)\right)\cdot{\bf w} + q\nabla_{\nabla{\bf u}}f_{\rm div}({\bf x},{\bf u},\nabla{\bf u},p):\nabla{\bf w}{\rm d}{\bf x} + \int_\Gamma q{\bf n}^\top\nabla_{\nabla{\bf u}}f_{\rm div}({\bf x},{\bf u},\nabla{\bf u},p){\bf w}{\rm d}\Gamma.
    \end{align*}
\end{enumerate}
Plug in to obtain:
\begin{align*}
    &\int_\Omega -\nabla{\bf v}:\left(\nabla_{\Delta{\bf u}}{\bf P}({\bf x},{\bf u},\nabla{\bf u},\Delta{\bf u},p,\nabla p)\cdot\nabla{\bf w}\right) + \nabla{\bf v}:\left(\nabla\nabla_{\Delta{\bf u}}{\bf P}({\bf x},{\bf u},\nabla{\bf u},\Delta{\bf u},p,\nabla p)\cdot{\bf w}\right)\\
    &\hspace{1cm} + \left(\nabla_{\nabla{\bf u}}{\bf P}({\bf x},{\bf u},\nabla{\bf u},\Delta{\bf u},p,\nabla p) + \nabla_{\nabla{\bf u}}{\bf f}({\bf x},{\bf u},\nabla{\bf u},p)\right):\nabla{\bf v}\cdot{\bf w}\\
    &\hspace{1cm} - \left[\nabla_{\bf u}{\bf P}({\bf x},{\bf u},\nabla{\bf u},\Delta{\bf u},p,\nabla p) + \Delta\nabla_{\Delta{\bf u}}{\bf P}({\bf x},{\bf u},\nabla{\bf u},\Delta{\bf u},p,\nabla p) - \nabla_{\bf u}{\bf f}({\bf x},{\bf u},\nabla{\bf u},p)\right]{\bf v}\cdot{\bf w}\\
    &\hspace{1cm}+ \left[\left[\nabla\cdot\left(\nabla_{\nabla{\bf u}}{\bf P}({\bf x},{\bf u},\nabla{\bf u},\Delta{\bf u},p,\nabla p)\right) - \nabla\cdot\left(\nabla_{\nabla{\bf u}}{\bf f}({\bf x},{\bf u},\nabla{\bf u},p)\right)\right]\cdot{\bf v}\right]\cdot{\bf w} + q\nabla\cdot{\bf w} - q\nabla\cdot\left(\nabla_{\nabla{\bf u}}f_{\rm div}({\bf x},{\bf u},\nabla{\bf u},p)\right)\cdot{\bf w}\\
    &\hspace{1cm}+ q\nabla_{\nabla{\bf u}}f_{\rm div}({\bf x},{\bf u},\nabla{\bf u},p):\nabla{\bf w} + q\left[\nabla\cdot\left(\nabla_{\nabla{\bf u}}f_{\rm div}({\bf x},{\bf u},\nabla{\bf u},p)\right) - \nabla_{\bf u}f_{\rm div}({\bf x},{\bf u},\nabla{\bf u},p)\right]\cdot{\bf w}{\rm d}{\bf x}\\
    &+ \int_\Gamma {\bf w}^\top\nabla_{\Delta{\bf u}}{\bf P}({\bf x},{\bf u},\nabla{\bf u},\Delta{\bf u},p,\nabla p)\partial_{\bf n}{\bf v} - q{\bf w}\cdot{\bf n} + q{\bf n}^\top\nabla_{\nabla{\bf u}}f_{\rm div}({\bf x},{\bf u},\nabla{\bf u},p){\bf w}{\rm d}\Gamma\\
    &\hspace{1cm} = \int_\Omega \nabla\cdot\left(\nabla_{\nabla{\bf u}}J_\Omega({\bf x},{\bf u},\nabla{\bf u},p)\right)\cdot{\bf w} - \nabla_{\bf u}J_\Omega({\bf x},{\bf u},\nabla{\bf u},p)\cdot{\bf w}{\rm d}{\bf x}.
\end{align*}
Note that
\begin{align*}
    &\int_{\Gamma\backslash\Gamma_{\rm D}^{\bf u}} {\bf w}^\top\nabla_{\Delta{\bf u}}{\bf P}({\bf x},{\bf u},\nabla{\bf u},\Delta{\bf u},p,\nabla p)\partial_{\bf n}{\bf v} - q{\bf w}\cdot{\bf n} + q{\bf n}^\top\nabla_{\nabla{\bf u}}f_{\rm div}({\bf x},{\bf u},\nabla{\bf u},p){\bf w}{\rm d}\Gamma\\
    =&\, \int_{\Gamma\backslash\Gamma_{\rm D}^{\bf u}} \left[\nabla_{\nabla{\bf u}}J_\Omega({\bf x},{\bf u},\nabla{\bf u},p)\cdot{\bf n} + \nabla_{\bf u}J_\Gamma({\bf x},{\bf u},\nabla{\bf u},p,{\bf n},{\bf t}) - \left(\nabla_{\nabla{\bf u}}{\bf P}({\bf x},{\bf u},\nabla{\bf u},\Delta{\bf u},p,\nabla p)\cdot{\bf n}\right)\cdot{\bf v}\right.\\
    &\hspace{1cm} \left.- \partial_{\bf n}\nabla_{\Delta{\bf u}}{\bf P}({\bf x},{\bf u},\nabla{\bf u},\Delta{\bf u},p,\nabla p){\bf v} + \left(\nabla_{\nabla{\bf u}}{\bf f}({\bf x},{\bf u},\nabla{\bf u},p)\cdot{\bf n}\right)\cdot{\bf v} - \delta_{\mathcal{L}}\nabla_{\bf u}{\bf Q}({\bf x},{\bf u},\nabla{\bf u},p,{\bf n},{\bf t}){\bf v}_{\rm bc}\right]\cdot{\bf w}{\rm d}\Gamma\\
    &+ \int_{\Gamma\backslash\Gamma_{\rm D}^p} r\left[D_pJ_\Gamma({\bf x},{\bf u},\nabla{\bf u},p,{\bf n},{\bf t}) - {\bf n}^\top\nabla_{\nabla p}{\bf P}({\bf x},{\bf u},\nabla{\bf u},\Delta{\bf u},p,\nabla p){\bf v} - \delta_{\mathcal{L}}D_p{\bf Q}({\bf x},{\bf u},\nabla{\bf u},p,{\bf n},{\bf t})\cdot{\bf v}_{\rm bc}\right]{\rm d}\Gamma\\
    &+ \int_\Gamma \nabla_{\nabla{\bf u}}J_\Gamma({\bf x},{\bf u},\nabla{\bf u},p,{\bf n},{\bf t}):\nabla{\bf w} + {\bf v}^\top D_{\Delta{\bf u}}{\bf P}({\bf x},{\bf u},\nabla{\bf u},\Delta{\bf u},p,\nabla p)\partial_{\bf n}{\bf w} - \delta_{\mathcal{L}}D_{\nabla{\bf u}}{\bf Q}({\bf x},{\bf u},\nabla{\bf u},p,{\bf n},{\bf t})\nabla{\bf w}\cdot{\bf v}_{\rm bc}{\rm d}\Gamma,
\end{align*}
Thus, obtain the following weak formulation of \eqref{adjoint general stationary fluid dynamics PDEs}:
\begin{equation}
    \label{weak form adjoint general stationary fluid dynamics PDEs}
    \tag{wf-adj-gfld}
    \left\{\begin{split}
        &\int_\Omega -\nabla{\bf v}:\left(\nabla_{\Delta{\bf u}}{\bf P}({\bf x},{\bf u},\nabla{\bf u},\Delta{\bf u},p,\nabla p)\cdot\nabla{\bf w}\right) + \nabla{\bf v}:\left(\nabla\nabla_{\Delta{\bf u}}{\bf P}({\bf x},{\bf u},\nabla{\bf u},\Delta{\bf u},p,\nabla p)\cdot{\bf w}\right)\\
        &\hspace{1cm} + \left(\nabla_{\nabla{\bf u}}{\bf P}({\bf x},{\bf u},\nabla{\bf u},\Delta{\bf u},p,\nabla p) - \nabla_{\nabla{\bf u}}{\bf f}({\bf x},{\bf u},\nabla{\bf u},p)\right):\nabla{\bf v}\cdot{\bf w}\\
        &\hspace{1cm} - \left[\nabla_{\bf u}{\bf P}({\bf x},{\bf u},\nabla{\bf u},\Delta{\bf u},p,\nabla p) + \Delta\nabla_{\Delta{\bf u}}{\bf P}({\bf x},{\bf u},\nabla{\bf u},\Delta{\bf u},p,\nabla p) - \nabla_{\bf u}{\bf f}({\bf x},{\bf u},\nabla{\bf u},p)\right]{\bf v}\cdot{\bf w}\\
        &\hspace{1cm}+ \left[\left[\nabla\cdot\left(\nabla_{\nabla{\bf u}}{\bf P}({\bf x},{\bf u},\nabla{\bf u},\Delta{\bf u},p,\nabla p)\right) - \nabla\cdot\left(\nabla_{\nabla{\bf u}}{\bf f}({\bf x},{\bf u},\nabla{\bf u},p)\right)\right]\cdot{\bf v}\right]\cdot{\bf w} + q\nabla\cdot{\bf w} - q\nabla\cdot\left(\nabla_{\nabla{\bf u}}f_{\rm div}({\bf x},{\bf u},\nabla{\bf u},p)\right)\cdot{\bf w}\\
        &\hspace{1cm}+ q\nabla_{\nabla{\bf u}}f_{\rm div}({\bf x},{\bf u},\nabla{\bf u},p):\nabla{\bf w} + q\left[\nabla\cdot\left(\nabla_{\nabla{\bf u}}f_{\rm div}({\bf x},{\bf u},\nabla{\bf u},p)\right) - \nabla_{\bf u}f_{\rm div}({\bf x},{\bf u},\nabla{\bf u},p)\right]\cdot{\bf w}{\rm d}{\bf x}\\
        &\hspace{5mm}+ \int_\Gamma {\bf w}^\top\nabla_{\Delta{\bf u}}{\bf P}({\bf x},{\bf u},\nabla{\bf u},\Delta{\bf u},p,\nabla p)\partial_{\bf n}{\bf v} - q{\bf w}\cdot{\bf n} + q{\bf n}^\top\nabla_{\nabla{\bf u}}f_{\rm div}({\bf x},{\bf u},\nabla{\bf u},p){\bf w}{\rm d}\Gamma\\
        &\hspace{1cm} = \int_\Omega \nabla\cdot\left(\nabla_{\nabla{\bf u}}J_\Omega({\bf x},{\bf u},\nabla{\bf u},p)\right)\cdot{\bf w} - \nabla_{\bf u}J_\Omega({\bf x},{\bf u},\nabla{\bf u},p)\cdot{\bf w}{\rm d}{\bf x},\\
        &\int_\Omega r\nabla_{\nabla p}{\bf P}({\bf x},{\bf u},\nabla{\bf u},\Delta{\bf u},p,\nabla p):\nabla{\bf v}\\
        &\hspace{1cm} + r\left[-D_p{\bf P}({\bf x},{\bf u},\nabla{\bf u},\Delta{\bf u},p,\nabla p) + \nabla\cdot\left(\nabla_{\nabla p}{\bf P}({\bf x},{\bf u},\nabla{\bf u},\Delta{\bf u},p,\nabla p)\right) + D_p{\bf f}({\bf x},{\bf u},\nabla{\bf u},p)\right]\cdot{\bf v}{\rm d}{\bf x}\\
        &\hspace{1cm} = \int_\Omega qrD_pf_{\rm div}({\bf x},{\bf u},\nabla{\bf u},p) - rD_pJ_\Omega({\bf x},{\bf u},\nabla{\bf u},p){\rm d}{\bf x}.
    \end{split}\right.
\end{equation}



\subsection{Shape derivatives of \eqref{general stationary fluid dynamics PDEs}-constrained \eqref{cost functional of general stationary fluid dynamics PDEs}}
To calculate the shape derivatives of \eqref{cost functional of general stationary fluid dynamics PDEs} under the constraint state equation \eqref{general stationary fluid dynamics PDEs}, consider the \textit{perturbed cost functional}:
\begin{align}
    \label{perturbed cost functional of general stationary fluid dynamics PDEs}
    \tag{ptb-cost-gfld}
    J({\bf u}_t,p_t,\Omega_t) := \int_{\Omega_t} J_\Omega({\bf x},{\bf u}_t,\nabla{\bf u}_t,p_t){\rm d}{\bf x} + \int_{\Gamma_t} J_\Gamma({\bf x},{\bf u}_t,\nabla{\bf u}_t,p_t,{\bf n}_t,{\bf t}_t){\rm d}\Gamma_t,
\end{align}
where $({\bf u}_t,p_t)$ denotes the strong/classical solution (if exist and unique) of \eqref{general stationary fluid dynamics PDEs} on the perturbed domain $\Omega_t := T_t(V)(\Omega)$, i.e.:
\begin{equation}
    \label{perturbed general stationary fluid dynamics PDEs}
    \tag{ptb-gfld}
    \left\{\begin{split}
        {\bf P}({\bf x},{\bf u}_t,\nabla{\bf u}_t,\Delta{\bf u}_t,p_t,\nabla p_t) &= {\bf f}({\bf x},{\bf u}_t,\nabla{\bf u}_t,p_t) &&\mbox{ in } \Omega_t,\\
        -\nabla\cdot{\bf u}_t &= f_{\rm div}({\bf x},{\bf u}_t,\nabla{\bf u}_t,p_t) &&\mbox{ in } \Omega_t,\\
        {\bf Q}({\bf x},{\bf u}_t,\nabla{\bf u}_t,p_t,{\bf n}_t,{\bf t}_t) &= {\bf f}_{\rm bc}({\bf x}) &&\mbox{ on } \Gamma_t,
    \end{split}\right.
\end{equation}
where $\Gamma_t := \partial\Omega_t$.

Define:
\begin{itemize}
    \item Local shape derivative:
    \begin{align*}
        {\bf u}'({\bf x};V) := \lim_{t\downarrow 0} \frac{{\bf u}_t({\bf x}) - {\bf u}({\bf x})}{t},\ p'({\bf x};V) := \lim_{t\downarrow 0} \frac{p_t({\bf x}) - p({\bf x})}{t},\ \forall{\bf x}\in D.
    \end{align*}
    \item Material derivative:
    \begin{align*}
        {\rm d}{\bf u}({\bf x};V) := \lim_{t\downarrow 0} \frac{{\bf u}_t({\bf x}_t) - {\bf u}({\bf x})}{t},\ {\rm d}p({\bf x};V) := \lim_{t\downarrow 0} \frac{p_t({\bf x}_t) - p({\bf x})}{t}, \mbox{ where } {\bf x}_t := T_t(V)({\bf x}),\ \forall{\bf x}\in D.
    \end{align*}
\end{itemize}
Now subtracting \eqref{perturbed general stationary fluid dynamics PDEs} to \eqref{general stationary fluid dynamics PDEs} side by side, taking $\lim_{t\downarrow 0}$, we obtain:
\begin{equation}
    \label{PDEs local shape derivative for general stationary fluid dynamics PDEs}
    \left\{\begin{split}
        &D_{\bf u}{\bf P}({\bf x},{\bf u},\nabla{\bf u},\Delta{\bf u},p,\nabla p){\bf u}'({\bf x};V) + D_{\nabla{\bf u}}{\bf P}({\bf x},{\bf u},\nabla{\bf u},\Delta{\bf u},p,\nabla p)\nabla{\bf u}'({\bf x};V) + D_{\Delta{\bf u}}{\bf P}({\bf x},{\bf u},\nabla{\bf u},\Delta{\bf u},p,\nabla p)\Delta{\bf u}'({\bf x};V)\\
        &\hspace{5mm}+ D_p{\bf P}({\bf x},{\bf u},\nabla{\bf u},\Delta{\bf u},p,\nabla p)p'({\bf x};V) + D_{\nabla p}{\bf P}({\bf x},{\bf u},\nabla{\bf u},\Delta{\bf u},p,\nabla p)\nabla p'({\bf x};V)\\
        &\hspace{1cm}= D_{\bf u}{\bf f}({\bf x},{\bf u},\nabla{\bf u},p){\bf u}'({\bf x};V) + D_{\nabla{\bf u}}{\bf f}({\bf x},{\bf u},\nabla{\bf u},p)\nabla{\bf u}'({\bf x};V) + D_p{\bf f}({\bf x},{\bf u},\nabla{\bf u},p)p'({\bf x};V) \mbox{ in } \Omega,\\
        &-\nabla\cdot{\bf u}'({\bf x};V) = D_{\bf u}f_{\rm div}({\bf x},{\bf u},\nabla{\bf u},p){\bf u}'({\bf x};V) + D_{\nabla{\bf u}}f_{\rm div}({\bf x},{\bf u},\nabla{\bf u},p)\nabla{\bf u}'({\bf x};V) + D_pf_{\rm div}({\bf x},{\bf u},\nabla{\bf u},p)p'({\bf x};V) \mbox{ in } \Omega,\\
        &D_{\bf u}{\bf Q}({\bf x},{\bf u},\nabla{\bf u},p,{\bf n},{\bf t}){\bf u}'({\bf x};V) + D_{\nabla{\bf u}}{\bf Q}({\bf x},{\bf u},\nabla{\bf u},p,{\bf n},{\bf t})\nabla{\bf u}'({\bf x};V) + D_p{\bf Q}({\bf x},{\bf u},\nabla{\bf u},p,{\bf n},{\bf t})p'({\bf x};V)\\
        &\hspace{5mm}+ D_{\bf n}{\bf Q}({\bf x},{\bf u},\nabla{\bf u},p,{\bf n},{\bf t}){\bf n}'({\bf x};V) + D_{\bf t}{\bf Q}({\bf x},{\bf u},\nabla{\bf u},p,{\bf n},{\bf t}){\bf t}'({\bf x};V) = {\bf 0} \mbox{ on } \Gamma.
    \end{split}\right.
\end{equation}
Now start to compute the 1st-order shape derivative for \eqref{cost functional of general stationary fluid dynamics PDEs}. Applying the domain and boundary formulas for the domain and boundary integrals, respectively, yields the following ``4 combinations'' (2 combinations for each of domain and boundary integrals, but only 2 of 4 presented here for brevity):
\begin{align*}
    dJ({\bf u},p,\Omega;V) =&\, \int_\Omega J_\Omega'({\bf x},{\bf u},\nabla{\bf u},p;V) + \nabla\cdot\left(J_\Omega({\bf x},{\bf u},\nabla{\bf u},p)V(0)\right){\rm d}{\bf x}\\
    &+ \int_\Gamma J_\Gamma'({\bf x},{\bf u},\nabla{\bf u},p,{\bf n},{\bf t};V) + \nabla\left(J_\Gamma({\bf x},{\bf u},\nabla{\bf u},p,{\bf n},{\bf t})\right)\cdot V(0) + J_\Gamma({\bf x},{\bf u},\nabla{\bf u},p,{\bf n},{\bf t})\left(\nabla\cdot V(0) - DV(0){\bf n}\cdot{\bf n}\right){\rm d}\Gamma\\
    =&\, \int_\Omega J_\Omega'({\bf x},{\bf u},\nabla{\bf u},p;V){\rm d}{\bf x} + \int_\Gamma J_\Omega({\bf x},{\bf u},\nabla{\bf u},p)V(0)\cdot{\bf n}{\rm d}\Gamma\\
    &+ \int_\Gamma J_\Gamma'({\bf x},{\bf u},\nabla{\bf u},p,{\bf n},{\bf t};V) + \left[\partial_{\bf n}(J_\Gamma({\bf x},{\bf u},\nabla{\bf u},p,{\bf n},{\bf t})) + HJ_\Gamma({\bf x},{\bf u},\nabla{\bf u},p,{\bf n},{\bf t})\right]V(0)\cdot{\bf n}{\rm d}\Gamma.
\end{align*}
We now compute these explicitly.
\begin{enumerate}[leftmargin=0in]
    \item \textit{1st representation of shape derivative.}
    \begin{align*}
        &dJ({\bf u},p,\Omega;V)\\
        =&\, \int_\Omega J_\Omega'({\bf x},{\bf u},\nabla{\bf u},p;V) + \nabla\cdot\left(J_\Omega({\bf x},{\bf u},\nabla{\bf u},p)V(0)\right){\rm d}{\bf x}\\
        &+ \int_\Gamma J_\Gamma'({\bf x},{\bf u},\nabla{\bf u},p,{\bf n},{\bf t};V) + \nabla\left(J_\Gamma({\bf x},{\bf u},\nabla{\bf u},p,{\bf n},{\bf t})\right)\cdot V(0) + J_\Gamma({\bf x},{\bf u},\nabla{\bf u},p,{\bf n},{\bf t})\left(\nabla\cdot V(0) - DV(0){\bf n}\cdot{\bf n}\right){\rm d}\Gamma\\
        =&\, \int_\Omega D_{\bf u}J_\Omega({\bf x},{\bf u},\nabla{\bf u},p){\bf u}'({\bf x};V) + D_{\nabla{\bf u}}J_\Omega({\bf x},{\bf u},\nabla{\bf u},p)\nabla{\bf u}'({\bf x};V) + \partial_pJ_\Omega({\bf x},{\bf u},\nabla{\bf u},p)p'({\bf x};V) + \nabla\cdot\left(J_\Omega({\bf x},{\bf u},\nabla{\bf u},p)V(0)\right){\rm d}{\bf x}\\
        &+ \int_\Gamma D_{\bf u}J_\Gamma({\bf x},{\bf u},\nabla{\bf u},p,{\bf n},{\bf t}){\bf u}'({\bf x};V) + D_{\nabla{\bf u}}J_\Gamma({\bf x},{\bf u},\nabla{\bf u},p,{\bf n},{\bf t})\nabla{\bf u}'({\bf x};V) + \partial_pJ_\Gamma({\bf x},{\bf u},\nabla{\bf u},p,{\bf n},{\bf t})p'({\bf x};V)\\
        &\hspace{1cm}+ D_{\bf n}J_\Gamma({\bf x},{\bf u},\nabla{\bf u},p,{\bf n},{\bf t}){\bf n}'({\bf x};V) + D_{\bf t}J_\Gamma({\bf x},{\bf u},\nabla{\bf u},p,{\bf n},{\bf t}){\bf t}'({\bf x};V)\\
        &\hspace{1cm}+ \left[DJ_\Gamma({\bf x},{\bf u},\nabla{\bf u},p,{\bf n},{\bf t}) + D_{\bf u}J_\Gamma({\bf x},{\bf u},\nabla{\bf u},p,{\bf n},{\bf t})D{\bf u} + D_{\nabla{\bf u}}J_\Gamma({\bf x},{\bf u},\nabla{\bf u},p,{\bf n},{\bf t})D\nabla{\bf u} + \partial_pJ_\Gamma({\bf x},{\bf u},\nabla{\bf u},p,{\bf n},{\bf t})Dp\right.\\
        &\hspace{15mm} \left.+ D_{\bf n}J_\Gamma({\bf x},{\bf u},\nabla{\bf u},p,{\bf n},{\bf t})D{\bf n} + D_{\bf t}J_\Gamma({\bf x},{\bf u},\nabla{\bf u},p,{\bf n},{\bf t})D{\bf t}\right]V(0)\\
        &\hspace{1cm}+ J_\Gamma({\bf x},{\bf u},\nabla{\bf u},p,{\bf n},{\bf t})\left(\nabla\cdot V(0) - DV(0){\bf n}\cdot{\bf n}\right){\rm d}\Gamma\\
        =&\, \int_\Omega \nabla_{\bf u}J_\Omega({\bf x},{\bf u},\nabla{\bf u},p)\cdot{\bf u}'({\bf x};V) + \nabla_{\nabla{\bf u}}J_\Omega({\bf x},{\bf u},\nabla{\bf u},p):\nabla{\bf u}'({\bf x};V) + \partial_pJ_\Omega({\bf x},{\bf u},\nabla{\bf u},p)p'({\bf x};V) + \nabla\cdot\left(J_\Omega({\bf x},{\bf u},\nabla{\bf u},p)V(0)\right){\rm d}{\bf x}\\
        &+ \int_\Gamma \nabla_{\bf u}J_\Gamma({\bf x},{\bf u},\nabla{\bf u},p,{\bf n},{\bf t})\cdot{\bf u}'({\bf x};V) + \nabla_{\nabla{\bf u}}J_\Gamma({\bf x},{\bf u},\nabla{\bf u},p,{\bf n},{\bf t}):\nabla{\bf u}'({\bf x};V) + \partial_pJ_\Gamma({\bf x},{\bf u},\nabla{\bf u},p,{\bf n},{\bf t})p'({\bf x};V)\\
        &\hspace{1cm}+ \nabla_{\bf n}J_\Gamma({\bf x},{\bf u},\nabla{\bf u},p,{\bf n},{\bf t})\cdot{\bf n}'({\bf x};V) + \nabla_{\bf t}J_\Gamma({\bf x},{\bf u},\nabla{\bf u},p,{\bf n},{\bf t}):{\bf t}'({\bf x};V)\\
        &\hspace{1cm}+ \left[\nabla J_\Gamma({\bf x},{\bf u},\nabla{\bf u},p,{\bf n},{\bf t}) + \nabla_{\bf u}J_\Gamma({\bf x},{\bf u},\nabla{\bf u},p,{\bf n},{\bf t})\cdot\nabla{\bf u} + \nabla\nabla{\bf u}:\nabla_{\nabla{\bf u}}J_\Gamma({\bf x},{\bf u},\nabla{\bf u},p,{\bf n},{\bf t}) + \partial_pJ_\Gamma({\bf x},{\bf u},\nabla{\bf u},p,{\bf n},{\bf t})\nabla p\right.\\
        &\hspace{15mm} \left.+ \nabla{\bf n}\nabla_{\bf n}J_\Gamma({\bf x},{\bf u},\nabla{\bf u},p,{\bf n},{\bf t}) + \nabla{\bf t}:\nabla_{\bf t}J_\Gamma({\bf x},{\bf u},\nabla{\bf u},p,{\bf n},{\bf t})\right]\cdot V(0)\\
        &\hspace{1cm}+ J_\Gamma({\bf x},{\bf u},\nabla{\bf u},p,{\bf n},{\bf t})\left(\nabla\cdot V(0) - DV(0){\bf n}\cdot{\bf n}\right){\rm d}\Gamma,
    \end{align*}
    where we have expanded $\nabla\left(J_\Gamma({\bf x},{\bf u},\nabla{\bf u},p,{\bf n},{\bf t})\right)$ as follows:
    \begin{align*}
        &J_\Gamma({\bf x},{\bf u},\nabla{\bf u},p,{\bf n},{\bf t})\\
        =&\, J(x_1,\ldots,x_N,u_1,\ldots,u_N,\partial_{x_1}u_1,\ldots,\partial_{x_N}u_1,\ldots,\partial_{x_1}u_N,\ldots,\partial_{x_N}u_N,p,n_1,\ldots,n_N,t_{1,1},\ldots,t_{1,N},\ldots,t_{N-1,1},\ldots,t_{N-1,N}),\\
        &\partial_{x_k}\left(J_\Gamma({\bf x},{\bf u},\nabla{\bf u},p,{\bf n},{\bf t})\right)\\
        =&\, \partial_{x_k}J_\Gamma({\bf x},{\bf u},\nabla{\bf u},p,{\bf n},{\bf t}) + \sum_{i=1}^N \partial_{u_i}J_\Gamma({\bf x},{\bf u},\nabla{\bf u},p,{\bf n},{\bf t})\partial_{x_k}u_i\\
        &+ \sum_{i=1}^N\sum_{j=1}^N \partial_{\partial_{x_i}u_j}J_\Gamma({\bf x},{\bf u},\nabla{\bf u},p,{\bf n},{\bf t})\partial_{x_k}\partial_{x_i}u_j + \partial_pJ_\Gamma({\bf x},{\bf u},\nabla{\bf u},p,{\bf n},{\bf t})\partial_{x_k}p + \sum_{i=1}^N \partial_{n_i}J_\Gamma({\bf x},{\bf u},\nabla{\bf u},p,{\bf n},{\bf t})\partial_{x_k}n_i\\
        &+ \sum_{i=1}^{N-1}\sum_{j=1}^N \partial_{t_{i,j}}J_\Gamma({\bf x},{\bf u},\nabla{\bf u},p,{\bf n},{\bf t})\partial_{x_k}t_{i,j}\\
        =&\, \partial_{x_k}J_\Gamma({\bf x},{\bf u},\nabla{\bf u},p,{\bf n},{\bf t}) + \nabla_{\bf u}J_\Gamma({\bf x},{\bf u},\nabla{\bf u},p,{\bf n},{\bf t})\cdot\partial_{x_k}{\bf u} + \nabla_{\nabla{\bf u}}J_\Gamma({\bf x},{\bf u},\nabla{\bf u},p,{\bf n},{\bf t}):\partial_{x_k}\nabla{\bf u} + \partial_pJ_\Gamma({\bf x},{\bf u},\nabla{\bf u},p,{\bf n},{\bf t})\partial_{x_k}p\\
        &+ \nabla_{\bf n}J_\Gamma({\bf x},{\bf u},\nabla{\bf u},p,{\bf n},{\bf t})\cdot\partial_{x_k}{\bf n} + \sum_{i=1}^{N-1} \nabla_{{\bf t}_i}J_\Gamma({\bf x},{\bf u},\nabla{\bf u},p,{\bf n},{\bf t})\cdot\partial_{x_k}{\bf t}_i\\
        =&\, \partial_{x_k}J_\Gamma({\bf x},{\bf u},\nabla{\bf u},p,{\bf n},{\bf t}) + \nabla_{\bf u}J_\Gamma({\bf x},{\bf u},\nabla{\bf u},p,{\bf n},{\bf t})\cdot\partial_{x_k}{\bf u} + \nabla_{\nabla{\bf u}}J_\Gamma({\bf x},{\bf u},\nabla{\bf u},p,{\bf n},{\bf t}):\partial_{x_k}\nabla{\bf u} + \partial_pJ_\Gamma({\bf x},{\bf u},\nabla{\bf u},p,{\bf n},{\bf t})\partial_{x_k}p\\
        &+ \nabla_{\bf n}J_\Gamma({\bf x},{\bf u},\nabla{\bf u},p,{\bf n},{\bf t})\cdot\partial_{x_k}{\bf n} + \nabla_{\bf t}J_\Gamma({\bf x},{\bf u},\nabla{\bf u},p,{\bf n},{\bf t}):\partial_{x_k}{\bf t},
    \end{align*}
    hence
    \begin{align*}
        \nabla\left(J_\Gamma({\bf x},{\bf u},\nabla{\bf u},p,{\bf n},{\bf t})\right) =&\, \nabla J_\Gamma({\bf x},{\bf u},\nabla{\bf u},p,{\bf n},{\bf t}) + \nabla{\bf u}\nabla_{\bf u}J_\Gamma({\bf x},{\bf u},\nabla{\bf u},p,{\bf n},{\bf t}) + \nabla\nabla{\bf u}:\nabla_{\nabla{\bf u}}J_\Gamma({\bf x},{\bf u},\nabla{\bf u},p,{\bf n},{\bf t})\\
        &+ \partial_pJ_\Gamma({\bf x},{\bf u},\nabla{\bf u},p,{\bf n},{\bf t})\nabla p + \nabla{\bf n}\nabla_{\bf n}J_\Gamma({\bf x},{\bf u},\nabla{\bf u},p,{\bf n},{\bf t}) + \nabla{\bf t}:\nabla_{\bf t}J_\Gamma({\bf x},{\bf u},\nabla{\bf u},p,{\bf n},{\bf t}).
    \end{align*}
    Integrate by parts the term $\nabla_{\nabla{\bf u}}J_\Omega({\bf x},{\bf u},\nabla{\bf u},p):\nabla{\bf u}'({\bf x};V)$ in $dJ({\bf u},p,\Omega;V)$:
    \begin{align*}
        &\int_\Omega \nabla_{\nabla{\bf u}}J_\Omega({\bf x},{\bf u},\nabla{\bf u},p):\nabla{\bf u}'({\bf x};V){\rm d}{\bf x} = \int_\Omega \sum_{i=1}^N\sum_{j=1}^N \partial_{\partial_{x_i}u_j}J_\Omega({\bf x},{\bf u},\nabla{\bf u},p)\partial_{x_i}u_j'({\bf x};V){\rm d}{\bf x}\\
        =&\, \sum_{i=1}^N\sum_{j=1}^N \int_\Omega \partial_{\partial_{x_i}u_j}J_\Omega({\bf x},{\bf u},\nabla{\bf u},p)\partial_{x_i}u_j'({\bf x};V){\rm d}{\bf x}\\
        =&\, \sum_{i=1}^N\sum_{j=1}^N -\int_\Omega \partial_{x_i}\partial_{\partial_{x_i}u_j}J_\Omega({\bf x},{\bf u},\nabla{\bf u},p)u_j'({\bf x};V){\rm d}{\bf x} + \int_\Gamma n_i\partial_{\partial_{x_i}u_j}J_\Omega({\bf x},{\bf u},\nabla{\bf u},p)u_j'({\bf x};V){\rm d}\Gamma\\
        =&-\int_\Omega \sum_{i=1}^N\sum_{j=1}^N \partial_{x_i}\partial_{\partial_{x_i}u_j}J_\Omega({\bf x},{\bf u},\nabla{\bf u},p)u_j'({\bf x};V){\rm d}{\bf x} + \int_\Gamma \sum_{i=1}^N\sum_{j=1}^N n_i\partial_{\partial_{x_i}u_j}J_\Omega({\bf x},{\bf u},\nabla{\bf u},p)u_j'({\bf x};V){\rm d}\Gamma\\
        =&-\int_\Omega \sum_{j=1}^N u_j'({\bf x};V)\sum_{i=1}^N \partial_{x_i}\partial_{\partial_{x_i}u_j}J_\Omega({\bf x},{\bf u},\nabla{\bf u},p){\rm d}{\bf x} + \int_\Gamma {\bf n}^\top\nabla_{\nabla{\bf u}}J_\Omega({\bf x},{\bf u},\nabla{\bf u},p){\bf u}'({\bf x};V){\rm d}\Gamma\\
        =&-\int_\Omega \sum_{j=1}^N u_j'({\bf x};V)\nabla\cdot\left(\nabla_{\nabla u_j}J_\Omega({\bf x},{\bf u},\nabla{\bf u},p)\right){\rm d}{\bf x} + \int_\Gamma {\bf n}^\top\nabla_{\nabla{\bf u}}J_\Omega({\bf x},{\bf u},\nabla{\bf u},p){\bf u}'({\bf x};V){\rm d}\Gamma\\
        =&-\int_\Omega \nabla\cdot\left(\nabla_{\nabla{\bf u}}J_\Omega({\bf x},{\bf u},\nabla{\bf u},p)\right)\cdot{\bf u}'({\bf x};V){\rm d}{\bf x} + \int_\Gamma
        {\bf n}^\top\nabla_{\nabla{\bf u}}J_\Omega({\bf x},{\bf u},\nabla{\bf u},p){\bf u}'({\bf x};V){\rm d}\Gamma.
    \end{align*}
    Then
    \begin{align*}
        &dJ({\bf u},p,\Omega;V)\\
        =&\, \int_\Omega \left[\nabla_{\bf u}J_\Omega({\bf x},{\bf u},\nabla{\bf u},p) - \nabla\cdot(\nabla_{\nabla{\bf u}}J_\Omega({\bf x},{\bf u},\nabla{\bf u},p))\right]\cdot{\bf u}'({\bf x};V) + \partial_pJ_\Omega({\bf x},{\bf u},\nabla{\bf u},p)p'({\bf x};V) + \nabla\cdot\left(J_\Omega({\bf x},{\bf u},\nabla{\bf u},p)V(0)\right){\rm d}{\bf x}\\
        &+ \int_\Gamma {\bf n}^\top\nabla_{\nabla{\bf u}}J_\Omega({\bf x},{\bf u},\nabla{\bf u},p){\bf u}'({\bf x};V) + \nabla_{\bf u}J_\Gamma({\bf x},{\bf u},\nabla{\bf u},p,{\bf n},{\bf t})\cdot{\bf u}'({\bf x};V) + \nabla_{\nabla{\bf u}}J_\Gamma({\bf x},{\bf u},\nabla{\bf u},p,{\bf n},{\bf t}):\nabla{\bf u}'({\bf x};V)\\
        &\hspace{1cm}+ \partial_pJ_\Gamma({\bf x},{\bf u},\nabla{\bf u},p,{\bf n},{\bf t})p'({\bf x};V) + \nabla_{\bf n}J_\Gamma({\bf x},{\bf u},\nabla{\bf u},p,{\bf n},{\bf t})\cdot{\bf n}'({\bf x};V) + \nabla_{\bf t}J_\Gamma({\bf x},{\bf u},\nabla{\bf u},p,{\bf n},{\bf t}):{\bf t}'({\bf x};V)\\
        &\hspace{1cm}+ \left[\nabla J_\Gamma({\bf x},{\bf u},\nabla{\bf u},p,{\bf n},{\bf t}) + \nabla_{\bf u}J_\Gamma({\bf x},{\bf u},\nabla{\bf u},p,{\bf n},{\bf t})\cdot\nabla{\bf u} + \nabla\nabla{\bf u}:\nabla_{\nabla{\bf u}}J_\Gamma({\bf x},{\bf u},\nabla{\bf u},p,{\bf n},{\bf t}) + \partial_pJ_\Gamma({\bf x},{\bf u},\nabla{\bf u},p,{\bf n},{\bf t})\nabla p\right.\\
        &\hspace{15mm} \left.+ \nabla{\bf n}\nabla_{\bf n}J_\Gamma({\bf x},{\bf u},\nabla{\bf u},p,{\bf n},{\bf t}) + \nabla{\bf t}:\nabla_{\bf t}J_\Gamma({\bf x},{\bf u},\nabla{\bf u},p,{\bf n},{\bf t})\right]\cdot V(0)\\
        &\hspace{1cm}+ J_\Gamma({\bf x},{\bf u},\nabla{\bf u},p,{\bf n},{\bf t})\left(\nabla\cdot V(0) - DV(0){\bf n}\cdot{\bf n}\right){\rm d}\Gamma,
    \end{align*}
    Test \eqref{PDEs local shape derivative for general stationary fluid dynamics PDEs} with the adjoint variable $({\bf v},q)$, obtain
    \begin{equation*}
        \left\{\begin{split}
            &\int_\Omega D_{\bf u}{\bf P}({\bf x},{\bf u},\nabla{\bf u},\Delta{\bf u},p,\nabla p){\bf u}'({\bf x};V)\cdot{\bf v} + {\color{red}D_{\nabla{\bf u}}{\bf P}({\bf x},{\bf u},\nabla{\bf u},\Delta{\bf u},p,\nabla p)\nabla{\bf u}'({\bf x};V)\cdot{\bf v}} + {\color{red}D_{\Delta{\bf u}}{\bf P}({\bf x},{\bf u},\nabla{\bf u},\Delta{\bf u},p,\nabla p)\Delta{\bf u}'({\bf x};V)\cdot{\bf v}}\\
            &\hspace{5mm}+ D_p{\bf P}({\bf x},{\bf u},\nabla{\bf u},\Delta{\bf u},p,\nabla p)p'({\bf x};V)\cdot{\bf v} + {\color{red}D_{\nabla p}{\bf P}({\bf x},{\bf u},\nabla{\bf u},\Delta{\bf u},p,\nabla p)\nabla p'({\bf x};V)\cdot{\bf v}}{\rm d}{\bf x}\\
            &\hspace{1cm}= \int_\Omega D_{\bf u}{\bf f}({\bf x},{\bf u},\nabla{\bf u},p){\bf u}'({\bf x};V)\cdot{\bf v} + {\color{red}D_{\nabla{\bf u}}{\bf f}({\bf x},{\bf u},\nabla{\bf u},p)\nabla{\bf u}'({\bf x};V)\cdot{\bf v}} + D_p{\bf f}({\bf x},{\bf u},\nabla{\bf u},p)p'({\bf x};V)\cdot{\bf v}{\rm d}{\bf x},\\
            &\int_\Omega {\color{red}-q\nabla\cdot{\bf u}'({\bf x};V)}{\rm d}{\bf x} = \int_\Omega qD_{\bf u}f_{\rm div}({\bf x},{\bf u},\nabla{\bf u},p){\bf u}'({\bf x};V) + {\color{red}qD_{\nabla{\bf u}}f_{\rm div}({\bf x},{\bf u},\nabla{\bf u},p)\nabla{\bf u}'({\bf x};V)} + qD_pf_{\rm div}({\bf x},{\bf u},\nabla{\bf u},p)p'({\bf x};V){\rm d}{\bf x},
        \end{split}\right.
    \end{equation*}
    Integrate by parts:
    \begin{enumerate}[leftmargin=0in]
        \item Term $D_{\nabla{\bf u}}{\bf P}({\bf x},{\bf u},\nabla{\bf u},\Delta{\bf u},p,\nabla p)\nabla{\bf u}'({\bf x};V)\cdot{\bf v}$: Use the result for term 2 before with $\tilde{\bf u} := {\bf u}'({\bf x};V)$:
        \begin{align*}
            &\int_\Omega D_{\nabla{\bf u}}{\bf P}({\bf x},{\bf u},\nabla{\bf u},\Delta{\bf u},p,\nabla p)\nabla{\bf u}'({\bf x};V)\cdot{\bf v}{\rm d}{\bf x}\\
            =&\, -\int_\Omega \left(\nabla\cdot\left(\nabla_{\nabla{\bf u}}{\bf P}({\bf x},{\bf u},\nabla{\bf u},\Delta{\bf u},p,\nabla p)\right)\cdot{\bf v}\right)\cdot{\bf u}'({\bf x};V) + \left(\nabla_{\nabla{\bf u}}{\bf P}({\bf x},{\bf u},\nabla{\bf u},\Delta{\bf u},p,\nabla p):\nabla{\bf v}\right)\cdot{\bf u}'({\bf x};V){\rm d}{\bf x}\\
            &+ \int_\Gamma \left(\left(\nabla_{\nabla{\bf u}}{\bf P}({\bf x},{\bf u},\nabla{\bf u},\Delta{\bf u},p,\nabla p)\cdot{\bf n}\right)\cdot{\bf v}\right)\cdot{\bf u}'({\bf x};V){\rm d}\Gamma.
        \end{align*}
        \item Term $D_{\Delta{\bf u}}{\bf P}({\bf x},{\bf u},\nabla{\bf u},\Delta{\bf u},p,\nabla p)\Delta{\bf u}'({\bf x};V)\cdot{\bf v}$: Use the result for term 3 before with $\tilde{\bf u} := {\bf u}'({\bf x};V)$:
        \begin{align*}
            &\int_\Omega D_{\Delta{\bf u}}{\bf P}({\bf x},{\bf u},\nabla{\bf u},\Delta{\bf u},p,\nabla p)\Delta{\bf u}'({\bf x};V)\cdot{\bf v}{\rm d}{\bf x}\\
            =&\, \int_\Omega {\bf v}^\top\Delta D_{\Delta{\bf u}}{\bf P}({\bf x},{\bf u},\nabla{\bf u},\Delta{\bf u},p,\nabla p){\bf u}'({\bf x};V) + \Delta{\bf v}^\top D_{\Delta{\bf u}}{\bf P}({\bf x},{\bf u},\nabla{\bf u},\Delta{\bf u},p,\nabla p){\bf u}'({\bf x};V){\rm d}{\bf x}\\
            &- \int_\Gamma {\bf v}^\top D_{\Delta{\bf u}}{\bf P}({\bf x},{\bf u},\nabla{\bf u},\Delta{\bf u},p,\nabla p)\partial_{\bf n}{\bf u}'({\bf x};V) - {\bf v}^\top\partial_{\bf n}D_{\Delta{\bf u}}{\bf P}({\bf x},{\bf u},\nabla{\bf u},\Delta{\bf u},p,\nabla p){\bf u}'({\bf x};V)\\
            &\hspace{1cm}- \partial_{\bf n}{\bf v}^\top D_{\Delta{\bf u}}{\bf P}({\bf x},{\bf u},\nabla{\bf u},\Delta{\bf u},p,\nabla p){\bf u}'({\bf x};V){\rm d}\Gamma.
        \end{align*}
        \item Term $D_{\nabla p}{\bf P}({\bf x},{\bf u},\nabla{\bf u},\Delta{\bf u},p,\nabla p)\nabla p'({\bf x};V)\cdot{\bf v}$: Use the result for term 5 before with $\tilde{\bf u} := {\bf u}'({\bf x};V)$:
        \begin{align*}
            &\int_\Omega D_{\nabla p}{\bf P}({\bf x},{\bf u},\nabla{\bf u},\Delta{\bf u},p,\nabla p)\nabla p'({\bf x};V)\cdot{\bf v}{\rm d}{\bf x}\\
            =&\, -\int_\Omega p'({\bf x};V)\nabla_{\nabla p}{\bf P}({\bf x},{\bf u},\nabla{\bf u},\Delta{\bf u},p,\nabla p):\nabla{\bf v} + p'({\bf x};V)\nabla\cdot\left(\nabla_{\nabla p}{\bf P}({\bf x},{\bf u},\nabla{\bf u},\Delta{\bf u},p,\nabla p)\right)\cdot{\bf v}{\rm d}{\bf x}\\
            &+ \int_\Gamma p'({\bf x};V){\bf n}^\top\nabla_{\nabla p}{\bf P}({\bf x},{\bf u},\nabla{\bf u},\Delta{\bf u},p,\nabla p){\bf v}{\rm d}\Gamma.
        \end{align*}
        \item Term $D_{\nabla{\bf u}}{\bf f}({\bf x},{\bf u},\nabla{\bf u},p)\nabla{\bf u}'({\bf x};V)\cdot{\bf v}$: Use the result for term 4 before with $\tilde{\bf u} := {\bf u}'({\bf x};V)$:
        \begin{align*}
            \int_\Omega D_{\nabla{\bf u}}{\bf f}({\bf x},{\bf u},\nabla{\bf u},p)\nabla{\bf u}'({\bf x};V)\cdot{\bf v}{\rm d}{\bf x} =& -\int_\Omega \left(\nabla\cdot\left(\nabla_{\nabla{\bf u}}{\bf f}({\bf x},{\bf u},\nabla{\bf u},p)\right)\cdot{\bf v}\right)\cdot{\bf u}'({\bf x};V) + \left(\nabla_{\nabla{\bf u}}{\bf f}({\bf x},{\bf u},\nabla{\bf u},p):\nabla{\bf v}\right)\cdot{\bf u}'({\bf x};V){\rm d}{\bf x}\\
            &+ \int_\Gamma \left(\left(\nabla_{\nabla{\bf u}}{\bf f}({\bf x},{\bf u},\nabla{\bf u},p)\cdot{\bf n}\right)\cdot{\bf v}\right)\cdot{\bf u}'({\bf x};V){\rm d}\Gamma.
        \end{align*}
        \item Term $q\nabla\cdot{\bf u}'({\bf x};V)$: Use the result for term 6 before with $\tilde{\bf u} := {\bf u}'({\bf x};V)$:
        \begin{align*}
            \int_\Omega -q\nabla\cdot{\bf u}'({\bf x};V){\rm d}{\bf x} = \int_\Omega \nabla q\cdot{\bf u}'({\bf x};V){\rm d}{\bf x} - \int_\Gamma q{\bf u}'({\bf x};V)\cdot{\bf n}{\rm d}\Gamma.
        \end{align*}
        \item Term $qD_{\nabla{\bf u}}f_{\rm div}({\bf x},{\bf u},\nabla{\bf u},p)\nabla{\bf u}'({\bf x};V)$: Use the result for term 7 before with $\tilde{\bf u} := {\bf u}'({\bf x};V)$:
        \begin{align*}
            \int_\Omega qD_{\nabla{\bf u}}f_{\rm div}({\bf x},{\bf u},\nabla{\bf u},p)\nabla{\bf u}'({\bf x};V){\rm d}{\bf x} =& -\int_\Omega \nabla^\top q\nabla_{\nabla{\bf u}}f_{\rm div}({\bf x},{\bf u},\nabla{\bf u},p){\bf u}'({\bf x};V) + q\left(\nabla\cdot\left(\nabla_{\nabla{\bf u}}f_{\rm div}({\bf x},{\bf u},\nabla{\bf u},p)\right)\right)\cdot{\bf u}'({\bf x};V){\rm d}{\bf x}\\
            &+ \int_\Gamma q{\bf n}^\top\nabla_{\nabla{\bf u}}f_{\rm div}({\bf x},{\bf u},\nabla{\bf u},p){\bf u}'({\bf x};V){\rm d}\Gamma.
        \end{align*}
    \end{enumerate}
    Plug in, obtain then
    \begin{equation*}
        \left\{\begin{split}
            &\int_\Omega \left[\nabla_{\Delta{\bf u}}{\bf P}({\bf x},{\bf u},\nabla{\bf u},\Delta{\bf u},p,\nabla p)\Delta{\bf v} - \left(\nabla_{\nabla{\bf u}}{\bf P}({\bf x},{\bf u},\nabla{\bf u},\Delta{\bf u},p,\nabla p) - \nabla_{\nabla{\bf u}}{\bf f}({\bf x},{\bf u},\nabla{\bf u},p)\right):\nabla{\bf v}\right.\\
            &\hspace{1cm}+ \left[\nabla_{\bf u}{\bf P}({\bf x},{\bf u},\nabla{\bf u},\Delta{\bf u},p,\nabla p) + \Delta\nabla_{\Delta{\bf u}}{\bf P}({\bf x},{\bf u},\nabla{\bf u},\Delta{\bf u},p,\nabla p) - \nabla_{\bf u}{\bf f}({\bf x},{\bf u},\nabla{\bf u},p)\right]{\bf v}\\
            &\hspace{1cm}\left.- \left[\nabla\cdot\left(\nabla_{\nabla{\bf u}}{\bf P}({\bf x},{\bf u},\nabla{\bf u},\Delta{\bf u},p,\nabla p)\right) - \nabla\cdot\left(\nabla_{\nabla{\bf u}}{\bf f}({\bf x},{\bf u},\nabla{\bf u},p)\right)\right]\cdot{\bf v}\right]\cdot{\bf u}'({\bf x};V){\rm d}{\bf x}\\
            &\hspace{5mm}+ \int_\Omega \left[\left[D_p{\bf P}({\bf x},{\bf u},\nabla{\bf u},\Delta{\bf u},p,\nabla p) - \nabla\cdot\left(\nabla_{\nabla p}{\bf P}({\bf x},{\bf u},\nabla{\bf u},\Delta{\bf u},p,\nabla p)\right) - D_p{\bf f}({\bf x},{\bf u},\nabla{\bf u},p)\right]\cdot{\bf v}\right.\\
            &\hspace{15mm}\left.-\nabla_{\nabla p}{\bf P}({\bf x},{\bf u},\nabla{\bf u},\Delta{\bf u},p,\nabla p):\nabla{\bf v}\right]p'({\bf x};V){\rm d}{\bf x}\\
            &\hspace{5mm}+ \int_\Gamma \left[\left(\nabla_{\nabla{\bf u}}{\bf P}({\bf x},{\bf u},\nabla{\bf u},\Delta{\bf u},p,\nabla p)\cdot{\bf n}\right)\cdot{\bf v} + \partial_{\bf n}\nabla_{\Delta{\bf u}}{\bf P}({\bf x},{\bf u},\nabla{\bf u},\Delta{\bf u},p,\nabla p){\bf v} + \nabla_{\Delta{\bf u}}{\bf P}({\bf x},{\bf u},\nabla{\bf u},\Delta{\bf u},p,\nabla p)\partial_{\bf n}{\bf v}\right.\\
            &\hspace{1cm}\left.- \left(\nabla_{\nabla{\bf u}}{\bf f}({\bf x},{\bf u},\nabla{\bf u},p)\cdot{\bf n}\right)\cdot{\bf v}\right]\cdot{\bf u}'({\bf x};V){\rm d}\Gamma\\
            &\hspace{5mm}+ \int_\Gamma -{\bf v}^\top D_{\Delta{\bf u}}{\bf P}({\bf x},{\bf u},\nabla{\bf u},\Delta{\bf u},p,\nabla p)\partial_{\bf n}{\bf u}'({\bf x};V) + p'({\bf x};V){\bf n}^\top\nabla_{\nabla p}{\bf P}({\bf x},{\bf u},\nabla{\bf u},\Delta{\bf u},p,\nabla p){\bf v}{\rm d}\Gamma = 0,\\
            &\int_\Omega \left[\nabla q - q\nabla_{\bf u}f_{\rm div}({\bf x},{\bf u},\nabla{\bf u},p) + D_{\nabla{\bf u}}f_{\rm div}({\bf x},{\bf u},\nabla{\bf u},p)\nabla q + q\nabla\cdot\left(\nabla_{\nabla{\bf u}}f_{\rm div}({\bf x},{\bf u},\nabla{\bf u},p)\right)\right]\cdot{\bf u}'({\bf x};V){\rm d}{\bf x}\\
            &\hspace{5mm}- \int_\Omega qD_pf_{\rm div}({\bf x},{\bf u},\nabla{\bf u},p)p'({\bf x};V){\rm d}{\bf x} - \int_\Gamma \left[qn + qD_{\nabla{\bf u}}f_{\rm div}({\bf x},{\bf u},\nabla{\bf u},p){\bf n}\right]\cdot{\bf u}'({\bf x};V){\rm d}\Gamma = 0.
        \end{split}\right.
    \end{equation*}
    Add them together, obtain
    \begin{align*}
        &\int_\Omega \left[\nabla_{\Delta{\bf u}}{\bf P}({\bf x},{\bf u},\nabla{\bf u},\Delta{\bf u},p,\nabla p)\Delta{\bf v} - \left(\nabla_{\nabla{\bf u}}{\bf P}({\bf x},{\bf u},\nabla{\bf u},\Delta{\bf u},p,\nabla p) - \nabla_{\nabla{\bf u}}{\bf f}({\bf x},{\bf u},\nabla{\bf u},p)\right):\nabla{\bf v}\right.\\
        &\hspace{1cm}+ \left[\nabla_{\bf u}{\bf P}({\bf x},{\bf u},\nabla{\bf u},\Delta{\bf u},p,\nabla p) + \Delta\nabla_{\Delta{\bf u}}{\bf P}({\bf x},{\bf u},\nabla{\bf u},\Delta{\bf u},p,\nabla p) - \nabla_{\bf u}{\bf f}({\bf x},{\bf u},\nabla{\bf u},p)\right]{\bf v}\\
        &\hspace{1cm}\left.- \left[\nabla\cdot\left(\nabla_{\nabla{\bf u}}{\bf P}({\bf x},{\bf u},\nabla{\bf u},\Delta{\bf u},p,\nabla p)\right) - \nabla\cdot\left(\nabla_{\nabla{\bf u}}{\bf f}({\bf x},{\bf u},\nabla{\bf u},p)\right)\right]\cdot{\bf v}\right.\\
        &\hspace{1cm}\left.+ \nabla q - q\nabla_{\bf u}f_{\rm div}({\bf x},{\bf u},\nabla{\bf u},p) + D_{\nabla{\bf u}}f_{\rm div}({\bf x},{\bf u},\nabla{\bf u},p)\nabla q + q\nabla\cdot\left(\nabla_{\nabla{\bf u}}f_{\rm div}({\bf x},{\bf u},\nabla{\bf u},p)\right)\right]\cdot{\bf u}'({\bf x};V){\rm d}{\bf x}\\
        &+ \int_\Omega \left[\left[D_p{\bf P}({\bf x},{\bf u},\nabla{\bf u},\Delta{\bf u},p,\nabla p) - \nabla\cdot\left(\nabla_{\nabla p}{\bf P}({\bf x},{\bf u},\nabla{\bf u},\Delta{\bf u},p,\nabla p)\right) - D_p{\bf f}({\bf x},{\bf u},\nabla{\bf u},p)\right]\cdot{\bf v}\right.\\
        &\hspace{1cm}\left.-\nabla_{\nabla p}{\bf P}({\bf x},{\bf u},\nabla{\bf u},\Delta{\bf u},p,\nabla p):\nabla{\bf v} - qD_pf_{\rm div}({\bf x},{\bf u},\nabla{\bf u},p)\right]p'({\bf x};V){\rm d}{\bf x}\\
        &+ \int_\Gamma \left[\left(\nabla_{\nabla{\bf u}}{\bf P}({\bf x},{\bf u},\nabla{\bf u},\Delta{\bf u},p,\nabla p)\cdot{\bf n}\right)\cdot{\bf v} + \partial_{\bf n}\nabla_{\Delta{\bf u}}{\bf P}({\bf x},{\bf u},\nabla{\bf u},\Delta{\bf u},p,\nabla p){\bf v} + \nabla_{\Delta{\bf u}}{\bf P}({\bf x},{\bf u},\nabla{\bf u},\Delta{\bf u},p,\nabla p)\partial_{\bf n}{\bf v}\right.\\
        &\hspace{1cm}\left.- \left(\nabla_{\nabla{\bf u}}{\bf f}({\bf x},{\bf u},\nabla{\bf u},p)\cdot{\bf n}\right)\cdot{\bf v} - qn - qD_{\nabla{\bf u}}f_{\rm div}({\bf x},{\bf u},\nabla{\bf u},p){\bf n}\right]\cdot{\bf u}'({\bf x};V){\rm d}\Gamma\\
        &+ \int_\Gamma -{\bf v}^\top D_{\Delta{\bf u}}{\bf P}({\bf x},{\bf u},\nabla{\bf u},\Delta{\bf u},p,\nabla p)\partial_{\bf n}{\bf u}'({\bf x};V) + p'({\bf x};V){\bf n}^\top\nabla_{\nabla p}{\bf P}({\bf x},{\bf u},\nabla{\bf u},\Delta{\bf u},p,\nabla p){\bf v}{\rm d}\Gamma = 0.
    \end{align*}
    Combine this with \eqref{adjoint general stationary fluid dynamics PDEs}, obtain
    \begin{align*}
        &\int_\Omega \left[\nabla_{\bf u}J_\Omega({\bf x},{\bf u},\nabla{\bf u},p) - \nabla\cdot\left(\nabla_{\nabla{\bf u}}J_\Omega({\bf x},{\bf u},\nabla{\bf u},p)\right)\right]\cdot{\bf u}'({\bf x};V) + D_pJ_\Omega({\bf x},{\bf u},\nabla{\bf u},p)p'({\bf x};V){\rm d}{\bf x}\\
        &+ \int_\Gamma \left[-\nabla_{\bf u}{\bf Q}({\bf x},{\bf u},\nabla{\bf u},p,{\bf n},{\bf t}){\bf v}_{\rm bc} + \nabla_{\nabla{\bf u}}J_\Omega({\bf x},{\bf u},\nabla{\bf u},p)\cdot{\bf n} + \nabla_{\bf u}J_\Gamma({\bf x},{\bf u},\nabla{\bf u},p,{\bf n},{\bf t})\right]\cdot{\bf u}'({\bf x};V){\rm d}\Gamma\\
        &+ \int_\Gamma -{\bf v}^\top D_{\Delta{\bf u}}{\bf P}({\bf x},{\bf u},\nabla{\bf u},\Delta{\bf u},p,\nabla p)\partial_{\bf n}{\bf u}'({\bf x};V) + p'({\bf x};V){\bf n}^\top\nabla_{\nabla p}{\bf P}({\bf x},{\bf u},\nabla{\bf u},\Delta{\bf u},p,\nabla p){\bf v}{\rm d}\Gamma = 0.
    \end{align*}
    Combine this equality with the formula of shape derivative $dJ({\bf u},p,\Omega)$, obtain:
    \begin{align*}
        dJ({\bf u},p,\Omega;V) =&\, \int_\Omega \nabla\cdot\left(J_\Omega({\bf x},{\bf u},\nabla{\bf u},p)V(0)\right){\rm d}{\bf x}\\
        &+ \int_\Gamma \nabla_{\nabla{\bf u}}J_\Gamma({\bf x},{\bf u},\nabla{\bf u},p,{\bf n},{\bf t}):\nabla{\bf u}'({\bf x};V) + \partial_pJ_\Gamma({\bf x},{\bf u},\nabla{\bf u},p,{\bf n},{\bf t})p'({\bf x};V) + \nabla_{\bf n}J_\Gamma({\bf x},{\bf u},\nabla{\bf u},p,{\bf n},{\bf t})\cdot{\bf n}'({\bf x};V)\\
        &\hspace{1cm}+ \nabla_{\bf t}J_\Gamma({\bf x},{\bf u},\nabla{\bf u},p,{\bf n},{\bf t}):{\bf t}'({\bf x};V)\\
        &\hspace{1cm}+ \left[\nabla J_\Gamma({\bf x},{\bf u},\nabla{\bf u},p,{\bf n},{\bf t}) + \nabla_{\bf u}J_\Gamma({\bf x},{\bf u},\nabla{\bf u},p,{\bf n},{\bf t})\cdot\nabla{\bf u} + \nabla\nabla{\bf u}:\nabla_{\nabla{\bf u}}J_\Gamma({\bf x},{\bf u},\nabla{\bf u},p,{\bf n},{\bf t})\right.\\
        &\hspace{15mm} \left.+ \partial_pJ_\Gamma({\bf x},{\bf u},\nabla{\bf u},p,{\bf n},{\bf t})\nabla p + \nabla{\bf n}\nabla_{\bf n}J_\Gamma({\bf x},{\bf u},\nabla{\bf u},p,{\bf n},{\bf t}) + \nabla{\bf t}:\nabla_{\bf t}J_\Gamma({\bf x},{\bf u},\nabla{\bf u},p,{\bf n},{\bf t})\right]\cdot V(0)\\
        &\hspace{1cm}+ J_\Gamma({\bf x},{\bf u},\nabla{\bf u},p,{\bf n},{\bf t})\left(\nabla\cdot V(0) - DV(0){\bf n}\cdot{\bf n}\right) + \left(\nabla_{\bf u}{\bf Q}({\bf x},{\bf u},\nabla{\bf u},p,{\bf n},{\bf t}){\bf v}_{\rm bc}\right)\cdot{\bf u}'({\bf x};V)\\
        &\hspace{1cm}+ {\bf v}^\top D_{\Delta{\bf u}}{\bf P}({\bf x},{\bf u},\nabla{\bf u},\Delta{\bf u},p,\nabla p)\partial_{\bf n}{\bf u}'({\bf x};V) - p'({\bf x};V){\bf n}^\top\nabla_{\nabla p}{\bf P}({\bf x},{\bf u},\nabla{\bf u},\Delta{\bf u},p,\nabla p){\bf v}{\rm d}\Gamma.
    \end{align*}
    To eliminate $({\bf u}'({\bf x};V),p({\bf x};V))$ in boundary integrals, we need the explicit formula of ${\bf Q}({\bf x},{\bf u},\nabla{\bf u},p,{\bf n},{\bf t})$ (but not ${\bf f}_{\rm bc}({\bf x})$).
    
    \item \textit{2nd representation of shape derivative.} Use the same technique:
    \begin{align*}
        &dJ({\bf u},p,\Omega;V)\\
        =&\, \int_\Omega J_\Omega'({\bf x},{\bf u},\nabla{\bf u},p;V){\rm d}{\bf x}\\
        &+ \int_\Gamma J_\Omega({\bf x},{\bf u},\nabla{\bf u},p)V(0)\cdot{\bf n} + J_\Gamma'({\bf x},{\bf u},\nabla{\bf u},p,{\bf n},{\bf t};V) + \left[\partial_{\bf n}(J_\Gamma({\bf x},{\bf u},\nabla{\bf u},p,{\bf n},{\bf t})) + HJ_\Gamma({\bf x},{\bf u},\nabla{\bf u},p,{\bf n},{\bf t})\right]V(0)\cdot{\bf n}{\rm d}\Gamma\\
        =&\, \int_\Omega D_{\bf u}J_\Omega({\bf x},{\bf u},\nabla{\bf u},p){\bf u}'({\bf x};V) + D_{\nabla{\bf u}}J_\Omega({\bf x},{\bf u},\nabla{\bf u},p)\nabla{\bf u}'({\bf x};V) + \partial_pJ_\Omega({\bf x},{\bf u},\nabla{\bf u},p)p'({\bf x};V){\rm d}{\bf x}\\
        &+ \int_\Gamma J_\Omega({\bf x},{\bf u},\nabla{\bf u},p)V(0)\cdot{\bf n} + D_{\bf u}J_\Gamma({\bf x},{\bf u},\nabla{\bf u},p,{\bf n},{\bf t}){\bf u}'({\bf x};V) + D_{\nabla{\bf u}}J_\Gamma({\bf x},{\bf u},\nabla{\bf u},p,{\bf n},{\bf t})\nabla{\bf u}'({\bf x};V)\\
        &\hspace{1cm}+ \partial_pJ_\Gamma({\bf x},{\bf u},\nabla{\bf u},p,{\bf n},{\bf t})p'({\bf x};V) + D_{\bf n}J_\Gamma({\bf x},{\bf u},\nabla{\bf u},p,{\bf n},{\bf t}){\bf n}'({\bf x};V) + D_{\bf t}J_\Gamma({\bf x},{\bf u},\nabla{\bf u},p,{\bf n},{\bf t}){\bf t}'({\bf x};V)\\
        &\hspace{1cm}+ \left[DJ_\Gamma({\bf x},{\bf u},\nabla{\bf u},p,{\bf n},{\bf t}) + D_{\bf u}J_\Gamma({\bf x},{\bf u},\nabla{\bf u},p,{\bf n},{\bf t})D{\bf u} + D_{\nabla{\bf u}}J_\Gamma({\bf x},{\bf u},\nabla{\bf u},p,{\bf n},{\bf t})D\nabla{\bf u} + \partial_pJ_\Gamma({\bf x},{\bf u},\nabla{\bf u},p,{\bf n},{\bf t})Dp\right.\\
        &\hspace{15mm} \left.+ D_{\bf n}J_\Gamma({\bf x},{\bf u},\nabla{\bf u},p,{\bf n},{\bf t})D{\bf n} + D_{\bf t}J_\Gamma({\bf x},{\bf u},\nabla{\bf u},p,{\bf n},{\bf t})D{\bf t}\right]\cdot{\bf n}V(0)\cdot{\bf n} + HJ_\Gamma({\bf x},{\bf u},\nabla{\bf u},p,{\bf n},{\bf t})V(0)\cdot{\bf n}{\rm d}\Gamma\\
        =&\, \int_\Omega \nabla_{\bf u}J_\Omega({\bf x},{\bf u},\nabla{\bf u},p)\cdot{\bf u}'({\bf x};V) + \nabla_{\nabla{\bf u}}J_\Omega({\bf x},{\bf u},\nabla{\bf u},p):\nabla{\bf u}'({\bf x};V) + \partial_pJ_\Omega({\bf x},{\bf u},\nabla{\bf u},p)p'({\bf x};V){\rm d}{\bf x}\\
        &+ \int_\Gamma J_\Omega({\bf x},{\bf u},\nabla{\bf u},p)V(0)\cdot{\bf n} + \nabla_{\bf u}J_\Gamma({\bf x},{\bf u},\nabla{\bf u},p,{\bf n},{\bf t})\cdot{\bf u}'({\bf x};V) + \nabla_{\nabla{\bf u}}J_\Gamma({\bf x},{\bf u},\nabla{\bf u},p,{\bf n},{\bf t}):\nabla{\bf u}'({\bf x};V)\\
        &\hspace{1cm}+ \partial_pJ_\Gamma({\bf x},{\bf u},\nabla{\bf u},p,{\bf n},{\bf t})p'({\bf x};V) + \nabla_{\bf n}J_\Gamma({\bf x},{\bf u},\nabla{\bf u},p,{\bf n},{\bf t})\cdot{\bf n}'({\bf x};V) + \nabla_{\bf t}J_\Gamma({\bf x},{\bf u},\nabla{\bf u},p,{\bf n},{\bf t}):{\bf t}'({\bf x};V)\\
        &\hspace{1cm}+ \left[\nabla J_\Gamma({\bf x},{\bf u},\nabla{\bf u},p,{\bf n},{\bf t}) + \nabla_{\bf u}J_\Gamma({\bf x},{\bf u},\nabla{\bf u},p,{\bf n},{\bf t})\cdot\nabla{\bf u} + \nabla\nabla{\bf u}:\nabla_{\nabla{\bf u}}J_\Gamma({\bf x},{\bf u},\nabla{\bf u},p,{\bf n},{\bf t}) + \partial_pJ_\Gamma({\bf x},{\bf u},\nabla{\bf u},p,{\bf n},{\bf t})\nabla p\right.\\
        &\hspace{15mm} \left.+ \nabla{\bf n}\nabla_{\bf n}J_\Gamma({\bf x},{\bf u},\nabla{\bf u},p,{\bf n},{\bf t}) + \nabla{\bf t}:\nabla_{\bf t}J_\Gamma({\bf x},{\bf u},\nabla{\bf u},p,{\bf n},{\bf t})\right]\cdot{\bf n}V(0)\cdot{\bf n} + HJ_\Gamma({\bf x},{\bf u},\nabla{\bf u},p,{\bf n},{\bf t})V(0)\cdot{\bf n}{\rm d}\Gamma\\
        =&\, \int_\Omega \left[\nabla_{\bf u}J_\Omega({\bf x},{\bf u},\nabla{\bf u},p) - \nabla\cdot(\nabla_{\nabla{\bf u}}J_\Omega({\bf x},{\bf u},\nabla{\bf u},p))\right]\cdot{\bf u}'({\bf x};V) + \partial_pJ_\Omega({\bf x},{\bf u},\nabla{\bf u},p)p'({\bf x};V){\rm d}{\bf x}\\
        &+ \int_\Gamma J_\Omega({\bf x},{\bf u},\nabla{\bf u},p)V(0)\cdot{\bf n} + {\bf n}^\top\nabla_{\nabla{\bf u}}J_\Omega({\bf x},{\bf u},\nabla{\bf u},p){\bf u}'({\bf x};V) + \nabla_{\bf u}J_\Gamma({\bf x},{\bf u},\nabla{\bf u},p,{\bf n},{\bf t})\cdot{\bf u}'({\bf x};V)\\
        &\hspace{1cm}+ \nabla_{\nabla{\bf u}}J_\Gamma({\bf x},{\bf u},\nabla{\bf u},p,{\bf n},{\bf t}):\nabla{\bf u}'({\bf x};V) + \partial_pJ_\Gamma({\bf x},{\bf u},\nabla{\bf u},p,{\bf n},{\bf t})p'({\bf x};V) + \nabla_{\bf n}J_\Gamma({\bf x},{\bf u},\nabla{\bf u},p,{\bf n},{\bf t})\cdot{\bf n}'({\bf x};V)\\
        &\hspace{1cm}+ \nabla_{\bf t}J_\Gamma({\bf x},{\bf u},\nabla{\bf u},p,{\bf n},{\bf t}):{\bf t}'({\bf x};V)\\
        &\hspace{1cm}+ \left[\nabla J_\Gamma({\bf x},{\bf u},\nabla{\bf u},p,{\bf n},{\bf t}) + \nabla_{\bf u}J_\Gamma({\bf x},{\bf u},\nabla{\bf u},p,{\bf n},{\bf t})\cdot\nabla{\bf u} + \nabla\nabla{\bf u}:\nabla_{\nabla{\bf u}}J_\Gamma({\bf x},{\bf u},\nabla{\bf u},p,{\bf n},{\bf t}) + \partial_pJ_\Gamma({\bf x},{\bf u},\nabla{\bf u},p,{\bf n},{\bf t})\nabla p\right.\\
        &\hspace{15mm} \left.+ \nabla{\bf n}\nabla_{\bf n}J_\Gamma({\bf x},{\bf u},\nabla{\bf u},p,{\bf n},{\bf t}) + \nabla{\bf t}:\nabla_{\bf t}J_\Gamma({\bf x},{\bf u},\nabla{\bf u},p,{\bf n},{\bf t})\right]\cdot{\bf n}V(0)\cdot{\bf n} + HJ_\Gamma({\bf x},{\bf u},\nabla{\bf u},p,{\bf n},{\bf t})V(0)\cdot{\bf n}{\rm d}\Gamma\\
        =&\, \int_\Gamma J_\Omega({\bf x},{\bf u},\nabla{\bf u},p)V(0)\cdot{\bf n} + \nabla_{\nabla{\bf u}}J_\Gamma({\bf x},{\bf u},\nabla{\bf u},p,{\bf n},{\bf t}):\nabla{\bf u}'({\bf x};V) + \partial_pJ_\Gamma({\bf x},{\bf u},\nabla{\bf u},p,{\bf n},{\bf t})p'({\bf x};V)\\
        &\hspace{1cm}+ \nabla_{\bf n}J_\Gamma({\bf x},{\bf u},\nabla{\bf u},p,{\bf n},{\bf t})\cdot{\bf n}'({\bf x};V) + \nabla_{\bf t}J_\Gamma({\bf x},{\bf u},\nabla{\bf u},p,{\bf n},{\bf t}):{\bf t}'({\bf x};V)\\
        &\hspace{1cm}+ \left[\nabla J_\Gamma({\bf x},{\bf u},\nabla{\bf u},p,{\bf n},{\bf t}) + \nabla_{\bf u}J_\Gamma({\bf x},{\bf u},\nabla{\bf u},p,{\bf n},{\bf t})\cdot\nabla{\bf u} + \nabla\nabla{\bf u}:\nabla_{\nabla{\bf u}}J_\Gamma({\bf x},{\bf u},\nabla{\bf u},p,{\bf n},{\bf t}) + \partial_pJ_\Gamma({\bf x},{\bf u},\nabla{\bf u},p,{\bf n},{\bf t})\nabla p\right.\\
        &\hspace{15mm} \left.+ \nabla{\bf n}\nabla_{\bf n}J_\Gamma({\bf x},{\bf u},\nabla{\bf u},p,{\bf n},{\bf t}) + \nabla{\bf t}:\nabla_{\bf t}J_\Gamma({\bf x},{\bf u},\nabla{\bf u},p,{\bf n},{\bf t})\right]\cdot{\bf n}V(0)\cdot{\bf n} + HJ_\Gamma({\bf x},{\bf u},\nabla{\bf u},p,{\bf n},{\bf t})V(0)\cdot{\bf n}\\
        &\hspace{1cm}+ (\nabla_{\bf u}{\bf Q}({\bf x},{\bf u},\nabla{\bf u},p,{\bf n},{\bf t}){\bf v}_{\rm bc})\cdot{\bf u}'({\bf x};V) + {\bf v}^\top D_{\Delta{\bf u}}{\bf P}({\bf x},{\bf u},\nabla{\bf u},\Delta{\bf u},p,\nabla p)\partial_{\bf n}{\bf u}'({\bf x};V)\\
        &\hspace{1cm}- p'({\bf x};V){\bf n}^\top\nabla_{\nabla p}{\bf P}({\bf x},{\bf u},\nabla{\bf u},\Delta{\bf u},p,\nabla p){\bf v}{\rm d}\Gamma.
    \end{align*}
    To eliminate $({\bf u}'({\bf x};V),p({\bf x};V))$ in boundary integrals, we need the explicit formula of ${\bf Q}({\bf x},{\bf u},\nabla{\bf u},p,{\bf n},{\bf t})$ (but not ${\bf f}_{\rm bc}({\bf x})$).
\end{enumerate}
Conclude: The 1st-order shape derivative of \eqref{cost functional of general stationary fluid dynamics PDEs} under the state constraint \eqref{general stationary fluid dynamics PDEs} s given by either of the following representations:
\begin{equation*}
    \boxed{\left.\begin{split}
            &dJ({\bf u},p,\Omega;V)\\
            =&\, \int_\Omega \nabla\cdot\left(J_\Omega({\bf x},{\bf u},\nabla{\bf u},p)V(0)\right){\rm d}{\bf x}\\
            &+ \int_\Gamma \nabla_{\nabla{\bf u}}J_\Gamma({\bf x},{\bf u},\nabla{\bf u},p,{\bf n},{\bf t}):\nabla{\bf u}'({\bf x};V) + \partial_pJ_\Gamma({\bf x},{\bf u},\nabla{\bf u},p,{\bf n},{\bf t})p'({\bf x};V) + \nabla_{\bf n}J_\Gamma({\bf x},{\bf u},\nabla{\bf u},p,{\bf n},{\bf t})\cdot{\bf n}'({\bf x};V)\\
            &\hspace{1cm}+ \nabla_{\bf t}J_\Gamma({\bf x},{\bf u},\nabla{\bf u},p,{\bf n},{\bf t}):{\bf t}'({\bf x};V) + \nabla\left(J_\Gamma({\bf x},{\bf u},\nabla{\bf u},p,{\bf n},{\bf t})\right)\cdot V(0)\\
            &\hspace{1cm}+ J_\Gamma({\bf x},{\bf u},\nabla{\bf u},p,{\bf n},{\bf t})\left(\nabla\cdot V(0) - DV(0){\bf n}\cdot{\bf n}\right) + \left(\nabla_{\bf u}{\bf Q}({\bf x},{\bf u},\nabla{\bf u},p,{\bf n},{\bf t}){\bf v}_{\rm bc}\right)\cdot{\bf u}'({\bf x};V)\\
            &\hspace{1cm}+ {\bf v}^\top D_{\Delta{\bf u}}{\bf P}({\bf x},{\bf u},\nabla{\bf u},\Delta{\bf u},p,\nabla p)\partial_{\bf n}{\bf u}'({\bf x};V) - p'({\bf x};V){\bf n}^\top\nabla_{\nabla p}{\bf P}({\bf x},{\bf u},\nabla{\bf u},\Delta{\bf u},p,\nabla p){\bf v}{\rm d}\Gamma\\
            =&\, \int_\Omega \nabla\cdot\left(J_\Omega({\bf x},{\bf u},\nabla{\bf u},p)V(0)\right){\rm d}{\bf x}\\
            &+ \int_\Gamma \nabla_{\nabla{\bf u}}J_\Gamma({\bf x},{\bf u},\nabla{\bf u},p,{\bf n},{\bf t}):\nabla{\bf u}'({\bf x};V) + \partial_pJ_\Gamma({\bf x},{\bf u},\nabla{\bf u},p,{\bf n},{\bf t})p'({\bf x};V)\\
            &\hspace{1cm}+ \nabla_{\bf n}J_\Gamma({\bf x},{\bf u},\nabla{\bf u},p,{\bf n},{\bf t})\cdot{\bf n}'({\bf x};V) + \nabla_{\bf t}J_\Gamma({\bf x},{\bf u},\nabla{\bf u},p,{\bf n},{\bf t}):{\bf t}'({\bf x};V)\\
            &\hspace{1cm}+ \left[\nabla J_\Gamma({\bf x},{\bf u},\nabla{\bf u},p,{\bf n},{\bf t}) + \nabla_{\bf u}J_\Gamma({\bf x},{\bf u},\nabla{\bf u},p,{\bf n},{\bf t})\cdot\nabla{\bf u} + \nabla\nabla{\bf u}:\nabla_{\nabla{\bf u}}J_\Gamma({\bf x},{\bf u},\nabla{\bf u},p,{\bf n},{\bf t})\right.\\
            &\hspace{15mm} \left.+ \partial_pJ_\Gamma({\bf x},{\bf u},\nabla{\bf u},p,{\bf n},{\bf t})\nabla p + \nabla{\bf n}\nabla_{\bf n}J_\Gamma({\bf x},{\bf u},\nabla{\bf u},p,{\bf n},{\bf t}) + \nabla{\bf t}:\nabla_{\bf t}J_\Gamma({\bf x},{\bf u},\nabla{\bf u},p,{\bf n},{\bf t})\right]\cdot V(0)\\
            &\hspace{1cm}+ J_\Gamma({\bf x},{\bf u},\nabla{\bf u},p,{\bf n},{\bf t})\left(\nabla\cdot V(0) - DV(0){\bf n}\cdot{\bf n}\right) + \left(\nabla_{\bf u}{\bf Q}({\bf x},{\bf u},\nabla{\bf u},p,{\bf n},{\bf t}){\bf v}_{\rm bc}\right)\cdot{\bf u}'({\bf x};V)\\
            &\hspace{1cm}+ {\bf v}^\top D_{\Delta{\bf u}}{\bf P}({\bf x},{\bf u},\nabla{\bf u},\Delta{\bf u},p,\nabla p)\partial_{\bf n}{\bf u}'({\bf x};V) - p'({\bf x};V){\bf n}^\top\nabla_{\nabla p}{\bf P}({\bf x},{\bf u},\nabla{\bf u},\Delta{\bf u},p,\nabla p){\bf v}{\rm d}\Gamma\\
            =&\, \int_\Gamma J_\Omega({\bf x},{\bf u},\nabla{\bf u},p)V(0)\cdot{\bf n} + \nabla_{\nabla{\bf u}}J_\Gamma({\bf x},{\bf u},\nabla{\bf u},p,{\bf n},{\bf t}):\nabla{\bf u}'({\bf x};V) + \partial_pJ_\Gamma({\bf x},{\bf u},\nabla{\bf u},p,{\bf n},{\bf t})p'({\bf x};V)\\
            &\hspace{1cm}+ \nabla_{\bf n}J_\Gamma({\bf x},{\bf u},\nabla{\bf u},p,{\bf n},{\bf t})\cdot{\bf n}'({\bf x};V) + \nabla_{\bf t}J_\Gamma({\bf x},{\bf u},\nabla{\bf u},p,{\bf n},{\bf t}):{\bf t}'({\bf x};V)\\
            &\hspace{1cm}+ \partial_{\bf n}\left(J_\Gamma({\bf x},{\bf u},\nabla{\bf u},p,{\bf n},{\bf t})\right)V(0)\cdot{\bf n} + HJ_\Gamma({\bf x},{\bf u},\nabla{\bf u},p,{\bf n},{\bf t})V(0)\cdot{\bf n}\\
            &\hspace{1cm}+ (\nabla_{\bf u}{\bf Q}({\bf x},{\bf u},\nabla{\bf u},p,{\bf n},{\bf t}){\bf v}_{\rm bc})\cdot{\bf u}'({\bf x};V) + {\bf v}^\top D_{\Delta{\bf u}}{\bf P}({\bf x},{\bf u},\nabla{\bf u},\Delta{\bf u},p,\nabla p)\partial_{\bf n}{\bf u}'({\bf x};V)\\
            &\hspace{1cm}- p'({\bf x};V){\bf n}^\top\nabla_{\nabla p}{\bf P}({\bf x},{\bf u},\nabla{\bf u},\Delta{\bf u},p,\nabla p){\bf v}{\rm d}\Gamma\\
            =&\, \int_\Gamma J_\Omega({\bf x},{\bf u},\nabla{\bf u},p)V(0)\cdot{\bf n} + \nabla_{\nabla{\bf u}}J_\Gamma({\bf x},{\bf u},\nabla{\bf u},p,{\bf n},{\bf t}):\nabla{\bf u}'({\bf x};V) + \partial_pJ_\Gamma({\bf x},{\bf u},\nabla{\bf u},p,{\bf n},{\bf t})p'({\bf x};V)\\
            &\hspace{1cm}+ \nabla_{\bf n}J_\Gamma({\bf x},{\bf u},\nabla{\bf u},p,{\bf n},{\bf t})\cdot{\bf n}'({\bf x};V) + \nabla_{\bf t}J_\Gamma({\bf x},{\bf u},\nabla{\bf u},p,{\bf n},{\bf t}):{\bf t}'({\bf x};V)\\
            &\hspace{1cm}+ \left[\nabla J_\Gamma({\bf x},{\bf u},\nabla{\bf u},p,{\bf n},{\bf t}) + \nabla_{\bf u}J_\Gamma({\bf x},{\bf u},\nabla{\bf u},p,{\bf n},{\bf t})\cdot\nabla{\bf u} + \nabla\nabla{\bf u}:\nabla_{\nabla{\bf u}}J_\Gamma({\bf x},{\bf u},\nabla{\bf u},p,{\bf n},{\bf t}) + \partial_pJ_\Gamma({\bf x},{\bf u},\nabla{\bf u},p,{\bf n},{\bf t})\nabla p\right.\\
            &\hspace{15mm} \left.+ \nabla{\bf n}\nabla_{\bf n}J_\Gamma({\bf x},{\bf u},\nabla{\bf u},p,{\bf n},{\bf t}) + \nabla{\bf t}:\nabla_{\bf t}J_\Gamma({\bf x},{\bf u},\nabla{\bf u},p,{\bf n},{\bf t})\right]\cdot{\bf n}V(0)\cdot{\bf n} + HJ_\Gamma({\bf x},{\bf u},\nabla{\bf u},p,{\bf n},{\bf t})V(0)\cdot{\bf n}\\
            &\hspace{1cm}+ (\nabla_{\bf u}{\bf Q}({\bf x},{\bf u},\nabla{\bf u},p,{\bf n},{\bf t}){\bf v}_{\rm bc})\cdot{\bf u}'({\bf x};V) + {\bf v}^\top D_{\Delta{\bf u}}{\bf P}({\bf x},{\bf u},\nabla{\bf u},\Delta{\bf u},p,\nabla p)\partial_{\bf n}{\bf u}'({\bf x};V)\\
            &\hspace{1cm}- p'({\bf x};V){\bf n}^\top\nabla_{\nabla p}{\bf P}({\bf x},{\bf u},\nabla{\bf u},\Delta{\bf u},p,\nabla p){\bf v}{\rm d}\Gamma
        \end{split}\right.}
\end{equation*}

\section{Stationary incompressible viscous Navier-Stokes equations}
Consider the following stationary incompressible viscous Navier-Stokes equations:
\begin{equation}
    \label{stationary incompressible viscous NSEs}
    \tag{NS}
    \left\{\begin{split}
        -\nabla\cdot\left(\nu({\bf x},{\bf u},\nabla{\bf u},p)\nabla{\bf u}\right) + ({\bf u}\cdot\nabla){\bf u} + \nabla p &= {\bf f}({\bf x},{\bf u},\nabla{\bf u},p) &&\mbox{ in } \Omega,\\
        \nabla\cdot{\bf u} &= f_{\rm div}({\bf x},{\bf u},\nabla{\bf u},p) &&\mbox{ in } \Omega.
    \end{split}\right.
\end{equation}

\section{Weak formulations for stationary incompressible viscous Navier-Stokes equations}
Test both sides of the 1st equation of \eqref{general stationary incompressible NSEs} with a test function ${\bf v}$ and those of the 2nd one with a test function $q$ over $\Omega$:
\begin{equation}
    \label{tested general stationary incompressible NSEs}
    \tag{test-gsincNS}
    \left\{\begin{split}
        \int_\Omega -\nabla\cdot\left(\nu({\bf x},{\bf u},\nabla{\bf u},p)\nabla{\bf u}\right)\cdot{\bf v} + \left({\bf u}\cdot\nabla\right){\bf u}\cdot{\bf v} + \nabla p\cdot{\bf v}{\rm d}{\bf x} &= \int_\Omega {\bf f}({\bf x},{\bf u},\nabla{\bf u},p)\cdot{\bf v}{\rm d}{\bf x},\\
        \int_\Omega q\nabla\cdot{\bf u}{\rm d}{\bf x} &= \int_\Omega qf_{\rm div}({\bf x},{\bf u},\nabla{\bf u},p){\rm d}{\bf x}.
    \end{split}\right.
\end{equation}
Apply \eqref{integration by parts for matrix 2} for the 1st term in the l.h.s. of the 1st equation of \eqref{tested general stationary incompressible NSEs}:
\begin{align*}
    \int_\Omega -\nabla\cdot\left(\nu({\bf x},{\bf u},\nabla{\bf u},p)\nabla{\bf u}\right)\cdot{\bf v}{\rm d}{\bf x} = \int_\Omega \nu({\bf x},{\bf u},\nabla{\bf u},p)\nabla{\bf u}:\nabla{\bf v}{\rm d}{\bf x} - \int_\Gamma \nu({\bf x},{\bf u},\nabla{\bf u},p){\bf n}^\top\nabla{\bf u}{\bf v}{\rm d}\Gamma.
\end{align*}
Apply \eqref{ibp} for the 3rd term in the l.h.s. of the 1st equation of \eqref{tested general stationary incompressible NSEs}:
\begin{align*}
    \int_\Omega \nabla p\cdot{\bf v}{\rm d}{\bf x} = -\int_\Omega p\nabla\cdot{\bf v}{\rm d}{\bf x} + \int_\Gamma p{\bf v}\cdot{\bf n}{\rm d}\Gamma.
\end{align*} 
Keep the 2nd equation of \eqref{tested general stationary incompressible NSEs}, then it becomes
\begin{equation}
    \label{weak formulation of general stationary incompressible NSEs}
    \tag{wf-gsincNS}
    \left\{\begin{split}
        \int_\Omega \nu({\bf x},{\bf u},\nabla{\bf u},p)\nabla{\bf u}:\nabla{\bf v} + ({\bf u}\cdot\nabla){\bf u}\cdot{\bf v} - p\nabla\cdot{\bf v} - {\bf f}({\bf x},{\bf u},\nabla{\bf u},p)\cdot{\bf v}{\rm d}{\bf x} + \int_\Gamma p{\bf v}\cdot{\bf n} - \nu({\bf x},{\bf u},\nabla{\bf u},p){\bf n}^\top\nabla{\bf u}{\bf v}{\rm d}\Gamma &= 0,\\
        \int_\Omega q\nabla\cdot{\bf u} - qf_{\rm div}({\bf x},{\bf u},\nabla{\bf u},p){\rm d}{\bf x} &= 0.
    \end{split}\right.    
\end{equation}

\section{A general cost functionals \& its associated optimization problem}
The optimization problem associated with the cost functional \eqref{cost functional of general stationary fluid dynamics PDEs} can be formulated as follows:

Find $\Omega$ over a class of admissible domain $\mathcal{O}_{\rm ad}$ s.t. the cost functional \eqref{cost functional of general stationary fluid dynamics PDEs} is minimized subject to \eqref{general stationary incompressible NSEs}, i.e.:
\begin{align}
    \label{optimization problem for general stationary incompressible NSEs}
    \tag{opt-gsincNS}
    \min_{\Omega\in\mathcal{O}_{\rm ad}} J({\bf u},p,\Omega) \mbox{ s.t. } ({\bf u},p) \mbox{ solves } \eqref{general stationary incompressible NSEs}.
\end{align}

%------------------------------------------------------------------------------%

\chapter{Shape Optimization for General Conservation Laws}

\section{General conservation equation}
We consider the following general conservation equation in differential form (see, e.g., \cite{Ferziger_Peric_Street2020}, \cite[Sect. 3.7]{Moukalled_Mangani_Darwish2016}):
\begin{align}
    \label{general conservation: differential form}
    \tag{gcl}
    \partial_t(\rho\phi) + \nabla\cdot(\rho{\bf u}\phi) = \nabla\cdot(\Gamma^\phi\nabla\phi) + q^\phi,
\end{align}
which may be rewritten as
\begin{align*}
    \partial_t(\rho\phi) + \nabla\cdot{\bf J}^\phi = q^\phi,
\end{align*}
where the \textit{total flux} ${\bf J}^\phi$ is the sum of the \textit{convective} and \textit{diffusive fluxes} given by
\begin{align*}
    {\bf J}^\phi\coloneqq{\bf J}_{\rm conv}^\phi + {\bf J}_{\rm diff}^\phi,\mbox{ where }{\bf J}_{\rm conv}^\phi\coloneqq\rho{\bf u}\phi,\ {\bf J}_{\rm diff}^\phi\coloneqq-\Gamma^\phi\nabla\phi.
\end{align*}
The integral form of the \textit{generic conservation equation} (see, e.g., \cite{Ferziger_Peric_Street2020}):
\begin{align*}
    \partial_t\int_V \rho\phi{\rm d}V + \int_S \rho\phi{\bf v}\cdot{\bf n}{\rm d}S = \int_S \Gamma\nabla\phi\cdot{\bf n}{\rm d}S + \int_V q_\phi{\rm d}V,
\end{align*}
where $q_\phi$ is the source or sink of $\phi$.

The coordinate-free vector form of this equation is:
\begin{align*}
    \partial_t(\rho\phi) + \nabla\cdot(\rho\phi{\bf v}) = \nabla\cdot(\Gamma\nabla\phi) + q_\phi.
\end{align*}
Special features of NSEs will be described afterwards as an extension of the methods for the generic equation.

We also consider the steady-state\texttt{/}stationary form of \eqref{general conservation: differential form}:
\begin{align}
    \label{stationary general conservation: differential form}
    \tag{sgcl}
    \nabla\cdot(\rho{\bf u}\phi) = \nabla\cdot(\Gamma^\phi\nabla\phi) + q^\phi.
\end{align}
Let $(\mathcal{T},\mathcal{E},\mathcal{P})$ be an admissible finite volume mesh defined in Definition \ref{def: admissible FV mesh}. Integrating \eqref{stationary general conservation: differential form} over an element $K\in\mathcal{T}$ yields
\begin{align}
    \label{stationary general conservation: differential form/1}
    \int_K \nabla\cdot(\rho{\bf u}\phi){\rm d}{\bf x} = \int_K \nabla\cdot(\Gamma^\phi\nabla\phi){\rm d}{\bf x} + \int_K q^\phi{\rm d}{\bf x}.
\end{align}
Applying \eqref{div} with $\Omega = K$, $\boldsymbol{\phi} = \rho{\bf u}\phi$ and $\boldsymbol{\phi} = \Gamma^\phi\nabla\phi$ yields
\begin{align*}
    \int_K \nabla\cdot(\rho{\bf u}\phi){\rm d}{\bf x} &= \int_{\partial K} \rho\phi{\bf u}\cdot{\bf n}{\rm d}\partial K = \int_{\partial K} \rho\phi{\bf u}_{\bf n}{\rm d}\partial K,\\
    \int_K \nabla\cdot(\Gamma^\phi\nabla\phi){\rm d}{\bf x} &= \int_{\partial K} \Gamma^\phi\nabla\phi\cdot{\bf n}{\rm d}\partial K = \int_{\partial K} \Gamma^\phi\partial_{\bf n}\phi{\rm d}\partial K.
\end{align*}
Then \eqref{stationary general conservation: differential form/1} becomes
\begin{align*}
    \int_{\partial K} \rho\phi{\bf u}\cdot{\bf n}{\rm d}\partial K = \int_{\partial K} \Gamma^\phi\nabla\phi\cdot{\bf n}{\rm d}\partial K + \int_K q^\phi{\rm d}{\bf x}.
\end{align*}
Replacing the surface integral over the cell $K$ by a summation of the flux terms over the edges\texttt{/}faces of element $K$, the surface integrals of the convection-, diffusion-, and total fluxes become
\begin{align*}
    \int_{\partial K} {\bf J}_{\rm conv}^\phi\cdot{\bf n}{\rm d}\partial K &= \int_{\partial K} \rho\phi{\bf u}\cdot{\bf n}{\rm d}\partial K = \sum_{\sigma\in\mathcal{E}_K} \int_\sigma \rho\phi{\bf u}\cdot{\bf n}{\rm d}\sigma,\\
    \int_{\partial K} {\bf J}_{\rm diff}^\phi\cdot{\bf n}{\rm d}\partial K &= \int_{\partial K} \Gamma^\phi\nabla\phi\cdot{\bf n}{\rm d}\partial K = \sum_{\sigma\in\mathcal{E}_K} \int_\sigma \Gamma^\phi\nabla\phi\cdot{\bf n}{\rm d}\sigma,\\
    \int_{\partial K} {\bf J}^\phi\cdot{\bf n}{\rm d}\partial K &= \int_{\partial K} ({\bf J}_{\rm conv}^\phi + {\bf J}_{\rm diff}^\phi)\cdot{\bf n}{\rm d}\partial K = \sum_{\sigma\in\mathcal{E}_K} \int_\sigma (\rho\phi{\bf u} + \Gamma^\phi\nabla\phi)\cdot{\bf n}{\rm d}\sigma.
\end{align*}
Then some appropriate \textit{numerical quadratures}\texttt{/}\textit{numerical integration formulas} can be used to approximate these surface integrals and also the volume integral for $q^\phi$ (see, e.g., \cite[Chap. 7]{Isaacson_Keller1994}).

%------------------------------------------------------------------------------%

\part{Shape Optimization for Large Eddy Simulation Turbulence Models}

In this part, we consider the following governing equations (see, e.g., \cite{John2004})
\begin{equation}
    \label{general LES}
    \left\{\begin{split}
        {\bf w}_t - \nabla\cdot((2\nu + \nu_{\rm t})\boldsymbol{\varepsilon}({\bf w})) + ({\bf w}\nabla){\bf w} + \nabla r + \nabla\cdot\frac{\delta^2}{2\gamma}\left(A(\nabla{\bf w}\otimes\nabla{\bf w})\right) &= {\bf f},&&\mbox{ in }(0,T)\times\Omega,\\
        \nabla\cdot{\bf w} &= 0,&&\mbox{ in }[0,T]\times\Omega.
    \end{split}\right.
\end{equation}

\chapter{Shape Optimization for Smagorinsky Turbulence Model}

In \cite{Mohammadi_Pironneau1994}: ``Mathematically the Smagorinsky system is better than Navier-Stokes' because there is existence, uniqueness and regularity even in 3D (Lions[1968])''.

\section{Notation}
Let $\Omega\subset\mathbb{R}^d$, $d\in\{2,3\}$, a bounded domain with Lipschitz boundary $\Gamma$ given as $\Gamma := \Gamma_{\rm in}\cup\Gamma_{\rm wall}\cup\Gamma_{\rm out}$ and $T > 0$. Denote by
\begin{align*}
    \varepsilon({\bf v}) := \frac{1}{2}(\nabla{\bf v} + \nabla{\bf v}^\top)
\end{align*}
the symmetrized gradient of ${\bf v}\in H^1(\Omega)$.

The Banach space
\begin{align*}
    W_{0,{\rm div}}^{1,3}(\Omega) = \{{\bf v}\in W^{1,3}(\Omega);{\bf v}|_{\partial\Omega} = {\bf 0},\ \nabla\cdot{\bf v} = 0 \mbox{ in } \Omega\}
\end{align*}
is equipped with the same norm as $W_0^{1,3}(\Omega)$. Let
\begin{align*}
    V = H^1(0,T;L^2(\Omega))\cap L^3(0,T;W_{0,{\rm div}}^{1,3}(\Omega))
\end{align*}
equipped with
\begin{align*}
    \|{\bf v}\|_V = \|\nabla{\bf v}\|_{L^3(0,T;L^3(\Omega))} + \|{\bf v}_t\|_{L^2(0,T;L^2(\Omega))}.
\end{align*}

\section{Smagorinsky turbulence model with homogeneous boundary conditions}
We consider the following instationary NSEs:
\begin{equation}
    \label{instationary NSEs}
    \left\{\begin{split}
        {\bf u}_t - \nu\Delta{\bf u} + ({\bf u}\cdot\nabla){\bf u} + \nabla p &= {\bf f} &&\mbox{ in } (0,T]\times\Omega,\\
        \nabla\cdot{\bf u} &= 0 &&\mbox{ in } [0,T]\times\Omega,\\
        {\bf u}(0,\cdot) &= {\bf u}_0 &&\mbox{ in } \Omega,\\
        {\bf u} &= {\bf 0} &&\mbox{ on } [0,T]\times\Gamma,\\
        \int_\Omega p{\rm d}x &= 0 &&\mbox{ in } (0,T]
    \end{split}\right.
\end{equation}
The initial flow field ${\bf u}_0(x)$ is also divergence-free, i.e. $\nabla\cdot{\bf u}_0 = 0$ in $\Omega$.

We define the Smagorinsky turbulence models
\begin{equation}
    \label{Smagorinsky}
    \left\{\begin{split}
        {\bf w}_t - \nabla\cdot((\nu + \nu_S\|\nabla{\bf w}\|_F)\nabla{\bf w}) + ({\bf w}\cdot\nabla){\bf w} + \nabla r &= {\bf f} &&\mbox{ in } (0,T]\times\Omega,\\
        \nabla\cdot{\bf w} &= 0 &&\mbox{ in } [0,T]\times\Omega,\\
        {\bf w} &= {\bf 0} &&\mbox{ on } [0,T]\times\Gamma,\\
        {\bf w}(0,\cdot) &= {\bf w}_0 &&\mbox{ in } \Omega,\\
        \int_\Omega r{\rm d}x &= 0 &&\mbox{ in } (0,T],
    \end{split}\right.
\end{equation}
with $\nu_S > 0$, ${\bf f}\in L^2(0,T;L^2(\Omega))$ and $T\in(0,\infty)$.

\subsection{Weak formulation}
The weak formulation of \eqref{Smagorinsky} reads:

Find ${\bf w}\in V$ s.t. ${\bf w}(0,x) = {\bf w}_0\in W_{0,{\rm div}}^{1,3}(\Omega)$ and for all ${\bf v}\in V$
\begin{align}
    \label{weak formulation Smagorinksy}
    \int_0^T {\bf w}_t\cdot{\bf v} + ({\bf w}\cdot\nabla){\bf w}\cdot{\bf v} + (\nu + \nu_S\|\nabla{\bf w}\|_F)\nabla{\bf w}\cdot\nabla{\bf v}{\rm d}t = \int_0^T {\bf f}\cdot{\bf v}{\rm d}t.
\end{align}

\begin{definition}
    A function ${\bf w}\in V$ satisfying \eqref{weak formulation Smagorinksy} is called a \emph{weak solution} of the Smagorinsky model.
\end{definition}
See \cite[Lemmas 6.1, 6.2 and 6.3, pp. 75--76]{John2004}:
\begin{lemma}
    Assume that \eqref{Smagorinsky} has a sufficiently smooth solution $({\bf w},r)$. Then, for $T > 0$, this solution satisfies
    \begin{align*}
        \|{\bf w}(T)\|_{L^2(\Omega)}&\le\|{\bf w}_0\|_{L^2(\Omega)} + \int_0^T \|{\bf f}(t,x)\|_{L^2(\Omega)}{\rm d}t,\\
        \|{\bf w}(T,x)\|_{L^2(\Omega)}^2 + 2\int_0^T (\nu + \nu_S\|\nabla{\bf w}\|_F)\nabla{\bf w}\cdot\nabla{\bf w}{\rm d}t&\le 2\|{\bf w}_0\|_{L^2(\Omega)}^2 + 3\left(\int_0^T \|{\bf f}(t)\|_{L^2(\Omega)}{\rm d}t\right)^2 = c_1(T),\\
        \|\nabla{\bf w}(T)\|_{L^3(\Omega)}^3 + \frac{3}{2\nu_S}\int_0^T \|{\bf w}_t\|_{L^2(\Omega)}^2{\rm d}t&\le c_2(T).
    \end{align*}
\end{lemma}
The nonlinear viscous operator ${\bf A}: L^3(\Omega)\to L^{3/2}(\Omega)$ with
\begin{align*}
    {\bf A}(\nabla{\bf w}^n) = (\nu + \nu_S\|\nabla{\bf w}^n\|_F)\nabla{\bf w}^n. 
\end{align*}
See \cite[Lemma 6.9, pp. 85]{John2004}:
\begin{lemma}
    For arbitrary functions ${\bf w}',{\bf w}''\in W^{1,3}(\Omega)$ holds the estimate
    \begin{align}
        \int_\Omega ({\bf A}(\nabla{\bf w}') - A(\nabla{\bf w}'')):(\nabla{\bf w}' - \nabla{\bf w}''){\rm d}x\ge\nu\|\nabla{\bf w}' - \nabla{\bf w}''\|_{L^2(\Omega)}^2.
    \end{align}
    Moreover, the Smagorinsky term defines a monotone operator from $L^3(\Omega)$ into $L^{3/2}(\Omega)$.
\end{lemma}
See \cite[Theorem 6.12, p. 89]{John2004}:
\begin{theorem}[Existence of a weak solution]
    Problem \eqref{weak formulation Smagorinksy} possesses at least one solution ${\bf w}\in V$ for arbitrary ${\bf f}\in L^2(0,T;L^2(\Omega))$ and ${\bf w}_0\in W_{0,{\rm div}}^{1,3}(\Omega)$.
\end{theorem}













\section{Smagorinsky turbulence model with mixed boundary conditions}
We consider the following instationary NSEs:
\begin{equation}
    \label{instationary NSEs mixed BCs}
    \left\{\begin{split}
        {\bf u}_t - \nu\Delta{\bf u} + ({\bf u}\cdot\nabla){\bf u} + \nabla p &= {\bf f} &&\mbox{ in } (0,T]\times\Omega,\\
        \nabla\cdot{\bf u} &= 0 &&\mbox{ in } [0,T]\times\Omega,\\
        {\bf u}(0,\cdot) &= {\bf u}_0 &&\mbox{ in } \Omega,\\
        {\bf u} &= {\bf f}_{\rm in} &&\mbox{ on } [0,T]\times\Gamma_{\rm in},\\
        {\bf u} &= {\bf 0} &&\mbox{ on } [0,T]\times\Gamma_{\rm wall},\\
        -\nu\partial_{\bf n}{\bf u} + p{\bf n} &= {\bf 0} &&\mbox{ on } [0,T]\times\Gamma_{\rm out}.
    \end{split}\right.
\end{equation}
We define the Smagorinsky turbulence models
\begin{equation}
    \label{Smagorinsky mixed BCs}
    \left\{\begin{split}
        {\bf w}_t - \nabla\cdot((2\nu + \nu_t)\varepsilon({\bf w})) + ({\bf w}\cdot\nabla){\bf w} + \nabla r &= {\bf f} &&\mbox{ in } (0,T]\times\Omega,\\
        \nabla\cdot{\bf w} &= 0 &&\mbox{ in } [0,T]\times\Omega,\\
        {\bf w}(0,\cdot) &= {\bf w}_0 &&\mbox{ in } \Omega,\\
        {\bf w} &= {\bf f}_{\rm in} &&\mbox{ on } [0,T]\times\Gamma_{\rm in},\\
        {\bf w} &= {\bf 0} &&\mbox{ on } [0,T]\times\Gamma_{\rm wall},\\
        -\nu\partial_{\bf n}{\bf w} + r{\bf n} &= {\bf 0} &&\mbox{ on } [0,T]\times\Gamma_{\rm out},
    \end{split}\right.
\end{equation}
where $\nu_t = \nu_S\|\nabla{\bf w}\|_{\rm F}$

%------------------------------------------------------------------------------%

\part{Shape Optimization for Reynolds-Averaged Navier-Stokes Equations}

\chapter{Shape Optimization for $k$-$\epsilon$ Turbulence Model}

\section{Derivation of $k$-$\epsilon$ turbulence model}
Reynolds hypothesis is that the turbulence in the flow is a local function of  $2\boldsymbol{\varepsilon}(\overline{\bf u}) = \nabla\overline{\bf u} + (\nabla\overline{\bf u})^\top$, i.e., $R(t,{\bf x}) = R(2\boldsymbol{\varepsilon}(\overline{\bf u}(t,{\bf x})))$ (see, e.g., \cite{Mohammadi_Pironneau1994, Mohammadi_Pironneau2010}).

Denote by $k$ the kinetic energy of small scales and $\epsilon$ their rate of viscous energy dissipation:
\begin{align}
    k\coloneqq\frac{1}{2}\overline{|{\bf u}'|^2},\ \epsilon\coloneqq 2\nu\overline{\|\boldsymbol{\varepsilon}({\bf u}')\|_{\rm F}^2} = \frac{\nu}{2}\overline{\|\nabla{\bf u}' + (\nabla{\bf u}')^\top\|_{\rm F}^2},
\end{align}
where $\|\cdot\|_{\rm F}$ denotes the Frobenius norm of a matrix.

For 2D mean flows and for some $\alpha(t,{\bf x})$,
\begin{align*}
    R = 2\nu_{\rm t}\boldsymbol{\varepsilon}(\overline{\bf u}) + \alpha I,\ \nu_{\rm t} = c_\mu\frac{k^2}{\epsilon},
\end{align*}
and $k$ and $\epsilon$ are modeled by
\begin{equation}
    \left\{\begin{split}
        \partial_tk + \overline{\bf u}\cdot\nabla k - 2c_\mu\frac{k^2}{\epsilon}\|\boldsymbol{\varepsilon}(\overline{\bf u})\|_{\rm F}^2 - \nabla\cdot\left(c_\mu\frac{k^2}{\epsilon}\nabla k\right) + \epsilon &= 0,\\
        \partial_t\epsilon + \overline{\bf u}\cdot\nabla\epsilon - 2c_1k\|\boldsymbol{\varepsilon}(\overline{\bf u})\|_{\rm F}^2 - \nabla\cdot\left(c_\epsilon\frac{k^2}{\epsilon}\nabla\epsilon\right) + c_2\frac{\epsilon^2}{k} &= 0,
    \end{split}\right.
\end{equation}
with $c_\mu = 0.09$, $c_1 = 0.126$, $c_2 = 1.92$, $c_\epsilon = 0.07$.

Given a filter $\langle\cdot\rangle$, the incompressible Reynolds averaged NSEs for the \textit{mean flow} $\overline{\bf u}$ and \textit{mean pressure} $\overline{p}$ are
\begin{equation}
    \left\{\begin{split}
        \overline{\bf u}_t - \nu\Delta \overline{\bf u} + (\overline{\bf u}\cdot\nabla)\overline{\bf u} + \nabla \overline{p} - \nabla\cdot R(k,\varepsilon,\nabla \overline{\bf u} + \nabla \overline{\bf u}^\top) &= 0 &&\mbox{ in } (0,T)\times\Omega,\\
        \nabla\cdot \overline{\bf u} &= 0 &&\mbox{ in } (0,T)\times\Omega,
    \end{split}\right.
\end{equation}
where $R_{ij} = -\langle u_iu_j\rangle$ is the Reynolds tensor. The kinetic energy of the turbulence $k$ and the rate of dissipation of turbulent energy $\varepsilon$ are defined by
\begin{align}
    k = \frac{1}{2}\langle|{\bf u}'|^2\rangle,\ \varepsilon = \frac{\nu}{2}\langle|\nabla{\bf u}' + \nabla{\bf u}'^\top|^2\rangle,
\end{align}
then $R$, $k$, $\epsilon$ are modeled in terms of the mean flow $\overline{\bf u}$ by
\begin{align}
    &R = -\frac{2}{3}kI + \left(\nu + c_\mu\frac{k^2}{\varepsilon}\right)(\nabla \overline{\bf u} + \nabla \overline{\bf u}^\top) \mbox{ in } (0,T)\times\Omega,\\
    &k_t + (\overline{\bf u}\cdot\nabla)k - \nabla\cdot\left(c_\mu\frac{k^2}{\varepsilon}\nabla k\right) - \frac{c_\mu}{2}\frac{k^2}{\varepsilon}|\nabla \overline{\bf u} + \nabla \overline{\bf u}^\top|^2 + \varepsilon = 0 \mbox{ in } (0,T)\times\Omega,\\
    &\varepsilon_t + (\overline{\bf u}\cdot\nabla)\varepsilon - \nabla\cdot\left(c_\varepsilon\frac{k^2}{\varepsilon}\nabla\varepsilon\right) - \frac{c_1}{2}k|\nabla \overline{\bf u} + \nabla \overline{\bf u}^\top|^2 + c_2\frac{\varepsilon^2}{k} = 0 \mbox{ in } (0,T)\times\Omega,
\end{align}
with $c_\mu = 0.09$, $c_1 = 0.126$, $c_2 = 1.92$, $c_\varepsilon = 0.07$.

Let $\nu_{\rm t} = c_\mu\frac{k^2}{\varepsilon}$, then the $k$-$\varepsilon$ and the NSEs can be rewritten as
\begin{align}
    k_t + (\overline{\bf u}\cdot\nabla)k - \nabla\cdot(\nu_{\rm t}\nabla k) - \frac{c_\mu}{2}\frac{k^2}{\varepsilon}|\nabla \overline{\bf u} + \nabla \overline{\bf u}^\top|^2 + \varepsilon &= 0,\\
    \varepsilon_t + (\overline{\bf u}\cdot\nabla)\varepsilon - \nabla\cdot\left(\frac{c_\varepsilon}{c_\mu}\nu_{\rm t}\nabla\varepsilon\right) - \frac{c_1}{2}k|\nabla \overline{\bf u} + \nabla \overline{\bf u}^\top|^2 + c_2\frac{\varepsilon^2}{k} &= 0,\\
    \overline{\bf u}_t + (\overline{\bf u}\cdot\nabla)\overline{\bf u} - \nabla\cdot\left((\nu + \nu_{\rm t})(\nabla \overline{\bf u} + \nabla \overline{\bf u}^\top)\right) + \nabla\left(\overline{p} + \frac{2}{3}k\right) &= \overline{\bf f},\\
    \nabla\cdot \overline{\bf u} &= 0.
\end{align}

\section{Cost functional}

\section{Formal Lagrangian}
We consider the following Lagrange function: [add ICs and BCs later]
\begin{align*}
    &\mathcal{L}(\overline{\bf u},\overline{p},k,\varepsilon,\Omega,{\bf v},q,r,\eta) := J_{12}^{\epsilon,\gamma}(\overline{\bf u},\overline{p},\Omega)\\
    &- \int_0^T\int_\Omega {\bf v}\cdot\left(\overline{\bf u}_t + (\overline{\bf u}\cdot\nabla)\overline{\bf u} - \nabla\cdot\left(\left(\nu + c_\mu\frac{k^2}{\varepsilon}\right)(\nabla \overline{\bf u} + \nabla \overline{\bf u}^\top)\right) + \nabla\left(\overline{p} + \frac{2}{3}k\right) - \overline{\bf f}\right) - q\nabla\cdot \overline{\bf u}{\rm d}x{\rm d}t\\
    &- \int_0^T\int_\Omega r\left(k_t + (\overline{\bf u}\cdot\nabla)k - \nabla\cdot\left(c_\mu\frac{k^2}{\varepsilon}\nabla k\right) - \frac{c_\mu}{2}\frac{k^2}{\varepsilon}|\nabla \overline{\bf u} + \nabla \overline{\bf u}^\top|^2 + \varepsilon\right){\rm d}x{\rm d}t\\
    &- \int_0^T\int_\Omega \eta\left(\varepsilon_t + (\overline{\bf u}\cdot\nabla)\varepsilon - \nabla\cdot\left(c_\varepsilon\frac{k^2}{\varepsilon}\nabla\varepsilon\right) - \frac{c_1}{2}k|\nabla \overline{\bf u} + \nabla \overline{\bf u}^\top|^2 + c_2\frac{\varepsilon^2}{k}\right){\rm d}x{\rm d}t,
\end{align*}
where ${\bf v},q,r,\eta$ are Lagrange multipliers.

Choose the Lagrange multiplier $({\bf v},q,r,\eta)$ such that the variation with respect to the state variables vanishes identically, i.e.,
\begin{align*}
    \label{6.4.5}
    \partial_{\overline{\bf u}}\mathcal{L}\cdot\delta \overline{\bf u} + \partial_{\overline{p}}\mathcal{L}\delta \overline{p} + \partial_k\mathcal{L}\delta k + \partial_\varepsilon\mathcal{L}\delta\varepsilon = 0.
\end{align*}
Note that 
\begin{align*}
    |\nabla\overline{\bf u} + \nabla\overline{\bf u}^\top|^2 = \left(\nabla\overline{\bf u} + \nabla\overline{\bf u}^\top\right):\left(\nabla\overline{\bf u} + \nabla\overline{\bf u}^\top\right) = 2\left(\nabla\overline{\bf u}:\nabla\overline{\bf u} + \nabla\overline{\bf u}:\nabla\overline{\bf u}^\top\right),
\end{align*}
hence
\begin{align*}
    \partial_{\overline{\bf u}}\left(|\nabla\overline{\bf u} + \nabla\overline{\bf u}^\top|^2\right)\delta\overline{\bf u} &= 2\left(\nabla\delta\overline{\bf u}:\nabla\overline{\bf u} +\nabla\overline{\bf u}:\nabla\delta\overline{\bf u} + \nabla\delta\overline{\bf u}:\nabla\overline{\bf u}^\top + \nabla\overline{\bf u}:\nabla\delta\overline{\bf u}^\top\right)\\
    &= 2\left(2\nabla\overline{\bf u}:\nabla\delta\overline{\bf u} + 2\nabla\overline{\bf u}^\top:\nabla\delta\overline{\bf u}\right) = 4\left(\nabla\overline{\bf u} + \nabla\overline{\bf u}^\top\right):\nabla\delta\overline{\bf u}.
\end{align*}
Then \eqref{6.4.5} reads as
{\color{blue}\begin{align*}
        &\partial_{\overline{\bf u}}J_{12}^{\epsilon,\gamma}(\overline{\bf u},\overline{p},\Omega)\cdot\delta\overline{\bf u} + \partial_{\overline{p}}J_{12}^{\epsilon,\gamma}(\overline{\bf u},\overline{p},\Omega)\delta\overline{p}\\
        &-\int_0^T\int_\Omega {\bf v}\cdot\left(\delta\overline{\bf u}_t + (\delta\overline{\bf u}\cdot\nabla)\overline{\bf u} + (\overline{\bf u}\cdot\nabla)\delta\overline{\bf u} - \nabla\cdot\left(\left(\nu + c_\mu\frac{k^2}{\varepsilon}\right)\left(\nabla\delta\overline{\bf u} + \nabla\delta\overline{\bf u}^\top\right)\right)\right) - q\nabla\cdot\delta\overline{\bf u}{\rm d}x{\rm d}t\\
        &-\int_0^T\int_\Omega r\left((\delta\overline{\bf u}\cdot\nabla)k - 2c_\mu\frac{k^2}{\varepsilon}\left(\nabla\overline{\bf u} + \nabla\overline{\bf u}^\top\right):\nabla\delta\overline{\bf u}\right){\rm d}x{\rm d}t\\
        &-\int_0^T\int_\Omega \eta\left((\delta\overline{\bf u}\cdot\nabla)\varepsilon - 2c_1k\left(\nabla\overline{\bf u} + \nabla\overline{\bf u}^\top\right):\nabla\delta\overline{\bf u}\right){\rm d}x{\rm d}t - \int_0^T\int_\Omega {\bf v}\cdot\nabla\delta\overline{p}{\rm d}x{\rm d}t\\
        &-\int_0^T\int_\Omega {\bf v}\cdot\left(-\nabla\cdot\left(2c_\mu\frac{k\delta k}{\varepsilon}\left(\nabla\overline{\bf u} + \nabla\overline{\bf u}^\top\right)\right) + \frac{2}{3}\nabla\delta k\right){\rm d}x{\rm d}t\\
        &-\int_0^T\int_\Omega r\left(\delta k_t + (\overline{\bf u}\cdot\nabla)\delta k - \nabla\cdot\left(2c_\mu\frac{k\delta k}{\varepsilon}\nabla k + c_\mu\frac{k^2}{\varepsilon}\nabla\delta k\right) - c_\mu\frac{k\delta k}{\varepsilon}|\nabla\overline{\bf u} + \nabla\overline{\bf u}^\top|^2\right){\rm d}x{\rm d}t\\
        &+\int_0^T\int_\Omega \eta\left(\nabla\cdot\left(2c_\varepsilon\frac{k\delta k}{\varepsilon}\nabla\varepsilon\right) + \frac{c_1}{2}\delta k|\nabla\overline{\bf u} + \nabla\overline{\bf u}^\top|^2 + c_2\frac{\varepsilon^2\delta k}{k^2}\right){\rm d}x{\rm d}t\\
        &-\int_0^T\int_\Omega {\bf v}\cdot\left(\nabla\cdot\left(c_\mu\frac{k^2\delta\varepsilon}{\varepsilon^2}\left(\nabla\overline{\bf u} + \nabla\overline{\bf u}^\top\right)\right)\right){\rm d}x{\rm d}t\\
        &-\int_0^T\int_\Omega r\left(\nabla\cdot\left(c_\mu\frac{k^2\delta\varepsilon}{\varepsilon^2}\nabla k\right) + \frac{c_\mu}{2}\frac{k^2\delta\varepsilon}{\varepsilon^2}|\nabla\overline{\bf u} + \nabla\overline{\bf u}^\top|^2 + \delta\varepsilon\right){\rm d}x{\rm d}t\\
        &-\int_0^T\int_\Omega \eta\left(\delta\varepsilon_t + (\overline{\bf u}\cdot\nabla)\delta\varepsilon - \nabla\cdot\left(-c_\varepsilon\frac{k^2\delta\varepsilon}{\varepsilon^2}\nabla\varepsilon + c_\varepsilon\frac{k^2}{\varepsilon}\nabla\delta\varepsilon\right) + 2c_2\frac{\varepsilon\delta\varepsilon}{k}\right){\rm d}x{\rm d}t.
\end{align*}}
We integrate by parts all the terms which involve the derivative of directions: the term involving time derivative:
\begin{align*}
    -\int_0^T\int_\Omega {\bf v}\cdot\delta\overline{\bf u}_t{\rm d}x{\rm d}t = -\int_\Omega {\bf v}(T)\cdot\delta\overline{\bf u}(T) - {\bf v}(0)\cdot\delta\overline{\bf u}(0){\rm d}x + \int_0^T\int_\Omega {\bf v}_t\cdot\delta\overline{\bf u}{\rm d}x{\rm d}t,
\end{align*}
the term produced by the nonlinear term $(\overline{\bf u}\cdot\nabla)\overline{\bf u}$:
\begin{align*}
    -\int_\Omega {\bf v}\cdot((\overline{\bf u}\cdot\nabla)\delta\overline{\bf u}){\rm d}x = -\int_\Gamma (\overline{\bf u}\cdot{\bf n})({\bf v}\cdot\delta\overline{\bf u}){\rm d}s + \int_\Omega \left[(\overline{\bf u}\cdot\nabla){\bf v}\cdot\delta\overline{\bf u} + \nabla\cdot\overline{\bf u}({\bf v}\cdot\delta\overline{\bf u})\right]{\rm d}x = -\int_\Gamma (\overline{\bf u}\cdot{\bf n})({\bf v}\cdot\delta\overline{\bf u}){\rm d}s + \int_\Omega (\overline{\bf u}\cdot\nabla){\bf v}\cdot\delta\overline{\bf u}{\rm d}x,
\end{align*}

\paragraph*{Second-order term.} Since
\begin{align*}
    &{\bf v}\cdot\left(\nabla\cdot\left(\left(\nu + c_\mu\frac{k^2}{\varepsilon}\right)\left(\nabla\delta\overline{\bf u} + \nabla\delta\overline{\bf u}^\top\right)\right)\right)\\
    &= {\bf v}\cdot\left(\nabla\cdot\left(\left(\nu + c_\mu\frac{k^2}{\varepsilon}\right)\left(\partial_{x_i}\delta\overline{u}_j + \partial_{x_j}\delta\overline{u}_i\right)_{i,j=1}^d\right)\right)\\
    &= {\bf v}\cdot\left(\left(\nu + c_\mu\frac{k^2}{\varepsilon}\right)\left(\sum_{j=1}^d \partial_{x_ix_j}\delta\overline{u}_j + \partial_{x_j}^2\delta\overline{u}_i\right)_{i=1}^d + \left(\sum_{j=1}^d \left(\partial_{x_i}\delta\overline{u}_j + \partial_{x_j}\delta\overline{u}_i\right)c_\mu\partial_{x_j}\left(\frac{k^2}{\varepsilon}\right)\right)_{i=1}^d\right)\\
    &= \sum_{i=1}^d \left(\nu + c_\mu\frac{k^2}{\varepsilon}\right)v_i\sum_{j=1}^d\left(\partial_{x_ix_j}\delta\overline{u}_j + \partial_{x_j}^2\delta\overline{u}_i\right) + v_i\sum_{j=1}^d \left(\partial_{x_i}\delta\overline{u}_j + \partial_{x_j}\delta\overline{u}_i\right)c_\mu\partial_{x_j}\left(\frac{k^2}{\varepsilon}\right)\\
    &= \sum_{i=1}^d\sum_{j=1}^d \left(\nu + c_\mu\frac{k^2}{\varepsilon}\right)v_i\left(\partial_{x_ix_j}\delta\overline{u}_j + \partial_{x_j}^2\delta\overline{u}_i\right) + c_\mu v_i\left(\partial_{x_i}\delta\overline{u}_j + \partial_{x_j}\delta\overline{u}_i\right)\partial_{x_j}\left(\frac{k^2}{\varepsilon}\right),
\end{align*}
hence
\begin{align*}
    &\int_\Omega {\bf v}\cdot\left(\nabla\cdot\left(\left(\nu + c_\mu\frac{k^2}{\varepsilon}\right)\left(\nabla\delta\overline{\bf u} + \nabla\delta\overline{\bf u}^\top\right)\right)\right){\rm d}x\\
    &= \int_\Omega \sum_{i=1}^d\sum_{j=1}^d \left(\nu + c_\mu\frac{k^2}{\varepsilon}\right)v_i\left(\partial_{x_ix_j}\delta\overline{u}_j + \partial_{x_j}^2\delta\overline{u}_i\right) + c_\mu v_i\left(\partial_{x_i}\delta\overline{u}_j + \partial_{x_j}\delta\overline{u}_i\right)\partial_{x_j}\left(\frac{k^2}{\varepsilon}\right){\rm d}x,
\end{align*}
We integrate by part each term:
\begin{align*}
    &\int_\Omega \sum_{i=1}^d\sum_{j=1}^d \left(\nu + c_\mu\frac{k^2}{\varepsilon}\right)v_i\partial_{x_ix_j}\delta\overline{u}_j{\rm d}x\\
    =&\ \int_\Gamma \sum_{i=1}^d\sum_{j=1}^d \left(\nu + c_\mu\frac{k^2}{\varepsilon}\right)v_i\partial_{x_i}\delta\overline{u}_jn_j{\rm d}s - \int_\Omega \sum_{i=1}^d\sum_{j=1}^d \partial_{x_i}\delta\overline{u}_j\left(c_\mu\partial_{x_j}\left(\frac{k^2}{\varepsilon}\right)v_i + \left(\nu+ c_\mu\frac{k^2}{\varepsilon}\right)\partial_{x_j}v_i\right){\rm d}x\\
    =&\ \int_\Gamma \sum_{i=1}^d\sum_{j=1}^d \left(\nu + c_\mu\frac{k^2}{\varepsilon}\right)v_i\partial_{x_i}\delta\overline{u}_jn_j{\rm d}s - \int_\Gamma \sum_{i=1}^d\sum_{j=1}^d \delta\overline{u}_j\left(c_\mu\partial_{x_j}\left(\frac{k^2}{\varepsilon}\right)v_i + \left(\nu + c_\mu\frac{k^2}{\varepsilon}\right)\partial_{x_j}v_i\right)n_i{\rm d}s\\
    &+\int_\Omega \sum_{i=1}^d\sum_{j=1}^d \delta\overline{u}_j\left(c_\mu\partial_{x_ix_j}\left(\frac{k^2}{\varepsilon}\right)v_i + c_\mu\partial_{x_j}\left(\frac{k^2}{\varepsilon}\right)\partial_{x_i}v_i + c_\mu\partial_{x_i}\left(\frac{k^2}{\varepsilon}\right)\partial_{x_j}v_i + \left(\nu + c_\mu\frac{k^2}{\varepsilon}\right)\partial_{x_ix_j}v_i\right){\rm d}x,\\
    &\int_\Omega \sum_{i=1}^d\sum_{j=1}^d \left(\nu + c_\mu\frac{k^2}{\varepsilon}\right)v_i\partial_{x_j}^2\delta\overline{u}_i{\rm d}x\\
    =&\ \int_\Gamma \sum_{i=1}^d\sum_{j=1}^d \left(\nu + c_\mu\frac{k^2}{\varepsilon}\right)v_i\partial_{x_j}\delta\overline{u}_in_j{\rm d}s - \int_\Omega \sum_{i=1}^d\sum_{j=1}^d \partial_{x_j}\delta\overline{u}_i\left(c_\mu\partial_{x_j}\left(\frac{k^2}{\varepsilon}\right)v_i + \left(\nu + c_\mu\frac{k^2}{\varepsilon}\right)\partial_{x_j}v_i\right){\rm d}x\\
    =&\ \int_\Gamma \sum_{i=1}^d\sum_{j=1}^d \left(\nu + c_\mu\frac{k^2}{\varepsilon}\right)v_i\partial_{x_j}\delta\overline{u}_in_j{\rm d}s - \int_\Gamma \sum_{i=1}^d\sum_{j=1}^d \delta\overline{u}_i\left(c_\mu\partial_{x_j}\left(\frac{k^2}{\varepsilon}\right)v_i + \left(\nu + c_\mu\frac{k^2}{\varepsilon}\right)\partial_{x_j}v_i\right)n_j{\rm d}s\\
    &+\int_\Omega \sum_{i=1}^d\sum_{j=1}^d \delta\overline{u}_i\left(c_\mu\partial_{x_j}^2\left(\frac{k^2}{\varepsilon}\right)v_i + c_\mu\partial_{x_j}\left(\frac{k^2}{\varepsilon}\right)\partial_{x_j}v_i + c_\mu\partial_{x_j}\left(\frac{k^2}{\varepsilon}\right)\partial_{x_j}v_i + \left(\nu + c_\mu\frac{k^2}{\varepsilon}\right)\partial_{x_j}^2v_i\right){\rm d}x,\\
    &\int_\Omega \sum_{i=1}^d\sum_{j=1}^d c_\mu v_i\partial_{x_i}\delta\overline{u}_j\partial_{x_j}\left(\frac{k^2}{\varepsilon}\right){\rm d}x\\
    =&\ \int_\Gamma \sum_{i=1}^d\sum_{j=1}^d c_\mu v_i\delta\overline{u}_j\partial_{x_j}\left(\frac{k^2}{\varepsilon}\right)n_i{\rm d}s - \int_\Omega \sum_{i=1}^d\sum_{j=1}^d \delta\overline{u}_j\left(c_\mu\partial_{x_i}v_i\partial_{x_j}\left(\frac{k^2}{\varepsilon}\right) + c_\mu v_i\partial_{x_ix_j}\left(\frac{k^2}{\varepsilon}\right)\right){\rm d}x,\\
    &\int_\Omega \sum_{i=1}^d\sum_{j=1}^d c_\mu v_i\partial_{x_j}\delta\overline{u}_i\partial_{x_j}\left(\frac{k^2}{\varepsilon}\right){\rm d}x\\
    =&\ \int_\Gamma \sum_{i=1}^d\sum_{j=1}^d c_\mu v_i\delta\overline{u}_i\partial_{x_j}\left(\frac{k^2}{\varepsilon}\right)n_j{\rm d}s - \int_\Omega \sum_{i=1}^d\sum_{j=1}^d \delta\overline{u}_i\left(c_\mu\partial_{x_j}v_i\partial_{x_j}\left(\frac{k^2}{\varepsilon}\right) + c_\mu v_i\partial_{x_j}^2\left(\frac{k^2}{\varepsilon}\right)\right){\rm d}x.
\end{align*}
Gathering up yields
\begin{align*}
    &\int_\Omega {\bf v}\cdot\left(\nabla\cdot\left(\left(\nu + c_\mu\frac{k^2}{\varepsilon}\right)\left(\nabla\delta\overline{\bf u} + \nabla\delta\overline{\bf u}^\top\right)\right)\right){\rm d}x\\
    =&\ \int_\Omega \sum_{i=1}^d\sum_{j=1}^d \delta\overline{u}_j\left(c_\mu\partial_{x_ix_j}\left(\frac{k^2}{\varepsilon}\right)v_i + c_\mu\partial_{x_j}\left(\frac{k^2}{\varepsilon}\right)\partial_{x_i}v_i + c_\mu\partial_{x_i}\left(\frac{k^2}{\varepsilon}\right)\partial_{x_j}v_i + \left(\nu + c_\mu\frac{k^2}{\varepsilon}\right)\partial_{x_ix_j}v_i\right)\\
    &\hspace{1.5cm} + \delta\overline{u}_i\left(c_\mu\partial_{x_j}^2\left(\frac{k^2}{\varepsilon}\right)v_i + c_\mu\partial_{x_j}\left(\frac{k^2}{\varepsilon}\right)\partial_{x_j}v_i + c_\mu\partial_{x_j}\left(\frac{k^2}{\varepsilon}\right)\partial_{x_j}v_i + \left(\nu + c_\mu\frac{k^2}{\varepsilon}\right)\partial_{x_j}^2v_i\right)\\
    &\hspace{1.5cm} -\delta\overline{u}_j\left(c_\mu\partial_{x_i}v_i\partial_{x_j}\left(\frac{k^2}{\varepsilon}\right) + c_\mu v_i\partial_{x_ix_j}\left(\frac{k^2}{\varepsilon}\right)\right) - \delta\overline{u}_i\left(c_\mu\partial_{x_j}v_i\partial_{x_j}\left(\frac{k^2}{\varepsilon}\right) + c_\mu v_i\partial_{x_j}^2\left(\frac{k^2}{\varepsilon}\right)\right){\rm d}x\\
    &+\int_\Gamma \sum_{i=1}^d\sum_{j=1}^d \left(\nu + c_\mu\frac{k^2}{\varepsilon}\right)v_i\partial_{x_i}\delta\overline{u}_jn_j - \delta\overline{u}_j\left(c_\mu\partial_{x_j}\left(\frac{k^2}{\varepsilon}\right)v_i + \left(\nu + c_\mu\frac{k^2}{\varepsilon}\right)\partial_{x_j}v_i\right)n_i\\
    &\hspace{1.5cm} + \left(\nu + c_\mu\frac{k^2}{\varepsilon}\right)v_i\partial_{x_j}\delta\overline{u}_in_j - \delta\overline{u}_i\left(c_\mu\partial_{x_j}\left(\frac{k^2}{\varepsilon}\right)v_i + \left(\nu + c_\mu\frac{k^2}{\varepsilon}\right)\partial_{x_j}v_i\right)n_j\\
    &\hspace{1.5cm} + c_\mu v_i\delta\overline{u}_j\partial_{x_j}\left(\frac{k^2}{\varepsilon}\right)n_i + c_\mu v_i\delta\overline{u}_i\partial_{x_j}\left(\frac{k^2}{\varepsilon}\right)n_j{\rm d}s\\
    =&\ \int_\Omega \sum_{i=1}^d\sum_{j=1}^d \delta\overline{u}_j\left(c_\mu\partial_{x_i}\left(\frac{k^2}{\varepsilon}\right)\partial_{x_j}v_i + \left(\nu + c_\mu\frac{k^2}{\varepsilon}\right)\partial_{x_ix_j}v_i\right) + \delta\overline{u}_i\left(c_\mu\partial_{x_j}\left(\frac{k^2}{\varepsilon}\right)\partial_{x_j}v_i + \left(\nu + c_\mu\frac{k^2}{\varepsilon}\right)\partial_{x_j}^2v_i\right){\rm d}x\\
    &+ \int_\Gamma \sum_{i=1}^d\sum_{j=1}^d \left(\nu + c_\mu\frac{k^2}{\varepsilon}\right)v_i\partial_{x_i}\delta\overline{u}_jn_j - \delta\overline{u}_j\left(\nu + c_\mu\frac{k^2}{\varepsilon}\right)\partial_{x_j}v_in_i + \left(\nu + c_\mu\frac{k^2}{\varepsilon}\right)v_i\partial_{x_j}\delta\overline{u}_in_j - \delta\overline{u}_i\left(\nu + c_\mu\frac{k^2}{\varepsilon}\right)\partial_{x_j}v_in_j{\rm d}s\\
    =&\ \int_\Omega \sum_{i=1}^d\sum_{j=1}^d \left(c_\mu\partial_{x_i}\left(\frac{k^2}{\varepsilon}\right)\left(\partial_{x_i}v_j + \partial_{x_j}v_i\right) + \left(\nu + c_\mu\frac{k^2}{\varepsilon}\right)\left(\partial_{x_ix_j}v_i + \partial_{x_i}^2v_j\right)\right)\delta\overline{u}_j{\rm d}x\\
    &+ \int_\Gamma \sum_{i=1}^d\sum_{j=1}^d \left(\nu + c_\mu\frac{k^2}{\varepsilon}\right)\left(v_i\left(\partial_{x_i}\delta\overline{u}_j + \partial_{x_j}\delta\overline{u}_i\right)n_j - \delta\overline{u}_i\left(\partial_{x_i}v_j + \partial_{x_j}v_i\right)n_j\right){\rm d}s\\
    =&\ \int_\Omega 2c_\mu\nabla\left(\frac{k^2}{\varepsilon}\right)^\top\varepsilon({\bf v})\delta\overline{\bf u} + \left(\nu + c_\mu\frac{k^2}{\varepsilon}\right)\left(\nabla(\nabla\cdot{\bf v}) + \Delta{\bf v}\right)\cdot\delta\overline{\bf u}{\rm d}x + \int_\Gamma 2\left(\nu + c_\mu\frac{k^2}{\varepsilon}\right){\bf v}^\top\varepsilon(\delta\overline{\bf u}){\bf n} - 2{\bf n}^\top\varepsilon({\bf v})\delta{\bf u}{\rm d}s.
\end{align*}
\paragraph*{Divergence term.} 
\begin{align*}
    \int_\Omega q\nabla\cdot\delta\overline{\bf u}{\rm d}x = -\int_\Omega \delta\overline{\bf u}\cdot\nabla q{\rm d}x + \int_\Gamma q\delta\overline{\bf u}\cdot{\bf n}{\rm d}s,
\end{align*}
\paragraph*{Term.}
\begin{align*}
    &\int_\Omega 2c_\mu r\frac{k^2}{\varepsilon}\left(\nabla\overline{\bf u} + \nabla\overline{\bf u}^\top\right):\nabla\delta\overline{\bf u}{\rm d}x = \int_\Omega 2c_\mu\sum_{i=1}^d\sum_{j=1}^d r\frac{k^2}{\varepsilon}\left(\partial_{x_i}\overline{u}_j + \partial_{x_j}\overline{u}_i\right)\partial_{x_i}\delta\overline{u}_j{\rm d}x\\
    =&\ \int_\Gamma 2c_\mu\sum_{i=1}^d\sum_{j=1}^d r\frac{k^2}{\varepsilon}\left(\partial_{x_i}\overline{u}_j + \partial_{x_j}\overline{u}_i\right)\delta\overline{u}_jn_i{\rm d}s\\
    &-\int_\Omega 2c_\mu\sum_{i=1}^d\sum_{j=1}^d \delta\overline{u}_j\left(\partial_{x_i}r\frac{k^2}{\varepsilon}\left(\partial_{x_i}\overline{u}_j + \partial_{x_j}\overline{u}_i\right) + r\partial_{x_i}\left(\frac{k^2}{\varepsilon}\right)\left(\partial_{x_i}\overline{u}_j + \partial_{x_j}\overline{u}_i\right) + r\frac{k^2}{\varepsilon}\left(\partial_{x_i}^2\overline{u}_j + \partial_{x_ix_j}\overline{u}_i\right)\right){\rm d}x\\
    =&\ \int_\Gamma 4c_\mu r\frac{k^2}{\varepsilon}{\bf n}^\top\varepsilon(\overline{\bf u})\delta\overline{\bf u}{\rm d}s - \int_\Omega 2c_\mu\left(2\frac{k^2}{\varepsilon}\nabla r^\top\varepsilon(\overline{\bf u})\delta\overline{\bf u} + 2r\nabla\left(\frac{k^2}{\varepsilon}\right)^\top\varepsilon(\overline{\bf u})\delta\overline{\bf u} + r\frac{k^2}{\varepsilon}\left(\Delta\overline{\bf u}\cdot\delta\overline{\bf u} + \nabla(\nabla\cdot\overline{\bf u})\cdot\delta\overline{\bf u}\right)\right){\rm d}x.
\end{align*}
\paragraph*{Term.}
\begin{align*}
    &\int_\Omega 2c_1\eta k\left(\nabla\overline{\bf u} + \nabla\overline{\bf u}^\top\right):\nabla\delta\overline{\bf u}{\rm d}x = \int_\Omega 2c_1\sum_{i=1}^d\sum_{j=1}^d \eta k\left(\partial_{x_i}\overline{u}_j + \partial_{x_j}\overline{u}_i\right)\partial_{x_i}\delta\overline{u}_j{\rm d}x\\
    =&\ \int_\Gamma 2c_1\sum_{i=1}^d\sum_{j=1}^d \eta k\left(\partial_{x_i}\overline{u}_j + \partial_{x_j}\overline{u}_i\right)\delta\overline{u}_jn_i{\rm d}s\\
    &-\int_\Omega 2c_1\sum_{i=1}^d\sum_{j=1}^d \delta\overline{u}_j\left(\partial_{x_i}\eta k\left(\partial_{x_i}\overline{u}_j + \partial_{x_j}\overline{u}_i\right) + \eta\partial_{x_i}k\left(\partial_{x_i}\overline{u}_j + \partial_{x_j}\overline{u}_i\right) + \eta k\left(\partial_{x_i}^2\overline{u}_j + \partial_{x_ix_j}\overline{u}_i\right)\right){\rm d}x\\
    =&\ \int_\Gamma 4c_1\eta k{\bf n}^\top\varepsilon(\overline{\bf u})\delta\overline{\bf u}{\rm d}s - \int_\Omega 2c_1\left(2k\nabla\eta^\top\varepsilon(\overline{\bf u})\delta\overline{\bf u} + \eta\nabla k^\top\varepsilon(\overline{\bf u})\delta\overline{\bf u} + \eta k\left(\Delta\overline{\bf u}\cdot\delta\overline{\bf u} + \nabla(\nabla\cdot\overline{\bf u})\cdot\delta\overline{\bf u}\right)\right){\rm d}x.
\end{align*}
\paragraph*{Term produced by $\nabla\overline{p}$.}
\begin{align*}
    -\int_\Omega {\bf v}\cdot\nabla\delta\overline{p}{\rm d}x = - \int_\Gamma \delta\overline{p}{\bf v}\cdot{\bf n}{\rm d}s + \int_\Omega \delta\overline{p}\nabla\cdot{\bf v}{\rm d}x.
\end{align*}
\paragraph*{Term.} Since
\begin{align*}
    &{\bf v}\cdot\left(\nabla\cdot\left(2c_\mu\frac{k\delta k}{\varepsilon}\left(\nabla\overline{\bf u} + \nabla\overline{\bf u}^\top\right)\right)\right) = {\bf v}\cdot\left(\nabla\cdot\left(2c_\mu\frac{k\delta k}{\varepsilon}\left(\partial_{x_i}\overline{u}_j + \partial_{x_j}\overline{u}_i\right)_{i,j=1}^d\right)\right)\\
    =&\ {\bf v}\cdot\left(2c_\mu\frac{k\delta k}{\varepsilon}\left(\sum_{j=1}^d \partial_{x_ix_j}\overline{u}_j + \partial_{x_j}^2\overline{u}_i\right)_{i=1}^d + \left(\sum_{j=1}^d \left(\partial_{x_i}\overline{u}_j + \partial_{x_j}\overline{u}_i\right)2c_\mu\partial_{x_j}\left(\frac{k\delta k}{\varepsilon}\right)\right)_{i=1}^d\right)\\
    =&\ \sum_{i=1}^d 2c_\mu\frac{k\delta k}{\varepsilon}v_i\sum_{j=1}^d \left(\partial_{x_ix_j}\overline{u}_j + \partial_{x_j}^2\overline{u}_i\right) + v_i\sum_{j=1}^d \left(\partial_{x_i}\overline{u}_j + \partial_{x_j}\overline{u}_i\right)2c_\mu\left(\partial_{x_j}\left(\frac{k}{\varepsilon}\right)\delta k + \frac{k}{\varepsilon}\partial_{x_j}\delta k\right)\\
    =&\ \sum_{i=1}^d\sum_{j=1}^d 2c_\mu\frac{k\delta k}{\varepsilon}v_i\left(\partial_{x_ix_j}\overline{u}_j + \partial_{x_j}^2\overline{u}_i\right) + 2c_\mu v_i\left(\partial_{x_i}\overline{u}_j + \partial_{x_j}\overline{u}_i\right)\left(\partial_{x_j}\left(\frac{k}{\varepsilon}\right)\delta k + \frac{k}{\varepsilon}\partial_{x_j}\delta k\right),
\end{align*}
hence
\begin{align*}
    &\int_\Omega {\bf v}\cdot\left(\nabla\cdot\left(2c_\mu\frac{k\delta k}{\varepsilon}\left(\nabla\overline{\bf u} + \nabla\overline{\bf u}^\top\right)\right)\right){\rm d}x\\
    =&\ \int_\Omega \sum_{i=1}^d\sum_{j=1}^d 2c_\mu\frac{k\delta k}{\varepsilon}v_i\left(\partial_{x_ix_j}\overline{u}_j + \partial_{x_j}^2\overline{u}_i\right) + 2c_\mu v_i\left(\partial_{x_i}\overline{u}_j + \partial_{x_j}\overline{u}_i\right)\left(\partial_{x_j}\left(\frac{k}{\varepsilon}\right)\delta k + \frac{k}{\varepsilon}\partial_{x_j}\delta k\right){\rm d}x.
\end{align*}
We only integrate by part the last term:
\begin{align*}
    &\int_\Omega \sum_{i=1}^d\sum_{j=1}^d 2c_\mu v_i\left(\partial_{x_i}\overline{u}_j + \partial_{x_j}\overline{u}_i\right)\frac{k}{\varepsilon}\partial_{x_j}\delta k{\rm d}x\\
    =&\ \int_\Gamma \sum_{i=1}^d\sum_{j=1}^d 2c_\mu v_i\left(\partial_{x_i}\overline{u}_j + \partial_{x_j}\overline{u}_i\right)\frac{k\delta k}{\varepsilon}n_j{\rm d}s\\
    &- \int_\Omega \sum_{i=1}^d\sum_{j=1}^d 2c_\mu\delta k\left(\partial_{x_j}v_i\left(\partial_{x_i}\overline{u}_j + \partial_{x_j}\overline{u}_i\right)\frac{k}{\varepsilon} + v_i\left(\partial_{x_ix_j}\overline{u}_j + \partial_{x_j}^2\overline{u}_i\right)\frac{k}{\varepsilon} + v_i\left(\partial_{x_i}\overline{u}_j + \partial_{x_j}\overline{u}_i\right)\partial_{x_j}\left(\frac{k}{\varepsilon}\right)\right){\rm d}x.
\end{align*}
Then
\begin{align*}
    &\int_\Omega {\bf v}\cdot\left(\nabla\cdot\left(2c_\mu\frac{k\delta k}{\varepsilon}\left(\nabla\overline{\bf u} + \nabla\overline{\bf u}^\top\right)\right)\right){\rm d}x\\
    =&\ \int_\Omega \sum_{i=1}^d\sum_{j=1}^d 2c_\mu\frac{k\delta k}{\varepsilon}v_i\left(\partial_{x_ix_j}\overline{u}_j + \partial_{x_j}^2\overline{u}_i\right) + 2c_\mu v_i\left(\partial_{x_i}\overline{u}_j + \partial_{x_j}\overline{u}_i\right)\partial_{x_j}\left(\frac{k}{\varepsilon}\right)\delta k\\
    &\hspace{1.5cm} - 2c_\mu\delta k\left(\partial_{x_j}v_i\left(\partial_{x_i}\overline{u}_j + \partial_{x_j}\overline{u}_i\right)\frac{k}{\varepsilon} + v_i\left(\partial_{x_ix_j}\overline{u}_j + \partial_{x_j}^2\overline{u}_i\right)\frac{k}{\varepsilon} + v_i\left(\partial_{x_i}\overline{u}_j +  \partial_{x_j}\overline{u}_i\right)\partial_{x_j}\left(\frac{k}{\varepsilon}\right)\right){\rm d}x\\
    &+\int_\Gamma \sum_{i=1}^d\sum_{j=1}^d 2c_\mu v_i\left(\partial_{x_i}\overline{u}_j + \partial_{x_j}\overline{u}_i\right)\frac{k\delta k}{\varepsilon}n_j{\rm d}s\\
    =&\ -\int_\Omega \sum_{i=1}^d\sum_{j=1}^d 2c_\mu\delta k\partial_{x_j}v_i\left(\partial_{x_i}\overline{u}_j + \partial_{x_j}\overline{u}_i\right)\frac{k}{\varepsilon}{\rm d}x + \int_\Gamma \sum_{i=1}^d\sum_{j=1}^d 2c_\mu v_i\left(\partial_{x_i}\overline{u}_j + \partial_{x_j}\overline{u}_i\right)\frac{k\delta k}{\varepsilon}n_j{\rm d}s\\
    =&\ -\int_\Omega 4c_\mu\frac{k\delta k}{\varepsilon}\varepsilon(\overline{\bf u}):\nabla{\bf v}{\rm d}x + \int_\Gamma 4c_\mu\frac{k\delta k}{\varepsilon}{\bf v}^\top\varepsilon(\overline{\bf u}){\bf n}{\rm d}s.
\end{align*}
\paragraph*{Term containing $\nabla\delta k$.}
\begin{align*}
    -\int_\Omega {\bf v}\cdot\frac{2}{3}\nabla\delta k{\rm d}x = - \int_\Gamma \frac{2}{3}\delta k{\bf v}\cdot{\bf n}{\rm d}s + \int_\Omega \frac{2}{3}\delta k\nabla\cdot{\bf v}{\rm d}x.
\end{align*}
\paragraph*{Term containing $\delta k_t$.}
\begin{align*}
    -\int_0^T\int_\Omega r\delta k_t{\rm d}x{\rm d}t = -\int_\Omega r(T)\delta k(T) - r(0)\delta k(0){\rm d}x + \int_0^T\int_\Omega r_t\delta k{\rm d}x{\rm d}t.
\end{align*}
\paragraph*{Term.}
\begin{align*}
    -\int_\Omega r\left(\left(\overline{\bf u}\cdot\nabla\right)\delta k\right){\rm d}x &= -\int_\Omega r\sum_{i=1}^d \overline{u}_i\partial_{x_i}\delta k{\rm d}x = -\int_\Gamma r\sum_{i=1}^d \overline{u}_i\delta kn_i{\rm d}s + \int_\Omega \sum_{i=1}^d \delta k\left(\partial_{x_i}r\overline{u}_i + r\partial_{x_i}\overline{u}_i\right){\rm d}x\\
    &= -\int_\Gamma r\delta k\overline{\bf u}\cdot{\bf n}{\rm d}s + \int_\Omega \delta k\left(\nabla r\cdot\overline{\bf u} + r\nabla\cdot\overline{\bf u}\right){\rm d}x = -\int_\Gamma r\delta k\overline{\bf u}\cdot{\bf n}{\rm d}s + \int_\Omega \delta k\nabla r\cdot\overline{\bf u}{\rm d}x.
\end{align*}
\paragraph*{Term.}
\begin{align*}
    \int_\Omega r\nabla\cdot\left(2c_\mu\frac{k\delta k}{\varepsilon}\nabla k\right){\rm d}x &= \int_\Gamma 2c_\mu r\frac{k\delta k}{\varepsilon}\nabla k\cdot{\bf n}{\rm d}s - \int_\Omega \nabla r\cdot\left(2c_\mu\frac{k\delta k}{\varepsilon}\nabla k\right){\rm d}x\\
    &= \int_\Gamma 2c_\mu r\frac{k\delta k}{\varepsilon}\nabla k\cdot{\bf n}{\rm d}s - \int_\Omega 2c_\mu\frac{k\delta k}{\varepsilon}\nabla r\cdot\nabla k{\rm d}x.
\end{align*}
\paragraph*{Term.}
\begin{align*}
    \int_\Omega r\nabla\cdot\left(c_\mu\frac{k^2}{\varepsilon}\nabla\delta k\right){\rm d}x &= \int_\Gamma c_\mu r\frac{k^2}{\varepsilon}\nabla\delta k\cdot{\bf n}{\rm d}s - \int_\Omega c_\mu\frac{k^2}{\varepsilon}\nabla r\cdot\nabla\delta k{\rm d}x.
\end{align*}
We integrate by parts the last term
\begin{align*}
    -\int_\Omega c_\mu\frac{k^2}{\varepsilon}\nabla r\cdot\nabla\delta k{\rm d}x &= -\int_\Omega c_\mu\frac{k^2}{\varepsilon}\sum_{i=1}^d \partial_{x_i}r\partial_{x_i}\delta k{\rm d}x\\
    &= -\int_\Gamma c_\mu\frac{k^2}{\varepsilon}\sum_{i=1}^d \partial_{x_i}r\delta kn_i{\rm d}s + \int_\Omega c_\mu\sum_{i=1}^d \delta k\left(\partial_{x_i}\left(\frac{k^2}{\varepsilon}\right)\partial_{x_i}r + \frac{k^2}{\varepsilon}\partial_{x_i}^2r\right){\rm d}x.
\end{align*}
Hence
\begin{align*}
    \int_\Omega r\nabla\cdot\left(c_\mu\frac{k^2}{\varepsilon}\nabla\delta k\right){\rm d}x =&\ \int_\Gamma rc_\mu\frac{k^2}{\varepsilon}\nabla\delta k\cdot{\bf n}{\rm d}s - \int_\Gamma c_\mu\frac{k^2}{\varepsilon}\delta k\nabla r\cdot{\bf n}{\rm d}s +\int_\Omega c_\mu\delta k\left(\nabla\left(\frac{k^2}{\varepsilon}\right)\cdot\nabla r + \frac{k^2}{\varepsilon}\Delta r\right){\rm d}x\\
    =&\ \int_\Gamma c_\mu r\frac{k^2}{\varepsilon}\nabla\delta k\cdot{\bf n} - c_\mu\frac{k^2\delta k}{\varepsilon}\nabla r\cdot{\bf n}{\rm d}s + \int_\Omega c_\mu\delta k\left(\nabla\left(\frac{k^2}{\varepsilon}\right)\cdot\nabla r + \frac{k^2}{\varepsilon}\Delta r\right){\rm d}x.
\end{align*}
\paragraph*{Term.}
\begin{align*}
    \int_\Omega \eta\nabla\cdot\left(2c_\varepsilon\frac{k\delta k}{\varepsilon}\nabla\varepsilon\right){\rm d}x = \int_\Gamma 2c_\varepsilon\eta\frac{k\delta k}{\varepsilon}\nabla\varepsilon\cdot{\bf n}{\rm d}s - \int_\Omega 2c_\varepsilon\frac{k\delta k}{\varepsilon}\nabla\eta\cdot\nabla\varepsilon{\rm d}x.
\end{align*}
\paragraph*{Second-order term.} Since
\begin{align*}
    &{\bf v}\cdot\left(\nabla\cdot\left(c_\mu\frac{k^2\delta\varepsilon}{\varepsilon^2}\left(\nabla\overline{\bf u} + \nabla\overline{\bf u}^\top\right)\right)\right) = {\bf v}\cdot\left(\nabla\cdot\left(c_\mu\frac{k^2\delta\varepsilon}{\varepsilon^2}\left(\partial_{x_i}\overline{u}_j + \partial_{x_j}\overline{u}_i\right)_{i,j=1}^d\right)\right)\\
    =&\ {\bf v}\cdot\left(c_\mu\frac{k^2\delta\varepsilon}{\varepsilon^2}\left(\sum_{j=1}^d \partial_{x_ix_j}\overline{u}_j + \partial_{x_j}^2\overline{u}_i\right)_{i=1}^d + \left(\sum_{j=1}^d \left(\partial_{x_i}\overline{u}_j + \partial_{x_j}\overline{u}_i\right)c_\mu\partial_{x_j}\left(\frac{k^2\delta\varepsilon}{\varepsilon^2}\right)\right)_{i=1}^d\right)\\
    =&\ \sum_{i=1}^d c_\mu\frac{k^2\delta\varepsilon}{\varepsilon^2}v_i\sum_{j=1}^d \left(\partial_{x_ix_j}\overline{u}_j + \partial_{x_j}^2\overline{u}_i\right) + v_i\sum_{j=1}^d c_\mu\left(\partial_{x_i}\overline{u}_j + \partial_{x_j}\overline{u}_i\right)\left(\partial_{x_j}\left(\frac{k^2}{\varepsilon^2}\right)\delta\varepsilon + \frac{k^2}{\varepsilon^2}\partial_{x_j}\delta\varepsilon\right)\\
    =&\ \sum_{i=1}^d\sum_{j=1}^d c_\mu\frac{k^2\delta\varepsilon}{\varepsilon^2}v_i\left(\partial_{x_ix_j}\overline{u}_j + \partial_{x_j}^2\overline{u}_i\right) + c_\mu v_i\left(\partial_{x_i}\overline{u}_j + \partial_{x_j}\overline{u}_i\right)\left(\partial_{x_j}\left(\frac{k^2}{\varepsilon^2}\right)\delta\varepsilon + \frac{k^2}{\varepsilon^2}\partial_{x_j}\delta\varepsilon\right),
\end{align*}
hence
\begin{align*}
    &\int_\Omega {\bf v}\cdot\left(\nabla\cdot\left(c_\mu\frac{k^2\delta\varepsilon}{\varepsilon^2}\left(\nabla\overline{\bf u} + \nabla\overline{\bf u}^\top\right)\right)\right){\rm d}x\\
    =&\ \int_\Omega \sum_{i=1}^d\sum_{j=1}^d c_\mu\frac{k^2\delta\varepsilon}{\varepsilon^2}v_i\left(\partial_{x_ix_j}\overline{u}_j + \partial_{x_j}^2\overline{u}_i\right) + c_\mu v_i\left(\partial_{x_i}\overline{u}_j + \partial_{x_j}\overline{u}_i\right)\left(\partial_{x_j}\left(\frac{k^2}{\varepsilon^2}\right)\delta\varepsilon + \frac{k^2}{\varepsilon^2}\partial_{x_j}\delta\varepsilon\right){\rm d}x.
\end{align*}
We only integrate by part the last term:
\begin{align*}
    &\int_\Omega \sum_{i=1}^d\sum_{j=1}^d c_\mu v_i\left(\partial_{x_i}\overline{u}_j + \partial_{x_j}\overline{u}_i\right)\frac{k^2}{\varepsilon^2}\partial_{x_j}\delta\varepsilon{\rm d}x\\
    =&\ \int_\Gamma \sum_{i=1}^d\sum_{j=1}^d c_\mu v_i\left(\partial_{x_i}\overline{u}_j + \partial_{x_j}\overline{u}_i\right)\frac{k^2}{\varepsilon^2}\delta\varepsilon n_j{\rm d}s\\
    &-\int_\Omega \sum_{i=1}^d\sum_{j=1}^d c_\mu\delta\varepsilon\left(\partial_{x_j}v_i\left(\partial_{x_i}\overline{u}_j + \partial_{x_j}\overline{u}_i\right)\frac{k^2}{\varepsilon^2} + v_i\left(\partial_{x_ix_j}\overline{u}_j + \partial_{x_j}^2\overline{u}_i\right)\frac{k^2}{\varepsilon^2} + v_i\left(\partial_{x_i}\overline{u}_j + \partial_{x_j}\overline{u}_i\right)\partial_{x_j}\left(\frac{k^2}{\varepsilon^2}\right)\right){\rm d}x.
\end{align*}
Plugging back, we obtain
\begin{align*}
    &-\int_\Omega {\bf v}\cdot\left(\nabla\cdot\left(c_\mu\frac{k^2\delta\varepsilon}{\varepsilon^2}\left(\nabla\overline{\bf u} + \nabla\overline{\bf u}^\top\right)\right)\right){\rm d}x\\
    =&-\int_\Omega \sum_{i=1}^d\sum_{j=1}^d c_\mu\frac{k^2\delta\varepsilon}{\varepsilon^2}v_i\left(\partial_{x_ix_j}\overline{u}_j + \partial_{x_j}^2\overline{u}_i\right) + c_\mu v_i\left(\partial_{x_i}\overline{u}_j + \partial_{x_j}\overline{u}_i\right)\partial_{x_j}\left(\frac{k^2}{\varepsilon^2}\right)\delta\varepsilon\\
    &\hspace{1.5cm} -c_\mu\delta\varepsilon\left(\partial_{x_j}v_i\left(\partial_{x_i}\overline{u}_j + \partial_{x_j}\overline{u}_i\right)\frac{k^2}{\varepsilon^2} + v_i\left(\partial_{x_ix_j}\overline{u}_j + \partial_{x_j}^2\overline{u}_i\right)\frac{k^2}{\varepsilon^2} + v_i\left(\partial_{x_i}\overline{u}_j + \partial_{x_j}\overline{u}_i\right)\partial_{x_j}\left(\frac{k^2}{\varepsilon^2}\right)\right){\rm d}x\\
    &-\int_\Gamma \sum_{i=1}^d\sum_{j=1}^d c_\mu v_i\left(\partial_{x_i}\overline{u}_j + \partial_{x_j}\overline{u}_i\right)\frac{k^2\delta\varepsilon}{\varepsilon^2}n_j{\rm d}s\\
    =& \int_\Omega \sum_{i=1}^d\sum_{j=1}^d c_\mu\frac{k^2\delta\varepsilon}{\varepsilon^2}\partial_{x_j}v_i\left(\partial_{x_i}\overline{u}_j + \partial_{x_j}\overline{u}_i\right){\rm d}x - \int_\Gamma \sum_{i=1}^d\sum_{j=1}^d c_\mu\frac{k^2\delta\varepsilon}{\varepsilon^2}v_i\left(\partial_{x_i}\overline{u}_j + \partial_{x_j}\overline{u}_i\right)n_j{\rm d}s\\
    =&\ \int_\Omega 2c_\mu\frac{k^2\delta\varepsilon}{\varepsilon^2}\varepsilon(\overline{\bf u}):\nabla{\bf v}{\rm d}x - \int_\Gamma 2c_\mu\frac{k^2\delta\varepsilon}{\varepsilon^2}{\bf v}^\top\varepsilon(\overline{\bf u}){\bf n}{\rm d}s.
\end{align*}
\paragraph*{Term.}
\begin{align*}
    -\int_\Omega r\nabla\cdot\left(c_\mu\frac{k^2\delta\varepsilon}{\varepsilon^2}\nabla k\right){\rm d}x = \int_\Gamma c_\mu r\frac{k^2\delta\varepsilon}{\varepsilon^2}\nabla k\cdot{\bf n}{\rm d}s - \int_\Omega c_\mu\frac{k^2\delta\varepsilon}{\varepsilon^2}\nabla r\cdot\nabla k{\rm d}x.
\end{align*}
\paragraph*{Term containing $\delta\varepsilon_t$.}
\begin{align*}
    -\int_0^T\int_\Omega \eta\delta\varepsilon_t{\rm d}x{\rm d}t = -\int_\Omega \eta(T)\delta\varepsilon(T) - \eta(0)\delta\varepsilon(0){\rm d}x + \int_0^T\int_\Omega \eta_t\delta\varepsilon{\rm d}x{\rm d}t.
\end{align*}
\paragraph*{Term.}
\begin{align*}
    -\int_\Omega \eta\left(\left(\overline{\bf u}\cdot\nabla\right)\delta\varepsilon\right){\rm d}x &= -\int_\Omega \eta\sum_{i=1}^d \overline{u}_i\partial_{x_i}\delta\varepsilon{\rm d}x = -\int_\Gamma \eta\sum_{i=1}^d \overline{u}_i\delta\varepsilon n_i{\rm d}s + \int_\Omega \sum_{i=1}^d \delta\varepsilon\left(\partial_{x_i}\eta\overline{u}_i + \eta\partial_{x_i}\overline{u}_i\right){\rm d}x\\
    &= -\int_\Gamma \eta\delta\varepsilon\overline{\bf u}\cdot{\bf n}{\rm d}s + \int_\Omega \delta\varepsilon\left(\nabla\eta\cdot\overline{\bf u} + \eta\nabla\cdot\overline{\bf u}\right){\rm d}x = -\int_\Gamma \eta\delta\varepsilon\overline{\bf u}\cdot{\bf n}{\rm d}s + \int_\Omega \delta\varepsilon\nabla\eta\cdot\overline{\bf u}{\rm d}x.
\end{align*}
\paragraph*{Term.}
\begin{align*}
    -\int_\Omega \eta\nabla\cdot\left(c_\varepsilon\frac{k^2\delta\varepsilon}{\varepsilon^2}\nabla\varepsilon\right){\rm d}x = -\int_\Gamma c_\varepsilon\eta\frac{k^2\delta\varepsilon}{\varepsilon^2}\nabla\varepsilon\cdot{\bf n}{\rm d}s + \int_\Omega c_\varepsilon\frac{k^2\delta\varepsilon}{\varepsilon^2}\nabla\eta\cdot\nabla\varepsilon{\rm d}x.
\end{align*}
\paragraph*{Last term.}
\begin{align*}
    \int_\Omega \eta\nabla\cdot\left(c_\varepsilon\frac{k^2}{\varepsilon}\nabla\delta\varepsilon\right){\rm d}x = \int_\Gamma \eta c_\varepsilon\frac{k^2}{\varepsilon}\nabla\delta\varepsilon\cdot{\bf n}{\rm d}s - \int_\Omega c_\varepsilon\frac{k^2}{\varepsilon}\nabla\eta\cdot\nabla\delta\varepsilon{\rm d}x.
\end{align*}
We integrate by part the last term:
\begin{align*}
    -\int_\Omega c_\varepsilon\frac{k^2}{\varepsilon}\nabla\eta\cdot\nabla\delta\varepsilon{\rm d}x &= -\int_\Omega c_\varepsilon\frac{k^2}{\varepsilon}\sum_{i=1}^d \partial_{x_i}\eta\partial_{x_i}\delta\varepsilon{\rm d}x\\
    &= -\int_\Gamma c_\varepsilon\frac{k^2}{\varepsilon}\sum_{i=1}^d \partial_{x_i}\eta\delta\varepsilon n_i{\rm d}s + \int_\Omega c_\varepsilon\sum_{i=1}^d \delta\varepsilon\left(\partial_{x_i}\left(\frac{k^2}{\varepsilon}\right)\partial_{x_i}\eta + \frac{k^2}{\varepsilon}\partial_{x_i}^2\eta\right){\rm d}x\\
    &= -\int_\Gamma c_\varepsilon\frac{k^2\delta\varepsilon}{\varepsilon}\nabla\eta\cdot{\bf n}{\rm d}s + \int_\Omega c_\varepsilon\delta\varepsilon\left(\nabla\left(\frac{k^2}{\varepsilon}\right)\cdot\nabla\eta + \frac{k^2}{\varepsilon}\Delta\eta\right){\rm d}x.
\end{align*}
Plugging back, we obtain
\begin{align*}
    \int_\Omega \eta\nabla\cdot\left(c_\varepsilon\frac{k^2}{\varepsilon}\nabla\delta\varepsilon\right){\rm d}x = \int_\Gamma c_\varepsilon\eta\frac{k^2}{\varepsilon}\nabla\delta\varepsilon\cdot{\bf n}{\rm d}s - \int_\Gamma c_\varepsilon\frac{k^2\delta\varepsilon}{\varepsilon}\nabla\eta\cdot{\bf n}{\rm d}s + \int_\Omega c_\varepsilon\delta\varepsilon\left(\nabla\left(\frac{k^2}{\varepsilon}\right)\cdot\nabla\eta + \frac{k^2}{\varepsilon}\Delta\eta\right){\rm d}x.
\end{align*}
Gathering all terms, we can reformulate \eqref{1.4.2} as
\begin{align*}
    &\int_0^T\int_\Omega [\partial_{\overline{\bf u}}J_\Omega + {\bf v}_t + \nabla\overline{\bf u}{\bf v} + (\overline{\bf u}\cdot\nabla){\bf v} + 2c_\mu\varepsilon({\bf v})\nabla\left(\frac{k^2}{\varepsilon}\right) + \left(\nu + c_\mu\frac{k^2}{\varepsilon}\right)\left(\nabla(\nabla\cdot{\bf v}) + \Delta{\bf v}\right) - \nabla q - r\nabla k\\
    &\hspace{2cm} - 4c_\mu\frac{k^2}{\varepsilon}\varepsilon(\overline{\bf u})\nabla r - 4c_\mu r\varepsilon(\overline{\bf u})\nabla\left(\frac{k^2}{\varepsilon}\right) - 2c_\mu r\frac{k^2}{\varepsilon}\Delta\overline{\bf u} - 2c_\mu\nabla\left(\nabla\cdot\overline{\bf u}\right) - \eta\nabla\varepsilon - 4c_1k\varepsilon(\overline{\bf u})\nabla\eta\\
    &\hspace{2cm} - 2c_1\eta\varepsilon(\overline{\bf u})\nabla k - 2c_1\eta k\left(\Delta\overline{\bf u} + \nabla\left(\nabla\cdot\overline{\bf u}\right)\right)]\cdot\delta\overline{\bf u}{\rm d}x{\rm d}t\\
    &+ \int_0^T\int_\Omega \left(\partial_{\overline{p}}J_\Omega + \nabla\cdot{\bf v}\right)\delta\overline{p}{\rm d}x{\rm d}t\\
    &+ \int_0^T\int_\Omega [-4c_\mu\frac{k}{\varepsilon}\varepsilon(\overline{\bf u}):\nabla{\bf v} + \frac{2}{3}\nabla\cdot{\bf v} + r_t + \nabla r\cdot\overline{\bf u} - 2c_\mu\frac{k}{\varepsilon}\nabla r\cdot\nabla k + c_\mu\nabla\left(\frac{k^2}{\varepsilon}\right)\cdot\nabla r + c_\mu\frac{k^2}{\varepsilon}\Delta r\\
    & \hspace{2cm} + c_\mu r\frac{k}{\varepsilon}\left|\nabla\overline{\bf u} + \nabla\overline{\bf u}^\top\right|^2 - 2c_\varepsilon\frac{k}{\varepsilon}\nabla\eta\cdot\nabla\varepsilon + \frac{c_1}{2}\eta\left|\nabla\overline{\bf u} + \nabla\overline{\bf u}^\top\right|^2 + c_2\eta\frac{\varepsilon^2}{k^2}]\delta k{\rm d}x{\rm d}t\\
    &+ \int_0^T\int_\Omega [2c_\mu\frac{k^2}{\varepsilon^2}\varepsilon(\overline{\bf u}):\nabla{\bf v} - c_\mu\frac{k^2}{\varepsilon^2}\nabla r\cdot\nabla k - \frac{c_\mu}{2}r\frac{k^2}{\varepsilon^2}\left|\nabla\overline{\bf u} + \nabla\overline{\bf u}^\top\right|^2 - r + \eta_t + \nabla\eta\cdot\overline{\bf u} + c_\varepsilon\frac{k^2}{\varepsilon^2}\nabla\eta\cdot\nabla\varepsilon\\
    &\hspace{2cm} + c_\varepsilon\nabla\left(\frac{k^2}{\varepsilon}\right)\cdot\nabla\eta + c_\varepsilon\frac{k^2}{\varepsilon}\Delta\eta - 2c_2\eta\frac{\varepsilon}{k}]\delta\varepsilon{\rm d}x{\rm d}t\\
    &+ \int_\Omega \left[-{\bf v}(T)\cdot\delta\overline{\bf u}(T) + {\bf v}(0)\cdot\delta\overline{\bf u}(0) - r(T)\delta k(T) + r(0)\delta k(0) - \eta(T)\delta\varepsilon(T) + \eta(0)\delta\varepsilon(0)\right]{\rm d}x\\
    &+ \int_0^T\int_\Gamma [\partial_{\overline{\bf u}}J_\Gamma\cdot\delta\overline{\bf u} + \partial_{\overline{p}}J_\Gamma\delta\overline{p} + 2\left(\nu + c_\mu\frac{k^2}{\varepsilon}\right){\bf v}^\top\varepsilon(\delta\overline{\bf u}){\bf n} - 2{\bf n}^\top\varepsilon({\bf v})\delta\overline{\bf u} + q\delta\overline{\bf u}\cdot{\bf n} + 4c_\mu r\frac{k^2}{\varepsilon}{\bf n}^\top\varepsilon(\overline{\bf u})\delta\overline{\bf u}\\
    &\hspace{2cm} + 4c_1\eta k{\bf n}^\top\varepsilon(\overline{\bf u})\delta\overline{\bf u} - \delta\overline{p}{\bf v}\cdot{\bf n} + 4c_\mu\frac{k\delta k}{\varepsilon}{\bf v}^\top\varepsilon(\overline{\bf u}){\bf n} - \frac{2}{3}\delta k{\bf v}\cdot{\bf n} - r\delta k\overline{\bf u}\cdot{\bf n} + 2c_\mu r\frac{k\delta k}{\varepsilon}\nabla k\cdot{\bf n}\\
    &\hspace{2cm} + c_\mu r\frac{k^2}{\varepsilon}\nabla\delta k\cdot{\bf n} - c_\mu\frac{k^2\delta k}{\varepsilon}\nabla r\cdot{\bf n} + 2c_\varepsilon\eta\frac{k\delta k}{\varepsilon}\nabla\varepsilon\cdot{\bf n} - 2c_\mu\frac{k^2\delta\varepsilon}{\varepsilon^2}{\bf v}^\top\varepsilon(\overline{\bf u}){\bf n} + c_\mu r\frac{k^2\delta\varepsilon}{\varepsilon^2}\nabla k\cdot{\bf n}\\
    &\hspace{2cm} - \eta\delta\varepsilon\overline{\bf u}\cdot{\bf n} - c_\varepsilon\eta\frac{k^2\delta \varepsilon}{\varepsilon^2}\nabla\varepsilon\cdot{\bf n} + c_\varepsilon\eta\frac{k^2}{\varepsilon}\nabla\delta\varepsilon\cdot{\bf n} - c_\varepsilon\frac{k^2\delta\varepsilon}{\varepsilon}\nabla\eta\cdot{\bf n}]{\rm d}x{\rm d}t.
\end{align*}
Since this holds for any $\delta\overline{\bf u},\delta\overline{p},\delta k$ and $\delta\varepsilon$ satisfying the primal NSEs, the integrals vanish individually. The vanishing of the integrals over the domain yields the adjoint $k$-$\varepsilon$:
\begin{equation}
    \label{adjoint k-epsilon}
    \left\{\begin{split}
        &{\bf v}_t + \left(\nu + c_\mu\frac{k^2}{\varepsilon}\right)\left(\nabla(\nabla\cdot{\bf v}) + \Delta{\bf v}\right) + 2c_\mu\varepsilon({\bf v})\nabla\left(\frac{k^2}{\varepsilon}\right) + \nabla\overline{\bf u}{\bf v} + (\overline{\bf u}\cdot\nabla){\bf v} - \nabla q\\
        &\hspace{1cm} = -\partial_{\overline{\bf u}}J_\Omega + r\nabla k + 4c_\mu\frac{k^2}{\varepsilon}\varepsilon(\overline{\bf u})\nabla r + 4c_\mu r\varepsilon(\overline{\bf u})\nabla\left(\frac{k^2}{\varepsilon}\right) + 2c_\mu r\frac{k^2}{\varepsilon}\Delta\overline{\bf u} + 2c_\mu\nabla\left(\nabla\cdot\overline{\bf u}\right)  + \eta\nabla\varepsilon\\
        & \hspace{1.5cm}+ 4c_1k\varepsilon(\overline{\bf u})\nabla\eta + 2c_1\eta\varepsilon(\overline{\bf u})\nabla k + 2c_1\eta k\left(\Delta\overline{\bf u} + \nabla\left(\nabla\cdot\overline{\bf u}\right)\right) \mbox{ in } \Omega,\\
        &\nabla\cdot{\bf v} = -\partial_{\overline{p}}J_\Omega \mbox{ in } \Omega,\\
        &r_t + c_\mu\frac{k^2}{\varepsilon}\Delta r + \nabla r\cdot\overline{\bf u} - 2c_\mu\frac{k}{\varepsilon}\nabla r\cdot\nabla k + c_\mu\nabla\left(\frac{k^2}{\varepsilon}\right)\cdot\nabla r\\
        &\hspace{1cm} = 4c_\mu\frac{k}{\varepsilon}\varepsilon(\overline{\bf u}):\nabla{\bf v} - \frac{2}{3}\nabla\cdot{\bf v} - c_\mu r\frac{k}{\varepsilon}\left|\nabla\overline{\bf u} + \nabla\overline{\bf u}^\top\right|^2 + 2c_\varepsilon\frac{k}{\varepsilon}\nabla\eta\cdot\nabla\varepsilon - \frac{c_1}{2}\eta\left|\nabla\overline{\bf u} + \nabla\overline{\bf u}^\top\right|^2 - c_2\eta\frac{\varepsilon^2}{k^2} \mbox{ in } \Omega,\\
        &\eta_t + c_\varepsilon\frac{k^2}{\varepsilon}\Delta\eta + \nabla\eta\cdot\overline{\bf u} + c_\varepsilon\frac{k^2}{\varepsilon^2}\nabla\eta\cdot\nabla\varepsilon + c_\varepsilon\nabla\left(\frac{k^2}{\varepsilon}\right)\cdot\nabla\eta\\
        &\hspace{1cm} = -2c_\mu\frac{k^2}{\varepsilon^2}\varepsilon(\overline{\bf u}):\nabla{\bf v} + c_\mu\frac{k^2}{\varepsilon^2}\nabla r\cdot\nabla k + \frac{c_\mu}{2}r\frac{k^2}{\varepsilon^2}\left|\nabla\overline{\bf u} + \nabla\overline{\bf u}^\top\right|^2 + r + 2c_2\eta\frac{\varepsilon}{k} \mbox{ in } \Omega.
    \end{split}\right.    
\end{equation}
Plugging explicit formulas of $J_\Omega$ yields
\begin{align*}
    J_\Omega(\overline{\bf u},p) &= \gamma k_\epsilon\left(p + \frac{1}{2}|\overline{\bf u}|^2\right)\overline{\bf u}\cdot{\bf n},\\
    \partial_{\overline{\bf u}}J_\Omega(\overline{\bf u},p) &= \gamma k_\epsilon\left(\left(p + \frac{1}{2}|\overline{\bf u}|^2\right){\bf n} + (\overline{\bf u}\cdot{\bf n})\overline{\bf u}\right),\\
    \partial_pJ_\Omega(\overline{\bf u},p) &= \gamma k_\epsilon\overline{\bf u}\cdot{\bf n}.
\end{align*}
Then \eqref{1.4.4} becomes
\begin{equation}
    \label{adjoint k-epsilon 1}
    \left\{\begin{split}
        &{\bf v}_t + \left(\nu + c_\mu\frac{k^2}{\varepsilon}\right)\left(\nabla(\nabla\cdot{\bf v}) + \Delta{\bf v}\right) + 2c_\mu\varepsilon({\bf v})\nabla\left(\frac{k^2}{\varepsilon}\right) + \nabla\overline{\bf u}{\bf v} + (\overline{\bf u}\cdot\nabla){\bf v} - \nabla q\\
        &\hspace{1cm} = -\gamma k_\epsilon\left(\left(p + \frac{1}{2}|\overline{\bf u}|^2\right){\bf n} + (\overline{\bf u}\cdot{\bf n})\overline{\bf u}\right) + r\nabla k + 4c_\mu\frac{k^2}{\varepsilon}\varepsilon(\overline{\bf u})\nabla r + 4c_\mu r\varepsilon(\overline{\bf u})\nabla\left(\frac{k^2}{\varepsilon}\right)\\
        & \hspace{1.5cm} + 2c_\mu r\frac{k^2}{\varepsilon}\Delta\overline{\bf u} + 2c_\mu\nabla\left(\nabla\cdot\overline{\bf u}\right)  + \eta\nabla\varepsilon + 4c_1k\varepsilon(\overline{\bf u})\nabla\eta + 2c_1\eta\varepsilon(\overline{\bf u})\nabla k + 2c_1\eta k\left(\Delta\overline{\bf u} + \nabla\left(\nabla\cdot\overline{\bf u}\right)\right) \mbox{ in } \Omega,\\
        &\nabla\cdot{\bf v} = -\gamma k_\epsilon\overline{\bf u}\cdot{\bf n} \mbox{ in } \Omega,\\
        &r_t + c_\mu\frac{k^2}{\varepsilon}\Delta r + \nabla r\cdot\overline{\bf u} - 2c_\mu\frac{k}{\varepsilon}\nabla r\cdot\nabla k + c_\mu\nabla\left(\frac{k^2}{\varepsilon}\right)\cdot\nabla r\\
        &\hspace{1cm} = 4c_\mu\frac{k}{\varepsilon}\varepsilon(\overline{\bf u}):\nabla{\bf v} - \frac{2}{3}\nabla\cdot{\bf v} - c_\mu r\frac{k}{\varepsilon}\left|\nabla\overline{\bf u} + \nabla\overline{\bf u}^\top\right|^2 + 2c_\varepsilon\frac{k}{\varepsilon}\nabla\eta\cdot\nabla\varepsilon - \frac{c_1}{2}\eta\left|\nabla\overline{\bf u} + \nabla\overline{\bf u}^\top\right|^2 - c_2\eta\frac{\varepsilon^2}{k^2} \mbox{ in } \Omega,\\
        &\eta_t + c_\varepsilon\frac{k^2}{\varepsilon}\Delta\eta + \nabla\eta\cdot\overline{\bf u} + c_\varepsilon\frac{k^2}{\varepsilon^2}\nabla\eta\cdot\nabla\varepsilon + c_\varepsilon\nabla\left(\frac{k^2}{\varepsilon}\right)\cdot\nabla\eta\\
        &\hspace{1cm} = -2c_\mu\frac{k^2}{\varepsilon^2}\varepsilon(\overline{\bf u}):\nabla{\bf v} + c_\mu\frac{k^2}{\varepsilon^2}\nabla r\cdot\nabla k + \frac{c_\mu}{2}r\frac{k^2}{\varepsilon^2}\left|\nabla\overline{\bf u} + \nabla\overline{\bf u}^\top\right|^2 + r + 2c_2\eta\frac{\varepsilon}{k} \mbox{ in } \Omega.
    \end{split}\right.    
\end{equation}
The remaining terms yields
\begin{align}
    \int_\Omega \left[-{\bf v}(T)\cdot\delta\overline{\bf u}(T) + {\bf v}(0)\cdot\delta\overline{\bf u}(0) - r(T)\delta k(T) + r(0)\delta k(0) - \eta(T)\delta\varepsilon(T) + \eta(0)\delta\varepsilon(0)\right]{\rm d}x + {\color{red}\mbox{ICs for $k$-$\epsilon$}}= 0,
\end{align}
{\color{red} Wait initial conditions for $k$-$\varepsilon$!}

and
\begin{align}
    \int_0^T\int_\Gamma [&\partial_{\overline{\bf u}}J_\Gamma\cdot\delta\overline{\bf u} + \partial_{\overline{p}}J_\Gamma\delta\overline{p} + 2\left(\nu + c_\mu\frac{k^2}{\varepsilon}\right){\bf v}^\top\varepsilon(\delta\overline{\bf u}){\bf n} - 2{\bf n}^\top\varepsilon({\bf v})\delta\overline{\bf u} + q\delta\overline{\bf u}\cdot{\bf n} + 4c_\mu r\frac{k^2}{\varepsilon}{\bf n}^\top\varepsilon(\overline{\bf u})\delta\overline{\bf u}\\
    & + 4c_1\eta k{\bf n}^\top\varepsilon(\overline{\bf u})\delta\overline{\bf u} - \delta\overline{p}{\bf v}\cdot{\bf n} + 4c_\mu\frac{k\delta k}{\varepsilon}{\bf v}^\top\varepsilon(\overline{\bf u}){\bf n} - \frac{2}{3}\delta k{\bf v}\cdot{\bf n} - r\delta k\overline{\bf u}\cdot{\bf n} + 2c_\mu r\frac{k\delta k}{\varepsilon}\nabla k\cdot{\bf n} + c_\mu r\frac{k^2}{\varepsilon}\nabla\delta k\cdot{\bf n}\\
    & - c_\mu\frac{k^2\delta k}{\varepsilon}\nabla r\cdot{\bf n} +2c_\varepsilon\eta\frac{k\delta k}{\varepsilon}\nabla\varepsilon\cdot{\bf n} - 2c_\mu\frac{k^2\delta\varepsilon}{\varepsilon^2}{\bf v}^\top\varepsilon(\overline{\bf u}){\bf n} + c_\mu r\frac{k^2\delta\varepsilon}{\varepsilon^2}\nabla k\cdot{\bf n} - \eta\delta\varepsilon\overline{\bf u}\cdot{\bf n} - c_\varepsilon\eta\frac{k^2\delta\varepsilon}{\varepsilon^2}\nabla\varepsilon\cdot{\bf n}\\
    &+ c_\varepsilon\eta\frac{k^2}{\varepsilon}\nabla\delta\varepsilon\cdot{\bf n} - c_\varepsilon\frac{k^2\delta\varepsilon}{\varepsilon}\nabla\eta\cdot{\bf n}]{\rm d}s{\rm d}t + {\color{red}\mbox{BCs for $k$-$\varepsilon$}} = 0.
\end{align}
{\color{red} Wait boundary conditions for $k$-$\varepsilon$!}
%------------------------------------------------------------------------------%

\appendix

\chapter{Tools in Mathematical Analysis*}

\section{Differentiability in Banach spaces}
See \cite[Chap. 2, Sect. 2.6]{Troltzsch2010}.

\begin{definition}[1st variation]
    Let $(U,\|\cdot\|_U)$ and $(V,\|\cdot\|_V)$ be Banach spaces, and $\mathcal{U}$ be a nonempty open subset of $U$, $u\in\mathcal{U}$ and $h\in U$, and $F:\mathcal{U}\subset U\to V$ be given. If the limit
    \begin{align*}
        \delta F(u,h)\coloneqq\lim_{t\downarrow 0} \frac{1}{t}\left(F(u + th) - F(u)\right)
    \end{align*}
    exists in $V$, then it is called the \emph{directional derivative} of $F$ at $u$ in the direction $h$.
    
    If this limit exists for all $h\in U$, then the mapping $h\mapsto\delta F(u,h)$ is termed the \emph{1st variation} of $F$ at $u$.
\end{definition}

%------------------------------------------------------------------------------%

\chapter{Numerical Methods}

In this chapter, we discuss some well-established methods for ODEs and PDEs.

\section{Approximation of a normed space}
First of all, we recall general concepts of approximation of a normed space presented in \cite[Subsect. I.3.1]{Temam2000}.

A normed space $W$ must be approximated by a family $(W_h)_{h\in\mathcal{H}}$, where $\mathcal{H}$ is called the \textit{set of indices}, of normed spaces $W_h$ in any computational methods.

\begin{definition}[Internal approximation]
    An \emph{internal approximation} of a normed vector space $W$ is a set consisting of a family of triples $\{W_h,p_h,r_h\}$, $h\in\mathcal{H}$ where
    \begin{itemize}
        \item[(i)] $W_h$ is a normed vector space;
        \item[(ii)] $p_h$ is a linear continuous operator from $W_h$ into $W$;
        \item[(iii)] $r_h$ is a (perhaps nonlinear) operator from $W$ into $W_h$.
    \end{itemize}
\end{definition}

\begin{definition}[External approximation]
    An \emph{external approximation} of a normed space $W$ is a set consisting of
    \begin{itemize}
        \item[(i)] a normed space $F$ and an isomorphism $\overline{\omega}$ of $W$ into $F$.
        \item[(ii)] a family of triples $\{W_h,p_h,r_h\}_{h\in\mathcal{H}}$, in which, for each $h$,
        \begin{itemize}
            \item $W_h$ is a normed space,
            \item $p_h$ a linear continuous mapping of $W_h$ into $F$,
            \item $r_h$ a (perhaps nonlinear) mapping of $W$ into $W_h$.
        \end{itemize}
    \end{itemize}
\end{definition}
The operators $p_h$ and $r_h$ are called \textit{prolongation} and \textit{restriction operators}, respectively. When the spaces $W$ and $F$ are Hilbert spaces, and when the spaces $W_h$ are likewise Hilbert spaces, the approximation is said to be a \textit{Hilbert approximation}.

\begin{definition}[(Discrete\texttt{/}truncation) error]
    For given $h,{\bf u}\in W$, ${\bf u}_h\in W_h$, we say that
    \begin{itemize}
        \item[(i)] $\|\overline{\omega}{\bf u} - p_h{\bf u}_h\|_F$ is the \emph{error} between ${\bf u}$ and ${\bf u}_h$,
        \item[(ii)] $\|{\bf u}_h - r_h{\bf u}\|_{W_h}$ is the \emph{discrete error} between ${\bf u}$ and ${\bf u}_h$,
        \item[(iii)] $\|\overline{\omega}{\bf u} - p_hr_h{\bf u}\|_F$ is the \emph{truncation error} of ${\bf u}$.
    \end{itemize}
\end{definition}
We now define \textit{stable} and \textit{convergent approximations}.

\begin{definition}[Stable\texttt{/}convergent approximation]
    The prolongation operators $p_h$ are said to be \emph{stable} if their norms
    \begin{align*}
        \|p_h\| = \sup_{u_h\in W_h,\,\|u_h\|_{W_h} = 1} \|p_h{\bf u}_h\|_F
    \end{align*}
    can be majorized independently of $h$.
    
    The approximation of the space $W$ is said to be \emph{stable} if the prolongation operators are stable.
\end{definition}

\begin{definition}
    We will say that a family ${\bf u}_h$ \emph{converges strongly} (or \emph{weakly}) to ${\bf u}$ if $p_h{\bf u}_h$ converges to $\overline{\omega}{\bf u}$ when $h\to 0$ in the strong (or weak) topology of $F$.
    
    We will say that the family ${\bf u}_h$ \emph{converges discretely} to ${\bf u}$ if
    \begin{align*}
        \lim_{h\to 0} \|{\bf u}_h - r_h{\bf u}\|_{W_h} = 0.
    \end{align*}
\end{definition}

\begin{definition}
    We will say that an external approximation of a normed space $W$ is \emph{convergent} if the 2 following conditions hold:
    \begin{itemize}
        \item[(C1)] for all ${\bf u}\in W$
        \begin{align*}
            \lim_{h\to 0} p_hr_h{\bf u} = \overline{\omega}{\bf u}
        \end{align*}
        in the strong topology of $F$.
        \item[(C2)] for each sequence ${\bf u}_h'$ of elements of $W_{h'}$ ($h'\to 0$), s.t. $p_{h'}{\bf u}_{h'}$ converges to some element $\phi$ in the weak topology of $F$, we have, $\phi\in\overline{\omega}W$; i.e., $\phi = \overline{\omega}{\bf u}$ for some ${\bf u}\in W$.
    \end{itemize}
\end{definition}

\begin{remark}
    Condition (C2) disappears when $\overline{\omega}$ is surjective and especially in the case of internal approximation.
\end{remark}
The following proposition shows that condition (C1) can in some sense be weakened for internal and external approximations.

\begin{proposition}
    Let there be given a stable external approximation of a space $W$ which is convergent in the following restrictive sense: the operators $r_h$ are defined only on a dense subset $\mathcal{W}$ of $W$ and condition (C1) in Definition 3.6 holds only for the ${\bf u}$ belonging to $\mathcal{W}$ (condition (C2) remains unchanged).
    
    Then it is possible to extend the definition of the restriction operators $r_h$ to the whole space $W$ so that condition (C1) is valid for each ${\bf u}\in W$ and hence the approximation of $W$ is stable and convergent without any restriction.
\end{proposition}

\begin{proof}
    See \cite[p. 30]{Temam2000}.
\end{proof}

\begin{remark}
    If the mappings $r_h$ are defined on the whole space $W$ and condition (C1) holds for all ${\bf u}\in\mathcal{W}$, Proposition 3.1 shows us that we can modify the value of $r_h{\bf u}$ on the complement of $\mathcal{W}$ so that condition (C1) is satisfied for all ${\bf u}\in W$.
\end{remark}

\subsection{A general convergence theorem}
Let $H$ be a Hilbert space, $a:H\times H\to\mathbb{R}$ be a coercive bilinear continuous form, and $\phi\in H^\star$. By Theorem \ref{theorem: Lax-Milgram}, let $u\in H$ denote the unique solution of \eqref{Lax-Milgram: Euler equation}.

Let $\{H_h,p_h,r_h\}_{h\in\mathcal{H}}$ be an external stable and convergent Hilbert approximation of $H$. For each $h\in\mathcal{H}$, let there be given
\begin{itemize}
    \item[(i)] a continuous coercive bilinear form $a_h:H_h\times H_h\to\mathbb{R}$ and satisfies
    \begin{align}
        \label{uniform coerciveness}
        \exists\alpha_0 > 0\mbox{ independent of } h,\mbox{ s.t. } a_h({\bf u}_h,{\bf u}_h)\ge\alpha_0\|{\bf u}_h\|_{H_h}^2,\ \forall{\bf u}_h\in H_h.
    \end{align}
    \item[(ii)] a continuous linear form $\phi_h\in H_h^\star$ such that
    \begin{align}
        \label{uniform bound of linear forms}
        \|\phi_h\|_{H_h^\star}\le\beta,\ \beta\mbox{ is independent of } h.
    \end{align}
\end{itemize}
Now \eqref{Lax-Milgram: Euler equation} is associated with the the following family of approximation equations:

For each $h\in\mathcal{H}$, find $u_h\in H_h$ such that
\begin{align}
    \label{Lax-Milgram: approximate Euler equation}
    a_h(u_h,v_h) = \langle\phi_h,v_h\rangle,\ \forall v_h\in H_h.
\end{align}
By the preceding hypotheses, applying Theorem \ref{theorem: Lax-Milgram} for $(a,\phi,H) = (a_h,\phi_h,H_h)$ yields that \eqref{Lax-Milgram: approximate Euler equation} has a unique solution $u_h$, and we will call $u_h$ an \textit{approximate solution} of \eqref{Lax-Milgram: Euler equation}.

\begin{assumption}[Roger Temam's consistency hypotheses]
    \begin{itemize}
        \item[(i)] If $v_h\rightharpoonup v$ as $h\to 0$, and if $w_h\to w$ as $h\to 0$, then
        \begin{align}
            \label{Temam's consistency hypotheses 1}
            \lim_{h\to 0} a_h(v_h,w_h) = a(v,w),\ \lim_{h\to 0} a_h(w_h,v_h) = a(w,v).
        \end{align}
        \item[(ii)] If $v_h\rightharpoonup v$ as $h\to 0$, then
        \begin{align}
            \label{Temam's consistency hypotheses 2}
            \lim_{h\to 0} \langle\phi_h,v_h\rangle = \langle\phi,v\rangle.
        \end{align}
    \end{itemize}
\end{assumption}
This is a manner in which the forms $a_h$ and $\phi_h$ are consistent with the forms $a$ and $l$. A general convergence theorem is stated as follows.

\begin{theorem}
    Let $H$ be a Hilbert space, $a(u,v)$ is a coercive bilinear continuous form on $H\times H$, $\phi\in H^\star$. Under the hypotheses \eqref{uniform coerciveness}, \eqref{uniform bound of linear forms}, \eqref{Temam's consistency hypotheses 1}, \eqref{Temam's consistency hypotheses 2}, the solution $u_h$ of \eqref{Lax-Milgram: approximate Euler equation} converges strongly to the solution $u$ of \eqref{Lax-Milgram: Euler equation} as $h\to 0$.
\end{theorem}

\begin{proof}
    See \cite[pp. 32--33]{Temam2000}.
\end{proof}

\begin{remark}
    For any $h\in\mathcal{H}$, if $\{w_{ih}\}_{1\le i\le N(h)}$ constitutes a basis of $H_h$, then the approximate problem \eqref{Lax-Milgram: approximate Euler equation} is equivalent to a regular linear system for the components of $u_h$ in this basis; i.e., if $u_h\in H_h$ has the following representation
    \begin{align*}
        u_h = \sum_{i=1}^{N(h)} \xi_{ih}w_{ih},
    \end{align*}
    then \eqref{Lax-Milgram: approximate Euler equation} is equivalent to solving the following linear system
    \begin{align*}
        \sum_{i=1}^{N(h)} \xi_{ih}a_h(w_{ih},w_{jh}) &= \langle\phi_h,w_{jh}\rangle,\ 1\le j\le N(h).
    \end{align*}
\end{remark}

\section{Finite Difference Methods}
See, e.g., \cite{LeVeque2007}. We denote ${\bf h} = (h_i)_{i=1}^N$ the \textit{vector-mesh}, where $h_i$ is the mesh in the $x_i$-direction and $0 < h_i\le h_i^0$, for $i = 1,\ldots,N$. In this case, $\mathcal{H} = \prod_{i=1}^N (0,h_i^0)$.

\section{Finite Element Methods*}

\section{Finite Volume Methods*}

\subsection{Finite Volume Meshes}

\subsubsection{Structured meshes}
If $\Omega$ is a rectangle ($N = 2$) or a parallelepiped ($N = 3$), then it can be meshed with rectangular or parallelepipedic control volumes. For a regular structured mesh, every interior cell\texttt{/}control volume in the domain has the same number of neighboring cells.

\begin{definition}[Box in $\mathbb{R}^N$]
    Let ${\bf a},{\bf b}\in\mathbb{R}^N$ such that $a_i < b_i$ for all $i = 1,\ldots,N$. We define the box associated with ${\bf a},{\bf b}$ as the Cartesian product of the intervals formed component-wise, i.e.,
    \begin{align*}
        \operatorname{Box}({\bf a},{\bf b})\coloneqq\prod_{i=1}^N [a_i,b_i] = [a_1,b_1]\times\cdots\times[a_N,b_N].
    \end{align*}
\end{definition}

\begin{definition}[Rectangular admissible finite volume meshes for boxes in $\mathbb{R}^N$]
    Let ${\bf a},{\bf b}\in\mathbb{R}^N$ such that $a_i < b_i$ for all $i = 1,\ldots,N$. Let $(n_i)_{i=1}^N\in\mathbb{Z}_{> 0}^N$. For each $i = 1,\ldots,N$, let $h_{i,j} > 0$, $j = 1,\ldots,n_i$ such that
    \begin{align*}
        \sum_{j=1}^{n_i} h_{i,j} = b_i - a_i,
    \end{align*}
    and let $h_{i,0}\coloneqq 0$, $h_{i,n_i+1}\coloneqq 0$, $x_{i,\frac{1}{2}}\coloneqq a_i$, $x_{i,j+\frac{1}{2}}\coloneqq x_{i,j-\frac{1}{2}} + h_{i,j}$ (so that $x_{i,n_i+\frac{1}{2}} = b_i$), and
    \begin{align*}
        \mathcal{T}\coloneqq(K_{j_1,\ldots,j_N})_{j_1,\ldots,j_N}^{n_1,\ldots,n_N},\mbox{ where } K_{j_1,\ldots,j_N}\coloneqq\prod_{i=1}^N [x_{i,j_i-\frac{1}{2}},x_{i,j_i+\frac{1}{2}}];
    \end{align*}
    let $(x_{i,j})_{j=0}^{n_i+1}$ such that $x_{i,j-\frac{1}{2}} < x_{i,j} < x_{i,j+\frac{1}{2}}$ for $j = 1,\ldots,n_i$, $x_{i,0}\coloneqq a_i$, $x_{i,n_i+1}\coloneqq b_i$, and let $x_{j_1,\ldots,j_N}\coloneqq(x_{1,j_1},\ldots,x_{N,j_N})$ for $j_1 = 1,\ldots,n_i$; set
    \begin{equation*}
        \left.\begin{split}
            h_{i,j}^-\coloneqq x_{i,j} - x_{i,j-\frac{1}{2}},\ h_{i,j}^+&\coloneqq x_{i,j+\frac{1}{2}} - x_j,&&\mbox{for } j = 1,\ldots,n_i,\\
            h_{i,j+\frac{1}{2}}&\coloneqq x_{i,j+1} - x_{i,j},&&\mbox{for } j = 0,\ldots,n_i.
        \end{split}\right.
    \end{equation*}
    Set $h\coloneqq\max\{h_{i,j};i = 1,\ldots,N;j = 1,\ldots,n_i\}$.
    
    Let $\mathcal{E}$ and $\mathcal{P}$ be the corresponding sets of edges and vertices of these rectangles. Then the triple $(\mathcal{T},\mathcal{E},\mathcal{P})$ is called a rectangular admissible finite volume meshes for $\operatorname{Box}({\bf a},{\bf b})$.
\end{definition}
A particular case in 2D of the above definition is given in \cite{Eymard_Gallouet_Herbin2019}.

\begin{definition}[Rectangular admissible meshes in $\mathbb{R}^2$]
    Let $N_1\in\mathbb{N}^\star$, $N_2\in\mathbb{N}^\star$, $h_1,\ldots,h_{N_1} > 0$, $k_1,\ldots,k_{N_2} > 0$ s.t.
    \begin{align*}
        \sum_{i=1}^{N_1} h_i = 1,\ \sum_{i=1}^{N_2} k_i = 1,
    \end{align*}
    and let $h_0 = 0$, $h_{N_1 + 1} = 0$, $k_0 = 0$, $k_{N_2 + 1} = 0$.
    
    For $i = 1,\ldots,N_1$ let $x_{\frac{1}{2}} = 0$, $x_{i+\frac{1}{2}} = x_{i-\frac{1}{2}} + h_i$, (so that $x_{N_1+\frac{1}{2}} = 1$), and for $j = 1,\ldots,N_2$, $y_{\frac{1}{2}} = 0$, $y_{j+\frac{1}{2}} = y_{i-\frac{1}{2}} + k_j$, (so that $y_{N_2+\frac{1}{2}} = 1$) and
    \begin{align*}
        K_{i,j} = [x_{i-\frac{1}{2}},x_{i+\frac{1}{2}}]\times[y_{j-\frac{1}{2}},y_{j+\frac{1}{2}}].
    \end{align*}
    Let $(x_i)_{i=0}^{N_1+1}$, and $(y_j)_{j=0}^{N_2+1}$, s.t.
    \begin{align*}
        x_{i-\frac{1}{2}} < x_i < x_{i+\frac{1}{2}},\mbox{ for } i = 1,\ldots,N_1,\ x_0 = 0,\ x_{N_1+1} = 1,\\
        y_{j-\frac{1}{2}} < y_j < y_{j+\frac{1}{2}},\mbox{ for } j = 1,\ldots,N_2,\ y_0 = 0,\ y_{N_2+1} = 1,
    \end{align*}
    and let $x_{i,j} = (x_i,y_j)$, for $i = 1,\ldots,N_1$, $j = 1,\ldots,N_2$; set
    \begin{align*}
        h_i^- = x_i - x_{i-\frac{1}{2}},\ h_i^+ = x_{i+\frac{1}{2}} - x_i,\mbox{ for } i = 1,\ldots,N_1,\ h_{i+\frac{1}{2}} = x_{i+1} - x_i,\mbox{ for } i = 0,\ldots,N_1,\\
        k_j^- = y_j - y_{j-\frac{1}{2}},\ k_j^+ = y_{j+\frac{1}{2}} - y_j,\mbox{ for } j = 1,\ldots,N_2,\ k_{j+\frac{1}{2}} = y_{j+1} - y_j,\mbox{ for } j = 0,\ldots,N_2.
    \end{align*}
    Let $h = \max\{(h_i,\ i = 1,\ldots,N_1),(k_j,\ j = 1,\ldots,N_2)\}$.
\end{definition}


\subsubsection{Unstructured meshes}
We recall the following definition from \cite[Definition 9.1, p. 37]{Eymard_Gallouet_Herbin2019}.

\begin{definition}[Admissible finite volume meshes]
    \label{def: admissible FV mesh}
    Let $\Omega$ be an open bounded polygonal subset of $\mathbb{R}^N$, $N\in\{2,3\}$. An \emph{admissible finite volume mesh} of $\Omega$ is defined as a triple $(\mathcal{T},\mathcal{E},\mathcal{P})$, where $\mathcal{T}$ is a family of ``control volumes'', which are open polygonal convex subsets of $\Omega$, $\mathcal{E}$ is a family of subsets of $\overline{\Omega}$ contained in hyperplanes of $\mathbb{R}^N$ (these are the edges (2D) or sides (3D) of the control volumes) with strictly positive $(N - 1)$-dimensional measure, and $\mathcal{P}$ is a family of points of $\Omega$ satisfying the following properties:
    \begin{itemize}
        \item[(i)] The closure of the union of all the control volumes is $\overline{\Omega}$;
        \item[(ii)] For any $K\in\mathcal{T}$, there exists a subset $\mathcal{E}_K$ of $\mathcal{E}$ s.t. $\partial K = \overline{K}\backslash K = \bigcup_{\sigma\in\mathcal{E}_K} \overline{\sigma}$. Furthermore, $\mathcal{E} = \bigcup_{K\in\mathcal{T}} \mathcal{E}_K$.
        \item[(iii)] For any $(K,L)\in\mathcal{T}^2$ with $K\ne L$, either the $(N - 1)$-dimensional Lebesgue measure of $\overline{K}\cap\overline{L}$ is 0 or $\overline{K}\cap\overline{L} = \overline{\sigma}$ for some $\sigma\in\mathcal{E}$, which will then be denoted by $K|L$.
        \item[(iv)] The family $\mathcal{P} = ({\bf x}_K)_{K\in\mathcal{T}}$ is s.t. ${\bf x}_K\in\overline{K}$ (for all $K\in\mathcal{T}$) and, if $\sigma = K|L$, it is assumed that ${\bf x}_K\ne{\bf x}_L$, and that the straight line $\mathcal{D}_{K,L}$ going through ${\bf x}_K$ and ${\bf x}_L$ is orthogonal to $K|L$.
        \item[(v)] For any $\sigma\in\mathcal{E}$ s.t. $\sigma\subset\partial\Omega$, let $K$ be the control volume s.t. $\sigma\in\mathcal{E}_K$. If ${\bf x}_K\notin\sigma$, let $\mathcal{D}_{K,\sigma}$ be the straight line going through ${\bf x}_K$ and orthogonal to $\sigma$, then the condition $\mathcal{D}_{K,\sigma}\cap\sigma\ne 0$ is assumed; let ${\bf y}_\sigma = \mathcal{D}_{K,\sigma}\cap\sigma$.
    \end{itemize}
\end{definition}
In the sequel, the following notations are used.
\begin{itemize}
    \item The mesh size is defined by: $\operatorname{size}(\mathcal{T})\coloneqq\sup\{\operatorname{diam}(K),\ K\in\mathcal{T}\}$.
    \item For any $K\in\mathcal{T}$ and $\sigma\in\mathcal{E}$, $\operatorname{m}(K)$ is the $N$-dimensional Lebesgue measure of $K$ (it is the area of $K$ in the 2D case and the volume in the 3D case) and $\operatorname{H}_{N-1}(\sigma)$ the $(N - 1)$-dimensional Hausdorff measure of $\sigma$.
    \item The set of interior (resp. boundary) edges is denoted by $\mathcal{E}_{\rm int}$ (resp., $\mathcal{E}_{\rm ext}$), i.e. $\mathcal{E}_{\rm int} = \{\sigma\in\mathcal{E};\sigma\not\subset\partial\Omega\}$ (resp., $\mathcal{E}_{\rm ext} = \{\sigma\in\mathcal{E};\sigma\subset\partial\Omega\}$).
    \item The set of neighbors of $K$ is denoted by $\mathcal{N}(K)$, i.e. $\mathcal{N}(K) = \{L\in\mathcal{T};\exists\sigma\in\mathcal{E}_K,\ \overline{\sigma} = \overline{K}\cap\overline{L}\}$.
    \item If $\sigma = K|L$, we denote $d_\sigma$ or $d_{K|L}$ the Euclidean distance between ${\bf x}_K$ and ${\bf x}_L$ (which is positive) and by $d_{K,\sigma}$ the distance from ${\bf x}_K$ to $\sigma$.
    \item If $\sigma\in\mathcal{E}_K\cap\mathcal{E}_{\rm ext}$, let $d_\sigma$ denote the Euclidean distance between $x_K$ and ${\bf y}_\sigma$ (then, $d_\sigma = d_{K,\sigma}$).
    \item For any $\sigma\in\mathcal{E}$; the ``transmissibility'' through $\sigma$ is defined by $\tau_\sigma\coloneqq\frac{\operatorname{H}_{N-1}(\sigma)}{d_\sigma}$ if $d_\sigma\ne 0$.
    \item In some results and proofs given below, there are summations over $\sigma\in\mathcal{E}_0$, with $\mathcal{E}_0\coloneqq\{\sigma\in\mathcal{E};d_\sigma\ne 0\}$.
    
    For simplicity, (in these results and proofs) $\mathcal{E} = \mathcal{E}_0$ is assumed.
\end{itemize}
In \cite[p. 673]{Mazya_Rossmann2009}:
\begin{definition}[Domain of polyhedral type]
    The bounded domain $\mathcal{G}\subset\mathbb{R}^3$ is said to be a \emph{domain of polyhedral type} if
    \begin{itemize}
        \item[(i)] The boundary $\partial\mathcal{G}$ consist of smooth (of class $C^\infty$) open 2D manifolds $\Gamma_j$ (the faces of $\mathcal{G}$), $j = 1,\ldots,N$, smooth curves $M_k$ (the edges), $k = 1,\ldots,m$, and vertices ${\bf x}^{(1)},\ldots,{\bf x}^{(d)}$.
        \item[(ii)] For every $\xi\in M_k$ there exist a neighborhood $U_\xi$ and a diffeomorphism (a $C^\infty$ mapping) $\kappa_\xi$ which maps $\mathcal{G}\cap\mathcal{U}_\xi$ onto $\mathcal{D}_\xi\cap B_1$, where $\mathcal{D}_\xi$ is a dihedron of the form 
        \begin{align*}
            \left\{{\bf x} = (x_1,x_2,x_3)\in\mathbb{R}^3;0 < r < \infty,\ -\frac{\theta}{2} < \varphi < \frac{\theta}{2},\ x_3\in\mathbb{R}\right\},
        \end{align*}
        and $B_1$ is the unit ball.
        \item[(iii)] For every vertex ${\bf x}^{(j)}$ there exists a neighborhood $\mathcal{U}_j$ and a diffeomorphism $\kappa_j$ mapping $\mathcal{G}\cap\mathcal{U}_j$ onto $\mathcal{K}_j\cap B_1$, where $\mathcal{K}_j$ is a polyhedral cone with vertex at the origin.
    \end{itemize}
\end{definition}
The set $M_1\cup\cdots\cup M_m\cup\{{\bf x}^{(1)},\ldots,{\bf x}^{(d)}\}$ of the singular boundary points is denoted by $\mathcal{S}$.

There are 3 types of indexing: local-, global-, and discretization indexings.

\section{Mesh Generations*}

\subsection{OpenFOAM \texttt{blockMesh} utility}
An duct geometry is generated by an OpenFOAM utility called \texttt{blockMesh}.



%------------------------------------------------------------------------------%

\chapter{Terms Integrated by Parts in the Derivations of Adjoint Systems}

This appendix is devoted to present a comprehensive list of the terms which are integrated by parts in the derivation of the adjoint systems. We consider below two cases: stationary and instationary cases, since in the latter, there will be the additional time integration.

\section{Integration by parts formulas}

\subsection{Divergence theorem}

\begin{theorem}[Divergence\texttt{/}Gauss-Green]
    \label{theorem: divergence}
    Given a bounded smooth domain $D$ in $\mathbb{R}^N$ and a Lipschitzian domain $\Omega$ in $D$, then
    \begin{align}
        \label{div}
        \tag{div}
        \forall\boldsymbol{\phi}\in C^1\left(\overline{D},\mathbb{R}^N\right),\ \int_\Omega \nabla\cdot\boldsymbol{\phi}{\rm d}x = \int_\Gamma \boldsymbol{\phi}\cdot{\bf n}{\rm d}\Gamma,
    \end{align}
    where ${\bf n}$ denotes the outward unit normal field.
\end{theorem}

\begin{lemma}[Integration by parts]
    If $\Omega$ is a bounded $C^1$ open set in $\mathbb{R}^N$ with boundary $\Gamma\coloneqq\partial\Omega$ and $v\in C^1(\overline{\Omega})$, $f\in C^1(\overline{\Omega})$, then
    \begin{align}
        \label{ibp}
        \tag{ibp}
        \int_\Omega \nabla f\cdot{\bf v}{\rm d}{\bf x} = -\int_\Omega f\nabla\cdot{\bf v}{\rm d}{\bf x} + \int_\Gamma f{\bf v}\cdot{\bf n}{\rm d}\Gamma.
    \end{align}
\end{lemma}

\begin{proof}
    Use integration by parts formula for each component:
    \begin{align*}
        \int_\Omega \nabla f\cdot{\bf v}{\rm d}{\bf x} &= \int_\Omega \sum_{i=1}^N \partial_{x_i}fv_i{\rm d}{\bf x} = \sum_{i=1}^N \int_\Omega \partial_{x_i}fv_i{\rm d}{\bf x} = \sum_{i=1}^N -\int_\Omega f\partial_{x_i}v_i{\rm d}{\bf x} + \int_\Gamma fv_in_i{\rm d}\Gamma\\
        &= -\int_\Omega f\sum_{i=1}^N \partial_{x_i}v_i{\rm d}{\bf x} + \int_\Gamma f\sum_{i=1}^N v_in_i{\rm d}\Gamma = -\int_\Omega f\nabla\cdot{\bf v}{\rm d}{\bf x} + \int_\Gamma f{\bf v}\cdot{\bf n}{\rm d}\Gamma.
    \end{align*}
\end{proof}

\subsection{Green's identities for scalar functions}

\begin{lemma}[Green's identities for scalar functions]
    If $\Omega$ is a bounded $C^1$ open set in $\mathbb{R}^N$ with boundary $\Gamma\coloneqq\partial\Omega$ and $u,v\in C^2(\overline{\Omega})$, then
    \begin{align}
        \int_\Omega \Delta uv{\rm d}{\bf x} &= -\int_\Omega \nabla u\cdot\nabla v{\rm d}{\bf x} + \int_\Gamma \partial_{\bf n}uv{\rm d}\Gamma,\label{Green's 1st identity}\tag{1st-Green}\\
        \int_\Omega \Delta uv{\rm d}{\bf x} &= \int_\Omega u\Delta v{\rm d}{\bf x} + \int_\Gamma \partial_{\bf n}uv - u\partial_{\bf n}v{\rm d}\Gamma.\label{Green's 2nd identity}\tag{2nd-Green}
    \end{align}
\end{lemma}
About notation, $\Delta uv$ means $(\Delta u)v$, not $\Delta(uv)$; also, $\partial_{\bf n}uv$ means $(\partial_{\bf n}u)v$, not $\partial_{\bf n}(uv)$.

%
Note that Green's first identity \eqref{Green's 1st identity} is a special case of the more general identity derived from the divergence theorem \ref{theorem: divergence} by substituting $\boldsymbol{\phi} = \phi{\bf f}$ into \eqref{Green's 1st identity}:
\begin{align}
    \label{divergence theorem's consequence}
    \int_\Omega \left(\phi\nabla\cdot{\bf f} + \nabla\phi\cdot{\bf f}\right){\rm d}{\bf x} = \int_\Gamma \phi{\bf f}\cdot{\bf n}{\rm d}\Gamma.
\end{align}

\subsection{Green's identities for vector fields}
\begin{lemma}[Green's identities for vector fields]
    If $\Omega$ is a bounded $C^1$ open set in $\mathbb{R}^N$ with boundary $\Gamma\coloneqq\partial\Omega$ and ${\bf u},{\bf v}\in C^2(\overline{\Omega},\mathbb{R}^N)$, then
    \begin{align}
        \int_\Omega \Delta{\bf u}\cdot{\bf v}{\rm d}{\bf x} &= -\int_\Omega \nabla{\bf u}:\nabla{\bf v}{\rm d}{\bf x} + \int_\Gamma \partial_{\bf n}{\bf u}\cdot{\bf v}{\rm d}\Gamma,\label{Green's 1st identity vector}\tag{1st-Green-vec}\\
        \int_\Omega \Delta{\bf u}\cdot{\bf v}{\rm d}{\bf x} &= \int_\Omega {\bf u}\cdot\Delta{\bf v}{\rm d}{\bf x} + \int_\Gamma \left(\partial_{\bf n}{\bf u}\cdot{\bf v} - {\bf u}\cdot\partial_{\bf n}{\bf v}\right){\rm d}\Gamma.\label{Green's 2nd identity vector}\tag{2nd-Green-vec}
    \end{align}
\end{lemma}

\begin{proof}
    Use Green's identities for scalar functions to obtain the former:
    \begin{align*}
        \int_\Omega \Delta{\bf u}\cdot{\bf v}{\rm d}{\bf x} &= \int_\Omega \sum_{i=1}^N \Delta u_iv_i{\rm d}{\bf x} = \sum_{i=1}^N \int_\Omega \Delta u_iv_i{\rm d}{\bf x} = \sum_{i=1}^N -\int_\Omega \nabla u_i\cdot\nabla v_i{\rm d}{\bf x} + \int_\Gamma \partial_{\bf n}u_iv_i{\rm d}\Gamma\\
        &= -\int_\Omega \sum_{i=1}^N \nabla u_i\cdot\nabla v_i{\rm d}{\bf x} + \int_\Gamma \sum_{i=1}^N \partial_{\bf n}u_iv_i{\rm d}\Gamma = -\int_\Omega \nabla{\bf u}:\nabla{\bf v}{\rm d}{\bf x} + \int_\Gamma \partial_{\bf n}{\bf u}\cdot{\bf v}{\rm d}\Gamma.
    \end{align*}
    The latter is obtained by subtracting the former with its version after interchanging the roles of ${\bf u}$ and ${\bf v}$.
\end{proof}

\subsection{Integration by parts involving matrices\texttt{/}2nd-order tensors}
For a matrix\texttt{/}second-order tensor $A = (a_{ij})_{i,j=1}^{M,N} = (A_{i \cdot})_{i=1}^M = ((A_{\cdot j})_{j=1}^N)^\top\in C^1(\overline{\Omega};\mathbb{R}^{M\times N})$ (where $A_{i \cdot}$ and $A_{\cdot j}$ is the $i$th row and $j$th column of $A$, respectively), its divergence is defined \textit{column-wise}\footnote{In \cite{John2004} and \cite{John2016}, the divergence of a matrix is defined \textit{row-wise} instead.} by
\begin{align*}
    \nabla\cdot A = \left(\nabla\cdot A_{\cdot j}\right)_{j=1}^N = \left(\sum_{i=1}^M \partial_{x_i}a_{ij}\right)_{j=1}^N.
\end{align*}

\begin{lemma}[Integration by parts involving 2nd-order tensors]
    Let $\Omega$ be a bounded $C^1$ open set in $\mathbb{R}^N$ with boundary $\Gamma\coloneqq\partial\Omega$ and $A = (a_{ij})_{i,j=1}^N\in C^1(\overline{\Omega},\mathbb{R}^{N^2})$, ${\bf v}\in C^1(\overline{\Omega},\mathbb{R}^N)$, $f\in C^1(\overline{\Omega})$, then
    \begin{align}
        \label{integration by parts for matrix 1}
        \tag{ibp-mat1}
        \int_\Omega \left(\nabla\cdot A\right)\cdot{\bf v}{\rm d}{\bf x}&= -\int_\Omega A:\nabla{\bf v}{\rm d}{\bf x} + \int_\Gamma {\bf n}^\top A{\bf v}{\rm d}\Gamma,\\
        \label{integration by parts for matrix 2}
        \tag{ibp-mat2}
        \int_\Omega \nabla\cdot(fA)\cdot{\bf v}{\rm d}{\bf x} &= -\int_\Omega fA:\nabla{\bf v}{\rm d}{\bf x} + \int_\Gamma f{\bf n}^\top A{\bf v}{\rm d}\Gamma.
    \end{align}
    Moreover, if $A$ is symmetric, then the following also holds:
    \begin{align}
        \int_\Omega \left(\nabla\cdot A\right)\cdot{\bf v}{\rm d}{\bf x} &= -\int_\Omega A:\boldsymbol{\varepsilon}({\bf v}){\rm d}{\bf x} + \int_\Gamma {\bf n}^\top A{\bf v}{\rm d}\Gamma,\label{integration by parts for matrix 3}\tag{ibp-mat3}\\
        \int_\Omega \nabla\cdot(fA)\cdot{\bf v}{\rm d}{\bf x} &= -\int_\Omega fA:\boldsymbol{\varepsilon}({\bf v}){\rm d}{\bf x} + \int_\Gamma f{\bf n}^\top A{\bf v}{\rm d}\Gamma,\label{integration by parts for matrix 4}\tag{ibp-mat4}
    \end{align}
\end{lemma}

\begin{proof}
    Use integration by parts formula to obtain \eqref{integration by parts for matrix 1}:
    \begin{align*}
        \int_\Omega (\nabla\cdot A)\cdot{\bf v}{\rm d}{\bf x} &= \int_\Omega \left(\sum_{i=1}^N \partial_{x_i}a_{ij}\right)_{j=1}^N\cdot{\bf v}{\rm d}{\bf x} = \int_\Omega \sum_{j=1}^N\sum_{i=1}^N \partial_{x_i}a_{ij}v_j{\rm d}{\bf x}\\
        &= \sum_{i=1}^N\sum_{j=1}^N \int_\Omega \partial_{x_i}a_{ij}v_j{\rm d}{\bf x} = \sum_{i=1}^N\sum_{j=1}^N -\int_\Omega a_{ij}\partial_{x_i}v_j{\rm d}{\bf x} + \int_\Gamma n_ia_{ij}v_j{\rm d}\Gamma\\
        &= -\int_\Omega \sum_{i=1}^N\sum_{j=1}^N a_{ij}\partial_{x_i}v_j{\rm d}{\bf x} + \int_\Gamma \sum_{i=1}^N\sum_{j=1}^N n_ia_{ij}v_j{\rm d}\Gamma = -\int_\Omega A:\nabla{\bf v}{\rm d}{\bf x} + \int_\Gamma {\bf n}^\top A{\bf v}{\rm d}\Gamma,
    \end{align*}
    and \eqref{integration by parts for matrix 2} is obtained by applying the first one for $\widetilde{A}\coloneqq fA$.
    
    If $A$ is symmetric, i.e., $a_{ij} = a_{ji}$ for all $i,j = 1,\ldots,N$, then
    \begin{align*}
        A:\nabla{\bf v} = \sum_{i=1}^N\sum_{j=1}^N a_{ij}\partial_{x_i}v_j = \sum_{i=1}^N\sum_{j=1}^N a_{ji}\partial_{x_j}v_i = \sum_{i=1}^N\sum_{j=1}^N a_{ij}\partial_{x_j}v_i,
    \end{align*}
    and thus
    \begin{align*}
        A:\nabla{\bf v} = \sum_{i=1}^N\sum_{j=1}^N \frac{1}{2}a_{ij}(\partial_{x_i}v_j + \partial_{x_j}v_i) = \sum_{i=1}^N\sum_{j=1}^N a_{ij}\varepsilon_{ij}({\bf v}) = A:\boldsymbol{\varepsilon}({\bf v}).
    \end{align*}
    Then \eqref{integration by parts for matrix 3} and \eqref{integration by parts for matrix 4} follow from \eqref{integration by parts for matrix 1} and \eqref{integration by parts for matrix 2}, respectively.
\end{proof}

\begin{corollary}[Integration by parts involving 2nd-order tensors]
    Let $\Omega$ be a bounded $C^1$ open set in $\mathbb{R}^N$ with boundary $\Gamma\coloneqq\partial\Omega$ and ${\bf u}\in C^2(\overline{\Omega},\mathbb{R}^N)$, ${\bf v}\in C^1(\overline{\Omega},\mathbb{R}^N)$, $f\in C^1(\overline{\Omega})$, then
    \begin{align}
        \label{integration by parts for matrix 5}
        \tag{ibp-mat5}
        \int_\Omega \nabla\cdot(f\nabla{\bf u})\cdot{\bf v}{\rm d}{\bf x} &= -\int_\Omega f\nabla{\bf u}:\nabla{\bf v}{\rm d}{\bf x} + \int_\Gamma f\partial_{\bf n}{\bf u}\cdot{\bf v}{\rm d}\Gamma,\\
        \int_\Omega \nabla\cdot(f\boldsymbol{\varepsilon}({\bf u}))\cdot{\bf v}{\rm d}{\bf x} &=  -\int_\Omega f\boldsymbol{\varepsilon}({\bf u}):\nabla{\bf v}{\rm d}{\bf x} + \int_\Gamma f\boldsymbol{\varepsilon}_{\bf n}({\bf u})\cdot{\bf v}{\rm d}\Gamma\nonumber\\
        &= -\int_\Omega f\boldsymbol{\varepsilon}({\bf u}):\boldsymbol{\varepsilon}({\bf v}){\rm d}{\bf x} + \int_\Gamma f\boldsymbol{\varepsilon}_{\bf n}({\bf u})\cdot{\bf v}{\rm d}\Gamma.
        \label{integration by parts for matrix 6}
        \tag{ibp-mat6}
    \end{align}
    If, in addition, ${\bf v}\in C^2(\overline{\Omega},\mathbb{R}^N)$, then
    \begin{align}
        \label{integration by parts for matrix 7}
        \tag{ibp-mat7}
        \int_\Omega \nabla\cdot(f\nabla{\bf u})\cdot{\bf v}{\rm d}{\bf x} &= \int_\Omega \nabla\cdot(f\nabla{\bf v})\cdot{\bf u}{\rm d}{\bf x} + \int_\Gamma f(\partial_{\bf n}{\bf u}\cdot{\bf v} - {\bf u}\cdot\partial_{\bf n}{\bf v}){\rm d}\Gamma,\\
        \label{integration by parts for matrix 8}
        \tag{ibp-mat8}
        \int_\Omega \nabla\cdot(f\boldsymbol{\varepsilon}({\bf u}))\cdot{\bf v}{\rm d}{\bf x} &= \int_\Omega \nabla\cdot(f\boldsymbol{\varepsilon}({\bf v}))\cdot{\bf u}{\rm d}{\bf x} + \int_\Gamma f(\boldsymbol{\varepsilon}_{\bf n}({\bf u})\cdot{\bf v} - {\bf u}\cdot\boldsymbol{\varepsilon}_{\bf n}({\bf v})){\rm d}\Gamma.
    \end{align}
\end{corollary}

\begin{proof}
    Applying \eqref{integration by parts for matrix 2} with $A = \nabla{\bf u}$ and $A = \boldsymbol{\varepsilon}({\bf u})$ yields \eqref{integration by parts for matrix 5} and the first equality of \eqref{integration by parts for matrix 6}, respectively. Applying \eqref{integration by parts for matrix 4} with $A = \boldsymbol{\varepsilon}({\bf u})$ yields the second equality of \eqref{integration by parts for matrix 6}.
    
    Interchanging the role of ${\bf u}$ and ${\bf v}$ in \eqref{integration by parts for matrix 5} and the second equality of \eqref{integration by parts for matrix 6} yields
    \begin{align*}
        \int_\Omega \nabla\cdot(f\nabla{\bf v})\cdot{\bf u}{\rm d}{\bf x} &= -\int_\Omega f\nabla{\bf u}:\nabla{\bf v}{\rm d}{\bf x} + \int_\Gamma f{\bf u}\cdot\partial_{\bf n}{\bf v}{\rm d}\Gamma,\\
        \int_\Omega \nabla\cdot(f\boldsymbol{\varepsilon}({\bf v}))\cdot{\bf u}{\rm d}{\bf x} &= -\int_\Omega f\boldsymbol{\varepsilon}({\bf u}):\boldsymbol{\varepsilon}({\bf v}){\rm d}{\bf x} + \int_\Gamma f{\bf u}\cdot\boldsymbol{\varepsilon}_{\bf n}({\bf v}){\rm d}\Gamma.
    \end{align*}
    Subtracting these with \eqref{integration by parts for matrix 5}, \eqref{integration by parts for matrix 6} yields \eqref{integration by parts for matrix 7}, \eqref{integration by parts for matrix 8}, respectively.
\end{proof}

\begin{remark}[An integral identity]
    Interchanging the role of ${\bf u}$ and ${\bf v}$ in the first equality of \eqref{integration by parts for matrix 6} and then subtracting them yield also
    \begin{align*}
        \int_\Omega \nabla\cdot(f\boldsymbol{\varepsilon}({\bf u}))\cdot{\bf v} - \nabla\cdot(f\boldsymbol{\varepsilon}({\bf v}))\cdot{\bf u}{\rm d}{\bf x} = \int_\Omega f(\nabla{\bf u}:\boldsymbol{\varepsilon}({\bf v}) - \boldsymbol{\varepsilon}({\bf u}):\nabla{\bf v}){\rm d}{\bf x} + \int_\Gamma f(\boldsymbol{\varepsilon}_{\bf n}({\bf u})\cdot{\bf v} - {\bf u}\cdot\boldsymbol{\varepsilon}_{\bf n}({\bf v})){\rm d}\Gamma.
    \end{align*}
    However, this identity is rarely used in this thesis.
\end{remark}

\section{Stationary Cases}
\label{ibp: stationary}
\begin{enumerate}[leftmargin=0mm]
    \item \textbf{Diffusion term $-\nabla\cdot(\nu\nabla{\bf u})$ with $\nu = \nu({\bf x})$.} Apply \eqref{integration by parts for matrix 5} with $f = \nu$ to obtain
    \begin{align*}
        -\int_\Omega \nabla\cdot(\nu\nabla{\bf u})\cdot{\bf v}{\rm d}{\bf x} = \int_\Omega \nu\nabla{\bf u}:\nabla{\bf v}{\rm d}{\bf x} - \int_\Gamma \nu\partial_{\bf n}{\bf u}\cdot{\bf v}{\rm d}\Gamma.
    \end{align*}
    Apply \eqref{integration by parts for matrix 7} with $f = \nu$ to obtain
    \begin{align*}
        -\int_\Omega \nabla\cdot(\nu\nabla{\bf u})\cdot{\bf v}{\rm d}{\bf x} = -\int_\Omega \nabla\cdot(\nu\nabla{\bf v})\cdot{\bf u}{\rm d}{\bf x} - \int_\Gamma \nu(\partial_{\bf n}{\bf u}\cdot{\bf v} - {\bf u}\cdot\partial_{\bf n}{\bf v}){\rm d}\Gamma.
    \end{align*}
    \item \textbf{Diffusion term $-\nu\Delta{\bf u}$, with $\nu = {\rm const}$.} Integration by parts formulas for this term can be deduced directly from the previous term when by letting $\nu = \mbox{const}$.
    
    Or, alternatively, we can apply \eqref{Green's 1st identity vector} to obtain
    \begin{align*}
        -\int_\Omega \nu\Delta{\bf u}\cdot{\bf v}{\rm d}{\bf x} = \int_\Omega \nu\nabla{\bf u}:\nabla{\bf v}{\rm d}{\bf x} - \int_\Gamma \nu\partial_{\bf n}{\bf u}\cdot{\bf v}{\rm d}\Gamma.
    \end{align*}
    Apply \eqref{Green's 2nd identity vector} to obtain
    \begin{align*}
        -\int_\Omega \nu\Delta{\bf u}\cdot{\bf v}{\rm d}{\bf x} = -\int_\Omega \nu{\bf u}\cdot\Delta{\bf v}{\rm d}{\bf x} - \int_\Gamma \nu(\partial_{\bf n}{\bf u}\cdot{\bf v} - {\bf u}\cdot\partial_{\bf n}{\bf v}){\rm d}\Gamma.
    \end{align*}
    \item \textbf{Diffusion term $-\nabla\cdot(2\nu\boldsymbol{\varepsilon}({\bf u}))$ with $\nu = \nu({\bf x})$.} Apply \eqref{integration by parts for matrix 6} with $f = 2\nu$ to obtain
    \begin{align*}
        -\int_\Omega \nabla\cdot(2\nu\boldsymbol{\varepsilon}({\bf u}))\cdot{\bf v}{\rm d}{\bf x} = \int_\Omega 2\nu\boldsymbol{\varepsilon}({\bf u}):\boldsymbol{\varepsilon}({\bf v}){\rm d}{\bf x} - \int_\Gamma 2\nu\boldsymbol{\varepsilon}_{\bf n}({\bf u})\cdot{\bf v}{\rm d}\Gamma.
    \end{align*}
    Apply \eqref{integration by parts for matrix 8} with $f = 2\nu$ to obtain
    \begin{align*}
        -\int_\Omega \nabla\cdot(2\nu\boldsymbol{\varepsilon}({\bf u}))\cdot{\bf v}{\rm d}{\bf x} = -\int_\Omega \nabla\cdot(2\nu\boldsymbol{\varepsilon}({\bf v}))\cdot{\bf u}{\rm d}{\bf x} - \int_\Gamma 2\nu(\boldsymbol{\varepsilon}_{\bf n}({\bf u})\cdot{\bf v} - {\bf u}\cdot\boldsymbol{\varepsilon}_{\bf n}({\bf v})){\rm d}\Gamma.
    \end{align*}
    \item \textbf{Diffusion term $-2\nu\nabla\cdot\boldsymbol{\varepsilon}({\bf u})$ with $\nu = {\rm const}$.} Integration by parts formulas for this term can be deduced directly from the previous term when by letting $\nu = \mbox{const}$:
    \begin{align*}
        -\int_\Omega 2\nu\nabla\cdot\boldsymbol{\varepsilon}({\bf u})\cdot{\bf v}{\rm d}{\bf x} &= \int_\Omega 2\nu\boldsymbol{\varepsilon}({\bf u}):\boldsymbol{\varepsilon}({\bf v}){\rm d}{\bf x} - \int_\Gamma 2\nu\boldsymbol{\varepsilon}_{\bf n}({\bf u})\cdot{\bf v}{\rm d}\Gamma,\\
        -\int_\Omega 2\nu\nabla\cdot\boldsymbol{\varepsilon}({\bf u})\cdot{\bf v}{\rm d}{\bf x} &= -\int_\Omega 2\nu\nabla\cdot\boldsymbol{\varepsilon}({\bf v})\cdot{\bf u}{\rm d}{\bf x} - \int_\Gamma 2\nu(\boldsymbol{\varepsilon}_{\bf n}({\bf u})\cdot{\bf v} - {\bf u}\cdot\boldsymbol{\varepsilon}_{\bf n}({\bf v})){\rm d}\Gamma.
    \end{align*}
    \item \textbf{Convection term $({\bf u}\cdot\nabla){\bf u}\cdot{\bf v}$.}
    \begin{align*}
        &\int_\Omega ({\bf u}\cdot\nabla){\bf u}\cdot{\bf v}{\rm d}{\bf x} = \int_\Omega \sum_{j=1}^N u_j\partial_{x_j}{\bf u}\cdot{\bf v}{\rm d}{\bf x} = \int_\Omega \sum_{j=1}^N\sum_{i=1}^N u_j\partial_{x_j}u_iv_i{\rm d}{\bf x} = \sum_{i=1}^N\sum_{j=1}^N \int_\Omega u_jv_i\partial_{x_j}u_i{\rm d}{\bf x}\\
        &= \sum_{i=1}^N\sum_{j=1}^N -\int_\Omega u_i\partial_{x_j}u_jv_i + u_iu_j\partial_{x_j}v_i{\rm d}{\bf x} + \int_\Gamma u_iv_iu_jn_j{\rm d}\Gamma\\
        &= -\int_\Omega \sum_{i=1}^N u_iv_i\sum_{j=1}^N \partial_{x_j}u_j + \sum_{i=1}^N\sum_{j=1}^N u_i\partial_{x_j}v_iu_j{\rm d}{\bf x} + \int_\Gamma \sum_{i=1}^N u_iv_i\sum_{j=1}^N u_jn_j{\rm d}\Gamma\\
        &= -\int_\Omega ({\bf u}\cdot{\bf v})\nabla\cdot{\bf u} + ({\bf u}\cdot\nabla){\bf v}\cdot{\bf u}{\rm d}{\bf x} + \int_\Gamma ({\bf u}\cdot{\bf v})({\bf u}\cdot{\bf n}){\rm d}\Gamma.
    \end{align*}
    \item \textbf{Convection term $\nabla\cdot({\bf u}\otimes{\bf u})$.} Apply \eqref{integration by parts for matrix 1} with $A = {\bf u}\otimes{\bf u}$ to obtain
    \begin{align*}
        \int_\Omega \nabla\cdot({\bf u}\otimes{\bf u})\cdot{\bf v}{\rm d}{\bf x} = -\int_\Omega ({\bf u}\otimes{\bf u}):\nabla{\bf v}{\rm d}{\bf x} + \int_\Gamma {\bf n}^\top({\bf u}\otimes{\bf u}){\bf v}{\rm d}\Gamma.
    \end{align*}
    \item \textbf{Gradient pressure term $\nabla p$.} Apply \eqref{ibp} with $f = p$ to obtain 
    \begin{align*}
        \int_\Omega \nabla p\cdot{\bf v}{\rm d}{\bf x} = -\int_\Omega p\nabla\cdot{\bf v}{\rm d}{\bf x} + \int_\Gamma p{\bf v}\cdot{\bf n}{\rm d}\Gamma.
    \end{align*}
    \item \textbf{Domain cost term $D_{\nabla{\bf u}}J_\Omega({\bf x},{\bf u},\nabla{\bf u},p)\nabla\tilde{\bf u}$.}
    \begin{align*}
        &\int_\Omega D_{\nabla{\bf u}} J_\Omega({\bf x},{\bf u},\nabla{\bf u},p)\nabla\tilde{\bf u}{\rm d}{\bf x} = \int_\Omega \nabla_{\nabla{\bf u}}J_\Omega({\bf x},{\bf u},\nabla{\bf u},p):\nabla\tilde{\bf u}{\rm d}{\bf x}\\
        =&\,\int_\Omega \sum_{i=1}^N\sum_{j=1}^N \partial_{\partial_{x_i}u_j}J_\Omega({\bf x},{\bf u},\nabla{\bf u},p)\partial_{x_i}\tilde{u}_j{\rm d}{\bf x} = \sum_{i=1}^N\sum_{j=1}^N \int_\Omega \partial_{\partial_{x_i}u_j}J_\Omega({\bf x},{\bf u},\nabla{\bf u},p)\partial_{x_i}\tilde{u}_j{\rm d}{\bf x}\\
        =&\, \sum_{i=1}^N\sum_{j=1}^N -\int_\Omega \partial_{x_i}\partial_{\partial_{x_i}u_j}J_\Omega({\bf x},{\bf u},\nabla{\bf u},p)\tilde{u}_j{\rm d}{\bf x} + \int_\Gamma n_i\partial_{\partial_{x_i}u_j}J_\Omega({\bf x},{\bf u},\nabla{\bf u},p)\tilde{u}_j{\rm d}\Gamma\\
        =&\, -\int_\Omega \sum_{j=1}^N \tilde{u}_j\sum_{i=1}^N \partial_{x_i}\partial_{\partial_{x_i}u_j}J_\Omega({\bf x},{\bf u},\nabla{\bf u},p){\rm d}{\bf x} + \int_\Gamma \sum_{i=1}^N\sum_{j=1}^N n_i\partial_{\partial_{x_i}u_j}J_\Omega({\bf x},{\bf u},\nabla{\bf u},p)\tilde{u}_j{\rm d}\Gamma\\
        =&\, -\int_\Omega \sum_{j=1}^N \tilde{u}_j\nabla\cdot\left(\nabla_{\nabla u_j}J_\Omega({\bf x},{\bf u},\nabla{\bf u},p)\right){\rm d}{\bf x} + \int_\Gamma {\bf n}^\top\nabla_{\nabla{\bf u}}J_\Omega({\bf x},{\bf u},\nabla{\bf u},p)\tilde{\bf u}{\rm d}\Gamma\\
        =&\, -\int_\Omega \nabla\cdot\left(\nabla_{\nabla{\bf u}}J_\Omega({\bf x},{\bf u},\nabla{\bf u},p)\right)\cdot\tilde{\bf u}{\rm d}{\bf x} + \int_\Gamma {\bf n}^\top\nabla_{\nabla{\bf u}}J_\Omega({\bf x},{\bf u},\nabla{\bf u},p)\tilde{\bf u}{\rm d}\Gamma,
    \end{align*}
    where $\nabla_{\nabla{\bf u}} f(\nabla{\bf u})\coloneqq\left(\partial_{\partial_{x_i}u_j}f(\nabla{\bf u})\right)_{i,j=1}^N$ for any scalar function $f:\mathbb{R}^{N^2}\to\mathbb{R}$.
    \item \textbf{Diffusion term $D_{\bf u}(\operatorname{diff}(\nu,{\bf u}))\tilde{\bf u}\cdot{\bf v}$.} Apply \eqref{integration by parts for matrix 7} with $f = \nu$ and ${\bf u} = \tilde{\bf u}$ to obtain:
    \begin{align*}
        \int_\Omega \nabla\cdot(\nu\nabla\tilde{\bf u})\cdot{\bf v}{\rm d}{\bf x} = \int_\Omega \nabla\cdot(\nu\nabla{\bf v})\cdot\tilde{\bf u}{\rm d}{\bf x} + \int_\Gamma \nu(\partial_{\bf n}\tilde{\bf u}\cdot{\bf v} - \tilde{\bf u}\cdot\partial_{\bf n}{\bf v}){\rm d}\Gamma.
    \end{align*}
    Apply \eqref{integration by parts for matrix 8} with $f = 2\nu$ and ${\bf u} = \tilde{\bf u}$ to obtain:
    \begin{align*}
        \int_\Omega \nabla\cdot(2\nu\boldsymbol{\varepsilon}(\tilde{\bf u}))\cdot{\bf v}{\rm d}{\bf x} = \int_\Omega \nabla\cdot(2\nu\boldsymbol{\varepsilon}({\bf v}))\cdot\tilde{\bf u}{\rm d}{\bf x} + \int_\Gamma 2\nu(\boldsymbol{\varepsilon}_{\bf n}(\tilde{\bf u})\cdot{\bf v} - \tilde{\bf u}\cdot\boldsymbol{\varepsilon}_{\bf n}({\bf v})){\rm d}\Gamma.
    \end{align*}
    Thus, $\int_\Omega D_{\bf u}(\operatorname{diff}(\nu,{\bf u}))\tilde{\bf u}\cdot{\bf v}{\rm d}{\bf x} =$
    \begin{equation*}
         = \left\{\begin{split}
            &\int_\Omega \nabla\cdot(\nu\nabla{\bf v})\cdot\tilde{\bf u}{\rm d}{\bf x} + \int_\Gamma \nu(\partial_{\bf n}\tilde{\bf u}\cdot{\bf v} - \tilde{\bf u}\cdot\partial_{\bf n}{\bf v}){\rm d}\Gamma,&&\mbox{ if }\operatorname{diff}(\nu,{\bf u}) = \nabla\cdot(\nu\nabla{\bf u}),\\
            &\int_\Omega \nabla\cdot(2\nu\boldsymbol{\varepsilon}({\bf v}))\cdot\tilde{\bf u}{\rm d}{\bf x} + \int_\Gamma 2\nu(\boldsymbol{\varepsilon}_{\bf n}(\tilde{\bf u})\cdot{\bf v} - \tilde{\bf u}\cdot\boldsymbol{\varepsilon}_{\bf n}({\bf v})){\rm d}\Gamma,&&\mbox{ if } \operatorname{diff}(\nu,{\bf u}) = \nabla\cdot(2\nu\boldsymbol{\varepsilon}({\bf u})).
        \end{split}\right.
    \end{equation*}
    \item \textbf{Domain cost term $D_{\nabla{\bf u}}{\bf f}({\bf x},{\bf u},\nabla{\bf u},p)\nabla\tilde{\bf u}\cdot{\bf v}$.}
    \begin{align*}
        &\int_\Omega D_{\nabla{\bf u}}{\bf f}({\bf x},{\bf u},\nabla{\bf u},p)\nabla\tilde{\bf u}\cdot{\bf v}{\rm d}{\bf x} = \int_\Omega \left(\nabla_{\nabla{\bf u}}f_k({\bf x},{\bf u},\nabla{\bf u},p):\nabla\tilde{\bf u}\right)_{k=1}^N\cdot{\bf v}{\rm d}{\bf x}\\
        =&\,\int_\Omega \sum_{k=1}^N \nabla_{\nabla{\bf u}}f_k({\bf x},{\bf u},\nabla{\bf u},p):\nabla\tilde{\bf u}v_k{\rm d}{\bf x} = \int_\Omega \sum_{k=1}^N\sum_{i=1}^N\sum_{j=1}^N \partial_{\partial_{x_i}u_j}f_k({\bf x},{\bf u},\nabla{\bf u},p)\partial_{x_i}\tilde{u}_jv_k{\rm d}{\bf x}\\
        =&\,\sum_{i=1}^N\sum_{j=1}^N\sum_{k=1}^N \int_\Omega \partial_{\partial_{x_i}u_j}f_k({\bf x},{\bf u},\nabla{\bf u},p)\partial_{x_i}\tilde{u}_jv_k{\rm d}{\bf x}\\
        =&\, \sum_{i=1}^N\sum_{j=1}^N\sum_{k=1}^N -\int_\Omega \partial_{x_i}\partial_{\partial_{x_i}u_j}f_k({\bf x},{\bf u},\nabla{\bf u},p)\tilde{u}_jv_k + \partial_{\partial_{x_i}u_j}f_k({\bf x},{\bf u},\nabla{\bf u},p)\tilde{u}_j\partial_{x_i}v_k{\rm d}{\bf x}\\
        &+ \int_\Gamma n_i\partial_{\partial_{x_i}u_j}f_k({\bf x},{\bf u},\nabla{\bf u},p)\tilde{u}_jv_k{\rm d}\Gamma\\
        =&\, -\int_\Omega \sum_{j=1}^N \tilde{u}_j\sum_{k=1}^N v_k\sum_{i=1}^N \partial_{x_i}\partial_{\partial_{x_i}u_j}f_k({\bf x},{\bf u},\nabla{\bf u},p) + \sum_{j=1}^N \tilde{u}_j\sum_{k=1}^N\sum_{i=1}^N \partial_{\partial_{x_i}u_j}f_k({\bf x},{\bf u},\nabla{\bf u},p)\partial_{x_i}v_k{\rm d}{\bf x}\\
        &+ \int_\Gamma \sum_{j=1}^N \tilde{u}_j\sum_{k=1}^N v_k\sum_{i=1}^N n_i\partial_{\partial_{x_i}u_j}f_k({\bf x},{\bf u},\nabla{\bf u},p){\rm d}\Gamma\\
        =&\, - \int_\Omega \sum_{j=1}^N \tilde{u}_j\sum_{k=1}^N v_k\nabla\cdot\left(\nabla_{\nabla u_j}f_k({\bf x},{\bf u},\nabla{\bf u},p)\right) + \sum_{j=1}^N \tilde{u}_j\sum_{k=1}^N \nabla_{\nabla u_j}f_k({\bf x},{\bf u},\nabla{\bf u},p)\cdot\nabla v_k{\rm d}{\bf x}\\
        &+ \int_\Gamma \sum_{j=1}^N \tilde{u}_j\sum_{k=1}^N v_k\nabla_{\nabla u_j}f_k({\bf x},{\bf u},\nabla{\bf u},p)\cdot{\bf n}{\rm d}\Gamma\\
        =&\, -\int_\Omega \sum_{j=1}^N \tilde{u}_j\nabla\cdot\left(\nabla_{\nabla u_j}{\bf f}({\bf x},{\bf u},\nabla{\bf u},p)\right)\cdot{\bf v} + \sum_{j=1}^N \tilde{u}_j\nabla_{\nabla u_j}{\bf f}({\bf x},{\bf u},\nabla{\bf u},p):\nabla{\bf v}{\rm d}{\bf x}\\
        &+ \int_\Gamma \sum_{j=1}^N \tilde{u}_j\left(\nabla_{\nabla u_j}{\bf f}({\bf x},{\bf u},\nabla{\bf u},p)\cdot{\bf n}\right)\cdot{\bf v}{\rm d}\Gamma\\
        =&\, -\int_\Omega \left(\nabla\cdot\left(\nabla_{\nabla{\bf u}}{\bf f}({\bf x},{\bf u},\nabla{\bf u},p)\right)\cdot{\bf v}\right)\cdot\tilde{\bf u} + \left(\nabla_{\nabla{\bf u}}{\bf f}({\bf x},{\bf u},\nabla{\bf u},p):\nabla{\bf v}\right)\cdot\tilde{\bf u}{\rm d}{\bf x}\\
        &+ \int_\Gamma \left(\left(\nabla_{\nabla{\bf u}}{\bf f}({\bf x},{\bf u},\nabla{\bf u},p)\cdot{\bf n}\right)\cdot{\bf v}\right)\cdot\tilde{\bf u}{\rm d}\Gamma.
    \end{align*}
    \item \textbf{Term $-\nabla\tilde{p}\cdot{\bf v}$.} Apply \eqref{ibp} with $f = \tilde{p}$ to obtain:
    \begin{align*}
        -\int_\Omega \nabla\tilde{p}\cdot{\bf v}{\rm d}{\bf x} = \int_\Omega \tilde{p}\nabla\cdot{\bf v}{\rm d}{\bf x} - \int_\Gamma \tilde{p}{\bf v}\cdot{\bf n}{\rm d}\Gamma.
    \end{align*}
    \item\textbf{ Term $-q\nabla\cdot\tilde{\bf u}$.} Apply \eqref{ibp} with $f = q$ and ${\bf v} = \tilde{\bf u}$ to obtain:
    \begin{align*}
        -\int_\Omega q\nabla\cdot\tilde{\bf u}{\rm d}{\bf x} = \int_\Omega \nabla q\cdot\tilde{\bf u}{\rm d}{\bf x} - \int_\Gamma q\tilde{\bf u}\cdot{\bf n}{\rm d}\Gamma.
    \end{align*}
    \item \textbf{Term $qD_{\nabla{\bf u}}f_{\rm div}({\bf x},{\bf u},\nabla{\bf u},p)\nabla\tilde{\bf u}$.}
    \begin{align*}
        &\int_\Omega qD_{\nabla{\bf u}}f_{\rm div}({\bf x},{\bf u},\nabla{\bf u},p)\nabla\tilde{\bf u}{\rm d}{\bf x} = \int_\Omega q\nabla_{\nabla{\bf u}}f_{\rm div}({\bf x},{\bf u},\nabla{\bf u},p):\nabla\tilde{\bf u}{\rm d}{\bf x}\\
        =&\, \int_\Omega q\sum_{i=1}^N\sum_{j=1}^N \partial_{\partial_{x_i}u_j}f_{\rm div}({\bf x},{\bf u},\nabla{\bf u},p)\partial_{x_i}\tilde{u}_j{\rm d}{\bf x} = \sum_{i=1}^N\sum_{j=1}^N \int_\Omega q\partial_{\partial_{x_i}u_j}f_{\rm div}({\bf x},{\bf u},\nabla{\bf u},p)\partial_{x_i}\tilde{u}_j{\rm d}{\bf x}\\
        =&\, \sum_{i=1}^N\sum_{j=1}^N -\int_\Omega \partial_{x_i}q\partial_{\partial_{x_i}u_j}f_{\rm div}({\bf x},{\bf u},\nabla{\bf u},p)\tilde{u}_j + q\partial_{x_i}\partial_{\partial_{x_i}u_j}f_{\rm div}({\bf x},{\bf u},\nabla{\bf u},p)\tilde{u}_j{\rm d}{\bf x}\\
        &+ \int_\Gamma qn_i\partial_{\partial_{x_i}u_j}f_{\rm div}({\bf x},{\bf u},\nabla{\bf u},p)\tilde{u}_j{\rm d}\Gamma\\
        =&\, -\int_\Omega \sum_{i=1}^N\sum_{j=1}^N \partial_{x_i}q\partial_{\partial_{x_i}u_j}f_{\rm div}({\bf x},{\bf u},\nabla{\bf u},p)\tilde{u}_j + q\sum_{j=1}^N \tilde{u}_j\sum_{i=1}^N \partial_{x_i}\partial_{\partial_{x_i}u_j}f_{\rm div}({\bf x},{\bf u},\nabla{\bf u},p){\rm d}{\bf x}\\
        &+ \int_\Gamma q\sum_{i=1}^N\sum_{j=1}^N n_i\partial_{\partial_{x_i}u_j}f_{\rm div}({\bf x},{\bf u},\nabla{\bf u},p)\tilde{u}_j{\rm d}\Gamma\\
        =&\, -\int_\Omega \nabla^\top q\nabla_{\nabla{\bf u}}f_{\rm div}({\bf x},{\bf u},\nabla{\bf u},p)\tilde{\bf u} + q\sum_{j=1}^N \tilde{u}_j\nabla\cdot\left(\nabla_{\nabla u_j}f_{\rm div}({\bf x},{\bf u},\nabla{\bf u},p)\right){\rm d}{\bf x}\\
        &+ \int_\Gamma q{\bf n}^\top\nabla_{\nabla{\bf u}}f_{\rm div}({\bf x},{\bf u},\nabla{\bf u},p)\tilde{\bf u}{\rm d}\Gamma\\
        =&\, -\int_\Omega \nabla^\top q\nabla_{\nabla{\bf u}}f_{\rm div}({\bf x},{\bf u},\nabla{\bf u},p)\tilde{\bf u} + q\left(\nabla\cdot\left(\nabla_{\nabla{\bf u}}f_{\rm div}({\bf x},{\bf u},\nabla{\bf u},p)\right)\right)\cdot\tilde{\bf u}{\rm d}{\bf x}\\
        &+ \int_\Gamma q{\bf n}^\top\nabla_{\nabla{\bf u}}f_{\rm div}({\bf x},{\bf u},\nabla{\bf u},p)\tilde{\bf u}{\rm d}\Gamma.
    \end{align*}
\end{enumerate}

\section{Instationary Cases}
\begin{enumerate}[leftmargin=5mm]
    \item \textbf{Term $\partial_t{\bf u}$.}
    \begin{align*}
        \int_0^T\int_\Omega \partial_t{\bf u}\cdot{\bf v}{\rm d}{\bf x}{\rm d}t &= \int_\Omega\int_0^T \sum_{i=1}^N \partial_tu_iv_i{\rm d}t{\rm d}{\bf x} = \int_\Omega\sum_{i=1}^N \int_0^T \partial_tu_iv_i{\rm d}t{\rm d}{\bf x}\\
        &= \int_\Omega \left(\sum_{i=1}^N u_i(T)v_i(T) - u_i(0)v_i(0) - \int_0^T u_i\partial_tv_i{\rm d}t\right){\rm d}{\bf x}\\
        &= \int_\Omega \left({\bf u}(T)\cdot{\bf v}(T) - {\bf u}(0)\cdot{\bf v}(0) - \int_0^T \sum_{i=1}^N u_i\partial_tv_i{\rm d}t\right){\rm d}{\bf x}\\
        &= \int_\Omega {\bf u}(T)\cdot{\bf v}(T) - {\bf u}_0\cdot{\bf v}(0){\rm d}{\bf x} - \int_0^T\int_\Omega {\bf u}\cdot\partial_t{\bf v}{\rm d}{\bf x}{\rm d}t.
    \end{align*}
    \item \textbf{Term $\partial_t(\rho{\bf u})$.} Apply the previous one with ${\bf u}$ replaced by $\rho{\bf u}$ to obtain
    \begin{align*}
        \int_0^T\int_\Omega \partial_t(\rho{\bf u})\cdot{\bf v}{\rm d}{\bf x}{\rm d}t = \int_\Omega \rho(T){\bf u}(T)\cdot{\bf v}(T) - \rho(0){\bf u}(0)\cdot{\bf v}(0){\rm d}{\bf x} - \int_0^T\int_\Omega \rho{\bf u}\cdot\partial_t{\bf v}{\rm d}{\bf x}{\rm d}t.
    \end{align*}
    \item \textbf{Convective term $\nabla\cdot(\rho{\bf u}\otimes{\bf u})$.} Apply \eqref{integration by parts for matrix 1} with $A = \rho{\bf u}\otimes{\bf u}$ to obtain
    \begin{align*}
        \int_0^T\int_\Omega \nabla\cdot(\rho{\bf u}\otimes{\bf u})\cdot{\bf v}{\rm d}{\bf x}{\rm d}t = -\int_0^T\int_\Omega \rho({\bf u}\otimes{\bf u}):\nabla{\bf v}{\rm d}{\bf x}{\rm d}t + \int_0^T\int_\Gamma {\bf n}^\top\rho({\bf u}\otimes{\bf u}){\bf v}{\rm d}\Gamma{\rm d}t.
    \end{align*}
\end{enumerate}

%------------------------------------------------------------------------------%

\printbibliography[heading=bibintoc]
\end{document}