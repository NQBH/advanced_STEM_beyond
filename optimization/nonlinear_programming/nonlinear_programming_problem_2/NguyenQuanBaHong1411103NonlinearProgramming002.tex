\documentclass[a4paper]{article}
\usepackage{longtable,float,hyperref,color,amsmath,amsxtra,amssymb,latexsym,amscd,amsthm,amsfonts,graphicx}
\numberwithin{equation}{section}
\allowdisplaybreaks
\usepackage{fancyhdr}
\pagestyle{fancy}
\fancyhf{}
\fancyhead[RE,LO]{\footnotesize \textsc \leftmark}
\cfoot{\thepage}
\renewcommand{\headrulewidth}{0.5pt}
\setcounter{tocdepth}{3}
\setcounter{secnumdepth}{3}
\usepackage{imakeidx}
\makeindex[columns=2, title=Alphabetical Index, 
           options= -s index.ist]
\title{\huge Nonlinear Programming Assignment 002}
\author{\textsc{Nguyen Quan Ba Hong}\\
{\small Students at Faculty of Math and Computer Science,}\\ 
{\small Ho Chi Minh University of Science, Vietnam} \\
{\small \texttt{email. nguyenquanbahong@gmail.com}}\\
{\small \texttt{blog. \url{www.nguyenquanbahong.com}} 
\footnote{Copyright \copyright\ 2016-2018 by Nguyen Quan Ba Hong, Student at Ho Chi Minh University of Science, Vietnam. This document may be copied freely for the purposes of education and non-commercial research. Visit my site \texttt{\url{www.nguyenquanbahong.com}} to get more.}}}
\begin{document}
\maketitle
\begin{abstract}
This assignment aims at solving some selected problems for the final exam of the course \textit{Theory of Nonlinear Programming}. 
\end{abstract}
\newpage
\tableofcontents
\newpage
\section{Directional, G\^{a}teaux \& Fr\'{e}chet Differentiable Functions}
\textbf{Problem 1.} \textit{Let $f:\mathbb{R}^2\to \mathbb{R}$ be a mapping defined by}
\begin{align}
f\left( {x,y} \right) = \left\{ {\begin{array}{*{20}{c}}
{\dfrac{{x{y^2}}}{{{x^2} + {y^4}}} \mbox{ if } \left( {x,y} \right) \ne \left( {0,0} \right),}\\
{0 \hspace{1cm} \mbox{ if } \left( {x,y} \right) = \left( {0,0} \right).}
\end{array}} \right.
\end{align}
\begin{enumerate}
\item \textit{Is $f$ directional differentiable at $x_0=\left(0,0\right)$?}
\item \textit{Is $f$ G\^{a}teaux differentiable at $x_0=\left(0,0\right)$?}
\end{enumerate}
\textsc{Solution.}
\begin{enumerate}
\item Let $d\in \mathbb{R}^2$ be a vector $d=\left(d_1,d_2\right)^T$, and $x_0=\left(0,0\right)$. If $d=\left(0,0\right)$, we have $f'\left(x_0;d\right) =0$ from the definition of directional derivative. If $d\ne \left(0,0\right)$, we compute
\begin{align}
\mathop {\lim }\limits_{t \to 0} \frac{{f\left( {{x_0} + td} \right) - f\left( {{x_0}} \right)}}{t} &= \mathop {\lim }\limits_{t \to 0} \frac{{f\left( {t{d_1},t{d_2}} \right) - f\left( {0,0} \right)}}{t} \\
 &= \mathop {\lim }\limits_{t \to 0} \frac{{{d_1}d_2^2}}{{d_1^2 + {t^2}d_2^4}}. \label{1.3}
\end{align}
We consider the following two cases depending on $d_1$. If $d_1 =0$ (hence $d_2\ne 0$ due to the assumption $d\ne \left(0,0\right)$), then the term in the limit in \eqref{1.3} equals zero, so this limit also equals zero. If $d_1\ne 0$, then
\begin{align}
\mathop {\lim }\limits_{t \to 0} \frac{{{d_1}d_2^2}}{{d_1^2 + {t^2}d_2^4}} = \frac{{d_2^2}}{{{d_1}}}.
\end{align}
Combining both cases, we deduces that $f$ is directional differentiable at $x_0$ and its directional derivative is given by 
\begin{align}
\label{1.5}
f'\left( {{x_0};d} \right) = \left\{ {\begin{array}{*{20}{c}}
{0,\mbox{ if } {d_1} = 0,}\\
{\frac{{d_2^2}}{{{d_1}}}\mbox{ if } {d_1} \ne 0.}
\end{array}} \right.
\end{align}
where $d:=\left(d_1,d_2\right) \in \mathbb{R}^2$. 
\item The directional derivative $f'\left( {{x_0};d} \right)$ given by \eqref{1.5} is not linear in term of the variable $d$. Hence, $f$ is not G\^{a}teaux differentiable at $x_0$.\end{enumerate}
This completes our solution. \hfill $\square$\\
\\
\textbf{Problem 2.} \textit{Let $f:\mathbb{R}^2\to \mathbb{R}$ be a mapping defined by}
\begin{align}
f\left( {x,y} \right) = \left\{ {\begin{array}{*{20}{c}}
{\dfrac{{{x^3}y}}{{{x^4} + {y^2}}}\mbox{ if } \left( {x,y} \right) \ne \left( {0,0} \right),}\\
{0 \hspace{1cm} \mbox{ if } \left( {x,y} \right) = \left( {0,0} \right).}
\end{array}} \right.
\end{align}
\begin{enumerate}
\item \textit{Is $f$ directional differentiable at $x_0=\left(0,0\right)$?}
\item \textit{Is $f$ G\^{a}teaux differentiable at $x_0=\left(0,0\right)$?}
\item \textit{Is $f$ Fr\'{e}chet differentiable at $x_0=\left(0,0\right)$?}
\end{enumerate}
\textsc{Solution.} 
\begin{enumerate}
\item Let $d\in \mathbb{R}^2$ be a vector $d=\left(d_1,d_2\right)^T$, and $x_0=\left(0,0\right)$. If $d=\left(0,0\right)$, we have $f'\left(x_0;d\right) =0$ by the definition of directional derivative. If $d\ne \left(0,0\right)$, we compute
\begin{align}
\mathop {\lim }\limits_{t \to 0} \frac{{f\left( {{x_0} + td} \right) - f\left( {{x_0}} \right)}}{t} &= \mathop {\lim }\limits_{t \to 0} \frac{{f\left( {t{d_1},t{d_2}} \right) - f\left( {0,0} \right)}}{t}\\
 &= \mathop {\lim }\limits_{t \to 0} \frac{{td_1^3{d_2}}}{{{t^2}d_1^4 + d_2^2}}. \label{1.8}
\end{align}
We consider the following two cases depending on $d_2$. If $d_2=0$ (hence $d_1\ne 0$ due to the assumption $d\ne \left(0,0\right)$), then the term in the limit in \eqref{1.8} equals zero, so this limit also equals zero. If $d_2 \ne 0$, then
\begin{align}
\mathop {\lim }\limits_{t \to 0} \left| {\frac{{td_1^3{d_2}}}{{{t^2}d_1^4 + d_2^2}}} \right| \le \mathop {\lim }\limits_{t \to 0} \left| {\frac{{td_1^3}}{{{d_2}}}} \right| = 0,
\end{align}
i.e., the limit in \eqref{1.8} also equals zero in this case. Combining both cases, we deduce that $f$ is directional differentiable at $x_0$ and its directional derivative is given by $f'\left(x_0;d\right) =0$ for all $d\in \mathbb{R}^2$.
\item From the above result, we have
\begin{align}
f'\left( {{x_0};d} \right) = 0 = \left( {\begin{array}{*{20}{c}}
0&0
\end{array}} \right)d,\hspace{0.2cm}\forall d \in {\mathbb{R}^2},
\end{align}
which is linear in $d$. Hence, $f$ is G\^{a}teaux differentiable at $x_0=\left(0,0\right)$.
\item We claim that $f$ is not Fr\'{e}chet differentiable at $x_0=\left(0,0\right)$. To this end, we suppose for the contrary that $f$ is Fr\'{e}chet differentiable at $x_0=\left(0,0\right)$, by definition of Fr\'{e}chet differentiability, there exists a linear function $l:\mathbb{R}^2\to \mathbb{R}$, $l\left( x \right) = \left\langle {l,x} \right\rangle  = {l_1}{x_1} + {l_2}{x_2}$ such that
\begin{align}
\label{1.11}
\mathop {\lim }\limits_{\left\| h \right\| \to 0} \frac{{f\left( {{x_0} + h} \right) - f\left( {{x_0}} \right) - \left\langle {l,h} \right\rangle }}{{\left\| h \right\|}} = 0.
\end{align}
Denote $h=\left(h_1,h_2\right)^T \in \mathbb{R}^2$, then \eqref{1.11} becomes
\begin{align}
\label{1.12}
\mathop {\lim }\limits_{\left\| h \right\| \to 0} \frac{1}{{\left\| h \right\|}}\left( {\frac{{h_1^3{h_2}}}{{h_1^4 + h_2^2}} - {l_1}{h_1} - {l_2}{h_2}} \right) = 0.
\end{align}
In particular, if we take $h=\left(h_1,0\right)$ for which $h_1\ne 0$ and $h_1\to 0$, then \eqref{1.12} gives $\mathop {\lim }\limits_{{h_1} \to 0} \frac{{{l_1}{h_1}}}{{\left| {{h_1}} \right|}} = 0$, thus, $l_1=0$. Similarly, taking $h=\left(0,h_2\right)$ for which $h_2\ne 0$ and $h_2\to 0$ gives $l_2 =0$. Substituting $l_1=l_2=0$ back to \eqref{1.12} gives
\begin{align}
\label{1.13}
\mathop {\lim }\limits_{\left\| h \right\| \to 0} \frac{{h_1^3{h_2}}}{{\left( {h_1^4 + h_2^2} \right)\sqrt {h_1^2 + h_2^2} }} = 0.
\end{align}
But \eqref{1.13} is not true since, for instance, taking $h_2=h_1^2$ in \eqref{1.13}, i.e., $h = \left( {{h_1},h_1^2} \right)$, for which $h_1\ne 0$ and $h_1\to 0$, gives
\begin{align}
\label{1.14}
\mathop {\lim }\limits_{{h_1} \to 0} \frac{1}{{2\sqrt {1 + h_1^2} }} = 0, 
\end{align}
which is absurd, since the limit in the left-hand side of \eqref{1.14} is $\frac{1}{2}$. 
\end{enumerate}
This contradiction ends our proof. \hfill $\square$\\
\\
\textbf{Problem 3.} \textit{Let $f:\mathbb{R}^2\to \mathbb{R}^2$ be a mapping defined by}
\begin{align}
f\left( {x,y} \right) = \left( {{x^3},{y^2}} \right).
\end{align}
\textit{Consider $x_0=\left(0,0\right), y_0=\left(1,1\right)$.}

\textit{Does there exist any $z \in \left[ {x_0,y_0} \right] = \left\{ {tx_0 + \left( {1 - t} \right)y_0|t \in \left[ {0,1} \right]} \right\}$ such that}
\begin{align}
\label{1.16}
\left\| {f\left( {{y_0}} \right) - f\left( {{x_0}} \right)} \right\| = \nabla f\left( z \right)\left[ {{y_0} - {x_0}} \right].
\end{align}
\textsc{Solution.} We compute 
\begin{align}
\nabla f\left( {x,y} \right) &= \left( {3{x^2},2y} \right),\hspace{0.2cm}\forall \left( {x,y} \right) \in {\mathbb{R}^2},\\
\left[ {{x_0},{y_0}} \right] &= \left\{ {t{x_0} + \left( {1 - t} \right){y_0}|t \in \left[ {0,1} \right]} \right\}\\
 &= \left\{ {\left( {1 - t,1 - t} \right)|t \in \left[ {0,1} \right]} \right\}\\
 &= \left\{ {\left( {t,t} \right)|t \in \left[ {0,1} \right]} \right\},\\
\left\| {f\left( {{y_0}} \right) - f\left( {{x_0}} \right)} \right\| &= \left\| {\left( {1,1} \right) - \left( {0,0} \right)} \right\| = \sqrt 2 ,
\end{align}
Hence, putting $z=\left(t,t\right)$ for $t\in \left[0,1\right]$, \eqref{1.16} is equivalent to the following quadratic equation
\begin{align}
3{t^2} + 2t = \sqrt 2 ,
\end{align}
which has a root $t_0 = \frac{1}{3}\left( {\sqrt {1 + 3\sqrt 2 }  - 1} \right) \in \left[ {0,1} \right]$. Hence, $z_0=\left(t_0,t_0\right)$ satisfies the requirement. \hfill $\square$
\section{Tangent Cones \& Asymptotic Contingent Cones}
\textbf{Definition 2.1 (Contingent set of first and second orders).} Let $X$ be a normed space, $M\subset X$ and $x_0\in X$.
\begin{enumerate}
\item The \textit{contingent cone} (or, \textit{tangent cone, Bouligand cone}) of $M$ at $x_0$ is determined by
\begin{align}
\label{2.1}
T\left( {M,{x_0}} \right) = \left\{ {u \in X|\exists {t_n} \to {0^ + },\exists {u_n} \to u,{x_0} + {t_n}{u_n} \in M,\hspace{0.2cm}\forall n \in \mathbb{N}} \right\}. 
\end{align}
\item The \textit{second-order contingent set} of $M$ at $x_0$ in the direction $u$ is determined by
\begin{align}
\label{2.2}
{T^2}\left( {M,{x_0},u} \right) = \left\{ \begin{array}{l}
w \in X|\exists {t_n} \to {0^ + },\exists {w_n} \to w,\\
{x_0} + {t_n}u + \frac{1}{2}t_n^2{w_n} \in M,\hspace{0.2cm}  \forall n \in \mathbb{N}
\end{array} \right\}.
\end{align} 
\item The \textit{asymptotic contingent cone of second order} of $M$ at $x_0$ in the direction $u$ is determined by
\begin{align}
\label{2.3}
T''\left( {M,{x_0},u} \right) = \left\{ \begin{array}{l}
w \in X|\exists \left( {{t_n},{r_n}} \right) \to \left( {{0^ + }{{,0}^ + }} \right):\frac{{{t_n}}}{{{r_n}}} \to 0,\\
\exists {w_n} \to w,{x_0} + {t_n}u + \frac{1}{2}{t_n}{r_n}{w_n} \in M, \hspace{0.2cm} \forall n \in \mathbb{N}
\end{array} \right\}.
\end{align}
\end{enumerate}
\textbf{Problem 4.} \textit{Let}
\begin{align}
M = \left\{ {\left( {{x_1},{x_2}} \right) \in {\mathbb{R}^2}|{x_1} + x_2^2 \le 0} \right\},
\end{align}
\textit{and $x_0=\left(0,0\right)$.}
\begin{enumerate}
\item \textit{Compute $T\left(M,x_0\right)$.}
\item \textit{Consider $u=\left(0,1\right)$, compute ${T^2}\left( {M,{x_0},u} \right)$ and $T''\left( {M,{x_0},u} \right)$.}
\end{enumerate}
\textsc{Solution.} 
\begin{enumerate}
\item Setting $X=\mathbb{R}^2$, we notice that $x_0=\left(0,0\right)\in M$. We claim that
\begin{align}
\label{2.5}
T\left( {M,{x_0}} \right) = \widehat T\left( {M,{x_0}} \right): = \left\{ {\left( {x,y} \right) \in {\mathbb{R}^2}|x \le 0} \right\}.
\end{align}
To prove \eqref{2.5}, we prove the following inclusions.
\begin{enumerate}
\item \textit{Prove $T\left( {M,{x_0}} \right) \subset \widehat T\left( {M,{x_0}} \right)$.} Taking $u=\left(x,y\right) \in T\left(M,x_0\right)$, by \eqref{2.1}, there exist a sequence of positive reals $\left\{ {{t_n}} \right\}_{n = 1}^\infty $ such that $t_n\to 0^+$ and a sequence $\left\{ {{u_n}} \right\}_{n = 1}^\infty  \subset {\mathbb{R}^2}$ such that $u_n\to u$ as $n\to \infty$ and $x_0+t_nu_n\in M$ for all $n\in \mathbb{N}$. Set $u_n:=\left(x_n,y_n\right)$, the fact $u_n\to u$ implies that $x_n\to x$ and $y_n\to y$, and the fact $x_0+t_n u_n\in M$ for all $n\in \mathbb{N}$ gives
\begin{align}
\label{2.6}
{t_n}{x_n} + t_n^2y_n^2 \le 0,\hspace{0.2cm}\forall n \in \mathbb{N}.
\end{align}
Since $t_n>0$ for all $n\in \mathbb{N}$, \eqref{2.6} then implies
\begin{align}
\label{2.7}
{x_n} + {t_n}y_n^2 \le 0,\hspace{0.2cm}\forall n \in \mathbb{N}.
\end{align}
We see at a glance from \eqref{2.7} that $x_n\le 0$ for all $n\in \mathbb{N}$. Hence $x\le 0$ (since $x_n\to x$ as $n\to \infty$). Now let $n\to \infty$ in \eqref{2.7} and use the given limits $x_n\to x,y_n\to y$ and $t_n\to 0^+$, we obtain $x\le 0$ as just mentioned. Hence, $u\in \widehat T \left(M,x_0\right)$ and our first inclusion is proved.
\item \textit{Prove $\widehat{T} \left( {M,{x_0}} \right) \subset T\left( {M,{x_0}} \right)$.} Taking $u=\left(x,y\right) \in \mathbb{R}^2$ satisfying $x\le 0$, we claim that $u\in T\left(M,x_0\right)$. To this end, we now choose $x_n=x-\frac{1}{n}<0, y_n=y$ for all $n\in \mathbb{N}$. This choice ensures that $u_n:=\left(
x_n,y_n\right) \to u:=\left(x,y\right)$ as $n\to \infty$. It then suffices to prove that there exists a sequence of positive reals $\left\{ {{t_n}} \right\}_{n = 1}^\infty $ such that $t_n\to 0^+$ and $x_0+t_nu_n\in M$ for all $n\in \mathbb{N}$. The latter gives, using \eqref{2.7} again, 
\begin{align}
\label{2.8}
x - \frac{1}{n} + {t_n}{y^2} \le 0,\hspace{0.2cm}\forall n \in \mathbb{N}.
\end{align}
If $y=0$, \eqref{2.8} holds obviously for all $t_n>0$, thus, we can choose an arbitrary sequence $t_n$'s of positive reals satisfying $t_n\to 0^+$. If $y\ne 0$, \eqref{2.8} is equivalent to
\begin{align}
\label{2.9}
{t_n} \le \frac{{\frac{1}{n} - x}}{{{y^2}}}, \hspace{0.2cm}\forall n \in \mathbb{N}.
\end{align}
The term in the right-hand side of \eqref{2.9} is positive for all $n\in \mathbb{N}$. Hence we can  choose $t_n$'s satisfying \eqref{2.9} and $t_n\to 0^+$ as $n\to \infty$. This choice implies that $u\in T\left(M,x_0\right)$, i.e., the second inclusion is also proved.
\end{enumerate}
Combining these, we conclude that \eqref{2.5} holds, i.e.,
\begin{align}
T\left( {M,{x_0}} \right) = \left\{ {\left( {x,y} \right) \in {\mathbb{R}^2}|x \le 0} \right\}.
\end{align}
\item \textbf{Compute ${T^2}\left( {M,{x_0},u} \right)$.} First, we claim that
\begin{align}
\label{2.11}
{T^2}\left( {M,{x_0},u} \right) = {\widehat T^2}\left( {M,{x_0},u} \right): = \left\{ {\left( {x,y} \right) \in {\mathbb{R}^2}|x \le  - 2} \right\}.
\end{align}
To prove \eqref{2.11}, we prove the following inclusions.
\begin{enumerate}
\item \textit{Prove ${T^2}\left( {M,{x_0},u} \right) \subset {\widehat T^2}\left( {M,{x_0},u} \right)$.} Taking $w=\left(x,y\right) \in T^2 \left(M,x_0,u\right)$, by \eqref{2.2}, there exist a sequence of positive reals $\left\{ {{t_n}} \right\}_{n = 1}^\infty $ such that $t_n\to 0^+$ and a sequence $\left\{ {{w_n}} \right\}_{n = 1}^\infty  \subset {\mathbb{R}^2}$ such that $w_n\to w$ as $n\to \infty$ and ${x_0} + {t_n}u + \frac{1}{2}t_n^2{w_n} \in M$ for all $n\in \mathbb{N}$. Set $w_n:=\left(x_n,y_n\right)$, the fact $w_n\to w$ implies that $x_n\to x$ and $y_n\to y$, and the fact ${x_0} + {t_n}u + \frac{1}{2}t_n^2{w_n} \in M$ for all $n\in \mathbb{N}$ gives
\begin{align}
\label{2.12}
\frac{1}{2}t_n^2{x_n} + {\left( {{t_n} + \frac{1}{2}t_n^2{y_n}} \right)^2} \le 0,\hspace{0.2cm}\forall n \in \mathbb{N}.
\end{align}
Since $t_n>0$ for all $n\in \mathbb{N}$, \eqref{2.12} implies that
\begin{align}
\label{2.13}
\frac{{{x_n}}}{2} + 1 + {t_n}{y_n} + \frac{1}{4}t_n^2y_n^2 \le 0,\hspace{0.2cm} \forall n \in \mathbb{N}.
\end{align}
Now let $n\to \infty$ in \eqref{2.13} and use the given limits $x_n\to x,y_n\to y$ and $t_n\to 0^+$, we obtain $x\le -2$. Hence, $w\in \widehat{T}\left(M,x_0,u\right)$ and our first inclusion is proved.
\item \textit{Prove ${\widehat T^2}\left( {M,{x_0},u} \right) \subset {T^2}\left( {M,{x_0},u} \right)$.}\\
\\
\textsc{Proof 1.} Taking $w=\left(x,y\right)$ satisfying $x\le -2$, we claim that $w\in T^2\left(M,x_0,u\right)$. To this end, we now choose $x_n:=x-\frac{1}{n}\le -2 -\frac{1}{n}$ and $y_n:=y$ for all $n\in \mathbb{N}$. This choice ensures that $w_n:=\left(x_n,y_n\right) \to w:=\left(x,y\right)$ as $n\to \infty$. It then suffices to prove that there exists a sequence of positive reals $\left\{ {{t_n}} \right\}_{n = 1}^\infty $ such that $t_n\to 0^+$ and ${x_0} + {t_n}u + \frac{1}{2}t_n^2{w_n} \in M$ for all $n\in \mathbb{N}$. The latter is equivalent to, using \eqref{2.13} again,
\begin{align}
\label{2.14}
t_n^2{y^2} + 4{t_n}y + 2\left( {x - \frac{1}{n}} \right) + 4 \le 0,\hspace{0.2cm}\forall n \in \mathbb{N}.
\end{align}
We consider the following cases depending on $y$. If $y=0$, then \eqref{2.14} obviously holds for all sequence $t_n$'s since $x\le -2$. If $y\ne 0$, consider the left-hand side of \eqref{2.14} as a quadratic equation in $t_n$, its discriminant is given by
\begin{align}
\Delta ' &= 4{y^2} - {y^2}\left[ {2\left( {x - \frac{1}{n}} \right) + 4} \right]\\
 &= 2\left( {\frac{1}{n} - x} \right){y^2} \ge 0.
\end{align}
Thus, its two roots are given by
\begin{align}
{t_{n,1}} &= \frac{{ - 2y - \left| y \right|\sqrt {2\left( {\frac{1}{n} - x} \right)} }}{{{y^2}}},\\
{t_{n,2}} &= \frac{{ - 2y + \left| y \right|\sqrt {2\left( {\frac{1}{n} - x} \right)} }}{{{y^2}}}.
\end{align}
Since $x\le -2$, it is easy to verify that $t_{n,1}<0<t_{n,2}$. If we choose a sequence $t_n$'s such that $t_n\to 0^+$ and $0 < {t_n} \le {t_{n,2}}$ then ${x_0} + {t_n}u + \frac{1}{2}t_n^2{w_n} \in M$  for all $n\in \mathbb{N}$. Hence, $w\in T^2\left(M,x_0,u\right)$ and the second inclusion is also proved.\\
\\
\textsc{Proof 2.} Use the same settings as Proof 1, we arrive at the inequality \eqref{2.14}. We now define, for each $n\in \mathbb{N}$, the function 
\begin{align}
F_n\left( t \right): = {y^2}{t^2} + 4yt + 2\left( {x - \frac{1}{n}} \right) + 4,\hspace{0.2cm} t > 0.
\end{align}
It is obvious to check $F\left(t\right)$ is continuous, and 
\begin{align}
{F_n}\left( 0 \right) = 2\left( {x - \frac{1}{n}} \right) + 4 = 2\left( {x + 2} \right) - \frac{2}{n} < 0.
\end{align}
Thus, by continuity of $F_n$, we can choose $t_n>0$ small enough such that \eqref{2.14} holds. And the chosen sequence $t_n$'s indicates that $w\in T^2\left(M,x_0,u\right)$, i.e., the second inclusion is proved.
\end{enumerate}
Combining these inclusions, we conclude that \eqref{2.11} holds, i.e.,
\begin{align}
\label{2.21}
{T^2}\left( {M,{x_0},u} \right) = \left\{ {\left( {x,y} \right) \in {\mathbb{R}^2}|x \le  - 2} \right\}.
\end{align}
\textbf{Compute $T''\left( {M,{x_0},u} \right)$.} We claim that
\begin{align}
\label{2.22}
T''\left( {M,{x_0},u} \right) = \widehat T''\left( {M,{x_0},u} \right): = \left\{ {\left( {x,y} \right) \in {\mathbb{R}^2}|x \le 0} \right\}.
\end{align}
To prove \eqref{2.22}, we also prove the following two inclusions as before.
\begin{enumerate}
\item \textit{Prove $T''\left( {M,{x_0},u} \right) \subset \widehat T''\left( {M,{x_0},u} \right)$.} Taking $w=\left(x,y\right)\in T''\left(M,x_0,u\right)$, by \eqref{2.3}, there exist two sequences of positive reals $\left\{ {{t_n}} \right\}_{n = 1}^\infty $ and $\left\{ {{r_n}} \right\}_{n = 1}^\infty $ such that $t_n\to 0^+,r_n\to 0^+$ and $\frac{{{t_n}}}{{{r_n}}} \to 0$ as $n\to \infty$ and a sequence $\left\{ {{w_n}} \right\}_{n = 1}^\infty  \subset {\mathbb{R}^2}$ such that $w_n\to w$ and ${x_0} + {t_n}u + \frac{1}{2}{t_n}{r_n}{w_n} \in M$ for all $n\in \mathbb{N}$. Set $w_n:=\left(x_n,y_n\right)$, the fact that $w_n\to w$ implies that $x_n\to x$ and $y_n\to y$ as $n\to \infty$, and the fact ${x_0} + {t_n}u + \frac{1}{2}{t_n}{r_n}{w_n} \in M$ for all $n\in \mathbb{N}$ gives
\begin{align}
\label{2.23}
\frac{1}{2}{t_n}{r_n}{x_n} + {\left( {{t_n} + \frac{1}{2}{t_n}{r_n}{y_n}} \right)^2} \le 0,\hspace{0.2cm}\forall n \in \mathbb{N}.
\end{align}
Since $t_n,r_n>0$ for all $n\in \mathbb{N}$, \eqref{2.23} implies that
\begin{align}
\label{2.24}
\frac{{{x_n}}}{2} + \frac{{{t_n}}}{{{r_n}}} + {t_n}{y_n} + \frac{1}{4}{t_n}{r_n}y_n^2 \le 0,\hspace{0.2cm}\forall n \in \mathbb{N}.
\end{align}
Now let $n\to \infty$ in \eqref{2.24} and use the given limits $x_n\to x, y_n\to y, t_n\to 0^+,r_n\to 0^+$ and $\frac{t_n}{r_n}\to 0^+$ as $n\to \infty$, we obtain $x\le 0$. Hence, $w\in \widehat{T}''\left(M,x_0,u\right)$ and our first inclusion is proved.
\item \textit{Prove $\widehat T''\left( {M,{x_0},u} \right) \subset T''\left( {M,{x_0},u} \right)$.}\\
\\
\textsc{Proof 1.} Taking $w=\left(x,y\right)$ satisfying $x\le 0$, we claim that $w\in T''\left(M,x_0,u\right)$. To this end, we choose $x_n:=x-\frac{1}{n} <0, y_n=y$ and $t_n\le \frac{1}{n^2}$ and $r_n=2nt_n$ for all $n\in \mathbb{N}$, where $t_n$ will be constrained more strictly as follows. This choice ensures that $w_n:=\left(x_n,y_n\right)\to w:=\left(x,y\right)$, $t_n\to 0^+$, $r_n\to 0^+$ and $\frac{t_n}{r_n}\to 0$ as $n\to \infty$. It then suffices to prove that there exists a sequence of positive reals $\left\{ {{t_n}} \right\}_{n = 1}^\infty $ such that $t_n\le \frac{1}{n^2}$ (in order for $r_n\to 0^+$) and ${x_0} + {t_n}u + \frac{1}{2}{t_n}{r_n}{w_n} \in M$ for all $n\in \mathbb{N}$. The latter is equivalent to, using \eqref{2.24} again, 
\begin{align}
\frac{1}{2}\left( {x - \frac{1}{n}} \right) + \frac{1}{{2n}} + {t_n}y + \frac{n}{2}t_n^2{y^2} \le 0,\hspace{0.2cm}\forall n \in \mathbb{N},
\end{align}
i.e.,
\begin{align}
\label{2.26}
n{y^2}t_n^2 + 2y{t_n} + x \le 0,\hspace{0.2cm} \forall n \in \mathbb{N}.
\end{align}
We consider the following cases depending on $y$. If $y=0$, then \eqref{2.26} obviously holds for all sequences $t_n$'s since $x\le 0$. Thus we can choose a sequence of positive reals $t_n$'s satisfying $t_n\le \frac{1}{n^2}$ for all $n\in \mathbb{N}$ arbitrarily in this case. If $y\ne 0$, consider the left-hand side of \eqref{2.26} as a quadratic equation in $t_n$, its discriminant is given by
\begin{align}
\Delta ' = \left( {1 - nx} \right){y^2} \ge 0.
\end{align}
Thus, its two roots are given by
\begin{align}
{t_{n,1}} &= \frac{{ - y - \left| y \right|\sqrt {1 - nx} }}{{n{y^2}}},\\
{t_{n,2}} &= \frac{{ - y + \left| y \right|\sqrt {1 - nx} }}{{n{y^2}}}.
\end{align}
Since $x\le 0$, it is easy to verify that $t_{n,1}<0<t_{n,2}$. If we choose a sequence $t_n$'s such that 
\begin{align}
0 < {t_n} \le \min \left\{ {\frac{1}{{{n^2}}},{t_{n,2}}} \right\},\hspace{0.2cm}\forall n \in \mathbb{N},
\end{align}
then ${x_0} + {t_n}u + \frac{1}{2}{t_n}{r_n}{w_n} \in M$ for all $n\in \mathbb{N}$. Hence, $w\in T''\left(M,x_0,u\right)$ and the second inclusion is also proved.\\
\\
\textsc{Proof 2.} We choose $x_n:=x-\frac{1}{n},y_n:=y,t_n=r_n^2$ for all $n\in \mathbb{N}$, where $r_n$'s is a sequence of positive reals such that $r_n\to 0^+$ as $n\to \infty$. This choice ensures that $w_n:=\left(x_n,y_n\right)\to w:=\left(x,y\right)$, $t_n\to 0^+,r_n\to 0^+$ and $\frac{t_n}{r_n}=r_n\to 0^+$ as $n\to \infty$. With these settings, \eqref{2.24} becomes
\begin{align}
\label{2.31}
\frac{1}{2}\left( {x - \frac{1}{n}} \right) + {r_n}{\left( {1 + \frac{1}{2}{r_n}y} \right)^2} \le 0,\hspace{0.2cm}\forall n \in \mathbb{N}.
\end{align}
Define 
\begin{align}
F_n\left( r \right) = \frac{1}{2}\left( {x - \frac{1}{n}} \right) + r{\left( {1 + \frac{1}{2}ry} \right)^2},\hspace{0.2cm} r > 0.
\end{align}
It is obvious that $F_n\left(r\right)$ is continuous and $F\left( 0 \right) = \frac{1}{2}\left( {x - \frac{1}{n}} \right) < 0$ since $x\le 0$. By continuity of $F$, we can choose $r_n>0$ small enough such that \eqref{2.31} holds. And the chosen sequence $r_n$'s indicates that $w\in T''\left(M,x_0,u\right)$, i.e., the second inclusion is also proved. 
\end{enumerate}
Combining these inclusions, we conclude that \eqref{2.22} holds, i.e.,
\begin{align}
T''\left( {M,{x_0},u} \right) = \left\{ {\left( {x,y} \right) \in {\mathbb{R}^2}|x \le 0} \right\}.
\end{align}
\end{enumerate}
This completes our solution. \hfill $\square$\\
\\
\textbf{Problem 5.} \textit{Let}
\begin{align}
M = \left\{ {\left( {{x_1},{x_2}} \right) \in {\mathbb{R}^2}|x_1^3 - x_2^2 = 0} \right\},
\end{align}
\textit{and $x_0=\left(0,0\right)$.}
\begin{enumerate}
\item \textit{Compute $T\left(M,x_0\right)$.}
\item \textit{Consider $u=\left(1,0\right)$, compute ${T^2}\left( {M,{x_0},u} \right)$ and $T''\left( {M,{x_0},u} \right)$.}
\end{enumerate}
\textsc{Solution.} Setting $X=\mathbb{R}^2$, we notice that $x_0=\left(0,0\right) \in M$.
\begin{enumerate}
\item We claim that
\begin{align}
\label{2.35}
T\left( {M,{x_0}} \right) = \widehat T\left( {M,{x_0}} \right): = \left\{ {\left( {x,0} \right) \in {\mathbb{R}^2}|x \ge 0} \right\}.
\end{align}
To prove \eqref{2.35}, we prove the following inclusions.
\begin{enumerate}
\item \textit{Prove $T\left( {M,{x_0}} \right) \subset \widehat T\left( {M,{x_0}} \right)$.} Taking $u=\left(x,y\right) \in T\left(M,x_0\right)$, by \eqref{2.1}, there exist a sequence of positive reals $\left\{ {{t_n}} \right\}_{n = 1}^\infty $ such that $t_n\to 0^+$ and a sequence $\left\{ {{u_n}} \right\}_{n = 1}^\infty  \subset {\mathbb{R}^2}$ such that $u_n\to u$ as $n\to \infty$ and $x_0+t_nu_n\in M$ for all $n\in \mathbb{N}$. Set $u_n:=\left(x_n,y_n\right)$, the fact that $u_n\to u$ implies that $x_n\to x$ and $y_n\to y$, and the fact $x_0+t_nu_n\in M$ for all $n\in \mathbb{N}$ gives
\begin{align}
\label{2.36}
t_n^3x_n^3 - t_n^2y_n^2 = 0,\hspace{0.2cm}\forall n \in \mathbb{N}.
\end{align}
Since $t_n>0$ for all $n\in \mathbb{N}$, \eqref{2.36} then implies 
\begin{align}
\label{2.37}
{t_n}x_n^3 = y_n^2 ,\hspace{0.2cm}\forall n \in \mathbb{N}.
\end{align}
We see at a glance from \eqref{2.37} that $x_n\ge 0$ for all $n\in \mathbb{N}$. Hence, $x\ge 0$ (since $x_n\to x$ as $n\to \infty$). Now let $n\to \infty$ in \eqref{2.37} and use the given limits $x_n\to x,y_n\to y$ and $t_n\to 0^+$, we obtain $y=0$. Hence, $u\in \widehat{T}\left(M,x_0\right)$ and our first inclusion is proved.
\item \textit{Prove $\widehat T\left( {M,{x_0}} \right) \subset T\left( {M,{x_0}} \right)$.} Taking $u=\left(x,0\right)$ for which $x\ge 0$, we claim that $u\in T\left(M,x_0\right)$. To this end, we now choose $x_n:=x+\frac{1}{n}>0, y_n=\frac{1}{n^2}$ for all $n\in \mathbb{N}$. This choice ensures that $u_n:=\left(x_n,y_n\right) \to u:=\left(x,0\right)$ as $n\to \infty$. It then suffices to prove that there exists a sequence of positive reals $\left\{ {{t_n}} \right\}_{n = 1}^\infty $ such that $t_n\to 0^+$ and $x_0+t_nu_n\in M$ for all $n\in \mathbb{N}$. The latter gives, using \eqref{2.37} again,
\begin{align}
{t_n}{\left( {x + \frac{1}{n}} \right)^3} = \frac{1}{{{n^4}}},\hspace{0.2cm}\forall n \in \mathbb{N}.
\end{align}
i.e.,
\begin{align}
{t_n} = \frac{1}{{{n^4}{{\left( {x + \frac{1}{n}} \right)}^3}}},\hspace{0.2cm}\forall n \in \mathbb{N}.
\end{align}
It is easy to check that $t_n>0$ (since $x\ge 0$) and $t_n\to 0^+$ as $n\to \infty$.\footnote{If $x=0$, then $t_n =\frac{1}{n}\to 0$ as $n\to \infty$. If $x<0$, then ${t_n} \to \frac{1}{{{x^3}}}\mathop {\lim }\limits_{n \to \infty } \frac{1}{{{n^4}}} = 0$ as $n\to \infty$.} Hence, $u\in T\left(M,x_0\right)$ and the second inclusion is also proved.
\end{enumerate}
Combining these inclusions, we conclude that \eqref{2.35} holds, i.e.,
\begin{align}
T\left( {M,{x_0}} \right) = \left\{ {\left( {x,0} \right) \in {\mathbb{R}^2}|x \ge 0} \right\}.
\end{align}
\item \textbf{Compute ${T^2}\left( {M,{x_0},u} \right)$.} First, we claim that
\begin{align}
\label{2.41}
{T^2}\left( {M,{x_0},u} \right) = \emptyset . 
\end{align}
Indeed, suppose for the contrary that there exists $w=\left(x,y\right) \in T^2\left(M,x_0,u\right)$, by \eqref{2.2}, there exist a sequence of positive reals $\left\{ {{t_n}} \right\}_{n = 1}^\infty $ such that $t_n\to 0^+$ and a sequence $\left\{ {{w_n}} \right\}_{n = 1}^\infty  \subset {\mathbb{R}^2}$ such that $w_n\to w$ as $n\to \infty$ and ${x_0} + {t_n}u + \frac{1}{2}t_n^2{w_n} \in M$ for all $n\in \mathbb{N}$. Set $w_n:=\left(x_n,y_n\right)$, the fact $w_n\to w$ implies that $x_n\to x$ and $y_n\to y$, and the fact ${x_0} + {t_n}u + \frac{1}{2}t_n^2{w_n} \in M$ for all $n\in \mathbb{N}$ gives
\begin{align}
\label{2.42}
{\left( {{t_n} + \frac{1}{2}t_n^2{x_n}} \right)^3} = \frac{1}{4}t_n^4y_n^2,\forall n \in \mathbb{N}.
\end{align}
Since $t_n>0$ for all $n\in \mathbb{N}$, \eqref{2.42} implies
\begin{align}
\label{2.43}
{\left( {1 + \frac{1}{2}{t_n}{x_n}} \right)^3} = \frac{1}{4}{t_n}y_n^2,\hspace{0.2cm}\forall n \in \mathbb{N}.
\end{align}
Now let $n\to \infty$ in \eqref{2.43} and use the given limits $x_n\to x,y_n\to y$ and $t_n\to 0^+$, we obtain $1=0$, which is absurd. This contradiction implies that \eqref{2.41} is true.\\
\\
\textbf{Compute $T''\left( {M,{x_0},u} \right)$.} We claim that $T''\left( {M,{x_0},u} \right) = \mathbb{R}^2$. To prove this, taking $w=\left(x,y\right)\in \mathbb{R}^2$ arbitrarily, we claim that $w\in T''\left(M,x_0,u\right)$. The event $x_0+t_nu+\frac{1}{2}t_nr_nw_n\in M$ for all $n\in \mathbb{N}$ is equivalent to the following equality
\begin{align}
\label{2.44}
{\left( {1 + \frac{1}{2}{r_n}{x_n}} \right)^3} = \frac{{r_n^2y_n^2}}{{4{t_n}}},\hspace{0.2cm}\forall n \in \mathbb{N}.
\end{align}
We now consider the following cases depending on $y$.
\begin{enumerate}
\item \textit{Case $y\ne 0$.} In this case, we choose $x_n:=x$  and $t_n,r_n$ for which
\begin{align}
\frac{{r_n^2}}{{{t_n}}} = \frac{4}{{{y^2}}},\hspace{0.2cm}\forall n \in \mathbb{N},
\end{align}
i.e., ${t_n} = \frac{{r_n^2{y^2}}}{4}$ for all $n\in \mathbb{N}$. This choice ensures that $x_n\to x,t_n\to 0^+$ and $\frac{t_n}{r_n}=\frac{r_n y^2}{4}\to 0^+$ as $n\to \infty$ provided $r_n\to 0^+$. If now suffices to choose $r_n$'s and $y_n$'s in order that $r_n\to 0^+,y_n\to y$ as $n\to \infty$ and \eqref{2.44} holds. We consider the following cases depending on $x$. 
\begin{itemize}
\item \textit{Case $x= 0$.} We choose $y_n:=y$ for all $n\in \mathbb{N}$ and an arbitrarily sequence of positive reals $r_n$'s such that $r_n\to 0^+$ as $n\to \infty$. With this setting, it is easy to verify that \eqref{2.44} holds (both sides of this equality equal $1$) and other conditions for $T''\left(M,x_0,u\right)$ meet. Hence, $w\in T''\left(M,x_0,u\right)$ in this case.
\item \textit{Case $x\ne 0$.} In this case, \eqref{2.44} becomes
\begin{align}
{\left( {1 + \frac{1}{2}{r_n}x} \right)^3} = \frac{{y_n^2}}{{{y^2}}},\hspace{0.2cm}\forall n \in \mathbb{N}.
\end{align}
i.e.,
\begin{align}
{r_n} = \frac{2}{x}\left( {\sqrt[3]{{\frac{{y_n^2}}{{{y^2}}}}} - 1} \right),\hspace{0.2cm}\forall n \in \mathbb{N}.
\end{align}
If $x>0$, we choose 
\begin{align}
\label{2.48}
{y_n} = y + \frac{{\mbox{sign}\left( y \right)}}{n},\hspace{0.2cm}\forall n \in \mathbb{N},
\end{align}
to ensure that $y_n\to y$ and
\begin{align}
{r_n} = \frac{2}{x}\left( {\sqrt[3]{{\frac{{{{\left( {\left| y \right| + \frac{1}{n}} \right)}^2}}}{{{y^2}}}}} - 1} \right) \to {0^ + },\hspace{0.2cm}\forall n \in \mathbb{N}.
\end{align}
Similarly, if $x<0$, the choice 
\begin{align}
\label{2.50}
{y_n} = y - \frac{{\mbox{sign}\left( y \right)}}{n},\forall n \in \mathbb{N},
\end{align}
ensures that $y_n\to y$ and
\begin{align}
{r_n} = \frac{2}{x}\left( {\sqrt[3]{{\frac{{{{\left( {\left| y \right| - \frac{1}{n}} \right)}^2}}}{{{y^2}}}}} - 1} \right) \to {0^ + },\hspace{0.2cm}\forall n \in \mathbb{N}.
\end{align}
We can write both \eqref{2.48} and \eqref{2.50} in a more compact form
\begin{align}
{y_n} = y + \frac{{\mbox{sign}\left( {xy} \right)}}{n},\hspace{0.2cm}\forall n \in \mathbb{N},
\end{align}
to ensure that $y_n\to y$ and $r_n \to 0^+$ as $n\to \infty$. Hence, we also have $w\in T''\left(M,x_0,u\right)$ in this case.
\end{itemize}
\item \textit{Case $y=0$.} In this case, we choose $x_n:=x, y_n:=\frac{1}{n}$ for all $n\in \mathbb{N}$, \eqref{2.44} then becomes
\begin{align}
\label{2.53}
{\left( {1 + \frac{1}{2}{r_n}x} \right)^3} = \frac{{r_n^2}}{{4{n^2}{t_n}}},\hspace{0.2cm}\forall n \in \mathbb{N}.
\end{align}
We now consider the following cases depending on $x$.
\begin{itemize}
\item \textit{Case $x=0$.} Choose $r_n$'s as an arbitrary sequence of positive reals satisfying $r_n\to 0^+$ as $n\to \infty$, and then choose $t_n$ as follows, 
\begin{align}
{t_n} = \frac{{r_n^2}}{{4{n^2}}},\hspace{0.2cm}\forall n \in \mathbb{N}.
\end{align}
This choice ensures that \eqref{2.53} holds and $t_n\to 0^+$ and $\frac{{{t_n}}}{{{r_n}}} = \frac{{{r_n}}}{{4{n^2}}} \to 0^+$ as $n\to \infty$. Thus, $w\in T''\left(M,x_0,u\right)$ in this case.
\item \textit{Case $x>0$.} We choose $t_n$ such that
\begin{align}
\frac{{r_n^2}}{{4{n^2}{t_n}}} = 1 + \frac{1}{n},\hspace{0.2cm}\forall n \in \mathbb{N},
\end{align}
i.e.,
\begin{align}
{t_n} = \frac{{r_n^2}}{{4n\left( {n + 1} \right)}},\hspace{0.2cm}\forall n \in \mathbb{N}.
\end{align}
Provided $r_n\to 0^+$ as $n\to \infty$, this choice of $t_n$'s ensures that $t_n\to 0^+$ and $\frac{{{t_n}}}{{{r_n}}} = \frac{{{r_n}}}{{4n\left( {n + 1} \right)}} \to {0^ + }$ as $n\to \infty$. Substituting the chosen $t_n$'s into \eqref{2.53} yields
\begin{align}
{\left( {1 + \frac{1}{2}{r_n}x} \right)^3} = 1 + \frac{1}{n},\hspace{0.2cm}\forall n \in \mathbb{N},
\end{align}
i.e.,
\begin{align}
{r_n} = \frac{2}{x}\left( {\sqrt[3]{{1 + \frac{1}{n}}} - 1} \right),\hspace{0.2cm}\forall n \in \mathbb{N}. 
\end{align}
It is easy to verify that $r_n\to 0^+$ as $n\to \infty$. Thus, $w\in T''\left(M,x_0,u\right)$ in this case.
\item \textit{Case $x<0$.} This case can be handled similarly with some modifications. We choose $t_n$ such that
\begin{align}
\frac{{r_n^2}}{{4{n^2}{t_n}}} = 1 - \frac{1}{n},\hspace{0.2cm}\forall n \in \mathbb{N}.
\end{align}
i.e.,
\begin{align}
{t_n} = \frac{{r_n^2}}{{4n\left( {n - 1} \right)}},\hspace{0.2cm}\forall n \in \mathbb{N}:n>1.
\end{align}
($t_1>0$ is arbitrary) Provided $r_n\to 0^+$ as $n\to \infty$, this choice of $t_n$'s ensures that $t_n\to 0^+$ and $\frac{{{t_n}}}{{{r_n}}} = \frac{{{r_n}}}{{4n\left( {n - 1} \right)}} \to {0^ + }$ as $n\to \infty$. Substituting the chosen $t_n$'s into \eqref{2.53} yields
\begin{align}
{\left( {1 + \frac{1}{2}{r_n}x} \right)^3} = 1 - \frac{1}{n},\hspace{0.2cm}\forall n \in \mathbb{N},
\end{align}
i.e.,
\begin{align}
{r_n} = \frac{2}{x}\left( {\sqrt[3]{{1 - \frac{1}{n}}} - 1} \right),\hspace{0.2cm}\forall n \in \mathbb{N}.
\end{align}
It is easy to verify that $r_n\to 0^+$ as $n\to \infty$. Thus, $w\in T''\left(M,x_0,u\right)$ in this case.\\
\\
\textbf{Remark 2.2.} In both cases $x>0,x<0$, we can set in a more compact form as follows. Choose
\begin{align}
{t_n} = \frac{{r_n^2}}{{4n\left( {n + \mbox{sign}\left( x \right)} \right)}},\hspace{0.2cm}\forall n \in \mathbb{N},
\end{align}
and then \eqref{2.53} gives 
\begin{align}
{r_n} = \frac{2}{x}\left( {\sqrt[3]{{1 + \frac{{\mbox{sign}\left( x \right)}}{n}}} - 1} \right),\hspace{0.2cm}\forall n \in \mathbb{N}.
\end{align}
\end{itemize}
\end{enumerate}
The second inclusion is also proved.
\end{enumerate}
Combining these inclusions, we conclude that $T''\left( {M,{x_0},u} \right) = \mathbb{R}^2$. This completes our solution. \hfill $\square$\\

The following problem gives us some basic properties of contingent cone of first order.\\
\\
\textbf{Problem 6.} \textit{Let $X$ be a normed space, $M\subset X$ and $x_0 \in X$.}
\begin{enumerate}
\item \textit{If $T\left(M,x_0\right) \ne \emptyset$ then $x_0 \in \overline{M}$ (where $\overline{M}$ is the closure of the set $M$).}
\item \textit{$T\left(M,x_0\right)$ is a closed cone.}
\item \textit{$T\left( {M,{x_0}} \right) \subset \overline {\mbox{cone}\left( {M - {x_0}} \right)}$.}

\textit{Moreover, if $M$ is a convex set then}
\item $T\left( {M,{x_0}} \right) = \overline {\mbox{cone}\left( {M - {x_0}} \right)} $, and hence, $T\left(M,x_0\right)$ is a convex set.
\item $T\left( {M,{x_0}} \right) = \left\{ {v \in X|\forall {t_n} \to {0^ + },\forall {v_n} \to v,{x_0} + {t_n}{v_n} \in M} \right\}$.
\end{enumerate}
\textsc{Solution.}
\begin{enumerate}
\item Suppose that $T\left( {M,{x_0}} \right) \ne \emptyset $, we can take, for instance, $u\in T\left(M,x_0\right)$. Then there exist a sequence of positive reals $\left\{ {{t_n}} \right\}_{n = 1}^\infty $ such that $t_n \to 0^+$ and a sequence $\left\{ {{u_n}} \right\}_{n = 1}^\infty  \subset X$ such that $u_n\to u$ and $x_0+t_n u_n\in M$ for all $n\in \mathbb{N}$. Set $x_n:=x_0+t_nu_n \in M$. Since $u_n\to u$, there exists $N\in \mathbb{N}$ such that 
\begin{align}
n \ge N \Rightarrow \left\| {{u_n} - u} \right\| \le 1.
\end{align}
We now prove that $x_n\to x_0$ as $n\to \infty$. Indeed, for $n\ge N$,
\begin{align}
\label{2.66}
\left\| {{x_n} - {x_0}} \right\| = \left\| {{t_n}{u_n}} \right\| = {t_n}\left\| {{u_n}} \right\| \le {t_n}\left( {\left\| u \right\| + 1} \right).
\end{align}
Since $t_n\to 0^+$, \eqref{2.66} implies that $x_n \to x_0$ as $n\to \infty$, i.e., $x_0\in \overline{M}$. 
\item We first prove that $T\left(M,x_0\right)$ is a cone. Let $u\in T\left(M,x_0\right)$ arbitrarily, we need to prove that $tu\in T\left(M,x_0\right)$ for all $t>0$. By \eqref{2.1}, there exist a sequence of positive reals $\left\{ {{t_n}} \right\}_{n = 1}^\infty $ such that $t_n\to 0^+$ and a sequence $\left\{ {{u_n}} \right\}_{n = 1}^\infty  \subset X$ such that $u_n\to u$ and ${x_0} + {t_n}{u_n} \in M$ for all $n\in \mathbb{N}$. Fix $t>0$ arbitrarily, if we set ${v_n}: = t{u_n}$ and ${s_n} = \frac{{{t_n}}}{t}$ for all $n\in \mathbb{N}$, then $s_n\to 0^+$, ${v_n} \to tu$ and ${x_0} + {s_n}{v_n} = {x_0} + {t_n}{u_n} \in M$ for all $n\in \mathbb{N}$, i.e., $tu\in T\left(M,x_0\right)$. Since $t>0$ and $u\in T\left(M,x_0\right)$ are chosen arbitrarily, this implies that $T\left(M,x_0\right)$ is a cone. 

To prove that $T\left(M,x_0\right)$ is closed, let $\left\{ {{u_n}} \right\}_{m = 0}^\infty  \subset T\left( {M,{x_0}} \right)$ such that $u_m \to u$ as $m\to \infty$. We need to prove that $u\in T\left(M,x_0\right)$. To this end, by definition \eqref{2.1}, for each $m\in \mathbb{N}$, there exist a sequence $t_{m,n} \to 0^+$ as $n\to \infty$ and a sequence $\left\{ {{u_{m,n}}} \right\}_{n = 0}^\infty  \subset X$ such that $u_{m,n}\to u_m$ as $n\to \infty$ and $x_0+t_{m,n}u_{m,n} \in M$ for all $n\in \mathbb{N}$, in addition, $\left\| {{u_{m,m}} - {u_m}} \right\| \le \frac{1}{m}$ for all $m\in \mathbb{N}$\footnote{This is possible, since for each $m\in \mathbb{N}$, there exists a sequence $\left\{ {{u_{m,n}}} \right\}_{n = 0}^\infty  \subset X$ such that $u_{m,n}\to u_m$ as $n\to \infty$. By definition of limits, there exists $N\in \mathbb{N}$ such that 
\begin{align}
n \ge N \Rightarrow \left\| {{u_{m,n}} - {u_m}} \right\| \le \frac{1}{m}.
\end{align}
Hence, we can drop all the terms $u_{m,1},\ldots,u_{m,n-1}$ from the sequence. Re-indexing ${\widehat u_{m,n}}: = {u_{m,N + n - 1}}$ for all $n\in \mathbb{N}$, we have, in particular, 
\begin{align}
\left\| {{{\widehat u}_{m,m}} - {u_m}} \right\| = \left\| {{{\widehat u}_{m,N + m - 1}} - {u_m}} \right\| \le \frac{1}{m}.
\end{align}
We now ignore the old sequence $\left\{ {{u_{m,n}}} \right\}_{n = 0}^\infty $ and use the new sequence, by abuse notation, $\left\{ {{u_{m,n}}} \right\}_{n = 0}^\infty $ which is exactly $\left\{ {{{\widehat u}_{m,n}}} \right\}_{n = 0}^\infty $ just defined.}. We claim that
\begin{align}
\label{2.69}
{u_{m,m}} \to u \mbox{ and } {x_0} + {t_{m,m}}{u_{m,m}} \in M,\hspace{0.2cm}\forall m \in \mathbb{N}.
\end{align} 
The latter is obvious since ${x_0} + {t_{m,n}}{u_{m,n}} \in M$ for all $m,n \in \mathbb{N}$. We now prove the former in \eqref{2.69}. With the help of triangle inequality for the norm of $X$, 
\begin{align}
\left\| {{u_{m,m}} - u} \right\| &\le \left\| {{u_{m,m}} - {u_m}} \right\| + \left\| {{u_m} - u} \right\|\\
& \le \frac{1}{m} + \left\| {{u_m} - u} \right\| \to 0\mbox{ as } m \to \infty ,
\end{align}
i.e., $u\in T\left(M,x_0\right)$. Hence, $T\left(M,x_0\right)$ is a closed cone. 
\item The convex conical hull of $M-x_0$ is given by (see, e.g., \cite{1}, Def. 4.19, p.94)
\begin{align}
\label{2.72}
\mbox{cone}\left( {M - {x_0}} \right): = \left\{ {\sum\limits_{i = 1}^k {{\lambda _i}{x_i}} :{x_i} \in M - {x_0},{\lambda _i} > 0,k \ge 1} \right\}.
\end{align}
Take $u\in T\left(M,x_0\right)$ arbitrarily, we need to prove that $u \in \overline {\mbox{cone}\left( {M - {x_0}} \right)} $. By \eqref{2.1} again, there exist a sequence of positive reals $\left\{ {{t_n}} \right\}_{n = 1}^\infty $ such that $t_n \to 0^+$ and a sequence $\left\{ {{u_n}} \right\}_{n = 1}^\infty  \subset X$ such that $u_n\to u$ and $x_0+t_n u_n\in M$ for all $n\in \mathbb{N}$. The fact that $x_0+t_nu_n\in M$ for all $n\in \mathbb{N}$ gives us $t_nu_n\in M- {x_0}$ for all $n\in \mathbb{N}$. Choosing $k = 1,{\lambda _1} = \frac{1}{{{t_n}}} > 0,{x_1} = {t_n}{u_n} \in M - {x_0}$ in \eqref{2.72} gives ${u_n} \in \mbox{cone}\left( {M - {x_0}} \right)$ for all $n\in \mathbb{N}$. Combining this with the fact that $u_n\to u$, we conclude that $u \in \overline {\mbox{cone}\left( {M - {x_0}} \right)} $. Therefore,
\begin{align}
\label{2.73}
T\left( {M,{x_0}} \right) \subset \overline {\mbox{cone}\left( {M - {x_0}} \right)} .
\end{align}
\item \textsc{First Proof.} We now assume (until the end of the proof of this problem) that $M$ is a convex set and $x_0\in M$\footnote{The definition of tangent cone in \cite{1} also requires this.}. To prove $T\left( {M,{x_0}} \right) = \overline {\mbox{cone}\left( {M - {x_0}} \right)} $, due to \eqref{2.73}, it suffices to prove that $T\left( {M,{x_0}} \right) \supset \overline {\mbox{cone}\left( {M - {x_0}} \right)} $. First, we need the following lemma (see, e.g., \cite{2}, Lemma 2.4.11, p.41).\\
\\
\textbf{Lemma 2.3.} \textit{Let $M$ be a nonempty convex set and $x_0\in M$. Then}
\begin{align}
M - {x_0} \subset T\left( {M,{x_0}} \right).
\end{align}
\textit{Proof of Lemma 2.3.} Let $u \in M$. We need to show that $u-{x_0} \in T\left(M,x_0\right)$. To this end, choose $\left\{ {{t_n}} \right\}_{n = 1}^\infty  \subset \left[ {0,1} \right]$ such that $t_n\to 0^+$, and put $u_n:=u-x_0$ (hence $u_n\to u-x_0$ obviously) and put
\begin{align}
{x_n}: &= {x_0} + {t_n}\left( {u - {x_0}} \right)\\
 &= \left( {1 - {t_n}} \right){x_0} + {t_n}u \in M ,\hspace{0.2cm} \forall n\in \mathbb{N},
\end{align}
as $M$ is convex. By \eqref{2.1}, $u-{x_0}\in T\left(M,x_0\right)$. \hfill $\square$\\

Return to our proof, since we have proved that $T\left(M,x_0\right)$ is a closed cone, we only need to prove prove that $T\left( {M,{x_0}} \right) \supset \mbox{cone}\left( {M - {x_0}} \right)$. Using the fact that the convex conical hull of an arbitrary nonempty set is the intersection of all closed convex cones that contain that sets, it suffices to prove that $T\left(M,x_0\right)$ is convex (and thus is a closed convex cone). Take $u,v\in T\left(M,x_0\right)$, we need to prove that $\lambda u + \left( {1 - \lambda } \right)v \in T\left( {M,{x_0}} \right)$ for all $\lambda \in \left[0,1\right]$. But since $T\left(M,x_0\right)$ is a cone, we deduce that $\lambda u \in T\left( {M,{x_0}} \right)$ and $\left( {1 - \lambda } \right)v \in T\left( {M,{x_0}} \right)$. Hence, it suffices to prove the following stronger statement\footnote{\textit{A cone $K$ is convex if and only if $K+K\subset K$.} (see, e.g., \cite{2}, Proposition 2.4.2, p.38.)}
\begin{align}
u + v \in T\left( {M,{x_0}} \right),\hspace{0.2cm} \forall u,v \in T\left( {M,{x_0}} \right).
\end{align}
By \eqref{2.1}, there exists sequences of positive reals $\left\{ {{t_n}} \right\}_{n = 1}^\infty ,\left\{ {{s_n}} \right\}_{n = 1}^\infty $ such that $t_n\to 0^+$ and $s_n\to 0^+$ and sequences $\left\{ {{u_n}} \right\}_{n = 1}^\infty ,\left\{ {{v_n}} \right\}_{n = 1}^\infty $ such that $u_n\to u,v_n\to v$ and
\begin{align}
\label{2.78}
{x_0} + {t_n}{u_n} \in M,\hspace{0.2cm} {x_0} + {s_n}{v_n} \in M ,\hspace{0.2cm} \forall n\in \mathbb{N}.
\end{align}
Since $M$ is convex, it is deduced from \eqref{2.78} that
\begin{align}
\label{2.79}
\alpha \left( {{x_0} + {t_n}{u_n}} \right) + \left( {1 - \alpha } \right)\left( {{x_0} + {s_n}{v_n}} \right) \in M, \hspace{0.2cm} \forall \alpha  \in \left[ {0,1} \right],n\in \mathbb{N}.
\end{align}
In particular, choosing $\alpha  = \frac{{{s_n}}}{{{t_n} + {s_n}}}$ in \eqref{2.79} gives
\begin{align}
{x_0} + \frac{{{t_n}{s_n}}}{{{t_n} + {s_n}}}\left( {{u_n} + {v_n}} \right) \in M, \hspace{0.2cm} \forall n  \in \mathbb{N}.
\end{align}
Hence, if we choose ${w_n}: = {u_n} + {v_n} \to u + v$ and ${r_n}: = \frac{{{t_n}{s_n}}}{{{t_n} + {s_n}}} \to {0^ + }$\footnote{Indeed, $0 < {r_n} = {t_n}\underbrace {\frac{{{s_n}}}{{{t_n} + {s_n}}}}_{ < 1} < {t_n} \to {0^ + }$ as $n\to \infty$.}. By \eqref{2.1}, $u+v \in T\left(M,x_0\right)$. This completes our proof. \hfill $\square$\\
\\
\textsc{Second Proof.} We have the following result (see, e.g., \cite{2}, Proposition 2.4.8, p.40)
\begin{align}
\mbox{cone}S = {\mathbb{R}_ + }\left( {\mbox{conv}S} \right) = \mbox{conv} \left( {{\mathbb{R}_ + }S} \right),
\end{align}
for an arbitrary nonempty set $S$. Since $M$ is convex, $M-x_0$ is also convex (as a Minkowski sum of convex sets), hence $\mbox{conv}\left( {M - {x_0}} \right) = M - {x_0}$ (see \cite{1}, Corollary 4.12, p.91) and 
\begin{align}
\overline {\mbox{cone}\left( {M - {x_0}} \right)}  = \overline {{\mathbb{R}_ + }\left( {\mbox{conv}\left( {M - {x_0}} \right)} \right)}  = \overline {{\mathbb{R}_ + }\left( {M - {x_0}} \right)}.
\end{align}
It suffices to prove $\overline {{\mathbb{R}_ + }\left( {M - {x_0}} \right)}  \subset T\left( {M,{x_0}} \right)$. By Lemma 2.3, we have $M - {x_0} \subset T\left( {M,{x_0}} \right)$. Since $T\left(M,x_0\right)$ is a closed cone, this yields $\overline {{\mathbb{R}_ + }\left( {M - {x_0}} \right)}  \subset T\left( {M,{x_0}} \right)$. A direct consequence of this fact is that $T\left(M,x_0\right)$ is a closed convex cone.
\item \textit{(Need correcting)} Suppose the set in the right-hand side is nonempty, i.e., there exists $v\in X$ such that 
\begin{align}
\label{2.83}
\forall {t_n} \to {0^ + },\forall {v_n} \to v,{x_0} + {t_n}{v_n} \in M,\hspace{0.2cm}\forall n \in \mathbb{N}.
\end{align}
If we take $t_1$ and $v_1$ arbitrarily, then $x_0+t_1 v_1$ still belongs to $M$. Hence, $M=X$? Should \eqref{2.83} be corrected as ``$\forall {t_n} \to {0^ + },\forall {v_n} \to v,{x_0} + {t_n}{v_n} \in M$ \textit{for $n$ large enough}''? This problem needs correcting.
\end{enumerate}
We end our proof here. \hfill $\square$\\

The following problem gives us some basic properties of second-order contingent set.\\
\\
\textbf{Problem 7.} \textit{Let $X$ be a normed space, $M\subset X$ and $x_0,u\in X$.}
\begin{enumerate}
\item \textit{If $u \notin T\left(M,x_0\right)$ then $T^2\left(M,x_0,u\right)$ and $T''\left(M,x_0,u\right)$ are empty sets.}
\item ${T^2}\left( {M,{x_0},0} \right) = T''\left( {M,{x_0},0} \right) = T\left( {M,{x_0}} \right)$.
\item \textit{$T''\left(M,x_0,u\right)$ is a cone while $T^2\left(M,x_0,u\right)$ is not a cone in general.}
\item \textit{If $X$ is finite dimensional and $u\in T\left(M,x_0\right)$ then}
\begin{align}
\label{2.84}
{T^2}\left( {M,{x_0},u} \right) \cup T''\left( {M,{x_0},u} \right) \ne \emptyset .
\end{align}
\end{enumerate}
\textsc{Solution.}
\begin{enumerate}
\item Suppose that $u\notin T\left(M,x_0\right)$, we claim that
\begin{align}
\label{2.85}
{T^2}\left( {M,{x_0},u} \right) = T''\left( {M,{x_0},u} \right) = \emptyset .
\end{align}
To prove ${T^2}\left( {M,{x_0},u} \right) = \emptyset $, we suppose for the contrary that there exists a $w\in T^2\left(M,x_0,u\right)$. By \eqref{2.2}, there exist a sequence of positive reals $\left\{ {{t_n}} \right\}_{n = 1}^\infty $ such that $t_n\to 0^+$ as $n\to \infty$ and a sequence $\left\{ {{w_n}} \right\}_{n = 1}^\infty  \subset X$ such that $w_n\to w$ as $n\to \infty$ and ${x_0} + {t_n}u + \frac{1}{2}t_n^2{w_n} \in M$   for all $n\in \mathbb{N}$. To obtain a contradiction, we now prove that $u\in T\left(M,x_0\right)$. Indeed, setting $u_n:=u+\frac{1}{2}t_nw_n$ for all $n\in \mathbb{N}$, it is obvious to verify that $u_n\to u$ as $n\to \infty$ and $x_0+t_nu_n\in M$ for all $n\in \mathbb{N}$, i.e., $u\in T\left(M,x_0\right)$. This contradiction implies that $T^2\left(M,x_0,u\right) =\emptyset$.

Similarly, to prove $T''\left(M,x_0,u\right) =\emptyset$, we suppose for the contrary that there exists a $w\in T''\left(M,x_0,u\right)$. By \eqref{2.3}, there exist two sequences of positive reals $\left\{ {{t_n}} \right\}_{n = 1}^\infty $, $\left\{ {{r_n}} \right\}_{n = 1}^\infty $ such that $t_n\to 0^+,r_n\to 0^+$ and $\frac{t_n}{r_n}\to 0$ as $n\to \infty$ and a sequence $\left\{ {{w_n}} \right\}_{n = 1}^\infty  \subset X$ such that $w_n\to w$ and $x_0+t_n u+\frac{1}{2}t_nr_nw_n\in M$ for all $n\in \mathbb{N}$. To obtain a contradiction, we now prove that $u\in T\left(M,x_0\right)$ as above. Indeed, setting $u_n:=u+\frac{1}{2}r_nw_n$ for all $n\in \mathbb{N}$, it is obvious to verify that $u_n\to u$ as $n\to \infty$ and $x_0+t_nu_n\in M$ for all $n\in \mathbb{N}$, i.e., $u\in T\left(M,x_0\right)$. This contradiction implies that $T''\left(M,x_0,u\right)=\emptyset$. Hence, \eqref{2.85} holds.
\item We claim that
\begin{align}
{T^2}\left( {M,{x_0},0} \right) = T''\left( {M,{x_0},0} \right) = T\left( {M,{x_0}} \right).
\end{align}
\textit{Prove ${T^2}\left( {M,{x_0},0} \right) = T''\left( {M,{x_0},0} \right)$.} Taking $w\in T^2\left(M,x_0,0\right)$, by \eqref{2.2}, there exist a sequence of positive reals $\left\{ {{t_n}} \right\}_{n = 1}^\infty $ such that $t_n\to 0^+$ as $n\to \infty$ and a sequence $\left\{ {{w_n}} \right\}_{n = 1}^\infty  \subset X$ such that $w_n\to w$ as $n\to \infty$ and 
\begin{align}
\label{2.87}
{x_0} + \frac{1}{2}t_n^2{w_n} \in M,\hspace{0.2cm}\forall n \in \mathbb{N}.
\end{align}
Setting ${\widehat t_n} = {t_n}\sqrt {{t_n}} ,{r_n} = \sqrt {{t_n}} $ for all $n\in \mathbb{N}$, this choice ensures that $\widehat{t}_n\to 0^+,r_n\to 0^+,\dfrac{\widehat{t}_n}{r_n}=t_n\to 0$ as $n\to \infty$. Moreover, \eqref{2.87} can be rewritten as
\begin{align}
{x_0} + \frac{1}{2}{\widehat t_n}{r_n}{w_n} \in M,\hspace{0.2cm}\forall n \in \mathbb{N}.
\end{align}
i.e., $w\in T''\left(M,x_0,u\right)$. Notice that this argument is reversible (choose ${t_n} = \sqrt {{{\widehat t}_n}{r_n}} $ for all $n\in \mathbb{N}$ for the converse inclusion). Hence ${T^2}\left( {M,{x_0},0} \right) = T''\left( {M,{x_0},0} \right)$.

\textit{Prove ${T^2}\left( {M,{x_0},0} \right) = T\left( {M,{x_0}} \right)$.} Briefly, this equality is easily deduced from
\begin{align}
{x_0} + {t_n}{w_n} \in M \Leftrightarrow {x_0} + \frac{1}{2}\widehat t_n^2{w_n} \in M,
\end{align}
which holds by choosing ${t_n} := \frac{1}{2}\widehat t_n^2 \to {0^ + }$ for the inclusion ${T^2}\left( {M,{x_0},0} \right) \subset T\left( {M,{x_0}} \right)$ and ${\widehat t_n} = \sqrt {2{t_n}} $ for the converse.

\textit{Prove $T''\left(M,x_0,0\right) =T\left(M,x_0\right)$.} (This part is unnecessary but I also provide it here for completeness) Similarly, this equality is easily deduced from
\begin{align}
{x_0} + {t_n}{w_n} \in M \Leftrightarrow {x_0} + \frac{1}{2}{\widehat t_n}{r_n}{w_n} \in M,
\end{align}
which holds by choosing ${t_n}: = \frac{1}{2}{\widehat t_n}{r_n}$ for the inclusion $T''\left( {M,{x_0},0} \right) \subset T\left( {M,{x_0}} \right)$ and, for instance, ${\widehat t_n} = 2t_n^{\frac{2}{3}},{r_n} = t_n^{\frac{1}{3}}$ for the converse. 
\item To prove that $T''\left(M,x_0,u\right)$ is a cone, taking $w\in T''\left(M,x_0,u\right)$, we will prove that $tw\in T''\left(M,x_0,u\right)$ for all $t>0$. Fix $t>0$ arbitrary, by \eqref{2.3}, there exist two sequences of positive reals $\left\{ {{t_n}} \right\}_{n = 1}^\infty $, $\left\{ {{r_n}} \right\}_{n = 1}^\infty $ such that $t_n\to 0^+,r_n\to 0^+$ and $\frac{t_n}{r_n}\to 0$ as $n\to \infty$ and a sequence $\left\{ {{w_n}} \right\}_{n = 1}^\infty  \subset X$ such that $w_n\to w$ and $x_0+t_n u+\frac{1}{2}t_nr_nw_n\in M$ for all $n\in \mathbb{N}$. By setting $\widehat{r}_n:=\frac{r_n}{t}$ and $\widehat{w}_n:=tw_n$, we have $\widehat{w}_n\to tw$ and
\begin{align}
{x_0} + {t_n}u + \frac{1}{2}{t_n}{\widehat r_n}{\widehat w_n} \in M,\hspace{0.2cm}\forall n \in \mathbb{N},
\end{align}
i.e., $tw\in T''\left(M,x_0,u\right)$. Since $t>0$ and $w\in T''\left(M,x_0,u\right)$ are chosen arbitrarily, we conclude that $T''\left(M,x_0,u\right)$ is a cone.

To prove that $T^2\left(M,x_0,u\right)$ is not a cone in general, we go back to the setting of Problem 4. We have proved that
\begin{align}
{T^2}\left( {M,{x_0},u} \right) = \left\{ {\left( {x,y} \right) \in {\mathbb{R}^2}|x \le  - 2} \right\}.
\end{align} 
Taking $w:=\left(x,y\right)\in T^2\left(M,x_0,u\right)$, we have $x\le -2$. But this does not implies that $tw \in T^2\left(M,x_0,u\right)$ for all $t>0$. Indeed, choosing $t:=-\frac{1}{x}>0$ yields $tx =-1>-2$, i.e., $tw\notin T^2\left(M,x_0,u\right)$. It follows that $T^2\left(M,x_0,u\right)$ is not a cone in general.
\item Since $X$ is finite-dimensional, we can assume that $X=\mathbb{R}^n$ without loss of generality. In \cite{4}, Remark 7, p.88, the authors have proved the following stronger results, from which \eqref{2.84} follows directly.
\end{enumerate}
\textbf{Theorem 2.4.} \textit{Let $M\subset \mathbb{R}^n$, $x_0 \in \overline{M}$ and $u\in \mathbb{R}^n$.}
\begin{enumerate}
\item $0 \in T''\left( {M,{x_0},u} \right) \Leftrightarrow u \in T\left( {M,{x_0}} \right)$.
\item \textit{If $T^2\left(M,x_0,u\right) =\emptyset$ and $u\in T\left(M,x_0\right)$, then there exists $w\in T''\left(M,x_0,u\right)$, $w\ne 0$, such that $w^Tu=0$.}
\end{enumerate}
\textsc{Proof of Theorem 2.4.} 
\begin{enumerate}
\item \textit{Prove $0 \in T''\left( {M,{x_0},u} \right) \Rightarrow u \in T\left( {M,{x_0}} \right)$.} Let $0\in T''\left(M,x_0,u\right)$, then there exist two sequences of positive reals $\left\{ {{t_n}} \right\}_{n = 1}^\infty $, $\left\{ {{r_n}} \right\}_{n = 1}^\infty $ such that $t_n\to 0^+,r_n\to 0^+$ and $\frac{t_n}{r_n}\to 0$ as $n\to \infty$ and a sequence $\left\{ {{w_n}} \right\}_{n = 1}^\infty  \subset X$ such that $w_n\to 0$ and $x_0+t_n u+\frac{1}{2}t_nr_nw_n\in M$ for all $n\in \mathbb{N}$. By setting ${u_n}: = u + \frac{1}{2}{r_n}{w_n}$ for all $n\in \mathbb{N}$, we have $u_n\to u$ as $n\to \infty$ and $x_0+t_n u_n\in M$ for all $n\in \mathbb{N}$, i.e., $u\in T\left(M,x_0\right)$.

\textit{Prove $u \in T\left( {M,{x_0}} \right) \Rightarrow 0 \in T''\left( {M,{x_0},u} \right)$.}  If $T\left( {M,{x_0}} \right) = \left\{ 0 \right\}$, then $u=0$ and by the result obtained in Problem 7.2, $T''\left( {M,{x_0},u} \right) = T\left( {M,{x_0}} \right) = \left\{ 0 \right\}$ and the conclusion is true. 

If $T\left( {M,{x_0}} \right) \ne \left\{ 0 \right\}$, choose $u \in T\left( {M,{x_0}} \right)\backslash \left\{ 0 \right\}$. We have two cases depending on $T^2\left(M,x_0,u\right)$.
\begin{enumerate}
\item \textit{Case $T^2\left(M,x_0,u\right) \ne \emptyset$.} Pick $w\in T^2\left(M,x_0,u\right)$, then there exist a sequence of positive reals $\left\{ {{t_n}} \right\}_{n = 1}^\infty $ such that $t_n\to 0^+$ as $n\to \infty$ and a sequence $\left\{ {{w_n}} \right\}_{n = 1}^\infty  \subset X$ such that $w_n\to w$ as $n\to \infty$ and ${x_0} + {t_n}u + \frac{1}{2}t_n^2{w_n} \in M$   for all $n\in \mathbb{N}$. Setting $r_n:=\sqrt{t_n}$, $\widehat{w}_n:=\sqrt{t_n}{w_n}$ for all $n\in \mathbb{N}$, we have $r_n\to 0^+$, $\frac{{{t_n}}}{{{r_n}}} = \sqrt {{t_n}}  \to {0^ + }$, $\widehat{w}_n \to 0$ as $n\to \infty$ and
\begin{align}
{x_0} + {t_n}u + \frac{1}{2}{t_n}{r_n}{\widehat w_n} = {x_0} + {t_n}u + \frac{1}{2}t_n^2{w_n} \in M,\hspace{0.2cm}\forall n \in \mathbb{N}.
\end{align}
i.e., $0\in T''\left(M,x_0,u\right)$.
\item \textit{Case $T^2\left(M,x_0,u\right) = \emptyset$.} As $u \in T\left( {M,{x_0}} \right)\backslash \left\{ 0 \right\}$, there exist a sequence of positive real $\left\{ {{t_n}} \right\}_{n = 1}^\infty $ such that $t_n\to 0^+$ and a sequence $\left\{ {{u_n}} \right\}_{n = 1}^\infty \in X$ such that $u_n\to u$ as $n\to \infty$ and $x_n:=x_0+t_nu_n\in M$ for all $n\in \mathbb{N}$. By Lemma 3.4, \cite{5}, p.129, there exists a subsequence, denoted again $t_n$'s (and also $x_n$'s), such that either
\begin{itemize}
\item $\dfrac{{{x_n} - {x_0} - {t_n}u}}{{\frac{1}{2}t_n^2}} \to w$ for some $w \in {T^2}\left( {M,{x_0},u} \right) \cap {u^ \bot }$ or
\item there exists a sequence $r_n\to 0^+$ such that $\frac{t_n}{r_n}\to 0$ and
\begin{align}
\frac{{{x_n} - {x_0} - {t_n}u}}{{\frac{1}{2}{t_n}{r_n}}} \to w,
\end{align}
for some $w \in T''\left( {M,{x_0},u} \right) \cap {u^ \bot }\backslash \left\{ 0 \right\}$.
\end{itemize}
As $T^2\left(M,x_0,u\right) =\emptyset$, the second condition is satisfied. Hence, as $T''\left(M,x_0,v\right) \ne \emptyset$, being this set a closed cone, we conclude that $0\in T''\left(M,x_),u\right)$. 
\end{enumerate}
\item At the same time we have proved this part of Theorem 2.4. Indeed, the assumptions of part $\left(1\right)$ imply that $u\ne 0$, as if $u=0$, then $T^2\left(M,x_0,0\right) =T\left(M,x_0\right) \ne \emptyset$, in contradiction with the assumption. 
\end{enumerate}
We end the proof of Theorem 2.4 here. \hfill $\square$
\section{Theory of Optimality Conditions}
In this section, we will discuss about the theory of optimality conditions for the following problems
\begin{align}
\left( P \right): \hspace{0.5cm}\min f\left( x \right)\mbox{ s.t. } x \in \Omega .
\end{align}
where $f:\mathbb{R}^n\to \mathbb{R}$ and $\Omega  \subseteq {\mathbb{R}^n}$.\\
\\
\textbf{Definition 3.1.} Consider the problem $\left(P\right), m\in \mathbb{N}^*$,
\begin{enumerate}
\item $x_0 \in \Omega$ is called \textit{local minimizer} of $\left(P\right)$ if there exists a neighborhood $U$ of $x_0$ such that
\begin{align}
f\left( x \right) \ge f\left( {{x_0}} \right),\hspace{0.2cm}\forall x \in U \cap \Omega .
\end{align}
\item $x_0\in \Omega$ is called \textit{strictly local minimizer of order $m$} of $\left(P\right)$ if there exist a neighborhood $U$ of $x_0$ and a positive real number $\alpha$ such that
\begin{align}
f\left( x \right) \ge f\left( {{x_0}} \right) + \alpha {\left\| {x - {x_0}} \right\|^m}, \hspace{0.2cm}\forall x \in U \cap \Omega .
\end{align}
\end{enumerate}
\textbf{Theorem 3.2 (First-order necessary optimality condition).} 

\textit{If $x_0$ is a local minimizer of $\left(P\right)$ then}
\begin{align}
\left\langle {\nabla f\left( {{x_0}} \right),u} \right\rangle  \ge 0,\hspace{0.2cm}\forall u \in T\left( {\Omega ,{x_0}} \right).
\end{align}
\textbf{Theorem 3.3 (First-order sufficient optimality condition).} 
\begin{enumerate}
\item \textit{If $f$ is a convex function, $\Omega$ is a convex set and}
\begin{align}
\left\langle {\nabla f\left( {{x_0}} \right),u} \right\rangle  \ge 0,\hspace{0.2cm}\forall u \in T\left( {\Omega ,{x_0}} \right),
\end{align}
\textit{then $x_0$ is a minimizer of $\left(P\right)$.}
\item \textit{If for all $u\in T\left(\Omega,x_0\right)$ satisfying $\left\| u \right\| = 1$, $\left\langle {\nabla f\left( {{x_0}} \right),u} \right\rangle  > 0$ holds then $x_0$ is a strictly local minimizer of first order of $\left(P\right)$.}
\end{enumerate}
We consider some exercises which apply the first order optimality conditions.\\
\\
\textbf{Problem 8 (Fermat's rule).} \textit{Use Theorem 3.2, prove that if $x_0$ is a local minimizer of} $\left(P\right)$ and $x_0\in \mbox{int } \Omega$ \textit{then ${\nabla f\left( {{x_0}} \right)}=0$, then apply this result to find solutions of the following problems.}
\begin{enumerate}
\item $\left( P \right):\hspace{0.5cm}\min {x^2} + 3{y^2} - 2xy - 4x - 8y$ s.t. $\left(x,y\right) \in \mathbb{R}^2$.
\item $\left( P \right):\hspace{0.5cm}\min xyz{e^{ - x - y - z}}$ s.t. $\left(x,y,z\right)\in \mathbb{R}^3$.
\end{enumerate}
\textsc{Solution.} Since $x_0$ is a local minimizer of $\left(P\right)$, by Definition 3.1 and Theorem 3.2, there exists a neighborhood $U$ of $x_0$ such that
\begin{align}
\label{3.6}
f\left( x \right) \ge f\left( {{x_0}} \right),\hspace{0.2cm}\forall x \in U \cap \Omega ,
\end{align}
and
\begin{align}
\label{3.7}
\left\langle {\nabla f\left( {{x_0}} \right),u} \right\rangle  \ge 0,\hspace{0.2cm}\forall u \in T\left( {\Omega ,{x_0}} \right).
\end{align}
Since $x_0 \in \mbox{int }\Omega$, there exists a positive real $r>0$ such that $B_r\left(x_0\right) \subset \Omega$. 

We claim that if $x_0\in \mbox{int }\Omega$ then $T\left(\Omega,x_0\right)=\mathbb{R}^n$. To prove this claim, for arbitrary $u\in \mathbb{R}^n$, we will prove that $u\in T\left(\Omega,x_0\right)$. Indeed, we can take an arbitrary sequence $\left\{ {{u_n}} \right\}_{n = 1}^\infty \in \mathbb{R}^n$ such that $u_n\to u$ as $n\to \infty$. We now choose a sequence of positive reals $\left\{ {{t_n}} \right\}_{n = 1}^\infty $ whose each term $t_n$ is small enough such that
\begin{align}
\label{3.8}
{x_0} + {t_n}{u_n} \in {B_r}\left( {{x_0}} \right).
\end{align}
For instance, we can take $t_n$'s satisfying
\begin{align}
0 < {t_n} < \frac{r}{{\left\| {{u_n}} \right\|}},\hspace{0.2cm}\forall n \in \mathbb{N},
\end{align}
in order that \eqref{3.8} holds for all $n\in \mathbb{N}$. Thus, $u\in T\left(\Omega,x_0\right)$. And this yields $T\left(\Omega,x_0\right)=\mathbb{R}^n$ as we claimed. 

Use this result, we now can proceed as the proof of Theorem 2.7, \cite{1}, p.35 as follows. If $d\in \mathbb{R}^n$, then
\begin{align}
f'\left( {{x_0};d} \right) = \mathop {\lim }\limits_{t \to 0} \frac{{f\left( {{x_0} + td} \right) - f\left( {{x_0}} \right)}}{t} = \left\langle {\nabla f\left( {{x_0}} \right),d} \right\rangle .
\end{align}
By theorem 3.2, since $T\left(M,x_0\right)=\mathbb{R}^n$, we have 
\begin{align}
\left\langle {\nabla f\left( {{x_0}} \right),d} \right\rangle  \ge 0,\hspace{0.2cm}\forall d \in T\left( {M,{x_0}} \right) = {\mathbb{R}^n}.
\end{align}
In particular, picking $d =  - \nabla f\left( {{x_0}} \right)$ gives $ - {\left\| {\nabla f\left( {{x_0}} \right)} \right\|^2} \ge 0$, that is, $\nabla f\left( {{x_0}} \right) = 0$.

We now apply this result to find solutions of the above problems.
\begin{enumerate}
\item Setting $\Omega =\mathbb{R}^n$, and
\begin{align}
f\left( {x,y} \right) = {x^2} + 3{y^2} - 2xy - 4x - 8y,\hspace{0.2cm} \forall \left( {x,y} \right) \in {\mathbb{R}^2}.
\end{align}
By Fermat's rule, we consider the equation $\nabla f\left( {{x_0},{y_0}} \right) = 0$, i.e.,
\begin{align}
\nabla f\left( {{x_0},{y_0}} \right) = \left( {2{x_0} - 2{y_0} - 4,6{y_0} - 2{x_0} - 8} \right) = 0.
\end{align}
This gives a linear system of equations
\begin{align}
{x_0} - {y_0} &= 2,\\
{x_0} - 3{y_0} &=  - 4.
\end{align}
Solving this yields $x_0=5,y_0=3$. 

\textit{Check.} The point $\left(x_0,y_0\right)=\left(5,2\right)$ is a global minimizer of $f\left(x,y\right)$. Indeed, since $f\left(x_0, y_0\right)=f\left(5,3\right)=-22$, it suffices to prove
\begin{align}
f\left( {x,y} \right) + 22 \ge 0,\hspace{0.2cm}\forall \left( {x,y} \right) \in {\mathbb{R}^2}.
\end{align}
Fixed $y\in \mathbb{R}$, we set
\begin{align}
{F_y}\left( x \right): = f\left( {x,y} \right) + 22 ,\hspace{0.2cm}\forall x \in \mathbb{R}.
\end{align}
The first derivative of $F_y\left(x\right)$ is 
\begin{align}
\label{3.18}
\frac{d}{{dx}}{F_y}\left( x \right) = 2\left( {x - y - 2} \right),\hspace{0.2cm}\forall x \in \mathbb{R}.
\end{align}
By surveying the sign of $\frac{d}{{dx}}{F_y}\left( x \right)$, it follows from \eqref{3.18} that
\begin{align}
\mathop {\min }\limits_{x \in \mathbb{R}} {F_y}\left( x \right) = {F_y}\left( {y + 2} \right) = 2{\left( {y - 3} \right)^2}\ge 0.
\end{align}
Hence, 
\begin{align}
\min \left( {f\left( {x,y} \right) + 22} \right) &= \mathop {\min }\limits_{y \in \mathbb{R}} \mathop {\min }\limits_{x \in \mathbb{R}} {F_y}\left( x \right)\\
& = \mathop {\min }\limits_{y \in \mathbb{R}} 2{\left( {y - 3} \right)^2}\\
& = 0,
\end{align}
which is attained at $y=3$ (and thus) $x=y+2=5$.
\item Setting $\Omega =\mathbb{R}^3$, and
\begin{align}
f\left( {x,y,z} \right) = xyz{e^{ - x - y - z}},\hspace{0.2cm}\forall \left( {x,y,z} \right) \in {\mathbb{R}^3}.
\end{align}
We have
\begin{align}
\nabla f\left( {x,y,z} \right) = {e^{ - x - y - z}}\left( {yz\left( {1 - x} \right),zx\left( {1 - y} \right),xy\left( {1 - z} \right)} \right),
\end{align}
for all $\left( {x,y,z} \right) \in {\mathbb{R}^3}$. The equation $\nabla f\left( {{x_0},{y_0},{z_0}} \right) = 0$ gives the following system of equations
\begin{align}
{y_0}{z_0}\left( {1 - {x_0}} \right) &= 0,\\
{z_0}{x_0}\left( {1 - {y_0}} \right) &= 0,\\
{x_0}{y_0}\left( {1 - {z_0}} \right) &= 0.
\end{align}
The roots of this system are $\left(1,1,1\right)$, $\left(a,0,0\right)$ for all $a\in \mathbb{R}$ and their permutations. 

If exactly two in three numbers $x_0,y_0,z_0$ is equal to $0$ (for instance, $\left(x_0,y_0,z_0\right)=\left(a,0,0\right)$ for some nonzero $a\in \mathbb{R}$) then we can choose in a neighborhood of $\left(x_0,y_0,z_0\right)$ a point $\left(x,y,z\right)$ for which $xyz<0$ \footnote{This is easily handled by considering the signs of the three coordinates.} (for instance, choose $\left( {x,y,z} \right) = \left( {a,\frac{{\mathrm{sign} \left( a \right)}}{n}, - \frac{1}{n}} \right)$ with $n$ small enough), and thus $f\left(x,y,z\right)<0$. But $f\left(x_0,y_0,z_0\right)=0$, we deduce that $\left(x_0,y_0,z_0\right)$ is not a local minimizer of $f$. We only need to consider the remaining cases, i.e., $\left(x_0,y_0,z_0\right) =\left(0,0,0\right)$ and $\left(x_0,y_0,z_0\right) =\left(1,1,1\right)$.

For $\left(x_0,y_0,z_0\right) =\left(0,0,0\right)$, we choose $x_n = y_n = \frac{1}{n},z_n =  - \frac{2}{n}$. This choice ensures that $x_n+y_n+z_n=0$ and $\left(x_n,y_n,z_n\right)$ lies in any given neighborhood of $\left(0,0,0\right)$ provided $n$ is large enough. We then have
\begin{align}
f\left( {{x_n},{y_n},{z_n}} \right) =  - \frac{2}{{{n^3}}} < 0 = f\left( {0,0,0} \right).
\end{align}
This implies that $\left(0,0,0\right)$ is not a local minimizer. 

For $\left(x_0,y_0,z_0\right) =\left(1,1,1\right)$, we choose $x_n=1+\frac{2}{n},y_n=z_n=1-\frac{1}{n}$. This choice ensures that $x_n+y_n+z_n=3$ and $\left(x_n,y_n,z_n\right)$ lies in any given neighborhood of $\left(1,1,1\right)$ if $n$ is large enough. By Cauchy inequality, we have
\begin{align}
x_ny_nz_n = \left( {1 + \frac{2}{n}} \right){\left( {1 - \frac{1}{n}} \right)^2} \le {\left( {\frac{{1 + \frac{2}{n} + 1 - \frac{1}{n} + 1 - \frac{1}{n}}}{3}} \right)^3} = 1.
\end{align}
Since $x_n\ne y_n$, the equality does not hold and thus this gives us $x_ny_nz_n<1$. We then have
\begin{align}
f\left( {{x_n},{y_n},{z_n}} \right) = \frac{{{x_n}{y_n}{z_n}}}{{{e^3}}} < \frac{1}{{{e^3}}} = f\left( {1,1,1} \right).
\end{align}
This implies that $\left(1,1,1\right)$ is not a local minimizer. In face, we can use Hessian matrix of $f$ to prove that $\left(1,1,1\right)$ is a local maximizer. Finally, we conclude that there does not exist any local minimizers of $\left(P\right)$.

Furthermore, the fact that $f$ has no global minimizer can be easily demonstrated by choosing $x=-n,y=n-1,y=2$ for $n \in \mathbb{N}$, 
\begin{align}
f\left( { - n,n - 1,2} \right) =  - 2n\left( {n - 1} \right)e \to  - \infty \mbox{ as } n \to  + \infty .
\end{align}
\end{enumerate}
This completes our solution. \hfill $\square$\\
\\
\textbf{Problem 9.} \textit{Consider the following problem}
\begin{align}
\left( P \right):\hspace{0.5cm}\min {x^2} - y\mbox{ s.t. } \left( {x,y} \right) \in \Omega  = \left\{ {\left( {x,y} \right) \in {\mathbb{R}^2}|x + {y^3} \ge 0} \right\}.
\end{align}
\begin{enumerate}
\item \textit{Compute the tangent cone of $\Omega$ at the point $x_0=\left(0,0\right)$.}
\item \textit{Apply the first-order necessary optimality condition, check whether $x_0=\left(0,0\right)$ is a local minimizer of $\left(P\right)$ or not.}
\end{enumerate}
\textsc{Solution.} Setting $X=\mathbb{R}^2$, $f\left(x,y\right)=x^2-y$ for $\left(x,y\right)\in \mathbb{R}^2$, we notice that $x_0=\left(0,0\right)\in \Omega$. 
\begin{enumerate}
\item We claim that
\begin{align}
\label{3.33}
T\left( {\Omega ,{x_0}} \right) = \widehat T\left( {\Omega ,{x_0}} \right): = \left\{ {\left( {x,y} \right) \in {\mathbb{R}^2}|x \ge 0} \right\}.
\end{align}
To prove \eqref{3.33}, we prove the following inclusions.
\begin{enumerate}
\item \textit{Prove $T\left( {\Omega ,{x_0}} \right) \subset \widehat T\left( {\Omega ,{x_0}} \right)$.} Taking $u=\left(x,y\right)\in T\left(M,x_0\right)$, there exist a sequence of positive reals $\left\{ {{t_n}} \right\}_{n = 1}^\infty $ such that $t_n\to 0^+$ and a sequence $\left\{ {{u_n}} \right\}_{n = 1}^\infty  \subset {\mathbb{R}^2}$ such that $u_n\to u$ as $n\to \infty$ and $x_0+t_nu_n\in \Omega$ for all $n\in \mathbb{N}$. Set $u_n:=\left(x_n,y_n\right)$, the face $u_n\to u$ implies that $x_n\to x$ and $y_n\to y$, and the fact $x_0+t_nu_n\in \Omega$ for all $n\in \mathbb{N}$ gives
\begin{align}
\label{3.34}
{t_n}{x_n} + t_n^3y_n^3 \ge 0,\hspace{0.2cm}\forall n \in \mathbb{N}.
\end{align}
Since $t_n>0$ for all $n\in \mathbb{N}$, \eqref{3.34} then implies
\begin{align}
\label{3.35}
{x_n} + t_n^2y_n^3 \ge 0,\hspace{0.2cm}\forall n \in \mathbb{N}.
\end{align}
Now let $n\to \infty$ and use the given limits $x_n\to x,y_n\to y$ and $t_n\to 0^+$, we obtain $x\ge 0$. Hence, $u\in \widehat{T}\left(\Omega,x_0\right)$ and our first inclusion is proved.
\item \textit{Prove $\widehat T\left( {\Omega ,{x_0}} \right) \subset T\left( {\Omega ,{x_0}} \right)$.} Taking $u=\left(x,y\right)\in \mathbb{R}^2$ satisfying $x\ge 0$, we claim that $u\in T\left(\Omega,x_0\right)$. To this end, we now choose $x_n=x+\frac{1}{n},y_n=y$ for all $n\in \mathbb{N}$. This choice ensures that $u_n:=\left(x_n,y_n\right)\to u:=\left(x,y\right)$ as $n\to \infty$. It then suffices to prove that there exists a sequence of positive reals $\left\{ {{t_n}} \right\}_{n = 1}^\infty $ such that $t_n\to 0^+$ and $x_0+t_nu_n\in \Omega$ for all $n\in \mathbb{N}$. The latter gives, using \eqref{3.35} again, 
\begin{align}
\label{3.36}
x + \frac{1}{n} + t_n^2{y^3} \ge 0,\hspace{0.2cm}\forall n \in \mathbb{N}.
\end{align}
We consider the following cases depending on the sign of $y$. If $y\ge0$, then \eqref{3.36} holds for all positive reals $t_n$'s. Thus we can take an arbitrary sequence of positive reals $\left\{ {{t_n}} \right\}_{n = 1}^\infty $ such that $t_n\to 0^+$, this means $u\in T\left(\Omega,x_0\right)$ in this case. If $y<0$, \eqref{3.36} gives
\begin{align}
\label{3.37}
{t_n} \le \sqrt { - \frac{1}{{{y^3}}}\left( {x + \frac{1}{n}} \right)} ,\hspace{0.2cm}\forall n \in \mathbb{N}.
\end{align}
The term in the right-hand side of \eqref{3.37} is positive for all $n\in \mathbb{N}$. Hence we can choose $t_n$'s satisfying \eqref{3.37} and $t_n\to 0^+$ as $n\to \infty$. This choice implies that $u\in T\left(\Omega,x_0\right)$, i.e., the second inclusion is also proved.
\end{enumerate}
Combining these, we conclude that \eqref{3.33} holds, i.e.,
\begin{align}
T\left( {\Omega ,{x_0}} \right) = \left\{ {\left( {x,y} \right) \in {\mathbb{R}^2}|x \ge 0} \right\}.
\end{align}
\item We have $\nabla f\left( {x,y} \right) = \left( {2x, - 1} \right)$ for all $\left(x,y\right)\in \mathbb{R}^2$. In particular, $\nabla f\left( {{x_0}} \right) = \nabla f\left( {0,0} \right) = \left( {0, - 1} \right)$. Consider $u:=\left(0,1\right)\in T\left(\Omega,x_0\right)$, we have
\begin{align}
\left\langle {\nabla f\left( {{x_0}} \right),u} \right\rangle  = \left\langle {\left( {0, - 1} \right),\left( {0,1} \right)} \right\rangle  =  - 1 < 0.
\end{align}
By the first-order necessary optimality condition, this implies that $x_0$ is not a local minimizer of $f$. 
\end{enumerate}
This completes our solution. \hfill $\square$\\
\\
\textbf{Problem 10.} \textit{Consider the following problem}
\begin{align}
\left( P \right):\hspace{0.5cm}\min x+y^2 \mbox{ s.t. } \left( {x,y} \right) \in \Omega  = \left\{ {\left( {x,y} \right) \in {\mathbb{R}^2}|x - \sqrt {\left| y \right|}  = 0} \right\}.
\end{align}
\begin{enumerate}
\item \textit{Compute the tangent cone of $\Omega$ at the point $x_0=\left(0,0\right)$.}
\item \textit{Apply the first-order sufficient optimality condition, check whether $x_0$ is a strictly local minimizer of first order of $\left(P\right)$ or not.}
\item \textit{Use definition, prove that $x_0$ is a strictly local minimizer of first order of $\left(P\right)$.}
\end{enumerate}
\textsc{Solution.} Setting $X=\mathbb{R}^2$, $f\left(x,y\right)=x+y^2$ for all $\left(x,y\right)\in \mathbb{R}^2$, we notice that $x_0=\left(0,0\right)\in \Omega$.
\begin{enumerate}
\item We claim that
\begin{align}
\label{3.41}
T\left( {\Omega ,{x_0}} \right) = \widehat T\left( {\Omega ,{x_0}} \right): = \left\{ {\left( {x,y} \right) \in {\mathbb{R}^2}|x \ge 0,y = 0} \right\}.
\end{align}
To prove \eqref{3.41}, we prove the following inclusions.
\begin{enumerate}
\item \textit{Prove $T\left( {\Omega ,{x_0}} \right) \subset \widehat T\left( {\Omega ,{x_0}} \right)$.} Taking $u:=\left(x,y\right)\in T\left(\Omega,x_0\right)$, there exist a sequence of positive reals $\left\{ {{t_n}} \right\}_{n = 1}^\infty $ such that $t_n\to 0^+$ and a sequence $\left\{ {{u_n}} \right\}_{n = 1}^\infty  \subset {\mathbb{R}^2}$ such that $u_n\to u$ as $n\to \infty$ and $x_0+t_nu_n\in M$ for all $n\in \mathbb{N}$. Set $u_n:=\left(x_n,y_n\right)$, the fact that $u_n\to u$ implies that $x_n\to x$ and $y_n\to y$, and the fact $x_0+t_nu_n\in M$ for all $n\in \mathbb{N}$ gives
\begin{align}
\label{3.42}
{t_n}{x_n} = \sqrt {\left| {{t_n}{y_n}} \right|} ,\hspace{0.2cm}\forall n \in \mathbb{N}.
\end{align}
We see at a glance from \eqref{3.42} that $x_n\ge 0$ for all $n\in \mathbb{N}$. Hence, $x\ge 0$ (since $x_n\to x$ as $n\to \infty$). Now squaring both sides of \eqref{3.42} yields
\begin{align}
\label{3.43}
t_n^2x_n^2 = {t_n}\left| {{y_n}} \right|,\hspace{0.2cm}\forall n \in \mathbb{N}.
\end{align}
Since $t_n>0$ for all $n\in \mathbb{N}$, \eqref{3.43} then implies
\begin{align}
\label{3.44}
{t_n}x_n^2 = \left| {{y_n}} \right|,\hspace{0.2cm}\forall n \in \mathbb{N}.
\end{align}
Now let $n\to \infty$ in \eqref{3.44} and use the given limits $x_n\to x,y_n\to y$ and $t_n\to 0^+$, we obtain $y=0$. Hence, $u\in \widehat{T}\left(\Omega,x_0\right)$ and our first inclusion is proved.
\item \textit{Prove $\widehat T\left( {\Omega ,{x_0}} \right) \subset T\left( {\Omega ,{x_0}} \right)$.} Taking $u:=\left(x,0\right)$ for which $x\ge 0$, we claim that $u\in T\left(\Omega,x_0\right)$. To this end, we now choose $x_n:=x+\frac{1}{n}, y_n:=\frac{1}{n^3}$ for all $n\in \mathbb{N}$. This choice ensures that $u_n:=\left(x_n,y_n\right)\to u:=\left(x,0\right)$ as $n\to \infty$. It then suffices to prove that there exists a sequence of positive reals $\left\{ {{t_n}} \right\}_{n = 1}^\infty $ such that $t_n\to 0^+$ and $x_0+t_nu_n\in \Omega$ for all $n\in \mathbb{N}$. The latter gives, using \eqref{3.44} again,
\begin{align}
{t_n}{\left( {x + \frac{1}{n}} \right)^2} = \frac{1}{{{n^3}}},\hspace{0.2cm}\forall n \in \mathbb{N}.
\end{align}
i.e.,
\begin{align}
{t_n} = \frac{1}{{{n^3}{{\left( {x + \frac{1}{n}} \right)}^2}}},\hspace{0.2cm}\forall n \in \mathbb{N}.
\end{align}
It is easy to check that $t_n>0$ (since $x\ge 0$) and $t_n\to 0^+$ as $n\to \infty$. Hence, $u\in T\left(\Omega,x_0\right)$ and the second inclusion is also proved.
\end{enumerate}
Combining these inclusions, we conclude that \eqref{3.41} holds, i.e.,
\begin{align}
T\left( {\Omega ,{x_0}} \right) = \left\{ {\left( {x,y} \right) \in {\mathbb{R}^2}|x \ge 0,y = 0} \right\}.
\end{align}
\item Taking $u\in T\left(\Omega,x_0\right)$ satisfying $\left\| u \right\| = 1$, i.e., $u:=\left(x,0\right),x\ge 0$ for which $\left| x \right| = 1$. The only point satisfying these assumptions is $u_0:=\left(1,0\right)$. 

We have $\nabla f\left( {x,y} \right) = \left( {1,2y} \right)$ for all $\left( {x,y} \right) \in {\mathbb{R}^2}$. In particular, $\nabla f\left( {{x_0}} \right) = \nabla f\left( {0,0} \right) = \left( {1,0} \right)$. Thus,
\begin{align}
\label{3.48}
\left\langle {\nabla f\left( {{x_0}} \right),{u_0}} \right\rangle  = \left\langle {\left( {1,0} \right),\left( {1,0} \right)} \right\rangle  = 1 > 0.
\end{align}
By the first-order sufficient optimality condition, \eqref{3.48} implies that $x_0$ is a strictly local minimizer of first order of $\left(P\right)$.
\item By definition 3.1.2, to show that $x_0\in \Omega$ is a strictly local minimizer of first order of $\left(P\right)$, it suffices to prove that there exists a neighborhood $U$ of $x_0$ and a positive real number $\alpha$ such that
\begin{align}
\label{3.49}
x + {y^2} \ge \alpha \sqrt {{x^2} + {y^2}} ,\hspace{0.2cm}\forall \left(x,y\right) \in U \cap \Omega .
\end{align}
We now choose 
\begin{align}
\alpha  &= \frac{1}{\sqrt{2}},\\
U: = B\left( {{x_0};1} \right) &= \left\{ {\left( {x,y} \right) \in {\mathbb{R}^2}|{x^2} + {y^2} \le 1} \right\},
\end{align}
and then prove that \eqref{3.49} holds in this setting. Indeed, since $\left(x,y\right)\in \Omega$, $x\ge 0$ and ${x^2} = \left| y \right|$. Substituting $y^2=x^4$ into \eqref{3.49}, we need to prove
\begin{align}
\label{3.52}
x + {x^4} \ge \frac{1}{\sqrt{2}}\sqrt {{x^2} + {x^4}} ,\hspace{0.2cm}\forall x \in \left[ {0,1} \right].
\end{align}
By squaring both sides,\eqref{3.52} is equivalent to
\begin{align}
2{\left( {1 + {x^3}} \right)^2} \ge 1 + {x^2},\hspace{0.2cm}\forall x \in \left[ {0,1} \right]
\end{align}
i.e.,
\begin{align}
\label{3.54}
1 + 4{x^3} + 2{x^6} \ge {x^2},\hspace{0.2cm}\forall x \in \left[ {0,1} \right].
\end{align}
The last inequality \eqref{3.54} is obviously true, since 
\begin{align}
\mbox{RHS} = {x^2} \le 1 \le 1 + \underbrace {4{x^3} + 2{x^6}}_{ \ge 0} = \mbox{LHS}.
\end{align}
Thus, \eqref{3.49} holds for our setting. 
\end{enumerate}
This completes our proof. \hfill $\square$\\

When we consider second-order optimality conditions (i.e., ``proposed solutions" satisfy first-order optimality conditions), we continue to consider in the ``critical direction'' $u$ satisfying
\begin{align}
\left\langle {\nabla f\left( {{x_0}} \right),u} \right\rangle  = 0.
\end{align}
\textbf{Theorem 3.4 (Second-order necessary optimality condition).} 

\textit{If $x_0$ is a local minimizer of $\left(P\right)$ and $u$ satisfies $\left\langle {\nabla f\left( {{x_0}} \right),u} \right\rangle  = 0$, then we have}
\begin{enumerate}
\item \textit{$\left\langle {\nabla f\left( {{x_0}} \right),w} \right\rangle  + {\nabla ^2}f\left( {{x_0}} \right)\left( {u,u} \right) > 0$ for all $w\in T^2\left(\Omega,x_0,u\right)$.}
\item \textit{$\left\langle {\nabla f\left( {{x_0}} \right),w} \right\rangle  \ge 0$ for all $w\in T''\left(\Omega,x_0,u\right)$.}
\end{enumerate}
\textbf{Theorem 3.5 (Second-order sufficient optimality condition).} 

\textit{If for all $u\in T\left(\Omega,x_0\right)$ for which $\left\| u \right\| = 1$ we have}
\begin{enumerate}
\item \textit{$\left\langle {\nabla f\left( {{x_0}} \right),w} \right\rangle  + {\nabla ^2}f\left( {{x_0}} \right)\left( {u,u} \right) > 0$ for all $w\in T^2\left(\Omega,x_0,u\right)$.}
\item \textit{$\left\langle {\nabla f\left( {{x_0}} \right),w} \right\rangle  > 0$ for all $w\in T''\left(\Omega,x_0,u\right)$ for which $\left\| w \right\| = 1$.}
\end{enumerate}
\textit{then $x_0$ is a strictly local minimizer of second order of $\left(P\right)$.}\\

Here are some problems to apply second-order optimality conditions.\\
\\
\textbf{Problem 11.} \textit{Consider the following problem}
\begin{align}
\left( P \right)\hspace{0.5cm}\min {x^3} - {y^2}\mbox{ s.t. }\left( {x,y} \right) \in \Omega  = \left\{ {\left( {x,y} \right) \in {\mathbb{R}^2}|xy \ge 0} \right\}.
\end{align}
\begin{enumerate}
\item \textit{Compute the tangent cone of $\Omega$ at $x_0=\left(0,0\right)$.}
\item \textit{Apply the first-order necessary optimality condition, check if $x_0=\left(0,0\right)$ is a local minimizer of $\left(P\right)$ or not.}
\item \textit{Apply the second-order necessary optimality condition, check if $x_0=\left(0,0\right)$ is a local minimizer of $\left(P\right)$ or not.}
\end{enumerate}
\textsc{Solution.} Setting $Xx=\mathbb{R}^2$, $f\left(x,y\right)=x^3-y^2$ for all $\left(x,y\right)\in \mathbb{R}^2$, we notice that $x_0=\left(0,0\right)\in \Omega$.
\begin{enumerate}
\item We claim that
\begin{align}
\label{3.58}
T\left( {\Omega ,{x_0}} \right) = \Omega  = \left\{ {\left( {x,y} \right) \in {\mathbb{R}^2}|xy \ge 0} \right\}
\end{align}
To prove \eqref{3.58}, we prove the following inclusions.
\begin{enumerate}
\item \textit{Prove $T\left( {\Omega ,{x_0}} \right) \subset \Omega $.} Taking $u:=\left(x,y\right)\in T\left(\Omega,x_0\right)$, there exist a sequence of positive reals $\left\{ {{t_n}} \right\}_{n = 1}^\infty $ such that $t_n\to 0^+$ and a sequence $\left\{ {{u_n}} \right\}_{n = 1}^\infty  \subset {\mathbb{R}^2}$ such that $u_n\to u$ as $n\to \infty$ and $x_0+t_nu_n\in M$ for all $n\in \mathbb{N}$. Set $u_n:=\left(x_n,y_n\right)$, the fact that $u_n\to u$ implies that $x_n\to x$ and $y_n\to y$, and the fact $x_0+t_nu_n\in M$ for all $n\in \mathbb{N}$ gives
\begin{align}
\label{3.59}
t_n^2{x_n}{y_n} \ge 0,\hspace{0.2cm}\forall n \in \mathbb{N}.
\end{align}
Since $t_n\ge 0$ for all $n\in \mathbb{N}$, \eqref{3.59} then implies
\begin{align}
\label{3.60}
{x_n}{y_n} \ge 0,\hspace{0.2cm}\forall n \in \mathbb{N}.
\end{align}
Now let $n\to \infty$ in \eqref{3.60} and use the given limits $x_n\to x, y_n\to y$ and $t_n\to 0^+$, we obtain $xy\ge 0$. Hence, $u\in \Omega$ and our first inclusion is proved.
\item \textit{Prove $\Omega  \subset T\left( {\Omega ,{x_0}} \right)$.} Taking $u:=\left(x,y\right)\in \Omega$, i.e., $xy\ge 0$, we claim that $u\in T\left(\Omega,x_0\right)$. To this end, we consider the following cases depending on the common sign of $x$ and $y$.
\begin{itemize}
\item \textit{Case $x\ge 0,y\ge 0$.} We can choose $x_n:=x+\frac{1}{n},y_n:=y+\frac{1}{n}$ for all $n\in \mathbb{N}$ and an arbitrary sequence of positive reals $t_n$'s such that $t_n\to 0^+$ as $n\to \infty$. These choices will ensure that \eqref{3.59} holds. Thus, $u\in T\left(\Omega,x_0\right)$ in this case.
\item \textit{Case $x\le 0,y\le 0$.} Similarly, we can choose that $x_n:=x-\frac{1}{n},y_n:=y-\frac{1}{n}$ for all $n\in \mathbb{N}$ and an arbitrary sequence of positive reals $t_n$'s such that $t_n\to 0^+$ as $n\to \infty$. Hence, we also deduce that $u\in T\left(\Omega,x_0\right)$ in this case.
\end{itemize}
What we have just proved is the second inclusion.
\end{enumerate}
Combining these inclusions, we conclude that \eqref{3.58} holds.
\item We have $\nabla f\left( {x,y} \right) = \left( {3{x^2}, - 2y} \right)$ for all $\left(x,y\right)\in \mathbb{R}^2$. In particular, $\nabla f\left( {{x_0}} \right) = \nabla f\left( {0,0} \right) = \left( {0,0} \right)$. Thus, 
\begin{align}
\label{3.61}
\left\langle {\nabla f\left( {{x_0}} \right),u} \right\rangle  = 0,\hspace{0.2cm}\forall u \in \Omega  \equiv T\left( {\Omega ,{x_0}} \right),
\end{align}
which satisfies the conclusion of Theorem 3.2. However, we can not deduce from \eqref{3.61} that $x_0$ is a local minimizer of $\left(P\right)$. 
\item We claim that $x_0$ is not a local minimizer of $\left(P\right)$. Suppose for the contrary that $x_0$ is a local minimizer of $\left(P\right)$, we have $\left\langle {\nabla f\left( {{x_0}} \right),u} \right\rangle  = 0$ for all $u\in \Omega$ by above argument. Compute
\begin{align}
{\nabla ^2}f\left( {x,y} \right) = \left( {\begin{array}{*{20}{c}}
{6x}&0\\
0&{ - 2}
\end{array}} \right),\forall \left( {x,y} \right) \in {\mathbb{R}^2}.
\end{align}
In particular, 
\begin{align}
{\nabla ^2}f\left( {{x_0}} \right) = {\nabla ^2}f\left( {0,0} \right) = \left( {\begin{array}{*{20}{c}}
0&0\\
0&{ - 2}
\end{array}} \right).
\end{align}
Denote $u:=\left(x,y\right)\in \Omega$, we have
\begin{align}
\left\langle {\nabla f\left( {{x_0}} \right),w} \right\rangle  + {\nabla ^2}f\left( {{x_0}} \right)\left( {u,u} \right) &= \left( {\begin{array}{*{20}{c}}
x&y
\end{array}} \right)\left( {\begin{array}{*{20}{c}}
0&0\\
0&{ - 2}
\end{array}} \right)\left( {\begin{array}{*{20}{c}}
x\\
y
\end{array}} \right)\\
 &=  - 2{y^2} \le 0, 
\end{align}
for all $w\in T^2\left(\Omega,x_0,u\right)$, which contradicts to the second-order necessary optimality condition. Therefore, $x_0$ is not a local minimizer of $\left(P\right)$.
\end{enumerate}
This completes our proof. \hfill $\square$\\
\\
\textbf{Problem 12.} \textit{Consider the following problem}
\begin{align}
\left( P \right)\hspace{0.5cm}\min {x^3} + {y^2}\mbox{ s.t. } \left( {x,y} \right) \in \Omega  = \left\{ {\left( {x,y} \right) \in {\mathbb{R}^2}|x - {y^2} \ge 0} \right\}.
\end{align}
\begin{enumerate}
\item \textit{Compute the tangent cone of $\Omega$ at $x_0=\left(0,0\right)$.}
\item \textit{Prove that $x_0$ satisfies the first-order necessary optimality condition.}
\item \textit{Use definition, prove that $x_0$ is not a strictly local minimizer of first order of $\left(P\right)$.}
\item \textit{Use definition, check if $x_0$ is a strictly local minimizer of second order of $\left(P\right)$.}
\item \textit{Apply the second-order necessary optimality condition, check if $x_0=\left(0,0\right)$ is a strictly local minimizer of second order of $\left(P\right)$ or not.}
\end{enumerate}
\textsc{Solution.} Setting $X=\mathbb{R}^2$, $f\left(x,y\right)=x^3+y^2$ for all $\left(x,y\right)\in \mathbb{R}^2$. We notice that $x_0=\left(0,0\right)\in \Omega$.
\begin{enumerate}
\item We claim that 
\begin{align}
\label{3.67}
T\left( {\Omega ,{x_0}} \right) = \widehat T\left( {\Omega ,{x_0}} \right): = \left\{ {\left( {x,y} \right) \in {\mathbb{R}^2}|x \ge 0} \right\}.
\end{align}
To prove \eqref{3.67}, we prove the following inclusions.
\begin{enumerate}
\item \textit{Prove $T\left( {\Omega ,{x_0}} \right) \subset \widehat T\left( {\Omega ,{x_0}} \right)$.} Taking $u:=\left(x,y\right)\in T\left(\Omega,x_0\right)$, there exist a sequence of positive reals $\left\{ {{t_n}} \right\}_{n = 1}^\infty $ such that $t_n\to 0^+$ and a sequence $\left\{ {{u_n}} \right\}_{n = 1}^\infty  \subset {\mathbb{R}^2}$ such that $u_n\to u$ as $n\to \infty$ and $x_0+t_nu_n\in M$ for all $n\in \mathbb{N}$. Set $u_n:=\left(x_n,y_n\right)$, the fact that $u_n\to u$ implies that $x_n\to x$ and $y_n\to y$, and the fact $x_0+t_nu_n\in M$ for all $n\in \mathbb{N}$ gives
\begin{align}
\label{3.68}
{t_n}{x_n} \ge t_n^2y_n^2,\hspace{0.2cm}\forall n \in \mathbb{N}.
\end{align}
Since $t_n>0$ for all $n\in \mathbb{N}$, \eqref{3.68} implies that
\begin{align}
\label{3.69}
{x_n} \ge {t_n}y_n^2,\hspace{0.2cm}\forall n \in \mathbb{N}.
\end{align}
Let $n\to \infty$ in \eqref{3.69}, we obtain $x\ge 0$. Hence, $u\in \widehat{T}\left(\Omega,x_0\right)$ and our first inclusion is proved.
\item \textit{Prove $\widehat T\left( {\Omega ,{x_0}} \right) \subset T\left( {\Omega ,{x_0}} \right)$.} Taking $u:=\left(x,y\right)$ for which $x\ge 0$, we claim that $u\in T\left(\Omega,x_0\right)$. To this end, we choose $x_n:=x+\frac{1}{n},y_n:=y$ for all $n\in \mathbb{N}$. If $y=0$ then \eqref{3.69} obviously holds, $x+\frac{1}{n}\ge 0$ for all $n\in \mathbb{N}$, and we can choose an arbitrary sequence of positive reals $t_n$'s for which $t_n\to 0^+$. If $y\ne 0$, \eqref{3.69} gives
\begin{align}
\label{3.70}
{t_n} \le \frac{1}{{{y^2}}}\left( {x + \frac{1}{n}} \right),\hspace{0.2cm}\forall n \in \mathbb{N}.
\end{align}
The right-hand side of \eqref{3.70} is positive. Thus, we can choose an arbitrary sequence of positive reals $t_n$'s satisfying \eqref{3.70} and $t_n\to 0^+$ as $n\to \infty$. In both cases, we deduce that $u\in T\left(\Omega,x_0\right)$ and our inclusion is also proved.
\end{enumerate}
Combining these inclusions, we conclude that \eqref{3.67} holds, i.e.,
\begin{align}
T\left( {\Omega ,{x_0}} \right) = \left\{ {\left( {x,y} \right) \in {\mathbb{R}^2}|x \ge 0} \right\}.
\end{align}
\item We have $\nabla f\left( {x,y} \right) = \left( {3{x^2},2y} \right)$ for all $\left(x,y\right)\in \mathbb{R}^2$. In particular, $\nabla f\left( {{x_0}} \right) = \nabla f\left( {0,0} \right) = \left( {0,0} \right)$. We then have
\begin{align}
\left\langle {\nabla f\left( {{x_0}} \right),u} \right\rangle  = \left\langle {\left( {0,0} \right),\left( {x,y} \right)} \right\rangle  = 0,\hspace{0.2cm} \forall u\in T\left(\Omega,x_0\right),
\end{align}
i.e., $x_0$ satisfies the first-order necessary optimality condition.
\item \textsc{Solution 1.} Suppose for the contrary that $x_0$ is a strictly local minimizer of first order of $\left(P\right)$, by definition 3.1, there exist a neighborhood $U$ of $x_0$ and a positive real number $\alpha$ such that
\begin{align}
\label{3.73}
{x^3} + {y^2} \ge \alpha \sqrt {{x^2} + {y^2}} ,\hspace{0.2cm}\forall \left( {x,y} \right) \in U \cap \Omega .
\end{align}
For $x>0$ small enough, $\left( {x,\sqrt{x}} \right) \in U \cap \Omega $ holds. Then \eqref{3.73} gives
\begin{align}
\alpha  \le \frac{{{x^3} + x}}{{\sqrt {{x^2} + x} }} = \sqrt x \cdot \frac{{{x^2} + 1}}{{\sqrt {x + 1} }} \to 0\mbox{ as } x \to 0,
\end{align}
which contradicts the positivity of $\alpha$. This contradiction implies that $x_0$ is not a strictly local minimizer of first order of $\left(P\right)$.\\
\\
\textsc{Solution 2.} Similarly, for $x>0$ small enough, $\left( x,0 \right) \in U \cap \Omega $ holds. Then \eqref{3.73} gives
\begin{align}
\alpha  \le {x^2} \to 0 \mbox{ as } x \to 0,
\end{align}
which also contradicts the positivity of $\alpha$. Therefore, $x_0$ is not a strictly local minimizer of first order of $\left(P\right)$.
\item Similarly, suppose for the contrary that $x_0$ is a strictly local minimizer of second order of $\left(P\right)$, by definition 3.1, there exists a neighborhood $U$ of $x_0$ and a positive real number $\alpha$ such that
\begin{align}
\label{3.76}
{x^3} + {y^2} \ge \alpha \left( {{x^2} + {y^2}} \right),\hspace{0.2cm}\forall \left( {x,y} \right) \in U \cap \Omega .
\end{align}
For $x>0$ small enough, $\left( x,0 \right) \in U \cap \Omega $ holds. Then \eqref{3.76} gives
\begin{align}
\alpha  \le x \to 0\mbox{ as } x \to 0,
\end{align}
which contradicts the positivity of $\alpha$. Therefore, $x_0$ is not a strictly local minimizer of second order of $\left(P\right)$.
\item We claim that $x_0$ is not a strictly local minimizer of second order of $\left(P\right)$. To show this, we suppose for the contrary that $x_0$ is a strictly local minimizer of second order of $\left(P\right)$, and thus, is a local minimizer of $\left(P\right)$. We has showed that $\left\langle {\nabla f\left( {{x_0}} \right),u} \right\rangle  = 0$ for all $u \in \Omega$ in $\left(2\right)$. Thus,
\begin{align}
\label{3.78}
\left\langle {\nabla f\left( {{x_0}} \right),w} \right\rangle  + {\nabla ^2}f\left( {{x_0}} \right)\left( {u,u} \right) &= \left( {\begin{array}{*{20}{c}}
x&y
\end{array}} \right)\left( {\begin{array}{*{20}{c}}
0&0\\
0&2
\end{array}} \right)\left( {\begin{array}{*{20}{c}}
x\\
y
\end{array}} \right)\\
& = 2{y^2}, \label{3.79}
\end{align}
for all $u:=\left(x,y\right) \in \Omega$ and for all $w\in T^2\left(\Omega,x_0,u\right)$. Hence, for $u=\left(x,0\right),x\ge 0$, \eqref{3.78}-\eqref{3.79} gives
\begin{align}
\left\langle {\nabla f\left( {{x_0}} \right),w} \right\rangle  + {\nabla ^2}f\left( {{x_0}} \right)\left( {u,u} \right) = 0,
\end{align}
which contradicts the second-order necessary condition. This contradiction illustrates that $x_0$ is not a strictly local minimizer of second order of $\left(P\right)$.
\end{enumerate}
This ends our proof. \hfill $\square$\\
\\
\\
\\
\begin{center}
\textsc{The End}
\end{center}
\newpage
\begin{thebibliography}{999}
\bibitem {1} O. G\"{u}ler, \textit{Foundations of Optimization}, Graduate Texts in Mathematics 258, Springer.
\bibitem {2} Tim Hoheisel, \textit{Convex Analysis}, University of W\"{u}rzburg, Germany, Lecture Notes, Summer term 2016.
\bibitem {3} Akhtar A. Khan, Christiane Tammer, Constantin Z\u{a}linescu, \textit{Set-valued Optimization, An Introduction with Applications}, Vector Optimization, Springer.
\bibitem {4} G. Giorgi, B. Jim\'{e}nez, V. Novo, \textit{An Overview of Second Order Tangent Sets and Their Application to Vector Optimization}, Bol. Soc. Esp. Mat. Apl. no \textbf{52} (2010), 73–96.
\bibitem {5} B. Jim\'{e}nez, V. Novo, \textit{Optimality conditions in differentiable vector optimization via second order tangent sets}, Appl. Math. Optim., \textbf{49} (2004), no. 2, 123-144.
\end{thebibliography}
\end{document}