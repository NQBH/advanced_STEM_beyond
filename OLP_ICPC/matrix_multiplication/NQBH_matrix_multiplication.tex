\documentclass{article}
\usepackage[backend=biber,natbib=true,style=alphabetic,maxbibnames=50]{biblatex}
\addbibresource{/home/nqbh/reference/bib.bib}
\usepackage[utf8]{vietnam}
\usepackage{tocloft}
\renewcommand{\cftsecleader}{\cftdotfill{\cftdotsep}}
\usepackage[colorlinks=true,linkcolor=blue,urlcolor=red,citecolor=magenta]{hyperref}
\usepackage{amsmath,amssymb,amsthm,enumitem,fancyvrb,float,graphicx,mathtools,tikz}
\usetikzlibrary{angles,calc,intersections,matrix,patterns,quotes,shadings}
\allowdisplaybreaks
\newtheorem{assumption}{Assumption}
\newtheorem{baitoan}{Bài toán}
\newtheorem{cauhoi}{Câu hỏi}
\newtheorem{conjecture}{Conjecture}
\newtheorem{corollary}{Corollary}
\newtheorem{dangtoan}{Dạng toán}
\newtheorem{definition}{Definition}
\newtheorem{dinhly}{Định lý}
\newtheorem{dinhnghia}{Định nghĩa}
\newtheorem{example}{Example}
\newtheorem{ghichu}{Ghi chú}
\newtheorem{hequa}{Hệ quả}
\newtheorem{hypothesis}{Hypothesis}
\newtheorem{lemma}{Lemma}
\newtheorem{luuy}{Lưu ý}
\newtheorem{nhanxet}{Nhận xét}
\newtheorem{notation}{Notation}
\newtheorem{note}{Note}
\newtheorem{principle}{Principle}
\newtheorem{problem}{Problem}
\newtheorem{proposition}{Proposition}
\newtheorem{question}{Question}
\newtheorem{remark}{Remark}
\newtheorem{theorem}{Theorem}
\newtheorem{vidu}{Ví dụ}
\usepackage[left=1cm,right=1cm,top=5mm,bottom=5mm,footskip=4mm]{geometry}
\def\labelitemii{$\circ$}
\DeclareRobustCommand{\divby}{%
    \mathrel{\vbox{\baselineskip.65ex\lineskiplimit0pt\hbox{.}\hbox{.}\hbox{.}}}%
}
\setlist[itemize]{leftmargin=*}
\setlist[enumerate]{leftmargin=*}

\title{Matrix Multiplication {\it\&} Fast Doubling Techniques in Competitive Programming}
\author{Nguyễn Quản Bá Hồng\footnote{A scientist- {\it\&} creative artist wannabe, a mathematics {\it\&} computer science lecturer of Department of Artificial Intelligence {\it\&} Data Science (AIDS), School of Technology (SOT), UMT Trường Đại học Quản lý {\it\&} Công nghệ TP.HCM, Hồ Chí Minh City, Việt Nam.\\E-mail: {\sf nguyenquanbahong@gmail.com} {\it\&} {\sf hong.nguyenquanba@umt.edu.vn}. Website: \url{https://nqbh.github.io/}. GitHub: \url{https://github.com/NQBH}.}}
\date{\today}

\begin{document}
\maketitle
\begin{abstract}
    This text is a part of the series {\it Some Topics in Advanced STEM \& Beyond}:

    {\sc url}: \url{https://nqbh.github.io/advanced_STEM/}.

    Latest version:
    \begin{itemize}
        \item {\it }.

        PDF: {\sc url}: \url{.pdf}.

        \TeX: {\sc url}: \url{.tex}.
        \item {\it }.

        PDF: {\sc url}: \url{.pdf}.

        \TeX: {\sc url}: \url{.tex}.
    \end{itemize}
\end{abstract}
\tableofcontents

%------------------------------------------------------------------------------%

\section{Introduction}
\textbf{\textsf{Resources -- Tài nguyên.}}
\begin{enumerate}
    \item {\sc Benjamin Qi, Harshini Rayasam, Neo Wang, Peng Bai}. \href{https://usaco.guide/plat/matrix-expo?lang=cpp}{USACO Guide{\tt/}matrix exponentatiation}.

    \item \href{https://codeforces.com/blog/entry/67776}{CodeForces{\tt/}lazyneuron{\tt/}a complete guide on matrix exponentiation}.

    \item
\end{enumerate}

\begin{problem}[\href{https://cses.fi/problemset/task/1722}{CSES Problem Set{\tt/}Fibonacci numbers}]
    The Fibonacci numbers can be defined as follows:
    \begin{equation}
        \label{Fibonacci}
        F_0 = 0,\ F_1 = 1,\ F_n = F_{n - 2} + F_{n - 1},\ \forall n\in\mathbb{N},\,n\ge2.
    \end{equation}
    Calculate the value of $F_n$ for a given $n$.
    \item {\sf Input.} The only input line has an integer $n$.
    \item {\sf Output.} Print the value of $F_n\mod(10^9 + 7)$.
    \item {\sf Constraints.} $0\le n\le10^{18}$.
\end{problem}

\begin{proof}[Solution]
    Đặt
    \begin{equation*}
        A\coloneqq\begin{bmatrix}
            1 & 1\\1 & 0
        \end{bmatrix}\in{\cal M}_2(\mathbb{Z}),
    \end{equation*}
    ta chứng minh
    \begin{equation}
        \label{matrix Fibonacci}
        A^n = \begin{bmatrix}
            1 & 1\\1 & 0
        \end{bmatrix}^n = \begin{bmatrix}
            F_{n + 1} & F_n\\ F_n & F_{n - 1}
        \end{bmatrix},\ \forall n\in\mathbb{N}^\star.
    \end{equation}
    Trường hợp cơ sở hiển nhiên đúng:
     \begin{equation*}
         A^1 = A = \begin{bmatrix}
             1 & 1\\1 & 0
         \end{bmatrix} = \begin{bmatrix}
             F_2 & F_1\\ F_1 & F_0
         \end{bmatrix}.
     \end{equation*}
     Bước chuyển quy nạp từ $n$ sang $n + 1$:
     \begin{equation*}
         A^{n + 1} = AA^n = \begin{bmatrix}
             1 & 1\\1 & 0
         \end{bmatrix}\begin{bmatrix}
            F_{n + 1} & F_n\\ F_n & F_{n - 1}
         \end{bmatrix} = \begin{bmatrix}
            F_{n + 1} + F_n & F_n + F_{n - 1}\\ F_{n + 1} & F_n
         \end{bmatrix} = \begin{bmatrix}
            F_{n + 2} & F_{n + 1}\\ F_{n + 1} & F_n
         \end{bmatrix},
     \end{equation*}
     suy ra \eqref{matrix Fibonacci} đúng theo nguyên lý quy nạp toán học.
     \item {\sf C++ implementation.}
     \begin{Verbatim}[numbers=left,xleftmargin=0mm]
#include <bits/stdc++.h>
using namespace std;
using ll = long long;
using Matrix = array<array<ll, 2>, 2>;
const ll MOD = 1e9 + 7;

Matrix mul(Matrix a, Matrix b) {
    Matrix res = {{{0, 0}, {0, 0}}};
    for (int i = 0; i < 2; ++i)
        for (int j = 0; j < 2; ++j)
            for (int k = 0; k < 2; ++k) {
                res[i][j] += a[i][k] * b[k][j];
                res[i][j] %= MOD;
            }
    return res;
}

int main() {
    ios_base::sync_with_stdio(false);
    cin.tie(nullptr);
    ll n;
    cin >> n;
    Matrix base = {{{1, 0}, {0, 1}}}, m = {{{1, 1}, {1, 0}}};
    for (; n > 0; n /= 2, m = mul(m, m))
        if (n & 1) base = mul(base, m);
    cout << base[0][1];
}
     \end{Verbatim}
\end{proof}
Ta có thể mở rộng bài toán này bằng cách mở rộng \eqref{Fibonacci} cho dãy dãy $\{f_n\}_{n\in\mathbb{N}}$ được định nghĩa bởi công thức truy hồi:
\begin{equation*}
    f_0 = 0,\ f_1 = 1,\ f_n = af_{n - 1} + f_{n - 2},\ \forall n\in\mathbb{N},\,n\ge2,
\end{equation*}
bằng cách đặt
\begin{equation*}
    A\coloneqq\begin{bmatrix}
        a & b\\1 & 0
    \end{bmatrix},
\end{equation*}
thì chứng minh được bằng quy nạp



\begin{baitoan}
    Cho 1 quan hệ hồi quy tuyến tính có dạng
    \begin{equation*}
        f_n = \sum_{i=1}^k c_if_{n - i},\ \forall n\in\mathbb{N},\ n\ge k.
    \end{equation*}
\end{baitoan}



%------------------------------------------------------------------------------%

\section{Miscellaneous}

%------------------------------------------------------------------------------%

\printbibliography[heading=bibintoc]

\end{document}