\documentclass{article}
\usepackage[backend=biber,natbib=true,style=alphabetic,maxbibnames=50]{biblatex}
\addbibresource{/home/nqbh/reference/bib.bib}
\usepackage[utf8]{vietnam}
\usepackage{tocloft}
\renewcommand{\cftsecleader}{\cftdotfill{\cftdotsep}}
\usepackage[colorlinks=true,linkcolor=blue,urlcolor=red,citecolor=magenta]{hyperref}
\usepackage{amsmath,amssymb,amsthm,enumitem,float,graphicx,mathtools,tikz}
\usetikzlibrary{angles,calc,intersections,matrix,patterns,quotes,shadings}
\allowdisplaybreaks
\newtheorem{assumption}{Assumption}
\newtheorem{baitoan}{Bài toán}
\newtheorem{cauhoi}{Câu hỏi}
\newtheorem{conjecture}{Conjecture}
\newtheorem{corollary}{Corollary}
\newtheorem{dangtoan}{Dạng toán}
\newtheorem{definition}{Definition}
\newtheorem{dinhly}{Định lý}
\newtheorem{dinhnghia}{Định nghĩa}
\newtheorem{example}{Example}
\newtheorem{ghichu}{Ghi chú}
\newtheorem{hequa}{Hệ quả}
\newtheorem{hypothesis}{Hypothesis}
\newtheorem{lemma}{Lemma}
\newtheorem{luuy}{Lưu ý}
\newtheorem{nhanxet}{Nhận xét}
\newtheorem{notation}{Notation}
\newtheorem{note}{Note}
\newtheorem{principle}{Principle}
\newtheorem{problem}{Problem}
\newtheorem{proposition}{Proposition}
\newtheorem{question}{Question}
\newtheorem{remark}{Remark}
\newtheorem{theorem}{Theorem}
\newtheorem{vidu}{Ví dụ}
\usepackage[left=1cm,right=1cm,top=5mm,bottom=5mm,footskip=4mm]{geometry}
\def\labelitemii{$\circ$}
\DeclareRobustCommand{\divby}{%
	\mathrel{\vbox{\baselineskip.65ex\lineskiplimit0pt\hbox{.}\hbox{.}\hbox{.}}}%
}
\setlist[itemize]{leftmargin=*}
\setlist[enumerate]{leftmargin=*}

\title{Lecture Note: Information Technology Fundamentals\\Bài Giảng: Nền Tảng Công Nghệ Thông Tin}
\author{Nguyễn Quản Bá Hồng\footnote{A scientist- {\it\&} creative artist wannabe, a mathematics {\it\&} computer science lecturer of Department of Artificial Intelligence {\it\&} Data Science (AIDS), School of Technology (SOT), UMT Trường Đại học Quản lý {\it\&} Công nghệ TP.HCM, Hồ Chí Minh City, Việt Nam.\\E-mail: {\sf nguyenquanbahong@gmail.com} {\it\&} {\sf hong.nguyenquanba@umt.edu.vn}. Website: \url{https://nqbh.github.io/}. GitHub: \url{https://github.com/NQBH}.}}
\date{\today}

\begin{document}
\maketitle
\begin{abstract}
	This text is a part of the series {\it Some Topics in Advanced STEM \& Beyond}:
	
	{\sc url}: \url{https://nqbh.github.io/advanced_STEM/}.
	
	Latest version:
	\begin{itemize}
		\item {\it Lecture Note: Information Technology Fundamentals -- Bài Giảng: Nền Tảng Công Nghệ Thông Tin}.
		
		PDF: {\sc url}: \url{https://github.com/NQBH/advanced_STEM_beyond/blob/main/IT_fundamentals/lecture/NQBH_IT_fundamentals_lecture.pdf}.
		
		\TeX: {\sc url}: \url{https://github.com/NQBH/advanced_STEM_beyond/blob/main/IT_fundamentals/lecture/NQBH_IT_fundamentals_lecture.tex}.
		\item {\it Codes}.
		
		PDF: {\sc url}: \url{.pdf}.
		
		\TeX: {\sc url}: \url{.tex}.
	\end{itemize}
\end{abstract}
\tableofcontents

%------------------------------------------------------------------------------%

\section{Basic}
\textbf{\textsf{Resources -- Tài nguyên.}}
\begin{enumerate}
	\item \cite{Ha_Python_co_ban}. {\sc Bùi Việt Hà}. {\it Python Cơ Bản}.
	\item \cite{Ha_loi_giai_BT_Python_co_ban}. {\sc Bùi Việt Hà}. {\it Lời Giải Bài Tập Python Cơ Bản}.
	\item \cite{Ha_Python_nang_cao}. {\sc Bùi Việt Hà}. {\it Python Nâng Cao}.
	\item \cite{Matthes2019,Matthes2023}. {\sc Eric Matthes}. {\it Python Crash Course: A Hands-on, Project-Based Introduction to Programming}.
\end{enumerate}

\subsection{Basic guides}

\begin{itemize}
	\item Truy cập trang web Google Colab: \url{https://colab.research.google.com}.
	\item {\it Cách tạo 1 file Jupyter notebook mới}: File $\to$ New notebook in Drive.
	\item Thêm ghi chú: Nhấp vào {\tt+ Text} $\to$ Gõ ghi chú, comment $\to$ Shift Enter.
	\item Thêm code: Nhấp vào {\tt+ Code} $\to$ Gõ code vào $\to$ Run.
\end{itemize}

\begin{itemize}
	\item {\tt int}: integer: kiểu dữ liệu số nguyên $\mathbb{Z} = \{0,\pm1,\pm2,\ldots\}$.
	\item {\tt float}: real number: integer: kiểu dữ liệu số thực $\mathbb{R}$.
\end{itemize}

\begin{baitoan}
	Viết chương trình Python để tính tổng, hiệu, tích, thương của 2 số $a,b\in\mathbb{R}$ được nhập từ bàn phím.
\end{baitoan}

\begin{proof}
	Python:
	\begin{verbatim}
a = float(input("a = "))
b = float(input("b = "))
print("Sum a + b = ", a + b)
print("Difference a - b = ", a - b)
print("Product ab = ", a * b)
if b == 0:
    print("Division by zero error")
else:
    print("Quotient a/b = ", a / b)
	\end{verbatim}
\end{proof}

\begin{remark}
	Nếu không xét trường hợp $b = 0$ thì sẽ bị lỗi chia cho $0$:
	\begin{verbatim}
Traceback (most recent call last):
  File "/home/nqbh/advanced_STEM_beyond/IT_fundamentals/Python/basic.py", line 6, in <module>
    print("Quotient a/b = ", a / b)
                             ~~^~~
ZeroDivisionError: float division by zero
	\end{verbatim}
\end{remark}

\begin{baitoan}[Even \& odd -- Chẵn \& lẻ. +0.5]
	Viết chương trình Python để xét tính chẵn lẻ của 1 số $a\in\mathbb{Z}$ được nhập từ bàn phím.
\end{baitoan}

\begin{baitoan}[Divisible by -- Tính chia hết, +0.5]
	Viết chương trình Python để xét xem $a\in\mathbb{Z}$ có chia hết cho $b\in\mathbb{Z}$ không, với $a,b$ được nhập từ bàn phím. Nếu có thông báo {\tt a is divisible by b}, nếu không thì in số dư $r$ của phép chia $a$ cho $b$, với $0\le r < |b|$.
\end{baitoan}

\begin{baitoan}
	Cho $n\in\mathbb{N}^\star$ được nhập từ bàn phím. Viết chương trình Python để tính: (a) Tổng của $n$ số nguyên dương đầu tiên: $\sum_{i=1}^n i = 1 + 2 + \cdots + n$ \& so sánh với $\dfrac{n(n + 1)}{2}$. (b) Tổng của $n$ số nguyên dương lẻ đầu tiên: $\sum_{i=1}^n (2i - 1) = 1 + 3 + 5 + \cdots + (2n - 1)$  \& so sánh với $n^2$. (c) Tổng của $n$ số lẻ nguyên dương chẵn đầu tiên: $\sum_{i=1}^n 2i = 2 + 4 + 6 + \cdots + 2n$ \& so sánh với $n(n + 1)$. (d) Tổng bình phương của $n$ số nguyên dương đầu tiên: $\sum_{i=1}^n i^2 = 1^2 + 2^2 + \cdots + n^2$ \& so sánh với $\dfrac{n(n + 1)(2n + 1)}{6}$. (e) Tổng bình phương của $n$ số nguyên dương lẻ đầu tiên: $\sum_{i=1}^n (2i - 1)^2 = 1^2 + 3^2 + 5^2 + \cdots + (2n - 1)^2$. (f) Tổng bình phương của $n$ số lẻ nguyên dương chẵn đầu tiên: $\sum_{i=1}^n (2i)^2 = 2^2 + 4^2 + 6^2 + \cdots + (2n)^2$ \& so sánh với $\dfrac{2n(n + 1)(2n + 1)}{3}$. (g) Tổng lập phương của $n$ số nguyên dương đầu tiên: $\sum_{i=1}^n i^3 = 1^3 + 2^3 + \cdots + n^3$ \& so sánh với $\dfrac{n^2(n + 1)^2}{4}$. (e) Tổng lập phương của $n$ số nguyên dương lẻ đầu tiên: $\sum_{i=1}^n (2i - 1)^3 = 1^3 + 3^3 + 5^3 + \cdots + (2n - 1)^3$. (f) Tổng lập phương của $n$ số nguyên dương chẵn đầu tiên: $\sum_{i=1}^n (2i)^3 = 2^3 + 4^3 + 6^3 + \cdots + (2n)^3$ \& so sánh với $2n^2(n + 1)^2$.
\end{baitoan}

\begin{baitoan}[Triangle -- Tam giác]
	 Viết chương trình Python để xét 3 số $a,b,c\in(0,\infty)$ được nhập từ bàn phím có phải là: (a) 3 cạnh của 1 tam giác hay không nhờ bất đẳng thức 3 cạnh tam giác $a < b + c,b < c + a,c < a + b$. (b) Nếu $a,b,c$ là 3 cạnh tam giác, phân loại tam giác đó: tam giác nhọn, tam giác vuông, tam giác tù, tam giác cân, tam giác đều, tam giác vuông cân.
\end{baitoan}

\begin{baitoan}
	Viết chương trình tính chu vi, diện tích, 3 đường cao của 1 tam giác với độ dài 3 cạnh $a,b,c\in(0,\infty)$ được nhập từ bàn phím.
\end{baitoan}

\begin{baitoan}
	Cho 1 mảng số thực $a_1,\ldots,a_n$ được nhập từ bàn phím ứng với số tiền thu được mỗi tháng, trong đó $a_i < 0$: lỗ, $a_i = 0$: huề vốn, $a_i > 0$: lời{\tt/}lãi. Viết chương trình Python xuất ra màn hình: (a) Số tháng lời, lỗ, huề vốn. (b) Tổng số tiền lời, tổng số tiền lỗ, tổng số thu nhập cuối cùng (sau khi lấy tổng số tiền lời $-$ tổng số tiền lỗ).
	\item {\sf Input.} Dãy số thực $a_1,\ldots,a_n$.
	\item {\sf Output.} Số tháng lời, số tháng lỗ, số tháng huề vốn. Dòng tiếp theo: Tổng số tiền lời, tổng số tiền lỗ.
	\item {\sf Sample.}
	\begin{table}[H]
		\centering
		\begin{tabular}{|l|l|}
			\hline
			{\tt money.inp} & {\tt money.out} \\
			\hline
			10.5 -2.3 3.6 4.5 -7.41 0 1.23 & 4 2 1 \\
			& 19.83 $-9.71$ \\
			& 10.12 \\
			\hline
		\end{tabular}
	\end{table}
\end{baitoan}

\begin{baitoan}[\cite{CDHT_Toan_10_CD}, VD3, p. 25, tính tiền vốn lẫn lãi nếu không rút tiền ra]
	1 người gửi số tiền $m_0\in(0,\infty)$ đồng vào ngân hàng với lãi suất $r\%${\tt/}năm. Biết nếu không rút tiền ra khỏi ngân hàng thì cứ sau mỗi năm, số tiền lãi sẽ được nhạp vào vốn ban đầu. Biết số tiền nhận được (bao gồm cả vốn lẫn lãi) sau $n$ năm là
	\begin{equation*}
		M(m_0,r,n) = m_0\left(1 + \frac{r}{100}\right)^n\mbox{ đồng},
	\end{equation*}
	nếu trong khoảng thời gian này người gửi không rút tiền ra \& lãi suất không thay đổi. Viết chương trình Python để: (a) Tính $M(m_0,r,n)$ với $m_0,r,n$ lần lượt được nhập vào. (b) Xuất ra số tiền nhận được (bao gồm cả vốn lẫn lãi) sau năm 1, năm 2, $\ldots$, năm $n$, i.e., xuất ra dãy số thực $\{M(m_0,r,i)\}_{i=1}^n = M(m_0,r,1),M(m_0,r,2),\ldots,M(m_0,r,n)$. (c) Với số tiền $m$ được nhập từ bàn phím, cho biết sau bao nhiêu năm thì số tiền cả vốn lẫn lãi vượt qua số tiền $m$ kỳ vọng này.
	\item {\sf Input.} Dòng 1 chứa lần lượt $m_0,r\in(0,\infty)$, $n\in\mathbb{N}^\star$.
	\item {\sf Output.} Dòng 1 chứa $T(m_0,r,n)$. Dòng 2 chứa dãy số $\{M(m_0,r,i)\}_{i=1}^n = M(m_0,r,1),M(m_0,r,2),\ldots,M(m_0,r,n)$.
	\item {\sf Sample.}
	\begin{table}[H]
		\centering
		\begin{tabular}{|l|l|}
			\hline
			{\tt rate.inp} & {\tt rate.out} \\
			\hline
			125000000 4.9 3 & 144290081.125 \\
			\hline
		\end{tabular}
	\end{table}
\end{baitoan}

\begin{baitoan}[\cite{CDHT_Toan_10_CD}, 10., p. 30, tính tiền vốn lẫn lãi nếu không rút tiền ra]
	Giả sử năm đầu tiên, A gửi vào ngân hàng $m_0\in(0,\infty)$ đồng với lãi suất $r\%$ năm. Hết năm đầu tiên, A không rút tiền ra \& gửi thêm $m_0$ đồng nữa. Hết năm thứ 2, A cũng không rút tiền ra \& lại gửi thêm $m_0$ đồng nữa. Cứ tiếp tục như vậy cho các năm sau. Biết số tiền cả vốn lẫn lãi mà A có được sau $n\in\mathbb{N}^\star$ năm là
	\begin{equation*}
		M(m_0,r,n) = \frac{m_0(100 + r)}{r}\left[\left(1 + \frac{r}{100}\right)^n - 1\right]\mbox{ đồng},
	\end{equation*}
	nếu trong khoảng thời gian này lãi suất không thay đổi. Viết chương trình Python để: (a) Tính $M(m_0,r,n)$với $m_0,r,n$ lần lượt được nhập vào. (b) Xuất ra số tiền nhận được (bao gồm cả vốn lẫn lãi) sau năm 1, năm 2, $\ldots$, năm $n$, i.e., xuất ra dãy số thực $\{M(m_0,r,i)\}_{i=1}^n = M(m_0,r,1),M(m_0,r,2),\ldots,M(m_0,r,n)$. (c) Với số tiền $m$ được nhập từ bàn phím, cho biết sau bao nhiêu năm thì số tiền cả vốn lẫn lãi vượt qua số tiền $m$ kỳ vọng này.
	\item {\sf Input.} Dòng 1 chứa lần lượt $m_0,r\in(0,\infty)$, $n\in\mathbb{N}^\star$.
	\item {\sf Output.} Dòng 1 chứa $T(m_0,r,n)$. Dòng 2 chứa dãy số $\{M(m_0,r,i)\}_{i=1}^n = M(m_0,r,1),M(m_0,r,2),\ldots,M(m_0,r,n)$.
	\item {\sf Sample.}
	\begin{table}[H]
		\centering
		\begin{tabular}{|l|l|}
			\hline
			{\tt rate.inp} & {\tt rate.out} \\
			\hline
			125000000 4.9 5 & 723102450.785  \\
			\hline
		\end{tabular}
	\end{table}
\end{baitoan}

\begin{baitoan}[\cite{CDHT_Toan_10_CD}, 11., p. 30, tính tiền vốn lẫn lãi nếu không rút tiền ra]
	1 người gửi số tiền $m_0\in(0,\infty)$ đồng vào ngân hàng. Biểu lãi suất của ngân hàng như sau: Chia mỗi năm thanh $m\in\mathbb{N}^\star$ kỳ hạn \& lãi suất $r\%${\tt/}năm. Biết nếu không rút tiền ra khỏi ngân hàng thì cứ sau mỗi kỳ hạn, số tiền lãi sẽ được nhập vào vốn ban đầu. Biết số tiền nhận được (bao gồm cả vốn lẫn lãi) sau $n$ năm gửi là
	\begin{equation*}
		M(m_0,r,m,n) = m_0\left(1 + \frac{r}{100m}\right)^{mn},
	\end{equation*}
	nếu trong khoảng thời gian này người gửi không rút tiền ra \& lãi suất không thay đổi. Viết chương trình Python để: (a) Tính $M(m_0,r,m,n)$với $m_0,r,m,n$ lần lượt được nhập vào. (b) Xuất ra số tiền nhận được (bao gồm cả vốn lẫn lãi) sau năm 1, năm 2, $\ldots$, năm $n$, i.e., xuất ra dãy số thực $\{M(m_0,r,m,i)\}_{i=1}^n = M(m_0,r,m,1),M(m_0,r,m,2),\ldots,M(m_0,r,m,n)$. (c) Với số tiền $M$ được nhập từ bàn phím, cho biết sau bao nhiêu năm thì số tiền cả vốn lẫn lãi vượt qua số tiền $M$ kỳ vọng này.
	\item {\sf Input.} Dòng 1 chứa lần lượt $m_0,r\in(0,\infty)$, $m,n\in\mathbb{N}^\star$.
	\item {\sf Output.} Dòng 1 chứa $T(m_0,r,m,n)$. Dòng 2 chứa dãy số $\{M(m_0,r,m,i)\}_{i=1}^n = M(m_0,r,m,1),M(m_0,r,m,2),\ldots,M(m_0,r,m,n)$.
\end{baitoan}

%------------------------------------------------------------------------------%

\section{Miscellaneous}

\begin{enumerate}
	\item Install Anaconda for Python: \url{https://www.anaconda.com/}.
\end{enumerate}

%------------------------------------------------------------------------------%

\printbibliography[heading=bibintoc]
	
\end{document}