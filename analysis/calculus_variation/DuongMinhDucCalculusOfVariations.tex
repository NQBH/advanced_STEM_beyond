\documentclass[a4paper]{article}
\usepackage{longtable,float,hyperref,color,amsmath,amsxtra,amssymb,latexsym,amscd,amsthm,amsfonts,graphicx}
\numberwithin{equation}{section}
\allowdisplaybreaks
\usepackage{fancyhdr}
\pagestyle{fancy}
\fancyhf{}
\fancyhead[RE,LO]{\footnotesize \textsc \leftmark}
\cfoot{\thepage}
\renewcommand{\headrulewidth}{0.5pt}
\setcounter{tocdepth}{3}
\setcounter{secnumdepth}{3}
\usepackage{imakeidx}
\makeindex[columns=2, title=Alphabetical Index, 
           options= -s index.ist]
\title{\huge Duong Minh Duc, Calculus of Variations}
\author{\textsc{Nguyen Quan Ba Hong}\footnote{Typer.}\\
{\small Students at Faculty of Math and Computer Science,}\\ 
{\small Ho Chi Minh University of Science, Vietnam} \\
{\small \texttt{email. nguyenquanbahong@gmail.com}}\\
{\small \texttt{blog. \url{www.nguyenquanbahong.com}} 
\footnote{Copyright \copyright\ 2016 by Nguyen Quan Ba Hong, Student at Ho Chi Minh University of Science, Vietnam. This document may be copied freely for the purposes of education and non-commercial research. Visit my site \texttt{\url{www.nguyenquanbahong.com}} to get more.}}}
\begin{document}
\maketitle
\begin{abstract}
I retype and correct some errors in \cite{1}.
\end{abstract}
\newpage
\tableofcontents
\newpage
\section{Integrals of Vector-Valued Mappings}
\textbf{Definition 1.1.} Let $a_0,\ldots,a_n;c_1,\ldots,c_n$ be $2n+1$ real numbers such that
\begin{align}
a &= {a_0} < {a_1} <  \cdots  < {a_{n - 1}} < {a_n} = b\\
{c_k} &\in \left[ {{a_{k - 1}},{a_k}} \right],\hspace{0.2cm} k = 1, \ldots ,n
\end{align}
We call $P = \left\{ {{a_0}, \ldots ,{a_n};{c_1}, \ldots ,{c_n}} \right\}$ a \textit{partition} of the interval $\left[a,b\right]$ and put
\begin{align}
\left| P \right| = \max \left\{ {{a_1} - {a_0},{a_2} - {a_1}, \ldots ,{a_n} - {a_{n - 1}}} \right\}
\end{align}
Define $\mathcal{P}\left(\left[a,b\right]\right)$ to be the \textit{set of all partitions of the interval $\left[a,b\right]$.}\\
\\
\textbf{Definition 1.2.} Let $\left( {E,\left\|  \cdot  \right\|} \right)$ be a Banach (normed) space. Let $u$ be a continuous mapping from a closed interval $\left[a,b\right]$ into $E$, and $P = \left\{ {{a_0}, \ldots ,{a_n};{c_1}, \ldots ,{c_n}} \right\}$ be a partition of the interval $\left[a,b\right]$. We put
\begin{align}
S\left( {u,P} \right) = \sum\limits_{k = 1}^n {u\left( {{c_k}} \right)\left( {{a_k} - {a_{k - 1}}} \right)} 
\end{align}
and call $S\left(u,P\right)$ a \textit{Riemann sum} with respect to the partition $P$.

Using uniform continuity of $u$ in $\left[a,b\right]$ and completeness of $E$, we can prove that there exists a vector $w$ such that
\begin{align}
\mathop {\lim }\limits_{\left| P \right| \to 0} S\left( {u,P} \right) = w
\end{align}
We call $w$ the \textit{integral} of $u$ in $\left[a,b\right]$ and denote it by $\int_a^b {u\left( t \right)dt} $.\\
\\
\textbf{Problem 1.3.} \textit{Let $\left( {E,\left\|  \cdot  \right\|} \right)$ be a Banach space. Let $u$ be a continuous mapping from a closed interval $\left[a,b\right]$ into $E$ and $T$ be a linear mapping from $E$ into $\mathbb{R}$. Prove that}
\begin{align}
\int_a^b {\left( {T \circ u} \right)\left( t \right)dt}  = T\left( {\int_a^b {u\left( t \right)dt} } \right)
\end{align}
\textsc{Hint.} Notice that
\begin{align}
S\left( {\left( {T \circ u} \right),P} \right) = T\left( {S\left( {u,P} \right)} \right)
\end{align}
for all partitions $P$ of $\left[a,b\right]$. \hfill $\square$\\
\\
\textbf{Problem 1.4.} \textit{Let $\left( {E,\left\|  \cdot  \right\|} \right)$ be a Banach space. Let $u,v$ be two continuous mappings from a closed interval $\left[a,b\right]$ into $E$ and $\alpha$ be a positive real number. Prove that}
\begin{align}
\int_a^b {\left( {u + \alpha v} \right)\left( t \right)dt}  = \int_a^b {u\left( t \right)dt}  + \alpha \int_a^b {v\left( t \right)dt} 
\end{align}
\textsc{Hint.} Prove 
\begin{align}
T\left( {\int_a^b {\left( {u + \alpha v} \right)\left( t \right)dt} } \right) = T\left( {\int_a^b {u\left( t \right)dt}  + \alpha \int_a^b {v\left( t \right)dt} } \right)
\end{align}
for all $T \in L\left( {E,\mathbb{R}} \right)$. \hfill $\square$\\
\\
\textbf{Problem 1.5.} \textit{Let $\left( {E,\left\|  \cdot  \right\|} \right)$ be a Banach space. Let $u$ be a continuous mapping from a closed interval $\left[a,b\right]$ into $E$. Prove that}
\begin{align}
\left\| {\int_a^b {u\left( t \right)dt} } \right\| \le \int_a^b {{{\left\| {u\left( t \right)} \right\|}_E}dt} 
\end{align}
\textsc{Hint.} Choose $T \in L\left( {E,\mathbb{R}} \right)$ such that $\left\| T \right\| = 1$ and
\begin{align}
\left\| {\int_a^b {u\left( t \right)dt} } \right\| = T\left( {\int_a^b {u\left( t \right)dt} } \right)
\end{align}
then applied Problem 1.3 and Problem 1.4. \hfill $\square$
\section{Differential Calculus in Normed Spaces}
Let $E$ and $F$ be two normed spaces equipped norms ${\left\|  \cdot  \right\|_E}$ and ${\left\|  \cdot  \right\|_F}$, respectively, $U$ be an open set in $E$, and $e$ be a vector in $E$ and $x\in U$.\\
\\
\textbf{Problem 2.1.} \textit{Prove that there exists a positive real number $\alpha$ such that the open interval $\left(-\alpha,\alpha\right)$ is contained in the following set}
\begin{align}
{I_{x,e}} = \left\{ {t:t \in R,x + te \in U} \right\}
\end{align}
Define
\begin{align}
{U_{x,h}} = \left\{ {y:y = x + te \in U,t \in \mathbb{R}} \right\}
\end{align}
\textbf{Definition 2.2.} Let $f$ be a mapping from $U$ into $F$, $e$ be a vector in $E$ and $x\in U$. We say
\begin{enumerate}
\item $f$ has a \textit{partial derivative with respect to direction} (\textit{directional derivative}) $e$ at the point $x$ if the following limit exists
\begin{align}
\mathop {\lim }\limits_{t \to 0} \frac{{f\left( {x + te} \right) - f\left( x \right)}}{t}
\end{align}
Then we denote this limit by $\frac{{\partial f}}{{\partial e}}\left( x \right)$.
\item $f$ is \textit{directional differentiable} at the point $x$ if $f$ has partial derivatives with respect to all directions in $E$ and there exists a linear mapping $Df\left(x\right)$ from $E$ into $F$ such that
\begin{align}
\frac{{\partial f}}{{\partial e}}\left( x \right) = Df\left( x \right)e,\hspace{0.2cm}\forall e \in E
\end{align}
\end{enumerate}
\textbf{Definition 2.3.} Let $f$ be a mapping from $U$ into $F$ and $x\in U$. We say
\begin{enumerate}
\item $f$ is \textit{G\^{a}teaux differentiable at the point $x$} if $f$ is directional differentiable at $x$ and $Df\left(x\right)$ belongs to $L\left(E,F\right)$.
\item $f$ is \textit{G\^{a}teaux differentiable in $U$} if $f$ is G\^{a}teaux differentiable at all points $x$ in $U$.
\end{enumerate}
\textbf{Problem 2.4.} \textit{Let $\left( {H,\left\|  \cdot  \right\|} \right)$ be a Hilbert space. Put}
\begin{align}
f\left( x \right) = {\left\| x \right\|^2},\hspace{0.2cm}\forall x \in H
\end{align}
\textit{Prove that $f$ is G\^{a}teaux differentiable in $H$.}\\
\\
\textsc{Hint.} Let $\left\langle { \cdot , \cdot } \right\rangle $ be the scalar product equipped for $H$. Notice that 
\begin{align}
{\left\| x \right\|^2} = \left\langle {x,x} \right\rangle ,\hspace{0.2cm}\forall x \in H
\end{align}
\textbf{Problem 2.5.} \textit{Let $f$ be a continuous linear mapping from a normed space $\left( {E,{{\left\|  \cdot  \right\|}_E}} \right)$ into a normed space $\left( {F,{{\left\|  \cdot  \right\|}_F}} \right)$. Prove that $f$ is G\^{a}teaux differentiable in $E$.}\\
\\
\textsc{Hint.} Use linearity and continuity of $f$. \hfill $\square$\\
\\
\textbf{Problem 2.6.} \textit{Let $\left( {E,{{\left\|  \cdot  \right\|}_E}} \right)$ and $\left( {F,{{\left\|  \cdot  \right\|}_F}} \right)$ be two normed spaces, $B$ be a continuous bilinear mapping from $E\times E$ into $F$. Put $f\left(x\right)=B\left(x,x\right)$ for all $x$ in $E$. Prove that $f$ is G\^{a}teaux differentiable in $E$.}\\
\\
\textsc{Hint.} Use the bilinearity and continuity of $B$.\hfill $\square$\\

Let $E$ and $F$ be two normed spaces and $U$ be a open set in $E$. Let $x\in U$. There exists a positive real number $r$ such that $B\left(x,r\right) \subset U$.\\
\\
\textbf{Definition 2.7.} Let $f$ be a mapping from $U$ into $F$, and $x\in U$. We say
\begin{enumerate}
\item $f$ is \textit{Fr\'{e}chet differentiable at the point $x$} if there exists a mapping $Df\left(x\right) \in L\left(E,F\right)$ and $\phi$ from $B\left(0,r\right) \subset E$ into $F$ such that 
\begin{align}
f\left( {x + h} \right) - f\left( x \right) = Df\left( x \right)\left( h \right) + {\left\| h \right\|_E}\phi \left( h \right),\hspace{0.2cm}\forall h \in B\left( {0,r} \right)
\end{align}
where $\phi \left( h \right) \to 0$ as $h \to 0$.
\item $f$ is \textit{Fr\'{e}chet differentiable in $U$} if $f$ is Fr\'{e}chet differentiable at all points $x$ in $U$.
\end{enumerate}
\textbf{Problem 2.8.} \textit{Let $\left( {H,\left\|  \cdot  \right\|} \right)$ be a Hilber space. Put}
\begin{align}
f\left( x \right) = {\left\| x \right\|^2},\hspace{0.2cm}\forall x \in H
\end{align}
\textit{Prove that $f$ is Fr\'{e}chet differentiable in $H$.}\\
\\
\textsc{Hint.} Let $\left\langle { \cdot , \cdot } \right\rangle $ be the scalar product equipped for $H$. Notice that 
\begin{align}
{\left\| x \right\|^2} = \left\langle {x,x} \right\rangle ,\hspace{0.2cm}\forall x \in H
\end{align}
Done. \hfill $\square$\\
\\
\textbf{Problem 2.9.} \textit{Let $f$ be a continuous linear mapping from a normed space $\left( {E,{{\left\|  \cdot  \right\|}_E}} \right)$ into a normed space $\left( {F,{{\left\|  \cdot  \right\|}_F}} \right)$. Prove that $f$ is Fr\'{e}chet differentiable in $E$.}\\
\\
\textsc{Hint.} Use the linearity and continuity of $f$. \hfill $\square$\\
\\
\textbf{Problem 2.10.} \textit{Let $f$ be a continuous bilinear mapping from a normed space $\left( {E,{{\left\|  \cdot  \right\|}_E}} \right)$ into a normed space $\left( {F,{{\left\|  \cdot  \right\|}_F}} \right)$. Prove that $f$ is Fr\'{e}chet differentiable in $E$.}\\
\\
\textsc{Hint.} use the bilinearity and continuity of $f$. \hfill $\square$\\
\\
\textbf{Problem 2.11.} \textit{Let $E$ and $F$ be two normed spaces, $U$ be an open set in $E$ and $x\in U$. Let $f$ be a mapping from $U$ into $F$ which is Fr\'{e}chet differentiable at $x$. Prove that $f$ is continuous at $x$.}\\
\\
\textsc{Hint.} Use definition. \hfill $\square$\\
\\
\textbf{Problem 2.12.} \textit{Let $U$ be an open set in a normed space $E$, $\alpha  \in \Phi $ and $f,g$ be two mappings from $U$ into a normed space $F$. Suppose that $f$ and $g$ are G\^{a}teaux differentiable (resp. Fr\'{e}chet differentiable) at some point $x$ in $U$. Prove that $f+g$ and $\alpha f$ are G\^{a}teaux differentiable (resp. Fr\'{e}chet differentiable) at $x$. In addition,}
\begin{align}
D\left( {f + g} \right)\left( x \right) &= Df\left( x \right) + Dg\left( x \right)\\
D\left( {\alpha f} \right)\left( x \right) &= \alpha Df\left( x \right)
\end{align}
\textsc{Hint.} Use definition. \hfill $\square$\\
\\
\textbf{Theorem 2.13.} \textit{Let $U$ and $O$ be a open subsets in normed spaces $E$ and $F$, respectively. Let $f:U\to O$ and $g:O\to G$ be mappings, where $G$ is a normed space. Let $x$ be some point in $U$. Suppose that $f$ is Fr\'{e}chet differentiable at $x$ and $g$ is Fr\'{e}chet differentiable at $f\left(x\right)$. Then $g \circ f$ is Fr\'{e}chet differentiable at $x$ and}
\begin{align}
D\left( {g \circ f} \right)\left( x \right) = Dg\left( {f\left( x \right)} \right) \circ Df\left( x \right)
\end{align}
\textbf{Problem 2.14 (Mean Value Theorem).} \textit{Let $f$ be a G\^{a}teaux differentiable mapping from an open set $U$ in a normed space $E$ into a normed space $F$. Let $a$ and $b$ in $U$ such that the set $\left[ {a,b} \right] \equiv \left\{ {a + t\left( {b - a} \right)|t \in \left[ {0,1} \right]} \right\}$ contained in $U$. Suppose that $f$ is continuous on $\left[a,b\right]$. Prove that}
\begin{align}
{\left\| {f\left( b \right) - f\left( a \right)} \right\|_F} \le {\left\| {b - a} \right\|_E}\sup \left\{ {\left\| {Df\left( y \right)} \right\|:y \in \left[ {a,b} \right]} \right\}
\end{align} 
\textsc{Hint.} Let $T \in L\left( {E,\mathbb{R}} \right)$ satisfy $\left\| T \right\| = 1$ and
\begin{align}
{\left\| {f\left( b \right) - f\left( a \right)} \right\|_F} = T\left( {f\left( b \right) - f\left( a \right)} \right)
\end{align}
Put 
\begin{align}
g\left( s \right) = T\left( {f\left( {a + s\left( {b - a} \right)} \right)} \right),\hspace{0.2cm}\forall s \in \left[ {0,1} \right]
\end{align}
Prove
\begin{align}
{\left\| {\left( {T \circ f} \right)\left( b \right) - \left( {T \circ f} \right)\left( a \right)} \right\|_F} \le {\left\| {b - a} \right\|_E}\sup \left\{ {\left\| {D\left( {T \circ f} \right)\left( y \right)} \right\||y \in \left[ {a,b} \right]} \right\}
\end{align}
Done. \hfill $\square$\\
\\
\textbf{Problem 2.15 (Mean Value Theorem).} \textit{Let $f$ be a G\^{a}teaux differentiable mapping from an open set $U$ in a normed space $E$ into a normed space $F$. Suppose that $x \mapsto Df\left( x \right)\left( h \right)$ is a continuous mapping in $U$ for all $h$ in $E$. Let $a$ and $b$ in $U$ such that the set $\left[ {a,b} \right] \equiv \left\{ {a + t\left( {b - a} \right)|t \in \left[ {0,1} \right]} \right\}$ contained in $U$. Prove that}
\begin{align}
f\left( b \right) - f\left( a \right) = \left( {b - a} \right)\int_0^1 {Df\left( {a + t\left( {b - a} \right)} \right)dt} 
\end{align}
\textsc{Hint.} Let $T \in L\left( {E,\mathbb{R}} \right)$, prove 
\begin{align}
D\left( {T \circ f} \right)\left( {a + t\left( {b - a} \right)} \right) &= T\left( {Df\left( {a + t\left( {b - a} \right)} \right)} \right),\hspace{0.2cm}\forall t \in \left[ {0,1} \right]\\
T\left( {f\left( b \right) - f\left( a \right)} \right) &= T\left( {\left( {b - a} \right)\int_0^1 {Df\left( {a + t\left( {b - a} \right)} \right)dt} } \right)
\end{align}
\textbf{Problem 2.16.} \textit{Let $f$ be a G\^{a}teaux differentiable mapping from a open set $U$ in a normed space $E$ into a normed space $F$. Suppose that $x \mapsto Df\left( x \right)$ is a continuous mapping from $U$ into $L\left(E,F\right)$. Prove that $f$ is Fr\'{e}chet differentiable in $U$.}\\
\\
\textsc{Hint.} Let $x\in U$ and a positive real number $r$ such that $B\left(x,r\right)\subset U$. Put
\begin{align}
\phi \left( h \right) = \int_0^1 {\left( {Df\left( {x + th} \right) - Df\left( x \right)} \right)\left( h \right)dt} ,\hspace{0.2cm}\forall h \in B\left( {0,r} \right)
\end{align}
Prove 
\begin{align}
\mathop {\lim }\limits_{h \to 0} \frac{{\phi \left( h \right)}}{{\left\| h \right\|}} = 0
\end{align}
then use Mean Value Theorem. \hfill $\square$\\
\\
\textbf{Definition 2.17.} Let $f$ be a mapping from $U$ into $F$. We say
\begin{enumerate}
\item $f$ is \textit{continuously G\^{a}teaux differentiable in $U$} if $f$ is G\^{a}teaux differentiable in $U$ and the mapping $x \mapsto Df\left( x \right)$ is continuous from $U$ into $L\left(E,F\right)$.
\item $f$ is \textit{continuously Fr\'{e}chet differentiable in $U$} if $f$ is Fr\'{e}chet differentiable in $U$ and the mapping $x \mapsto Df\left( x \right)$ is continuous from $U$ into $L\left(E,F\right)$. We also say that \textit{$f$ is of class $C^1\left(U\right)$.}
\end{enumerate}
\textbf{Definition 2.18.} Let $E$ and $F$ be two normed spaces, $U$ be an open set in $E$ and $f$ be of class $C^1\left(U\right)$. Then $Df$ is a mapping from $U$ into the normed space $L\left(E,F\right)$. If $Df$ is Fr\'{e}chet differentiable in $U$, we say that $f$ is \textit{two-time Fr\'{e}chet differentiable in $U$} and denote $D\left(Df\right)$ by $D^2f$ and call it \textit{second derivative of $f$.}\\

Using mathematical induction, we can define \textit{$n$-time Fr\'{e}chet differentiable} concept in $U$ and denote $D\left(D^{n-1}f\right)$ by $D^nf$ where $D^0f=f$ for all positive integer $n$, and call it \textit{$n$th derivative of $f$}.

We denote by $C^r\left(U,F\right)$ the \textit{set of all $r$-time Fr\'{e}chet differentiable in $U$} such that $D^nf$ is continuous in $U$ for all $n\le r$. If \textit{$f$ is of class $C^r\left(U,F\right)$}, we say that $f$ is \textit{continuously $r$-time Fr\'{e}chet differentiable in $U$}. We define
\begin{align}
{C^\infty }\left( {U,F} \right) = \bigcap\limits_{r = 1}^\infty  {{C^r}\left( {U,F} \right)} 
\end{align}
\textbf{Theorem 2.19.} \textit{Let $U$ be an open set in a Banach space $E$, $u$ in $C^2\left(U,F\right)$, $x\in U$ and $h,k\in E$. Then}
\begin{align}
{D^2}u\left( x \right)\left( {h,k} \right) = {D^2}u\left( x \right)\left( {k,h} \right)
\end{align}
\textbf{Problem 2.20.} \textit{Let $A$ be a bounded measurable set in $\mathbb{R}^n$. Let $\mu$ be Lebesgue measure in $\mathbb{R}^n$. Define}
\begin{align}
f\left( u \right) = \int_A {{u^3}d\mu } ,\hspace{0.2cm}\forall u \in {L^3}\left( A \right)
\end{align}
\textit{Prove that $f$ is of class ${C^1}\left( {{L^3}\left( A \right),\mathbb{R}} \right)$.}\\
\\
\textsc{Hint.} Let $u$ and $v$ be in $L^3\left(A\right)$ and $t \in \left( { - 1,1} \right)\backslash \left\{ 0 \right\}$. Prove
\begin{align}
\frac{{f\left( {u + tv} \right) - f\left( u \right)}}{t} = \int_A {\left( {3{u^2}v + 3tu{v^2} + {t^2}{v^3}} \right)d\mu } 
\end{align}
Thus, $f$ is G\^{a}teaux differentiable in $L^3\left(A\right)$ and
\begin{align}
Df\left( u \right)\left( v \right) = 3\int_A {{u^2}vd\mu } ,\hspace{0.2cm}\forall u,v \in {L^3}\left( A \right)
\end{align}
Let $u,v,w \in L^3\left(A\right)$. Prove
\begin{align}
\left| {\left( {Df\left( u \right) - Df\left( w \right)} \right)\left( v \right)} \right| \le 3{\left\| {u + w} \right\|_{{L^3}\left( A \right)}}{\left\| {u - w} \right\|_{{L^3}\left( A \right)}}{\left\| v \right\|_{{L^3}\left( A \right)}}
\end{align}
\textbf{Problem 2.21.} \textit{Let $A$ be a bounded measurable set in $\mathbb{R}^n$. Let $\mu$ be the Lebesgue measure in $\mathbb{R}^n$. Define}
\begin{align}
f\left( u \right) = \int_A {{u^3}d\mu } ,\hspace{0.2cm}\forall u \in {L^3}\left( A \right)
\end{align}
\textit{Prove that $f$ is of class ${C^2}\left( {{L^3}\left( A \right),\mathbb{R}} \right)$.}\\
\\
\textsc{Hint.} Let $u,v,w \in L^3\left(A\right)$. Prove that
\begin{align}
\frac{{\left( {Df\left( {u + tw} \right) - Df\left( u \right)} \right)\left( v \right)}}{t} = 3\int_A {\left( {2uvw + t{w^2}v} \right)d\mu } 
\end{align}
Thus,
\begin{align}
{D^2}f\left( u \right)\left( {v,w} \right) = 6\int_A {uvwd\mu } 
\end{align}
Let $u,v,w,z\in L^3\left(A\right)$. Prove
\begin{align}
\left| {\left( {{D^2}f\left( u \right) - {D^2}f\left( z \right)} \right)\left( {v,w} \right)} \right| \le 6{\left\| {u - z} \right\|_{{L^3}\left( A \right)}}{\left\| v \right\|_{{L^3}\left( A \right)}}{\left\| w \right\|_{{L^3}\left( A \right)}}
\end{align}
\textbf{Problem 2.22.} \textit{Let $A$ be a bounded measurable in $\mathbb{R}^n$. Let $\mu$ be the Lebesgue measure in $\mathbb{R}^n$. Define}
\begin{align}
f\left( u \right) = \int_A {{u^3}d\mu } ,\hspace{0.2cm}\forall u \in {L^3}\left( A \right)
\end{align}
\textit{Prove that $f$ is of class ${C^3}\left( {{L^3}\left( A \right),R} \right)$.}\\
\\
\textsc{Hint.} Let $u,v,w,z \in L^3\left(A\right)$. Prove 
\begin{align}
\frac{{\left( {{D^2}f\left( {u + tz} \right) - {D^2}f\left( u \right)} \right)\left( {v,w} \right)}}{t} = 6\int_A {vwzd\mu } 
\end{align}
Thus, 
\begin{align}
{D^3}f\left( u \right)\left( {v,w,z} \right) = 6\int_A {vwzd\mu } 
\end{align}
\textbf{Problem 2.23.} \textit{Let $A$ be a bounded measurable set in $\mathbb{R}^n$. Let $\mu$ be the Lebesgue measure in $\mathbb{R}^n$. Define}
\begin{align}
f\left( u \right) = \int_A {\sin \left( {u\left( t \right)} \right)d\mu } ,\hspace{0.2cm}\forall u \in {L^5}\left( A \right)
\end{align}
\textit{Prove that $f$ is of class ${C^1}\left( {{L^5}\left( A \right),R} \right)$.}\\
\\
\textsc{Hint.} Let $u,v\in L^5\left(A\right)$ and $t \in \left( { - 1,1} \right)\backslash \left\{ 0 \right\}$. Prove \begin{align}
\frac{{f\left( {u + tv} \right) - f\left( u \right)}}{t} = \int_A {\frac{{\sin \left( {u\left( s \right) + tv\left( s \right)} \right) - \sin \left( {u\left( s \right)} \right)}}{t}d\mu } 
\end{align}
Use the Lebesgue dominated convergence theorem, prove
\begin{align}
\mathop {\lim }\limits_{t \to 0} \frac{{f\left( {u + tv} \right) - f\left( u \right)}}{t} = \int_A {\cos \left( u \right)vd\mu } 
\end{align}
\textbf{Definition 2.24.} Let $f$ be a real function from a subset $A$ in a normed space $E$ and $a$ is a point in $A$. We say
\begin{enumerate}
\item $f$ attains \textit{maximum} at $a$ if $f\left(x\right)\le f\left(a\right)$ for all $x \in A$. Then $a$ is called a \textit{maximum point} of $f$.
\item $f$ attains \textit{minimum} at $a$ if $f\left(x\right) \ge f\left(a\right)$ for all $x in A$. Then $a$ is called a \textit{minimum point} of $f$.
\item $f$ attains \textit{extremity} at $a$ if $f$ attains maximum or minimum at $a$. Then $a$ is called \textit{extreme point} of $f$.
\item $f$ attains \textit{local maximum} at $a$ if there exists a positive real number $r$ such that $f\left(x\right)\le f\left(a\right)$ for all $x \in A \cap B\left( {a,r} \right)$. Then $a$ is called a \textit{local maximum point} of $f$.
\item $f$ attains \textit{local minimum} at $a$ if there exists a positive real number $r$ such that $f\left(x\right)\ge f\left(a\right)$ for all $x \in A \cap B\left( {a,r} \right)$. Then $a$ is called a \textit{local minimum point} of $f$.
\item $f$ attains \textit{local extremity} at $a$ if $f$ attains local maximum or local minimum at $a$. Then $a$ is called a \textit{local extreme point} of $f$.
\item $a$ is called a \textit{critical point} of $f$ if $A$ is an open set and $f$ is directional differentiable at $a$ and $Df\left(a\right)\left(h\right)=0$ for all $h\in E$.
\end{enumerate}
\textbf{Problem 2.25.} \textit{Let $f$ be a real function on an open set $U$ in a normed space $E$ which attains extremity at $a$ in $U$. Let $h$ in $E$ such that $f$ is directional differentiable with respect to direction $h$ at $a$. Prove that}
\begin{align}
\frac{{\partial f}}{{\partial h}}\left( a \right) = 0
\end{align}
\textbf{Problem 2.26.} \textit{Let $A$ be a bounded measurable in $\mathbb{R}^n$. Let $\mu$ be the Lebesgue measure in $\mathbb{R}^n$. Define}
\begin{align}
f\left( u \right) = \int_A {\sin {{\left( {u\left( t \right)} \right)}^2}d\mu } ,\hspace{0.2cm}\forall u \in {L^7}\left( A \right)
\end{align}
\textit{Accept that $f$ is of class ${C^1}\left( {{L^7}\left( A \right),\mathbb{R}} \right)$, find some critical points of $f$ without calculating derivative of $f$.}\\
\\
\textsc{Hint.} Find extremity of $f$. \hfill $\square$\\
\\
\textbf{Theorem (Lagrange Multipliers).} \textit{Let $f$ and $g$ be two real functions which are continuously Fr\'{e}chet differentiable on an open set $U$ in a normed space $E$. Define}
\begin{align}
M = \left\{ {x \in U|g\left( x \right) = 0} \right\}
\end{align}
\textit{Suppose that there exists $a$ in $M$ such that $f\left(a\right)$ is an extreme value of $f\left(M\right)$ and $Dg\left( a \right)\not  \equiv 0$. Then there exists a real number $\lambda$ such that}
\begin{align}
Df\left( a \right) = \lambda Dg\left( a \right)
\end{align}
\section{Lower Semicontinuous Functions}
\textbf{Definition 3.1.} Let $\left( {M,\delta } \right)$ be a metric space and $f$ be a real function on $M$. We say that $f$ is \textit{lower semicontinuous in $M$} if for all sequence $\left\{ {{x_m}} \right\}_{m = 1}^\infty $ converging to $x$ in $M$, the following inequality holds
\begin{align}
f\left( x \right) \le \mathop {\lim \inf }\limits_{m \to \infty } f\left( {{x_m}} \right)
\end{align}
\textbf{Problem 3.1.} \textit{Let $\left( {M,\delta } \right)$ be a metric space, $f$ be a real lower semicontinuous function in $M$, and $\alpha$ be a real number. Prove}
\begin{align}
\left\{ {x \in M|f\left( x \right) > \alpha } \right\}
\end{align}
\textit{is an open set in $M$.}\\
\\
\textsc{Hint.} Prove that 
\begin{align}
\left\{ {x \in M|f\left( x \right) \le \alpha } \right\}
\end{align}
is a closed set in $M$. \hfill $\square$\\
\\
\textbf{Problem 3.2.} \textit{Let $\left( {M,\delta } \right)$ be a metric space, $f$ be a real function in $M$. Suppose that}
\begin{align}
\left\{ {x \in M|f\left( x \right) > \alpha } \right\}
\end{align}
\textit{is an open set in $M$ for all real number $\alpha$. Prove that $f$ is lower semicontinuous in $M$.}\\
\\
\textsc{Hint.} Let $\left\{ {{x_m}} \right\}_{m = 1}^\infty $ be a sequence converging to $x$ in $M$. Let $\beta$ be real number for which $\beta <f\left(x\right)$. Prove
\begin{align}
\beta  < \mathop {\lim \inf }\limits_{m \to \infty } f\left( {{x_m}} \right)
\end{align}
\hfill $\square$\\
\\
\textbf{Problem 3.3.} \textit{Let $\left( {M,\delta } \right)$ be a metric space, $f$ be a real lower semicontinuous function in $M$, and $\alpha$ in $f\left(M\right)$. Suppose that}
\begin{align}
\left\{ {x \in M|f\left( x \right) \le \alpha } \right\}
\end{align}
\textit{is compact in $M$. Prove that there exists $u$ in $M$ such that $f\left(u\right)=\min f\left(M\right)$.}\\
\\
\textsc{Hint.} Let $\left\{ {{x_m}} \right\}_{m = 1}^\infty $ 
be a sequence in $M$ such that
\begin{align}
\mathop {\lim }\limits_{m \to \infty } f\left( {{x_m}} \right) = \inf f\left( M \right)
\end{align}
\textbf{Problem 3.4.} \textit{Let $\left( {M,\delta } \right)$ be a metric space, $f$ be a real function in $M$, and $\alpha$ in $f\left(M\right)$. Suppose that for all $\beta \le \alpha$ the set}
\begin{align}
{K_\beta } = \left\{ {x \in M:f\left( x \right) \le \beta } \right\}
\end{align}
\textit{is compact. Prove that there exists some $u$ in $M$ such that}
\begin{align}
f\left( u \right) = \min f\left( M \right)
\end{align}
\textsc{Hint.} Let $\left\{ {{x_m}} \right\}_{m = 1}^\infty $ 
be a sequence in $M$ such that $\left\{ {f\left( {{x_m}} \right)} \right\}_{m = 1}^\infty $ converges to $\inf f\left( M \right)$. Suppose 
\begin{align}
{\beta _m} = f\left( {{x_m}} \right) \le \alpha 
\end{align}
Prove that there exists a subsequence $\left\{ {{x_{{m_k}}}} \right\}_{k = 1}^\infty $ of the sequence $\left\{ {{x_m}} \right\}_{m = 1}^\infty $ which converges to $u$ in the set $\bigcap\limits_{m = 1}^\infty  {{K_{{\beta _m}}}} $. \hfill $\square$\\
\\
\textbf{Definition 3.5.} Let $\left\{ {{x_m}} \right\}_{m = 1}^\infty $ be a sequence in a normed space $E$. We say that $\left\{ {{x_m}} \right\}_{m = 1}^\infty $\textit{ weakly converges} to $x$ in $E$ if $\left\{ {T\left( {{x_m}} \right)} \right\}_{m = 1}^\infty $ converges to $T\left(x\right)$ for all $T \in L\left(E,\mathbb{R}\right)$.\\
\\
\textbf{Problem 3.6.} \textit{Let $\left\{ {{x_m}} \right\}_{m = 1}^\infty $ be a sequence which weakly converges to $x$ in a Banach space $\left( {E,\left\|  \cdot  \right\|} \right)$. Prove that $\left\{ {\left\| {{x_m}} \right\|} \right\}_{m = 1}^\infty $ is bounded in $\mathbb{R}$.}\\
\\
\textsc{Hint.} Define 
\begin{align}
{\Lambda _m}\left( S \right) = S\left( {{x_m}} \right),\hspace{0.2cm}\forall m \in {\mathbb{Z}_ + },\forall S \in F: = L\left( {E,\mathbb{R}} \right)
\end{align}
and
\begin{align}
|||S||| &= \sup \left\{ {\left| {S\left( u \right)} \right||u \in E,\left\| u \right\| \le 1} \right\}\\
|||{\Lambda _m}||| &= \sup \left\{ {\left| {\Lambda \left( T \right)} \right||T \in L\left( {E,\mathbb{R}} \right),|||T||| \le 1} \right\}
\end{align}
Use Hahn-Banach theorem, prove $|||{\Lambda _m}||| = \left\| {{x_m}} \right\|$. Then use Banach-Steinhaus theorem, prove that $\left\{ {|||{\Lambda _m}|||} \right\}_{m = 1}^\infty $ is bounded. \hfill $\square$\\
\\
\textbf{Definition 3.7.} Let $M$ be a nonempty subset in a Banach space $E$ and $f$ be a real function in $M$. We say that $f$ is \textit{weakly lower semicontinuous in $M$} if for all sequences $\left\{ {{x_m}} \right\}_{m = 1}^\infty $ converging to $x$ in $M$, the following inequality holds
\begin{align}
f\left( x \right) \le \mathop {\lim \inf }\limits_{m \to \infty } f\left( {{x_m}} \right)
\end{align}
\textbf{Definition 3.7.} Let $M$ be a nonempty subset in a Banach space $\left( {E,\left\|  \cdot  \right\|} \right)$ and $f$ be a real function in $M$. We say that $f$ is \textit{coercive} in $M$ if for all sequences $\left\{ {{x_m}} \right\}_{m = 1}^\infty $ in $M$ such that $\left\{ {\left\| {{x_m}} \right\|} \right\}_{m = 1}^\infty $ converges to $\infty$, then $\left\{ {f\left( {{x_m}} \right)} \right\}_{m = 1}^\infty $ converges to $\infty$.\\
\\
\textbf{Definition 3.8.} Let $M$ be a subset in a Banach space $E$. We say that $M$ is \textit{weakly closed in $E$} if for all sequences $\left\{ {{x_m}} \right\}_{m = 1}^\infty $ in $M$ which weakly converges to $x$ in $E$, then $x \in M$.\\
\\
\textbf{Problem 3.9.} \textit{Let $E$ be a Banach space. Suppose that all each bounded sequence $\left\{ {{x_m}} \right\}_{m = 1}^\infty $ in $E$ always has a weakly convergent subsequence in $E$. Let $M$ be a weakly closed subset in $E$. Let $f$ be a real coercive lower semicontinuous function in $M$. Prove that there exists some $u$ in $M$ such that}
\begin{align}
f\left( u \right) = \min f\left( M \right)
\end{align}
\textsc{Hint.} Let $\left\{ {{u_m}} \right\}_{m = 1}^\infty $ be a sequence in $M$ such that $\left\{ {f\left( {{u_m}} \right)} \right\}_{m = 1}^\infty $ converges to $\inf f\left( M \right)$. Prove that $\left\{ {{u_m}} \right\}_{m = 1}^\infty $ is a bounded sequence whose a subsequence weakly converges to $u$ in $M$. \hfill $\square$\\
\\
\textbf{Theorem 3.10.} \textit{Let $\Omega$ be an open set in $\mathbb{R}^n$ and $F$ be a real function in $\Omega  \times \mathbb{R} \times {\mathbb{R}^n}$. Suppose}
\begin{enumerate}
\item \textit{For all $\left( {s,z} \right) \in \mathbb{R} \times {\mathbb{R}^n}$, the function $x \mapsto F\left( {x,s,z} \right)$ is measurable in $\Omega$.}
\item \textit{For all $x\in \Omega$, the function $\left( {s,z} \right) \mapsto F\left( {x,s,z} \right)$ is continuous in $\mathbb{R} \times {\mathbb{R}^n}$.}
\item \textit{For all $\left( {x,s} \right) \in \Omega  \times \mathbb{R}$, the function $z \mapsto F\left( {x,s,z} \right)$ is convex in $\mathbb{R}^n$.}
\item \textit{There exists an integrable function $\phi$ in $\Omega$ such that}
\begin{align}
F\left( {x,s,z} \right) \ge \phi \left( x \right),\hspace{0.2cm}\forall \left( {x,s,z} \right) \in \Omega  \times \mathbb{R} \times {\mathbb{R}^n}.
\end{align}
\end{enumerate}
\textit{Define}
\begin{align}
J\left( u \right) = \int_\Omega  {F\left( {x,u\left( x \right),\nabla u\left( x \right)} \right)dx} ,\hspace{0.2cm}\forall u \in {W^{1,1}}\left( \Omega  \right)
\end{align}
\textit{Then $J$ is weakly lower semicontinuous in $W^{1,1}\left(\Omega\right)$.}\\
\\
\\
\\
\begin{center}
\textsc{The End}
\end{center}
\newpage
\printindex
\newpage
\begin{thebibliography}{999}
\bibitem {1} Duong Minh Duc, \textit{Lectures: Real Analysis}, Faculty of Math and Computer Science, Ho Chi Minh University of Science.
\end{thebibliography}
\end{document}