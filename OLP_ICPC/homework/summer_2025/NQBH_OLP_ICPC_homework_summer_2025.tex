\documentclass{article}
\usepackage[backend=biber,natbib=true,style=alphabetic,maxbibnames=50]{biblatex}
\addbibresource{/home/nqbh/reference/bib.bib}
\usepackage[utf8]{vietnam}
\usepackage{tocloft}
\renewcommand{\cftsecleader}{\cftdotfill{\cftdotsep}}
\usepackage[colorlinks=true,linkcolor=blue,urlcolor=red,citecolor=magenta]{hyperref}
\usepackage{amsmath,amssymb,amsthm,enumitem,fancyvrb,float,graphicx,mathtools,tikz}
\usetikzlibrary{angles,calc,intersections,matrix,patterns,quotes,shadings}
\allowdisplaybreaks
\newtheorem{assumption}{Assumption}
\newtheorem{baitoan}{Bài toán}
\newtheorem{cauhoi}{Câu hỏi}
\newtheorem{conjecture}{Conjecture}
\newtheorem{corollary}{Corollary}
\newtheorem{dangtoan}{Dạng toán}
\newtheorem{definition}{Definition}
\newtheorem{dinhly}{Định lý}
\newtheorem{dinhnghia}{Định nghĩa}
\newtheorem{example}{Example}
\newtheorem{ghichu}{Ghi chú}
\newtheorem{hequa}{Hệ quả}
\newtheorem{hypothesis}{Hypothesis}
\newtheorem{lemma}{Lemma}
\newtheorem{luuy}{Lưu ý}
\newtheorem{nhanxet}{Nhận xét}
\newtheorem{notation}{Notation}
\newtheorem{note}{Note}
\newtheorem{principle}{Principle}
\newtheorem{problem}{Problem}
\newtheorem{proposition}{Proposition}
\newtheorem{question}{Question}
\newtheorem{remark}{Remark}
\newtheorem{theorem}{Theorem}
\newtheorem{vidu}{Ví dụ}
\usepackage[left=1cm,right=1cm,top=5mm,bottom=5mm,footskip=4mm]{geometry}
\def\labelitemii{$\circ$}
\DeclareRobustCommand{\divby}{%
    \mathrel{\vbox{\baselineskip.65ex\lineskiplimit0pt\hbox{.}\hbox{.}\hbox{.}}}%
}
\setlist[itemize]{leftmargin=*}
\setlist[enumerate]{leftmargin=*}

\title{OLP {\it\&} ICPC Homework: Summer 2025}
\author{Nguyễn Quản Bá Hồng\footnote{A scientist- {\it\&} creative artist wannabe, a mathematics {\it\&} computer science lecturer of Department of Artificial Intelligence {\it\&} Data Science (AIDS), School of Technology (SOT), UMT Trường Đại học Quản lý {\it\&} Công nghệ TP.HCM, Hồ Chí Minh City, Việt Nam.\\E-mail: {\sf nguyenquanbahong@gmail.com} {\it\&} {\sf hong.nguyenquanba@umt.edu.vn}. Website: \url{https://nqbh.github.io/}. GitHub: \url{https://github.com/NQBH}.}}
\date{\today}

\begin{document}
\maketitle
\begin{abstract}
    This text is a part of the series {\it Some Topics in Advanced STEM \& Beyond}:

    {\sc url}: \url{https://nqbh.github.io/advanced_STEM/}.

    Latest version:
    \begin{itemize}
        \item {\it OLP \& ICPC Homework: Summer 2025}.

        PDF: {\sc url}: \url{https://github.com/NQBH/advanced_STEM_beyond/blob/main/OLP_ICPC/homework/summer_2025/NQBH_OLP_ICPC_homework_summer_2025.pdf.pdf}.

        \TeX: {\sc url}: \url{https://github.com/NQBH/advanced_STEM_beyond/blob/main/OLP_ICPC/homework/summer_2025/NQBH_OLP_ICPC_homework_summer_2025.tex}.
    \end{itemize}
\end{abstract}
\tableofcontents

%------------------------------------------------------------------------------%

\section{Tasks}
Giải bài tập nhiều nhất \& viết reports cho mỗi tasks:
\begin{enumerate}
    \item Cày nhiều nhất có thể các lộ trình của VNOI roadmap:  \url{https://roadmap.sh/r/vnoi-roadmap}.
    \item Đọc kiến thức về CP trên trang \url{https://cp-algorithms.com/} \& giải các bài tập liên quan kèm theo.
    \item Học khuôn codes \& ý tưởng code 1 cách có hệ thống từ 2 cuốn \cite{Wu_Wang2016, Wu_Wang2018}.
    \item Task của thầy {\sc Trần Đan Thư}: Giải đề Olympic Tin học \& ICPC của 3 bảng không chuyên, chuyên Tin, \& siêu cúp của 3 năm gần nhất \& viết report.
\end{enumerate}
Quá trình biến kiến thức học được thành của bản thân cần đặt các câu hỏi:
\begin{enumerate}
    \item Mình học được gì từ bài toán hoặc code này?
    \item Cấu trúc dữ liệu cần thiết cho bài toán này là gì? Có cấu trúc dữ liệu nào tốt hơn không?
    \item Tư tưởng cốt lõi của bài toán này là gì? Tư tưởng này có thể áp dụng cho các loại bài toán nào?
    \item Có thể mở rộng code, hoặc tư tưởng code này không? Có thể tổng quát hóa bài toán này được không?
    \item Mình có thể đặt những câu hỏi tương tự để nâng cao cách học của bản thân không? Hay hiện tại chỉ cần chú trọng kỹ thuật lập trình \& kiến thức nhanh nhất có thể để vét được nhiều subtasks \& testcase nhất có thể?
\end{enumerate}
Cố gắng trả lời 1 cách rõ ràng, xúc tích nhất có thể để viết vào reports.

%------------------------------------------------------------------------------%

\section{Rewards -- Phần Thưởng}
Slogan: ``Nếu sinh viên không quay lưng với giảng viên thì không có bất cứ lý do gì mà giảng viên lại quay lưng với sinh viên.''
\begin{enumerate}
    \item Được bao ăn gà rán Texas Chicken, Popeyes hoặc Lotteria hoặc bất cứ món ăn, thức uống nào mà giảng viên nghèo như NQBH có khả năng bao nổi.
    \item Nếu bản thân sinh viên là rich kid, hoặc đủ giàu để không thèm thể loại thức ăn mà giảng viên (nghèo) bao, thì tặng kèm nhiều thời gian tư vấn \& kiến thức bổ ích về Khóa Luận Tốt Nghiệp hoặc Nghiên Cứu Khoa Học, cũng như các mối quan hệ trong mạng lưới.
    \item Nếu sinh viên nữ hoặc sinh viên nam nhưng cũng có nhu cầu làm đẹp, sẽ được nhận khóa học làm đẹp miễn phí của chị thư ký Khoa Công Nghệ {\sc Lê Thị Xuân Thưởng} thưởng trực tiếp.
    \item Nhiều quyền lợi sẽ được phát sinh trong suốt quá trình sinh viên học tập tại UMT, đặc biệt là mỗi khi có biến lớn, e.g., rắc rối khi thực tập tại Doanh nghiệp, Khóa luận Tốt nghiệp, etc.
\end{enumerate}

%------------------------------------------------------------------------------%

\section{Miscellaneous}

%------------------------------------------------------------------------------%

\printbibliography[heading=bibintoc]

\end{document}