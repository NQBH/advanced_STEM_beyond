\documentclass[oneside]{book}
\usepackage[backend=biber,natbib=true,style=alphabetic,maxbibnames=50]{biblatex}
\addbibresource{/home/nqbh/reference/bib.bib}
\usepackage[utf8]{vietnam}
\usepackage{tocloft}
\renewcommand{\cftsecleader}{\cftdotfill{\cftdotsep}}
\usepackage[colorlinks=true,linkcolor=blue,urlcolor=red,citecolor=magenta]{hyperref}
\usepackage{amsmath,amssymb,amsthm,enumitem,fancyvrb,float,graphicx,mathtools,minitoc,tikz,tipa}
\usetikzlibrary{angles,calc,intersections,matrix,patterns,quotes,shadings}
\usepackage{fancyhdr}
\pagestyle{fancy}
\fancyhf{}
\addtolength{\headheight}{0pt}% obsolete
\lhead{\scshape\small\chaptername~\thechapter}
\rhead{\small\nouppercase{\leftmark}}
\renewcommand{\chaptermark}[1]{\markboth{#1}{}}
\cfoot{\thepage}
\renewcommand{\headrulewidth}{0.5pt}
\renewcommand{\footrulewidth}{0pt}
\fancyheadoffset[RE,LO]{-0.0\textwidth}

\usepackage{textcase}

\makeatletter
\def\@makechapterhead#1{%
    \vspace*{50\p@}%
    {\parindent \z@ \centering\normalfont
        \ifnum \c@secnumdepth >\m@ne
        \if@mainmatter
        \huge\bfseries \MakeTextUppercase{\@chapapp}\space \thechapter
        \par\nobreak
        \vskip 20\p@
        \fi
        \fi
        \interlinepenalty\@M
        \huge \bfseries \MakeTextUppercase{#1}\par\nobreak
        \vskip 40\p@
}}
\def\@makeschapterhead#1{%
    \vspace*{50\p@}%
    {\parindent \z@ \centering
        \normalfont
        \interlinepenalty\@M
        \huge \bfseries  \MakeTextUppercase{#1}\par\nobreak
        \vskip 40\p@
}}
\makeatother


\DeclareMathSymbol{\mathinvertedexclamationmark}{\mathclose}{operators}{'074}
\DeclareMathSymbol{\mathexclamationmark}{\mathclose}{operators}{'041}

\makeatletter
\newcommand{\raisedmathinvertedexclamationmark}{%
    \mathclose{\mathpalette\raised@mathinvertedexclamationmark\relax}%
}
\newcommand{\raised@mathinvertedexclamationmark}[2]{%
    \raisebox{\depth}{$\m@th#1\mathinvertedexclamationmark$}%
}
\begingroup\lccode`~=`! \lowercase{\endgroup
    \def~}{\@ifnextchar`{\raisedmathinvertedexclamationmark\@gobble}{\mathexclamationmark}}
\mathcode`!="8000
\makeatother

\usepackage{sectsty}
\allsectionsfont{\sffamily}
\allowdisplaybreaks
\newtheorem{assumption}{Assumption}
\newtheorem{baitoan}{Bài toán}
\newtheorem{cauhoi}{Câu hỏi}
\newtheorem{conjecture}{Conjecture}
\newtheorem{corollary}{Corollary}
\newtheorem{dangtoan}{Dạng toán}
\newtheorem{definition}{Definition}
\newtheorem{dinhly}{Định lý}
\newtheorem{dinhnghia}{Định nghĩa}
\newtheorem{example}{Example}
\newtheorem{ghichu}{Ghi chú}
\newtheorem{goal}{Goal}
\newtheorem{hequa}{Hệ quả}
\newtheorem{hypothesis}{Hypothesis}
\newtheorem{intuition}{Intuition}
\newtheorem{lemma}{Lemma}
\newtheorem{luuy}{Lưu ý}
\newtheorem{nhanxet}{Nhận xét}
\newtheorem{notation}{Notation}
\newtheorem{note}{Note}
\newtheorem{principle}{Principle}
\newtheorem{problem}{Problem}
\newtheorem{proposition}{Proposition}
\newtheorem{question}{Question}
\newtheorem{remark}{Remark}
\newtheorem{theorem}{Theorem}
\newtheorem{vidu}{Ví dụ}
\usepackage[left=1cm,right=1cm,top=1.5cm,bottom=1.5cm]{geometry}
\def\labelitemii{$\circ$}
\DeclareRobustCommand{\divby}{%
    \mathrel{\vbox{\baselineskip.65ex\lineskiplimit0pt\hbox{.}\hbox{.}\hbox{.}}}%
}
\setlist[itemize]{leftmargin=*}
\setlist[enumerate]{leftmargin=*}
\newcommand{\genstirlingI}[3]{%
    \genfrac{[}{]}{0pt}{#1}{#2}{#3}%
}
\newcommand{\genstirlingII}[3]{%
    \genfrac{\{}{\}}{0pt}{#1}{#2}{#3}%
}
\newcommand{\stirlingI}[2]{\genstirlingI{}{#1}{#2}}
\newcommand{\dstirlingI}[2]{\genstirlingI{0}{#1}{#2}}
\newcommand{\tstirlingI}[2]{\genstirlingI{1}{#1}{#2}}
\newcommand{\stirlingII}[2]{\genstirlingII{}{#1}{#2}}
\newcommand{\dstirlingII}[2]{\genstirlingII{0}{#1}{#2}}
\newcommand{\tstirlingII}[2]{\genstirlingII{1}{#1}{#2}}

\title{Lecture Note: Mathematical Analysis {\it\&} Numerical Analysis\\Bài Giảng: Giải Tích Toán Học {\it\&} Giải Tích Số}
\author{Nguyễn Quản Bá Hồng\footnote{A scientist- {\it\&} creative artist wannabe, a mathematics {\it\&} computer science lecturer of Department of Artificial Intelligence {\it\&} Data Science (AIDS), School of Technology (SOT), UMT Trường Đại học Quản lý {\it\&} Công nghệ TP.HCM, Hồ Chí Minh City, Việt Nam.\\E-mail: {\sf nguyenquanbahong@gmail.com} {\it\&} {\sf hong.nguyenquanba@umt.edu.vn}. Website: \url{https://nqbh.github.io/}. GitHub: \url{https://github.com/NQBH}.}}
\date{\today}

\begin{document}
\maketitle
\setcounter{secnumdepth}{4}
\setcounter{tocdepth}{4}
\dominitoc % Initialization
\tableofcontents

%------------------------------------------------------------------------------%

\chapter*{Preface}
\minitoc
\section*{Abstract}
This text is a part of the series {\it Some Topics in Advanced STEM \& Beyond}:

{\sc url}: \url{https://nqbh.github.io/advanced_STEM/}.

Latest version:
\begin{itemize}
	\item {\it Lecture Note: Mathematical Analysis \& Numerical Analysis -- Bài Giảng: Giải Tích Toán Học \& Giải Tích Số}.
	
	PDF: {\sc url}: \url{https://github.com/NQBH/advanced_STEM_beyond/blob/main/analysis/lecture/NQBH_mathematical_analysis_lecture.pdf}.
	
	\TeX: {\sc url}: \url{https://github.com/NQBH/advanced_STEM_beyond/blob/main/analysis/lecture/NQBH_mathematical_analysis_lecture.tex}.
	\item {\it Slide: Mathematical Analysis -- Slide: Giải Tích Toán Học}.
	
	PDF: {\sc url}: \url{https://github.com/NQBH/advanced_STEM_beyond/blob/main/analysis/slide/NQBH_mathematical_analysis_slide.pdf}.
	
	\TeX: {\sc url}: \url{https://github.com/NQBH/advanced_STEM_beyond/blob/main/analysis/slide/NQBH_mathematical_analysis_slide.tex}.
	\item Codes:
	\begin{itemize}
		\item C++: \url{https://github.com/NQBH/advanced_STEM_beyond/tree/main/analysis/C++}.
		\item Python: \url{https://github.com/NQBH/advanced_STEM_beyond/tree/main/analysis/Python}.
	\end{itemize}
\end{itemize}

%------------------------------------------------------------------------------%

\part{Mathematical Analysis -- Giải Tích Toán Học}

\chapter{Basic Mathematical Analysis -- Giải Tích Toán Học Cơ Bản}
\minitoc
\textbf{\textsf{Resources -- Tài nguyên.}}
\begin{enumerate}
	\item {\sc Đặng Đình Áng}. {\it Nhập Môn Giải Tích}.
	\item \cite{Rudin1976}. {\sc Walter Rudin}. {\it Principles of Mathematical Analysis}.
	
	\item \cite{Tao_analysis_1}. {\sc Terence Tao}. {\it Analysis I}.
	
	\item \cite{Tao_analysis_2}. {\sc Terence Tao}. {\it Analysis II}.
\end{enumerate}

\begin{question}[Definition of mathematical analysis]
	What is mathematical analysis? Cf. mathematical analysis with other types of analysis.
\end{question}
For answers, see, e.g., \cite[Chap. 1, Sect. 1.1: {\it What Is Analysis?}, pp. 1--2]{Tao_analysis_1}, \href{https://en.wikipedia.org/wiki/Mathematical_analysis}{Wikipedia{\tt/}mathematical analysis}. For other types of analysis, see, e.g., \href{https://en.wikipedia.org/wiki/Analysis}{Wikipedia{\tt/}analysis}.

\begin{question}[Motivation of mathematical analysis]
	Why do mathematical analysis?
\end{question}
For answers, see, e.g., \cite[Chap. 1, Sect. 1.2: {\it Why Do Analysis?}, pp. 2--10]{Tao_analysis_1}

\begin{example}[Division by zero \& infinity]
	The cancellation law for multiplication $ac = bc\Rightarrow a = b$ does not work when $c = 0$ \& $c = \pm\infty$. The cancellation law for addition $a + c = b + c\Rightarrow a = b$.
\end{example}

\begin{example}[Cancellation properties]
	
\end{example}
See, e.g., \href{https://en.wikipedia.org/wiki/Cancellation_property}{Wikipedia{\tt/}cancellation property}.

\begin{example}[Geometric series -- Chuỗi hình học]
	When does the geometric series $G(a)\coloneqq\sum_{i=0}^\infty \frac{1}{a^i}$ converge? When does $G(a)$ diverge? 
\end{example}

%------------------------------------------------------------------------------%

\section{Numbers -- Các loại số}
Trong chương trình Toán phổ thông, học sinh đã được học: số tự nhiên ở chương trình Toán 6 \cite{SGK_Toan_6_CD_tap_1, SGK_Toan_6_CD_tap_2}, \& số hữu tỷ \& số thực ở chương trình Toán 7,

%------------------------------------------------------------------------------%

\section{Notations \& conventions -- Ký hiệu \& quy ước}
Đặt tập hợp các đa thức (polynomial) 1 biến với hệ số nguyên, hệ số hữu tỷ, hệ số thực, hệ số phức lần lượt cho bởi:
\begin{align*}
	\mathbb{Z}[x]&\coloneqq\left\{\sum_{i=0}^n a_ix^i;n\in\mathbb{N},\ a_i\in\mathbb{Z},\ \forall i = 0,\ldots,n,\ a_n\ne0\right\},\\
	\mathbb{Q}[x]&\coloneqq\left\{\sum_{i=0}^n a_ix^i;n\in\mathbb{N},\ a_i\in\mathbb{Q},\ \forall i = 0,\ldots,n,\ a_n\ne0\right\},\\
	\mathbb{R}[x]&\coloneqq\left\{\sum_{i=0}^n a_ix^i;n\in\mathbb{N},\ a_i\in\mathbb{R},\ \forall i = 0,\ldots,n,\ a_n\ne0\right\},\\
	\mathbb{C}[x]&\coloneqq\left\{\sum_{i=0}^n a_ix^i;n\in\mathbb{N},\ a_i\in\mathbb{C},\ \forall i = 0,\ldots,n,\ a_n\ne0\right\}.
\end{align*}
Ta có quan hệ hiển nhiên $\mathbb{N}[x]\subset\mathbb{Z}[x]\subset\mathbb{Q}[x]\subset\mathbb{R}[x]\subset\mathbb{C}[x]$. Tổng quát, với $\mathbb{F}$ là 1 trường bất kỳ, tập hợp các đa thức 1 biến với hệ số thuộc trường $\mathbb{F}$ (e.g., $\mathbb{Z},\mathbb{Z}_p,\mathbb{Q},\mathbb{R},\mathbb{C}$) cho bởi:
\begin{equation*}
	\mathbb{F}[x]\coloneqq\left\{\sum_{i=0}^n a_ix^i;n\in\mathbb{N},\ a_i\in\mathbb{F},\ \forall i = 0,\ldots,n,\ a_n\ne0\right\}.
\end{equation*}
Tập xác định của đa thức có thể là toàn bộ trường số thực $\mathbb{R}$ hoặc trường số phức $\mathbb{C}$, i.e., $D_P = {\rm dom}(P) = \mathbb{R}$ or $D_P = {\rm dom}(P) = \mathbb{C}$, tùy vào trường $\mathbb{F}$ của các hệ số \& mục đích sử dụng đa thức.

\begin{problem}[Cf: Calculus vs. Mathematical Analysis]
	Distinguish \& compare Calculus vs. Mathematical Analysis.
\end{problem}
Analysis is more pure mathematics. Calculus is more applied mathematics.

\begin{problem}[Examples \& counterexamples in mathematical analysis -- Ví dụ \& phản ví dụ trong phân tích toán học]
	Find, from simple to advanced, examples \& counterexamples to each mathematical concepts \& mathematical results, including lemmas, propositions, theorems, \& consequences.
	
	-- Tìm các ví dụ \& phản ví dụ từ đơn giản đến nâng cao cho mỗi khái niệm toán học \& kết quả toán học, bao gồm các bổ đề, mệnh đề, định lý, \& hệ quả.
\end{problem}

\begin{problem}[Python {\tt SymPy}]
	Study {\tt SymPy} to support calculus \& mathematical analysis.
\end{problem}

\begin{definition}[Neighborhood, \cite{Wrede_Spiegel2010}, p. 6]
	The set of all points $x$ s.t. $|x - a| < \delta$, where $\delta > 0$, is called a $\delta$ {\rm neighborhood} of the point $a$. The set of all points $x$ s.t. $0 < |x - a| < \delta$, in which $x = a$ is excluded, is called a {\rm deleted $\delta$ neighborhood} of $a$ or an open ball of radius $\delta$ about $a$.
\end{definition}

\begin{theorem}[Bolzano--Weierstrass theorem]
	Every bounded infinite set has at least 1 limit point.
\end{theorem}

\begin{definition}[Algebraic- \& transcendental numbers -- số đại số \& số siêu việt]
	A number $x\in\mathbb{R}$ which is a solution to the {\rm polynomial equation}
	\begin{equation}
		\label{polynomial eqn}
		\sum_{i=0}^n a_ix^i = a_nx^n + a_{n-1}x^{n-1} + \cdots + a_1x + a_0 = 0,
	\end{equation}
	where $n\in\mathbb{N}^\star$, called the {\rm degree} of the equation, $a_i\in\mathbb{Z}$, $\forall i = 0,1,\ldots,n$, $a_n\ne0$, is called an {\rm algebraic number}. A number which cannot be expressed as a solution of any polynomial equation with integer coefficients is called a {\rm transcendental number}.
\end{definition}

\begin{theorem}[Common transcendental numbers]
	$\pi,e$ are transcendental.
\end{theorem}

\begin{theorem}[Countability of sets of algebraic- \& transcendental numbers]
	(i) The set of algebraic numbers is a countably infinite set. (ii) The set of transcendental numbers is noncountably infinite.
\end{theorem}

%------------------------------------------------------------------------------%

\chapter{Sequence -- Dãy Số}
\minitoc
\begin{itemize}\sf\small
	\item \textbf{sequence} [n] {\tt/}\textipa{'si:kw@ns}{\tt/} 1. [countable] \textit{sequence (of sth)} a set of events, actions, numbers, etc. which have a particular order \& which lead to a particular result; 2. [countable, uncountable] the order that events, actions, etc. happen in or should happen in; 3. [countable] a part of a film that deals with 1 subject or topic or consists of 1 scene. [v] 1. \textit{sequence sth} (specialist) to arrange things into a sequence; 2. \textit{sequence sth} (biology) to identify the order in which a set of genes or parts of molecules are arranged.
\end{itemize}
\textbf{\textsf{Resources -- Tài nguyên.}}
\begin{enumerate}
	\item \cite{Rudin1976}. {\sc Walter Rudin}. {\it Principles of Mathematical Analysis}. Chap. 3: Numerical Sequences \& Series.
	
	\item \cite{Tao_analysis_1}. {\sc Terence Tao}. {\it Analysis I}.
	
	\item \cite{Tao_analysis_2}. {\sc Terence Tao}. {\it Analysis II}.
	
	\item \cite{Wrede_Spiegel2010}. {\sc Robert Wrede, Murray R. Spiegel}. {\it Advanced Calculus}. 3e. Schaum's Outline Series. Chap. 2: Sequences.
\end{enumerate}
This section deals primarily with sequences of real- \& complex numbers, sequences in Euclidean spaces, or even in metric spaces.

-- Phần này chủ yếu đề cập đến các dãy số thực \& phức, các dãy trong không gian Euclid hoặc thậm chí trong không gian metric.

%------------------------------------------------------------------------------%

\section{Definition of a sequence -- Định nghĩa của dãy số}

\begin{definition}[Numerical sequence -- dãy số, \cite{Wrede_Spiegel2010}, p. 25]
	A {\rm sequence} is a set of numbers $u_1,u_2,\ldots$ in a definite order of arrangement (i.e., a {\rm correspondence} with the natural numbers or a subset thereof) \& formed according to a definite rule. Each number in the sequence is called a {\rm term}; $u_n$ is called the {\rm$n$th term}. The sequence is called {\rm finite} or {\rm infinite} according as there are or are not a finite number of terms. The sequence $u_1,u_2,\ldots$ is also designated briefly by $\{u_n\}$.
\end{definition}
Có thể hiểu khái niệm dãy (sequence) ở đây 1 cách tổng quát hơn là 1 dãy các đối tượng Toán học hoặc Tin học, e.g., dãy số phức $\{a_n\}_{n=1}^\infty$ là 1 dãy gồm các số $a_n\in\mathbb{C}$, $\forall n = 1,2,\ldots$, dãy các hàm số thực $\{f_n\}_{n=1}^\infty$ là 1 dãy gồm các hàm số $f_n:\mathbb{R}\to\mathbb{R}$, $\forall n = 1,2,\ldots$, hay dãy các dãy $\{\{a_{m,n}\}_{n=1}^\infty\}_{m=1}^\infty$ tức 1 dãy gồm các phần tử của dãy lại là các dãy số $\{a_{m,n}\}_{n=1}^\infty$, $\forall m = 1,2,\ldots$ Trước hết, ta tập trung là khái niệm dãy đơn giản nhất: dãy số -- numerical sequence, trước khi đến với khái niệm {\it hội tụ đều} của dãy hàm (uniform convergence of sequences of functions).

%------------------------------------------------------------------------------%

\section{Convergent- \& divergent sequences -- Dãy số hội tụ \& dãy số phân kỳ}

\begin{definition}[Limit of a sequence, \cite{Wrede_Spiegel2010}, p. 25]
	A number $l\in\mathbb{R}$ is called the {\rm limit} of an infinite sequence $u_1,u_2,\ldots$ if for any positive number $\epsilon$ we can find a positive number $N$ depending on $\epsilon$ s.t. $|u_n - l| < \epsilon$, $\forall n\in\mathbb{N}$, $n > N$. In such case we write $\lim_{n\to\infty} u_n = l$.
\end{definition}

\begin{definition}[Convergent sequences, \cite{Rudin1976}, Def. 3.1, p. 47]
	\label{def: convergent sequence in metric space}
	A sequence $\{p_n\}$ in a metric space $X$ is said to {\rm converge} if there is a point $p\in X$ with the following property: For every $\varepsilon > 0$ there is an integer $N_\varepsilon$ such that $n\ge N_\varepsilon = N(\varepsilon)$ implies that $d(p_n,p) < \varepsilon$. (Here $d$ denotes the {\rm distance} in $X$.) In this case we also say that $\{p_n\}$ converges to $p$, or that $p$ is the {\rm limit} of $\{p_n\}$, \& we write $p_n\to p$, or $p_n\to p$ as $n\to\infty$, or $\lim_{n\to\infty} p_n = p$. If $\{p_n\}$ does not converge, it is said to {\rm diverge}.
\end{definition}

\begin{remark}
	Định nghĩa \ref{def: convergent sequence in metric space} về dãy hội tụ trong các không gian metric không chỉ phụ thuộc vào bản thân dãy $\{p_n\}$ mà còn vào chính không gian metric $X$. Nhân tiện, vì ở đây đang xét không gian metric mà mỗi phần tử của nó được coi là 1 điểm (point), nên thành phần của dãy số được ký hiệu là $p_n$ để ám chỉ bản chất của mỗi phần tử của dãy là 1 điểm trong không gian metric tổng quát $X$. Nếu $X = \mathbb{R}$ hoặc $X = \mathbb{C}$ thì mỗi điểm trên trục số thực hoặc 1 số phức $z = a + bi$ tương ứng với điểm $(a,b)$ trên mặt phẳng phức $\mathbb{R}^2$, khi đó ký hiệu $p_n$ có thể được thay bởi các ký hiệu quen thuộc hơn cho số (numerals), e.g., $a_n,x_n,\ldots$
\end{remark}
In cases of possible ambiguity, we can be more precise \& specify ``convergent in $X$'' rather than ``convergent''.

-- Trong trường hợp có thể có sự mơ hồ, chúng ta có thể chính xác hơn \& cụ thể hơn ``hội tụ trong $X$'' thay vì ``hội tụ''.

\begin{definition}[Range of a sequence, bounded sequence]
	The set of all points $p_n$, $n = 1,2,\ldots$, is the {\rm range} of $\{p_n\}$. The range of a sequence may be a finite set, or it may be infinite. The sequence $\{p_n\}$ is said to be {\rm bounded} if its range is bounded.
\end{definition}

\begin{problem}
	Prove: (a) If $s_n = \frac{1}{n}$, then $\lim_{n\to\infty} s_n = 0$; the range is infinite, \& the sequence is bounded. (b) If $s_n = n^2$, the sequence $\{s_n\}$ is unbounded, is divergent, \& has infinite range. (c) If $s_n = 1 + \frac{(-1)^n}{n}$, the sequence $\{s_n\}$ converges to $1$, is bounded, \& has infinite range. (d) If $s_n = i^n$, the sequence $\{s_n\}$ is divergent, is bounded, \& has finite range. (e) If $s_n = 1$, $\forall n\in\mathbb{N}^\star$, then $\{s_n\}$ converges to $1$, is bounded, \& has finite range. (f) Find similar examples.
\end{problem}

\begin{theorem}[Some important properties of convergent sequences in metric spaces, \cite{Rudin1976}, Thm. 3.2, p. 48]
	Let $\{p_n\}$ be a sequence in a metric space $X$.
	\item(a) $\{p_n\}$ converges to $p\in X$ iff every neighborhood of $p$ contains all but finitely many of the terms of $\{p_n\}$.
	\item(b) {\rm(Uniqueness of limit)} If $p\in X,p'\in X$, \& if $\{p_n\}$ converges to $p$ \& to $p'$, then $p' = p$.
	\item(c) If $\{p_n\}$ converges, then $\{p_n\}$ is bounded.
	\item(d) If $E\subset X$ \& if $p$ is a limit point of $E$, then there is a sequence $\{p_n\}$ in $E$ such that $p = \lim_{n\to\infty} p_n$.
\end{theorem}
For sequences in Euclidean spaces $\mathbb{R}^d$, we can study the relation between convergence \& the algebraic operations.

\begin{theorem}[Algebraic operations on limit of sequences of complex numbers, \cite{Rudin1976}, Thm. 3.3, p. 49]
	Suppose $\{a_n\},\{b_n\}$ are complex sequences, \& $\lim_{n\to\infty} a_n = a$,$\lim_{n\to\infty} b_n = b$. Then:
	\item(a) $\lim_{n\to\infty} (a_n + b_n) = \lim_{n\to\infty} a_n + \lim_{n\to\infty} b_n = a + b$.
	\item(b) $\lim_{n\to\infty} ca_n = ca$, $\lim_{n\to\infty} (c + a_n) = c + \lim_{n\to\infty} a_n = c + a$, $\forall c\in\mathbb{C}$.
	\item(c) $\lim_{n\to\infty} a_nb_n = \lim_{n\to\infty} a_n\lim_{n\to\infty} b_n = ab$.
	\item(d) $\lim_{n\to\infty} \dfrac{1}{a_n} = \dfrac{1}{a}$, provided $a_n\ne0$, $\forall n\in\mathbb{N}^\star$, \& $a\ne0$.
\end{theorem}

\begin{theorem}[Algebraic operations on limit of sequences in Euclidean spaces, \cite{Rudin1976}, Thm. 3.4, p. 50]
	\item(a) Suppose ${\bf x}_n\in\mathbb{R}^d$, $\forall n\in\mathbb{N}^\star$, \& ${\bf x}_n = (x_{1,n},\ldots,x_{d,n})$. Then $\{{\bf x}_n\}$ converges to ${\bf x} = (x_1,\ldots,x_n)$ iff $\lim_{n\to\infty} x_{i,n} = x_i$, $\forall i = 1,\ldots,k$.
	\item(b) Suppose $\{{\bf x}_n\}_{n=1}^\infty,\{{\bf y}_n\}_{n=1}^\infty$ are sequences in $\mathbb{R}^d$, $\{a_n\}_{n=1}^\infty$ is a sequence of reals, \& ${\bf x}_n\to{\bf x},{\bf y}_n\to{\bf y},a_n\to a$. Then
	\begin{equation*}
		\lim_{n\to\infty} {\bf x}_n + {\bf y}_n = {\bf x} + {\bf y},\ \lim_{n\to\infty} {\bf x}_n\cdot{\bf y}_n = {\bf x}\cdot{\bf y},\ \lim_{n\to\infty} a_n{\bf x}_n = a{\bf x}.
	\end{equation*}
\end{theorem}

\begin{baitoan}[\cite{TLCT_dai_so_giai_tich_11}, 1.]
	Tìm $5$ số hạng đầu của dãy số: (a) $\{u_n\}_{n=1}^\infty$ với $u_n = \left(1 + \frac{1}{n}\right)^n$, $\forall n\in\mathbb{N}^\star$. (b) $\{u_n\}_{n=1}^\infty$ với $u_1 = 1,u_2 = -2,u_{n+1} = u_n - 2u_{n-1}$, $\forall n\in\mathbb{N}$, $n\ge2$. (c) Dãy $\{u_n\}_{n=1}^\infty$ các hợp số nguyên dương theo thứ tự tăng dần.
\end{baitoan}

\begin{baitoan}[\cite{TLCT_dai_so_giai_tich_11}, 2.]
	Cho dãy số $\{u_n\}_{n=1}^\infty$ xác định bởi $u_0 = 2,u_1 = 5,u_{n+1} = 5u_n - 6u_{n-1}$, $\forall n\in\mathbb{N}^\star$. Chứng minh, bằng phương pháp quy nạp Toán học, $u_n = 2^n + 3^n$, $\forall n\in\mathbb{N}^\star$.
\end{baitoan}

\begin{baitoan}[\cite{TLCT_dai_so_giai_tich_11}, 3.]
	Cho $\Delta ABC$ đều có cạnh bằng $4$. Trên cạnh $BC$ lấy điểm $A_1$ sao $CA_1 = 1$. Gọi $B_1$ là hình chiếu của $A_1$ trên $CA$, $C_1$ là hình chiếu của $B_1$ trên $AB$, $A_2$ là hình chiếu của $C_1$ trên $BC$, $B_2$ là hình chiếu của $A_2$ trên $CA,\ldots$ \& cứ thế tiếp tục. Đặt $u_n = CA_n$. Cho dãy số $\{u_n\}_{n=1}^\infty$ bởi hệ thức truy hồi.
\end{baitoan}

\begin{baitoan}[\cite{TLCT_dai_so_giai_tich_11}, 4.]
	Xét tính tăng, giảm, bị chặn trên, bị chặn dưới, bị chặn của dãy số $\{u_n\}_{n=1}^\infty$ với: (a) $u_n = n^3 - 3n^2 + 5n - 7$, $\forall n\in\mathbb{N}^\star$. (b) $u_n = \dfrac{n + 1}{3^n}$, $\forall n\in\mathbb{N}^\star$. (c) $u_n = \sqrt{n} - \sqrt{n - 1}$, $\forall n\in\mathbb{N}^\star$.
\end{baitoan}

\begin{baitoan}[\cite{TLCT_dai_so_giai_tich_11}, 5.]
	Xét 2 dãy số $\{u_n\}_{n=1}^\infty$, $\{v_n\}_{n=1}^\infty$ xác định bởi
	\begin{equation*}
		u_n = \left(1 + \frac{1}{n}\right)^n,\ v_n = \left(1 + \frac{1}{n}\right)^{n+1},\ \forall n\in\mathbb{N}^\star.
	\end{equation*}
	Chứng minh: (a) $\{u_n\}_{n=1}^\infty$ tăng, $\{v_n\}_{n=1}^\infty$ giảm. (b) $\{u_n\}_{n=1}^\infty$, $\{v_n\}_{n=1}^\infty$ đều bị chặn.
\end{baitoan}

\begin{baitoan}[\cite{TLCT_dai_so_giai_tich_11}, 6.]
	Cho dãy số $\{u_n\}_{n=1}^\infty$ xác định bởi $u_1 = 1,u_{n+1} = u_n + (n + 1)2^n$, $\forall n\in\mathbb{N}^\star$. Chứng minh: (a) $\{u_n\}_{n=1}^\infty$ tăng. (b) $u_n = 1 + (n - 1)2^n$, $\forall n\in\mathbb{N}^\star$.
\end{baitoan}

\begin{proof}
	(a) $u_{n+1} - u_n = (n + 1)2^n > 0$, $\forall n\in\mathbb{N}^\star$, nên $\{u_n\}_{n=1}^\infty$ tăng.
\end{proof}

\begin{baitoan}[\cite{TLCT_dai_so_giai_tich_11}, 7.]
	Cho dãy số $\{u_n\}_{n=1}^\infty$ xác định bởi $u_1 = 1,u_2 = 2,u_{n+1} = au_n - u_{n-1}$, $\forall n\in\mathbb{N}^\star$, $n\ge2$. Chứng minh: (a) Với $a = \sqrt{3}$ thì $\{u_n\}_{n=1}^\infty$ tuần hoàn. (b) Với $a = \frac{3}{2}$ thì $\{u_n\}_{n=1}^\infty$ không tuần hoàn.
\end{baitoan}

%------------------------------------------------------------------------------%

\subsection{Arithmetic progression -- Cấp số cộng}
\textbf{\textsf{Resources -- Tài nguyên.}}
\begin{enumerate}
	\item \href{https://en.wikipedia.org/wiki/Arithmetic_progression}{Wikipedia{\tt/}arithmetic progression}.
	\item \cite{SGK_Toan_11_CD_tap_1}. {\sc Đỗ Đức Thái, Phạm Xuân Chung, Nguyễn Sơn Hà, Nguyễn Thị Phương Loan, Phạm Sỹ Nam, Phạm Minh Phương}. {\it Toán 11 Tập 1. Cánh Diều}.
	\item \cite{SBT_Toan_11_CD_tap_1}. {\sc Đỗ Đức Thái, Phạm Xuân Chung, Nguyễn Sơn Hà, Nguyễn Thị Phương Loan, Phạm Sỹ Nam, Phạm Minh Phương}. {\it Bài Tập Toán 11 Tập 1. Cánh Diều}.
\end{enumerate}
An {\it arithmetic progression} or {\it arithmetic sequence} is a sequence of numbers (real or complex) such that the difference from any succeeding term to its preceding term remains constant throughout the sequence. The constant difference is called {\it common difference} of that arithmetic progression.

-- 1 {\it cấp số cộng} hoặc {\it dãy số cộng} là 1 dãy số (thực hoặc phức) sao cho hiệu số từ bất kỳ số hạng tiếp theo nào đến số hạng trước nó vẫn không đổi trong suốt dãy số. Hiệu số không đổi được gọi là {\it  công sai}, i.e., hiệu số chung của cấp số cộng đó.

If the initial term of an arithmetic progression is $a_1$ \& the common difference of successive members is $d\in\mathbb{C}$, then the $n$th term of the sequence $\{a_n\}_{n=1}^\infty$ is given by
\begin{equation*}
	a_n = a_1 + (n - 1)d,\ \forall n\in\mathbb{N}^\star.
\end{equation*}
A finite portion of an arithmetic progression is called a {\it finite arithmetic progression} \& sometimes just called an {\it arithmetic progression}. The sum of a finite arithmetic progression is called an {\it arithmetic series}:
\begin{equation*}
	S_n = \sum_{i=1}^n a_i = a_1 + a_2 + \cdots + a_n = \frac{n(a_1 + a_2)}{2},\ \forall n\in\mathbb{N}^\star.
\end{equation*}

\begin{proof}
	We can prove by mathematical induction: $a_i + a_{n-i} = a_1 + a_n$, $\forall n\in\mathbb{N}^\star$, hence $2S_n = \sum_{i=1}^n a_i + \sum_{i=n}^1 a_i = \sum_{i=1}^n (a_i + a_{n-i}) = \sum_{i=1}^n (a_1 + a_n) = n(a_1 + a_n)\Rightarrow S_n = \frac{n(a_1 + a_n)}{2}$, $\forall n\in\mathbb{N}^\star$.
\end{proof}
-- Nếu số hạng đầu của 1 cấp số cộng là $a_1$ \& hiệu chung của các phần tử liên tiếp là $d\in\mathbb{C}$, thì số hạng thứ $n$ của dãy $\{a_n\}_{n=1}^\infty$ được cho bởi
\begin{equation*}
	a_n = a_1 + (n - 1)d,\ \forall n\in\mathbb{N}^\star.
\end{equation*}
Một phần hữu hạn của 1 cấp số cộng được gọi là {\it cấp số cộng hữu hạn} \& đôi khi chỉ được gọi là {\it cấp số cộng}. Tổng của 1 cấp số cộng hữu hạn được gọi là {\it cấp số cộng}:
\begin{equation*}
	S_n = \sum_{i=1}^n a_i = a_1 + a_2 + \cdots + a_n = \frac{n(a_1 + a_2)}{2},\ \forall n\in\mathbb{N}^\star.
\end{equation*}

\begin{baitoan}[\cite{TLCT_dai_so_giai_tich_11}, 8.]
	(a) Cho cấp số cộng $\{u_n\}_{n=1}^\infty$ có $u_{13} = 31,u_{31} = -13$. Tìm số hạng tổng quát của cấp số đó. (b) Cho cấp số cộng $\{u_n\}_{n=1}^\infty$ có $u_m = a,u_n = b$, với $m,n\in\mathbb{N}$, $m\ne n$, $a,b\in\mathbb{C}$. Tìm số hạng tổng quát của cấp số đó.
\end{baitoan}

\begin{proof}[Giải]
	(a) Giải hệ phương trình
	\begin{equation*}
		\left\{\begin{split}
			u_1 + 12d &= u_{13} = 31,\\
			u_1 + 30d &= -13
		\end{split}\right.\Leftarrow\left\{\begin{split}
			u_1 &= \frac{181}{3},\\
			d &= -\frac{22}{9}.
		\end{split}\right.\Rightarrow u_n = \frac{181}{3} -  \frac{22}{9}(n - 1) = \frac{565}{9} - \frac{22}{9}n,\ \forall n\in\mathbb{N}^\star.
	\end{equation*}
	(b) Giải hệ phương trình
	\begin{equation*}
		\left\{\begin{split}
			u_1 + (m - 1)d &= u_m = a,\\
			u_1 + (n - 1)d &= u_n = b,
		\end{split}\right.\Leftarrow\left\{\begin{split}
			u_1 &= \frac{a - b - an + bm}{m - n},\\
			d &= \frac{a - b}{m - n}.
		\end{split}\right.
	\end{equation*}
	Suy ra
	\begin{equation*}
		u_k = \frac{a - b - an + bm}{m - n} + (k - 1)\frac{a - b}{m - n} = \frac{(a - b)k - an + bm}{m - n},\ \forall k\in\mathbb{N}^\star.
	\end{equation*}
	(Kiểm tra lại: $u_m = \dfrac{(a - b)m - an + bm}{m - n} = \dfrac{a(m - n)}{m - n} = a,u_n = \dfrac{(a - b)n - an + bm}{m - n} = \dfrac{b(m - n)}{m - n} = b$.)
\end{proof}

\begin{baitoan}[\cite{TLCT_dai_so_giai_tich_11}, 9.]
	Số đo 3 góc của 1 tam giác vuông lập thành 1 cấp số cộng. Tìm số đo 3 góc đó.
\end{baitoan}

\begin{proof}[Giải]
	$\Delta ABC$ vuông tại $A$, $\widehat{B} > \widehat{C}$, giải hệ phương trình $\widehat{B} + \widehat{C} = 90^\circ$, $\widehat{A} + \widehat{C} = 90^\circ + \widehat{C} = 2\widehat{B}$ được $\widehat{B} = 60^\circ,\widehat{C} = 30^\circ$.
\end{proof}

\begin{baitoan}[Mở rộng \cite{TLCT_dai_so_giai_tich_11}, 9.]
	(a) Tìm điều kiện để số đo 3 góc của 1 tam giác lập thành 1 cấp số cộng. (b) Cho $n\in\mathbb{N},n\ge3$. Tìm điều kiện để số đo $n$ góc của 1 đa giác lồi lập thành 1 cấp số cộng. Suy ra cho tứ giác lồi.
\end{baitoan}

\begin{proof}[Giải]
	(a) Gọi $a,a + d,a + 2d$ là 3 góc của tam giác thỏa giả thiết, có $a + (a + d ) + (a + 2d) = 180^\circ\Leftrightarrow a + d = 60^\circ$. Cho $a = d = 30^\circ$ thu được bài toán trước. (b) Gọi $\{a + id\}_{i=0}^{n-1}$ là $n$ góc của đa giác lồi $n$ cạnh thỏa giả thiết, có $\sum_{i=0}^{n-1} a + id = na + \dfrac{n(n - 1)}{2}d = (n - 2)180^\circ$. Với tứ giác lồi, cho $n = 4$ được $4a + 6d = 2\cdot180^\circ\Leftrightarrow 2a + 3d = 180^\circ$.
\end{proof}

\begin{baitoan}[\cite{TLCT_dai_so_giai_tich_11}, 10.]
	(a) Tổng của số hạng thứ $3$ \& số hạng thứ $9$ của 1 cấp số cộng bằng $8$. Tính tổng của $11$ số hạng đầu tiên của cấp số đó. (b) Tổng của số hạng thứ $m$ \& số hạng thứ $n$ của 1 cấp số cộng bằng $a$, với $m,n\in\mathbb{N}$, $m\ne n$, $a\in\mathbb{C}$. Tính tổng của $N\in\mathbb{N}$ số hạng đầu tiên của cấp số đó.
\end{baitoan}

\begin{baitoan}[\cite{TLCT_dai_so_giai_tich_11}, 11.]
	Gọi $S_n$ là tổng $n\in\mathbb{N}^\star$ số hạng đầu tiên của 1 cấp số cộng. Biết $m,n\in\mathbb{N}^\star$, $m\ne n$ thỏa $S_m = S_n$. Chứng minh $S_{m+n} = 0$.
\end{baitoan}

\begin{baitoan}[\cite{TLCT_dai_so_giai_tich_11}, 12.]
	Chu kỳ bán rã của nguyên tố phóng xạ poloni $210$ là $138$ ngày, i.e., sau $138$ ngày, khối lượng của nguyên tố đó chỉ còn 1 nửa. Tính khối lượng còn lại của $20$ gram poloni $210$ sau $7314$ ngày (khoảng $20$ năm).
\end{baitoan}

\begin{proof}[Giải]
	Gọi $t_{1/2}\in\mathbb{N}^\star$ (ngày) là chu kỳ bán rã (half-life) của 1 chất, $m_0\in(0,\infty)$ (g) là khối lượng ban đầu (initial mass) của poloni. Sau $n\in\mathbb{N}$ ngày, khối lượng còn lại (remaining mass) của poloni bằng
	\begin{equation*}
		m(m_0,t_{1/2},n) = m_02^{-\frac{n}{t_{1/2}}},\ \forall n\in\mathbb{N}.
	\end{equation*}
	Áp dụng cho poloni với $m_0 = 20$ g, $t_{1/2}  =138$ ngày, $n = 7314$ ngày: $m(20,138,7314) = 20\cdot2^{-\frac{7314}{138}} = 20\cdot2^{53} = \frac{5}{2^{51}}$ g.
\end{proof}
Về chu kỳ bán rã, see, e.g., \href{https://en.m.wikipedia.org/wiki/Half-life}{Wikipedia{\tt/}half-life}, or \href{https://vi.wikipedia.org/wiki/Chu_k%E1%BB%B3_b%C3%A1n_r%C3%A3}{Wikipedia{\tt/}chu kỳ bán rã}.

%------------------------------------------------------------------------------%

\subsection{Geometric progression -- Cấp số nhân}
\textbf{\textsf{Resources -- Tài nguyên.}}
\begin{enumerate}
	\item \href{https://en.wikipedia.org/wiki/Geometric_progression}{Wikipedia{\tt/}geometric progression}.
\end{enumerate}

\begin{baitoan}[\cite{TLCT_dai_so_giai_tich_11}, 13.]
	Tính: (a) Tổng tất cả các số hạng của 1 cấp số nhân có $100$ số hạng với số hạng đầu là $1$, công bội là $\frac{1}{2}$. (b) Tổng tất cả các số hạng của 1 cấp số nhân biết số hạng đầu bằng $18$, số hạng thứ 2 bằng $54$ \& số hạng cuối bằng $39366$.
\end{baitoan}

\begin{proof}[Giải]
	(a) $S_{100} = u_1\cdot\dfrac{1 - q^{100}}{1 - q} = 2(1 - 2^{-100}) = 2 - \dfrac{1}{2^{99}}$. (b) $q = \dfrac{u_2}{u_1} = 3$. $u_n = u_1q^{n-1} = 18\cdot3^{n-1} = 39366\Rightarrow n = 8$. $S_8 = u_1\cdot\dfrac{1 - q^8}{1 - q} = 18\cdot\dfrac{1 - 3^8}{1 - 3} = 59040$.
\end{proof}

\begin{baitoan}[Mở rộng \cite{TLCT_dai_so_giai_tich_11}, 13.]
	Tính: (a) Tổng tất cả các số hạng của 1 cấp số nhân có $n\in\mathbb{N}^\star$ số hạng với số hạng đầu là $u_1\in\mathbb{R}$, công bội là $q\in\mathbb{R}^\star$. (b) Tổng tất cả các số hạng của 1 cấp số nhân biết số hạng đầu bằng $u_1 = a\in\mathbb{R}$, số hạng thứ 2 bằng $u_2\in\mathbb{R}$ \& số hạng cuối bằng $u_n\in\mathbb{R}$.
\end{baitoan}

\begin{baitoan}[\cite{TLCT_dai_so_giai_tich_11}, 14.]
	Số hạng thứ 2, số hạng đầu, \& số hạng thứ 3 của 1 cấp số cộng với công sai $\ne0$ theo thứ tự đó lập thành 1 cấp số nhân. Tìm công bội của cấp số nhân đó.
\end{baitoan}

\begin{proof}[Giải]
	Gọi $u_1,u_2,u_3$ lần lượt là 3 số hạng đầu tiên của cấp số cộng thì $u_1^2 = u_2u_3\Leftrightarrow u_1^2 = (u_1 + d)(u_1 + 2d)\Leftrightarrow u_1 = -\frac{2d}{3}\Rightarrow u_2 = \frac{d}{3}\Rightarrow q = \frac{u_1}{u_2} = -\frac{2d}{3}:\frac{d}{3} = -2$.
\end{proof}

\begin{baitoan}[\cite{TLCT_dai_so_giai_tich_11}, 15.]
	Tìm số hạng tổng quát \& tính tổng $n\in\mathbb{N}^\star$ số hạng đầu tiên của dãy số $\{u_n\}_{n=1}^\infty$ xác định bởi $u_1 = 1,u_{n+1} = \frac{1}{2}u_n + 1$, $\forall n\in\mathbb{N}^\star$.
\end{baitoan}

\begin{proof}[Giải]
	Đặt $v_n\coloneqq u_n - 2$, $\forall n\in\mathbb{N}^\star$ thì $v_{n+1} = \frac{1}{2}v_n$ nên $\{v_n\}_{n=1}^\infty$ là 1 cấp số nhân với $v_1 = u_1 - 2 = 1 - 2 = -1$ \& công bội $q = \frac{1}{2}$, suy ra $v_n = -\frac{1}{2^{n-1}}\Rightarrow u_n = 2 - \frac{1}{2^{n-1}}$, $\forall n\in\mathbb{N}^\star$, nên
	\begin{equation*}
		S_n = \sum_{i=1}^n u_i = S_n = \sum_{i=1}^n 2 - \frac{1}{2^{i-1}} = 2n - \sum_{i=0}^{n-1} \frac{1}{2^i} = 2n - \frac{1 - \frac{1}{2^n}}{1 - \frac{1}{2}} = 2n - 2\left(1 - \frac{1}{2^n}\right) = 2(n - 1) + \frac{1}{2^{n-1}},\ \forall n\in\mathbb{N}^\star.
	\end{equation*}
	(Công thức truy hồi của dãy tổng $\{S_n\}_{n=1}^\infty$: $S_1 = u_1 = 1$, $S_{n+1} = S_n + u_{n+1} = S_n + 2 - \frac{1}{2^n}$.)
\end{proof}

\begin{baitoan}[\cite{TLCT_dai_so_giai_tich_11}, 16.]
	Tìm công thức cho số hạng tổng quát của dãy số xác định bởi $a_1 = a,a_{n+1} = qa_n + d\alpha^n$, $\forall n\in\mathbb{N}^\star$, $\alpha\ne q$.
\end{baitoan}

\begin{proof}[Giải]
	Đặt $v_n\coloneq a_n + \dfrac{d}{q - \alpha}\alpha^n$, $\forall n\in\mathbb{N}^\star$, $\Rightarrow v_{n+1} = qv_n$, $\forall n\in\mathbb{N}^\star$, i.e., $\{v_n\}_{n=1}^\infty$ là 1 cấp số nhân với số hạng đầu $v_1 = a + \dfrac{d\alpha}{q - \alpha}$ \& công bội $q$, suy ra $v_n = v_1q^{n-1} = \left(a + \dfrac{d\alpha}{q - \alpha}\right)q^{n-1}\Rightarrow a_n = \left(a + \dfrac{d\alpha}{q - \alpha}\right)q^{n-1} - \dfrac{d}{q - \alpha}\alpha^n$, $\forall  n\in\mathbb{N}^\star$.
\end{proof}

\begin{baitoan}[\cite{TLCT_dai_so_giai_tich_11}, 17.]
	Gọi $F_n$ là số hạng thứ $n$ của dãy Fibonacci, xác định bởi $F_0 = 0,F_1 = 1,F_{n+1} = F_n + F_{n-1}$, $\forall n\in\mathbb{N}^\star$. Chứng minh: (a) $F_n^2 + F_{n+1}^2 = F_{2n+1}$, $\forall n\in\mathbb{N}^\star$. (b) $F_{m+n+1} = F_{m+1}F_{n+1} + F_mF_n$, $\forall m,n\in\mathbb{N}$. (c) $F_{3n} = F_{n+1}^3 + F_n^3 - F_{n-1}^3$, $\forall m,n\in\mathbb{N}$.
\end{baitoan}

\begin{baitoan}[\cite{TLCT_dai_so_giai_tich_11}, 18.]
	Dãy Lucas là dãy số xá định bởi $L_1 = 1,L_2 = 3,L_{n+2} = L_{n+1} + L_n$, $\forall n\in\mathbb{N}^\star$. Tìm công thức tổng quát cho $L_n$.
\end{baitoan}

\begin{baitoan}[\cite{TLCT_dai_so_giai_tich_11}, 19.]
	Giả sử $F_n,L_n$ tương ứng là số hạng thứ $n$ của dãy Fibonnaci \& dãy Lucas. Chứng minh $F_{2n} = F_nL_n$, $\forall n\in\mathbb{N}^\star$.
\end{baitoan}

\begin{baitoan}[\cite{TLCT_dai_so_giai_tich_11}, 20.]
	Lập dãy số Farey bậc $9$, bậc $10$.
\end{baitoan}

\begin{baitoan}[\cite{TLCT_dai_so_giai_tich_11}, 21.]
	Chứng minh nếu $\dfrac{a}{b},\dfrac{c}{d}$ là 2 phân số với $ad - bc = 1$, $d\ge b$ thì $\dfrac{c}{d} < \dfrac{a}{b}$ \& $\dfrac{c}{d},\dfrac{a}{b}$ là 2 số hạng liên tiếp trong dãy số Farey bậc $d$.
\end{baitoan}

\begin{baitoan}[\cite{TLCT_dai_so_giai_tich_11}, 22.]
	Số hạng thứ 3, thứ 4, thứ 7, \& cuối cùng của 1 cấp số cộng không hằng lập thành 1 cấp số nhân. Tìm số số hạng của cấp số này.
\end{baitoan}

\begin{baitoan}[\cite{TLCT_dai_so_giai_tich_11}, 23.]
	Số hạng thứ 4 của 1 cấp số cộng bằng $4$. Tìm {\rm GTNN} của tổng các tích đôi 1 của 3 số hạng đầu.
\end{baitoan}

\begin{baitoan}[\cite{TLCT_dai_so_giai_tich_11}, 24.]
	2 cấp số cộng có cùng số phần tử. Tỷ lệ giữ số hạng cuối của cấp số đầu \& số hạng đầu của cấp số thứ 2 bằng tỷ lệ giữa số hạng cuối của cấp số thứ 2 \& số hạng đầu của cấp số thứ nhất \& bằng $4$. Tỷ lệ giữa tổng các số hạng của cấp số thứ nhất \& tổng các số hạng của cấp số thứ 2 bằng $2$. Tìm tỷ lệ giữa các công sai của 2 cấp số.
\end{baitoan}

\begin{baitoan}[\cite{TLCT_dai_so_giai_tich_11}, 25.]
	3 số lập thành 1 cấp số nhân. Nếu ta trừ số hạng thứ 3 cho $4$  thì ta được 1 cấp số cộng. Nếu lại trừ các số hạng thứ 2 \& thứ 3 của cấp số cộng thu được cho $1$, ta lại được 1 cấp số nhân. Tìm 3 số ban đầu.
\end{baitoan}

\begin{baitoan}[\cite{TLCT_dai_so_giai_tich_11}, 26.]
	Tính tổng: (a) $\sum_{i=1}^n \dfrac{1}{i(i + 1)}$. (b) $\sum_{i=1}^n i(i + 1)$. (c) $\sum_{i=1}^n \dfrac{i}{2^i}$.
\end{baitoan}

\begin{baitoan}[\cite{TLCT_dai_so_giai_tich_11}, 27.]
	Tìm đa thức $P(x)$ thỏa $P(x) - P(x - 1) = x^3$, $\forall x\in\mathbb{R}$. Từ đó lập công thức tính tổng $S_n^{(3)} = \sum_{i=1}^n i^3$.
\end{baitoan}

\begin{baitoan}[\cite{TLCT_dai_so_giai_tich_11}, 28.]
	Cho dãy số thực $\{x_n\}_{n=1}^\infty$ xác định bởi $x_0 = 1,x_{n+1} = 2 + \sqrt{x_n} - 2\sqrt{1 + \sqrt{x_n}}$, $\forall n\in\mathbb{N}^\star$. Xác định dãy $\{y_n\}_{n=1}^\infty$ bởi công thức $y_n = \sum_{i=1}^n x_i2^i$, $\forall n\in\mathbb{N}^\star$. Tìm công thức tổng quát của dãy $\{y_n\}_{n=1}^\infty$.
\end{baitoan}

\begin{baitoan}[\cite{TLCT_dai_so_giai_tich_11}, 29.]
	2 dãy số nguyên $\{a_n\}_{n=0}^\infty,\{b_n\}_{n=0}^\infty$ được xác định bởi:
	\begin{align*}
		a_0 &= 0,\ a_1 = 1,\ a_{n+2} = 4a_{n+1} - a_n,\\
		b_0 &= 1,\ b_1 = 2,\ b_{n+2} = 4b_{n+1} - b_n.
	\end{align*}
	Chứng minh $b_n^2 = 3a_n^2 + 1$, $\forall n\in\mathbb{N}^\star$.
\end{baitoan}

\begin{baitoan}[\cite{TLCT_dai_so_giai_tich_11}, 30.]
	Cho dãy số $\{x_n\}_{n=1}^\infty$ xác định bởi $x_1 = \dfrac{2}{3},x_{n+1} = \dfrac{x_n}{2(2n + 1)x_n + 1}$, $\forall n\in\mathbb{N}^\star$. Tính tổng của: (a) $2010$ số hạng đầu tiên của $\{x_n\}_{n=1}^\infty$. (b) $n\in\mathbb{N}^\star$ số hạng đầu tiên của $\{x_n\}_{n=1}^\infty$.
\end{baitoan}

\begin{baitoan}[\cite{TLCT_dai_so_giai_tich_11}, 31.]
	Tính: (a) $\sum_{i=1}^{101} \dfrac{a_i^3}{1 - 3a_i + 3a_i^2}$ với $a_n = \dfrac{n}{101}$. (b) $\sum_{i=1}^n \dfrac{a_i^3}{1 - 3a_i + 3a_i^2}$ với $a_i = \dfrac{i}{n}$.
\end{baitoan}

\begin{baitoan}[\cite{TLCT_dai_so_giai_tich_11}, 32.]
	Cho dãy số $\{a_n\}_{n=1}^\infty$ xác định bởi $a_1 = \dfrac{1}{2}$, $a_{n+1} = \dfrac{a_n^2}{a_n^2 - a_n + 1}$. Chứng minh $\sum_{i=1}^n a_i < 1$, $n\in\mathbb{N}^\star$.
\end{baitoan}

\begin{baitoan}[\cite{TLCT_dai_so_giai_tich_11}, 33.]
	Cho dãy số $\{x_n\}_{n=0}^\infty$ xác định bởi $x_0 = x_1 = 1$,$n(n + 1)x_{n+1} = n(n - 1)x_n - (n - 2)x_{n-1}$. Tìm $\sum_{i=0}^n \dfrac{x_i}{x_{i+1}}$.
\end{baitoan}

\begin{baitoan}[\cite{TLCT_dai_so_giai_tich_11}, 34.]
	Cho dãy số $\{x_n\}_{n=1}^\infty$ xác định bởi $x_1 = 2,x_{n+1} = \dfrac{2 + x_n}{1 - 2x_n}$, $\forall n\in\mathbb{N}^\star$. Chứng minh: (a) $x_n\ne0$, $\forall n\in\mathbb{N}^\star$. (b) $\{x_n\}_{n=1}^\infty$ không tuần hoàn.
\end{baitoan}

%------------------------------------------------------------------------------%

\section{Subsequences -- Dãy con}

\begin{definition}
	Given a sequence $\{p_n\}_{n=1}^\infty$, consider a sequence $\{n_k\}$ of positive integers, s.t. $n_1 < n_2 < \cdots$. Then the sequence $\{p_{n_i}\}_{i=1}^\infty$ is called a {\rm subsequence} of $\{p_n\}_{n=1}^\infty$. If $\{p_{n_i}\}_{i=1}^\infty$ converges, its limit is called a {\rm subsequential limit} of $\{p_n\}_{n=1}^\infty$.
\end{definition}

\begin{problem}
	Prove that $\{p_n\}_{n=1}^\infty$ converges to $p$ iff every subsequence of $\{p_n\}_{n=1}^\infty$ converges to $p$.
\end{problem}

\begin{theorem}[\cite{Rudin1976}, Thm. 3.6, p. 50]
	\item(a) If $\{p_n\}_{n=1}^\infty$ is a sequence in a compact metric space $X$, then some subsequence of $\{p_n\}_{n=1}^\infty$ converges to a point of $X$.
	\item(b) Every bounded sequence in $\mathbb{R}^d$ contains a convergent subsequence.
\end{theorem}

\begin{theorem}[\cite{Rudin1976}, Thm. 3.7, p. 52]
	The subsequential limits of a sequence $\{p_n\}_{n=1}^\infty$ in a metric space $X$ form a closed subset of $X$.
\end{theorem}

%------------------------------------------------------------------------------%

\section{Limit of sequences -- Giới hạn của dãy số}

\begin{dinhnghia}[Dãy số thực có giới hạn $0$,  \cite{SGK_Toan_11_CD_tap_1}, p. 60]
	\label{def: sequence lim 0}
	Dãy số thực $\{u_n\}_{n=1}^\infty\subset\mathbb{R}$ có giới hạn $0$ khi $n$ dần tới dương vô cực nếu $|u_n|$ có thể nhỏ hơn 1 số dương bé tùy ý, kể từ 1 số hạng nào đó trở đi, ký hiệu $\lim_{n\to\infty} u_n = 0$.
\end{dinhnghia}
{\sf Notation.} Ngoài ký hiệu, $\lim_{n\to\infty} u_n = 0$, ta cũng sử dụng các ký hiệu: $\lim u_n = 0$ hay $u_n\to0$ khi $n\to\infty$.

\begin{nhanxet}
	Nếu $u_n$ ngày càng gần tới $0$ khi $n$ ngày càng lớn thì $\lim u_n = 0$.
\end{nhanxet}

\begin{dinhnghia}[Dãy số thực có giới hạn $0$ theo ngôn ngữ $\varepsilon$-$\delta$]
	\label{def: sequence lim 0: epsilon-delta}
	1 dãy số thực $\{u_n\}_{n=1}^\infty$ có giới hạn $0$ nếu \& chỉ nếu với mọi số nguyên dương $\varepsilon$, tồn tại 1 số nguyên dương $N_\varepsilon\in\mathbb{N}^\star$ để $|u_n| < \varepsilon$ kể từ chỉ số $N_\varepsilon$ đó trở đi:	
	\begin{equation*}
		\forall\varepsilon\in(0,\infty),\ \exists N_\varepsilon\in\mathbb{N}^\star,\ |u_n| < \varepsilon,\ \forall n\ge N_\varepsilon,
	\end{equation*}
	hay tương đương:
	\begin{equation*}
		\forall\varepsilon\in(0,\infty),\ \exists N_\varepsilon\in\mathbb{N}^\star,\ n\ge N_\varepsilon\Rightarrow|u_n| < \varepsilon.
	\end{equation*}
\end{dinhnghia}

\begin{remark}[Optimal{\tt/}smallest{\tt/}best indices -- Các chỉ số tối ưu{\tt/}nhỏ nhất{\tt/}tốt nhất]
	Định nghĩa \ref{def: sequence lim 0: epsilon-delta} chỉ yêu cầu tồn tại $N_\varepsilon\in\mathbb{N}^\star$ đủ lớn với mỗi $\varepsilon\in(0,\infty)$. Tuy nhiên nếu tìm được chỉ số $N_\varepsilon\in\mathbb{N}^\star$ tối ưu, i.e., chỉ số nhỏ nhất trong các chỉ số $N_\varepsilon$ thỏa mãn, i.e.:
	\begin{equation*}
		N_\varepsilon^{\rm opt}\coloneqq\min\{N_\varepsilon\in\mathbb{N};|u_n| < \varepsilon,\ \forall n\ge N_\varepsilon\} = \min\{N_\varepsilon\in\mathbb{N};n\ge N_\varepsilon\Rightarrow|u_n| < \varepsilon\}.
	\end{equation*}
	thì ta có thể sử dụng ký hiệu $N_\varepsilon^{\rm opt}$ để chỉ rõ tính tối ưu (i.e., nhỏ nhất, chặt{\tt/}ngặt nhất) của $N_\varepsilon$.
\end{remark}

\begin{remark}[Ceil- vs. floor functions]
	\begin{equation*}
		\lceil x\rceil = \left\{\begin{split}
			&\lfloor x\rfloor&&\mbox{if } x\in\mathbb{Z},\\
			&\lfloor x\rfloor + 1&&\mbox{if } x\in\mathbb{R}\backslash\mathbb{Z}.
		\end{split}\right. = \lfloor x\rfloor + \chi_{\mathbb{R}\backslash\mathbb{Z}}(x),\ \forall x\in\mathbb{R}.
	\end{equation*}
\end{remark}

\begin{baitoan}[\cite{SGK_Toan_11_CD_tap_1}, p. 60]
	Chứng minh $\lim_{n\to\infty} u_n = 0$ \& chỉ ra $N_\varepsilon^{\rm opt}$ với $\varepsilon = 0.1,0.01,10^{-n}$, $\forall n\in\mathbb{N}$, \& với $\varepsilon > 0$ bất kỳ: (a) $u_n = 0$. (b) $u_n = \dfrac{(-1)^n}{n}$. (c) $u_n = \dfrac{1}{\sqrt{n}}$. (d) $u_n = -\dfrac{1}{\sqrt{n}}$. (e) $u_n = \dfrac{(-1)^n}{\sqrt{n}}$ (f) $u_n = \dfrac{a\epsilon_n}{n^b}$ với $\{\epsilon_n\}_{n=1}^\infty\subset\{\pm1\}$, $a\in\mathbb{R},b\in(0,\infty)$.
\end{baitoan}

\begin{proof}
	(a) Lấy $\varepsilon > 0$ bất kỳ, có $|u_n| = |0| = 0 < \varepsilon$, $\forall n\ge1$. Theo định nghĩa giới hạn theo ngôn ngữ $\varepsilon$-$\delta$, suy ra $\lim_{n\to\infty} u_n = 0$. Ta có thể chọn $N_\varepsilon\coloneqq N_\varepsilon^{\rm opt} = 1$, $\forall\varepsilon > 0$, nên $N_{0.1}^{\rm opt} = N_{0.01}^{\rm opt} = N_{10^{-n}}^{\rm opt} = 1$, $\forall n\in\mathbb{N}$.
	
	\item(b) Lấy $\varepsilon > 0$ bất kỳ, có $|u_n| = \left|\dfrac{(-1)^n}{n}\right| = \dfrac{1}{n}$, nên $|u_n| < \varepsilon\Leftrightarrow\dfrac{1}{n} < \varepsilon\Leftrightarrow n > \dfrac{1}{\varepsilon}\Rightarrow N_\varepsilon^{\rm opt} = \left\lfloor\dfrac{1}{\varepsilon}\right\rfloor + 1$, nên nếu chọn $N_\varepsilon\coloneqq N_\varepsilon^{\rm opt} = \left\lfloor\dfrac{1}{\varepsilon}\right\rfloor + 1$ thì $|u_n| < \varepsilon,\ \forall n\ge N_\varepsilon$. Theo định nghĩa giới hạn theo ngôn ngữ $\varepsilon$-$\delta$, suy ra $\lim_{n\to\infty} u_n = 0$.
	
	\item(c) Lấy $\varepsilon > 0$ bất kỳ, có $|u_n| = \left|\dfrac{1}{\sqrt{n}}\right| = \dfrac{1}{\sqrt{n}}$, nên $|u_n| < \varepsilon\Leftrightarrow\dfrac{1}{\sqrt{n}} < \varepsilon\Leftrightarrow\sqrt{n} > \dfrac{1}{\varepsilon}\Leftrightarrow n > \dfrac{1}{\varepsilon^2}\Rightarrow N_\varepsilon^{\rm opt} = \left\lfloor\dfrac{1}{\varepsilon^2}\right\rfloor + 1$, nên nếu chọn $N_\varepsilon\coloneqq N_\varepsilon^{\rm opt} = \left\lfloor\dfrac{1}{\varepsilon^2}\right\rfloor + 1$ thì $|u_n| < \varepsilon,\ \forall n\ge N_\varepsilon$. Theo định nghĩa giới hạn theo ngôn ngữ $\varepsilon$-$\delta$, suy ra $\lim_{n\to\infty} u_n = 0$.
	
	\item(d) Lấy $\varepsilon > 0$ bất kỳ, có $|u_n| = \left|-\dfrac{1}{\sqrt{n}}\right| = \dfrac{1}{\sqrt{n}}$, nên $|u_n| < \varepsilon\Leftrightarrow\frac{1}{\sqrt{n}} < \varepsilon\Leftrightarrow\sqrt{n} > \dfrac{1}{\varepsilon}\Leftrightarrow n > \dfrac{1}{\varepsilon^2}\Rightarrow N_\varepsilon^{\rm opt} = \left\lfloor\dfrac{1}{\varepsilon^2}\right\rfloor + 1$, nên nếu chọn $N_\varepsilon\coloneqq N_\varepsilon^{\rm opt} = \left\lfloor\dfrac{1}{\varepsilon^2}\right\rfloor + 1$ thì $|u_n| < \varepsilon,\ \forall n\ge N_\varepsilon$. Theo định nghĩa giới hạn theo ngôn ngữ $\varepsilon$-$\delta$, suy ra $\lim_{n\to\infty} u_n = 0$.
	
	\item(e) Lấy $\varepsilon > 0$ bất kỳ, có $|u_n| = \left|\dfrac{a\epsilon_n}{n^b}\right| = \dfrac{|a|}{n^b}$, nên $|u_n| < \varepsilon\Leftrightarrow\dfrac{|a|}{n^b} < \varepsilon\Leftrightarrow n^b > \dfrac{|a|}{\varepsilon}\Leftrightarrow n > \left(\dfrac{|a|}{\varepsilon}\right)^{\frac{1}{b}}\Rightarrow N_\varepsilon^{\rm opt} = \left\lfloor\left(\dfrac{|a|}{\varepsilon}\right)^{\frac{1}{b}}\right\rfloor + 1$, nên nếu chọn $N_\varepsilon\coloneqq N_\varepsilon^{\rm opt} = \left\lfloor\left(\dfrac{|a|}{\varepsilon}\right)^{\frac{1}{b}}\right\rfloor + 1$ thì $|u_n| < \varepsilon,\ \forall n\ge N_\varepsilon$. Theo định nghĩa giới hạn theo ngôn ngữ $\varepsilon$-$\delta$, suy ra $\lim_{n\to\infty} u_n = 0$.
\end{proof}

\begin{remark}[Dấu của số hạng của dãy số có giới hạn $0$]
	Đối với bài toán chứng minh dãy $\{u_n\}_{n=1}^\infty$ có $\lim_{n\to\infty} u_n = 0$ thì dấu của từng số hạng $u_n$ của dãy $\{u_n\}_{n=1}^\infty$ không quan trọng lắm, i.e., ${\rm sgn}\,u_n$ không làm ảnh hưởng tới bất đẳng thức $|u_n| < \varepsilon$ trong định nghĩa giới hạn theo ngôn ngữ $\varepsilon$-$\delta$ vì sau khi lấy giá trị tuyệt đối, $|u_n|\ge0$, $\forall n\in\mathbb{N}^\star$.
\end{remark}

\begin{baitoan}
	(a) Chứng minh $\lim_{n\to\infty} \dfrac{1}{2^n} = 0$. (b) Viết chương trình {\sf C{\tt/}C++, Python} để tính $N_\varepsilon^{\rm opt}$ với $\varepsilon\in(0,\infty)$ được nhập từ bàn phím.
\end{baitoan}

\begin{baitoan}
	(a) Chứng minh $\lim_{n\to\infty} \dfrac{n}{n + 1} = 1$. (b) Viết chương trình {\sf C{\tt/}C++, Python} để tính $N_\varepsilon^{\rm opt}$ với $\varepsilon\in(0,\infty)$ được nhập từ bàn phím.
\end{baitoan}

\begin{baitoan}
	Cho dãy $\{u_n\}_{n=1}^\infty$ có $\lim_{n\to\infty} u_n = l\in\mathbb{R}$. Chứng minh $\lim_{n\to\infty} v_n = 0$ với $v_n = u_n - u_{n-1}$. (b) $\lim_{n\to\infty} u_n - u_{n-1} = 0$ có suy ra được $\lim_{n\to\infty} u_n = l\in\mathbb{R}$ không?
\end{baitoan}

\begin{dinhnghia}[Dãy số thực có giới hạn hữu hạn,  \cite{SGK_Toan_11_CD_tap_1}, p. 61]
	\label{def: sequence lim}
	Dãy số thực $\{u_n\}_{n=1}^\infty\subset\mathbb{R}$ có giới hạn hữu là $l\in\mathbb{R}$ khi $n$ dần tới dương vô cực nếu $\lim_{n\to\infty} (u_n - l) = 0$, ký hiệu $\lim_{n\to\infty} u_n = L$.  
\end{dinhnghia}
{\sf Notation.} Ngoài ký hiệu $\lim_{n\to\infty} u_n = l$, ta cũng sử dụng các ký hiệu $\lim u_n = L$ hay $u_n\to l$ khi $n\to\infty$.

\begin{nhanxet}
	Nếu $u_n$ ngày càng gần tới $l$ khi $n$ ngày càng lớn thì $\lim u_n = l$.
\end{nhanxet}

\begin{dinhnghia}[Dãy số thực có giới hạn thực theo ngôn ngữ $\varepsilon$-$\delta$]
	\label{def: sequence lim: epsilon-delta}
	1 dãy số thực $\{u_n\}_{n=1}^\infty$ có giới hạn hữu hạn là $l\in\mathbb{R}$ nếu \& chỉ nếu với mọi số nguyên dương $\varepsilon$, tồn tại 1 số nguyên dương $N_\varepsilon\in\mathbb{N}^\star$ để $|u_n - l| < \varepsilon$ kể từ chỉ số $N_\varepsilon$ đó trở đi:	
	\begin{equation*}
		\forall\varepsilon\in(0,\infty),\ \exists N_\varepsilon\in\mathbb{N}^\star,\ |u_n| < \varepsilon,\ \forall n\ge N_\varepsilon,
	\end{equation*}
	hay tương đương:
	\begin{equation*}
		\forall\varepsilon > 0,\ \exists N_\varepsilon\in\mathbb{N}^\star,\ n\ge N_\varepsilon\Rightarrow|u_n| < \varepsilon.
	\end{equation*}
\end{dinhnghia}

\begin{remark}[Optimal{\tt/}smallest{\tt/}best indices -- Các chỉ số tối ưu{\tt/}nhỏ nhất{\tt/}tốt nhất]
	Định nghĩa \ref{def: sequence lim 0: epsilon-delta} chỉ yêu cầu tồn tại $N_\varepsilon\in\mathbb{N}^\star$ đủ lớn với mỗi $\varepsilon\in(0,\infty)$. Tuy nhiên nếu tìm được chỉ số $N_\varepsilon\in\mathbb{N}^\star$ tối ưu, i.e., chỉ số nhỏ nhất trong các chỉ số $N_\varepsilon$ thỏa mãn, i.e.:
	\begin{equation*}
		N_\varepsilon^{\rm opt}\coloneqq\min\{N_\varepsilon\in\mathbb{N};|u_n| < \varepsilon,\ \forall n\ge N_\varepsilon\} = \min\{N_\varepsilon\in\mathbb{N};n\ge N_\varepsilon\Rightarrow|u_n| < \varepsilon\}.
	\end{equation*}
	thì ta có thể sử dụng ký hiệu $N_\varepsilon^{\rm opt}$ để chỉ rõ tính tối ưu (i.e., nhỏ nhất, chặt{\tt/}ngặt nhất) của $N_\varepsilon$.
\end{remark}

\begin{baitoan}
	Tính $\lim_{n\to\infty} u_n$ với: (a) $u_n = c\in\mathbb{R}$, $\forall n\in\mathbb{N}^\star$. (b) $u_n = \dfrac{an + b}{n}$, $\forall n\in\mathbb{N}^\star$, với $a,b\in\mathbb{R}$. (c) $u_n = \dfrac{an + b}{cn + d}$, $\forall n\in\mathbb{N}^\star$ với $a,b,c,d\in\mathbb{R}$ thỏa $cn + d\ne0$, $\forall n\in\mathbb{N}^\star$.
\end{baitoan}

\begin{proof}
	(a) Lấy $\varepsilon > 0$ bất kỳ, có $|u_n - c| = |c - c| = 0 < \varepsilon$, $\forall n\ge1$, suy ra $N_\varepsilon^{\rm opt} = 1$, $\forall\varepsilon\in(0,\infty)$. Theo định nghĩa giới hạn theo ngôn ngữ $\varepsilon$-$\delta$, suy ra $\lim_{n\to\infty} u_n = 0$.
	
	\item(b) Lấy $\varepsilon > 0$ bất kỳ, có $|u_n - a| = \left|\dfrac{an + b}{n} - a\right| = \left|\dfrac{b}{n}\right| = \dfrac{|b|}{n}$, nên $|u_n| < \varepsilon\Leftrightarrow\dfrac{|b|}{n}< \varepsilon\Leftrightarrow n > \dfrac{|b|}{\varepsilon}\Rightarrow N_\varepsilon^{\rm opt} = \left\lfloor\dfrac{|b|}{\varepsilon}\right\rfloor + 1$, nên nếu chọn $N_\varepsilon\coloneqq N_\varepsilon^{\rm opt} = \left\lfloor\dfrac{|b|}{\varepsilon}\right\rfloor + 1$ thì $|u_n| < \varepsilon,\ \forall n\ge N_\varepsilon$. Theo định nghĩa giới hạn theo ngôn ngữ $\varepsilon$-$\delta$, suy ra $\lim_{n\to\infty} u_n = a$.
\end{proof}

\begin{baitoan}[Programming: Compute $N_\varepsilon^{\rm opt}$]
	Cho $\{u_n\}_{n=1}^\infty$ có giới hạn $\lim_{n\to\infty} u_n = L$. Viết chương trình {\sf C{\tt/}C++, Python}, với $\varepsilon\in(0,\infty)$ được nhập từ bàn phím, output $N_\varepsilon$: (a) $u_n = \dfrac{(-1)^n}{n}$. (b) $u_n = \dfrac{1}{\sqrt{n}}$ \& $u_n = -\dfrac{1}{\sqrt{n}}$. (c) $u_n = \dfrac{(-1)^n}{\sqrt{n}}$.
\end{baitoan}
Python: {\sc url}: \url{https://github.com/NQBH/advanced_STEM_beyond/blob/main/analysis/Python/limit.py}.
\begin{verbatim}
from math import sqrt

def ua(n):
    return (-1)**n / n

def ub(n):
    return 1 / sqrt(n)

def uc(n):
    return -1 / sqrt(n)

def ud(n):
    return (-1)**n / sqrt(n)

MAX_LOOP = 100000
epsilon = float(input())

for i in range(1, MAX_LOOP + 1):
    if abs(ua(i)) < epsilon:
        print(i) # N_epsilon
        break

for i in range(1, MAX_LOOP + 1):
    if abs(ub(i)) < epsilon:
        print(i) # N_epsilon
        break

for i in range(1, MAX_LOOP + 1):
    if abs(uc(i)) < epsilon:
        print(i) # N_epsilon
        break

for i in range(1, MAX_LOOP + 1):
    if abs(ud(i)) < epsilon:
        print(i) # N_epsilon
        break
\end{verbatim}
C++:
\begin{itemize}
	\item NLDK's C++ script to compute $N_\varepsilon^{\rm opt}$:
	
	{\sc url}: \url{https://github.com/NQBH/advanced_STEM_beyond/blob/main/analysis/C%2B%2B/NLDK_limit.cpp}.
	\begin{verbatim}
#include<bits/stdc++.h>
#define Sanic_speed ios_base::sync_with_stdio(false);cin.tie(NULL);cout.tie(NULL);
#define el "\n";
#define fre(i, a, b) for(int i = a; i <= b; ++i)
using namespace std;

double long qa(int n) {
    return (pow(-1, n)/n);
}
double long qb(int n) {
    double long deno = sqrt(n);
    return (1/deno);
}
double long qc(int n) {
    double long deno = sqrt(n);
    return (-1/deno);
}
double long qd(int n) {
    double long deno = sqrt(n);
    return (pow(-1, n)/deno);
}

void solve() {
    double long epsilon;
    cin >> epsilon;
    int maxN = 100000;
    fre(i, 1 ,maxN) {
        if (abs(qa(i)) < epsilon) {
            cout << "a) " << i << el
            break;
        }
    }
    fre(i, 1 ,maxN) {
        if (abs(qb(i)) < epsilon) {
            cout << "b) " << i << el
            break;
        }
    }
    fre(i, 1 ,maxN) {
        if (abs(qc(i)) < epsilon) {
            cout << "c) " << i << el
            break;
        }
    }
    fre(i, 1 ,maxN) {
        if (abs(qd(i)) < epsilon) {
            cout << "d) " << i << el
            break;
        }
    }
}

int main() {
    Sanic_speed
    int t = 1;// cin >> t;
    while(t > 0) {
    	solve();
    	--t;
   	}
}
	\end{verbatim}
\end{itemize}
Tính giới hạn:

\begin{baitoan}[\cite{TLCT_dai_so_giai_tich_11}, 1.]
	(a) $\lim_{n\to\infty} \dfrac{1}{n(n + 1)}$. (b) $\lim_{n\to\infty} \dfrac{\sin n}{\sqrt{n}}$. (c) $\lim_{n\to\infty} \dfrac{2n - 1}{2n + 2}$. (d) Mở rộng bài toán.
\end{baitoan}

\begin{baitoan}[\cite{TLCT_dai_so_giai_tich_11}, 2.]
	(a) $\lim_{n\to\infty} \sqrt{\dfrac{2n^2 - 1}{n^2 + n}}$. (b) $\lim_{n\to\infty} \dfrac{3^n}{1 + 2^n + 3^n}$. (c) $\lim_{n\to\infty} \dfrac{\sum_{i=1}^n i}{n^2}$.
\end{baitoan}

\begin{baitoan}[\cite{TLCT_dai_so_giai_tich_11}, 3.]
	Chứng minh: (a) $\lim_{n\to\infty} \sqrt[n]{2} = 1$. (b) $\lim_{n\to\infty} \sqrt[n]{n} = 1$. 
\end{baitoan}
{\sf Hint.} Sử dụng định lý kẹp.

\begin{baitoan}[\cite{TLCT_dai_so_giai_tich_11}, 4.]
	Biểu diễn số thập phân vô hạn tuần hoàn $0.(1428571)$ dưới dạng phân số.
\end{baitoan}
	Biểu diễn số thập phân vô hạn tuần hoàn $\overline{a_na_{n-1}\ldots a_1a_0.a_{-1}a_{-2}\ldots a_{-m}(b_1b_2\ldots b_p)}$ dưới dạng phân số. Viết chương trình {\sf C{\tt/}C++, Pascal, Python} để mô phỏng.
\begin{baitoan}
	
\end{baitoan}

\begin{baitoan}[\cite{TLCT_dai_so_giai_tich_11}, 5.]
	(a) $\lim_{n\to\infty} 2^n - 3^n$. (b) $\lim_{n\to\infty} n + \sin n$. (c) $\lim_{n\to\infty} \sqrt[3]{n^3 + 3n + 1}$. (d) Mở rộng bài toán.
\end{baitoan}

\begin{baitoan}[\cite{TLCT_dai_so_giai_tich_11}, 6.]
	(a) $\lim_{n\to\infty} \sqrt{n^2 + n + 1} - n$. (b) $\lim_{n\to\infty} n(\sqrt{n + 1} - \sqrt{n})$.
\end{baitoan}

\begin{baitoan}[\cite{TLCT_dai_so_giai_tich_11}, 7.]
	Cho $\Delta A_0B_0C_0$ đều cạnh $a\in(0,\infty)$. $\Delta A_{n+1}B_{n+1}C_{n+1}$có 3 đỉnh là trung điểm của $\Delta A_nB_nC_n$, $\forall n\in\mathbb{N}$. Gọi $P_n,S_n$ lần lượt là chu vi \& diện tích $\Delta A_nB_nC_n$, $\forall n\in\mathbb{N}$. Tính: (a) $\lim_{n\to\infty} p_n,\lim_{n\to\infty} S_n$. (b) $\sum_{i=0}^\infty p_i,\sum_{i=0}^\infty S_i$.
\end{baitoan}

\begin{baitoan}[\cite{TLCT_BT_dai_so_giai_tich_11}, 22., p. 47]
	Tính $\lim_{n\to\infty} u_n$ với $u_n = \sum_{i=1}^n \dfrac{1}{i(i + 1)}$.
\end{baitoan}

\begin{baitoan}[\cite{TLCT_BT_dai_so_giai_tich_11}, 23., p. 47]
	Tính $\lim_{n\to\infty} \dfrac{\sum_{i=1}^n \sqrt{i}}{n\sqrt{n}} = \lim_{n\to\infty} \dfrac{1 + \sqrt{2} + \cdots + \sqrt{n}}{n\sqrt{n}}$.
\end{baitoan}
{\sf Hint.} sử dụng định lý kẹp \& đánh giá:
\begin{equation*}
	\frac{3\sqrt{n}}{2} < (n + 1)\sqrt{n + 1} - n\sqrt{n} < \frac{3\sqrt{n + 1}}{2}.
\end{equation*}

\begin{baitoan}[\cite{TLCT_BT_dai_so_giai_tich_11}, 24., p. 48]
	Chứng minh dãy số $x_n = \cos n$ không có giới hạn khi $n\to\infty$.
\end{baitoan}
{\sf Hint.} Chứng minh phản chứng.

\begin{baitoan}
	$\lim_{n\to\infty} x_n$ với $x_n = \sum_{i=1}^n \dfrac{1}{i(i + 1)}$.
\end{baitoan}

\begin{proof}
	$x_n = \sum_{i=1}^n \dfrac{1}{i} - \dfrac{1}{i + 1} = 1 - \dfrac{1}{n + 1}\to1$ as $n\to\infty$ nên $\lim_{n\to\infty} x_n = 1$.
\end{proof}

\begin{baitoan}
	$\lim_{n\to\infty} \dfrac{4^n - 5^{-n}}{3^n - 2^{2n} - 5n^6}$.
\end{baitoan}

\begin{baitoan}
	$\lim_{n\to\infty} \dfrac{\ln(3n^2 - 2n)}{n^9 + 3n^2}$.
\end{baitoan}

\begin{baitoan}
	$\lim_{n\to\infty} \left(\dfrac{2n - 3}{2n + 5}\right)^{\dfrac{n^2 + 1}{n + 1}}$.
\end{baitoan}

\begin{baitoan}
	$\lim_{n\to\infty} \sqrt[n]{n + (-1)^n}$.
\end{baitoan}

\begin{baitoan}
	$\lim_{n\to\infty} \left(\dfrac{n - 2}{n + 2}\right)^{\dfrac{1 + n}{2 - \sqrt{n}}}$.
\end{baitoan}

\begin{baitoan}
	$\lim_{n\to\infty} \left(\dfrac{2n - 1}{5n + 2}\right)^n$.
\end{baitoan}

\begin{baitoan}
	$\lim_{n\to\infty} \left(\dfrac{n + 1}{n + 2}\right)^{\dfrac{1 + n}{2 - n^2}}$.
\end{baitoan}

\begin{baitoan}
	$\lim_{n\to\infty} \sqrt[n]{\dfrac{n^2 + 4^n}{n + 5^n}}$.
\end{baitoan}

\begin{baitoan}
	$\lim_{n\to\infty} \sqrt[n]{\dfrac{5n + 1}{n^{10} + 2n}}$.
\end{baitoan}

\begin{baitoan}
	$\lim_{n\to\infty} \left(\dfrac{2n + 1}{n^2 - 1}\right)^{\dfrac{1}{n - 2}}$.
\end{baitoan}

\begin{baitoan}
	$\lim_{n\to\infty} \left(\dfrac{n - 1}{n^2 + 1}\right)^{1 - n}$.
\end{baitoan}

\begin{baitoan}
	$\lim_{n\to\infty} \dfrac{1}{\sqrt[n]{n!}}$.
\end{baitoan}

\begin{baitoan}
	$\lim_{n\to\infty} \dfrac{n}{\sqrt[n]{n!}}$.
\end{baitoan}

\begin{baitoan}
	$\lim_{n\to\infty} u_n$ với $u_n = \sum_{i=1}^n \dfrac{1}{(2i - 1)(2i + 1)}$.
\end{baitoan}

\begin{baitoan}
	$\lim_{n\to\infty} u_n$ với $u_1 = \sqrt{3},u_{n+1} = \sqrt{3 + u_n}$, $\forall n\in\mathbb{N}^\star$.
\end{baitoan}

\begin{baitoan}[\cite{Hung_nang_cao_phat_trien_Toan_11_tap_1}, VD1, p. 86]
	Cho dãy số $a_n = \dfrac{n}{n + 1}$, $n = 1,2,\ldots$ Chứng minh dãy $(a_n)$ có giới hạn là $1$.
\end{baitoan}

\begin{baitoan}[\cite{Hung_nang_cao_phat_trien_Toan_11_tap_1}, VD2, p. 87]
	Chứng minh $\lim_{n\to\infty} \dfrac{1}{n} = 0$.
\end{baitoan}

\begin{baitoan}[\cite{Hung_nang_cao_phat_trien_Toan_11_tap_1}, VD3, p. 87]
	Chứng minh $\lim_{n\to\infty} q^n = 0$ nếu $0 < |q| < 1$.
\end{baitoan}

\begin{baitoan}[\cite{Hung_nang_cao_phat_trien_Toan_11_tap_1}, VD4, p. 87]
	Chứng minh dãy $u_n = (-1)^n$ phân kỳ.
\end{baitoan}

\begin{baitoan}[\cite{Hung_nang_cao_phat_trien_Toan_11_tap_1}, VD5, p. 88]
	Tìm $\lim_{n\to\infty} \dfrac{n^3 + 3n + 1}{2n^3 - 1}$.
\end{baitoan}

\begin{baitoan}[\cite{Hung_nang_cao_phat_trien_Toan_11_tap_1}, VD6, p. 88]
	Tìm $\lim_{n\to\infty} \dfrac{n^4 + 2n^3 + 7n^2 + 8n + 9}{2n^4 + 3n^3 + n + 10}$.
\end{baitoan}

\begin{baitoan}[\cite{Hung_nang_cao_phat_trien_Toan_11_tap_1}, VD7, p. 88]
	Tìm $\lim_{n\to\infty} (n - \sqrt[3]{n} - \sqrt{n})$.
\end{baitoan}

\begin{baitoan}[\cite{Hung_nang_cao_phat_trien_Toan_11_tap_1}, VD1, p. 89]
	Tìm $\lim_{n\to\infty} \dfrac{\sin n}{n}$.
\end{baitoan}

\begin{baitoan}[\cite{Hung_nang_cao_phat_trien_Toan_11_tap_1}, VD2, p. 89]
	Chứng minh nếu $\lim_{n\to\infty} |a_n| = 0$ thì $\lim_{n\to\infty} a_n = 0$.
\end{baitoan}

\begin{baitoan}[\cite{Hung_nang_cao_phat_trien_Toan_11_tap_1}, VD3, p. 89]
	Chứng minh $\lim_{n\to\infty} \sqrt[n]{n} = 1$.
\end{baitoan}

\begin{baitoan}[\cite{Hung_nang_cao_phat_trien_Toan_11_tap_1}, VD4, p. 89]
	Cho dãy số nguyên dương $(u_n)$ thỏa mãn $u_n > u_{n-1}u_{n+1}$, $\forall n\in\mathbb{N}$, $n\ge2$. Tính giới hạn $\lim_{n\to\infty} \dfrac{1}{n^2}\sum_{i=1}^n \dfrac{i}{u_i} = \lim_{n\to\infty} \dfrac{1}{n^2}\left(\dfrac{1}{u_1} + \dfrac{2}{u_2} + \cdots + \dfrac{n}{u_n}\right)$.
\end{baitoan}

\begin{baitoan}[\cite{Hung_nang_cao_phat_trien_Toan_11_tap_1}, VD5, p. 90]
	Tính $\lim_{n\to\infty} \dfrac{1}{n^2}\sum_{i=2}^n i\cos\dfrac{\pi}{i}$.
\end{baitoan}

\begin{baitoan}[\cite{Hung_nang_cao_phat_trien_Toan_11_tap_1}, VD1, p. 90]
	Cho dãy số $(u_n)$ được xác định theo công thức $u_n = f(u_{n-1})$. Giả sử $u_n\in[a,b]$ với mọi chỉ số $n$ \& $f$ là hàm tăng trên $[a,b]$. Chứng minh: (a) Nếu $u_1\le u_2$ thì $(u_n)$ là dãy tăng. (b) Nếu $u_1\ge u_2$ thì $(u_n)$ là dãy giảm. (c) Nếu hàm $f$ bị chặn thì $(u_n)$ hội tụ.
\end{baitoan}

\begin{baitoan}[\cite{Hung_nang_cao_phat_trien_Toan_11_tap_1}, VD2, p. 90]
	Cho dãy $(u_n)$ được xác định bởi $u_n = \dfrac{1}{3}\left(2u_{n-1} + \dfrac{1}{u_{n-1}^2}\right)$, $\forall n\in\mathbb{N}$, $n\ge2$, $u_1 > 0$. Chứng minh dãy $(u_n)$ hội tụ \& tìm giới hạn của dãy.
\end{baitoan}

\begin{baitoan}[\cite{Hung_nang_cao_phat_trien_Toan_11_tap_1}, VD3, p. 91]
	Tìm $u_1$ để dãy $u_n = u_{n-1}^2 + 3u_{n-1} + 1$ hội tụ.
\end{baitoan}

\begin{baitoan}[\cite{Hung_nang_cao_phat_trien_Toan_11_tap_1}, VD4, p. 92]
	Chứng minh tồn tại $\lim_{n\to\infty} \left(1 + \dfrac{1}{n}\right)^n$.
\end{baitoan}

\begin{baitoan}[Số Napier $e$]
	Đặt $e\coloneqq\lim_{n\to\infty} \left(1 + \dfrac{1}{n}\right)^n$. Chứng minh: (a) $ \left(1 + \dfrac{1}{n}\right)^n < e < \left(1 + \dfrac{1}{n}\right)^{n+1}$, $\forall n\in\mathbb{N}^\star$. (b) $\dfrac{1}{n + 1} < \ln\left(1 + \dfrac{1}{n}\right) < \dfrac{1}{n}$, trong đó $\ln x$ là logarith cơ số $e$ của $x$.
\end{baitoan}

\begin{baitoan}[\cite{Hung_nang_cao_phat_trien_Toan_11_tap_1}, VD5, p. 91]
	Chứng minh dãy $u_n = \sum_{i=1}^n \dfrac{1}{i} - \ln n = 1 + \dfrac{1}{2} + \dfrac{1}{3} + \cdots + \dfrac{1}{n} - \ln n$ có giới hạn hữu hạn.
\end{baitoan}

\begin{luuy}
	$C = \lim_{n\to\infty} \sum_{i=1}^n \dfrac{1}{i} - \ln n = \lim_{n\to\infty}  1 + \dfrac{1}{2} + \dfrac{1}{3} + \cdots + \dfrac{1}{n} - \ln n$ được gọi là {\rm hằng số Euler}.
\end{luuy}

\begin{baitoan}[\cite{Hung_nang_cao_phat_trien_Toan_11_tap_1}, VD1, p. 92]
	Chứng minh không tồn tại $\lim_{n\to\infty} \cos\dfrac{n\pi}{2}$.
\end{baitoan}

\begin{baitoan}[\cite{Hung_nang_cao_phat_trien_Toan_11_tap_1}, VD2, p. 92]
	Cho hàm $f:[0,+\infty)\to(0,b)$ liên tục \& nghịch biến. Giả sử hệ phương trình
	\begin{equation*}
		\left\{\begin{split}
			y &= f(x),\\
			x &= f(y),
		\end{split}\right.
	\end{equation*}
	có nghiệm duy nhất $x = y = q$. Chứng minh dãy $u_n = f(u_{n-1})$ hội tụ tới $q$ với $u_1 > 0$.
\end{baitoan}

\begin{baitoan}[\cite{Hung_nang_cao_phat_trien_Toan_11_tap_1}, VD3, p. 93]
	Cho dãy số $u_n = 1 + \dfrac{2}{1 + u_{n-1}}$, $u_1 > 0$. Chứng minh dãy hội tụ \& tìm giới hạn.
\end{baitoan}

\begin{baitoan}[\cite{Hung_nang_cao_phat_trien_Toan_11_tap_1}, VD1, p. 93]
	Cho dãy $a_n = \sum_{i=1}^n \dfrac{1}{i^2} = 1 + \dfrac{1}{2^2} + \cdots + \dfrac{1}{n^2}$, $\forall n\in\mathbb{N}^\star$. Chứng minh dãy này hội tụ.
\end{baitoan}

\begin{baitoan}[\cite{Hung_nang_cao_phat_trien_Toan_11_tap_1}, VD2, p. 93]
	Cho dãy $a_n = \sum_{i=1}^n \dfrac{1}{i} = 1 + \dfrac{1}{2} + \cdots + \dfrac{1}{n}$, $\forall n\in\mathbb{N}^\star$. Chứng minh dãy này phân kỳ.
\end{baitoan}

\begin{baitoan}[\cite{Hung_nang_cao_phat_trien_Toan_11_tap_1}, VD3, p. 94]
	Chứng minh $\lim_{n\to\infty} \dfrac{1^p + 2^p + \cdots + n^p}{n^{p + 1}} = \dfrac{1}{p + 1}$, $\forall p\in\mathbb{N}$.
\end{baitoan}

\begin{baitoan}[\cite{Hung_nang_cao_phat_trien_Toan_11_tap_1}, VD1, p. 94]
	Khảo sát sự hội tụ của {\rm dãy H\'eron} $(u_n)$ được xác định bởi $u_1 = 1$, $u_n = \dfrac{1}{2}\left(u_{n-1} + \dfrac{2}{u_{n-1}}\right)$, $\forall n\in\mathbb{N}$, $n\ge2$.
\end{baitoan}

\begin{baitoan}[\cite{Hung_nang_cao_phat_trien_Toan_11_tap_1}, VD2, p. 95]
	Cho dãy số $(x_n)$ thỏa mãn $|x_{n+1} - a|\le\alpha|x_n - a|$, $\forall n\in\mathbb{N}$, trong đó $a\in\mathbb{R}$ \& $0 < \alpha < 1$. Chứng minh dãy số $(x_n)$ hội tụ về $a$.
\end{baitoan}

\begin{baitoan}[\cite{Hung_nang_cao_phat_trien_Toan_11_tap_1}, VD3, p. 95]
	Cho dãy số $(x_n)$ xác định bởi $x_1 = a\in\mathbb{R}$, $x_{n+1} = \cos x_n$, $\forall n\in\mathbb{N}^\star$. Chứng minh $(x_n)$ hội tụ.
\end{baitoan}

\begin{baitoan}[\cite{Hung_nang_cao_phat_trien_Toan_11_tap_1}, VD4, p. 95, Canada 1985]
	Dãy số $(x_n)$ thỏa mãn $1 < x_1 < 2$ \& $x_{n+1} = 1 + x_n - \dfrac{1}{2}x_n^2$, $\forall n\in\mathbb{N}^\star$. Chứng minh $(x_n)$ hội tụ. Tìm $\lim_{n\to\infty} x_n$.
\end{baitoan}

\begin{baitoan}[\cite{Hung_nang_cao_phat_trien_Toan_11_tap_1}, VD5, p. 95, VMO2023]
	Xét dãy số $(a_n)$ thỏa mãn $a_1 = \dfrac{1}{2}$, $a_{n+1} = \sqrt[3]{3a_{n+1} - a_n}$ \& $0\le a_n\le1$, $\forall n\in\mathbb{N}^\star$. Chứng minh dãy $(a_n)$ có giới hạn hữu hạn.
\end{baitoan}

\begin{baitoan}[\cite{Hung_nang_cao_phat_trien_Toan_11_tap_1}, VD6, p. 96, VMO2022]
	Cho dãy số $(u_n)$ xác định bởi $u_1 = 6$, $u_{n+1} = 2 + \sqrt{u_n + 4}$, $\forall n\in\mathbb{N}^\star$.  Chứng minh dãy $(u_n)$ có giới hạn hữu hạn.
\end{baitoan}

\begin{baitoan}[\cite{Hung_nang_cao_phat_trien_Toan_11_tap_1}, VD7, p. 96, VMO2019]
	Cho dãy số $(x_n)$ xác định bởi $x_1 = 1$ \& $x_{n+1} = x_n + 3\sqrt{x_n} + \dfrac{n}{\sqrt{x_n}}$, $\forall n\in\mathbb{N}^\star$. (a) Chứng minh $\lim_{n\to\infty} \dfrac{n}{x_n} = 0$. (b) Tính giới hạn $\lim_{n\to\infty} \dfrac{n^2}{x_n}$.
\end{baitoan}

\begin{baitoan}[\cite{Hung_nang_cao_phat_trien_Toan_11_tap_1}, VD1, p. 97, VMO1984]
	Dãy số $(u_n)$ được xác định như sau: $u_1 = 1$, $u_2 = 2$, $u_{n+1} = 3u_n - u_{n-1}$. Dãy số $(v_n)$ được xác định như sau: $v_n = \sum_{i=1}^n {\rm arccot}u_i$. Tìm giới hạn $\lim_{n\to\infty} v_n$.
\end{baitoan}

\begin{baitoan}[\cite{Hung_nang_cao_phat_trien_Toan_11_tap_1}, VD2, p. 97, VMO1988]
	Dãy số $(u_n)$ bị chặn thỏa mãn điều kiện $u_n + u_{n+1}\ge2u_{n+2}$, $\forall n\in\mathbb{N}^\star$ có nhất thiết hội tụ không?
\end{baitoan}

\begin{baitoan}[\cite{Hung_nang_cao_phat_trien_Toan_11_tap_1}, VD3, p. 98, Olympic 30.4 lần V]
	Cho $x_k = \sum_{i=1}^k \dfrac{i}{(i + 1)!} = \dfrac{1}{2!} + \dfrac{2}{3!} + \cdots + \dfrac{k}{(k + 1)!}$. Tính $\lim_{n\to\infty} \sqrt[n]{\sum_{i=1}^{1999} x_i^n} = \lim_{n\to\infty} \sqrt[n]{x_1^n + x_2^n + \cdots + x_{1999}^n}$.
\end{baitoan}

\begin{baitoan}[\cite{Hung_nang_cao_phat_trien_Toan_11_tap_1}, VD4, p. 98, VMO2013A]
	Gọi $F$ là tập hợp tất cả các hàm số $f:(0,+\infty)\to(0,+\infty)$ thỏa mãn $f(3x)\ge f(f(2x)) + x$, $\forall x > 0$. Tìm hằng số $A$ lớn nhất để $f(x)\ge Ax$, $\forall f\in F$, $\forall x > 0$.
\end{baitoan}

\begin{baitoan}[\cite{Hung_nang_cao_phat_trien_Toan_11_tap_1}, VD5, p. 98, Hải Dương 2019--2020]
	Cho dãy số thực $(x_n)$ thỏa mãn $x_1 = \dfrac{1}{6}$, $x_{n+1} = \dfrac{3x_n}{2x_n + 1}$, $\forall n\in\mathbb{N}^\star$. Tìm số hạng tổng quát của dãy số \& tính giới hạn của dãy số đó.
\end{baitoan}

\begin{baitoan}[\cite{Hung_nang_cao_phat_trien_Toan_11_tap_1}, VD6, p. 99, Hải Dương 2015--2016]
	Cho dãy số $(u_n)$ thỏa mãn $u_1 = -1$, $u_{n+1} = \dfrac{u_n}{2} + \dfrac{2}{u_n}$, $\forall n\in\mathbb{N}^\star$ \& dãy số $(v_n)$ thỏa mãn $u_nv_n - u_n + 2v_n + 2 = 0$, $\forall n\in\mathbb{N}^\star$. Tính $v_{2015}$ \& $\lim_{n\to\infty} u_n$.
\end{baitoan}

\begin{baitoan}[\cite{Hung_nang_cao_phat_trien_Toan_11_tap_1}, VD7, p. 99, Hải Dương 2013--2014]
	Cho dãy số $(u_n)$ thỏa mãn $u_1 = \dfrac{5}{2}$, $u_{n+1} = \dfrac{1}{2}u_n^2 - u_n + 2$. Tính $\lim_{n\to\infty} \sum_{i=1}^n \dfrac{1}{u_i}$.
\end{baitoan}

\begin{baitoan}[\cite{Hung_nang_cao_phat_trien_Toan_11_tap_1}, VD1, p. 99]
	Cho dãy số $(u_n)$ được xác định: $u_1$, $u_n = \alpha u_{n-1} + \beta$. Biện luận theo tham số $\alpha,\beta$ giá trị giới hạn của dãy số.
\end{baitoan}

\begin{baitoan}[\cite{Hung_nang_cao_phat_trien_Toan_11_tap_1}, VD1, p. 100]
	Cho $(u_n)$ là dãy số hội tụ \& $\lim_{n\to\infty} u_n = u$. Khi đó, dãy trung bình cộng $v_n =  \dfrac{1}{n}\sum_{i=1}^n u_i$ cũng hội tụ \& $\lim_{n\to\infty} v_n = u$.
\end{baitoan}

\begin{baitoan}[\cite{Hung_nang_cao_phat_trien_Toan_11_tap_1}, VD2, p. 100]
	Giả sử $\lim_{n\to\infty} a_n = a$, $\lim_{n\to\infty} b_n = b$. Chứng minh $\lim_{n\to\infty} \dfrac{1}{n}\sum_{i=1}^n a_ib_{n+1-i} = \lim_{n\to\infty} \dfrac{a_1b_n + a_2b_{n-1} + \cdots + a_nb_1}{n} = ab$.  Từ đó, suy ra $\lim_{n\to\infty} \dfrac{1}{n}\sum_{i=1}^n a_i = \lim_{n\to\infty} \dfrac{a_1 + a_2 + \cdots + a_n}{n}= a$.
\end{baitoan}

\begin{baitoan}[\cite{Hung_nang_cao_phat_trien_Toan_11_tap_1}, VD3, p. 101]
	Giả sử $a_n > 0$, $\forall n\in\mathbb{N}^\star$. Chứng minh nếu $\lim_{n\to\infty} a_n = a > 0$ thì $\lim_{n\to\infty} \sqrt[n]{\prod_{i=1}^n a_i} = \lim_{n\to\infty} \sqrt[n]{a_1a_2\cdots a_n} = a$.
\end{baitoan}

%------------------------------------------------------------------------------%

\section{Cauchy sequences -- Dãy Cauchy}

\begin{definition}[\cite{Rudin1976}, Def. 3.8, p. 52]
	A sequence $\{p_n\}$ in a metric space $X$ is said to be a {\rm Cauchy sequence} if for every $\epsilon > 0$ there is an integer $N$ s.t. $d_X(p_n,p_m) < \epsilon$ if $n\ge N$ \& $m\ge N$.
\end{definition}
Briefly:
\begin{equation*}
	\{p_n\}\mbox{ is a Cauchy sequence in a metric space } X\Leftrightarrow\forall\varepsilon > 0,\exists N_\varepsilon\mbox{ s.t. }\min\{m,n\}\ge N_\varepsilon\Rightarrow d_X(p_n,p_m) < \varepsilon, 
\end{equation*}
or equivalently,
\begin{equation*}
	\{p_n\}\mbox{ is a Cauchy sequence in a metric space } X\Leftrightarrow\forall\varepsilon > 0,\exists N_\varepsilon\mbox{ s.t. }d_X(p_n,p_m) < \varepsilon,\ \forall m\ge N_\varepsilon,\,\forall n\ge N_\varepsilon.
\end{equation*}

\begin{definition}
	Let $E$ be a subset of a metric space $X$, \& let $S$ be the set of all real numbers of the form $d(p,q)$, with $p\in E,q\in E$. The sup of $S$ is called the {\rm diameter} of $E$.
\end{definition}

\begin{problem}[\cite{Rudin1976}, p. 48, +1]
	(a) Prove that the sequence $\{\frac{1}{n}\}$ converges in $\mathbb{R} = \mathbb{R}^1$ (to $0$), but fails to converge in the set of all positive real numbers, with $d(x,y)\coloneqq|x - y|$, $\forall x,y\in X$. (b) Find similar or more advanced examples.
\end{problem}

%-----------------------------------------------------------------------------%

\section{Sequences with {\tt SymPy}}
A sequence is a finite or infinite lazily evaluated list.
\begin{verbatim}
	sympy.series.sequences.sequence(seq, limits=None)
\end{verbatim}
returns appropriate {\tt sequence object}.

{\it Explanation}: If {\tt seq} is a SymPy sequence, returns {\tt SeqPer} object otherwise returns {\tt SeqFormula} object. E.g.:
\begin{verbatim}
	from sympy import sequence
	from sympy.abc import n
	sequence(n**2, (n, 0, 5))
	# output: SeqFormula(n**2, (n, 0, 5))
	sequence((1, 2, 3), (n, 0, 5))
	# output: SeqPer((1, 2, 3), (n, 0, 5))
\end{verbatim}

%-----------------------------------------------------------------------------%

\subsection{Sequence Base}
{\tt class sympy.series.sequences.SeqBase(*args)}: Base class for sequences.
\begin{itemize}
	\item {\tt coeff(pt)}: returns the coefficient at point {\tt pt}.
	\item \verb|coeff_mul(other)|: should be used when {\tt other} is not a sequence. Should be defined to define custom behavior.
	\begin{verbatim}
		from sympy import SeqFormula			
		from sympy.abc import n			
		SeqFormula(n**2).coeff_mul(2)
		# output: SeqFormula(2*n**2, (n, 0, oo))
	\end{verbatim}
	{\tt*} defines multiplication of sequences with sequences only.
	\item \verb|find_linear_recurrence(n, d = None, gfvar = None, )|: Finds the shortest linear recurrence that satisfies the 1st $n$ terms of sequence of order $\le\frac{n}{2}$ if possible. If {\tt d} is specified, find shortest linear recurrence of order $\le\min\{d,\frac{n}{2}\}$ if possible. Returns list of coefficients {\tt[b(1), b(2), ...]} corresponding to recurrence relation {\tt x(n) = b(1)*x(n - 1) + b(2)*x(n - 2) + ...}. Return {\tt[]} if no recurrence is found. If {\tt gfvar} is specified, also returns ordinary generating function as a function of {\tt gfvar}.
\end{itemize}

%-----------------------------------------------------------------------------%

\section{Problems: Sequences}

\begin{baitoan}
	Tính $\lim_{n\to\infty} \dfrac{an + b}{cn + d}$ theo $a,b,c,d\in\mathbb{R}$, $(c,d)\ne(0,0)$.
\end{baitoan}

\begin{baitoan}
	Tính $\lim_{n\to\infty} \dfrac{an^2 + bn + c}{dn^2 + en + f}$ theo $a,b,c,d,e,f\in\mathbb{R}$, $(d,e,f)\ne(0,0,0)$.
\end{baitoan}

\begin{baitoan}
	Tính $\lim_{n\to\infty} \dfrac{P(n)}{Q(n)}$ với: (a) $P,Q\in\mathbb{R}[x]$, $Q\not\equiv0$. (b) $P,Q\in\mathbb{C}[x]$, $Q\not\equiv0$.
\end{baitoan}

\begin{baitoan}
	Cho $a,b,c,d,\alpha\in\mathbb{R}$, $\alpha\ne0$. Tính: (a) $\lim_{n\to\infty} \dfrac{a + b\alpha^n}{c + d\alpha^n}$. (b) $\lim_{n\to\infty} \dfrac{an + b\alpha^n}{cn + d\alpha^n}$. (c) $\lim_{n\to\infty} \dfrac{an^2 + b\alpha^n}{cn^2 + d\alpha^n}$. (d) $\lim_{n\to\infty} \dfrac{P(x) + a\alpha^n}{Q(x) + b\alpha^n}$ với $P,Q\in\mathbb{R}[x]$.
\end{baitoan}

\begin{baitoan}[\cite{TLCT_dai_so_giai_tich_11}, 1.]
	Cho dãy số $\{a_n\}_{n=1}^\infty$ thỏa $\lim_{n\to\infty} a_n\sum_{i=1}^n a_i^2 = 1$. Tính $\lim_{n\to\infty} a_n\sqrt[3]{3n} = 1$.
\end{baitoan}

\begin{baitoan}[\cite{TLCT_dai_so_giai_tich_11}, 2.]
	Cho dãy số $\{a_n\}_{n=1}^\infty$ thỏa $a_1\in(0,1),a_{n+1} = a_n - a_n^2$, $\forall n\in\mathbb{N}^\star$. Chứng minh $\lim_{n\to\infty} na_n = 1$.
\end{baitoan}

\begin{baitoan}[\cite{TLCT_dai_so_giai_tich_11}, 3.]
	Cho dãy số $\{x_n\}_{n=1}^\infty$ thỏa $x_0 = 2,x_{n+1} = \dfrac{2x_n + 1}{x_n + 2}$, $n\in\mathbb{N}$. Tính $\left\lfloor\sum_{i=1}^n x_i\right\rfloor$.
\end{baitoan}

\begin{baitoan}[\cite{TLCT_dai_so_giai_tich_11}, 4.]
	Chứng minh $\lim_{n\to\infty} \sum_{i=1}^n \left(\sqrt{1 + \dfrac{i}{n^2}} - 1\right) = \dfrac{1}{4}$.
\end{baitoan}

\begin{baitoan}[\cite{TLCT_dai_so_giai_tich_11}, 5.]
	Cho dãy số $\{x_n\}_{n=0}^\infty$ xác định bởi $x_0 = 0,x_1 = 2,x_{n+2} = 2^{-x_n} + \dfrac{1}{2}$, $\forall n\in\mathbb{N}$. Chứng minh $\exists\lim_{n\to\infty} x_n\in\mathbb{R}$ \& tính $\lim_{n\to\infty} x_n$.
\end{baitoan}

\begin{baitoan}[\cite{TLCT_dai_so_giai_tich_11}, 6.]
	Xét tính hội tụ của dãy theo giá trị của $a\in\mathbb{R}$:
	\begin{equation*}
		\left\{\begin{split}
			x_1 &= a\ne-1,\\
			x_{n+1} &= \dfrac{3\sqrt{2x_n^2 + 2} - 2}{2x_n + \sqrt{2x_n^2 + 2}},\ \forall n\in\mathbb{N}^\star.
		\end{split}\right.
	\end{equation*}
\end{baitoan}

\begin{baitoan}[\cite{TLCT_dai_so_giai_tich_11}, 7.]
	Cho $a,b,c,d\in\mathbb{R}$. Xét hàm số $f(x) = \dfrac{ax + b}{cx + d}$, $f:\mathbb{R}\backslash\left\{-\dfrac{d}{c}\right\}\to\mathbb{R}\backslash\left\{-\dfrac{a}{c}\right\}$ \& dãy $\{u_n\}_{n=0}^\infty$ thỏa $u_0 = a\in\mathbb{R}$, $u_{n+1} = f(u_n)$, $\forall n\in\mathbb{N}$. (a) Chứng minh $f(x)$ là 1 song ánh \& dãy $\{u_n\}_{n=0}^\infty$ đã cho xác định khi \& chỉ khi $a\ne v_n$, $\forall n\in\mathbb{N}$, trong đó $\{v_n\}_{n=0}^\infty$ được xác định bởi $v_0 = -\dfrac{d}{c},v_{n+1} = f^{-1}(v_n)$, $\forall n\in\mathbb{N}$ (lưu ý: dãy $\{v_n\}_{n=0}^\infty$ có thể không xác định kề từ 1 chỉ số nào đó). (b) Đặt $\Delta\coloneqq(a - d)^2 + 4bc$. Biện luận theo $\Delta$ sự hội tụ của dãy $\{u_n\}_{n=0}^\infty$.
\end{baitoan}

\begin{baitoan}[\cite{TLCT_dai_so_giai_tich_11}, 8.]
	Cho $\{a_n\}_{n=1}^\infty$ là dãy bị chặn thỏa $F_{n+2}a_{n+2}\le F_{n+1}a_{n+1} + F_na_n$, $\forall n\in\mathbb{N}^\star$. Chứng minh $\{a_n\}_{n=1}^\infty$ hội tụ.
\end{baitoan}

\begin{baitoan}[\cite{TLCT_dai_so_giai_tich_11}, 9.]
	Dãy $\{a_n\}_{n=1}^\infty$ được xác định bởi $a_1 > 0,a_2 > 0,a_{n+1} = \sqrt{a_n} + \sqrt{a_{n-1}}$. Chứng minh dãy số $\{a_n\}_{n=1}^\infty$ hội tụ \& tìm giới hạn của dãy số đó.
\end{baitoan}

\begin{baitoan}[\cite{TLCT_dai_so_giai_tich_11}, 10.]
	Cho $a,b,A,B\in(0,\infty)$. Xét dãy số $\{x_n\}_{n=1}^\infty$ xác định bởi $x_1 = a,x_2 = b,x_{n+2} = A\sqrt[3]{x_{n+1}^2} + B\sqrt[3]{x_n^2}$, $\forall n\in\mathbb{N}^\star$. Chứng minh $\exists\lim_{n\to\infty} x_n$ \& tính $\lim_{n\to\infty} x_n$.
\end{baitoan}

\begin{baitoan}[\cite{TLCT_dai_so_giai_tich_11}, 11.]
	Tìm $a\in\mathbb{R}$ để dãy $\{x_n\}_{n=1}^\infty$ xác định bởi:
	\begin{equation*}
		x_0 = a,\ x_{n+1} = \frac{4x_n^5 + x_n^2 - x_n - 1}{5x_n^4 + x_n},\ \forall n\in\mathbb{N},
	\end{equation*}
	hội tụ.
\end{baitoan}

\begin{baitoan}[\cite{TLCT_dai_so_giai_tich_11}, 12.]
	Cho $a,b,c\in(0,\infty)$ \& 3 dãy số $\{a_n\}_{n=0}^\infty,\{b_n\}_{n=0}^\infty,\{c_n\}_{n=0}^\infty$ được xác định bởi:
	\begin{equation*}
		a_0 = a,\ b_0 = b,\ c_0 = c,\ a_{n+1} = a_n + \frac{2}{b_n + c_n},\ b_{n+1} = b_n + \frac{2}{c_n + a_n},\ c_{n+1} = c_n + \frac{2}{a_n + b_n},\ \forall n\in\mathbb{N}.
	\end{equation*}
	Chứng minh $\lim_{n\to\infty} a_n = \lim_{n\to\infty} b_n = \lim_{n\to\infty} c_n = \infty$.
\end{baitoan}

\begin{baitoan}[\cite{TLCT_dai_so_giai_tich_11}, 13.]
	Cho $n\in\mathbb{N},n\ge2$. Chứng minh phương trình $x^n = x + 1$ có 1 nghiệm dương duy nhất, ký hiệu là $x_n$. (a) Chứng minh $\lim_{n\to\infty} x_n = 1$. (b) Tính $\lim_{n\to\infty} n(x_n - 1)$.
\end{baitoan}

\begin{baitoan}[\cite{TLCT_dai_so_giai_tich_11}, 14.]
	Cho $n\in\mathbb{N},n\ge2$. Chứng minh phương trình $x^n = x^2 + x + 1$ có 1 nghiệm dương duy nhất, ký hiệu là $x_n$. Tìm $a\in\mathbb{R}$ để giới hạn $\lim_{n\to\infty} n^a(x_n - x_{n+1})$ tồn tại, hữu hạn, \& $\ne0$.
\end{baitoan}

\begin{baitoan}[\cite{TLCT_dai_so_giai_tich_11}, 15.]
	Cho dãy số $\{a_n\}_{n=1}^\infty$ thỏa $a_1 = 5,a_{n+1} = a_n + \dfrac{1}{a_n}$, $\forall n\in\mathbb{N}^\star$. Chứng minh $45 < a_{1000} < 45.1$.
\end{baitoan}

\begin{baitoan}[\cite{TLCT_dai_so_giai_tich_11}, 16.]
	Cho dãy số $\{u_n\}_{n=1}^\infty$ xác định bởi
	\begin{equation*}
		u_1 = 1,\ u_2 = 2,\ u_{n+2} = 2u_{n+1} + u_n,\ \forall n\in\mathbb{N}^\star.
	\end{equation*}
	Đặt $x_n\coloneqq\dfrac{u_{n+1}}{u_n}$, $\forall n\in\mathbb{N}^\star$. Tính $\lim_{n\to\infty} x_n$.
\end{baitoan}

\begin{baitoan}[\cite{TLCT_dai_so_giai_tich_11}, 17.]
	Cho dãy số $\{u_n\}_{n=1}^\infty$ thỏa $\lim_{n\to\infty} u_{2n} + u_{2n+1} = 2010,\lim_{n\to\infty} u_{2n} + u_{2n-1} = 2011$. Tính $\lim_{n\to\infty} \dfrac{u_{2n}}{u_{2n+1}}$.
\end{baitoan}

\begin{baitoan}[\cite{TLCT_dai_so_giai_tich_11}, 18.]
	Cho dãy số $\{u_n\}_{n=1}^\infty$ xác định bởi $u_1 = u_2 = 1$, $u_{n+2} = 4u_{n+1} - 5u_n$, $\forall n\in\mathbb{N}^\star$. Chứng minh $\forall a\in(\sqrt{5},\infty)$, $\lim_{n\to\infty} \dfrac{u_n}{a^n} = 0$. 
\end{baitoan}

\begin{baitoan}[\cite{TLCT_dai_so_giai_tich_11}, 19.]
	Tính $\lim_{n\to\infty} \dfrac{1}{\sqrt{n}}\sum_{i=1}^n \dfrac{1}{\sqrt{i}}$.
\end{baitoan}

\begin{baitoan}[\cite{TLCT_dai_so_giai_tich_11}, 20.]
	Cho dãy số $\{u_n\}_{n=1}^\infty\subset[1,\infty)$ thỏa $u_{m+n}\le u_mu_n$. Đặt $v_n\coloneqq\dfrac{\ln u_n}{n}$, $\forall n\in\mathbb{N}^\star$. Chứng minh $\{v_n\}_{n=1}^\infty$ hội tụ.
\end{baitoan}

\begin{baitoan}[\cite{TLCT_dai_so_giai_tich_11}, 21.]
	Cho dãy số dương $\{a_n\}_{n=1}^\infty$ thỏa $a_1 > 0,a_{n+1}^p\ge\sum_{i=1}^n a_i$, $\forall n\in\mathbb{N}^\star$, với $p\in(0,2)$ cho trước. Chứng minh tồn tại $c > 0$ để $a_n > nc$, $\forall n\in\mathbb{N}^\star$.
\end{baitoan}

\begin{baitoan}[\cite{TLCT_dai_so_giai_tich_11}, 22.]
	Khảo sát sự hội tụ của dãy $u_0 = a\in\mathbb{R}$, $u_{n+1} = \sqrt[3]{7u_n - 6}$, $\forall n\in\mathbb{N}$.
\end{baitoan}

\begin{baitoan}[\cite{TLCT_dai_so_giai_tich_11}, 23.]
	Cho $\alpha\in(0,2)$. Tính giới hạn của dãy $\{u_n\}_{n=1}^\infty$ đặt bởi $u_{n+2} = \alpha u_{n+1} + (1 - \alpha)u_n$, $\forall n\in\mathbb{N}$, theo 2 giá trị $u_0,u_1$ cho trước.
\end{baitoan}

\begin{baitoan}[\cite{TLCT_dai_so_giai_tich_11}, 24.]
	Cho $a\in(1,\infty)$. Tính $\lim_{n\to\infty} \dfrac{n}{a^{n+1}}\sum_{i=1}^n \dfrac{a^i}{i}$.
\end{baitoan}

\begin{baitoan}[\cite{TLCT_dai_so_giai_tich_11}, 25.]
	Tìm $a\in\mathbb{R}$ để dãy số $\{x_n\}_{n=1}^\infty$ được xác định bởi $x_-0 = \sqrt{1996},x_{n+1} = \dfrac{a}{x_n^2 + 1}$, $\forall n\in\mathbb{N}$, có giới hạn $\lim_{n\to\infty} u_n\in\mathbb{R}$.
\end{baitoan}

\begin{baitoan}[\cite{TLCT_dai_so_giai_tich_11}, 26.]
	Cho dãy số thực $\{a_n\}_{n=1}^\infty$ thỏa $e^{a_n} + na_n = 2$, $\forall n\in\mathbb{N}^\star$. Chứng minh $\lim_{n\to\infty} n(1 - na_n) = 1$.
\end{baitoan}

\begin{baitoan}[\cite{TLCT_dai_so_giai_tich_11}, 27.]
	Cho dãy số thực $\{x_n\}_{n=1}^\infty\subset(0,\infty)$ được xác định bởi $x_1 = 1,x_2 = 9,x_3 = 9,x_4 = 1,x_{n+4} = \sqrt[4]{x_nx_{n+1}x_{n+2}x_{n+3}}$, $\forall n\in\mathbb{N}^\star$. Chứng minh dãy này có giới hạn hữu hạn \& tính giới hạn đó.
\end{baitoan}

\begin{baitoan}[\cite{VMS_VMC2023}, 1.1, p. 30, HCMUT]
	Cho $f\in C^1(\mathbb{R},\mathbb{R})$ thỏa $f'(x) < 0$, $\forall x\in\mathbb{R}$. Xét dãy số $\{a_n\}$:
	\begin{equation*}
		\left\{\begin{split}
			a_1 &= 1,\\
			a_{n+1} &= a_n - \frac{f(a_n)}{f'(a_n)},\ \forall n\in\mathbb{N}^\star.
		\end{split}\right.
	\end{equation*}
	(a) Nếu $f(x) > 0$, $\forall x\in\mathbb{R}$, tính $\lim_{n\to\infty} a_n$. (b) Nếu $f(2023) = 0$ \& $f\in C^2(\mathbb{R})$ thỏa $f''(x) > 0$, $\forall x\in\mathbb{R}$, tính $\lim_{n\to\infty} a_n$.
\end{baitoan}

\begin{baitoan}[\cite{VMS_VMC2023}, 1.2, p. 30, VNUHCM UIT]
	Cho dãy số $\{u_n\}_{n=1}^\infty$ thỏa
	\begin{equation*}
		\left\{\begin{split}
			u_0&\ge-2,\\
			u_n &= \sqrt{2 + u_{n-1}},\ \forall n\in\mathbb{N}^\star.
		\end{split}\right.
	\end{equation*}
	(a) Chứng minh $\{u_n\}$ có giới hạn hữu hạn. Tính $\lim_{n\to\infty} u_n$. (b) Cho 2 dãy $\{v_n\}_{n=1}^\infty,\{w_n\}_{n=1}^\infty$ đặt bởi
	\begin{equation*}
		\left\{\begin{split}
			v_n &= 4^n|u_n - 2|,\\
			w_n &= \frac{u_1u_2\cdots u_n}{2^n},\ \forall n\in\mathbb{N}^\star.
		\end{split}\right.
	\end{equation*}
	Tính $\lim_{n\to\infty} v_n,\lim_{n\to\infty} w_n$.
\end{baitoan}

\begin{baitoan}[\cite{VMS_VMC2023}, 1.3, p. 30, ĐH Đồng Tháp]
	Xét dãy số $\{u_n\}_{n=1}^\infty$ đặt bởi
	\begin{equation*}
		u_1 = \frac{3}{2},\ u_n = 1 + \frac{1}{2}\arctan u_{n-1},\ \forall n\in\mathbb{N}^\star.
	\end{equation*}
	Chứng minh $\{u_n\}_{n=1}^\infty$ hội tụ.
\end{baitoan}

\begin{baitoan}[\cite{VMS_VMC2023}, 1.4, p. 31, ĐH Đồng Tháp]
	Cho dãy số $\{a_n\}_{n=1}^\infty$ đặt bởi
	\begin{equation*}
		a_1 = 1,\ a_{n+1} = \frac{n^2 - 1}{a_n} + 2,\ \forall n\in\mathbb{N}^\star.
	\end{equation*}
	(a) Chứng minh $n\le a_n\le n + 1$, $\forall n\in\mathbb{N}^\star$. (b) Đặt $S_n^{(3)}\coloneqq\sum_{i=1}^n a_i^3$. Tính $\lim_{n\to\infty} \dfrac{S_n^{(3)}}{n^4}$.
\end{baitoan}

\begin{baitoan}[\cite{VMS_VMC2023}, 1.5, p. 31, ĐHGTVT]
	Cho dãy số $\{a_n\}_{n=1}^\infty$ đặt bởi
	\begin{equation*}
		a_1 > 0,\ a_{n+1} = \frac{a_n^2}{a_n^2 - a_n + 1},\ \forall n\in\mathbb{N}^\star.
	\end{equation*}
	Chứng minh $\{a_n\}_{n=1}^\infty$ giảm \& tính $\lim_{n\to\infty} a_n$.
\end{baitoan}

\begin{baitoan}[\cite{VMS_VMC2023}, 1.6, p. 31, ĐH Hùng Vương, Phú Thọ]
	Cho dãy số $\{u_n\}_{n=1}^\infty$ đặt bởi
	\begin{equation*}
		\left\{\begin{split}
			u_0 &= 0,\ u_1 = \beta,\\
			u_{n+1} &= \frac{u_n + u_{n-1}}{2},\ \forall n\in\mathbb{N}^\star.
		\end{split}\right.
	\end{equation*}
	(a) Tìm công thức số hạng tổng quát của $\{u_n\}_{n=1}^\infty$. (b) Tính $\lim_{n\to\infty} u_n$.
\end{baitoan}

\begin{baitoan}[\cite{VMS_VMC2023}, 1.7, p. 31, ĐHKH, Thái Nguyên]
	Cho dãy số $\{u_n\}_{n=1}^\infty$ đặt bởi
	\begin{equation*}
		x_n = \sum_{i=1}^n \frac{i}{(i + 1)!} = \frac{1}{2!} + \frac{2}{3!} + \cdots + \frac{n}{(n + 1)!},\ \forall n\in\mathbb{N}^\star.
	\end{equation*}
	Tính $\lim_{n\to\infty} \sqrt[n]{\sum_{i=1}^{2023} x_i^n} = \lim_{n\to\infty} \sqrt[n]{x_1^n + x_2^n + \cdots + x_{2023}^n}$.
\end{baitoan}

\begin{baitoan}[\cite{VMS_VMC2023}, 1.8, p. 31, ĐH Mỏ--Địa chất]
	Tính
	\begin{equation*}
		\lim_{n\to\infty} \frac{\left(\prod_{i=1}^n i^{i^{2021}}\right)^{\frac{1}{n^{2022}}}}{n^{\frac{1}{2022}}} = \lim_{n\to\infty} \frac{\left(1^{1^{2021}}\cdot2^{2^{2021}}\cdots n^{n^{2021}}\right)^{\frac{1}{n^{2022}}}}{n^{\frac{1}{2022}}}.
	\end{equation*}
\end{baitoan}

\begin{baitoan}[\cite{VMS_VMC2023}, 1.9, pp. 31--32, ĐHSPHN2]
	Cho dãy số $\{x_n\}_{n=1}^\infty$ đặt bởi
	\begin{equation*}
		x_1\in(0,1),\ x_{n+1} = \frac{1}{n}\sum_{i=1}^n \ln(1 + x_i),\ \forall n\in\mathbb{N}^\star.
	\end{equation*}
	(a) Chứng minh dãy $\{x_n\}_{n=1}^\infty$ có giới hạn hữu hạn. (b) Chứng minh $\lim_{n\to\infty} \dfrac{n(x_n - x_{n+1})}{x_n^2} = \dfrac{1}{2}$.
\end{baitoan}

\begin{baitoan}[\cite{VMS_VMC2023}, 1.10, p. 32, ĐH Trà Vinh]
	Cho dãy số $\{x_n\}_{n=1}^\infty$ đặt bởi
	\begin{equation*}
		a_1 = a_2 = 1,\ a_{n+2} = \frac{1}{a_{n+1}} + a_n,\ \forall n\in\mathbb{N}^\star.
	\end{equation*}
	Tính $x_{2022}$.
\end{baitoan}

\begin{baitoan}[\cite{VMS_VMC2023}, 1.11, p. 32, ĐH Trà Vinh]
	Cho 2 dãy số $\{x_n\}_{n=1}^\infty,\{y_n\}_{n=1}^\infty$ đặt bởi
	\begin{equation*}
		x_1 = y_1 = \sqrt{3},\ x_{n+1} = x_n + \sqrt{1 + x_n^2},\ y_{n+1} = \frac{1}{1 + \sqrt{1 + y_n^2}},\ \forall n\in\mathbb{N}^\star.
	\end{equation*}
	Chứng minh $x_ny_n\in(2,3)$, $\forall n\ge2$ \& $\lim_{n\to\infty} y_n = 0$.
\end{baitoan}

\begin{baitoan}[\cite{VMS_VMC2023}, 1.11, p. 32, ĐH Vinh]
	Cho dãy số $\{x_n\}_{n=1}^\infty$ đặt bởi
	\begin{equation*}
		x_n = \prod_{i=1}^n \left(1 + \frac{1}{2^i}\right) = \left(1 + \frac{1}{2}\right)\left(1 + \frac{1}{2^2}\right)\cdots\left(1 + \frac{1}{2^n}\right),\ \forall n\in\mathbb{N}^\star.
	\end{equation*}
	(a) Tìm tất cả $n\in\mathbb{N}^\star$ thỏa $x_n > \frac{15}{8}$. (b) Chứng minh $\{x_n\}_{n=1}^\infty$ hội tụ.
\end{baitoan}

\begin{baitoan}[\cite{VMS_VMC2024}, p. 32, 1.1, VNUHCM UIT]
	Cho $a,b\in\mathbb{R}$, $a < b$. Xét dãy số
	\begin{equation*}
		\left\{\begin{split}
			x_0 &= a,\ x_1 = b,\\
			x_{n+1} &= x_n + \frac{1}{2}x_{n-1}\left(1 - \cos\frac{\pi}{n}\right).
		\end{split}\right.
	\end{equation*}
	Chứng minh $\{x_n\}$ hội tụ.
\end{baitoan}

\begin{baitoan}[\cite{VMS_VMC2024}, p. 32, 1.2, ĐH Đồng Tháp]
	Cho dãy số $\{u_n\}_{n=1}^\infty$ đặt bởi
	\begin{equation*}
		u_n = \sum_{i=1}^n \frac{i}{(i + 1)!} = \frac{1}{2!} + \frac{2}{3!} + \frac{3}{4!} + \cdots + \frac{n}{(n + 1)!},\ \forall n\in\mathbb{N}^\star.
	\end{equation*}
	(a) Tìm $n\in\mathbb{N}$ lớn nhất để $u_n < \dfrac{2023}{2024}$. (b) Tính giới hạn $\lim_{n\to\infty} \sqrt[n]{\sum_{i=1}^{2024} u_i^n} = \sqrt[n]{u_1^n + u_2^n + \cdots + u_{2024}^n}$.
\end{baitoan}

\begin{baitoan}[\cite{VMS_VMC2024}, p. 32, 1.3, ĐHGTVT]
	Cho dãy số $\{a_n\}_{n=1}^\infty$ thỏa $\frac{1}{2} < a_n < 1$, $\forall n\in\mathbb{N}^\star$. Dãy số $\{x_n\}$ đặt bởi
	\begin{equation*}
		x_1 = a_1,\ x_{n+1} = \frac{2(a_{n+1} + x_n) - 1}{1 + 2a_{n+1}x_n},\ \forall n\in\mathbb{N}^\star.
	\end{equation*}
	(a) Chứng minh dãy số $\{x_n\}_{n=1}^\infty$ tăng \& bị chặn trên. (b) Tìm $\lim_{n\to\infty} x_n$.
\end{baitoan}

\begin{baitoan}[\cite{VMS_VMC2024}, p. 33, 1.4, ĐH Vinh]
	Cho dãy số $\{x_n\}_{n=1}^\infty$ đặt bởi
	\begin{equation*}
		\left\{\begin{split}
			x_1 &= 2024,\\
			x_{n+1} &= \frac{x_n^2}{3\lfloor x_n\rfloor + 4},\ \forall n\in\mathbb{N}^\star.
		\end{split}\right.
	\end{equation*}
	(a) Chứng minh $x_8 < 1$. (b) Chứng minh $\{x_n\}_{n=1}^\infty$ hội tụ \& tìm giới hạn.
\end{baitoan}

%------------------------------------------------------------------------------%

\chapter{Function -- Hàm Số}
\minitoc

%------------------------------------------------------------------------------%

\section{Limit of function -- Giới hạn hàm số}

\begin{equation*}
	\lim_{x\to x_0} f(x) = l\Leftrightarrow\forall\varepsilon > 0,\ \exists\delta_\varepsilon > 0,\ |x - x_0| < \delta_\varepsilon\Rightarrow|f(x) - l| < \varepsilon.
\end{equation*}
hay tương đương với:
\begin{equation*}
	\lim_{x\to x_0} f(x) = l\Leftrightarrow\forall\varepsilon > 0,\ \exists\delta_\varepsilon > 0,\ |f(x) - l| < \varepsilon,\ \forall x\in\mathbb{R},\ |x - x_0| < \delta_\varepsilon.
\end{equation*}

\begin{baitoan}[\cite{TLCT_dai_so_giai_tich_11}, 8.]
	Áp dụng định nghĩa giới hạn của hàm số, tính giới hạn: (a) $\lim_{x\to-1} \dfrac{x^2 - 3x - 4}{x + 1}$. (b) $\lim_{x\to2} \sqrt{x + 2}$.
\end{baitoan}

\begin{proof}
	(a) $\lim_{x\to-1} \dfrac{x^2 - 3x - 4}{x + 1} = \lim_{x\to-1} \dfrac{(x + 1)(x - 4)}{x + 1} = \lim_{x\to-1} (x - 4)$. Chứng minh $\lim_{x\to-1} (x - 4) = -5$: Xét $\varepsilon > 0$ bất kỳ, xét bất phương trình $|x - 4 - (-5)| < \varepsilon\Leftrightarrow|x + 1| < \varepsilon\Leftrightarrow|x - (-1)| < \varepsilon$. Theo định nghĩa giới hạn của hàm số, suy ra $\lim_{x\to-1} (x - 4) = -5$.
	
	\item(b)  Xét $\varepsilon > 0$ bất kỳ, xét bất phương trình $|\sqrt{x + 2} - 2| < \varepsilon\Leftrightarrow2 - \varepsilon < \sqrt{x + 2} < 2 + \varepsilon\Leftrightarrow4 - 4\varepsilon + \varepsilon^2 < x + 2 < 4 + 4\varepsilon + \varepsilon^2\Leftrightarrow-4\varepsilon + \varepsilon^2 < x - 2 < 4\varepsilon + \varepsilon^2$. Nếu chọn $\delta_\varepsilon = \min\{|-4\varepsilon + \varepsilon^2|,|4\varepsilon + \varepsilon^2|\} = 4\varepsilon - \varepsilon^2$ thì $|x - 2| < \delta\Rightarrow|\sqrt{x + 2} - 2| < \varepsilon$, nên theo định nghĩa giới hạn của hàm số, $\lim_{x\to2} \sqrt{x + 2} = 2$.
\end{proof}

\begin{baitoan}
	Viết chương trình {\sf C{\tt/}C++, Pascal, Python} để tính giới hạn $\lim_{x\to x_0} \dfrac{P(x)}{Q(x)} = \lim_{x\to x_0} \dfrac{\sum_{i=0}^m a_ix^i}{\sum_{i=0}^n b_ix^i}$.
	
	\item {\sf Input.} Dòng 1 chứa $x_0\in\overline{\mathbb{R}} = \mathbb{R}\cup\{\pm\infty\}$, $m,n$. Dòng 2 chứa $a_0,a_1,\ldots,a_m$. Dòng 3 chứa $b_0,b_1,\ldots,b_n$.
	\item {\sf Output.} Giới hạn $\lim_{x\to x_0} \dfrac{\sum_{i=0}^m a_ix^i}{\sum_{i=0}^n b_ix^i}\in\overline{\mathbb{R}}$.
	\item {\sf Sample.}
	\begin{table}[H]
		\centering
	\end{table}
\end{baitoan}

\begin{baitoan}
	Viết chương trình {\sf C{\tt/}C++, Pascal, Python} để tính giới hạn $\lim_{x\to x_0} \dfrac{P(x)}{Q(x)} = \lim_{x\to x_0} \dfrac{\sum_{i=0}^m a_ix^{\alpha_i}}{\sum_{i=0}^n b_ix^{\beta_i}}$.
	
	\item {\sf Input.} Dòng 1 chứa $x_0\in\overline{\mathbb{R}} = \mathbb{R}\cup\{\pm\infty\}$, $m,n$. Dòng 2 chứa $a_0,a_1,\ldots,a_m$. Dòng 3 chứa $\alpha_0,\alpha_1,\ldots,\alpha_m\in\mathbb{R}$. Dòng 4 chứa $b_0,b_1,\ldots,b_n$. Dòng 5 chứa $\beta_0,\beta_1,\ldots,\beta_n\mathbb{R}$.
	\item {\sf Output.} Giới hạn $\lim_{x\to x_0} \dfrac{P(x)}{Q(x)} = \lim_{x\to x_0} \dfrac{\sum_{i=0}^m a_ix^{\alpha_i}}{\sum_{i=0}^n b_ix^{\beta_i}}$.
	\item {\sf Sample.}
	\begin{table}[H]
		\centering
	\end{table}
\end{baitoan}

\begin{baitoan}[\cite{TLCT_dai_so_giai_tich_11}, 9.]
	Cho hàm số $f(x) = \cos\frac{1}{x}$ \& 2 dãy số $\{x_n\}_{n=1}^\infty,\{y_n\}_{n=1}^\infty$:
	\begin{equation*}
		x_n = \frac{1}{2n\pi},\ y_n = \frac{1}{(2n + 1)\frac{\pi}{2}}.
	\end{equation*}
	(a) Tìm giới hạn của 4 dãy số $\{x_n\}_{n=1}^\infty,\{y_n\}_{n=1}^\infty,\{f(x_n)\}_{n=1}^\infty,\{f(y_n)\}_{n=1}^\infty$. (b) Tồn tại hay không giới hạn $\lim_{x\to0} \cos\frac{1}{x}$?
\end{baitoan}

\begin{proof}
	(a) $\lim_{n\to\infty} x_n = \lim_{n\to\infty} \frac{1}{2n\pi} = 0$, $\lim_{n\to\infty} y_n = \lim_{n\to\infty} \frac{1}{(2n + 1)\frac{\pi}{2}} = 0$, $\lim_{n\to\infty} f(x_n) = \lim_{n\to\infty} \cos2n\pi = \lim_{n\to\infty} 1 = 1$, $\lim_{n\to\infty} f(y_n) = \lim_{n\to\infty} \cos(2n + 1)\frac{\pi}{2} = \lim_{n\to\infty} 0 = 0$.
	
	\item(b) Ta có 2 dãy $\{x_n\}_{n=1}^\infty,\{y_n\}_{n=1}^\infty$ cùng tiến về 0 nhưng $\lim_{n\to\infty} f(x_n) = 1\ne0 = \lim_{n\to\infty} f(y_n)$, suy ra không tồn tại $\lim_{x\to0} f(x)$.
\end{proof}

\begin{baitoan}[\cite{TLCT_dai_so_giai_tich_11}, 10.]
	Tính: (a) $\lim_{x\to0} \dfrac{x^2 - 1}{2x^2 - x - 1}$. (b) $\lim_{x\to1} \dfrac{x^2 - 1}{2x^2 - x - 1}$. (c) $\lim_{x\to\infty} \dfrac{x^2 - 1}{2x^2 - x - 1}$.
\end{baitoan}

\begin{proof}
	(a) $1$. (b) $\frac{2}{3}$. (c) $\frac{1}{2}$.
\end{proof}

\begin{baitoan}[\cite{TLCT_dai_so_giai_tich_11}, 11.]
	Tính: (a) $\lim_{x\to0} \dfrac{(1 + x)(1 + 2x)(1 + 3x) - 1}{x}$. (b) $\lim_{x\to3} \dfrac{x^2 - 5x + 6}{x^2 - 8x + 15}$.\\(c) $\lim_{x\to\infty} \dfrac{(1 + x)(1 + 2x)(1 + 3x)(1 + 4x)(1 + 5x)}{(2x + 3)^5}$. (d) $\lim_{x\to2} \sqrt{\dfrac{x^2 - 4}{x^3 - 3x - 2}}$. (e) $\lim_{x\to0} \dfrac{\sqrt{1 + 2x} - \sqrt[3]{1 + 3x}}{x}$.\\(f) $\lim_{x\to1} \dfrac{3}{1 - \sqrt{x}} - \dfrac{3}{1 - \sqrt[3]{x}}$.
\end{baitoan}

\begin{baitoan}[\cite{TLCT_dai_so_giai_tich_11}, 12.]
	Cho hàm số
	\begin{equation*}
		f(x) = \left\{\begin{split}
			&x^2 - 2x + 3&&\mbox{if } x\le2,\\
			&4x - 3&&\mbox{if } x > 2.
		\end{split}\right.
	\end{equation*}
	Tính $\lim_{x\to2^+} f(x),\lim_{x\to2^-} f(x),\lim_{x\to2} f(x)$.
\end{baitoan}

\begin{baitoan}[\cite{TLCT_dai_so_giai_tich_11}, 13.]
	Tính: (a) $\lim_{x\to a} \dfrac{\sin x - \sin a}{x - a}$. (b) $\lim_{x\to0} \dfrac{1 - \cos x\cos2x\cos3x}{1 - \cos x}$. (c) $\lim_{x\to\frac{\pi}{3}} \dfrac{\sin\left(x - \frac{\pi}{3}\right)}{1 - 2\cos x}$.
\end{baitoan}

\begin{baitoan}[\cite{TLCT_dai_so_giai_tich_11}, 14.]
	Tính: (a) $\lim_{x\to\infty} \left(\dfrac{x + a}{x - a}\right)^x$. (b) $\lim_{x\to1} (x - 1)\log_x2$. (c) $\lim_{x\to2} \dfrac{2^x - x^2}{x - 2}$.
\end{baitoan}

%------------------------------------------------------------------------------%

\section{Continuous function -- Hàm số liên tục}

\begin{baitoan}[\cite{TLCT_dai_so_giai_tich_11}, 15.]
	Chứng minh: (a) 2 hàm số $f(x) = x^3 - x + 2,g(x) = \dfrac{x^3 + 1}{x^2 + 1}$ liên tục tại mọi điểm $x\in\mathbb{R}$. (b) Hàm số
	\begin{equation*}
		f(x) = \left\{\begin{split}
			&\dfrac{x^2 - x - 2}{x - 2}&&\mbox{if } x\ne2,\\
			&3&&\mbox{if } x = 2,
		\end{split}\right.
	\end{equation*}
	liên tục tại điểm $x = 2$. (c) Hàm số
	\begin{equation*}
		f(x) = \left\{\begin{split}
			&\dfrac{x^3 - 1}{x - 1}&&\mbox{if } x\ne1,\\
			&2&&\mbox{if } x = 1,
		\end{split}\right.
	\end{equation*}
	gián đoạn tại điểm $x = 1$.
\end{baitoan}

\begin{baitoan}[\cite{TLCT_dai_so_giai_tich_11}, 16.]
	Chứng minh: (a) Hàm số $f(x) = (x^2 - 2)^2 + 2$ liên tục trên $\mathbb{R}$. (b) Hàm số $f(x) = \dfrac{1}{\sqrt{1 - x^2}}$ liên tục trên $(-1,1)$. (c) Hàm số $f(x) = \sqrt{4 - x^2}$ liên tục trên $[-2,2]$. (d) Hàm số $f(x) = \sqrt{2x - 1}$ liên tục trên $[\frac{1}{2},\infty)$.
\end{baitoan}

\begin{baitoan}[\cite{TLCT_dai_so_giai_tich_11}, 17.]
	Sử dụng bất đẳng thức $|\sin x|\le|x|$, $\forall x\in\mathbb{R}$, chứng minh tính liên tục của hàm số $y = \cos x$ tại điểm $x = x_0$ bất kỳ.
\end{baitoan}

\begin{baitoan}[\cite{TLCT_dai_so_giai_tich_11}, 18.]
	Tìm tất cả các điểm gián đoạn của hàm số: (a) $y = \dfrac{1 + x}{1 + x^3}$. (b) $y = \sqrt{\dfrac{1 - \cos\pi x}{4 - x^2}}$. (c) $y = x - \lfloor x\rfloor$. (d) $y = \dfrac{1}{\ln x}$.
\end{baitoan}

\begin{baitoan}[\cite{TLCT_dai_so_giai_tich_11}, 19.]
	(a) Chứng minh phương trình bậc 3 $x^3 + ax^2 + bx + c = 0$ luôn có ít nhất 1 nghiệm thực $\forall a,b,c\in\mathbb{R}$. (b) Mở rộng bài toán.
\end{baitoan}

\begin{baitoan}[\cite{TLCT_dai_so_giai_tich_11}, 20.]
	Tìm tất cả $m\in\mathbb{R}$ để phương trình $\sqrt{1 + x} + \sqrt{1 - x} = m$ có nghiệm.
\end{baitoan}

\begin{baitoan}[\cite{TLCT_dai_so_giai_tich_11}, 21.]
	Giải bất phương trình $\sqrt{x + 1} + \sqrt[3]{7 - x} > 2$.
\end{baitoan}

\begin{baitoan}[\cite{TLCT_BT_dai_so_giai_tich_11}, 25., p. 48]
	Tính $\lim_{x\to-\infty} \sqrt{x^2 + x + 1} + x$.
\end{baitoan}

\begin{baitoan}[\cite{TLCT_BT_dai_so_giai_tich_11}, 26., p. 48]
	Tính $\lim_{x\to0} \dfrac{\tan x - \sin x}{x^3}$.
\end{baitoan}

\begin{baitoan}[\cite{TLCT_BT_dai_so_giai_tich_11}, 27., p. 48]
	Sử dụng giới hạn đặc biệt $\lim_{x\to0} \dfrac{e^x - 1}{x} = 1$, chứng minh hàm số $y = e^x\in C(\mathbb{R})$.
\end{baitoan}

\begin{baitoan}[\cite{TLCT_BT_dai_so_giai_tich_11}, 28., p. 48]
	Tìm tất cả $m\in\mathbb{R}$ để hàm số
	\begin{equation*}
		f(x) = \left\{\begin{split}
			&\dfrac{x^2 - 3x + 2}{x^2 - 2x}&&\mbox{if } x < 2,\\
			&mx + m + 1&&\mbox{if } x\ge2,
		\end{split}\right.\in C(\mathbb{R}).
	\end{equation*}
\end{baitoan}

\begin{baitoan}[\cite{TLCT_BT_dai_so_giai_tich_11}, 29., p. 48]
	Tìm tất cả hàm số $f:\mathbb{R}\to\mathbb{R}$ liên tục tại điểm $0$ \& thỏa $f(3x) = f(x)$, $\forall x\in\mathbb{R}$.
\end{baitoan}

\begin{baitoan}[\cite{TLCT_BT_dai_so_giai_tich_11}, 30., p. 48]
	Tìm tất cả hàm số $f:\mathbb{R}\to\mathbb{R}$ liên tục tại điểm $0$ \& thỏa $f(x + y) = f(x) + f(y) $, $\forall x,y\in\mathbb{R}$.
\end{baitoan}

\begin{baitoan}[\cite{TLCT_BT_dai_so_giai_tich_11}, 31., p. 48]
	Tìm ví dụ về 1 hàm số $f:\mathbb{R}\to\mathbb{R}$ thỏa $f$ gián đoạn tại mọi điểm thuộc $\mathbb{R}$ nhưng $f\circ f$ liên tục tại mọi điểm thuộc $\mathbb{R}$.
\end{baitoan}

\begin{baitoan}[\cite{TLCT_BT_dai_so_giai_tich_11}, 32., p. 48]
	Chứng minh parabol $(P):y = x^2 - 2x$ \& ellipse $(E):\dfrac{x^2}{9} + y^2 = 1$ cắt nhau tại 4 điểm phân biệt nằm trên 1 đường tròn.
\end{baitoan}

\begin{baitoan}[\cite{TLCT_BT_dai_so_giai_tich_11}, 33., p. 48]
	Cho $f:\in C([0,1],[0,1])$. Chứng minh tồn tại điểm $x_0\in[0,1]$ thỏa $f(x_0) = x_0$.
\end{baitoan}

\begin{baitoan}[\cite{TLCT_BT_dai_so_giai_tich_11}, 34., p. 48]
	Dùng phương pháp chia đôi, tìm nghiệm của phương trình $x^5 + x + 1 = 0$ với độ chính xác $0.1$.
\end{baitoan}
Xem code C{\tt/}C++ của bài toán này ở \cite{Thu_Phuong_Tien_Triet_NMLT}.

%------------------------------------------------------------------------------%

\section{Weight Function -- Hàm Trọng Số}
A {\it weight function} is a mathematical device used when performing a sum, integral, or average to give some elements more ``weight'' or influence on the result than other elements in the same set. The result of this application of a weight function is a {\it weighted sum} or \href{https://en.wikipedia.org/wiki/Weighted_average}{weighted average}. Weight functions occur frequently in statistics \& analysis, \& are closely related to the concept of a measure. Weight functions can be employed in both discrete \& continuous settings. They can be used to construct systems of calculus called ``weighted calculus'' \& ``meta-calculus''.

-- {\it weight function} là một công cụ toán học được sử dụng khi thực hiện tổng, tích phân hoặc trung bình để đưa ra một số phần tử có ``trọng số'' hoặc ảnh hưởng đến kết quả nhiều hơn các phần tử khác trong cùng một tập hợp. Kết quả của ứng dụng này của hàm trọng số là {\it weighted sum} hoặc trung bình có trọng số. Các hàm trọng số thường xuất hiện trong thống kê \& phân tích, \& có liên quan chặt chẽ đến khái niệm về phép đo. Các hàm trọng số có thể được sử dụng trong cả thiết lập rời rạc \& liên tục. Chúng có thể được sử dụng để xây dựng các hệ thống phép tính được gọi là ``phép tính có trọng số'' \& ``siêu phép tính''.

%------------------------------------------------------------------------------%

\subsection{Discrete weight -- Trọng số rời rạc}

%------------------------------------------------------------------------------%

\subsubsection{General discrete weights -- Trọng số rời rạc tổng quát}

\begin{definition}[Discrete weight function]
    In the discrete setting, a {\rm weight function} $w:A\to\mathbb{R}^+$ is a positive function defined on a discrete set $A$, which is typically finite or countable. The weight function $w(a)\coloneqq1$ corresponds to the {\rm unweighted} situation in which all elements have equal weight. One can then apply this weight to various concepts.
\end{definition}

\begin{dinhnghia}[Hàm trọng số rời rạc]
    Trong bối cảnh rời rạc, {\rm weight function} $w:A\to\mathbb{R}^+$ là một hàm dương được xác định trên một tập rời rạc $A$, thường là hữu hạn hoặc đếm được. Hàm trọng số $w(a)\coloneqq1$ tương ứng với tình huống {\rm unweighted} trong đó tất cả các phần tử đều có trọng số bằng nhau. Sau đó, người ta có thể áp dụng trọng số này cho nhiều khái niệm khác nhau.    
\end{dinhnghia}
If the function $f:A\to\mathbb{R}$ is a real-valued function, then the {\it unweighted sum of $f$ on $A$} is defined as $\sum_{a\in A} f(a)$ but given a {\it weight function} $w:A\to\mathbb{R}^+$, the {\rm weighted sum} or \href{https://en.wikipedia.org/wiki/Conical_combination}{conical combination} is defined as $\sum_{a\in A} f(a)w(a)$. 1 common application of weighted sum arises in \href{https://en.wikipedia.org/wiki/Numerical_integration}{numerical integration}.

-- Nếu hàm $f:A\to\mathbb{R}$ là hàm có giá trị thực, thì {\it tổng không trọng số của $f$ trên $A$} được định nghĩa là $\sum_{a\in A} f(a)$ nhưng với một {\it hàm trọng số} $w:A\to\mathbb{R}^+$, {\rm tổng trọng số} hoặc tổ hợp hình nón được định nghĩa là $\sum_{a\in A} f(a)w(a)$. 1 ứng dụng phổ biến của tổng trọng số phát sinh trong tích phân số, i.e., xấp xỉ tích phân .

If $B$ is a finite subset of $A$, one can replace the unweighted cardinality $|B|$ of $B$ by the {\it weighted cardinality} $\sum_{a\in B} w(a)$. If $A$ is a finite nonempty set, one can replace the unweighted mean or average $\frac{1}{|A|}\sum_{a\in A} f(a)$ by the \href{https://en.wikipedia.org/wiki/Weighted_mean}{weighted mean} or \href{https://en.wikipedia.org/wiki/Weighted_average}{weighted average}
\begin{equation*}
    \frac{\sum_{a\in A} f(a)w(a)}{\sum_{a\in A} w(a)}.
\end{equation*}
In this case only the {\it relative} weights are relevant.

-- Nếu $B$ là một tập con hữu hạn của $A$, ta có thể thay thế số lượng không trọng số $|B|$ của $B$ bằng {\it số lượng có trọng số} $\sum_{a\in B} w(a)$. Nếu $A$ là một tập hữu hạn không rỗng, ta có thể thay thế trung bình hoặc trung bình không trọng số $\frac{1}{|A|}\sum_{a\in A} f(a)$ bằng trung bình có trọng số hoặc trung bình có trọng số
\begin{equation*}
    \frac{\sum_{a\in A} f(a)w(a)}{\sum_{a\in A} w(a)}.
\end{equation*}
Trong trường hợp này, chỉ có trọng số {\it tương đối} là có liên quan.

%------------------------------------------------------------------------------%

\subsubsection{Statistical discrete weight -- Trọng số rời rạc thống kê}
Weighted means are commonly used in statistics to compensate for the presence of \href{https://en.wikipedia.org/wiki/Bias_(statistics)}{bias}. For a quantity $f$ measured multiple independent times $f_i$ with \href{https://en.wikipedia.org/wiki/Variance}{variance} $\sigma_i^2$, the best estimate of the signal is obtained by averaging all the measurements with weight $w_i = \frac{1}{\sigma_i^2}$, \& the resulting variance is smaller than each of the independent measurements $\sigma^2 = \frac{1}{\sum_i w_i}$. The \href{https://en.wikipedia.org/wiki/Maximum_likelihood}{maximum likelihood} method weights the difference between fit \& data using the same weights $w_i$.

-- Các giá trị trung bình có trọng số thường được sử dụng trong thống kê để bù đắp cho sự hiện diện của độ lệch. Đối với một lượng $f$ được đo nhiều lần độc lập $f_i$ với phương sai $\sigma_i^2$, ước tính tốt nhất của tín hiệu thu được bằng cách lấy trung bình tất cả các phép đo với trọng số $w_i = \frac{1}{\sigma_i^2}$, \& phương sai kết quả nhỏ hơn mỗi phép đo độc lập $\sigma^2 = \frac{1}{\sum_i w_i}$. Phương pháp độ tin cậy tối đa tính trọng số cho sự khác biệt giữa dữ liệu phù hợp \& bằng cách sử dụng cùng trọng số $w_i$.

The \href{https://en.wikipedia.org/wiki/Expected_value}{expected values} of a random variable is the weighted average of the possible values it might take on, with the weights being the respective probabilities. More generally, the expected value of a function of a random variable is the probability-weighted average of the values the function takes on for each possible value of the random variable.

-- Giá trị kỳ vọng của một biến ngẫu nhiên là giá trị trung bình có trọng số của các giá trị có thể mà nó có thể nhận, với các trọng số là các xác suất tương ứng. Tổng quát hơn, giá trị kỳ vọng của một hàm của một biến ngẫu nhiên là giá trị trung bình có trọng số xác suất của các giá trị mà hàm nhận đối với mỗi giá trị có thể của biến ngẫu nhiên.

In \href{https://en.wikipedia.org/wiki/Linear_regression}{regressions} in which the \href{https://en.wikipedia.org/wiki/Dependent_variable}{dependent variable} is assumed to be affected by both current \& lagged (past) values of the \href{https://en.wikipedia.org/wiki/Independent_variable}{independent variable}, a \href{https://en.wikipedia.org/wiki/Distributed_lag}{distributed lag} function is estimated, this function being a weighted average of the current \& various lagged independent variable values. Similarly, a \href{https://en.wikipedia.org/wiki/Moving_average_model}{moving average model} specifies an evolving variable as a weighted average of current \& various lagged values of a random variable.

-- Trong các hồi quy mà biến phụ thuộc được cho là bị ảnh hưởng bởi cả giá trị hiện tại \& trễ (quá khứ) của biến độc lập, một hàm trễ phân phối được ước tính, hàm này là giá trị trung bình có trọng số của các giá trị hiện tại \& trễ khác nhau của biến độc lập. Tương tự như vậy, một mô hình trung bình động chỉ định một biến đang tiến hóa là giá trị trung bình có trọng số của các giá trị hiện tại \& trễ khác nhau của một biến ngẫu nhiên.

%------------------------------------------------------------------------------%

\subsubsection{Mechanical discrete weight function -- Hàm trọng số rời rạc cơ học}
The terminology {\it weight function} arises from mechanics: if one has a collection of $n\in\mathbb{N}^\star$ objects on a \href{https://en.wikipedia.org/wiki/Lever}{lever}, with weights $w_1,\ldots,w_n$ (where weight is now interpreted in the physical sense) \& locations ${\bf x}_1,\ldots,{\bf x}_n$, then the lever will be in balance if the fulcrum of the lever is at the \href{https://en.wikipedia.org/wiki/Center_of_mass}{center of mass}
\begin{equation*}
    \frac{\sum_{i=0}^n w_i{\bf x}_i}{\sum_{i=0}^n w_i},
\end{equation*}
which is also the weighted average of the positions ${\bf x}_i$.

-- Thuật ngữ {\it weight function} phát sinh từ cơ học: nếu ta có một tập hợp $n\in\mathbb{N}^\star$ vật thể trên một đòn bẩy, với các trọng số $w_1,\ldots,w_n$ (trong đó trọng số hiện được diễn giải theo nghĩa vật lý) \& các vị trí ${\bf x}_1,\ldots,{\bf x}_n$, thì đòn bẩy sẽ cân bằng nếu điểm tựa của đòn bẩy nằm ở tâm khối lượng
\begin{equation*}
    \frac{\sum_{i=0}^n w_i{\bf x}_i}{\sum_{i=0}^n w_i},
\end{equation*}
cũng là giá trị trung bình có trọng số của các vị trí ${\bf x}_i$.

%------------------------------------------------------------------------------%

\subsection{Continuous weight -- Trọng số liên tục}
In the continuous setting, a weight is a positive measure e.g. $w(x)\,{\rm d}x$ on some domain $\Omega$, which is typically a subset of a Euclidean space $\mathbb{R}^n$, e.g. $\Omega$ could be an interval $[a,b]$. Here ${\rm d}x$ is Lebesgue measure \& $w:\Omega\to\mathbb{R}^+$ is a nonnegative measurable function. In this context, the weight function $w(x)$ is sometimes referred to as a density.

-- Trong bối cảnh liên tục, trọng số là một phép đo dương, ví dụ $w(x)\,{\rm d}x$ trên một miền nào đó $\Omega$, thường là một tập con của không gian Euclidean $\mathbb{R}^n$, ví dụ $\Omega$ có thể là một khoảng $[a,b]$. Ở đây ${\rm d}x$ là phép đo Lebesgue \& $w:\Omega\to\mathbb{R}^+$ là một hàm đo lường không âm. Trong bối cảnh này, hàm trọng số $w(x)$ đôi khi được gọi là mật độ.

\subsubsection{General definition of continuous weight}

\begin{definition}
    If $f:\Omega\to\mathbb{R}$ is a real-valued function, then the {\rm unweighted} integral $\int_\Omega f(x)\,{\rm d}x$ can be generalized to the {\rm unweighted integral} $\int_\Omega f(x)w(x)\,{\rm d}x$.
\end{definition}
One may need to require $f$ to be absolutely integrable w.r.t. the weight $w(x)\,{\rm d}x$ in order for this integral to be finite.

%------------------------------------------------------------------------------%

\subsubsection{Weighted volume -- Thể tích có trọng số}

\begin{definition}[Weighted volume]
    If $E\subset\Omega$, then the volume ${\rm vol}(E)$ of $E$ can be generalized to the {\rm weighted volume} $\int_E w({\bf x})\,{\rm d}{\bf x}$.
\end{definition}

\begin{dinhnghia}[Thể tích có trọng số]
    Nếu $E\subset\Omega$, thì thể tích ${\rm vol}(E)$ của $E$ có thể được khái quát thành {\rm thể tích có trọng số} $\int_E w({\bf x})\,{\rm d}{\bf x}$.
\end{dinhnghia}

%------------------------------------------------------------------------------%

\subsubsection{Weighted average -- Trung bình có trọng số}

\begin{definition}[Weighted average]
    If $\Omega$ has finite nonzero weighted volume, then we can replace the unweighted average $\frac{1}{{\rm vol}(\Omega)}\int_\Omega f({\bf x})\,{\rm d}{\bf x}$ by the {\rm weighted average}
    \begin{equation*}
        \frac{\int_\Omega f({\bf x})w({\bf x})\,{\rm d}{\bf x}}{\int_\Omega w({\bf x})\,{\rm d}{\bf x}}.
    \end{equation*}
\end{definition}

\begin{dinhnghia}[Trung bình có trọng số]
    Nếu $\Omega$ có thể tích hữu hạn có trọng số khác không, thì chúng ta có thể thay thế trung bình không có trọng số $\frac{1}{{\rm vol}(\Omega)}\int_\Omega f({\bf x})\,{\rm d}{\bf x}$ bằng {\rm trung bình có trọng số}
    \begin{equation*}
        \frac{\int_\Omega f({\bf x})w({\bf x})\,{\rm d}{\bf x}}{\int_\Omega w({\bf x})\,{\rm d}{\bf x}}.
    \end{equation*}
\end{dinhnghia}

%------------------------------------------------------------------------------%

\subsubsection{Bilinear form -- Dạng song tuyến tính}
If $f:\Omega\to\mathbb{R},g:\Omega\to\mathbb{R}$ are 2 functions, one can generalize the unweighted \href{https://en.wikipedia.org/wiki/Bilinear_form}{bilinear form} $\langle f,g\rangle\coloneqq\int_\Omega f({\bf x})g({\bf x})\,{\rm d}{\bf x}$ to a weighted bilinear form $\langle f,g\rangle_w\coloneqq\int_\Omega f({\bf x})g({\bf x})w({\bf x})\,{\rm d}{\bf x}$. See \href{https://en.wikipedia.org/wiki/Orthogonal_polynomials}{orthogonal polynomials} for examples of weighted \href{https://en.wikipedia.org/wiki/Orthogonal_functions}{orthogonal functions}.

%------------------------------------------------------------------------------%

\section{Problem: Function -- Bài tập: Hàm số}

\begin{baitoan}[\cite{VMS_VMC2023}, 3.1, p. 33, HCMUT]
	(a) Chứng minh tồn tại hàm số $f\in C^2(\mathbb{R},\mathbb{R})$ thỏa $xf''(x) + 2f'(x) = x^{2023}$, $\forall x\in\mathbb{R}$. (b) Giả sử $g\in C^2(\mathbb{R},\mathbb{R})$ thỏa $xg''(x) + 2g'(x)\ge x^{2023}$, $\forall x\in\mathbb{R}$. Chứng minh $\int_{-1}^1 x(g(x) + x^{2023})\,{\rm d}x\ge\dfrac{2}{2025}$.
\end{baitoan}

\begin{baitoan}[\cite{VMS_VMC2023}, 3.2, p. 33, ĐH Đồng Tháp]
	Cho hàm $f(x)x = 2(x - 1) - \arctan x$, $\forall x\in\mathbb{R}$. Chứng minh phương trình $f(x) = 0$ có nghiệm duy nhất là $a\in(1,\sqrt{3})$.
\end{baitoan}

\begin{proposition}[Luật bình phương nghịch đảo]
	Mỗi sự gia tăng khoảng cách từ nguồn cho ra kết quả giảm mức độ âm thanh theo tỷ lệ nghịch với bình phương của sự gia tăng khoảng cách.
\end{proposition}

\begin{baitoan}[\cite{VMS_VMC2023}, 3.3, pp. 33--34, ĐH Đồng Tháp]
	Sử dụng luật bình phương nghịch đảo, giải quyết bài toán: 1 người có 1 mảnh đất lớn có chiều dài mặt tiền là $l$ {\rm m} ở giữa 2 quán karaoke thường phát ra âm thanh có cường độ lần lượt là $I_1,I_2$. Người này định xây 1 ngôi nhà nhỏ trên mảnh đất đó nhưng muốn tìm vị trí sao cho chịu ảnh hưởng của âm thanh từ 2 quán karaoke là ít nhất. Giúp người này nếu biết: (a) Cường độ âm thanh $I_1 = I_2$. (b) Cường độ âm thanh $I_1 = 8I_2$. (c) $I_1 = aI_2$ với $a\in(0,\infty)$ cho trước.
\end{baitoan}

\begin{baitoan}[\cite{VMS_VMC2023}, 3.5, p. 34, ĐH Hùng Vương, Phú Thọ]
	Cho hàm
	\begin{equation*}
		f(x) = \left\{\begin{split}
			&x^2\sin\frac{1}{x} + \alpha x&&\mbox{if } x\ne0,\\
			&0&&\mbox{if } x = 0.
		\end{split}\right.
	\end{equation*}
	(a) Tính $f'(x)$ khi $x\ne0$. (b) Tính $f'(0)$. (c) Chứng minh hàm $f(x)$ không đơn điệu trên mỗi khoảng mở chứa điểm $0$.
\end{baitoan}

\begin{baitoan}[\cite{VMS_VMC2023}, 3.6, p. 34, ĐH Hùng Vương, Phú Thọ]
	(a) Gia đình bác Nam muốn xây 1 cái bể hình hộp với đáy là hình vuông có thể tích $V = 10\ {\rm m}^3$. Biết giá thành để xây mỗi $\rm m^2$ mặt đấy là $a = 700000$ đồng \& 1 mặt bên là $b = 500000$ đồng. Để tổng chi phí xây dựng là nhỏ nhất thì bác Nam nên xây bể với kích thước như thế nào? (b) Giải bài toán với $a,b,V\in(0,\infty)$ bất kỳ.
\end{baitoan}

\begin{baitoan}[\cite{VMS_VMC2023}, 3.7, pp. 34--35, ĐHKH Thái Nguyên]
	Tìm các hàm liên tục $f:\mathbb{R}\to\mathbb{R}$, $f\not\equiv0$, thỏa
	\begin{equation*}
		f(x + y) = 2023^yf(x) + 2023^xf(y),\ \forall x,y\in\mathbb{R}.
	\end{equation*}
	Từ đó tính
	\begin{equation*}
		\lim_{x\to0} \frac{e^{f(x)} - 1}{\sin f(x)},\ \lim_{n\to\infty} \frac{n}{f^{(n)}(0)}.
	\end{equation*}
\end{baitoan}

\begin{baitoan}[\cite{VMS_VMC2023}, 3.8, p. 35, ĐH Mỏ--Địa chất]
	Tính
	\begin{equation*}
		\lim_{(x,y,z)\to(0,0,0)} \frac{\sin x^2 + \sin y^2 + \sin z^2}{x^2 + y^2 + z^2}.
	\end{equation*}
\end{baitoan}

\begin{baitoan}[\cite{VMS_VMC2023}, 3.9, p. 35, ĐH Mỏ--Địa chất]
	Gọi $y_1(x),y_2(x),y_3(x)$ là 3 nghiệm của phương trình vi phân $y''' + a(x)y'' + b(x)y' c(x)y = 0$ thỏa $y_1^2(x) + y_2^2(x) + y_3^2(x) = 1$, $\forall x\in\mathbb{R}$. Tìm các hằng số $\alpha,\beta$ để hàm $z = (y_1'(x))^2 + (y_2'(x))^2 + (y_3'(x))^2$ là nghiệm của phương trình vi phân $z' + \alpha a(x)z + \beta c(x) = 0$.
\end{baitoan}

\begin{baitoan}[\cite{VMS_VMC2023}, 3.10, p. 35, ĐH Mỏ--Địa chất]
	Trên hình ellipse $\dfrac{x^2}{a^2} + \dfrac{y^2}{b^2} = 1$, tìm tất cả các điểm $T = (x_0,y_0)$ thỏa: tam giác bị giới hạn bởi các đường thẳng $x = 0,y = 0$ \& tiếp tuyến với ellipse tại điểm $T$ có diện tích nhỏ nhất.
\end{baitoan}

\begin{baitoan}[\cite{VMS_VMC2023}, 3.11, p. 35, FTU Hà Nội]
	Chứng minh đa thức $f(x) = \sum_{i=0}^{2022} (-1)^i\frac{x^i}{i!} = 1 - x + \dfrac{x^2}{2!} - \dfrac{x^3}{3!} + \cdots + \dfrac{x^{2022}}{2022!}$ không có nghiệm thực.
\end{baitoan}

\begin{baitoan}[\cite{VMS_VMC2023}, 3.12, p. 35, ĐHSPHN2]
	Cho $f\in C(\mathbb{R},\mathbb{R})$, $a,b\in\mathbb{R}$, $a < b$. 1 điểm $x$ được gọi là 1 {\rm điểm mù} nếu tồn tại 1 điểm $y\in\mathbb{R}$ với $y > x$ sao cho $f(y) > f(x)$. Giả sử tất cả các điểm thuộc khoảng mở $I = (a,b)$ là các điểm mù \& $a,b$ không phải là 2 điểm mù. Chứng minh $f(a) = f(b)$.
\end{baitoan}

\begin{baitoan}[\cite{VMS_VMC2023}, 3.13, p. 36, ĐH Trà Vinh]
	Chứng minh hàm số $f(x) = x^{x^x}$ đồng biến trên $(0,\infty)$ \& $\lim_{x\to0^+} f(x) = 0$.
\end{baitoan}

\begin{baitoan}[\cite{VMS_VMC2023}, 3.14, p. 36, ĐH Vinh]
	Cho hàm
	\begin{equation*}
		f(x) = \left\{\begin{split}
			&\sqrt[3]{x^2}\sin\frac{1}{x^{2023}}&&\mbox{if } x\ne0,\\
			&0&&\mbox{if } x = 0.
		\end{split}\right.
	\end{equation*}
	(a) Chứng minh hàm số $f$ liên tục tại $x = 0$. (b) Hàm số $f$ có khả vi tại $x = 0$ hay không?
\end{baitoan}

\begin{baitoan}[\cite{VMS_VMC2023}, 3.15, p. 36, ĐH Vinh]
	Cho hàm $f\in C([0,1],\mathbb{R})$, khả vi trên khoảng $(0,1)$, thỏa $f(0) = 0$, \& $|f'(x)|\le2023|f(x)|$, $\forall x\in(0,1)$. Chứng minh $f(x) = 0$, $\forall x\in[0,1]$.
\end{baitoan}

\begin{baitoan}[\cite{VMS_VMC2023}, 3.16, p. 36, ĐH Vinh]
	Giả sử hàm $f:(0,\infty)\to\mathbb{R}$ khả vi trên khoảng $(0,\infty)$ \& thỏa 2 điều kiện: (i) $|f(x)|\le2023$, $\forall x\in(0,\infty)$; (ii) $f(x)f'(x)\ge2022\cos x$, $\forall x\in(0,\infty)$. Có tồn tại $\lim_{x\to\infty} f(x)$ không?
\end{baitoan}

%------------------------------------------------------------------------------%

\chapter{Continuity -- Sự Liên Tục}
\minitoc

\begin{definition}[\cite{Tao_analysis_1}, Def. 6.1.1, p. 109: distance between 2 reals]
	Given $x,y\in\mathbb{R}$, their distance $d(x,y)$ is defined to be $d(x,y)\coloneqq|x - y|\in[0,\infty)$.
\end{definition}

\begin{definition}[\cite{Tao_analysis_1}, Def. 6.1.2, p. 109: $\varepsilon$-close reals]
	Let $\varepsilon > 0$ be a real number. $x,y\in\mathbb{R}$ is said to be {\rm$\varepsilon$-close} iff $d(x,y)\le\varepsilon$.
\end{definition}

%------------------------------------------------------------------------------%

\chapter{Series -- Chuỗi Số}
\minitoc

\begin{baitoan}[\cite{VMS_VMC2023}, 2.1, p. 32, VNUHCM UIT]
	Cho dãy số $\{x_n\}_{n=1}^\infty\subset(0,\infty)$ thỏa $\sum_{n=1}^\infty \dfrac{x_n}{(2n - 1)^2} < 1$. Chứng minh $\sum_{k=1}\sum_{n=1}^k \dfrac{x_n}{k^3} < 2$.
\end{baitoan}

\begin{baitoan}[\cite{VMS_VMC2023}, 2.2, p. 32, ĐHGTVT]
	Cho dãy số $\{a_n\}_{n=1}^\infty\subset(0,\infty)$ đặt bởi
	\begin{equation*}
		a_1 > 0,\ a_{n+1} = \frac{a_n^2}{a_n^2 - a_n + 1},\ \forall n\in\mathbb{N}^\star.
	\end{equation*}
	Tính $\sum_{n=1}^\infty a_n$.
\end{baitoan}

\begin{baitoan}[\cite{VMS_VMC2023}, 2.2, p. 32, ĐH Mỏ--Địa chất]
	Gọi $S$ là dãy con của dãy điều hòa $\left\{\dfrac{1}{n}\right\}_{n=1}^\infty = 1,\frac{1}{2},\frac{1}{3}\ldots,\frac{1}{n},\ldots$ \& có tổng hữu hạn. Gọi $c(n)$ là số lượng các phần tử của $S$ có số thứ tự trong dãy mẹ (điều hòa) ban đầu không vượt quá $n$. Chứng minh $\lim_{n\to\infty} \dfrac{c(n)}{n} = 0$.
\end{baitoan}

\begin{baitoan}[\cite{VMS_VMC2024}, p. 33, 2.1, ĐHCNTT TpHCM]
	Khảo sát sự hội tụ của chuỗi số
	\begin{equation*}
		\sum_{i=1}^{+\infty} \frac{\beta\sin^2l\alpha}{1 + \beta\sin^2k\alpha},\ \alpha\notin\{k\pi:k\in\mathbb{Z}\},\,\beta > 0.
	\end{equation*}
\end{baitoan}

%------------------------------------------------------------------------------%

\chapter{Derivative \& Differentiability -- Đạo Hàm \& Tính Khả Vi}
\minitoc

%------------------------------------------------------------------------------%

\section{Định nghĩa đạo hàm. Ý nghĩa hình học của đạo hàm}
\textbf{\textsf{Resources -- Tài nguyên.}}
\begin{enumerate}
	\item \href{https://en.wikipedia.org/wiki/Derivative}{Wikipedia{\tt/}derivative}.
\end{enumerate}
In mathematics, the {\it derivative} is a fundamental tool that quantifies the sensitivity to change of a function's output w.r.t. its input. The derivative of a function of a single variable at a chosen input value, when it exists, is the \href{https://en.wikipedia.org/wiki/Slope}{slope} of the \href{https://en.wikipedia.org/wiki/Tangent}{tangent line} to the \href{https://en.wikipedia.org/wiki/Graph_of_a_function}{graph of the function} at that point. The tangent line is the best \href{https://en.wikipedia.org/wiki/Linear_approximation}{linear approximation} of the function near that input value. For this reason, the derivative is often described as the {\it instantaneous rate of change}, the ratio of the instantaneous change in the dependent variable to that of the independent variable. The process of finding a derivative is called {\it differentiation}.

-- Trong toán học, {\it đạo hàm} là 1 công cụ cơ bản định lượng độ nhạy với sự thay đổi của đầu ra của 1 hàm so với đầu vào của nó. Đạo hàm của 1 hàm của 1 biến duy nhất tại 1 giá trị đầu vào đã chọn, khi nó tồn tại, là độ dốc của đường tiếp tuyến với đồ thị của hàm tại điểm đó. Đường tiếp tuyến là phép xấp xỉ tuyến tính tốt nhất của hàm gần giá trị đầu vào đó. Vì lý do này, đạo hàm thường được mô tả là {\it tốc độ thay đổi tức thời}, tỷ lệ giữa sự thay đổi tức thời của biến phụ thuộc với biến độc lập. Quá trình tìm đạo hàm được gọi là {\it phép vi phân}.

There are multiple different notations for differentiation. \href{https://en.wikipedia.org/wiki/Leibniz_notation}{Leibniz notation}, named after \href{https://en.wikipedia.org/wiki/Gottfried_Wilhelm_Leibniz}{\sc Gottfried Wilhelm Leibniz}, is represented as the ratio of 2 \href{https://en.wikipedia.org/wiki/Differential_(mathematics)}{differentials}, whereas {\it prime notation} is written by adding a prime mark. Higher order notations represent repeated differentiation, \& they are usually denoted in Leibniz notation by adding superscripts to the differentials, \& in prime notation by adding additional prime marks. The \href{https://en.wikipedia.org/wiki/Higher_order_derivative}{higher order derivatives} can be applied in physics, e.g., while the 1st derivative of the position of a moving object w.r.t. time is the object's \href{https://en.wikipedia.org/wiki/Velocity}{velocity}, how the position changes as time advances, the 2nd derivative is the object's \href{https://en.wikipedia.org/wiki/Acceleration}{acceleration}, how the velocity changes as time advances.

-- Có nhiều ký hiệu khác nhau cho phép tính vi phân. Ký hiệu Leibniz, được đặt theo tên của Gottfried Wilhelm Leibniz, được biểu diễn dưới dạng tỷ số của hai phép tính vi phân, trong khi ký hiệu nguyên tố được viết bằng cách thêm 1 dấu nguyên tố. Ký hiệu bậc cao hơn biểu diễn phép tính vi phân lặp lại và chúng thường được biểu diễn trong ký hiệu Leibniz bằng cách thêm các chữ số mũ vào các phép tính vi phân, và trong ký hiệu nguyên tố bằng cách thêm các dấu nguyên tố bổ sung. Các đạo hàm bậc cao hơn có thể được áp dụng trong vật lý; ví dụ, trong khi đạo hàm bậc nhất của vị trí của 1 vật chuyển động theo thời gian là vận tốc của vật, cách vị trí thay đổi khi thời gian trôi qua, thì đạo hàm bậc hai là gia tốc của vật, cách vận tốc thay đổi khi thời gian trôi qua.

Derivatives can be generalized to \href{https://en.wikipedia.org/wiki/Function_of_several_real_variables}{functions of several real variables}. In this case, the derivative is reinterpreted as a \href{https://en.wikipedia.org/wiki/Linear_transformation}{linear transformation} whose graph is (after an appropriate translation) the best linear approximation to the graph of the original function. The \href{https://en.wikipedia.org/wiki/Jacobian_matrix}{Jacobian matrix} is the matrix that represents this linear transformation w.r.t. the basis given by the choice of independent \& dependent variables. It can be calculated in terms of the \href{https://en.wikipedia.org/wiki/Partial_derivative}{partial derivatives} w.r.t. the independent variables. For a \href{https://en.wikipedia.org/wiki/Real-valued_function}{real-valued function} of several variables, the Jacobian matrix reduces to the \href{https://en.wikipedia.org/wiki/Gradient_vector}{gradient vector}.

Nếu quỹ đạo chuyển động của 1 vật hay 1 chất điểm được miêu tả bằng hàm số ${\bf x}(t)$ theo thời gian thì vận tốc ${\bf v}(t) = {\bf x}'(t)$ biểu thị độ nhanh chậm của chuyển động tại 1 thời điểm $t$.

%------------------------------------------------------------------------------%

\subsection{Definition of derivative as a limit -- Định nghĩa đạo hàm như 1 giới hạn}

\begin{definition}[$\varepsilon$-$\delta$ definition of derivative]
	A \href{https://en.wikipedia.org/wiki/Function_of_a_real_variable}{function of a real variable} $f(x)$ is \href{https://en.wikipedia.org/wiki/Differentiable_function}{differentiable} at a point $a$ of its domain, if its domain contains an open interval containing $a$, \& the limit
	\begin{equation*}
		l = \lim_{h\to0} \frac{f(a + h) - f(a)}{h}
	\end{equation*}
	exists. I.e., for every positive real number $\varepsilon\in(0,\infty)$, there exists a positive real number $\delta = \delta(\varepsilon)\in(0,\infty)$ s.t., for every $h\ne0$ s.t. $|h| < \delta$ then $f(a + h)$ is defined, \&
	\begin{equation*}
		\left|\frac{f(a + h) - f(a)}{h} - l\right| < \varepsilon.
	\end{equation*}
\end{definition}
If the function $f$ is differentiable at $a$, i.e., if the limit $l$ exists, then this limit is called the {\it derivative} of $f$ at $a$. Multiple notations for the derivative exist. The derivative of $f$ at $a$ can be denoted $f'(a)$, read as ``$f$ prime of $a$''; or it can be denoted $\frac{df}{dx}(a)$, read as ``the derivative of $f$ w.r.t. $x$ at $a$'' or ``$df$ by (or over) $dx$ at $a$''. If $f$ is a function that has a derivative at every point in its domain, then a function can be defined by mapping every point $x$ to the value of the derivative of $f$ at $x$. This function is written $f'$ \& is called the {\it derivative function} or the {\it derivative of $f$}. The function $f$ sometimes has a derivative at most, but not all, points of its domain. The function whose value at $a$ equals $f'(a)$ whenever $f'(a)$ is defined \& elsewhere is undefined is also called the derivative of $f$. It is still a function, but its domain may be smaller than the domain of $f$.

-- Nếu hàm $f$ khả vi tại $a$, tức là nếu giới hạn $l$ tồn tại, thì giới hạn này được gọi là {\it đạo hàm} của $f$ tại $a$. Có nhiều ký hiệu cho đạo hàm. Đạo hàm của $f$ tại $a$ có thể được ký hiệu là $f'(a)$, đọc là ``$f$ phẩy của $a$''; hoặc có thể được ký hiệu là $\frac{df}{dx}(a)$, đọc là ``đạo hàm của $f$ theo $x$ tại $a$'' hoặc ``$df$ theo (hoặc trên) $dx$ tại $a$''. Nếu $f$ là hàm có đạo hàm tại mọi điểm trong tập xác định của nó, thì hàm có thể được định nghĩa bằng cách ánh xạ mọi điểm $x$ tới giá trị của đạo hàm của $f$ tại $x$. Hàm này được ký hiệu là $f'$ \& được gọi là {\it đạo hàm hàm} hoặc {\it đạo hàm của $f$}. Hàm $f$ đôi khi có đạo hàm tại nhiều nhất, nhưng không phải tất cả, các điểm của tập xác định của nó. Hàm có giá trị tại $a$ bằng $f'(a)$ bất cứ khi nào $f'(a)$ được xác định \& ở nơi khác không xác định cũng được gọi là đạo hàm của $f$. Nó vẫn là 1 hàm, nhưng tập xác định của nó có thể nhỏ hơn tập xác định của $f$.

The ratio in the definition of the derivative is the slope of the line through 2 points on the graph of the function $f$, specially the points $(a,f(a))$ \& $(a + h,f(a + h))$. As $h$ is made smaller, these points grow closer together, \& the slope of this line approaches the limiting value, the slope of the tangent to the graph of $f$ at $a$. I.e., the derivative is the slope of the tangent.

-- Tỷ lệ trong định nghĩa của đạo hàm là độ dốc của đường thẳng qua 2 điểm trên đồ thị của hàm số $f$, đặc biệt là các điểm $(a,f(a))$ \& $(a + h,f(a + h))$. Khi $h$ nhỏ hơn, các điểm này sẽ gần nhau hơn, \& độ dốc của đường thẳng này tiến tới giá trị giới hạn, độ dốc của tiếp tuyến với đồ thị của $f$ tại $a$. I.e., đạo hàm là độ dốc của tiếp tuyến.

%------------------------------------------------------------------------------%

\subsubsection{Definition of derivative using infinitesimals -- Định nghĩa đạo hàm sử dụng vô cùng nhỏ}
1 way to think of the derivative $\frac{df}{dx}(a)$ is as the ratio of an \href{https://en.wikipedia.org/wiki/Infinitesimal}{infinitesimal} change in the output of the function $f$ to an infinitesimal change in its input. In order to make this intuition rigorous, a system of rules for manipulating infinitesimal quantities is required. The system of \href{https://en.wikipedia.org/wiki/Hyperreal_number}{hyperreal numbers} is a way of treating infinite \& infinitesimal quantities. The hyperreals are an \href{https://en.wikipedia.org/wiki/Field_extension}{extension} of the real numbers that contain numbers greater than anything of the form $1 + 1 + \cdots + 1$ for any finite number of terms. Such numbers are infinite, \& their \href{https://en.wikipedia.org/wiki/Multiplicative_inverse}{reciprocals} are infinitesimals. The application if hyperreal numbers to the foundations of calculus is called \href{https://en.wikipedia.org/wiki/Nonstandard_analysis}{nonstandard analysis}. This provides a way to define the basic concepts of calculus such as the derivative \& integral in terms of infinitesimals, thereby giving a precise meaning to the $d$ in the Leibniz notation. Thus, the derivative of $f(x)$ becomes
\begin{equation*}
	f'(x) = {\rm st}\left(\frac{f(x + dx) - f(x)}{dx}\right)
\end{equation*}
for an arbitrary infinitesimal $dx$, where st denotes the \href{https://en.wikipedia.org/wiki/Standard_part_function}{standard part function}, which ``rounds off'' each finite hyperreal to the nearest real.

%------------------------------------------------------------------------------%

\subsection{Continuity \& differentiability -- Liên tục \& khả vi}
If $f$ is differentiable at $a$, then $f$ must also be continuous at $a$. E.g., choose a point $a$ \& let $f$ be the \href{https://en.wikipedia.org/wiki/Step_function}{step function}
\begin{equation*}
	{\rm step}(x;a,a_l,a_r) = \left\{\begin{split}
		&a_l&&\mbox{if } x < a,\\
		&a_r&&\mbox{if } x\ge a.
	\end{split}\right.\mbox{ for }a\in\mathbb{R},a_l,a_r\in\mathbb{C},a_l\ne a_r.
\end{equation*}

\begin{remark}
	This step function is very common in the mathematical analysis of hyperbolic Partial Differential Equations (abbr., hyperbolic PDEs), especially in the shock waves \& rarefaction waves -- solutions of Riemann problem.
	
	-- Hàm bước này rất phổ biến trong phân tích toán học của Phương trình đạo hàm riêng hypebolic (viết tắt là PDE hypebolic), đặc biệt là trong sóng xung kích \& sóng loãng -- các nghiệm của bài toán Riemann.
\end{remark}
The function ${\rm step}(x;a,a_l,a_r)$ cannot have a derivative at $a$. If $h < 0$, then $a + h$ is on the low part of the step, so the secant line from $a$ to $a + h$ is very steep; as $h$ tends to 0, the slope tends to $\infty$. If $h > 0$, then $a + h$ is on the high part of the step, so the secant line from $a$ to $a + h$ has slope 0. Consequently, the secant lines do not approach any single slope, so the limit of the difference quotient does not exist. However, even if a function is continuous at a point, it may not be differentiable there.

-- Hàm ${\rm step}(x;a,a_l,a_r)$ không thể có đạo hàm tại $a$. Nếu $h < 0$, thì $a + h$ nằm ở phần thấp của bậc, do đó đường cắt từ $a$ đến $a + h$ rất dốc; khi $h$ tiến tới 0, độ dốc tiến tới $\infty$. Nếu $h > 0$, thì $a + h$ nằm ở phần cao của bậc, do đó đường cắt từ $a$ đến $a + h$ có độ dốc 0. Do đó, các đường cắt không tiến tới bất kỳ độ dốc đơn nào, do đó giới hạn của thương hiệu không tồn tại. Tuy nhiên, ngay cả khi 1 hàm liên tục tại 1 điểm, thì nó có thể không khả vi tại đó.

\begin{problem}
	Prove that the \href{https://en.wikipedia.org/wiki/Absolute_value}{absolute value} function given by $f(x) = |x|$ is continuous at $x = 0$, but it is not differentiable there.
\end{problem}

\begin{proof}
	If $h > 0$, then the slope of the secant line from 0 to $h$ is 1. If $h < 0$, then the slope of the secant line from 0 to $h$ is $-1$. This can be seen graphically as a ``kink'' or a ``cusp'' in the graph at $x = 0$.
\end{proof}

\begin{problem}
	Investigate the continuity \& differentiability of functions: (a) $x^a|x|$, $a > 0$. (b) $x^a|x|$, $a < 0$. (c) $x^a|x|^b$ for $a,b\in\mathbb{R}$.
\end{problem}
Even a function with a smooth graph is not differentiable at a point where its \href{https://en.wikipedia.org/wiki/Vertical_tangent}{tangent is vertical}, e.g.:

-- Ngay cả 1 hàm số có đồ thị trơn cũng không khả vi tại điểm mà tiếp tuyến của nó thẳng đứng, e.g.:

\begin{problem}
	Prove that the function $f(x) = x^{\frac{1}{3}} = \sqrt[3]{x}$ is continuous on the whole $\mathbb{R}$ but not differentiable at $x = 0$.
\end{problem}
In summary, a function that has a derivative is continuous, but there are continuous functions that do not have a derivative.

-- Tóm lại, 1 hàm số có đạo hàm là hàm số liên tục, nhưng có những hàm số liên tục nhưng không có đạo hàm.

Most functions that occur in practice have derivatives at all points or \href{https://en.wikipedia.org/wiki/Almost_everywhere}{almost every} point. Early in the \href{https://en.wikipedia.org/wiki/History_of_calculus}{history of calculus}, many mathematicians assumed that a continuous function was differentiable at most points. Under mild conditions (e.g., if the function is a \href{https://en.wikipedia.org/wiki/Monotone_function}{monotone} or a \href{https://en.wikipedia.org/wiki/Lipschitz_function}{Lipschitz function}), this is true. However, in 1872, {\sc Weierstrass} found the 1st example of a function, called the \href{https://en.wikipedia.org/wiki/Weierstrass_function}{Weierstrass function}, that is continuous everywhere but differentiable nowhere. In 1931, \href{https://en.wikipedia.org/wiki/Stefan_Banach}{\sc Stefan Banach} proved that the set of functions that have a derivative at some point is a \href{https://en.wikipedia.org/wiki/Meager_set}{meager set} in the space of all continuous functions. Informally, this means that hardly any random continuous functions have a derivative at even 1 point.

-- Hầu hết các hàm xuất hiện trong thực tế đều có đạo hàm tại mọi điểm hoặc hầu như mọi điểm. Vào đầu lịch sử phép tính, nhiều nhà toán học cho rằng 1 hàm liên tục có thể vi phân tại hầu hết các điểm. Trong điều kiện nhẹ nhàng (ví dụ, nếu hàm là hàm đơn điệu hoặc hàm Lipschitz), điều này là đúng. Tuy nhiên, vào năm 1872, {\sc Weierstrass} đã tìm thấy ví dụ đầu tiên về 1 hàm, được gọi là hàm Weierstrass, liên tục ở mọi nơi nhưng không thể vi phân ở bất kỳ đâu. Vào năm 1931, {\sc Stefan Banach} đã chứng minh rằng tập hợp các hàm có đạo hàm tại 1 điểm nào đó là 1 tập hợp ít ỏi trong không gian của tất cả các hàm liên tục. Nói 1 cách không chính thức, điều này có nghĩa là hầu như không có hàm liên tục ngẫu nhiên nào có đạo hàm tại 1 điểm.

%------------------------------------------------------------------------------%

\subsection{Notation for differentiation -- Ký hiệu cho phép lấy đạo hàm}
\textbf{\textsf{Resources -- Tài nguyên.}}
\begin{enumerate}
	\item \href{https://en.wikipedia.org/wiki/Notation_for_differentiation}{Wikipedia{\tt/}notation for differentiation}.
\end{enumerate}
1 common way of writing the derivative of a function is \href{https://en.wikipedia.org/wiki/Leibniz_notation}{Leibniz notation}, introduced by \href{https://en.wikipedia.org/wiki/Gottfried_Leibniz}{\sc Gottfried Wilhelm Leibniz} in 1675, which denotes a derivative as the quotient of 2 \href{https://en.wikipedia.org/wiki/Differential_(mathematics)}{differentials} e.g. $dx,dy$. It is still common used when the equation $y = f(x)$ is viewed as a functional relationship between \href{https://en.wikipedia.org/wiki/Dependent_and_independent_variables}{dependent \& independent variables}. The 1st derivative is denoted by $\frac{dy}{dx}$, read as ``the derivative of $y$ w.r.t. $x$''. This derivative can alternately be treated as the application of a \href{https://en.wikipedia.org/wiki/Differential_operator}{differential operator} to a function, $\frac{dy}{dx} = \frac{d}{dx}f(x)$. Higher derivatives are expressed using the notation $\frac{d^ny}{dx^n}$ for the $n$th derivative of $y = f(x)$. These are abbreviations for multiple applications of the 

%------------------------------------------------------------------------------%

\section{L'H\^ospital's rule -- Quy tắc l'H\^ospital}
\textbf{\textsf{Resources -- Tài nguyên.}}
\begin{enumerate}
	\item \href{https://en.wikipedia.org/wiki/L%27H%C3%B4pital%27s_rule}{Wikipedia{\tt/}L'H\^ospital rule}.
	\item \cite{Rudin1976}. {\sc Walter Rudin}. {\it Principles of Mathematical Analysis}. Sect: L'Hospital's rule, pp. 109--110.
	\item \cite{Tao_analysis_1}. {\sc Terence Tao}. {\it Analysis I}. Sect. 10.5: L'H\^ospital's Rule, pp. 228--229.
\end{enumerate}
{\it L'Hôpital's rule}, also known as {\it Bernoulli's rule}. is a mathematical theorem that allows evaluating limits of \href{https://en.wikipedia.org/wiki/Indeterminate_form}{indeterminate forms} using derivatives. Application (or repeated application) of the rule often converts an indeterminate form to an expression that can be easily evaluated by substitution. The rule is named after the 17th-century French mathematician \href{https://en.wikipedia.org/wiki/Guillaume_de_l%27H%C3%B4pital}{Guillaume de l'H\^opital}. Although the rule is often attributed to de l''H\^opital, the theorem was 1st introduced to him in 1694 by the Swiss mathematician \href{https://en.wikipedia.org/wiki/Johann_Bernoulli}{\sc Johann Bernoulli}.

The following theorem is frequently useful in the evaluation of limits. The differentiation of the numerator \& denominator often simplifies the quotient or converts it to a limit that can be directly evaluated by \href{https://en.wikipedia.org/wiki/Continuous_function}{continuity}.

\begin{theorem}[\cite{Rudin1976}, Thm. 5.13, p. 109, l'H\^ospital's rule]
	\label{thm: l'H\^ospital rule}
	suppose $f,g$ are real \& differentiable in $(a,b)$, \& $g'(x)\ne0$, $\forall x\in(a,b)$, where $-\infty\le a < b\le\infty$. Suppose $\frac{f'(x)}{g'(x)}\to l$ as $x\to a$. If $f(x)\to0$ \& $g(x)\to0$ as $x\to a$, or if $g(x)\to\infty$ as $x\to a$, then $\frac{f(x)}{g(x)}\to l$ as $x\to a$. The analogous statement is also true if $x\to b$, or if $g(x)\to-\infty$ instead of $g(x)\to\infty$.
\end{theorem}

\begin{theorem}[\cite{Tao_analysis_1}, Prop. 10.5.1, p. 228, L'H\^ospital's rule I]
	Let $X\subset\mathbb{R}$, let $f:X\to\mathbb{R},g:X\to\mathbb{R}$ be functions, \& let $x_0\in X$ be a limit point of $X$. Suppose that $f(x_0) = g(x_0) = 0$, that $f,g$ are both differentiable at $x_0$, but $g'(x_0)\ne0$. Then there exists a $\delta > 0$ such that $g(x)\ne0$, $\forall x\in(X\cap(x_0 - \delta,x_0 + \delta))\backslash\{x_0\}$, \&
	\begin{equation*}
		\lim_{x\to x_0,\ x\in(X\cap(x_0 - \delta,x_0 + \delta))\backslash\{x_0\}} \frac{f(x)}{g(x)} = \frac{f'(x_0)}{g'(x_0)}.
	\end{equation*}
\end{theorem}

\begin{baitoan}
	Cho hàm số $f(x) = \sin x,g(x) = -\frac{1}{2}x$. Chứng minh hàm số
	\begin{equation*}
		h(x)\coloneqq\left\{\begin{split}
			&\frac{f(x)}{g(x)}&&\mbox{if } x\ne0,\\
			&\frac{f'(0)}{g'(0)} = -2&&\mbox{if } x = 0.
		\end{split}\right.\in C(\mathbb{R}).
	\end{equation*}
\end{baitoan}

\begin{problem}
	Suppose $f,g$ are complex differentiable functions on $(0,1)$, $f(x)\to0,g(x)\to0,f'(x)\to A,g'(x)\to B$ as $x\to0$, where $A,B\in\mathbb{C}$, $B\ne0$. Prove that $\lim_{x\to0} \frac{f(x)}{g(x)} = \frac{A}{B}$.
\end{problem}
{\sf Hint.} $\frac{f(x)}{g(x)} = \left(\frac{f(x)}{x} - A\right)\frac{x}{g(x)} + A\frac{x}{g(x)}$ then apply l'H\^ospital rule, i.e., Thm. \ref{thm: l'H\^ospital rule} to the real- \& imaginary parts of $\frac{f(x)}{x},\frac{g(x)}{x}$.

%------------------------------------------------------------------------------%

\subsection{Problems: Derivative -- Bài tập: Đạo hàm}

\begin{baitoan}
	Tính đạo hàm bằng định nghĩa $\forall x_0\in\mathbb{C}$: (a) $(x)'|_{x=x_0}$. (b) $(x^2)'|_{x=x_0}$. (c) $(x^n)'|_{x=x_0}$, $\forall n\in\mathbb{N}$. (d) $(x^{-n})'|_{x=x_0}$, $\forall n\in\mathbb{N}$. (e) $(\sqrt{x})'|_{x=x_0}$. (f) $(\sqrt[3]{x})'|_{x=x_0}$. (g) $(\sqrt[n]{x})'|_{x=x_0}$, $\forall n\in\mathbb{N}$. (h) $(\sqrt[n]{x^m})|_{x=x_0}$, $\forall m,n\in\mathbb{N}$. (i) $(x^a)'|_{x=x_0}$, $\forall a\in\mathbb{R}$.
\end{baitoan}

\begin{baitoan}[Derivative of polynomials -- Đạo hàm của các đa thức]
	Tính đạo hàm của hàm số đa thức
	\begin{equation}
		\label{polynomial}
		\tag{P}
		P(x;n,{\bf a})\coloneqq\sum_{i=0}^n a_ix^i = a_nx^n + a_{n-1}x^{n-1} + \cdots + a_1x + a_0,
	\end{equation}
	tại $x = x_0$ bằng định nghĩa, với $\deg P(x;n,{\bf a}) = n\in\mathbb{N}$ \& vector chứa các hệ số của đa thức $P(x;n,{\bf a})$ là ${\bf a}\coloneqq(a_0,a_1,\ldots,a_n)\in\mathbb{R}^n\times\mathbb{R}^\star$.
\end{baitoan}

\begin{baitoan}[Derivative of rational function -- Đạo hàm của phân thức]
	Tính đạo hàm của hàm số phân thức
	\begin{equation}
		\label{rational function}
		\tag{Q}
		Q(x;m,n,{\bf a},{\bf b})\coloneqq\frac{\sum_{i=0}^m a_ix^i}{\sum_{i=0}^n b_ix^i} = \frac{a_mx^m + a_{m-1}x^{m-1} + \cdots + a_1x + a_0}{b_nx^n + b_{n-1}x^{n-1} + \cdots + b_1x + b_0},
	\end{equation}
	tại $x = x_0$ bằng định nghĩa.
\end{baitoan}

\begin{baitoan}[Đạo hàm của căn thức]
	Tính đạo hàm của hàm số căn thức $f(x) = \sqrt[n]{x} = x^{\frac{1}{n}}$, với $n\in\mathbb{N}^\star$, tại $x = x_0$ bằng định nghĩa.
\end{baitoan}
Ta có 3 dạng hàm số sơ cấp thường gặp: hàm đa thức $P(x;n,{\bf a})\coloneqq\sum_{i=0}^n a_ix^i$, hàm phân thức $Q(x;m,n,{\bf a},{\bf b})\coloneqq\dfrac{\sum_{i=0}^m a_ix^i}{\sum_{i=0}^n b_ix^i}$, hàm căn thức $R_n(x)\coloneqq\sqrt[n]{x}$.

\begin{baitoan}[\cite{TLCT_BT_dai_so_giai_tich_11}, 1., p. 49]
	Dùng định nghĩa, tính đạo hàm của hàm số tại điểm $x_0$: (a) $y = 2x + 1,x_0 = 2$. (b) $y = x^2 + 3x,x_0 = 1$. (c) $y = ax + b$ tại $x = x_0$. (d) $y = ax^2 + bx + c$ tại $x = x_0$.
\end{baitoan}

\begin{baitoan}[\cite{TLCT_BT_dai_so_giai_tich_11}, 2., p. 49]
	Cho parabol $y = x^2$ \& 2 điểm $A(2,4),B(2 + \Delta x,4 + \Delta y)$ trên parabol đó. (a) Tính hệ số góc của cát tuyến $AB$ biết $\Delta x\in\{1,0.1,0.01\}$. (b) Tính hệ số góc của tiếp tuyến của parabol đã cho tại điểm $A$. (c) Mở rộng cho parabol $y = ax^2 + bx + c$ \& 2 điểm $A(x_0,y_0),B(x_0 + \Delta x,y_0 + \Delta y)$.
\end{baitoan}

\begin{baitoan}[\cite{TLCT_BT_dai_so_giai_tich_11}, 3., p. 49]
	Viết phương trình tiếp tuyến của đồ thị hàm số $y = x^3$ biết: (a) Tiếp tuyến có hoành độ bằng $1$. (b) Tiếp điểm của tung độ bằng $8$. (c) Hệ số góc của tiếp tuyến bằng $3$.
\end{baitoan}

\begin{baitoan}[\cite{TLCT_BT_dai_so_giai_tich_11}, 4., p. 49]
	1 vật rơi tự do có phương trình chuyển động $S = \frac{gt^2}{2}$ với $g\approx9.8{\rm m{\tt/}s^2}$ \& $t$ (s). Tính: (a) Vận tốc trung bình trong khoảng thời gian từ $t$ đến $t + \Delta t$ với độ chính xác $0.001$, biết $t = 5$ \& $\Delta t\in\{0.1,0.001,0.001\}$. (b) Vận tốc tại thời điểm $t = 5$.
\end{baitoan}

\begin{baitoan}[\cite{TLCT_BT_dai_so_giai_tich_11}, 5., p. 49]
	Tính đạo hàm của hàm số $y = \sqrt[3]{x}$ trên $(0,\infty)$.
\end{baitoan}

\begin{baitoan}[\cite{TLCT_BT_dai_so_giai_tich_11}, 6., p. 49]
	Tính đạo hàm của hàm số $y = x|x|$ tại điểm $x_0 = 0$ (nếu có).
\end{baitoan}

\begin{baitoan}[\cite{TLCT_BT_dai_so_giai_tich_11}, 7., p. 49]
	Tính $f'(x)$ với
	\begin{equation}
		f(x) = \left\{\begin{split}
			&2x + 1&&\mbox{if } x < 1,\\
			&x^2 + 2&&\mbox{if } 1\le x\le2,\\
			&x^3 - x^2 - 8x + 10&&\mbox{if } x > 2.
		\end{split}\right.
	\end{equation}
\end{baitoan}

%------------------------------------------------------------------------------%

\section{Differentiation Rules -- Các Quy Tắc Tính Đạo Hàm}

\begin{baitoan}[\cite{TLCT_BT_dai_so_giai_tich_11}, 8., p. 50]
	Tính đạo hàm của hàm số: (a) $y = x^4 - 3x^3 + 5x^2 - 7x + 9$. (b) $y = (x - 1)^5(x + 1)^7$. (c) $y = \dfrac{x^2 + 1}{x^4 + 1}$. (d) $y = (x + 1)^3(x + 2)^4(x + 3)^5$.
\end{baitoan}

\begin{baitoan}[\cite{TLCT_BT_dai_so_giai_tich_11}, 9., p. 50]
	Tính đạo hàm của hàm số: (a) $y = \sqrt{\dfrac{1 - x}{1 + x}}$. (b) $y = \sin x^2 + x\cos x^2$. (c) $y = \ln(x + \sqrt{x^2 + 1})$. (d) $y = (x^3 + x^2 + x + 1)e^{x^2 + x}$.
\end{baitoan}

\begin{baitoan}[\cite{TLCT_BT_dai_so_giai_tich_11}, 10., p. 50]
	Tính đạo hàm của hàm số: (a) $y = \dfrac{\sin x - \cos x}{\sin x + \cos x}$. (b) $y = \dfrac{\sin x - 1}{\sin x + \cos x}$.
\end{baitoan}

\begin{baitoan}[\cite{TLCT_BT_dai_so_giai_tich_11}, 11., p. 50]
	Viết phương trình tiếp tuyến của đồ thị hàm số: (a) $y = \dfrac{x}{x^2 + 1}$ biết hoành độ tiếp điểm là $x_0 = \frac{1}{2}$. (b) $y = \sqrt{x + 2}$ biết tung độ tiếp điểm là $y_0 = 2$.
\end{baitoan}

\begin{baitoan}[\cite{TLCT_BT_dai_so_giai_tich_11}, 12., p. 50]
	(a) Chứng minh hàm số $y = f(x) = \sin^6x + \cos^6x + 3\sin^2x\cos^2x$ có đạo hàm bằng $0$. (b) Mở rộng bài toán.
\end{baitoan}

\begin{proof}
    (a) $y = f(x) = \sin^6x + \cos^6x + 3\sin^2x\cos^2x = \sin^6x + \cos^6x + 3\sin^2x\cos^2x(\sin^2x + \cos^2x) = (\sin^2x + \cos^2x)^3 = 1\Rightarrow f'(x) = 0$, $\forall x\in\mathbb{R}$. (b) Xét hàm $f:D_f\subset\mathbb{R}\to\mathbb{R}$ bất kỳ sao cho $1\in D_f$ \& $f$ khả vi tại $x = 1$. Đặt $g(x)\coloneqq\sin^2x + \cos^2x\equiv1$, thì $g'(x) = 1' = 0$ (hoặc tính cụ thể $g'(x) = 2\sin x\cos x - 2\cos x\sin x = 0$), $\forall x\in\mathbb{R}$. Khi đó đạo hàm của hàm hợp $f\circ g$ sẽ bằng $0$ vì $(f\circ g)'(x) = (f(1))' = 0$, hoặc theo quy tắc xích $(f\circ g)'(x) = f'(g(x))g'(x) = f'(1)\cdot0 = 0$.
\end{proof}

\begin{remark}
    Mấu chốt của bài toán chỉ là dùng đẳng thức lượng giác cơ bản $\sin^2x + \cos^2x = 1$, $\forall x\in\mathbb{R}$, để đưa về đạo hàm của hàm hằng bằng $0$. Tương tự nếu ta có 1 đẳng thức đại số hoặc 1 đẳng thức lượng giác nói riêng, hoặc 1 đẳng thức toán học có dạng $g(x) = a\in\mathbb{R}$. Xét hàm $f:D_f\subset\mathbb{R}\to\mathbb{R}$ bất kỳ sao cho $a\in D_f$ \& $f$ khả vi tại $x = a$. Khi đó đạo hàm của hàm hợp $f\circ g$ sẽ bằng $0$ vì $(f\circ g)'(x) = (f(a))' = 0$, hoặc theo quy tắc xích $(f\circ g)'(x) = f'(g(x))g'(x) = f'(a)(a)' = f'(a)\cdot0 = 0$, $\forall x\in\mathbb{R}$.
\end{remark}

\begin{baitoan}[\cite{TLCT_BT_dai_so_giai_tich_11}, 13., p. 50]
	Viết phương trình tiếp tuyến của parabol $y = x^2$ biết tiếp tuyến đó đi qua điểm $A(0,-1)$.
\end{baitoan}

\begin{baitoan}[\cite{TLCT_BT_dai_so_giai_tich_11}, 14., p. 50]
	1 viên đạn được bắn lên từ mặt đất theo phương thẳng đứng với tốc độ ban đầu $v_0 = 196$ {\rm m{\tt/}s} (bỏ qua sức cản của không khí). Tìm thời điểm tại đó tốc độ của viên đạn bằng $0$. Khi đó viên đạn cách mặt đất bao nhiêu {\rm m}?
\end{baitoan}

%------------------------------------------------------------------------------%

\section{Numerical Differentiation -- Xấp Xỉ Đạo Hàm}
\textbf{\textsf{Resources -- Tài nguyên.}}
\begin{enumerate}
	\item \href{https://en.wikipedia.org/wiki/Numerical_differentiation}{Wikipedia{\tt/}numerical differentiation}.
    
	\item \cite{Scheid1989}. {\sc Francis Scheid}. {\it Schaum's Outline of Numerical Analysis}. Chap. 13: Numerical Differentiation.
    
	\item \cite{LeVeque2007}. {\sc Randall J. LeVeque}. {\it Finite Difference Methods for Ordinary \& Partial Differential Equations: Steady-State \& Time-Dependent Problems}.
\end{enumerate}
In \href{https://en.wikipedia.org/wiki/Numerical_analysis}{numerical analysis}, {\it numerical differentiation} algorithms estimate the derivative of a mathematical function or \href{https://en.wikipedia.org/wiki/Function_(computer_programming)}{subroutine} using values of the function \& perhaps other knowledge about the function.

-- Trong phân tích số, thuật toán phân biệt số ước tính đạo hàm của 1 hàm toán học hoặc chương trình con bằng cách sử dụng các giá trị của hàm \& có thể là các kiến thức khác về hàm.

%------------------------------------------------------------------------------%

\subsection{Approximate 1st-order derivatives -- Xấp xỉ đạo hàm bậc nhất}
Approximate derivatives of a function $y = f(x)$ may be found from a polynomial approximation $P(x)$ simply by accepting $P',P'',P^{(3)},\ldots$ in place of $y' = f'(x),y'' = f''(x),y^{(3)} = f^{(3)}(x),\ldots$. Our collocation polynomials lead to a broad variety of useful formulas of this sort. The 3 well-known formulas:
\begin{enumerate}
	\item Newton forward differentiation:
	\begin{equation*}
		y'|_{x = x_0} = f'(x_0)\approx\frac{f(x_0 + h) - f(x_0)}{h}\mbox{ for all } h > 0\mbox{ small enough}.
	\end{equation*}
	\item Newton backward differentiation:
	\begin{equation*}
		y'|_{x = x_0} = f'(x_0)\approx\frac{f(x_0) - f(x_0 - h)}{h}\mbox{ for all } h > 0\mbox{ small enough}.
	\end{equation*}
	\item Stirling differentiation:
	\begin{equation*}
		y'|_{x = x_0} = f'(x_0)\approx\frac{f(x_0 + h) - f(x_0 - h)}{2h}\mbox{ for all } h > 0\mbox{ small enough}.
	\end{equation*}
\end{enumerate}
More complicated formulas are available simply by using more terms. Thus [fill in more details]
\begin{equation*}
	f'(x)\approx\frac{1}{h}\left[\Delta y_0 + \left(k - \frac{1}{2}\right)\Delta^2y_0 + \frac{3k^2 - 6k + 2}{6}\Delta^3y_0 + \cdots\right]
\end{equation*}
comes from the Newton formula, while
\begin{equation*}
	f'(x)\approx\frac{1}{h}\left(\delta\mu y_0 + k\delta^2y_0 + \frac{3k^2 - 1}{6}\delta^3\mu y_0 + \cdots\right)
\end{equation*}
results from differentiating Stirling's. Other collocation formulas produce similar approximations.

\begin{baitoan}[Error estimate in numerical differentiation -- Đánh giá sai số trong xấp xỉ đạo hàm]
	Giả sử $f\in C^1(\mathbb{R})$. Dùng khai triển Taylor tìm sai số của quy tắc xấp xỉ đạo hàm: (a) Newton forward $f'(x)\approx\dfrac{f(x + h) - f(x)}{h}$. (b) Newton backward $f'(x)\approx\dfrac{f(x) - f(x - h)}{h}$. (c) Stirling $f'(x)\approx\dfrac{f(x + h) - f(x - h)}{2h}$.
\end{baitoan}

\begin{baitoan}[Numerical differentiation of polynomials -- Xấp xỉ đạo hàm của đa thức]
	Viết thuật toán \& chương trình {\sf C{\tt/}C++, Pascal, Python} để xấp xỉ đạo hàm $P'(x_0)$ của đa thức $P(x) = \sum_{i=0}^n a_ix^i$.
	\item {\sf Input.} Dòng 1 của file input chứa $n\in\mathbb{N}$ là bậc của đa thức $P$, i.e., $n = \deg P$, $x_0\in\mathbb{R}$, $h\in(0,\infty)$. Dòng 2 chứa $n + 1$ hệ số thực theo thứ tự $a_n,a_{n-1},\ldots,a_1,a_0\in\mathbb{R}$.
	\item {\sf Output.} In ra 3 giá trị xấp xỉ của $P'(x_0)$ lần lượt bằng 3 phương pháp Newton forward, Newton backward, \& Stirling.
	\item {\sf Sample.}
	\begin{table}[H]
		\centering
		\begin{tabular}{|l|l|}
			\hline
			\verb|numerical_differentiation.inp| & \verb|numerical_differentiation.out| \\
			\hline
			3 1.4142135623730951 0.013 & $-0.1767932308$ \\
			$-1 3$ $-2.6457513110645907$ 3.141592653589793 & $-0.1444846154$ \\
			& $-0.1606389231$ \\
			\hline
		\end{tabular}
	\end{table}
\end{baitoan}

\begin{proof}[Solution]
    Đa thức $P(x) = \sum_{i=0}^n a_ix^i$ có đạo hàm $P'(x) = \sum_{i=1}^n ia_ix^{i - 1}$ nên $P'(x_0) = \sum_{i=1}^n ia_ix_0^{i - 1}$, $\forall x_0\in\mathbb{R}$.
    
    C++:
    \begin{enumerate}
        \item VNTA's C++: numerical differentiation:
        \begin{Verbatim}[numbers=left,xleftmargin=5mm]
#include <bits/stdc++.h>
using namespace std;

// P(x) = an.x^n + ... + a1.x + a0
double Px(vector<double>a, double x) {
    double res = 0;
    for (int i = 0; i < a.size(); i++) {
        res += a[i] * pow(x, a.size() - 1 - i);
    }
    return res;
}

double forward(vector<double>a, double x, double h) {
    double res;
    res = (double)(Px(a, x + h) - Px(a, x)) / (h * 1.0);
    return res;
}

double stirling(vector<double>a, double x, double h) {
    double res;
    res = (double)(Px(a, x + h) - Px(a, x - h)) / (h * 2.0);
    return res;
}

double backward(vector<double>a, double x, double h) {
    double res;
    res = (double)(Px(a, x) - Px(a, x - h)) / (h * 1.0);
    return res;
}

int main() {
    ios_base::sync_with_stdio(false);
    cin.tie(0); cout.tie(0);
    int n;
    double x, h;
    cin >> n >> x >> h;
    vector<double>a(n + 1);
    for (int i = 0; i <= n; i++) cin >> a[i];
    cout << "Newton forward: " << fixed << setprecision(10) << forward(a, x, h) << "\n";
    cout << "Stirling: " << fixed << setprecision(10) << stirling(a, x, h) << "\n";
    cout << "Newton backward: " << fixed << setprecision(10) << backward(a, x, h) << "\n";
}
        \end{Verbatim}
        \item DXH's C++: numerical differentiation:
        \begin{Verbatim}[numbers=left,xleftmargin=5mm]
#include <iostream>
#include <cmath>
#include <iomanip>
#include <functional>
using namespace std;

class DerivativeApproximator {
    private:
    double m_x; // Tên thuộc tính nên có tiền tố để dễ phân biệt, ví dụ m_ (member)
    double m_h;
    // Hàm & đạo hàm của hàm đó
    std::function<double(double)> m_func;
    std::function<double(double)> m_func_prime;
    
    public:
    // Constructor nhận các hàm & giá trị x, h
    DerivativeApproximator(std::function<double(double)> func,
    std::function<double(double)> func_prime,
    double x_val, double h_val)
    : m_func(func), m_func_prime(func_prime), m_x(x_val), m_h(h_val) {}
    // Phương thức tính hàm
    double caculateNewtonForward() {
        return (m_func(m_x + m_h) - m_func(m_x)) / m_h;
    }
    double caculateNewtonBackward() {
        return (m_func(m_x) - m_func(m_x - m_h)) / m_h;
    }
    double caculateStirling(double x, double h, std::function<double(double)> func) {
        return (func(x + h) - func(x - h)) / (2 * h);
    }
};
int main() {
    // Định nghĩa hàm & đạo hàm của nó
    auto func = [](double x) { return x * x; }; // Hàm f(x) = x^2
    auto func_prime = [](double x) { return 2 * x; }; // Đạo hàm f'(x) = 2x
    
    double x_val = 1.0; // Giá trị x tại đó ta muốn tính đạo hàm
    double h_val = 0.01; // H
    
    DerivativeApproximator approximator(func, func_prime, x_val, h_val);
    
    cout << fixed << setprecision(4);
    cout  << approximator.caculateNewtonForward() << endl;
    cout  << approximator.caculateNewtonBackward() << endl;
    cout  << approximator.caculateStirling(x_val, h_val, func) << endl;
    
    return 0;
}  
        \end{Verbatim}
        \item DPAK's C++: numerical differentiation.
        \begin{Verbatim}[numbers=left,xleftmargin=5mm]
#include <bits/stdc++.h>
using namespace std;
int n;
double h;

double forward(double xo, double x, vector<double> &a) {
    double ans =  0;
    for (int i = 0; i <= n; i++) {
        ans = ans + a[i] * pow(xo, i);
    }
    cout << ans << endl;
    for (int i = 0; i <= n; i++) {
        ans = ans - a[i] * pow(x, i);
    }
    cout << ans << endl;
    ans = ans / h;
    cout << ans << endl;
    return ans;
}

double backward(double xo, double x, vector<double> &a) {
    double ans =  0;
    for (int i = 0; i <= n; i++) {
        ans += a[i] * pow(xo, i);
    }
    for (int i = 0; i <= n; i++) {
        ans -= a[i] * pow(x, i);
    }
    ans = ans / h;
    return ans;
}

double stirling(double xo, double x, vector<double> &a) {
    double ans =  0;
    for (int i = 0; i <= n; i++) {
        ans += a[i] * pow(xo, i);
    }
    for (int i = 0; i <= n; i++) {
        ans -= a[i] * pow(x, i);
    }
    ans = ans / (2 * h);
    return ans;
}

int main() {
    vector<double>a;
    a.resize(n + 1);
    double xo;
    cin >> n >> xo >> h;
    for (int i = n; i >= 0; i--) {
        cin >> a[i];
    }
    double FW = forward(xo + h, xo, a);
    double BW = backward(xo, xo - h, a);
    double ST = stirling(xo + h, xo - h, a);
    cout << "Newton Forward = " << FW << endl;
    cout << "Newton Backward = " << BW << endl;
    cout << "Stirling = " << ST << endl;
} 
        \end{Verbatim}
        \item NHT's C++ numerical differentiation:
\begin{Verbatim}[numbers=left,xleftmargin=5mm]
#include <bits/stdc++.h>
#define ll long long
#define ld long double
using namespace std;

const int mod = 1e9 + 7;
int n;
ld x, h, res;
ld a[1000005];

/*ll Power(ll a, ll b){
    ll ans(1);
    for(; b; b >>= 1){
        if(b & 1) ans = (ans * a) % mod;
        a = (a * a) % mod;
    }
    return ans;
}*/

ld Qx(int x) {
    ld ans(0);
    for (int i = 0; i <= n; ++i)
    ans += (i * a[i] * pow(x, i));
    return ans;
}

ld forward(ld x, ld h) {
    ld ans(0), fx(0);
    for (int i = 0; i <= n; ++i)
    ans += (a[i] * pow(x + h, i));
    for (int i = 0; i <= n; ++i)
    fx += (a[i] * pow(x, i));
    ans -= fx;
    return ans / h;
}

ld backward(ld x, ld h) {
    ld ans(0), fx(0);
    for (int i = 0; i <= n; ++i)
    ans += (a[i] * pow(x - h, i));
    for (int i = 0; i <= n; ++i)
    fx += (a[i] * pow(x, i));
    ans = fx - ans;
    return ans / h;
}

ld stirling(ld x, ld h) {
    ld ans1(0), ans2(0);
    for (int i = 0; i <= n; ++i)
    ans1 += (a[i] * pow(x + h, i));
    
    for (int i = 0; i <= n; ++i)
    ans2 += (a[i] * pow(x - h, i));
    
    ld ans = ans1 - ans2;
    return ans / (2 * h);
}

int main() {
    ios::sync_with_stdio(false);
    cin.tie(nullptr);
    
    cin >> n >> x >> h;
    for (int i = n; i >= 0; i--) cin >> a[i];
    
    res = Qx(x);
    //cout << "P'(xo) = " << res << endl;
    ld FORWARD = forward(x, h), BACKWARD = backward(x, h), STIRLING = stirling(x, h);
    cout << "Forward: " << FORWARD << endl;
    cout << "Backward: " << BACKWARD << endl;
    cout << "Stirling: " << STIRLING << endl;
    
    return 0;
}
\end{Verbatim}
    \end{enumerate}
\end{proof}

\begin{baitoan}[Numerical differentiation of polynomial-like function -- Xấp xỉ đạo hàm của hàm tựa-đa thức]
	Viết thuật toán \& chương trình {\sf C{\tt/}C++, Pascal, Python} để xấp xỉ đạo hàm $f'(x_0)$ của hàm số tựa-đa thức $f(x) = \sum_{i=0}^n a_ix^{\alpha_i}$.
	\item {\sf Input.} Dòng 1 của file input chứa $n\in\mathbb{N}$ là bậc của đa thức $P$, i.e., $n = \deg P$, $x_0\in\mathbb{R}$, $h\in(0,\infty)$. Dòng 2 chứa $n + 1$ hệ số thực theo thứ tự $a_n,a_{n-1},\ldots,a_1,a_0\in\mathbb{R}$. Dòng 3 chứa $n + 1$ số mũ thực theo thứ tự $\alpha_n,\alpha_{n-1},\ldots,\alpha_1,\alpha_0\in\mathbb{R}$.
	\item {\sf Output.} In ra 3 giá trị xấp xỉ của $f'(x_0)$ lần lượt bằng 3 phương pháp Newton forward, Newton backward, \& Stirling.
	\item {\sf Sample.}
	\begin{table}[H]
		\centering
		\begin{tabular}{|l|l|}
			\hline
			\verb|numerical_differentiation_1.inp| & \verb|numerical_differentiation_1.out| \\
			\hline
			3 1.4142135623730951 0.013 & -25.80051177 \\
			-1 3 -2.6457513110645907 3.141592653589793 & -25.40984308 \\
			3 -2 3.141592653589793 -22.16716829679195 & -25.60517742 \\
			\hline
		\end{tabular}
	\end{table}
\end{baitoan}

\begin{baitoan}[Numerical differentiation of rational functions -- Xấp xỉ đạo hàm của phân thức]
	Viết thuật toán \& chương trình {\sf C{\tt/}C++, Pascal, Python} để xấp xỉ đạo hàm $f'(x_0)$ của hàm số phân thức
	\begin{equation*}
		f(x) = \frac{P(x)}{Q(x)} = \frac{\sum_{i=0}^m a_ix^i}{\sum_{i=0}^n b_ix^i},\ \forall x\in\mathbb{R}\backslash{\rm ker}\,Q.
	\end{equation*}
	\item {\sf Input.} Dòng 1 của file input chứa $m,n\in\mathbb{N}$ lần lượt là bậc của 2 đa thức $P,Q$, i.e., $m = \deg P,n = \deg Q$, $x_0\in\mathbb{R}$, $h\in(0,\infty)$. Dòng 2 chứa $m + 1$ hệ số thực theo thứ tự $a_m,a_{m-1},\ldots,a_1,a_0\in\mathbb{R}$. Dòng 3 chứa $n + 1$ hệ số thực theo thứ tự $b_n,b_{n-1},\ldots,b_1,b_0\in\mathbb{R}$.
	\item {\sf Output.} In ra 3 giá trị xấp xỉ của $f'(x_0)$ lần lượt bằng 3 phương pháp Newton forward, Newton backward, \& Stirling.
	\item {\sf Sample.}
	\begin{table}[H]
		\centering
		\begin{tabular}{|l|l|}
			\hline
			\verb|numerical_differentiation_2.inp| & \verb|numerical_differentiation_2.out| \\
			\hline
			& \\
			\hline
		\end{tabular}
	\end{table}
\end{baitoan}

\begin{baitoan}[Numerical differentiation of rational-like functions -- Xấp xỉ đạo hàm của hàm tựa-phân thức]
	Viết thuật toán \& chương trình {\sf C{\tt/}C++, Pascal, Python} để xấp xỉ đạo hàm $f'(x_0)$ của hàm số tựa-phân thức 
	\begin{equation*}
		f(x) = \frac{f_{\rm num}(x)}{f_{\rm den}(x)} = \frac{\sum_{i=0}^m a_ix^{\alpha_i}}{\sum_{i=0}^n b_ix^{\beta_i}},\ \forall x\in\mathbb{R}\backslash{\rm ker}\,f_{\rm den}.
	\end{equation*}
	\item {\sf Input.} Dòng 1 của file input chứa $m,n\in\mathbb{N}$, $x_0\in\mathbb{R}$, $h\in(0,\infty)$. Dòng 2 chứa $m + 1$ hệ số thực theo thứ tự $a_m,a_{m-1},\ldots,a_1,a_0\in\mathbb{R}$. Dòng 3 chứa $n + 1$ số mũ thực theo thứ tự $\alpha_n,\alpha_{n-1},\ldots,\alpha_1,\alpha_0\in\mathbb{R}$. Dòng 4 chứa $n + 1$ hệ số thực theo thứ tự $b_n,b_{n-1},\ldots,b_1,b_0\in\mathbb{R}$. Dòng 5 chứa $n + 1$ số mũ thực theo thứ tự $\beta_n,\beta_{n-1},\ldots,\beta_1,\beta_0\in\mathbb{R}$.
	\item {\sf Output.} In ra 3 giá trị xấp xỉ của $f'(x_0)$ lần lượt bằng 3 phương pháp Newton forward, Newton backward, \& Stirling.
	\item {\sf Sample.}
	\begin{table}[H]
		\centering
		\begin{tabular}{|l|l|}
			\hline
			\verb|numerical_differentiation_3.inp| & \verb|numerical_differentiation_3.out| \\
			\hline
			& \\
			\hline
		\end{tabular}
	\end{table}
\end{baitoan}

%------------------------------------------------------------------------------%

\subsection{Approximate 2nd-order derivatives -- Xấp xỉ đạo hàm bậc 2}
For 2nd derivatives 1 popular result is
\begin{equation*}
	f''(x)\approx\frac{1}{h}\left(\delta^2y_0 + k\delta^3\mu y_0 + \frac{6k^2 - 1}{12}\delta^4y_0 + \cdots\right)
\end{equation*}
\& comes from the Stirling formula. Retaining only the 1st term, we have the familiar
\begin{equation*}
	y''|_{x = x_0} = f''(x_0)\approx\frac{f(x + h) - 2f(x) + f(x - h)}{h^2}\mbox{ for all } h > 0\mbox{ small enough}.
\end{equation*}

%------------------------------------------------------------------------------%

\subsection{Approximate higher order derivatives -- Xấp xỉ đạo hàm bậc cao}

%------------------------------------------------------------------------------%

\subsection{Sources of error in approximate differentiation -- Nguồn lỗi trong xấp xỉ đạo hàm}
The study of test cases suggests that approximate derivatives obtained from collocation polynomials be viewed with skepticism unless very accurate data are available. Even then the accuracy diminishes with increasing order of the derivatives.

-- Nghiên cứu các trường hợp thử nghiệm cho thấy rằng các đạo hàm gần đúng thu được từ các đa thức sắp xếp được xem xét với sự hoài nghi trừ khi có dữ liệu rất chính xác. Ngay cả khi đó, độ chính xác giảm dần theo thứ tự tăng dần của các đạo hàm.

The basic difficulty is that $y(x) - P(x)$ may be very small while $y'(x) - P'(x)$ is very large. In geometrical language, 2 curves may be close together but still have very different slopes. All the other familiar sources of error are also present, including input errors in the $y_i$ values, truncation errors e.g. $y' - P',y'' - P''$, etc., \& internal roundoffs.

-- Khó khăn cơ bản là $y(x) - P(x)$ có thể rất nhỏ trong khi $y'(x) - P'(x)$ lại rất lớn. Trong ngôn ngữ hình học, 2 đường cong có thể gần nhau nhưng vẫn có độ dốc rất khác nhau. Tất cả các nguồn lỗi quen thuộc khác cũng có mặt, bao gồm lỗi đầu vào trong các giá trị $y_i$, lỗi cắt cụt ví dụ $y' - P',y'' - P''$, v.v., \& làm tròn nội bộ.

%------------------------------------------------------------------------------%

\subsection{Problems: Numerical approximation -- Bài tập: Xấp xỉ đạo hàm}


%------------------------------------------------------------------------------%

\chapter{Các định lý giá trị trung bình}
\minitoc

\begin{baitoan}[\cite{TLCT_BT_dai_so_giai_tich_11}, 15., p. 50]
	Cho $a,b,c\in\mathbb{R},2a + 3b + 6c = 0$. Chứng minh phương trình $ax^2 + bx + c = 0$ có ít nhất 1 nghiệm thuộc $(0,1)$.
\end{baitoan}

\begin{baitoan}[\cite{TLCT_BT_dai_so_giai_tich_11}, 16., p. 50]
	Cho $f(x) = x(x - 1)(x - 2)(x - 3)(x - 4)(x - 5)(x - 6)$. Đếm số nghiệm của phương trình $f'(x) = 0$.
\end{baitoan}

\begin{baitoan}[\cite{TLCT_BT_dai_so_giai_tich_11}, 17., p. 51]
	Xét hàm số $f(x)$ liên tục trên đoạn $[a,b]$ có đạo hàm trên $(a,b)$. Giả sử phương trình $f(x) = 0$ có đúng 2 nghiệm $x_1,x_2$ với $x_1\ne x_2$. Chứng minh phương trình $f'(x) = 0$ có nghiệm, hơn nữa biểu thức $f'(x)$ phải đổi dấu.
\end{baitoan}

\begin{baitoan}[\cite{TLCT_BT_dai_so_giai_tich_11}, 18., p. 51]
	Chứng minh $2(\sqrt{n + 1} - \sqrt{n}) < \frac{1}{\sqrt{n}} < 2(\sqrt{n} - \sqrt{n - 1})$, $\forall n\in\mathbb{N}^\star$.
\end{baitoan}

\begin{baitoan}[\cite{TLCT_BT_dai_so_giai_tich_11}, 19., p. 51]
	Cho $0 < a < b$ \& $f$ là 1 hàm liên tục trên $[a,b]$, có đạo hàm trên $(a,b)$. Chứng minh tồn tại $c\in(a,b)$ thỏa $\dfrac{af(b) - bf(a)}{a - b} = f(c) - f'(c)$.
\end{baitoan}

\begin{baitoan}[\cite{TLCT_BT_dai_so_giai_tich_11}, 20., p. 51]
	Tính giới hạn: (a) $\lim_{x\to0} \dfrac{\tan x - \sin x}{x^3}$. (b) $\lim_{x\to0} \dfrac{\sqrt[m]{1 + x} - 1}{\sqrt[n]{1 + x} - 1}$. (c) $\lim_{x\to0} \dfrac{1 - \cos x}{x\sin x}$.
\end{baitoan}

\begin{baitoan}[\cite{TLCT_BT_dai_so_giai_tich_11}, 21., p. 51]
	Tính giới hạn: (a) $\lim_{x\to1} \left(\dfrac{1}{x - 1} - \dfrac{1}{\ln x}\right)$. (b) $\lim_{x\to0} (1 + x)^{\cot x}$.
\end{baitoan}

%------------------------------------------------------------------------------%

\chapter{2nd-Order Derivative -- Đạo Hàm Cấp 2}
\minitoc

%------------------------------------------------------------------------------%

\chapter{Vi Phân \& Đạo Hàm Cấp Cao}
\minitoc

\begin{baitoan}[\cite{TLCT_BT_dai_so_giai_tich_11}, 22., p. 51]
	Tính vi phân của hàm số: (a) $y = \sqrt{x^2 + a^2}$. (b) $y = x\sin x$. (c) $y = x^2 + \sin^2x$. (d) $y = e^x\ln x$.
\end{baitoan}

\begin{baitoan}[\cite{TLCT_BT_dai_so_giai_tich_11}, 23., p. 51]
	Làm tròn đến hàng phần nghìn: (a) $\dfrac{1}{0.9995}$. (b) $\ln1.001$. (c) $\cos61^\circ$.
\end{baitoan}

\begin{baitoan}[\cite{TLCT_BT_dai_so_giai_tich_11}, 24., p. 51]
	Chứng minh nếu $f,g$ là 2 hàm số có đạo hàm đến cấp 2 thì $fg$ cũng có đạo hàm đến cấp 2 \& có công thức $(f(x)g(x))'' = f''(x)g(x) + 2f'(x)g'(x) + g''(x)$.
\end{baitoan}

\begin{baitoan}[\cite{TLCT_BT_dai_so_giai_tich_11}, 25., p. 51]
	Tính đạo hàm: (a) $f(x) = x^4 - \cos2x$, tính $f^{(4)}(x)$. (b) $f(x) = \cos^2x$, tính $f^{(5)}(x)$. (c) $f(x) = (x + 10)^6$, tính $f^{(n)}(x)$.
\end{baitoan}

\begin{baitoan}[\cite{TLCT_BT_dai_so_giai_tich_11}, 26., p. 52]
	Vận tốc của 1 chất điểm chuyển động được biểu thị bởi công thức $v(t) = 8t + 3t^2$, với $t > 0$, $t$ được tính bằng giây {\rm s} \& $v(t)$ tính bằng {\rm m{\tt/}s}. Tính gia tốc của chất điểm: (a) Lúc $t = 4$. (b) Lúc vận tốc chuyển động bằng $11$.
\end{baitoan}

\begin{baitoan}[\cite{TLCT_BT_dai_so_giai_tich_11}, 27., p. 52]
	Chứng minh $\forall n\ge1$: (a) Nếu $f(x) = \frac{1}{x}$ thì $f^{(n)}(x) = \dfrac{(-1)^nn!}{x^{n+1}}$. (b) Nếu $f(x) = \cos x$ thì $f^{(n)}(x) = \cos\left(x + \frac{n\pi}{2}\right)$.
\end{baitoan}

\begin{baitoan}[\cite{TLCT_BT_dai_so_giai_tich_11}, 28., p. 52]
	Cho $f(x) = \sqrt{x}$. Tính $f^{(n)}(x)$.
\end{baitoan}

%------------------------------------------------------------------------------%

\section{Miscellaneous}

\begin{baitoan}[\cite{TLCT_BT_dai_so_giai_tich_11}, 29., p. 52]
	Tính $f'(x)$ với
	\begin{equation}
		f(x) = \left\{\begin{split}
			&2x + 1&&\mbox{if } x < 1,\\
			&x^2 + 1&&\mbox{if } 1\le x\le2,\\
			&x^3 - x^2 - 4x + 10&&\mbox{if } x > 2.
		\end{split}\right.
	\end{equation}
\end{baitoan}

\begin{baitoan}[\cite{TLCT_BT_dai_so_giai_tich_11}, 30., p. 52]
	Tính $f'(x) + f(x) + 2$ nếu $f(x) = x\sin2x$.
\end{baitoan}

\begin{baitoan}[\cite{TLCT_BT_dai_so_giai_tich_11}, 31., p. 52]
	Chứng minh nếu $f(x) = 3e^{x^2}$ thì $f'(x) - 2xf(x) + \frac{1}{3}f(0) - f'(0) = 1$.
\end{baitoan}

\begin{baitoan}[\cite{TLCT_BT_dai_so_giai_tich_11}, 32., p. 52]
	Viết phương trình tiếp tuyến của đường cong $y = 4x - x^2$ tại các điểm mà đường cong cắt trục hoành.
\end{baitoan}

\begin{baitoan}[\cite{TLCT_BT_dai_so_giai_tich_11}, 33., p. 52]
	Cho đa thức bậc 4 $P(x)$ thỏa mãn điều kiện $P(x)\ge0$, $\forall x\in\mathbb{R}$. Chứng minh $P(x) + P'(x) + P''(x) + P^{(3)}(x) + P^{(4)}(x)\ge0$, $\forall x\in\mathbb{R}$.
\end{baitoan}

\begin{baitoan}[\cite{TLCT_BT_dai_so_giai_tich_11}, 34., p. 53]
	Áp dụng định lý Rolle cho hàm số $f(x) = e^xP(x)$ để chứng minh nếu đa thức $P(x)$ bậc $n$ có $n$ nghiệm thực phân biệt thì đa thức $P(x) + P'(x)$ cũng có $n$ nghiệm thực phân biệt.
\end{baitoan}

\begin{baitoan}[\cite{TLCT_BT_dai_so_giai_tich_11}, 35., p. 53]
	Cho hàm số $f(x)$ khả vi trên đoạn $[0,1]$ \& $f'(0)f'(1) < 0$. Chứng minh tồn tại $c\in(0,1)$ thỏa $f'(c) = 0$.
\end{baitoan}

\begin{baitoan}[\cite{TLCT_BT_dai_so_giai_tich_11}, 36., p. 53]
	Giả sử $f(x)$ là 1 hàm số lẻ \& khả vi trên $\mathbb{R}$. Chứng minh $f'(x)$ là 1 hàm số chẵn.
\end{baitoan}

\begin{baitoan}[\cite{TLCT_BT_dai_so_giai_tich_11}, 37., p. 53]
	Tính đạo hàm cấp $100$ của hàm số $f(x) = \dfrac{x}{x^2 - 1}$.
\end{baitoan}

\begin{baitoan}[\cite{TLCT_BT_dai_so_giai_tich_11}, 38., p. 53]
	Tính giới hạn: (a) $\lim_{x\to0} \cos^{\frac{1}{2x^2}} x$. (b) $\lim_{x\to0} \cos^{\frac{5}{x}} 3x$.
\end{baitoan}

\begin{baitoan}[\cite{TLCT_BT_dai_so_giai_tich_11}, 39., p. 53]
	Chứng minh: (a) {\rm (Phương trình dao động điều hòa)} Nếu $y = A\sin(\omega t + \varphi) + B\cos(\omega t + \varphi)$ với $A,B,\omega,\varphi$ là 4 hằng số thì $y'' + \omega^2y = 0$. (b) Nếu $y = \sqrt{2x - x^2}$ thì $y^3y'' + 1 = 0$.
\end{baitoan}

\begin{baitoan}[\cite{TLCT_BT_dai_so_giai_tich_11}, 40., p. 53, công thức Newton--Leibnitz]
	Cho $f,g$ là 2 hàm số có đạo hàm đến cấp $n$, chứng minh công thức: $(f(x)g(x))^{(n)} = \sum_{k=0}^n C_n^kf^{(k)}(x)g^{(n-k)}(x)$.
\end{baitoan}

\begin{baitoan}[\cite{TLCT_BT_dai_so_giai_tich_11}, 41., p. 53]
	Cho hàm số $f(x) = \dfrac{x}{x^2 + 1}$. Tính $f^{(100)}(0),f^{(101)}(0)$.
\end{baitoan}

%------------------------------------------------------------------------------%

\begin{baitoan}[\cite{VMS_VMC2023}, p. 36, 4.1, VNUHCM UIT]
	Cho hàm $f\in C^2(\mathbb{R})$ thỏa $f(0) = 2$, $f'(0) = -2$, $f(1) = 1$. Chứng minh tồn tại $c\in(0,1)$ thỏa $f(c)f'(c) + f''(c) = 0$.
\end{baitoan}

\begin{baitoan}[\cite{VMS_VMC2023}, p. 37, 4.2, ĐH Đồng Tháp]
	Cho $f$ khả vi trên $(a,\infty)$, $\forall a\in(0,\infty)$ \& $\lim_{x\to\infty} f'(x) = 0$. Chứng minh $\lim_{x\to\infty} \dfrac{f(x)}{x} = 0$.
\end{baitoan}

\begin{baitoan}[\cite{VMS_VMC2023}, p. 37, 4.3, ĐH Đồng Tháp]
	Cho $f$ là hàm số có đạo hàm $f'$ đồng biến trên $[0,2]$ \& $f(0) = -1,f(2) = 1$. Chứng minh tồn tại $a,b,c\in[0,2]$ thỏa $f'(a)f'(b)f'(c) = 1$.
\end{baitoan}

\begin{baitoan}[\cite{VMS_VMC2023}, p. 37, 4.4, ĐHGTVT]
	Cho $f\in C^\infty(\mathbb{R})$ thỏa $f^{(n)}(0) = 0$, $\forall n\in\mathbb{N}$ \& $f^{(n)}(x)x\ge0$, $\forall k\in\mathbb{N}^\star$, $\forall x\in(0,\infty)$. Chứng minh $f(x) = 0$, $\forall x\in(0,\infty)$.	 
\end{baitoan}

\begin{baitoan}[\cite{VMS_VMC2023}, p. 37, 4.5, ĐH Hùng Vương, Phú Thọ]
	Giả sử hàm $f\in C([1,2023])$, khả vi trong khoảng $(1,2023)$, \& $f(2023) = 0$. Chứng minh tồn tại $c\in(1,2023)$ thỏa
	\begin{equation*}
		f'(c) = \frac{2024 - 2023c}{1 - c}f(c).
	\end{equation*}
\end{baitoan}

\begin{baitoan}[\cite{VMS_VMC2023}, p. 37, 4.6, ĐHKH Thái Nguyên]
	Giả sử $f(x)\in C^\infty([-1,1])$, $f^{(n)}(0) = 0$, $\forall n\in\mathbb{N}$, \& tồn tại $\alpha\in(0,1)$ thỏa $\sup_{x\in[-1,1]} |f^{(n)}(x)|\le\alpha^nn!$, $\forall n\in\mathbb{N}$. Chứng minh $f(x)\equiv0$ trên đoạn $[-1,1]$.
\end{baitoan}

\begin{baitoan}[\cite{VMS_VMC2023}, p. 37, 4.7, ĐHSPHN2]
	Cho $f\in C([a,b])$ khả vi trên $(a,b)$. Giả sử $f'(x) > 0$, $\forall x\in(a,b)$. Chứng minh $\forall x_1,x_2\in\mathbb{R}$ thỏa $a\le x_1 < x_2\le b$ \& $f(x_1)f(x_2) > 0$ thì luôn tồn tại $c\in(x_1,x_2)$ thỏa
	\begin{equation*}
		\frac{x_1f(x_2) - x_2f(x_1)}{f(x_2) - f(x_1)} = c - \frac{f(c)}{f'(c)}.
	\end{equation*}
\end{baitoan}

\begin{baitoan}[\cite{VMS_VMC2024}, p. 33, 3.1, VNUHCM UIT]
	Cho $f$ là hàm số thực trên $(0,\infty)$. Giả sử
	\begin{equation*}
		f(x^\alpha) = f(x)\sin^2\alpha + f(1)\cos^2\alpha,\ \forall x\in(0,\infty),\ \forall\alpha\in\mathbb{R}.
	\end{equation*}
	Chứng minh $f$ khả vi tại $1$.
\end{baitoan}

\begin{baitoan}[\cite{VMS_VMC2024}, p. 34, 3.2, ĐH Đồng Tháp]
	(a) Chứng minh với mỗi $n\in\mathbb{N}^\star$, phương trình $2x = \sqrt{x + n} + \sqrt{x + n + 1}$ có nghiệm dương duy nhất, ký hiệu là $x_n$. (b) Tính $a\coloneqq\lim_{n\to\infty} \dfrac{x_n}{\sqrt{n}},b\coloneqq \lim_{n\to\infty} x_n - a\sqrt{n}$.
\end{baitoan}

\begin{baitoan}[\cite{VMS_VMC2024}, p. 34, 3.3, ĐH Đồng Tháp]
	Cho
	\begin{equation*}
		f(x) = \left\{\begin{split}
			&x^2\left|\cos\frac{\pi}{x}\right|&&\mbox{if } x\ne0,\\
			&0&&\mbox{if } x = 0.
		\end{split}\right.
	\end{equation*}
	Chứng minh $f$ khả vi tại $0$ nhưng $f$ không khả vi tại các điểm $x_n\coloneqq\dfrac{2}{2n + 1}$ với $n\in\mathbb{Z}$.
\end{baitoan}

\begin{baitoan}[\cite{VMS_VMC2024}, p. 34, 3.4, ĐH Đồng Tháp]
	Giả sử $f$ khả vi liên tục trên $(0,\infty)$, $f(0) = 1$. Chứng minh nếu $|f(x)|\le e^{-x}$, $\forall x\ge0$ thì tồn tại $x_0 > 0$ để $f'(x_0) = -e^{-x_0}$.
\end{baitoan}

\begin{baitoan}[\cite{VMS_VMC2024}, p. 34, 3.5, ĐHGTVT]
	Cho $a\in\mathbb{R}$, $b\in(0,\infty)$. Hàm $f$ xác định trên $[-1,1]$, được cho bởi
	\begin{equation*}
		f(x) = \left\{\begin{split}
			&x^a\sin x^{-b}&&\mbox{if } x\ne0,\\
			&0&&\mbox{if } x = 0.
		\end{split}\right.
	\end{equation*}
	(a) Tìm tất cả các giá trị của $a$ để hàm $f$ liên tục trên $[-1,1]$. (b) Tìm tất cả các giá trị của $a$ để tồn tại $f'(0)$. (c) Tìm điều kiện của $a,b$ để tồn tại $f''(0)$.
\end{baitoan}

\begin{baitoan}[\cite{VMS_VMC2024}, p. 35, 3.7, HUS]
	Cho $f:\mathbb{R}\to\mathbb{R}$ là hàm số được xác định bởi công thức
	\begin{equation*}
		f(x) = \left\{\begin{split}
			&x^2 + a&&\mbox{if } x\le0,\\
			&be^x + x&&\mbox{if } x > 0,
		\end{split}\right.
	\end{equation*}
	với $a,b\in\mathbb{R}$: tham số. Xác định $a,b$ để $f$ có nguyên hàm trên $\mathbb{R}$.
\end{baitoan}

\begin{baitoan}[\cite{VMS_VMC2024}, p. 35, 3.8, ĐH Vinh]
	Cho hàm $f\in C(\mathbb{R},\mathbb{R})$ thỏa $f_{2024}(x) = x$, $\forall x\in\mathbb{R}$ với
	\begin{equation*}
		\left\{\begin{split}
			f_{n+1}(x) &= f(f_n(x)),\ \forall x\in\mathbb{R},\,\forall n\in\mathbb{N}^\star,\\
			f_1(x) &= f(x),\ \forall x\in\mathbb{R}
		\end{split}\right.
	\end{equation*}
	Chứng minh $f_2(x) = x$, $\forall x\in\mathbb{R}$.
\end{baitoan}

\begin{baitoan}[\cite{VMS_VMC2024}, p. 35, 3.9, ĐH Vinh]
	Cho hàm
	\begin{equation*}
		f(x) = \left(\frac{2023^x + 2024^x}{2}\right)^{\frac{1}{x}},\ x > 0.
	\end{equation*}
	(a) Tìm $\lim_{x\to0^+} f(x)$. (b) Chứng minh $f$ là hàm số đơn điệu tăng trên $(0,+\infty)$.
\end{baitoan}

\begin{baitoan}[\cite{VMS_VMC2024}, p. 36, 4.1, HCMUT]
	(a) Cho $f\in C^3(\mathbb{R},[0,+\infty))$ thỏa $\max_{x\in\mathbb{R}} |f'''(x)|\le1$. Chứng minh
	\begin{equation*}
		f''(x)\ge-\sqrt[3]{\frac{3}{2}f(x)},\ \forall x\in\mathbb{R}.
	\end{equation*}
	(b) Tìm tất cả các hàm số $f$ thỏa mãn điều kiện của (a) thỏa
	\begin{equation*}
		f''(x) = -\sqrt[3]{\frac{3}{2}f(x)},\ \forall x\in\mathbb{R}.
	\end{equation*}
\end{baitoan}

\begin{baitoan}[\cite{VMS_VMC2024}, p. 36, 4.2, VNUHCM UIT]
	Cho hàm số $f:[0,1]\to\mathbb{R})$ liên tục trên $[0,1]$, khả vi trên $(0,1)$ sao cho $\exists M > 0$, $\exists c\in[0,1]$ thỏa $f(c) = 0$ \&
	\begin{equation*}
		|f'(x)|\le M|f(x)|,\ \forall x\in(0,1).
	\end{equation*}
	Chứng minh $f(x) = 0$, $\forall x\in[0,1]$.
\end{baitoan}

\begin{baitoan}[\cite{VMS_VMC2024}, p. 36, 4.3, ĐH Đồng Tháp]
	Cho $f$ khả vi trên $\mathbb{R}$ \& $f'$ giảm ngặt trên $\mathbb{R}$. (a) Chứng minh
	\begin{equation*}
		f(x + 1) - f(x) < f'(x) < f(x) - f(x - 1),\ \forall x\in\mathbb{R}.
	\end{equation*}
	(b) Chứng minh nếu tồn tại $\lim_{x\to\infty} f(x) = L$ thì $\lim_{x\to\infty} f'(x) = 0$. (c) Tìm hàm số $g$ khả vi trên $\mathbb{R}$ \& tồn tại $\lim_{x\to\infty} g(x) = L$ nhưng $\lim_{x\to\infty} g'(x)\ne0$.
\end{baitoan}

\begin{baitoan}[\cite{VMS_VMC2024}, p. 37, 4.4, ĐHGTVT]
	Giả sử $V$ là tập hợp các hàm liên tục $f:[0,1]\to\mathbb{R}$ \& khả vi trên $(0,1)$ thỏa $f(0) = 0,f(1) = 1$. Xác định các giá trị $\alpha\in\mathbb{R}$ để với mỗi $f\in V$, luôn tồn tại $\xi\in(0,1)$ thỏa $f(\xi) + \alpha = f'(\alpha)$.
\end{baitoan}

\begin{baitoan}[\cite{VMS_VMC2024}, p. 37, 4.5, HUS]
	Cho $f:[0,3]\to\mathbb{R}$ là hàm liên tục trên $[0,3]$ \& khả vi trong $(0,3)$. Chứng minh tồn tại $c\in(0,3)$ thỏa $2f'(c) = f(3) - f(2) + f(1) - f(0)$.
\end{baitoan}

\begin{baitoan}[\cite{VMS_VMC2024}, p. 37, 4.6, ĐH Mỏ--Địa chất]
	Giả sử có chuỗi có 2 đầu hướng ra vô cực
	\begin{equation*}
		\cdots + f''(x) + f'(x) + f(x) + \int_0^x f(t)\,{\rm d}t + \int_0^x\int_0^t f(s)\,{\rm d}s\,{\rm d}t + \cdots
	\end{equation*}
	\& hội tụ đều trên khoảng $(-1,1)$. Chuỗi là biểu diễn của số nào?
\end{baitoan}

\begin{baitoan}[\cite{VMS_VMC2024}, p. 37, 4.7, ĐH Vinh]
	Cho hàm $f\in C^2(\mathbb{R},\mathbb{R})$ \& thỏa $f(x)\le2024$, $\forall x\in\mathbb{R}$. Chứng minh tồn tại $x\in\mathbb{R}$ thỏa $f''(x) = 0$.
\end{baitoan}

%------------------------------------------------------------------------------%

\chapter{Integral -- Tích Phân}
\minitoc

%------------------------------------------------------------------------------%

%------------------------------------------------------------------------------%

\section{Antiderivative -- Nguyên Hàm}
\textbf{\textsf{Resources -- Tài nguyên.}}
\begin{enumerate}
    \item \href{https://en.wikipedia.org/wiki/Antiderivative}{Wikipedia{\tt/}antiderivative}.
    
    \item \href{https://en.wikipedia.org/wiki/Lists_of_integrals}{Wikipedia{\tt/}lists of integrals}.
    
    \item \cite{SGK_Toan_12_CD_tap_2}. {\sc Đỗ Đức Thái, Phạm Xuân Chung, Nguyễn Sơn Hà, Nguyễn Thị Phương Loan, Phạm Sỹ Nam, Phạm Minh Phương}. {\it Toán 12 Tập 2}. Cánh Diều. Chương IV: Nguyên Hàm. Tích Phân.
    
    \item \cite{SBT_Toan_10_CD_tap_2}. {\sc Đỗ Đức Thái, Phạm Xuân Chung, Nguyễn Sơn Hà, Nguyễn Thị Phương Loan, Phạm Sỹ Nam, Phạm Minh Phương}. {\it Bài Tập Toán 12 Tập 2}. Cánh Diều. Chương IV: Nguyên Hàm. Tích Phân.
\end{enumerate}
\fbox{1} $\left(\int f(x)dx\right)' = f(x)$. \fbox{2} {\sf Tính chất của nguyên hàm}: $\int [f(x) + g(x)]dx = \int f(x)dx + \int g(x)dx$. $\int af(x)dx = a\int f(x)dx$, $\forall a\in\mathbb{R}$. $d\left(\int f(x)dx\right) = f(x)dx$.

\begin{dinhnghia}[Nguyên hàm]
    Cho hàm số $:K\to\mathbb{R}$ (với $K$ là khoảng, đoạn hay nửa khoảng). Hàm số $F(x)$ được gọi là {\rm nguyên hàm} của hàm số $f(x)$ trên $K$ nếu $F'(x) = f(x)$, $\forall x\in K$.
\end{dinhnghia}

\begin{dinhly}[Family of antiderivatives -- Họ nguyên hàm]
    Nếu $F(x)$ là 1 nguyên hàm của hàm số $f(x)$ trên $K$ thì:
    \item(i) Vvới mỗi hằng số $C\in\mathbb{R}$, hàm số $G(x,C) = F(x) + C$ cũng là 1 nguyên hàm của $f(x)$ trên $K$.
    \item(ii) Mọi nguyên hàm của $f(x)$ trên $K$ đều có dạng $F(x) + C$, với $C\in\mathbb{R}$ là 1 hằng số.
    
    Do đó, $\{F(x) + C\}_{C\in\mathbb{R}}$ là họ tất cả các nguyên hàm của $f(x)$ trên $K$. Ký hiệu $\int f(x)\,{\rm d}x = F(x) + C$.
\end{dinhly}

\begin{dinhly}[Some properties of antiderivative -- Vài tính chất của nguyên hàm]
    \begin{align*}
        \left(\int f(x)\,{\rm d}x\right)' &= f(x),\ \forall f\in C(\mathbb{R}),\\
        \int f'(x)\,{\rm d}x &= f(x) + C,\ \forall f\in C^1(\mathbb{R}),\\
        \int kf(x)\,{\rm d}x &= k\int f(x)\,{\rm d}x,\ \forall k\in\mathbb{R},\\
        \int (f(x)\pm g(x))\,{\rm d}x &= \int f(x)\,{\rm d}x\pm\int g(x)\,{\rm d}x.
    \end{align*}
    Nguyên hàm của 1 tổ hợp tuyến tính các hàm liên tục là tổ hợp tuyến tính của các nguyên hàm của các hàm liên tục đó:
    \begin{equation*}
        \int \sum_{i=1}^n a_if_i(x)\,{\rm d}x = \sum_{i=1}^n a_i\int f_i(x)\,{\rm d}x,\ \forall n\in\mathbb{N}^\star,\,\forall a_i\in\mathbb{R},\,\forall f_i\in C(\mathbb{R}),\,\forall i\in[n].
    \end{equation*}
\end{dinhly}

\begin{dinhly}[Continuity $\Rightarrow$ integrability -- Liên tục $\Rightarrow$ khả tích or existence of antiderivative -- sự tồn tại của nguyên hàm]
    Mọi hàm số $f\in C(K)$ đều có nguyên hàm trên $K$.
\end{dinhly}

\begin{baitoan}[\cite{SGK_Toan_12_giai_tich_co_ban}, 1, p. 93]
    Tìm hàm số $F(x)$ sao cho $F'(x) = f(x)$ nếu: (a) $f(x) = 3x^2$, $\forall x\in\mathbb{R}$; (b) $f(x) = \frac{1}{\cos^2x}$, $\forall x\in\left(-\frac{\pi}{2};\frac{\pi}{2}\right)$.
\end{baitoan}

\begin{baitoan}[\cite{SGK_Toan_12_giai_tich_co_ban}, Ví dụ 6, p. 97]
    Tính: (a) $\int \left(2x^ + \frac{1}{\sqrt[3]{x^2}}\right)\,{\rm d}x$ trên khoảng $(0;+\infty)$; (b) $\int (3\cos x - 3^{x-1})\,{\rm d}x$ trên khoảng $(-\infty;+\infty)$.
\end{baitoan}

\begin{baitoan}[\cite{SGK_Toan_12_giai_tich_co_ban}, 6, p. 98]
    (a) Cho $\int (x - 1)^{10}\,{\rm d}x$. Đặt $u = x - 1$, viết $(x - 1)^{10}\,{\rm d}x$ theo $u$ \& ${\rm d}u$. (b) Cho $\int\frac{\ln x}{x}\,{\rm d}x$. Đặt $x = e^t$, viết $\frac{\ln x}{x}\,{\rm d}x$ theo $t$ \& ${\rm d}t$.
\end{baitoan}

\begin{baitoan}[\cite{SGK_Toan_12_giai_tich_co_ban}, Ví dụ 7, p. 98]
    Tính: (a) $\int \sin(3x - 1)\,{\rm d}x$. (b) $\int \sin(ax + b)\,{\rm d}x$. (c) $\int \cos(ax + b)\,{\rm d}x$
\end{baitoan}

\begin{baitoan}[\cite{SGK_Toan_12_giai_tich_co_ban}, Ví dụ 8, p. 99]
    Tính $\int \frac{x}{(x + 1)^5}\,{\rm d}x$.
\end{baitoan}

\begin{baitoan}[Mở rộng \cite{SGK_Toan_12_giai_tich_co_ban}, Ví dụ 8, p. 99]
    Tính $\int \frac{x}{(x + 1)^n}\,{\rm d}x$ với $n\in\mathbb{N}$.
\end{baitoan}

\begin{baitoan}[\cite{SGK_Toan_12_giai_tich_co_ban}, Ví dụ 8, p. 100]
    Tính: (a) $\int xe^x\,{\rm d}x$; (b) $\int x\cos x\,{\rm d}x$; (c) $\int \ln x\,{\rm d}x$.
\end{baitoan}

\begin{baitoan}[\cite{SGK_Toan_12_giai_tich_co_ban}, 8, p. 100]
    Cho $P(x)$ là đa thức của $x$. Tính $\int P(x)e^x\,{\rm d}x$, $\int P(x)\cos x\,{\rm d}x$, $\int P(x)\ln x\,{\rm d}x$.
\end{baitoan}

\begin{baitoan}[\cite{SGK_Toan_12_giai_tich_co_ban}, 1., p. 100]
    Trong các cặp hàm số dưới đây, hàm số nào là 1 nguyên hàm của hàm số còn lại? (a) $e^{-x}$ \& $-e^{-x}$; (b) $\sin2x$ \& $\sin^2x$; (c) $\left(1 - \frac{2}{x}\right)^2e^x$ \& $\left(1 - \frac{4}{x}\right)e^x$.
\end{baitoan}

\begin{baitoan}[\cite{SGK_Toan_12_giai_tich_co_ban}, 2., pp. 100--101]
    Tìm nguyên hàm của các hàm số sau: (a) $f(x) = \frac{x + \sqrt{x} + 1}{\sqrt[3]{x}}$; (b) $f(x) = \frac{2^x - 1}{e^x}$; (c) $f(x) = \frac{1}{\sin^2x\cos^2x}$; (d) $f(x) = \sin5x\cos3x$; (e) $f(x) = \tan^2x$; (g) $f(x) = e^{3-2x}$; (h) $f(x) = \frac{1}{(1 + x)(1 - 2x)}$.
\end{baitoan}

\begin{baitoan}[\cite{SGK_Toan_12_giai_tich_co_ban}, 3., p. 101]
    Sử dụng phương pháp đổi biến số, tính: (a) $\int (1 - x)^9\,{\rm d}x$ (đặt $u = 1 - x$); (b) $\int x(1 + x^2)^{\frac{3}{2}}\,{\rm d}x$ (đặt $u = 1 + x^2$); (c) $\int \cos^3x\sin x\,{\rm d}x$ (đặt $t = \cos x$); (d) $\int \frac{{\rm d}x}{e^x + e^{-x} + 2}$ (đặt $u = e^x + 1$).
\end{baitoan}

\begin{baitoan}[\cite{SGK_Toan_12_giai_tich_co_ban}, 4., p. 101]
    Sử dụng phương pháp tính nguyên hàm từng phần, tính: (a) $\int x\ln(1 + x)\,{\rm d}x$; (b) $\int (x^2 + 2x - 1)e^x\,{\rm d}x$; (c) $\int x\sin(2x + 1)\,{\rm d}x$; (d) $\int (1 - x)\cos x\,{\rm d}x$.
\end{baitoan}

\begin{baitoan}[\cite{SBT_Toan_12_giai_tich_co_ban}, Ví dụ 1, p. 144]
    Tính: $\int \frac{\sin^3x}{\cos^4x}\,{\rm d}x$.
\end{baitoan}

\begin{proof}[Giải]
    Có $\int \frac{\sin^3x}{\cos^4x}\,{\rm d}x = \int \left(\frac{1}{\cos^4x} - \frac{1}{\cos^2x}\right)\sin x\,{\rm d}x$. Đặt $t = \cos x$, được $t' = -\sin x$ hay $dt = -\sin x\,{\rm d}x$ \& $\frac{\sin^3x}{\cos^4x}\,{\rm d}x = \left(\frac{1}{\cos^4x} - \frac{1}{\cos^2x}\right)\sin x\,{\rm d}x$ viết thành $-\left(\frac{1}{t^4} - \frac{1}{t^2}\right)\,{\rm d}t$. Do đó, nguyên hàm đã cho viết thành: $-\int \left(\frac{1}{t^4} - \frac{1}{t^2}\right)\,{\rm d}t = \frac{1}{3t^3} - \frac{1}{t} + C$. Thay $t = \cos x$, được: $\int \frac{\sin^3x}{\cos^4x}\,{\rm d}x = \frac{1}{3\cos^3x} - \frac{1}{\cos x} + C$.
\end{proof}

\begin{baitoan}[\cite{SBT_Toan_12_giai_tich_co_ban}, Ví dụ 2, p. 144]
    Tính: $\int \frac{\ln(\sin x)}{\cos^2x}\,{\rm d}x$.
\end{baitoan}

\begin{baitoan}[\cite{SBT_Toan_12_giai_tich_co_ban}, Ví dụ 3, p. 145]
    Tính: $\int \cos\sqrt{x}\,{\rm d}x$.
\end{baitoan}

\begin{baitoan}[\cite{SBT_Toan_12_giai_tich_co_ban}, 3.1., p. 145]
    Kiểm tra xem hàm số nào là 1 nguyên hàm của hàm số còn lại trong mỗi cặp hàm số sau: (a) $f(x) = \ln\left(x + \sqrt{1 + x^2}\right)$ \& $g(x) = \frac{1}{\sqrt{1 + x^2}}$. (b) $f(x) = e^{\sin x}\cos x$ \& $g(x) = e^{\sin x}$. (c) $f(x) = \sin^2\frac{1}{x}$ \& $g(x) = -\frac{1}{x^2}\sin\frac{2}{x}$. (d) $f(x) = \frac{x - 1}{\sqrt{x^2 - 2x + 2}}$ \& $g(x) = \sqrt{x^2 - 2x + 2}$. (e) $f(x) = x^2e^{\frac{1}{x}}$ \& $g(x) = (2x - 1)e^{\frac{1}{x}}$.
\end{baitoan}

\begin{baitoan}[\cite{SBT_Toan_12_giai_tich_co_ban}, 3.2., pp. 145--146]
    Chứng minh các hàm số $F(x),G(x)$ sau đều là 1 nguyên hàm của cùng 1 hàm số: (a) $F(x) = \frac{x^2 + 6x + 1}{2x - 3}$ \& $G(x) = \frac{x^2 + 10}{2x - 3}$. (b) $F(x) = \frac{1}{\sin^2x}$ \& $G(x) = 10 + \cot^2x$. (c) $F(x) = 5 + 2\sin^2x$ \& $G(x) = 1 - \cos2x$.
\end{baitoan}

\begin{baitoan}[\cite{SBT_Toan_12_giai_tich_co_ban}, 3.3, p. 146]
    Tìm nguyên hàm của các hàm số sau: (a) $f(x) = (x - 9)^4$; (b) $f(x) = \frac{1}{(2 - x)^2}$; (c) $f(x) = \frac{x}{\sqrt{1 - x^2}}$; (d) $f(x) = \frac{1}{\sqrt{2x + 1}}$; (e) $f(x) = \frac{1 - \cos2x}{\cos^2x}$; (f) $f(x) = \frac{2x + 1}{x^2 + x + 1}$.
\end{baitoan}

\begin{baitoan}[\cite{SBT_Toan_12_giai_tich_co_ban}, 3.4, p. 146]
    Tính các nguyên hàm sau bằng phương pháp đổi biến số: (a) $\int x^2\sqrt[3]{1 + x^3}\,{\rm d}x$ với $x > -1$ (đặt $t = 1 + x^3$); (b) $\int xe^{-x^2}\,{\rm d}x$ (đặt $t = x^2$); (c) $\int \frac{x}{(1 + x^2)^2}\,{\rm d}x$ (đặt $t = 1 + x^2$); (d) $\int \frac{1}{(1 - x)\sqrt{x}}\,{\rm d}x$ (đặt $t = \sqrt{x}$); (e) $\int \sin\frac{1}{x}\cdot\frac{1}{x^2}\,{\rm d}x$ (đặt $t = \frac{1}{x}$); (f) $\int \frac{(\ln x)^2}{x}\,{\rm d}x$ (đặt $t = \ln x$); (g) $\int \frac{\sin x}{\sqrt[3]{\cos^2x}}\,{\rm d}x$ (đặt $t = \cos x$); (h) $\int \cos x\sin^3x\,{\rm d}x$ (đặt $t = \sin x$); (i) $\int \frac{1}{e^x - e^{-x}}\,{\rm d}x$ (đặt $t = e^x$); (j) $\int \frac{\cos x + \sin x}{\sqrt{\sin x - \cos x}}\,{\rm d}x$ (đặt $t = \sin x - \cos x$).
\end{baitoan}

\begin{baitoan}[\cite{SBT_Toan_12_giai_tich_co_ban}, 3.5, p. 146]
    Áp dụng phương pháp tính nguyên hàm từng phần, tính: (a) $\int (1 - 2x)e^x\,{\rm d}x$; (b) $\int xe^{-x}\,{\rm d}x$; (c) $\int x\ln(1 - x)\,{\rm d}x$; (d) $\int x\sin^2x\,{\rm d}x$; (e) $\int \ln(1 + \sqrt{1 + x^2})$
\end{baitoan}

\begin{baitoan}[\cite{TLCT_giai_tich_12}, VD1, p. 106]
    Tính $\int \cos^23xdx$.
\end{baitoan}

\begin{baitoan}[\cite{TLCT_giai_tich_12}, VD2, p. 106]
    Tìm hàm số $f$ thỏa $f''(x) = 12x^2 + 6x - 4,f(0) = 4,f(1) = 1$.
\end{baitoan}

\begin{baitoan}
    Tìm hàm số $f$ thỏa $f(a) = b$ \&: (a) $f'(x) = c$. (b) $f'(x) = cx + d$. (c) $f'(x) = cx^2 + dx + e$. (d) $f'(x) = \sum_{i=0}^n a_ix^i$.
\end{baitoan}

\begin{baitoan}
    Tìm hàm số $f$ thỏa $f(a) = m,f(b) = n$ \&: (a) $f''(x) = c$. (b) $f''(x) = cx + d$. (c) $f''(x) = cx^2 + dx + e$. (d) $f''(x) = \sum_{i=0}^n a_ix^i$.
\end{baitoan}

\begin{baitoan}[\cite{TLCT_giai_tich_12}, VD3, p. 106]
    Cho $f(x) = \dfrac{x^3 + 2}{x^2 - 1}$. (a) Viết $f(x)$ dưới dạng $f(x) = ax + \dfrac{b}{x + 1} + \dfrac{c}{x - 1}$. (b) Tính $\int f(x)dx$.
\end{baitoan}

\begin{baitoan}[\cite{TLCT_giai_tich_12}, VD4, p. 108]
    Tính $\int x^2(1 - x)^7dx$.
\end{baitoan}

\begin{baitoan}[\cite{TLCT_giai_tich_12}, VD5, p. 108]
    Tính: (a) $\int \dfrac{\cos x - \sin x}{\cos x + \sin x}dx$. (b) $\int \dfrac{7\cos x - 4\sin x}{\cos x + \sin x}dx$.
\end{baitoan}

\begin{baitoan}[\cite{TLCT_giai_tich_12}, VD6, p. 109]
    Tính: (a) $\int xe^{-x}dx$. (b) $\int \sqrt{x}\ln xdx$.
\end{baitoan}

\begin{baitoan}[\cite{TLCT_giai_tich_12}, VD7, p. 110]
    Tính $\int \dfrac{x^2}{(\cos x + x\sin x)^2}dx$.
\end{baitoan}

\begin{baitoan}[\cite{TLCT_giai_tich_12}, VD8, p. 110]
    Tính $\int \sin x\cos xdx$.
\end{baitoan}

\begin{baitoan}[\cite{TLCT_giai_tich_12}, 1., p. 110]
    Tính $\int \dfrac{e^{\tan x}}{\cos^2x}dx$.
\end{baitoan}

\begin{baitoan}[\cite{TLCT_giai_tich_12}, 2., p. 110]
    Tính: (a) $\int \sin2x\cos xdx$. (b) $\int \cot^22xdx$.
\end{baitoan}

\begin{baitoan}[\cite{TLCT_giai_tich_12}, 3., p. 111]
    Tìm hàm số $f(x)$ thỏa: (a) $f'(x) = 4\sqrt{x} - x,f(4) = 0$. (b) $f'(x) = x - \dfrac{1}{x^2} + 2,f(1) = 2$.
\end{baitoan}

\begin{baitoan}[\cite{TLCT_giai_tich_12}, 4., p. 111]
    Tính: (a) $\int 3x^2\sqrt{x^3 + 1}dx$. (b) $\int \dfrac{2x + 4}{x^2 + 4x - 5}dx$.
\end{baitoan}

\begin{baitoan}[\cite{TLCT_giai_tich_12}, 5., p. 111]
    Tính $\int xe^{x^2}dx$.
\end{baitoan}

\begin{baitoan}[\cite{TLCT_giai_tich_12}, 6., p. 111]
    Tính: (a) $\int x^3\ln2xdx$. (b) $\int x^2\cos2xdx$.
\end{baitoan}

\begin{baitoan}[\cite{TLCT_giai_tich_12}, 7., p. 111]
    Tính: (a) $\int \dfrac{x^3}{(6x^4 + 5)^5}dx$. (b) $\int x^2e^xdx$.
\end{baitoan}

%------------------------------------------------------------------------------%

\section{Antivative of Some Elementary Functions -- Nguyên Hàm Của 1 Số Hàm Số Sơ Cấp}
\fbox{1} (a) $\int dx = x + C$. (b) $\int (x + a)^\alpha dx = \dfrac{(x + a)^{\alpha + 1}}{\alpha + 1} + C$, $\forall a,\alpha\in\mathbb{R},\alpha\ne-1$. (c) $\int \dfrac{1}{x + a}dx = \ln|x + a| + C$, $\forall a\in\mathbb{R}$. (d) $\int \sin\alpha dx = -\dfrac{\cos\alpha x}{\alpha} + C$, $\int \cos\alpha xdx = \dfrac{\sin\alpha x}{\alpha} + C$, $\forall\alpha\in\mathbb{R}^\star$. (e) $\int a^xdx = \dfrac{a^x}{\ln a} + C$, $\forall a\in(0,\infty),a\ne-1$. (f) $\int \dfrac{1}{\cos^2x}dx = \tan x + C$, $\int\dfrac{1}{\sin^2x}dx = -\cot x + C$. \fbox{2} {\sf Công thức đổi biến}: \fbox{$\int f(u(x))u'(x)dx = F(u(x)) + C$}, \fbox{$\int f(u)du = F(u(x)) + C$}. \fbox{5} {\sf Công thức nguyên hàm từng phần}: \fbox{$\int u(x)v'(x)dx = u(x)v(x) - \int v(x)u'(x)dx$}, \fbox{$\int udv = uv - \int vdu$}.\\
\\
\cite[Chap. IV, \S2, pp. 9--16]{SGK_Toan_12_Canh_Dieu_tap_2}: HD1. LT1. LT2. HD2. LT3. HD3. LT4. LT5. HD4. LT6. 1. 2. 3. 4. 5. 6. 7. 8.

%------------------------------------------------------------------------------%

\section{Integral -- Tích Phân}
\fbox{1} $\int_a^b f(x)dx = F(x)|_a^b = \left(\int f(x)dx\right)|_a^b$. \fbox{2} (a) {\sf Tính chất của tích phân}: (a) $\int_a^a f(x)dx = 0$. (b) $\int_a^b f(x)dx = -\int_b^a f(x)$. (c) $\int_a^b f(x)dx + \int_b^c f(x)dx = \int_a^c f(x)dx$. (d) $\int_a^b (f(x) + g(x))dx = \int_a^b f(x)dx + \int_a^b g(x)dx$. (f) $\int_a^b kf(x)dx = k\int_a^b f(x)dx$, $\forall k\in\mathbb{R}$. \fbox{3} {\sf Công thức đổi biến}: \fbox{$\int_a^b f(u(x))u'(x)dx = \int_{u(a)}^{u(b)} f(u)du$}. \fbox{4} {\sf Công thức tích phân từng phần}: $\int_a^b udv = uv|_a^b - \int_a^b vdu$, \fbox{$\int_a^b u(x)v'(x)dx = u(b)v(b) - u(a)v(a) - \int_a^b u'(x)v(x)dx$}.

\begin{baitoan}[\cite{TLCT_giai_tich_12}, VD1, p. 113]
    Tính: (a) $\int_4^5 \left(x^2 + \dfrac{1}{x}\right)^2dx$. (b) $I = \int_{\frac{\pi}{4}}^{\frac{\pi}{3}} \dfrac{dx}{\sin2x}$. (c) $I = \int_1^e x^2\ln xdx$.
\end{baitoan}

\begin{baitoan}[\cite{TLCT_giai_tich_12}, VD2, p. 114]
    Cho $a\in\left(0,\dfrac{\pi}{2}\right)$. Chứng minh $\int_e^{\tan a} \dfrac{xdx}{1 + x^2} + \int_e^{\cot a} \dfrac{dx}{x(1 + x^2)} = -1$.
\end{baitoan}

\begin{baitoan}[\cite{TLCT_giai_tich_12}, VD3, p. 114]
    Tìm nguyên hàm của hàm số
    \begin{equation*}
        f(x) = \left\{\begin{split}
            -&x,&&\mbox{if } x < -1,\\
            &1,&&\mbox{if } -1\le x\le1,\\
            &x,&&\mbox{if } x > 1.
        \end{split}\right.
    \end{equation*}
\end{baitoan}

\begin{baitoan}[\cite{TLCT_giai_tich_12}, VD4, p. 115]
    Cho hàm số $g(x) = \int_{\sqrt{x}}^{x^2} \sqrt{t}\sin tdt$ xác định với $x > 0$. Tìm $g'(x)$.
\end{baitoan}

\begin{baitoan}[\cite{TLCT_giai_tich_12}, VD5, p. 117]
    Cho dãy $(u_n)$ xác định bởi công thức $u_n = \dfrac{1}{n}\sum_{i=1}^n \sqrt{\dfrac{i}{n}}$. Tính $\lim_{n\to\infty} u_n$.
\end{baitoan}

\begin{baitoan}[\cite{TLCT_giai_tich_12}, VD6, p. 118]
    Cho dãy $(u_n)$ xác định bởi công thức $u_n = \sum_{i=1}^n \dfrac{1}{2n + 2i - 1} = \dfrac{1}{2n + 1} + \dfrac{1}{2n + 3} + \cdots + \dfrac{1}{4n - 1}$. Tính $\lim_{n\to\infty} u_n$.
\end{baitoan}

\begin{baitoan}[\cite{TLCT_giai_tich_12}, VD7, p. 119]
    Tính $I = \int_1^2 xe^{x^2}dx$.
\end{baitoan}

\begin{baitoan}[\cite{TLCT_giai_tich_12}, VD8, p. 120]
    Tính: (a) $I = \int_{-1}^1 \dfrac{dx}{x^2 + 1}$. (b) $I = \int_{\pi}^{2\pi} \dfrac{x\sin x}{1 + \cos^2x}dx$.
\end{baitoan}

\begin{baitoan}[\cite{TLCT_giai_tich_12}, VD9, p. 121]
    Tính $I = \int_0^{\frac{\pi}{4}} \dfrac{(1 + \sin x\cos x)e^x}{1 + \cos2x}dx$.
\end{baitoan}

\begin{baitoan}[\cite{TLCT_giai_tich_12}, VD10, p. 121]
    Tính $u_n = \int_0^\pi \cos^nx\cos nxdx$.
\end{baitoan}

\begin{baitoan}[\cite{TLCT_giai_tich_12}, VD11, p. 122]
    Giả sử f là hàm liên tục. Chứng minh $\int_0^a f(x)(a - x)dx = \int_0^a\left(\int_0^x f(t)dt\right)dx$.
\end{baitoan}

\begin{baitoan}[\cite{TLCT_giai_tich_12}, 8., p. 123]
    Tính: (a) $I = \int_0^1 x^3e^{x^2}dx$. (b) $I = \int_0^{\ln2} e^{7x}dx$.
\end{baitoan}

\begin{baitoan}[\cite{TLCT_giai_tich_12}, 9., p. 123]
    Tính: (a) $I = \int_0^{\frac{\pi}{3}} \tan xdx$. (b) $I = \int_0^3 \dfrac{xdx}{1 + x^2}$.
\end{baitoan}

\begin{baitoan}[\cite{TLCT_giai_tich_12}, 10., p. 123]
    Tính: (a) $I = \int_0^{\frac{\pi}{3}} \tan^2xdx$. (b) $I = \int_1^e (\ln x)^2dx$.
\end{baitoan}

\begin{baitoan}[\cite{TLCT_giai_tich_12}, 11., p. 123]
    Tính: (a) $I = \int_0^1 x^2e^{4x}dx$. (b) $I = \int_4^7 \dfrac{dx}{\sqrt{(x - 4)(7 - x)}}$.
\end{baitoan}

\begin{baitoan}[\cite{TLCT_giai_tich_12}, 12., p. 123]
    Cho hàm số
    \begin{equation*}
        f(x) = \left\{\begin{split}
            -&2(x + 1),&&\mbox{khi } x\le0,\\
            &k(1 - x^2),&&\mbox{khi } x > 0.
        \end{split}\right.
    \end{equation*}
    Tìm $k\in\mathbb{R}$ để $\int_{-1}^1 f(x)dx = 1$.
\end{baitoan}

\begin{baitoan}[\cite{TLCT_giai_tich_12}, 13., p. 123]
    Cho hàm số $g(x) = \int_{2x}^{3x} \dfrac{t^2 - 1}{t^2 + 1}dt$. Tìm $g'(x)$.
\end{baitoan}

\begin{baitoan}[\cite{TLCT_giai_tich_12}, 14., p. 123]
    Tìm hàm số f \& $a\in(0,\infty)$ thỏa $\int_a^x \dfrac{f(t)}{t^2}dt + 6 = 2\sqrt{x}$, $\forall x\in(0,\infty)$.
\end{baitoan}

\begin{baitoan}[\cite{TLCT_giai_tich_12}, 15., p. 123]
    Cho hàm $f(x)$ liên tục \& $a\in(0,\infty)$. Giả sử $\forall x\in[0,a]$, có $f(x) > 0,f(x)f(a - x) = 1$. Tính $I = \int_0^a \dfrac{dx}{1 + f(x)}$ theo a.
\end{baitoan}

\begin{baitoan}[\cite{TLCT_giai_tich_12}, 16., p. 123]
    Tính $I = \int_{-1}^1 \dfrac{dx}{(e^x + 1)(x^2 + 1)}$.
\end{baitoan}

\begin{baitoan}[\cite{TLCT_giai_tich_12}, 17., p. 123]
    Cho dãy $(u_n)$ xác định bởi công thức $u_n = \sum_{i=1}^n \dfrac{i^3}{n^4}$. Tính $\lim_{n\to\infty} u_n$.
\end{baitoan}

\begin{baitoan}[\cite{TLCT_giai_tich_12}, 18., p. 123]
    Cho dãy $(u_n)$ xác định bởi công thức $u_n = \sum_{i=1}^n \dfrac{i^2}{i^3 + n^3}$. Tính $\lim_{n\to\infty} u_n$.
\end{baitoan}

%------------------------------------------------------------------------------%

\section{Geometrical Application of Integral -- Ứng Dụng Hình Học Của Tích Phân}
Cho các hàm $f,g\in C(\mathbb{R})$. \fbox{1} Hình phẳng giới hạn bởi đồ thị hàm số $y = f(x),y = g(x)$ \& 2 đường thẳng $x = a,x = b$ có diện tích $S = \int_a^b |f(x) - g(x)|dx$. \fbox{2} Hình phẳng giới hạn bởi các đường cong với phương trình $x = f(y),x = g(y)$ \& 2 đường thẳng $y = c,y = d$, $c < d$ có diện tích $S = \int_c^d |f(y) - g(y)|dy$. \fbox{3} Đường cong $\mathcal{C}:y = f(x),f\in C^2([a,b])$ từ điểm $A(a,f(a))$ đến điểm $B(b,f(b))$ có độ dài $L = \int_a^b \sqrt{1 + (f'(x))^2}dx$. \fbox{4} Đường cong $\mathcal{C}:x = f(y),f\in C^2([c,d])$ từ điểm $C(g(c),c)$ đến điểm $D(g(d),d)$ có độ dài $L = \int_c^d \sqrt{1 + (g'(y))^2}dy$.

\begin{baitoan}[\cite{SBT_Toan_12_giai_tich_co_ban}, Ví dụ 1, p. 156]
    Tính diện tích hình phẳng được giới hạn bởi các đường $y = x^2 - 2x$ \& $y = x$.
\end{baitoan}

\begin{baitoan}[\cite{SBT_Toan_12_giai_tich_co_ban}, Ví dụ 2, p. 156]
    Tính diện tích hình phẳng được giới hạn bởi các đường $y = \frac{10}{3}x - x^2$ \&
    \begin{equation*}
        y = \left\{\begin{split}
            &-x,&&\mbox{nếu }x\le 1,\\
            &x - 2,&&\mbox{nếu }x > 1.
        \end{split}\right.
    \end{equation*}
\end{baitoan}

\begin{baitoan}[\cite{TLCT_giai_tich_12}, VD1, p. 126]
    Tính diện tích hình phẳng giới hạn bởi đồ thị 2 hàm số $y = \sin x,y = \cos x$ \& 2 đường thẳng $x = 0,x = \dfrac{\pi}{2}$.
\end{baitoan}

\begin{baitoan}[\cite{TLCT_giai_tich_12}, VD2, p. 126]
    Tính diện tích hình phẳng $\mathcal{H}$ giới hạn bởi đường thẳng $y = x - 1$ \& parabol $y^2 = 2x + 6$.
\end{baitoan}

\begin{baitoan}[\cite{TLCT_giai_tich_12}, VD3, p. 128]
    Tính độ dài đường cong $\mathcal{C}:y^2 = x^3$ đi từ điểm $A(1,1)$ đến điểm $B(4,8)$.
\end{baitoan}

\begin{baitoan}[\cite{TLCT_giai_tich_12}, VD4, p. 129]
    Tìm độ dài cung parabol $\mathcal{C}:y^2 = x$ từ điểm $A(0,0)$ đến điểm $B\left(\frac{1}{4},\frac{1}{2}\right)$.
\end{baitoan}

%------------------------------------------------------------------------------%

\begin{baitoan}[\cite{VMS_VMC2023}, p. 38, 5.1, VNUHCM UIT]
	Cho hàm $f:(-1,1)\to\mathbb{R}$ khả vi đến cấp 2 thỏa $f(0) = 1$ \& $f''(x) + 2f'(x) + f(x)\ge1$, $\forall x\in(-1,1)$. Tìm {\rm GTNN} của $\int_{-1}^1 e^xf(x)\,{\rm d}x$.
\end{baitoan}

\begin{baitoan}[\cite{VMS_VMC2023}, p. 38, 5.2, ĐH Đồng Tháp]
	Cho hàm $f:[0,2023]\to(0,\infty)$ khả tích \& $f(x)f(2023 - x) = 1$, $\forall x\in[0,2023]$. Chứng minh $\int_0^{2023} f(x)\,{\rm d}x\ge2023$.
\end{baitoan}

\begin{baitoan}[\cite{VMS_VMC2023}, p. 38, 5.3, ĐHGTVT]
	Cho hàm $f\in C([0,1])$ thỏa $\int_0^1 f(x)\,{\rm d}x = \int_0^1 xf(x)\,{\rm d}x$. Chứng minh tồn tại $c\in(0,1)$ thỏa $cf(c) + 2023\int_0^c f(x)\,{\rm d}x = 0$.
\end{baitoan}

\begin{baitoan}[\cite{VMS_VMC2023}, p. 38, 5.4, ĐHGTVT]
	Tính
	\begin{equation*}
		I\coloneqq\int_{-\pi}^\pi \frac{\sin nx}{(1 + 2023^x)\sin x}\,{\rm d}x.
	\end{equation*}
\end{baitoan}

\begin{baitoan}[\cite{VMS_VMC2023}, p. 38, 5.5, ĐHGTVT]
	Cho hàm $f$ dương, khả tích trên $[a,b]$, $0 < m\le f(x)\le M$, $\forall x\in[a,b]$. Chứng minh
	\begin{equation*}
		(b - a)^2\le\int_a^b f(x)\,{\rm d}x\int_a^b \frac{{\rm dx}}{f(x)}\le\frac{(m + M)^2}{4mM}(b - a)^2.
	\end{equation*}
\end{baitoan}

\begin{baitoan}[\cite{VMS_VMC2023}, p. 39, 5.6, ĐHKH Thái Nguyên]
	Cho hàm $h\in C([0,1])$ thỏa $\int_0^1 xh(x)\,{\rm d}x = \int_0^1 h(x)\,{\rm d}x$. Chứng minh tồn tại $\beta\in(0,1)$ thỏa $\beta h(\beta^2) = \frac{2023}{2}\int_0^{\beta^2} h(x)\,{\rm d}x$.
\end{baitoan}

\begin{baitoan}[\cite{VMS_VMC2023}, p. 39, 5.7, ĐHKH Thái Nguyên]
	Cho $f\in C([0,\pi])$ thỏa $f(0) > 0$ \& $\int_0^\pi f(x)\,{\rm d}x < 2$. Chứng minh phương trình $f(x) = \sin x$ có ít nhất 1 nghiệm trong khoảng $(0,\pi)$.
\end{baitoan}

\begin{baitoan}[\cite{VMS_VMC2023}, p. 39, 5.8, ĐH Mỏ--Địa chất]
	Cho $f\in C([0,1]),g\in C([0,1],(0,\infty))$ với $f$ không giảm. Chứng minh
	\begin{equation*}
		\left(\int_0^t f(x)g(x)\,{\rm d}x\right)\left(\int_0^1 g(x)\,{\rm d}x\right)\le\left(\int_0^t g(x)\,{\rm d}x\right)\left(\int_0^1 f(x)g(x)\,{\rm d}x\right),\ \forall t\in[0,1].
	\end{equation*}
\end{baitoan}

\begin{baitoan}[\cite{VMS_VMC2023}, p. 39, 5.9, ĐH Mỏ--Địa chất]
	Cho $f\in C([0,1])$ thỏa $\int_0^1 f(x)\,{\rm d}x = 0$. Chứng minh tồn tại điểm $c\in(0,1)$ thỏa $\int_0^c xf(x)\,{\rm d}x = 0$.
\end{baitoan}

\begin{baitoan}[\cite{VMS_VMC2023}, p. 39, 5.10, ĐHSPHN2]
	Gọi ${\cal F}$ là lớp tất cả các hàm khả vi $f:\mathbb{R}\to(0,\infty)$ thỏa
	\begin{equation*}
		|f'(x) - f'(y)|\le2023|x - y|,\ \forall x,y\in\mathbb{R}.
	\end{equation*}
	Chứng minh
	\begin{equation*}
		(f'(x))^2 < 4046f(x),\ \forall x\in\mathbb{R}.
	\end{equation*}
\end{baitoan}

\begin{baitoan}[\cite{VMS_VMC2023}, p. 40, 5.11, ĐHSPHN2]
	Giả sử $f\in C^2([a,b])$ thỏa $f(a)\ne-f(b)$ \& $\int_a^b f(x)\,{\rm d}x = 0$. Tìm {\rm GTNN} của
	\begin{equation*}
		A\coloneqq\frac{(b - a)^3}{(f(a) + f(b))^2}\int_a^b (f''(x))^2\,{\rm d}x.
	\end{equation*}
\end{baitoan}

\begin{baitoan}[\cite{VMS_VMC2023}, p. 40, 5.12, ĐH Trà Vinh]
	Tính
	\begin{equation*}
		I\coloneqq\int_0^{2\pi} \ln(\sin x + \sqrt{1 + \sin^2x})\,{\rm d}x.
	\end{equation*}
\end{baitoan}

\begin{baitoan}[\cite{VMS_VMC2023}, p. 40, 5.12, ĐH Vinh]
	Cho $f\in C([0,1])$ thỏa $xf(y) + yf(x)\le1$, $\forall x,y\in[0,1]$. Chứng minh: (a) $f(x)\le\dfrac{1}{2x}$, $\forall x\in(0,1]$. (b) $\int_0^1 f(x)\,{\rm d}x\le\dfrac{\pi}{4}$.
\end{baitoan}

\begin{baitoan}[\cite{VMS_VMC2024}, p. 37, 5.1, VNUHCM UIT]
	Cho $\alpha\in(0,\infty)$ \& $f\in C([0,1])$ nghịch biến, $a\in(0,1)$ thỏa
	\begin{equation*}
		\int_0^a f(t)\,{\rm d}t < \frac{a}{2025},\ f(0) = \beta > 0.
	\end{equation*}
	Chứng minh phương trình $f(x) = x^{2024}$ có nghiệm trong $[0,1]$.
\end{baitoan}

\begin{baitoan}[\cite{VMS_VMC2024}, p. 38, 5.2, ĐH Đồng Tháp]
	Giả sử $f\in C^1([0,1])$ thỏa $f(0) = 0$, $0\le f'(x)\le1$, $\forall x\in[0,1]$. Xét hàm số
	\begin{equation*}
		F(t) = \left(\int_0^t f(x)\,{\rm d}x\right)^2 - \int_0^t (f(x))^3\,{\rm d}x,\ \forall t\in[0,1].
	\end{equation*}
	(a) Chứng minh $F$ đồng biến trên $[0,1]$. (b) Chứng minh
	\begin{equation*}
		\left(\int_0^1 f(x)\,{\rm d}x\right)^2\ge\int_0^1 (f(x))^3\,{\rm d}x.
	\end{equation*}
	Cho vài ví dụ về hàm $f$ để đẳng thức xảy ra.
\end{baitoan}

\begin{baitoan}[\cite{VMS_VMC2024}, p. 38, 5.3, ĐHGTVT]
	Cho $f:[0,1]\to(0,+\infty)$ là 1 hàm khả tích thỏa $f(x)f(1 - x) = 1$, $\forall x\in[0,1]$. Chứng minh $\int_0^1 f(x)\,{\rm d}x\ge1$.
\end{baitoan}

\begin{baitoan}[\cite{VMS_VMC2024}, p. 38, 5.4, HUS]
	Cho $f:[0,1]\to\mathbb{R}$ là hàm khả tích trên $[0,1]$ \& liên tục trên $(0,1)$. Chứng minh tồn tại $a,b\in(0,1)$ phân biệt sao cho
	\begin{equation*}
		\int_0^1 f(x)\,{\rm d}x = \frac{f(a) + f(b)}{2}.
	\end{equation*}
\end{baitoan}

\begin{baitoan}[\cite{VMS_VMC2024}, p. 38, 5.5, ĐH Mỏ--Địa chất]
	Tính tích phân
	\begin{equation*}
		\iiiint_{x^2 + y^2 + z^2 + t^2\le1} e^{x^2 + y^2 - z^2 - t^2}\,{\rm d}x\,{\rm d}y\,{\rm d}z\,{\rm d}t.
	\end{equation*}
\end{baitoan}

\begin{baitoan}[\cite{VMS_VMC2024}, p. 38, 5.6, ĐH Vinh]
	Chứng minh
	\begin{equation*}
		\frac{9}{8\pi} < \int_{\frac{\pi}{6}}^{\frac{\pi}{3}} \left(\frac{\sin x}{x}\right)^2\,{\rm d}x < \frac{3}{2\pi}.
	\end{equation*}
\end{baitoan}

%------------------------------------------------------------------------------%

\section{SymPy{\tt/integrals} module}
See \url{https://docs.sympy.org/latest/modules/integrals/integrals.html}. The {\tt integrals} module in {\tt SymPy} implements methods to calculate definite \& indefinite integrals of expressions. Principal method in this module is {\tt integrate()}:
\begin{itemize}
	\item {\tt integrate(f, x)} returns the indefinite integral $\int f\,{\rm d}x$
	\item {\tt integrate(f, (x, a, v))} returns the definite integral $\int_a^b f\,{\rm d}x$.
\end{itemize}

\begin{problem}[Integration of elementary functions]
	Use {\tt SymPy} to compute definite- \& indefinite integrals of elementary functions as many as possible.
\end{problem}

\begin{problem}[Integration of nonelementary functions]
	Use {\tt SymPy} to compute definite- \& indefinite integrals of nonelementary functions as many as possible.
\end{problem}

\begin{example}[Integral of error function]
	The indefinite integral of the nonelementary function $e^{-x^2}{\rm erf}(x)$, where ${\rm erf}(x)$ is the {\rm error function}, is given by
	\begin{equation*}
		\int e^{-x^2}{\rm erf}(x)\,{\rm d}x = \frac{\sqrt{\pi}}{4}{\rm erf}(x).
	\end{equation*}
\end{example}
Run the following Python code:
\begin{verbatim}
	from sympy import *
	x = Symbol('x')
	print(integrate(exp(-x**2)*erf(x), x))
\end{verbatim}
to obtain the following output:
\begin{verbatim}
	sqrt(pi)*erf(x)**2/4
\end{verbatim}
For more information about the error function, see, e.g., \href{https://en.wikipedia.org/wiki/Error_function}{Wikipedia{\tt/}error function}.

%------------------------------------------------------------------------------%

\section{Leibniz integral rule -- Quy tắc tích phân Leibniz}
In \href{https://en.wikipedia.org/wiki/Calculus}{calculus}, the {\it Leibniz integral rule} for differentiation under the integral sign, named after \href{https://en.wikipedia.org/wiki/Gottfried_Wilhelm_Leibniz}{\sc Gottfried Wilhelm Leibniz}.

\begin{theorem}[Leibniz integral rule -- Quy tắc tích phân Leibniz]
	For an integral of the form $\int_{a(x)}^{b(x)} f(t,x)\,{\rm d}t$ where $a(x),b(x)\in\mathbb{R}$ \& the integrands are functions dependent on $x$, the derivative of this integral is expressible as
	\begin{equation}
		\label{Leibniz integral rule}
		\tag{Lintr}
		\frac{d}{dx}\left(\int_{a(x)}^{b(x)} f(t,x)\,{\rm d}t\right) = f(b(x),x)\frac{d}{dx}b(x) - f(a(x),x)\frac{d}{dx}a(x) + \int_{a(x)}^{b(x)} \partial_xf(t,x)\,{\rm d}t,
	\end{equation}
	where the \href{https://en.wikipedia.org/wiki/Partial_derivative}{partial derivative} $\partial_x = \frac{\partial}{\partial x}$ indicates that inside the integral, only the variation of $f(t,x)$ with $x$ is considered in taking the derivative.
\end{theorem}
Cho tiện, ký hiệu
\begin{align*}
    I(f(x),a,b)&\coloneqq\int_a^b f(x)\,{\rm d}x,\ \forall f\in{\rm Riemann}([a,b]),\ \forall a,b\in\mathbb{R},\\
    I(f(t,x),a(x),b(x))&\coloneqq\int_{a(x)}^{b(x)} f(t,x)\,{\rm d}t,\ \forall f\mbox{ s.t. } f(\cdot,x)\in{\rm Riemann}([a,b]),\ \forall a,b\in C^1(\mathbb{R}),\\
\end{align*}

\begin{baitoan}
    Sử dụng quy tắc tích phân Leibniz, tính đạo hàm theo biến $x$ của tích phân:
    \begin{align*}
        I\left(\sum_{i=1}^n f_i(t,x),a(x),b(x)\right)&\coloneqq\int_{a(x)}^{b(x)} \sum_{i=1}^n f_i(t,x)\,{\rm d}t,\\
        I\left(f(t,x),\sum_{i=1}^m a_i(x),\sum_{i=1}^n b_i(x)\right)&\coloneqq\int_{\sum_{i=1}^m a_i(x)}^{\sum_{i=1}^n b_i(x)} f(t,x)\,{\rm d}t,\\
        I\left(\sum_{i=1}^p f_i(t,x),\sum_{i=1}^m a_i(x),\sum_{i=1}^n b_i(x)\right)&\coloneqq\int_{\sum_{i=1}^m a_i(x)}^{\sum_{i=1}^n b_i(x)} \sum_{i=1}^p f_i(t,x)\,{\rm d}t,\\
        I(f(t,x),I(A(x,y),a(x),b(x)),I(B(x,y),c(x),d(x)))&\coloneqq\int_{\int_{a(x)}^{b(x)} A(x,y)\,{\rm d}y}^{\int_{c(x)}^{d(x)} B(x,y)\,{\rm d}y} f(t,x)\,{\rm d}t.
    \end{align*}
    Cho dãy hàm $\{a_n\}_{n=1}^\infty,\{b_n\}_{n=1}^\infty,\{f_n\}_{n=1}^\infty\subset C_t^\infty C_x^\infty$ tính đạo hàm của tích phân lặp chồng chất:
    \begin{equation*}
        I_n(f_n(t,x),I_{n-1}(f_{n-1}(t,x),I_{n-2}(f_{n-2}(t,x),\ldots)).)***
    \end{equation*}
\end{baitoan}

%------------------------------------------------------------------------------%

\section{Numerical Integration{\tt/}Quadrature -- Xấp Xỉ Tích Phân}
\textbf{\textsf{Resources -- Tài nguyên.}}
\begin{enumerate}
    \item \href{https://en.wikipedia.org/wiki/Numerical_integration}{Wikipedia{\tt/}numerical integration}.
    
    \item \cite{Scheid1989}. {\sc Francis Scheid}. {\it Schaum's Outline of Numerical Analysis}. Chap. 14: Numerical Integration.
\end{enumerate}
In numerical analysis, {\it numerical integration} comprises a broad family of algorithms for calculating the numerical values of a definite integral. The term {\it numerical quadrature} (often abbreviated to {\it quadrature}) is more or less a synonym for ``numerical integration'', especially as applied to 1D integrals. Some authors refer to numerical integration over $> 1$ dimension as {\it cubature}; others take ``quadrature'' to include higher-dimensional integration.

-- Trong phân tích số, {\it tích phân số} bao gồm 1 họ rộng các thuật toán để tính toán các giá trị số của 1 tích phân xác định. Thuật ngữ {\it tích phân số} (thường được viết tắt là {\it quadrature}) ít nhiều là từ đồng nghĩa với ``tích phân số'', đặc biệt khi áp dụng cho tích phân 1 chiều. Một số tác giả gọi tích phân số trên $> 1$ chiều là {\it cubature}; những tác giả khác coi ``tích phân'' bao gồm tích phân chiều cao hơn.

\begin{problem}[Approximate indefinite integrals, R]
    Find a way to approximate indefinite integral via limit.
    
    -- Tìm cách tính gần đúng tích phân không xác định thông qua giới hạn.
\end{problem}
The basic problem in numerical integration is to compute an approximate solution to a definite integral $\int_a^b f(x)\,{\rm d}x$ to a given degree of accuracy. If $f(x)$ is a smooth function integrated over a small number of dimensions, \& the domain of integration is bounded, there are many methods for approximating the integral to the desired precision.

-- Vấn đề cơ bản trong tích phân số là tính toán 1 giải pháp gần đúng cho 1 tích phân xác định $\int_a^b f(x)\,{\rm d}x$ với 1 mức độ chính xác nhất định. Nếu $f(x)$ là 1 hàm trơn tích phân trên 1 số lượng nhỏ chiều, \& miền tích phân bị chặn, thì có nhiều phương pháp để xấp xỉ tích phân với độ chính xác mong muốn.

Numerical integration has roots in the geometrical problem of finding a square with the same area as a given plane figure (\href{https://en.wikipedia.org/wiki/Quadrature_(geometry)}{\it quadrature} or {\it squaring}), as in the \href{https://en.wikipedia.org/wiki/Quadrature_of_the_circle}{quadrature of the circle}. The term is also sometimes used to describe the \href{https://en.wikipedia.org/wiki/Numerical_ordinary_differential_equations}{numerical solution of differential equations}.

-- Tích phân số có nguồn gốc từ bài toán hình học tìm 1 hình vuông có cùng diện tích với 1 hình phẳng cho trước ({\it quadrature} hoặc {\it squarering}), như trong tích phân của hình tròn. Thuật ngữ này đôi khi cũng được dùng để mô tả giải pháp số của các phương trình vi phân.

%------------------------------------------------------------------------------%

\subsection{Motivation \& need of numerical integration -- Động lực \& nhu cầu xấp xỉ tích phân}
There are several reasons for carrying out numerical integration, as opposed to analytical integration by finding the antiderivative:
\begin{enumerate}
    \item The integrand $f(x)$ may be known only at certain points, e.g. obtained by sampling. Some embedded systems \& other computer applications may need numerical integration for this reason.
    \item A formula for the integrand may be known, but it may be difficult or impossible to find an antiderivative that is an elementary function, e.g., $f(x) = e^{-x^2}$, the antiderivative of which (the error function, times a constant) cannot be written in elementary form.
    \item It may be possible to find an antiderivative symbolically, but it may be easier to compute a numerical approximation than to compute the antiderivative. That may be the case if the antiderivative is given as an infinite series or product, or if its evaluation requires a special function that is not available.
\end{enumerate}
-- Có 1 số lý do để thực hiện tích phân số, trái ngược với tích phân phân tích bằng cách tìm nguyên hàm:
\begin{enumerate}
    \item Tích phân $f(x)$ chỉ có thể được biết tại 1 số điểm nhất định, ví dụ như thu được bằng cách lấy mẫu. Một số hệ thống nhúng \& các ứng dụng máy tính khác có thể cần tích phân số vì lý do này.
    \item Có thể biết công thức của tích phân, nhưng có thể khó hoặc không thể tìm được nguyên hàm là hàm cơ bản, ví dụ, $f(x) = e^{-x^2}$, nguyên hàm của hàm này (hàm lỗi, nhân với 1 hằng số) không thể viết ở dạng cơ bản.
    \item Có thể tìm được nguyên hàm theo ký hiệu, nhưng có thể dễ tính toán xấp xỉ số hơn là tính nguyên hàm. Trường hợp đó có thể xảy ra nếu nguyên hàm được đưa ra dưới dạng chuỗi hoặc tích vô hạn, hoặc nếu việc đánh giá của nó yêu cầu 1 hàm đặc biệt không có sẵn.
\end{enumerate}

%------------------------------------------------------------------------------%

\subsection{Methods for 1D integrals -- Các phương pháp cho tích phân 1 chiều}
A {\it quadrature rule} is an approximation of the definite integral of a function, usually stated as a \href{https://en.wikipedia.org/wiki/Weighted_sum}{weighted sum} of function values at specified points within the domain of integration.

-- Quy tắc tích phân là 1 phép tính gần đúng của tích phân xác định của 1 hàm, thường được biểu thị dưới dạng tổng có trọng số của các giá trị hàm tại các điểm xác định trong miền tích phân.

Numerical integration methods can generally be described as combining evaluations of the integrand to get an approximation to the integral. The integrand is evaluated at a finite set of points called {\it integration points} \& a weighted sum of these values is used to approximate the integral. The integrand points \& weights depend on the specific method used \& the accuracy required from the approximation. 

-- Các phương pháp tích phân số thường có thể được mô tả như là kết hợp các đánh giá của tích phân để có được 1 phép tính gần đúng của tích phân. Tích phân được đánh giá tại 1 tập hợp hữu hạn các điểm được gọi là {\it integration points} \& tổng có trọng số của các giá trị này được sử dụng để xấp xỉ tích phân. Các điểm tích phân \& trọng số phụ thuộc vào phương pháp cụ thể được sử dụng \& độ chính xác cần thiết từ phép tính gần đúng.

An important part of the analysis of any numerical integration method is to study the behavior of the approximation error as a function of the number of integrand evaluations. A method that yields a small error for a small number of evaluations is usually considered superior. Reducing the number of evaluations of the integrand reduces the number of arithmetic operations involved, \& therefore reduces the total error. Also, each evaluation takes time, \& the integrand may be arbitrarily complicated.

-- 1 phần quan trọng trong phân tích của bất kỳ phương pháp tích phân số nào là nghiên cứu hành vi của lỗi xấp xỉ như 1 hàm của số lượng đánh giá tích phân. Một phương pháp tạo ra lỗi nhỏ cho 1 số lượng nhỏ các đánh giá thường được coi là vượt trội. Giảm số lượng đánh giá của tích phân sẽ giảm số lượng các phép toán số học liên quan, \& do đó giảm tổng lỗi. Ngoài ra, mỗi đánh giá đều mất thời gian, \& tích phân có thể phức tạp tùy ý.

%------------------------------------------------------------------------------%

\subsubsection{Quadrature rules based on step functions -- Các quy tắc xấp xỉ tích phân dựa trên hàm bước nhảy}
A ``brute force'' kind of numerical integration can be done, if the integrand is reasonably well-behaved (i.e., piecewise continuous \& \href{https://en.wikipedia.org/wiki/Bounded_variation}{bounded variation}), by evaluating the integrand with very small increments.

-- Có thể thực hiện tích phân số theo kiểu ``thô bạo'', nếu tích phân có hành vi khá tốt (tức là liên tục từng phần \& biến thiên bị chặn), bằng cách đánh giá tích phân với các gia số rất nhỏ.

This simplest method approximates the function by a \href{https://en.wikipedia.org/wiki/Step_function}{step function} (a piecewise constant function, or a segmented polynomial of degree 0) that passes through the point $\left(\frac{a + b}{2},f\left(\frac{a + b}{2}\right)\right)$. This is called the {\it midpoint rule} or \href{https://en.wikipedia.org/wiki/Rectangle_method}{\it rectangle rule}
\begin{equation*}
    \int_a^b f(x)\,{\rm d}x\approx(b - a)f\left(\frac{a + b}{2}\right).
\end{equation*}
-- Phương pháp đơn giản nhất này xấp xỉ hàm bằng 1 hàm bước (một hàm hằng từng phần hoặc 1 đa thức phân đoạn bậc 0) đi qua điểm $\left(\frac{a + b}{2},f\left(\frac{a + b}{2}\right)\right)$. Phương pháp này được gọi là {\it quy tắc điểm giữa} hoặc {\it quy tắc hình chữ nhật}
\begin{equation*}
    \int_a^b f(x)\,{\rm d}x\approx(b - a)f\left(\frac{a + b}{2}\right).
\end{equation*}

%------------------------------------------------------------------------------%

\subsubsection{Quadrature rules based on interpolating functions-- Các quy tắc xấp xỉ tích phân dựa trên các hàm nội suy}
A large class of quadrature rules can be derived by constructing interpolating functions that are easy to integrate. Typically these interpolating functions are polynomials. In practice, since polynomials of very high degree tend to oscillate wildly, only polynomials of low degree are used, typically linear \& quadratic.

-- 1 lớp lớn các quy tắc tích phân có thể được suy ra bằng cách xây dựng các hàm nội suy dễ tích phân. Thông thường các hàm nội suy này là các đa thức. Trong thực tế, vì các đa thức có bậc rất cao có xu hướng dao động mạnh, nên chỉ các đa thức có bậc thấp mới được sử dụng, thường là tuyến tính \& bậc hai.

The interpolating function may be a straight line (an affine function, i.e., a polynomial of degree 1) passing through the points $(a,f(a)),(b,f(b))$. This is called the {\it trapezoidal rule}
\begin{equation*}
    \int_a^b f(x)\,{\rm d}x\approx(b - a)\frac{f(a) + f(b)}{2}.
\end{equation*}
-- Hàm nội suy có thể là 1 đường thẳng (một hàm afin, tức là 1 đa thức bậc 1) đi qua các điểm $(a,f(a)),(b,f(b))$. Đây được gọi là {\it trapezoidal rule}
\begin{equation*}
    \int_a^b f(x)\,{\rm d}x\approx(b - a)\frac{f(a) + f(b)}{2}.
\end{equation*}
For either 1 of these rules, we can make a more accurate approximation by breaking up the interval $[a,b]$ into some number $n\in\mathbb{N},n\ge2$ of subintervals, computing an approximation for each subinterval, then adding up all the results. This is called a {\it composite rule, extended rule}, or {\it iterated rule}. E.g., the composite trapezoidal rule can be stated as
\begin{equation*}
    \int_a^b f(x)\,{\rm d}x\approx\frac{b - a}{n}\left(\frac{f(a)}{2} + \sum_{i=1}^{n-1} f\left(a + i\frac{b - a}{n}\right) + \frac{f(b)}{2}\right),
\end{equation*}
where the subintervals have the form $[a + ih,a + (i + 1)h]\subset[a,b]$, with $h\coloneqq\frac{b - a}{n}$ \& $i = 0,1,\ldots,n - 1$. Here we used subintervals of the same length $h$ but one could also use intervals of varying length $\{h_i\}_{i=0}^{n-1}$.

-- Đối với bất kỳ 1 trong những quy tắc này, chúng ta có thể thực hiện phép xấp xỉ chính xác hơn bằng cách chia khoảng $[a,b]$ thành 1 số $n\in\mathbb{N},n\ge2$ các khoảng con, tính toán phép xấp xỉ cho mỗi khoảng con, sau đó cộng tất cả các kết quả lại. Đây được gọi là {\it quy tắc hợp thành, quy tắc mở rộng} hoặc {\it quy tắc lặp lại}. Ví dụ, quy tắc hình thang hợp thành có thể được phát biểu như sau
\begin{equation*}
    \int_a^b f(x)\,{\rm d}x\approx\frac{b - a}{n}\left(\frac{f(a)}{2} + \sum_{i=1}^{n-1} f\left(a + i\frac{b - a}{n}\right) + \frac{f(b)}{2}\right),
\end{equation*}
trong đó các khoảng con có dạng $[a + ih,a + (i + 1)h]\subset[a,b]$, với $h\coloneqq\frac{b - a}{n}$ \& $i = 0,1,\ldots,n - 1$. Ở đây chúng ta sử dụng các khoảng con có cùng độ dài $h$ nhưng người ta cũng có thể sử dụng các khoảng có độ dài thay đổi $\{h_i\}_{i=0}^{n-1}$.

Interpolation with polynomials evaluated at equally spaced points in $[a,b]$ yields the \href{https://en.wikipedia.org/wiki/Newton%E2%80%93Cotes_formulas}{Newton--Cotes formulas}, of which the rectangle rule \& the trapezoidal rule are examples. \href{https://en.wikipedia.org/wiki/Simpson%27s_rule}{Simpson's rule}, which is based on a polynomial of order 2, is also a Newton--Cotes formula.

-- Nội suy với các đa thức được đánh giá tại các điểm cách đều nhau trong $[a,b]$ tạo ra các công thức Newton-Cotes, trong đó quy tắc hình chữ nhật \& quy tắc hình thang là các ví dụ. Quy tắc Simpson, dựa trên đa thức bậc 2, cũng là 1 công thức Newton-Cotes.

Quadrature rules with equally spaced points have very convenient property of {\it nesting}. The correspodning rule with each interval subdivided includes all the current points, so these integrand values can be re-used.

-- Quy tắc tích phân với các điểm cách đều nhau có tính chất rất thuận tiện là {\it lồng nhau}. Quy tắc tương ứng với mỗi khoảng chia nhỏ bao gồm tất cả các điểm hiện tại, do đó các giá trị tích phân này có thể được sử dụng lại.

If we allow the intervals between interpolation points to vary, we find another group of quadrature formulas, e.g. the \href{https://en.wikipedia.org/wiki/Gaussian_quadrature}{Gaussian quadrature formulas}. A Gaussian quadrature rule is typically more accurate than a Newton--Cotes rule that uses the same number of function evaluations, if the integrand is smooth  (i.e., if it is sufficiently differentiable). Other quadrature methods with varying intervals include \href{https://en.wikipedia.org/wiki/Clenshaw%E2%80%93Curtis_quadrature}{Clenshaw--Curtis quadrature} (also called Fej\'er quadrature) methods, which do nest.

-- Nếu chúng ta cho phép các khoảng giữa các điểm nội suy thay đổi, chúng ta sẽ tìm thấy 1 nhóm công thức tích phân khác, ví dụ như công thức tích phân Gauss. Quy tắc tích phân Gauss thường chính xác hơn quy tắc Newton-Cotes sử dụng cùng số lượng đánh giá hàm, nếu tích phân là trơn (tức là nếu nó đủ khả vi). Các phương pháp tích phân khác với các khoảng thay đổi bao gồm phương pháp tích phân Clenshaw-Curtis (còn gọi là phương pháp tích phân Fej\'er), có lồng nhau.

Gaussian quadrature rules do not nest, but the related Gauss--Kronrod quadrature formulas do.

-- Các quy tắc tích phân Gauss không lồng nhau, nhưng các công thức tích phân Gauss-Kronrod liên quan thì có.

%------------------------------------------------------------------------------%

\subsubsection{Adaptive algorithms}
\href{https://en.wikipedia.org/wiki/Adaptive_quadrature}{Adaptive quadrature} is a numerical integration method in which the integrand of a function $f(x)$ is approximated using static quadrature rules on adaptively refined subintervals of the region of integration. Generally, adaptive algorithms are just as efficient \& effective as traditional algorithms for ``well behaved'' integrands, but are also effective for ``badly behaved'' integrands for which traditional algorithms may fail.

-- Tích phân thích nghi là 1 phương pháp tích phân số trong đó tích phân của 1 hàm $f(x)$ được xấp xỉ bằng cách sử dụng các quy tắc tích phân tĩnh trên các khoảng con được tinh chỉnh thích nghi của vùng tích phân. Nhìn chung, các thuật toán thích nghi cũng hiệu quả \& hiệu quả như các thuật toán truyền thống đối với các tích phân ``hoạt động tốt'', nhưng cũng hiệu quả đối với các tích phân ``hoạt động kém'' mà các thuật toán truyền thống có thể không thành công.

For more information, see, e.g., \href{https://en.wikipedia.org/wiki/Adaptive_quadrature}{Wikipedia{\tt/}adaptive quadrature}.

%------------------------------------------------------------------------------%

\subsubsection{Extrapolation methods -- Các phương pháp ngoại suy}
The accuracy of a quadrature rule of the Newton--Cotes type is generally a function of the number of evaluation points. The result is usually more accurate as the number of evaluation points increases, or, equivalently, as the width of the step size between the points decreases. It is natural to ask what the result would be if the step size were allowed to approach 0. This can be answered by extrapolating the result from $\ge2$ nonzero step sizes, using \href{https://en.wikipedia.org/wiki/Series_acceleration}{series acceleration} methods e.g. Richardson extrapolation. The extrapolation function may be a polynomial or rational function. Extrapolation methods are implemented in many of the routines in the \href{https://en.wikipedia.org/wiki/QUADPACK}{QUADPACK} library.

-- Độ chính xác của quy tắc tích phân theo kiểu Newton-Cotes thường là 1 hàm của số điểm đánh giá. Kết quả thường chính xác hơn khi số điểm đánh giá tăng lên hoặc tương đương, khi chiều rộng của kích thước bước giữa các điểm giảm xuống. Thật tự nhiên khi hỏi kết quả sẽ như thế nào nếu kích thước bước được phép tiến tới 0. Câu hỏi này có thể được trả lời bằng cách ngoại suy kết quả từ $\ge2$ kích thước bước khác không, sử dụng các phương pháp tăng tốc chuỗi, ví dụ như ngoại suy Richardson. Hàm ngoại suy có thể là hàm đa thức hoặc hàm hữu tỉ. Các phương pháp ngoại suy được triển khai trong nhiều chương trình con trong thư viện QUADPACK.

%------------------------------------------------------------------------------%

\subsubsection{Conservative (a priori) error estimate -- Ước tính tiên nghiệm sai số bảo toàn }
Let $f$ have a bounded 1st derivative over $[a,b]$, i.e., $f\in C^1([a,b])$. The mean value theorem for $f$, where $x\in[a,b)$, gives $(x - a)f'(\xi_x) = f(b) - f(a)$, for some $\xi_x\in(a,x]$ depending on $x$. If we integrate in $x$ from $a$ to $b$ on both sides \& take the absolute values \& then further approximate the integral on the RHS of obtained equality by bringing the absolute value into the integrand, \& replacing the term in $f'$ by an upper bound, we obtain
\begin{equation*}
    \left|\int_a^b f(x)\,{\rm d}x - (b - a)f(a)\right| = \left|\int_a^b (x - a)f'(\xi_x)\,{\rm d}x\right|\le\frac{(b - a)^2}{2}\sup_{x\in[a,b]} |f'(x)|,
\end{equation*}
where the supremum was used to approximate. Hence, if we approximate the integral $\int_a^b f(x)\,{\rm d}x$ by the quadrature rule $(b - a)f(a)$ our error is no greater than the last RHS. We an convert this into an error analysis for the \href{https://en.wikipedia.org/wiki/Riemann_sum#Definition}{Riemann sum}, giving an upper bound of $\frac{1}{2n}\sup_{x\in[0,1]} |f'(x)|$ for the error term of that particular approximation. Note that this is precisely the error we calculated for the example $f(x) = x$. Using more derivatives, \& by tweaking the quadrature, we can do a similar error analysis using a Taylor series (using a partial sum with remainder term) for $f$. This error analysis gives a strict upper bound on the error, if the derivatives of $f$ are available.

-- Cho $f$ có đạo hàm bậc 1 bị chặn trên $[a,b]$, tức là $f\in C^1([a,b])$. Định lý giá trị trung bình cho $f$, trong đó $x\in[a,b)$, đưa ra $(x - a)f'(\xi_x) = f(b) - f(a)$, đối với 1 số $\xi_x\in(a,x]$ phụ thuộc vào $x$. Nếu chúng ta tích phân trong $x$ từ $a$ đến $b$ ở cả hai vế \& lấy các giá trị tuyệt đối \& sau đó tiếp tục xấp xỉ tích phân trên Vế phải của phép tính bằng nhau thu được bằng cách đưa giá trị tuyệt đối vào tích phân, \& thay thế số hạng trong $f'$ bằng 1 giới hạn trên, chúng ta thu được
\begin{equation*}
    \left|\int_a^b f(x)\,{\rm d}x - (b - a)f(a)\right| = \left|\int_a^b (x - a)f'(\xi_x)\,{\rm d}x\right|\le\frac{(b - a)^2}{2}\sup_{x\in[a,b]} |f'(x)|,
\end{equation*}
trong đó cực đại được sử dụng để xấp xỉ. Do đó, nếu chúng ta xấp xỉ tích phân $\int_a^b f(x)\,{\rm d}x$ theo quy tắc tích phân $(b - a)f(a)$ thì lỗi của chúng ta không lớn hơn RHS cuối cùng. Chúng ta có thể chuyển đổi điều này thành 1 phân tích lỗi cho tổng Riemann, đưa ra 1 giới hạn trên của $\frac{1}{2n}\sup_{x\in[0,1]} |f'(x)|$ cho số hạng lỗi của phép xấp xỉ cụ thể đó. Lưu ý rằng đây chính xác là lỗi mà chúng ta đã tính toán cho ví dụ $f(x) = x$. Sử dụng nhiều đạo hàm hơn, \& bằng cách điều chỉnh tích phân, chúng ta có thể thực hiện 1 phân tích lỗi tương tự bằng cách sử dụng chuỗi Taylor (sử dụng tổng 1 phần với số hạng còn lại) cho $f$. Phân tích lỗi này đưa ra 1 giới hạn trên nghiêm ngặt cho lỗi, nếu các đạo hàm của $f$ có sẵn.

This integration method can be combined with \href{https://en.wikipedia.org/wiki/Interval_arithmetic}{interval arithmetic} to produce \href{https://en.wikipedia.org/wiki/Computer_proof}{computer proofs} \& {\it verified} calculations. 

-- Phương pháp tích hợp này có thể kết hợp với phép tính khoảng để đưa ra bằng chứng máy tính \& các phép tính đã được xác minh.

%------------------------------------------------------------------------------%

\subsubsection{Integrals over infinite intervals -- Tích phân trên các đoạn dài vô hạn}
Several methods exist for approximate integration over unbounded intervals. The standard technique involves specially derived quadrature rules, e.g., Gauss--Hermite quadrature (for integrals on the whole real line $\mathbb{R}$ \& Gauss--Laguerre quadrature)for integrals on the positive reals $(0,\infty)$. Monte Carlo methods can also be used, or a change of variables to a finite interval; e.g., for the whole line one could use
\begin{equation*}
    \int_{-\infty}^\infty f(x)\,{\rm d}x = \int_{-1}^1 f\left(\frac{t}{1 - t^2}\right)\frac{1 + t^2}{(1 - t^2)^2}\,{\rm d}t,
\end{equation*}
\& for semi-infinite intervals one could use
\begin{align*}
    \int_a^\infty f(x)\,{\rm d}x &= \int_0^1 f\left(a + \frac{t}{1 - t}\right)\frac{{\rm d}t}{(1 - t)^2},\\
    \int_{-\infty}^a f(x)\,{\rm d}x &= \int_0^1 f\left(a - \frac{1 - t}{t}\right)\frac{{\rm d}t}{t^2},
\end{align*}
as possible transformations.

-- Có 1 số phương pháp để tích phân xấp xỉ trên các khoảng không giới hạn. Kỹ thuật chuẩn liên quan đến các quy tắc tích phân được suy ra đặc biệt, ví dụ, tích phân Gauss--Hermite (đối với tích phân trên toàn bộ đường thẳng thực $\mathbb{R}$ \& tích phân Gauss--Laguerre) đối với tích phân trên các số thực dương $(0,\infty)$. Các phương pháp Monte Carlo cũng có thể được sử dụng, hoặc thay đổi các biến thành 1 khoảng hữu hạn; ví dụ, đối với toàn bộ đường thẳng, ta có thể sử dụng
\begin{equation*}
    \int_{-\infty}^\infty f(x)\,{\rm d}x = \int_{-1}^1 f\left(\frac{t}{1 - t^2}\right)\frac{1 + t^2}{(1 - t^2)^2}\,{\rm d}t,
\end{equation*}
\& đối với các khoảng bán vô hạn, ta có thể sử dụng
\begin{align*}
    \int_a^\infty f(x)\,{\rm d}x &= \int_0^1 f\left(a + \frac{t}{1 - t}\right)\frac{{\rm d}t}{(1 - t)^2},\\
    \int_{-\infty}^a f(x)\,{\rm d}x &= \int_0^1 f\left(a - \frac{1 - t}{t}\right)\frac{{\rm d}t}{t^2},
\end{align*}
là các phép biến đổi có thể.

%------------------------------------------------------------------------------%

\subsection{Multidimensional integrals -- Tích phân nhiều chiều}
The quadrature rules discussed so far are all designed to compute 1D integrals. To compute integrals in multiple dimensions, 1 approach is to phrase the multiple integral as repeated 1D integrals by applying \href{https://en.wikipedia.org/wiki/Fubini%27s_theorem}{Fubibi's theorem} (the tensor product rule). This approach requires the function evaluations to grow exponentially as the number of dimensions increases. 3 methods are known to overcome this so-called \href{https://en.wikipedia.org/wiki/Curse_of_dimensionality}{\it curse of dimensionality}.

-- Các quy tắc tích phân vuông góc đã thảo luận cho đến nay đều được thiết kế để tính tích phân 1 chiều. Để tính tích phân trong nhiều chiều, 1 cách tiếp cận là diễn đạt tích phân bội thành tích phân 1 chiều lặp lại bằng cách áp dụng định lý Fubibi (quy tắc tích tenxơ). Cách tiếp cận này yêu cầu các đánh giá hàm tăng theo cấp số nhân khi số chiều tăng lên. Có 3 phương pháp được biết là có thể khắc phục cái gọi là {\it lời nguyền về chiều} này.

A great many additional techniques for forming multidimensional cubature integration rules for a variety of weighting functions are given in the monograph by Stroud. Integration on the sphere has been reviewed by Hesse et al. (2015).

%------------------------------------------------------------------------------%

\subsubsection{Monte Carlo integration}
\href{https://en.wikipedia.org/wiki/Monte_Carlo_method}{Monte Carlo methods} \& \href{https://en.wikipedia.org/wiki/Quasi-Monte_Carlo_method}{quasi-Monte Carlo methods} are easy to apply to multidimensional integrals. They may yield greater accuracy for the same number of function evaluations than repeated integrations using 1D methods.

-- Phương pháp Monte Carlo \& phương pháp Monte Carlo gần đúng dễ áp dụng cho tích phân đa chiều. Chúng có thể mang lại độ chính xác cao hơn cho cùng số lượng đánh giá hàm so với tích phân lặp lại sử dụng phương pháp 1D.

A large class of useful Monte Carlo methods are the so-called \href{https://en.wikipedia.org/wiki/Markov_chain_Monte_Carlo}{Markov chain Monte Carlo} algorithms, which include the \href{https://en.wikipedia.org/wiki/Metropolis%E2%80%93Hastings_algorithm}{Metropolis--Hastings algorithm} \& \href{https://en.wikipedia.org/wiki/Gibbs_sampling}{Gibbs sampling}.

-- 1 lớp lớn các phương pháp Monte Carlo hữu ích là các thuật toán Monte Carlo chuỗi Markov, bao gồm thuật toán Metropolis--Hastings \& lấy mẫu Gibbs.

%------------------------------------------------------------------------------%

\subsubsection{Sparse grids}
\href{https://en.wikipedia.org/wiki/Sparse_grid}{Sparse grids} were originally developed by {\sc Smolyak} for the quadrature of high-dimensional functions. The method is always based on a 1D quadrature rule, but performs a more sophisticated combination of univariate results. However, whereas the tensor product rule guarantees that the weights of all of the cubature points will be positive if the weights of the quadrature points were positive, Smolyak's rule does not guarantee that the weights will all be positive.

-- Lưới thưa ban đầu được phát triển bởi {\sc Smolyak} cho tích phân của các hàm số chiều cao. Phương pháp này luôn dựa trên quy tắc tích phân 1D, nhưng thực hiện kết hợp tinh vi hơn các kết quả đơn biến. Tuy nhiên, trong khi quy tắc tích tenxơ đảm bảo rằng trọng số của tất cả các điểm tích phân sẽ dương nếu trọng số của các điểm tích phân là dương, quy tắc Smolyak không đảm bảo rằng tất cả các trọng số sẽ dương.

%------------------------------------------------------------------------------%

\subsubsection{Bayesian quadrature}
\href{https://en.wikipedia.org/wiki/Bayesian_quadrature}{Bayesian quadrature} is a statistical approach to the numerical problem of computing integrals \& falls under the field of \href{https://en.wikipedia.org/wiki/Probabilistic_numerics}{probabilistic numerics}. It can provide a full handling of the uncertainty over the solution of the integral expressed as a \href{https://en.wikipedia.org/wiki/Gaussian_process}{Gaussian process} posterior variance.

-- Bayesian quadrature là 1 phương pháp thống kê đối với bài toán số về tính tích phân \& thuộc lĩnh vực số học xác suất. Nó có thể cung cấp khả năng xử lý đầy đủ sự không chắc chắn đối với giải pháp của tích phân được biểu thị dưới dạng phương sai sau của quá trình Gaussian.

%------------------------------------------------------------------------------%

\subsection{Connection with differential equations -- Kết nối với phương trình vi phân}
The problem of evaluating the definite integral $F(x) = \int_a^x f(t)\,{\rm d}t$ can be reduced to an initial value problem (IVP) for an ordinary differential equation (ODE) by applying the 1st part of the fundamental theorem of calculus. By differentiating both sides of the above w.r.t. the argument $x$, it is seen that the function $F$ satisfies
\begin{equation*}
    \frac{{\rm d}F(x)}{{\rm d}x} = f(x),\ F(a) = 0.
\end{equation*}
\href{https://en.wikipedia.org/wiki/Numerical_methods_for_ordinary_differential_equations}{Numerical methods for ODEs}, e.g. \href{https://en.wikipedia.org/wiki/Runge%E2%80%93Kutta_methods}{Runge--Kutta methods}, can be applied to the restated problem \& thus be used to evaluate the integral. E.g., the standard 4th-order Runge--Kutta method applied to the differential equation yields Simpson's rule from above.

-- Bài toán đánh giá tích phân xác định $F(x) = \int_a^x f(t)\,{\rm d}t$ có thể được rút gọn thành bài toán giá trị ban đầu (IVP) cho phương trình vi phân thường (ODE) bằng cách áp dụng phần 1 của định lý cơ bản của phép tính vi phân. Bằng cách vi phân cả hai vế của phương trình trên đối với đối số $x$, ta thấy rằng hàm $F$ thỏa mãn
\begin{equation*}
    \frac{{\rm d}F(x)}{{\rm d}x} = f(x),\ F(a) = 0.
\end{equation*}
Các phương pháp số cho ODE, ví dụ như phương pháp Runge--Kutta, có thể được áp dụng cho bài toán được phát biểu lại \& do đó có thể được sử dụng để đánh giá tích phân. Ví dụ, phương pháp Runge--Kutta bậc 4 chuẩn được áp dụng cho phương trình vi phân sẽ đưa ra quy tắc Simpson.

The differential equation $F'(x) = f(x)$ has a special form: the RHS contains only the independent variable (here $x$) \& not dependent variable (here $F$). This simplifies the theory \& algorithms considerably. The problem of evaluating integrals is thus best studied in its own right.

-- Phương trình vi phân $F'(x) = f(x)$ có dạng đặc biệt: RHS chỉ chứa biến độc lập (ở đây là $x$) \& không chứa biến phụ thuộc (ở đây là $F$). Điều này đơn giản hóa lý thuyết \& thuật toán đáng kể. Do đó, vấn đề đánh giá tích phân được nghiên cứu tốt nhất theo cách riêng của nó.

Conversely, the term ``quadrature'' may also be used for the solution of differential equations; ``\href{https://en.wikipedia.org/wiki/Linear_differential_equation#Types_of_solution}{solving by quadrature}'' or ``\href{https://en.wikipedia.org/wiki/Ordinary_differential_equation#Reduction_to_quadratures}{reduction to quadrature}'' means expressing its solution in terms of integrals.

-- Ngược lại, thuật ngữ ``bậc hai'' cũng có thể được sử dụng để giải phương trình vi phân; ``giải bằng bậc hai'' hoặc ``rút gọn thành bậc hai'' có nghĩa là thể hiện lời giải theo tích phân.

%------------------------------------------------------------------------------%

\section{Problem: Mixture of Sequence, Differentiation, \& Integration -- Bài Tập: Trộn Dãy Số, Vi Phân, \& Tích Phân}

\begin{baitoan}
    Xấp xỉ bản thân tích phân \& đạo hàm của tích phân, đánh giá sai số nếu có thể:
    \begin{align*}
        I\left(\sum_{i=1}^n f_i(t,x),a(x),b(x)\right)&\coloneqq\int_{a(x)}^{b(x)} \sum_{i=1}^n f_i(t,x)\,{\rm d}t,\\
        I\left(f(t,x),\sum_{i=1}^m a_i(x),\sum_{i=1}^n b_i(x)\right)&\coloneqq\int_{\sum_{i=1}^m a_i(x)}^{\sum_{i=1}^n b_i(x)} f(t,x)\,{\rm d}t,\\
        I\left(\sum_{i=1}^p f_i(t,x),\sum_{i=1}^m a_i(x),\sum_{i=1}^n b_i(x)\right)&\coloneqq\int_{\sum_{i=1}^m a_i(x)}^{\sum_{i=1}^n b_i(x)} \sum_{i=1}^p f_i(t,x)\,{\rm d}t,\\
        I(f(t,x),I(A(x,y),a(x),b(x)),I(B(x,y),c(x),d(x)))&\coloneqq\int_{\int_{a(x)}^{b(x)} A(x,y)\,{\rm d}y}^{\int_{c(x)}^{d(x)} B(x,y)\,{\rm d}y} f(t,x)\,{\rm d}t.
    \end{align*}
    Cho dãy hàm $\{a_n\}_{n=1}^\infty,\{b_n\}_{n=1}^\infty,\{f_n\}_{n=1}^\infty\subset C_t^\infty C_x^\infty$ tính đạo hàm của tích phân lặp chồng chất:
    \begin{equation*}
        I_n(f_n(t,x),I_{n-1}(f_{n-1}(t,x),I_{n-2}(f_{n-2}(t,x),\ldots)).)***
    \end{equation*}
\end{baitoan}

\begin{baitoan}[Tổng hợp kiến thức]
    Cho 1 dãy số $\{a_n\}_{n=1}^\infty$ với số hạng được xác định bởi
    \begin{equation*}
        a_n = f(n) + g'(n) + \int_{a(n)}^{b(n)} h(x)\,{\rm d}x,\ \forall n\in\mathbb{N}^\star.
    \end{equation*}
    Tìm điều kiện để dãy số: (a) hội tụ. (b) bị chặn. (c) đơn điệu.
\end{baitoan}

%------------------------------------------------------------------------------%

\chapter{Dynamic Problem -- Bài Toán Động Học}
\minitoc

%------------------------------------------------------------------------------%

\section{Characteristics of Movements -- Các Đặc Trưng của Chuyển Động}
\textbf{\textsf{Resources -- Tài nguyên.}}
\begin{enumerate}
    \item \cite{Cuong_symbolic_MATLAB}. {\sc Vũ Đỗ Huy Cường}. {\it Lập Trình Symbolic Với MATLAB Cho Các Bài Toán Ứng Dụng}. HCMUS. Chap. 2: Bài Toán Động Học.
\end{enumerate}
Trong phần này, ta xét các bài toán trong không gian 1D, 2D, 3D, còn trường hợp tổng quát $d$D cần cấu trúc dữ liệu thích hợp. Trong không gian Euclidean $d$ chiều $\mathbb{R}^d$, 3 đặc trưng chuyển động cơ bản nhất của 1 vật thể được ký hiệu bởi:
\begin{equation*}
    \mbox{Coordinate -- tọa độ } {\bf x}(x_1,x_2,\ldots,x_d),\mbox{ velocity -- vận tốc }
\end{equation*}

%------------------------------------------------------------------------------%

\chapter{Functional Equation -- Phương Trình Hàm}
\minitoc

\begin{baitoan}[\cite{VMS_VMC2023}, 6.1, p. 40, VNUHCM UIT]
	Tìm tất cả các hàm số $f\in C^2(\mathbb{R},(0,\infty))$ thỏa
	\begin{equation*}
		f''(x)f(x)\ge2(f'(x))^2,\ \forall x\in\mathbb{R}.
	\end{equation*}
\end{baitoan}

\begin{baitoan}[\cite{VMS_VMC2023}, 6.2, p. 40, ĐH Hùng Vương, Phú Thọ]
	Tìm tất cả các hàm số $f\in C(\mathbb{R})$ thỏa $f(1) = 2023$ \& $f(x + y) = 2023^xf(y) + 2023^yf(x)$, $\forall x,y\in\mathbb{R}$.
\end{baitoan}

\begin{baitoan}[\cite{VMS_VMC2023}, 6.3, p. 40, ĐH Hùng Vương, Phú Thọ]
	Tìm tất cả các hàm số $f(x)\in C^1([0,1])$ có $f(1) = f(0$ \& thỏa
	\begin{equation*}
		\int_0^1 \left(\dfrac{f'(x)}{f(x)}\right)^2\,{\rm d}x\le1.
	\end{equation*}
\end{baitoan}

\begin{baitoan}[\cite{VMS_VMC2023}, 6.4, p. 41, ĐH Mỏ--Địa chất]
	Cho $r,s\in\mathbb{Q}$. Tìm tất cả các hàm số $f:\mathbb{Q}\to\mathbb{Q}$ thỏa
	\begin{equation*}
		f(x + f(y)) = f(x + r) + y + s,\ \forall x,y\in\mathbb{Q}.
	\end{equation*}
\end{baitoan}

\begin{baitoan}[\cite{VMS_VMC2023}, 6.5, p. 41, FTU Hà Nội]
	Tìm tất cả các hàm số thực $f:(0,\infty)\to(0,\infty)$ thỏa
	\begin{equation*}
		f(x + f(y)) = xf\left(1 + f\left(\frac{y}{x}\right)\right),\ \forall x,y\in(0,\infty).
	\end{equation*}
\end{baitoan}

\begin{baitoan}[\cite{VMS_VMC2023}, 6.6, p. 41, ĐH Trà Vinh]
	Tìm tất cả các hàm số $f(x)$ thỏa
	\begin{equation*}
		f\left(\frac{x + 1}{x - 1}\right) = 2f(x) + \frac{3}{x - 1},\ \forall x\ne1.
	\end{equation*}
\end{baitoan}

\begin{baitoan}[\cite{VMS_VMC2023}, 6.7, p. 41, ĐH Trà Vinh]
	Tìm tất cả các hàm số $f(x)\in C^1([0,1])$ thỏa $f(1) = ef(0)$ \&
	\begin{equation*}
		\int_0^1 \left(\frac{f'(x)}{f(x)}\right)^2\,{\rm d}x\le1.
	\end{equation*}
\end{baitoan}

\begin{baitoan}[\cite{VMS_VMC2024}, p. 38, 6.1, HUS]
	Cho $f:(0,1)\to\mathbb{R}$ là 1 hàm khả vi thỏa $(f'(x))^2 - 3f'(x) + 2 = 0$, $\forall x\in(0,1)$. Tìm $f$. (b) Mở rộng bài toán cho dạng phương trình hàm phức tạp hơn.
\end{baitoan}

%------------------------------------------------------------------------------%

\part{Numerical Analysis -- Giải Tích Số}

%------------------------------------------------------------------------------%

\chapter{Basic Numerical Analysis -- Giải Tích Số Cơ Bản}
\minitoc
\textbf{\textsf{Resources -- Tài nguyên.}}
\begin{enumerate}
	\item \cite{Atkinson_Han2009}. {\sc Kendall Atkinson, Weimin Han}. {\it Theoretical Numerical Analysis}. 3e.
	
	\item \cite{Burden_Faires_Burden2015}. {\sc Richard L. Burden, J. Douglas Faires, Annette M. Burden}. 8e.
	
	\item \cite{Isaacson_Keller1994}. {\sc Eugene Isaacson, Herbert Bishop Keller}. {\it Analysis of Numerical Methods}.
	
	\item \cite{Scheid1989}. {\sc Francis Scheid}. {\it Schaum's Outline of Numerical Analysis}. 2e.
\end{enumerate}

%------------------------------------------------------------------------------%

\section{Some Basic Concepts -- Vài Khái Niệm Cơ Bản}
%------------------------------------------------------------------------------%

\subsection{Algorithms -- Thuật Toán}
``The objective of numerical analysis is to solve complex numerical problems using only the simplest operations of arithmetic, to develop \& evaluate methods for computing numerical results from given data. The methods of computation are called {\it algorithms}.'' -- \cite[p. 1]{Scheid1989}

-- Mục tiêu của phân tích số là giải quyết các vấn đề số phức tạp chỉ bằng các phép toán số học đơn giản nhất, để phát triển \& đánh giá các phương pháp tính toán kết quả số từ dữ liệu đã cho. Các phương pháp tính toán được gọi là {\it thuật toán}.

``Our efforts will be focused on the search for algorithms. For some problem no satisfactory algorithm has yet been found, while for others there are several \& we must choose among them. There are various reasons for choosing 1 algorithm over another, 2 obvious criteria being {\it speed \& accuracy}. Speed is clearly an advantage, though for problems of modest size this advantage is almost eliminated by the power of the computer. For larger scale problems speed is still a major factor, \& a slow algorithm may have to be rejected as impractical. However, other things being equal, the faster method surely gets the nod.'' -- \cite[p. 1]{Scheid1989}

-- Nỗ lực của chúng tôi sẽ tập trung vào việc tìm kiếm các thuật toán. Đối với 1 số vấn đề, chưa có thuật toán nào thỏa đáng được tìm thấy, trong khi đối với những vấn đề khác, có 1 số \& chúng ta phải lựa chọn trong số chúng. Có nhiều lý do để chọn 1 thuật toán hơn thuật toán khác, 2 tiêu chí rõ ràng là {\it tốc độ \& độ chính xác}. Tốc độ rõ ràng là 1 lợi thế, mặc dù đối với các vấn đề có quy mô khiêm tốn, lợi thế này gần như bị loại bỏ bởi sức mạnh của máy tính. Đối với các vấn đề quy mô lớn hơn, tốc độ vẫn là 1 yếu tố chính, \& 1 thuật toán chậm có thể phải bị loại bỏ vì không thực tế. Tuy nhiên, các yếu tố khác đều như nhau, phương pháp nhanh hơn chắc chắn sẽ được chấp nhận.

\begin{problem}[\cite{Scheid1989}, Ex. 1.1, p. 1, approximate square root]
	Use the sequence $\{x_n\}_{n=1}^\infty$ defined by
	\begin{equation*}
		x_0 = 1,\ x_{n+1} = \frac{1}{2}\left(x_n + \frac{a}{x_n}\right),\ \forall n\in\mathbb{N}.
	\end{equation*}
	to approximate $\sqrt{a}$ with $a\in(0,\infty)$ to $m\in\mathbb{N}^\star$ decimal places.
\end{problem}

\begin{problem}[Approximate $n$th root]
	Find algorithms to approximate $\sqrt[n]{a}$ with $a\in\mathbb{R},n\in\mathbb{N}^\star$ given. Investigate accuracy \& efficiencies of these algorithms.
\end{problem}

\begin{problem}[Approximate real-order root of complexes, R]
	Find algorithms to approximate $\sqrt[n]{a}$ with $a\in\mathbb{C},n\in\mathbb{R}$ given. Investigate accuracy \& efficiencies of these algorithms.
\end{problem}

\begin{remark}
	More than 1 algorithm, using only 4 basic operations of arithmetic $\pm,\cdot,:$, exists.
\end{remark}

%------------------------------------------------------------------------------%

\subsection{Error -- Sai Số}
``The numerical optimist asks how accurate the computed results; the numerical pessimist asks how much error has been introduced. The 2 questions are, of course, 1 \& the same. Only rarely will the given data be exact, since it often originates in measurement process. So there is probably error in the input information. \& usually the algorithm itself introduces error, perhaps unavoidable roundoffs. The output information will then contain error from both of these sources.'' -- \cite[p. 1]{Scheid1989}

-- Người lạc quan về số học hỏi kết quả tính toán chính xác đến mức nào; người bi quan về số học hỏi có bao nhiêu lỗi đã được đưa vào. Tất nhiên, 2 câu hỏi là 1 \& giống nhau. Chỉ hiếm khi dữ liệu được đưa ra là chính xác, vì nó thường bắt nguồn từ quá trình đo lường. Vì vậy, có thể có lỗi trong thông tin đầu vào. \& thường thì chính thuật toán sẽ đưa ra lỗi, có lẽ là làm tròn không thể tránh khỏi. Thông tin đầu ra sau đó sẽ chứa lỗi từ cả hai nguồn này.

\begin{problem}[\cite{Scheid1989}, 1.1., p. 4]
	Calculate the value of the polynomial $P(x) = 2x^3 - 3x^2 + 5x - 4$ for the argument $x = 3$. Count each of 4 elementary operations.
\end{problem}

\begin{baitoan}[Evaluate value of polynomials with real coefficients -- Tính giá trị đa thức hệ số thực]
	Viết chương trình {\sf C{\tt/}C++, Pascal, Python} để tính giá trị $P(x_0)$ của đa thức $P(x) = \sum_{i=0}^n a_ix^i\in\mathbb{R}[x]$ (i.e., $\forall n\in\mathbb{N}$, $\forall a_i\in\mathbb{R}$, $\forall i = 0,1,\ldots,n$) tại $x = x_0\in\mathbb{R}$. Đếm số phép $\pm,\cdot,:$ đã thực hiện.
	\item {\sf Input.} Dòng 1 lần lượt chứa $n = \deg P\in\mathbb{N}$ là bậc (degree) của đa thức $P(x)$ \& $x_0\in\mathbb{R}$. Dòng 2 chứa  $a_n,a_{n-1},\ldots,a_1,a_0\in\mathbb{R}$.
	\item {\sf Output.} Giá trị $P(x_0)$. Số lần thực hiện phép $\pm,\cdot,:$.
\end{baitoan}

\begin{baitoan}[Evaluate value of polynomials with complex coefficients -- Tính giá trị đa thức hệ số phức]
	Viết chương trình {\sf C{\tt/}C++, Pascal, Python} để tính giá trị $P(x_0)$ của đa thức $P(x) = \sum_{i=0}^n a_ix^i\in\mathbb{C}[x]$ (i.e., $\forall n\in\mathbb{N}$, $\forall a_i\in\mathbb{C}$, $\forall i = 0,1,\ldots,n$) tại $x = x_0\in\mathbb{C}$. Đếm số phép $\pm,\cdot,:$ đã thực hiện.
	\item {\sf Input.} Dòng 1 lần lượt chứa $n = \deg P\in\mathbb{N}$ là bậc (degree) của đa thức $P(x)$ \& 2 số $a,b\in\mathbb{R}$ lần lượt là phần thực \& phần ảo của $x_0\in\mathbb{C}$, i.e., $a = \Re x_0,b = \Im x_0$. Dòng 2 chứa $n + 1$ phần thực $\Re a_n,\Re a_{n-1},\ldots,\Re a_1,\Re a_0$ của $n + 1$ số phức $a_n,a_{n-1},\ldots,a_1,a_0\in\mathbb{C}$. Dòng 3 chứa phần ảo của $a_n,a_{n-1},\ldots,a_1,a_0\in\mathbb{C}$.
	\item {\sf Output.} Giá trị $P(x_0)$ theo format $\Re P(x_0) + i\Im P(x_0)$. Số lần thực hiện phép $\pm,\cdot,:$.
\end{baitoan}

\begin{problem}[\cite{Scheid1989}, 1.2., p. 5]
	Define the error of an approximation.
\end{problem}

\begin{proof}[Solution.]
	The tradition definition is: true value $=$ approximation $+$ error.
\end{proof}

\begin{problem}[\cite{Scheid1989}, 1.3., p. 5]
	What is relative error?
\end{problem}

%------------------------------------------------------------------------------%

\part{Introduction to Ordinary Differential Equations (ODEs) -- Nhập Môn Phương Trình Vi Phân Đạo Hàm Thường}
\minitoc
\textbf{\textsf{Resources -- Tài nguyên.}}
\begin{enumerate}
	\item \cite{Teschl2012}. {\sc Gerald Teschl}. {\it Ordinary Differential Equations \& Dynamical Systems}.
\end{enumerate}
Lecture Notes chi tiết hơn về ODEs sẽ được viết riêng. Phần này chỉ giới thiệu những khái niệm cơ bản của ODEs.

%------------------------------------------------------------------------------%

\part{Introduction to Partial Differential Equations (PDEs) -- Nhập Môn Phương Trình Vi Phân Đạo Hàm Riêng}
\minitoc
\textbf{\textsf{Resources -- Tài nguyên.}}
\begin{enumerate}
	\item \cite{Anh_Ke_semigroup}. {\sc Cung Thế Anh, Trần Đình Kế}. {\it Nửa Nhóm Các Toán Tử Tuyến Tính \& Ứng Dụng}.
	
	\item \cite{DuChateau_Zachmann_PDEs}. {\sc Paul DuChateau, David W. Zachmann}. {\it Schaum's Outlines of Theory and Problems of Partial Differential Equations}.
	
	\item \cite{Evans2010}. {\sc Lawrence C. Evans}. {\it Partial Differential Equations}.
	
	\item \cite{Klainerman2000}. {\sc Sergiu Klainerman}. {\it PDE as a unified subject}.
	
	\item \cite{Taylor2011}. {\sc Michael E. Taylor}. {\it PDEs III: Nonlinear Equations}.
\end{enumerate}
Lecture Notes chi tiết hơn về PDEs sẽ được viết riêng. Phần này chỉ giới thiệu những khái niệm cơ bản của PDEs.

%------------------------------------------------------------------------------%

\part{Introduction to Differential Geometry -- Nhập Môn Hình Học Vi Phân}
\minitoc
\textbf{\textsf{Resources -- Tài nguyên.}}
\begin{enumerate}
	\item \cite{Carmo2016}. {\sc Manfredo P. do Carmo}. {\it Differential Geometry of Curves \& Surfaces}.
	
	\item \cite{Kuhnel2015}. {\sc Wolfgang K\"{u}hnel}. {\it Differential Geometry: Curves -- Surfaces -- Manifolds}.
	
	\item \cite{Walker2015}. {\sc Shawn W. Walker}. {\it The Shapes of Things}.
\end{enumerate}
Lecture Notes chi tiết hơn về Hình Học Vi Phân sẽ được viết riêng. Phần này chỉ giới thiệu những khái niệm cơ bản của Hình Học Vi Phân.

%------------------------------------------------------------------------------%

\part{Introduction to Functional Analysis -- Nhập Môn Giải Tích Hàm}
\minitoc
\textbf{\textsf{Resources -- Tài nguyên.}}
\begin{enumerate}
	\item \cite{Alt2016}. {\sc Hans Wilhelm Alt}. {\it Linear Functional Analysis}.
	
	\item \cite{Brezis2011}. {\sc Ha\"im Brezis}. {\it Functional Analysis, Sobolev Spaces \& PDEs}.
	
	\item \cite{Thanh_Thanh_Vu_gth}. {\sc Đinh Ngọc Thanh, Bùi Lê Trọng Thanh, Huỳnh Quang Vũ}. {\it Bài Giảng Giải Tích Hàm}. HCMUS.
\end{enumerate}
Lecture Notes chi tiết hơn về Giải Tích Hàm sẽ được viết riêng. Phần này chỉ giới thiệu những khái niệm cơ bản của Giải Tích Hàm.

%------------------------------------------------------------------------------%

\part{Fourier Transform -- Biến Đổi Fourier}
\minitoc
\textbf{\textsf{Resources -- Tài nguyên.}}
\begin{enumerate}
	\item \cite{Tao_Fourier_analysis}. {\sc Terence Tao}. {\it Higher Order Fourier Analysis}.
\end{enumerate}

%------------------------------------------------------------------------------%

\section{Discrete Fourier transform -- Biến đổi Fourier rời rạc}
See, e.g., \href{https://en.wikipedia.org/wiki/Discrete_Fourier_transform}{Wikipedia{\tt/}discrete Fourier transform}. In mathematics, the {\it discrete Fourier transform (DFT)} converts a finite sequence of equally-spaced \href{https://en.wikipedia.org/wiki/Sampling_(signal_processing)}{samples} of a function into a same-length sequence of equally-spaced samples of the \href{https://en.wikipedia.org/wiki/Discrete-time_Fourier_transform}{discrete-time Fourier transform} (DTFT), which is a complex-valued function of frequency. The interval at which the DTFT is sampled is the reciprocal of the duration of the input sequence.

\begin{definition}[Discrete Fourier transform]
	The {\rm discrete Fourier transform} transforms a \href{https://en.wikipedia.org/wiki/Sequence}{sequence} of $N$ complex numbers ${\bf x} = \{x_n\}_{n=0}^{N-1}\coloneqq x_0,x_1,\ldots,x_{N-1}$ into another sequence of complex numbers, ${\bf X} = \{X_n\}_{n=0}^{N-1}\coloneqq X_0,X_1,\ldots,X_{N-1}$ defined by
	\begin{equation}
		\label{discrete Fourier transform}
		\tag{dFt}
		X_k\coloneqq\sum_{n=0}^{N-1} x_ne^{-i2\pi\frac{k}{N}n}.
	\end{equation}
	The transform is sometimes denoted by the symbol ${\cal F}$, as in ${\bf X} = {\cal F}\{{\bf x}\}$ or ${\cal F}({\bf x})$ or ${\cal F}{\bf x}$.
\end{definition}

%------------------------------------------------------------------------------%

\chapter{Miscellaneous}
\minitoc

%------------------------------------------------------------------------------%

\section{Contributors}

\begin{enumerate}
	\item {\sc Võ Ngọc Trâm Anh [VNTA].} Code C{\tt/}C++.
	\item {\sc Nguyễn Lê Đăng Khoa [NLDK].} Code C{\tt/}C++.
	\item {\sc Phan Vĩnh Tiến [PVT].} Proofs of some results in Mathematical Analysis.
\end{enumerate}

\section{See also}

\begin{enumerate}
	\item \cite{Strogatz_infinite_power}. {\sc Steven Strogatz}. {\it Infinite Powers: How Calculus Reveals the Secrets of the Universe}.
	
	\item \cite{Strogatz_infinite_power_VN}. {\sc Steven Strogatz}. {\it Infinite Powers: How Calculus Reveals the Secrets of the Universe -- Sức Mạnh Vô Hạn: Giải Tích Toán Khám Phá Bí Mật Của Vũ Trụ Như Thế Nào?}.
	
	{\sf Nhận xét.} 1 quyển sách hay về thường thức về lịch sử phát triển của Giải tích Toán học \& các ý tưởng cơ bản nhất của Giải tích. Khuyến khích đọc thử, cũng như các tác phẩm thường thức Khoa học Tự nhiên nói chung \& Toán học nói riêng khác của tác giả {\sc Steven Strogatz}.
	\item TS. {\sc Huỳnh Quang Vũ}. {\it Các Bài Giảng Giải Tích}. \url{https://sites.google.com/view/hqvu/teaching}.
	\begin{itemize}
		\item Bộ Môn Giải Tích, Khoa Toán - Tin học, Faculty of Mathematics \& Computer Science, HCMUS. \href{https://drive.google.com/file/d/1NA1G0NSIVjnu_zG7e0JTnOvGfFqmuuVg/view}{\it Giáo Trình Vi Tích Phân 1}.
		\item Bộ Môn Giải Tích, Khoa Toán - Tin học, Faculty of Mathematics \& Computer Science, HCMUS. \href{https://drive.google.com/file/d/1Td7-zDZYFdop6f1IXvsPO0S4Cxc7ccd3/view}{\it Giáo Trình Vi Tích Phân 2}.
	\end{itemize}
	\item {\it Vietnamese Mathematical Olympiad for High School- \& College Students (VMC) -- Olympic Toán Học Học Sinh \& Sinh Viên Toàn Quốc}.
	
	PDF: {\sc url}: \url{https://github.com/NQBH/advanced_STEM_beyond/blob/main/VMC/NQBH_VMC.pdf}.
	
	\TeX: {\sc url}: \url{https://github.com/NQBH/advanced_STEM_beyond/blob/main/VMC/NQBH_VMC.tex}.
	\begin{itemize}
		\item Codes:
		\begin{itemize}
			\item C++ code: \url{https://github.com/NQBH/advanced_STEM_beyond/tree/main/VMC/C++}.
			\item Python code: \url{https://github.com/NQBH/advanced_STEM_beyond/tree/main/VMC/Python}.
		\end{itemize}
		\item Resource: \url{https://github.com/NQBH/advanced_STEM_beyond/tree/main/VMC/resource}.
		\item Figures: \url{https://github.com/NQBH/advanced_STEM_beyond/tree/main/VMC/figure}.
	\end{itemize}
	\item {\it Olympic Tin Học Sinh Viên OLP \& ICPC}.
	
	PDF: {\sc url}: \url{https://github.com/NQBH/advanced_STEM_beyond/blob/main/OLP_ICPC/NQBH_OLP_ICPC.pdf}.
	
	\TeX: {\sc url}: \url{https://github.com/NQBH/advanced_STEM_beyond/blob/main/OLP_ICPC/NQBH_OLP_ICPC.tex}.
	\begin{itemize}
		\item Codes:
		\begin{itemize}
			\item C: \url{https://github.com/NQBH/advanced_STEM_beyond/tree/main/OLP_ICPC/C}.
			\item C++: \url{https://github.com/NQBH/advanced_STEM_beyond/tree/main/OLP_ICPC/C++}.
			\item Python: \url{https://github.com/NQBH/advanced_STEM_beyond/tree/main/OLP_ICPC/Python}.
		\end{itemize}
	\end{itemize}
\end{enumerate}

%------------------------------------------------------------------------------%

\printbibliography[heading=bibintoc]
	
\end{document}