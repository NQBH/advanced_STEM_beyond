\documentclass{article}
\usepackage[backend=biber,natbib=true,style=alphabetic,maxbibnames=50]{biblatex}
\addbibresource{/home/nqbh/reference/bib.bib}
\usepackage[utf8]{vietnam}
\usepackage{tocloft}
\renewcommand{\cftsecleader}{\cftdotfill{\cftdotsep}}
\usepackage[colorlinks=true,linkcolor=blue,urlcolor=red,citecolor=magenta]{hyperref}
\usepackage{amsmath,amssymb,amsthm,enumitem,fancyvrb,float,graphicx,mathtools,tikz}
\usetikzlibrary{angles,calc,intersections,matrix,patterns,quotes,shadings}
\allowdisplaybreaks
\newtheorem{assumption}{Assumption}
\newtheorem{baitoan}{Bài toán}
\newtheorem{cauhoi}{Câu hỏi}
\newtheorem{conjecture}{Conjecture}
\newtheorem{corollary}{Corollary}
\newtheorem{dangtoan}{Dạng toán}
\newtheorem{definition}{Definition}
\newtheorem{dinhly}{Định lý}
\newtheorem{dinhnghia}{Định nghĩa}
\newtheorem{example}{Example}
\newtheorem{ghichu}{Ghi chú}
\newtheorem{hequa}{Hệ quả}
\newtheorem{hypothesis}{Hypothesis}
\newtheorem{lemma}{Lemma}
\newtheorem{luuy}{Lưu ý}
\newtheorem{nhanxet}{Nhận xét}
\newtheorem{notation}{Notation}
\newtheorem{note}{Note}
\newtheorem{principle}{Principle}
\newtheorem{problem}{Problem}
\newtheorem{proposition}{Proposition}
\newtheorem{question}{Question}
\newtheorem{remark}{Remark}
\newtheorem{theorem}{Theorem}
\newtheorem{vidu}{Ví dụ}
\usepackage[left=1cm,right=1cm,top=5mm,bottom=5mm,footskip=4mm]{geometry}
\def\labelitemii{$\circ$}
\DeclareRobustCommand{\divby}{%
    \mathrel{\vbox{\baselineskip.65ex\lineskiplimit0pt\hbox{.}\hbox{.}\hbox{.}}}%
}
\setlist[itemize]{leftmargin=*}
\setlist[enumerate]{leftmargin=*}

\title{Lecture: Linear Algebra -- Bài Giảng: Đại Số Tuyến Tính}
\author{Nguyễn Quản Bá Hồng\footnote{A scientist- {\it\&} creative artist wannabe, a mathematics {\it\&} computer science lecturer of Department of Artificial Intelligence {\it\&} Data Science (AIDS), School of Technology (SOT), UMT Trường Đại học Quản lý {\it\&} Công nghệ TP.HCM, Hồ Chí Minh City, Việt Nam.\\E-mail: {\sf nguyenquanbahong@gmail.com} {\it\&} {\sf hong.nguyenquanba@umt.edu.vn}. Website: \url{https://nqbh.github.io/}. GitHub: \url{https://github.com/NQBH}.}}
\date{\today}

\begin{document}
\maketitle
\begin{abstract}
    This text is a part of the series {\it Some Topics in Advanced STEM \& Beyond}:

    {\sc url}: \url{https://nqbh.github.io/advanced_STEM/}.

    Latest version:
    \begin{itemize}
        \item {\it Lecture: Linear Algebra -- Bài Giảng: Đại Số Tuyến Tính}.

        PDF: {\sc url}: \url{.pdf}.

        \TeX: {\sc url}: \url{.tex}.
        \item {\it }.

        PDF: {\sc url}: \url{.pdf}.

        \TeX: {\sc url}: \url{.tex}.
    \end{itemize}
\end{abstract}
\tableofcontents

%------------------------------------------------------------------------------%

\section{Linear System of Equations \& Matrices -- Hệ Phương Trình Tuyến Tính \& Ma Trận}

%------------------------------------------------------------------------------%

\subsection{Matrix -- Ma trận}

\begin{baitoan}[Nhập xuất ma trận]
    Viết chương trình {\sf Python, C++} để nhập vào số hàng, số cột, \& các phần tử của 1 ma trận $A\in\mathbb{R}^{m\times n}$. Sau đó xuất ma trận ra màn hình.
\end{baitoan}

\begin{baitoan}[Cộng, trừ ma trận]
    Viết chương trình {\sf Python, C++} để cộng, trừ 2 ma trận $A,B\in\mathbb{R}^{m\times n}$.
\end{baitoan}

\begin{baitoan}[Phép nhân vô hướng của ma trận với 1 hằng số]
    Viết chương trình {\sf Python, C++} để thực hiện phép nhân vô hướng ma trận $A\in\mathbb{R}^{m\times n}$ với 1 số thực $c\in\mathbb{R}$.
\end{baitoan}

\begin{baitoan}[Nhân 2 ma trận]
    Viết chương trình {\sf Python, C++} để nhân 2 ma trận $A\in\mathbb{R}^{m\times n},B\in\mathbb{R}^{n\times p}$.
\end{baitoan}

\begin{baitoan}
    Viết chương trình tìm ma trận chuyển vị $A^\top\in\mathbb{R}^{n\times m}$ của 1 ma trận $A\in\mathbb{R}^{m\times n}$ cho trước.
\end{baitoan}

%------------------------------------------------------------------------------%

\section{Miscellaneous}

%------------------------------------------------------------------------------%

\printbibliography[heading=bibintoc]

\end{document}