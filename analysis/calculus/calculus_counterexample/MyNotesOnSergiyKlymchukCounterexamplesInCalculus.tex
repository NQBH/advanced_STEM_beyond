\documentclass[a4paper]{article}
\usepackage{tabu,float,hyperref,color,amsmath,amsxtra,amssymb,latexsym,amscd,amsthm,amsfonts,graphicx}
\numberwithin{equation}{section}
\usepackage{fancyhdr}
\pagestyle{fancy}
\fancyhf{}
\fancyhead[RE,LO]{\footnotesize \textsc \leftmark}
\cfoot{\thepage}
\renewcommand{\headrulewidth}{0.5pt}
\setcounter{tocdepth}{3}
\setcounter{secnumdepth}{3}
\usepackage{imakeidx}
\makeindex[columns=2, title=Alphabetical Index, 
           options= -s index.ist]
\title{My notes on $\star$ Sergiy Klymchuk, Counter-examples in Calculus }
\author{\textsc{Nguyen Quan Ba Hong}\\
{\small Students at Faculty of Math and Computer Science,}\\ 
{\small Ho Chi Minh University of Science, Vietnam} \\
{\small \texttt{email. nguyenquanbahong@gmail.com}}\\
{\small \texttt{blog. \url{http://hongnguyenquanba.wordpress.com}} 
\footnote{Copyright \copyright\ 2016 by Nguyen Quan Ba Hong, Student at Ho Chi Minh University of Science, Vietnam. This document may be copied freely for the purposes of education and non-commercial research. Visit my site \texttt{\url{http://hongnguyenquanba.wordpress.com}} to get more.}}}
\begin{document}
\maketitle
\begin{abstract}
I take some notes when I learn the book \cite{1}.
\end{abstract}
\newpage
\tableofcontents
\newpage
\section{Selected Counter-examples}
\textbf{Definition.} A function $f(x)$ is said to be \textit{increasing at the point $x=a$} if in a certain neighborhood $\left( { - \delta ,\delta } \right)$, where $\delta >0$ the following is true: if $x<a$ then $f(x)<f(a)$ and if $x>a$ then $f(x)>f(a)$.\\
\\
\textbf{Problem 1.} \textit{If a function $f(x)$ is continuous and increasing at the point $x=a$ then there is a neighborhood $\left( {x - \delta ,x + \delta } \right),\delta  > 0$ where the function is also increasing.}\\
\\
\textsc{Counter-examples.} The function 
\begin{align}
f\left( x \right) = \left\{ {\begin{array}{*{20}{c}}
{x + {x^2}\sin \frac{2}{x},\mbox{ if }x \ne 0}\\
{0,\mbox{ if }x = 0}
\end{array}} \right.
\end{align}
is increasing at the point $x=0$ but it is not increasing in any neighborhood $\left( { - \delta ,\delta } \right)$, $\delta >0$. \hfill $\square$\\
\\
\textbf{Problem 2.} \textit{If a function is not monotone then it doesn't have an inverse function.}\\
\\
\textsc{Counter-example.} 
\begin{align}
f\left( x \right) = \left\{ {\begin{array}{*{20}{c}}
{x,\mbox{ if } x \in \mathbb{Q}}\\
{ - x,\mbox{ if } x \in \mathbb{R}\backslash \mathbb{Q} }
\end{array}} \right. 
\end{align}
\hfill $\square$\\
\\
\textbf{Problem 3.} \textit{If a function is not monotone in $(a,b)$ then its square cannot be monotone on $(a,b)$.}\\
\\
\textsc{Counter-example.} 
\begin{align}
f\left( x \right) = \left\{ {\begin{array}{*{20}{c}}
{x,\mbox{ if } x \in \mathbb{Q}}\\
{ - x,\mbox{ if } x \in \mathbb{R}\backslash \mathbb{Q}}
\end{array}} \right.
\end{align}
defined on $\left( {0,\infty } \right)$. \hfill $\square$\\
\\
\textbf{Problem 4.} \textit{If on the closed interval $[a,b]$ a function is}
\begin{enumerate}
\item \textit{bounded.}
\item \textit{takes its maximum and minimum values.}
\item \textit{takes all its values between the maximum and minimum values.\\
then this function is continuous at some points or subintervals on $[a,b]$.}
\end{enumerate}
\textsc{Counter-example.}
\begin{align}
f\left( x \right) = \left\{ {\begin{array}{*{20}{c}}
{1,\mbox{ if } x = 0}\\
{x,\mbox{ if } x \in \mathbb{Q}\backslash \left\{ {0,1} \right\}}\\
{ - x,\mbox{ if } x \in \mathbb{R}\backslash \mathbb{Q}}\\
{0,\mbox{ if } x = 1}
\end{array}} \right.
\end{align}
\hfill $\square$\\
\\
\textbf{Problem 5.} \textit{If a function is discontinuous at every point in its domain then the square and the absolute value of this function cannot be continuous.}\\
\\
\textsc{Counter-example.} 
\begin{align}
f\left( x \right) = \left\{ {\begin{array}{*{20}{c}}
{1,\mbox{ if } x \in \mathbb{Q}}\\
{ -1,\mbox{ if } x \in \mathbb{R}\backslash \mathbb{Q} }
\end{array}} \right.
\end{align}
\hfill $\square$\\
\\
\textbf{Problem 6.} \textit{A function cannot be continuous at only one point in its domain and discontinuous everywhere else.}\\
\\
\textsc{Counter-example.} 
\begin{align}
f\left( x \right) = \left\{ {\begin{array}{*{20}{c}}
{x,\mbox{ if } x \in \mathbb{Q}}\\
{ - x,\mbox{ if } x \in \mathbb{R}\backslash \mathbb{Q} }
\end{array}} \right.
\end{align}
\hfill $\square$\\
\\
\textbf{Problem 7.} \textit{If a function $ f(x)$ is differentiable on $\left( {0,\infty } \right)$ and $\mathop {\lim }\limits_{x \to \infty } f\left( x \right)$ exists then $\mathop {\lim }\limits_{x \to \infty } f'\left( x \right)$ also exists.}\\
\\
\textsc{Counter-example.}
\begin{align}
f\left( x \right) = \frac{{\sin \left( {{x^2}} \right)}}{x}
\end{align}
\hfill $\square$\\
\\
\textbf{Problem 8.} \textit{If a function $ f(x)$ is differentiable and bounded on $\left( {0,\infty } \right)$ and $\mathop {\lim }\limits_{x \to \infty } f'\left( x \right)$ exists then $\mathop {\lim }\limits_{x \to \infty } f\left( x \right)$ also exists.}\\
\\
\textsc{Counter-example.}
\begin{align}
f\left( x \right) = \cos \left( {\ln x} \right)
\end{align}
\hfill $\square$\\
\\
\textbf{Problem 9.} \textit{If a function $ f(x)$ is differentiable at the point $x=a$ then its derivative is continuous at $x=a$.}\\
\\
\textsc{Counter-example.} 
\begin{align}
f\left( x \right) = \left\{ {\begin{array}{*{20}{c}}
{{x^2}\sin \frac{1}{x},\mbox{ if } x \ne 0}\\
{0,\mbox{ if } x = 0}
\end{array}} \right.
\end{align}
at the point $x=0$. \hfill $\square$\\
\\
\textbf{Problem 10.} \textit{If the derivative of a function $ f(x)$ is positive at the point $x=a$ then there is a neighborhood about $x=a$ (no matter how small) where the function is increasing.}\\
\\
\textsc{Counter-example.} 
\begin{align}
f\left( x \right) = \left\{ {\begin{array}{*{20}{c}}
{x+2{x^2}\sin \frac{1}{x},\mbox{ if } x \ne 0}\\
{0,\mbox{ if } x = 0}
\end{array}} \right.
\end{align}
\hfill $\square$\\
\\
\textbf{Problem 11.} \textit{If a function $ f(x)$ is continuous on $(a,b)$ and has a local maximum at the point $c \in \left( {a,b} \right)$ then in a sufficiently small neighborhood of the point $x=c$ the function is increasing on the left and decreasing on the right from $x=c$.}\\
\\
\textsc{Counter-example.} 
\begin{align}
f\left( x \right) = \left\{ {\begin{array}{*{20}{c}}
{2 - {x^2}\left( {2 + \sin \frac{1}{x}} \right),\mbox{ if } x \ne 0}\\
{2,\mbox{ if } x = 0}
\end{array}} \right.
\end{align}
\hfill $\square$\\
\\
\textbf{Problem 12.} \textit{If a function $ f(x)$ is differentiable at the point $x=a$ then there is a certain neighborhood of the point $x=a$ where the derivative of the function $ f(x)$ is bounded.}\\
\\
\textsc{Counter-example.} 
\begin{align}
f\left( x \right) = \left\{ {\begin{array}{*{20}{c}}
{{x^2}\sin \frac{1}{x},\mbox{ if } x \ne 0}\\
{0,\mbox{ if } x = 0}
\end{array}} \right.
\end{align}
\hfill $\square$\\
\\
\textbf{Problem 13.} \textit{If a function $ f(x)$ at any neighborhood of the point $x=a$ has points where $f'(x)$ doesn't exist then $f'(a)$ doesn't exist.}\\
\\
\textsc{Counter-example.} 
\begin{align}
f\left( x \right) = \left\{ {\begin{array}{*{20}{c}}
{{x^2}\left| {\cos \frac{\pi }{x}} \right|,\mbox{ if } x \ne 0}\\
{0,\mbox{ if } x = 0}
\end{array}} \right.
\end{align}
\hfill $\square$\\
\\
\textbf{Problem 14.} \textit{A function cannot be differentiable only at one point in its domain and non-differentiable everywhere else in its domain.} \\
\\
\textsc{Counter-example.} 
\begin{align}
f\left( x \right) = \left\{ {\begin{array}{*{20}{c}}
{x^2,\mbox{ if } x \in \mathbb{Q}}\\
{ 0,\mbox{ if } x \in \mathbb{R}\backslash \mathbb{Q} }
\end{array}} \right.
\end{align}
\hfill $\square$\\
\\
\textbf{Problem 15.} \textit{A continuous function cannot be non-differentiable at every point in its domain.}\\
\\
\textsc{Counter-example.} The Weierstrass' function 
\begin{align}
f\left( x \right) = \sum\limits_{n = 0}^\infty  {{{\left( {\frac{1}{2}} \right)}^n}\cos \left( {{3^n}x} \right)}
\end{align}
\hfill $\square$\\
\\
\textbf{Problem 16.} \textit{If a function $f(x)$ is continuous and $\int\limits_a^\infty  {f\left( x \right)dx} $ converges then} 
\begin{align}
\mathop {\lim }\limits_{x \to \infty } f\left( x \right) = 0
\end{align}
\textsc{Counter-example.} The Fresnel integral
\begin{align}
\int\limits_a^\infty  {\sin {x^2}dx}
\end{align}
\hfill $\square$\\
\\
\textbf{Problem 17.} \textit{If a function $ f(x)$ is continuous and non-negative and $\int\limits_a^\infty  {f\left( x \right)dx} $ converges then $\mathop {\lim }\limits_{x \to \infty } f\left( x \right) = 0$. }\\
\\
\textbf{Problem 18.} \textit{If a function $ f(x)$ is positive and unbounded for all real $x$ then the integral $\int\limits_a^\infty  {f\left( x \right)dx} $ diverges.}\\
\\
\textbf{Problem 19.} \textit{If a function $ f(x)$ is continuous and not bounded for all real $x$ then the integral $\int\limits_a^\infty  {f\left( x \right)dx} $ diverges.}\\
\\
\textsc{Counter-example.} 
\begin{align}
f\left( x \right) = x\sin {x^4}
\end{align}
\hfill $\square$\\
\\
\textbf{Problem 20.} \textit{If a function $ f(x)$ is continuous on $\left[ {1,\infty } \right)$ and  $\int\limits_1^\infty  {f\left( x \right)dx} $ converges then $\int\limits_1^\infty  {\left| {f\left( x \right)} \right|dx} $ also converges.}\\
\\
\textsc{Counter-example.} 
\begin{align}
f\left( x \right) = \frac{{\sin x}}{x}
\end{align}
\hfill $\square$\\
\\
\textbf{Problem 21.} \textit{If the integral $\int\limits_a^\infty  {f\left( x \right)dx} $ converges and a function $g(x)$ is bounded then the integral $\int\limits_a^\infty  {f\left( x \right)} g\left( x \right)dx$ converges. }\\
\\
\textsc{Counter-example.} 
\begin{align}
f\left( x \right) = \frac{{\sin x}}{x},g\left( x \right) = \sin x
\end{align}
\hfill $\square$
\section{Selected examples}
\textbf{Problem 22.} \textit{A continuous curve that has a sharp corner at every point.}\\
\\
\textsc{Example.} The Koch's snowflake.\hfill $\square$\\
\newpage
\begin{thebibliography}{9}
\bibitem {1} Sergiy Klymchuk, \textit{Counter-examples in Calculus}, November 2004.
\end{thebibliography}
\end{document}



