\documentclass{article}
\usepackage[backend=biber,natbib=true,style=alphabetic,maxbibnames=50]{biblatex}
\addbibresource{/home/nqbh/reference/bib.bib}
\usepackage[utf8]{vietnam}
\usepackage{tocloft}
\renewcommand{\cftsecleader}{\cftdotfill{\cftdotsep}}
\usepackage[colorlinks=true,linkcolor=blue,urlcolor=red,citecolor=magenta]{hyperref}
\usepackage{amsmath,amssymb,amsthm,float,graphicx,mathtools,tikz}
\usetikzlibrary{angles,calc,intersections,matrix,patterns,quotes,shadings}
\allowdisplaybreaks
\newtheorem{assumption}{Assumption}
\newtheorem{baitoan}{}
\newtheorem{cauhoi}{Câu hỏi}
\newtheorem{conjecture}{Conjecture}
\newtheorem{corollary}{Corollary}
\newtheorem{dangtoan}{Dạng toán}
\newtheorem{definition}{Definition}
\newtheorem{dinhly}{Định lý}
\newtheorem{dinhnghia}{Định nghĩa}
\newtheorem{example}{Example}
\newtheorem{ghichu}{Ghi chú}
\newtheorem{hequa}{Hệ quả}
\newtheorem{hypothesis}{Hypothesis}
\newtheorem{lemma}{Lemma}
\newtheorem{luuy}{Lưu ý}
\newtheorem{nhanxet}{Nhận xét}
\newtheorem{notation}{Notation}
\newtheorem{note}{Note}
\newtheorem{principle}{Principle}
\newtheorem{problem}{Problem}
\newtheorem{proposition}{Proposition}
\newtheorem{question}{Question}
\newtheorem{remark}{Remark}
\newtheorem{theorem}{Theorem}
\newtheorem{vidu}{Ví dụ}
\usepackage[left=1cm,right=1cm,top=5mm,bottom=5mm,footskip=4mm]{geometry}
\def\labelitemii{$\circ$}
\DeclareRobustCommand{\divby}{%
	\mathrel{\vbox{\baselineskip.65ex\lineskiplimit0pt\hbox{.}\hbox{.}\hbox{.}}}%
}

\title{Mathematical Analysis -- Giải Tích Toán Học}
\author{Nguyễn Quản Bá Hồng\footnote{A Scientist {\it\&} Creative Artist Wannabe. E-mail: {\tt nguyenquanbahong@gmail.com}. Bến Tre City, Việt Nam.}}
\date{\today}

\begin{document}
\maketitle
\begin{abstract}
	This text is a part of the series {\it Some Topics in Advanced STEM \& Beyond}:
	
	{\sc url}: \url{https://nqbh.github.io/advanced_STEM/}.
	
	Latest version:
	\begin{itemize}
		\item {\it Mathematical Analysis -- Giải Tích Toán Học}.
		
		PDF: {\sc url}: \url{https://github.com/NQBH/advanced_STEM_beyond/blob/main/analysis/NQBH_mathematical_analysis.pdf}.
		
		\TeX: {\sc url}: \url{https://github.com/NQBH/advanced_STEM_beyond/blob/main/analysis/NQBH_mathematical_analysis.tex}.
	\end{itemize}
\end{abstract}
\tableofcontents

%------------------------------------------------------------------------------%

\section{Basic}
Tôi được giải Nhì Giải tích Olympic Toán Sinh viên 2014 (VMC2014) khi còn học năm nhất Đại học \& được giải Nhất Giải tích Olympic Toán Sinh viên 2015 (VMC2015) khi học năm 2 Đại học. Nhưng điều đó không có nghĩa là tôi giỏi Giải tích. Bằng chứng là 10 năm sau khi nhận các giải đó, tôi đang tự học lại Giải tích Toán học với hy vọng có 1 hay nhiều cách nhìn mới mẻ hơn \& mang tính ứng dụng hơn cho các đề tài cá nhân của tôi.

\noindent\textbf{\textsf{Resources -- Tài nguyên.}}
\begin{itemize}
	\item \cite{Rudin1976}. {\sc Walter Rudin}. {\it Principles Principles of Mathematical Analysis}.
	\item \cite{Rudin1973,Rudin1987}. {\sc Walter Rudin}. {\it Real \& Complex Analysis}.
	\item \cite{Rudin1991}. {\sc Walter Rudin}. {\it Functional Analysis}.
	\item \cite{Lieb_Loss2001}. {\sc Elliott Lieb, Michael Loss}. {\it Analysis}.
	\item \cite{Evans_Gariepy2015}. {\sc Lawrence C. Evans, Ronald F. Gariepy}. {\it Measure Theory \& Fine Properties of Functions}.
\end{itemize}
``Analysis is the art of taking limits, \& the constraint of having to deal with an integration theory that does not allow taking limits is much like having to do mathematics only with rational numbers \& excluding the irrational ones.'' -- \cite[Chap. 1, p. 1]{Lieb_Loss2001}

\subsection{Measure \& integration -- Độ đo \& tích phân}
The point of view of integration defined as a Riemann integral may be historically grounded \& useful in many areas of mathematics but is far from being adequate for the requirements of modern analysis since Riemann integral can be defined only for a special class of functions \& this class is not closed under the process of taking pointwise limits of sequence (not even monotonic sequences) of functions in this class.

``The useful \& far-reaching idea of Lebesgue \& others was to compute the $(n + 1)$-dimensional volume `in the other direction' by 1st computing the $n$-dimensional volume of the set where the function $> y$. This volume is a well-behaved, monotone nonincreasing function of $y$, which then can be integrated in the manner of Riemann. This method of integration not only works for a large class of functions (which is closed under taking pointwise limits), but it also greatly simplifies a problem that used to plague analysts: {\it Is it permissible to exchange limits \& integration?}'' -- \cite[Chap. 1, pp. 1--2]{Lieb_Loss2001}

%------------------------------------------------------------------------------%

\section{Partial Differential Equations (PDEs) -- Phương Trình Vi Phân Đạo Hàm Riêng}
\noindent\textbf{\textsf{Resources -- Tài nguyên.}}
\begin{itemize}
	\item \cite{Brezis2011}. {\sc Ha\"im Brezis}. {\it Functional Analysis, Sobolev Spaces, \& Partial Differential Equations}.
	\item \cite{Evans2010}. {\sc Lawrence C. Evans}. {\it Partial Differential Equations}.
	\item \cite{Gilbarg_Trudinger2001}. {\sc David Gilbarg, Neil S. Trudinger}. {\it Elliptic Partial Differential Equations of 2nd Order}.
\end{itemize}

%------------------------------------------------------------------------------%

\section{Miscellaneous}

%------------------------------------------------------------------------------%

\printbibliography[heading=bibintoc]
	
\end{document}