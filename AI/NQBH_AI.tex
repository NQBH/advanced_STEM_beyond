\documentclass{article}
\usepackage[backend=biber,natbib=true,style=alphabetic,maxbibnames=50]{biblatex}
\addbibresource{/home/nqbh/reference/bib.bib}
\usepackage[utf8]{vietnam}
\usepackage{tocloft}
\renewcommand{\cftsecleader}{\cftdotfill{\cftdotsep}}
\usepackage[colorlinks=true,linkcolor=blue,urlcolor=red,citecolor=magenta]{hyperref}
\usepackage{amsmath,amssymb,amsthm,enumitem,float,graphicx,mathtools,tikz}
\usetikzlibrary{angles,calc,intersections,matrix,patterns,quotes,shadings}
\allowdisplaybreaks
\newtheorem{assumption}{Assumption}
\newtheorem{baitoan}{}
\newtheorem{cauhoi}{Câu hỏi}
\newtheorem{conjecture}{Conjecture}
\newtheorem{corollary}{Corollary}
\newtheorem{dangtoan}{Dạng toán}
\newtheorem{definition}{Definition}
\newtheorem{dinhly}{Định lý}
\newtheorem{dinhnghia}{Định nghĩa}
\newtheorem{example}{Example}
\newtheorem{ghichu}{Ghi chú}
\newtheorem{hequa}{Hệ quả}
\newtheorem{hypothesis}{Hypothesis}
\newtheorem{lemma}{Lemma}
\newtheorem{luuy}{Lưu ý}
\newtheorem{nhanxet}{Nhận xét}
\newtheorem{notation}{Notation}
\newtheorem{note}{Note}
\newtheorem{principle}{Principle}
\newtheorem{problem}{Problem}
\newtheorem{proposition}{Proposition}
\newtheorem{question}{Question}
\newtheorem{remark}{Remark}
\newtheorem{theorem}{Theorem}
\newtheorem{vidu}{Ví dụ}
\usepackage[left=1cm,right=1cm,top=5mm,bottom=5mm,footskip=4mm]{geometry}
\def\labelitemii{$\circ$}
\DeclareRobustCommand{\divby}{%
	\mathrel{\vbox{\baselineskip.65ex\lineskiplimit0pt\hbox{.}\hbox{.}\hbox{.}}}%
}
\setlist[itemize]{leftmargin=*}
\setlist[enumerate]{leftmargin=*}

\title{Survey: Artificial Intelligence -- Khảo Sát: Trí Tuệ Nhân Tạo}
\author{Nguyễn Quản Bá Hồng\footnote{A scientist- {\it\&} creative artist wannabe, a mathematics {\it\&} computer science lecturer of Department of Artificial Intelligence {\it\&} Data Science (AIDS), School of Technology (SOT), UMT Trường Đại học Quản lý {\it\&} Công nghệ TP.HCM, Hồ Chí Minh City, Việt Nam.\\E-mail: {\sf nguyenquanbahong@gmail.com} {\it\&} {\sf hong.nguyenquanba@umt.edu.vn}. Website: \url{https://nqbh.github.io/}. GitHub: \url{https://github.com/NQBH}.}}
\date{\today}

\begin{document}
\maketitle
\begin{abstract}
	This text is a part of the series {\it Some Topics in Advanced STEM \& Beyond}:
	
	{\sc url}: \url{https://nqbh.github.io/advanced_STEM/}.
	
	Latest version:
	\begin{itemize}
		\item {\it Survey: Artificial Intelligence -- Khảo Sát: Trí Tuệ Nhân Tạo}.
		
		PDF: {\sc url}: \url{https://github.com/NQBH/advanced_STEM_beyond/blob/main/AI/NQBH_AI.pdf}.
		
		\TeX: {\sc url}: \url{https://github.com/NQBH/advanced_STEM_beyond/blob/main/AI/NQBH_AI.tex}.
		\item {\it Lecture Note: Introduction to Artificial Intelligence -- Bài Giảng: Nhập Môn Trí Tuệ Nhân Tạo}.
		
		PDF: {\sc url}: \url{https://github.com/NQBH/advanced_STEM_beyond/blob/main/AI/lecture/NQBH_introduction_AI_lecture.pdf}.
		
		\TeX: {\sc url}: \url{https://github.com/NQBH/advanced_STEM_beyond/blob/main/AI/lecture/NQBH_introduction_AI_lecture.tex}.
		\item Codes:
		\begin{itemize}
			\item C++: \url{https://github.com/NQBH/advanced_STEM_beyond/tree/main/AI/C++}.
			\item Python: \url{https://github.com/NQBH/advanced_STEM_beyond/tree/main/AI/Python}.
		\end{itemize}
	\end{itemize}
\end{abstract}
\tableofcontents

%------------------------------------------------------------------------------%

\section{Basic AI}

\subsection{\cite{Norvig_Russel2021}. {\sc Peter Norvig, Stuart Russell}. Artificial Intelligence: A Modern Approach}
{\sf[479 Amazon ratings][4365 Goodreads ratings]}
\begin{itemize}
	\item {\sf Amazon review.} The long-anticipated revision of {\it AI: A Modern Approach} explores full breadth \& depth of field of AI. 4e brings readers up to date on latest technologies, presents concepts in a more unified manner, \& offers new or expanded coverage of ML, DL, transfer learning, multi agent systems, robotics, NLP, causality, probabilistic programming, privacy, fairness, \& safe AI.
	\item {\sf Preface.} AI is a big field, \& this is a big book. Have tried to explore full breadth of field, which encompasses logic, probability, \& continuous mathematics; perception, reasoning, learning, \& actions; fairness, trust, social good, \& safety; \& applications that range from microelectronic devices to robotic planetary explorers to online services with billions of users.
	
	Subtitle of this book is ``A Modern Approach''. I.e., have chosen to tell story from a current perspective. Synthesize what is now known into a common framework, recasting early work using ideas \& terminology that are prevalent today. Apologize to those whose subfields are, as a result, less recognizable.
	\begin{itemize}
		\item {\sf New to 4e.} This edition reflects changes of AI since last edition in 2010:
		\begin{itemize}
			\item Focus more on ML rather than hand-crafted knowledge engineering, due to increased availability of data, computing resources, \& new algorithms.
			\item DL, probabilistic programming, \& multiagent systems receive expanded coverage, each with their own chap.
			\item Coverage of natural language understanding, robotics, \& computer vision has been revised to reflect impact of DL.
			\item Robotics chap now includes robots that interact with humans \& application of reinforcement learning to robotics.
			\item Previously defined goal of AI as creating systems that try to maximize expected utility, where specific utility information -- objective -- is supplied by human designers of system. Now we no longer assume: objective is fixed \& known by AI system; instead, system may be uncertain about true objectives of humans on whose behalf it operates. It must learn what to maximize \& must function appropriately even while uncertain about objective.
			\item Increase coverage of impact of AI on society, including vital issues of ethics, fairness, trust, \& safety.
			\item Have moved exercises from end of each chap to an online site. This allows us to continuously add to, update, \& improve exercises, to meet needs of instructors \& to reflect advances in field \& in AI-related software tools.
			\item Overall, about 25\% of material in book is brand new. Remaining 75\% has been largely rewritten to present a more unified picture of field. 22\% of citations of 4e are to works published after 2010.
		\end{itemize}
		\item {\sf Overview of book.} Main unifying theme is idea of an {\it intelligent agent}. Define AI as study of agents that receive percepts from environment \& perform actions. Each such agent implements a function that maps percept sequences to actions, \& cover different ways to represent these functions, e.g. reactive agents, real-time planners, decision-theoretic systems, \& DL systems. Emphasize learning both as a construction method for competent systems \& as a way of extending reach of designer into unknown environments. Treat robotics \& vision not as independently defined problems, but as occurring in service of achieving goals. Stress importance of task environment in determining appropriate agent design.
		
		-- Chủ đề thống nhất chính là ý tưởng về 1 {\it tác nhân thông minh}. Định nghĩa AI là nghiên cứu về các tác nhân tiếp nhận các nhận thức từ môi trường \& thực hiện các hành động. Mỗi tác nhân như vậy triển khai 1 chức năng ánh xạ các chuỗi nhận thức thành các hành động, \& bao gồm các cách khác nhau để biểu diễn các chức năng này, ví dụ như các tác nhân phản ứng, các nhà lập kế hoạch thời gian thực, các hệ thống lý thuyết quyết định, \& hệ thống DL. Nhấn mạnh việc học vừa là phương pháp xây dựng cho các hệ thống có năng lực \& như 1 cách mở rộng phạm vi của nhà thiết kế vào các môi trường chưa biết. Xử lý robot \& tầm nhìn không phải là các vấn đề được xác định độc lập, mà là diễn ra để phục vụ cho việc đạt được các mục tiêu. Nhấn mạnh tầm quan trọng của môi trường nhiệm vụ trong việc xác định thiết kế tác nhân phù hợp.
		
		Primary aim: convey {\it ideas} that have emerged over past 70 years of AI research \& past 2 millennia of related work. Have tried to avoid excessive formality in presentation of these ideas, while retaining precision. Have included mathematical formulas \& pseudocode algorithms to make key ideas concrete; mathematical concepts \& notation are described in Appendix A \& our pseudocode is described in Appendix B.
		
		-- Mục tiêu chính: truyền đạt {\it ý tưởng} đã xuất hiện trong hơn 70 năm nghiên cứu AI \& 2 thiên niên kỷ công trình liên quan. Đã cố gắng tránh sự trang trọng quá mức trong việc trình bày những ý tưởng này, đồng thời vẫn giữ được độ chính xác. Đã bao gồm các công thức toán học \& thuật toán mã giả để làm cho các ý tưởng chính trở nên cụ thể; các khái niệm toán học \& ký hiệu được mô tả trong Phụ lục A \& mã giả của chúng tôi được mô tả trong Phụ lục B.
		
		This book is primarily intended for use in an undergraduate course or course sequence. Book has 29 chaps, each requiring about a week's worth of lectures, so working through whole book requires a 2-semester sequence. 1 1-semester course can use selected chaps to suit interests of instructor \& students. Book can also be used in a graduate-level course (perhaps with addition of some of primary courses suggested in bibliographical notes), or for self-study or as a reference.
		
		Only prerequisite is familiarity with basic concepts of CS (algorithms, data structures, complexity) at a sophomore level. Freshman calculus \& linear algebra are useful for some of topics.
	\end{itemize}
	PART I: AI.
	\begin{itemize}
		\item {\sf1. Introduction.} In which we try to explain why we consider AI to be a subject most worthy of study, \& in which we try to decide what exactly it is, this being a good thing to decide before embarking.
		
		Call ourselves {\it Homo sapiens} -- man the wise -- because our {\it intelligence} is so important to us. For thousands of years, have tried to understand {\it how we think \& act} -- i.e., how our brain, a mere handful of matter, can perceive, understand, predict, \& manipulate a world far larger \& more complicated than itself. Field of AI is concerned with not just understanding but also {\it building} intelligent entities -- machines that can compute how to act effectively \& safely in a wide variety of novel situations.
		
		Surveys regularly rank AI as 1 of most interesting \& fastest-growing fields, \& already generating over a trillion dollars a year in revenue. AI expert {\sc Kai-Fu Lee} predicts: its impact will be ``more than anything in history of mankind''. Moreover, intellectual frontiers of AI are wide open. Whereas a student of an older science e.g. physics might feel best ideas have already been discovered by {\sc Galileo, Newton, Curie, Einstein}, \& the rest, AI still has many openings for full-time masterminds.
		
		AI currently encompasses a huge variety of subfields, ranging from general (learning, reasoning, perception, etc.) to specific, e.g. playing chess, proving mathematical theorems, writing poetry, driving a car, or diagnosing diseases. AI is relevant to any intellectual task; it is truly a universal field.
		\begin{itemize}
			\item {\sf1.1. What Is AI?} Have claimed: AI is interesting, but have not said what it is. Historically, researchers have pursued several different versions of AI. Some have defined intelligence in terms of fidelity to {\it human} performance, while others prefer an abstract, formal def of intelligence called {\it rationality} -- loosely speaking, doing ``right thing''. Subject matter itself also varies: some consider intelligence to be a property of internal {\it thought processes \& reasoning}, while others focus on intelligent {\it behavior}, an external characterization. [In public eye, there is sometimes confusion between terms ``AI'' \& ``ML''. ML is a subfield of AI that studies ability to improve performance based on experience. Some AI systems use ML methods to achieve competence, but some do not.]
			
			-- Đã tuyên bố: AI rất thú vị, nhưng chưa nói rõ nó là gì. Theo truyền thống, các nhà nghiên cứu đã theo đuổi 1 số phiên bản khác nhau của AI. Một số người đã định nghĩa trí thông minh theo nghĩa là sự trung thành với hiệu suất của {\it con người}, trong khi những người khác thích 1 định nghĩa trừu tượng, chính thức về trí thông minh được gọi là {\it lý trí} -- nói 1 cách rộng rãi, làm ``điều đúng đắn''. Bản thân chủ đề cũng khác nhau: 1 số coi trí thông minh là 1 đặc tính của {\it quá trình suy nghĩ \& lý luận} bên trong, trong khi những người khác tập trung vào {\it hành vi} thông minh, 1 đặc điểm bên ngoài. [Trong mắt công chúng, đôi khi có sự nhầm lẫn giữa các thuật ngữ ``AI'' \& ``ML''. ML là 1 lĩnh vực phụ của AI nghiên cứu khả năng cải thiện hiệu suất dựa trên kinh nghiệm. Một số hệ thống AI sử dụng các phương pháp ML để đạt được năng lực, nhưng 1 số thì không.]
			
			From these 2 dimensions -- human vs. rational [We are not suggesting that humans are ``irrational'' in dictionary sense of ``deprived of normal mental clarity''. We are merely conceding that human decisions are not always mathematically perfect.] \& thought vs. behavior -- there are 4 possible combinations, \& there have been adherents \& research programs $\forall$ 4. Methods used are necessarily different: pursuit of human-like intelligence must be in part an empirical science related to psychology, involving observations \& hypotheses about actual human behavior \& thought processes; a rationalist approach, on other hand, involves a combination of mathematics \& engineering, \& connects to statistics, control theory, \& economics. Various groups have both disparaged \& helped each other. Look at 4 approaches in more detail.
			
			--  Từ 2 chiều này -- con người so với lý trí [Chúng tôi không ám chỉ rằng con người ``phi lý trí'' theo nghĩa trong từ điển là ``thiếu sự minh mẫn bình thường về mặt tinh thần''. Chúng tôi chỉ thừa nhận rằng các quyết định của con người không phải lúc nào cũng hoàn hảo về mặt toán học.] \& suy nghĩ so với hành vi -- có 4 sự kết hợp có thể, \& đã có những người ủng hộ \& các chương trình nghiên cứu $\forall$ 4. Các phương pháp được sử dụng nhất thiết phải khác nhau: việc theo đuổi trí thông minh giống con người phải là 1 phần khoa học thực nghiệm liên quan đến tâm lý học, bao gồm các quan sát \& giả thuyết về hành vi thực tế của con người \& các quá trình suy nghĩ; mặt khác, 1 cách tiếp cận duy lý bao gồm sự kết hợp của toán học \& kỹ thuật, \& kết nối với thống kê, lý thuyết kiểm soát, \& kinh tế học. Nhiều nhóm đã coi thường \& giúp đỡ lẫn nhau. Hãy xem xét 4 cách tiếp cận chi tiết hơn.
			\begin{itemize}
				\item {\sf1.1.1. Acting humanly: Turing test approach.} {\it Turing test}, proposed by {\sc Alan Turing} (1950) was designed as a thought experiment that would sidestep philosophical vagueness of question ``Can a machine think?'' A computer passes test if a human interrogator, after posing some written questions, cannot tell whether written responses come from a person or from a computer. Chap. 28 discusses details of test \& whether a computer would really be intelligent if it passed. For now, note: programming a computer to pass a rigorously applied test provides plenty to work on. Computer would need following capabilities:
				\begin{enumerate}
					\item {\it natural language processing} to communicate successfully in a human language
					\item {\it knowledge representation} to store what it knows or hears
					\item {\it automated reasoning} to answer questions \& to draw new conclusions
					\item {\it ML} to adapt to new circumstances \& to detect \& extrapolate patterns.
				\end{enumerate}
				{\sc Turing} viewed {\it physical} simulation of a person as unnecessary to demonstrate intelligence. However, other researchers have proposed a {\it total Turing test}, which requires interaction with objects \& people in real world. To pass total Turing test, a robot will need
				\begin{enumerate}
					\item {\it computer vision} \& speech recognition to perceive world
					\item {\it robotics} to manipulate objects \& move about.
				\end{enumerate}
				These 6 disciplines compose most of AI. Yet AI researchers have devoted little effort to passing Turing test, believing: more important to study underlying principles of intelligence. Quest for ``artificial flight'' succeeded when engineers \& investors stopped imitating birds \& started using wind tunnels \& learning about aerodynamics. Aeronautical engineering texts do not define goal of their fields as making ``machines that fly so exactly like pigeons that they can fool even other pigeons''.
				
				-- 6 chuyên ngành này tạo nên phần lớn AI. Tuy nhiên, các nhà nghiên cứu AI đã dành ít nỗ lực để vượt qua bài kiểm tra Turing, tin rằng: quan trọng hơn là nghiên cứu các nguyên tắc cơ bản của trí thông minh. Nhiệm vụ tìm kiếm ``chuyến bay nhân tạo'' đã thành công khi các kỹ sư \& nhà đầu tư ngừng bắt chước chim \& bắt đầu sử dụng đường hầm gió \& tìm hiểu về khí động học. Các văn bản về kỹ thuật hàng không không xác định mục tiêu của lĩnh vực này là tạo ra ``những cỗ máy bay giống hệt chim bồ câu đến mức chúng có thể đánh lừa cả những con bồ câu khác''.
				\item {\sf1.1.2. Thinking humanly: cognitive modeling approach.} To say a program thinks like a human, must know how humans think. Can learn about human though in 3 ways:
				\begin{enumerate}
					\item {\it introspection} -- trying to catch our own thoughts as they go by
					\item {\it psychological experiments} -- observing a person in action
					\item {\it brain imaging} -- observing brain in action.
				\end{enumerate}
				Once have a sufficiently precise theory of mind, it becomes possible to express theory as a computer program. If program's input--output behavior matches corresponding human behavior, that is evidence: some of program's mechanisms could also be operating in humans.
				
				-- Một khi có 1 lý thuyết đủ chính xác về tâm trí, có thể diễn đạt lý thuyết như 1 chương trình máy tính. Nếu hành vi đầu vào-đầu ra của chương trình khớp với hành vi tương ứng của con người, thì đó là bằng chứng: 1 số cơ chế của chương trình cũng có thể hoạt động ở con người.
				
				E.g., {\sc Allen Newell \& Herbert Simon}, who developed GPS, ``General Problem Solver'' (Newell \& Simon, 1961), were not content merely to have their program solve problems correctly. They were more concerned with comparing sequence \& timing of its reasoning steps to those of human subjects solving same problems. Interdisciplinary field of {\it cognitive science} brings together computer models from AI \& experimental techniques from psychology to construct precise \& testable theories of human mind.
				
				-- Ví dụ, {\sc Allen Newell \& Herbert Simon}, người đã phát triển GPS, ``General Problem Solver'' (Newell \& Simon, 1961), không chỉ hài lòng với việc chương trình của họ giải quyết vấn đề 1 cách chính xác. Họ quan tâm nhiều hơn đến việc so sánh trình tự \& thời gian của các bước lý luận của nó với trình tự của con người giải quyết cùng 1 vấn đề. Lĩnh vực liên ngành của {\it khoa học nhận thức} tập hợp các mô hình máy tính từ AI \& các kỹ thuật thử nghiệm từ tâm lý học để xây dựng các lý thuyết chính xác \& có thể kiểm chứng về tâm trí con người.
				
				Cognitive science is a fascinating field in itself, worthy of several textbooks \& at least 1 encyclopedia (Wilson \& Keil, 1999). Will occasionally comment on similarities or differences between AI techniques \& human cognition. Real cognition science, however, is necessary based on experimental investigation of actual humans or animals. Leave that for other books, as assume reader has only a computer for experimentation.
				
				-- Khoa học nhận thức là 1 lĩnh vực hấp dẫn, xứng đáng với 1 số sách giáo khoa \& ít nhất 1 bách khoa toàn thư (Wilson \& Keil, 1999). Thỉnh thoảng sẽ bình luận về điểm tương đồng hoặc khác biệt giữa các kỹ thuật AI \& nhận thức của con người. Tuy nhiên, khoa học nhận thức thực sự là cần thiết dựa trên nghiên cứu thực nghiệm trên con người hoặc động vật thực tế. Hãy để dành điều đó cho các cuốn sách khác, vì giả sử người đọc chỉ có máy tính để thử nghiệm.
				
				In early days of AI there was often confusion between approaches. An author would argue: an algorithm performs well on a task \& therefor a good model of human performance, or vice versa. Modern authors separate 2 kinds of claims; this distinction has allowed both AI \& cognitive science to develop more rapidly. 2 fields fertilize each other, most notably in computer vision, which incorporates neurophysiological evidence into computational models. Recently, combination of neuroimaging methods combined with ML techniques for analyzing such data has led to beginnings of a capability to ``read minds'' -- i.e., to ascertain semantic content of a person's inner thoughts. This capability could, in turn, shed further light on how human cognitive works.
				
				-- Vào những ngày đầu của AI, thường có sự nhầm lẫn giữa các cách tiếp cận. Một tác giả sẽ lập luận: 1 thuật toán thực hiện tốt 1 nhiệm vụ \& do đó là 1 mô hình tốt về hiệu suất của con người, hoặc ngược lại. Các tác giả hiện đại tách biệt 2 loại tuyên bố; sự phân biệt này đã cho phép cả AI \& khoa học nhận thức phát triển nhanh hơn. 2 lĩnh vực này hỗ trợ lẫn nhau, đáng chú ý nhất là trong lĩnh vực thị giác máy tính, nơi kết hợp bằng chứng thần kinh sinh lý vào các mô hình tính toán. Gần đây, sự kết hợp của các phương pháp chụp ảnh thần kinh kết hợp với các kỹ thuật ML để phân tích dữ liệu như vậy đã dẫn đến sự khởi đầu của khả năng ``đọc suy nghĩ'' -- tức là xác định nội dung ngữ nghĩa của những suy nghĩ bên trong của 1 người. Đến lượt mình, khả năng này có thể làm sáng tỏ thêm cách thức hoạt động của nhận thức của con người.
				\item {\sf1.1.3. Thinking rationally: ``law of thought'' approach.} Greek philosopher {\sc Aristotle} was 1 of 1st attempt to codify ``right thinking'' -- i.e., irrefutable reasoning processes. His {\it syllogisms} provided patterns for argument structures that always yielded correct conclusions when given correct premises. Canonical example starts with {\it Socrates is a man \& all men are moral} \& concludes {\it Socrates is mortal}. (This example is probably due to {\sc Sextus Empiricus} rather than {\sc Aristotle}). These laws of thought were supposed to govern operation of mind; their study initiated field called {\it logic}.
				
				Logicians in 19th century developed a precise notation for statements about objects in world \& relations among them. (Contrast this with ordinary arithmetic notation, which provides only for statements about {\it numbers}.) By 1965, programs could, in principle, solve {\it any} solvable problem described in logical notation. So-called {\it logicist} tradition within AI hopes to build on such programs to create intelligent systems.
				
				-- Các nhà logic học vào thế kỷ 19 đã phát triển 1 ký hiệu chính xác cho các phát biểu về các đối tượng trong thế giới \& các mối quan hệ giữa chúng. (Đối chiếu điều này với ký hiệu số học thông thường, chỉ cung cấp các phát biểu về {\it số}.) Đến năm 1965, về nguyên tắc, các chương trình có thể giải quyết {\it bất kỳ} vấn đề có thể giải quyết nào được mô tả bằng ký hiệu logic. Cái gọi là truyền thống logic trong AI hy vọng sẽ xây dựng trên các chương trình như vậy để tạo ra các hệ thống thông minh.
				
				Logic as conventionally understood requires knowledge of world that is {\it certain} -- a condition that, in reality, is seldom achieved. Simply don't know rules of, say, politics or warfare in same way that we know rules of chess or arithmetic. Theory of {\it probability} fills this gap, allowing rigorous reasoning with uncertain information. In principle, it allows construction of a comprehensive model of rational thought, leading from raw perceptual information to an understanding of how world works to predictions about future. What it does not do, is generate intelligent {\it behavior}. For that, we need a theory of rational action. Rational thought, by itself, is not enough.
				
				-- Logic theo cách hiểu thông thường đòi hỏi kiến thức về thế giới {\it chắc chắn} -- 1 điều kiện mà trên thực tế hiếm khi đạt được. Đơn giản là không biết các quy tắc của, chẳng hạn, chính trị hay chiến tranh theo cùng cách mà chúng ta biết các quy tắc của cờ vua hay số học. Lý thuyết về {\it xác suất} lấp đầy khoảng trống này, cho phép lý luận chặt chẽ với thông tin không chắc chắn. Về nguyên tắc, nó cho phép xây dựng 1 mô hình toàn diện về tư duy hợp lý, dẫn từ thông tin nhận thức thô sơ đến sự hiểu biết về cách thế giới hoạt động để dự đoán về tương lai. Điều mà nó không làm được là tạo ra {\it hành vi} thông minh. Đối với điều đó, chúng ta cần 1 lý thuyết về hành động hợp lý. Tư duy hợp lý, tự nó, là không đủ.
				\item {\sf1.1.4. Acting rationally: rational agent approach.} An {\it agent} is juts sth that acts ({\it agent} comes from Latin {\it agere}, to do). Of course, all computer programs do sth, but computer agents are expected to do more: operate autonomously, perceive their environment, persist over a prolonged time period, adapt to change, \& create \& pursue goals. A {\it rational agent} is one that acts so as to achieve best outcome or, when there is uncertainty, best expected outcome.
				
				-- {\it Hành động hợp lý: phương pháp tiếp cận tác nhân hợp lý.} Một tác nhân chỉ là cái gì đó hành động (tác nhân bắt nguồn từ tiếng Latin agere, nghĩa là làm). Tất nhiên, tất cả các chương trình máy tính đều làm cái gì đó, nhưng các tác nhân máy tính được kỳ vọng sẽ làm nhiều hơn thế: hoạt động tự chủ, nhận thức môi trường của chúng, tồn tại trong 1 khoảng thời gian dài, thích nghi với sự thay đổi, \& tạo ra \& theo đuổi mục tiêu. Một tác nhân hợp lý là tác nhân hành động để đạt được kết quả tốt nhất hoặc, khi có sự không chắc chắn, kết quả mong đợi tốt nhất.
				
				In ``laws of thought'' approach to AI, emphasis was on correct inferences. Making correct inferences is sometimes {\it part}  of being a rational agent, because 1 way to act rationally: deduce that a given action is best \& then to act on that conclusion. On other hand, there are ways of acting rationally that cannot be said to involve inference. E.g., recoiling from a hot stove is a reflex action that is usually more successful than a slower action taken after careful deliberation.
				
				-- Trong cách tiếp cận ``luật tư duy'' đối với AI, trọng tâm là suy luận đúng. Việc đưa ra suy luận đúng đôi khi là {\it 1 phần} của việc trở thành 1 tác nhân lý trí, bởi vì 1 cách để hành động hợp lý: suy ra rằng 1 hành động nhất định là tốt nhất \& sau đó hành động theo kết luận đó. Mặt khác, có những cách hành động hợp lý không thể nói là liên quan đến suy luận. Ví dụ, lùi lại khỏi bếp nóng là 1 hành động phản xạ thường thành công hơn so với hành động chậm hơn được thực hiện sau khi cân nhắc kỹ lưỡng.
				
				All skills needed for Turing test also allow an agent to act rationally. Knowledge representation \& reasoning enable agents to reach good decisions. Need to be able to generate comprehensible sentences in natural language to get by in a complex society. Need learning not only for erudition, but also because it improves our ability to generate effective behavior, especially in circumstances that are new.
				
				-- Tất cả các kỹ năng cần thiết cho bài kiểm tra Turing cũng cho phép 1 tác nhân hành động hợp lý. Biểu diễn kiến thức \& lý luận cho phép các tác nhân đưa ra quyết định đúng đắn. Cần có khả năng tạo ra các câu dễ hiểu bằng ngôn ngữ tự nhiên để tồn tại trong 1 xã hội phức tạp. Cần học không chỉ để có kiến thức uyên bác mà còn vì nó cải thiện khả năng tạo ra hành vi hiệu quả của chúng ta, đặc biệt là trong những hoàn cảnh mới.
				
				Rational-agent approach to AI has 2 advantages over other approaches. 1st, it is more general than ``law of thought'' approach because correct inference is just 1 of several possible mechanism for achieving rationality. 2nd, more amenable to scientific development. Standard of rationality is mathematically well defined \& completely general. Can often work back from this specification to derive agent designs that provably achieve it -- sth that is largely impossible if goal: imitate human behavior or thought processes.
				
				-- Phương pháp tiếp cận tác nhân hợp lý đối với AI có 2 ưu điểm so với các phương pháp tiếp cận khác. Thứ nhất, nó tổng quát hơn phương pháp tiếp cận ``luật tư duy'' vì suy luận đúng chỉ là 1 trong số nhiều cơ chế có thể đạt được tính hợp lý. Thứ hai, dễ tiếp thu hơn đối với sự phát triển khoa học. Tiêu chuẩn của tính hợp lý được định nghĩa rõ ràng về mặt toán học \& hoàn toàn tổng quát. Thường có thể làm việc ngược lại từ thông số kỹ thuật này để đưa ra các thiết kế tác nhân có thể chứng minh được là đạt được nó -- điều mà phần lớn là không thể nếu mục tiêu: bắt chước hành vi hoặc quá trình suy nghĩ của con người.
				
				For these reasons, rational-agent approach to AI has prevailed throughout most of field's history. In early decades, rational agents were built on logical foundations \& formed definite plans to achieve specific goals. Later, methods based on probability theory \& ML allowed creation of agents that could make decisions under uncertainty to attain best expected outcome. In a nutshell, {\it AI has focused on study \& construction of agents that do right thing}. What counts as right thing is defined by objective that we provide to agent. This general paradigm is so pervasive that we might call it {\it standard model}. It prevails not only in AI, but also in control theory, where a controller minimizes a cost function; in operations research, where a policy maximizes a sum of rewards; in statistics, where a decision rule minimizes a loss function; \& in economics, where a decision maker maximizes utility or some measure of social welfare.
				
				-- Vì những lý do này, cách tiếp cận tác nhân hợp lý đối với AI đã chiếm ưu thế trong hầu hết lịch sử của lĩnh vực này. Trong những thập kỷ đầu, các tác nhân hợp lý được xây dựng trên nền tảng logic \& hình thành các kế hoạch chắc chắn để đạt được các mục tiêu cụ thể. Sau đó, các phương pháp dựa trên lý thuyết xác suất \& ML cho phép tạo ra các tác nhân có thể đưa ra quyết định trong điều kiện không chắc chắn để đạt được kết quả mong đợi tốt nhất. Tóm lại, {\it AI đã tập trung vào nghiên cứu \& xây dựng các tác nhân làm điều đúng đắn}. Những gì được coi là điều đúng đắn được xác định bởi mục tiêu mà chúng ta cung cấp cho tác nhân. Mô hình chung này rất phổ biến đến mức chúng ta có thể gọi nó là {\it mô hình chuẩn}. Nó không chỉ chiếm ưu thế trong AI mà còn trong lý thuyết điều khiển, trong đó bộ điều khiển giảm thiểu hàm chi phí; trong nghiên cứu hoạt động, trong đó chính sách tối đa hóa tổng phần thưởng; trong thống kê, trong đó quy tắc quyết định giảm thiểu hàm mất mát; \& trong kinh tế, trong đó người ra quyết định tối đa hóa tiện ích hoặc 1 số biện pháp phúc lợi xã hội.
				
				Need to make 1 important refinement to standard model to account for fact that perfect rationality -- always taking exactly optimal action -- is not feasible in complex environments. Computational demands are just too high. Chaps. 6 \& 16 deals with issue of {\it limited rationality} -- acting appropriately when there is not enough time to do all computations one might like. However, perfect rationality often remains a good starting point for theoretical analysis.
				
				-- Cần thực hiện 1 cải tiến quan trọng đối với mô hình chuẩn để tính đến thực tế là tính hợp lý hoàn hảo -- luôn thực hiện hành động tối ưu chính xác -- là không khả thi trong các môi trường phức tạp. Yêu cầu tính toán quá cao. Chương 6 \& 16 giải quyết vấn đề về {\it tính hợp lý hạn chế} -- hành động phù hợp khi không có đủ thời gian để thực hiện tất cả các phép tính mà người ta có thể thích. Tuy nhiên, tính hợp lý hoàn hảo thường vẫn là điểm khởi đầu tốt cho phân tích lý thuyết.				
				\item {\sf1.1.5. Beneficial machines.} Standard model has been a useful guide for AI research since its inception, but it is probably not right model in long run. Reason: standard model assumes: will supply a fully specified objective to machine.
				
				-- Mô hình chuẩn đã là 1 hướng dẫn hữu ích cho nghiên cứu AI kể từ khi ra đời, nhưng có lẽ về lâu dài, nó không phải là mô hình phù hợp. Lý do: mô hình chuẩn giả định: sẽ cung cấp 1 mục tiêu được chỉ định đầy đủ cho máy.
								
				For an artificially defined task e.g. chess or shortest-path computation, task comes with an objective built in -- so standard model is applicable. As move into real world, however, it becomes more \& more difficult to specify objective completely \& correctly. E.g., in designing a self-driving car, one might think: objective is to reach destination safely. But driving along any road incurs a risk of injury due to other errant drivers, equipment failure, etc.; thus, a strict goal of safety requires staying in garage. There is a tradeoff between making progress towards destination \& incurring a risk of injury. How should this tradeoff be made? Furthermore, to what extent can we allow car to take actions that would annoy other drivers? How much should car moderate its acceleration, steering, \& braking to avoid shaking up passenger? These kinds of questions are difficult to answer a priori. They are particularly problematic in general area of human--robot interaction, of which self-driving car is 1 example.
				
				-- Đối với 1 nhiệm vụ được xác định nhân tạo, ví dụ như cờ vua hoặc tính toán đường đi ngắn nhất, nhiệm vụ đi kèm với 1 mục tiêu được tích hợp sẵn -- do đó, mô hình chuẩn có thể áp dụng được. Tuy nhiên, khi chuyển sang thế giới thực, việc xác định mục tiêu 1 cách hoàn chỉnh \& chính xác trở nên \& khó khăn hơn. Ví dụ, khi thiết kế 1 chiếc xe tự lái, người ta có thể nghĩ: mục tiêu là đến đích an toàn. Nhưng việc lái xe trên bất kỳ con đường nào cũng có nguy cơ bị thương do những người lái xe khác đi sai đường, hỏng thiết bị, v.v.; do đó, mục tiêu an toàn nghiêm ngặt đòi hỏi phải ở trong gara. Có 1 sự đánh đổi giữa việc tiến về đích \& với nguy cơ bị thương. Sự đánh đổi này nên được thực hiện như thế nào? Hơn nữa, chúng ta có thể cho phép xe thực hiện những hành động có thể làm phiền những người lái xe khác ở mức độ nào? Xe nên điều chỉnh gia tốc, đánh lái, \& phanh ở mức nào để tránh làm rung chuyển hành khách? Những loại câu hỏi này rất khó trả lời trước. Chúng đặc biệt có vấn đề trong lĩnh vực tương tác giữa con người và rô-bốt nói chung, trong đó xe tự lái là 1 ví dụ.
				
				Problem of achieving agreement between our true preferences \& objective we put into machine is called {\it value alignment problem}: values or objectives put into machine must be aligned with those of human. In we are developing an AI system in lab or in a simulator -- as has been case for most of field's history -- there is an easy fix for an incorrectly specified objective: reset system, fix objective, \& try again. As field progresses towards increasingly capable intelligent systems that are deployed in rewal world, this approach is no longer viable. A system deployed with an incorrect objective will have negative consequences. Moreover, more intelligent system, more negative consequences.
				
				-- Vấn đề đạt được sự đồng thuận giữa sở thích thực sự của chúng ta \& mục tiêu mà chúng ta đưa vào máy được gọi là vấn đề căn chỉnh giá trị: các giá trị hoặc mục tiêu đưa vào máy phải phù hợp với mục tiêu hoặc mục tiêu của con người. Khi chúng ta đang phát triển 1 hệ thống AI trong phòng thí nghiệm hoặc trong trình mô phỏng -- như đã từng xảy ra trong hầu hết lịch sử của lĩnh vực này -- có 1 cách khắc phục dễ dàng cho 1 mục tiêu được chỉ định không chính xác: đặt lại hệ thống, sửa mục tiêu, \& thử lại. Khi lĩnh vực này tiến triển theo hướng các hệ thống thông minh ngày càng có khả năng được triển khai trong thế giới mới, cách tiếp cận này không còn khả thi nữa. Một hệ thống được triển khai với mục tiêu không chính xác sẽ có hậu quả tiêu cực. Hơn nữa, hệ thống càng thông minh thì hậu quả tiêu cực càng nhiều.
				
				Returning to apparently unproblematic example of chess, consider what happens if machine is intelligent enough to reason \& act beyond confines of chessboard. In that case, it might attempt to increase its chances of winning by such ruses as hypnotizing or blackmailing its opponent or bribing audience to make rustling noises during its opponent's thinking time. [In 1 of 1st books on chess, {\sc Ruy Lopez} (1561) wrote, ``Always place board so sun is in your opponent's eyes.''] It might also attempt to hijack additional computing power for itself. {\it These behaviors are not ``unintelligent'' or ``insane''; they are a logical consequence of defining winning as the sole objective for machine}.
				
				-- Quay trở lại ví dụ cờ vua có vẻ không có vấn đề gì, hãy xem xét điều gì xảy ra nếu máy đủ thông minh để lý luận \& hành động vượt ra ngoài giới hạn của bàn cờ. Trong trường hợp đó, nó có thể cố gắng tăng cơ hội chiến thắng bằng những mánh khóe như thôi miên hoặc tống tiền đối thủ hoặc hối lộ khán giả tạo ra tiếng sột soạt trong thời gian suy nghĩ của đối thủ. [Trong 1 trong những cuốn sách đầu tiên về cờ vua, {\sc Ruy Lopez} (1561) đã viết, ``Luôn đặt bàn cờ sao cho mặt trời chiếu vào mắt đối thủ.''] Nó cũng có thể cố gắng chiếm đoạt thêm sức mạnh tính toán cho chính nó. {\it Những hành vi này không phải là ``không thông minh'' hay ``điên rồ''; chúng là hệ quả hợp lý của việc xác định chiến thắng là mục tiêu duy nhất của máy}.
				
				Impossible to anticipate all ways in which a machine pursuing a fixed objective might misbehave. There is good reason, then, to think: standard model is inadequate. We don't want machines that are intelligent in sense of pursuing {\it their} objectives; want them to pursue {\it our} objectives. If cannot transfer those objectives perfectly to machine, then need a new formulation -- one in which machine is pursuing our objectives, but is necessarily {\it uncertain} as to what they are. When a machine knows that it doesn't know complete objective, it has an incentive to act cautiously, to ask permission, to learn more about our preferences through observation, \& to defer to human control. Ultimately, want agents that are {\it provably beneficial} to humans.
				
				-- Không thể lường trước được mọi cách mà 1 cỗ máy theo đuổi 1 mục tiêu cố định có thể hoạt động không đúng. Do đó, có lý do chính đáng để nghĩ rằng: mô hình chuẩn là không đủ. Chúng ta không muốn những cỗ máy thông minh theo nghĩa theo đuổi {\it là mục tiêu của chúng}; muốn chúng theo đuổi {\it là mục tiêu của chúng ta}. Nếu không thể chuyển những mục tiêu đó 1 cách hoàn hảo cho cỗ máy, thì cần 1 công thức mới -- công thức mà trong đó cỗ máy đang theo đuổi mục tiêu của chúng ta, nhưng nhất thiết {\it không chắc chắn} về mục tiêu đó là gì. Khi 1 cỗ máy biết rằng nó không biết mục tiêu hoàn chỉnh, nó có động cơ để hành động thận trọng, để xin phép, để tìm hiểu thêm về sở thích của chúng ta thông qua quan sát, \& để tuân theo sự kiểm soát của con người. Cuối cùng, muốn các tác nhân {\it có thể chứng minh được là có lợi} cho con người.
			\end{itemize}
			\item {\sf1.2. Foundations of AI.} In this sect, provide a brief history of disciplines that contributed ideas, viewpoints, \& techniques to AI. Like any history, this one concentrates on a small number of people, events, \& ideas \& ignores others that also were important. Organize history around a series of questions. Certainly would not wish to give impression that these questions are only ones the disciplines address or disciplines have all been working toward AI as their ultimate fruition.
			
			-- {\it Nền tảng của AI.} Trong phần này, hãy cung cấp 1 lịch sử tóm tắt về các ngành đã đóng góp ý tưởng, quan điểm, \& kỹ thuật cho AI. Giống như bất kỳ lịch sử nào, phần này tập trung vào 1 số ít người, sự kiện, \& ý tưởng \& bỏ qua những người khác cũng quan trọng. Sắp xếp lịch sử xung quanh 1 loạt các câu hỏi. Chắc chắn không muốn tạo ấn tượng rằng đây chỉ là những câu hỏi mà các ngành giải quyết hoặc tất cả các ngành đều hướng tới AI như là thành quả cuối cùng của họ.
			\begin{itemize}
				\item {\sf1.2.1. Philosophy.}
				\begin{enumerate}
					\item Can formal rules be used to draw valid conclusions?
					\item How does mind arise from a physical brain?
					\item Where does knowledge come from?
					\item How does knowledge lead to action?
				\end{enumerate}
				p. 24+++
			\end{itemize}
		\end{itemize}
		\item {\sf2. Intelligent Agents.} In which we discuss nature of agents, perfect or otherwise, diversity of environments, \& resulting menagerie of agent types.
		
		-- {\it Các tác nhân thông minh.} Trong đó chúng ta thảo luận về bản chất của các tác nhân, hoàn hảo hay không, sự đa dạng của môi trường, \& sự kết hợp của các loại tác nhân.
		
		Chap. 1 identified concept of {\it rational agents} as central as our approach to AI. In this chap, make this notion more concrete. See: concept of rationality can be applied to a wide variety of agents operating in any imaginable environment. Our plan in this book: use this concept to develop a small set of design principles for building successful agents -- systems that can reasonably be called {\it intelligent}.
		
		-- Chương 1 xác định khái niệm về các tác nhân hợp lý là trung tâm trong cách tiếp cận của chúng ta đối với AI. Trong chương này, hãy cụ thể hóa khái niệm này hơn. Xem: khái niệm về tính hợp lý có thể được áp dụng cho nhiều loại tác nhân hoạt động trong bất kỳ môi trường nào có thể tưởng tượng được. Kế hoạch của chúng tôi trong cuốn sách này: sử dụng khái niệm này để phát triển 1 tập hợp nhỏ các nguyên tắc thiết kế nhằm xây dựng các tác nhân thành công -- các hệ thống có thể được gọi 1 cách hợp lý là thông minh.
		
		Begin by examining agents, environments, \& coupling between them. Observation that some agents behave better than others leads naturally to idea of a rational agent -- one that behaves as well as possible. How well an agent can behave depends on nature of environment; some environments are more difficult than others. Give a crude categorization of environments \& show how properties of an environment influence design of suitable agents for that environment. Describe a number of basic ``skeleton'' agent designs, which flesh out in rest of book.
		
		-- Bắt đầu bằng cách xem xét các tác nhân, môi trường, \& sự kết hợp giữa chúng. Quan sát thấy 1 số tác nhân hành xử tốt hơn những tác nhân khác dẫn đến ý tưởng về 1 tác nhân hợp lý -- 1 tác nhân hành xử tốt nhất có thể. Mức độ 1 tác nhân có thể hành xử tốt như thế nào phụ thuộc vào bản chất của môi trường; 1 số môi trường khó hơn những môi trường khác. Đưa ra 1 phân loại thô sơ về các môi trường \& cho thấy các đặc tính của 1 môi trường ảnh hưởng đến thiết kế các tác nhân phù hợp cho môi trường đó như thế nào. Mô tả 1 số thiết kế tác nhân ``bộ xương'' cơ bản, được trình bày chi tiết trong phần còn lại của cuốn sách.
		\begin{itemize}
			\item {\sf2.1. Agents \& Environments.} An {\it agent} is anything that can be viewed as perceiving its {\it environment} through {\it sensors} \& acting upon that environment through {\it actuators}. This simple idea is illustrated in {\sf Fig. 2.1: Agents interact with environments through sensors \& actuators}. A human agent has eyes, ears, \& other organs for sensors \& hands, legs, vocal tract, \& so on for actuators. A robotic agent might have cameras \& infrared range finders for sensors \& various motors for actuators. A software agent receives file contents, network packets, \& human input (keyboard{\tt/}mouse{\tt/}touchscreen{\tt/}voice) as sensory inputs \& acts on environment by writing files, sending network packets, \& displaying information or generating sounds. Environment could be everything -- entire universe! In practice it is just that part of universe whose state we care about when designing this agent -- part that affects what agent perceives \& is affected by agent's actions.
			
			-- {\it Tác nhân \& Môi trường.} Một tác nhân là bất kỳ thứ gì có thể được xem như nhận thức môi trường của nó thông qua các cảm biến \& tác động lên môi trường đó thông qua các bộ truyền động. Ý tưởng đơn giản này được minh họa trong {\sf Hình 2.1: Các tác nhân tương tác với môi trường thông qua các cảm biến \& bộ truyền động}. Một tác nhân con người có mắt, tai, \& các cơ quan khác để cảm biến \& tay, chân, đường thanh quản, \& v.v. để truyền động. Một tác nhân rô bốt có thể có camera \& máy đo khoảng cách hồng ngoại để cảm biến \& nhiều động cơ khác nhau để truyền động. Một tác nhân phần mềm nhận nội dung tệp, các gói mạng, \& đầu vào của con người (bàn phím{\tt/}chuột{\tt/}màn hình cảm ứng{\tt/}giọng nói) làm đầu vào cảm biến \& tác động lên môi trường bằng cách ghi tệp, gửi các gói mạng, \& hiển thị thông tin hoặc tạo ra âm thanh. Môi trường có thể là tất cả mọi thứ -- toàn bộ vũ trụ! Trên thực tế, chỉ có 1 phần của vũ trụ mà chúng ta quan tâm đến trạng thái của nó khi thiết kế tác nhân này -- phần ảnh hưởng đến những gì tác nhân nhận thức \& bị ảnh hưởng bởi các hành động của tác nhân.
			
			Use term {\it percept} to refer to content an agent's sensors are perceiving. An agent's {\it percept sequence} is complete history of everything agent has ever perceived. In general, {\it an agent's choice of action at any given instant can depend on its built-in knowledge \& on entire percept sequence observed to date, but not on anything it hasn't perceived}. By specifying agent's choice of action for every possible percept sequence, have said more or less everything there is to say about agent. Mathematically speaking, say: an agent's behavior is described by {\it agent function} that maps any given percept sequence to an action.
			
			-- Sử dụng thuật ngữ percept để chỉ nội dung mà các cảm biến của tác nhân đang nhận thức. Chuỗi nhận thức của tác nhân là lịch sử hoàn chỉnh về mọi thứ mà tác nhân từng nhận thức. Nhìn chung, {\it lựa chọn hành động của tác nhân tại bất kỳ thời điểm nào có thể phụ thuộc vào kiến thức tích hợp của tác nhân \& vào toàn bộ chuỗi nhận thức được quan sát cho đến nay, nhưng không phụ thuộc vào bất kỳ điều gì mà tác nhân chưa nhận thức}. Bằng cách chỉ định lựa chọn hành động của tác nhân cho mọi chuỗi nhận thức có thể, đã nói ít nhiều mọi thứ cần nói về tác nhân. Về mặt toán học, nói: hành vi của tác nhân được mô tả bởi {\it hàm tác nhân} ánh xạ bất kỳ chuỗi nhận thức nào cho 1 hành động.
			
			Can imagine {\it tabulating} agent function that describes any given agent; for most agents, this would be a very large table -- infinite, in fact, unless we place a bound on length of percept sequences we want to consider. Given an agent to experiment with, we can, in principle, construct this table by trying out all possible percept sequences \& recording which actions agent does in response. [If agent uses some randomization to choose its actions, then would have to try each sequence many times to identify probability of each action. One might imagine: acting randomly is rather silly, but show later in this chap: it can be very intelligent.] Table is, of course, an {\it external} characterization of agent. {\it Internally}, agent function for an artificial agent will be implemented by an {\it agent program}. Important to keep these 2 ideas distinct. Agent function is an abstract mathematical description; agent program is a concrete implementation, running within some physical system.
			
			-- Có thể tưởng tượng việc lập bảng hàm tác nhân mô tả bất kỳ tác nhân nào; đối với hầu hết các tác nhân, đây sẽ là 1 bảng rất lớn -- vô hạn, trên thực tế, trừ khi chúng ta đặt 1 giới hạn về độ dài của các chuỗi nhận thức mà chúng ta muốn xem xét. Với 1 tác nhân để thử nghiệm, về nguyên tắc, chúng ta có thể xây dựng bảng này bằng cách thử tất cả các chuỗi nhận thức có thể \& ghi lại hành động mà tác nhân thực hiện để phản hồi. [Nếu tác nhân sử dụng 1 số ngẫu nhiên để chọn hành động của mình, thì sẽ phải thử từng chuỗi nhiều lần để xác định xác suất của từng hành động. Người ta có thể tưởng tượng: hành động ngẫu nhiên khá ngớ ngẩn, nhưng sẽ hiển thị sau trong chương này: nó có thể rất thông minh.] Tất nhiên, bảng là 1 đặc điểm {\it bên ngoài} của tác nhân. {\it Bên trong}, hàm tác nhân cho 1 tác nhân nhân tạo sẽ được triển khai bởi 1 {\it chương trình tác nhân}. Điều quan trọng là phải giữ cho 2 ý tưởng này riêng biệt. Hàm tác nhân là 1 mô tả toán học trừu tượng; chương trình tác nhân là 1 triển khai cụ thể, chạy trong 1 số hệ thống vật lý.
			
			To illustrate these ideas, use a simple example -- vacuum-cleaner world, which consists of a robotic vacuum-cleaning agent in a world consisting squares that can be either dirty or clean. {\sf Fig. 2.2: A vacuum-cleaner world with just 2 locations. Each location can be clean or dirty, \& agent can move left or right \& can clean square that it occupies. Different versions of vacuum world allow for different rules about what agent can perceive, whether its actions always succeed, \& so on.} shows a configuration with just 2 squares, A \& B. Vacuum agent perceives which square it is in \& whether there is dirt in square. Agent starts in square A. Available actions are to move to right, move to left, suck up dirt, or do nothing. [In a real robot, it would be unlikely to have an actions like ``move right'' \& ``move left''. Instead actions would be ``spin wheels forward'' \& ``spin wheels backward''. Have chosen actions to be easier to follow on page, not for ease of implementation in an actual robot.] 1 very simple agent function is following: if current square is dirty, then suck; otherwise, move to other square. A partial tabulation of this agent function is shown in {\sf Fig. 2.3: Partial tabulation of a simple agent function for vacuum-cleaner world shown in Fig. 2.2. Agent cleans current square if it is dirty, otherwise it moves to other square. Note: table is of unbounded size unless there is a restriction on length of possible percept sequences.} \& an agent program that implements it appears in {\sf Fig. 2.8: Agent program for a simple reflex agent in 2-location vacuum environment. This program implements agent function tabulated in Fig. 2.3.}
			
			Looking at Fig. 2.3, see: various vacuum-world agents can be defined simply by filling in RH column in various ways. Obvious question, then: {\it What is right way to fill out table?} I.e., what makes an agent good or bad, intelligent or stupid? Answer these questions in next sect.
			
			Before close this sect, should emphasize: notion of an agent is meant to be a tool for analyzing systems, not an absolute characterization that divides world into agents \& non-agents. One could view a hand-held calculator as an agent that chooses action of displaying ``4'' when given percept sequence ``$2 + 2 =''$, but such an analysis would hardly aid our understanding of calculator. In a sense, all areas of engineering can be seen as designing artifacts that interact with world; AI operates at (what authors consider to be) most interesting end of spectrum, where artifacts have significant computational resources \& task environment requires nontrivial decision making.
			
			-- Trước khi kết thúc phần này, cần nhấn mạnh: khái niệm về tác nhân được hiểu là 1 công cụ để phân tích hệ thống, không phải là 1 đặc điểm tuyệt đối chia thế giới thành tác nhân \& không phải tác nhân. Người ta có thể xem máy tính cầm tay như 1 tác nhân chọn hành động hiển thị ``4'' khi đưa ra chuỗi nhận thức ``$2 + 2 =''$, nhưng 1 phân tích như vậy khó có thể giúp chúng ta hiểu máy tính. Theo 1 nghĩa nào đó, tất cả các lĩnh vực kỹ thuật đều có thể được coi là thiết kế các hiện vật tương tác với thế giới; AI hoạt động ở (những gì các tác giả coi là) phần cuối của quang phổ thú vị nhất, nơi các hiện vật có tài nguyên tính toán đáng kể \& môi trường tác vụ đòi hỏi phải đưa ra quyết định không tầm thường.
			\item {\sf2.2. Good Behavior: Concept of Rationality.} A {\it rational agent} is one that does right thing. Obviously, doing right thing is better than doing wrong thing, but what does it mean to do right thing?
			\begin{itemize}
				\item {\sf2.2.1. Performance measures.} Moral philosophy has developed several different notions of ``right thing'', but AI has generally stuck to 1 notion called {\it consequentialism}: evaluate an agent's behavior by its consequences. When an agent is plunked down in an environment, it generates a sequence of actions according to percepts it receives. This sequence of actions causes environment to go through a sequence of states. If sequence is desirable, then agent has performed well. This notion of desirability is captured by a {\it performance measure} that evaluates any given sequence of environment states.
				
				-- {\it Các biện pháp hiệu suất.} Triết học đạo đức đã phát triển 1 số khái niệm khác nhau về ``điều đúng đắn'', nhưng AI thường gắn bó với 1 khái niệm gọi là chủ nghĩa hậu quả: đánh giá hành vi của tác nhân theo hậu quả của nó. Khi 1 tác nhân được đặt xuống trong 1 môi trường, nó sẽ tạo ra 1 chuỗi hành động theo các nhận thức mà nó nhận được. Chuỗi hành động này khiến môi trường trải qua 1 chuỗi trạng thái. Nếu chuỗi là mong muốn, thì tác nhân đã hoạt động tốt. Khái niệm mong muốn này được nắm bắt bằng 1 {\it biện pháp hiệu suất} đánh giá bất kỳ chuỗi trạng thái môi trường nào được đưa ra.
				
				Humans have desires \& preferences of their own, so notion of rationality as applied to humans has to do with their success in choosing actions that produce sequences of environment states that are desirable {\it from their point of view}. Machines, on other hand, do not have desires \& preferences of their own; performance measure is, initially at least, in mind of designer of machine, or in mind of users machine is designed for. See: some agent designs have an explicit representation of (a version of) performance measure, while in other designs performance measure is entirely implicit -- agent may do right thing, but it doesn't know why.
				
				-- Con người có ham muốn \& sở thích riêng, vì vậy khái niệm về tính hợp lý khi áp dụng cho con người có liên quan đến thành công của họ trong việc lựa chọn các hành động tạo ra chuỗi trạng thái môi trường mong muốn {\it theo quan điểm của họ}. Mặt khác, máy móc không có ham muốn \& sở thích riêng; thước đo hiệu suất, ít nhất là ban đầu, nằm trong tâm trí của nhà thiết kế máy móc hoặc trong tâm trí của người dùng mà máy móc được thiết kế cho. Xem: 1 số thiết kế tác nhân có biểu diễn rõ ràng về (một phiên bản) thước đo hiệu suất, trong khi trong các thiết kế khác, thước đo hiệu suất hoàn toàn ngầm định -- tác nhân có thể làm đúng, nhưng không biết tại sao.
				
				Recalling {\sc Norbert Wiener}'s warning to ensure ``purpose put into machine is purpose which we really desire'', notice: it can be quite hard to formulate a performance measure correctly. Consider, e.g., vacuum--cleaner agent from preceding sect. Might propose to measure performance by amount of dirt cleaned up in a single 8-hour shift. With a rational agent, of course, what you ask for is what you get. A rational agent can maximize this performance measure by cleaning up dirt, then dumping it all on floor, then cleaning it up again, \& so on. A more suitable performance measure would reward agent for having a clean floor. E.g., 1 point could be awarded for each clean square at each time step (perhaps with a penalty for electricity consumed \& noise generated). {\it As a general rule, better to design performance measures according to what one actually wants to be achieved in environment, rather than according to how one thinks agent should behave}.
				
				-- Nhắc lại lời cảnh báo của {\sc Norbert Wiener} để đảm bảo ``mục đích đưa vào máy là mục đích mà chúng ta thực sự mong muốn'', hãy lưu ý: có thể khá khó để xây dựng 1 thước đo hiệu suất chính xác. Ví dụ, hãy xem xét tác nhân máy hút bụi từ phần trước. Có thể đề xuất đo hiệu suất theo lượng bụi bẩn được làm sạch trong 1 ca làm việc 8 giờ. Tất nhiên, với 1 tác nhân hợp lý, những gì bạn yêu cầu là những gì bạn nhận được. Một tác nhân hợp lý có thể tối đa hóa thước đo hiệu suất này bằng cách làm sạch bụi bẩn, sau đó đổ hết xuống sàn, rồi lại làm sạch, \& cứ như vậy. Một thước đo hiệu suất phù hợp hơn sẽ thưởng cho tác nhân vì có sàn sạch. Ví dụ, có thể thưởng 1 điểm cho mỗi ô vuông sạch tại mỗi bước thời gian (có thể kèm theo hình phạt cho lượng điện tiêu thụ \& tiếng ồn tạo ra). {\it Theo nguyên tắc chung, tốt hơn là thiết kế các thước đo hiệu suất theo những gì người ta thực sự muốn đạt được trong môi trường, thay vì theo cách người ta nghĩ tác nhân nên hành xử}.
				
				Even when obvious pitfalls are avoided, some knotty problems remain. E.g., notion of ``clean floor'' in preceding paragraph is based on average cleanliness over time. Yet same average cleanliness can be achieved by 2 different agents, one of which does a mediocre job all time while other cleans energetically but takes long breaks. Which is preferable might seem to be a fine point of janitorial science, but in fact it is a deep philosophical question with far-reaching implications. Which is better -- an economy where everyone lives in moderate poverty, or one in which some live in plenty while others are very poor? Leave these questions as an exercise for diligent reader.
				
				-- Ngay cả khi tránh được những cạm bẫy rõ ràng, 1 số vấn đề nan giải vẫn còn tồn tại. Ví dụ, khái niệm ``sàn nhà sạch'' trong đoạn trước dựa trên mức độ sạch trung bình theo thời gian. Tuy nhiên, mức độ sạch trung bình tương tự có thể đạt được bởi 2 tác nhân khác nhau, 1 trong số đó làm việc tầm thường mọi lúc trong khi tác nhân kia làm việc rất hăng hái nhưng lại nghỉ giải lao rất lâu. Cái nào tốt hơn có vẻ là 1 điểm tinh tế của khoa học vệ sinh, nhưng trên thực tế, đó là 1 câu hỏi triết học sâu sắc với những hàm ý sâu xa. Cái nào tốt hơn -- 1 nền kinh tế mà mọi người đều sống trong cảnh nghèo đói vừa phải, hay 1 nền kinh tế mà 1 số người sống trong cảnh sung túc trong khi những người khác lại rất nghèo? Hãy để những câu hỏi này như 1 bài tập cho người đọc siêng năng.
				
				For most of book, assume: performance measure can be specified correctly. For reasons given above, however, must accept possibility that we might put wrong purpose into machine -- precisely King Midas problem described on p. 51. Moreover, when designing 1 piece of software, copies of which will belong to different users, cannot anticipate exact preferences of each individual user. Thus, may need to build agents that reflect initial uncertainty about true performance measure \& learn more about it as time goes by; such agents are described in Chaps. 15, 17, 23.
				\item {\sf2.2.2. Rationality.} What is rational at any given time depends on 4 things:
				\begin{enumerate}
					\item Performance measure that defines criterion of success.
					\item Agent's prior knowledge of environment.
					\item Actions that agent can perform.
					\item Agent's percept sequence to date.
				\end{enumerate}
				This leads to a def of a rational agent:
				\begin{definition}[Rational agent]
					For each possible percept sequence, a {\rm rational agent} should select an action that is expected to maximize its performance measure, given evidence provided by percept sequence \& whatever built-in knowledge agent has.
					
					-- Đối với mỗi chuỗi nhận thức có thể, 1 {\rm tác nhân hợp lý} nên chọn 1 hành động dự kiến sẽ tối đa hóa thước đo hiệu suất của nó, dựa trên bằng chứng được cung cấp bởi chuỗi nhận thức \& bất kỳ tác nhân kiến thức tích hợp nào.
				\end{definition}
				Consider simple vacuum-cleaner agent that cleans a square if it is dirty \& moves to other square if not; this is agent function tabulated in {\sf Fig. 2.3}. Is this a rational agent? That depends! 1st, need to say what performance measure is, what is known about environment, \& what sensors \& actuators agent has. Assume:
				\begin{enumerate}
					\item Performance measure awards 1 point for each clean square at each time step, over a ``lifetime'' of 1000 time steps.
					\item ``Geography'' of environment is known {\it a priori} {\sf Fig. 2.2} but dirt distribution \& initial location of agent are not. Clean squares stay clean \& sucking cleans current square. {\it Right \& Left} actions move agent 1 square except when this would take agent outside environment, in which case agent remains where it is.
					\item Only available actions are {\it Right, Left, \& Suck}.
					\item Agent correctly perceives its location \& whether that location contains dirt.
				\end{enumerate}
				Under these circumstances agent is indeed rational; its expected performance is at least as good as other agent's.
				
				One can see easily: same agent would be irrational under different circumstances. E.g., once all dirt is cleaned up, agent will oscillate needlessly back \& forth; if performance measure includes a penalty of 1 point for each movement, agent will fare poorly.tA better agent for this case would do nothing once it is sure: all squares are clean. If clean squares can become dirty again, agent should occasionally check \& reclean them if needed. If geography of environment is unknown, agent will need to {\it explore} it. Exercise 2.VACR asks you to design agents for these cases.
				
				-- Người ta có thể dễ dàng thấy: cùng 1 tác nhân sẽ không hợp lý trong những hoàn cảnh khác nhau. Ví dụ, sau khi tất cả bụi bẩn được dọn sạch, tác nhân sẽ dao động qua lại không cần thiết \&; nếu thước đo hiệu suất bao gồm hình phạt 1 điểm cho mỗi chuyển động, tác nhân sẽ hoạt động kém. Một tác nhân tốt hơn cho trường hợp này sẽ không làm gì cả khi chắc chắn: tất cả các ô vuông đều sạch. Nếu các ô vuông sạch có thể bị bẩn trở lại, tác nhân nên thỉnh thoảng kiểm tra \& làm sạch lại chúng nếu cần. Nếu địa lý của môi trường không xác định, tác nhân sẽ cần phải {\it khám phá} nó. Bài tập 2.VACR yêu cầu bạn thiết kế các tác nhân cho những trường hợp này.
				\item {\sf2.2.3. Omniscience, learning, \& autonomy.} Need to be careful to distinguish between rationality \& {\it omniscience}. An omniscient agent knows {\it actual} outcome of its actions \& can act accordingly; but omniscience is impossible in reality. Consider example: I am walking along Champs Elysées 1 day \& I see an old friend across street. There is no traffic nearby \& I'm not otherwise engaged, so, being rational, I start to cross street. Meanwhile, at 33000 feet, a cargo door falls off a passing airliner [See {\sc N. Henderson}, ``New door latches urged for Boeing 747 jumbo jets,'' Washington Post, Aug 24, 1989.], \& before I make it to other side of street I am flattened. Was I irrational to cross street? It is unlikely: my obituary would read ``Idiot attempts to cross street.''
				
				This example shows: rationality is not same as perfection. Rationality maximizes {\it expected} performance, while perfection maximizes {\it actual} performance. Retreating from a requirement of perfection is not just a question of being fair to agents. Point: if expect an agent to do what turns out after fact to be best action, it will be impossible to design an agent to fulfill this specification -- unless improve performance of crystal balls or time machines.
				
				-- Ví dụ này cho thấy: tính hợp lý không giống với sự hoàn hảo. Tính hợp lý tối đa hóa hiệu suất {\it mong đợi}, trong khi sự hoàn hảo tối đa hóa hiệu suất {\it thực tế}. Việc rút lui khỏi yêu cầu về sự hoàn hảo không chỉ là vấn đề công bằng với các tác nhân. Điểm chính: nếu mong đợi 1 tác nhân thực hiện những gì sau này trở thành hành động tốt nhất, thì sẽ không thể thiết kế 1 tác nhân để đáp ứng thông số kỹ thuật này -- trừ khi cải thiện hiệu suất của quả cầu pha lê hoặc cỗ máy thời gian.
				
				Our definition of rationality does not require omniscience, then, because rational choice depends only on percept sequence {\it to date}. Must also ensure: we haven't inadvertently allowed agent to engage in decidedly underintelligent activities. E.g., if an agent does not look both ways before crossing a busy road, then its percept sequence will not tell it that there is a large truck approaching at high speed. Does our definition of rationality say that it's now OK to cross road? Far from it!
				
				-- Định nghĩa của chúng ta về tính hợp lý không đòi hỏi sự toàn năng, vì sự lựa chọn hợp lý chỉ phụ thuộc vào chuỗi nhận thức {\it cho đến nay}. Cũng phải đảm bảo: chúng ta không vô tình cho phép tác nhân tham gia vào các hoạt động rõ ràng là thiếu thông minh. Ví dụ, nếu 1 tác nhân không nhìn cả hai hướng trước khi băng qua 1 con đường đông đúc, thì chuỗi nhận thức của tác nhân đó sẽ không cho tác nhân biết rằng có 1 chiếc xe tải lớn đang lao tới với tốc độ cao. Định nghĩa của chúng ta về tính hợp lý có nói rằng bây giờ có thể băng qua đường không? Hoàn toàn không phải vậy!
				
				1st, it would not be rational to cross road given this uninformative percept sequence: risk of accident from crossing without looking is too great. 2nd, a rational agent should choose ``looking'' action before stepping into street, because looking helps maximize expected performance. Doing actions {\it in order to modify future percepts} -- sometimes called {\it information gathering} -- is an important part of rationality \& is covered in depth in Chap. 15. A 2nd example of information gathering is provided by {\it exploration} that must be undertaken by a vacuum-cleaning agent in an initially unknown environment.
				
				-- 1. Sẽ không hợp lý khi băng qua đường khi xét đến chuỗi nhận thức không cung cấp thông tin này: nguy cơ tai nạn khi băng qua đường mà không nhìn là quá lớn. 2. Một tác nhân hợp lý nên chọn hành động ``nhìn'' trước khi bước vào đường, vì nhìn giúp tối đa hóa hiệu suất mong đợi. Thực hiện các hành động {\it để sửa đổi các nhận thức trong tương lai} -- đôi khi được gọi là {\it thu thập thông tin} -- là 1 phần quan trọng của tính hợp lý \& được trình bày sâu trong Chương 15. Ví dụ thứ 2 về việc thu thập thông tin được cung cấp bởi {\it khám phá} phải được thực hiện bởi 1 tác nhân hút bụi trong 1 môi trường ban đầu không xác định.
				
				Our definition requires a rational agent not only to gather information but also to {\it learn} as much as possible from what it perceives. Agent's initial configuration could reflect some prior knowledge of environment, but as agent gains experience this may be modified \& augmented. There are extreme cases in which environment is completely known {\it a priori} \& completely predictable. In such cases, agent need not perceive or learn; it simply acts correctly.
				
				-- Định nghĩa của chúng tôi yêu cầu 1 tác nhân lý trí không chỉ thu thập thông tin mà còn {\it học} càng nhiều càng tốt từ những gì nó nhận thức. Cấu hình ban đầu của tác nhân có thể phản ánh 1 số kiến thức trước đó về môi trường, nhưng khi tác nhân có thêm kinh nghiệm, điều này có thể được sửa đổi \& tăng cường. Có những trường hợp cực đoan trong đó môi trường được biết đến hoàn toàn {\it a priori} \& hoàn toàn có thể dự đoán được. Trong những trường hợp như vậy, tác nhân không cần phải nhận thức hoặc học; nó chỉ đơn giản là hành động đúng.
				
				Of course, such agents are fragile. Consider lowly dung beetle. After digging its nest \& laying its eggs, it fetches a ball of dung from a nearby heap to plug entrance. If ball of dung is removed from its grasp {\it en route}, beetle continues its task \& pantomimes plugging nest with nonexistent dung ball, never noticing that it is missing. Evolution has built an assumption into beetle's behavior, \& when it is violated, unsuccessful behavior results.
				
				-- Tất nhiên, những tác nhân như vậy rất mong manh. Hãy xem xét loài bọ hung thấp kém. Sau khi đào tổ \& đẻ trứng, nó lấy 1 cục phân từ đống phân gần đó để chặn lối vào. Nếu cục phân bị lấy khỏi tay nó trên đường đi, bọ hung tiếp tục nhiệm vụ \& làm trò hề bằng cách chặn tổ bằng cục phân không tồn tại, không bao giờ nhận ra rằng cục phân đã mất. Sự tiến hóa đã xây dựng 1 giả định vào hành vi của bọ hung, \& khi giả định đó bị vi phạm, hành vi không thành công sẽ xảy ra.
				
				Slightly more intelligent is sphex wasp. Female sphex will dig a burrow, go out \& sting a caterpillar \& drag it to burrow, enter burrow again to check all is well, drag caterpillar inside, \& lay its eggs. Caterpillar serves as a food source when eggs hatch. So far so good, but if an entomologist moves caterpillar a few inches away while sphex is doing check, it will revert to ``drag caterpillar'' step of its plan \& will continue plan without modification, re--checking burrow, even after dozens of caterpillar-moving interventions. Sphex is unable to learn that its innate plan is failing, \& thus will not change it.
				
				-- Thông minh hơn 1 chút là ong bắp cày sphex. Ong sphex cái sẽ đào hang, chui ra \& đốt sâu bướm \& kéo nó vào hang, chui vào hang lần nữa để kiểm tra mọi thứ ổn thỏa, kéo sâu bướm vào trong, \& đẻ trứng. Sâu bướm đóng vai trò là nguồn thức ăn khi trứng nở. Cho đến giờ thì mọi thứ vẫn ổn, nhưng nếu 1 nhà côn trùng học di chuyển sâu bướm ra xa vài inch trong khi sphex đang kiểm tra, nó sẽ quay lại bước ``kéo sâu bướm'' trong kế hoạch của nó \& sẽ tiếp tục kế hoạch mà không cần sửa đổi, kiểm tra lại hang, ngay cả sau hàng chục lần can thiệp di chuyển sâu bướm. Sphex không thể học được rằng kế hoạch bẩm sinh của nó đang thất bại, \& do đó sẽ không thay đổi nó.
				
				To extent that an agent relies on prior knowledge of its designer rather than on its own percepts \& learning processes, say: agent lacks {\it autonomy}. A rational agent should be autonomous -- it should learn what it can to compensate for partial or incorrect prior knowledge. E.g., a vacuum-cleaning agent that learns to predict where \& when additional dirt will appear will do better than one that does not.
				
				-- Trong phạm vi mà 1 tác nhân dựa vào kiến thức trước đó của người thiết kế nó hơn là vào các nhận thức \& quá trình học tập của chính nó, hãy nói: tác nhân thiếu {\it tự chủ}. Một tác nhân hợp lý phải tự chủ -- nó phải học những gì nó có thể để bù đắp cho kiến thức trước đó không đầy đủ hoặc không chính xác. Ví dụ, 1 tác nhân hút bụi học cách dự đoán nơi \& khi nào bụi bẩn sẽ xuất hiện sẽ hoạt động tốt hơn 1 tác nhân không làm như vậy.
				
				As a practical matter, one seldom requires complete autonomy from start: when agent has had little or no experience, it would have to act randomly unless designer gave some assistance. Just as evolution provides animals with enough built-in reflexes to survive long enough to learn for themselves, it would be reasonable to provide an AI agent with some initial knowledge as well as an ability to learn. After sufficient experience of its environment, behavior of a rational agent can become effectively {\it independent} of its prior knowledge. Hence, incorporation of learning allows one to design a single rational agent that will succeed in a vast variety of environments.
				
				-- Trên thực tế, người ta hiếm khi đòi hỏi sự tự chủ hoàn toàn ngay từ đầu: khi tác nhân có ít hoặc không có kinh nghiệm, nó sẽ phải hành động ngẫu nhiên trừ khi nhà thiết kế cung cấp 1 số hỗ trợ. Cũng giống như quá trình tiến hóa cung cấp cho động vật đủ phản xạ tích hợp để tồn tại đủ lâu để tự học, sẽ hợp lý khi cung cấp cho tác nhân AI 1 số kiến thức ban đầu cũng như khả năng học hỏi. Sau khi có đủ kinh nghiệm về môi trường của mình, hành vi của tác nhân hợp lý có thể trở nên {\it độc lập} hiệu quả với kiến thức trước đó của nó. Do đó, việc kết hợp học tập cho phép người ta thiết kế 1 tác nhân hợp lý duy nhất sẽ thành công trong nhiều môi trường khác nhau.
			\end{itemize}
			\item {\sf 2.3. Nature of Environments.} Now have a definition of rationality, almost ready to think about building rational agents. 1st, however, must think about {\it task environments}, which are essentially ``problems'' to which rational agents are ``solutions''. Begin by showing how to specify a task environment, illustrating process with a number of examples. Then show: task environments come in a variety of flavors. Nature of task environment directly affects appropriate design for agent program.
			
			-- {\it Bản chất của Môi trường.} Bây giờ đã có định nghĩa về tính hợp lý, gần như đã sẵn sàng để nghĩ về việc xây dựng các tác nhân hợp lý. Tuy nhiên, trước tiên, phải nghĩ về {\it task environments}, về cơ bản là ``các vấn đề'' mà các tác nhân hợp lý là ``giải pháp''. Bắt đầu bằng cách chỉ ra cách chỉ định 1 môi trường tác vụ, minh họa quy trình bằng 1 số ví dụ. Sau đó chỉ ra: các môi trường tác vụ có nhiều loại khác nhau. Bản chất của môi trường tác vụ ảnh hưởng trực tiếp đến thiết kế phù hợp cho chương trình tác nhân.
			\begin{itemize}
				\item {\sf2.3.1. Specifying task environment.} In our discussion of rationality of simple vacuum-cleaner agent, had to specify performance measure, environment, \& agent's actuators \& sensors. Group all these under heading of {\it task environment}. For acronymically minded, call this PEAS (Performance, Environment, Actuators, Sensors) description. In designing an agent, 1st step mus always be to specify task environment as fully as possible.
				
				Vacuum world was a simple example; consider a more complex problem: an automated taxi driver. {\sf Fig. 2.4: PEAS description of task environment for an automated taxi driver.} summarizes PEAS description for taxi's task environment. Discuss each element in more detail.
				
				1st, what is {\it performance measure} to which we would like our automated driver to aspire? Desirable qualities include getting to correct destination; minimizing fuel consumption \& wear \& tear; minimizing trip time or cost; minimizing violations of traffic laws \& disturbances to other drivers; maximizing safety \& passenger comfort; maximizing profits. Obviously, some of these goals conflict, so tradeoffs will be required.
				
				-- 1, {\it tiêu chuẩn hiệu suất} mà chúng ta muốn trình điều khiển tự động của mình hướng tới là gì? Các phẩm chất mong muốn bao gồm đến đúng đích; giảm thiểu mức tiêu thụ nhiên liệu \& hao mòn \& rách; giảm thiểu thời gian hoặc chi phí chuyến đi; giảm thiểu vi phạm luật giao thông \& gây phiền nhiễu cho những người lái xe khác; tối đa hóa sự an toàn \& sự thoải mái của hành khách; tối đa hóa lợi nhuận. Rõ ràng, 1 số mục tiêu này xung đột với nhau, vì vậy cần phải đánh đổi.
				
				Next, what is driving {\it environment} that taxi will face? Any taxi driver must deal with a variety of roads, ranging from rural lanes \& urban alleys to 12-lane freeways. Roads contain other traffic, pedestrians, stray animals, road works, police cars, puddles, \& potholes. Taxi must also interact with potential \& actual passengers. There are also some optional choices. Taxi might need to operate in Southern California, where snow is seldom a problem, or in Alaska, where it seldom is not. It could always driving on right, or we might want it to be flexible enough to drive on left when in Britain or Japan. Obviously, more restricted environment, easier design problem.
				
				-- Tiếp theo, môi trường lái xe mà taxi sẽ phải đối mặt là gì? Bất kỳ tài xế taxi nào cũng phải đối mặt với nhiều loại đường khác nhau, từ làn đường nông thôn \& ngõ phố đến đường cao tốc 12 làn. Đường có nhiều phương tiện giao thông khác, người đi bộ, động vật hoang dã, công trình đường bộ, xe cảnh sát, vũng nước, \& ổ gà. Taxi cũng phải tương tác với hành khách tiềm năng \& thực tế. Ngoài ra còn có 1 số lựa chọn tùy chọn. Taxi có thể cần hoạt động ở Nam California, nơi tuyết hiếm khi là vấn đề, hoặc ở Alaska, nơi tuyết hiếm khi không là vấn đề. Nó luôn có thể lái xe bên phải, hoặc chúng ta có thể muốn nó đủ linh hoạt để lái xe bên trái khi ở Anh hoặc Nhật Bản. Rõ ràng, môi trường hạn chế hơn, vấn đề thiết kế dễ dàng hơn.
				
				{\it Actuators} for an automated taxi include those available to a human driver: control over engine through accelerator \& control over steering \& braking. In addition, it will need output to a display screen or voice synthesizer to talk back to passengers, \& perhaps some way to communicate with other vehicles, politely or otherwise.
				
				-- {\it Bộ truyền động} cho 1 chiếc taxi tự động bao gồm những bộ truyền động có sẵn cho người lái: điều khiển động cơ thông qua chân ga \& điều khiển tay lái \& phanh. Ngoài ra, nó sẽ cần xuất ra màn hình hiển thị hoặc bộ tổng hợp giọng nói để nói chuyện với hành khách, \& có lẽ là 1 số cách để giao tiếp với các phương tiện khác, theo cách lịch sự hoặc không.
				
				Basic {\it sensors} for taxi will include 1 or more video cameras so that it can see, as well as lidar \& ultrasound sensors to detect distances to other cars \& obstacles. To avoid speeding tickets, taxi should have a speedometer, \& to control vehicle properly, especially on curves, it should have an accelerometer. To determine mechanical state of vehicle, it will need usual array of engine, fuel, \& electrical system sensors. Like many human drivers, it might want to access GPT signals so that it doesn't get lost. Finally, it will need touchscreen or voice input for passenger to request a destination.
				
				-- Các cảm biến cơ bản của taxi sẽ bao gồm 1 hoặc nhiều camera video để có thể nhìn thấy, cũng như cảm biến lidar \& siêu âm để phát hiện khoảng cách đến các xe khác \& chướng ngại vật. Để tránh bị phạt vì chạy quá tốc độ, taxi nên có đồng hồ đo tốc độ, \& để điều khiển xe đúng cách, đặc biệt là khi vào cua, taxi nên có máy đo gia tốc. Để xác định trạng thái cơ học của xe, taxi sẽ cần 1 loạt các cảm biến thông thường về động cơ, nhiên liệu, \& hệ thống điện. Giống như nhiều tài xế khác, taxi có thể muốn truy cập tín hiệu GPT để không bị lạc đường. Cuối cùng, taxi sẽ cần màn hình cảm ứng hoặc giọng nói để hành khách yêu cầu điểm đến.
				
				In {\sf Fig. 2.5: Examples of agent types \& their PEAS descriptions.}, have sketched basic PEAS elements for a number of additional agent types. Further examples appear in Exercise 2.PEAS. Examples include physical as well as virtual environments. Note: virtual task environments can be just as complex as ``real'' world: e.g., a {\it software agent} (or software robot or {\it softbot}) that trades on auction \& reselling Web sites deals with millions of other users \& billions of objects, many with real images.
				
				-- Trong {\sf Hình 2.5: Ví dụ về các loại tác nhân \& mô tả PEAS của chúng.}, đã phác thảo các thành phần PEAS cơ bản cho 1 số loại tác nhân bổ sung. Các ví dụ khác xuất hiện trong Bài tập 2.PEAS. Các ví dụ bao gồm cả môi trường vật lý cũng như môi trường ảo. Lưu ý: môi trường tác vụ ảo có thể phức tạp như thế giới ``thực'': ví dụ, 1 {\it phần mềm tác nhân} (hoặc rô-bốt phần mềm hoặc {\it softbot}) giao dịch trên các trang web đấu giá \& bán lại giao dịch với hàng triệu người dùng khác \& hàng tỷ đối tượng, nhiều đối tượng có hình ảnh thực.				
				\item {\sf2.3.2. Properties of task environments.} Range of task environments that might arise in AI is obviously vast. Can, however, identify a fairly small number of dimensions along which task environments can be categorized. These dimensions determine, to a large extent, appropriate agent design \& applicability of each of principal families of techniques for agent implementation. 1st list dimensions, then analyze several task environments to illustrate ideas. Definitions here are informal; later chaps provide more precise statements \& examples of each kind of environment.
				
				-- {\it Thuộc tính của môi trường tác vụ.} Phạm vi các môi trường tác vụ có thể phát sinh trong AI rõ ràng là rất lớn. Tuy nhiên, có thể xác định 1 số lượng khá nhỏ các chiều mà môi trường tác vụ có thể được phân loại theo. Các chiều này xác định, ở 1 mức độ lớn, thiết kế tác nhân phù hợp \& khả năng áp dụng của từng họ kỹ thuật chính để triển khai tác nhân. Đầu tiên, hãy liệt kê các chiều, sau đó phân tích 1 số môi trường tác vụ để minh họa các ý tưởng. Các định nghĩa ở đây là không chính thức; các chương sau cung cấp các tuyên bố chính xác hơn \& ví dụ về từng loại môi trường.
				
				p. 61+++				
			\end{itemize}
		\end{itemize}
		\item 
	\end{itemize}
	PART II: PROBLEM-SOLVING.
	\begin{itemize}
		\item {\sf3. Solving Problems by Searching.}
		\item {\sf4. Search in Complex Environments.}
		\item {\sf5. Constraint Satisfaction Problems.}
		\item {\sf6. Adversarial Search \& Games.}
	\end{itemize}
	PART III: KNOWLEDGE, REASONING, \& PLANNING.
	\begin{itemize}
		\item {\sf7. Logical Agents.}
		\item {\sf8. 1st-Order Logic.}
		\item {\sf9. Inference in 1st-Order Logic.}
		\item {\sf10. Knowledge Representation.}
		\item {\sf11. Automated Planning.}
	\end{itemize}
	PART IV: UNCERTAIN KNOWLEDGE \& REASONING.
	\begin{itemize}
		\item {\sf12. Quantifying Uncertainty.}
		\item {\sf13. Probabilistic Reasoning.}
		\item {\sf14. Probabilistic Reasoning over Time.}
		\item {\sf15. Making Simple Decisions.}
		\item {\sf16. Making Complex Decisions.}
		\item {\sf17. Multiagent Decision Making.}
		\item {\sf18. Probabilistic Programming.}
	\end{itemize}
	PART V: ML
	\begin{itemize}
		\item {\sf19. Learning from Examples.}
		\item {\sf20. Knowledge in Learning.}
		\item {\sf21. Learning Probabilistic Models.}
		\item {\sf22. DL.}
		\item {\sf23. Reinforcement Learning.}
	\end{itemize}
	PART VI: COMMUNICATING, PERCEIVING, \& ACTING.
	\begin{itemize}
		\item {\sf24. Natural Language Processing.}
		\item {\sf25. DL for NLP.}
		\item {\sf26. Robotics.}
		\item {\sf27. Computer Vision.}
	\end{itemize}
	PART VII: CONCLUSIONS.
	\begin{itemize}
		\item {\sf28. Philosophy, Ethics, \& Safety of AI.}
		\item {\sf29. Future of AI.}
	\end{itemize}
	\item {\sf Appendix A: Mathematical Background.}
	\item {\sf Appendix B: Notes on Languages \& Algorithms.}
\end{itemize}

%------------------------------------------------------------------------------%

\section{KnapSack Problem}

%------------------------------------------------------------------------------%

\subsection{\cite{Kellerer_Pferschy_Pisinger2004}. {\sc Hans Kellerer, Ulrich Pferschy, David Pisinger}. Knapsack Problems}

\begin{itemize}
    \item {\sf Preface.} 30 years have passed since seminal book on knapsack problems by {\sc Martello \& Toth} appeared. On this occasion a former colleague exclaimed back in 1990: ``How can you write 250 pages on knapsack problem?'' Indeed, def of knapsack problem is easily understood even by a non-expert who will not suspect presence of challenging research topics in this area at 1st glance.
    
    However, in last decade a large number of research publications contributed new results for knapsack problem in all areas of interest e.g. exact algorithms, heuristics, \& approximation schemes. Moreover, extension of knapsack problem to higher dimensions both in number of constraints \& in number of knapsacks, as well as modification of problem structure concerning available item seet \& objective function, leads to a number of interesting variations of practical relevance which were subject of intensive research during last few years.
    
    -- Tuy nhiên, trong thập kỷ qua, 1 số lượng lớn các ấn phẩm nghiên cứu đã đóng góp những kết quả mới cho bài toán ba lô trong mọi lĩnh vực quan tâm, ví dụ như thuật toán chính xác, phương pháp tìm kiếm, \& sơ đồ xấp xỉ. Hơn nữa, việc mở rộng bài toán ba lô lên các chiều cao hơn về cả số lượng ràng buộc \& số lượng ba lô, cũng như việc sửa đổi cấu trúc bài toán liên quan đến mục có sẵn \& hàm mục tiêu, dẫn đến 1 số biến thể thú vị có liên quan thực tế, là chủ đề của nghiên cứu chuyên sâu trong vài năm qua.
    
    Hence, 2 years ago idea arose to produce a new monograph covering not only most recent developments of standard knapsack problem, but also giving a comprehensive treatment of whole knapsack family including siblings e.g. subset sum problem \& bounded \& unbounded knapsack problem, \& also more distant relatives e.g. multidimensional, multiple, multiple-choice \& quadratic knapsack problems in dedicated chaps.
    
    -- Do đó, 2 năm trước, 1 ý tưởng đã nảy sinh là biên soạn 1 chuyên khảo mới không chỉ bao gồm những phát triển gần đây nhất của bài toán ba lô chuẩn mà còn đưa ra cách xử lý toàn diện cho toàn bộ họ ba lô bao gồm cả các phần tử cùng nhóm, ví dụ như bài toán tổng tập hợp \& bài toán ba lô có giới hạn \& không giới hạn, \& cả những họ hàng xa hơn, ví dụ như bài toán ba lô đa chiều, nhiều, nhiều lựa chọn \& bậc hai trong các chương chuyên đề.
    
    Furthermore, attention is paid to a number of less frequently considered variants of knapsack problem \& to study of stochastic aspects of problem. To illustrate high practical relevance of knapsack family for many industrial \& economic problems, a number of applications are described in more detail. They are selected subjectively from innumerable occurrences of knapsack problems reported in literature.
    
    -- Hơn nữa, sự chú ý được dành cho 1 số biến thể ít được xem xét hơn của bài toán ba lô \& để nghiên cứu các khía cạnh ngẫu nhiên của bài toán. Để minh họa tính liên quan thực tế cao của họ ba lô đối với nhiều bài toán công nghiệp \& kinh tế, 1 số ứng dụng được mô tả chi tiết hơn. Chúng được lựa chọn 1 cách chủ quan từ vô số trường hợp xảy ra của bài toán ba lô được báo cáo trong tài liệu.
    
    Our above-mentioned colleague will be surprised to notice that even on $> 500$ pages of this book not all relevant topics could be treated in equal depth but decisions had to be made on where to go into details of constructions \& proofs \& where to concentrate on stating results \& refer to appropriate publications. Moreover, an editorial deadline had to be drawn at some point. In our case, stopped looking for new publications at end of Jun 2003.
    
    -- Người đồng nghiệp được đề cập ở trên của chúng tôi sẽ ngạc nhiên khi nhận thấy rằng ngay cả trên $> 500$ trang của cuốn sách này, không phải tất cả các chủ đề có liên quan đều có thể được xử lý ở mức độ sâu như nhau mà phải đưa ra quyết định về việc đi sâu vào chi tiết về các cấu trúc \& bằng chứng \& tập trung vào việc nêu kết quả \& tham khảo các ấn phẩm phù hợp. Hơn nữa, phải đưa ra thời hạn biên tập tại 1 thời điểm nào đó. Trong trường hợp của chúng tôi, đã ngừng tìm kiếm các ấn phẩm mới vào cuối tháng 6 năm 2003.
    
    Audience we envision for this book is 3fold: 1st 2 chaps offer a very basic introduction to knapsack problem \& main algorithmic concepts to derive optimal \& approximate solution. Chap. 3 presents a number of advanced algorithmic techniques which are used throughout later chaps of book. Style of presentation in these 3 chaps is kept rather simple \& assumes only minimal prerequisites. They should be accessible to students \& graduates of business administration, economics \& engineering as well as practitioners with little knowledge of algorithms \& optimization.
    
    -- Đối tượng mà chúng tôi hình dung cho cuốn sách này là 3 phần: 2 chương đầu tiên cung cấp phần giới thiệu rất cơ bản về bài toán ba lô \& các khái niệm thuật toán chính để đưa ra giải pháp tối ưu \& gần đúng. Chương 3 trình bày 1 số kỹ thuật thuật toán nâng cao được sử dụng trong các chương sau của cuốn sách. Phong cách trình bày trong 3 chương này được giữ khá đơn giản \& chỉ giả định các điều kiện tiên quyết tối thiểu. Chúng nên dễ tiếp cận đối với sinh viên \& tốt nghiệp ngành quản trị kinh doanh, kinh tế \& kỹ thuật cũng như những người hành nghề có ít kiến thức về thuật toán \& tối ưu hóa.
    
    This 1st part of book is also well suited to introduce classical concepts of optimization in a classroom, since knapsack problem is easy to understand \& is probably the least difficult but most illustrative problem where dynamic programming, branch-\&-bound, relaxations \& approximation schemes can be applied.
    
    -- Phần 1 của cuốn sách này cũng rất phù hợp để giới thiệu các khái niệm cổ điển về tối ưu hóa trong lớp học, vì bài toán ba lô dễ hiểu \& có lẽ là bài toán ít khó nhất nhưng minh họa rõ nhất, trong đó có thể áp dụng các lược đồ lập trình động, nhánh\&-bound, nới lỏng \& xấp xỉ.
    
    In these chaps no knowledge of linear or integer programming \& only a minimal familiarity with basic elements of graph theory is assumed. Issue of NP-completeness is dealt with by an intuitive introduction in Sect. 1.5, whereas a thorough \& rigorous treatment is deferred to Appendix.
    
    -- Trong các chương này không có kiến thức về lập trình tuyến tính hoặc số nguyên \& chỉ giả định có sự quen thuộc tối thiểu với các yếu tố cơ bản của lý thuyết đồ thị. Vấn đề về tính đầy đủ của NP được giải quyết bằng phần giới thiệu trực quan trong Phần 1.5, trong khi phần xử lý kỹ lưỡng \& chặt chẽ được chuyển đến Phụ lục.
    
    Remaining chaps of book addresses 2 different audiences. On 1 hand, a student or graduate of mathematics or CS, or a successful reader of 1st 3 chaps willing to go into more depth, can use this book to study advanced algorithms for knapsack problem \& its relatives. On other hand, hope scientific researchers or expert practitioners will find book a valuable source of reference for a quick update on state of art \& on most efficient algorithms currently available. In particular, a collection of computational experiments, many of them published for 1st time in this book, should serve as a valuable tool to pick algorithm best suited for a given problem instance. To facilitate use of book as a reference, tried to keep these chaps self-contained as far as possible.
    
    -- Các chương còn lại của cuốn sách hướng đến 2 đối tượng độc giả khác nhau. Một mặt, 1 sinh viên hoặc tốt nghiệp ngành toán học hoặc khoa học máy tính, hoặc 1 độc giả thành công của 3 chương đầu tiên muốn đi sâu hơn, có thể sử dụng cuốn sách này để nghiên cứu các thuật toán nâng cao cho bài toán ba lô \& các chương liên quan. Mặt khác, hy vọng các nhà nghiên cứu khoa học hoặc các chuyên gia thực hành sẽ thấy cuốn sách là 1 nguồn tham khảo có giá trị để cập nhật nhanh về tình hình nghệ thuật \& các thuật toán hiệu quả nhất hiện có. Đặc biệt, 1 bộ sưu tập các thí nghiệm tính toán, nhiều trong số chúng được công bố lần đầu tiên trong cuốn sách này, sẽ đóng vai trò là 1 công cụ có giá trị để chọn thuật toán phù hợp nhất cho 1 trường hợp bài toán nhất định. Để tạo điều kiện thuận lợi cho việc sử dụng cuốn sách làm tài liệu tham khảo, chúng tôi đã cố gắng giữ các chương này độc lập nhất có thể.
    
    For these advanced audiences assume familiarity with basic theory of linear programming, elementary elements of graph theory, \& concepts of algorithms \& data structures as far as they are generally taught in basic courses on these subjects.
    
    -- Đối với những đối tượng nâng cao này, giả sử họ đã quen thuộc với lý thuyết cơ bản về lập trình tuyến tính, các yếu tố cơ bản của lý thuyết đồ thị, \& các khái niệm về thuật toán \& cấu trúc dữ liệu theo như những gì thường được dạy trong các khóa học cơ bản về các chủ đề này.
    
    Chaps. 4--12 give detailed presentations of knapsack problem \& its variants in increasing order of structural difficulty. Hence, start with subset sum problem in Chap. 4, move on to standard knapsack problem which is discussed extensively in 2 chaps, 1 for exact \& 1 for approximate algorithms, \& finish this 2nd part of book with bounded \& unbounded knapsack problem in Chaps. 7--8.
    
    3rd part of book contains more complicated generalizations of knapsack problems. It starts with multidimensional knapsack problem (a knapsack problem with $d$ constraints) in Chap. 9, then considers multiple knapsack problem ($m$ knapsacks are available for packing) in Chap. 10, goes on to multiple-choice knapsack problem (items are partitioned into classes \& exactly 1 item of each class must be packed), \& extends linear objective function to a quadratic one yielding quadratic knapsack problem in Chap. 12. This chap also contains an excursion to semidefinite programming giving a mostly self-contained short introduction to this topic.
    
    -- Phần thứ 3 của cuốn sách bao gồm những khái quát phức tạp hơn về các bài toán ba lô. Nó bắt đầu với bài toán ba lô đa chiều (một bài toán ba lô với $d$ ràng buộc) trong Chương 9, sau đó xem xét nhiều bài toán ba lô ($m$ ba lô có sẵn để đóng gói) trong Chương 10, tiếp tục đến bài toán ba lô trắc nghiệm (các mục được phân vùng thành các lớp \& chính xác 1 mục của mỗi lớp phải được đóng gói), \& mở rộng hàm mục tiêu tuyến tính thành 1 hàm bậc hai tạo ra bài toán ba lô bậc hai trong Chương 12. Chương này cũng bao gồm 1 chuyến tham quan đến lập trình bán xác định cung cấp phần giới thiệu ngắn gọn chủ yếu là độc lập về chủ đề này.
    
    All these 6 chaps can be seen as survey articles, most of them being 1st survey on their subject, containing many pointers to literature \& some examples of application.
    
    Particular effort was put into description of interesting applications of knapsack type problems. Decided to avoid a boring listing of umpteen papers with a 2-line description of occurrence of a knapsack problem for each of them, but selected a smaller number of application areas where knapsack models play a prominent role. These areas are discussed in more detail in Chap. 15 to give reader a full understanding of situations presented. They should be particularly useful for teaching purposes.
    
    -- Nỗ lực đặc biệt đã được đưa vào việc mô tả các ứng dụng thú vị của các bài toán loại ba lô. Quyết định tránh 1 danh sách nhàm chán của hàng tá bài báo với mô tả 2 dòng về sự xuất hiện của 1 bài toán ba lô cho mỗi bài báo, nhưng đã chọn 1 số ít các lĩnh vực ứng dụng mà các mô hình ba lô đóng vai trò nổi bật. Các lĩnh vực này được thảo luận chi tiết hơn trong Chương 15 để cung cấp cho người đọc sự hiểu biết đầy đủ về các tình huống được trình bày. Chúng sẽ đặc biệt hữu ích cho mục đích giảng dạy.
    
    Appendix gives a short presentation of NP-completeness with focus on knapsack problems. Without venturing into depths of theoretical CS \& avoiding topics e.g. Turing machines \& unary encoding, a rather informal introduction to NP-completeness is given, however with formal proofs for NP-hardness of subset sum \& knapsack problem.
    
    -- Phụ lục trình bày ngắn gọn về NP-completeness tập trung vào các bài toán knapsack. Không đi sâu vào CS lý thuyết \& tránh các chủ đề như máy Turing \& mã hóa đơn phân, phần giới thiệu khá không chính thức về NP-completeness được đưa ra, tuy nhiên có các bằng chứng chính thức cho NP-hardness của tổng tập con \& bài toán knapsack.
    
    Some assumptions \& conventions concerning notation \& style are kept throughout book. Most algorithms are stated in a flexible pseudocode style putting emphasis on readability instead of formal uniformity. I.e., simpler algorithms are given in style of an known but easily understandable programming language, whereas more complex algorithms are introduced by a structured, but verbal description. Commands \& verbal instructions are given in {\sf Sans Serif} font, whereas comments follow in {\it Italic} letters. As a general reference \& guideline to algorithms we used book by Cormen, Leiserson, Rivest \& Stein [92].
    
    -- 1 số giả định \& quy ước liên quan đến ký hiệu \& phong cách được giữ nguyên trong toàn bộ cuốn sách. Hầu hết các thuật toán được nêu theo phong cách mã giả linh hoạt, nhấn mạnh vào khả năng đọc được thay vì tính thống nhất về mặt hình thức. Tức là, các thuật toán đơn giản hơn được đưa ra theo phong cách của 1 ngôn ngữ lập trình đã biết nhưng dễ hiểu, trong khi các thuật toán phức tạp hơn được giới thiệu bằng mô tả có cấu trúc nhưng bằng lời. Các lệnh \& hướng dẫn bằng lời được đưa ra bằng phông chữ {\sf Sans Serif}, trong khi các bình luận theo sau bằng chữ {\it Italic}. Là tài liệu tham khảo chung \& hướng dẫn cho các thuật toán, chúng tôi đã sử dụng cuốn sách của Cormen, Leiserson, Rivest \& Stein [92].
    
    For sake of readability \& personal taste follow non-standard convention of using term {\it increasing} instead of mathematically correct {\it nondecreasing} \& in same way {\it decreasing} instead of {\it nonincreasing}. Whereever use log function always refer to base 2 logarithm unless stated otherwise. After Preface give a short list of notations containing only those terms which are used throughout book. Many more naming conventions will be introduced on a local level during individual chaps \& sects.
    
    A number of computational experiments were performed for exact algorithms. These were performed on following machines: AMD Athlon, Intel Pentium 4, III. Performance index was obtained from SPEC \url{www.specbench.org}. As can be seen 3 machines have reasonably similar performance, making it possible to compare running times across chaps. Codes have been compiled using GNU project C \& C++ Compiler gcc-2.96, which also compiles Fortran77 code, thus preventing differences in computation times due to alternative compilers.
    
    -- 1 số thí nghiệm tính toán đã được thực hiện cho các thuật toán chính xác. Những thí nghiệm này được thực hiện trên các máy sau: AMD Athlon, Intel Pentium 4, III. Chỉ số hiệu suất được lấy từ SPEC \url{www.specbench.org}. Như có thể thấy, 3 máy có hiệu suất khá giống nhau, giúp có thể so sánh thời gian chạy giữa các chap. Các mã đã được biên dịch bằng dự án GNU C \& C++ Compiler gcc-2.96, cũng biên dịch mã Fortran77, do đó ngăn ngừa sự khác biệt về thời gian tính toán do các trình biên dịch thay thế.
    \item {\sf Acknowledgments.} Authors strongly beieve in necessity to do research not with an island mentality but in an open exchange of knowledge, opinions \& ideas within an international research community. Clearly, none of us would have been able to contribute to this book without innumerable personal exchanges with colleagues on conferences \& workshops, in person, by email or even by surface mail. Therefore, like to start our acknowledgments by thanking global research community for providing spirit necessary for joint projects of collection \& presentation. Importance of solving knapsack problems for all values of capacity.
    
    -- Các tác giả tin tưởng mạnh mẽ vào sự cần thiết phải nghiên cứu không phải với tâm lý đảo mà trong sự trao đổi cởi mở về kiến thức, ý kiến \& ý tưởng trong cộng đồng nghiên cứu quốc tế. Rõ ràng, không ai trong chúng ta có thể đóng góp cho cuốn sách này nếu không có vô số cuộc trao đổi cá nhân với các đồng nghiệp tại các hội nghị \& hội thảo, trực tiếp, qua email hoặc thậm chí qua thư. Do đó, chúng tôi muốn bắt đầu lời cảm ơn của mình bằng cách cảm ơn cộng đồng nghiên cứu toàn cầu đã cung cấp tinh thần cần thiết cho các dự án chung về thu thập \& trình bày. Tầm quan trọng của việc giải quyết các vấn đề ba lô cho tất cả các giá trị của năng lực.
    \item {\sf1. Introduction.}
    \begin{itemize}
        \item {\sf1.1. Introducing Knapsack Problem.} Every aspect of human life is crucially determined by result of decisions. Whereas private decisions may be based on emotions or personal taste, complex professional environment of 21st century requires a decision process which can be formalized \& validated independently from involved individuals. Therefore, a quantitative formulation of all factors influencing a decision \& also of result of decision process is sought.
        
        -- Mọi khía cạnh của cuộc sống con người đều được quyết định 1 cách quan trọng bởi kết quả của các quyết định. Trong khi các quyết định riêng tư có thể dựa trên cảm xúc hoặc sở thích cá nhân, môi trường chuyên nghiệp phức tạp của thế kỷ 21 đòi hỏi 1 quy trình ra quyết định có thể được chính thức hóa \& xác thực độc lập với các cá nhân liên quan. Do đó, cần tìm kiếm 1 công thức định lượng của tất cả các yếu tố ảnh hưởng đến quyết định \& cũng như kết quả của quy trình ra quyết định.
        
        In order to meet this goal it must be possible to represent effect of any decision by numerical values. In most basic case outcome of decision can be measured by a single value representing gain, profit, loss, cost or some other category of data. Comparison of these values induces a total order on set of all options which are available for a decision. Finding option with highest or lowest value can be difficult because set of available options may be extremely large \&{\tt/}or not explicitly known. Frequently, only conditions are known which characterize feasible options out of a very general ground set of theoretically available choices.
        
        -- Để đạt được mục tiêu này, phải có thể biểu diễn hiệu ứng của bất kỳ quyết định nào bằng các giá trị số. Trong hầu hết các trường hợp cơ bản, kết quả của quyết định có thể được đo bằng 1 giá trị duy nhất biểu diễn mức tăng, lợi nhuận, tổn thất, chi phí hoặc 1 số loại dữ liệu khác. So sánh các giá trị này tạo ra thứ tự tổng thể trên tập hợp tất cả các tùy chọn có sẵn cho 1 quyết định. Việc tìm ra tùy chọn có giá trị cao nhất hoặc thấp nhất có thể khó khăn vì tập hợp các tùy chọn có sẵn có thể cực kỳ lớn \&{\tt}hoặc không được biết rõ ràng. Thông thường, chỉ có các điều kiện được biết là đặc trưng cho các tùy chọn khả thi trong 1 tập hợp cơ sở rất chung của các lựa chọn có sẵn về mặt lý thuyết.
        
        Simplest possible form of a decision is choice between 2 alternatives. Such a {\it binary decision} is formulated in a quantitative model as a {\it binary variable} $x\in\{0,1\}$ with obvious meaning that $x = 1$ means taking 1st alternative whereas $x = 0$ indicates rejection of 1st alternative \& hence selection of 2nd option.
        
        -- Dạng đơn giản nhất có thể của 1 quyết định là lựa chọn giữa 2 phương án thay thế. Một {\it quyết định nhị phân} như vậy được xây dựng trong 1 mô hình định lượng như 1 {\it biến nhị phân} $x\in\{0,1\}$ với ý nghĩa rõ ràng là $x = 1$ có nghĩa là chọn phương án thay thế thứ nhất trong khi $x = 0$ biểu thị việc từ chối phương án thay thế thứ nhất \& do đó lựa chọn phương án thứ 2.
        
        Many practical decision processes can be represented by an appropriate combination of several binary decisions. I.e., overall decision problem consists of choosing 1 of 2 alternatives for a large number of binary decisions which may all influence each other. In basic version of a {\it linear decision model} outcome of complete decision process is evaluated by a linear combination of values associated with each of binary decisions. In order to take pairwise interdependencies between decisions into account also a {\it quadratic function} can be used to represent outcome of a decision process. Feasibility of a particular selection of alternatives may be very complicated to establish in practice because binary decisions may influence or even contract each other.
        
        -- Nhiều quá trình ra quyết định thực tế có thể được biểu diễn bằng sự kết hợp thích hợp của 1 số quyết định nhị phân. Tức là, vấn đề quyết định tổng thể bao gồm việc lựa chọn 1 trong 2 phương án thay thế cho 1 số lượng lớn các quyết định nhị phân mà tất cả đều có thể ảnh hưởng lẫn nhau. Trong phiên bản cơ bản của {\it mô hình quyết định tuyến tính}, kết quả của toàn bộ quá trình ra quyết định được đánh giá bằng sự kết hợp tuyến tính của các giá trị liên quan đến từng quyết định nhị phân. Để tính đến sự phụ thuộc lẫn nhau theo cặp giữa các quyết định, cũng có thể sử dụng {\it hàm bậc hai} để biểu diễn kết quả của 1 quá trình ra quyết định. Tính khả thi của 1 lựa chọn thay thế cụ thể có thể rất phức tạp để thiết lập trong thực tế vì các quyết định nhị phân có thể ảnh hưởng hoặc thậm chí thu hẹp lẫn nhau.
        
        Formally speaking, linear decision model is defined by $n$ binary variables $x_j\in\{0,1\}$ which correspond to selection in $j$th binary decision \& by {\it profit values} $p_j$ which indicate difference of value attained by choosing 1st alternative, i.e., $x_j = 1$, instead of 2nd alternative $x_j = 0$. W.l.o.g. can assume: after a suitable assignment of 2 options to 2 cases $x_j = 1$ \& $x_j = 0$, always have $p_j\ge0$. Overall profit value associated with a particular  choice $\forall n$ binary decisions is given by sum of all values $p_j$ for all decisions where 1st alternative was selected.
        
        -- Về mặt hình thức, mô hình quyết định tuyến tính được định nghĩa bởi $n$ biến nhị phân $x_j\in\{0,1\}$ tương ứng với lựa chọn trong quyết định nhị phân $j$th \& bởi {\it giá trị lợi nhuận} $p_j$ biểu thị sự khác biệt về giá trị đạt được khi chọn phương án thứ nhất, tức là $x_j = 1$, thay vì phương án thứ hai $x_j = 0$. W.l.o.g. có thể giả sử: sau khi gán phù hợp 2 tùy chọn cho 2 trường hợp $x_j = 1$ \& $x_j = 0$, luôn có $p_j\ge0$. Tổng giá trị lợi nhuận liên quan đến một lựa chọn cụ thể $\forall n$ quyết định nhị phân được đưa ra bởi tổng của tất cả các giá trị $p_j$ cho tất cả các quyết định trong đó phương án thứ nhất được chọn.
        
        Consider decision problems where feasibility of a particular selection of alternatives can be evaluated by a linear combination of coefficients for each binary decision. In this model feasibility of a selection of alternatives is determined by a {\it capacity restriction} in following way. In every binary decision $j$ selection of 1st alternative $x_j = 1$ requires a {\it weight} or {\it resource} $w_j$ whereas choosing 2nd alternative $x_j = 0$ does not. A selection of alternatives is feasible if sum of weights over all binary decisions does not exceed a given threshold capacity value $c$. This condition can be written as $\sum_{i=1}^n w_ix_i\le c$. Considering this decision process as an optimization problem, where overall profit should be as large as possible, yields {\it knapsack problem} (KP), core problem of this book.
        
        -- Hãy xem xét các bài toán quyết định trong đó tính khả thi của một lựa chọn thay thế cụ thể có thể được đánh giá bằng tổ hợp tuyến tính các hệ số cho mỗi quyết định nhị phân. Trong mô hình này, tính khả thi của một lựa chọn thay thế được xác định bằng {\it hạn chế năng lực} theo cách sau. Trong mọi quyết định nhị phân $j$, lựa chọn phương án thứ nhất $x_j = 1$ yêu cầu {\it trọng số} hoặc {\it tài nguyên} $w_j$ trong khi việc chọn phương án thứ hai $x_j = 0$ thì không. Một lựa chọn thay thế là khả thi nếu tổng trọng số trên tất cả các quyết định nhị phân không vượt quá giá trị ngưỡng năng lực $c$ đã cho. Điều kiện này có thể được viết là $\sum_{i=1}^n w_ix_i\le c$. Xem xét quá trình quyết định này như một bài toán tối ưu hóa, trong đó lợi nhuận tổng thể phải lớn nhất có thể, sẽ tạo ra {\it bài toán ba lô} (KP), bài toán cốt lõi của cuốn sách này.
        
        This characteristic of problem gives rise to following interpretation of (KP) which is more colorful than combination of binary decision problems. Consider a mountaineer who is packing his knapsack (or rucksack) for a mountain tour \& has to decide which items he should take with him. He has a large number of objects available which may be useful on his tour. Each of these items numbered from 1 to $n$ would give him a certain amount of comfort or benefit which is measured by a positive number $p_j$. Of course, weight $w_j$ of every object which mountaineer puts into his knapsack increases load he has to carry. For obvious reasons, he wants to limit total weight of his knapsack \& hence fixes maximum load by capacity value $c$.
        
        -- Đặc điểm này của bài toán dẫn đến cách giải thích sau đây về (KP) có nhiều màu sắc hơn so với việc kết hợp các bài toán quyết định nhị phân. Hãy xem xét một người leo núi đang đóng gói ba lô (hoặc ba lô du lịch) của mình cho một chuyến leo núi \& phải quyết định những vật dụng nào anh ta nên mang theo. Anh ta có một số lượng lớn các vật dụng có sẵn có thể hữu ích trong chuyến đi của mình. Mỗi vật dụng trong số này được đánh số từ 1 đến $n$ sẽ mang lại cho anh ta một lượng thoải mái hoặc lợi ích nhất định được đo bằng một số dương $p_j$. Tất nhiên, trọng lượng $w_j$ của mỗi vật dụng mà người leo núi để vào ba lô của mình sẽ làm tăng tải trọng mà anh ta phải mang. Vì những lý do hiển nhiên, anh ta muốn giới hạn tổng trọng lượng của ba lô của mình \& do đó cố định tải trọng tối đa theo giá trị sức chứa $c$.
        
        In order to give a more intuitive presentation, use this knapsack interpretation \& usually refer to a ``packing of items into a knapsack'' rather than to ``combination of binary decisions'' throughout this book. Also terms ``profit'' \& ``weight'' are based on this interpretation. Instead of making a number of binary decisions will speak of selection of a subset of items from {\it item set} $N\coloneqq[n]$.
        
        -- Để đưa ra một bài trình bày trực quan hơn, hãy sử dụng cách diễn giải ba lô này \& thường ám chỉ ``đóng gói các mặt hàng vào ba lô'' thay vì ``kết hợp các quyết định nhị phân'' trong suốt cuốn sách này. Ngoài ra, các thuật ngữ ``lợi nhuận'' \& ``trọng số'' cũng dựa trên cách diễn giải này. Thay vì đưa ra một số quyết định nhị phân sẽ nói về việc lựa chọn một tập hợp con các mặt hàng từ {\it item set} $N\coloneqq[n]$.
        
        Knapsack problem (KP) can be formally defined as follows: We are given an {\it instance} of knapsack problem with item set $N$, consisting of $n$ {\it items} $j$ with {\it profit} $p_j$ \& {\it weight} $w_j$, \& {\it capacity value} $c$. (Usually, all these values are taken from positive integer numbers.) Then objective: select a subset of $N$ s.t. total profit of selected items is maximized \& total weight does not exceed $c$.
        
        Alternatively, a knapsack problem can be formulated as a solution of linear integer programming formulation:
        \begin{equation*}
            (KP)\ \max\sum_{i=1}^n p_ix_i\mbox{ subject to }\sum_{i=1}^n w_ix_i\le c,\ x_i\in\{0,1\}\,\forall j\in[n].
        \end{equation*}
        Denote {\it optimal solution vector} by $x^* = (x_1^*,\ldots,x_n^*)$ \& {\it optimal solution value} by $z^*$. Set $X^*$ denotes {\it optimal solution set}, i.e., set of items corresponding to optimal solution vector.
        
        Problem (KP) is simplest nontrivial integer programming model with binary variables, only 1 single constraint \& only positive coefficients. Nevertheless, adding integrality condition (1.3) $x_i\in\{0,1\}\,\forall j\in[n]$ to simple linear program (1.1)--(1.2) already puts (KP) into class of ``difficult'' problems.
        
        -- Bài toán (KP) là mô hình lập trình số nguyên phi tầm thường đơn giản nhất với các biến nhị phân, chỉ có 1 ràng buộc duy nhất \& chỉ có hệ số dương. Tuy nhiên, việc thêm điều kiện tích phân (1.3) $x_i\in\{0,1\}\,\forall j\in[n]$ vào chương trình tuyến tính đơn giản (1.1)--(1.2) đã đưa (KP) vào lớp các bài toán ``khó''.
        
        Knapsack problem has been studied for centuries as it is simplest prototype of a maximization problem. Already in 1897 Mathews [338] showed how several constraints may be aggregated into 1 single knapsack constraint. This is somehow a prototype of a reduction of a general integer program to (KP), thus proving: (KP is at least as hard to solve as an integer program. It is however unclear how name ``Knapsack Problem''  was invented. {\sc Dantzig} is using expression in his early work \& thus name could be a kind of folklore.
        
        -- Bài toán ba lô đã được nghiên cứu trong nhiều thế kỷ vì đây là nguyên mẫu đơn giản nhất của bài toán tối đa hóa. Ngay từ năm 1897, Mathews [338] đã chỉ ra cách tổng hợp nhiều ràng buộc thành 1 ràng buộc ba lô duy nhất. Đây bằng cách nào đó là nguyên mẫu của phép rút gọn một chương trình số nguyên tổng quát thành (KP), do đó chứng minh rằng: (KP khó giải quyết ít nhất cũng như một chương trình số nguyên. Tuy nhiên, vẫn chưa rõ tên ``Bài toán ba lô'' được phát minh như thế nào. {\sc Dantzig} đang sử dụng cách diễn đạt trong tác phẩm đầu của mình \& do đó tên có thể là một loại văn hóa dân gian.
        
        Consider above characteristic of mountaineer in context of business instead of leisure leads to a 2nd classical interpretation of (KP) as an investment problem. A wealthy individual or institutional investor has a certain amount of money $c$ available which she wants to put into profitable business projects. As a basis for her decisions she compiles a long list of possible investments including for every investment required amount $w_i$ \& expected net return $p_i$ over a fixed period. Aspect of risk is not explicitly taken into account here. Obviously, combination of binary decisions for every investment s.t. overall return on investment is as large as possible can be formulated by (KP).
        
        -- Xem xét đặc điểm trên của người leo núi trong bối cảnh kinh doanh thay vì giải trí dẫn đến cách giải thích cổ điển thứ 2 của (KP) như một vấn đề đầu tư. Một cá nhân giàu có hoặc nhà đầu tư tổ chức có một số tiền nhất định $c$ có sẵn mà cô ấy muốn đầu tư vào các dự án kinh doanh có lợi nhuận. Để làm cơ sở cho các quyết định của mình, cô ấy biên soạn một danh sách dài các khoản đầu tư khả thi bao gồm cho mọi khoản đầu tư cần thiết là $w_i$ \& lợi nhuận ròng dự kiến $p_i$ trong một khoảng thời gian cố định. Khía cạnh rủi ro không được tính đến một cách rõ ràng ở đây. Rõ ràng, sự kết hợp của các quyết định nhị phân cho mọi khoản đầu tư s.t. tổng lợi nhuận đầu tư lớn nhất có thể có thể được xây dựng bằng (KP).
        
        A 3rd illustrating example of a real-world economic situation which is captured by (KP) is taken from airline cargo business. Dispatcher of a cargo airline has to decide which of transportation requests posed by customers he should fulfill, i.e., how to load a particular plane. His decision is based on a list of requests which contain weight $w_i$ of every package \& rate per weight unit charged for each request. Note: this rate is not fixed but depends on particular long-term arrangements with every customer. Hence profit $p_i$ made by company by accepting a request \& by putting corresponding package on plane is not directly proportional to weight of package. Naturally, every plane has a specified maximum capacity $c$ which may not be exceeded by total weight of selected packages. This logistic problem is a direct analogon to packing of mountaineers knapsack.
        
        -- Ví dụ minh họa thứ 3 về tình hình kinh tế thực tế được (KP) nắm bắt được lấy từ hoạt động vận chuyển hàng hóa của hãng hàng không. Người điều phối của một hãng hàng không phải quyết định nên đáp ứng yêu cầu vận chuyển nào của khách hàng, tức là cách chất hàng lên một chiếc máy bay cụ thể. Quyết định của anh ta dựa trên danh sách các yêu cầu có chứa trọng lượng $w_i$ của mọi kiện hàng \& giá cước theo đơn vị trọng lượng được tính cho mỗi yêu cầu. Lưu ý: giá cước này không cố định mà phụ thuộc vào các thỏa thuận dài hạn cụ thể với từng khách hàng. Do đó, lợi nhuận $p_i$ mà công ty kiếm được khi chấp nhận một yêu cầu \& khi đặt kiện hàng tương ứng lên máy bay không tỷ lệ thuận với trọng lượng của kiện hàng. Đương nhiên, mỗi máy bay đều có sức chứa tối đa quy định $c$ không được vượt quá tổng trọng lượng của các kiện hàng đã chọn. Vấn đề hậu cần này tương tự trực tiếp với việc đóng gói ba lô của người leo núi.
        
        While previous examples all contain elements of ``packing'', one may also view (KP) as a ``cutting'' problem. Assume: a sawmill has to cut a log into shorter pieces. Pieces must however be cut into some predefined standard-lengths $w_i$, where each length has an associated selling price $p_i$. In order to maximize profit of log, sawmill can formulate problem as a (KP) where length of log defines capacity $c$.
        
        -- Trong khi các ví dụ trước đều chứa các yếu tố của ``đóng gói'', người ta cũng có thể xem (KP) là một vấn đề ``cắt''. Giả sử: một xưởng cưa phải cắt một khúc gỗ thành các đoạn ngắn hơn. Tuy nhiên, các đoạn gỗ phải được cắt thành một số chiều dài chuẩn được xác định trước $w_i$, trong đó mỗi chiều dài có giá bán liên quan là $p_i$. Để tối đa hóa lợi nhuận của khúc gỗ, xưởng cưa có thể xây dựng vấn đề như một (KP) trong đó chiều dài của khúc gỗ xác định công suất $c$.
        
        An interesting example of knapsack problem from academia which may appeal to teachers \& students was reported by Feuerman \& Weiss [144]. They describe a test procedure from a college in Norwalk, Connecticut, where students may select a subset of given question. To be more precise, students receive $n$ questions each with a certain ``weight'' indicating number of points that can be scored for that question with a total of e.g. 125 points. However, after exam all questions answered by students are graded by instructor who assigns points to teach answer. Then a subset of questions is selected to determine overall grade s.t. maximum number of reachable points for this subset is below a certain threshold, e.g., 100. To give students best possible marks subset should be chosen automatically s.t. scored points are as large as possible. This task is clearly equivalent to solving a knapsack problem with an item $i$ for $i$th question with $w_i$ representing reachable points \& $p_i$ actually scored points. Capacity $c$ gives threshold for limit of points of selected questions.
        
        -- 1 ví dụ thú vị về bài toán ba lô từ học viện có thể hấp dẫn giáo viên \& học sinh đã được Feuerman \& Weiss [144] báo cáo. Họ mô tả một quy trình kiểm tra từ một trường cao đẳng ở Norwalk, Connecticut, trong đó học sinh có thể chọn một tập hợp con của câu hỏi cho sẵn. Để chính xác hơn, học sinh nhận được $n$ câu hỏi, mỗi câu hỏi có một ``trọng số'' nhất định biểu thị số điểm có thể đạt được cho câu hỏi đó với tổng số điểm là 125 điểm. Tuy nhiên, sau khi thi, tất cả các câu hỏi do học sinh trả lời đều được giáo viên chấm điểm, người này sẽ chỉ định điểm để dạy trả lời. Sau đó, một tập hợp con các câu hỏi được chọn để xác định điểm tổng thể s.t. số điểm tối đa có thể đạt được cho tập hợp con này thấp hơn một ngưỡng nhất định, ví dụ: 100. Để cho học sinh điểm tốt nhất có thể, tập hợp con nên được chọn tự động s.t. điểm ghi được càng lớn càng tốt. Nhiệm vụ này rõ ràng tương đương với việc giải bài toán ba lô với một mục $i$ cho câu hỏi $i$th với $w_i$ biểu thị điểm có thể đạt được \& $p_i$ điểm thực tế đã ghi được. Dung lượng $c$ cung cấp ngưỡng giới hạn điểm của các câu hỏi đã chọn.
        
        However, as it is frequently case with industrial applications, in practice several additional constraints, e.g. urgency \& priority of requests, time windows for every request, packages with low weight but high volume, etc., have to be fulfilled. This leads to various extensions \& variations of basic model (KP). Because this need for extension of basic knapsack model arose in many practical optimization problems, some of more general variants of (KP) have become standard problems of their own. Introduce several of them in following sect \& deal with many others in later chaps of this book.
        
        -- Tuy nhiên, như thường thấy trong các ứng dụng công nghiệp, trong thực tế, một số ràng buộc bổ sung, ví dụ như tính cấp bách \& mức độ ưu tiên của các yêu cầu, khung thời gian cho mọi yêu cầu, các gói có trọng lượng thấp nhưng khối lượng lớn, v.v., phải được đáp ứng. Điều này dẫn đến nhiều phần mở rộng \& biến thể của mô hình cơ bản (KP). Vì nhu cầu mở rộng mô hình ba lô cơ bản này nảy sinh trong nhiều bài toán tối ưu hóa thực tế, một số biến thể tổng quát hơn của (KP) đã trở thành các bài toán chuẩn của riêng chúng. Giới thiệu một số trong số chúng trong phần sau \& giải quyết nhiều bài toán khác trong các chương sau của cuốn sách này.
        
        Beside these explicit occurrences of knapsack problems it should be noted: many solution methods of more complex problems employ knapsack problem (sometimes iteratively) as a subproblem. Therefore, a comprehensive study of knapsack problem carries many advantages for a wide range of mathematical models.
        
        -- Bên cạnh những trường hợp rõ ràng này của bài toán ba lô, cần lưu ý: nhiều phương pháp giải bài toán phức tạp hơn sử dụng bài toán ba lô (đôi khi lặp lại) như một bài toán con. Do đó, một nghiên cứu toàn diện về bài toán ba lô mang lại nhiều lợi thế cho nhiều mô hình toán học.
        
        From a didactic \& historic point of view it is worth mentioning: many techniques of combinatorial optimization \& also of CS were introduced in context of, or in connection with knapsack problems. 1 of 1st optimization problems to be considered in development of NP-hardness (Sect. 1.5) was subset sum problem (Sect. 1.2). Other concepts e.g. approximation schemes, reduction algorithms \& dynamic programming were established in their beginning based on or illustrated by knapsack problem.
        
        -- Theo quan điểm giáo khoa \& lịch sử, điều đáng đề cập là: nhiều kỹ thuật tối ưu hóa tổ hợp \& cũng như của CS đã được giới thiệu trong bối cảnh hoặc liên quan đến các bài toán ba lô. 1 trong những bài toán tối ưu hóa đầu tiên được xem xét trong quá trình phát triển độ khó NP (Phần 1.5) là bài toán tổng tập con (Phần 1.2). Các khái niệm khác ví dụ như các lược đồ xấp xỉ, thuật toán rút gọn \& lập trình động đã được thiết lập ngay từ đầu dựa trên hoặc minh họa bằng bài toán ba lô.
        
        Research in combinatorial optimization \& operational research can be carried out either in a top-down or bottom-up fashion. In top-down approach, researchers develop solution methods for most difficult optimization problems like {\it traveling salesman problem, quadratic assignment problem} or {\it scheduling problem}. If developed methods work for these difficult problems, may assume: they will also work for a large variety of other problems. Opposite approach: develop new methods for most simple model, like e.g. {\it knapsack problem}, hoping: techniques can be generalized to more complex models. Since both approaches are {\it method developing}, they justify a considerable research effort for solving a relatively simple problem.
        
        -- Nghiên cứu về tối ưu hóa tổ hợp \& nghiên cứu vận hành có thể được thực hiện theo cách từ trên xuống hoặc từ dưới lên. Theo cách tiếp cận từ trên xuống, các nhà nghiên cứu phát triển các phương pháp giải cho hầu hết các bài toán tối ưu hóa khó như {\it bài toán người bán hàng du lịch, bài toán gán bậc hai} hoặc {\it bài toán lập lịch}. Nếu các phương pháp đã phát triển có hiệu quả đối với các bài toán khó này, có thể cho rằng: chúng cũng sẽ có hiệu quả đối với nhiều bài toán khác. Cách tiếp cận ngược lại: phát triển các phương pháp mới cho hầu hết các mô hình đơn giản, chẳng hạn như {\it bài toán ba lô}, hy vọng: các kỹ thuật có thể được khái quát hóa thành các mô hình phức tạp hơn. Vì cả hai cách tiếp cận đều là {\it phương pháp đang phát triển}, chúng biện minh cho nỗ lực nghiên cứu đáng kể để giải quyết một bài toán tương đối đơn giản.
        
        
        {\sc Skiena} [437 reports an analysis of a quarter of a million requests to Stony Brook Algorithms Repository, to determine relative level of interest among 75 algorithmic problems. In this analysis it turns out: codes for knapsack problem are among top-20 of most most requested algorithms. When comparing interest to number of actual knapsack implementations, {\sc Skiena} concludes: knapsack algorithms are 3rd most needed implementations. Research should not be driven by demand figures alone, but analysis indicates: knapsack problems occur in many real-life applications \& solution of these problems is of vital interest both to industry \& administration.
        
        -- {\sc Skiena} [437 báo cáo phân tích một phần tư triệu yêu cầu gửi đến Stony Brook Algorithms Repository, để xác định mức độ quan tâm tương đối trong số 75 vấn đề thuật toán. Trong phân tích này, kết quả cho thấy: mã cho vấn đề knapsack nằm trong top 20 thuật toán được yêu cầu nhiều nhất. Khi so sánh mức độ quan tâm với số lượng triển khai knapsack thực tế, {\sc Skiena} kết luận: thuật toán knapsack là triển khai cần thiết thứ 3. Nghiên cứu không nên chỉ dựa trên số liệu nhu cầu, nhưng phân tích chỉ ra rằng: vấn đề knapsack xảy ra trong nhiều ứng dụng thực tế \& giải pháp của những vấn đề này có ý nghĩa sống còn đối với cả ngành \& quản lý.
        \item {\sf1.2. Variants \& Extensions of Knapsack Problem.} Consider again previous problem of cargo airline dispatcher. In a different setting profit made by accepting a package may be directly proportional to its weight. In this case optimal loading of plane is achieved by filling as much weight as possible into plane or equivalently setting $p_i = w_i$ in (KP). Resulting optimization problem is known as {\it subset sum problem} (SSP) because looking for a {\it subset} of values $w_i$ with {\it sum} being as close as possible to, but not exceeding given target value $c$.
        \begin{equation*}
            (SSP)\ \max\sum_{i=1}^n w_ix_i\mbox{ subject to }\sum_{i=1}^n w_ix_i\le c,\ x_i\in\{0,1\},\,\forall i\in[n].
        \end{equation*}
        
        -- Xem xét lại bài toán trước của người điều phối hàng không vận chuyển hàng hóa. Trong một bối cảnh khác, lợi nhuận thu được khi chấp nhận một gói hàng có thể tỷ lệ thuận với trọng lượng của nó. Trong trường hợp này, tải trọng tối ưu của máy bay đạt được bằng cách chất càng nhiều trọng lượng càng tốt vào máy bay hoặc tương đương là đặt $p_i = w_i$ trong (KP). Bài toán tối ưu hóa kết quả được gọi là {\it subset sum problem} (SSP) vì tìm kiếm một {\it subset} các giá trị $w_i$ với {\it sum} càng gần càng tốt nhưng không vượt quá giá trị mục tiêu $c$ đã cho.
        
        In original cargo problem described above it will frequently be case that not all packages are different from each other. In particular, in practice there may be given a number $b_i$ of identical copies of each item to be transported. If either have to accept a request $\forall b_i$ packages or reject them all then we can generate an artificial request with weight $b_iw_i$ generating a profit of $b_ip_i$> If possible to select also a subset of $b_i$ items from a request we can either represent each individual package by a binary variable or more efficiently represent whole set of identical packages by an integer variable $x_i\ge0$ indicating number of packages of this type which are put into plane. In this case number of variables is $=$ number of different packages instead of total number of packages. This may decrease size of model considerably if numbers $b_i$ are relatively large. Formally, constraint (1.3) in (KP) is replaced by (1.4)
        \begin{equation*}
            0\le x_i\le b_i,\ x_i\in\mathbb{N},\ \forall i\in[n].
        \end{equation*}
        Resulting problem is called {\it bounded knapsack problem} (BKP). Chap. 7 is devoted to (BKP). A special variant thereof is {\it unbounded knapsack problem} (UKP) (Chap. 8). Also known as {\it integer knapsack problem} where instead of a fixed number $b_i$ a very large or an infinite amount of identical copies of each item is given. In this case, constraint (1.3) in (KP) is simply replaced by (1.5)
        \begin{equation*}
            x_i\in\mathbb{N},\ \forall i\in[n].
        \end{equation*}
        Moving in a different direction, consider again above cargo problem \& now take into account not only weight constraint but also limited space available to transport packages. For practical purposes only volume of packages is considered \& not their different shapes.
        
        -- Trong bài toán vận chuyển hàng hóa ban đầu được mô tả ở trên, thường xảy ra trường hợp không phải tất cả các gói hàng đều khác nhau. Đặc biệt, trong thực tế, có thể đưa ra số lượng $b_i$ bản sao giống hệt nhau của mỗi mặt hàng cần vận chuyển. Nếu phải chấp nhận yêu cầu $\forall b_i$ gói hàng hoặc từ chối tất cả thì chúng ta có thể tạo một yêu cầu nhân tạo với trọng số $b_iw_i$ tạo ra lợi nhuận là $b_ip_i$> Nếu có thể chọn cả một tập hợp con của $b_i$ mặt hàng từ một yêu cầu, chúng ta có thể biểu diễn từng gói hàng riêng lẻ bằng một biến nhị phân hoặc hiệu quả hơn là biểu diễn toàn bộ tập hợp các gói hàng giống hệt nhau bằng một biến số nguyên $x_i\ge0$ biểu thị số lượng các gói hàng cùng loại được đưa vào mặt phẳng. Trong trường hợp này, số lượng biến là $=$ số lượng các gói hàng khác nhau thay vì tổng số các gói hàng. Điều này có thể làm giảm đáng kể kích thước của mô hình nếu số lượng $b_i$ tương đối lớn. Về mặt hình thức, ràng buộc (1.3) trong (KP) được thay thế bằng (1.4)
        \begin{equation*}
            0\le x_i\le b_i,\ x_i\in\mathbb{N},\ \forall i\in[n].
        \end{equation*}
        Bài toán kết quả được gọi là {\it bounded knapsack problem} (BKP). Chương 7 dành riêng cho (BKP). Một biến thể đặc biệt của nó là {\it unbounded knapsack problem} (UKP) (Chương 8). Còn được gọi là {\it integer knapsack problem} trong đó thay vì một số cố định $b_i$, một số lượng rất lớn hoặc vô hạn các bản sao giống hệt nhau của mỗi mục được đưa ra. Trong trường hợp này, ràng buộc (1.3) trong (KP) chỉ được thay thế bằng (1.5)
        \begin{equation*}
            x_i\in\mathbb{N},\ \forall i\in[n].
        \end{equation*}
        Di chuyển theo một hướng khác, hãy xem xét lại bài toán hàng hóa ở trên \& bây giờ hãy tính đến không chỉ hạn chế về trọng lượng mà còn cả không gian hạn chế có sẵn để vận chuyển các gói hàng. Đối với mục đích thực tế, chỉ thể tích của các gói hàng được xem xét \& không phải hình dạng khác nhau của chúng.
        
        Denoting weight of every item by $w_{1i}$ \& its volume by $w_{2i}$ \& introducing weight capacity of plane as $c_1$ \& upper bound on volume as $c_2$ we can formulate extended cargo problem by replacing constraint (1.2) in (KP) by 2 inequalities:
        \begin{equation*}
            \sum_{i=1}^n w_{1i}x_i\le c_1,\ \sum_{i=1}^n w_{2i}x_i\le c_2,.
        \end{equation*}
        Obvious generalization of this approach, where $d$ instead of 2 inequalities are introduced, yields {\it$d$-dimensional knapsack problem} or {\it multidimensional knapsack problem} (Chap. 9) formally defined by
        \begin{equation*}
            (d-KP)\ \max\sum_{i=1}^n p_ix_i\mbox{ subject to }\sum_{j=1}^n w_{ij}x_j\le c_i,\ \forall i\in[d],\ x_j\in\{0,1\},\ \forall j\in[n].
        \end{equation*}
        Another interesting variant of cargo problem arises from original version described above if consider a very busy flight route, e.g. Frankfurt--New York, which is flown by several planes everyday. In this case dispatcher has to decide on loading of a number of planes in parallel, i.e., it has to be decided whether to accept a particular transportation request \& in positive case on which plane to put corresponding package. Concepts of profit, weight, \& capacity remain unchanged. This can be formulated by introducing a binary decision variable for every combination of a package with a plane. If there are $n$ items on list of transportation requests \& $m$ planes available on this route we use $mn$ binary variables $x_{ij}$ for $i\in[m],j\in[n]$ with (1.6)
        \begin{equation*}
            x_{ij} = \left\{\begin{split}
                &1&&\mbox{if item } j\mbox{ is put into plane } i,\\
                &0&&\mbox{otherwise}.
            \end{split}\right.
        \end{equation*}
        Mathematical programming formulation of this {\it multiple knapsack problem} (MKP) is given by
        \begin{equation*}
            (MKP)\ \max\sum_{i=1}^m\sum_{j=1}^n p_jx_{ij}\mbox{ subject to }\sum_{j=1}^n w_jx_{ij}\le c_i,\ \forall i\in[m],\ \sum_{i=1}^m x_{ij}\le1,\ \forall j\in[n],\ x_{ij}\in\{0,1\},\ \forall i\in[m],\,\forall j\in[n].
        \end{equation*}
        p. 27+++
        \item {\sf1.3. Single-Capacity vs. All-Capacities Problem.}
    \end{itemize}
    \item {\sf2. Basic Algorithmic Concepts.}
    \item {\sf3. Advanced Algorithmic Concepts.}
    \item {\sf4. Subset Sum Problem.}
    \item {\sf5. Exact Solution of Knapsack Problem.}
    \item {\sf6. Approximation Algorithms for Knapsack Problem.}
    \item {\sf7. Bounded Knapsack Problem.}
    \item {\sf8. Unbounded Knapsack Problem.}
    \item {\sf9. Multidimensional Knapsack Problems.}
    \item {\sf10. Multiple Knapsack Problems.}
    \item {\sf11. Multiple-Choice Knapsack Problem.}
    \item {\sf12. Quadratic Knapsack Problem.}
    \item {\sf13. Other Knapsack Problems.}
    \item {\sf14. Stochastic Aspects of Knapsack Problems.}
    \item {\sf15. Some Selected Applications.}
    \item {\sf Appendix A: Introduction to NP-Completeness of Knapsack Problems.}
\end{itemize}


%------------------------------------------------------------------------------%

\section{Scheduling Problems -- Bài Toán Phân Công Thời Gian}

%------------------------------------------------------------------------------%

\subsection{{\sc Michael L. Pinedo}. Scheduling: Theory, Algorithms, \& Systems. 6e}

\begin{itemize}
    \item {\sf Preface to 6e.} Since release of 1e in 1994, scheduling field has seen many new developments that are of interest to theoreticians \& practitioners alike. Clearly, new editions of book, with extensions in different directions, were to be expected. However, basic general setup \& structure of book has not changed over years. It still consists of 3 main parts: Deterministic Models, Stochastic Models, \& Scheduling in Practice. There are also 4 Appendices that present basics of mathematical programming, dynamic programming, constraint programming, \& complexity theory, as well as 3 Appendixes that provide overviews of complexity statuses of several classes of deterministic scheduling problems, of tractability of a variety of stochastic scheduling problems, \& of latest developments in scheduling system designs \& implementations.
    
    Since its introduction in 1994 this book has undergone a number of expansions, with extensions in areas that have aroused interest in academia as well as industry. These extensions have included over years multi-objective scheduling, batch scheduling, \& proportionate flow shop scheduling. Additions in this 6e include deterministic flow shop models with reentry (i.e., scheduling models that are of interest to semiconductor manufacturing industry), stochastic models with due date related objective functions, Fixed Parameter Tractability (FPT) of deterministic scheduling problems, as well as discussions regarding recent scheduling system implementations. Latest updates of various tables in Appendixes E, F, G have been extensive.
    
    Part I of book (Deterministic Models) can still be used as basis for a course in deterministic scheduling at a Senior or Masters level in an Engineering school or in an Applied Mathematics department. Parts I \& II together (Deterministic \& Stochastic Models) can be used as a basis for a Masters or PhD level course in an engineering or a business school.
    
    A solution manual is still available. However, because of specific requests made by several of faculty who have contributed to contents of this manual, it can only be sent out to instructors who actually teach a course at an established university. There is also a fair amount of supplementary material available, closely related to content of book (e.g., PowerPoint presentations, scheduling cases, etc.), that can be downloaded from author's homepage at \url{wp.nyu.edu/michaelpinedo/books/}.
    \item {\sf Gantt Chart: Its Originators.} Every practitioner \& researcher in scheduling field has used Gantt charts as a scheduling tool. Origins of these famous charts are interesting; they were developed independently in Poland \& in US in late 19th \& early 20th century.
    
    In 1896 Polish engineer \& economist {\sc Karol Adamiecki} developed what he called {\it harmonogram}, which he used in management of a steel rolling mill in southern Poland. {\sc Adamiecki} published his techniques in Polish magazine {\it Przeglad Techniczny} (Technical Review) in 1909.
    
    Shortly thereafter, {\sc Henry Laurence Gantt}, an industrial engineer working in steel industry in US, developed his charts for evaluating production schedules. {\sc Gantt} discussed principles underlying his charts in his book ``Organizing for Work'', which was published just before his death in 1919. Since {\sc Gantt} communicated in English \& {\sc Adamiecki} in Polish, these types of charts are in English literature usually referred to as Gantt charts.
    
    Charts currently being used in real world decision support systems are at times somewhat different from originals developed by {\sc Adamiecki \& Gantt}, in their design as well as in their purpose.
    \item {\sf Supplementary Electronic Material.} Supplementary electronic material listed below is available for download from author's homepage at \url{wp.nyu.edu/michaelpinedo/books/}: slides from Academia, scheduling systems: LEKIN, LiSA, TORSCHE, scheduling case: scheduling in time-shared jet business, mini-cases, handouts, movies: SAIGA -- scheduling at Paris Airports (ILOG), scheduling at United Airlines, preactor international.
    \item {\sf1. Introduction.}
    \begin{itemize}
        \item {\sf1.1. Role of Scheduling.} Scheduling is a decision-making process that is used on a regular basis in many manufacturing \& services industries. It deals with allocation of resources to tasks over given time periods \& its goal is to optimize 1 or more objectives.
        
        -- {\it Vai trò của Lập lịch.} Lập lịch là 1 quá trình ra quyết định được sử dụng thường xuyên trong nhiều ngành sản xuất \& dịch vụ. Nó liên quan đến việc phân bổ nguồn lực cho các nhiệm vụ trong khoảng thời gian nhất định \& mục tiêu của nó là tối ưu hóa 1 hoặc nhiều mục tiêu.
        
        Resources \& tasks in an organization can take many different forms. Resources may be machines in a workshop, runways at an airport, crews at a construction site, processing units in a computing environment, etc. Tasks may be operations in a production process, take-offs \& landings at an airport, stages in a construction project, executions of computer programs, etc. Each task may have a certain priority level, an earliest possible starting time, \& a due date. Objectives can also take many different forms. 1 objective may be minimization of completion time of last tasks \& another may be minimization of number of tasks completed after their respective due dates.
        
        -- Tài nguyên \& nhiệm vụ trong 1 tổ chức có thể có nhiều hình thức khác nhau. Tài nguyên có thể là máy móc trong xưởng, đường băng tại sân bay, phi hành đoàn tại công trường xây dựng, đơn vị xử lý trong môi trường máy tính, v.v. Nhiệm vụ có thể là các hoạt động trong quy trình sản xuất, cất cánh \& hạ cánh tại sân bay, các giai đoạn trong dự án xây dựng, thực hiện các chương trình máy tính, v.v. Mỗi nhiệm vụ có thể có 1 mức độ ưu tiên nhất định, thời gian bắt đầu sớm nhất có thể, \& ngày đến hạn. Mục tiêu cũng có thể có nhiều hình thức khác nhau. 1 mục tiêu có thể là giảm thiểu thời gian hoàn thành của các nhiệm vụ cuối cùng \& mục tiêu khác có thể là giảm thiểu số lượng nhiệm vụ hoàn thành sau ngày đến hạn tương ứng của chúng.
        
        Scheduling, as a decision-making process, plays an important role in most manufacturing \& production systems as well as in most information processing environments. Also important in transportation \& distribution settings \& in other types of service industries. Following examples illustrate role of scheduling in a number of real-world environments.
        
        -- Lên lịch, như 1 quá trình ra quyết định, đóng vai trò quan trọng trong hầu hết các hệ thống sản xuất \& cũng như trong hầu hết các môi trường xử lý thông tin. Cũng quan trọng trong các thiết lập vận chuyển \& phân phối \& trong các loại ngành dịch vụ khác. Các ví dụ sau minh họa vai trò của việc lên lịch trong 1 số môi trường thực tế.
        
        \begin{example}[A paper bag factory]
            Consider a factory that produces paper bags for cement, charcoal, dog food, etc. Basic raw material for such an operation are rolls of paper. Production process consists of 3 stages: printing of logo, gluing of side of bag, \& sewing of 1 end or both ends of bag. Each stage consists of a number of machines which are not necessarily identical. Machines at a stage may differ slightly in speed at which they operate, number of colors they can print, or size of bag they can produce. Each production order indicates a given quantity of a specific bag that has to be produced \& shipped by a committed shipping date or due date. Processing time for different operations are proportional to size of order, i.e., number of bags ordered.
            
            -- Hãy xem xét 1 nhà máy sản xuất túi giấy đựng xi măng, than củi, thức ăn cho chó, v.v. Nguyên liệu thô cơ bản cho hoạt động như vậy là cuộn giấy. Quy trình sản xuất bao gồm 3 công đoạn: in logo, dán mép túi, \& may 1 đầu hoặc cả hai đầu túi. Mỗi công đoạn bao gồm 1 số máy không nhất thiết phải giống hệt nhau. Các máy ở 1 công đoạn có thể hơi khác nhau về tốc độ hoạt động, số lượng màu có thể in hoặc kích thước túi có thể sản xuất. Mỗi lệnh sản xuất chỉ ra số lượng nhất định của 1 loại túi cụ thể phải được sản xuất \& vận chuyển theo ngày giao hàng hoặc ngày đến hạn đã cam kết. Thời gian xử lý cho các hoạt động khác nhau tỷ lệ thuận với quy mô đơn hàng, tức là số lượng túi đã đặt hàng.
            
            A late delivery implies a penalty in form of loss of goodwill \& magnitude of penalty depends on importance of order or client \& tardiness of delivery. 1 of objectives of scheduling system is to minimize sum of these penalties.
            
            -- Giao hàng trễ sẽ phải chịu hình phạt dưới hình thức mất uy tín \& mức độ phạt phụ thuộc vào tầm quan trọng của đơn hàng hoặc khách hàng \& thời gian giao hàng chậm. 1 trong những mục tiêu của hệ thống lập lịch là giảm thiểu tổng số tiền phạt này.
            
            When a machine is switched over from 1 type of bag to another, a setup is required. Length of setup time on machine depends on similarities between 2 consecutive orders (number of colors in common, differences in bag size, etc.). An important objective of scheduling system is minimization of total time spent on setups.
            
            -- Khi máy được chuyển từ loại túi này sang loại túi khác, cần phải thiết lập. Thời gian thiết lập trên máy phụ thuộc vào điểm tương đồng giữa 2 đơn hàng liên tiếp (số lượng màu chung, sự khác biệt về kích thước túi, v.v.). Một mục tiêu quan trọng của hệ thống lập lịch là giảm thiểu tổng thời gian dành cho việc thiết lập.
        \end{example}
        
        \begin{example}[A semiconductor manufacturing facility]
            Semiconductors are manufactured in highly specialized facilities. This is case with memory chips as well as with microprocessors. Production process in these facilities usually consists of 4 phases: wafer fabrication, wafer probe, assembly or packaging, \& final testing.
            
            -- Chất bán dẫn được sản xuất tại các cơ sở chuyên dụng cao. Trường hợp này cũng xảy ra với chip nhớ cũng như với bộ vi xử lý. Quy trình sản xuất tại các cơ sở này thường bao gồm 4 giai đoạn: chế tạo wafer, đầu dò wafer, lắp ráp hoặc đóng gói, \& thử nghiệm cuối cùng.
            
            Wafer fabrication is technologically the most complex phase. Layers of metal \& wafer material are built up in patterns on wafers of silicon or gallium arsenide to produce circuitry. Each layer requires a number of operations, which typically include (i) cleaning, (ii) oxidation, deposition, \& metallization, (iii) lithography, (iv) etching, (v) ion implantation, (vi) photoresist stripping, (vii) inspection \& measurement. Because it consists of various layers, each wafer has to undergo these operations several times. Thus, there is a significant amount of recirculation in process. Wafers move through facility in lots of $24$. Some machines may require setups to prepare them for incoming jobs; setup time often depends on configurations of lot just completed \& lot about to start.
            
            -- Chế tạo wafer là giai đoạn phức tạp nhất về mặt công nghệ. Các lớp kim loại \& vật liệu wafer được tạo thành các mẫu trên wafer silicon hoặc gali arsenide để tạo ra mạch điện. Mỗi lớp yêu cầu 1 số thao tác, thường bao gồm (i) làm sạch, (ii) oxy hóa, lắng đọng, \& kim loại hóa, (iii) quang khắc, (iv) khắc, (v) cấy ion, (vi) loại bỏ chất cản quang, (vii) kiểm tra \& đo lường. Vì bao gồm nhiều lớp khác nhau, mỗi wafer phải trải qua các thao tác này nhiều lần. Do đó, có 1 lượng tuần hoàn đáng kể trong quá trình này. Các wafer di chuyển qua cơ sở theo từng lô $24$. Một số máy có thể yêu cầu thiết lập để chuẩn bị cho các công việc sắp tới; thời gian thiết lập thường phụ thuộc vào cấu hình của lô vừa hoàn thành \& lô sắp bắt đầu.
            
            Number of orders in production process is often in hundreds \& each has its own release date \& a committed shipping or due date. Scheduler's objective: meet as many of committed shipping dates as possible, which maximizing throughput. Latter goal is achieved by maximizing equipment utilization, especially of bottleneck machines, requiring thus a minimization of idle times \& setup times.
            
            -- Số lượng đơn hàng trong quá trình sản xuất thường lên tới hàng trăm \& mỗi đơn hàng có ngày phát hành riêng \& ngày giao hàng hoặc ngày đến hạn đã cam kết. Mục tiêu của người lập lịch: đáp ứng càng nhiều ngày giao hàng đã cam kết càng tốt, qua đó tối đa hóa thông lượng. Mục tiêu sau đạt được bằng cách tối đa hóa việc sử dụng thiết bị, đặc biệt là các máy móc bị tắc nghẽn, do đó đòi hỏi phải giảm thiểu thời gian nhàn rỗi \& thời gian thiết lập.
        \end{example}
        
        \begin{example}[Gate assignments at an airport]
            Consider an airline terminal at a major airport. There are dozens of gates \& hundreds of planes arriving \& departing each day. Gates are not all identical \& neither are planes. Some of gates are in locations with a lot of space where large planes (widebodies) can be accommodated easily. Other gates are in locations where it is difficult to bring in the planes; certain planes may actually have to be towed to their gates.
            
            -- Hãy xem xét 1 nhà ga hàng không tại 1 sân bay lớn. Có hàng chục cổng \& hàng trăm máy bay đến \& khởi hành mỗi ngày. Không phải tất cả các cổng đều giống nhau \& máy bay cũng vậy. Một số cổng nằm ở những vị trí có nhiều không gian, nơi có thể dễ dàng chứa được những chiếc máy bay lớn (máy bay thân rộng). Những cổng khác nằm ở những vị trí khó đưa máy bay vào; 1 số máy bay thực sự có thể phải được kéo đến cổng của chúng.
            
            Planes arrive \& depart according to a certain schedule. However, schedule is subject to a certain amount of randomness, which may be weather related or caused by unforeseen events at other airports. During time that a plane occupies a gate the arriving passengers have to be deplaned, plane has to be serviced, \& departing passengers have to be boarded. Scheduled departure time can be viewed as a due date \& airline's performance is measured accordingly. However, if it is known in advance: plane cannot land at next airport because of anticipated congestion at its scheduled arrival time, then plane does not take off (such a policy is followed to conserve fuel). If a plane is not allowed to take off, operating policies usually prescribe that passengers remain in terminal rather than on plane. If boarding is postponed, a plane may remain at a gate for an extended period of time, thus preventing other planes from using that gate.
            
            -- Máy bay đến \& khởi hành theo 1 lịch trình nhất định. Tuy nhiên, lịch trình phụ thuộc vào 1 lượng ngẫu nhiên nhất định, có thể liên quan đến thời tiết hoặc do các sự kiện không lường trước được tại các sân bay khác. Trong thời gian máy bay chiếm 1 cổng, hành khách đến phải xuống máy bay, máy bay phải được phục vụ, \& hành khách khởi hành phải lên máy bay. Thời gian khởi hành theo lịch trình có thể được xem là ngày đến hạn \& hiệu suất của hãng hàng không được đo lường theo đó. Tuy nhiên, nếu biết trước: máy bay không thể hạ cánh tại sân bay tiếp theo do dự đoán có tình trạng tắc nghẽn tại thời gian đến theo lịch trình, thì máy bay sẽ không cất cánh (chính sách như vậy được áp dụng để tiết kiệm nhiên liệu). Nếu máy bay không được phép cất cánh, các chính sách khai thác thường quy định hành khách phải ở lại nhà ga thay vì trên máy bay. Nếu việc lên máy bay bị hoãn, máy bay có thể ở lại 1 cổng trong 1 thời gian dài, do đó ngăn không cho các máy bay khác sử dụng cổng đó.
                        
            Scheduler has to assign planes to gates in such a way that assignment is physically feasible while optimizing a number of objectives. This implies: scheduler has to assign planes to suitable gates available at respective arrival times. Objective include minimization of work for airline personnel \& minimization of airplane delays.
            
            -- Người lập lịch phải chỉ định máy bay đến các cổng theo cách mà việc chỉ định khả thi về mặt vật lý trong khi tối ưu hóa 1 số mục tiêu. Điều này ngụ ý: người lập lịch phải chỉ định máy bay đến các cổng phù hợp có sẵn tại thời điểm đến tương ứng. Mục tiêu bao gồm giảm thiểu công việc cho nhân viên hãng hàng không \& giảm thiểu sự chậm trễ của máy bay.
            
            In this scenario gates are resources \& handling \& servicing of planes are tasks. Arrival of a plane at a gate represents starting time of a tasks \& departure represents its completion time.
            
            -- Trong kịch bản này, cổng là tài nguyên \& xử lý \& bảo dưỡng máy bay là nhiệm vụ. Máy bay đến cổng biểu thị thời gian bắt đầu của nhiệm vụ \& khởi hành biểu thị thời gian hoàn thành.            
        \end{example}
        
        \begin{example}[Scheduling tasks in a central processing unit (CPU)]
            1 of functions of a multi-tasking computer OS: schedule time CPU devotes to different programs that have to be executed. Exact processing times are usually not known in advance. However, distribution of these random processing times may be known in advance, including their means \& their variances. In addition, each task usually has a certain priority level (OS typically allows operators \& users to specify priority level or weight of each task). In such a case, objective: minimize expected sum of weighted completion times of all tasks.
            
            -- 1 trong những chức năng của máy tính đa nhiệm HĐH: thời gian lập lịch CPU dành cho các chương trình khác nhau phải được thực thi. Thời gian xử lý chính xác thường không được biết trước. Tuy nhiên, sự phân bố của các thời gian xử lý ngẫu nhiên này có thể được biết trước, bao gồm cả phương tiện \& phương sai của chúng. Ngoài ra, mỗi tác vụ thường có 1 mức độ ưu tiên nhất định (HĐH thường cho phép người vận hành \& người dùng chỉ định mức độ ưu tiên hoặc trọng số của từng tác vụ). Trong trường hợp như vậy, mục tiêu: giảm thiểu tổng thời gian hoàn thành có trọng số dự kiến của tất cả các tác vụ.
            
            To avoid situation where relatively short tasks remain in system for a long time waiting for much longer tasks that have a higher priority, OS ``slides'' each task into little pieces. OS then rotates these slices on CPU so that in any given time interval, CPU spends some amount of time on each task. This way, if by chance processing time of 1 of tasks is very short, task will be able to leave system relatively quickly.
            
            -- Để tránh tình huống các tác vụ tương đối ngắn vẫn nằm trong hệ thống trong thời gian dài chờ đợi các tác vụ dài hơn nhiều có mức độ ưu tiên cao hơn, HĐH ``trượt'' từng tác vụ thành các phần nhỏ. Sau đó, HĐH xoay các phần này trên CPU để trong bất kỳ khoảng thời gian nào, CPU dành 1 khoảng thời gian nhất định cho từng tác vụ. Theo cách này, nếu thời gian xử lý của 1 trong các tác vụ rất ngắn, tác vụ sẽ có thể rời khỏi hệ thống tương đối nhanh.
            
            An interruption of processing of a tasks is often referred to as a {\it preemption}. Clear: optimal policy in such an environment makes heavy use of preemptions.
            
            -- Việc gián đoạn xử lý tác vụ thường được gọi là quyền ưu tiên. Rõ ràng: chính sách tối ưu trong môi trường như vậy sử dụng nhiều quyền ưu tiên.
        \end{example}
        It may not be immediately clear what impact schedules may have on objectives of interest. Does it make sense to invest time \& effort searching for a good schedule rather than just choosing a schedule at random? In practice, it often turns out: choice of schedule {\it does} have a significant impact on system's performance \& it {\it does} make sense to spend some time \& effort searching for a suitable schedule.
        
        -- Có thể không rõ ràng ngay lập tức về tác động của lịch trình lên các mục tiêu quan tâm. Có hợp lý không khi đầu tư thời gian \& công sức tìm kiếm 1 lịch trình tốt thay vì chỉ chọn 1 lịch trình ngẫu nhiên? Trong thực tế, thường thì: lựa chọn lịch trình có tác động đáng kể đến hiệu suất của hệ thống \& việc dành thời gian \& công sức tìm kiếm 1 lịch trình phù hợp là hợp lý.
        
        Scheduling can be difficult from a technical as well as an implementation point of view. Type of difficulties encountered on technical side are similar to difficulties encountered in other forms of combinatorial optimization \& stochastic modeling. Difficulties on implementation side are of a completely different kind. They may depend on accuracy of model used for analysis of actual scheduling problem \& on reliability of input data needed.
        
        -- Lên lịch có thể khó khăn từ góc độ kỹ thuật cũng như góc độ triển khai. Loại khó khăn gặp phải về mặt kỹ thuật tương tự như khó khăn gặp phải trong các hình thức tối ưu hóa tổ hợp khác \& mô hình ngẫu nhiên. Khó khăn về mặt triển khai có loại hoàn toàn khác. Chúng có thể phụ thuộc vào độ chính xác của mô hình được sử dụng để phân tích vấn đề lập lịch thực tế \& độ tin cậy của dữ liệu đầu vào cần thiết.
        \item {\sf1.2. Scheduling Function in an Enterprise.} Scheduling function in a production system or service organization must interact with many other functions. These interactions are system-dependent \& may differ substantially from 1 situation to another. They often take place within an enterprise-wide information system.
        
        -- Chức năng lập lịch trong 1 hệ thống sản xuất hoặc tổ chức dịch vụ phải tương tác với nhiều chức năng khác. Những tương tác này phụ thuộc vào hệ thống \& có thể khác nhau đáng kể tùy từng tình huống. Chúng thường diễn ra trong 1 hệ thống thông tin toàn doanh nghiệp.
        
        A modern factory or service organization often has an elaborate information system in place that includes a central computer \& database. Local area networks of personal computers, workstations, \& data entry terminals, which are connected to this central computer, may be used either to retrieve data from database or to enter new data. Software controlling e.g. elaborate information system is typically referred to as an Enterprise Resource Planning (ERP) system. A number of software companies specialize in development of such systems, including SAP, J.D. Edwards, \& PeopleSoft. Such an ERP system plays role of an information highway that traverses enterprise with, at al organizational levels, links to decision support systems.
        
        -- 1 nhà máy hiện đại hoặc tổ chức dịch vụ thường có 1 hệ thống thông tin phức tạp bao gồm 1 máy tính trung tâm \& cơ sở dữ liệu. Mạng cục bộ của máy tính cá nhân, máy trạm, \& thiết bị đầu cuối nhập dữ liệu, được kết nối với máy tính trung tâm này, có thể được sử dụng để truy xuất dữ liệu từ cơ sở dữ liệu hoặc để nhập dữ liệu mới. Phần mềm kiểm soát ví dụ như hệ thống thông tin phức tạp thường được gọi là hệ thống Lập kế hoạch nguồn lực doanh nghiệp (ERP). Một số công ty phần mềm chuyên phát triển các hệ thống như vậy, bao gồm SAP, J.D. Edwards, \& PeopleSoft. Một hệ thống ERP như vậy đóng vai trò là 1 xa lộ thông tin đi qua doanh nghiệp với, ở mọi cấp độ tổ chức, liên kết đến các hệ thống hỗ trợ quyết định.
        
        Scheduling is often done interactively via a decision support system installed on a personal computer or workstation linked to ERP system. Terrminals at key locations connected to ERP system can give departments throughout enterprise access to all current scheduling information. These departments, in turn, can provide scheduling system with up-to-date information concerning statuses of jobs \& machines.
        
        -- Việc lập lịch thường được thực hiện tương tác thông qua hệ thống hỗ trợ quyết định được cài đặt trên máy tính cá nhân hoặc máy trạm được liên kết với hệ thống ERP. Các thiết bị đầu cuối tại các vị trí quan trọng được kết nối với hệ thống ERP có thể cung cấp cho các phòng ban trong toàn doanh nghiệp quyền truy cập vào tất cả thông tin lập lịch hiện tại. Đổi lại, các phòng ban này có thể cung cấp cho hệ thống lập lịch thông tin cập nhật liên quan đến trạng thái của công việc \& máy móc.
        
        There are, of course, still environments where communication between scheduling function \& other decision-making entities occurs in meetings or through memos.
        
        -- Tất nhiên, vẫn có những môi trường mà việc giao tiếp giữa chức năng lập lịch trình \& các thực thể ra quyết định khác diễn ra trong các cuộc họp hoặc thông qua bản ghi nhớ.
        \begin{itemize}
            \item {\sf Scheduling in Manufacturing.} Consider following generic manufacturing environment \& role of its scheduling. Orders that are released in a manufacturing setting have to be translated into jobs with associated due dates. These jobs often have to be processed on machines in a workcenter in a given order or sequence. Processing of jobs may sometimes be delayed if certain machines are busy \& preemptions may occur when high-priority jobs arrive at machines that are busy. Unforeseen events on shop floor, e.g. machine breakdowns or longer-than-expected processing times, also have to be taken into account, since they may have a major impact on schedules. In such an environment, development of a detailed task schedule helps maintain efficiency \& control of operations.
            
            -- {\it Lên lịch trong sản xuất.} Hãy cân nhắc đến môi trường sản xuất chung sau \& vai trò của lịch trình. Các đơn hàng được phát hành trong môi trường sản xuất phải được chuyển thành các công việc có ngày đến hạn liên quan. Những công việc này thường phải được xử lý trên các máy trong 1 trung tâm làm việc theo thứ tự hoặc trình tự nhất định. Việc xử lý các công việc đôi khi có thể bị chậm trễ nếu 1 số máy nhất định đang bận \& có thể xảy ra tình trạng chiếm dụng trước khi các công việc có mức độ ưu tiên cao đến với các máy đang bận. Các sự kiện không lường trước được trên sàn nhà máy, ví dụ như máy hỏng hoặc thời gian xử lý dài hơn dự kiến, cũng phải được tính đến, vì chúng có thể ảnh hưởng lớn đến lịch trình. Trong môi trường như vậy, việc phát triển 1 lịch trình công việc chi tiết giúp duy trì hiệu quả \& kiểm soát hoạt động.
            
            Shop floor is not only part of organization that impacts scheduling process. It is also affected by production planning process that handles medium- to long-term planning for entire organization. This process attempts to optimize firm's overall product mix \& long-term resource allocation based on its inventory levels, demand forecasts, \& resource requirements. Decisions made at this higher planning level may impact scheduling process directly. {\sf Fig. 1.1: Information flow diagram in a manufacturing system} depicts a diagram of information flow in a manufacturing system.
            
            -- Xưởng sản xuất không chỉ là 1 phần của tổ chức tác động đến quy trình lập lịch trình. Nó cũng bị ảnh hưởng bởi quy trình lập kế hoạch sản xuất xử lý kế hoạch trung hạn đến dài hạn cho toàn bộ tổ chức. Quy trình này cố gắng tối ưu hóa hỗn hợp sản phẩm tổng thể của công ty \& phân bổ nguồn lực dài hạn dựa trên mức tồn kho, dự báo nhu cầu, \& yêu cầu về nguồn lực. Các quyết định được đưa ra ở cấp độ lập kế hoạch cao hơn này có thể tác động trực tiếp đến quy trình lập lịch trình. {\sf Hình 1.1: Sơ đồ luồng thông tin trong hệ thống sản xuất} mô tả sơ đồ luồng thông tin trong hệ thống sản xuất.
            
            In a manufacturing environment, scheduling function has to interact with other decision-making functions. 1 popular system that is widely used is Material Requirements Planning (MRP) system. After a schedule has been generated, necessary: all raw materials \& resources are available at specified times. Ready dates of all jobs have to be determined jointly by production planning{\tt/}scheduling system \& MRP system.
            
            -- Trong môi trường sản xuất, chức năng lập lịch phải tương tác với các chức năng ra quyết định khác. 1 hệ thống phổ biến được sử dụng rộng rãi là hệ thống Lập kế hoạch yêu cầu vật liệu (MRP). Sau khi lập lịch, cần: tất cả nguyên vật liệu \& tài nguyên đều có sẵn tại thời điểm cụ thể. Ngày sẵn sàng của tất cả các công việc phải được xác định chung bởi hệ thống lập kế hoạch sản xuất{\tt/}lập lịch \& hệ thống MRP.
            
            MRP systems are normally fairly elaborate. Each job has a Bill of Materials (BOM) itemizing parts required for production. MRP system keeps track of inventory of each part. Furthermore, it determines timing of purchases of each 1 of materials. In doing so, it uses techniques e.g. lot sizing \& lot scheduling that are similar to those used in scheduling systems. There are many commercial MRP software packages available \&, as a result, there are many manufacturing facilities with MRP systems. In cases where facility does not have a scheduling system, MRP system may be used for production planning purposes. However, in complex settings, not easy for an MRP system to do detailed scheduling satisfactorily.
            
            -- Hệ thống MRP thường khá phức tạp. Mỗi công việc đều có 1 Danh mục vật liệu (BOM) liệt kê các bộ phận cần thiết cho sản xuất. Hệ thống MRP theo dõi hàng tồn kho của từng bộ phận. Hơn nữa, nó xác định thời điểm mua từng loại vật liệu. Khi làm như vậy, nó sử dụng các kỹ thuật ví dụ như định cỡ lô \& lập lịch lô tương tự như các kỹ thuật được sử dụng trong hệ thống lập lịch. Có nhiều gói phần mềm MRP thương mại có sẵn \& do đó, có nhiều cơ sở sản xuất có hệ thống MRP. Trong trường hợp cơ sở không có hệ thống lập lịch, hệ thống MRP có thể được sử dụng cho mục đích lập kế hoạch sản xuất. Tuy nhiên, trong các bối cảnh phức tạp, không dễ để hệ thống MRP lập lịch chi tiết 1 cách thỏa đáng.
            \item {\sf Scheduling in Services.} Describing a generic service organization \& a typical scheduling system is not as easy as describing a generic manufacturing organization. Scheduling function in a service organization may face a variety of problems. It may have to deal with reservation of resources, e.g., assignment of planes to gates, or reservation of meeting rooms or other facilities. Models used are at times somewhat different from those used in manufacturing settings. Scheduling in a service environment must be coordinated with other decision-making functions, usually within elaborate information systems, much in same way as scheduling function in a manufacturing setting. These information systems usually rely on extensive databases that contain all relevant information w.r.t. availability of resources \& (potential) customers. Scheduling system interacts often with forecasting \& yield management modules. {\sf Fig. 1.2: Information flow diagram in a service system} depicts information flow in a service organization e.g. a car rental agency. In contrast to manufacturing settings, there is usually no MRP system in a service environment.
            
            -- {\it Lên lịch trong Dịch vụ.} Việc mô tả 1 tổ chức dịch vụ chung \& 1 hệ thống lên lịch thông thường không dễ như mô tả 1 tổ chức sản xuất chung. Chức năng lên lịch trong 1 tổ chức dịch vụ có thể gặp phải nhiều vấn đề. Nó có thể phải xử lý việc đặt trước tài nguyên, ví dụ như phân công máy bay đến cổng, hoặc đặt trước phòng họp hoặc các cơ sở khác. Các mô hình được sử dụng đôi khi hơi khác so với các mô hình được sử dụng trong môi trường sản xuất. Lên lịch trong môi trường dịch vụ phải được phối hợp với các chức năng ra quyết định khác, thường là trong các hệ thống thông tin phức tạp, tương tự như chức năng lên lịch trong môi trường sản xuất. Các hệ thống thông tin này thường dựa vào các cơ sở dữ liệu mở rộng chứa tất cả thông tin có liên quan đến tính khả dụng của tài nguyên \& (khách hàng tiềm năng). Hệ thống lên lịch thường tương tác với các mô-đun dự báo \& quản lý năng suất. {\sf Hình 1.2: Sơ đồ luồng thông tin trong hệ thống dịch vụ} mô tả luồng thông tin trong 1 tổ chức dịch vụ, ví dụ như 1 công ty cho thuê ô tô. Trái ngược với môi trường sản xuất, thường không có hệ thống MRP trong môi trường dịch vụ.
        \end{itemize}
        \item {\sf1.3. Outline of Book.} This book focuses on both theory \& applications of scheduling. Theoretical side deals with detailed sequencing \& scheduling of jobs. Given a collection of jobs requiring processing in a certain machine environment, problem: sequence these jobs, subject to given constraints, in such a way that 1 or more performance criteria are optimized. Scheduler may have to deal with various forms of uncertainties, e.g. random job processing times, machines subject to breakdowns, rush orders, etc.
        
        -- Cuốn sách này tập trung vào cả lý thuyết \& ứng dụng của việc lập lịch. Mặt lý thuyết liên quan đến việc sắp xếp chi tiết \& lập lịch các công việc. Với 1 tập hợp các công việc cần xử lý trong 1 môi trường máy móc nhất định, vấn đề: sắp xếp các công việc này, tuân theo các ràng buộc nhất định, theo cách mà 1 hoặc nhiều tiêu chí hiệu suất được tối ưu hóa. Người lập lịch có thể phải xử lý nhiều dạng bất định khác nhau, ví dụ: thời gian xử lý công việc ngẫu nhiên, máy móc dễ hỏng hóc, đơn hàng gấp, v.v.
        
        Thousands of scheduling problems \& models have been studied \& analyzed in past. Obviously, only a limited number are considered in this book; selection is based on insight they provide, methodology needed for their analysis \& their importance in applications.
        
        -- Hàng ngàn vấn đề lập lịch \& mô hình đã được nghiên cứu \& phân tích trong quá khứ. Rõ ràng, chỉ có 1 số lượng hạn chế được xem xét trong cuốn sách này; việc lựa chọn dựa trên hiểu biết mà chúng cung cấp, phương pháp cần thiết cho việc phân tích \& tầm quan trọng của chúng trong các ứng dụng.
        
        Although applications driving models in this book come mainly from manufacturing \& production environments, clear from examples: scheduling plays a role in a wide variety of situations. Models \& concepts considered in this book are applicable in other settings as well.
        
        This book is divided into 3 parts. Part I (Chaps. 2--8) deals with deterministic scheduling models. In these chaps, assumed: there are a finite number of jobs that have to be schedule with 1 or more objectives to be minimized. Emphasis is placed on analysis of relatively simple priority or dispatching rules. Chap. 2 discusses notation \& gives an overview of models considered in subsequent chaps. Chaps. 3--8 consider various machine environments. Chaps. 3--4 deal with single machine, Chaps. 5 with machines in parallel, Chap. 6 with machines in series, \& Chap. 7 with more complicated job shop models. Chap. 8 focuses on open shops in which there are no restrictions on routings of jobs in shop.
        
        Part II (Chaps. 9--13) deals with stochastic scheduling models. These chaps, in most cases, also assume: a given (finite) number of jobs have to be scheduled. Job data, e.g. processing times, release dates, \& due dates, may not be exactly known in advance; only their distributions are known in advance. Actual processing times, release dates, \& due dates become known only at {\it completion} of processing or at actual occurrence of release or due date. In these models, a single objective has to be minimized, usually in expectation. Again, an emphasis is placed on analysis of relatively simple priority or dispatching rules. Chap. 9 contains preliminary material. Chap. 10 covers single machine environment. Chap. 11 also covers single machine, but in this chap, assumed: jobs are released at different points in time. This chap establishes relationship between stochastic scheduling \& theory of priority queues. Chap. 12 focuses on machines in parallel \& Chap. 13 describes more complicated flow shop, job shop, \& open shop models.
        
        Part III (Chaps. 14--20) deals with applications \& implementation issues. Algorithms are described for a number of real-world scheduling problems. Design issues for scheduling systems are discussed \& some examples of scheduling systems are given. Chaps. 14--15 describe various general purpose procedures that have proven to be useful in industrial scheduling systems. Chap. 16 describes a number of real-world scheduling problems \& how they have been dealt with in practice. Chap. 17 focuses on basic issues concerning design, development, \& implementation of scheduling systems, \& Chap. 18 discusses more advanced concepts in design \& implementation of scheduling systems. Chap. 19 gives some examples of actual implementations. Chaps. 20 ponders on what lies ahead in scheduling.
        
        Appendices A--D present short overviews of some of basic methodologies, namely mathematical programming, dynamic programming, constraint programming, \& complexity theory. Appendix E contains a complexity classification of deterministic scheduling problems, while Appendix F presents an overview of stochastic scheduling problems. Appendix G lists a number of scheduling systems that have been developed in industry \& academia. Appendix H provides some guidelines for using LEKIN scheduling system which can be downloaded from author's homepage.
        
        This book has been designed for either a master's level course or a beginning PhD level course in Production Scheduling with prerequisites being an elementary course in Operations Research \& an elementary course in stochastic processes. A senior-level course can cover some of sects in Parts I \& III. Such a course can be given without getting into complexity theory: one can go through chaps of Part I skipping all complexity proofs without loss of continuity. A master's level course may cover some sects in Part II as well. Even though all 3 parts are fairly self-contained, helpful to go through Chap. 2 before venturing into Part II.
        
        -- Cuốn sách này được thiết kế cho khóa học trình độ thạc sĩ hoặc khóa học trình độ tiến sĩ cơ bản về Lập lịch sản xuất với các điều kiện tiên quyết là khóa học cơ bản về Nghiên cứu hoạt động \& khóa học cơ bản về quy trình ngẫu nhiên. Một khóa học trình độ cao cấp có thể bao gồm 1 số giáo phái trong Phần I \& III. Một khóa học như vậy có thể được cung cấp mà không cần đi sâu vào lý thuyết phức tạp: người ta có thể học các chương của Phần I bỏ qua tất cả các bằng chứng về độ phức tạp mà không mất tính liên tục. Một khóa học trình độ thạc sĩ cũng có thể bao gồm 1 số giáo phái trong Phần II. Mặc dù cả 3 phần đều khá độc lập, nhưng vẫn hữu ích khi học Chương 2 trước khi bắt đầu Phần II.
        \item {\sf Comments \& Refs.} Over last 5 decade many books have appeared that focus on sequencing \& scheduling. These books range from elementary to very advanced.
        
        A volume edited by Muth and Thompson (1963) contains a collection of papers focusing primarily on computational aspects of scheduling. 1 of better known textbooks is the one by Conway, Maxwell, and Miller (1967) (which, even though slightly out of date, is still very interesting); this book also deals with some of stochastic aspects \& with priority queues. A more recent text by Baker (1974) gives an excellent overview of many aspects of deterministic scheduling. However, this book does not deal with computational complexity issues since it appeared just before research in computational complexity started to become popular. Book by Coffman1976 is a compendium of papers on deterministic scheduling; it does cover computational complexity. An introductory textbook by French1982 covers most of techniques used in deterministic scheduling. Proceedings of a NATO workshop, edited by Dempster, Lenstra, and Rinnooy Kan (1982), contains a number of advanced papers on deterministic $+$ on stochastic scheduling. p. 9+++
    \end{itemize}    
    PART I: DETERMINISTIC MODELS.
    \item {\sf2. Deterministic Models: Preliminaries.} Over last 50 years a considerable amount of research effort has been focused on deterministic scheduling. Number \& variety of models considered is astounding. During this time a notation has evolved that succinctly captures structure of many (but for sure not all) deterministic models that have been considered in literature.
    
    1st sect in this chap presents an adapted version of this notation. 2nd sect contains a number of examples \& describes some of shortcomings of framework \& notation. 3rd sect describes several classes of schedules. A class of schedules is typically characterized by freedom scheduler has in decision-making process. Last sect discusses complexity of scheduling problems introduced in 1st sect. This last sect can be used, together with Appendices D--E, to classify scheduling problems according to their complexity.
    \begin{itemize}
        \item {\sf2.1. Framework \& Notation.} In all scheduling problems considered number of jobs \& number of machines are assumed to be finite. Number of jobs is denoted by $n$ \& number of machines by $m$. Usually, subscript $j$ refers to a job while subscript $i$ refers to a machine. If a job requires a number of processing steps or operations, then pair $(i,j)$ refers to processing step or operation of job $j$ on machine $i$. Following pieces of data are associated with job $j$.
        \begin{enumerate}
            \item {\bf Processing time} $p_{ij}$ represents processing time of job $j$ on machine $i$. Subscript $i$ is omitted if processing time of job $j$ does not depend on machine or if job $j$ is only to be processed on 1 given machine.
            p. 14+++
        \end{enumerate}
        \item {\sf2.2. Examples.}
        \item {\sf2.3. Classes of Schedules.}
        \item {\sf2.4. Complexity Hierarchy.}
    \end{itemize}
    \item {\sf3. Single Machine Models (Deterministic).}
    \item {\sf4. Advanced Single Machine Models (Deterministic).}
    \item {\sf5. Parallel Machine Models (Deterministic).}
    \item {\sf6. Flow Shops \& Flexible Flow Shops (Deterministic).}
    \item {\sf7. Job Shops (Deterministic).}
    \item {\sf8. Open Shops (Deterministic).}
    
    PART II: STOCHASTIC MODELS.
    \item {\sf9. Stochastic Models: Preliminaries.}
    \item {\sf10. Single Machine Models (Stochastic).}
    \item {\sf11. Single Machine Models with Release Dates (Stochastic).}
    \item {\sf12. Parallel Machine Models (Stochastic).}
    \item {\sf13. Flow Shops, Job Shops, \& Open Shops (Stochastic).}
    
    PART III: SCHEDULING IN PRACTICE.
    \item {\sf14. General Purpose Procedures for Deterministic Scheduling.}
    \item {\sf15. More Advanced General Purpose Procedures.}
    \item {\sf16. Modeling \& Solving Scheduling Problems in Practice.}
    \item {\sf17. Design \& Implementation of Scheduling Systems: Basic Concepts.}
    \item {\sf18. Design \& Implementation of Scheduling Systems: More Advanced Concepts.}
    \item {\sf19. Examples of System Designs \& Implementations.}
    \item {\sf20. What Lies Ahead?}
    \item {\sf Appendix A: Mathematical Programming: Formulations \& Applications.}
    \item {\sf Appendix B: Deterministic \& Stochastic Dynamic Programming.}
    \item {\sf Appendix C: Constraint Programming.}
    \item {\sf Appendix D: Complexity Theory.}
    \item {\sf Appendix E: Complexity Classification of Deterministic Scheduling Problems.}
    \item {\sf Appendix F: Overview of Stochastic Scheduling Problems.}
    \item {\sf Appendix G: Selected Scheduling Systems.}
    \item {\sf Appendix H: Lekin System.}
\end{itemize}

%------------------------------------------------------------------------------%

\section{Miscellaneous}

%------------------------------------------------------------------------------%

\printbibliography[heading=bibintoc]
	
\end{document}