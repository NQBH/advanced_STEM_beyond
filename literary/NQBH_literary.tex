\documentclass{article}
\usepackage[backend=biber,natbib=true,style=alphabetic,maxbibnames=50]{biblatex}
\addbibresource{/home/nqbh/reference/bib.bib}
\usepackage[utf8]{vietnam}
\usepackage{tocloft}
\renewcommand{\cftsecleader}{\cftdotfill{\cftdotsep}}
\usepackage[colorlinks=true,linkcolor=blue,urlcolor=red,citecolor=magenta]{hyperref}
\usepackage{amsmath,amssymb,amsthm,enumitem,float,graphicx,mathtools,tikz}
\usetikzlibrary{angles,calc,intersections,matrix,patterns,quotes,shadings}
\allowdisplaybreaks
\newtheorem{assumption}{Assumption}
\newtheorem{baitoan}{}
\newtheorem{cauhoi}{Câu hỏi}
\newtheorem{conjecture}{Conjecture}
\newtheorem{corollary}{Corollary}
\newtheorem{dangtoan}{Dạng toán}
\newtheorem{definition}{Definition}
\newtheorem{dinhly}{Định lý}
\newtheorem{dinhnghia}{Định nghĩa}
\newtheorem{example}{Example}
\newtheorem{ghichu}{Ghi chú}
\newtheorem{hequa}{Hệ quả}
\newtheorem{hypothesis}{Hypothesis}
\newtheorem{lemma}{Lemma}
\newtheorem{luuy}{Lưu ý}
\newtheorem{nhanxet}{Nhận xét}
\newtheorem{notation}{Notation}
\newtheorem{note}{Note}
\newtheorem{principle}{Principle}
\newtheorem{problem}{Problem}
\newtheorem{proposition}{Proposition}
\newtheorem{question}{Question}
\newtheorem{remark}{Remark}
\newtheorem{theorem}{Theorem}
\newtheorem{vidu}{Ví dụ}
\usepackage[left=1cm,right=1cm,top=5mm,bottom=5mm,footskip=4mm]{geometry}
\def\labelitemii{$\circ$}
\DeclareRobustCommand{\divby}{%
	\mathrel{\vbox{\baselineskip.65ex\lineskiplimit0pt\hbox{.}\hbox{.}\hbox{.}}}%
}
\setlist[itemize]{leftmargin=*}
\setlist[enumerate]{leftmargin=*}

\title{Literary -- Văn Chương}
\author{Nguyễn Quản Bá Hồng\footnote{A Scientist {\it\&} Creative Artist Wannabe. E-mail: {\tt nguyenquanbahong@gmail.com}. Bến Tre City, Việt Nam.}}
\date{\today}

\begin{document}
\maketitle
\begin{abstract}
	This text is a part of the series {\it Some Topics in Advanced STEM \& Beyond}:
	
	{\sc url}: \url{https://nqbh.github.io/advanced_STEM/}.
	
	Latest version:
	\begin{itemize}
		\item {\it Literary -- Văn Chương}.
		
		PDF: {\sc url}: \url{https://github.com/NQBH/advanced_STEM_beyond/blob/main/literary/NQBH_literary.pdf}.
		
		\TeX: {\sc url}: \url{https://github.com/NQBH/advanced_STEM_beyond/blob/main/literary/NQBH_literary.tex}.
	\end{itemize}
\end{abstract}
\tableofcontents

%------------------------------------------------------------------------------%

\section{Wikipedia}

\subsection{Wikipedia{\tt/}style guide}
``A {\it style guide} is a set of standards for the writing, \href{https://en.wikipedia.org/wiki/Typesetting}{formatting}, \& design of \href{https://en.wikipedia.org/wiki/Document}{documents}. A book-length style guide is often called a {\it style manual} or a {\it manual of style} (MoS or MOS). A short style guide, typically ranging from several to several dozen pages, is often called a {\it style sheet}. The standards documented in a style guide are applicable either for general use, or prescribed use for an individual publication, particular organization, or specific field.

A style guide establishes standard style requirements to improve \href{https://en.wikipedia.org/wiki/Communication}{communication} by ensuring consistency within \& across documents. They may require certain \href{https://en.wikipedia.org/wiki/Best_practice}{best practices} in \href{https://en.wikipedia.org/wiki/Writing_style}{writing style}, \href{https://en.wikipedia.org/wiki/Usage_(language)}{usage}, \href{https://en.wikipedia.org/wiki/Composition_(language)}{language composition}, \href{https://en.wikipedia.org/wiki/Composition_(visual_arts)}{visual composition}, \href{https://en.wikipedia.org/wiki/Orthography}{orthography}, \& \href{https://en.wikipedia.org/wiki/Typography}{typography} by setting standards of usage in areas such as \href{https://en.wikipedia.org/wiki/Punctuation}{punctuation}, \href{https://en.wikipedia.org/wiki/Capitalization}{capitalization}, \href{https://en.wikipedia.org/wiki/Citation#Styles}{citing sources}, formatting of numbers \& dates, \href{https://en.wikipedia.org/wiki/Table_(information)}{table} appearance \& other areas. For \href{https://en.wikipedia.org/wiki/Academic_publishing}{academic} \& \href{https://en.wikipedia.org/wiki/Technical_communication}{technical} documents, a guide may also enforce the best practice in \href{https://en.wikipedia.org/wiki/Ethics}{ethics} (e.g. \href{https://en.wikipedia.org/wiki/Author}{authorship}, \href{https://en.wikipedia.org/wiki/Research_ethics}{research ethics}, \& disclosure) \& compliance (\href{https://en.wikipedia.org/wiki/Technical_standard}{technical} \& \href{https://en.wikipedia.org/wiki/Regulatory_compliance}{regulatory}). For translation, a style guide may even be used to enforce consistent grammar, tones, \& localization decisions such as \href{https://en.wikipedia.org/wiki/Unit_of_measurement}{units of measure}.

Style guides are specialized in a variety of ways, from the general use of a broad public audience, to a wide variety of specialized uses (such as for students \& scholars of various \href{https://en.wikipedia.org/wiki/Academic_discipline}{acadmeic disciplines}, \href{https://en.wikipedia.org/wiki/Medical_publishing}{medicine}, \href{https://en.wikipedia.org/wiki/Journalism}{journalism}, the \href{https://en.wikipedia.org/wiki/Legal_publication}{law}, \href{https://en.wikipedia.org/wiki/Government_Publishing_Office}{government}, business, \& specific \href{https://en.wikipedia.org/wiki/Industry_(economics)}{industries}). The term {\it house style} refers to the conventions defined by the style guide of a particular \href{https://en.wikipedia.org/wiki/Publisher}{publisher} or other organization.

\subsubsection{Varieties}
Style guides vary widely in scope \& size. Writers working in many large industries or professional sectors reference a specific style guide, written for their usage in specialized documents within their fields. For the most part, these guides are relevant \& useful for peer-to-peer specialist documentation or to help writers working in specific industries or sectors communicate highly technical information in scholarly articles or industry \href{https://en.wikipedia.org/wiki/White_paper}{white papers}.

Professional style guides of different countries can be referenced for authoritative advice on their respective language(s), e.g., the \href{https://en.wikipedia.org/wiki/New_Oxford_Style_Manual}{New Oxford Style Manual} from \href{https://en.wikipedia.org/wiki/Oxford_University_Press}{Oxford University Press}, UK; \& \href{https://en.wikipedia.org/wiki/The_Chicago_Manual_of_Style}{The Chicago Manual of Style} from the \href{https://en.wikipedia.org/wiki/University_of_Chicago_Press}{University of Chicago Press}, US; both Australia \& Canada have style guides -- available online -- created by their governments.

\paragraph{Sizes.} See also \href{https://en.wikipedia.org/wiki/List_of_style_guides}{Wikipedia{\tt/}list of style guides}. The variety in scope \& length is enabled by the cascading of 1 style over another, analogous to how styles cascade \href{https://en.wikipedia.org/wiki/Style_sheet_(web_development)}{in web development} \& \href{https://en.wikipedia.org/wiki/Style_sheet_(desktop_publishing)}{in desktop} cascade over \href{https://en.wikipedia.org/wiki/CSS}{CSS} styles.

In many cases, a project such as a \href{https://en.wikipedia.org/wiki/Book}{book}, \href{https://en.wikipedia.org/wiki/Academic_journal}{journal}, or \href{https://en.wikipedia.org/wiki/Monograph}{monograph} series typically has a short style sheet that cascades over the somewhat larger style guide of an organization such as a \href{https://en.wikipedia.org/wiki/Publishing}{publishing} company, whose specific content is usually called {\it house style}. Most house styles, in turn, cascade over an industry-wide style manual that is even more comprehensive. Examples of industry style guides include:
\begin{enumerate}
	\item \href{https://en.wikipedia.org/wiki/The_Associated_Press_Stylebook}{\it The Associated Press Stylebook} (AP Stylebook) \& {\it The Canadian Press Stylebook} for journalism
	\item \href{https://en.wikipedia.org/wiki/The_Chicago_Manual_of_Style}{\it The Chicago Manual of Style} (CMoS) \& \href{https://en.wikipedia.org/wiki/Hart%27s_Rules}{Oxford style} for general academic writing \& publishing
	\item \href{https://en.wikipedia.org/wiki/MHRA_Style_Guide}{Modern Humanities Research Association (MHRA) style} \& \href{https://en.wikipedia.org/wiki/ASA_style}{American Sociological Association (ASA) style} for the arts \& humanities
	\item \href{https://en.wikipedia.org/wiki/Oxford_Standard_for_Citation_of_Legal_Authorities}{Oxford Standard for Citation of Legal Authorities} (OSCOLA) \& \href{https://en.wikipedia.org/wiki/Bluebook}{Bluebook style} for law
	\item \href{https://en.wikipedia.org/wiki/GPO_style_manual}{US Government Publishing Office (USGPO) style} \& \href{https://en.wikipedia.org/wiki/Australian_Government_Publishing_Service}{Australian Government Publishing Service (AGPS) style} for government publication
\end{enumerate}
Finally, these reference works cascade over the \href{https://en.wikipedia.org/wiki/Orthography}{orthographic} norms of the language in use (e.g., \href{https://en.wikipedia.org/wiki/English_orthography}{English orthography} for English-language publications). This, of course, may be subject to national variety, e.g., \href{https://en.wikipedia.org/wiki/Comparison_of_American_and_British_English}{British, American, Canadian, \& Australian English}.

\paragraph{Topics.} Some style guides focus on specific topic areas e.g. \href{https://en.wikipedia.org/wiki/Graphic_design}{graphic design}, including \href{https://en.wikipedia.org/wiki/Typography}{typography}. Website style guides cover a publication's visual \& technical aspects as well as text.

Guides in specific scientific \& technical fields may cover \href{https://en.wikipedia.org/wiki/Nomenclature}{nomenclature} to specify names of classifying labels that are clear, standardized, \& \href{https://en.wikipedia.org/wiki/Ontology}{ontologically} sound (e.g., \href{https://en.wikipedia.org/wiki/Taxonomy_(biology)}{taxonomy}, \href{https://en.wikipedia.org/wiki/Chemical_nomenclature}{chemical nomenclature}, \& \href{https://en.wikipedia.org/wiki/Gene_nomenclature}{gene nomenclature}).

Style guides that cover \href{https://en.wikipedia.org/wiki/Usage_(language)}{usage} may suggest descriptive terms for people which avoid \href{https://en.wikipedia.org/wiki/Racism}{racism}, \href{https://en.wikipedia.org/wiki/Sexism}{sexism}, \href{https://en.wikipedia.org/wiki/Homophobia}{homophobia}, etc. Style guides increasingly incorporate \href{https://en.wikipedia.org/wiki/Accessibility}{accessibility} conventions for audience members with visual, mobility, or other disabilities.

\paragraph{Web style guides.} Since the rise of the digital age, websites have allowed for an expansion of style guide conventions that account for digital behavior such as screen reading (reading from a digitalized screen rather than a physical document). Screen reading requires web style guides to focus more intently on a user experience subjected to multichannel surfing. Though web style guides can also vary widely, they tend to prioritize similar values concerning brevity, terminology, syntax, tone, structure, typography, graphics, \& errors.

\subsubsection{Updating}
Most style guides are revised periodically to accommodate changes in conventions \& usage. The frequency of updating \& the \href{https://en.wikipedia.org/wiki/Revision_control}{revision control} are determined by the subject. For style manuals in \href{https://en.wikipedia.org/wiki/Reference_work}{reference-work} format, new \href{https://en.wikipedia.org/wiki/Edition_(book)}{editions} typically appear every 1--20 years. E.g., the {\it Ap Stylebook} is revised annually, \& the Chicago, \& ASA manuals are in their 17th, 7th, \& 6th editions, respectively, as of 2023. Many house styles \& individual project styles change more frequently, especially for new projects.'' -- \href{https://en.wikipedia.org/wiki/Style_guide}{Wikipedia{\tt/}style guide}

%------------------------------------------------------------------------------%

\subsection{Wikipedia{\tt/}English writing style}
``An {\it English writing style} is a combination of features in an \href{https://en.wikipedia.org/wiki/English_language}{English language} \href{https://en.wikipedia.org/wiki/Writing}{composition} that has become characteristics of a particular writer, a genre, a particular organization, or a profession more broadly (e.g., \href{https://en.wikipedia.org/wiki/Legal_writing}{legal writing}).

An individual's writing style may be distinctive for particular themes, personal idiosyncrasies of phrasing \&{\tt/}or \href{https://en.wikipedia.org/wiki/Idiolect}{idiolet}; recognizable combinations of these patterns may be defined metaphorically as a writer's ``voice.''

Organizations that employ writers or commission written work from individuals may require that writers conform to a ``\href{https://en.wikipedia.org/wiki/Style_guide}{house style}'' defined by the organization. This conformity enables a more consistent readability of composite works produced by many authors \& promotes usability of, e.g., references to other cited works.

In many kinds of professional writing aiming for effective transfer of information, adherence to a standardized style can facilitate the comprehension of readers who are already accustomed to it. Many of these standardized styles are documented in \href{https://en.wikipedia.org/wiki/Style_guide}{style guides}.

\subsubsection{Personal styles}
All writing has some style, even if the author is not thinking about a personal style. It is important to understand that style reflects meaning. E.g., if a writer wants to express a sense of euphoria\footnote{an extremely strong feeling of happiness \& excitement that usually lasts only a short time.}, he or she might write in a style overflowing with expressive modifiers. Some writers use styles that are very specific, e.g. in pursuit of an artistic effect. Stylistic rule-breaking is exemplified by the poet. An example is \href{https://en.wikipedia.org/wiki/E._E._Cummings}{E. E. Cummings}, whose writing consists mainly of only \href{https://en.wikipedia.org/wiki/Lower_case}{lower case} letters, \& often uses unconventional \href{https://en.wikipedia.org/wiki/Typography}{typography}, \href{https://en.wikipedia.org/wiki/White_space_(visual_arts)}{spacing}, \& \href{https://en.wikipedia.org/wiki/Punctuation}{punctuation}. Even in non-artistic writing, every person who writes has his or her own personal style.

\subsubsection{Proprietary styles}
Many large publications define a house style to be used throughout the publication, a practice almost universal among newspapers \& well-known magazines. These styles can cover the means of expression \& sentence structures, such as those adopted by \href{https://en.wikipedia.org/wiki/Time_(magazine)#Style}{\it Time}. They may also include features peculiar to a publication; the practice at \href{https://en.wikipedia.org/wiki/The_Economist#Tone_and_voice}{\it The Economist}, e.g., is that articles are rarely attributed to an individual author. General characteristics have also been prescribed for different categories of writing, e.g. in \href{https://en.wikipedia.org/wiki/News_style}{journalism}, the use of \href{https://en.wikipedia.org/wiki/SI#SI_writing_style}{SI units}, or \href{https://en.wikipedia.org/wiki/Questionnaire_construction}{questionnaire construction}.

\subsubsection{Academic styles}
University students, especially graduate students, are encouraged to write papers in an approved style. This practice promotes readability \& ensures that references to cited works are noted in a uniform way. Typically, students are encouraged to use a style commonly adopted by journals publishing articles in the field of study. The list of {\it Style Manuals \& Guides}, from the \href{https://en.wikipedia.org/wiki/University_of_Memphis}{University of Memphis} Libraries, includes 30 academic style manuals that are currently in print, \& 12 that are available online. Citation of referenced works is a key element in academic style.

The requirements for writing \& citing articles accessed online may sometimes differ from those for writing \& citing printed works. Some of the details are covered in {\it The Columbia Guide to Online Style}.'' -- \href{https://en.wikipedia.org/wiki/English_writing_style}{Wikipedia{\tt/}English writing style}

%------------------------------------------------------------------------------%

\subsection{Wikipedia{\tt/}rat race}
``A {\it rat race} is an endless, self-defeating, or pointless pursuit. The phrase equates humans to rats attempting to earn a reward such as cheese, in vain. It may also refer to a competitive struggle to get ahead financially or routinely.

The term is commonly associated with an exhausting, repetitive lifestyle that leaves no time for relaxation or enjoyment.

\subsubsection{Etymology}
In the late 1800s, the term ``rat-run'' was used meaning ``maze-like passages by which rats move about their territory'', commonly used in a derogatory sense.

By the 1930s actual rat races of some sort are frequently mentioned among carnival \& gambling attractions.

By 1934, ``rat-race'' was also used in reference to aviation training, referring to a ``\href{https://en.wikipedia.org/wiki/Follow-the-leader}{follow-the-leader}'' game in which a trainee fighter pilot had to copy all the actions (loops, rolls, spins, \href{https://en.wikipedia.org/wiki/Immelmann_turn}{Immelmann turns} etc.) performed by an experienced pilot.

From 1939, the phrase took on the meaning of ``competitive struggle'' referring to a person's work \& life.

\subsubsection{Historical usage}
\href{https://en.wikipedia.org/wiki/The_Rat_Race_(novel)}{\it The Rat Race} was used as a title for a novel written by \href{https://en.wikipedia.org/wiki/Jay_Franklin}{\sc Jay Franklin} in 1947 for \href{https://en.wikipedia.org/wiki/Colliers_Magazine}{Colliers Magazine} \& 1st published in book form in 1950. It is dedicated {\it To those few rats in Washington who do not carry brief-cases}.

The term ``rat race'' was used in an article about \href{https://en.wikipedia.org/wiki/Samuel_Goudsmit}{\sc Samuel Goudsmit} published in 1953 entitled: {\it A Farewell to String \& Sealing Wax\~I} in which \href{https://en.wikipedia.org/wiki/Daniel_Lang_(writer)}{\sc Daniel Lang} wrote:
\begin{quote}
	Sometimes when his sardonic mood is on him, he wonders whether the \href{https://en.wikipedia.org/wiki/Synchrotron}{synchrotrons}, the \href{https://en.wikipedia.org/wiki/Betatron}{betatrons}, the \href{https://en.wikipedia.org/wiki/Cosmotron}{cosmotrons}, \& all the other contrivances physicists have lately rigged up to create energy by accelerating particles of matter aren't playing a wry joke on their inventors. ``They are accelerating us too,'' he says, in a voice that still betrays a trace of the accent of his native Holland. In protesting against the speedup, {\sc Goudsmit} can speak with authority, for in the course of only a few years, he, like many other contemporary physicists, has seen his way of life change from a tranquil one of contemplation of a rat race.
\end{quote}
\href{https://en.wikipedia.org/wiki/Philip_K._Dick}{\sc Philip K. Dick} used the term in ``\href{https://en.wikipedia.org/wiki/The_Last_of_the_Masters}{The Last of the Masters}'' published in 1954:
\begin{quote}
	``Maybe,'' McLean said softly, ``you \& I can then get off this rat race. You \& I \& all the rest of us. \& live like human beings.'' ``Rat race,'' Fowler murmured. ``Rats in a maze. Doing tricks. Performing chores thought up by somebody else.'' McClean caught Fowler's eye. ``By somebody of another species.''
\end{quote}
\href{https://en.wikipedia.org/wiki/Jim_Bishop}{\sc Jim Bishop} used the term rat race in his book {\it The Golden Ham: A Candid Biography of Jackie Gleason}. The term occurs in a letter \href{https://en.wikipedia.org/wiki/Jackie_Gleason}{\sc Jackie Gleason} wrote to his wife in which he says: ``Television is a rat race, \& remember this, even if you win you are still a rat.''

\href{https://en.wikipedia.org/wiki/William_H._Whyte}{\sc William H. Whyte} used the term rat race in \href{https://en.wikipedia.org/wiki/The_Organization_Man}{The Organization Man} published in 1956:
\begin{quote}
	The word {\it collective} most of them can't bring themselves to use -- except to describe foreign countries or organizations they don't work for -- but they are keenly aware of how much more deeply beholden they are to organization than were their elders. They are wry about it, to be sure; they talk of the ``treadmill,'' the ``rat race,'' of the inability to control one's direction.
\end{quote}
\href{https://en.wikipedia.org/wiki/Merle_A._Tuve}{\sc Merle A. Tuve} used the term rat race in a 1959 article entitled ``Is Science Too Big for the Scientist?'', writing:
\begin{quote}
	There is a growing conviction among many of my friends in academic circles that the university today is no place for a scholar in science. A professor's life nowadays is a rat-race of busyness \& activity, managing contracts \& projects, guiding teams of assistants, bossing crews of technicians, making numerous trips, sitting on committees for government agencies, \& engaging in other distractions necessary to keep the whole frenetic business from collapse.
\end{quote}

\subsubsection{Solutions}
``Escaping the rat race'' can have a number of different meanings:
\begin{itemize}
	\item Movement from work or geographical location into (typically) a more rural area
	\item \href{https://en.wikipedia.org/wiki/Retirement}{Retirement}, quitting or ceasing work
	\item Moving from a job of high strenuosity to 1 of lesser strenuosity, like the \href{https://en.wikipedia.org/wiki/Tang_ping}{tang ping} lifestyle of young Chinese
	\item Adopting a \href{https://en.wikipedia.org/wiki/Buddha-like_mindset}{Buddha-like mindset}
	\item Changing to a different job altogether
	\item \href{https://en.wikipedia.org/wiki/Remote_work}{Remote work}
	\item Becoming \href{https://en.wikipedia.org/wiki/Financial_independence}{financially independent} from an employer
	\item Living in harmony with nature
	\item Developing an inner attitude of detachment from materialistic pursuits
	\item Alienation from the norms of society
\end{itemize}

\subsubsection{Music}
[$\ldots$]'' -- \href{https://en.wikipedia.org/wiki/Rat_race}{Wikipedia{\tt/}rat race}

%------------------------------------------------------------------------------%

\subsection{Wikipedia{\tt/}Town Mouse \& Country Mouse}
``{\it The Town Mouse \& the Country Mouse}'' is ` of \href{https://en.wikipedia.org/wiki/Aesop%27s_Fables}{Aesop's Fables}. It is number 352 in the \href{https://en.wikipedia.org/wiki/Perry_Index}{Perry Index} \& type 112 in \href{https://en.wikipedia.org/wiki/Aarne%E2%80%93Thompson}{Aarne--Thompson}'s folk tale index. Like several other elements in Aesop's fables, ``town mouse \& country mouse'' has become an English idiom.

\subsubsection{Story}
In the original tale, a proud town mouse visits his cousin in the country. The country mouse offers the city mouse a meal of simple country cuisine, at which the visitor scoffs \& invites the country mouse back to the city for a taste of the ``fine life'' \& the 2 cousins dine on \href{https://en.wikipedia.org/wiki/White_bread}{white bread} \& other fine foods. But their rich feast is interrupted by a cat which fores the rodent cousins to abandon their meal \& retreat back into their mouse hole for safety. The town mouse tells the country mouse that that cat killed his mother \& father \& that he is frequently the target of attacks. After hearing this, the country mouse decides to return home, preferring security to opulence or, as the 13th-century preacher \href{https://en.wikipedia.org/wiki/Odo_of_Cheriton}{Odo of Cheriton} phrased it, ``I'd rather gnaw a bean than be gnawed by continual fear''.

\subsubsection{Spread}
The story was widespread in Classical times \& there is an early Greek version by \href{https://en.wikipedia.org/wiki/Babrius}{\sc Babrius} (Fable 108). \href{https://en.wikipedia.org/wiki/Horace}{\sc Horace} included it as part of 1 of his satires (II.6), ending on this story in a poem comparing town living unfavorably to life in the country. \href{https://en.wikipedia.org/wiki/Marcus_Aurelius}{\sc Marcus Aurelius} alludes to it in his \href{https://en.wikipedia.org/wiki/Meditations}{Meditations}, Book 11.22; ``Think of the country mouse \& of the town mouse, \& of the alarm \& trepidation\footnote{great worry or fear about something unpleasant that may happen.} of the town mouse''.

However, it seems to have been the 12th century Anglo-Norman write \href{https://en.wikipedia.org/wiki/Walter_of_England}{Walter of England} who contributed most to the spread of the fable throughout medieval Europe. His Latin version (or that of Odo of Cheriton) has been credited as the source of the fable that appeared in the Spanish {\it Libro de Buen Amor} of \href{https://en.wikipedia.org/wiki/Juan_Ruiz}{\sc Juan Ruiz} in the 1t half of the 14th century. Walter has also the source for several manuscript collections of Aesop's fables in Italian \& equally of the popular {\it Esopi fabulas} by Accio Zucco da Sommacampagna, the 1st printed collection of Aesop's fables in that language (Verona, 1479), in which the story of the town mouse \& the country mouse appears as fable 12. This consists of 2 sonnets, the 1st of which tells the story \& the 2nd contains a moral reflection.

\subsubsection{British variations}

\subsubsection{Eastern analogies}

\subsubsection{Later adaptations}
[$\ldots$]'' -- \href{https://en.wikipedia.org/wiki/The_Town_Mouse_and_the_Country_Mouse}{Wikipedia{\tt/}Town Mouse \& Country Mouse}

%------------------------------------------------------------------------------%

\section{{\sc Brian W. Kernighan, P. J. Plauger}. The Elements of Programming Style}

\subsection{Wikipedia{\tt/}The Elements of Programming Style}
``{\it The Elements of Programming Style}, by \href{https://en.wikipedia.org/wiki/Brian_W._Kernighan}{Brian W. Kernighan} \& \href{https://en.wikipedia.org/wiki/P._J._Plauger}{P. J. Plauger}, is a study of \href{https://en.wikipedia.org/wiki/Programming_style}{programming style}, advocating the notion that computer programs should be written not only to satisfy the compiler or personal programming ``style'', but also for ``readability'' by humans, specially \href{https://en.wikipedia.org/wiki/Software_maintenance}{software maintenance} engineers, \href{https://en.wikipedia.org/wiki/Programmers}{programmers}, \& \href{https://en.wikipedia.org/wiki/Technical_writers}{technical writers}. It was originally published in 1974.

The book pays explicit homage\footnote{{\it homage (to somebody{\tt/}something)} something that is said or done to show respect for somebody.}, in title \& tone, to \href{https://en.wikipedia.org/wiki/The_Elements_of_Style}{The Elements of Style}, by Strunk \& White \& is considered a practical template promoting \href{https://en.wikipedia.org/wiki/Edsger_Dijkstra}{Edsger Dijkstra's structured programming} discussions. It has been influential \& has spawned a series of similar texts tailored to individual languages, such as {\it The Elements of C Programming Style, The Elements of C\# Style, The Elements of Java(TM) Style, The Elements of MATLAB Style}, etc.

The book is built on short examples from actual, published programs in programming textbooks. This results in a practical treatment rather than an abstract or academic discussion. The style is diplomatic \& generally sympathetic in its criticism, \& unabashedly honest as well -- some of the examples with which it finds fault are from the authors's own work (1 example in the 2nd edition is from the 1st edition).'' -- \href{https://en.wikipedia.org/wiki/The_Elements_of_Programming_Style}{Wikipedia{\tt/}The Elements of Programming Style}

\subsubsection{Lessons}
``Its lessons are summarized at the end of each section in \href{https://en.wikipedia.org/wiki/Aphorism}{pithy maxims}, such as ``Let the machine do the dirty work'':
\begin{enumerate}
	\item Write clearly -- don't be too clever.
	\item Say what you mean, simply \& directly.
	\item Use library functions whenever feasible.
	\item Avoid too many temporary variables.
	\item Write clearly -- don't sacrifice clarity for efficiency.
	\item Let the machine do the dirty work.
	\item Replace repetitive expressions by calls to common functions.
	\item Parenthesize to avoid ambiguity.
	\item Choose variable names that don't be confused.
	\item Avoid unnecessary branches.
	\item If a logical expression is hard to understand, try transforming it.
	\item Choose a data representation that makes the program simple.
	\item Write 1st in easy-to-understand pseudo language; then translate into whatever language you have to use.
	\item Modularize. Use procedures \& functions.
	\item Avoid gotos completely if you can keep the program readable.
	\item Don't patch bad code -- rewrite t.
	\item Write \& test a big program in small pieces.
	\item Use recursive procedures for recursively-defined data structures.
	\item Test input for plausibility \& validity.
	\item Make sure input doesn't violate the limits of the program.
	\item Terminate input by end-of-file marker, not by count.
	\item Identify bad input; recover if possible.
	\item Make input easy to prepare \& output self-explanatory.
	\item Use uniform input formats.
	\item Make input easy to proofread.
	\item Use self-identifying input. Allow defaults. Echo both on output.
	\item Make sure all variables are initialized before use.
	\item Don't stop at 1 bug.
	\item Use debugging compilers.
	\item Watch out for off-by-1 errors.
	\item Take care to branch the right way on equality.
	\item Be careful if a loop exits to the same place from the middle \& the bottom.
	\item Make sure you code does ``nothing'' gracefully\footnote{1. in an attractive way that shows control; showing a smooth, attractive form; 2. in a polite \& kind way, especially in a difficult situation.}.
	\item Test programs at their boundary values.
	\item Check some answers by hand.
	\item {\tt10.0} times {\tt0.1} is hardly ever {\tt1.0}.
	\item {\tt7/8} is zero while {\tt7.0/8.0} is not zero.
	\item Don't compare floating point numbers solely for equality.
	\item Make it right before you make it faster.
	\item Make it fail-safe before you make it faster.
	\item Make it clear before you make it faster.
	\item Don't sacrifice clarity for small gains in efficiency.
	\item Let your compiler do the simple optimizations.
	\item Don't strain to reuse code; reorganize instead.
	\item Make sure special cases are truly special.
	\item Keep it simple to make it faster.
	\item Don't diddle code to make it faster -- find a better algorithm.
	\item Instrument your programs. Measure before making efficiency changes.
	\item Make sure comments \& code agree.
	\item Don't just echo the code with comments -- make every comment count.
	\item Don't comment bad code -- rewrite it.
	\item Use variable names that mean something.
	\item Use statement labels that mean something.
	\item Format a program to help the reader understand it.
	\item Document your data layouts.
	\item Don't over-comment.
\end{enumerate}
Modern readers may find it a shortcoming that its examples use older \href{https://en.wikipedia.org/wiki/Procedural_programming_languages}{procedural programming languages} (\href{https://en.wikipedia.org/wiki/Fortran}{Fortran} \& \href{https://en.wikipedia.org/wiki/PL/I}{PL{\tt/}I}) that are quite different from those popular today. Few of today's popular languages had been invented when this book was written. However, many of the book's points that generally concern stylistic \& structural issues transcend \& details of particular languages.

\href{https://en.wikipedia.org/wiki/Kilobaud_Microcomputing}{Kilobaud Microcomputing} stated that ``If you intend to write programs to be used by other people, then you should read this book. If you expect to become a professional programmer, this book is mandatory reading.''

\subsection{Software Quotes{\tt/}P. J. Plauger}

\begin{enumerate}
	\item Make your programs read from top to bottom.
	\item Let the machine do the dirty work.
	\item Where there are 2 bugs, there is likely to be a 3rd.
	\item Choose a data representation that makes the program simple.
	\item Take care to branch the right way on equality.
	\item Let the data structure the program.
	\item Test input for validity \& plausibility.
	\item Make sure your code `does nothing' gracefully.
	\item Don't patch bad code -- rewrite it.
	\item His major concern is: ``The principle of 1 Right Place -- there should be 1 Right Place to look for any nontrivial piece of code, \& 1 Right Place to make a likely maintenance change.''
	\item Don't stop with your 1st draft.
	\item The more dogmatic\footnote{being certain that your beliefs are right \& that others should accept them, without paying attention to evidence or other opinions.} you are about applying a design method, the fever real-life problems you are going to solve.
	\item Make it right before you make it fast. Make it clear before you make it fast. Keep it right when you make it faster.
	\item People who preach software design as a disciplined activity spend considerable energy making us all feel guilty. We can never be structured enough or object-oriented enough to achieve nirvana in this lifetime. We all truck around a kind of original sin from having learned Basic at an impressionable age. But my bet is that most of us are better designers than the purists will ever acknowledge.
	
	-- Những người rao giảng thiết kế phần mềm như một hoạt động có kỷ luật đã tiêu tốn rất nhiều công sức khiến tất cả chúng ta đều cảm thấy tội lỗi. Chúng ta không bao giờ có thể có đủ cấu trúc hoặc đủ hướng đến đối tượng để đạt được niết bàn trong cuộc đời này. Tất cả chúng ta đều mắc phải một loại tội lỗi nguyên thủy do đã học ngôn ngữ Basic (Cơ bản) ở độ tuổi dễ bị ảnh hưởng. Nhưng tôi cá rằng hầu hết chúng ta đều là những nhà thiết kế giỏi hơn những gì những người theo chủ nghĩa thuần túy thừa nhận.
\end{enumerate}

%------------------------------------------------------------------------------%

\section{{\sc William Strunk Jr., E. B. White}. The Elements of Style}
\textbf{\textsf{Resources -- Tài nguyên.}}
\begin{enumerate}
	\item \cite{Strunk_element_style}. {\sc William Strunk Jr.} {\it The Elements of Style}.
	\item \cite{Strunk_White_element_style}. {\sc William Strunk Jr., E. B. White}. {\it The Elements of Style}.
\end{enumerate}

\subsection{Wikipedia{\tt/}The Elements of Style}
``{\it The Elements of Style} (also called {\it Strunk \& White}) is a \href{https://en.wikipedia.org/wiki/Style_guide}{style guide} for formal grammar used in \href{https://en.wikipedia.org/wiki/American_English}{American English} writing. The 1st publishing was written by \href{https://en.wikipedia.org/wiki/William_Strunk_Jr.} n 1918, \& published by \href{https://en.wikipedia.org/wiki/Harcourt_(publisher)}{Harcourt} in 1920, comprising 8 ``elementary rules of usage,'' 10 ``elementary principles of composition,'' ``a few matters of form,'' a list of 49 ``words \& expressions commonly misused,'' \& a list of 57 ``words often misspelled.'' Writer \& editor \href{https://en.wikipedia.org/wiki/E._B._White}{E. B. White} greatly enlarged \& revised the book for publication by \href{https://en.wikipedia.org/wiki/Macmillan_Publishers}{Macmillan} in 1959. That was the 1st edition of the book, which \href{https://en.wikipedia.org/wiki/Time_(magazine)}{Time} recognized in 2011 as 1 of the 100 best \& most influenced non-fiction books written in English since 1923. American wit Dorothy Parker said, regarding the book:
\begin{quote}
	``If you have any young friends who aspire to become writers, the 2nd-greatest favor you can do them is to present them with copies of {\it The Elements of Style}. The 1st greatest, of course, is to shoot them now, while they're happy.'' -- \href{https://en.wikipedia.org/wiki/The_Elements_of_Style}{Wikipedia{\tt/}Elements of Style}
\end{quote}

\subsubsection{Content}
See \href{https://en.wikipedia.org/wiki/The_Elements_of_Style}{Wikipedia{\tt/}The Elements of Style}. ``Strunk concentrated on the cultivation of good writing \& composition; the original 1918 edition exhorted writers to ``omit needless words'', use the \href{https://en.wikipedia.org/wiki/Active_voice}{active voice}, \& employ \href{https://en.wikipedia.org/wiki/Parallelism_(grammar)}{parallelism} appropriately.'' [$\ldots$] ``The 3rd edition of {\it The Elements of Style} (1979) features 54 points: a list of common word-usage errors; 11 rules of punctuation \& grammar; 11 principles of writing; 11 matters of form; \&, in Chap. V, 21 reminders for better style. The final reminder, the 21st, ``Prefer the standard to the offbeat\footnote{{\bf offbeat} [a] [usually before noun] ({\it informal}) different from what most people expect, {\sc synonym}: {\bf unconventional}.}'', is thematically integral\footnote{{\bf integral} [a] {\bf 1.} being an essential part of something; {\bf 2.} [usually before noun] included as part of something, rather than supplied separately; {\bf 3.} [usually before noun] having all the parts that are necessary for something to be complete.} to the subject of {\it The Elements of Style}, yet does stand as a discrete\footnote{{\bf discrete} [a] ({\it formal or specialist}) independent of other things of the same type, {\sc synonym}: {\bf separate}.} essay about writing lucid\footnote{{\bf lucid} [a] {\bf 1.} clearly expressed; easy to understand, {\sc synonym}: clear; {\bf 2.} able to think clearly, especially when somebody cannot usually do this.} prose\footnote{{\bf prose} [n] [uncountable] writing that is not poetry.}. To write well, White advises writers to have the proper\footnote{{\bf proper} [a] {\bf 1.} [only before noun] ({\it especially British English}) right, appropriate or correct; according to the rules, {\sc opposite}: {\bf improper}; {\bf 2.} [only before noun] {\it British English}) considered to be real \& of a good enough standard; {\bf 3.} socially \& morally acceptable, {\sc opposite}: {\bf improper}; {\bf 4.} [after noun] according to the most exact meaning of the word; {\bf 5. proper to somebody{\tt/}something} belonging to a particular type of person or thing; natural in a particular situation or place.} mind-set, that they write to please themselves, \& that they aim for ``1 moment of felicity\footnote{{\bf felicity} [n] {\bf 1.} [uncountable] great happiness; {\bf 2.} [uncountable] the quality of being well chosen or suitable; {\bf 3. felicities} [plural] well-chosen or successful features, especially in a speech or piece of writing.}'', a phrase by \href{https://en.wikipedia.org/wiki/Robert_Louis_Stevenson}{Robert Louis Stevenson}. Thus Strunk's 1918 recommendation:
\begin{quotation}
	``Vigorous\footnote{{\bf vigorous} [a] {\bf 1.} involving physical strength, effort or energy; {\bf 2.} done with determination, energy or enthusiasm; {\bf 3.} strong \& healthy.} writing is concise\footnote{{\bf concise} [a] giving only the information that is necessary \& important, using few words.}. A sentence should contain no unnecessary words, a paragraph no unnecessary sentences, for the same reason that a drawing should have no unnecessary lines \& a machine no unnecessary parts. This requires not that the writer make all his sentences short, or that he avoid all detail \& treat his subjects only in outline, but that he make every word tell.'' -- ``Elementary Principles of Composition'', {\it The Element of Style} \cite{Strunk_element_style}''
\end{quotation}
[$\ldots$] ``The 4th edition of {\it The Elements of Style} (2000), published 54 years after Strunk's death, omits his stylistic\footnote{{\bf stylistic} [a] [only before noun] connected with the style that a writer, artist or musician uses.} advice about masculine\footnote{{\bf masculine} [a] {\bf 1.} having the qualities or appearance considered to be typical of men; connected with or like men; {\bf 2.} (in some languages) belonging to a class of nouns, pronouns or adjectives that have masculine gender, not feminine or neuter.} pronouns: ``unless the antecedent\footnote{{\bf antecedent} [n] a thing or an event that exists or comes before something else \& has an influence on it; [a] existing or coming before something else, \& having an influence on it.} is or must be feminine''. In its place, the following sentence has been added: ``many writers find the use of the generic {\it he} or {\it his} to rename indefinite antecedents limiting or offensive.'' Further, the retitled entry ``They. He or she'', in Chap. IV: {\it Misused Words \& Expressions}, advises the writer to avoid an ``unintentional emphasis on the masculine''.'' -- \href{https://en.wikipedia.org/wiki/The_Elements_of_Style#Content}{Wikipedia{\tt/}The Element of Style{\tt/}content}

\subsubsection{Reception}
``{\it The Elements of Style} was listed as 1 of the 100 best \& most influential\footnote{{\bf influential} [a] having a lot of influence on the way that somebody{\tt/}something behaves or develops, or on the way that somebody thinks.} books written in English since 1923 by {\it Time} in its 2011 list. Upon its release, Charles Poor, writing for \href{https://en.wikipedia.org/wiki/The_New_York_Times}{{\it The New York Times}}, called it ``a splendid\footnote{{\bf splendid} [a] ({\it especially British English}) {\bf 1.} very impressive; very beautiful; {\bf 2.} ({\it old-fashioned}) excellent; very good, {\sc synonym}: great.} trophy for all who are interested in reading \& writing.'' American poet \href{https://en.wikipedia.org/wiki/Dorothy_Parker}{Dorothy Parker} has, regarding the book, said:
\begin{quotation}
	``If you have any young friends who aspire to become writers, the 2nd-greatest favor you can do them is to present them with copies of {\it The Elements of Style}. The 1st-greatest, of course, is to shoot them now, while they're happy.''
\end{quotation}
Criticism\footnote{{\bf criticism} [n] {\bf 1.} [uncountable, countable] the act of expressing disapproval of somebody{\tt/}something \& opinions about their faults or bad qualities; a statement showing disapproval; {\bf 2.} [uncountable] the work or activity of analyzing \& making fair, careful judgments about somebody{\tt/}something, especially books, music, etc.} of {\it Strunk \& White} has largely focused on claims that it has a \href{https://en.wikipedia.org/wiki/Linguistic_prescriptivism}{prescriptivist}\footnote{{\bf prescriptive} [a] {\bf 1.} telling people what should be done or how something should be done; {\bf 2.} ({\it linguistics}) telling people how a language should be used, rather than describing how it is used, {\sc opposite}: {\bf descriptive}.} nature, or that it has become a general \href{https://en.wikipedia.org/wiki/Anachronism}{anachronism}\footnote{{\bf anachronism} [n] {\bf 1.} [countable] a person, a custom or an idea that seems old-fashioned \& does not belong to the present; {\bf 2.} [countable, uncountable] something that is placed, e.g., in a book or play, in the wrong period of history; the fact of placing something in the wrong period of history.} in the face of modern English usage.

In criticizing {\it The Elements of Style}, \href{https://en.wikipedia.org/wiki/Geoffrey_Pullum}{Geoffrey Pullum}, professor of \href{https://en.wikipedia.org/wiki/Linguistics}{linguistics} at the \href{https://en.wikipedia.org/wiki/University_of_Edinburgh}{University of Edinburgh}, \& co-author of \href{https://en.wikipedia.org/wiki/The_Cambridge_Grammar_of_the_English_Language}{{\it The Cambridge Grammar of the English Language}} (2002), said that:
\begin{quotation}
	``The book's toxic mix of \href{https://en.wikipedia.org/wiki/Linguistic_purism}{purism}\footnote{{\bf purism} [n] [uncountable] the belief that things should be done in the traditional way \& that there are correct forms in languages, art, etc. that should be followed.}, \href{https://en.wikipedia.org/wiki/Atavism}{atavism}, \& personal \href{https://en.wikipedia.org/wiki/Eccentricity_(behavior)}{eccentricity}\footnote{{\bf eccentricity} [n] {\bf 1.} [uncountable] behavior that people think is strange or unusual; the quality of being unusual \& different from other people; {\bf 2.} [countable, usually plural] an unusual act or habit.} is not underpinned\footnote{{\bf underpin} [v] to support or form the basis of something.} by a proper grounding\footnote{{\bf grounding} [n] [singular, uncountable] knowledge \& understanding of the basic parts of a subject; a basis for something.} in English grammar. It is often so misguided that the authors appear not to notice their own egregious\footnote{{\bf egregious} [a] ({\it formal}) extremely bad.} flouting\footnote{{\bf flout} [v] {\bf flout something} to show that you have no respect for a law, etc. by openly not obeying it, {\sc synonym}: {\bf defy}.} of its own rules $\ldots$ It's sad. Several generations of college students learned their grammar from the uninformed\footnote{{\bf uninformed} [a] having or showing a lack of knowledge or information about something, {\sc opposite}: informed.} bossiness\footnote{{\bf bossiness} [n] [uncountable] ({\it disapproving}) bossy behavior.} of {\it Strunk \& White}, \& the result is a nation of educated people who know they feel vaguely\footnote{{\bf vaguely} [adv] {\bf 1.} in a way that is not detailed or exact; {\bf 2.} slightly.} anxious\footnote{{\bf anxious} [a] {\bf 1. anxious (about something)} feeling worried or nervous; {\bf 2.} wanting something very much.} \& insecure\footnote{{\bf insecure} [a] {\bf 1.} not confident, especially about yourself or your abilities, {\sc opposite}: {\bf secure}; {\bf 2.} not safe or protected, {\sc opposite}: {\bf secure}.} whenever they write {\it however} or {\it than me} or {\it was} or {\it which}, but can't tell you why.''
\end{quotation}
Pullum has argued, e.g., that the authors misunderstood what constitutes the \href{https://en.wikipedia.org/wiki/English_passive_voice}{passive voice}\footnote{NQBH: Personally, I prefer the passive voice to the active one.}, \& he criticized their proscription\footnote{{\bf proscription} [n] [countable, uncountable] ({\it formal}) {\bf proscription (against{\tt/}on something)} the act of saying officially that something is banned; the stat of being banned.} of established \& unproblematic\footnote{{\bf unproblematic} [a] not having or causing problems, {\sc opposite}: {\bf problematic}.} English usages, e.g. the \href{https://en.wikipedia.org/wiki/Split_infinitive}{split infinitive} \& the use of {\it which} in a restrictive \href{https://en.wikipedia.org/wiki/English_relative_clause#That_or_which}{relative clause}. On \href{https://en.wikipedia.org/wiki/Language_Log}{Language Log}, a blog about language written by \href{https://en.wikipedia.org/wiki/Linguists}{linguists}, he further criticized {\it The Elements of Style} for promoting \href{https://en.wikipedia.org/wiki/Linguistic_prescriptivism}{linguistic precriptivism} \& \href{https://en.wikipedia.org/wiki/Hypercorrection}{hypercorrection} among \href{https://en.wikipedia.org/wiki/Anglophones}{Anglophones}, \& called it ``the book that ate American's brain''.

\href{https://en.wikipedia.org/wiki/The_Boston_Globe}{{\it The Boston Globe}}'s review described {\it The Elements of Style Illustrated} (2005), with illustrations by Maira Kalman, as an ``aging zombie of a book $\ldots$ a hodgepodge\footnote{{\bf hodgepodge} [n] ({\it North American English}) (also {\bf hotchpotch}, {\it especially in British English}) [singular] ({\it informal}) a number of things mixed together without any particular order or reason.}, its now-antiquated\footnote{{\bf antiquated} [a] ({\it usually disapproving}) (of things or ideas) old-fashioned \& no longer suitable for modern conditions, {\sc synonym}: {\bf outdated}.} \href{https://en.wikipedia.org/wiki/Pet_peeve}{pet peeves} jostling for\footnote{{\bf jostle for} [phrasal verb] {\bf jostle for something} to compete strongly \& with force for something.} space with 1970s taboos\footnote{{\bf taboo} [n] {\bf 1. taboo (against{\tt/}on something)} a cultural or religious custom that does not allow people to do, use or talk about a particular thing; {\bf 2. taboo (against{\tt/}on something)} a general agreement not to do something or talk about something.} \& 1990s computer advice''.

Nevertheless, many contemporary\footnote{{\bf contemporary} [a] {\bf 1.} belonging to the present time, {\sc synonym} {\bf modern}; {\bf 2.} (especially of people \& society) belonging to the same time as somebody{\tt/}something else.} authors still recommend it highly. Their praise\footnote{{\bf praise} [v] {\bf 1.} to express your approval or admiration for somebody{\tt/}something; {\bf 2. praise God} to express your thanks to or your respect for God.} tends to focus on its characterization\footnote{{\bf characterization} [n] [uncountable, countable] {\bf 1. characterization (of something)} the process of discovering or describing the qualities or features of something; the result of this process; {\bf 2.} the way in which the characters in a story, play or film are made to seem real.} of \fbox{good writing \& how to achieve it}, grammar being just 1 element of that purpose. In \href{https://en.wikipedia.org/wiki/On_Writing:_A_Memoir_of_the_Craft}{On writing} (2000, p. 11), \href{https://en.wikipedia.org/wiki/Stephen_King}{Stephen King} writes:
\begin{quotation}
	``There is little or no detectable \href{https://en.wikipedia.org/wiki/Bullshit}{bullshit} in that book. (Of course, it's short; at 85 pages it's much shorter than this one.) I'll tell you right now that every aspiring writer should read {\it The Elements of Style}. Rule 17 in the chapter titled {\it Principles of Composition} is `Omit needless words.' I will try to do that here.''
\end{quotation}
In 2011, Tim Skern remarked that {\it The Elements of Style} ``remains the best book available on writing good English.''

In 2013, \href{https://en.wikipedia.org/wiki/Nevile_Gwynne}{Nevile Gwynne} reproduced {\it The Elements of Style} in his work \href{https://en.wikipedia.org/wiki/Gwynne%27s_Grammar}{{\it Gwynne's Grammar}}. Britt Peterson of the \href{https://en.wikipedia.org/wiki/Boston_Globe}{{\it Boston Globe}} wrote that his inclusion of the book was a ``curious\footnote{{\bf curious} [a] {\bf 1.} having a strong desire to know about something; {\bf 2.} strange \& unusual.} addition''.

In 2016, the Open Syllabus Project lists {\it The Elements of Style} as the most frequently assigned text in US academic \href{https://en.wikipedia.org/wiki/Syllabus}{syllabuses}, based on an analysis of 933,635 texts appearing in over 1 million syllabuses.'' -- \href{https://en.wikipedia.org/wiki/The_Elements_of_Style#Reception}{Wikipedia{\tt/}The Elements of Style{\tt/}reception}

``The 1st writer I watched at work was my stepfather, E. B. White.\footnote{Sự ảnh hưởng, đặc biệt đến nhân cách \& việc lựa chọn nghề nghiệp, của những hình mẫu đầu tiên mà ta, 1 cách tình cờ hay được số phận sắp đặt, gặp gỡ trong cuộc đời.} Each Tuesday morning, he would close his study door \& sit down to write the ``Notes \& Comment'' page for {\it The New Yorker}. The task was familiar to him -- he was required to file a few hundred words of editorial\footnote{{\bf editorial} [a] [usually before noun] connected with the task of preparing something e.g. a newspaper, a book, or a television or radio programme, to be published or broadcast; [n] an important article in a journal or a newspaper, that expresses the editor's opinion about an issue.} of personal commentary on some topic in or out of the news that week -- but the sounds of his typewriter\footnote{{\bf typewriter} [n] a machine that produces writing similar to print. It has keys that you press to make metal letters or signs hit a piece of paper through a long, narrow piece of cloth covered with ink ($=$ colored liquid).} \footnote{NQBH: I like the term ``typewriter'' in any literary scene., which sounds traditional \& sexy, opposite to personal notebooks{\tt/}laptop now: modern \& robust.} from his room came in hesitant\footnote{{\bf hesitant} [a] slow to speak or act because you feel uncertain, embarrassed or unwilling.} bursts\footnote{{\bf burst} [v] {\bf 1.} [intransitive, transitive] to break open or apart, especially because of pressure from inside; to make something break in this way; {\bf 2.} [intransitive] {\bf $+$ adv.{\tt/}prep.} to go or come from somewhere suddenly; {\bf burst into something} [phrasal verb] to start producing something suddenly \& with great force; [n] a short period of a particular activity or strong emotion that often starts suddenly.}, with long silences in between. Hours went by. Summoned at last for lunch, he was silent \& preoccupied\footnote{{\bf preoccupied} [a] thinking \&{\tt/}or worrying continuously about something so that you do not pay attention to other things.}, \& soon excused himself to get back to the job. When the copy went off at last, in the afternoon RFD pouch\footnote{{\bf pouch} [n] {\bf 1.} a small bag, usually made of leather, \& often carried in a pocket or attached to a belt; {\bf 2.} a large bag for carrying letters, especially official ones; {\bf 3.} a pocket of skin on the stomach of some female marsupial animals, e.g. kangaroos, in which they carry their young; {\bf 4.} a pocket of skin in the cheeks of some animals, e.g. hamsters, in which they store food.} -- we were in Maine, a day's mail away from New York -- he rarely seemed satisfied. \fbox{``It isn't good enough.''}\footnote{``The quest for perfection can never end.''} he said sometimes, \fbox{``I wish it were better.''}

\fbox{Writing is hard}, even for authors who do it all the time. Less frequent practitioners -- the job applicant; the business executive with an annual report to get out; the high school senior with a Faulkner assignment; the graduate-school student with her thesis proposal; the writer of a letter of condolence\footnote{{\bf condolence} [n] [countable, usually plural, uncountable] sympathy that you feel for somebody when a person in their family or that they know well has died; an expression of this sympathy.} -- often get stuck in an awkward\footnote{{\bf awkward} [a] {\bf 1.} embarrassed; making you feel embarrassed; {\bf 2.} difficult to deal with, {\sc synonym}: {\bf difficult}; {\bf 3.} not convenient; {\bf 4.} difficult because of its shape or design; {\bf 5.} not moving in an easy way; not comfortable or elegant.} passage or find a muddle\footnote{{\bf muddle} [v] ({\it especially British English}) {\bf 1.} to put things in the wrong order or mix them up; {\bf 2.} muddle somebody (up) to confuse somebody; {\bf 3.} muddle somebody{\tt/}something (up)$|$ {\bf muddle A (up) with B} to confuse 1 person or thing with another, {\sc synonym}: {\bf mix up}.} on their screens, \& then blame themselves. What should be easy \& flowing looks tangled\footnote{{\bf tangled} [a] {\bf 1.} twisted together in an untidy way; {\bf 2.} complicated, \& not easy to understand.} or feeble\footnote{{\bf feeble} [a] {\bf 1.} very weak; {\bf 2.} not effective; not showing energy or effort.} or overblown\footnote{{\bf overblown} [a] {\bf 1.} that is made to seem larger, more impressive or more important than it really is, {\sc synonym}: {\bf exaggerated}; {\bf 2.} (of flowers) past the best, most beautiful stage.} -- not what was meant at all. \fbox{What's wrong with me}, each one thinks. \fbox{Why can't I get this right?}''

[$\ldots$] White knew that a compendium\footnote{{\bf compendium} [n] (plural {\bf compendia, compendiums}) a collection of facts, drawings \& photographs on a particular subject, especially in a book.} of specific tips -- about singular \& plural verbs, parentheses, the ``that'' -- ``which'' scuffle\footnote{{\bf scuffle} [n] {\bf scuffle (with somebody) $|$ scuffle (between A \& B)} a short \& not very violent fight or struggle; [v] {\bf 1.} [intransitive] {\bf scuffle (with somebody)} (of 2 or more people) to fight or struggle with each other for a short time, in a way that is not very serious; {\bf 2.} [intransitive] {\bf $+$ adv.{\tt/}prep.} to move quickly making a quiet rubbing noise.}, \& many others -- could clear up a recalcitrant\footnote{{\bf recalcitrant} [a] ({\it formal}) unwilling to obey rules or follow instructions; difficult to control.} sentence or subclause when quickly reconsulted\footnote{{\bf consult} [v] {\bf 1.} [transitive, intransitive] to discuss something with somebody to get their permission for something, or to help you make a decision; {\bf 2.} [transitive, intransitive] to go to somebody for information or advice, especially an expert e.g. a doctor or lawyer; {\bf 3.} [transitive] {\bf consult something} to look in or at something to get information, {\sc synonym}: {\bf refer to something}.}, \& that the larger principles needed to be kept in plain sight, like a wall sampler.

How simple they look, set down here in White's last chapter: ``\fbox{Write in a way that comes naturally},'' ``\fbox{Revise \& rewrite},'' ``\fbox{Do not explain too much},'' \& the rest; above all, the cleansing\footnote{{\bf cleanse} [v] {\bf 1.} [transitive, intransitive] {\bf cleanse (something)} to clean your skin or a wound; {\bf 2.} [transitive] {\bf cleanse somebody (of{\tt/}from something}) ({\it literary}) to take away somebody's guilty feelings or sin.}, clarion\footnote{{\bf clarion} [n] {\bf 1.} a medieval trumpet with clear shrill tones; {\bf 2.} the sound of or as if of a clarion' [a] brilliantly clear; loud \& clear.} ``Be clear.'' How often I have turned to them, in the book or in my mind, while trying to start or unblock or revise some piece of my own writing! They help -- they really do. They work. They are the way.

E. B. White's prose is celebrated for its ease\footnote{{\bf ease} [n] [uncountable] {\bf 1.} lack of difficulty or effort, {\sc opposite}: {\bf difficulty}; {\bf 2.} the state of feeling relaxed or comfortable, without anxiety, problems or pain.} \& clarity\footnote{{\bf clarity} [n] [uncountable] {\bf 1.} the quality of being expressed clearly; {\bf 2.} the ability to think about or understand something clearly; {\bf 3.} if a picture, substance or sound has clarity, you can see or hear it very clearly, or see through it easily.} -- just think of {\it Charlotte's Web} -- but maintaining this standard required endless attention. When the new issue of {\it The New Yorker} turned up in Maine, I sometimes saw him reading his ``Comment'' piece over to himself, with only a slightly different expression than the one he'd worn on the day it went off. Well, O.K., he seemed to be saying. \fbox{At least I got the elements right.}

This edition has been modestly\footnote{{\bf modest} [a] {\bf 1.} fairly limited or small in amout; {\bf 2.} not expensive, rich or impressive; {\bf 3.} (of people, especially women, or their clothes) not showing too much of the body; not intended to attract attention, especially in a sexual way; {\bf 4.} ({\it approving}) not talking much about your own abilities or possessions.} updated, with word processors \& air conditioners making their 1st appearance among White's references, \& with a light redistribution of genders to permit a feminine pronoun or female farmer to take their places among the males who once innocently\footnote{{\bf innocent} [a] {\bf 1.} not guilty of a crime, etc.; not having done something wrong, {\sc opposite}: {\bf guilty}; {\bf 2.} [only before noun] suffering harm or being killed because of a crime, war, etc. although not directly involved in it; {\bf 3.} having little experience of evil or unpleasant things, or of sexual matters; {\bf 4.} not intended to cause harm or upset somebody, {\sc synonym}: {\bf harmless}.} served him.'' [$\ldots$] ``What is not here is anything about E-mail -- the rules-free, lower-case flow that cheerfully keeps us in touch these days. E-mail is conversation, \& it may be replacing the sweet \& endless talking we once sustained\footnote{{\bf sustain} [v] {\bf 1. sustain somebody{\tt/}something} to provide enough of what somebody{\tt/}something needs in order to live or exist; {\bf 2.} to make something continue for some time without becoming less, {\sc synonym}: {\bf maintain}; {\bf 3. sustain something} ({\it formal}) to experience something bad, {\sc synonym}: {\bf suffer}; {\bf 4. sustain something} to provide evidence to support an opinion, a theory, etc., {\sc synonym}: {\bf uphold}; {\bf 5. sustain something} ({\it law}) to decide that a claim, etc. is valid, {\sc synonym}: {\bf uphold}.} (\& tucked away\footnote{{\bf tuck away} [phrasal verb] {\bf tuck something $\leftrightarrow$ away} {\bf 1. be tucked away} to be located in a quiet place, where not many people go; {\bf 2.} to hide something somewhere or keep it in a safe place; {\bf 3.} ({\it British English, informal}) to eat a lot of food.}) within the informal letter. But we are all writers \& readers as well as communicators, with \fbox{the need at times to please \& satisfy ourselves} (as White put it) with the \fbox{clear \& almost perfect thought}.'' -- \cite[{\it Foreword} by Roger Angell]{Strunk_White_element_style}

``I [E. B. White] passed the course, graduated from the university, \& \fbox{forgot the book but not the professor}.'' [$\ldots$]

``{\it The Elements of Style}, when I [E. B. White] reexamined it in 1957, seemed to me to contain \fbox{rich deposits\footnote{{\bf deposit} [n] {\bf 1.} a layer of a substance that has been left somewhere, especially by a river or flood, or is found at the bottom of a liquid; {\bf 2.} a layer of a substance that has formed naturally underground; {\bf 3.} [usually singular] {\bf a deposit (on something)} a sum of money that is given as the 1st part of a larger payment; {\bf 4.} (in the British political system) the amount of money that a candidate in an election to Parliament has to pay, \& that is returned if they get enough votes.} of gold}. It was Will Strunk's {\it parvum opus}\footnote{{\bf parvum opus} [from Latin] [n] a little work, a small but meaningful work of an artist or writer.}, his attempt to cut the vast tangle\footnote{{\bf tangle} [n] {\bf 1.} a twisted mass of threads, hair, etc. that cannot be easily separated; {\bf 2.} a lack of order; a confused state; {\bf 3.} ({\it informal}) a disagreement or fight; [v] [transitive, intransitive] {\bf tangle (something) up} to twist something into an untidy mass; to become twisted in this way.} of English rhetoric\footnote{{\bf rhetoric} [n] [uncountable] {\bf 1.} ({\it often disapproving} speech or writing that is intended to influence people, but that is not completely honest or sincere; {\bf 2.} the skill of using language in speech or writing in a special way that influences or entertains people.)} down to size \& write its rules \& principles on the head of a pin\footnote{{\bf pin} [n] {\bf 1.} a short thin piece of stiff wire with a sharp point at 1 end \& a round head at the other, used to hold or attach things; {\bf 2.} a short piece of metal or other material, used to hold things together; {\bf 3.} a piece of metal with a sharp point, worn for decoration; {\bf 4.} 1 of the metal parts that stick out of an electric plug \& fit into a socket; [v] {\bf pin something ($+$ adv.{\tt/}prep.)} to attach something onto another thing or join things together with a pin, etc.; {\bf pin something down} [phrasal verb] to explain or understand something exactly.}. Will himself had hung the tag ``little'' on the book; he referred to it sardonically\footnote{{\bf sardonically} [adv] ({\it disapproving}) in a way that shows that you think that you are better than other people \& do not take them seriously, {\sc synonym}: {\bf mockingly}.} \& with secret pride as ``the {\it little book},'' always giving the word ``little'' a special twist, as though he were putting a spin on a ball. In its original form, it was a 43 page summation of the case for cleanliness, accuracy\footnote{{\bf accuracy} [n] {\bf 1.} [uncountable] the state of being exact or correct, {\sc opposite}: {\bf inaccuracy}; {\bf 2.} [uncountable, countable] ({\it specialist}) the degree to which the result of a measurement or calculation matches the correct value or a standard, {\sc opposite}: {\bf inaccuracy}.}, \& brevity\footnote{{\bf brevity} [n] [uncountable] {\bf 1.} the quality of using few words when speaking or writing; {\bf 2. brevity (of something)} the fact of lasting a short time.} in the use of English. Today, 52 years later, its vigor\footnote{{\bf vigor} [n] [uncountable] {\bf 1.} effort, energy, \& enthusiasm; {\bf 2. vigor (of something)} physical strength; good health.} is unimpaired\footnote{{\bf unimpaired} [a] ({\it formal}) not damaged or made less good, {\sc opposite}: {\bf impaired}.}, \& for sheer\footnote{{\bf sheer} [a] {\bf 1.} [only before noun] used to emphasize the size, degree or amount of something; nothing but; {\bf 2.} very steep.} pith\footnote{{\bf pith} [n] [uncountable] {\bf 1.} a soft dry white substance inside the skin of oranges \& some other fruits; {\bf 2.} the essential or most important part of something.} I think it probably sets a record that is not likely to be broken. Even after I got through tampering with\footnote{{\bf tamper with} [phrasal verb] {\bf tamper with something} to make changes to something without permission, especially in order to damage it, {\sc synonym}: interfere with.} it, it was still a tiny thing, \fbox{a barely tarnished\footnote{{\bf tarnished} [v] {\bf 1.} [intransitive, transitive] if mental tarnishes or something tarnishes it, it no longer looks bright \& shiny; {\bf 2.} [transitive, often passive] to damage the good opinion people have of somebody{\tt/}something, {\sc synonym}: {\bf taint}; [n] [singular, uncountable] a thin layer on the surface of a metal that makes it look darker \& less bright.} gem\footnote{{\bf gem} [n] {\bf 1.} (also less frequent {\bf gemstone}) a precious stone that has been cut \& polished \& is used in jewellery, {\sc synonym}: {\bf jewel, precious stone}; {\bf 2.} a person, place or thing that is especially good.}}. 7 rules of usage, 11 principles of composition\footnote{{\bf composition} [n] {\bf 1.} [uncountable] the different parts that something is made of; the way in which the different parts are organized; {\bf 2.} [countable] a piece of music or a poem; {\bf 3.} [uncountable] the act of writing a piece of music or a poem; {\bf 4.} [uncountable] ({\it art}) the arrangement of people of objects in a painting, photograph or scene of a film.}, a few matters of form, \& a list of words \& expressions commonly misused -- that was the sum \& substance\footnote{{\bf substance} [n] {\bf 1.} a type of solid, liquid or gas that has particular qualities; {\bf 2.} [countable] a drug or chemical, especially an illegal one, that has a particular effect on the mind or body; {\bf 3.} [uncountable] the most important or main part of something; {\bf 4.} [uncountable] ({\it formal}) importance; {\bf 5.} [uncountable] the quality of being based on facts or the truth.} of Prof. Strunk's work. Somewhat audaciously\footnote{{\bf audaciously} [adv] ({\it formal}) in a way that shows you are willing to take risks or to do something that shocks people.}, \& in an attempt to give my publisher his money's worth, I [E. B. White] added a chapter called ``An Approach to Style,'' setting forth my own prejudices\footnote{{\bf prejudice} [n] [uncountable, countable] an unreasonable dislike of a person, group, etc., especially when it is based on their race, religion, sex, etc.}, my notions of error, my articles of faith. This chapter (Chap. V) is addressed particularly to those who feel that English prose composition is not only a necessary skill but a sensible pursuit as well -- a way to spend one's days. I think Prof. Strunk would not object to that.''

[$\ldots$] ``I have now completed a 3rd revision. Chap. IV has been refurbished\footnote{{\bf refurbish} [v] {\bf refurbish something} to clean \& decorate a room, building, etc. in order to make it more attractive, more useful, etc.} with words \& expressions of a recent vintage\footnote{{\bf vintage} [n] {\bf 1.} the wine that was produced in a particular year or place; the year in which it was produced; {\bf 2.} [usually singular] the period or season of gathering grapes for making wine; [a] [only before noun] {\bf 1. vintage} wine is of very good quality \& has been stored for several years; {\bf 2.} (British English) (of a vehicle) made between 1919 \& 1930 \& admired for its style \& interest; {\bf 3.} typical of a period in the past \& of high quality; the best work of the particular person; {\bf 4. vintage year} a particular good \& successful year.}; 4 rules of usage have been added to Chap. I. Fresh examples have been added to some of the rules \& principles, amplification\footnote{{\bf amplification} [n] [uncountable] {\bf 1. amplification (of something)} the process of increasing the amplitude of an electrical signal; {\bf 2.} (biochemistry) {\bf amplification (of something)} the process by which many copies of something, e.g. a gene, are made; {\bf 3. amplification (of something)} the action of making something greater or easier to notice; {\bf 4.} the action of adding details to a story, statement, etc.; details added to a story, statement, etc.} has reared\footnote{{\bf rear} [v] {\bf 1. rear somebody{\tt/}something} [often passive] to care for young children or animals until they are fully grown, {\sc synonym}: {\bf raise}; {\bf 2. rear something} to breed or keep animals or birds, e.g. on a farm; {\bf something rears its head} [idiom] (of something unpleasant) to appear or happen; [n] (usually {\bf the rear}) [singular] the back part of something; [a] [only before noun] at or near the back of something.} its head in a few places in the text where I felt an assault\footnote{{\bf assault} [n] {\bf 1.} [uncountable, countable] the crime of attacking somebody physically; in law, {\bf assault} is an act that threatens physical harm to somebody, whether or not actual harm is done: {\it to commit}{\tt/}{\it be charged with assault}; {\bf 2.} [countable] (by an army, etc.) the act of attacking somebody{\tt/}something, {\sc synonym}: {\bf attack}; {\bf 3.} [countable, usually singular, uncountable] an act of criticizing or attacking somebody{\tt/}something severely; [v] {\bf assault somebody} to attack somebody physically.} could successfully be made on the bastions\footnote{{\bf bastion} [n] {\bf 1.} ({\it formal}) a group of people or a system that protects a way of life or a belief when it seems that it may disappear; {\bf 2.} a place that military forces are defending.} of its brevity, \& in general the book has received a thorough overhaul\footnote{{\bf overhaul} [n] an examination of a machine or system, including doing repairs on it or making changes to it; [v] {\bf 1. overhaul something} to examine every part of a machine, system, etc. \& make any necessary changes or repairs; {\bf 2. overhaul somebody} to come from behind a person you are competing against in a race \& go past them, {\sc synonym}: {\bf overtake}.} -- to correct errors, delete bewhiskered\footnote{{\bf bewhiskered} [a] {\bf 1.} having whiskers; bearded; {\bf 2.} ancient, as a witticism, expression, etc.; pass\'e; hoary.} entries, \& enliven\footnote{{\bf enliven} [v] ({\it formal}) {\bf enliven something} to make something more interesting or more fun.} the argument.

Prof. Strunk was a positive man. His book contains rules of grammar phrased as direct orders. In the main I [E. B. White] have not tried to soften his commands, or modify his pronouncements\footnote{{\bf pronouncement} [n] a formal public statement.}, or remove the special objects of his scorn\footnote{{\bf scorn} [n] [uncountable] a strong feeling that somebody{\tt/}something is stupid or not good enough, usually shown by the way you speak, {\sc synonym}: {\bf contempt}; [v] {\bf 1. scorn somebody{\tt/}something} to feel or show that you think somebody{\tt/}something is stupid \& you do not respect them or it, {\sc synonym}: {\bf dismiss}; {\bf 2.} ({\it formal}) to refuse to have or do something because you are too proud.}. I have tried, instead, to preserve\footnote{{\bf preserve} [v] {\bf 1. preserve something} to keep a particular quality or feature; {\bf 2.} to keep something safe from harm, in good condition or in its original state; {\bf 3.} to prevent something from decaying, by treating it in a particular way; [n] [singular] an activity, job or interest that is thought to be suitable for 1 particular person or group of people.} the flavor\footnote{{\bf flavor} [n] {\bf 1.} [uncountable] {\bf flavor (of something)} how food or drink tastes, {\sc synonym}: {\bf taste}; {\bf 2.} [countable] a particular type of taste; {\bf 3.} [singular] a particular quality or atmosphere; {\bf 4.} [singular] {\bf a{\tt/}the flavor of something} an idea of what something is like.} of his discontent\footnote{{\bf discontent} [n] (also {\bf discontentment}) {\bf 1.} [uncountable] a feeling of being unhappy because you are not satisfied with a particular situation, {\sc synonym}: {\bf dissatisfaction}; {\bf 2.} [countable] {\bf discontent (of somebody)} a thing that makes you feel unhappy \& not satisfied with a particular situation, {\sc synonym}: {\bf dissatisfaction}.} while slightly enlarging the scope of the discussion. {\it The Elements of Style} does not pretend\footnote{{\bf pretend} [v] {\bf 1.} to behave in a particular way, in order to make other people believe something that is not true; {\bf 2.} (usually used in negative sentences \& questions) to claim to be, do or have something, especially when this is not true.} to survey\footnote{{\bf survey} [n] {\bf 1. survey} (of somebody{\tt/}something) an investigation of the opinions, behavior, etc. of a particular group of people, which is usually done by asking them questions; {\bf 2.} an act of examining \& recording the measurements, features, etc. of an area of land in order to make a map or plan of it; {\bf 3. survey (of something)} a general study, view or description of something; [v] {\bf 1. survey somebody{\tt/}something} to investigate the opinions or behavior of a group of people by asking them a series of questions; {\bf 2. survey something} to study \& give a general description of something; {\bf 3. survey something} to measure \& record the features of an area of land, e.g. in order to make a map or in preparation for building; {\bf 4. survey something} to look carefully at the whole of something, especially in order to get a general impression of it, {\sc synonym}: {\bf inspect}.} the whole field. Rather it proposes\footnote{{\bf propose} [v] {\bf 1.} to suggest a plan or an idea for people to consider \& decide on; {\bf 2.} to suggest an explanation of something for people to consider.} to give in brief space the principal\footnote{{\bf principal} [a] [only before noun] main; most important.} requirements of plain\footnote{{\bf plain} [a] {\bf 1.} easy to see or understand, {\sc synonym}: {\bf clear}; {\bf 2.} [only before noun] expressed in a clear \& simple way, without using technical language; {\bf 3.} not trying to deceive anyone; honest \& direct; {\bf 4.} not decorated or complicated; simple; in computing, {\bf plain text} is data representing text that is not written in code or using special formatting \& can be read, displayed or printed without much processing: {\it Mathematical formulae are an example of content that cannot be represented satisfactorily via plain text.}; {\bf 5.} without marks or a pattern on it; {\bf 6.} [only before noun] (used for emphasis) simple; nothing but. {\sc synonym}: {\bf sheer}.} English style. It concentrates\footnote{{\bf concentrate} [v] {\bf 1.} [transitive, often passive] {\bf concentrate something $+$ adv.{\tt/}prep.} to bring something together in 1 place; {\bf 2.} [intransitive, transitive] to give all your attention to something \& not think about anything else; {\bf 3.} [transitive] {\bf concentrate something} to increase the strength of a substance by reducing its volume, e.g. by boiling it; {\bf concentrate on something} [phrasal verb] to spend more time doing 1 particular thing than others; [n] [countable, uncountable] {\bf concentrate (of something)} a substance that is made stronger because water or other substances have been removed.} on fundamentals\footnote{{\bf fundamentals} [n] [plural] {\bf fundamentals (of something)} the basic \& most important parts of something.}: the rules of usage \& principles of composition most commonly violated\footnote{{\bf violet} [v] {\bf 1. violate something} to go against or refuse to obey a law, an agreement, etc.; {\bf 2. violate something} to not treat something with respect.}.

The reader will soon discover that these rules \& principles are in the form of sharp commands, Sergeant\footnote{{\bf sergeant} [n] (abbr., {\bf Sergt, Sgt}) {\bf 1.} a member of 1 of the middle ranks in the army \& the air force, below an officer; {\bf 2.} (in a UK) a police officer just below the rank of an inspector; {\bf 3.} (in the US) a police officer just below the rank of a lieutenant or caption.} Strunk snapping\footnote{{\bf snap} [v] {\it break} {\bf 1.} [transitive, intransitive] to break something suddenly with a sharp noise; to be broken in this way; {\it take photograph} {\bf 2.} [transitive, intransitive] ({\it informal}) to take a photograph; {\it open}{\tt/}{\it close}{\tt/}{\it move into position} {\bf 3.} [intransitive, transitive] to move, or to move something, into a particular position quickly, especially with a sudden sharp noise; {\it speak impatiently} {\bf 4.} [transitive, intransitive] to speak or say something in an impatient, usually angry, voice; {\it of animal} {\bf 5.} [intransitive] {\bf snap (at somebody{\tt/}something)} to try to bite somebody{\tt/}something, {\sc synonym}: {\bf nip}; {\it lose control} {\bf 6.} [intransitive] to suddenly be unable to control your feelings any longer because the situation has become too difficult; {\it fasten clothing} {\bf 7.} [intransitive, transitive] {\bf snap (something)} ({\it North American English}) to fasten a piece of clothing with a snap; {\it in American football} {\bf 8.} [transitive] {\bf snap something} ({\it sport}) (in American football) to start play by passing the ball back between your legs.} orders to his platoon\footnote{{\bf platoon} [n] a small group of soldiers that is part of a company \& commanded by a lieutenant.}. ``Do not join independent clauses with a comma.'' (Rule 5.) ``Do not break sentences in 2.'' (Rule 6.) ``Use the active voice.'' (Rule 14.) ``Omit\footnote{{\bf omit} [v] {\bf 1.} to not include something{\tt/}somebody, either deliberately or because you have forgotten it{\tt/}them, {\sc synonym}: {\bf leave somebody{\tt/}something out (of something)}; {\bf 2. omit to do something} to not do or fail to do something.} needless\footnote{{\bf needless} [a] (of something bad) not necessary; that could be avoided, {\sc synonym}: unnecessary.} words.'' (Rule 17.) ``Avoid a succession\footnote{{\bf succession} [n] {\bf 1.} [countable, usually singular] a number of things or people that follow each other in time or order, {\sc synonym}: {\bf series}; {\bf 2.} [uncountable] the act of taking over an official position or title; {\bf 3.} [uncountable] the right to take over an official position or title, especially to become the king or queen of a country.} of loose\footnote{{\bf loose} [a] {\bf 1.} not firmly fixed where it should be; that can become separated from something; {\bf 2.} not tightly packed together; not solid or hard; {\bf 3.} not strictly organized or controlled; {\bf 4.} not exact; not very careful; {\bf 5.} (of clothes) not fitting closely, {\sc opposite}: {\bf tight}; {\bf 6.} not tied together; not held in position by anything or contained in anything; {\bf 7.} ({\it medical}) (of body waste) having too much liquid in it.} sentences.'' (Rule 18.) ``In summaries, keep to 1 tense.'' (Rule 21.) Each rule or principle is followed by a short hortatory\footnote{{\bf hortatory} [a] trying to strongly encourage or persuade someone to do something.} essay, \& usually the exhortation\footnote{{\bf exhortation} [n] [countable, uncountable] ({\it formal}) {\bf exhortation (to do something)} an act of trying very hard to persuade somebody to do something.} is followed by, or interlarded\footnote{{\bf interlard} [v] (used with object) {\bf 1.} to diversify by adding or interjecting something unique, striking, or contrasting (usually followed by {\it with}); {\bf 2.} (of things) to be intermixed in.} with, examples in parallel columns -- the true vs. the false, the right vs. the wrong, the timid\footnote{{\bf timid} [a] shy \& nervous; not brave.} vs. the bold, the ragged\footnote{{\bf ragged} [a] {\bf 1.} (of clothes) old \& torn, {\sc synonym}: {\bf shabby}; {\bf 2.} (of people) wearing old or torn clothes; {\bf 3.} having an outline, an edge or a surface that is not straight or even; {\bf 4.} not smooth or regular; not showing control or careful preparation; {\bf 5.} ({\it informal}) very tired, especially after physical effort.} vs. the trim\footnote{{\bf trim} [v] {\bf 1. trim something} to make something neater, smaller, better, etc., by cutting parts from it; {\bf 2.} to cut away unnecessary parts from something; {\bf 3.} [usually passive] {\bf trim something (with something)} to decorate something, especially around its edges.}. From every line there peers out at me the puckish\footnote{{\bf puckish} [a] [usually before noun] ({\it literary}) enjoying playing tricks on other people, {\sc synonym}: {\bf mischievous}.} face of my professor, his short hair parted neatly\footnote{{\bf neat} [a] {\bf 1.} in good order; carefully done or arranged; {\bf 2.} simple but clever; {\bf 3.} containing or made out of just 1 substance; not mixed with anything else.} in the middle \& combed down over his forehead, his eyes blinking incessantly\footnote{{\bf incessantly} [adv] ({\it usually disapproving}) without stopping, {\sc synonym}: {\bf constantly}.} behind steel-rimmed spectacles\footnote{{\bf spectacle} [n] {\bf 1.} [countable, uncountable] {\bf spectacle (of something)} a performance or an event that is very impressive \& exciting to look at; {\bf 2.} [singular] {\bf spectacle (of something)} an unusual, embarrassing or sad sight or situation that attracts a lot of attention; {\bf 3.} ({\bf spectacles}) [plural] [{\it formal}] $=$ {\bf glass}.} as though he had just emerged into strong light, his lips nibbling each other like nervous horses, his smile shuttling to \& fro under a carefully edged mustache.

``Omit needless words!'' cries the author on p. 23, \& into that imperative\footnote{{\bf imperative} [n] a thing that is very important \& needs immediate attention or action; [a] [not usually before noun] very important \& needing immediate attention or action, {\sc synonym}: {\bf vital}.} Will Strunk \fbox{really put his heart \& soul}. In the days when I was sitting in his class, he omitted so many needless words, \& omitted them so forcibly\footnote{{\bf forcibly} [adv] {\bf 1.} in a way that involves the use of physical force; {\bf 2.} in a way that makes something very clear.} \& with such eagerness\footnote{{\bf eager} [a] very interested \& excited by something that is going to happen or about something that you want to do, {\sc synonym}: {\bf keen}.} \& obvious relish\footnote{{\bf relish} [v] to get great pleasure from something; to want very much to do or have something, {\sc synonym}: {\bf enjoy}; [n] {\bf 1.} [uncountable] great pleasure; {\bf 2.} [uncountable, countable] a cold, thick, spicy sauce made from fruit \& vegetables that have been boiled, that is served with meat, cheese, etc.}, that he often seemed in the position of having shortchanged\footnote{{\bf short-change} [v] [often passive] {\bf 1. short-change somebody} to give back less than the correct amount of money to somebody who has paid for something with more than the exact price; {\bf 2. short-change somebody} to treat somebody unfairly by not giving them what they have earned or deserve.} himself -- a man left with nothing more to say yet with time to fill, a radio prophet who had outdistanced\footnote{{\bf outdistance} [v] {\bf outdistance somebody{\tt/}something} to leave somebody{\tt/}something behind by going faster, further, etc.; to be better than somebody{\tt/}something, {\sc synonym}: {\bf outstrip}.} the clock. Will Strunk got out of this predicament\footnote{{\bf predicament} [n] a difficult or an unpleasant situation, especially one where it is difficult to know what to do, {\sc synonym}: {\bf quandary}.} by a simple trick: he uttered\footnote{{\bf utter} [v] {\bf utter something} to make a sound with your voice; to say something.} every sentence 3 times. When he delivered his oration\footnote{{\bf oration} [n] ({\it formal}) a formal speech made on a public occasion, especially as part of a ceremony.} on brevity to the class, he leaned forward over his desk, grasped his coat lapels\footnote{{\bf lapel} [n] 1 of the 2 front parts of the top of a coat or jacket that are joined to the collar \& are folded back.} in his hands, \&, in a husky\footnote{{\bf husky} [a] {\bf 1.} (of a person of their voice) sounding deep, quiet \& rough, sometimes in an attractive way; {\bf 2.} ({\it North American English}) with a large, strong body; [n] (North American English also {\bf huskie}) a large strong dog with thick hair, used for pulling sledges across snow.}, conspiratorial\footnote{{\bf conspiratorial} [a] {\bf 1.} connected with, or making you think of, a conspiracy ($=$ a secret plan to do something illegal); {\bf 2.} (of a person's behavior) suggesting that a secret is being shared.} voice, said, ``Rule 17. Omit needless words! Omit needless words! Omit needless word!''

He was a memorable\footnote{{\bf memorable} [a] special, good or unusual \& therefore worth remembering; easy to remember.} man, friendly \& funny. Under the remembered sting of his kindly lash\footnote{{\bf lash} [v] {\bf 1.} [intransitive, transitive] to hit somebody{\tt/}something with great force, {\sc synonym}: {\bf pound}; {\bf 2.} [transitive] {\bf lash somebody{\tt/}something} to hit a person or an animal with a whip, rope, stick, etc., {\sc synonym}: {\bf beat}.}, I have been trying to omit needless words since 1919, \& although there are still many words that cry for omission \& the huge task will never be accomplished, it is exciting to me to reread to masterly Strunkian elaboration\footnote{{\bf elaboration} [n] [uncountable, countable] {\bf 1.} the act of explaining or describing something in a more detailed way; {\bf 2.} the process of developing a plan, an idea, etc. \& making it complicated or detailed; {\bf 3. elaboration (of something)} ({\it biology}) the production of a substance or structure from elements or simpler constituents in a natural process.} of this noble\footnote{{\bf noble} [a] {\bf 1.} belonging to a family of high social rank, {\sc synonym}: {\bf aristocratic}; {\bf 2.} having or showing fine personal qualities that people admire, e.g. courage, honesty \& care for others; [n] a person who comes from a family of high social rank; a member of the nobility, {\sc synonym}: {\bf aristocratic}.} theme\footnote{{\bf theme} [n] the subject of a talk, piece of writing, exhibition, etc.; an idea that keeps returning in a piece of research or a work of art or literature.}. It goes:
\begin{quotation}
	{\it Vigorous writing is concise. A sentence should contain no unnecessary words, a paragraph no unnecessary sentences, for the same reason that a drawing should have no unnecessary lines \& a machine no unnecessary parts. This requires not that the writer make all sentences short or avoid all detail \& treat subjects only in outline, but that every word tell.}
\end{quotation}
There you have a short, valuable essay on the nature \& beauty of brevity -- 59 words that could change the world. Having recovered from his adventure in prolixity\footnote{{\bf prolixity} [n] [uncountable] ({\it formal}) the fact of using too many words \& therefore creating a piece of writing, a speech, etc., that is boring.} (59 words were a lot of words in the tight world of William Strunk Jr.), the professor proceeds to give a few quick lessons in pruning\footnote{{\bf pruning} [n] [uncountable] {\bf 1.} the activity of cutting off some of the branches from a tree, bush, etc. so that it will grow better \& stronger; {\bf 2.} the act of making something smaller by removing parts; the act of cutting out parts of something.}. Students learn to cut the dead-wood from ``this is a subject that,'' reducing it to ``this subject,'' a saving of 3 words. They learn to trim\footnote{{\bf trim} [v] {\bf 1. trim something} to make something neater, smaller, better, etc., by cutting parts from it; {\bf 2.} to cut away unnecessary parts from something; {\bf 3.} [usually passive] {\bf trim something (with something)} to decorate something, especially around its edges.} ``used for fuel purposes'' down to ``used for fuel.'' They learn that they are being chatterboxes\footnote{{\bf chatterbox} [n] ({\it informal}) a person who talks a lot, especially a child.} when they say ``the question as to whether'' \& that they should just say ``whether'' -- a saving of 4 words out of a possible 5.

The professor devotes\footnote{{\bf devote} [v] {\bf devote yourself to somebody{\tt/}something} to give most of your time, energy or attention to somebody{\tt/}something, {\sc synonym}: {\bf dedicate}; {\bf devote something to something}: to give an amount of time, attention or resources to something.} a special paragraph to the vile\footnote{{\bf vile} [a] {\bf 1.} ({\it informal}) extremely unpleasant or bad, {\sc synonym}: {\bf disgusting}; {\bf 2.} ({\it formal}) morally bad; completely unacceptable, {\sc synonym}: {\bf wicked}.} expression {\it the fact that}, a phrase that causes him to quiver\footnote{{\bf quiver} [v] to shake slightly; to make a slight movement, {\sc synonym}: {\bf tremble}; [n] {\bf 1.} an emotion that has an effect on your body; a slight movement in part of your body; {\bf 2.} a case for carrying arrows.} with revulsion\footnote{{\bf revulsion} [n] [uncountable, singular] ({\it formal}) a strong feeling of horror, {\sc synonym}: {\bf disgust, repugnance}.}. The expression, he says, should be ``revised out of every sentence in which it occurs.'' But a shadow\footnote{{\bf shadow} [n] {\bf 1.} [countable] the dark area or shape produced by somebody{\tt/}something coming between light \& a surface; {\bf 2.} [uncountable] ({\bf shadows} [plural]) darkness, especially that produced by somebody{\tt/}something coming between light \& a surface; {\bf 3.} [singular] the strong (usually bad) influence of somebody{\tt/}something.} of gloom\footnote{{\bf gloom} [n] {\bf 1.} [uncountable, singular] a feeling of being sad \& without hope, {\sc synonym}: {\bf depression}; {\bf 2.} [uncountable] ({\it literary}) almost total darkness.} seems to hang over the page, \& you feel that he knows how hopeless his cause is. I suppose I have written {\it the fact that} a thousand times in the heat of composition, revised it out maybe 500 times in the cool aftermath\footnote{{\bf aftermath} [n] [usually singular] the situation that exists as a result of an important (\& usually unpleasant) event, especially a war, an accident, etc.}. To be batting only .500 this late in the season, to fail half the time to connect with this fat pitch, saddens me, for it seems a betrayal of the man who showed me how to swing\footnote{{\bf swing} [v] {\bf 1.} [intransitive, transitive] to change to make somebody{\tt/}something change from 1 opinion or mood to another; {\bf 2.} [intransitive, transitive] to turn or change direction suddenly; to make something do this; {\bf 3.} [intransitive, transitive] to move backwards or forwards or from side to side while hanging from a fixed point; to make something do this; {\bf 4.} [intransitive, transitive] to move or make something move with a wide curved movement; [n] a change from 1 opinion or situation to another; the amount by which something changes.} at it \& made the swinging seem worthwhile.

I treasure\footnote{{\bf treasure} [n] {\bf 1.} [uncountable] a collection of valuable things e.g. gold, silver \& jewelery; {\bf 2.} [countable, usually plural] a highly valued object; {\bf 3.} [singular] a person who is much loved or valued; [v] {\bf treasure something} to have or keep something that you love \& that is extremely valuable to you, {\sc synonym}: {\bf cherish}.} {\it The Elements of Style} for its sharp\footnote{{\bf sharp} [a] {\bf 1.} [usually before noun] (especially of a change in something) sudden \& fast; {\bf 2.} [usually before noun] (especially of a difference in something) clear \& definite; {\bf 3.} (especially of something that can cut or make a hole in something) having a fine edge or point, {\sc opposite}: {\bf blunt}; {\bf 4.} (of a person or what they say) critical or severe; {\bf 5.} (of a physical feeling or an emotion) very strong \& sudden, often like being cut or wounded, {\sc synonym}: {\bf intense}; {\bf 6.} changing direction suddenly; {\bf 7.} (of people or their minds or eyes) quick to notice or understand things or to react.} advice, but I treasure it even more for the \fbox{audacity}\footnote{{\bf audacity} [n] [uncountable] behavior that is brave but likely to shock or offend people, {\sc synonym}: {\bf nerve}.} \& self-confidence\footnote{{\bf self-confidence} [n] [uncountable] confidence in yourself \& your abilities, {\sc synonym}: {\bf self-assurance, confidence}.} of its author. \fbox{Will knew where he stood.} He was so sure of where he stood, \& made his position so clear \& so plausible, that his peculiar\footnote{{\bf peculiar} [a] belonging to or connected with 1 particular place, situation, person, etc., \& not others.} stance\footnote{{\bf stance} [n] the opinions that somebody has about something \& expresses publicly, {\sc synonym}: {\bf position}.} has continued to invigorate\footnote{{\bf invigorate} [v] {\bf 1. invigorate somebody} to make somebody feel healthy \& full of energy; {\bf 2. invigorate something} to make a situation, an organization, etc. efficient \& successful.} me -- \&, I am sure, thousands of other ex-students -- during the years that have intervened\footnote{{\bf intervene} [v] {\bf 1.} [intransitive] to become involved in a situation in order to improve it or stop it from getting worse; {\bf 2.} [intransitive] to happen in the time between events; {\bf 3.} [intransitive] to exist or be found in the space between things; {\bf 4.} [intransitive] to happen in a way  that delays something or prevents it from happening.} since our 1st encounter\footnote{{\bf encounter} [v] {\bf 1. encounter something} to experience something, especially something unpleasant or difficult, while you are trying to do something else, {\sc synonym}: {\bf run into something}; {\bf 2. encounter something{\tt/}somebody} to discover or experience something, or meet somebody, especially something{\tt/}somebody new, unusual or unexpected, {\sc synonym}: {\bf come across somebody{\tt/}something}; [n] a meeting, especially one that is sudden or unexpected.}. He had a number of likes \& dislikes that were almost as whimsical\footnote{{\bf whimsical} [a] unusual \& not serious in a way that is either funny or annoying.} as the choice of a necktie, yet he made them seem utterly\footnote{{\bf utter} [a] [only before noun] used to emphasize how complete something is, {\sc synonym}: {\bf total}; [v] {\bf utter something} to make a sound with your voice; to say something.} convincing. He disliked the word {\it forceful}\footnote{{\bf forceful} [a] {\bf 1.} (of people) expressing opinion firmly \& clearly in a way that persuades other people to believe them, {\sc synonym}: {\bf assertive}; {\bf 2.} (of opinions, etc.) expressed firmly \& clearly so that other people believe them; {\bf 3.} using force; {\bf 4.} (of action) strong \& effective.} \& advised us to use {\it forcible}\footnote{{\bf forcible} [a] [only before noun] involving the use of physical force.} instead. He felt that the word {\it clever}\footnote{{\bf clever} [a] {\bf 1.} (especially British English) quick at learning \& understanding things, {\sc synonym}: {\bf intelligent}; {\bf 2. clever (at something{\tt/}doing somethign)} (especially British English) skillful; {\bf 3.} showing intelligence or skill, e.g. in the design of an object, in an idea or somebody's actions.} was greatly overused: ``It is best restricted to ingenuity\footnote{{\bf ingenuity} [n] [uncountable] the ability to invent things or solve problems in clever new ways, {\sc synonym}: {\bf inventiveness}.} displayed in small matters.'' He despised\footnote{{\bf despise} [v] (not used in the progressive tenses) to dislike \& have no respect for somebody{\tt/}something.} the expression {\it student body}, which he termed gruesome\footnote{{\bf gruesome} [a] very unpleasant \& filling you with horror, usually because it is connected with death or injury.}, \& made a special trip downtown to the {\it Alumni News} office 1 day to protest\footnote{{\bf protest} [n] [uncountable, countable] the expression of strong disagreement with or opposition to something; a statement or an action that shows this.} the expression \& suggest that {\it studentry} be substituted\footnote{{\bf substitute} [v] [intransitive, transitive] to take the place of somebody{\tt/}something else; to use somebody{\tt/}something instead of somebody{\tt/}something else; [n] a person or thing that you use or have instead of the usual one.} -- a coinage\footnote{{\bf coinage} [n] {\bf 1.} [uncountable] the coins used in a particular place or at a particular time; coins of a particular type; {\bf 2.} [countable, uncountable] a word or phrase that has been invented recently; the process of inventing a word or phrase.} of his own, which he felt was similar to {\it citizenry}\footnote{{\bf citizenry} [n] [singular $+$ singular or plural verb] ({\it formal}) all the citizens of a particular town, country, etc.}. I am told that the {\it News} editor was so charmed by the visit, if not by the word, that he ordered the student body buried, never to rise again. {\it Studentry} has taken its place. It's not much of an improvement, but it does sound less cadaverous\footnote{{\bf cadaverous} [a] ({\it literary}) (of a person) extremely pale, thin \& looking ill.}, \& it made Will Strunk quite happy.

Some years ago, when the heir\footnote{{\bf heir} [n] {\bf 1.} a person who has the legal right to receive somebody's property, money or title when that person dies; {\bf 2.} a person who is thought to continue the work or a tradition started by somebody else.} to the throne of England was a child, I noticed a headline in the {\it Times} about Bonnie Prince Charlie: ``CHARLES' TONSILS OOUT.'' Immediately Rule 1 leapt to mind.
\begin{quotation}
	{\bf 1.} Form the possessive singular of nouns by adding {\it 's}. Follow this rule whatever the final consonant\footnote{{\bf consonant} [n] {\bf 1.} (phonetics) a speech sound made by completely or partly stopping the flow of air being breathed out through the mouth; {\bf 2.} a letter of the alphabet that represents a consonant sound.}. Thus write, {\it Charles's friend, Burns's poems, the witch's malice\footnote{{\bf malice} [n] [uncountable] a desire to harm somebody caused by a feeling of hate.}}.
\end{quotation}
Clearly, Will Strunk had foreseen\footnote{{\bf foreseen} [v] to know about something before it happens.}, as far back as 1918, the dangerous tonsillectomy\footnote{{\bf tonsillectomy} [n] ({\it medical}) a medical operation to remove the tonsils.} of a prince, in which the surgeon removes the tonsils \& the {\it Times} copy desk removes the final {\it s}. He started his book with it. I commend Rule 1 to the {\it Times}, \& I trust that Charles's throat, not Charles' throat, is in fine shape today.

Style rules of this sort are, of course, somewhat a matter of individual preference\footnote{{\bf preference} [n] {\bf 1.} [countable, usually singular, uncountable] a greater interest in or desire for somebody{\tt/}something than somebody{\tt/}something else; {\bf 2.} [countable] a thing that is liked better or best.}, \& even the established rules of grammar are open to challenge. Prof. Strunk, although 1 of the most inflexible\footnote{{\bf inflexible} [a] {\bf 1.} ({\it disapproving}) that cannot be changed or made more suitable for a particular situation, {\sc synonym}: {\bf rigid}; {\bf 2.} ({\it disapproving}) (of people or organizations) unwilling to change their opinions, decision or behavior.} \& choosy\footnote{{\bf choosy} [a] ({\it informal}) careful in choosing; difficult to please, {\sc synonym}: {\bf fussy, picky}.} of men, was quick to acknowledge\footnote{{\bf acknowledge} [v]  {\bf 1.} to accept that something is true or exists; {\bf 2.} to accept that somebody{\tt/}something has a particular quality, importance or status, {\sc synonym}: {\bf recognize}; {\bf 3. acknowledge somebody{\tt/}something} to publicly express thanks fo help or inspiration; {\bf 4. acknowledge something} to tell somebody that you have received something that they sent to you.} the fallacy\footnote{{\bf fallacy} [n] {\bf 1.} [countable] a false idea that many people believe is true; {\bf 2.} [uncountable, countable] a false way of thinking about something.} of inflexibility \& the danger of doctrine\footnote{{\bf doctrine} [n] {\bf 1.} [countable, uncountable] {\bf doctrine (of something)} a belief or principle, or set of beliefs or principles, held by a religion, a political party or a legal system; {\bf 2.} ({\bf Doctrine}) [countable] (US) a statement of government policy, especially foreign policy.}. ``It is an old observation,'' he wrote, ``that the best writers sometimes disregard\footnote{{\bf disregard} [v] {\bf disregard something} to not consider something; to treat something as unimportant, {\sc synonym}: {\bf ignore}.} the rules of rhetoric\footnote{{\bf rhetoric} [n] [uncountable] {\bf 1.} ({\it often disapproving}) speech or writing that is intended to influence people, but that is not completely honest or sincere; {\bf 2.} the skill of using language in speech or writing in a special way that influences or entertains people.}. \texttt{[stop translating here]} When they do so, however, the reader will usually find in the sentence some compensating merit, attained at the cost of the violation. Unless he is certain of doing as well, he will probably do best to follow the rules.''

It is encouraging to see how perfectly a book, even a dusty rule book, perpetuates \& extends the spirit of a man. Will Strunk loved the clear, the brief, the bold, \& his book is clear, brief, bold. Boldness is perhaps its chief distinguishing mark. On p. 26, explaining 1 of his parallels, he says, ``The lefthand version gives the impression that the writer is undecided or timid, apparently unable or afraid to choose 1 form of expression \& hold to it.'' \& his original Rule 11 was ``Make definite assertions.'' That was Will all over. He scorned the vague, the tame, the colorless, the irresolute. He felt it was worse to be irresolute than to be wrong. I remember a day in class when he leaned far forward, in his characteristic pose -- the pose of a man about to impart a secret -- \& croaked, ``If you don't know how to pronounce a word, say it loud! If you don't know how to pronounce a word, say it loud!'' This comical piece of advice struck me as sound at the time, \& I still respect it.\fbox{ Why compound ignorance with inaudibility?} \fbox{Why run \& hide?}

All through {\it The Elements of Style} one finds evidence of the author's deep sympathy for the reader. Will felt that the reader was in serious trouble most of the time, floundering in a swamp, \& that it was the duty of anyone attempting to write English to drain this swamp quickly \& get the reader up on dry ground, or at least to throw a rope. In revising the text, I have tried to hold steadily in mind this belief of his, this concern for the bewildered reader.

In the English classes of today, ``the little book'' is surrounded by longer, lower textbooks -- books with permissive steering \& automatic transitions. Perhaps the book has become something of a curiosity. To me, it still seems to maintain its original poise, standing, in a drafty time, erect, resolute, \& assured. I still find the Strunkian wisdom a comfort, the Strunkian humor a delight, \& the Strunkian attitude forward right-\&-wrong a blessing undisguised.'' -- \cite[Introduction (by E. B. White)]{Strunk_White_element_style}

%------------------------------------------------------------------------------%

\subsection*{Foreword}
``This classic work, still used by college students as a guide to succinct \& clear writing, was formulated by a university English teacher, {\sc William Strunk, Jr.}, for the benefit of his students. The little work's great longevity \& continuing relevance is a credit to the man who conceived it.

William Strunk, Jr. was the oldest of 4 children born \& raised in Cincinnati, Ohio, by his parents William \& Ella Gerretson Strunk. he took his bachelor's degree from the University of Cincinnati with a Bachelor of Arts in 1890. After achieving his degree he was employed by Rose Polytechnical Institute to teach mathematics from 1890--1891. From there he went on to teach at Cornell University while also earning his PhD, which he took in 1896. Following his PhD, he traveled to France for the academic year of 1898--99 where he spent time at the University of Paris: the Sorbonne \& the Coll\`ege de France where he studied philosophy $\Phi$ \& morphology\footnote{1. (biology) the form \& structure of animals \& plants, studied as a science; 2. (linguistics) the forms of words, studied as a branch of linguistics.}.

Upon his return to the United States, Strunk began teaching English at his alma mater, Cornell University. He taught there for 46 years \& was elected to Phi Beta Kappa $\phi\beta\kappa$, America's most prestigious honor society in the liberal arts, before his retirement. During his time teaching Strunk specialized in both English \& non-English literature.

He published several books, the 1st being {\it The Elements of Style} in 1918. In 1922, Strunk published {\it English Metres} which concentrated on the study of the poetic metrical form. Following the success of {\it The Elements of Style}, Strunk revised it with Edward A. Tenney in 1935, \& renamed it {\it The Element \& Practice of Composition}. As well as being an educator, Strunk served as a literary consultant for the film studio Metro-Goldwyn-Mayer from 1935--1936 on the set of the 1936 production of {\it Romeo \& Juliet}. William Strunk Jr. retired from teaching in 1937 \& died a few years later in 1945.

{\it The Elements of Style} was written \& privately published for his Cornell students in 1918. It was a guide to writing \& editing with the intention ``to lighten the task of instructor \& student by concentrating attention $\ldots$ on a few essentials, the rules of usage \& principles of composition most commonly violated.'' On the part of the instructor, it allowed him to simply refer to the rule which was broken when grading rather than having to explain it. For the student it became a guide or how-to for writing essays.

The short introduction to ``style'' included grammar insights such as how to form a possessive singular noun \& does \& don'ts such as, ``do use the active voice'' but ``don't break 1 sentence into 2.'' Strunk even had the brilliant foresight to add a section on commonly misspelt words.

In 1957, E.B. White, as a previous student of Strunk, praised what had become known as ``the little book''. He was then commissioned to revise \& update the book by Macmillian \& Company. That book, commonly known as ``Strunk \& White'' went on to sell over 2 million copies \& is still used in colleges \& universities today. This edition contains just Strunk's original 1918 advice to his students.''

%------------------------------------------------------------------------------%

\subsection*{Introduction}
``This book aims to give a brief space the principal requirements of plain English style. It aims to lighten the task of instructor \& student by concentrating attention (in Chaps. II \& III) on a few essentials, the rules of usage \& principles of composition most commonly violated. In accordance with this plan it lays down 3 rules for the use of the common, instead of a score or more, \& 1 for the use of the semicolon, in the belief that these 4 rules provide for all the internal punctuation that is required by 19 sentences out of 20. Similarly, it gives in Chap. III only those principles of the paragraph \& the sentence which are of the widest application. The book thus covers only a small portion of the field of English style. The experience of its writer has been that once past the essentials, students profit most by individual instruction based on the problems of their own work, \& that each instructors has his own body of theory, which he may prefer to that offered by any textbook.

The numbers of the sections may be used as references in correcting manuscript.

The writer's colleagues in the Department of English in Cornell University have greatly helped him in the preparation of his manuscript. Mr. George McLane Wood has kindly consented to the inclusion under Rule 10 of some material from his Suggestions to Authors.The following books are recommended for reference or further study: in connection with Chaps. II \& IV:
\begin{enumerate}
	\item F. Howard Collins. {\it Author \& Printer} (Henry Frowde).
	\item Chicago University Press. {\it Manual of Style}.
	\item T. L. De Vinne. {\it Correct Composition (The Century Company)}.
	\item Horace Hart. {\it Rules for Compositors \& Printers} (Oxford University Press).
	\item George McLane Wood. {\it Extracts from the Style-Book of the Government Printing Office} (United States Geological Survey).
\end{enumerate}
in connection with Chaps. III \& V
\begin{enumerate}
	\item {\it The King's English} (Oxford University Press).
	\item Sir Arthur Quiller-Couch. {\it The Art of Writing} (Putnam), especially the chapter, {\it Interlude on Jargon}.
	\item George McLane Wood. {\it Suggestions to Authors} (United States Geological Survey).
	\item John Lesslie Hall. {\it English Usage} (Scott, Foresman \& Co.).
	\item James P. Kelley. {\it Workmanship in Words} (Little, Brown \& Co.).
\end{enumerate}
In these will be found full discussions of many points here briefly treated \& an abundant store of illustrations to supplement those given in this book.

It is an old observation that the best writers sometimes disregard the rules of rhetoric. When they do so, however, the reader will usually find in the sentence some compensating merit, attained at the cost of the violation. Unless he is certain of doing as well, he will probably do best to follow the rules. After he has learned, by their guidance, to write plain English adequate for everyday uses, let him look, for the secrets of style, to the study of the masters of literature.''

%------------------------------------------------------------------------------%

%------------------------------------------------------------------------------%

\section*{Amazon/reviews}
\begin{quotation}
	`The Elements of Style' (1918), by William Strunk, Jr., is an American English writing style guide.
	
	It is the best-known, most influential prescriptive treatment of English grammar \& usage, \& often is required reading \& usage in U.S. high school \& university composition classes.
	
	This edition of `The Elements of Style' details 8 elementary rules of usage, 10 elementary principles of composition, ``a few matters of form'', \& a list of commonly misused words \& expressions.
\end{quotation}

\begin{quotation}
	``{\it $\ldots$ a marvelous \& timeless little book$\ldots$ Here, succinctly, elegantly \& without fuss are the essentials of writing clear, correct English}.'' - John Clare, {\it The Telegraph}
\end{quotation}

%------------------------------------------------------------------------------%

\section*{Forword}

The 1st writer Roger Angell watched at work was my stepfather, E. B. White.

Each Tuesday morning, he would close his study door \& sit down to write the ``Notes \& Comment'' page for {\it The New Yorker}.

The task was familiar with him - he was required to file a few hundred words of editorial or personal commentary on some topic in or out of the news that week - but the sounds of his typewriter from his room came in hesitant bursts, with long silences in between.

Hours went by.

Summoned at last for lunch, he was silent \& preoccupied, \& soon excused himself to get back to the job.

When the copy went off at last, in the afternoon RFD pouch - we were in Maine, a day's mail away from New York - he rarely seemed satisfied.

``It isn't good enough,'' he said sometimes.

``I wish it were better.''

%
Writing is hard, even for authors who do it all the time.

Less frequent practitioners - the job applicant; the business executive with an annual report to get out; the high school senior with a Faulkner assignment; the graduate-school student with her thesis proposal; the writer of a letter of condolence - often get stuck in an awkward passage of find a muddle on their screens, \& then blame themselves.

What should be easy \& flowing looks tangled or feeable or overblown - not what was meant at all.

What's wrong with me, each one thinks.

Why can't I get this right?

%
It was this recurring question, put to himself, that must have inspired White to revive \& add to a textbook by an English professor of his, Will Strunk Jr., that he had 1st read in college, \& to get it published.

The result, this quiet book, has been in print for 40 years, \& has offered more than 10 million writers a helping hand.

White knew that a compendium of specific tips - about singular \& plural verbs, parentheses, that ``that'' - ``which'' scuffle, \& many others - could clear up a recalcitrant sentence or subclause when quickly reconsulted, \& that the larger principles needed to be kept in plain sight, like a wall sampler.

%
How simple they look, set down here in White's last chapter: ``Write in a way that comes naturally,'' ``Revise \& rewrite,'' ``Do not explain too much,'' \& the rest; above all, the cleansing, clarion ``Be clear.''

How often Roger Angell has turned to them, in the book or in my mind, while trying to start or unblock or revise some piece of my own writing!

They help - they really do.

They work.

They are the way.

%
E. B. White's prose is celebrated for its ease \& clarity - just think of {\it Charlotte's Web} - but maintaining this standard required endless attention.

When the new issue of {\it The New Yorker} turned up in Maine, Roger Angell sometimes saw him reading his ``Comment'' piece over to himself, with only a slightly different expression than the one he'd worn on the day it went off.

Well, O.K., he seemed to be saying.

{\it At least I got the elements right}.

%
This edition has been modestly updated, with word processors \& air conditioners making their 1st appearance among White's references, \& with a light redistribution of genders to permit a feminine pronoun or female farmer to take their places among the males who once innocently served him.

Sylvia Plath has knocked Keats out of the box, \& Roger Angell notices that ``America'' has become ``this country'' in a sample text, to forestall a subsequent \& possibly demeaning ``she'' in the same paragraph.

What is not here is anything about E-mail - the rules-free, lower-case flow that cheerfully keeps us in touch these days.

E-mail is conversation, \& it may be replacing the sweet \& endless talking we once sustained (and tucked away) within the informal letter.

But we are all writers \& readers as well as communicators, with the need at times to please \& satisfy ourselves (as White put it) with the clear \& almost perfect thought.

\begin{flushright}
	Roger Angell
\end{flushright}

%------------------------------------------------------------------------------%

\subsection*{Introduction}

At the close of the 1st World War, when E. B. White was a student at Cornell, E. B. White took a course called {\it English 8}.

My professor was William Strunk Jr.

A textbook required for the course was a slim volume called {\it The Elements of Style}, whose authors was the professor himself.

The year was 1919.

The book was known on the campus in those days as ``the little book,'' with the stress on the word ``little.''

It had been privately printed by the author.

%
E. B. White passed the course, graduated from the university, \& forgot the book but not the professor.

Some 38 years later, the book bobbed up again in my life when Macmillan commissioned me to revise it for the college market \& the general trade.

Meantime, Prof. Strunk had died.

%
{\it The Elements of Style}, when E. B. White reexamined it in 1957, seemed to me to contain rich deposits of gold.

It was Will Strunk's {\it parvum opus}, his attempt to cut the vast tangle of English rhetoric down to size \& write its rules \& principles on the head of a pint.

Will himself had hung the tag ``little'' on the book; he referred to it sardonically \& with secret pride as ``the {\it little} book,'' always giving the word ``little'' a special twist, as though he were putting a spin on a ball.

In its original form, it was a 43 page summation of the case for cleanliness, accuracy, \& brevity in the use of English.

Today, 52 years later, its vigor is unimpaired, \& for sheer pith E. B. White thinks it probably sets a record that is not likely to be broken.

Even after E. B. White got through tampering with it, it was still a tiny thing, a barely tarnished gem.

7 rules of usage, 11 principles of composition, a few matters of form, \& a list of words \& expressions commonly misused - that was the sum \& substance of Prof. Strunk's work.

Somewhat audaciously, \& in an attempt to give my publisher his money's worth, E. B. White added a chapter called ``An Approach to Style,'' setting forth my own prejudices, my notions of error, my articles of faith.

This chapter (Chap. V) is addressed particularly to those who feel that English prose composition is not only a necessary skills but a sensible pursuit as well - a way to spend one's days.

E. B. White thinks Prof. Strunk would not object to that.

%
A 2nd edition of the book was published in 1972.

E. B. White has now completed a 3rd revision.

Chap. IV has been refurbished with words \& expressions of a recent vintage; 4 rules of usage have been added to Chap. I.

Fresh examples have been added to some of the rules \& principles, amplification has reared its head in a few places in the text where E. B. White felt an assault could successfully be made on the bastions of its brevity, \& in general the book has received a thorough overhaul - to correct errors, delete bewhiskered entries, \& enliven the argument.

%
Prof. Strunk was a positive man.

His book contains rules of grammar phrased as direct orders.

In the main E. B. White has not tried to soften his commands, or modify his pronouncements, or remove the special objects of his scorn.

E. B. White has tried, instead, to preserve the flavor of his discontent while slightly enlarging the scope of the discussion.

{\it The Element of Style} does not pretend to survey the whole field.

Rather it proposes to give in brief space the principal requirements of plain English style.

It concentrates on fundamentals: the rules of usage \& principles of composition most commonly violated.

%
The reader will soon discover that these rules \& principles are in the form of sharp commands, Sergeant Strunk snapping orders to his platoon.

``Do not joint independent clauses with a comma.'' (Rule 5.)

``Do not break sentences in 2.'' (Rule 6.)

``Use the active voice.'' (Rule 14.)

``Omit needless words.'' (Rule 17.)

``Avoid a succession of loose sentences.'' (Rule 18.)

``In summaries, keep to 1 tense.'' (Rule 21.)

Each rule or principle is followed by a short hortatory essay, \& usually the exhortation is followed by, or interlarded with, examples in parallel columns - the true vs. the false, the right vs. the wrong, the timid vs. the bold, the ragged vs. the trim.

From every line there peers out at me the puckish face of my professor, his short hair pared neatly in the middle \& combed down over his forehead, his eyes blinking incessantly behind steel-rimmed spectacles as though he had just emerged into strong light, his lips nibbling each other like nervous horses, his smile shuttling to \& fro under a carefully edged mustache.

%
``Omit needless words!'' cries the author on p. 24, \& into that imperative Will Strunk really put his heart \& soul.

In the days when E. B. White was sitting in his class, he omitted so many needless words, \& omitted them so forcibly \& with such eagerness \& obvious relish, that he often seemed in the position of having shortchanged himself - a man left with nothing more to say yet with time to fill, a radio prophet who had out-distanced the clock.

Will Strunk got out of this predicament by a simple trick: he uttered every sentence 3 times.

When he delivered his oration on brevity to the class, he leaned forward over his desk, grasped his coat lapels in his hands, and, in a husky, conspiratorial voice, said, ``Rule 17. Omit needless words! Omit needless words! Omit needless words!''

%
He was a memorable man, friendly \& funny.

Under the remembered string of his kindly lash, E. B. White has been trying to omit needless words since 1919, \& although there are still many words that cry for omission \& the huge task will never be accomplished, it is exciting to me to reread the masterly Strunkian elaboration of this noble theme.

It goes:

\begin{quotation}
	\it
	Vigorous writing is concise. A sentence should contain no unnecessary words, a paragraph no unnecessary sentences, for the same reason that a drawing should have no unnecessary lines \& a machine no unnecessary parts.
	
	This requires not that the writer make all sentences short or avoid all detail \& treat subjects only in outline, but that every word tell.
\end{quotation}
There you have a short, valuable essay on the nature \& beauty of brevity - 59 words that could change the world.

Having recovered from his adventure in prolixity (59 words were a lot of words in the tight world of William Strunk Jr.), the professor proceeds to give a few quick lessons in pruning.

Students learn to cut the dead-wood from ``this is a subject that,'' reducing it to ``this subject,'' a saving of 3 words.

They learn to trim ``used for fuel purposes'' down to ``used for fuel.''

They learn that they are being chatterboxes when they say ``the question as to whether'' \& they should just day ``whether'' - a saving of 4 words out of a possible 5.

%
The professor devotes a special paragraph to the vile expression {\it the fact that}, a phrase that causes him to quiver with revulsion.

The expression, he says, should be ``revised out of every sentence in which it occurs.''

But a shadow of gloom seems to hang over the page, \& you feel that he knows how hopeless his cause is.

E. B. White supposes E. B. White has written {\it the fact that} a thousand times in the heat of composition, revised it out maybe 500 times in the cool aftermath.

To be batting only .500 this late in the season, to fail half the time to connect with this fat pitch, saddens me, for it seems a betrayal of the man who showed me how to swing at it \& made the swinging seem worthwhile.

%
E. B. White treasures {\it The Elements of Style} for its shape advice, but E. B. White treasures it even more for the audacity \& self-confidence of its author.

Will knew where he stood.

He was so sure of where he stood, \& made his position so clear \& so plausible, that his peculiar stance has continued to invigorate me - and, E. B. White is sure, thousands of other ex-students - during the years that have intervened since our 1st encounter.

He had a number of likes \& dislikes that were almost as whimsical as the choice of a necktie, yet he made them seem utterly convincing.

He disliked the word {\it forceful} \& advised us to use {\it forcible} instead.

He felt that the word {\it clever} was greatly overused: ``It is best restricted to ingenuity displayed in small matters.''

He despised the expression {\it student body}, which he termed gruesome, \& made a special trip downtown to the {\it Alumni News} office 1 day to protest the expression \& suggest that {\it studentry} be substituted - a coinage of is own, which he felt was similar to {\it citizenry}.

E. B. White is told that the {\it News} editor was so charmed by the visit, if not by the word, that he ordered the student body buried, never to rise again.

{\it Studentry} has taken its place.

It's not much of an improvement, but it does sound less cadaverous, \& it made Will Strunk quite happy.

%
Some years ago, when the heir to the throne of England was a child, E. B. White noticed a headline in the {\it Times} about Bonnie Prince Charlie: ``CHARLES' TONSILS OUT.''

Immediately Rule 1 leapt to mind.
\begin{enumerate}
	\item Form the possessive singular of nuns by adding {\it 's}.
	
	Follow this rule whatever the final consonant.
	
	Thus write,
	\begin{example}
		Charles's friend
		
		Burns's poems
		
		the witch's malice
	\end{example}
\end{enumerate}
Clearly, Will Strunk had foreseen, as far back as 1918, the dangerous tonsillectomy of a prince, in which the surgeon removes the tonsils \& the {\it Times} copy desk removes the final {\it s}.

He started his book with it.

E. B. White recommends Rule 1 to the {\it Times}, \& E. B. White trusts that Charles's throat, not Charles' throat, is in fine shape today.

%
Style rules of this sort are, of course, somewhat a matter of individual preference, \& even the established rules of grammar are open to challenge.

Professor Strunk, although 1 of the most inflexible \& choosy of men, was quick to acknowledge the fallacy of inflexibility \& the danger of doctrine.
\begin{quotation}
	``{\it It is an old observation},'' he wrote, ``{\it that the best writers sometimes disregard the rules of rhetoric. When they do so, however, the reader will usually find in the sentence some compensating merit, attained at the cost of the violation. Unless he is certain of doing as well, he will probably do best to follow the rules}.''
\end{quotation}
It is encouraging to see how perfectly a book, even a dusty rule book, perpetuates \& extends the spirit of a man.

Will Strunk loved the clear, the brief, the bold, \& his book is clear, brief, bold.

Boldness is perhaps its chief distinguishing mark.

On p. 26, explaining 1 of his parallels, he says,
\begin{quotation}
	``{\it The lefthand version gives the impression that the writer is undecided or timid, apparently unable or afraid to choose 1 form of expression \& hold to it}.''
\end{quotation}
\& his original Rule 11 was ``Make definite assertions.''

That was Will all over.

He scorned the vague, the tame, the colorless, the irresolute.

He felt it was worse to be irresolute than to be wrong.

E. B. White remembers a day in class when he leaned far forward, in this characteristic pose - the pose of a man about to impart a secret - \& croaked,
\begin{quotation}
	``{\it If you don't know how to pronounce a word, say it loud! If you don't know how to pronounce a word, say it loud!}''
\end{quotation}
This comical piece of advice struck me as sound at the time, \& E. B. White still respects it.

{\it Why compound ignorance with inaudibility?}

{\it Why run \& hide?}

%
All through {\it The Elements of Style} one finds evidences of the author's deep sympathy for the reader.

Will felt that the reader was in serious trouble most of the time, floundering in a swamp, \& that it was the duty of anyone attempting to write English to drain this swamp quickly \& get the reader up on dry ground, or at least to throw a rope.

In revising the text, E. B. White has tried to hold steadily in mind this belief of his, this concern for the bewildered reader.

%
In the English classes of today, ``the little book'' is surrounded be longer, lower textbooks - books with permissive steering \& automatic transitions.

Perhaps the book has become something of a curiosity.

To E. B. White, it still seems to maintain its original poise, standing, in a drafty time, erect, resolute, \& assured.

E. B. White still finds the Strunkian wisdom a comfort, the Strunkian humor a delight, \& the Strunkian attitude toward right-and-wrong a blessing undisguised.

\begin{flushright}
	E. B. White
\end{flushright}

%------------------------------------------------------------------------------%

\subsection{Elementary Rules of Usage}

\subsubsection{Form the possessive singular of nouns by adding 's.}
Follow this rule whatever the final consonant.

Thus write,
\begin{example}
	Charles's friend
	
	Burns's poems
	
	the witch's malice
\end{example}
Exceptions are the possessive of ancient proper names in {\it -es} an {\it -is}, the possessive {\it Jesus'}, \& such forms as {\it for conscience' sake, for righteousness' sake}.

But such forms as {\it Achilles' heel, Moses' laws, Isis' temple} are commonly replaced by
\begin{example}
	the laws of Moses
	
	the temple of Isis
\end{example}
The pronominal possessives {\it hers, its, theirs, yours}, \& {\it ours} have no apostrophe.

Indefinite pronouns, however, use the apostrophe to show possession.
\begin{example}
	one's rights
	
	somebody else's umbrella
\end{example}
A common error is to write {\it it's} for {\it its}, or vice versa.

The 1st is a contraction, meaning ``it is.''

The 2nd is a possessive.
\begin{example}
	It's wise dog that scratches its own fleas.
\end{example}

%------------------------------------------------------------------------------%

\subsubsection{In a series of 3 or more terms with a single conjunction, use a comma after each term except the last}
Thus write,
\begin{example}
	red, white, \& blue
	
	gold, silver, or copper
	
	He opened the letter, read it, \& made a note of its contents.
\end{example}
This comma is often referred to as the ``serial'' comma.

In the names of business firms the last comma is usually omitted.

Follow the usage of the individual firm.
\begin{example}
	Little, Brown \& Company
	
	Donaldson, Lufkin \& Jenrette
\end{example}

%------------------------------------------------------------------------------%

\subsubsection{Enclose parenthetic expressions between commas}
\begin{example}
	The best way to see a country, unless you are pressed for time, is to travel on foot.
\end{example}
This rule is difficult to apply; it is frequently hard to decide whether a single word, e.g., {\it however}, or a brief phrase is or is not parenthetic.

If the interruption to the flow of the sentence is but slight, the commas may be safely omitted.

But whether the interruption is slight or considerable, never omit 1 comma \& leave the other.

There is no defense for such punctuation as
\begin{example}
	Marjories husband, Colonel Nelson paid us a visit yesterday.
\end{example}
or
\begin{example}
	My brother you will be pleased to hear, is now in perfect health.
\end{example}
Dates usually contain parenthetic words or figures.

Punctuate as follows:
\begin{example}
	February to July, 1992
	
	April 6, 1986
	
	Wednesday, November 14, 1990
\end{example}
Note that it is customary to omit the comma in
\begin{example}
	6 April 1988
\end{example}
The last form is an excellent way to write a date; the figures are separated by a word \& are, for that reason, quickly grasped.

%
A name or a title in direct address is parenthetic.
\begin{example}
	If, Sir, you refuse, I cannot predict what will happen.
	
	Well, Susan, this is a fine mess you are in.
\end{example}
The abbreviation {\it etc., i.e.}, \& {\it e.g.}, the abbreviations for academic degrees, \& titles that follow a name are parenthetic \& should be punctuated accordingly.
\begin{example}
	Letters, packages, etc., should go here.
	
	Horace Fulsome, Ph.D., presided.
	
	Rachel Simonds, Attorney
	
	The Reverend Harry Lang, S.J.
\end{example}
No comma, however, should separate a noun from a restrictive term of identification.
\begin{example}
	Billy the Kid
	
	The novelist Jane Austen
	
	William the Conqueror
	
	The poet Sappho
\end{example}
Although {\it Junior}, with its abbreviation {\it Jr.}, has commonly been regarded as parenthetic, logic suggests that it is, in fact, restrictive \& therefore not in need of a comma.
\begin{example}
	James Wright Jr.
\end{example}
Nonrestrictive relative clauses are parenthetic, as are similar clauses introduced by conjunctions indicating time or place.

Commas are therefore needed.

A nonrestrictive clauses is one that does not serve to identify or define the antecedent noun.
\begin{example}
	The audience, which had at 1st been indifferent, became more \& more interested.
	
	In 1769, when Napoleon was born, Corsica had but recently been acquired by France.
	
	Nether Stowey, where Coleridge wrote The Rime of the Ancient Mariner, is a few miles from Bridgewater.
\end{example}
In these sentences, the clauses introduced by {\it which, when}, \& {\it where} are nonrestrictive; they do not limit or define, they merely add something.

In the 1st example, the clause introduced by {\it which} does not serve to tell which of several possible audiences is meant; the reader presumably knows that already.

The clause adds, parenthetically, a statement supplementing that in the main clause.

Each of the 3 sentences is a combination of 2 statements that might have been made independently.
\begin{example}
	The audience was at 1st indifferent. Later it became more \& more interested.
	
	Napoleon was born in 1769. At that time Corsica had but recently been acquired by France.
	
	Coleridge wrote The Rime of the Ancient Mariner at Nether Stowey. Nether Stowey is a few miles from Bridgewater.
\end{example}
Restrictive clauses, by contrast, are not parenthetic \& not set off by commas.

Thus.
\begin{example}
	People who live in glass houses shouldn't throw stones.
\end{example}
Here the clause introduced by {\it who} does serve to tell which people are meant; the sentence, unlike the sentences above, cannot be split into 2 independent statements.

The same principle of comma use applies to participial phrases \& to appositives.
\begin{example}
	People sitting in the rear couldn't hear, \emph{(restrictive)}
	
	Uncle Bert, being slightly deaf, moved forward, \emph{(non-restrictive)}
	
	My cousin Bob is a talented harpist, \emph{(restrictive)}
	
	Our oldest daughter, Mary, sings, \emph{nonrestrictive}
\end{example}
When the main clause of a sentence is preceded by a phrase or a subordinate clause, use a comma to set off these elements.
\begin{example}
	Partly by hard fighting, partly by diplomatic skill, they enlarged their dominions to the east \& rose to royal rank with the possession of Sicily.
\end{example}

%------------------------------------------------------------------------------%

\subsubsection{Place a comma before a conjunction introducing an independent clause}
\begin{example}
	The early records of the city have disappeared, \& the story of its 1st years can no longer be reconstructed.
	
	The situation is perilous, but there is still 1 chance of escape.
\end{example}
2-part sentences of which the 2nd member is introduced by as (in the sense of ``because''), {\it for, or, nor}, or {\it while} (in the sense of ``and at the same time'') likewise require a comma before the conjunction.

%
If a dependent clause, or an introductory phrase requiring to be set off by a comma, precedes the 2nd independent clause, no comma is needed after the conjunction.
\begin{example}
	The situation is perilous, but if we are prepared to act promptly, there is still 1 chance of escape.
\end{example}
When the subject is the same for both clauses \& is expressed only once, a comma is useful if the connective is {\it but}.

When the connective is {\it and}, the comma should be omitted if the relation between the 2 statements is close or immediate.
\begin{example}
	I have heard the arguments, but am still unconvinced.
	
	He has had several year's experience \& is thoroughly competent.
\end{example}

%------------------------------------------------------------------------------%

\subsubsection{Do not join independent clauses with a comma}
If 2 or more clauses grammatically complete \& not joined by a conjunction are to form a single compound sentence, the proper mark of punctuation is a semicolon.
\begin{example}
	Mary Shelley's works are entertaining; they are full of engaging ideas.
	
	It is nearly half past 5; we cannot reach town before dark.
\end{example}
It is, of course, equally correct to write each of these as 2 sentences, replacing the semicolons with periods.
\begin{example}
	Mary Shelley's works are entertaining. They are full of engaging ideas.
	
	It is nearly half past 5. We cannot reach town before dark.
\end{example}
If a conjunction is inserted, the proper mark is a comma. (Rule 4.)
\begin{example}
	Mary Shelley's works are entertaining, for they are full of engaging ideas.
	
	It is nearly half past 5, \& we cannot reach town before dark.
\end{example}
A comparison of the 3 forms given above will show clearly the advantage of the 1st.

It is, at least in the examples given, better than the 2nd form because it suggests the close relationship between the 2 statements in a way that the 2nd does not attempt, \& better than the 3rd because it is briefer \& therefore more forcible.

Indeed, this simple method of indicating relationship between statements is 1 of the most useful devices of composition.

The relationship, as above, is commonly 1 of cause \& consequence.

%
Note that if the 2nd clause is preceded by an adverb, e.g., {\it accordingly, besides, then, therefore}, or {\it thus}, \& not by a conjunction, the semicolon is still required.
\begin{example}
	I had never been in the place before; besides, it was dark as a tomb.
\end{example}
An exception to the semicolon rule is worth noting here.

A comma is preferable when the clauses are very short \& alike in form, or when the tone of the sentence is easy \& conversational.
\begin{example}
	Man proposes, God disposes.
	
	The gates swung apart, the bridge fell, the portcullis was drawn up.
	
	I hardly knew him, he was so changed.
	
	Here today, gone tomorrow.
\end{example}

%------------------------------------------------------------------------------%

\subsubsection{Do not break sentences in 2}
In other words, do not use periods for commas.
\begin{example}
	I met them on a Cunard liner many years ago. Coming home from Liverpool to New York.
	
	She was an interesting talker. A woman who had traveled all over the world \& lived in half of dozen countries.
\end{example}
In both these examples, the 1st period should be replaced by a comma \& the following word begun with a small letter.

%
It is permissible to make an emphatic word or expression serve the purpose of a sentence \& to punctuate it accordingly:
\begin{example}
	Again \& again he called out. No reply.
\end{example}
The writer must, however, be certain that the emphasis is warranted, lest a clipped sentence seem merely a blunder in syntax or in punctuation.

Generally speaking, the place for broken sentences is in dialogue, when a character happens to speak in a clipped or fragmentary way.

%
Rules 3, 4, 5, \& 6 cover the most important principles that govern punctuation.

They should be so thoroughly mastered that their application becomes 2nd nature.

%------------------------------------------------------------------------------%

\subsubsection{Use a colon after an independent clause to introduce a list of particulars, an appositive, an amplification, or an illustrative quotation}
A colon tells the reader that what follows is closely related to the preceding clause.

The colon has more effect than the comma, less power to separate than the semicolon, \& more formality than the dash.

It usually follows an independent clause \& should not separate a verb from its complement or a preposition from its object.

The examples in the lefthand column, below, are wrong; they should be rewritten as in the righthand column.
\begin{example}
	Your dedicated whittler requires: a knife, a piece of wood, \& a back porch.
	
	Understanding is that penetrating quality of knowledge that grows from: theory, practice, conviction, assertion, error, \& humiliation.
	
	Your dedicated whittler requires 3 props: a knife, a piece of wood, \& a back porch.
	
	Understanding is that penetrating quality of knowledge that grows from theory, practice, conviction, assertion, error, \& humiliation.
\end{example}
Join 2 independent clauses with a colon if the 2nd interprets or amplifies the 1st.
\begin{example}
	But even so, there was a directness \& dispatch about animal burial: there was no stopover in the undertaker's foul parlor, no wreath or spray.
\end{example}
A colon may introduce a quotation that supports or contributes to the preceding clause.
\begin{example}
	The squalor of the streets reminded her of a line from Oscar Wilde: ``We are all in the gutter, but some of us are looking at the star.''
\end{example}
The colon also has certain functions of form: to follow the salutation of a former letter, to separate hour from minute in a notation of time, \& to separate the title of work from its subtitle or a Bible chapter from a verse.
\begin{example}
	Dear Mr. Montague:
	
	departs at 10:48 P.M.
	
	Practical Calligraphy: An Introduction to Italic Script
\end{example}

%------------------------------------------------------------------------------%

\subsubsection{Use a dash to set off an abrupt break or interruption \& to announce a long appositive or summary}
A dash is a mark of separation stronger than a comma, less formal than a colon, \& more relaxed than parentheses.
\begin{example}
	His 1st thought on getting out of bed - if he had any thought at all - was to get back in again.
	
	The rear axle began to make a noise - a grinding, chattering, teeth-gritting rasp.
	
	The increasing reluctance of the sun to rise, the extra nip in the breeze, the patter of shed leaves dropping - all the evidences of fall drifting into winter were clearer each day.
\end{example}
Use a dash only when a more common mark of punctuation seems inadequate.
\begin{example}
	Her father's suspicions proved well-founded - it was not Edward she cared for - it was San Francisco.
	
	$\to$ Her father's suspicions proved well-founded. It was not Edward she cared for, it was San Francisco.
	
	Violence - the kind you see on television - is not honestly violent - there lies its harm.
	
	$\to$ Violence, the kind you see on television, is not honestly violent. There lies its harm.
\end{example}

%------------------------------------------------------------------------------%

\subsubsection{The number of the subject determines the number of the verb}
Words that intervene between subject \& verb do not affect the number of the verb.
\begin{example}
	The bittersweet flavor of youth - its trials, its joys, its adventures, its challenges - are not soon forgotten.
	
	$\to$ The bittersweet flavor of youth - its trials, its joys, its adventures, its challenges - is not soon forgotten.
\end{example}
A common blunder is the use of a singular verb form in a relative clause following ``one of$\ldots$'' or a similar expression when the relative is the subject.
\begin{example}
	1 of the the ablest scientists who has attacked this problem.
	
	$\to$ 1 of the ablest scientists who have attached this problem.
	
	1 of those people who is never ready on time $\to$ 1 of those people who are never ready on time
\end{example}
Use a singular verb form after {\it each, either, everyone, everybody, neither, nobody, someone}.
\begin{example}
	Everybody thinks he has a unique sense of humor.
	
	Although both clocks strike cheerfully, neither keeps good time.
\end{example}
With {\it none}, use the singular verb when the word means ``no one'' or ``not one.''
\begin{example}
	None of us are perfect.
	
	None of us is perfect.
\end{example}
A plural verb is commonly used when {\it none} suggests more than 1 thing or person.
\begin{example}
	None are so fallible as those who are sure they're right.
\end{example}
A compound subject formed of 2 or more nouns joined by {\it and} almost always requires a plural verb.
\begin{example}
	The walrus \& the carpenter were walking close at hand.
\end{example}
But certain compounds, often cliches, are so inseparable they are considered a unit \& so take a singular verb, as do compound subjects qualified by {\it each} or {\it every}.
\begin{example}
	The long \& the short of it is$\ldots$
	
	Bread \& butter was all she served.
	
	Give \& take is essential to a happy household.
	
	Every window, picture, \& mirror was smashed.
\end{example}
A singular subject remains singular even if other nouns are connected to it by {\it with, as well as, in addition to, except, together with}, \& {\it no less than}.
\begin{example}
	His speech as well as his manner is objectionable.
\end{example}
A linking verb agrees with the number of its subject.
\begin{example}
	What is wanted is a few more pairs of hands.
	
	The trouble with truth is its many varieties.
\end{example}
Some nouns that appear to be plural are usually construed as singular \& given a singular verb.
\begin{example}
	Politics is an art, not a science.
	
	The Republican Headquarters is on this side of the tracks.
\end{example}
But
\begin{example}
	The general's quarters are across the river.
\end{example}
In these cases the writer must simply learn the idioms.

The contents of a book is singular.

The contents of a jar may be either singular or plural, depending on what's in the jar - jam or marbles.

%------------------------------------------------------------------------------%

\subsubsection{Use the proper case of pronoun}
The personal pronouns, as well as the pronoun {\it who}, change form as they function as subject or object.
\begin{example}
	Will Jane or he be hired, do you think?
	
	The culprit, it turned out, was he.
	
	We heavy eaters would rather walk than ride.
	
	Who knocks?
	
	Give this work to whoever looks idle.
\end{example}
In the last example, {\it whoever} is the subject of {\it look idle}; the object of the preposition {\it to} is the entire clause {\it whoever looks idle}.

When {\it who} introduces a subordinate clause, its case depends on its function in that clause.
\begin{example}
	Virgil Soames is the candidate whom we think will win.
	
	$\to$ Virgil Soames is the candidate who we think will win. [We think \emph{he} will win.]
	
	Virgil Soames is the candidate who we hope to elect.
	
	$\to$ Virgil Soames is the candidate whom we hope to elect. [We hope to elect \emph{him}.]
\end{example}
A pronoun in a comparison is nominative if it is the subject of a stated or understood verb.
\begin{example}
	Sandy writes better than I. (Than I write.)
\end{example}
In general, avoid ``understood'' verbs by supplying them.
\begin{example}
	I think Horace admires Jessica more than I.
	
	$\to$ I think Horace admires Jessica more than I do.
	
	Polly loves cake more than me.
	
	$\to$ Polly loves cake more than she loves me.
\end{example}
The objective case is correct in the following examples.
\begin{example}
	The ranger offered Shirley \& him some advice on campsites.
	
	They came to meet the Baldwins \& us.
	
	Let's talk it over between us, then, you \& me.
	
	Whom should I ask?
\end{example}

\begin{example}
	A group of us taxpayers protested.
\end{example}
{\it Us} in the last example is in apposition to taxpayers, the object of the proposition {\it of}.

The wording, although grammatically defensible, is rarely apt.

``A group of us protested as taxpayers'' is better, if not exactly equivalent.

%
Use the simple personal pronoun as a subject.
\begin{example}
	Blake \& myself stayed home.
	
	$\to$ Blake \& I stayed home.
	
	Howard \& yourself brought the lunch, I thought.
	
	$\to$ Howard \& you brought the lunch, I thought.
\end{example}
The possessive case of pronouns is used to show ownership.

It has 2 forms: the adjectival modifier, {\it your} hat, \& the noun form, a hat {\it of yours}.
\begin{example}
	The dog has buried 1 of your gloves \& 1 of mine in the flower bed.
\end{example}
Gerunds usually require the possessive case.
\begin{example}
	Mother objected to our driving on the icy roads.
\end{example}
A present participle as a verbal, on the other hand, takes the objective case.
\begin{example}
	They heard him singing in the shower.
\end{example}
The difference between a verbal participle \& a gerund is not always obvious, but note what is really said in each of the following.
\begin{example}
	Do you mind me asking a question?
	
	Do you mind my asking a question?
\end{example}
In the 1st sentence, the queried objection is to {\it me}, as opposed to other members of the group, asking a question.

In the 2nd example, the issue is whether a question may be asked at all.

%------------------------------------------------------------------------------%

\subsubsection{A participial phrase at the beginning of a sentence must refer to the grammatical subject}
\begin{example}
	Walking slowly down the road, he saw a woman accompanied by 2 children.
\end{example}
The word {\it walking} refers to the subject to the sentence, not to the woman.

To make it refer to the woman, the writer must recast the sentence.
\begin{example}
	He saw a woman, accompanied by 2 children, walking slowly down the road.
\end{example}
Participial phrases preceded by a conjunction or by a preposition, nouns in apposition, adjectives, \& adjective phrases come under the same rule if they begin the sentence.
\begin{example}
	On arriving in Chicago, his friends met him at the station.
	
	$\to$ On arriving in Chicago, he was met at the station by his friends.
	
	A soldier of proved valor, they entrusted him with the defense of the city.
	
	$\to$ A soldier of proved valor, he was entrusted with the defense of the city.
	
	Young \& inexperienced, the task seemed easy to me.
	
	$\to$ Young \& inexperienced, I thought the task easy.
	
	Without a friend to counsel him, the temptation proved irresistible.
	
	$\to$ Without a friend to counsel him, he found the temptation irresistible.
\end{example}
Sentences violating Rule 11 are often ludicrous:
\begin{example}
	Being in a dilapidated condition, I was able to buy the house very cheap.
	
	Wondering irresolutely what to do next, the clock struck 12.
\end{example}

%------------------------------------------------------------------------------%

\subsection{Elementary Principles of Composition}

\subsubsection{Choose a suitable design \& hold to it.}
``A basic structural design underlies every kind of writing. Writers will in part follow this design, in part deviate from it, according to their skills, their needs, \& the unexpected events that accompany the act of composition. Writing, to be effective, must follow closely the thoughts of the writer, but not necessarily in the order in which those thoughts occur. This calls for a scheme of procedure. In some cases, the best design is no design, as with a love letter, which is simply an outpouring, or with a casual essay, which is a ramble. But in most cases, planning must be a deliberate prelude to writing. The 1st principle of composition, therefore, is to foresee or determine the shape of what is to come \& pursue that shape.

A sonnet is built on a 14-line frame, each line containing 5 feet. Hence, sonneteers know exactly where they are headed, although they may not know how to get there. Most forms of composition are less clearly defined, more flexible, but all have skeletons to which the writer will bring the flesh \& the blood. The more clearly the writer perceives the shape, the better are the chances of success.'' -- \cite[p. 29]{Strunk_White_element_style}

%------------------------------------------------------------------------------%

\subsubsection{Make the paragraph the unit of composition: 1 paragraph to each topic.}
``The paragraph is a convenient unit; it serves all forms of literary work. As long as it holds together, a paragraph may be of any length -- a single, short sentence or a passage of great duration.

If the subject on which you are writing is of slight extent, or if you intend to treat it briefly, there may be no need to divide it into topics. Thus, a brief description, a brief book review, a brief account of a single incident, a narrative merely outlining an action, the setting forth of a single idea -- any 1 of these is best written in a single paragraph. After the paragraph has been written, examine it to see whether division will improve it.

Ordinarily, however, a subject requires division into topics, each of which should be dealt with in a paragraph. The object of treating each topic in a paragraph by itself is, of course, to aid the reader. The beginning of each paragraph is a signal that a new step in the development of the subject has been reached.

As a rule, single sentences should not be written or printed as paragraphs. An exception may be made of sentences of transition, indicating the relation between the parts of an exposition or argument.

In dialogue, each speech, even if only a single word, is usually a paragraph by itself; i.e., a new paragraph begins with each change of speaker. The application of this rule when dialogue \& narrative are combined is best learned from examples in well-edited works of fiction. Sometimes a writer, seeking to create an effect of rapid talk or for some other reason, will elect not to set off each speech in a separate paragraph \& instead will run speeches together. The common practice, however, \& the one that serves best in most instances, is to give each speech a paragraph of its own.

As a rule, begin each paragraph either with a sentence that suggests the topic or with a sentence that helps the transition. If a paragraph forms part of a larger composition, its relation to what precedes, or its function as a part of the whole, may need to be expressed. This can sometimes be done by a mere word or phrase ({\it again, therefore, for the same reason}) in the 1st sentence. Sometimes, however, it is expedient to get into the topic slowly, by way of a sentence or 2 of introduction or transition.

In narration \& description, the paragraph sometimes begins with a concise, comprehensive statement serving to hold together the details that follows.
\begin{quotation}\it
	The breeze served us admirably.
	
	The campaign opened with a series of reverses.
	
	The next 10 or 12 pages were filled with a curious set of entries.
\end{quotation}
But when this device, or any device, is too often used, it becomes a mannerism. More commonly, the opening sentence simply indicates by its subject the direction the paragraph is to take.
\begin{quotation}\it
	At length I thought I might return toward the stockade.
	
	He picked up the heavy lamp from the table \& began to explore.
	
	Another flight of steps, \& they emerged on the roof.
\end{quotation}
In animated narrative, the paragraphs are likely to be short \& without any semblance of a topic sentence, the writer rushing headlong, event following event in rapid succession. The break between such paragraphs merely serves the purpose of a rhetorical pause, throwing into prominence some detail of the action.

In general, remember that paragraphing calls for a good eye as well as a logical mind. Enormous blocks of print look formidable to readers, who are often reluctant to tackle them. Therefore, breaking long paragraphs in 2, even if it is not necessary to do so for sense, meaning, or logical development, is often a visual help. But remember, too, that firing off many short paragraphs in quick succession can be distracting. Paragraph breaks used only for show read like the writing of commerce or of display advertising. Moderation \& a sense of order should be the main considerations in paragraphing.'' -- \cite[pp. 30--31]{Strunk_White_element_style}

%------------------------------------------------------------------------------%

\subsubsection{Use the active voice.}
``The active voice is usually more direct \& vigorous than the passive:
\begin{example}
	I shall always remember my 1st visit to Boston.
\end{example}
This is much better than
\begin{example}
	My 1st visit to Boston will always be remembered by me.
\end{example}
The latter sentence is less direct, less bold, \& less concise. If the writer tries to make it more concise by omitting ``by me,''
\begin{example}
	My 1st visit to Boston will always be remembered,
\end{example}
it becomes indefinite: is it the writer or some undisclosed person or the world at large that will always remember this visit?

This rule does not, of course, mean that the writer should entirely discard the passive voice, which is frequently convenient \& sometimes necessary.
\begin{example}
	The dramatists of the Restoration are little esteemed today.
	
	Modern readers have little esteem for the dramatists of the Restoration.
\end{example}
The 1st would be the preferred form in a paragraph on the dramatists of the Restoration, the 2nd in a paragraph on the tastes of modern readers. The need to make a particular word the subject of the sentence will often, as in these examples, determine which voice is to be used.

The habitual use of the active voice, however, makes for forcible writing. This is true not only in narrative concerned principally with action but in writing of any kind. Many a tame sentence of description or exposition can be made lively \& emphatic by substituting a transitive in the active voice for some such perfunctory expression as {\it there is} or {\it could be heard}.
\begin{example}
	There were a great number of dead leaves lying on the ground. $\to$ Dead leaves covered the ground.
	
	At dawn the crowing of a rooster could be heard. $\to$ The cock's crow came with dawn.
	
	The reason he left college was that his health became impaired. $\to$ Failing health compelled him to leave college.
	
	It was not long before she was very sorry that she had said what she had. $\to$ She soon repented her words.
\end{example}
Note, in the examples above, that when a sentence is made stronger, it usually becomes shorter. Thus, brevity is a by-product of vigor.'' -- \cite[p. 32]{Strunk_White_element_style}

%------------------------------------------------------------------------------%

\subsubsection{Put statements in positive form.}
``Make definite assertions. Avoid tame, colorless, hesitating, noncommittal language. Use the word {\it not} as a means of denial or in antithesis, never as a means of evasion.
\begin{example}
	He was not very often on time. $\to$ He usually came late.
	
	She did not think that studying Latin was a sensible way to use one's time. $\to$ She thought the study of Latin a waste of time.
	
	\emph{The Taming of the Shrew} is rather weak in spots. Shakespeare does not portray Katharine as a very admirable character, nor does Bianca remain long in memory as an important character in Shakespeare's works. $\to$ The women in \emph{The Taming of the Shrew} are unattractive. Katharine is disagreeable, Bianca insignificant.
\end{example}
The last example, before correction, is indefinite as well as negative. The corrected version, consequently, is simply a guess at the writer's intention.

All 3 examples show the weakness inherent in the word {\it not}. Consciously or unconsciously, the reader is dissatisfied with being told only what is not; the reader wishes to be told what is. Hence, as a rule, it is better to express even a negative in positive form.
\begin{example}
	not honest $\to$ dishonest, not important $\to$ trifling, did not remember $\to$ forgot, did not pay any attention to $\to$ ignored, did not have much confidence in $\to$ distrusted.
\end{example}
Placing negative \& positive in opposition makes for a stronger structure.
\begin{example}
	Not charity, but simple justice.
	
	Not that I loved Caesar less, but that I loved Rome more.
	
	Ask not what your country can do for you -- ask what you can do for your country.
\end{example}
Negative words other than {\it not} are usually strong.
\begin{example}
	Her loveliness I never knew
	
	Until she smiled on me.
\end{example}
Statements qualified with unnecessary auxiliaries or conditionals sound irresolute.
\begin{example}
	If you would let us know the time of your arrival, we would be happy to arrange your transportation from the airport. $\to$ If you will let us know the time of your arrival, we shall be happy to arrange your transportation from the airport.
	
	Applicants can make a good impression by being neat \& punctual. $\to$  Applicants will make a good impression if they are neat \& punctual.
	
	Plath may be ranked among those modem poets who died young. $\to$ Plath was one of those modern poets who died young.
\end{example}
If your every sentence admits a doubt, your writing will lack authority. Save the auxiliaries {\it would, should, could, may, might}, \& {\it can} for situations involving real uncertainty.'' -- \cite[pp. 33--34]{Strunk_White_element_style}

%------------------------------------------------------------------------------%

\subsubsection{Use definite, specific, concrete  language.}
``Prefer the specific to the general, the definite to the vague, the concrete to the abstract.
\begin{example}
	A period of unfavorable weather set in. $\to$ It rained every day for a week.
	
	He showed satisfaction as he took possession of his well-earned reward. $\to$ He grinned as he pocketed the coin.
\end{example}
If those who have studied the art of writing are in accord on any 1 point, it is this: the surest way to arouse \& hold the readers attention is by being specific, definite, \& concrete. The greatest writers -- Homer, Dante, Shakespeare -- are effective largely because they deal in particulars \& report the details that matter. Their words call up pictures.

Jean Stafford, to cite a more modern author, demonstrates in her short story ``In the Zoo'' how prose is made vivid by the use of words that evoke images \& sensations:
\begin{example}
	$\ldots$ Daisy \& I in time found asylum in a small menagerie down by the railroad tracks. It belonged to a gentle alcoholic ne'er-do- well, who did nothing all day long but drink bathtub gin in rickeys \&  play solitaire \&  smile to himself \&  talk to his animals. He had a little, stunted red vixen \&  a deodorized skunk, a parrot from Tahiti that spoke Parisian French, a woebegone coyote, \&  two capuchin monkeys, so serious \&  humanized, so small \&  sad \&  sweet, \&  so religious-looking with their tonsured heads that it was impossible not to think their gibberish was really an ordered language with a grammar that someday some philologist would understand.
	
	Gran knew about our visits to Mr. Murphy \&  she did not object, for it gave her keen pleasure to excoriate him when we came home. His vice was not a matter of guesswork; it was an established fact that he was half-seas over from dawn till midnight. ``With the black Irish,'' said Gran, ``the taste for drink is taken in with the mother's milk \&  is never mastered. Oh, I know all about those promises to join the temperance movement \&  not to touch another drop. The way to Hell is paved with good intentions.'' -- Excerpt from ``In the Zoo'' from Bad Characters by Jean Stafford.
\end{example}
If the experiences of Walter Mitty, of Molly Bloom, of Rabbit Angstrom have seemed for the moment real to countless readers, if in reading Faulkner we have almost the sense of inhabiting Yoknapatawpha County during the decline of the South, it is because the details used are definite, the terms concrete. It is not that every detail is given -- that would be impossible, as well as to no purpose -- but that all the significant details are given, \&  with such accuracy \&  vigor that readers, in imagination, can project themselves into the scene.

In exposition \&  in argument, the writer must likewise never lose hold of the concrete; \&  even when dealing with general principles, the writer must furnish particular instances of their application.

In his {\it Philosophy of Style}, Herbert Spencer gives 2 sentences to illustrate how the vague \&  general can be turned into the vivid \&  particular:
\begin{example}
	In proportion as the manners, customs, \&  amusements of a nation are cruel \&  barbarous, the regulations of their penal code will be severe. $\to$ In proportion as men delight in battles, bullfights, \&  combats of gladiators, will they punish by hanging, burning, \&  the rack.
\end{example}
To show what happens when strong writing is deprived of its vigor, George Orwell once took a passage from the Bible \&  drained it of its blood. On the left, below, is Orwell’s translation; on the right, the verse from Ecclesiastes (King James Version).
\begin{example}
	Objective consideration of contemporary phenomena compels the conclusion that success or failure in competitive activities exhibits no tendency to be commensurate with innate capacity, but that a considerable element of the unpredictable must inevitably be taken into account. $\to$ I returned, \&  saw under the sun, that the race is not to the swift, nor the battle to the strong, neither yet bread to the wise, nor yet riches to men of understanding, nor yet favor to men of skill; but time \&  chance happeneth to them all.'' -- \cite[pp. 35--36]{Strunk_White_element_style}
\end{example}

%------------------------------------------------------------------------------%

\subsubsection{Omit needless words.}
``Vigorous writing is concise. A sentence should contain no unnecessary words, a paragraph no unnecessary sentences, for the same reason that a drawing should have no unnecessary lines \&  a machine no unnecessary parts. This requires not that the writer make all sentences short, or avoid all detail \&  treat subjects only in outline, but that every word tell.

Many expressions in common use violate this principle.
\begin{example}
	the question as to whether $\to$ whether (the question whether), there is no doubt but that $\to$ no doubt (doubtless), used for fuel purposes $\to$ used for fuel, he is a man who $\to$ he, in a hasty manner $\to$ hastily, this is a subject that $\to$ this subject, Her story is a strange one. $\to$ Her story is strange. the reason why is that $\to$ because.
\end{example}
{\it The fact that} is an especially debilitating expression. It should be revised out of every sentence in which it occurs.
\begin{example}
	owing to the fact that $\to$ since (because), in spite of the fact that $\to$ though (although), call your attention to the fact that $\to$ remind you (notify you), I was unaware of the fact that $\to$ I was unaware that (did not know), the fact that he had not succeeded $\to$ his failure, the fact that I had arrived $\to$ my arrival.
\end{example}
See also the words {\it case, character, nature} in Chap. IV. {\it Who is, which was}, \&  the like are often superfluous.
\begin{example}
	His cousin, who is a member of the same firm $\to$ His cousin, a member of the same firm
	
	Trafalgar, which was Nelson's last battle $\to$ Trafalgar, Nelson’s last battle.
\end{example}
As the active voice is more concise than the passive, \&  a positive statement more concise than a negative one, many of the examples given under Rules 14 \&  15 illustrate this rule as well.

A common way to fall into wordiness is to present a single complex idea, step by step, in a series of sentences that might to advantage be combined into one.
\begin{example}
	Macbeth was very ambitious. This led him to wish to become king of Scotland. The witches told him that this wish of his would come true. The king of Scotland at this time was Duncan. Encouraged by his wife, Macbeth murdered Duncan. He was thus enabled to succeed Duncan as king. (51 words)
	
	$to$ Encouraged by his wife, Macbeth achieved his ambition \&  realized the prediction of the witches by murdering Duncan \&  becoming king of Scotland in his place. (26 words)'' -- \cite[pp. 37--38]{Strunk_White_element_style}
\end{example}

%------------------------------------------------------------------------------%

\subsubsection{Avoid a succession of loose sentences.}
``This rule refers especially to loose sentences of a particular type: those consisting of 2 clauses, the 2nd introduced by a conjunction or relative. A writer may err by making sentences too compact \& periodic. An occasional loose sentence prevents the style from becoming too formal \& gives the reader a certain relief. Consequently, loose sentences are common in easy, unstudied writing. The danger is that there may be too many of them.

An unskilled writer will sometimes construct a whole paragraph of sentences of this kind, using as connectives {\it and, but}, and, less frequently, {\it who, which, when, where}, \& {\it while}, these last in nonrestrictive senses. (See Rule 3.)
\begin{example}
	The 3rd concert of the subscription series was given last evening, \& a large audience was in attendance. Mr. Edward Appleton was the soloist, \& the Boston Symphony Orchestra furnished the instrumental music. The former showed himself to be an artist of the 1st rank, while the latter proved itself fully deserving of its high reputation. The interest aroused by the series has been very gratifying to the Committee, \& it is planned to give a similar series annually hereafter. The 4th concert will be given on Tuesday, May 10, when an equally attractive program will be presented.
\end{example}
Apart from its triteness \& emptiness, the paragraph above is bad because of the structure of its sentences, with their mechanical symmetry \& singsong. Compare these sentences from the chapter ``What I Believe'' in E. M. Forster's {\it 2 Cheers for Democracy}:
\begin{example}
	I believe in aristocracy, though -- if that is the right word, \& if a democrat may use it. Not an aristocracy of power, based upon rank \& influence, but an aristocracy of the sensitive, the considerate \& the plucky. Its members are to be found in all nations \& classes, \& all through the ages, \& there is a secret understanding between them when they meet. They represent the true human tradition, the 1 permanent victory of our queer race over cruelty \& chaos. Thousands of them perish in obscurity, a few are great names. They are sensitive for others as well as for themselves, they are considerate without being fussy, their pluck is not swankiness but the power to endure, \& they can take a joke.
\end{example}
A writer who has written a series of loose sentences should recast enough of them to remove the monotony, replacing them with simple sentences, sentences of 2 clauses joined by a semicolon, periodic sentences of 2 clauses, or sentences (loose or periodic) of 3 clauses -- whichever best represent the real relations of the thought.'' -- \cite[pp. 39--40]{Strunk_White_element_style}

%------------------------------------------------------------------------------%

\subsubsection{Express coordinate ideas in similar form.}
``This principle, that of parallel construction, requires that expressions similar in content \& function be outwardly similar. The likeness of form enables the reader to recognize more readily the likeness of content \& function. The familiar Beatitudes exemplify the virtue of parallel construction.
\begin{quotation}
	Blessed are the poor in spirit: for theirs is the kingdom of heaven.
	
	Blessed are they that mourn: for they shall be comforted.
	
	Blessed are the meek: for they shall inherit the earth.
	
	Blessed are they which do hunger \& thirst after righteousness: for they shall be filled.
\end{quotation}
The unskilled writer often violates this principle, mistakenly believing in the value of constantly varying the form of expression. When repeating a statement to emphasize it, the writer may need to vary its form. Otherwise, the writer should follow the principle of parallel construction.
\begin{example}
	Formerly, science was taught by the textbook method, while now the laboratory method is employed. $\to$ Formerly, science was taught by the textbook method; now it is taught by the laboratory method.
\end{example}
The lefthand version gives the impression that the writer is undecided or timid, apparently unable or afraid to choose one form of expression \& hold to it. The righthand version shows that the writer has at least made a choice \& abided by it.

By this principle, an article or a preposition applying to all the members of a series must either be used only before the first term or else be repeated before each term.
\begin{example}
	The French, the Italians, Spanish, \& Portuguese $\to$ The French, the Italians, the Spanish, \& the Portuguese
	
	In spring, summer, or in winter $\to$ In spring, summer, or winter (In spring, in summer, or in winter).
\end{example}
Some words require a particular preposition in certain idiomatic uses. When such words are joined in a compound construction, all the appropriate prepositions must be included, unless they are the same.
\begin{example}
	His speech was marked by disagreement \& scorn for his opponent's position. $\to$ His speech was marked by disagreement with \& scorn for his opponent's position.
\end{example}
Correlative expressions ({\it both, \&; not, but; not only, but also; either, or; 1st, 2nd, 3rd}; \& the like) should be followed by the same grammatical construction. Many violations of this rule can be corrected by rearranging the sentence.
\begin{example}
	It was both a long ceremony \& very tedious. $\to$ The ceremony was both long \& tedious.
	
	A time for not words but action. $\to$ A time not for words but for action.
	
	Either you must grant his request or incur his ill will. $\to$ You must either grant his request or incur his ill will.
	
	My objections are, 1st, the injustice of the measure; 2nd, that it is unconstitutional. $\to$ My objections are, 1st, that the measure is unjust; 2nd, that it is unconstitutional.
\end{example}
It may be asked, what if you need to express a rather large number of similar ideas -- say, 20? Must you write 20 consecutive sentences of the same pattern? On closer examination, you will probably find that the difficulty is imaginary -- that these 20 ideas can be classified in groups, \& that you need apply the principle only within each group. Otherwise, it is best to avoid the difficulty by putting statements in the form of a table.'' -- \cite[pp. 41--42]{Strunk_White_element_style}

%------------------------------------------------------------------------------%

\subsubsection{Keep related words together.}
The position of the words in a sentence is the principal means of showing their relationship.

Confusion \& ambiguity result when words are badly placed.

The writer must, therefore, bring together the words \& groups of words that are related in thought \& keep apart those that are not so related.
\begin{example}
	He noticed a large stain in the rug that was right in the center.
	
	$\to$ He noticed a large stain right in the center of the rug.
	
	You can call your mother in London \& tell her all about George's taking you out to dinner for just 2 dollars.
	
	$\to$ For just 2 dollars you can call your mother in London \& tell her all about George's taking you out to dinner.
	
	New York's 1st commercial human-sperm bank opened Friday with semen samples from 18 men frozen in a stainless steel tank.
	
	$\to$ New York's 1st commercial human-sperm bank opened Friday when semen samples were taken from 18 men.
	
	The samples were then frozen \& stored in a stainless steel tank.
\end{example}
In the lefthand version of the 1st example, the reader has no way of knowing whether the stain was in the center of the rug or the rug was in the center of the room.

In the lefthand version of the 2nd example, the reader may well wonder which cost 2 dollars - the phone call or the dinner.

In the lefthand version of the 3rd example, the reader's heart goes out to those 18 poor fellows frozen in a steel tank.

%
The subject of a sentence \& the principal verb should not, as a rule, be separated by a phrase or clause that can be transferred to the beginning.
\begin{example}
	Toni Morrison, in \emph{Beloved}, writes about characters who have escaped from slavery but are haunted by its heritage.
	
	$\to$ In \emph{Beloved}, Toni Morrison writes about characters who have escaped from slavery but are haunted by its heritage.
	
	A dog, if you fail to discipline him, becomes a household pest.
	
	$\to$ Unless disciplined, a dog becomes a household pest.
\end{example}
Interposing a phrase or a clause, as in the lefthand examples above, interrupts the flow of the main clause.

This interruption, however, is not usually bothersome when the flow is checked only by a relative clause or by an expression in apposition.

Sometimes, in periodic sentences, the interruption is a deliberate device for creating suspense. (See example under Rule 22.)

%
The relative pronoun should come, it most instances, immediately after its antecedent.
\begin{example}
	There was a stir in the audience that suggested disapproval.
	
	$\to$ A stir that suggested disapproval swept the audience.
	
	He wrote 3 articles about his adventures in Spain, which were published in \emph{Harper's Magazine}.
	
	$\to$ He published 3 articles in \emph{Harper's Magazine} about his adventures in Spain.
	
	This is a portrait of Benjamin Harrison, who became President in 1889.
	
	He was the grandson of William Henry Harrison.
	
	$\to$ This is a portrait of Benjamin Harrison, grandson of William Henry Harrison, who became President in 1889.
\end{example}
If the antecedent consists of a group of words, the relative comes at the end of the group, unless this would cause ambiguity.
\begin{example}
	The Superintendent of the Chicago Division, who
\end{example}
No ambiguity results from the above.

But
\begin{example}
	A proposal to amend the Sherman Act, which has been variously judged
\end{example}
leaves the reader wondering whether it is the proposal or the Act that has been variously judged.

The relative clause must be moved forward, to read, ``A proposal, which has been variously judged, to amend the Sherman Act$\ldots$''.

Similarly
\begin{example}
	The grandson of William Henry Harrison, who $\to$ William Henry Harrison's grandson, Benjamin Harrison, who
\end{example}
A noun in apposition may come between antecedent \& relative, because in such a combination no real ambiguity can arise.
\begin{example}
	The Duke of York, his brother, who was regarded with hostility by the Whigs
\end{example}
Modifiers should come, if possible, next to the words they modify.

If several expressions modify the same word, they should be ranged so that no wrong relation is suggested.
\begin{example}
	All the members were not present.
	
	$\to$ Not all the members were present.
	
	She only found 2 mistakes.
	
	$\to$ She found only 2 mistakes.
	
	The director said he hoped all members would give generously to the Fund at a meeting of the committee yesterday.
	
	$\to$ At a meeting of the committee yesterday, the director said he hoped all members would give generously to the Fund.
	
	Major R. E. Joyce will give a lecture on Tuesday evening in Bailey Hall, to which the public is invited on ``My Experiences in Mesopotamia'' at 8:00 P.M.
	
	$\to$ On Tuesday evening at 8, Major R. E. Joyce will give a lecture in Bailey Hall on ``My Experiences in Mesopotamia.''
	
	The public is invited.
\end{example}
Note, in the last lefthand example, how swiftly meaning departs when words are wrongly juxtaposed.

%------------------------------------------------------------------------------%

\subsubsection{In summaries, keep to 1 tense.}
In summarizing the action of a drama, use the present tense.

In summarizing a poem, story, or novel, also use the present, though you may use the past if it seems more natural to do so.

If the summary is in the present tense, antecedent action should be expressed by the perfect; if in the past, by the past perfect.
\begin{example}
	Chance prevents Friar John from delivering Friar Lawrence's letter to Romeo.
	
	Meanwhile, owing to her father's arbitrary change of the day set for her wedding, Juliet has been compelled to drink the potion on Tuesday night, with the result that Balthasar informs Romeo of her supposed death before Friar Lawrence learns of the nondelivery of the letter.
\end{example}
But whichever tense is used in the summary, a past tense in indirect discourse or in indirect question remains unchanged.
\begin{example}
	The Friar confessed that it was he who married them.
\end{example}
Apart from the exceptions noted, the writer should use the same tense throughout.

Shifting from 1 tense to another gives the appearance of uncertainty \& irresolution.

%
In presenting the statements or the thought of someone else, as in summarizing an essay or reporting a speech, do not overwork such expressions as ``he said,'' ``she stated,'' ``the speaker added,'' ``the speaker then went on to say,'' ``the author also thinks.''

Indicate clearly at the outset, once for all, that what follows is summary, \& then waste no words in repeating the notification.

%
In notebooks, in newspapers, in handbooks of literature, summaries of 1 kind or another be indispensable, \& for children in primary schools retelling a story in their own words is a useful exercise.

But in the criticism or interpretation of literature, be careful to avoid dropping into summary.

It may be necessary to devote 1 or 2 sentences to indicating the subject, or the opening situation, of the work being discussed, or to cite numerous details to illustrate its qualities.

But you should aim at writing an orderly discussion supported by evidence, not a summary with occasional comment.

Similarly, if the scope of the discussion includes a number of works, as a rule it is better not to take them up singly in chronological order but to aim from the beginning at establishing general conclusions.

%------------------------------------------------------------------------------%

\subsubsection{Place the emphatic words of a sentence at the end.}
The proper place in the sentence for the word or group of words that the writer desires to make most prominent is usually the end.
\begin{example}
	Humanity has hardly advanced in fortitude since that time, though it has advanced in many other ways.
	
	$\to$ Since that time, humanity has advanced in many ways, but it has hardly advanced in fortitude.
	
	This steel is principally used for making razors, because of its hardness.
	
	$\to$ Because of its hardness, this steel is used principally or making razors.
\end{example}
The word or group of words entitled to this position of prominence is usually the logical predicate - i.e., the {\it new} element in the sentence, as it is in the 2nd example.

The effectiveness of the periodic sentence arises from the prominence it gives to the main statement.
\begin{example}
	4 centuries ago, Christopher Columbus, 1 of the Italian mariners whom the decline of their own republics had put at the service of the world \& of adventure, seeking for Spain a westward passage to the Indies to offset the achievement of Portuguese discoverers, lighted on America.
	
	With these hopes \& in this belief I would urge you, laying aside all hindrance, thrusting away all private aims, to devote yourself unswervingly to the vigorous \& successful prosecution of this war.
\end{example}
The other prominent position in the sentence is the beginning.

Any element in the sentence other than the subject becomes emphatic when placed 1st.
\begin{example}
	Deceit or treachery she could never forgive.
	
	Vast \& rude, fretted by the action of nearly 3000 years, the fragments of this architecture may often seem, at 1st sight, like works of nature.
	
	Home is the sailor.
\end{example}
A subject coming 1st in its sentence may be emphatic, but hardly by its position alone.

In the sentence
\begin{example}
	Great kings worshiped at his shrine
\end{example}
the emphasis upon {\it kings} arises largely from its meaning \& from the context.

To receive special emphasis, the subject of a sentence must take the position of the predicate.
\begin{example}
	Through the middle of the valley flowed a winding stream.
\end{example}
The principle that the proper place for what is to be made most prominent is the end applies equally to the words of a sentence, to the sentences of a paragraph, \& to the paragraphs of a composition.

%------------------------------------------------------------------------------%

\subsection{A Few Matters of Form}

\begin{enumerate}
	\item {\bf Colloquialisms.} If you use a colloquialism or a slang word or phrase, simply use it; do not draw attention to it by enclosing it in quotation marks.
	
	To do so is to put on airs, as through you were inviting the reader to join you in a select society of those who know better.
	\item {\bf Exclamations.} Do not attempt to emphasize simple statements by using a mark of exclamation.
	\begin{example}
		It was wonderful show!
		
		$\to$ It was a wonderful show.
	\end{example}
	The exclamation mark is to be reserved for use after true exclamations or commands.
	\begin{example}
		What a wonderful show!
		
		Halt!
	\end{example}
	\item {\bf Headings.} If a manuscript is to be submitted for publication, leave plenty of space at the top of page 1.
	
	The editor will need this space to write directions to the compositor.
	
	Place the heading, or title, at least a 4th of the way down the page.
	
	Leave a blank line, or its equivalent in space, after the heading.
	
	On succeeding pages, begin near the top, but not so near as to give a crowded appearance.
	
	Omit the period after a title or heading.
	
	A question mark or an exclamation point may be used if the heading calls for it.
	\item {\bf Hyphen.} When 2 or more words are combined to form a compound adjective, a hyphen is usually required.
	\begin{example}
		``He belonged to the leisure class \& enjoyed leisure-class pursuits.''
		
		``She entered her boat in the round-the-island race.''
	\end{example}
	Do not use a hyphen between words that can better be written as 1 word: {\it water-fowl, waterfowl}.
	
	Common sense will aid you in the decision, but a dictionary is more reliable.
	
	The steady evolution of the language seems to favor union: 2 words eventually become one, usually after a period of hyphenation.
	\begin{example}
		bed chamber $\to$ bed-chamber $\to$ bedchamber
		
		wild life $\to$ wild-life $\to$ wildlife
		
		bell boy $\to$ bell-boy $\to$ bellboy
	\end{example}
	The hyphen can play tricks on the unwary, as it did it Chattanooga when 2 newspapers merged - the {\it News} \& the {\it Free Press}.
	
	Someone introduced a hyphen into the merger, \& the paper became {\it The Chattanooga News-Free Press}, which sounds as though the paper were news-free, or devoid of news.
	
	Obviously, we ask too much of a hyphen when we ask it to cast its spell over words it does not adjoint.
	\item {\bf Margins.} Keep righthand \& lefthand margins roughly the same width.
	
	\begin{remark}[Exception]
		If a great deal of annotating or editing is anticipated, the lefthand margin should be roomy enough to accommodate this work.
	\end{remark}
	\item {\bf Numerals.} Do not spell out dates or other serial numbers.
	
	Write them in figures or in Roman notation, as appropriate.
	\begin{example}
		August 9, 1988
		
		Part XII
		
		Rule 3
		
		352d Infantry
	\end{example}
	
	\begin{remark}[Exception]
		When they occur in dialogue, most dates \& numbers are best spelled out.
		\begin{example}
			``I arrived home on August ninth.''
			
			``In the year 1990, I turned twenty-one.''
			
			``Read Chapter Twelve.''
		\end{example}
	\end{remark}
	\item {\bf Parentheses.} A sentence containing an expression in parentheses is punctuated outside the last mark of parenthesis exactly as if the parenthetical expression were absent.
	
	The expression within the marks is punctuated as if it stood by itself, except that the final stop is omitted unless it is question mark or an exclamation point.
	\begin{example}
		I went to her house yesterday (my 3rd attempt to see her), but she had left town.
		
		He declares (and why should we doubt his good faith?) that he is now certain of success.
	\end{example}
	(When a wholly detached expression or sentence is parenthesized, the final stop comes before the last mark of parenthesis.)
	\item {\bf Quotations.} Formal quotations cited as documentary evidence are introduced by a colon \& enclosed in quotation marks.
	\begin{example}
		The United States Coast Pilot has this to say of the place: ``Bracy Cove, 0.5 mile eastward of Bear Island, is exposed to southeast winds, has a rocky \& uneven bottom, \& is unfit for anchorage.''
	\end{example}
	A quotation grammatically in apposition or the direct object of a verb is preceded by a comma \& enclosed in quotation marks.
	\begin{example}
		I am reminded of the advice of my neighbor, ``Never worry about your heart till it stops beating.''
		
		Mark Twain says, ``{\bf A classic is something that everybody wants to have read \& nobody wants to read.}''
	\end{example}
	When a quotation is followed by an attributive phrase, the comma is enclosed within the quotation marks.
	\begin{example}
		``I can't attend,'' she said.
	\end{example}
	Typographical usage dictates that the comma be inside the marks, though logically it often seems not to belong there.
	\begin{example}
		``The Fish,'' ``Poetry,'' \& ``The Monkeys'' are in Marianne Moore's Selected Poems.
	\end{example}
	When quotations of an entire line, or more, of either verse or prose are to be distinguished typographically from text matter, as are the quotations in this book, begin on a fresh line \& indent.
	
	Quotation marks should not be used unless they appear in the original, as in dialogue.
	\begin{example}
		Wordworth's enthusiasm for the French Revolution was at 1st unbounded:
		
		Bliss was it in that dawn to be alive,
		
		But to be young was very heaven!
	\end{example}
	Quotations introduced by {\it that} are indirect discourse \& not enclosed in quotation marks.
	\begin{example}
		Keats declares that beauty is truth, truth beauty.
		
		Dickinson states that a coffin is a small domain.
	\end{example}
	Proverbial expressions \& familiar phrases of literary origin require no quotation marks.
	\begin{example}
		These are the times that try men's souls.
		
		He lives far from the madding crowd.
	\end{example}
	\item {\bf References.} In scholarly work requiring exact references, abbreviate titles that occur frequently, giving the full forms in an alphabetical list at the end.
	
	As a general practice, give the references in parentheses or in footnotes, not in the body of the sentence.
	
	Omit the words {\it act, scene, line, book, volume, page}, except when referring to only 1 of them.
	
	Punctuate as indicated below.
	\begin{example}
		in the 2nd scene of the 3rd act $\to$ in III.ii (Better still, simply insert m.ii in parenthesis at the proper place in the sentence.)
		
		After the killing of Polonius, Hamlet is placed under guard (IV.ii.14).
		
		2 Samuel i: 17--27
		
		Othello II.iii. 264--267, III.iii. 155--161
	\end{example}
	\item {\bf Syllabication.} When a word must be divided at the end of a line, consult a dictionary to learn the syllables between which division should be made.
	
	The student will do well to examine the syllable division in a number of pages of any carefully printed book.
	\item {\bf Titles.} For the titles of literary works, scholarly usage prefers italics with capitalized initials.
	
	The usage of editors \& publishers varies, some using italics with capitalized initials, others using Roman with capitalized initials \& with or without quotation marks.
	
	Use italics (indicated in manuscript by underscoring) except in writing for a periodical that follows a different practice.
	
	Omit initial {\it A} or {\it The} from titles when you place the possessive before them.
	\begin{example}
		\emph{A} Tale of 2 Cities; \emph{Dickens's} Tale of 2 Cities.
		
		The Age of Innocence; \emph{Wharton's} Age of Innocence.
	\end{example}
\end{enumerate}

%------------------------------------------------------------------------------%

\subsection{Words \& Expressions Commonly Misused}
Many of the words \& expressions listed here are not so much bad English as bad style, the commonplaces of careless writing.

As illustrated under {\it Feature}, the proper correction is likely to be not the replacement of 1 word or set of words by another but the {\it replacement of vague generality by definite statement}.

%
The shape of our language is not rigid; in questions of usage we have no lawgiver whose word is final.

Students whose curiosity is aroused by the interpretations that follow, or whose doubts are raised, will wish to pursue their investigations further.

Books useful in such pursuits are {\it Merriam Webster's Collegiate Dictionary}, 10th Edition; {\it The American Heritage Dictionary of the English Language}, 3rd Edition; {\it Webster's 3rd New International Dictionary; The New Fowler's Modern English Usage}, 3rd Edition, edited by R. W. Burchfield; {\it Modern American Usage: A Guide} by Wilson Follett \& Erik Wensberg; \& {\it The Careful Writer} by Theodore M. Bernstein.
\begin{enumerate}
	\item {\bf Aggravate. Irritate.} The 1st means ``to add to'' an already troublesome or vexing matter or condition.
	
	The 2nd means ``to vex'' or ``to annoy'' or ``to chafe.''
	\item {\bf All right.} Idiomatic in familiar speech as a detached phrase in the sense ``Agreed,'' or ``Go ahead,'' or ``O.K.''
	
	Properly written as 2 words - {\it all right}.
	\item {\bf Allude.} Do not confuse with {\it elude}.
	
	You {\it allude} to a book; you {\it elude} a pursuer.
	
	Note, too, that {\it allude} is not synonymous with {\it refer}.
	
	An allusion is an indirect mention, a reference is a specific one.
	\item {\bf Allusion.} Easily confused with {\it illusion}.
	
	The 1st means ``an indirect reference''; the 2nd means ``an unreal image'' or ``a false impression.''
	\item {\bf Alternate. Alternative.} The words are not always interchangeable as nouns or adjectives.
	
	The 1st means every other one in a series; the 2nd, 1 of 2 possibilities.
	
	As the other one of a series of 2, an {\it alternate} may stand for ``a substitute,'' but an {\it alternative}, although used in a similar sense, connotes a matter of choice that is never present with {\it alternate}.
	\begin{example}
		As the flooded road left them no alternative, they took the alternate route.
	\end{example}
	\item {\bf Among. Between.} When more than 2 things or persons are involved, {\it among} is usually called for: ``The money was divided among the 4 players.''
	
	When, however, more than 2 are involved but each is considered  individually, {\it between} is preferred: ``an agreement between the 6 heirs.''
	\item {\bf And/or.} A device, or shortcut, that damages a sentence \& often leads to confusion or ambiguity.
	\begin{example}
		1st of all, would an honor system successfully cut down on the amount of stealing and/or cheating?
		
		$\to$ 1st of all, would an honor system reduce the incidence of stealing or cheating or both?
	\end{example}
	\item {\bf Anticipate.} Use {\it expect} in the sense of simple expectation.
	\begin{example}
		I anticipated that he would look older.
		
		$\to$ I expected that he would look older.
		
		My brother anticipated the upturn in the market.
		
		$\to$ My brother expected the upturn in the market.
	\end{example}
	In the 2nd example, the word {\it anticipated} is ambiguous.
	
	It could mean simply that the brother believed the upturn would occur, or it could mean that he acted in advance of the expected upturn - by buying stock, perhaps.
	\item {\bf Anybody.} In the sense of ``any person,'' not to be written as 2 words.
	
	{\it Anybody.} In the sense of ``any person,'' not to be written as 2 words.
	
	{\it Any body} means ``any corpse,'' or ``any human form,'' or ``any group.''
	
	The rule holds equally for {\it everybody, nobody}, \& {\it somebody}.
	\item {\bf Anyone.} In the sense of ``anybody,'' written as 1 word.
	
	{\it Any one} means ``any single person'' or ``any single thing.''
	\item {\bf As good or better than.} Expressions of this type should be corrected by rearranging the sentences.
	\begin{example}
		My opinion is as good or better than his.
		
		$\to$ My opinion is as good as his, or better (if not better).
	\end{example}
	\item {\bf As to whether.} {\it Whether} is sufficient.
	\item {\bf As yet.} {\it Yet} nearly always is as good, if not better.
	\begin{example}
		No agreement has been reached as yet.
		
		$\to$ No agreement has yet been reached.
	\end{example}
	The chief exception is at the beginning of a sentence, where {\it yet} means something different.
	\begin{example}
		Yet (\emph{or} despite everything) he has not succeeded.
		
		As yet (\emph{or} so far) he has not succeeded.
	\end{example}
	\item {\bf Being.} Not appropriate after {\it regard$\ldots$} as.
	\begin{example}
		He is regarded as being the best dancer in the club.
		
		$\to$ He is regarded as the best dancer in the club.
	\end{example}
	\item {\bf But.} Unnecessary after {\it doubt} \& {\it help}.
	\begin{example}
		I have no doubt but that $\to$ I have no doubt that
		
		He could not help but see that $\to$ He could not help seeing that
	\end{example}
	The too-frequent use of {\it but} as a conjunction leads to the fault discussed under Rule 18.
	
	A loose sentence formed with {\it but} can usually be converted into a periodic sentence formed with {\it although}.
	
	Particularly awkward is one {\it but} closely following another, thus making a contrast to a contrast, or a reservation to a reservation.
	
	This is easily corrected by rearrangement.
	\begin{example}
		Our country had vast resources but seemed almost wholly unprepared for war.
		
		But within a year it had created an army of 4 million.
		
		$\to$ Our country seemed almost wholly unprepared for war, but it had vast resources.
		
		Within a year it had created an army of 4 million.
	\end{example}
	\item {\bf Can.} Means ``am (is, are) able.''
	
	Not to be used as a substitute for {\it may}.
	\item {\bf Care less.} The dismissive ``I couldn't care less'' is often used with the shortened ``not'' mistakenly (and mysteriously) omitted: ``I could care less.''
	
	The error destroys the meaning of the sentence \& is careless indeed.
	\item {\bf Case.} Often unnecessary.
	\begin{example}
		In many cases, the rooms lacked air conditioning.
		
		$\to$ Many of the rooms lacked air conditioning.
		
		It has rarely been the case that any mistake has been made.
		
		$\to$ Few mistakes have been made.
	\end{example}
	\item {\bf Certainly.} Used indiscriminately by some speakers, much as others use {\it very}, in an attempt to intensify any \& every statement.
	
	A mannerism of this kind, bad in speech, is even worse in writing.
	\item {\bf Character.} Often simply redundant, used from a mere habit of wordiness.
	\begin{example}
		acts of a hostile character $\to$ hostile acts
	\end{example}
	\item {\bf Claim.} [v] With object-noun, means ``lay claim to.''
	
	May be used with a dependent clause if this sense is clearly intended: ``She claimed that she was the sole heir.''
	
	(But even here {\it claimed to be} would be better.)
	
	Not to be used as a substitute for {\it declare, maintain}, or {\it charge}.
	\begin{example}
		He claimed he knew how.
		
		$\to$ He declared he knew how.
	\end{example}
	\item {\bf Clever.} Note that the word means 1 thing when applied to people, another when applied to horses.
	
	A clever horse is a good-natured one, not an ingenious one.
	\item {\bf Compare.} To {\it compare to} is to point out or imply resemblances between objects regarded as essentially of a different order; to {\it compare with} is mainly to point out differences between objects regarded as essentially of the same order.
	
	Thus, life has been {\it compared to} a pilgrimage, {\it to} a drama, {\it to} a battle; Congress may be {\it compared with} the British Parliament.
	
	Paris has been {\it compared to} ancient Athens; it may be {\it compared with} modern London.
	\item {\bf Comprise.} Literally, ``embrace'': A zoo comprises mammals, reptiles, \& birds (because it ``embraces,'' or ``includes,'' them).
	
	But animals do not comprise (``embrace'') a zoo - they constitute a zoo.
	\item {\bf Consider.} Not followed by {\it as} when it means ``believe to be.''
	\begin{example}
		I consider him as competent.
		
		$\to$ I consider him competent.
	\end{example}
	When {\it considered} means ``examined'' or ``discussed,'' it is followed by {\it as}:
	\begin{example}
		The lecturer considered Eisenhower 1st as soldier \& 2nd as administrator.
	\end{example}
	\item {\bf Contact.} As a transitive verb, the word is vague \& self-important.
	
	Do not {\it contact} people; get in touch with them, look them up, phone them, find them, or meet them.
	\item {\bf Cope.} An intransitive verb used with {\it with}.
	
	In formal writing, one doesn't ``cope,'' one ``copes with'' something or somebody.
	\begin{example}
		I knew they'd cope. (jocular)
		
		$\to$ I knew they would cope with the situation.
	\end{example}
	\item {\bf Currently.} In the sense of {\it now} with a verb in the present tense, {\it currently} is usually redundant; emphasis is better achieved through a more precise reference to time.
	\begin{example}
		We are currently reviewing your application.
		
		$\to$ We are at this moment reviewing your application.
	\end{example}
	\item {\bf Data.} Like {\it strata, phenomena}, \& {\it media}, {\it data} is a plural \& is best used with a plural verb.
	
	The word, however, is slowly gaining acceptance as a singular.
	\begin{example}
		The data is misleading.
		
		$\to$ These data are misleading.
	\end{example}
	\item {\bf Different than.} Here logic supports established usage: one thing differs {\it from} another, hence, {\it different from}.
	
	Or, {\it other than, unlike}.
	\item {\bf Disinterested.} Means ``impartial.''
	
	Do not confuse it {\it with uninterested}, which means ``not interested in.''
	\begin{example}
		Let a disinterested person judge our dispute, (an impartial person)
		
		This man is obviously uninterested in our dispute, (couldn't care less)
	\end{example}
	\item {\bf Divided into.} Not to be misused for {\it composed of}.
	
	The line is sometimes difficult to draw; doubtless plays are divided into acts, but poems are composed of stanzas.
	
	An apple, halved, is divided into sections, but an apple is composed of seeds, flesh, \& skin.
	\item {\bf Due to.} Loosely used for {\it through, because of}, or {\it owning to}, in adverbial phrases.
	\begin{example}
		He lost the 1st game due to carelessness.
		
		$\to$ He lost the 1st game because of carelessness.
	\end{example}
	In correct use, synonymous with {\it attributable to}: ``The accident was due to bad weather''; ``losses due to preventable fires.''
	\item {\bf Each \& every one.} Pitchman's jargon.
	
	Avoid, except in dialogue.
	\begin{example}
		It should be a lesson to each \& every one of us.
		
		$\to$ It should be a lesson to every one of us (to us all).
	\end{example}
	\item {\bf Effect.} As a noun, means ``result''; as a verb, means ``to bring about,'' ``to accomplish'' (not to be confused with {\it affect}, which means ``to influence'').
	
	As a noun, often loosely used in perfunctory writing about fashions, music, painting, \& other arts: ``a Southwestern effect''; ``effects in pale green''; ``very delicate effects''; ``subtle effects''; ``a charming effect was produced.''
	
	The writer who has a definite meaning to express will not take refuge in such vagueness.
	\item {\bf Enormity.} Use only in the sense of ``monstrous wickedness.''
	
	Misleading, if not wrong, when used to express bigness.
	\item {\bf Enthuse.} An annoying verb growing out of the noun {\it enthusiasm}.
	
	Not recommended.
	\begin{example}
		She was enthused about her new car.
		
		$\to$ She was enthusiastic about her new car.
		
		She enthused about her new car.
		
		$\to$ She talked enthusiastically (expressed enthusiasm) about her new car.
	\end{example}
	\item {\bf Etc.} Literally, ``and other things''; sometimes loosely used to mean ``and other persons.''
	
	The phrase is equivalent to {\it \& the rest, \& so forth}, \& hence is not to be used if 1 of these would be insufficient - i.e., if the reader would be left in doubt as to any important particulars.
	
	Least open to objection when it represents the last terms of a list already given almost in full, or immaterial words at the end of a quotation.
	
	%
	At the end of a list introduced by {\it such as, for example}, or any similar expression, {\it etc.} is incorrect.
	
	In formal writing, {\it etc.} is a misfit.
	
	An item important enough to call for {\it etc.} is probably important enough to be named.
	\item {\bf Fact.} Use this word only of matters capable of direct verification, not of matters of judgment.
	
	That a particular event happened on a given date \& that lead melts at a certain temperature are facts.
	
	But such conclusions as that Napoleon was the greatest of modern generals or that the climate of California is delightful, however defensible they may be, are not properly called facts.
	\item {\bf Facility.} Why must jails, hospitals, \& schools suddenly become ``facilities''?
	\begin{example}
		Parents complained bitterly about the fire hazard in the wooden facility.
		
		$\to$ Parents complained bitterly about the fire hazard in the wooden schoolhouse.
		
		He has been appointed warden of the new facility.
		
		$\to$ He has been appointed warden of the new prison.
	\end{example}
	\item {\bf Factor.} A hackneyed word; the expressions of which it is a part can usually be replaced by something more direct \& idiomatic.
	\begin{example}
		Her superior training was the great factor in her winning the match.
		
		$\to$ She won the match by being better trained.
		
		Air power is becoming an increasingly important factor in deciding battles.
		
		$\to$ Air power is playing a larger \& larger part in deciding battles.
	\end{example}
	\item {\bf Farther. Further.} The 2 words are commonly interchanged, but there is a distinction worth observing: {\it farther} serves best as a distance word, {\it further} as a time or quantity word.
	
	You chase a ball {\it farther} than the other fellow; you pursue a subject {\it further}.
	\item {\bf Feature.} Another hackneyed word; like {\it factor}, it usually adds nothing to the sentence in which it occurs.
	\begin{example}
		A feature of the entertainment especially worthy of mention was the singing of Allison Jones.
		$\to$ (Better use the same number of words to tell what Allison Jones sang \& how she sang it.)
	\end{example}
	As a verb, in the sense of ``offer as a special attraction,'' it is to be avoided.
	\item {\bf Finalize.} A pompous, ambiguous verb.
	
	(See Chap. V, Reminder 21.)
	\item {\bf Fix.} Colloquial in America for {\it arrange, prepare, mend}.
	
	The usage is well established.
	
	But bear in mind that this verb is from {\it figere:} ``to make firm,'' ``to place definitely.''
	
	These are the preferred meanings of the word.
	\item {\bf Flammable.} An oddity, chiefly useful in saving lives.
	
	The common word meaning ``combustible'' is {\it inflammable}.
	
	But some people are thrown off by the {\it in-} \& think {\it inflammable} means ``not combustible.''
	
	For this reason, trucks carrying gasoline or explosives are now marked FLAMMABLE.
	
	Unless you are operating such a truck \& hence are concerned with the safety of children \& illiterates, use {\it inflammable}.
	\item {\bf Folk.} A collective noun, equivalent to {\it people}.
	
	Use the singular form only.
	
	{\it Folks}, in the sense of ``parents,'' ``family,'' ``those present,'' is colloquial \& foo folksy for formal writing.
	\begin{example}
		Her folks arrived by the afternoon train.
		
		$\to$ Her father \& mother arrived by the afternoon train.
	\end{example}
	\item {\bf Fortuitous.} Limited to what happens by chance.
	
	Not to be used for {\it fortunate} or {\it lucky}.
	\item {\bf Get.} The colloquial {\it have got} for {\it have} should not be used in writing.
	
	The preferable form of the participle is {\it got}, not {\it gotten}.
	\begin{example}
		He has not got any sense.
		
		$\to$ He has no sense.
		
		They returned without having gotten any.
		
		$\to$ They returned without having got any.
	\end{example}
	\item {\bf Gratuitous.} Means ``unearned,'' or ``unwarranted.''
	\begin{example}
		The insult seemed gratuitous, (undeserved)
	\end{example}
	\item {\bf He is a man who.} A common type of redundant expression; see Rule 17.
	\begin{example}
		He is a man who is very ambitious.
		
		$\to$ He is very ambitious.
		
		Vermont is a state that attracts visitors because of its winter sports.
		
		$\to$ Vermont attracts visitors because of its winter sports.
	\end{example}
	\item {\bf Hopefully.} This once-useful adverb meaning ``with hope'' has been distorted \& is now widely used to mean ``I hope'' or ``it is to be hoped.''
	
	Such use is not merely wrong, it is silly.
	
	To say, ``Hopefully I'll leave on the noon plane'' is to talk nonsense.
	
	Do you mean you'll leave on the noon plane in a hopeful frame of mind?
	
	Or do you mean you hope you'll leave on the noon plane?
	
	Whichever you mean, you haven't said it clearly.
	
	Although the word in its new, free-floating capacity may be pleasurable \& even useful to many, it offends the ear of many others, who do not like to see words dulled or eroded, particularly when the erosion leads to ambiguity, softness, or nonsense.
	\item {\bf However.} Avoid starting a sentence with {\it however} when the meaning is ``nevertheless.''
	
	The word usually serves better when not in 1st position.
	\begin{example}
		The roads were almost impassable.
		
		However, we at least succeeded in reaching camp.
		
		$\to$ The roads were almost impassable.
		
		At last, however, we succeeded in reaching camp.
	\end{example}
	When {\it however} comes 1st, it means ``in whatever way'' or ``to whatever extent.''
	\begin{example}
		However you advise him, he will probably do as he thinks best.
		
		However discouraging the prospect, they never lost heart.
	\end{example}
	\item {\bf Illusion.} See allusion.
	\item {\bf Imply. Infer.} Not interchangeable.
	
	Something implied is something suggested or indicated, though not expressed.
	
	Something inferred is something deduced from evidence at hand.
	\begin{example}
		Farming implies early rising.
		
		Since she was a a farmer, we inferred that she got up early.
	\end{example}
	\item {\bf Importantly.} Avoid by rephrasing.
	\begin{example}
		More importantly, he paid for the damages.
		
		$\to$ What's more, he paid for the damages.
		
		With the breeze freshening, he altered course to pass inside the island.
		
		More importantly, as things turned out, he tucked in a reef.
		
		$\to$ With the breeze freshening, he altered course to pass inside the island.
		
		More important, as things turned out, he tucked in a reef.
	\end{example}
	\item {\bf In regard to.} Often wrongly written {\it in regards to}.
	
	But {\it as regards} is correct, \& means the same thing.
	\item {\bf In the last analysis.} A bankrupt expression.
	\item {\bf Inside of. Inside.} The {\it of} following {\it inside} is correct in the adverbial meaning ``in less than.''
	
	In other meanings, {\it of} is unnecessary.
	\begin{example}
		Inside of 5 minutes I'll be inside the bank.
	\end{example}
	\item {\bf Insightful.} The word is a suspicious overstatement for ``perceptive.''
	
	If it is to be used at all, it should be used for instances of remarkably penetrating vision.
	
	Usually, it crops up merely to inflate the commonplace.
	\begin{example}
		That was an insightful remark you made.
		
		$\to$ That was a perceptive remark you made.
	\end{example}
	\item {\bf In terms of.} A piece of padding usually best omitted.
	\begin{example}
		The job was unattractive in terms of salary.
		
		$\to$ The salary made the job unattractive.
	\end{example}
	\item {\bf Interesting.} An unconvincing word; avoid it as a means of introduction.
	
	Instead of announcing that what you are about to tell is interesting, make it so.
	\begin{example}
		An interesting story is told of $\to$ (Tell the story without preamble.)
		
		In connection with the forthcoming visit of Mr. B. to America, it is interesting to recall that he
		
		$\to$ Mr. B., who will soon visit America
	\end{example}
	Also to be avoided in introduction is the word {\it funny}.
	
	Nothing becomes funny by being labeled so.
	\item {\bf Irregardless.} Should be {\it regardless}.
	
	The error results from failure to see the negative in {\it -less} \& from a desire to get it in as a prefix, suggested by such words as {\it irregular, irresponsible}, and, perhaps especially, {\it irrespective}.
	\item {\bf -ize.} Do not coin verbs by adding this tempting suffix.
	
	Many good \& useful verbs do end in {\it -ize}: {\it summarize, fraternize, harmonize, fertilize}.
	
	But there is a growing list of abominations: {\it containerize, prioritize, finalize}, to name 3.
	
	Be suspicious of {\it -ize}; let your ear \& your eye guide you.
	
	Never tack {\it -ize} onto a noun to create a verb.
	
	Usually you will discover that a useful verb already exists.
	
	Why say ``utilize'' when there is the simple, unpretentious word {\it use}?
	\item {\bf Kind of.} Except in familiar style, not to be used as a substitute for {\it rather} or {\it something like}.
	
	Restrict it to its literal sense: ``Amber is kind of fossil resin''; ``I dislike that kind of publicity.''
	
	The same holds true for {\it sort of}.
	\item {\bf Lay.} A transitive verb.
	
	Except in slang (``Let it lay''), do not misuse it for the intransitive verb {\it lie}.
	
	The hen, or the play, {\it lays} an egg; the llama {\it lies} down.
	
	The playwright went home \& {\it lay} down.
	\begin{example}
		lie, lay, lain, lying
		
		lay, laid, laid, laying
	\end{example}
	\item {\bf Leave.} Not to be misused for {\it let}.
	\begin{example}
		Leave it stand the way it is.
		
		$\to$ Let it stand the way it is.
		
		Leave go of that rope!
		
		$\to$ Let go of that rope!
	\end{example}
	\item {\bf Less.} Should not be misused {\it for fewer}.
	\begin{example}
		They had less workers than in the previous campaign.
		
		$\to$ They had fewer workers than in the previous campaign.
	\end{example}
	{\it Less} refers to quantity, {\it fewer} to number.
	
	``His troubles are less than mine'' means ``His troubles are not so great as mine.''
	
	``His troubles are fewer than mine'' means ``His troubles are not so numerous as mine.''
	\item {\bf Like.} Not to be used for the conjunction {\it as}.
	
	{\it Like} governs nouns \& pronouns; before phrases \& clauses the equivalent word is as.
	\begin{example}
		We spent the evening like in the old days.
		
		$\to$ We spent the evening as in the old days.
		
		Chlo\"e smells good, like a baby should.
		
		$\to$ Chlo\"e smells good, as a baby should.
	\end{example}
	The use of {\it like} for {\it as} has its defenders; they argue that any usage that achieves currency becomes valid automatically.
	
	This, they say, is the way the language is formed.
	
	It is \& it isn't.
	
	An expression sometimes enjoys a vogue, much as an article of apparel does.
	
	{\it Like} has long been widely misused by the illiterate; lately it has been taken up by the knowing \& the well-informed, who find it catchy, or liberating, \& who use it as though they were slumming.
	
	If every word or device that achieved currency were immediately authenticated, simply on the ground of popularity, the language would be as chaotic as a ball game with no foul lines.
	
	For the students, perhaps the most useful thing to know about {\it like} is that most carefully edited publications regard its use before phrases \& clauses as simple error.
	\item {\bf Line. Along these lines.} {\it Line} in the sense of ``course of procedure, conduct, thought'' is allowable but has been so overworked, particularly in the phrase {\it along these lines}, that a writer who aims at freshness or originality had better discard it entirely.
	\begin{example}
		Mr. B. also spoke along the same lines.
		
		$\to$ Mr. B. also spoke to the same effect.
		
		She is studying along the line of French literature.
		
		$\to$ She is studying French literature.
	\end{example}
	{\bf Literal. Literally.} Often incorrectly used in support of exaggeration or violent metaphor.
	\begin{example}
		a literal flood of abuse $\to$ a flood of abuse
		
		literally dead with fatigue $\to$ almost dead with fatigue
	\end{example}
	\item {\bf Loan.} A noun. As a verb, prefer {\it lend}.
	\begin{example}
		Lend me your ears.
		
		the loan of your ears
	\end{example}
	\item {\bf Meaningful.} A bankrupt adjective.
	
	Choose another, or rephrase.
	\begin{example}
		His was a meaningful contribution.
		
		$\to$ His contribution counted heavily.
		
		We are instituting many meaningful changes in the curriculum.
		
		$\to$ We are improving the curriculum in many ways.
	\end{example}
	\item {\bf Memento.} Often incorrectly written {\it momento}.
	\item {\bf Most.} Not to be used for {\it almost} in formal composition.
	\begin{example}
		most everybody $\to$ almost everybody
		
		most all the time $\to$ almost all the time
	\end{example}
	{\bf Nature.} Often simply redundant, used like {\it character}.
	\begin{example}
		acts of a hostile nature $\to$ hostile acts
	\end{example}
	{\it Nature} should be avoided in such vague expressions as ``a lover of nature,'' ``poems about nature.''
	
	Unless more specific statements follow, the reader cannot tell whether the poems have to do with natural scenery, rural life, the sunset, the untracked wilderness, or the habits of squirrels.
	\item {\bf Nauseous. Nauseated.} The 1st means ``sickening to contemplate''; the 2nd means ``sick at the stomach.''
	
	Do not, therefore, say, ``I feel nauseous,'' unless you are sure you have that effect on others.
	\item {\bf Nice.} A shaggy, all-purpose word, to be used sparingly in formal composition.
	
	``I had a nice time.''
	
	``It was nice weather.''
	
	``She was so nice to her mother.''
	
	The meanings are indistinct.
	
	{\it Nice} is most useful in the sense of ``precise'' or ``dedicate'': ``a nice distinction.''
	\item {\bf Nor.} Often used wrongly for {\it or} after negative expressions.
	\begin{example}
		He cannot eat nor sleep.
		
		$\to$ He cannot eat or sleep./He can neither eat nor sleep./He cannot eat nor can he sleep.
	\end{example}
	\item {\bf Noun used as verb.} Many nouns have lately been pressed into service as verbs.
	
	Not all are bad, but all are suspect.
	\begin{example}
		Be prepared for kisses when you gift your girlfriend with this merry scent.
		
		$\to$ Be prepared for kisses when you give your girlfriend this merry scent.
		
		The candidate hosted a dinner for 50 of her workers.
		
		$\to$ The candidate gave a dinner for 50 of her workers.
		
		The meeting was chaired by Mr. Oglethorp.
		
		$\to$ Mr. Oglethorp was chair of the meeting.
		
		She headquarters in Newark.
		
		$\to$ She has headquarters in Newark.
		
		The theater troupe debuted last fall.
		
		$\to$ The theater troupe made its debut last fall.
	\end{example}
	\item {\bf Offputting. Ongoing.} Newfound adjectives, to be avoided because they are inexact \& clumsy.
	
	{\it Ongoing} is a mix of ``continuing'' \& ``active'' \& is usually superfluous.
	\begin{example}
		He devoted all his spare time to the ongoing program for aid to the elderly.
		
		$\to$ He devoted all his spare time to the program for aid to the elderly.
	\end{example}
	{\it Offputting} might mean ``objectionable,'' ``disconcerting,'' ``distasteful.''
	
	Select instead a word whose meaning is clear.
	
	As a simple test, transform the participles to verbs.
	
	It is possible to {\it upset} something.
	
	But to {\it offput}?
	
	To {\it ongo}?
	\item {\bf One.} In the sense of ``a person,'' not to be followed by {\it his} or {\it her}.
	\begin{example}
		One must watch his step.
		
		$\to$ One must watch one's step. (You must watch your step.)
	\end{example}
	\item {\bf One of the most.} Avoid this feeable formula.
	
	``1 of the most exciting developments of modern science is $\ldots$''; ``Switzerland is 1 of the most beautiful countries of Europe.''
	
	There is nothing wrong with the grammar; the formula is simply threadbare.
	\item {\bf -oriented.} A clumsy, pretentious device, much in vogue.
	
	Find a better way of indicating orientation or alignment or direction.
	\begin{example}
		It was a manufacturing-oriented company.
		
		$\to$ It was a company chiefly concerned with manufacturing.
		
		Many of the skits are situation-oriented.
		
		$\to$ Many of the skits rely on situation.
	\end{example}
	\item {\bf Partially.} Not always interchangeable with {\it partly}.
	
	Best used in the sense of ``to a certain degree,'' when speaking of a condition or state: ``I'm partially resigned to it.''
	
	{\it Partly} carries the idea of a part as distinct from the whole - usually a physical object.
	\begin{example}
		The log was partially submerged.
		
		$\to$ The log was partly submerged.
		
		She was partially in \& partially out.
		
		$\to$ She was partly in \& partly out./She was part in, part out.
	\end{example}
	\item {\bf Participle for verbal noun.}
	\begin{example}
		There was little prospect of the Senate accepting even this compromise.
		
		$\to$ There was little prospect of the Senate's accepting even this compromise.
	\end{example}
	In the lefthand column, {\it accepting} is a present participle; in the righthand column, it is a verbal noun (gerund).
	
	The construction shown in the lefthand column is occasionally found, \& has its defenders.
	
	Yet it is easy to see that the 2nd sentence has to do not with a prospect of the Senate but with a prospect of accepting.
	
	%
	Any sentence in which the use of the possessive is awkward or impossible should of course be recast.
	\begin{example}
		In the event of a reconsideration of the whole matters becoming necessary.
		
		$\to$ If it should become necessary to reconsider the whole matter.
		
		There was great dissatisfaction with the decision of the arbitrators being favorable to the company.
		
		$\to$ There was great dissatisfaction with the arbitrators' decision in favor of the company.
	\end{example}
	\item {\bf People.} A word with many meanings.
	
	({\it The American Heritage Dictionary}, 3rd Edition, gives 9.)
	
	{\it The people} is a political term, not to be confused with {\it the public}.
	
	From the people comes political support or opposition; from the public comes artistic appreciation or commercial patronage.
	
	%
	The word {\it people} is best not used with words of number, in place of {\it persons}.
	
	If of ``6 people'' 5 went away, how many people would be left?
	
	Answer: 1 people.
	\item {\bf Personalize.} A pretentious word, often carrying bad advice.
	
	Do not {\it personalize} your prose; simply make it good \& keep it clean.
	
	See Chap. V, Reminder 1.
	\begin{example}
		a highly personalized affair $\to$ a highly personal affair
		
		Personalize your stationery. $\to$ Design a letterhead.
	\end{example}
	\item {\bf Personally.} Often unnecessary.
	\begin{example}
		Personally, I thought it was a good book.
		
		$\to$ I thought it a good book.
	\end{example}
	\item {\bf Possess.} Often used because to the writer it sounds more impressive than {\it have} or {\it own}.
	
	Such usage is not incorrect but is to be guarded against.
	\begin{example}
		She possessed great courage.
		
		$\to$ She had great courage (was very brave).
		
		He was the fortunate possessor of $\to$ He was lucky enough to own
	\end{example}
	\item {\bf Presently.} Has 2 meanings: ``in a short while'' \& ``currently.''
	
	Because of this ambiguity it is best restricted to the 1st meaning: ``She'll be here presently'' (``soon,'' or ``in a short time'').
	\item {\bf Prestigious.} Often an adjective of last resort.
	
	{\it It's in the dictionary, but that doesn't mean you have to use it}.
	\item {\bf Refer.} See {\it allude}.
	\item {\bf Regretful.} Sometimes carelessly used for {\it regrettable}: ``The mixup was due to a regretful breakdown in communications.''
	\item {\bf Relate.} Not to be used intransitively to suggest rapport.
	\begin{example}
		I relate well to Janet.
		
		$\to$ Janet \& I see things the same way./Janet \& I have a lot in common.
	\end{example}
	\item {\bf Respective. Respectively.} These words may usually be omitted with advantage.
	\begin{example}
		Works of fiction are listed under the names of their respective authors.
		
		$\to$ Works of fiction are listed under the names of their authors.
		
		The mile run \& the 2-mile run were won by Jones \& Cummings respectively.
		
		$\to$ The mile run was won by Jones, the 2-mile run by Cummings.
	\end{example}
	\item {\bf Secondly, thirdly, etc.} Unless you are prepared to begin {\it with 1stly} \& defend it (which will be difficult), do not prettify numbers with {\it -ly}.
	
	Modern usage prefers {\it second, third}, \& so on.
	\item {\bf Shall. Will.} In formal writing, the future tense requires {\it shall} for the 1st person, {\it will} for the 2nd \& 3rd.
	
	The formula to express the speaker's belief regarding a future action or state is {\it I shall}; {\it I will} expresses determination or consent.
	
	A swimmer in distress cries, ``I shall drown; no one will save me!''
	
	A suicide puts it the other way: ``I will drown; no one shall save me!''
	
	In relaxed speech, however, the words {\it shall} \& {\it will} are seldom used precisely; our ear guides us or fails to guide us, as the case may be, \& we are quite likely to drown when we want to survive \& survive when we want to drown.
	\item {\bf So.} Avoid, in writing, the use of so as an intensifier: ``so good''; ``so warm''; ``so delightful.''
	\item {\bf Sort of.} See {\it kind of}.
	\item {\bf Split infinitive.} There is precedent from the 14th century down for interposing an adverb between {\it to} \& the infinitive it governs, but the construction should be avoided unless the writer wishes to place unusual stress on the adverb.
	\begin{example}
		to diligently inquire $\to$ to inquire diligently
	\end{example}
	For another side to the split infinitive, see Chap. V, Reminder 14.
	\item {\bf State.} Not to be used as a mere substitute for {\it say, remark}.
	
	Restrict it to the sense of ``express fully or clearly'': ``He refused to state his objections.''
	\item {\bf Student body.} 9 times out of 10 a needles \& awkward expression, meaning no more than the simple word {\it students}.
	\begin{example}
		a member of the student body $\to$ a student
		
		popular with the student body $\to$ liked by the students
	\end{example}
	\item {\bf Than.} Any sentence with {\it than} (to express comparison) should be examined to make sure to essential words are missing.
	\begin{example}
		I'm probably closer to my mother than my father. (Ambiguous.)
		
		$\to$ I'm probably closer to my mother than to my father./I'm probably closer to my mother than my father is.
		
		It looked more like a cormorant than a heron.
		
		$\to$ It looked more like a cormorant than like a heron.
	\end{example}
	\item {\bf Thanking you in advance.} This sounds as if the writer meant, ``It will not be worth my while to write to you again.''
	
	In making your request, write ``Will you please,'' or ``I shall be obliged.''
	
	Then, later, if you feel moved to do so, or if the circumstances call for it, write a letter of acknowledgment.
	\item {\bf That. Which.} {\it That} is the defining, or restrictive, pronoun, {\it which} the nondefining, or nonrestrictive. (See Rule 3.)
	\begin{example}
		The lawn mower that is broken is in the garage. (Tells which one.)
		
		The lawn mower, which is broken, is in the garage. (Adds a fact about the only mower in question.)
	\end{example}
	The use of {\it which} for {\it that} is common in written \& spoken language (``Let us now go even unto Bethlehem, \& see this thing which is come to pass.'').
	
	Occasionally {\it which} seems preferable to {\it that}, as in the sentence from the Bible.
	
	But it would be a convenience to all if these 2 pronouns were used with precision.
	
	Careful writers, watchful for small conveniences, go {\it which}-hunting, remove the defining {\it whiches}, \& by so doing improve their work.
	\item {\bf The foreseeable future.} A cliche, \& a fuzzy one.
	
	How much of the future is foreseeable?
	
	10 minutes?
	
	10 years?
	
	Any of it?
	
	By whom is it foreseeable?
	
	Seers?
	
	Experts?
	
	Everybody?
	\item {\bf The truth} is$\ldots$ {\bf The fact} is$\ldots$ A bad beginning for a sentence.
	
	If you feel you are possessed of the truth, or of the fact, simply state it.
	
	Do not give it advance billing.
	\item {\bf They. He or She.} Do not use {\it they} when the antecedent is a distributive expression such as {\it each, each one, everybody, everyone, many a man}.
	
	Use the singular pronoun.
	\begin{example}
		Every one of us knows they are fallible.
		
		$\to$ Everyone in the community, whether they are a member of the Association or not, is invited to attend.
		
		$\to$ Everyone in the community, whether he is a member of the Association or not, is invited to attend.
	\end{example}
	A similar fault is the use of the plural pronoun with the antecedent {\it anybody, somebody, someone}, the intention being either to avoid the awkward {\it he or she} or to avoid committing oneself to one or the other.
	
	Some bashful speakers even say, ``A friend of mine told me that they$\ldots$''
	
	%
	The use of {\it he} as a pronoun for nouns embracing both genders is a simple, practical convention rooted in the beginnings of the English language.
	
	Currently, however, many writers find the use of the generic {\it he} or {\it his} to rename indefinite antecedents limiting or offensive.
	
	Substituting {\it he or she} in its place is the logical thing to do if it works.
	
	But it often doesn't work, if only because repetition makes it sound boring or silly.
	
	%
	Consider these strategies to avoid an awkward overuse of {\it he or she} or an unintentional emphasis on the masculine:
	
	Use the plural rather than the singular.
	\begin{example}
		The writer must address his readers' concerns.
		
		$\to$ Writers must address their readers' concerns.
	\end{example}
	Eliminate the pronoun altogether.
	\begin{example}
		The writer must address his readers' concerns.
		
		$\to$ The writer must address reader's concerns.
	\end{example}
	Substitute the 2nd person for the 3rd person.
	\begin{example}
		The writer must address his readers' concerns.
		
		$\to$ As a writer, you must address your readers' concerns.
	\end{example}
	No one need fear to use {\it he} if common sense supports it.
	
	If you think {\it she} is a handy substitute for {\it he}, try it \& see what happens.
	
	Alternatively, put all controversial nouns in the plural \& avoid the choice of sex altogether, although you may find your prose sounding general \& diffuse as a result.
	\item {\bf This.} The pronoun {\it this}, referring to the complete sense of a preceding sentence or clause, can't always carry the load \& so may produce an imprecise statement.
	\begin{example}
		Visiting dignitaries watched yesterday as ground was broken for the new high-energy physics laboratory with a blowout safety wall.
		
		This is the 1st visible evidence of the university's plans for modernization \& expansion.
		
		$\to$ Visiting dignitaries watched yesterday as ground was broken for the new high-energy physics laboratory with a blowout safety wall.
		
		The ceremony afforded the 1st visible evidence of the university's plans for modernization \& expansion.
	\end{example}
	In the lefthand example above, {\it this} does not immediately make clear what the 1st visible evidence is.
	\item {\bf Thrust.} This showy noun, suggestive of power, hinting of sex, is the darling of executives, politicos, \& speech-writers.
	
	Use it sparingly.
	
	Save it for specific application.
	\begin{example}
		Our reorganization plan has a tremendous thrust.
		
		$\to$ The piston has a 5-inch thrust.
		
		The thrust of his letter was that he was working more hours than he'd bargained for.
		
		$\to$ The point he made in his letter was that he was working more hours than he'd bargained for.
	\end{example}
	\item {\bf Tortuous. Torturous.} A winding road is {\it tortuous}, a painful ordeal is {\it torturous}.
	
	Both words carry the idea of ``twist,'' the twist having been a form of torture.
	\item {\bf Transpire.} Not to be used in the sense of ``happen,'' ``come to pass.''
	
	Many writers so use it (usually when groping toward imagined elegance), but their usage finds little support in the Latin ``breathe across or through.''
	
	It is correct, however, in the sense of ``become known.''
	
	``Eventually, the grim account of his villainy transpired'' (literally, ``leaked through or out'').
	\item {\bf Try.} Takes the infinitive: ``try to mend it,'' not ``try \& mend it.''
	
	Students of the language will argue that {\it try and} has won through \& become idiom.
	
	Indeed it has, \& it is relaxed \& acceptable.
	
	But {\it try to} is precise, \& when you are writing formal prose, try \& write {\it try to}.
	\item {\bf Type.} Not a synonym for {\it kind of}.
	
	The examples below are common vulgarisms.
	\begin{example}
		that type employee $\to$ that kind of employee
		
		I dislike that type publicity.
		
		$\to$ I dislike that kind of publicity.
		
		small, home-type hotels $\to$ small, homelike hotels
		
		a new type plane $\to$ a plane of a new design (new kind)
	\end{example}
	\item {\bf Unique.} Means ``without like or equal.''
	
	Hence, there can be no degrees of uniqueness.
	\begin{example}
		It was the most unique coffee maker on the market.
		
		$\to$ It was a unique coffee maker.
		
		The balancing act was very unique.
		
		$\to$ The balancing act was unique.
		
		Of all the spiders, the one that lives in a bubble under water is the most unique.
		
		$\to$ Among spiders, the one that lives in a bubble under water is unique.
	\end{example}
	\item {\bf Utilize.} Prefer {\it use}.
	\begin{example}
		I utilized the facilities.
		
		$\to$ I used the toilet.
		
		He utilized the dishwasher.
		
		$\to$ He used the dishwasher.
	\end{example}
	\item {\bf Verbal.} Sometimes means ``word for word'' \& in this sense may refer to something expressed in writing.
	
	{\it Oral} (from Latin {\it os}, ``mouth'') limits the meaning to what is transmitted by speech.
	
	{\it Oral agreement} is more precise than {\it verbal agreement}.
	\item {\bf Very.} Use this word sparingly.
	
	Where emphasis is necessary, use words strong in themselves.
	\item {\bf While.} Avoid the indiscriminate use of this word for {\it and, but}, \& {\it although}.
	
	Many writers use it frequently as a substitute for {\it and} or {\it but}, either from a mere desire to vary the connective or from doubt about which of the 2 connectives is more appropriate.
	
	In this use it is best replaced by a semicolon.
	\begin{example}
		The office \& salesrooms are on the ground floor, while the rest of the building is used for manufacturing.
		
		$\to$ The office \& salesrooms are on the ground floor; the rest of the building is used for manufacturing.
	\end{example}
	Its use as a virtual equivalent {\it of although} is allowable in sentences where this leads to no ambiguity or absurdity.
	\begin{example}
		While I admire his energy, I wish it were employed in a better cause.
	\end{example}
	This is entirely correct, as shown by the paraphrase
	\begin{example}
		I admire his energy; at the same time, I wish it were employed in a better cause.
	\end{example}
	Compare:
	\begin{example}
		While the temperature reaches 90 or 95 degrees in the daytime, the nights are often chilly.
	\end{example}
	The paraphrase shows why the use of {\it while} is incorrect:
	\begin{example}
		The temperature reaches 90 or 95 degrees in the daytime; at the same time the nights are often chilly.
	\end{example}
	In general, the writer will do well to use {\it while} only with strict literalness, in the sense of ``during the time that.''
	\item {\bf -wise.} Not to be used indiscriminately as a pseudosuffix: {\it taxwise, pricewise, marriagewise, prosewise, saltwater taffy-wise}.
	
	Chiefly useful when it means ``in the manner of'': {\it clockwise}.
	
	There is not a noun in the language to which {\it -wise} cannot be added if the spirit moves one to add it.
	
	The sober writer will abstain from the use of this wild additive.
	\item {\bf Worth while.} Overworked as a term of vague approval \& (with {\it not}) of disapproval.
	
	Strictly applicable only to actions: ``Is it worth while to telegraph?''
	\begin{example}
		His books are not worth while.
		
		$\to$ His books are not worth reading (are not worth one's while to read; do not repay reading).
	\end{example}
	The adjective {\it worthwhile} (1 word) is acceptable but emaciated.
	
	Use a stronger word.
	\begin{example}
		a worthwhile project $\to$ a promising (useful, valuable, exciting) project
	\end{example}
	\item {\bf Would.} Commonly used to express habitual or repeated action.
	
	(``He would get up early \& prepare his own breakfast before he went to work.'')
	
	But when the idea of habit or repetition is expressed, in such phrases as {\it once a year, every day, each Sunday}, the past tense, without {\it would}, is usually sufficient, and, from its brevity, more emphatic.
	\begin{example}
		Once a year he would visit the old mansion.
		
		$\to$ Once a year he visited the old mansion.
	\end{example}
	In narrative writing, always indicate the transition from the general to the particular - i.e., from sentences that merely state a general habit to those that express the action of a specific day or period.
	
	Failure to indicate the change will cause confusion.
	\begin{example}
		Townsend would get up early \& prepare his own breakfast.
		
		If the day was cold, he filled the stove \& had a warm fire burning before he left the house.
		
		On his way to the garbage, he noticed that there were footprints in the new-fallen snow on the porch.
	\end{example}
\end{enumerate}

%------------------------------------------------------------------------------%

\subsection{An Approach to Style (With a List of Reminders)}
Up to this point, the book has been concerned with what is correct, or acceptable, in the use of English.

In this final chapter, we approach style in its broader meaning: style in the sense of what is distinguished \& distinguishing.

Here we leave solid ground.

Who can confidently say what ignites a certain combination of words, causing them to explode in the mind?

Who knows why certain notes in music are capable of stirring the listener deeply, though the same notes slightly rearranged are impotent?

These are high mysteries, \& this chapter is a mystery story, thinly disguised.

There is no satisfactory explanation of style, no infallible guide to good writing, no assurance that a person who thinks clearly will be able to write clearly, no key that unlocks the door, no inflexible rule by which writers may shape their course.

Writers will often find themselves steering by stars that are disturbingly in motion.

%
The preceding chapters contain instructions drawn from established English usage; this one contains advice drawn from a writer's experience of writing.

Since the book is a rule book, these cautionary remarks, these subtly dangerous hints, are presented in the form of rules, but they are, in essence, mere gentle reminders: they state what most of us know \& at times forget.

%
Style is an increment in writing.

When we speak of Fitzgerald's style, we don't mean his command of the relative pronoun, we mean th sound his words make on paper.

All writers, by the way they use the language, reveal something of their spirits, their habits, their capacities, \& their biases.

This is inevitable as well as enjoyable.

All writing is communication; creative writing is communication through revelation - it is the Self escaping into the open.

No writer long remains incognito.

%
If you doubt that style is something of a mystery, try rewriting a familiar sentence \& see what happens.

Any much-quoted sentence will do.

Suppose we take ``These are the times that try men's souls.''

Here we have 8 short, easy words, forming a simple declarative sentence.

The sentence contains no flashy ingredient such as ``Damn the torpedoes!'' \& the words, as you see, are ordinary.

Yet in that arrangement, they have shown great durability; the sentence is into its 3rd century.

Now compare a few variations:
\begin{example}
	Times like these try men's souls.
	
	How trying it is to live in these times!
	
	These are trying times for men's souls.
	
	Soulwise, these are trying times.
\end{example}
It seems unlikely that Thomas Paine could have made his sentiment stick if he had couched it in any of these forms.

By why not?

No fault of grammar can be detected in them, \& in every case the meaning is clear.

Each version is correct, \& each, for some reason that we can't readily put our finger on, is marked for oblivion.

We could, of course, talk about ``rhythm'' \& ``cadence,'' but the talk would be vague \& unconvincing.

We could declare {\it soulwise} to be a silly word,    inappropriate to the occasion; but even that won't do - it does not answer the main question.

Are we even sure {\it soulwise} is silly?

If {\it otherwise} is a serviceable word, what's the matter with {\it soulwise}?

%
Here is another sentence, this one by a later Tom.

It is not a famous sentence, although its author (Thomas Wolfe) is well known.

``Quick are the mouths of earth, \& quick the teeth that fed upon this loveliness.''

The sentence would not take a prize for clarity, \& rhetorically it is at the opposite pole from ``These are the times.''

Try it in a different form, without the inversions:
\begin{example}
	The mouths of earth are quick, \& the teeth that fed upon this loveliness are quick, too.
\end{example}
The author's meaning is still intact, but not his overpowering emotion.

What was poetical \& sensuous has become prosy \& wooden; instead of the secret sounds of beauty, we are left with the simple crunch of mastication.

(Whether Mr. Wolfe was guilty of overwriting is, of course, another question - one that is not pertinent here.)

%
With some writers, style not only reveals the spirit of the man but reveals his identity, as surely as would his fingerprints.

Here, following, are 2 brief passages from the works of 2 American novelists.

The subject in each case is languor.

In both, the words used are ordinary, \& there is nothing eccentric about the construction.
\begin{example}
	He did not still feel weak, he was merely luxuriating in that supremely gutful lassitude of convalescence in which time, hurry, doing, did not exist, the accumulating seconds \& minutes \& hours to which in its well state the body is slave both waking \& sleeping, now reversed \& time now the lip-server \& mendicant to the body's pleasure instead of the body thrall to time's headlong course.
	\\
	
	Manuel drank his brandy.
	
	He felt sleepy himself.
	
	It was too hot to go out into the town.
	
	Besides there was nothing to do.
	
	He wanted to see Zurito.
	
	He would go to sleep while he waited.
\end{example}
Anyone acquainted with Faulkner \& Hemingway will have recognized them in these passages \& perceived which was which.

How different are their languors!

%
Or take 2 American poets, stopping at evening.

One stops by woods, the other by laughing flesh.
\begin{example}
	My little horse must think it queer
	
	To stop without a farmhouse near
	
	Between the woods \& frozen lake
	
	The darkest evening of the year.
	\\
	
	I have perceived that to be with those I like is enough,
	
	To stop in company with the rest at evening is enough,
	
	To be surrounded by beautiful, curious, breathing,
	
	laughing flesh is enough$\ldots$
\end{example}
Because of the characteristic styles, there is little question about identity here, \& if the situations were reversed, with Whitman stopping by woods \& Frost by laughing flesh (not one of his regularly scheduled stops), the reader would know who was who.

%
Young writer often suppose that style is a garnish for the meat of prose, a sauce by which a dull dish is made palatable.

Style has no such separate entity; it is nondetachable, unfilterable.

The beginner should approach style warily, realizing that it is an expression of self, \& should turn resolutely away from all devices that are popularly believed to indicate style - all mannerisms, tricks, adornments.

The approach to style is by way of plainness, simplicity, orderliness, sincerity.

%
Writing is, for most, laborious \& slow.

The mind travels faster than the pen; consequently, writing becomes a question of learning to make occasional wing shots, bringing down the bird of thought as it flashes by.

A writer is a gunner, sometimes waiting in the blind for something to come in, sometimes roaming the countryside hoping to scare something up.

Like other gunners, the writer must cultivate patience, working many covers to bring down 1 partridge.

Here, following, are some suggestions \& cautionary hints that may help the beginner find the way to a satisfactory style.

\subsubsection{Place yourself in the background.}
``Write in a way that draws the reader's attention to the sense \& substance of the writing, rather than to the mood \& temper of the author. If the writing is solid \& good, the mood \& temper of the writer will eventually be revealed \& not at the expense of the work. Therefore, the 1st piece of advice is this: to achieve style, begin by affecting none -- i.e., place yourself in the background. A careful \& honest writer does not need to worry about style. As you become proficient in the use of language, your style will emerge, because you yourself will emerge, \& when this happens you will find it increasingly easy to break through the barriers that separate you from other minds, other hearts -- which is, of course, the purpose of writing, as well as its principal reward. Fortunately, the act of composition, or creation, disciplines the mind; writing is 1 way to go about thinking, \& the practice \& habit of writing not only drain the mind but supply it, too.'' -- \cite[p. 78]{Strunk_White_element_style}

%------------------------------------------------------------------------------%

\subsubsection{Write in a way that comes naturally.}
``Write in a way that comes easily \& naturally to you, using words \& phrases that come readily to hand. But do not assume that becaues you have acted naturally your product is without flaw.

The use of language begins with imitation. The infant imitates the sounds made by its parents; the child imitates 1st the spoken language, then the stuff of books. The imitative life continues long after the writer is secure in the language, for it is almost impossible to avoid imitating what one admires. Never imitate consciously, but do not worry about being an imitator; take pains instead to admire what is good. Then when you write in a way that comes naturally, you will echo the halloos that bear repeating.'' -- \cite[p. 79]{Strunk_White_element_style}

%------------------------------------------------------------------------------%

\subsubsection{Work from a suitable design.}
``Before beginning to compose something, gauge the nature \& extent of the enterprise \& work from a suitable design. (See Chap. II, Rule 12.) Design informs even the simplest structure, whether of brick \& steel or of prose. You raise a pup tent from 1 sort of vision, a cathedral from another. This does not mean that you must sit with a blueprint always in front of you, merely that you had best anticipate what you are getting into. To compose a laundry list, you can work directly from the pile of soiled garments, ticking them off 1 by 1. By to write a biography, you will need at least a rough scheme; you cannot plunge in blindly \& start ticking off fact after fact about your subject, lest you miss the forest for the trees \& there be no end to your labors.

Sometimes, of course, impulse \& emotion are more compelling than design. If you are deeply troubled \& are composing a letter appealing for mercy or for love, you had best not attempt to organize your emotions; the prose will have a better chance if the emotions are left in disarray -- which you'll probably have to do anyway, since feelings do not usually lend themselves to rearrangement. But even the kind of writing that is essentially adventurous \& impetuous will on examination be found to have a secret plan: Columbus didn't just sail, he sailed west, \& the New World took shape from this simple \&, we now think, sensible design.'' -- \cite[p. 80]{Strunk_White_element_style}

%------------------------------------------------------------------------------%

\subsubsection{Write with nouns \& verbs.}
``Write with nouns \& verbs, not with adjectives \& adverbs. The adjective hasn't been built that can pull a weak or inaccurate noun out of a tight place. This is not to disparage adjectives \& adverbs; they are indispensable parts of speech. Occasionally they surprise us with their power, as in
\begin{quotation}\it
	Up the airy mountain,
	
	Down the rushy glen,
	
	We daren't go a-hunting
	
	For fear of little men $\ldots$
\end{quotation}
The nouns {\it mountain} \& {\it glen} are accurate enough, but had the mountain not become airy, the glen rushy, William Ailing-ham might never have got off the ground with this poem. In general, however, it is nouns \& verbs, not their assistants, that give good writing its toughness \& color.'' -- \cite[p. 81]{Strunk_White_element_style}

%------------------------------------------------------------------------------%

\subsubsection{Revise \& rewrite.}
``Revising is part of writing. Few writers are so expert that they can produce what they are after on the 1st try. Quite often you will discover, on examining the completed work, that there are serious flaws in the arrangement of the material, calling for transpositions. When this is the case, a word processor can save you time \& labor as you rearrange the manuscript. You can select material on your screen \& move it to a more appropriate spot, or, if you cannot find the right spot, you can move the material to the end of the manuscript until you decide whether to delete it. Some writers find that working with a printed copy of the manuscript helps them to visualize the process of change; others prefer to revise entirely on screen. Above all, do not be afraid to experiment with what you have written. Save both the original \& the revised versions; you can always use the computer to restore the manuscript to its original condition, should that course seem best. Remember, it is no sign of weakness or defeat that your manuscript ends up in need of major surgery. This is a common occurrence in all writing, \& among the best writers.'' -- \cite[p. 82]{Strunk_White_element_style}

%------------------------------------------------------------------------------%

\subsubsection{Do not overwrite.}
``Rich, ornate prose is hard to digest, generally unwholesome, \& sometimes nauseating. If the sickly-sweet word, the overblown phrase are your natural form of expression, as is sometimes the case, you will have to compensate for it by a show of vigor, \& by writing something as meritorious as the Songs of Songs, which is Solomon's.

When writing with a computer, you must guard against wordiness. The click \& flow of a word processor can be seductive, \& you may find yourself adding a few unnecessary words or even a whole passage just to experience the pleasure of running your fingers over the keyboard \& watching your words appear on the screen. It is always a good idea to reread your writing later \& ruthlessly delete the excess.'' -- \cite[p. 83]{Strunk_White_element_style}

%------------------------------------------------------------------------------%

\subsubsection{Do not overstate.}
``When you overstate, readers will be instantly on guard, \& everything that has preceded your overstatement as well as everything that follows it will be suspect in their minds because they have lost confidence in your judgment or your poise. Overstatement is 1 of the common faults. A single overstatement, wherever or however it occurs, diminishes the whole, \& a single carefree superlative has the power to destroy, for readers, the object of your enthusiasm.'' -- \cite[p. 84]{Strunk_White_element_style}

%------------------------------------------------------------------------------%

\subsubsection{Avoid the use of qualifiers.}
``{\it Rather, very, little, pretty} -- these are the leeches that infest the pond of prose, sucking the blood of words. The constant use of the adjective {\it little} (except to indicate size) is particularly debilitating; we should all try to do a little better, we should all be very watchful of this rule, for it is a rather important one, \& we are pretty sure to violate it now \& then.'' -- \cite[p. 85]{Strunk_White_element_style}

%------------------------------------------------------------------------------%

\subsubsection{Do not affect a breezy manner.}
``The volume of writing is enormous, these days, \& much of it has a sort of windiness about it, almost as though the author were in a state of euphoria. ``Spontaneous me,'' say Whitman, \&, in his innocence, let loose the hordes of uninspired scribblers who would 1 day confuse spontaneity with genius.

The breezy style is often the work of an egocentric, the person who imagines that everything that comes to mind is of general interest \& that uninhibited prose creates high spirits \& carries the day. Open any alumni magazine, turn to the class notes, \& you are quite likely to encounter old Spontaneous Me at work -- an aging collegian who writes something like this:
\begin{quotation}\it
	Well, guys, here I am again dishing the dirt about your disorderly classmates, after passing a week in the Big Apple trying to catch the Columbia hoops tilt \& then a cab-ride from hell through the West Side casbah. \& speaking of news, howzabout tossing a few primo items this way?
\end{quotation}
This is an extreme example, but the same wind blows, at lesser velocities, across vast expanses of journalistic prose. The author in this case has managed in 2 sentences to commit most of the unpardonable sins: he obviously has nothing to say, he is showing off \& directing the attention of the reader to himself, he is using slang with neither provocation nor ingenuity, he adopts a patronizing air by throwing in the word {\it primo}, he is humorless (though full of fun), dull, \& empty. He has not done his work. Compare his opening remarks with the following -- a plunge directly into the news:
\begin{quotation}\it
	Clyde Crawford, who stroked the varsity shell in 1958, is swinging an oar again after a lapse of 40 years. Clyde resigned last spring as executive sales manager of the Indiana Flotex Company \& is now a gondolier in Venice.
\end{quotation}
This, although conventional, is compact, informative, unpretentious. The writer has dug up an item of news \& presented it in a straightforward manner. What the 1st writer tried to accomplish by cutting rhetorical capers \& by breeziness, the 2nd writer managed to achieve by good reporting, by keeping a tight rein on his material, \& by staying out of the act.'' -- \cite[p. 87]{Strunk_White_element_style}

%------------------------------------------------------------------------------%

\subsubsection{Use orthodox spelling.}
``In ordinary composition, use orthodox spelling. Do not write {\it nite} for {\it night, thru} for {\it through, pleez} for {\it please}, unless you plan to introduce a complete system of simplified spelling \& are prepared to take the consequences.

In the original edition of {\it The Elements of Style}, there was a chapter on spelling. In it, the author had this to say:
\begin{quotation}\it
	The spelling of English words is not fixed \& invariable, nor does it depend on any other authority than general agreement. At the present day there is practically unanimous agreement as to the spelling of most words $\ldots$ At any given moment, however, a relatively small number of words may be spelled in more than 1 way. Gradually, as a rule, 1 of these forms comes to be generally preferred, \& the less customary form comes to look obsolete \& is discarded. From time to time new forms, mostly simplifications, are introduced by innovators, \& either win their place or die of neglect.
	
	The practical objection to unaccepted \& oversimplified spellings is the disfavor with which they are received by the reader. They distract his attention \& exhaust his patience. He reads the form though automatically, without thought of its needless complexity; he reads the abbreviation tho \& mentally supplies the missing letters, at the cost of a fraction of his attention. The writer has defeated his own purposed.
\end{quotation}
The language manages somehow to keep pace with events. A word that has taken hold in our century is {\it thru-way}; it was born of necessity \& is apparently here to stay. In combination with {\it way, thru} is more serviceable than {\it through}; it is a high-speed word for readers who are going 65. {\it Throughway} would be too long to fit on a road sign, too slow to serve the speeding eye. It is conceivable that because of our thruways, {\it through} will eventually become {\it thru} -- after many more thousands of miles of travel.'' -- \cite[p. 88]{Strunk_White_element_style}

%------------------------------------------------------------------------------%

\subsubsection{Do not explain too much.}
``It is seldom advisable to tell all. Be sparing, e.g., in the use of adverbs after ``he said,'' ``she replied,'' \& the like: ``he said consolingly''; ``she replied grumblingly.'' Let the conversation itself disclose the speaker's manner of condition. Dialogue heavily weighted with adverbs after the attributive verb is cluttery \& annoying. Inexperienced writers not only overwork their adverbs but load their attributives with explanatory verbs: ``he consoled,'' ``she congratulated.'' They do this, apparently, in the belief that the word {\it said} is always in need of support, or because they have been told to do it by experts in the art of bad writing.'' -- \cite[p. 89]{Strunk_White_element_style}

%------------------------------------------------------------------------------%

\subsubsection{Do not construct awkward adverbs.}
``Adverbs are easy to build. Take an adjective or a participle, add {\it -ly}, \& behold! you have an adverb. But you'd probably be better off without it. Do not write {\it tangledly}. The word itself is a tangle. Do not even write {\it tiredly}. Nobody says {\it tangledly} \& not many people say {\it tiredly}. Words that are not used orally are seldom the ones to put on paper.
\begin{example}
	He climbed tiredly to bed. $\to$ He climbed wearily to bed.
	
	The lamp cord lay tangledly beneath her chair. $\to$ The lamp cord lay in tangles beneath her chair.
\end{example}
Do not dress words up by adding {\it -ly} to them, as though putting a hat on a horse.
\begin{example}
	overly $\to$ over, muchly $\to$ much, thusly $\to$ thus.'' -- \cite[p. 90]{Strunk_White_element_style}
\end{example}


%------------------------------------------------------------------------------%

\subsubsection{Make sure the reader knows who is speaking.}
``Dialogue is a total loss unless you indicate who the speaker is. In long dialogue passages containing no attributives, the reader may become lost \& be compelled to go back \& reread in order to puzzle the thing out. Obscurity is an imposition on the reader, to say nothing of its damage to the work.

In dialogue, make sure that your attributives do not awkwardly interrupt a spoken sentence. Place them where the break would come naturally in speech -- i.e., where the speaker would pause for emphasis, or take a breath. The best test for locating an attributive is to speak the sentence aloud.
\begin{example}
	``Now, my boy, we shall see,'' he said, ``how well you have learned your lesson.'' $\to$ ``Now, my boy,'' he said, ``we shall see how well you have learned your lesson.''
	
	``What's more, they would never,'' she added, ``consent to the plan.'' $\to$  ``What's more,'' she added, ``they would never consent to the plan.'''' -- \cite[p. 91]{Strunk_White_element_style}
\end{example}

%------------------------------------------------------------------------------%

\subsubsection{Avoid fancy words.}
``Avoid the elaborate, the pretentious, the coy, \& the cute. Do not be tempted by a 20-dollar word when there is a 10-center handy, ready \& able. Anglo-Saxon is a livelier tongue than Latin, so use Anglo-Saxon words. In this, as in so many matters pertaining to style, one's ear must be one's guide: {\it gut} is a lustier noun than {\it intestine}, but the 2 words are not interchangeable, because {\it gut} is often inappropriate, being too coarse for the context. Never call a stomach a tummy without good reason.

If you admire fancy words, if every sky is {\it beauteous}, every blonde {\it curvaceous}, every intelligent child prodigious, if you are tickled by {\it discombobulate}, you will have a bad time with Reminder 14. What is wrong, you ask, with {\it beauteous?} No one knows, for sure. There is nothing wrong, really, with any word -- all are good, but some are better than others. A matter of ear, a matter of reading the books that sharpen the ear.

The line between the fancy \& the plain, between the atrocious \& the felicitous, is sometimes alarmingly fine. The opening phrase of the Gettysburg address is close to the line, at least by our standards today, \& Mr. Lincoln, knowingly or unknowingly, was flirting with disaster when he wrote ``4 score \& 7 years ago.'' The President could have got into his sentence with plain ``87'' -- a saving of 2 words \& less of a strain on the listeners' powers of multiplication. But Lincoln's ear must have told him to go ahead with 4 score \& 7. By doing so, he achieved cadence while skirting the edge of fanciness. Suppose he had blundered over the line \& written, ``In the year of our Lord seventeen hundred \& seventy-six.'' His speech would have sustained a heavy blow. Or suppose he had settle for ``87.'' In that case he would have got into his introductory sentence too quickly; the timing would have been bad.

The question of ear is vital. Only the writer whose ear is reliable is in a position to use bad grammar deliberately; this writer knows for sure when a colloquialism is better than formal phrasing \& is able to sustain the work at a level of good taste. So cock your ear. Years ago, students were warned not to end a sentence with a preposition; time, of course, has softened that rigid decree. Not only is the preposition acceptable at the end, sometimes it is more effective in that spot than anywhere else. ``A claw hammer, not an ax, was the tool he murdered her with.'' This is preferable to ``A claw hammer, not an ax, was the tool with which he murdered her.'' Why? Because it sounds more violent, more like murder. A matter of ear.

\& would you write ``The worst tennis player around here is I'' or ``The The worst tennis player around here is me''? The 1st is good grammar, the 2nd is good judgment -- although the {\it me} might not do in all contexts.

The split infinitive is another trick of rhetoric in which the ear must be quicker than the handbook. Some infinitives seem to improve on being split, just as a stick of round stovewood does. ``I cannot bring myself to really like the fellow.'' The sentence is relaxed, the meaning is clear, the violation is harmless \& scarcely perceptible. Put the other way, the sentence becomes stiff, needlessly formal. A matter of ear.

There are times when the ear not only guides us through difficult situations but also saves us from minor or major embarrassments of prose. The ear, e.g., must decide when to omit {\it that} from a sentence, when to retain it. ``She knew she could do it'' is preferable to ``She knew that she could do it'' -- simpler \& just as clear. Bu tin many cases the {\it that} is needed. ``He felt that his big nose, which was sunburned, made him look ridiculous.'' Omit the {\it that} \& you have ``He felt his big nose $\ldots$'''' -- \cite[p. 93]{Strunk_White_element_style}

%------------------------------------------------------------------------------%

\subsubsection{Do not use dialect unless your ear is good.}
``Do not attempt to use dialect unless you are a devoted student of the tongue you hope to reproduce. If you use dialect, be consistent. The reader will become impatient or confused upon finding 2 or more versions of the same word or expression. In dialect it is necessary to spell phonetically, or at least ingeniously, to capture unusual inflections. Take, e.g., the word {\it once}. It often appears in dialect writing as {\it oncet}, but {\it oncet} looks as though it should be pronounced ``onset.'' A better spelling would be {\it wunst}. But if you write it {\it oncet} once, write it that way throughout. The best dialect writers, by \& large, are economical of their talents; they use the minimum, not the maximum, of deviation from the norm, thus sparing their readers as well as convincing them.'' -- \cite[p. 94]{Strunk_White_element_style}

%------------------------------------------------------------------------------%

\subsubsection{Be clear.}
``Clarity is not the prize in writing, nor it is always the principal mark of a good style. There are occasions when obscurity serves a literary yearning, if not a literary purpose, \& there are writers whose mien is more overcast than clear. But since writing is communication, clarity can only be a virtue. \& although there is no substitute for merit in writing, clarity comes closest to being one. Even to a writer who is being intentionally obscure or wild of tongue we can say, ``be obscure clearly! Be wild of tongue in a way we can understand!'' Even to writers of market letters, telling us (but not telling us) which securities are promising, we can say, ``Be cagey plainly! Be elliptical in a straightforward fashion!''

Clarity, clarity, clarity. When you become hopelessly mired in a sentence, it is best to start fresh; do not try to fight your way through against the terrible odds of syntax. Usually what is wrong is that the construction has become too involved at some point; the sentence needs to be broken apart \& replaced by 2 or more shorter sentences.

Muddiness is not merely a disturber of prose, it is also destroyer of life, of hope: death on the highway caused by a badly worded road sign, heartbreak among lovers caused by a misplaced phrase in a well-intentioned letter, anguish of a traveler expecting to be met at a railroad station \& not being met because of a slipshod telegram. Think of the tragedies that are rooted in ambiguity, \& be clear! When you say something, make sure you have said it. The chances of your having said it are only fair.'' -- \cite[p. 95]{Strunk_White_element_style}

%------------------------------------------------------------------------------%

\subsubsection{Do not inject opinion.}
``Unless there is a good reason for its being there, do not inject opinion into a piece of writing. We all have opinions about almost everything, \& the temptation to toss them in is great. To air one's views gratuitously, however, is to imply that the demand for them is brisk, which may not be the case, \& which, in any event, may not be relevant to the discussion. Opinions scattered indiscriminately about leave the mark of egotism on a work. Similarly, to air one's views at an improper time may be in bad taste. If you have received a letter inviting you to speak at the dedication of a new cat hospital, \& you have cats, your reply, declining the invitation, does not necessarily have to cover the full range of your emotions. You must make it clear that you will not attend, but you do not have to let fly at cats. The writer of the letter asked a civil question; attack cats, then, only if you can do so with good humor, good taste, \& in such a way that your answer will be courteous as well as responsive. Since you are out of sympathy with cats, you may quite properly give this as a reason for not appearing at the dedicatory ceremonies of a cat hospital. But bear in mind that your opinion of cats was not sought, only your services as a speaker. Try to keep things straight.'' -- \cite[p. 96]{Strunk_White_element_style}

%------------------------------------------------------------------------------%

\subsubsection{Use figures of speech sparingly.}
``The simile is a common device \& a useful one, but similes coming in rapid fire, one right on top of another, are more distracting than illuminating. Readers need time to catch their breath; they can't be expected to compare everything with something else, \& no relief in sight.

When you use metaphor, do not mix it up. I.e., don't start by calling something a swordfish \& end by calling it an hourglass.'' -- \cite[p. 97]{Strunk_White_element_style}

%------------------------------------------------------------------------------%

\subsubsection{Do not take shortcuts at the cost of clarity.}
``Do not use initials for the names of organizations or movements unless you are certain the initials will be readily understood. Write things out. Not everyone knows that MADD means Mothers Against Drunk Driving, \& even if everyone did, there are babies being born every minute who will someday encounter the name for the 1st time. They deserve to see the words, not simply the initials. A good rule is to start your article by writing out names in full, \& then, later, when your readers have got their bearings, to shorten them.

Many shortcuts are self-defeating; they waste the reader's time instead of conserving it. There are all sorts of rhetorical stratagems \& devices that attract writers who hope to be pithy, but most of them are simply bothersome. The longest way round is usually the shortest home, \& the one truly reliable shortcut in writing is to choose words that are strong \& surefooted to carry readers on their way.'' -- \cite[p. 98]{Strunk_White_element_style}

%------------------------------------------------------------------------------%

\subsubsection{Avoid foreign languages.}
``The writer will occasionally find it convenient or necessary to borrow from other languages. Some writers, however, from sheer exuberance or a desire to show off, sprinkle their work liberally with foreign expressions, with no regard for the reader's comfort. It is a bad habit. Write in English.'' -- \cite[p. 99]{Strunk_White_element_style}

%------------------------------------------------------------------------------%

\subsubsection{Prefer the standard to the offbeat.}
``Young writers will be drawn at every turn toward eccentricities in language. They will hear the beat of new vocabularies, the exciting rhythms of special segments of their society, each speaking a language of its own. All of us come under the spell of these unsettling drums; the problem for beginners is to listen to them, learn the words, feel the vibrations, \& not be carried away.

Youths invariably speak to other youths in a tongue of their own devising: they renovate the language with a wild vigor, as they would a basement apartment. By the time this paragraph sees print, {\it psyched, nerd, ripoff, dude, geek}, \& {\it funky} will be the words of yesteryear, \& we will be fielding more recent ones that have come bouncing into our speech -- some of them into our dictionary as well. A new word is always up for survival. Many do survive. Others grow stale \& disappear. Most are, at least in their infancy, more approximate to conversation than to composition.

Today, the language of advertising enjoys an enormous circulation. With its deliberate infractions of grammatical rules \& its crossbreeding of the parts of speech, it profoundly influences the tongues \& pens of children \& adults. Your new kitchen range is so revolutionary it {\it obsoletes} all other ranges. Your counter top is beautiful because it is {\it accessorized} with gold-plated faucets. Your cigarette tastes good {\it like} a cigarette should. \&, {\it like the man says}, you will want to try one. You will also, in all probability, want to try writing that way, using that language. You do so at your peril, for it is the language of mutilation.

Advertisers are quite understandably interested in what they call ``attention getting.'' The man photographed must have lost an eye or grown a pink beard, or he must have 3 arms or be sitting wrong-end-to on a horse. This technique is proper in its place, which is the world of selling, but the young writer had best not adopt the device of mutilation in ordinary composition, whose purpose is to engage, not paralyze, the readers senses. Buy the gold-plated faucets if you will, but do not accessorize your prose. To use the language well, do not begin by hacking it to bits; accept the whole body of it, cherish its classic form, its variety, \& its richness.

Another segment of society that has constructed a language of its own is business. People in business say that toner cartridges are {\it in short supply}, that they have {\it updated} the next shipment of these cartridges, \& that they will {\it finalize} their recommendations at the next meeting of the board. They are speaking a language familiar \& dear to them. Its portentous nouns \& verbs invest ordinary events with high adventure; executives walk among toner cartridges, caparisoned like knights. We should tolerate them -- every person of spirit wants to ride a white horse. The only question is whether business vocabulary is helpful to ordinary prose. Usually, the same ideas can be expressed less formidably, if one makes the effort. A good many of the special words of business seem designed more to express the user's dreams than to express a precise meaning. Not all such words, of course, can be dismissed summarily; indeed, no word in the language can be dismissed offhand by anyone who has a healthy curiosity. {\it Update} isn't a bad word; in the right setting it is useful. In the wrong setting, though, it is destructive, \& the trouble with adopting coinages too quickly is that they will bedevil one by insinuating themselves where they do not belong. This may sound like rhetorical snobbery, or plain stuffiness; but you will discover, in the course of your work, that the setting of a word is just as restrictive as the setting of a jewel. The general rule here is to prefer the standard. {\it Finalize}, for instance, is not standard; it is special, \& it is a peculiarly fuzzy \& silly word. Does it mean ``terminate,'' or does it mean ``put into final form''? One can't be sure, really, what it means, \& one gets the impression that the person using it doesn't know, either, \& doesn't want to know.

The special vocabularies of the law, of the military, of government are familiar to most of us. Even the world of criticism has a modest pouch of private words ({\it luminous, taut}), whose only virtue is that they are exceptionally nimble \& can escape from the garden of meaning over the wall. Of these critical words, Wilcott Gibbs once wrote, ``$\ldots$ they are detached from the language \& inflated like little balloons.'' The young writer should learn to spot them -- words that at 1st glance seem freighted with delicious meaning but that soon burst in air, leaving nothing but a memory of bright sound.

The language is perpetually in flux: it is a living stream, shifting, changing, receiving new strength from a thousand tributaries, losing old forms in the backwaters of time. To suggest that a young writer not swim in the main stream of this turbulence would be foolish indeed, \& such is not the intent of these cautionary remarks. The intent is to suggest that in choosing between the formal \& the informal, the regular \& the offbeat, the general \& the special, the orthodox \& the heretical, the beginner err on the side of conservatism, on the side of established usage. No idiom is taboo, no accent forbidden; there is simply a better chance of doing well if the writer holds a steady course, enters the stream of English quietly, \& does not thrash about.

``But,'' you may ask, ``what if it comes natural to me to experiment rather than conform? What if I am a pioneer, or even a genius?'' Answer: then be one. But do not forget that what may seem like pioneering may be merely evasion, or laziness -- the disinclination to submit to discipline. Writing good standard English is no cinch, \& before you have managed it you will have encountered enough rough country to satisfy even the most adventurous spirit.

Style takes its final shape more from attitudes of mind than from principles of composition, for, as an elderly practitioner once remarked, ``Writing is an act of faith, not a trick of grammar.'' This moral observation would have no place in a rule book were it not that style {\it is} the writer, \& therefore what you are, rather than what you know, will at last determine your style. If you write, you must believe -- in the truth \& worth of the scrawl, in the ability of the reader to receive \& decode the message. No one can write decently who is distrustful of the reader's intelligence, or whose attitude is patronizing.

Many references have been made in this book to ``the reader,'' who has been much in the news. It is now necessary to warn you that you concern for the reader must be pure: you must sympathize with the reader's plight (most readers are in trouble about half the time) but never seek to know the reader's wants. Your whole duty as a writer is to please \& satisfy yourself, \& the true writer always plays to an audience of one. Start sniffing the air, or glancing at the Trend Machine, \& you are as good as dead, although you may make a nice living.

Full of belief, sustained \& elevated by the power of purpose, armed with the rules of grammar, you are ready to exposure. At this point, you may well pattern yourself on the fully exposed cow of Robert Louis Stevenson's rhyme. This friendly \& commendable animal, you may recall, was ``blown by all the winds that pass{\tt/}\& wet with all the showers.'' \& so must you as a young writer be. In our modern idiom, we should say that you must get wet all over. Mr. Stevenson, working in a plainer style, said it with felicity, \& suddenly 1 cow, out of so many, received the gift of immortality. Like the steadfast writer, she is at home in the wind \& the rain; \&, thanks to 1 moment of felicity, she will live on \& on \& one.'' -- \cite[pp. 101--103]{Strunk_White_element_style}

%------------------------------------------------------------------------------%

\subsection{Afterword}
``Will Strunk \& E. B. White were unique collaborators. Unlike Gilbert \& Sullivan, or Woodward \& Bernstein, they worked separately \& decades apart.

We have no way of knowing whether Prof. Strunk took particular notice of Elwyn Brooks White, a student of his at Cornell University in 1919. Neither teacher nor pupil could have realized that their names would be linked as they now are. Nor could they have imagined that 38 years after they met, White would take this little gem of a textbook that Strunk had written for his students, polish it, expand it, \& transform it into a classic.

E. B. White shared Strunk's sympathy for the reader. To Strunk's do's \& don'ts he added passages about the power of words \& the clear expression of thoughts \& feelings. To the nuts \& bolts of grammar he added a rhetorical dimension.

The editors of this edition have followed in White's footsteps, once again providing fresh examples \& modernizing usage where appropriate. {\it The Elements of Style} is still a little book, small enough \& important enough to carry in your pocket, as I carry mine. It has helped me to write better. I believe it can do the same for you.

{\sc Charles Osgood}'' -- \cite[p. 104]{Strunk_White_element_style}

%------------------------------------------------------------------------------%

\subsection{Glossary}

\begin{enumerate}
	\item {\bf adjectival modifier.} A word, phrase, or clause that acts as an adjective in qualifying the meaning of a noun or pronoun, {\it Your} country; a {\it turn-of-the-century} style; people {\it who are always late}.
	\item {\bf adjective.} A word that modifies, quantifies, or otherwise describes a noun or pronoun.
	
	{\it Drizzly} November; midnight {\it dreary}; {\it only} requirement.
	\item {\bf adverb.} A word that modifies or otherwise qualifies a verb, an adjective, or another adverb.
	
	Gestures {\it gracefully}; {\it exceptionally} quiet engine.
	\item {\bf adverbial phrase.} A phrase that functions as an adverb. (See {\it phrase}.)
	
	Landon laughs {\it with abandon}.
	\item {\bf agreement.} The correspondence of a verb with its subject in person \& number (Karen {\it goes} to Cal Tech; her sisters {\it go} to UCLA), \& of a pronoun with its antecedent in person, number, \& gender (Ass soon as Karen finished the exam, {\it she} picked up {\it her} books \& left the room).
	\item {\bf antecedent.} The noun to which a pronoun refers.
	
	A pronoun \& its antecedent must agree in person, number, \& gender.
	
	Michael \& {\it his} teammates moved off campus.
	\item {\bf appositive.} A noun or noun phrase that renames or adds identifying information to a noun it immediately follows.
	
	His brother, {\it an accountant with Arthur Andersen}, was recently promoted.
	\item {\bf articles.} The words {\it a, an}, \& {\it the}, which signal or introduce nouns.
	
	The definite article {\it the} refers to a particular item: {\it the} report.
	
	The indefinite articles {\it a} \& {\it an} refer to a general item or one not already mentioned: {\it an} apple.
	\item {\bf auxiliary verb.} A verb that combines with the main verb to show differences in tense, person, \& voice.
	
	The most common auxiliaries are forms of {\it be, do}, \& {\it have}.
	
	I {\it am} going; we {\it did} not go; they {\it have} gone. (See also {\it modal auxiliaries}.)
	\item {\bf case.} The form of a noun or pronoun that reflects its grammatical function in a sentence as subject ({\it they}), object ({\it them}), or possessor ({\it their}).
	
	{\it She} gave {\it her} employees a raise that pleased {\it them} greatly.
	\item {\bf clause.} A group of related words that contains a subject \& predicate.
	
	{\it Moths swarm} around a burning candle.
	
	While {\it she was taking} the test, {\it Karen muttered} to herself.
	\item {\bf colloquialism.} A word or expression appropriate to informal conversation but not usually suitable for academic or business writing.
	
	They wanted to {\it get even} (instead of they wanted to {\it retaliate}).
	\item {\bf complement.} A word or phrase (especially a noun or adjective) that completes the predicate.
	\begin{itemize}
		\item {\bf Subject complements} complete linking verbs \& rename or describe the subject: Martha is my {\it neighbor}.
		
		She seems {\it shy}.
		\item {\bf Object complements} complete transitive verbs by describing or renaming the direct object: They found the play {\it exciting}.
		
		Robert considers Mary {\it a wonderful wife}.
	\end{itemize}
	\item {\bf compound sentence.} 2 or more independent clauses joined by a coordinating conjunction, a correlative conjunction, or a semicolon.
	
	{\it Caesar conquered Gaul}, but {\it Alexander the Great conquered the world}.
	\item {\bf compound subject.} 2 or more simple subjects joined by a coordinating or correlative conjunction.
	
	{\it Hemingway \& Fitzgerald} had little in common.
	\item {\bf conjunction.} A word that joins words, phrases, clauses, or sentences.
	
	The coordinating conjunctions, {\it and, but, or, nor, yet, so, for}, join grammatically equivalent elements.
	
	Correlative conjunctions ({\it both, and; either, or; neither, nor}) join the same kinds of elements.
	\item {\bf contraction.} A shortened form of a word or group of words: {\it can't} for cannot; {\it they're} for they are.
	\item {\bf correlative expression.} See {\it conjunction}.
	\item {\bf dependent clause.} A group of words that includes a subject \& verb but is subordinate to an independent clause in a sentence.
	
	Dependent clauses begin with either a subordinating conjunction, e.g., {\it if, because, since}, or a relative pronoun, e.g., {\it who, which, that}.
	
	{\it When it gets dark}, we'll find a restaurant {\it that has music}.
	\item {\bf direct object.} A noun or pronoun that receives the action of a transitive verb.
	
	Pearson publishes {\it books}.
	\item {\bf gerund.} The {\it -ing} form of a verb that functions as a noun: {\it Hiking} is good exercise.
	
	She was praised for her {\it playing}.
	\item {\bf indefinite pronoun.} A pronoun that refers to an unspecified person ({\it anybody}) or thing ({\it something}).
	\item {\bf independent clause.} A group of words with a subject \& verb that can stand alone as a sentence.
	
	{\it Raccoons steal food}.
	\item {\bf indirect object.} A noun or pronoun that indicates to whom or for whom, to what or for what the action of a transitive verb is performed.
	
	I asked {\it her} a question.
	
	Ed gave {\it the door} a kick.
	\item {\bf infinitive/split infinitive.} In the present tense, a verb phrase consisting of {\it to} followed by the base form of the verb ({\it to write}).
	
	A split infinitive occurs when 1 or more words separate {\it to} \& the verb ({\it to boldly go}).
	\item {\bf intransitive verb.} A verb that does not take a direct object.
	
	His {\it nerve failed}.
	\item {\bf linking verb.} A verb that joins the subject of a sentence to its complement.
	
	Prof. Chapman {\it is a} philosophy teacher.
	
	They {\it were} ecstatic.
	\item {\bf loose sentence.} A sentence that begins with the main idea \& then attaches modifiers, qualifiers, \& additional details: He was determined to succeed, with or without the promotion he was hoping for \& in spite of the difficulties he was confronting at every turn.
	\item {\bf main clause.} An independent clause, which can stand alone as a grammatically complete sentence.
	
	Grammarians quibble.
	\item {\bf modal auxiliaries.} Any of the verbs that combine with the main verb to express obligation ({\it must}), necessity ({\it should}), permission ({\it may}), probability ({\it might}), possibility ({\it could}), ability ({\it can}), or tentativeness ({\it would}).
	
	{\it Mary might} wash the car.
	\item {\bf modifier.} A word or phrase that qualifies, describes, or limits the meaning of a word, phrase, or clause.
	
	{\it Frayed} ribbon, {\it dancing} flowers, {\it worldly} wisdom.
	\item {\bf nominative pronoun.} A pronoun that functions as a subject or subject complement: {\it I, we, you, he, she, it, they, who}.
	\item {\bf nonrestrictive modifier.} A phrase or clause that does not limit or restrict the essential meaning of the element it modifies.
	
	My youngest niece, {\it who lives in Ann Arbor}, is a magazine editor.
	\item {\bf noun.} A word that names a person, place, thing, or idea.
	
	Most nouns have a plural form \& a possessive form.
	
	{\it Carol}; the {\it park}; the {\it cup}; {\it democracy}.
	\item {\bf number.} A feature of nouns, pronouns, \& a few verbs, referring to singular or plural.
	
	A subject \& its corresponding verb must be consistent in number; a pronoun should agree in number with its antecedent.
	
	A {\it solo flute plays}; 2 {\it oboes join} in.
	\item {\bf object.} The noun or pronoun that completes a prepositional phrase or the meaning of a transitive verb.
	
	(See also {\it direct object, indirect object}, \& {\it preposition}.)
	
	Frost offered {\it his audience a poetic performance} they would likely never forget.
	\item {\bf participial phrase.} A present or past participle with accompanying modifiers, objects, or complements.
	
	The buzzards, {\it circling with sinister determination}, squawked loudly.
	\item {\bf participle.} A verb that functions as an adjective.
	
	Present participles end in {\it -ing} ({\it brimming}); past participles typically end in {\it -d} or {\it -ed} ({\it injured}) or {\it -en} ({\it broken}) but may appear in other forms ({\it brought, been, gone}).
	\item {\bf periodic sentence.} A sentence that expresses the main idea at the end.
	
	With or without their parents' consent, \& whether or not they receive the assignment relocation they requested, {\it they are determined to get married}.
	\item {\bf phrase.} A group of related words that functions as a unit but lacks a subject, a verb, or both.
	
	{\it Without the resources to continue}.
	\item {\bf possessive.} The case of nouns \& pronouns that indicates ownership or possession ({\it Harold's, ours, mine}).
	\item {\bf predicate.} The verb \& its related words in a clause or sentence.
	
	The predicate expresses what the subject does, experiences, or is.
	
	{\it Birds fly}.
	
	{\it The partygoers celebrated wildly for a long time}.
	\item {\bf preposition.} A word that relates its object (a noun, pronoun, or {\it -ing} verb form) to another word in the sentence.
	
	She is the leader {\it of} our group.
	
	We opened the door {\it by} picking the lock.
	
	She went {\it out} the window.
	\item {\bf prepositional phrase.} A group of words consisting of a preposition, its object, \& any of the object's modifiers.
	
	Georgia {\it on my mind}.
	\item {\bf principal verb.} The predicating verb in a main clause or sentence.
	\item {\bf pronominal possessive.} Possessive pronouns, e.g., {\it hers, its}, \& {\it theirs}.
	\item {\bf proper noun.} The name of a particular person ({\it Frank Sinatra}), place ({\it Boston}), or thing ({\it Moby Dick}).
	
	Proper nouns are capitalized.
	
	Common nouns name classes of people ({\it singers}), places ({\it cities}), or things ({\it books}) \& are not capitalized.
	\item {\bf relative clause.} A clause introduced by a relative pronoun, e.g., {\it who, which, that}, or by a relative adverb, e.g., {\it where, when, why}.
	\item {\bf relative pronoun.} A pronoun that connects a dependent clause to a main clause in a sentence: {\it who, whom, whose, which, that, what, whoever, whomever, whichever}, \& {\it whatever}.
	\item {\bf restrictive term, element, clause.} A phrase or clause that limits the essential meaning of the sentence element it modifies or identifies.
	
	Professional athletes {\it who perform exceptionally} should earn stratospheric salaries.
	
	Since there are no commas before \& after the italicized clause, the italicized clause is restrictive \& suggests that only those athletes who perform exceptionally are entitled to such salaries.
	
	If commas were added before {\it who} \& after {\it exceptionally}, the clause would be nonrestrictive \& would suggest that {\it all} professional athletes should receive stratospheric salaries.
	\item {\bf sentence fragment.} A group of words that is not grammatically a complete sentence but is punctuated as one: {\it Because it mattered greatly}.
	\item {\bf subject.} The noun or pronoun that indicates what a sentence is about, \& which the principal verb of a sentence elaborates.
	
	{\it The new Steven Spielberg movie} is a box office hit.
	\item {\bf subordinate clause.} A clause dependent on the main clause in a sentence.
	
	{\it After we finish our work}, we will go out for dinner.
	\item {\bf syntax.} The order or arrangement of words in a sentence.
	
	Syntax may exhibit parallelism ({\it I came, I saw, I conquered}), inversion ({\it Whose woods these are I think I know}), or other formal characteristics.
	\item {\bf tense.} The time of a verb's action or state of being, e.g., past, present, or future.
	
	{\it Saw, see, will see}.
	\item {\bf transition.} A word or group of words that aids coherence in writing by showing the connections between ideas.
	
	William Carlos Williams was influenced by the poetry of Walt Whitman.
	
	{\it Moreover}, Williams's emphasis on the present \& the immediacy of the ordinary represented a rejection of the poetic stance \& style of his contemporary T. S. Eliot.
	
	{\it In addition}, William's poetry$\ldots$
	\item {\bf transitive verb.} A verb that requires a direct object to complete its meaning: They {\it washed} their new car.
	
	An {\it intransitive verb} does not require an object to complete its meaning: The audience {\it laughed}.
	
	Many verbs can be both: The wind {\it blew} furiously.
	
	My car {\it blew} a gasket.
	\item {\bf verb.} A word or group of words that expresses the action or indicates the state of being of the subject.
	
	\fbox{Verbs {\it activate} sentences.}
	\item {\bf verbal.} A verb form that functions in a sentence as a noun, an adjective, or an adverb rather than as a principal verb.
	
	{\it Thinking} can be fun.
	
	An {\it embroidered} handkerchief.
	
	(See {\it also gerund, infinitive}, \& {\it participle}.)
	\item {\bf voice.} The attribute of a verb that indicates whether its subject is active (Janet {\it played} the guitar) or passive (The guitar {\it was played} by Janet).\hfill$\square$
\end{enumerate}

%------------------------------------------------------------------------------%

\section{{\sc David Foster Wallace}. This Is Water: Some Thoughts, Delivered on a Significant Occasion, about Living a Compassionate Life}
\textbf{\textsf{Resources -- Tài nguyên.}}
\begin{enumerate}
	\item \cite{Wallace_water}. {\sc David Foster Wallace}. {\it This Is Water: Some Thoughts, Delivered on a Significant Occasion, about Living a Compassionate Life}
\end{enumerate}
``This Is Water

There are these 2 young fish swimming along \& they happen to meet an older fish swimming the other way, who nods at them \& says, ``Morning, boys, How's the water?''

\& the 2 young fish swim on for a bit, \& then eventually 1 of them looks at the other \& goes, ``What the hell is water?''

This is a standard requirement of US commencement speeches, the deployment of didactic little parable-ish stories.

The story thing turns out to be 1 of the better, less bullshitty conventions of the genre $\ldots$ but if you're worried that I plan to present myself here as the wise old fish explaining what water is to you younger fish, please don't be.

I am not the wise old fish.

The immediate point of the fish story is merely that the most obvious, ubiquitous, important realities are often the ones that are hardest to see \& talk about.

Stated as an English sentence, of course, this is just a banal platitude -- but the fact is that, in the day to day trenches of adult existence, banal platitudes can have a life-or-death importance.

Or so I wish to suggest to you on this dry \& lovely morning.

Of course the main requirement of speeches like this is that I'm supposed to talk about your liberal arts education's meaning, to try to explain why the degree you're about to receive has actual human value instead of just a material payoff.

So let's talk about the single most pervasive clich\'e in the commencement speech genre, which is that a liberal arts education is not so much about filling you up with knowledge as it is about, quote, ``teaching you how to think.''

If you're like me as a college student, you've never liked hearing this, \& you tend to feel a bit insulted by the claim that you've needed anybody to teach you how to think, since the fact that you even got admitted to a college this good seems like proof that you already know how to think.

But I'm going to posit to you that the liberal arts clich\'e turns out not to be insulting at all, because the really significant education in thinking that we are supposed to get in a place like this isn't really about the capacity to think, but rather about the choice of what to think about.

If your complete freedom of choice regarding what to think about seems too obvious to waste time talking about, I'd ask you to think about fish \& water, \& to bracket, for just a few minutes, your skepticism about the value of the totally obvious.

Here's another didactic little story.

There are these 2 guys sitting together in a bar in the remote Alaskan wilderness.

1 of the guys is religious, the other's an atheist, \& they're arguing about the existence of God with that special intensity that comes after abut the 4th beer.

\& the atheist says, ``Look, it's not like I don't have actual reasons for not believing in God.

It's not like I haven't ever experimented with the whole God-\&-prayer thing.

Just last month, I got caught off away from the camp in that terrible blizzard, \& I couldn't see a thing, \& I was totally lost, \& it was 50 below, \& so I did, I tried it: I fell to my knees in the snow \& cried out, `God, if there is a God, I'm lost in this blizzard, \& I'm gonna die if you don't help me!''

\& now, in the bar, the religious guy looks at the atheist all puzzled: ``Well then, you must believe now,'' he says. ``After all, here you are, alive.''

The atheist rolls his eyes like the religious guy is a total simp: ``No, man, all that happened was that a couple Eskimos just happened to come wandering by, \& they showed me the way back to the camp.''

It's easy to run this story through a kind of standard liberal arts analysis: The exact same experience can mean 2 completely different things to 2 different people, given those people's 2 different belief templates \& 2 different ways of constructing meaning from experience.

Because we prize tolerance \& diversity of belief, nowhere in our liberal arts analysis do we want to claim that one guy's interpretation is true \& the other guy's is false or bad.

Which is fine, except we also never end up talking about just where these individual templates \& beliefs come from, meaning, where they come from {\it inside} the 2 guys.

As if a person's most basic orientation toward the world \& the meaning of his experience were somehow automatically hardwired, like height or shoe size, or absorbed from the culture, like language.

As if how we construct meaning were not actually a matter of personal, intentional choice, of conscious decision.

Plus, there's the matter of arrogance.

The nonreligious guy is so totally, obnoxiously confident in his dismissal of the possibility that the Eskimos had anything to do with his prayer for help.

True, there are plenty of religious people who seem arrogantly certain of their own interpretations, too.

They're probably even more repulsive than atheists, at least to most of us here, but the fact is that religious dogmatists' problem is exactly the same as the story's atheist's -- arrogance, blind certainty, a closed-mindedness that's like an imprisonment so complete that the prisoner doesn't even know he's locked up.

The point here is that I think this is 1 part of what the liberal arts mantra of ``teaching me how to think'' is really supposed to mean: To be just a little less arrogant, to have some ``critical awareness'' abut myself \& my certainties $\ldots$ because a huge percentage of the stuff that I tend to be automatically certain of is, it turns out, totally wrong \& deluded.

I have learned this the hard way, as I predict you graduates will, too.

Here's 1 example of the utter wrongness of something I tend to be automatically sure of.

Everything in my own immediate experience supports my deep belief that I am the absolute center of the universe, the realest, most vivid \& important person in existence.

We rarely think about this sort of natural, basic self-centeredness, because it's so socially repulsive, but it's pretty much the same for all of us, deep down.

It is our default setting, hardwired into our boards at birth.

Think about it: There is no experience you've had that you were not at the absolute center of.

The world as you experience it is there in front of you, or behind you, to the left or right of you, on your TV, on your monitor, or whatever.

Other people's thoughts \& feelings have to be communicated to you somehow, but your own are so immediate, urgent, {\it real}.

You get the idea.

But please don't worry that I'm getting ready to preach to you about compassion or other-directedness or all the so-called ``virtues.''

This is not a matter of virtue -- It's a matter of my choosing to do the work of somehow altering or getting free of my natural, hardwired default setting, which is to be deeply \& literally self-centered, \& to see \& interpret everything through this lens of self.

People who {\it can} adjust their natural default setting this way are often described as being, quote, ``well-adjusted,'' which I suggest to you is not an accidental term.

Given the academic setting here, an obvious question is how much of this work of adjusting our default setting involves actual knowledge or intellect.

The answer, not surprisingly, is that it depends on what kind of knowledge we're talking about.

Probably the most dangerous thing about an academic education, at least in my own case, is that it enables my tendency to over-intellectualize stuff, to get lost in abstract thinking instead of simply paying attention to what's going on in front of me.

Instead of paying attention to what's going on {\it inside} me.

As I'm sure you guys know by now, it is extremely difficult to stay alert \& attentive instead of getting hypnotized by the constant monologue inside your head.

What you don't yet know are the stakes of this struggle.

In the 20 years since my own graduation, I have come gradually to understand these stakes, \& to see that the liberal arts clich\'e about ``teaching you how to think'' was actually shorthand for a very deep \& important truth.

``Learning how to think'' really means learning how to exercise some control over {\it how} \& {\it what} you think.

It means being conscious \& aware enough to {\it choose} what you pay attention to \& to {\it choose} how you construct meaning from experience.

Because if you cannot or will not exercise this kind of choice in adult life, you will be totally hosed.

Think of the old clich\'e about the mind being ``an excellent servant but a terrible master.''

This, like many clich\'es, so lame \& banal on the surface, actually expresses a great \& terrible truth.

It is not the least bit coincidental that adults who commit suicide with firearms nearly always shoot themselves in $\ldots$ the {\it head}.

\& the truth is that most of these suicides are actually dead long before they pull the trigger.

\& I submit that this is what the real, no-shit value of your liberal arts education is supposed to be about: How to keep from going through your comfortable, prosperous, respectable adult life dead, unconscious, a slave to your head \& to your natural default setting of being uniquely, completely, imperially alone, day in \& day out.

That may sound like hyperbole, or abstract nonsense.

So let's get concrete.

The plain fact is that you graduating seniors do not yet have any clue what ``day in, day out'' really means.

There happen to be whole large parts of adult American life that nobody talks about in commencement speeches.

1 such part involves \fbox{boredom, routine, \& petty frustration.}

The parents \& older folks here will know all too well what I am talking about.

By way of example, let's say it's an average adult day, \& you get up in the morning, go to your challenging, white-collar college-graduate job, \& you work hard for 9--10 hours, \& at the end of the day you are tired, \& you're stressed out, \& all you want is to go home \& have a good supper \& maybe unwind for a couple hours \& then hit the rack early because you have to get up the next day \& do it all again.

But then you remember there's no food at home -- you haven't had time to shop this week because of your challenging job -- \& so now after work you have to get in your car \& drive to the supermarket.

It's the end of the workday, \& the traffic's very bad, so getting to the store takes way longer than it should, \& when you finally get there, the supermarket is very crowded, because of course it's the time of day when all the other people with jobs also try to squeeze in some grocery shopping, \& the store is hideously, fluorescently lit, \& infused with soul-killing Muzak or corporate pop, \& it's pretty much the last place you want to be, but you can't just get in \& quickly out.

You have to wander all over the huge, overlit store's crowded aisles to find the stuff you want, \& you have to maneuver your junky cart through all these other tired, hurried people with carts, \& of course there are also the glacially slow old people \& the spacey people \& the ADHD kids who all block the aisle, \& you have to grit your teeth \& try to be polite as you ask them to let you by, \& eventually, finally, you get all your supper supplies, except now it turns out there aren't enough checkout lanes open even though it's the end-of-the-day rush, so the checkout line is incredibly long.

Which is stupid \& infuriating, but you can't take your fury out on the frantic lady working the register, who is overworked at a job whose daily tedium \& meaninglessness surpass the imagination of any of us here at a prestigious college $\ldots$ but anyway, you finally get to the checkout line's front, \& you pay for your food, \& wait to get your check or card authenticated by a machine, \& you get told to ``Have a nice day'' in a voice that is the absolute voice of {\it death}.

\& then you have to take your creepy flimsy plastic bags of groceries in your cart with the one crazy wheel that pulls maddeningly to the left, all the way out through the crowded, bumpy, littery parking lot, \& try to load the bags in your car in such a way that everything doesn't roll out of the bags \& roll around in the trunk on the way home, \& then you have to drive all the way home through slow, heavy, SUV-intensive rush-hour traffic, et cetera, et cetera.

Everyone here has done this, of course -- but it hasn't yet been part of your graduates' actual life routine, day after week after month after year.

But it will be, \& many more dreary, annoying, seemingly meaningless routines besides $\ldots$

Except that's not the point.

The point is that petty, frustrating crap like this is exactly where the work of choosing comes in.

Because the traffic jams \& crowded aisles \& long checkout lines give me time to think, \& if I don't make a conscious decision about how to think \& what to pay attention to, I'm gonna be pissed \& miserable every time I have to food-shop, because my natural default setting is that situations like this are really all about {\it me}, about my hungriness \& my fatigue \& my desire to just get home, \& it's going to seem, for all the world, like everybody else is just {\it in my way}, \& who the fuck are all these people in my way?

\& look at how repulsive most of them are \& how stupid \& cow-like \& dead-eyed \& nonhuman they seem here in the checkout line, or at how annoying \& rude it is that people are talking loudly on cell phones in the middle of the line, \& look at how deeply unfair this is: I've worked hard all day \& I am starved \& tired \& I can't even get home to eat \& unwind because of all these stupid goddamn {\it people}.

Or, of course, if I'm in a more socially conscious, liberal arts form of my default setting, I can spend time in the end-of-the-day traffic jam being angry \& disgusted at all the huge, stupid, lane-blocking SUVs \& Hummers \& V-12 pickup trucks burning their wasteful, selfish, 40-gallon tanks of gas, \& I can dwell on the fact that the patriotic or religious bumper stickers always seem to be on the biggest, most disgustingly selfish vehicles driven by the ugliest, most inconsiderate \& aggressive drivers, who are usually talking on cell phones as they cut people off in order to get just 20 stupid feet ahead in the traffic jam, \& I can think about how our children's children will despise us for wasting all the future's fuel \& probably screwing up the climate, \& how spoiled \& stupid \& selfish \& disgusting we all are, \& how it all {\it sucks}, \& so on \& so forth $\ldots$

Look, if I choose to think this way, fine, lots of us do -- except that thinking this way tends to be so easy \& automatic it doesn't {\it have} to be a choice.

Thinking this way is my natural default setting.

It's the automatic, unconscious way that I experience the boring, frustrating, crowded parts of adult life when I'm operating on the automatic, unconscious belief that I am the center of the world \& that my immediate needs \& feelings are what should determine the world's priorities.

The thing is that there are obviously different ways to think about these kinds of situations.

In this traffic, all these vehicles stuck \& idling in my way: It's not impossible that some of these people in SUVs have been in horrible auto accidents in the past \& now find driving so traumatic that their therapist has all but ordered them to get a huge, heavy SUV so they can feel safe enough to drive; or that the Hummer that just cut me off is maybe being driven by a father whose little child is hurt or sick in the seat next to him, \& he's trying to rush to the hospital, \& he is in a way bigger, more legitimate hurry than I am -- it is actually {\it I} who aim in {\it his} way.

Or I can choose to force myself to consider the likelihood that everyone else in the supermarket's checkout line is probably just as bored \& frustrated as I am, \& that some of these people actually have much harder, more tedious or painful lives than I do, overall.

\& so on.

Again, please don't think that I'm giving you moral advice, or that I'm saying you are ``supposed to'' think this way, or that anyone expects you to just automatically do it, because it's hard, it takes will \& mental effort, \& if you're like me, some days you won't be able to do it, or else you just flat-out won't want to.

But most days, if you're aware enough to give yourself a choice, you can choose to look differently at this fat, dead-eyed, over-made-up lady who just screamed at her kid in the checkout line -- maybe she's not usually like this; maybe she's been up 3 straight nights holding the hand of her husband, who's dying of bone cancer, or maybe this very lady is the low-wage clerk at the motor vehicles department who just yesterday helped your spouse resolve a nightmarish red-tape problem through some small act of bureaucratic kindness.

Of course, none of this is likely, but it's also not impossible -- it just depends what you want to consider.

If you're automatically sure that you know what reality is \& who \& what is really important -- if you want to operate on your default setting -- then you, like me, probably will not consider possibilities that aren't pointless \& annoying.

But if you've really learned how to think, how to pay attention, then you will know you have other options.

It will actually be within your power to experience a crowded, hot, slow, consumer-hell-type situation as not only meaningful, but sacred, on fire with the same force that lit the stars -- compassion, love, the subsurface unity of all things.

Not that that mystical stuff's necessarily true: The only thing that's capital-T True is that you get to {\it decide} how you're going to try to see it.

This, I submit, is the freedom of real education, of learning how to be well-adjusted: You get to consciously decide what has meaning \& what doesn't.

You get to decide what to worship $\ldots$

Because here's something else that's true.

In the day-to-day trenches of adult life, there is actually no such thing as atheism.

There is no such thing as not worshiping.

Everybody worships.

The only choice we get is {\it what} to worship.

\& an outstanding reason for choosing some sort of god or spiritual-type thing to worship -- be it J.C. Or Allah, be it Yahweh or the Wiccan mother-goddess or the 4 Noble Truths or some infrangible set of ethical principles -- is that pretty much anything else you worship will eat you alive.

If you worship money \& things -- if they are where you tap real meaning in life -- then you will never have enough.

Never feel you have enough.

It's the truth.

Worship your own body \& beauty \& sexual allure \& you will always feel ugly, \& when time \& age start showing, you will die a million deaths before they finally plant you.

On 1 level we all know this stuff already -- it's been codified as myths, proverbs, clich\'es, bromides, epigrams, parables: the skeleton of every great story.

Worship your intellect, being seen as smart -- you will end up feeling stupid, a fraud, always on the verge of being found out.

\& so on.

Look, the insidious thing about these forms of worship is not that they're evil or sinful; it is that they are {\it unconscious}.

They are default settings.

They're the kind of worship you just gradually slip into, day after day, getting more \& more selective about what you see \& how you measure value without ever being fully aware that that's what you're doing.

\& the so-called ``real world'' will not discourage you from operating on your default settings, because the so-called ``real world'' of men \& money \& power hums along quite nicely on the fuel of fear \& contempt \& frustration \& craving \& the worship of self.

Our own present culture has harnessed these forces in ways that have yielded extraordinary wealth \& comfort \& personal freedom.

The freedom all to be lords of our tiny skull-sized kingdoms, alone at the center of all creation.

This kind of freedom has much to recommend it.

But of course there are all different kinds of freedom, \& the kind that is most precious you will not hear much talked about in the great outside world of winning \& achieving \& displaying.

The really important kind of freedom involves attention, \& awareness, \& discipline, \& effort, \& being able truly to care about other people \& to sacrifice for them, over \& over, in myriad petty little unsexy ways, every day.

That is real freedom.

That is being taught how to think.

The alternative is unconsciousness, the default setting, the ``rat race'' -- the constant, gnawing sense of having had \& lost some infinite thing.

I know that this stuff probably doesn't sound fun \& breezy or grandly inspirational the way a commencement speech's central stuff should sound.

What it is, so far as I can see, is the truth, with a whole lot of rhetorical bullshit pared away.

Obviously, you can think of it whatever you wish.

But please don't dismiss it as some finger-wagging Dr. Laura sermon.

None of this is about morality, or religion, or dogma, or big fancy questions of life after death.

The capital-T Truth is about life {\it before} death.

It is about making it to 30, or maybe even 50, without wanting to shoot yourself in the head.

It is about the real value of a real education, which has nothing to do with grades or degrees \& everything to do with simple awareness -- awareness of what is so real \& essential, so hidden in plain sight all around us, that we have to keep reminding ourselves over \& over:

``This is water.''

``This is water.''

``These Eskimos might be much more than they seem.''

It is unimaginably hard to do this -- to live consciously, adultly, day in \& day out.

Which means yet another clich\'e is true: Your education really {\it is} the job of a lifetime, \& it commences -- now.

I wish you way more than luck.

{\sc David Foster Wallace} wrote the acclaimed novels {\it Infinite Jest} \& {\it The Broom of the System} \& the story collections {\it Oblivion, Brief Interviews with Hideous Men}, \& {\it Girl with Curious Hair}. His nonfiction includes the essay collections {\it Consider the Lobster} \& {\it A supposedly Fun Thing I'll Never Do  Again}, \& the full-length work {\it Everything \& More}. He died in 2008.'' -- \cite{Wallace_water}

%------------------------------------------------------------------------------%


%------------------------------------------------------------------------------%

\section{{\sc William Zinsser}. On Well Writing}

%------------------------------------------------------------------------------%

%------------------------------------------------------------------------------%

%------------------------------------------------------------------------------%

%------------------------------------------------------------------------------%

%------------------------------------------------------------------------------%

\section{Miscellaneous}

%------------------------------------------------------------------------------%

\printbibliography[heading=bibintoc]
	
\end{document}