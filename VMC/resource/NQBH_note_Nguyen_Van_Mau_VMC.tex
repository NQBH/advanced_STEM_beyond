\documentclass{article}
\usepackage[utf8]{inputenc}
\usepackage[vietnamese,english]{babel}
\usepackage{tabu,float,hyperref,color,amsmath,amsxtra,amssymb,latexsym,amscd,amsthm,amsfonts,graphicx}
\numberwithin{equation}{section}
\usepackage{fancyhdr}
\pagestyle{fancy}
\fancyhf{}
\fancyhead[RE,LO]{\footnotesize \textsc \leftmark}
\cfoot{\thepage}
\renewcommand{\headrulewidth}{0.5pt}
\setcounter{tocdepth}{3}
\setcounter{secnumdepth}{3}
\usepackage{imakeidx}
\makeindex[columns=2, title=Alphabetical Index, 
           options= -s index.ist]
\title{My notes on $\star$ \foreignlanguage{vietnamese}{Nguyễn Văn Mậu, Lê Ngọc Lăng, Phạm Thế Long, Nguyễn Minh Tuấn, Các đề thi Olympic toán sinh viên toàn quốc}}
\author{\textsc{Nguyen Quan Ba Hong}\\
{\small Students at Faculty of Math and Computer Science,}\\ 
{\small Ho Chi Minh University of Science, Vietnam} \\
{\small \texttt{email. nguyenquanbahong@gmail.com}}\\
{\small \texttt{blog. \url{http://hongnguyenquanba.wordpress.com}} 
\footnote{Copyright \copyright\ 2016 by Nguyen Quan Ba Hong, Student at Ho Chi Minh University of Science, Vietnam. This document may be copied freely for the purposes of education and non-commercial research. Visit my site \texttt{\url{http://hongnguyenquanba.wordpress.com}} to get more.}}}
\begin{document}
\maketitle
\begin{abstract}
I take some notes when I learn the book \cite{1}.
\end{abstract}
\newpage
\tableofcontents
\newpage
\section{Selected VMC problems}
\textbf{Problem 1. (VMC 1993, Day 2, Problem 3).} \\
\textit{Let $p(x)$ be a nonconstant polynomial with real coefficients. Prove that if the system of equations
\begin{align}
\left\{ {\begin{array}{*{20}{c}}
{\int\limits_0^x {p\left( t \right)\sin tdt = 0} }\\
{\int\limits_0^x {p\left( t \right)\cos tdt = 0} }
\end{array}} \right.
\end{align}
has real roots then the number of real roots must be finite.}\\\\
\textsc{Hint 1.} Let 
\begin{align}
\begin{array}{l}
{U_k} = \int\limits_0^x {{p^{\left( k \right)}}\left( t \right)\sin t} dt\\
{V_k} = \int\limits_0^x {{p^{\left( k \right)}}\left( t \right)\cos t} dt
\end{array}
\end{align}
Suppose that $\deg p=n$, then we have ${U_k} = 0,{V_k} = 0,\forall k > n$. Use integration by parts formula, get
\begin{equation}
\label{2.1}
\left\{ \begin{array}{l}
{U_k} = \left. { - {p^{\left( k \right)}}\left( t \right)\cos t} \right|_0^x + \int\limits_0^x {{p^{\left( {k + 1} \right)}}\left( t \right)\cos t} dt\\
{V_k} = \left. {{p^{\left( k \right)}}\left( t \right)\sin t} \right|_0^x - \int\limits_0^x {{p^{\left( {k + 1} \right)}}\left( t \right)\sin t} dt
\end{array} \right.
\end{equation}
Deduce that
\begin{align}
\left\{ \begin{array}{l}
{U_k} = \left. { - {p^{\left( k \right)}}\left( t \right)\cos t} \right|_0^x + {V_{k + 1}}\\
{V_k} = \left. {{p^{\left( k \right)}}\left( t \right)\sin t} \right|_0^x - {U_{k + 1}}
\end{array} \right.
\end{align}
\begin{align}
\left\{ \begin{array}{l}
{U_k} = \left. { - {p^{\left( k \right)}}\left( t \right)\cos t} \right|_0^x + \left. {{p^{\left( {k + 1} \right)}}\left( t \right)\sin t} \right|_0^x - {U_{k + 2}}\\
{V_k} = \left. {{p^{\left( k \right)}}\left( t \right)\sin t} \right|_0^x + \left. {{p^{\left( {k + 1} \right)}}\left( t \right)\cos t} \right|_0^x - {V_{k + 2}}
\end{array} \right.,\forall k \in \mathbb{N}
\end{align}
\begin{align}
\left\{ \begin{array}{l}
{U_0} =  - \sum\limits_{k = 0}^{2k \le n} {\left. {{p^{\left( {2k} \right)}}\left( t \right)\cos t} \right|_0^x}  + \sum\limits_{k = 0}^{2k + 1 \le n} {\left. {{p^{\left( {2k + 1} \right)}}\left( t \right)\sin t} \right|_0^x} \\
{V_k} = \sum\limits_{k = 0}^{2k \le n} {\left. {{p^{\left( {2k} \right)}}\left( t \right)\sin t} \right|_0^x}  + \sum\limits_{k = 0}^{2k + 1 \le n} {\left. {{p^{\left( {2k + 1} \right)}}\left( t \right)\cos t} \right|_0^x} 
\end{array} \right.
\end{align}
Put 
\begin{align}
\left\{ \begin{array}{l}
{p_1}\left( t \right) = \sum\limits_{k = 0}^{2k \le n} {{p^{\left( {2k} \right)}}\left( t \right)} \\
{p_2}\left( t \right) = \sum\limits_{k = 0}^{2k + 1 \le n} {{p^{\left( {2k + 1} \right)}}\left( t \right)} 
\end{array} \right.
\end{align}
Consider the case $n$ is even (the case $n$ odd is similar). Since $n$ is even, easy to get $\deg {p_1} = n,\deg {p_2} = n - 1$. Rewrite the formula (\ref{2.1}) in the form
\begin{align}
\left\{ {\begin{array}{*{20}{c}}
{{U_0} =  - {p_1}\left( t \right)\left. {\cos t} \right|_0^x + {p_2}\left( t \right)\left. {\sin t} \right|_0^x}\\
{{V_0} = {p_1}\left( t \right)\left. {\sin t} \right|_0^x + {p_2}\left( t \right)\left. {\cos t} \right|_0^x}
\end{array}} \right.
\end{align}
Call $X$ is the solution of the given system of equations 
\begin{align}
\left\{ {\begin{array}{*{20}{c}}
{{U_0} = 0}\\
{{V_0} = 0}
\end{array}} \right.
\end{align}
For $\forall x \in X$ we have
\begin{equation}
\left\{ {\begin{array}{*{20}{c}}
{ - {p_1}\left( t \right)\left. {\cos t} \right|_0^x + {p_2}\left( t \right)\left. {\sin t} \right|_0^x = 0}\\
{{p_1}\left( t \right)\left. {\sin t} \right|_0^x + {p_2}\left( t \right)\left. {\cos t} \right|_0^x = 0}
\end{array}} \right.
\end{equation}
Put ${p_1}\left( 0 \right) = a,{p_2}\left( 0 \right) = b$. Then 
\begin{align}
\left\{ {\begin{array}{*{20}{c}}
{{p_2}\left( x \right)\sin x - {p_1}\left( x \right)\cos x =  - a}\\
{{p_2}\left( x \right)\cos x + {p_1}\left( x \right)\sin x = b}
\end{array}} \right.
\end{align}
Deduce that
\begin{align}
{\left( {{p_2}\left( x \right)\sin x - {p_1}\left( x \right)\cos x} \right)^2} + {\left( {{p_2}\left( x \right)\cos x + {p_1}\left( x \right)\sin x} \right)^2} = {a^2} + {b^2}
\end{align}
Hence
\begin{align}
p_1^2\left( x \right) + p_2^2\left( x \right) - \left( {{a^2} + {b^2}} \right) = 0
\end{align}
Call $Y$ is the set of solutions of poynomial
\begin{align}
Q\left( x \right) = p_1^2\left( x \right) + p_2^2\left( x \right) - \left( {{a^2} + {b^2}} \right)
\end{align}
Deduce that $X \subset Y$. Since $\deg Q\left( x \right) = 2n$, $\left| X \right| \le \left| Y \right| \le 2n$, that means $X$ has finite elements. \hfill $\square$\\
\\
\textsc{Hint 2.} Rewrite the system of equations in the form
\begin{align}
F\left( x \right): = \int\limits_0^x {p\left( t \right){e^{it}}dt}  = 0
\end{align}
We have
\begin{align}
F'\left( x \right) = p\left( x \right){e^{it}}
\end{align}
so the equation $F'\left( x \right) = 0$ has finite solutions. Deduce that the equation $F\left( x \right) = 0$ has finite solutions. \hfill $\square$\\
\\
\textbf{Problem 2. (VMC 1994, Algebra, Problem 6).}\\
 \textit{Let $A \in {\mathcal{M}_2}\left( K \right)$ for which $A^2=A$. Prove that the necessary and sufficient condition for $AX-XA=0_2$ (where $X \in {\mathcal{M}_2}\left( K \right)$) is that there exists a $X_0 \in {\mathcal{M}_2}\left( K \right)$ for which}
\begin{align}
 X = A{X_0} + {X_0}A - {X_0}
\end{align}
\textsc{Hint.} ${X_0} = 2AX - X$. \hfill $\square$\\
\\
\textbf{Problem 3. (VMC 1994, Analysis, Problem 3b).}\\
 \textit{Let a function $f(x)$ which is differentiable on $[a,b]$ and $\forall x \in \left[ {a,b} \right],\left| {f'\left( x \right)} \right| \le \left| {f\left( x \right)} \right|$. Prove that}
\begin{align}
f\left( x \right) = 0,\forall x \in \left[ {a,b} \right]
\end{align}
\textsc{Hint.} Suppose that $x_0$ is the solution of the equation $f(x)=0$ for $x_0 \in [a,b]$. Use Taylor's expansion of $f$ with $x_0$, get
\begin{equation}
f\left( x \right) = f\left( {{x_0}} \right) + f'\left( c \right)\left( {x - {x_0}} \right) = f'\left( c \right)\left( {x - {x_0}} \right)
\end{equation}
Consider the close interval $G: = \left[ {{x_0} - \frac{1}{2},{x_0} + \frac{1}{2}} \right] \cap \left[ {a,b} \right]$. Since $f(x)$ is differentiable in $[a,b]$, $f(x)$ has maximum in $G$. Suppose that
\begin{align}
\left| {f\left( {{x_m}} \right)} \right| = \mathop {\max }\limits_{x \in G} \left| {f\left( x \right)} \right|,{x_m} \in G
\end{align}
Deduce that
\begin{align}
\left| {f\left( {{x_m}} \right)} \right| = \left| {f'\left( {{c_m}} \right)} \right|\left| {{x_m} - {x_0}} \right| \le \left| {f\left( {{c_m}} \right)} \right|\left| {{x_m} - {x_0}} \right| \le \frac{1}{2}\left| {f\left( {{c_m}} \right)} \right| \le \frac{1}{2}\left| {f\left( {{x_m}} \right)} \right|
\end{align}
So $f\left( x \right) = 0,\forall x \in G$. We have proved that\\ \textit{If for a point $x$ in $[a,b]$ for which $f(x)=0$, then $f(x)=0$ in the neighborhood of $x$ with radius $\frac{1}{2}$.}\\
By considering different points $x_0$ for which $f(x_0)=0$, tends to $a$ and $b$, after finite steps, we have $f\left( x \right) = 0,\forall x \in \left[ {a,b} \right]$. \hfill $\square$\\
\\
\textbf{Problem 4. (VMC 1995, Algebra, Problem 5).}\\
\textit{Let $A = {\left( {{a_{ij}}} \right)_{n \times n}}$ ($n>1$) have rank $r$. Consider ${A^*} = \left( {{A_{ij}}} \right)$ where ${A_{ij}}$ is cofactor of element $a_{ij}$ in $A$. Find the rank of $A^{*}$.}\\
\\
\textsc{Hint.} Divide into three cases
\begin{itemize}
\item \textbf{Case $r \le n - 2$.} $A_{ij}$ is the determinant of $\left( {n - 1} \right) \times \left( {n - 1} \right)$ matrix have rank $\le n-2$, so $A_{ij}=0, \forall i,j$ and ${A^*} = 0$. Deduce that $rank {A^*} = 0$. In addition, $\det A = 0,A{A^*} = 0$.
\item \textbf{Case $r=n-1$.} $A$ have a sub-determinant degree $n-1$ is nonzero, so ${A^*} \ne 0$. Deduce that $rank {A^*} \ge 1$. Conclude that $rank {A^*} = 1$. (?)
\item \textbf{Case $r=n$.} $\det A \ne 0$. Since ${A^t}{A^*} = \left| A \right|E$, $\left| {{A^*}} \right| \ne 0$ and $rank\left( {{A^*}} \right) = n$. 
\end{itemize}
Three case are true. Done. \hfill $\square$\\
\\
\textbf{Problem 5. (VMC 1995, Analysis, Problem 5).}\\
\textit{Let a continuous function $f(x)$ on $[a,b]$. Prove that}
\begin{align}
\mathop {\max }\limits_{x \in \left[ {a,b} \right]} \left| {f\left( x \right)} \right| = \mathop {\lim }\limits_{p \to \infty } {\left( {\int\limits_0^1 {{{\left| {f\left( x \right)} \right|}^p}} dx} \right)^{\frac{1}{p}}}
\end{align}
\textsc{Hint.} Since $f(x)$ is continuous on the compact set $[a,b]$ so $\exists c \in \left[ {a,b} \right]$ for which
\begin{align}
\left| {f\left( c \right)} \right| = \mathop {\max }\limits_{x \in \left[ {0,1} \right]} \left| {f\left( x \right)} \right| = M
\end{align}
Suppose that $a<c<b$, have
\begin{align}
\forall \varepsilon  > 0,\exists \delta  > 0,\left| {x - c} \right| < \delta  \Rightarrow \left| {f\left( x \right) - f\left( c \right)} \right| < \frac{\varepsilon }{4}
\end{align}
Then 
\begin{align}
{\left( {\int\limits_{c - \delta }^{c + \delta } {\left( {M - \frac{\varepsilon }{4}} \right)} dx} \right)^{\frac{1}{n}}} &< {\left( {\int\limits_{c - \delta }^{c + \delta } {\left| {{f^n}\left( x \right)} \right|} dx} \right)^{\frac{1}{n}}} \\
&< {\left( {\int\limits_a^b {\left| {{f^n}\left( x \right)} \right|} dx} \right)^{\frac{1}{n}}} \\
&< {\left( {\int\limits_a^b {{M^n}} dx} \right)^{\frac{1}{n}}}
\end{align}
hence 
\begin{align}
{\left( {2\delta } \right)^{\frac{1}{n}}}\left( {M - \frac{\varepsilon }{4}} \right) < {\left( {\int\limits_a^b {\left| {{f^n}\left( x \right)} \right|} dx} \right)^{\frac{1}{n}}} < M{\left( {b - a} \right)^{\frac{1}{n}}}
\end{align}
Have 
\begin{align}
\mathop {\lim }\limits_{n \to \infty } {\left( {2\delta } \right)^{\frac{1}{n}}} = 1,\mathop {\lim }\limits_{n \to \infty } {\left( {b - a} \right)^{\frac{1}{n}}} = 1
\end{align}
so 
\begin{align}
{\left( {\int\limits_a^b {\left| {{f^n}\left( x \right)} \right|} dx} \right)^{\frac{1}{n}}} = M
\end{align}
The case $c=a$, we have 
\begin{align}
\forall \varepsilon  > 0,\exists \delta  > 0,a < x < a + \delta  \Rightarrow \left| {f\left( x \right) - f\left( a \right)} \right| < \varepsilon
\end{align}
then we proceeds similarly the first case. The case $c=b$ is similar, too. \hfill $\square$\\
\\
\textbf{Problem 6. (VMC 1998, Algebra, Problem 3).}\\
\textit{Suppose that $A$ is $n+1 \times n+2$ matrix defined by
\begin{align}
A = \left( {\begin{array}{*{20}{c}}
{C_0^0}&{C_1^0}&{C_2^0}& \cdots &{C_n^0}&{C_{n + 1}^0}\\
0&{C_1^1}&{C_2^1}& \cdots &{C_n^1}&{C_{n + 1}^1}\\
0&0&{C_2^2}& \cdots &{C_n^2}&{C_{n + 1}^2}\\
 \vdots & \vdots & \vdots & \ddots & \vdots & \vdots \\
0&0&0& \cdots &{C_n^n}&{C_{n + 1}^n}
\end{array}} \right)
\end{align}
where $C_n^k = \left( {\begin{array}{*{20}{c}}
n\\
k
\end{array}} \right)$ (the left one is used in Vietnam, the right is used internationally).\\
Call $D_k$ is the determinant which is obtained from $A$ by deleting the $k$-th column ($k=1,2,...,n+2$).\\ Prove that ${D_k} = C_{n + 1}^{k - 1}$}\\
\\
\textsc{Hint.} Add the last row 
\begin{align}
1,x,{x^2}, \ldots ,{x^{n + 1}}
\end{align}
to matrix $A$, we obtain the square matrix degree $n+2$. Call $D_{n+2}$ is the determinant of the \lq new matrix\rq. Transform $D_{n+2}$: Take the $k$-th, multiply with $-1$, then add to $k+1$-th column, for each $k = n + 1,n, \ldots ,1$, we obtain
\begin{align}
{D_{n + 2}} = \left( {x - 1} \right){D_{n + 1}} =  \cdots  = {\left( {x - 1} \right)^{n + 1}}
\end{align}
Deduce that 
\begin{align}
{D_{n + 2}} = \sum\limits_{k = 1}^{n + 2} {{{\left( { - 1} \right)}^{n - k}}C_{n + 1}^{k - 1}{x^{k - 1}}} 
\end{align}
On the other hand, if we expand $D_{n+2}$ respect to the last row, we have
\begin{align}
{D_{n + 2}} = \sum\limits_{k = 1}^{n + 2} {{{\left( { - 1} \right)}^{n - k}}{D_k}{x^{k - 1}}}
\end{align}
Conclude that ${D_k} = C_{n + 1}^{k - 1}$. \hfill $\square$\\
\\
\textbf{Problem 7. (VMC 1998, Analysis, Problem 1)}\\
\textit{Let $f\left( x \right) \in {C^1}\left( {\left[ {0,1} \right]} \right)$ and $f(0)=0$. Prove that}
\begin{align}
\int\limits_0^1 {\left| {f\left( t \right)f'\left( t \right)} \right|} dt \le \frac{1}{2}\int\limits_0^1 {{{\left( {f'\left( t \right)} \right)}^2}} dt
\end{align}
\textsc{Hint.} Consider 
\begin{align}
F\left( x \right) = \int\limits_0^x {\left| {f\left( t \right)} \right|\left| {f'\left( t \right)} \right|} dt\\
G\left( x \right) = \frac{x}{2}\int\limits_0^x {{{\left( {f'\left( t \right)} \right)}^2}} dt
\end{align}
Then
\begin{align}
F'\left( x \right) = \left| {f\left( x \right)} \right|\left| {f'\left( x \right)} \right|\\
G'\left( x \right) = \frac{x}{2}{\left( {f'\left( x \right)} \right)^2} + \frac{1}{2}\int\limits_0^x {{{\left( {f'\left( t \right)} \right)}^2}} dt
\end{align}
On the other hand
\begin{align}
f\left( x \right) = \int\limits_0^x {f'\left( t \right)} dt,\forall x \in \left[ {0,1} \right]
\end{align}
Use Cauchy inequality
\begin{align}
\left| {f\left( x \right)} \right| = \left| {\int\limits_0^x {f'\left( t \right)} dt} \right| \le {\left( {\int\limits_0^x {dx} } \right)^{\frac{1}{2}}}{\left( {\int\limits_0^x {{{\left( {f'\left( t \right)} \right)}^2}} dt} \right)^{\frac{1}{2}}}
\end{align}
Then 
\begin{align}
\left| {f\left( x \right)f'\left( x \right)} \right| \le \sqrt x {\left( {\int\limits_0^x {{{\left( {f'\left( t \right)} \right)}^2}} dt} \right)^{\frac{1}{2}}}\left| {f'\left( x \right)} \right|
\end{align}
Deduce that
\begin{align}
\left| {f\left( x \right)f'\left( x \right)} \right| \le \frac{1}{2}\int\limits_0^x {{{\left( {f'\left( t \right)} \right)}^2}dt}  + \frac{x}{2}{\left( {f'\left( x \right)} \right)^2}
\end{align}
That means
\begin{align}
F'\left( x \right) \le G'\left( x \right),\forall x \in \left[ {0,1} \right]
\end{align}
\begin{align}
F\left( 1 \right) - F\left( 0 \right) \le G\left( 1 \right) - G\left( 0 \right)
\end{align}
and
\begin{align}
\left| {\int\limits_0^1 {f\left( t \right)f'\left( t \right)} dt} \right| \le \frac{1}{2}\int\limits_0^1 {{{\left( {f'\left( t \right)} \right)}^2}} dt
\end{align}
Done. \hfill $\square$\\
\\
\textbf{Problem 8. (VMC 2000, Algebra, Problem 4).}\\
\textit{Let $A$ be a square matrix degree $n$ which has all elements in the main diagonal equal to $0$, and other elements are equal to $1$ or $2000$.\\
Prove that $rank A = n$ or $rank A = n-1$.}\\
\\
\textsc{Hint.} Consider the matrix $B \in {\mathcal{M}_n}$ which has all elements equals to 1. Then 
\begin{align}
A - B = \left\{ {{c_{ij}}} \right\},{c_{ij}} \in \left\{ { - 1,0,1999} \right\}
\end{align}
Then 
\begin{align}
\det \left( {A - B} \right) = {\left( { - 1} \right)^n}\left( {\bmod 1999} \right)
\end{align}
Deduce that $\det \left( {A - B} \right) \ne 0$ and $n = rank\left( {A - B} \right) \le rankA + rank\left( { - B} \right) = rankA + 1$. \hfill $\square$\\
\\
\textbf{Problem 9. (VMC 2001, Algebra, Problem 4).}\\
\textit{Denote $\left\langle {a,b} \right\rangle $ for scalar product of two vectors $a,b \in \mathbb{R}$. Let ${a_1},{a_2}, \ldots ,{a_k} \in {\mathbb{R}^n}$. Put
\begin{align}
A = \left( {\begin{array}{*{20}{c}}
{\left\langle {{a_1},{a_1}} \right\rangle }&{\left\langle {{a_1},{a_2}} \right\rangle }& \cdots &{\left\langle {{a_1},{a_{k - 1}}} \right\rangle }&{\left\langle {{a_1},{a_k}} \right\rangle }\\
{\left\langle {{a_2},{a_1}} \right\rangle }&{\left\langle {{a_2},{a_2}} \right\rangle }& \cdots &{\left\langle {{a_2},{a_{k - 1}}} \right\rangle }&{\left\langle {{a_2},{a_{k - 1}}} \right\rangle }\\
 \vdots & \vdots & \ddots & \vdots & \vdots \\
{\left\langle {{a_{k - 1}},{a_1}} \right\rangle }&{\left\langle {{a_{k - 1}},{a_2}} \right\rangle }& \cdots &{\left\langle {{a_{k - 1}},{a_{k - 1}}} \right\rangle }&{\left\langle {{a_{k - 1}},{a_k}} \right\rangle }\\
{\left\langle {{a_k},{a_1}} \right\rangle }&{\left\langle {{a_k},{a_2}} \right\rangle }& \cdots &{\left\langle {{a_k},{a_{k - 1}}} \right\rangle }&{\left\langle {{a_k},{a_k}} \right\rangle }
\end{array}} \right)
\end{align}
Prove that}
\begin{enumerate}
\item $\det A \ge 0$
\item $A$ is a symmetric matrix and its all eigenvalues are nonzero. 
\end{enumerate}
\textsc{Hint.} \\
\textbf{1.} Write $a_s$ as
\begin{align}
{a_s} = \left( {{a_{s1}},{a_{s2}}, \ldots ,{a_{sn}}} \right),s = 1,2, \ldots ,k
\end{align}
Then 
\begin{align}
A = \left( {\begin{array}{*{20}{c}}
{\sum\limits_{j = 1}^n {a_{1j}^2} }&{\sum\limits_{j = 1}^n {{a_{1j}}{a_{2j}}} }& \cdots &{\sum\limits_{j = 1}^n {{a_{1j}}{a_{kj}}} }\\
{\sum\limits_{j = 1}^n {{a_{2j}}{a_{1j}}} }&{\sum\limits_{j = 1}^n {a_{2j}^2} }& \cdots &{\sum\limits_{j = 1}^n {{a_{2j}}{a_{kj}}} }\\
 \vdots & \vdots & \ddots & \vdots \\
{\sum\limits_{j = 1}^n {{a_{kj}}{a_{1j}}} }&{\sum\limits_{j = 1}^n {{a_{kj}}{a_{2j}}} }& \cdots &{\sum\limits_{j = 1}^n {a_{kj}^2} }
\end{array}} \right)
\end{align}
Deduce that 
\begin{equation}
\label{1}
\det A = \sum\limits_{1 \le {j_1},{j_2}, \ldots ,{j_k} \le n} {{a_{1{j_1}}}{a_{2{j_2}}} \ldots {a_{k{j_k}}}} \left| {\begin{array}{*{20}{c}}
{{a_{1{j_1}}}}&{{a_{1{j_2}}}}& \cdots &{{a_{1{j_k}}}}\\
{{a_{2{j_1}}}}&{{a_{2{j_2}}}}& \cdots &{{a_{2{j_k}}}}\\
 \vdots & \vdots & \ddots & \vdots \\
{{a_{k{j_1}}}}&{{a_{k{j_2}}}}& \cdots &{{a_{k{j_k}}}}
\end{array}} \right|
\end{equation}
Let ${A_{{j_1} < {j_2} <  \cdots  < {j_k}}}$ are matrices in the right hand side of the expression (\ref{1}) respect to $\left( {{j_1},{j_2}, \ldots ,{j_k}} \right)$ which is fixed and sorted increasingly. Then, all the terms have the same index set $\left( {{j_1},{j_2}, \ldots ,{j_k}} \right)$ have the sum equals to
\begin{align}
\begin{array}{l}
\sum\limits_{{j_1} < {j_2} <  \cdots  < {j_k}} {{{\left( { - 1} \right)}^{inv\left( {{j_1} < {j_2} <  \cdots  < {j_k}} \right)}}{a_{1{j_1}}}{a_{2{j_2}}} \ldots {a_{k{j_k}}}\det \left( {{A_{{j_1} < {j_2} <  \cdots  < {j_k}}}} \right)} \\
 = {\left( {\det \left( {{A_{{j_1} < {j_2} <  \cdots  < {j_k}}}} \right)} \right)^2}
\end{array}
\end{align}
From the expression (\ref{1}), we obtain 
\begin{align}
\det A = \sum\limits_{1 \le {j_1} < {j_2} <  \cdots  < {j_k} \le n} {{{\left( {\det \left( {{A_{{j_1} < {j_2} <  \cdots  < {j_k}}}} \right)} \right)}^2}}  \ge 0
\end{align}
\textbf{2.} Scalar product is a symmetric bi-linear form, that means $\left\langle {{a_i},{a_j}} \right\rangle  = \left\langle {{a_j},{a_i}} \right\rangle $. So, $A$ is symmetric matrix and all its eigenvalues are real. By the proof above, all sub-determinants of $A$ are nonzero. Therefore, the characteristic polynomial of $A$ has the form
\begin{align}
{P_A}\left( t \right) = {\left( { - 1} \right)^k}{t^k} + {\left( { - 1} \right)^{k - 1}}{a_1}{t^{k - 1}} +  \cdots  - {a_{k - 1}}t + {a_k}
\end{align}
where coefficients ${a_1},{a_2}, \ldots ,{a_k}$ are nonzero. Deduce that ${P_A}\left( t \right) > 0$ when $t<0$. So all eigenvalues of $A$ are nonzero. \hfill $\square$\\
\\
\textbf{Problem 10. (VMC 2001, Analysis, Problem 4).}\\
\textit{Given $f$ is defined on $\mathbb{R}$ and have $2$-th derivative on $\mathbb{R}$ satisfying $f\left( x \right) + f''\left( x \right) \ge 0,\forall x \in R$.\\
Prove}
\begin{align}
f\left( x \right) + f\left( {x + \pi } \right) \ge 0,\forall x \in R
\end{align}
\textsc{Hint.} For each fixed $x \in \mathbb{R}$, consider the function
\begin{align}
g\left( y \right) = f'\left( y \right)\sin \left( {y - x} \right) - f\left( y \right)\cos \left( {y - x} \right)
\end{align}
Have
\begin{align}
g'\left( y \right) = \left[ {f''\left( y \right) + f\left( y \right)} \right]\sin \left( {y - x} \right) \ge 0,\forall y \in \left[ {x,x + \pi } \right]
\end{align}
Deduce that $g(y)$ is monotonically increasing on $\left[ {x,x + \pi } \right]$. So $g\left( x \right) \le g\left( {x + \pi } \right)$, that means $f\left( x \right) + f\left( {x + \pi } \right) \ge 0$. \hfill $\square$\\
\\
\textbf{Problem 11. (VMC 2003, Algebra, Problem 7).}\\
\textit{Given a real coefficients polynomial $P(x)$ have degree $n$ ($n \ge 1$) have $m$ real roots. Prove that the polynomial
\begin{align}
Q\left( x \right) = \left( {{x^2} + 1} \right)P\left( x \right) + P'\left( x \right)
\end{align}
has at least $m$ real roots.}\\
\\
\textsc{Hint.} Consider the function
\begin{align}
f\left( x \right) = {e^{\frac{{{x^3}}}{3} + x}}P\left( x \right)
\end{align}
The set of real roots of $f(x)=0$ and the set of real roots of $P(x)$ is the same. By Rolle theorem, the equation 
\begin{align}
f'\left( x \right) = {e^{\frac{{{x^3}}}{3} + x}}\left[ {P\left( x \right) + \left( {{x^2} + 1} \right)P'\left( x \right)} \right] = 0
\end{align}
has at least $m-1$ real roots, or the equation
\begin{align}
P'\left( x \right) + \left( {{x^2} + 1} \right)P\left( x \right) = 0
\end{align}
has as least $m-1$ real roots. Consider two cases
\begin{itemize}
\item \textbf{Case 1. $m$ is even.} If $n$ is odd then $P(x)$ has at least $m+1$ real roots, absurd. So, $n$ must be even. Then 
\begin{align}
P'\left( x \right) + \left( {{x^2} + 1} \right)P\left( x \right)
\end{align}
has degree equal to $n+2$ (even number) and has $m-1$ (odd number) real roots. Deduce that this polynomial must have at least $(m-1)+1=m$ real roots.
\item \textbf{Case 2. $m$ is odd. } If $n$ is even then $P(x)$ has at least $m+1$ real roots, absurd. So $n$ must be odd. Then
\begin{align}
P'\left( x \right) + \left( {{x^2} + 1} \right)P\left( x \right)
\end{align}
has degree $n+2$ (odd number) and has $m-1$ (even number) real roots. Deduce that this polynomial must have at least $(m-1)+1=m$ real roots.
\end{itemize}
In both cases, we have desired result. \hfill $\square$\\
\\
\textbf{Problem 12. (VMC 2004, Analysis, Problem 5).}\\
\textit{Let $P(x),Q(x),R(x)$ are real coefficients polynomials which have degree $3,2,3$ respectively, satisfying the condition 
\begin{align}
{\left( {P\left( x \right)} \right)^2} + {\left( {Q\left( x \right)} \right)^2} = {\left( {R\left( x \right)} \right)^2}
\end{align}
How many real roots which the polynomial 
\begin{align}
T\left( x \right) = P\left( x \right)Q\left( x \right)R\left( x \right)
\end{align}
has at least (including repetition roots)?}\\
\\
\textsc{Hint.} Without loss of generality (w.l.o.g.), we can suppose that the coefficients respect of the maximal degree terms of polynomials $P,Q,R$ are positive. Firstly, we prove that $Q(x)$ always have 2 real roots. We have ${Q^2} = \left( {R - P} \right)\left( {R + P} \right)$. Since $\det P = \det R = 3$, $\det (R+P)=3$. Since $\deg {Q^2} = 4$, $\deg (R-P)=1$. Therefore, $Q^2$ has real roots, so $Q$ also has real roots. Since $\deg Q =2$, $Q$ has exactly 2 roots. Next, we prove that $P(x)$ always have 3 real roots. We have ${P^2} = \left( {R - Q} \right)\left( {R + Q} \right)$. Since $\deg \left( {R - Q} \right) = \deg \left( {R + Q} \right) = 3$, $(R-Q)$ and $(R+Q)$ have real roots. If these two roots is distinct, $P$ has 
2 distinct real roots and the third root of $P$ is also real. If $(R-Q)$ and $(R+Q)$ have a common real root $x=a$ then $x=a$ is root of $R$ and $Q$. So, 
\begin{align}
R\left( x \right) = \left( {x - a} \right){R_1}\left( x \right)\\
Q\left( x \right) = \left( {x - a} \right){Q_1}\left( x \right)\\
P\left( x \right) = \left( {x - a} \right){P_1}\left( x \right)
\end{align}
Put these formula into ${P^2} = \left( {R - Q} \right)\left( {R + Q} \right)$, we get $P_1^2 = R_1^2 - Q_1^2$, where $P_1,R_1$ is quadratic, $Q_1$ is linear. We have 
\begin{align}
Q_1^2 = \left( {{R_1} - {P_1}} \right)\left( {{R_1} + {P_1}} \right)
\end{align}
Since $Q_1^2$ is quadratic and $R_1+P_1$ is quadratic, $R_1-P_1$ is constant polynomial. So, if ${P_1}\left( x \right) = a{x^2} + bx + c,\left( {a > 0} \right),{Q_1}\left( x \right) = dx + e$, then ${R_1}\left( x \right) = a{x^2} + bx + c + k$ and 
\begin{equation}
\label{2}
k\left[ {{R_1}\left( x \right) + {P_1}\left( x \right)} \right] = {\left( {dx + e} \right)^2}
\end{equation}
Hence $k>0$. Take $x =  - \frac{e}{d}$ in (\ref{2}), we obtain
\begin{align}
{R_1}\left( { - \frac{e}{d}} \right) + {P_1}\left( { - \frac{e}{d}} \right) = 0
\end{align}
so ${P_1}\left( { - \frac{e}{d}} \right) =  - \frac{k}{2} < 0$. Therefore, quadratic $P_1(x)$ has 2 real roots and $P(x)$ has 3 real roots.\\
Since $P$ has 3 real roots, $Q$ has 2 real roots and $R$ is cubic (has at least 1 real root), the number of roots of $T(x)$ is larger or equal to 6. For example, if we choose
\begin{align}
P\left( x \right) = {x^3} + 3{x^2} + 2x\\
Q\left( x \right) = 2\left( {{x^2} + 2x + 1} \right)\\
R\left( x \right) = {x^3} + 3{x^2} + 4x + 2
\end{align}
then $P^2+Q^2=R^2$ and $PQR$ has exactly 6 real roots. \hfill $\square$\\
\\
\textbf{Problem 13. (VMC 2005, Analysis, Problem 4).}\\
\textit{Let $f$ be a continuous function on $[0,1]$ satisfying the condition}
\begin{align}
\int\limits_x^1 {f\left( t \right)} dt \ge \frac{{1 - {x^2}}}{2},\forall x \in \left[ {0,1} \right]
\end{align}
Prove that
\begin{align}
\int\limits_0^1 {{{\left( {f\left( x \right)} \right)}^2}} dx \ge \int\limits_0^1 {xf\left( x \right)} dx
\end{align}
\textsc{Hint.} We have
\begin{align}
0 \le \int\limits_0^1 {{{\left( {f\left( x \right) - x} \right)}^2}} dx = \int\limits_0^1 {{{\left( {f\left( x \right)} \right)}^2}} dx - 2\int\limits_0^1 {xf\left( x \right)} dx + \frac{1}{3}
\end{align}
Hence
\begin{equation}
\label{3}
\int\limits_0^1 {{{\left( {f\left( x \right)} \right)}^2}} dx \ge 2\int\limits_0^1 {xf\left( x \right)} dx - \frac{1}{3}
\end{equation}
Put 
\begin{align}
A = \int\limits_0^1 {\left( {\int\limits_x^1 {f\left( t \right)} dt} \right)dx} 
\end{align}
We have
\begin{align}
A = \int\limits_0^1 {\left( {\int\limits_x^1 {f\left( t \right)} dt} \right)dx}  \ge \int\limits_0^1 {\frac{{1 - {x^2}}}{2}} dx = \frac{1}{3}
\end{align}
On the other hand
\begin{align}
A = \int\limits_0^1 {\left( {\int\limits_x^1 {f\left( t \right)} dt} \right)dx}  = \left. {x\int\limits_0^1 {f\left( t \right)} dt} \right|_0^1 + \int\limits_0^1 {xf\left( x \right)} dx = \int\limits_0^1 {xf\left( x \right)} dx
\end{align}
Therefore
\begin{equation}
\label{4}
\int\limits_0^1 {xf\left( x \right)} dx \ge \frac{1}{3}
\end{equation}
(\ref{3}) and (\ref{4})complete the proof. \hfill $\square$\\
\\
\textbf{Problem 15. (VMC 2005, Analysis, Problem 5).}\\
\textit{Let $f(x)$ be a function which has continuous $2$-th derivative on $\mathbb{R}$, satisfying the condition $f\left( \alpha  \right) = f\left( \beta  \right) = a$. Prove that
\begin{align}
\mathop {\max }\limits_{x \in \left[ {\alpha ,\beta } \right]} \left\{ {f''\left( x \right)} \right\} \ge \frac{{8\left( {a - b} \right)}}{{{{\left( {\alpha  - \beta } \right)}^2}}}
\end{align}
where $b = \mathop {\min }\limits_{x \in \left[ {\alpha ,\beta } \right]} \left\{ {f\left( x \right)} \right\}$}\\
\\
\textsc{Hint.} Since $f$ is continuous on compact set $\left[ {\alpha ,\beta } \right]$, $f$ attains its minimum on $\left[ {\alpha ,\beta } \right]$, that means there exists a $c \in (\alpha,\beta)$ for which $f'(c)=0$ and $f(c) = \mathop {\min }\limits_{x \in \left[ {\alpha ,\beta } \right]} \left\{ {f\left( x \right)} \right\}=b$. Use the Taylor expanding of function $f(x) $ respect to $c$
\begin{align}
f\left( x \right) = f\left( c \right) + f'\left( c \right)\left( {x - c} \right) + \frac{{f''\left( {\theta \left( x \right)} \right)}}{2}{\left( {x - c} \right)^2}
\end{align}
Take $x=\alpha$ and $x=\beta$ into the above equality, we obtain
\begin{align}
a = b + \frac{{f''\left( {\theta \left( \alpha  \right)} \right)}}{2}{\left( {\alpha  - c} \right)^2}\\
a = b + \frac{{f''\left( {\theta \left( \beta  \right)} \right)}}{2}{\left( {\beta  - c} \right)^2}
\end{align}
That is
\begin{align}
\begin{array}{l}
f''\left( {\theta \left( \alpha  \right)} \right) = \frac{{2\left( {a - b} \right)}}{{{{\left( {\alpha  - c} \right)}^2}}}\\
f''\left( {\theta \left( \beta  \right)} \right) = \frac{{2\left( {a - b} \right)}}{{{{\left( {\beta  - c} \right)}^2}}}
\end{array}
\end{align}
Multiply two inequalities, we obtain
\begin{align}
f''\left( {\theta \left( \alpha  \right)} \right)f''\left( {\theta \left( \beta  \right)} \right) = \frac{{4{{\left( {a - b} \right)}^2}}}{{{{\left( {\alpha  - c} \right)}^2}{{\left( {\beta  - c} \right)}^2}}} \ge \frac{{64{{\left( {a - b} \right)}^2}}}{{{{\left( {\alpha  - \beta } \right)}^2}}}
\end{align}
This inequality completes our proof. \hfill $\square$.
\section{Training problems}
\textbf{Problem 1.} \textit{Suppose that functional equation
\begin{align}
f\left( {ax + y} \right) = Af\left( x \right) + f\left( y \right),\left( {aA \ne 0} \right),\forall x,y \in \mathbb{R}
\end{align}
has non-constant solution. Prove that if $a$ (or $A$) is algebraic number with minimal polynomial $P_a(t)$ (respectively, $P_A(t)$, then $A$ is algebraic number and
\begin{equation}
\label{7}
{P_a}\left( t \right) \equiv {P_A}\left( t \right)
\end{equation}
}\\
\textsc{Hint.} $f(0)=0$ and $f(ax)=Af(x)$ and by induction:
\begin{equation}
\label{5}
f\left( {{a^k}x} \right) = {A^k}f\left( x \right),k \in \mathbb{N}
\end{equation}
Suppose that
\begin{equation}
{P_a}\left( t \right) = {t^n} + \sum\limits_{i = 0}^{n - 1} {{r_i}{t^i},\left( {{r_0}, \ldots ,{r_{n - 1}} \in \mathbb{Q}} \right)} 
\end{equation}
Then, by (\ref{5})
\begin{align}
f\left[ {\left( {{a^n} + \sum\limits_{i = 0}^{n - 1} {{r_i}{a^i}} } \right)x} \right] = f\left( {{a^n}x} \right) + \sum\limits_{i = 0}^{n - 1} {{r_i}f\left( {{a^i}x} \right)} \\
 = \left( {{A^n} + \sum\limits_{i = 0}^{n - 1} {{r_i}{A^i}} } \right)f\left( x \right)
\end{align}
Since $f(x)$ is non-constant, 
\begin{equation}
\label{6}
{A^n} + \sum\limits_{i = 0}^{n - 1} {{r_i}{A^i}}  = 0
\end{equation}
Hence $A$ is algebraic number. Deduce that $P_a(t)$ is divisor of $P_A(t)$ and since $P_A(t)$ is minimal polynomial, we obtain
\begin{equation}
{P_a}\left( t \right) \equiv {P_A}\left( t \right)
\end{equation}
Conversely, if $A$ is algebraic number satisfy (\ref{6}) then do the reverse process, we obtain
\begin{align}
{a^n} + \sum\limits_{i = 0}^{n - 1} {{r_i}{a^i}}  = 0
\end{align}
hence (\ref{7}). \hfill $\square$\\
\\
\textbf{Problem 2.} \textit{Do there exist functions $f: \mathbb{R} \to \mathbb{R}$ and $g: \mathbb{R} \to \mathbb{R}$, where $g$ is periodically function satisfying}
\begin{align}
{x^3} = f\left( {\left\lfloor x \right\rfloor } \right) + g\left( {\left\lfloor x \right\rfloor } \right),\forall x \in \mathbb{R}
\end{align}
\textsc{Hint.} Suppose for the contrary, $f$ and $g$ satisfy all above conditions. Let $T>0$ denote the periodic of $g$. We have
\begin{align}
{\left( {x + T} \right)^3} = f\left( {\left\lfloor {x + T} \right\rfloor } \right) + g\left( {\left\lfloor {x + T} \right\rfloor } \right)
\end{align}
Hence
\begin{equation}
\label{8}
f\left( {\left\lfloor {x + T} \right\rfloor } \right) - f\left( {\left\lfloor x \right\rfloor } \right) \equiv {T^3} + 3{T^2}x + 3T{x^2}
\end{equation}
For $x \in \left[ {0,\left\lfloor T \right\rfloor  + 1 - T} \right)$, the LHS of (\ref{8}) is constant, so it give a quadratic which has infinity roots. Therefore, $T=0$, absurd. \hfill $\square$\\
\\
\textbf{Problem 3.} \textit{Let $\alpha_1,\alpha_2$ and $\beta_1,\beta_2$ satisfy
\begin{align}
\frac{{{\alpha _2}}}{{{\alpha _1}}} \in \mathbb{Q},\frac{{{\alpha _2}}}{{{\alpha _1}}} \le \frac{{{\beta _2}}}{{{\beta _1}}}
\end{align}
If $f: \mathbb{R} \to \mathbb{R}$ satisfies
\begin{align}
\left\{ {\begin{array}{*{20}{c}}
{f\left( {x + {\alpha _1}} \right) \le f\left( x \right) + {\beta _1}}\\
{f\left( {x + {\alpha _2}} \right) \ge f\left( x \right) + {\beta _2}}
\end{array}} \right.,\forall x \in \mathbb{R}
\end{align}
then $g\left( x \right) = f\left( x \right) - \frac{{{\beta _2}}}{{{\alpha _1}}}x$ is periodical function.}
\section{Interpolating polynomial}
\textbf{Interpolation 1 (Lagrange interpolating polynomial).}\\
\textit{Given ${x_1},{x_2}, \ldots ,{x_n}$ be distinct numbers. Find all polynomials $P(x)$ with $\deg P(x) \le n-1$ satisfying $P(x_k)=a_k \in \mathbb{R}, k=1,...,n$ ($a_1,...,a_n$ are given).}\\
\\
\textsc{Hint.} The Lagrange interpolating polynomial
\begin{equation}
P\left( x \right) = \sum\limits_{i = 1}^n {P\left( {{x_i}} \right)\frac{{\prod\limits_{k = 1,k \ne i}^n {\left( {x - {x_k}} \right)} }}{{\prod\limits_{k = 1,k \ne i}^n {\left( {{x_i} - {x_k}} \right)} }}}
\end{equation}
Uniqueness is obtained by: \textit{If two polynomials degree less than or equal to $n-1$ have same values at $n$ points, then they are identical.} \hfill $\square$\\
\\
\textbf{Interpolation 2 (Taylor - Gontcharov expanding or Newton interpolating formula).}\\
\textit{Given $\left( {{x_0},{x_1}, \ldots ,{x_n}} \right)\mbox{ and }\left( {{a_0},{a_1}, \ldots ,{a_n}} \right)$. Find all polynomials $P(x)$ with $\deg P(x) \le n$ satisfying}
\begin{align}
{P^{\left( k \right)}}\left( {{x_k}} \right) = {a_k},k \in \left\{ {0,1, \ldots ,n} \right\}
\end{align}
\textsc{Hint.}
Easy to prove
\begin{align}
P\left( x \right) &= P\left( {{x_0}} \right) + \int\limits_{{x_0}}^x {P'\left( t \right)} dt\\
P'\left( x \right) &= P'\left( {{x_1}} \right) + \int\limits_{{x_1}}^t {P''\left( {{t_1}} \right)} d{t_1}\\
P''\left( x \right) &= P''\left( {{x_2}} \right) + \int\limits_{{x_2}}^{{t_1}} {P'''\left( {{t_2}} \right)} d{t_2}\\
&\hdots
\end{align}
We obtain the desired polynomial
\begin{align}
P\left( x \right) = {a_n}\int\limits_{{x_0}}^x {\int\limits_{{x_0}}^{{t_1}} {\int\limits_{{x_0}}^{{t_2}} { \cdots \int\limits_{{x_0}}^{{t_{n - 1}}} {d{t_n}d{t_{n - 1}} \ldots d{t_1}} } } } \\
 + {a_{n - 1}}\int\limits_{{x_0}}^x {\int\limits_{{x_0}}^{{t_1}} {\int\limits_{{x_0}}^{{t_2}} { \cdots \int\limits_{{x_0}}^{{t_{n - 2}}} {d{t_{n - 1}} \ldots d{t_1}} }  +  \cdots  + {a_1}\int\limits_{{x_0}}^x {d{t_1}} }  + {a_0}} 
\end{align}
Done. \hfill $\square$\\ 
\\
\textbf{Interpolation 3 (Hermite interpolating formula).} \\
\textit{Given two distinct numbers $x_0$ and $x_1$. Find all polynomials $P(x)$ with $\deg P(x) \le n (n \in \mathbb{N}^{*})$ satisfying the conditions}
\begin{align}
P\left( {{x_0}} \right) = 1\\
{P^{\left( k \right)}}\left( {{x_1}} \right) = 0,k \in \left\{ {0,1, \ldots ,n - 1} \right\}
\end{align}
\textsc{Hint.} 
\begin{equation}
P\left( x \right) = \frac{{{{\left( {x - {x_1}} \right)}^n}}}{{{{\left( {{x_0} - {x_1}} \right)}^n}}}
\end{equation}
Done. \hfill $\square$\\
\\
\textbf{Interpolation 4 (Hermite interpolating formula).} \\
\textit{Given two distinct numbers $x_0$ and $x_1$. Find all polynomials $P(x)$ with $\deg P(x) \le n+1 (n \in \mathbb{N}^{*})$ satisfying the conditions}
\begin{align}
P\left( {{x_0}} \right) = 1,P'\left( {{x_0}} \right) = 1\\
{P^{\left( k \right)}}\left( {{x_1}} \right) = 0,k \in \left\{ {0,1, \ldots ,n - 1} \right\}
\end{align}
\textsc{Hint.} 
\begin{equation}
P\left( x \right) = {\left( {x - {x_1}} \right)^n}\left( {\frac{{\left( {{x_0} - {x_1} - n} \right)x + \left( {{x_0} - {x_1}} \right)\left( {1 - {x_0}} \right) + n{x_0}}}{{{{\left( {{x_0} - {x_1}} \right)}^{n + 1}}}}} \right)
\end{equation}
Done. \hfill $\square$\\
\\
\newpage
\begin{thebibliography}{999}
\bibitem {1} \foreignlanguage{vietnamese}{Nguyễn Văn Mậu (chủ biên), Lê Ngọc Lăng, Phạm Thế Long, Nguyễn Minh Tuấn, \textit{Các đề thi Olympic toán sinh viên toàn quốc}, NXB Giáo dục, 2006}.
\end{thebibliography}
\end{document}
