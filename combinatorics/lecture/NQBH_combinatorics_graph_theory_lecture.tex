\documentclass{article}
\usepackage[backend=biber,natbib=true,style=alphabetic,maxbibnames=50]{biblatex}
\addbibresource{/home/nqbh/reference/bib.bib}
\usepackage[utf8]{vietnam}
\usepackage{tocloft}
\renewcommand{\cftsecleader}{\cftdotfill{\cftdotsep}}
\usepackage[colorlinks=true,linkcolor=blue,urlcolor=red,citecolor=magenta]{hyperref}
\usepackage{amsmath,amssymb,amsthm,enumitem,float,graphicx,mathtools,tikz}
\usetikzlibrary{angles,calc,intersections,matrix,patterns,quotes,shadings}
\allowdisplaybreaks
\newtheorem{assumption}{Assumption}
\newtheorem{baitoan}{}
\newtheorem{conjecture}{Conjecture}
\newtheorem{corollary}{Corollary}
\newtheorem{definition}{Definition}[section]
\newtheorem{dinhnghia}{Định nghĩa}[section]
\newtheorem{dinhly}{Định lý}[section]
\newtheorem{example}{Example}
\newtheorem{goal}{Goal}
\newtheorem{hypothesis}{Hypothesis}
\newtheorem{lemma}{Lemma}
\newtheorem{notation}{Notation}
\newtheorem{note}{Note}
\newtheorem{principle}{Principle}
\newtheorem{problem}{Problem}
\newtheorem{proposition}{Proposition}
\newtheorem{question}{Question}
\newtheorem{quyuoc}{Quy ước}
\newtheorem{remark}{Remark}
\newtheorem{Rule}{Rule}
\newtheorem{theorem}{Theorem}
\newtheorem{psy-theorem}{$\Psi$-Theorem}
\newtheorem{phi-theorem}{$\Phi$-Theorem}
\newtheorem{vidu}{Ví dụ}
\usepackage[left=1cm,right=1cm,top=5mm,bottom=5mm,footskip=4mm]{geometry}
\def\labelitemii{$\circ$}
\DeclareRobustCommand{\divby}{%
	\mathrel{\vbox{\baselineskip.65ex\lineskiplimit0pt\hbox{.}\hbox{.}\hbox{.}}}%
}
\setlist[itemize]{leftmargin=*}
\setlist[enumerate]{leftmargin=*}

\title{Lecture Note: Combinatorics {\it\&} Graph Theory\\Bài Giảng: Tổ Hợp {\it\&} Lý Thuyết Đồ Thị}
\author{Nguyễn Quản Bá Hồng\footnote{A Scientist {\it\&} Creative Artist Wannabe. E-mail: {\tt nguyenquanbahong@gmail.com, hong.nguyenquanba@umt.edu.vn}. Bến Tre City, Việt Nam.}}
\date{\today}

\begin{document}
\maketitle
\begin{abstract}
	This text is a part of the series {\it Some Topics in Advanced STEM \& Beyond}:
	
	{\sc url}: \url{https://nqbh.github.io/advanced_STEM/}.
	
	Latest version:
	\begin{itemize}
		\item {\it Lecture Note: Combinatorics \& Graph Theory -- Bài Giảng: Tổ Hợp \& Lý Thuyết Đồ Thị}.
		
		PDF: {\sc url}: \url{https://github.com/NQBH/advanced_STEM_beyond/blob/main/combinatorics/lecture/NQBH_combinatorics_graph_theory_lecture.pdf}.
		
		\TeX: {\sc url}: \url{https://github.com/NQBH/advanced_STEM_beyond/blob/main/combinatorics/lecture/NQBH_combinatorics_graph_theory_lecture.tex}.
		\item {\it Slide: Combinatorics \& Graph Theory -- Slide Bài Giảng: Tổ Hợp \& Lý Thuyết Đồ Thị}.
		
		PDF: {\sc url}: \url{https://github.com/NQBH/advanced_STEM_beyond/blob/main/combinatorics/slide/NQBH_combinatorics_graph_theory_slide.pdf}.
		
		\TeX: {\sc url}: \url{https://github.com/NQBH/advanced_STEM_beyond/blob/main/combinatorics/slide/NQBH_combinatorics_graph_theory_slide.tex}.
		\item Codes:
		\begin{itemize}
			\item C{\tt/}C++: \url{https://github.com/NQBH/advanced_STEM_beyond/blob/main/combinatorics/C++}.
			\item Python: \url{https://github.com/NQBH/advanced_STEM_beyond/blob/main/combinatorics/Python}.
		\end{itemize}
	\end{itemize}
\end{abstract}
\tableofcontents

%------------------------------------------------------------------------------%

\section{Basic Combinatorics -- Tổ Hợp Cơ Bản}

\subsection{Mathematical induction \& recurrence -- Quy nạp \& truy hồi}

%------------------------------------------------------------------------------%

\subsection{Pigeonhole principle \& Ramsey theory -- Nguyên lý chuồng bồ câu \& lý thuyết Ramsey}

%------------------------------------------------------------------------------%

\subsection{Counting rules \& Stirling number of type 1 \& type 2}

%------------------------------------------------------------------------------%

\subsection{Hoán vị \& tổ hợp}

%------------------------------------------------------------------------------%

\subsection{Hệ số nhị thức \& đa thức}

%------------------------------------------------------------------------------%

\subsection{Phân vùng số nguyên \& nguyên tắc loại suy}

%------------------------------------------------------------------------------%

\section{Graph Theory -- Lý Thuyết Đồ Thị}

\begin{dinhnghia}[\cite{Ha_Thanh_to_hop}, Def. 7.2, p. 249, Đỉnh cô lập, lá]
	Cho $G$ là 1 đồ thị. Đỉnh có bậc $0$ được gọi là {\rm đỉnh cô lập}, đỉnh có bậc $1$ được gọi là {\rm lá}.
\end{dinhnghia}

\begin{dinhnghia}[\cite{Ha_Thanh_to_hop}, Def. 7.3, p. 249, Đồ thị chính quy]
	1 đồ thị được gọi là {\rm chính quy bậc $d$} hoặc {\rm$d$-chính quy} nếu mỗi đỉnh có bậc bằng $d\in\mathbb{N}$.
\end{dinhnghia}

\begin{dinhnghia}[\cite{Ha_Thanh_to_hop}, Def. 7.3, p. 249, Đỉnh thị khối]
	1 đồ thị được gọi là {\rm đồ thị bậc $3$} nếu nó chính quy bậc $3$, i.e., mỗi đỉnh đồ thị có bậc bằng $3$.
\end{dinhnghia}

\begin{goal}[Tính khả dĩ của dãy bậc của đồ thị]
	Tìm vài dấu hiệu hoặc vài điều kiện cần \& đủ để có thể quyết định liệu 1 dãy số nguyên dương $(a_i)_{i=1}^n\subset\mathbb{N}$ cho trươc có  thể thể là dãy bậc của đồ thị mà không phải vẽ biểu đồ.
\end{goal}

\begin{dinhnghia}[\cite{Ha_Thanh_to_hop}, Def. 7.6, p. 249, Dãy bậc của đồ thị, chuỗi đồ thị]
	{\rm Chuỗi bậc} của đồ thị là dãy bậc của các đỉnh của nó theo thứ tự không tăng. 1 dãy số nguyên không âm không tăng được gọi là {\rm đồ thị} nếu tồn tại 1 đồ thị có chuỗi bậc chính xác là dãy số nguyên không âm đó.
\end{dinhnghia}

\begin{vidu}[Sequence $1,1,\ldots,1$]
	$1,1,1$ không phải là 1 dãy đồ thị vì không thể xây dựng 1 đồ thị có 3 đỉnh sao cho tất cả 3 bậc là $1$. Nhưng $1,1$ \& $1,1,1,1$, hay nói chung các dãy chỉ toàn số $1$ với độ dài là 1 số chẵn, i.e., $\{1\}_{i=1}^{2n}$, $\forall n\in\mathbb{N}^\star$, là các dãy đồ thị, nhưng bất kỳ dãy chỉ toàn số $1$ với độ dài là 1 số lẻ, i.e., $\{1\}_{i=1}^{2n+1}$, $\forall n\in\mathbb{N}^\star$, thì không phải là 1 dãy đồ thị (why?)
\end{vidu}

\begin{dinhly}[Euler's, \cite{Ha_Thanh_to_hop}, Thm. 7.9, p. 250]
	\label{thm: Euler}
	Cho $G = (V,E)$ là đồ thị tổng quát với $d_1,\ldots,d_{|V|}\in\mathbb{N}$ là bậc của các đỉnh. Khi đó $\sum_{i=1}^{|V|} d_i = 2|E|$. Nói riêng, số đỉnh của $G$ có bậc lẻ là số chẵn.
\end{dinhly}
Briefly:
\begin{equation*}
	d_1,\ldots,d_{|V|}\mbox{ are degrees of vertices of a graph } G = (V,E)\Rightarrow\sum_{i=1}^{|V|} d_i = 2|E|\Rightarrow|\{i;d_i\not{\divby}\ 2\}|\divby2.
\end{equation*}
Chú ý chiều ngược lại chưa chắc đúng:
\begin{vidu}
	Dãy số $7,5,5,4,3,2,2,0$ không mâu thuẫn với Định lý \ref{thm: Euler} nhưng nó không phải là đồ thị (why?).
\end{vidu}

\begin{question}
	Có thể suy ra được những hệ quả nào từ đẳng thức $\sum_{i=1}^{|V|} d_i = 2|E|$?
\end{question}

\begin{baitoan}
	Cho $G = (V,E)$ là đồ thị tổng quát với $d_1,\ldots,d_p\in\mathbb{N}$ là bậc của các đỉnh. Chứng minh: (i) Bậc cao nhất $d_{\max}\coloneqq\max_{1\le i\le p} d_i$ thỏa $d_{\max}\ge\dfrac{2|E|}{|V|}$. (ii)
\end{baitoan}


%------------------------------------------------------------------------------%

\section{Posets, Kết Nối, Lưới Boolean}

%------------------------------------------------------------------------------%

\section{Miscellaneous}

%------------------------------------------------------------------------------%

\printbibliography[heading=bibintoc]
	
\end{document}