\documentclass[oneside]{book}
\usepackage[backend=biber,natbib=true,style=alphabetic,maxbibnames=50]{biblatex}
\addbibresource{/home/nqbh/reference/bib.bib}
\usepackage[utf8]{vietnam}
\usepackage{tocloft}
\renewcommand{\cftsecleader}{\cftdotfill{\cftdotsep}}
\usepackage[colorlinks=true,linkcolor=blue,urlcolor=red,citecolor=magenta]{hyperref}
\usepackage{amsmath,amssymb,amsthm,enumitem,fancyvrb,float,graphicx,mathtools,minitoc,tikz}
\usetikzlibrary{angles,calc,intersections,matrix,patterns,quotes,shadings}
\usepackage{fancyhdr}
\pagestyle{fancy}
\fancyhf{}
%  \addtolength{\headheight}{0pt}% obsolete
\lhead{\itshape  \chaptername~\thechapter}
\rhead{\itshape  \nouppercase{\leftmark}} %\nouppercase !
\renewcommand{\chaptermark}[1]{\markboth{#1}{}}
\cfoot{\thepage}

\usepackage{textcase}

\makeatletter
\def\@makechapterhead#1{%
    \vspace*{50\p@}%
    {\parindent \z@ \centering\normalfont
        \ifnum \c@secnumdepth >\m@ne
        \if@mainmatter
        \huge\bfseries \MakeTextUppercase{\@chapapp}\space \thechapter
        \par\nobreak
        \vskip 20\p@
        \fi
        \fi
        \interlinepenalty\@M
        \huge \bfseries \MakeTextUppercase{#1}\par\nobreak
        \vskip 40\p@
}}
\def\@makeschapterhead#1{%
    \vspace*{50\p@}%
    {\parindent \z@ \centering
        \normalfont
        \interlinepenalty\@M
        \huge \bfseries  \MakeTextUppercase{#1}\par\nobreak
        \vskip 40\p@
}}
\makeatother


\DeclareMathSymbol{\mathinvertedexclamationmark}{\mathclose}{operators}{'074}
\DeclareMathSymbol{\mathexclamationmark}{\mathclose}{operators}{'041}

\makeatletter
\newcommand{\raisedmathinvertedexclamationmark}{%
    \mathclose{\mathpalette\raised@mathinvertedexclamationmark\relax}%
}
\newcommand{\raised@mathinvertedexclamationmark}[2]{%
    \raisebox{\depth}{$\m@th#1\mathinvertedexclamationmark$}%
}
\begingroup\lccode`~=`! \lowercase{\endgroup
    \def~}{\@ifnextchar`{\raisedmathinvertedexclamationmark\@gobble}{\mathexclamationmark}}
\mathcode`!="8000
\makeatother

\usepackage{sectsty}
\allsectionsfont{\sffamily}
\allowdisplaybreaks
\newtheorem{assumption}{Assumption}
\newtheorem{baitoan}{Bài toán}
\newtheorem{cauhoi}{Câu hỏi}
\newtheorem{conjecture}{Conjecture}
\newtheorem{corollary}{Corollary}
\newtheorem{dangtoan}{Dạng toán}
\newtheorem{definition}{Definition}
\newtheorem{dinhly}{Định lý}
\newtheorem{dinhnghia}{Định nghĩa}
\newtheorem{example}{Example}
\newtheorem{ghichu}{Ghi chú}
\newtheorem{goal}{Goal}
\newtheorem{hequa}{Hệ quả}
\newtheorem{hypothesis}{Hypothesis}
\newtheorem{intuition}{Intuition}
\newtheorem{lemma}{Lemma}
\newtheorem{luuy}{Lưu ý}
\newtheorem{nhanxet}{Nhận xét}
\newtheorem{notation}{Notation}
\newtheorem{note}{Note}
\newtheorem{principle}{Principle}
\newtheorem{problem}{Problem}
\newtheorem{proposition}{Proposition}
\newtheorem{question}{Question}
\newtheorem{remark}{Remark}
\newtheorem{theorem}{Theorem}
\newtheorem{vidu}{Ví dụ}
\usepackage[left=1cm,right=1cm,top=1.5cm,bottom=1.5cm]{geometry}
\def\labelitemii{$\circ$}
\DeclareRobustCommand{\divby}{%
    \mathrel{\vbox{\baselineskip.65ex\lineskiplimit0pt\hbox{.}\hbox{.}\hbox{.}}}%
}
\setlist[itemize]{leftmargin=*}
\setlist[enumerate]{leftmargin=*}
\newcommand{\genstirlingI}[3]{%
    \genfrac{[}{]}{0pt}{#1}{#2}{#3}%
}
\newcommand{\genstirlingII}[3]{%
    \genfrac{\{}{\}}{0pt}{#1}{#2}{#3}%
}
\newcommand{\stirlingI}[2]{\genstirlingI{}{#1}{#2}}
\newcommand{\dstirlingI}[2]{\genstirlingI{0}{#1}{#2}}
\newcommand{\tstirlingI}[2]{\genstirlingI{1}{#1}{#2}}
\newcommand{\stirlingII}[2]{\genstirlingII{}{#1}{#2}}
\newcommand{\dstirlingII}[2]{\genstirlingII{0}{#1}{#2}}
\newcommand{\tstirlingII}[2]{\genstirlingII{1}{#1}{#2}}

\title{Lecture Note: Combinatorics {\it\&} Graph Theory\\Bài Giảng: Tổ Hợp {\it\&} Lý Thuyết Đồ Thị}
\author{Nguyễn Quản Bá Hồng\footnote{A scientist- {\it\&} creative artist wannabe, a mathematics {\it\&} computer science lecturer of Department of Artificial Intelligence {\it\&} Data Science (AIDS), School of Technology (SOT), UMT Trường Đại học Quản lý {\it\&} Công nghệ TP.HCM, Hồ Chí Minh City, Việt Nam.\\E-mail: {\sf nguyenquanbahong@gmail.com} {\it\&} {\sf hong.nguyenquanba@umt.edu.vn}. Website: \url{https://nqbh.github.io/}. GitHub: \url{https://github.com/NQBH}.}}
\date{\today}

\begin{document}
\maketitle
\setcounter{secnumdepth}{4}
\setcounter{tocdepth}{4}
\dominitoc % Initialization
\tableofcontents

%------------------------------------------------------------------------------%

\chapter*{Preface}

\section*{Abstract}
This text is a part of the series {\it Some Topics in Advanced STEM \& Beyond}:

{\sc url}: \url{https://nqbh.github.io/advanced_STEM/}.

Latest version:
\begin{itemize}
    \item {\it Lecture Note: Combinatorics \& Graph Theory -- Bài Giảng: Tổ Hợp \& Lý Thuyết Đồ Thị}.
    
    PDF: {\sc url}: \url{https://github.com/NQBH/advanced_STEM_beyond/blob/main/combinatorics/lecture/NQBH_combinatorics_graph_theory_lecture.pdf}.
    
    \TeX: {\sc url}: \url{https://github.com/NQBH/advanced_STEM_beyond/blob/main/combinatorics/lecture/NQBH_combinatorics_graph_theory_lecture.tex}.
    \item {\it Slide: Combinatorics \& Graph Theory -- Slide Bài Giảng: Tổ Hợp \& Lý Thuyết Đồ Thị}.
    
    PDF: {\sc url}: \url{https://github.com/NQBH/advanced_STEM_beyond/blob/main/combinatorics/slide/NQBH_combinatorics_graph_theory_slide.pdf}.
    
    \TeX: {\sc url}: \url{https://github.com/NQBH/advanced_STEM_beyond/blob/main/combinatorics/slide/NQBH_combinatorics_graph_theory_slide.tex}.
    \item {\it Survey: Combinatorics \& Graph Theory -- Khảo Sát: Tổ Hợp \& Lý Thuyết Đồ Thị}.
    
    PDF: {\sc url}: \url{https://github.com/NQBH/advanced_STEM_beyond/blob/main/combinatorics/NQBH_combinatorics.pdf}.
    
    \TeX: {\sc url}: \url{https://github.com/NQBH/advanced_STEM_beyond/blob/main/combinatorics/NQBH_combinatorics.tex}.
    \item Codes:
    \begin{itemize}
        \item C{\tt/}C++: \url{https://github.com/NQBH/advanced_STEM_beyond/blob/main/combinatorics/C++}.
        \item Pascal: \url{https://github.com/NQBH/advanced_STEM_beyond/blob/main/combinatorics/Pascal}.
        \item Python: \url{https://github.com/NQBH/advanced_STEM_beyond/blob/main/combinatorics/Python}.
    \end{itemize}
\end{itemize}
Tài liệu này là bài giảng tôi dạy cho sinh viên Khoa Công Nghệ (undegraduate Computer Science students) chuyên ngành Kỹ Thuật Phần Mềm (Software Engineering, abbr., SE) \& Trí Tuệ Nhân Tạo--Khoa Học Dữ Liệu (Artificial Intelligence--Data Science, abbr., AIDS) nên sẽ tập trung vào phương diện lập trình cho các khái niệm Tổ hợp \& Lý thuyết đồ thị được nghiên cứu. Bài giảng này gồm 2 phần chính:
\begin{itemize}
    \item {\bf Part I: Combinatorics -- Tổ Hợp}.
    \item {\bf Part II: Graph Theory -- Lý Thuyết Đồ Thị}. Tập trung vào các thuật toán trên cây (algorithms on trees) \& thuật toán trên đồ thị (algorithms on graphs)
\end{itemize}

%------------------------------------------------------------------------------%

\section*{Preliminaries}

%------------------------------------------------------------------------------%

\chapter{Miscellaneous}
\minitoc

%------------------------------------------------------------------------------%

\section{Contributors}

\begin{enumerate}
    \item {\sc Võ Ngọc Trâm Anh [VNTA].}
    \item {\sc Nguyễn Ngọc Thạch [NNT].}
    \item {\sc Phan Vĩnh Tiến [PVT].}
\end{enumerate}

%------------------------------------------------------------------------------%

\printbibliography[heading=bibintoc]

\end{document}