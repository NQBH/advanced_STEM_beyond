\documentclass{article}
\usepackage[backend=biber,natbib=true,style=alphabetic,maxbibnames=50]{biblatex}
\addbibresource{/home/nqbh/reference/bib.bib}
\usepackage[utf8]{vietnam}
\usepackage{tocloft}
\renewcommand{\cftsecleader}{\cftdotfill{\cftdotsep}}
\usepackage[colorlinks=true,linkcolor=blue,urlcolor=red,citecolor=magenta]{hyperref}
\usepackage{amsmath,amssymb,amsthm,enumitem,float,graphicx,mathtools,tikz}
\usetikzlibrary{angles,calc,intersections,matrix,patterns,quotes,shadings}
\allowdisplaybreaks
\newtheorem{assumption}{Assumption}
\newtheorem{baitoan}{}
\newtheorem{cauhoi}{Câu hỏi}
\newtheorem{conjecture}{Conjecture}
\newtheorem{corollary}{Corollary}
\newtheorem{dangtoan}{Dạng toán}
\newtheorem{definition}{Definition}
\newtheorem{dinhly}{Định lý}
\newtheorem{dinhnghia}{Định nghĩa}
\newtheorem{example}{Example}
\newtheorem{ghichu}{Ghi chú}
\newtheorem{hequa}{Hệ quả}
\newtheorem{hypothesis}{Hypothesis}
\newtheorem{lemma}{Lemma}
\newtheorem{luuy}{Lưu ý}
\newtheorem{nhanxet}{Nhận xét}
\newtheorem{notation}{Notation}
\newtheorem{note}{Note}
\newtheorem{principle}{Principle}
\newtheorem{problem}{Problem}
\newtheorem{proposition}{Proposition}
\newtheorem{question}{Question}
\newtheorem{remark}{Remark}
\newtheorem{theorem}{Theorem}
\newtheorem{vidu}{Ví dụ}
\usepackage[left=1cm,right=1cm,top=5mm,bottom=5mm,footskip=4mm]{geometry}
\def\labelitemii{$\circ$}
\DeclareRobustCommand{\divby}{%
	\mathrel{\vbox{\baselineskip.65ex\lineskiplimit0pt\hbox{.}\hbox{.}\hbox{.}}}%
}
\setlist[itemize]{leftmargin=*}
\setlist[enumerate]{leftmargin=*}

\title{Philosophy $\Phi$ -- Triết Học $\Phi$}
\author{Nguyễn Quản Bá Hồng\footnote{A Scientist {\it\&} Creative Artist Wannabe. E-mail: {\tt nguyenquanbahong@gmail.com}. Bến Tre City, Việt Nam.}}
\date{\today}

\begin{document}
\maketitle
\begin{abstract}
	This text is a part of the series {\it Some Topics in Advanced STEM \& Beyond}:
	
	{\sc url}: \url{https://nqbh.github.io/advanced_STEM/}.
	
	Latest version:
	\begin{itemize}
		\item {\it Philosophy $\Phi$ -- Triết Học $\Phi$}.
		
		PDF: {\sc url}: \url{https://github.com/NQBH/advanced_STEM_beyond/blob/main/philosophy/NQBH_philosophy.pdf}.
		
		\TeX: {\sc url}: \url{https://github.com/NQBH/advanced_STEM_beyond/blob/main/philosophy/NQBH_philosophy.tex}.
	\end{itemize}
\end{abstract}
\tableofcontents

%------------------------------------------------------------------------------%

\section{Wikipedia}

\subsection{Wikipedia{\tt/}logical intuition}
``{\it Logical Intuition}, or {\it mathematical intuition} or {\it rational intuition}, is a series of instinctive foresight, know-how, \& savviness\footnote{{\it savvy} [a] having practical knowledge \& understanding of something; having common sense.} often associated with the ability to perceive \href{https://en.wikipedia.org/wiki/Logic}{logical} or mathematical truth -- \& the ability to solve mathematical challenges efficiency. Humans apply logical intuition in proving mathematical \href{https://en.wikipedia.org/wiki/Theorem}{theorems}, validating logical arguments, developing algorithms \& \href{https://en.wikipedia.org/wiki/Heuristic}{heuristics}, \& in related contexts where mathematical challenges are involved. The ability to recognize logical or mathematical truth \& identify viable methods may vary from person to person, \& may even be a result of knowledge \& experience, which are subject to cultivation. The ability may not be realizable in a computer program by means other than \href{https://en.wikipedia.org/wiki/Genetic_programming}{genetic programming} or \href{https://en.wikipedia.org/wiki/Evolutionary_programming}{evolutionary programming}.

\subsubsection{History}
\href{https://en.wikipedia.org/wiki/Plato}{Plato} \& \href{https://en.wikipedia.org/wiki/Aristotle}{Aristotle} considered intuition a means for perceiving ideas, significant enough that for Aristotle, intuition comprised the only means of knowing principles that are \href{https://en.wikipedia.org/wiki/A_priori_and_a_posteriori}{not subject to argument}.

\href{https://en.wikipedia.org/wiki/Henri_Poincar%C3%A9}{\sc Henri Poincar\'e} distinguished logical intuition from \href{https://en.wikipedia.org/wiki/Intuition}{other forms of intuition}. In his book \href{https://en.wikipedia.org/wiki/The_Value_of_Science}{The Value of Science}, he points out that:
\begin{quote}
	{\it``$\ldots$ There are many kinds of intuition. I have said how much the intuition of pure number, when comes rigorous mathematical induction, differs from sensible intuition to which the imagination, properly so called, is the {\it principal contributor}.''}
	
	-- $\ldots$ Có nhiều loại trực giác. Tôi đã nói rằng trực quan về số thuần túy, khi có quy nạp toán học nghiêm ngặt, khác biệt đến mức nào với trực giác nhạy cảm mà trí tưởng tượng, được gọi một cách chính xác, là {\it đóng góp chính}.	
\end{quote}
The passage goes on to assign 2 roles to logical intuition: to permit one to choose which \href{https://en.wikipedia.org/wiki/Axiom}{route} (axiom) to follow in search of scientific \href{https://en.wikipedia.org/wiki/Truth}{truth}, \& to allow one to comprehend logical developments.

\href{https://en.wikipedia.org/wiki/Bertrand_Russell}{\sc Bertrand Russell}, though critical of intuitive \href{https://en.wikipedia.org/wiki/Mysticism#Intuitive_insight_and_enlightenment}{mysticism}, pointed out that the degree to which a truth is \href{https://en.wikipedia.org/wiki/Self-evidence}{self-evident} according to logical intuition can vary, from 1 situation to another, \& stated that some self-evident truths are practically \href{https://en.wikipedia.org/wiki/Infallibility#In_philosophy}{infallible}\footnote{{\it infallible} [a] 1. never wrong; never making mistakes, $\ne$ {\it fallible}; 2. that never fails; always doing what it is supposed to do.}:
\begin{quote}
	{\it``When a certain number of logical principles have been admitted, the rest can be deduced from them; but the propositions deduced are often just as self-evident as those that were assumed without proof. All arithmetic, moreover, can be deduced from the general principles of logic, yet the simple propositions of arithmetic, such as `$2 + 2 = 4$', are just as self-evident as the principles of logic.''}
	
	-- Khi một số nguyên tắc logic nhất định đã được thừa nhận, phần còn lại có thể được suy ra từ chúng; nhưng các mệnh đề được suy ra thường hiển nhiên như những mệnh đề được giả định mà không có bằng chứng. Hơn nữa, tất cả số học đều có thể được suy ra từ các nguyên tắc chung của logic, tuy nhiên các mệnh đề đơn giản của số học, chẳng hạn như `$2 + 2 = 4$', cũng hiển nhiên như các nguyên tắc logic.
\end{quote}
\href{https://en.wikipedia.org/wiki/Kurt_G%C3%B6del}{\sc Kurt G\"odel} demonstrated based on his \href{https://en.wikipedia.org/wiki/G%C3%B6del%27s_incompleteness_theorems}{incompleteness theorems} that intuition-based \href{https://en.wikipedia.org/wiki/Propositional_calculus}{propositional calculus} cannot be \href{https://en.wikipedia.org/wiki/Many-valued_logic}{finitely valued}. G\"odel also likened logical intuition to sense perception, \& considered the mathematical constructs that humans perceive to have an independent \href{https://en.wikipedia.org/wiki/Philosophical_realism}{existence} of their own. Under this line of reasoning, the human mind's ability to sense such abstract constructs may not be finitely implementable.

\subsubsection{Discussion}
Dissent (bất đồng chính kiến) regarding the value of intuition in a logical or mathematical context may often hinge on the breadth of the definition of intuition \& the psychological underpinning of the word. Dissent regarding the implications of logical intuition in the fields of AI \& \href{https://en.wikipedia.org/wiki/Cognitive_computing}{cognitive computing} may similarly hinge on definitions. However, similarity between the potentially infinite nature of logical intuition posited by G\"odel \& the \href{https://en.wikipedia.org/wiki/Hard_problem_of_consciousness}{hard problem of consciousness} posited by \href{https://en.wikipedia.org/wiki/David_Chalmers}{\sc David Chalmers} suggest that the realms of intuitive knowledge \& experiential consciousness may both have aspects that are not reducible to classical physics concepts.'' -- \href{https://en.wikipedia.org/wiki/Logical_intuition}{Wikipedia{\tt/}logical intuition}

%------------------------------------------------------------------------------%

\subsection{Wikipedia{\tt/}abstract \& concrete}
``In philosophy \& the \href{https://en.wikipedia.org/wiki/The_arts}{arts},  a fundamental distinction is between things that are {\it abstract} \& things that are {\it concrete}. While there is no general \href{https://en.wikipedia.org/wiki/Consensus_decision-making}{consensus} as to how to precisely defined the 2, examples include that things like \href{https://en.wikipedia.org/wiki/Number}{numbers}, \href{https://en.wikipedia.org/wiki/Set_(mathematics)}{sets}, \& \href{https://en.wikipedia.org/wiki/Idea}{ideas} are abstract objects, while plants, dogs, \& planets are concrete objects. Popular suggestions for a definition include that the distinction between concreteness vs. abstractness is, respectively: between
\begin{enumerate}
	\item existence inside vs. outside \href{https://en.wikipedia.org/wiki/Spacetime}{space-time};
	\item having causes \& effects vs. not;
	\item being related, in \href{https://en.wikipedia.org/wiki/Metaphysics}{metaphysics}, to \href{https://en.wikipedia.org/wiki/Particulars}{particulars} vs. \href{https://en.wikipedia.org/wiki/Universal_(metaphysics)}{universals};
	\item belonging to either the physical vs. the mental realm (or the mental-\&-physical realm vs. neither).
\end{enumerate}
Another view is that it is the distinction between contingent existence vs. necessary existence; however, philosophers differ on which type of existence here defines abstractness, as opposed to concreteness. Despite this diversity of views, there is broad agreement concerning most objects as to whether they are abstract or concrete, s.t. most interpretations agree, e.g., that rocks are concrete objects while numbers are abstract objects.

Abstract objects are most commonly used in philosophy, particularly metaphysics, \& \href{https://en.wikipedia.org/wiki/Semantics}{semantics}. They are sometimes called {\it abstracta} in contrast to {\it concreta}. The term {\it abstract object} is said to have been coined by \href{https://en.wikipedia.org/wiki/Willard_Van_Orman_Quine}{\sc Willard Van Orman Quine}. \href{https://en.wikipedia.org/wiki/Abstract_object_theory}{Abstract object theory} is a discipline that studies the nature \& role of abstract objects. It holds that properties can be related to objects in 2 ways: through exemplification \& through encoding. Concrete objects exemplify their properties while abstract objects merely encode them. This approach is also known as the \href{https://en.wikipedia.org/wiki/Dual_copula_strategy}{dual copula strategy}.

\subsubsection{In philosophy}
The \href{https://en.wikipedia.org/wiki/Type%E2%80%93token_distinction}{type-token distinction} identifies physical objects that are tokens of a particular type of thing. The ``type'' of which it is a part is in itself an abstract object. The abstract-concrete distinction is often introduced \& initially understood in terms of \href{https://en.wikipedia.org/wiki/Paradigm}{paradigmatic} examples of objects of each kind. Examples of abstract vs. concrete objects: tennis vs. a tennis match, redness vs. red light reflected off of an apple \& hitting one's eyes, 5 vs. 5 cars, justice vs. a just action, humanity (the property of being human) vs. human population (the set of all humans).

Abstract objects have often garnered the interest of philosophers because they raise problems for popular theories. In \href{https://en.wikipedia.org/wiki/Ontology}{ontology}, abstract objects are considered problematic for \href{https://en.wikipedia.org/wiki/Physicalism}{physicalism} \& some forms of \href{https://en.wikipedia.org/wiki/Metaphysical_naturalism}{naturalism}. Historically, the most important ontological dispute about abstract objects has been the \href{https://en.wikipedia.org/wiki/Problem_of_universals}{problem of universals}. In \href{https://en.wikipedia.org/wiki/Epistemology}{epistemology}, abstract objects are considered problematic for \href{https://en.wikipedia.org/wiki/Empiricism}{empiricism}. If abstracta lack causal powers \& spatial location, how do we know about them? It is hard to say how they can affect our sensory experiences, \& yet we seem to agree on a wide range of claims about them.

Some, e.g. \href{https://en.wikipedia.org/wiki/Ernst_Mally}{\sc Ernst Mally}, \href{https://en.wikipedia.org/wiki/Edward_Zalta}{\sc Edward Zalta}, \& arguably, \href{https://en.wikipedia.org/wiki/Plato}{\sc Plato} in his \href{https://en.wikipedia.org/wiki/Theory_of_Forms}{\it Theory of Forms}, have held that abstract objects constitute the defining subject matter of \href{https://en.wikipedia.org/wiki/Metaphysics}{metaphysics} or philosophical inquiry more broadly. To the extent that philosophy is independent of empirical research, \& to the extent that empirical questions do not inform questions about abstracta, philosophy would seem especially suited to answering these latter questions.

In \href{https://en.wikipedia.org/wiki/Modern_philosophy}{modern philosophy}, the distinction between abstract \& concrete was explored by \href{https://en.wikipedia.org/wiki/Immanuel_Kant}{\sc Immanuel Kant} \& \href{https://en.wikipedia.org/wiki/G._W._F._Hegel}{G. W. F. Hegel}.

\href{https://en.wikipedia.org/wiki/Gottlob_Frege}{\sc Gottlob Frege} said that abstract objects, e.g., propositions, were members of $\frac{1}{3}$ realm, different from the external world or from internal \href{https://en.wikipedia.org/wiki/Consciousness}{consciousness}. See \href{https://en.wikipedia.org/wiki/Popper%27s_three_worlds}{{\sc Popper}'s 3 worlds}.

\paragraph{Abstract objects \& causality.} Another popular proposal for drawing the abstract-concrete distinction contends that an object is abstract if it lacks \href{https://en.wikipedia.org/wiki/Causality}{causal} power. A causal power has the ability to affect something causally. Thus, the empty set is abstract because it cannot act on other objects. 1 problem with this view is that it is not clear exactly what it is to have causal power. For a more detailed exploration of the abstract-concrete distinction, see the relevant \href{https://en.wikipedia.org/wiki/Stanford_Encyclopedia_of_Philosophy}{\it Stanford Encyclopedia of Philosophy} article.

\paragraph{Quasi-abstract entities.} Recently, there has been some philosophical interest in the development of a 3rd category of objects known as the quasi-abstract. Quasi-abstract objects have drawn particular attention in the area of \href{https://en.wikipedia.org/wiki/Social_ontology}{social ontology} \& \href{https://en.wikipedia.org/wiki/Documentality}{documentality}. Some argue that the over-adherence to the \href{https://en.wikipedia.org/wiki/Platonism}{platonist} duality of the concrete \& the abstract has led to a large category of social objects having been overlooked or rejected as \href{https://en.wikipedia.org/wiki/Nonexistent_object}{nonexistent} because they exhibit characteristics that the traditional duality between concrete \& abstract regards as incompatible. Specifically, the ability to have temporal location, but not spatial location, \& have causal agency (if only by acting through representatives). These characteristics are exhibited by a number of social objects, including states of the international legal system.

\subsubsection{Concrete \& abstract thought in psychology}
\href{https://en.wikipedia.org/wiki/Jean_Piaget}{\sc Jean Piaget} uses the terms ``concrete'' \& ``formal'' to describe 2 different types of learning. Concrete thinking involves facts \& descriptions about everyday, tangible objects, while abstract (\href{https://en.wikipedia.org/wiki/Formal_Operational#Formal_operational_stage}{formal operational}) thinking involves a mental process. Abstract idea vs. Concrete idea: dense things sink vs. It will sink if its density is greater than the density of the fluid, you breathe in oxygen \& breathe out carbon dioxide vs. gas exchange takes place between the air in the alveoli \& the blood, plants get water through their roots vs. water diffuses through the cell membrane of the root hair cells.'' -- \href{https://en.wikipedia.org/wiki/Abstract_and_concrete}{Wikipedia{\tt/}abstract \& concrete}

%------------------------------------------------------------------------------%

\subsection{Wikipedia{\tt/}natural law}
``{\it Natural law} (Latin: {\it ius naturale, lex naturalis}) is a system of law based on a close observation of \href{https://en.wikipedia.org/wiki/Natural_order_(philosophy)}{natural order} \& \href{https://en.wikipedia.org/wiki/Human_nature}{human nature}, from which values, thought by natural law's proponents to be intrinsic to human nature, can be deduced \& applied independently of \href{https://en.wikipedia.org/wiki/Positive_law}{positive law} (the express enacted laws of a \href{https://en.wikipedia.org/wiki/Sovereign_state}{state} or \href{https://en.wikipedia.org/wiki/Society}{society}). According to the theory of law called \href{https://en.wikipedia.org/wiki/Jusnaturalism}{jusnaturalism}, all people have inherent rights, conferred not by act of legislation but by ``\href{https://en.wikipedia.org/wiki/God}{God}, \href{https://en.wikipedia.org/wiki/Nature}{nature}, or \href{https://en.wikipedia.org/wiki/Reason}{reason}''. Natural law theory can also refer to ``theories of ethics, theories of politics, theories of civil law, \& theories of religious morality.''
	
\subsubsection{History}

\subsubsection{Contemporary jurisprudence}

\subsubsection{Methodology}

'' -- \href{https://en.wikipedia.org/wiki/Natural_law}{Wikipedia{\tt/}natural law}

%------------------------------------------------------------------------------%

\subsection{Wikipedia{\tt/}universal law}
``In \href{https://en.wikipedia.org/wiki/Law}{law} \& \href{https://en.wikipedia.org/wiki/Ethics}{ethics}, {\it universal law} or {\it universal principle} refers to concepts of legal \href{https://en.wikipedia.org/wiki/Legitimacy_(political)}{legitimacy} actions, whereby those \href{https://en.wikipedia.org/wiki/Value_(personal_and_cultural)}{principles} \& rules for governing human beings' conduct which are most universal in their acceptability, their applicability, translation, \& \href{https://en.wikipedia.org/wiki/Philosophical}{philosophical} basis, are therefore considered to be most legitimate.

\subsubsection{Debate}
Cognition, experiences \& intuition are the starting points of legal thought, which has to be seen through the glasses of universality \& abstractness. Notwithstanding this assumption, ``legal principles
\begin{enumerate}
	\item do not contain only logic \& reason \&
	\item they can be different in different situations despite their equal naming.
\end{enumerate}
The legal rules can be identical in different legal orders while they carry different wants''.

On 1 side ``universality, abstraction, \& theory itself are defined in a way that undermines the perspectives of some while privileging the perspectives of others''; on the other side, ``the aspiration to universality itself may stand in the way of its realization if it seals off from view the bias built into legal norms, public practices, \& established institutions''.

\subsubsection{Examples}
There are 12 universal laws.
\begin{enumerate}
	\item Law of Divine Oneness
	\item Law of Vibration
	\item Law of Action
	\item Law of Correspondence
	\item Law of Cause \& Effect
	\item Law of Compensation
	\item Law of Attraction
	\item Law of Perpetual Transmutation of Energy
	\item Law of Relativity
	\item Law of Polarity
	\item Law of Rhythm
	\item Law of Gender
\end{enumerate}
'' -- \href{https://en.wikipedia.org/wiki/Universal_law}{Wikipedia{\tt/}universal law}

%------------------------------------------------------------------------------%

\subsection{Wikipedia{\tt/}universality (philosophy)}
``In philosophy, {\it universality} or {\it absolutism} is the idea that universal facts exist \& can be progressively discovered, as opposed to \href{https://en.wikipedia.org/wiki/Relativism}{relativism}, which asserts that all facts are relative to one's perspective. Absolutism \& relativism have been explored at length in contemporary \href{https://en.wikipedia.org/wiki/Analytic_philosophy}{analytic philosophy}.

\subsubsection{Universality in ethics}
Main article: \href{https://en.wikipedia.org/wiki/Moral_universalism}{Wikipedia{\tt/}moral universalism}. When used in the context of ethics, the meaning of {\it universal} refers to that which is true for ``all similarly situated individuals''. \href{https://en.wikipedia.org/wiki/Rights}{Rights}, e.g. in \href{https://en.wikipedia.org/wiki/Natural_rights}{natural rights}, or in the 1789 \href{https://en.wikipedia.org/wiki/Declaration_of_the_Rights_of_Man_and_of_the_Citizen}{Declaration of the Rights of Man \& of the Citizen}, for those heavily influenced by the philosophy of the \href{https://en.wikipedia.org/wiki/Age_of_Enlightenment}{Enlightenment} \& its conception of a \href{https://en.wikipedia.org/wiki/Human_nature}{human nature}, could be considered universal. The 1948 \href{https://en.wikipedia.org/wiki/Universal_Declaration_of_Human_Rights}{Universal Declaration of Human Rights} is inspired by such principles.

Universal moralities contrast with \href{https://en.wikipedia.org/wiki/Moral_relativism}{moral relativisms}, which seek to account for differing ethical positions between people \& \href{https://en.wikipedia.org/wiki/Social_norm}{cultural norms}.

\subsubsection{Universality about truth}
In logic, or the consideration of valid arguments, a \href{https://en.wikipedia.org/wiki/Proposition}{proposition} is said to have universality if it can be conceived as being true in all possible contexts without creating a \href{https://en.wikipedia.org/wiki/Contradiction}{contradiction}. A \href{https://en.wikipedia.org/wiki/Universalism}{universalist conception of truth} accepts 1 or more universals, whereas a \href{https://en.wikipedia.org/wiki/Relativism}{relativist conception of truth} denies the existence of some or all \href{https://en.wikipedia.org/wiki/Problem_of_universals}{universals}.

\subsubsection{Universal in metaphysics}
Main article: \href{https://en.wikipedia.org/wiki/Universal_(metaphysics)}{Wikipedia{\tt/}universal (metaphysics)}. In \href{https://en.wikipedia.org/wiki/Metaphysics}{metaphysics}, a \href{https://en.wikipedia.org/wiki/Universal_(metaphysics)}{universal} is a proposed \href{https://en.wikipedia.org/wiki/Type_(metaphysics)}{type}, \href{https://en.wikipedia.org/wiki/Property_(metaphysics)}{property}, or \href{https://en.wikipedia.org/wiki/Relation_(metaphysics)}{relation} which can be instantiated by many different \href{https://en.wikipedia.org/wiki/Particular}{particulars}. While universals are related to the concept o universality, the concept is importantly distinct, see the main page on universals for a full treatment of the topic.'' -- \href{https://en.wikipedia.org/wiki/Universality_(philosophy)}{Wikipedia{\tt/}universality (philosophy)}

%------------------------------------------------------------------------------%

\section{Jordan Peterson. 12 Rules for Life}

\begin{quotation}
	\textit{``The most influential public intellectual in the Western world right now.''} -- New York Times
\end{quotation}

\section*{Introduction}
``\textit{12 Rules for Life: An Antidote\footnote{\textbf{antidote} [n] \textbf{1.} \textbf{antidote (to something)} a substance that controls the effects of a poison or disease; \textbf{2.} \textbf{antidote (to something)} anything that takes away the effects of something unpleasant.} to Chaos\footnote{\textbf{chaos} [n] [uncountable] a state of complete confusion \& lack of order; in physics, \textbf{chaos} is the property of a complex system whose behavior is so unpredictable that it appears random, especially because small changes in conditions can have very large effects; \textbf{chaos theory} is the branch of mathematics that deals with these complex systems.}} is a 2018 \href{https://en.wikipedia.org/wiki/Self-help_book}{self-help book} by the Canadian clinical\footnote{\textbf{clinical} [a] [only before noun] connected with the examination \& treatment of patients \& their illnesses.} psychologist\footnote{\textbf{psychologist} [n] a scientist who studies psychology.} \href{https://en.wikipedia.org/wiki/Jordan_Peterson}{Jordan Peterson}. It provides life advice through essays in abstract ethical\footnote{\textbf{ethical} [a] \textbf{1.} connected with beliefs \& principles about what is right \& wrong; \textbf{2.} morally correct or acceptable.} principles, psychology, mythology\footnote{\textbf{mythology} [n] [uncountable, countable] \textbf{1.} ancient myths in general; the ancient myths of a particular culture, society, etc.; \textbf{2.} \textbf{mythology (of something)} ideas that many people think are true but are in fact false.}, religion\footnote{\textbf{religion} [n] \textbf{1.} [uncountable] the belief in the existence of a god or gods, \& the activities that are connected with the worship of them; \textbf{2.} [countable] 1 of the systems of belief that are based on the belief in the existence of a particular god or gods.}, \& personal anecdotes\footnote{\textbf{anecdote} [n] [countable, uncountable] \textbf{1.} \textbf{anecdote (about somebody{\tt/}something)} a short, interesting or funny story about a real person or event; \textbf{2.} a personal account of an event, especially one that is considered as possibly not true or accurate.}.''[$\ldots$] ``The book is written in a more accessible style than his previous academic book, \href{https://en.wikipedia.org/wiki/Maps_of_Meaning:_The_Architecture_of_Belief}{Maps of Meaning: The Artchitecture of Belief} (1999). A sequel, \href{https://en.wikipedia.org/wiki/Beyond_Order}{Beyond Order: 12 More Rules for Life}, was published in Mar 2021.'' -- \href{https://en.wikipedia.org/wiki/12_Rules_for_Life}{Wikipedia{\tt/}12 Rules for Life}

\subsection*{Overview}

\paragraph*{Background.} ``Peterson's interest in writing the book grew out of a personal hobby of answering questions posted on \href{https://en.wikipedia.org/wiki/Quora}{Quora}; 1 such question being
\begin{question}
	\fbox{``What are the most valuable things everyone should know?'',}
\end{question}
to which his answer comprised 42 rules. The early vision \& promotion of the book aimed to include all rules, with the title ``42''. Peterson stated that it ``isn't only written for other people. It's warning to me.'''' -- \href{https://en.wikipedia.org/wiki/12_Rules_for_Life#Background}{Wikipedia{\tt/}12 Rules for Life{\tt/}overview{\tt/}background}

\paragraph*{12 Rules.} ``The book is divided into chapters with each title representing 1 of the following 12 specific rules for life as explained through an essay.
\begin{enumerate}
	\item ``Stand up straight with your shoulders back.''
	\item ``Treat yourself like you are someone you are responsible for helping.''
	\item ``Make friends with people who want the best for you.''
	\item ``Compare yourself to who you were yesterday, not to who someone else is today.''
	\item ``Do not let your children do anything that makes you dislike them.''
	\item ``Set your house in perfect order before you criticize the world.''
	\item ``Pursue what is meaningful (not what is expedient\footnote{\textbf{expedient} [n] an action that is useful or necessary for a particular purpose, but not always fair or right.}).''
	\item ``Tell the truth -- or, at least, don't lie.''
	\item ``Assume that the person you are listening to might know something you don't.''
	\item ``Be precise in your speech.''
	\item ``Do not bother children when they are skate-boarding.''
	\item ``Pet a cat when you encounter\footnote{\textbf{encounter} [v] \textbf{1.} \textbf{encounter something} to experience something, especially something unpleasant or difficult, while you are trying to do something else, \textsc{synonym}: \textbf{run into something}; \textbf{2.} \textbf{encounter something{\tt/}somebody} to discover or experience something, or meet somebody, especially something{\tt/}somebody new, unusual or unexpected, \textsc{synonym}: \textbf{come across somebody{\tt/}something}; [n] a meeting, especially one that is sudden or unexpected.} one on the street.'''' -- \href{https://en.wikipedia.org/wiki/12_Rules_for_Life#12_Rules}{Wikipedia{\tt/}12 Rules for Life{\tt/}overview{\tt/}content}
\end{enumerate} 

\paragraph*{Content.} ``The book's central idea is that ``\fbox{suffering is built into the structure of \href{https://en.wikipedia.org/wiki/Being}{being}}'' \& although it can be unbearable\footnote{\textbf{unbearable} [a] too painful, annoying or unpleasant to deal with or accept, \textsc{synonym}: \textbf{intolerable}, \textsc{opposite}: \textbf{bearable}.}, people have a choice either to withdraw\footnote{\textbf{withdraw} [v] \textbf{1.} [transitive, intransitive] (used especially about armed forces) to make people leave a place; to leave a place; \textbf{2.} [intransitive] \textbf{withdraw (to something)} to leave a room; to go away from other people; \textbf{3.} [transitive] to move something back, out or away from something; \textbf{4.} [transitive] to take money out of a bank account or financial institution; \textbf{5.} [intransitive] to stop taking part in something; \textbf{6.} [intransitive] to stop wanting to speak to, or be with, other people; \textbf{7.} [transitive] to no longer provide or offer something; to no longer make something available; \textbf{8.} [transitive] \textbf{withdraw something} to say that you no longer agree with what you said before.}, which is a ``suicidal\footnote{\textbf{suicidal} [a] (of people) very unhappy or depressed \& feeling that they want to kill themselves; (of behavior) showing this.} gesture\footnote{\textbf{gesture} [n] \textbf{1.} [countable, uncountable] \textbf{gesture (of something)} something that you do or say to show a particular feeling or intention; \textbf{2.} [countable, uncountable] a movement that you make with your hands, your head or your face to show a particular meaning.}'', or to face \& transcend\footnote{\textbf{transcend} [v] \textbf{transcend something} to be or go beyond the usual limits of something.} it. Living in a world of chaos \& order,\fbox{ everyone has ``darkness''} that can \fbox{``turn them into the monsters they're capable of being''} to satisfy their \fbox{dark impulses\footnote{\textbf{impulse} [n] \textbf{1.} [countable, usually singular, uncountable] a sudden strong wish or need to do something, without stopping to think about the results; \textbf{2.} [countable, usually singular] something that causes somebody{\tt/}something to do something or to develop \& make progress; \textbf{3.} [countable] a brief electric current, e.g. one that travels from a nerve to a muscle; \textbf{4.} [countable] (\textit{physics}) the change in momentum of an object due to a force.} in the right situations}. Scientific experiments like the \href{https://en.wikipedia.org/wiki/Inattentional_blindness#Invisible_Gorilla_Test}{Invisible Gorilla Test} show that perception\footnote{\textbf{perception} [n] \textbf{1.} [uncountable, countable] an idea, a belief or an image you have as a result of how you see or understand something; \textbf{2.} [uncountable] the way you notice things or the ability to notice things with the senses; in biology, \textbf{perception} refers to the processes in the nervous system by which a living thing becomes aware of events \& things outside itself; \textbf{3.} [uncountable] the ability to understand the true nature of something, \textsc{synonym}: \textbf{insight}.} is adjusted to aims, \& it is \fbox{better to seek \href{https://en.wikipedia.org/wiki/Meaning_(psychology)}{meaning} rather than happiness}. Peterson notes:
\begin{quotation}
	``It's all very well to think the meaning of life is happiness, but what happens when you're unhappy? Happiness is a great side effect. When it comes, accept it gratefully\footnote{\textbf{grateful} [a] \textbf{1.} feeling or showing thanks because somebody has done something kind for you or has done as you asked; \textbf{2.} used to make a request, especially in a letter or in a formal situation.}. But it's fleeting\footnote{\textbf{fleeting} [a] [usually before noun] lasting only a short time, \textsc{synonym}: \textbf{brief}.} \& unpredictable\footnote{\textbf{unpredictable} [a] that cannot be predicted because it changes a lot or depends on too many different things, \textsc{opposite}: \textbf{predictable}.}. It's not something to aim at -- because it's not an aim. \& if happiness is the purpose of life, what happens when you're unhappy? Then you're a failure.''
\end{quotation}
The book advances the idea that \fbox{people are born with an instinct\footnote{\textbf{instinct} [n] [uncountable, countable] a natural tendency for people \& animals to behave in a particular way, using the knowledge \& abilities that they were born with rather than thought or training.} for ethics \& meaning}, \& should take responsibility\footnote{\textbf{responsibility} [n] \textbf{1.} [uncountable, countable] a duty to deal with or take care of somebody{\tt/}something, so that you may be blamed if something goes wrong; \textbf{2.} [uncountable] \textbf{responsibility (for something)} blame for something bad that has happened; \textbf{3.} [countable, uncountable] a moral duty to behave well with regard to somebody{\tt/}something.} to search for meaning above their own interests (Rule 7, ``Pursue what is meaningful, not what is expedient''). Such thinking is reflected both in contemporary\footnote{\textbf{contemporary} [a] \textbf{1.} belonging to the present time, \textsc{synonym}: \textbf{modern}; \textbf{2.} (especially of people \& society) belonging to the same time as somebody{\tt/}something else; [n] a person or thing living or existing at the same time as somebody{\tt/}something else, especially somebody who is about the same age as somebody else.} stories e.g. \href{https://en.wikipedia.org/wiki/Pinocchio_(1940_film)}{Pinocchio}, \href{https://en.wikipedia.org/wiki/The_Lion_King}{The Lion King}, \& \href{https://en.wikipedia.org/wiki/Harry_Potter}{Harry Potter}, \& in ancient stories from the \href{https://en.wikipedia.org/wiki/Bible}{Bible}. To ``stand up straight with your shoulders back'' (Rule 1) is to ``accept the terrible responsibility of life'', to make self-sacrifice\footnote{\textbf{self-sacrifice} [n] [uncountable] (\textit{approving}) the act of not allowing yourself to have or do something in order to help other people.}, because the individual must rise above \href{https://en.wikipedia.org/wiki/Victimisation}{victimization}\footnote{\textbf{victimize} [v] [often passive] \textbf{victimize somebody} to make somebody suffer unfairly because you do not like them, their opinions or something that they have done.} \& ``conduct his or her life in a manner that requires the rejection\footnote{\textbf{rejection} [n] [uncountable, countable] \textbf{1.} the act of refusing to accept or consider something; \textbf{2.} the act of refusing to accept somebody for a job or position; \textbf{3.} the decision not to use, sell, publish, etc. something because its quality is not good enough; \textbf{4.} \textbf{rejection (of something)} an occasion when somebody's body does not accept a new organ after a transplant operation, by producing substances that attack the organ; \textbf{5.} the act of failing to give a person or an animal enough care or affection.} of immediate gratification\footnote{\textbf{gratification} [n] [uncountable, countable] (\textit{formal}) the state of feeling pleasure when something goes well for you or when your desires are satisfied; something that gives you pleasure, \textsc{synonym}: \textbf{satisfaction}.}, of natural \& perverse\footnote{\textbf{perverse} [a] showing a deliberate \& determined desire to behave in a way that most people think is wrong, unacceptable or unreasonable.} desires alike.'' The comparison to \href{https://en.wikipedia.org/wiki/Neurology}{neurological}\footnote{\textbf{neurological} [a] relating to nerves or to the science of neurology.} structures \& behavior of \href{https://en.wikipedia.org/wiki/Lobsters}{lobsters} is used as a natural example to the formation\footnote{\textbf{formation} [n] \textbf{1.} [uncountable] the action of forming something; the process of being formed; \textbf{2.} [countable] a thing that has been formed, especially in a particular place or in a particular way; \textbf{3.} [countable, uncountable] a particular arrangement or pattern of people or things.} of \href{https://en.wikipedia.org/wiki/Hierarchy}{social hierarchies}\footnote{\textbf{hierarchy} [n] \textbf{1.} [countable, uncountable] a system, especially in a society or an organization, in which people are organized into different levels of importance from highest to lowest; \textbf{2.} [countable] a system that ideas or beliefs can be arranged into.}.

The other parts of the work explore \& criticize the state of young men; the upbringing\footnote{\textbf{upbringing} [n] [singular, uncountable] the way in which a child is cared for \& taught how to behave while it is growing up.} that ignores \href{https://en.wikipedia.org/wiki/Sex_differences_in_humans}{sex differences} between boys \& girls (criticism of \href{https://en.wikipedia.org/wiki/Overprotective}{over-protection} \& \href{https://en.wikipedia.org/wiki/Tabula_rasa}{tabula rasa} model in \href{https://en.wikipedia.org/wiki/Social_science}{social sciences}); male-female \href{https://en.wikipedia.org/wiki/Interpersonal_relationship}{interpersonal relationships}; \href{https://en.wikipedia.org/wiki/School_shooting}{school shootings}; religion \& moral \href{https://en.wikipedia.org/wiki/Nihilism}{nihilism}\footnote{\textbf{nihilism} [n] [uncountable] (\textit{philosophy}) the belief that life has no meaning or purpose \& that religious \& moral principles have no value.}; \href{https://en.wikipedia.org/wiki/Relativism}{relativism}\footnote{\textbf{relativism} [n] [uncountable] the belief that truth is not always \& generally valid, but can be judged only in relation to other things, e.g. your personal situation.}; \& lack of respect for the values that built \href{https://en.wikipedia.org/wiki/Western_world}{Western society}.

In the last chapter, Peterson outlines the ways in which one can cope with the most tragic\footnote{\textbf{tragic} [a] \textbf{1.} making you feel very sad, usually because somebody has died or suffered a lot; \textbf{2.} [usually before noun] connected with tragedy ($=$ the style of literature).} events, events that are often \fbox{out of one's control}. In it, he describes his own personal struggle upon discovering that his daughter, Mikhaila, had a rare bone disease. The chapter is a meditation\footnote{\textbf{meditation} [n] \textbf{1.} [uncountable] the practice of thinking deeply, usually in silence, especially for religious reasons or in order to make your mind calm; \textbf{2.} [countable, usually plural] \textbf{meditation (on something)} serious thoughts on a particular subject that somebody writes down or speaks.} on how to maintain\footnote{\textbf{maintain} [v] \textbf{1.} \textbf{maintain something} to cause or enable a condition or situation to continue, \textsc{synonym}: \textbf{preserve}; \textbf{2.} \textbf{maintain something} to keep something at the same level or rate; \textbf{3.} to state strongly that something is true, even when some other people may not believe it; \textbf{4.} \textbf{maintain somebody{\tt/}something} to support somebody{\tt/}something over a long period of time by providing money, paying for food, etc.; \textbf{5.} \textbf{maintain something} to keep a building, machine, etc. in good condition by checking or repairing it regularly; \textbf{6.} \textbf{maintain a record} to write something down as a record \& keep adding the most recent information, \textsc{synonym}: \textbf{keep}.} a watchful\footnote{\textbf{watchful} [a] paying attention to what is happening in case of danger, accidents, etc.} eye on, \& cherish\footnote{\textbf{cherish} [v] (\textit{formal}) \textbf{1.} \textbf{cherish somebody{\tt/}something} to love somebody{\tt/}something very much \& want to protect them or it; \textbf{2.} \textbf{cherish something} to keep an idea, a hope or a pleasant feeling in your mind for a long time.}, life's small redeemable\footnote{\textbf{redeemable} [a] \textbf{redeemable (against something)} that can be exchanged for money or goods.} qualities (i.e., ``pet a cat when you encounter one''). It also outlines a practical way to deal with hardship\footnote{\textbf{hardship} [n] [uncountable, countable] a situation that is difficult \& unpleasant because you do not have enough money, food, clothes, etc.}: to shorten one's temporal\footnote{\textbf{temporal} [a] \textbf{1.} connected with or limited by time; \textbf{2.} connected with the real physical world, not spiritual matters; \textbf{3.} (\textit{anatomy}) near the temples at the side of the head.} scope of responsibility (e.g., focusing on the next minute rather than the next 3 months).

Canadian psychiatrist \& psychoanalyst \href{https://en.wikipedia.org/wiki/Norman_Doidge}{Norman Doidge} wrote \cite{Peterson2018}'s foreword.'' -- \href{https://en.wikipedia.org/wiki/12_Rules_for_Life#Content}{Wikipedia{\tt/}12 Rules for Life{\tt/}overview{\tt/}content}

\begin{quotation}
	``The most influential public intellectual\footnote{\textbf{intellectual} [a] [usually before noun] connected with or using a person's ability to think in a logical way \& understand things, \textsc{synonym}: \textbf{mental}; [n] a person who is well educated \& enjoys activities in which they have to think seriously about things.} in the Western world right now.'' -- New York Times
\end{quotation}

%------------------------------------------------------------------------------%

\subsection{Foreword by Noman Doidge}
``Rules? More rules? Really? Isn't life complicated enough, restricting enough, without abstract rules that don't take our unique, individual situations into account? \& given that our brains are plastic, \& all develop differently based on our life experiences, why even expect that a few rules might be helpful to us all?

People don't clamor for rules, even in the Bible $\ldots$ as when Moses comes down the mountain, after a long absence, bearing the tablets inscribed with 10 commandments, \& finds the Children of Israel in revelry. They'd been Pharaoh's slaves \& subject to his tyrannical regulations for 400 years, \& after that Moses subjected them to the harsh desert wilderness for another 40 years, to purify them of their slavishness. Now, free at last, they are unbridled, \& have lost all control as they dance wildly around an idol, a golden calf, displaying all manner of corporeal corruption.

``I've got some good news $\ldots$ \& I've got some bad news,'' the lawgiver yells to them. ``Which do you want 1st?''

``The good news!'' the hedonists reply.

``I got Him from 15 commandments down to 10!''

``Hallelujah!'' cries the unruly crowd. ``\& the bad?''

``Adultery is still in.''

So rules there will be -- but, please, not too many. We are ambivalent about rules, even when we know they are good for us. If we are spirited souls, if we have character, rules seem restrictive, an affront to our sense of agency \& our pride in working out our own lives. Why should we be judged according to another's rule?

\& judged we are. After all, God didn't give Moses ``The 10 Suggestions,'' he gave Commandments; \& if I'm a free agent, my 1st reaction to a command might just be that nobody, not even God, tells me what to do, even if it's good for me. But the story of the golden calf also reminds us that without rules we quickly become slaves to our passions -- \& there's nothing freeing about that.

\& the story suggests something more: unchaperoned, \& left to our own untutored judgment, we are quick to aim low \& worship qualities that are beneath us -- in this case, an artificial animal that brings out our own animal instincts in a completely unregulated way. The old Hebrew story makes it clear how the ancients felt about our prospects for civilized behavior in the absence of rules that seek to elevate our gaze \& raise our standards.

1 neat thing about the Bible story is that it doesn't simply list its rules, as lawyers or legislators or administrators might; it embeds them in a dramatic tale that illustrates why we need them, thereby making them easier to understand. Similarly, in this book Prof. Peterson doesn't just propose his 12 rules, he tells stories, too, bringing to bear his knowledge of many fields as he illustrates \& explains why the best rules do not ultimately restrict us but instead facilitate our goals \& make for fuller, freer lives. p. 6

'' -- \cite[pp. 5--]{Peterson2018}



%------------------------------------------------------------------------------%

\subsection{Overture}

%------------------------------------------------------------------------------%

\subsection{Rule 1: Stand Up Straight with Your Shoulders Back}

%------------------------------------------------------------------------------%

\subsection{Rule 2: Treat Yourself Like Someone You Are Responsible for Helping}

%------------------------------------------------------------------------------%

\subsection{Rule 3: Make Friends with People Who Want The Best for You}

%------------------------------------------------------------------------------%

\subsection{Rule 4: Compare Yourself to Who You Were Yesterday, Not to Who Someone Else Is Today}

%------------------------------------------------------------------------------%

\subsection{Rule 5: Do Not Let Your Children Do Anything That Makes You Dislike Them}

%------------------------------------------------------------------------------%

\subsection{Rule 6: Set Your House In Perfect Order Before You Criticize The World}

%------------------------------------------------------------------------------%

\subsection{Rule 7: Pursue What Is Meaningful (Not What Is Expedient)}

\subsubsection{Get While The Getting's Good}
``Life is suffering. That's clear. There is no more basic, irrefutable truth. It's basically what God tells Adam \& Eve, immediately before he kicks them out of Paradise.
\begin{quotation}
	Unto the woman he said, I will greatly multiply thy sorrow \& thy conception; in sorrow thou shalt bring 4th children; \& thy desire shall be to thy husband, \& he shall rule over thee.
	
	\& unto Adam he said, Because thou hast hearkened unto the voice of thy wife, \& hast eaten of the tree, of which I commanded thee, saying, Thou shalt not eat of it: cursed is the ground for thy sake; in sorrow shalt thou eat of it all the days of thy life;
	
	Thorns also \& thistles shall it bring 4th to thee; \& thou shalt eat the herb of the field;
	
	By the sweat of your brow you will eat your food until you return to the ground, since from it you were taken; for dust you are \& to dust you will return.'' (Genesis 3:16-19. KJV)
\end{quotation}
What in the world should be done about that?

The simplest, most obvious, \& most direct answer? Pursue pleasure. Follow your impulses. Live for the moment. Do what's expedient. Lie, cheat, steal, deceive, manipulate -- but don't get caught. In an ultimately meaningless universe, what possible difference could it make? \& this is by no means a new idea. The fact of life's tragedy \& the suffering that is part of it has been used to justify the pursuit of immediate selfish gratification for a very long time.
\begin{quotation}
	Short \& sorrowful is our life, \& there is no remedy when a man comes to his end, \& no one has been known to return from Hades.
	
	Because we were born by mere chance, \& hereafter we shall be as though we had never been; because the breath in our nostrils is smoke, \& reason is a spark kindled by the beating of our hearts.
	
	When it is extinguished, the body will turn to ashes, \& the spirit will dissolve like empty air. Our name will be forgotten in time \& no one will remember our works; our life will pass away like the traces of a cloud, \& be scattered like mist that is chased by the rays of the sun \& overcome by its heat.
	
	For our allotted time is the passing of a shadow, \& there is no return from our death, because it is sealed up \& no one turns back.
	
	Come, therefore, let us enjoy the good things that exist, \& make use of the creation to the full as in youth.
	
	Let us take our fill of costly wine \& perfumes, \& let no flower of spring pass by us.
	
	Let us crown ourselves with rosebuds before they wither.
	
	Let none of us fail to share in our revelry, everywhere let us leave signs of enjoyment, because this is our portion, \& this our lot.
	
	Let us oppress the righteous poor man; let us not spare the widow nor regard the gray hairs of the aged.
	
	But let our might be our law of right, for what is weak proves itself to be useless.
	
	(Wisdom 2:1-11, RSV).
\end{quotation}
The pleasure of expediency may be fleeing, but it's pleasure, nonetheless, \& that's something to stack up against the terror \& pain of existence. Every man for himself, \& the devil take the hindmost, as the old proverb has it. Why not simply take everything you can get, whenever the opportunity arises? Why not determine to live in that manner?

Or is there an alternative, more powerful \& more compelling?

Our ancestors worked out very sophisticated answers to such questions, but we still don't understand them very well. This is because they are in large part still implicit -- manifest primarily in ritual \& myth \&, as of yet, incompletely articulated. We act them out \& represent them in stories, but we're not yet wise enough to formulate them explicitly. We're still chimps in a troupe, or wolves in a pack. we know how to behave. We know who's who, \& why. We've learned that through experience. Our knowledge has been shaped by our interaction with others. We've established predictable routines \& patterns of behavior -- but we don't really understand them, or know where they originated. They've evolved over great expanses of time. No one was formulating them explicitly (at least not in the dimmest reaches of the past), even though we've been telling each other how to act forever. 1 day, however, not so long ago, we woke up. We were already doing, but we started \textit{noticing} what we were doing. We started using our bodies as devices to represent their own actions. We started imitating \& dramatizing. We invented ritual. We started acting out our own experiences. Then we started to tell stories. We coded our observations of our own drama in these stories. In this manner, the information that was 1st only embedded in our behavior became represented in our stories. But we didn't \& still don't understand what it all means.

The Biblical narrative of Paradise \& the Fall is 1 such story, fabricated by our collective imagination, working over the centuries. It provides a profound account of the nature of Being, \& points the way to a mode of conceptualization \& action well-matched to that nature. In the Garden of Eden, prior to the dawn of self-consciousness -- so goes the story -- human beings were sinless. Our primordial parents, Adam \& Eve, walked with God. Then, tempted by the snake, the 1st couple ate from the tree of the knowledge of good \& evil, discovered Death \& vulnerability, \& turned away from God. Mankind was exiled from Paradise, \& began its effortful mortal existence. The idea of sacrifice enters soon afterward, beginning with the account of Cain \& Abel, \& developing through the Abrahamic adventures \& the Exodus: After much contemplation, struggling humanity learns that God's favor could be gained, \& his wrath averted, through proper sacrifice -- \&, also, that bloody murder might be motivated among those unwilling or unable to succeed in this manner.'' -- \cite[pp. 183--185]{Peterson2018}

\subsubsection{The Delay of Gratification}

%------------------------------------------------------------------------------%

\subsection{Rule 8: Tell The Truth -- Or, At Least, Don't Lie}

%------------------------------------------------------------------------------%

\subsection{Rule 9: Assume That The Person You Are Listening to Might Know Something You Don't}

%------------------------------------------------------------------------------%

\subsection{Rule 10: Be Precise In Your Speech}

%------------------------------------------------------------------------------%

\subsection{Rule 11: Do Not Bother Children When They Are Skateboarding}

%------------------------------------------------------------------------------%

\subsection{Rule 12: Pet A Cat When You Encounter One On The Street}

%------------------------------------------------------------------------------%

\subsection{Coda}

%------------------------------------------------------------------------------%

\subsection{Endnotes}

%------------------------------------------------------------------------------%

\section{Jordan Peterson. Beyond Order}

%------------------------------------------------------------------------------%

\section{Meaning of Life -- Giá Trị Của Sự Sống}

%------------------------------------------------------------------------------%

\section{Meaning of Death -- Giá Trị Của Cái Chết}

%------------------------------------------------------------------------------%

\section{Miscellaneous}

%------------------------------------------------------------------------------%

\printbibliography[heading=bibintoc]
	
\end{document}