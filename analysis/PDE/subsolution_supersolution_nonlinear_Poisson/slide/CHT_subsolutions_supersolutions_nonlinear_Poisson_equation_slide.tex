%%%%%%%%%%%%%%%%%%%%%%%%%%%%%%%%%%%%%%%%%
% Beamer Presentation
% LaTeX Template
% Version 1.0 (10/11/12)
%
% This template has been downloaded from:
% http://www.LaTeXTemplates.com
%
% License:
% CC BY-NC-SA 3.0 (http://creativecommons.org/licenses/by-nc-sa/3.0/)
%
%%%%%%%%%%%%%%%%%%%%%%%%%%%%%%%%%%%%%%%%%

%----------------------------------------------------------------------------------------
%	PACKAGES AND THEMES
%----------------------------------------------------------------------------------------

\documentclass{beamer}

\mode<presentation> {

% The Beamer class comes with a number of default slide themes
% which change the colors and layouts of slides. Below this is a list
% of all the themes, uncomment each in turn to see what they look like.

%\usetheme{default}
%\usetheme{AnnArbor}
%\usetheme{Antibes}
%\usetheme{Bergen}
%\usetheme{Berkeley}
\usetheme{Berlin}
%\usetheme{Boadilla}
%\usetheme{CambridgeUS}
%\usetheme{Copenhagen}
\usetheme{Darmstadt}
%\usetheme{Dresden}
%\usetheme{Frankfurt}
%\usetheme{Goettingen}
%\usetheme{Hannover}
%\usetheme{Ilmenau}
%\usetheme{JuanLesPins}
%\usetheme{Luebeck}
%\usetheme{Madrid}
%\usetheme{Malmoe}
%\usetheme{Marburg}
%\usetheme{Montpellier}
%\usetheme{PaloAlto}
%\usetheme{Pittsburgh}
%\usetheme{Rochester}
%\usetheme{Singapore}
%\usetheme{Szeged}
%\usetheme{Warsaw}

% As well as themes, the Beamer class has a number of color themes
% for any slide theme. Uncomment each of these in turn to see how it
% changes the colors of your current slide theme.

%\usecolortheme{albatross}
%\usecolortheme{beaver}
%\usecolortheme{beetle}
%\usecolortheme{crane}
%\usecolortheme{dolphin}
%\usecolortheme{dove}
%\usecolortheme{fly}
%\usecolortheme{lily}
%\usecolortheme{orchid}
%\usecolortheme{rose}
%\usecolortheme{seagull}
%\usecolortheme{seahorse}
%\usecolortheme{whale}
%\usecolortheme{wolverine}

%\setbeamertemplate{footline} % To remove the footer line in all slides uncomment this line
%\setbeamertemplate{footline}[page number] % To replace the footer line in all slides with a simple slide count uncomment this line

%\setbeamertemplate{navigation symbols}{} % To remove the navigation symbols from the bottom of all slides uncomment this line
}
\usepackage{color}
\usepackage{hyperref,amsmath,amssymb,amsthm,eufrak,latexsym,makeidx}
% For insert figure
\usepackage{graphicx} % Allows including images
\usepackage{booktabs} % Allows the use of \toprule, \midrule and \bottomrule in tables
%\usepackage[backend=biber]{biblatex}
\renewcommand{\thefootnote}{$\star$}
%----------------------------------------------------------------------------------------
%	TITLE PAGE
%----------------------------------------------------------------------------------------

\title[Methods of Subsolutions and Supersolutions for a Nonlinear Poisson Equation]{Method of Subsolutions and Supersolutions for a Nonlinear Poisson Equation} % The short title appears at the bottom of every slide, the full title is only on the title page

\author{\small NGUYEN Ngoc Minh Chau, VU Anh Tuan, NGUYEN Quan Ba Hong} % Your name
\institute[Universit\'e de Rennes 1] % Your institution as it will appear on the bottom of every slide, may be shorthand to save space
{
Universit\'e de Rennes 1, France \\ % Your institution for the title page
%E-mail: %\texttt{nguyenquanbahong@gmail.com} \\
% Your email address 
%\medskip
%\textbf{Supervisors:} \textsc{Dr. Dao Nguyen Anh, Dr. Bui Le Trong Thanh.}
}

\date{\today} % Date, can be changed to a custom date
\begin{document}

\begin{frame}
\titlepage % Print the title page as the first slide


\end{frame}

\begin{frame}
\frametitle{Overview} % Table of contents slide, comment this block out to remove it
\tableofcontents % Throughout your presentation, if you choose to use \section{} and \subsection{} commands, these will automatically be printed on this slide as an overview of your presentation
\end{frame}

%----------------------------------------------------------------------------------------
%	PRESENTATION SLIDES
%----------------------------------------------------------------------------------------

%------------------------------------------------
\section{Main PDE \& Weak Solutions} % Sections can be created in order to organize your presentation into discrete blocks, all sections and subsections are automatically printed in the table of contents as an overview of the talk
%------------------------------------------------
% A subsection can be created just before a set of slides with a common theme to further break down your presentation into chunks

\begin{frame}
\frametitle{Nonlinear Poisson equation}
Consider the following BVP for the nonlinear Poisson equation
\begin{equation}
\label{1.1}
\left\{ \begin{split}
 - \Delta u &= f\left( u \right), & \mbox{ in } U,\\
u &= 0, & \mbox{ on } \partial U,
\end{split} \right.
\end{equation}
where $f:\mathbb{R}\to \mathbb{R}$ is smooth, with 
\begin{align}
\label{1.2}
\left| {f'\left( z \right)} \right| \le C_f, \hspace{2mm}\forall z \in \mathbb{R},
\end{align}
for some constant $C_f$.

\end{frame}

\begin{frame}
\frametitle{Weak subsolution, supersolution, \& solution}
\begin{definition}[Weak sub- \& supersolutions]
\begin{itemize}
\item[(i)] We say that $\overline{u} \in H^1\left(U\right)$ is a \emph{weak supersolution} of \eqref{1.1} if 
\begin{align}
\label{1.3}
\int_U {D\overline u  \cdot Dvdx}  \ge \int_U {f\left( {\overline u } \right)vdx} ,\hspace{2mm}\forall v \in H_0^1\left( U \right),\hspace{1mm}v \ge 0 \mbox{ a.e.}
\end{align}
\item[(ii)] Similarly, $\underline{u} \in H^1\left(U\right)$ is a \emph{weak subsolution} of \eqref{1.1} provided
\begin{align}
\label{1.4}
\int_U {D\underline u  \cdot Dvdx}  \le \int_U {f\left( {\underline u } \right)vdx} ,\hspace{2mm}\forall v \in H_0^1\left( U \right),\hspace{1mm} v \ge 0 \mbox{ a.e.}
\end{align}
\item[(iii)] We say $u\in H_0^1\left(U\right)$ is a \emph{weak solution} of \eqref{1.1} if
\begin{align}
\int_U {Du \cdot Dvdx}  = \int_U {f\left( u \right)vdx} ,\hspace{2mm} \forall v \in H_0^1\left( U \right).
\end{align}
\end{itemize}
\end{definition}

\end{frame}
%------------------------------------------------

\begin{frame}
\frametitle{Quick notes}
\textbf{Question.} \textit{Why does ``$v\ge 0$ a.e.'' appear in the definitions of weak subsolution and supersolution?}
\medskip

\noindent
\textbf{Remark.} If $\overline{u}, \underline{u} \in C^2\left(U\right)$, then from \eqref{1.3} \& \eqref{1.4} it follows that
\begin{align}
\label{1.6}
 - \Delta \overline u  \ge f\left( {\overline u } \right),\hspace{1mm} - \Delta \underline u  \le f\left( {\underline u } \right), \mbox{ in } U.
\end{align}
When $f=0$, this gives the notions of subharmonicity and superharmonicity.
\end{frame}

%------------------------------------------------
\section{An Existence Theorem}

\begin{frame}
\frametitle{Method of subsolutions \& supersolutions}
\begin{theorem}[Existence of a solution between sub- and supersolutions]\label{theorem1.1}
Assume there exists a weak supersolution $\overline{u}$ and a weak subsolution $\underline{u}$ of \eqref{1.1} satisfying
\begin{align}
\label{1.8}
\underline u  \le 0,\hspace{1mm}\overline u  \ge 0 \mbox{ on } \partial U \mbox{ in the trace sense}, \hspace{1mm}\underline u  \le \overline u \mbox{ a.e. in } U.
\end{align}
Then there exists a weak solution $u$ of \eqref{1.1}, such that 
\begin{align}
\underline{u} \le u\le \overline{u} \mbox{ a.e. in } U.
\end{align}
\end{theorem}
%There are two strategies to prove this theorem:
%\begin{itemize}
%\item Integration by parts approach: see \cite{Evans2010}, p. 543.
%\item Maximum principle approach.
%\end{itemize}

\end{frame}

\begin{frame}
\frametitle{Outline of the proof}
\textit{Proof.} The proof in \cite{Evans2010}, p. 543 consists of 5 main steps: 
\begin{enumerate}
\item[1] Fix a number $\lambda >0$ so large that ${h_\lambda }\left( z \right): = f\left( z \right) + \lambda z$ is nondecreasing. Set $u_0=\underline{u}$, $u_k \in H_0^1\left(U\right)$, $k\in \mathbb{Z}^+$ is defined inductively to be the unique weak solution of the linear BVP
\begin{equation}
\label{1.11}
\left({\rm P}_{k+1} \right) \hspace{2mm} \left\{ \begin{split}
 - \Delta {u_{k + 1}} + \lambda {u_{k + 1}} &= f\left( {{u_k}} \right) + \lambda {u_k}, & \mbox{ in } U,\\
{u_{k + 1}} &= 0, & \mbox{ on } \partial U.
\end{split} \right.
\end{equation}
\item[2] Prove $\underline{u} = u_0 \le u_1 \le \ldots \le u_k\le \ldots$  a.e. in U.
\item[3] Prove $u_k\le \overline{u}$ a.e. in $U$, $\forall k\in \mathbb{N}$.


\end{enumerate}
\end{frame}

%------------------------------------------------
\begin{frame}
\frametitle{Outline of the proof (cont.)}
\begin{enumerate}
\item[4] Combining Step 2 \& 3 yields \begin{align}
\label{1.34}
\underline{u}\le \ldots \le u_k\le u_{k+1} \le \ldots \overline{u} \mbox{ a.e. in } U.
\end{align}
Thus, $u\left( x \right): = \mathop {\lim }\limits_{k \to \infty } {u_k}\left( x \right)$ exists for a.e. $x\in U$, and $u_k\to u$ in $L^2\left(U\right)$ by Dominated Convergence Theorem.

Prove ${\sup _{k\in \mathbb{N}}}{\left\| {{u_k}} \right\|_{H_0^1\left( U \right)}} < \infty $. Hence there is a subsequence $\left\{ {{u_{{k_j}}}} \right\}_{j = 1}^\infty $ which converges weakly in $H_0^1\left(U\right)$ to $u\in H_0^1\left(U\right)$. 
\item[5] Passing to limit $j\to \infty$, $u$ is a weak solution of problem \eqref{1.1}.\hfill $\square$
\end{enumerate}
 
\end{frame}

%------------------------------------------------

\begin{frame}
\frametitle{References}
\footnotesize{
\begin{thebibliography}{99} % Beamer does not support BibTeX so references must be inserted manually as below
\bibitem[Evans, 2010]{Evans2010} Lawrence C. Evans (2010) Partial Differential Equations, 2e
\newblock \emph{Graduate Studies in Mathematics} Volume 19, AMS.

\end{thebibliography}
}
\end{frame}

%------------------------------------------------

%----------------------------------------------------------------------------------------

\end{document}