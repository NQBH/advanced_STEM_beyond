\documentclass{article}
\usepackage[backend=biber,natbib=true,style=alphabetic,maxbibnames=50]{biblatex}
\addbibresource{/home/nqbh/reference/bib.bib}
\usepackage[utf8]{vietnam}
\usepackage{tocloft}
\renewcommand{\cftsecleader}{\cftdotfill{\cftdotsep}}
\usepackage[colorlinks=true,linkcolor=blue,urlcolor=red,citecolor=magenta]{hyperref}
\usepackage{amsmath,amssymb,amsthm,enumitem,float,graphicx,mathtools,tikz}
\usetikzlibrary{angles,calc,intersections,matrix,patterns,quotes,shadings}
\allowdisplaybreaks
\newtheorem{assumption}{Assumption}
\newtheorem{baitoan}{}
\newtheorem{cauhoi}{Câu hỏi}
\newtheorem{conjecture}{Conjecture}
\newtheorem{corollary}{Corollary}
\newtheorem{dangtoan}{Dạng toán}
\newtheorem{definition}{Definition}
\newtheorem{dinhly}{Định lý}
\newtheorem{dinhnghia}{Định nghĩa}
\newtheorem{example}{Example}
\newtheorem{ghichu}{Ghi chú}
\newtheorem{hequa}{Hệ quả}
\newtheorem{hypothesis}{Hypothesis}
\newtheorem{lemma}{Lemma}
\newtheorem{luuy}{Lưu ý}
\newtheorem{nhanxet}{Nhận xét}
\newtheorem{notation}{Notation}
\newtheorem{note}{Note}
\newtheorem{principle}{Principle}
\newtheorem{problem}{Problem}
\newtheorem{proposition}{Proposition}
\newtheorem{question}{Question}
\newtheorem{remark}{Remark}
\newtheorem{theorem}{Theorem}
\newtheorem{vidu}{Ví dụ}
\usepackage[left=1cm,right=1cm,top=5mm,bottom=5mm,footskip=4mm]{geometry}
\def\labelitemii{$\circ$}
\DeclareRobustCommand{\divby}{%
	\mathrel{\vbox{\baselineskip.65ex\lineskiplimit0pt\hbox{.}\hbox{.}\hbox{.}}}%
}
\setlist[itemize]{leftmargin=*}
\setlist[enumerate]{leftmargin=*}

\title{Lecture Note: Discrete Mathematics for Computer Science\\Bài Giảng: Toán Rời Rạc Cho Khoa Học Máy Tính}
\author{Nguyễn Quản Bá Hồng\footnote{A Scientist {\it\&} Creative Artist Wannabe. E-mail: {\tt nguyenquanbahong@gmail.com}. Bến Tre City, Việt Nam.}}
\date{\today}

\begin{document}
\maketitle
\begin{abstract}
	This text is a part of the series {\it Some Topics in Advanced STEM \& Beyond}:
	
	{\sc url}: \url{https://nqbh.github.io/advanced_STEM/}.
	
	Latest version:
	\begin{itemize}
		\item {\it Lecture Note: Discrete Mathematics for Computer Science -- Bài Giảng: Toán Rời Rạc Cho Khoa Học Máy Tính}.
		
		PDF: {\sc url}: \url{https://github.com/NQBH/advanced_STEM_beyond/blob/main/discrete_mathematics/lecture/NQBH_discrete_mathematics_lecture.pdf}.
		
		\TeX: {\sc url}: \url{https://github.com/NQBH/advanced_STEM_beyond/blob/main/discrete_mathematics/lecture/NQBH_discrete_mathematics_lecture.tex}.
	\end{itemize}
	Slide:
	\begin{itemize}
		\item {\it Discrete Mathematics for Computer Science -- Toán Rời Rạc Cho Khoa Học Máy Tính}.
		
		PDF: {\sc url}: \url{https://github.com/NQBH/advanced_STEM_beyond/blob/main/discrete_mathematics/slide/NQBH_discrete_mathematics_slide.pdf}.
		
		\TeX: {\sc url}: \url{https://github.com/NQBH/advanced_STEM_beyond/blob/main/discrete_mathematics/slide/NQBH_discrete_mathematics_slide.tex}.
	\end{itemize}
\end{abstract}
\tableofcontents

%------------------------------------------------------------------------------%

\section{Basic}

%------------------------------------------------------------------------------%

\section{Combinatorics -- Tổ Hợp}

\subsection{Combinatorics using {\tt SciPy}}

\begin{problem}[Permutation, arrangement, combination]
	Given $n,k\in\mathbb{N}^\star$, $k\le n$. Write {\sf Pascal{\tt/}Python{\tt/}C{\tt/}C++} programs to compute the numbers of permutations $P_n$, of arrangements $A_n^k$, of combinations $C_n^k$.
\end{problem}

\begin{proof}[Solution]
	$P_n = n!,A_n^k = \frac{n!}{(n - k)!},C_n^k = \frac{n!}{k!(n - k)!}$. Run {\tt combinatorics.py}.		
\end{proof}

\begin{problem}[Pascal triangle \& Newton binomial expansion]
	Given $m,n\in\mathbb{N}^\star$. Write {\sf Pascal{\tt/}Python{\tt/}C{\tt/}C++} programs to print the 1st $n + 1$ lines of the Pascal triangle \& Newton binomial expansion of $(a + b)^n,(a + b + c)^n,\left(\sum_{i=1}^m a_i\right)^n$,  $\forall a,b,c,a_i\in\mathbb{R}$, $\forall i = 1,\ldots,m$.
\end{problem}

\begin{problem}[Count number of lines formed by some points]
	Write {\sf Pascal{\tt/}Python{\tt/}C{\tt/}C++} programs to count the number of lines formed by $n\in\mathbb{N}^\star$ distinguished points in (2D) plane.
\end{problem}
{\sf Hint.} There are
\begin{equation}
	C_n^2 - \sum_{i=1}^m C_{a_i}^2 + m = \frac{n(n - 1)}{2} - \sum_{i=1}^m \frac{a_i(a_i - 1)}{2} + m
\end{equation}
lines, where $n$ given points is partitioned into exactly $m\in\mathbb{N}$ disjoint subsets $A_i$ of collinear points, where $a_i\coloneqq|A_i| = {\rm card}\,A_i$, $\forall i = 1,\ldots,m$.

\begin{problem}[Count number of intersections formed by some lines]
	Write {\sf Pascal{\tt/}Python{\tt/}C{\tt/}C++} programs to count the number of intersections of $n\in\mathbb{N}^\star$ distinguished lines in (2D) plane.
\end{problem}
{\sf Hint.} Nếu trong $n$ đường thẳng đã cho có đúng $m\in\mathbb{N}$ bộ lần lượt gồm $a_1,\ldots,a_m$ đường thẳng song song đôi một \& $k\in\mathbb{N}$ bộ lần lượt gồm $b_1,\ldots,b_k$ đường thẳng đồng quy thì số giao điểm:
\begin{align}
	C_n^2 - \sum_{i=1}^m C_{a_i}^2 - \sum_{i=1}^m C_{b_i}^2 + k = \frac{n(n - 1)}{2} - \sum_{i=1}^m \frac{a_i(a_i - 1)}{2} - \sum_{i=1}^k \frac{b_i(b_i - 1)}{2} + k
\end{align}

%------------------------------------------------------------------------------%

\section{Graph Theory -- Lý Thuyết Đồ Thị}

%------------------------------------------------------------------------------%

\section{Number Theory -- Số Học{\tt/}Lý Thuyết Số}

%------------------------------------------------------------------------------%


\section{Miscellaneous}

%------------------------------------------------------------------------------%

\printbibliography[heading=bibintoc]
	
\end{document}