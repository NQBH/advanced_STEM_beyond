\documentclass{article}
\usepackage[backend=biber,natbib=true,style=alphabetic,maxbibnames=50]{biblatex}
\addbibresource{/home/nqbh/reference/bib.bib}
\usepackage[utf8]{vietnam}
\usepackage{tocloft}
\renewcommand{\cftsecleader}{\cftdotfill{\cftdotsep}}
\usepackage[colorlinks=true,linkcolor=blue,urlcolor=red,citecolor=magenta]{hyperref}
\usepackage{amsmath,amssymb,amsthm,float,graphicx,mathtools,tikz}
\usetikzlibrary{angles,calc,intersections,matrix,patterns,quotes,shadings}
\allowdisplaybreaks
\newtheorem{assumption}{Assumption}
\newtheorem{baitoan}{}
\newtheorem{cauhoi}{Câu hỏi}
\newtheorem{conjecture}{Conjecture}
\newtheorem{corollary}{Corollary}
\newtheorem{dangtoan}{Dạng toán}
\newtheorem{definition}{Definition}
\newtheorem{dinhly}{Định lý}
\newtheorem{dinhnghia}{Định nghĩa}
\newtheorem{example}{Example}
\newtheorem{ghichu}{Ghi chú}
\newtheorem{hequa}{Hệ quả}
\newtheorem{hypothesis}{Hypothesis}
\newtheorem{lemma}{Lemma}
\newtheorem{luuy}{Lưu ý}
\newtheorem{nhanxet}{Nhận xét}
\newtheorem{notation}{Notation}
\newtheorem{note}{Note}
\newtheorem{principle}{Principle}
\newtheorem{problem}{Problem}
\newtheorem{proposition}{Proposition}
\newtheorem{question}{Question}
\newtheorem{remark}{Remark}
\newtheorem{theorem}{Theorem}
\newtheorem{vidu}{Ví dụ}
\usepackage[left=1cm,right=1cm,top=5mm,bottom=5mm,footskip=4mm]{geometry}
\def\labelitemii{$\circ$}
\DeclareRobustCommand{\divby}{%
	\mathrel{\vbox{\baselineskip.65ex\lineskiplimit0pt\hbox{.}\hbox{.}\hbox{.}}}%
}

\title{Probability {\it\&} Statistics -- Xác Suất {\it\&} Thống Kê}
\author{Nguyễn Quản Bá Hồng\footnote{A Scientist {\it\&} Creative Artist Wannabe. E-mail: {\tt nguyenquanbahong@gmail.com}. Bến Tre City, Việt Nam.}}
\date{\today}

\begin{document}
\maketitle
\begin{abstract}
	This text is a part of the series {\it Some Topics in Advanced STEM \& Beyond}:
	
	{\sc url}: \url{https://nqbh.github.io/advanced_STEM/}.
	
	Latest version:
	\begin{itemize}
		\item {\it Probability \& Statistics -- Xác Suất \& Thống Kê}.
		
		PDF: {\sc url}: \url{https://github.com/NQBH/advanced_STEM_beyond/blob/main/probability_statistics/NQBH_probability_statistics.pdf}.
		
		\TeX: {\sc url}: \url{https://github.com/NQBH/advanced_STEM_beyond/blob/main/probability_statistics/NQBH_probability_statistics.tex}.
	\end{itemize}
\end{abstract}
\tableofcontents

%------------------------------------------------------------------------------%

\section{Basic}

%------------------------------------------------------------------------------%

\section{Data Science (DS)}

%------------------------------------------------------------------------------%

\section{Deep Learning (DL)}
\textbf{\textsf{Resources -- Tài nguyên.}}
\begin{enumerate}
	\item \cite{LeCun_Bengio_Hinton2015}. {\sc Yann LeCun, Yoshua Bengio, Geoffrey Hinton}. {\it Deep Learning}.
\end{enumerate}

%------------------------------------------------------------------------------%

\section{Machine Learning (ML)}
\textbf{\textsf{Resources -- Tài nguyên.}}
\begin{enumerate}
	\item Machine Learning cơ bản:
	 \url{https://machinelearningcoban.com/}.
	\item \cite{Tiep_ML_co_ban}. {\sc Vũ Hữu Tiệp}. {\it Machine Learning Cơ Bản}.
	
	Mã nguồn cuốn ebook ``Machine Learning Cơ Bản'': \url{https://github.com/tiepvupsu/ebookMLCB}.
\end{enumerate}

\begin{definition}
	``\emph{Machine learning (ML)} is a field of study in \href{https://en.wikipedia.org/wiki/Artificial_intelligence}{AI} concerned with the development \& study of \href{https://en.wikipedia.org/wiki/Computational_statistics}{statistical algorithms} that can learn from \href{https://en.wikipedia.org/wiki/Data}{data} \& generalize to unseen data, \& thus perform \href{https://en.wikipedia.org/wiki/Task_(computing)}{tasks} without explicit \href{https://en.wikipedia.org/wiki/Machine_code}{instructions}. Quick progress in the fields of \href{https://en.wikipedia.org/wiki/Deep_learning}{deep learning}, beginning in 2010s, allowed neural networks to surpass many previous approaches in performance.'' -- \href{https://en.wikipedia.org/wiki/Machine_learning}{Wikipedia{\tt/}machine learning}
\end{definition}
``ML finds application in many fields, including \href{https://en.wikipedia.org/wiki/Natural_language_processing}{natural language processing}, \href{https://en.wikipedia.org/wiki/Computer_vision}{computer vision}, \href{https://en.wikipedia.org/wiki/Speech_recognition}{speech recognition}, \href{https://en.wikipedia.org/wiki/Email_filtering}{email filtering}, \href{https://en.wikipedia.org/wiki/Agriculture}{agriculture}, \& \href{https://en.wikipedia.org/wiki/Medicine}{medicine}. The application of ML to business problems is known as \href{https://en.wikipedia.org/wiki/Predictive_analytics}{predictive analysis}.

Statistics \& mathematical optimization{\tt/}mathematical programming methods comprise the foundations of machine learning. \href{https://en.wikipedia.org/wiki/Data_mining}{Data mining} is related field of study, focusing on \href{https://en.wikipedia.org/wiki/Exploratory_data_analysis}{exploratory data analysis} (EDA) via \href{https://en.wikipedia.org/wiki/Unsupervised_learning}{unsupervised learning}.

From a theoretical viewpoint, \href{https://en.wikipedia.org/wiki/Probably_approximately_correct_learning}{probably approximately correct (PAC) learning} provides a framework for describing machine learning.'' -- \href{https://en.wikipedia.org/wiki/Machine_learning}{Wikipedia{\tt/}machine learning}

{\bf Relationships of ML to AI.} As a scientific endeavor, machine learning grew out of the quest for AI. In the early days of AI as an \href{https://en.wikipedia.org/wiki/Discipline_(academia)}{academic discipline}, some researchers were interested in having machines learn from data. They attempted to approach the problem with various symbolic methods, as well as what were then termed ``\href{https://en.wikipedia.org/wiki/Artificial_neural_network}{neural networks}''; these were mostly \href{https://en.wikipedia.org/wiki/Perceptron}{perceptrons} \& other models e.g. \href{https://en.wikipedia.org/wiki/ADALINE}{ADALINE} that were later found to be reinventions of the \href{https://en.wikipedia.org/wiki/Generalized_linear_model}{generalized linear models} of statistics. \href{https://en.wikipedia.org/wiki/Probabilistic_reasoning}{Probabilistic reasoning} was also employed, especially in \href{https://en.wikipedia.org/wiki/Automated_medical_diagnosis}{automated medical diagnosis}. However, an increasing emphasis on the \href{https://en.wikipedia.org/wiki/Symbolic_AI}{logical, knowledge-based approach} caused a rift between AI \& machine learning. Probabilistic systems were plagued by theoretical \& practical problems of data acquisition \& representation. 

%------------------------------------------------------------------------------%

\section{Artificial Intelligence (AI)}
\textbf{\textsf{Resources -- Tài nguyên.}}
\begin{enumerate}
	\item \cite{Bac_Viet_AI}. {\sc Lê Hoài Bắc, Tô Hoài Việt}. {\it Cơ Sở Trí Tuệ Nhân Tạo}.
	\item \cite{Aoun_robot-proof}. {\sc Joseph E. Aoun}. {\it Robot-Proof: Higher Education in the Age of Artificial Intelligence}.
	\item \cite{Aoun_robot-proof_VN}. {\sc Joseph E. Aoun}. {\it Robot-Proof: Higher Education in the Age of Artificial Intelligence -- Chạy Đua Với Robot: Học Tập Thời Trí Tuệ Nhân Tạo}.
\end{enumerate}

%------------------------------------------------------------------------------%

\section{Miscellaneous}

%------------------------------------------------------------------------------%

\printbibliography[heading=bibintoc]
	
\end{document}