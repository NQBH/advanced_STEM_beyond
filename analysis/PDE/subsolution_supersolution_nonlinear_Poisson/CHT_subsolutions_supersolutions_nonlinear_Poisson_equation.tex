\documentclass[11pt,a4paper,center,notitlepage]{article}
\usepackage[backend=biber]{biblatex}

% Use Natbib reference style
%\usepackage{natbib}
 %\bibliographystyle{abbrvnat}

%\usepackage[backend=biber,style=authoryear,natbib=true]{biblatex} % Use the bibtex backend with the authoryear citation style (which resembles APA)

%\addbibresource{mybib.bib} % The filename of the bibliography
% For tabular
\usepackage{tabularx}
%\usepackage{arydshln,leftidx,mathtools}
%
%\setlength{\dashlinedash}{.4pt}
%\setlength{\dashlinegap}{.8pt}

\usepackage[autostyle=true]{csquotes} % Required to generate language-dependent quotes in the bibliography

\usepackage{algorithm}
\usepackage{algpseudocode}
\usepackage[utf8]{inputenc} 
%\usepackage[T1]{fontenc}
\usepackage[english]{babel} 
\usepackage{color}
\usepackage{textcomp,multicol,enumerate,amsmath,amssymb,amsthm,eufrak,latexsym,makeidx}
\newcommand{\vertiii}[1]{{\left\vert\kern-0.25ex\left\vert\kern-0.25ex\left\vert #1 
    \right\vert\kern-0.25ex\right\vert\kern-0.25ex\right\vert}}
% For insert figure
\usepackage{subfig}
\usepackage{graphicx,epstopdf}

%% Color Reference
\usepackage[usenames,dvipsnames,svgnames,table]{xcolor}
\usepackage[colorlinks=true,
            linkcolor=blue,
            urlcolor=gray,
            citecolor=magenta]{hyperref}
            
\allowdisplaybreaks
% No page break in Bibliography
\numberwithin{equation}{section}
\addto{\captionsenglish}{%
  \renewcommand{\bibname}{References}
}


\textwidth=16 cm
\textheight=22 cm
\topmargin= -1 cm
\oddsidemargin=0 cm
\evensidemargin=1 cm
\parindent=0.6 cm
\parskip=1.5 mm
\newtheorem{lemma}{Lemma}[section]
\newtheorem{corollary}{Corollary}[section]
\newtheorem{definition}{Definition}[section]
\newtheorem{prop}{Proposition}[section]
\newtheorem{theorem}{Theorem}[section]
\newtheorem{notation}{Notation}[section]
\newtheorem{remark}{Remark}[section]
\newtheorem{example}{Example}[section]
\newtheorem{ques}{Question}[section]
\newtheorem{sol}{Solution}[section]
\newtheorem{prob}{Problem}[section]
\renewcommand{\thenotation}{}
\renewcommand{\thesection}{\arabic{section}}
\renewcommand{\thesubsection}
{\arabic{section}.\arabic{subsection}}
\pagestyle{plain}

% Reference
\newcommand{\er}{\eqref}

%\newtheorem{theorem}{Theorem}[section]
%\newtheorem{corollary}[theorem]{Corollary}
%\newtheorem{lemma}[theorem]{Lemma}
%\newtheorem{proposition}[theorem]{Proposition}

%\theoremstyle{definition}
%\newtheorem{definition}[theorem]{Definition}
%\newtheorem{remark}{Remark}
%\newtheorem{properties}{Properties}
%\newtheorem*{notation}{Notation}
%\newtheorem{counter}{Counter-example}
%\newtheorem{open}{Open problem}
%\newtheorem{conjecture}{Conjecture}


%% FONT commands
\newcommand{\txt}[1]{\;\text{ #1 }\;}%% Used in math only
\newcommand{\tbf}{\textbf}%% Bold face. Usage: \tbf{...}
\newcommand{\tit}{\textit}%% Italic
\newcommand{\tsc}{\textsc}%% Small caps
\newcommand{\trm}{\textrm}
\newcommand{\mbf}{\mathbf}%% Math bold
\newcommand{\mrm}{\mathrm}%% Math Roman
\newcommand{\bsym}{\boldsymbol}%% Bold math symbol
%%Macros for changing font size in math.
\newcommand{\scs}{\scriptstyle}%% as in subscript
\newcommand{\sss}{\scriptscriptstyle}%% as in sub-subscript
\newcommand{\txts}{\textstyle}
\newcommand{\dsps}{\displaystyle}
%%Macros for changing font size in text.
\newcommand{\fnz}{\footnotesize}
\newcommand{\scz}{\scriptsize}
%%\tiny<\scz<\fsz<\small<\large<\Large<\huge<\Huge
%%%%%%%%%%%%
%%%%%%%%%%%%
%% EQUATION commands
\newcommand{\be}{\begin{equation}}
\newcommand{\bel}[1]{\begin{equation}\label{#1}}
\newcommand{\ee}{\end{equation}}
%% This macro does not work with amstex.
\newcommand{\eqnl}[2]{\begin{equation}\label{#1}{#2}\end{equation}}
%%use not advisable; confusing
\newcommand{\barr}{\begin{eqnarray}}
\newcommand{\earr}{\end{eqnarray}}
\newcommand{\bars}{\begin{eqnarray*}}
\newcommand{\ears}{\end{eqnarray*}}
\newcommand{\nnu}{\nonumber \\}
%%%%%%%%%%%%%%%
%% Unnumbered THEOREM env.
%% New env. to be used for unnumbered theorem, lemma etc.
%%(but with specified name)

\newtheorem{subn}{\name}
\renewcommand{\thesubn}{}
\newcommand{\bsn}[1]{\def\name{#1}\begin{subn}}
\newcommand{\esn}{\end{subn}}
%%%%%%%%%%%%%%
%% NUMBERED THEOREM env.
%% Environments: theorem, lemma, corollary defintion and
%%related commands,
%% designed to provide consecutive numbering of these forms.


\newtheorem{sub}{\name}[section]
\newcommand{\dn}[1]{\def\name{#1}}

%%used in conjuction with sub or subn.

\newcommand{\bs}{\begin{sub}}
\newcommand{\es}{\end{sub}}
\newcommand{\bsl}[1]{\begin{sub}\label{#1}}
	
%% the above must be preceeded by \dn (name definition),
%% however this is superceded by the list of commands bth etc. below.
%%%%%%%%%%%%
%% NUMBERED THEOREM env. (cont.)
%% List of commands derived from 'sub' env. for theorem, lemma etc.
%% designed to provide consecutive numbering of these forms.
\newcommand{\bth}[1]{\def\name{Theorem}\begin{sub}\label{t:#1}}
\newcommand{\blemma}[1]{\def\name{Lemma}\begin{sub}\label{l:#1}}
\newcommand{\bcor}[1]{\def\name{Corollary}\begin{sub}\label{c:#1}}	
\newcommand{\bdef}[1]{\def\name{Definition}\begin{sub}\label{d:#1}}
\newcommand{\bprop}[1]{\def\name{Proposition}\begin{sub}\label{p:#1}}	
%% ARRAY commands.%%%%%%%%%%%%%%%%%%%%%%%%%%%%%%%%%%
%% RERERENCE commands.
%% \newcommand{\R}[1]{(\ref{#1})}

\newcommand{\R}{\eqref}
\newcommand{\re}{\eqref}
\newcommand{\rth}[1]{Theorem~\ref{t:#1}}
\newcommand{\rlemma}[1]{Lemma~\ref{l:#1}}
\newcommand{\rcor}[1]{Corollary~\ref{c:#1}}
\newcommand{\rdef}[1]{Definition~\ref{d:#1}}
\newcommand{\rprop}[1]{Proposition~\ref{p:#1}}
%%%%%%%%%%%
\newcommand{\BA}{\begin{array}}
\newcommand{\EA}{\end{array}}
\newcommand{\BAN}{\renewcommand{\arraystretch}{1.2}
\setlength{\arraycolsep}{2pt}\begin{array}}
\newcommand{\BAV}[2]{\renewcommand{\arraystretch}{#1}
\setlength{\arraycolsep}{#2}\begin{array}}
%Note: The first variable gives the amount of stretching:
%(#1) x default.
%For instance #1=1.2 means a 20% stretching.
%The second variable should be
%written for instance in the form  4pt ; here the default is 5pt
%\newcommand{\EAN}{\end{array}\setlength{\arraycolsep}{5pt}}
\newcommand{\BSA}{\begin{subarray}}
\newcommand{\ESA}{\end{subarray}}	
%Note: These are used in subscripts as well as superscripts.
%They work essentially like 'array'.

\newcommand{\BAL}{\begin{aligned}}
	\newcommand{\EAL}{\end{aligned}}
\newcommand{\BALG}{\begin{alignat}}
	\newcommand{\EALG}{\end{alignat}}
%% the abbrev. does not work with latex2e
\newcommand{\BALGN}{\begin{alignat*}}
	\newcommand{\EALGN}{\end{alignat*}}
%% the abbrev. does not work with latex2e
%% The 'aligned' environment must be placed inside an 'equation' env.
%% in the same way as the array.
%% One could use also the 'align' env. or the 'alignat' env.
%% However in this case each line is numbered, unless '\notag' is used.
%% The 'alignat'
%% has a slightly different format (the number of columns must be %%specified in advance)
%% but it has the advantage that the distance between columns
%%is at our disposition.
%% (The default would be zero distance.) Using 'alignat*' we can have %%the advantages
%% of alignat plus the situation where separate lines are not numbered.
%% However in this case there is no numbering at all
%%(unless we provide a tag).
%%%%%%%%%%
%% PROOF, REMARK etc.
\newcommand{\note}[1]{\noindent\textit{#1.}\hspace{2mm}}
\newcommand{\Proof}{\note{Proof}}
%\newcommand{\qed}{\hspace{10mm}\hfill $\square$}
%\newcommand{\qed}{\\${}$ \hfill $\square$}
\newcommand{\Remark}{\note{Remark}}
%%%%%%%% Style command.
\newcommand{\modin}{$\,$\\[-4mm] \indent}
%% To be used after \section in order to start new line with \indent.
%%%%%%%%%%%%
%% MATHEMATICAL symbols
\newcommand{\forevery}{\quad \forall}
\newcommand{\set}[1]{\{#1\}}
\newcommand{\setdef}[2]{\{\,#1:\,#2\,\}}
\newcommand{\setm}[2]{\{\,#1\mid #2\,\}}
%% Arrows
\newcommand{\mt}{\mapsto}
\newcommand{\lra}{\longrightarrow}
\newcommand{\lla}{\longleftarrow}
\newcommand{\llra}{\longleftrightarrow}
\newcommand{\Lra}{\Longrightarrow}
\newcommand{\Lla}{\Longleftarrow}
\newcommand{\Llra}{\Longleftrightarrow}
\newcommand{\warrow}{\rightharpoonup}

%% Brackets, delimiters
\newcommand{\paran}[1]{\left (#1 \right )}
%% adjustable parantheses
\newcommand{\sqbr}[1]{\left [#1 \right ]}
%% adjustable square brackets
\newcommand{\curlybr}[1]{\left \{#1 \right \}}
%% adjustable curly brackets
\newcommand{\abs}[1]{\left |#1\right |}

%% adjustable vertical delimiters
\newcommand{\norm}[1]{\left \|#1\right \|}

%% adjustable norm
\newcommand{\paranb}[1]{\big (#1 \big )}

%% non-adjustable parantheses (big)
\newcommand{\lsqbrb}[1]{\big [#1 \big ]}

%% non-adjustable square brackets (big)
\newcommand{\lcurlybrb}[1]{\big \{#1 \big \}}

%% non-adjustable curly brackets(big)
\newcommand{\absb}[1]{\big |#1\big |}

%% non-adjustable vertical delimiters(big)
\newcommand{\normb}[1]{\big \|#1\big \|}

%% non-adjustable norm (big)
\newcommand{	\paranB}[1]{\Big (#1 \Big )}

%% non-adjustable parantheses (Big)
\newcommand{\absB}[1]{\Big |#1\Big |}

%% non-adjustable vertical delimiters(Big)
\newcommand{\normB}[1]{\Big \|#1\Big \|}%% non-adjustable norm (Big)
\newcommand{\produal}[1]{\langle #1 \rangle}%% the pairing of X' and X
%%%%%%%%%%%%%%%%%
%% Adjustable parantheses etc. in a different DEFINITION format.
%\def\adp(#1){\left (#1 \right )}%% adjustable parantheses
%\def\adsb(#1){\left [#1\right ]}%% adjustable square brackets
%\def\adcb(#1){\left \{#1\right \}}%% adjustable curly brackets
%\def\abs|#1|{\left |#1\right |}%% adjustable vertical delimiters
%%%%%%%%%%%%%%%%
%% More mathematical symbols
\newcommand{\thkl}{\rule[-.5mm]{.3mm}{3mm}}
\newcommand{\thknorm}[1]{\thkl #1 \thkl\,}
\newcommand{\trinorm}[1]{|\!|\!| #1 |\!|\!|\,}
\newcommand{\bang}[1]{\langle #1 \rangle}%% angle bracket
\def\angb<#1>{\langle #1 \rangle}%% angle bracket
%% The two last lines yield the same result.
%% The second is used as follows: \angb<a,b>
\newcommand{\vstrut}[1]{\rule{0mm}{#1}}
\newcommand{\rec}[1]{\frac{1}{#1}}
%% OPERATOR names.
%% OPERATOR names.
\newcommand{\opname}[1]{\mbox{\rm #1}\,}
\newcommand{\supp}{\opname{supp}}
\newcommand{\dist}{\opname{dist}}
\newcommand{\myfrac}[2]{{\displaystyle \frac{#1}{#2} }}
\newcommand{\myint}[2]{{\displaystyle \int_{#1}^{#2}}}
\newcommand{\mysum}[2]{{\displaystyle \sum_{#1}^{#2}}}
\newcommand {\dint}{{\displaystyle \myint\!\!\myint}}%%%%%%%%%%
%%%%%%% SPACE commands
\newcommand{\q}{\quad}
\newcommand{\qq}{\qquad}
\newcommand{\hsp}[1]{\hspace{#1mm}}
\newcommand{\vsp}[1]{\vspace{#1mm}}
%%%%%%%%%%%
%% ABREVIATIONS
\newcommand{\ity}{\infty}
\newcommand{\prt}{\partial}
\newcommand{\sms}{\setminus}
\newcommand{\ems}{\emptyset}
\newcommand{\ti}{\times}
\newcommand{\pr}{^\prime}
\newcommand{\ppr}{^{\prime\prime}}
\newcommand{\tl}{\tilde}
\newcommand{\sbs}{\subset}
\newcommand{\sbeq}{\subseteq}
\newcommand{\nind}{\noindent}
\newcommand{\ind}{\indent}
\newcommand{\ovl}{\overline}
\newcommand{\unl}{\underline}
\newcommand{\nin}{\not\in}
\newcommand{\pfrac}[2]{\genfrac{(}{)}{}{}{#1}{#2}}

%% frac with parantheses.
%%%%%%%%%%%
%%%%%%%%%%%%%

%%Macros for Greek letters.
\def\ga{\alpha}     \def\gb{\beta}       \def\gg{\gamma}
\def\gc{\chi}       \def\gd{\delta}      \def\gep{\epsilon}
\def\gth{\theta}                         \def\vge{\varepsilon}
\def\gf{\phi}       \def\vgf{\phi}    \def\gh{\eta}
\def\gi{\iota}      \def\gk{\kappa}      \def\gl{\lambda}
\def\gm{\mu}        \def\gn{\nu}         \def\gp{\pi}
\def\vgp{\varpi}    \def\gr{\gd}        \def\vgr{\varrho}
\def\gs{\sigma}     \def\vgs{\varsigma}  \def\gt{\tau}
\def\gu{\upsilon}   \def\gv{\vartheta}   \def\gw{\omega}
\def\gx{\xi}        \def\gy{\psi}        \def\gz{\zeta}
\def\Gg{\Gamma}     \def\Gd{\Delta}      \def\Gf{\Phi}
\def\Gth{\Theta}
\def\Gl{\Lambda}    \def\Gs{\Sigma}      \def\Gp{\Pi}
\def\Gw{\Omega}     \def\Gx{\Xi}         \def\Gy{\Psi}

%%Macros for calligraphic letters.
\def\CS{{\mathcal S}}   \def\CM{{\mathcal M}}   \def\CN{{\mathcal N}}
\def\CR{{\mathcal R}}   \def\CO{{\mathcal O}}   \def\CP{{\mathcal P}}
\def\CA{{\mathcal A}}   \def\CB{{\mathcal B}}   \def\CC{{\mathcal C}}
\def\CD{{\mathcal D}}   \def\CE{{\mathcal E}}   \def\CF{{\mathcal F}}
\def\CG{{\mathcal G}}   \def\CH{{\mathcal H}}   \def\CI{{\mathcal I}}
\def\CJ{{\mathcal J}}   \def\CK{{\mathcal K}}   \def\CL{{\mathcal L}}
\def\CT{{\mathcal T}}   \def\CU{{\mathcal U}}   \def\CV{{\mathcal V}}
\def\CZ{{\mathcal Z}}   \def\CX{{\mathcal X}}   \def\CY{{\mathcal Y}}
\def\CW{{\mathcal W}} \def\CQ{{\mathcal Q}}
%%%%%
%%Macros for 'blackboard' letters (See (27) for display.)
\def\BBA {\mathbb A}   \def\BBb {\mathbb B}    \def\BBC {\mathbb C}
\def\BBD {\mathbb D}   \def\BBE {\mathbb E}    \def\BBF {\mathbb F}
\def\BBG {\mathbb G}   \def\BBH {\mathbb H}    \def\BBI {\mathbb I}
\def\BBJ {\mathbb J}   \def\BBK {\mathbb K}    \def\BBL {\mathbb L}
\def\BBM {\mathbb M}   \def\BBN {\mathbb N}    \def\BBO {\mathbb O}
\def\BBP {\mathbb P}   \def\BBR {\mathbb R}    \def\BBS {\mathbb S}
\def\BBT {\mathbb T}   \def\BBU {\mathbb U}    \def\BBV {\mathbb V}
\def\BBW {\mathbb W}   \def\BBX {\mathbb X}    \def\BBY {\mathbb Y}
\def\BBZ {\mathbb Z}

%%Macros for Ghotic (Fraktur) letters.
\def\GTA {\mathfrak A}   \def\GTB {\mathfrak B}    \def\GTC {\mathfrak C}
\def\GTD {\mathfrak D}   \def\GTE {\mathfrak E}    \def\GTF {\mathfrak F}
\def\GTG {\mathfrak G}   \def\GTH {\mathfrak H}    \def\GTI {\mathfrak I}
\def\GTJ {\mathfrak J}   \def\GTK {\mathfrak K}    \def\GTL {\mathfrak L}
\def\GTM {\mathfrak M}   \def\GTN {\mathfrak N}    \def\GTO {\mathfrak O}
\def\GTP {\mathfrak P}   \def\GTR {\mathfrak R}    \def\GTS {\mathfrak S}
\def\GTT {\mathfrak T}   \def\GTU {\mathfrak U}    \def\GTV {\mathfrak V}
\def\GTW {\mathfrak W}   \def\GTX {\mathfrak X}    \def\GTY {\mathfrak Y}
\def\GTZ {\mathfrak Z}   \def\GTQ {\mathfrak Q}
\def\sign{\mathrm{sign\,}}
\def\bdw{\prt\Gw\xspace}
\def\nabu{|\nabla u|}
\def\tr{\mathrm{tr\,}}
\def\gap{{\ga_+}}
\def\gan{{\ga_-}}

\def\N{\mathbb{N}}
\def\Z{\mathbb{Z}}
\def\Q{\mathbb{Q}}
\def\R{\mathbb{R}}


\def\Proof.{{\bf{Proof. }}}
\def\End{\hspace{1cm} $\Box$\\}


\renewcommand{\baselinestretch}{1.1}

\let\e=\varepsilon
\let\vp=\phi
\let\t=\tilde
\let\ol=\overline
\let\ul=\underline
\let\.=\cdot
\let\0=\emptyset
\let\mc=\mathcal
\def\ex{\exists\;}
\def\fa{\forall\;}
\def\se{\ \Leftarrow\ }
\def\solose{\ \Rightarrow\ }
\def\sse{\ \Leftrightarrow\ }
\def\meno{\,\backslash\,}
\def\pp{,\dots,}
\def\D{\mc{D}}
\def\O{\Omega}


\def\loc{\text{\rm loc}}
\def\diam{\text{\rm diam}}
\def\dist{\text{\rm dist}}
\def\dv{\text{\rm div}}
\def\sign{\text{\rm sign}}
\def\supp{\text{\rm supp}}
\def\tr{\text{\rm Tr}}
\def\vec{\text{\rm vec}}
\def\inter{\text{\rm int\,}}
\def\norma#1{\|#1\|_\infty}

\newcommand{\esssup}{\mathop{\rm ess{\,}sup}}
\newcommand{\essinf}{\mathop{\rm ess{\,}inf}}
\newcommand{\su}[2]{\genfrac{}{}{0pt}{}{#1}{#2}}

\def\eq#1{{\rm(\ref{eq:#1})}}
\def\thm#1{Theorem \ref{thm:#1}}
\def\seq#1{(#1_n)_{n\in\N}}
\def\limn{\lim_{n\to\infty}}


\def\PP{\mc{P}}
\def\pe{principal eigenvalue}
\def\MP{maximum principle}
\def\SMP{strong maximum principle}
\def\l{\lambda_1}

\def\bq{\begin{equation}}
\def\eq{\end{equation}}

\def\l{\label}

\newenvironment{formula}[1]{\begin{equation}\label{eq:#1}}	{\end{equation}\noindent}

\title{Method of Subsolutions and Supersolutions for a Nonlinear Poisson Equation}
\author{\textsc{Nguyen} Ngoc Minh Chau \and \textsc{Vu} Anh Tuan \and \textsc{Nguyen} Quan Ba Hong\footnote{Master 2 Students at UFR math\'ematiques, Universit\'e de Rennes 1, Beaulieu - B\^atiment 22 et 23, 263 avenue du G\'en\'eral Leclerc, 35042 Rennes CEDEX, France.\newline
E-mail: \texttt{nguyenquanbahong@gmail.com} \newline
Blog: \texttt{\url{www.nguyenquanbahong.com}} \newline 
Copyright \copyright\ 2016-2018 by Nguyen Quan Ba Hong. This document may be copied freely for the purposes of education and non-commercial research. Visit my site to get more.}}
\begin{document}
\maketitle
\begin{abstract}
In this context, we are interested in the method of subsolutions \& supersolutions for a non-linear Poisson equation, which is presented in \cite{1}, p. 543. This material is used for our representation in the class \textit{Sobolev spaces and elliptic equations} which is taught by Prof. Nicoletta Tchou in Universit\'e de Rennes 1, 2018.
\end{abstract}
%\textbf{Mathematics Subject Classification (2010):} 
%
%\noindent
%\textbf{Keywords:} \emph{}
\newpage
\tableofcontents
\newpage
\section{Main Results}
Let $U$ be a bounded open subset of $\mathbb{R}^N$ with smooth boundary.

In this context, we focus on the maximum principle, which is a basic property of elliptic PDE, and demonstrate how various resulting comparison arguments can be used to solve certain semilinear problems.

The idea is to exploit \textit{order properties} for solutions. More precisely, we will show that if we can find a subsolution $\underline u$ \& a supersolution $\overline u$ of a particular boundary-value problem and if furthermore $\underline{u} \le \overline{u}$, then there in fact exists a solution satisfying $\underline u  \le u \le \overline u $. 

We will investigate this boundary-value problem for the nonlinear Poisson equation
\begin{equation}
\label{1.1}
\left\{ \begin{split}
 - \Delta u &= f\left( u \right), & \mbox{ in } U,\\
u &= 0, & \mbox{ on } \partial U,
\end{split} \right.
\end{equation}
where $f:\mathbb{R}\to \mathbb{R}$ is smooth, with 
\begin{align}
\label{1.2}
\left| {f'\left( z \right)} \right| \le C_f, \hspace{2mm}\forall z \in \mathbb{R},
\end{align}
for some constant $C_f$.
\begin{definition}
\begin{itemize}
\item[(i)] We say that $\overline{u} \in H^1\left(U\right)$ is a \emph{weak supersolution} of problem \eqref{1.1} if 
\begin{align}
\label{1.3}
\int_U {D\overline u  \cdot Dvdx}  \ge \int_U {f\left( {\overline u } \right)vdx} ,\hspace{2mm}\forall v \in H_0^1\left( U \right),\hspace{1mm}v \ge 0 \mbox{ a.e.}
\end{align}
\item[(ii)] Similarly, $\underline{u} \in H^1\left(U\right)$ is a \emph{weak subsolution} of \eqref{1.1} provided
\begin{align}
\label{1.4}
\int_U {D\underline u  \cdot Dvdx}  \le \int_U {f\left( {\underline u } \right)vdx} ,\hspace{2mm}\forall v \in H_0^1\left( U \right),\hspace{1mm} v \ge 0 \mbox{ a.e.}
\end{align}
\item[(iii)] We say $u\in H_0^1\left(U\right)$ is a \emph{weak solution} of \eqref{1.1} if
\begin{align}
\int_U {Du \cdot Dvdx}  = \int_U {f\left( u \right)vdx} ,\hspace{2mm} \forall v \in H_0^1\left( U \right).
\end{align}
\end{itemize}
\end{definition}

\begin{remark}\label{remark1.1}
If $\overline{u}, \underline{u} \in C^2\left(U\right)$, then from \eqref{1.3} \& \eqref{1.4} it follows that
\begin{align}
\label{1.6}
 - \Delta \overline u  \ge f\left( {\overline u } \right),\hspace{1mm} - \Delta \underline u  \le f\left( {\underline u } \right), \mbox{ in } U.
\end{align}
\end{remark}

\begin{proof}[Proof of Remark \ref{remark1.1}]
It suffices to prove the first inequality in \eqref{1.6}, the second one is treated similarly. Since $\overline{u}$ is a weak supersolution of \eqref{1.1}, the integral inequality \eqref{1.3} holds. Applying Green's formula to the LHS of \eqref{1.3} yields
\begin{align}
\int_U {\left( { - \Delta \overline u  - f\left( {\overline u } \right)} \right)vdx}  \ge 0,\hspace{2mm}\forall v \in H_0^1\left( U \right),\hspace{1mm}v \ge 0 \mbox{ a.e.},
\end{align}
in particular,
\begin{align}
\label{1.8*}
\int_U {\left( { - \Delta \overline u  - f\left( {\overline u } \right)} \right)vdx}  \ge 0,\hspace{2mm}\forall v \in C_c^\infty \left( U \right),\hspace{1mm}v \ge 0.
\end{align}
We suppose for the contrary that there exists a point $x_0 \in U$ such that $ - \Delta \overline u \left( {{x_0}} \right) < f\left( {\overline u \left( {{x_0}} \right)} \right)$. Since $\overline{u} \in C^2 \left(U\right)$ and $f$ is smooth, the last inequality implies that there exists a ball $B\left(x_0,r\right) \in U$ such that 
\begin{align}
 - \Delta \overline u  < f\left( {\overline u } \right),\mbox{ in } B\left( {{x_0},r} \right).
\end{align}
Then plugging an arbitrary function $v\in C_c^\infty \left(U\right)$, $v\ge 0$ satisfying $v > 0$ in $B\left(x_0,\frac{r}{2}\right)$ into \eqref{1.8*} yields a contradiction. Therefore, the desired result follows.
\end{proof}
\begin{theorem}[Existence of a solution between sub- and supersolutions]\label{theorem1.1}
Assume there exists a weak supersolution $\overline{u}$ and a weak subsolution $\underline{u}$ of \eqref{1.1} satisfying
\begin{align}
\label{1.8}
\underline u  \le 0,\hspace{1mm}\overline u  \ge 0 \mbox{ on } \partial U \mbox{ in the trace sense}, \hspace{1mm}\underline u  \le \overline u \mbox{ a.e. in } U.
\end{align}
Then there exists a weak solution $u$ of \eqref{1.1}, such that 
\begin{align}
\underline{u} \le u\le \overline{u} \mbox{ a.e. in } U.
\end{align}
\end{theorem}

\begin{proof} 1. Fix a number $\lambda >0$ so large that
\begin{align}
\label{1.10}
\mbox{the mapping } z \mapsto f\left(z\right) + \lambda z \mbox{ is nondecreasing},
\end{align}
this is possible as a consequence of hypothesis \eqref{1.2}\footnote{Indeed, consider the (smooth) mappings $h_\lambda: \mathbb{R}\to \mathbb{R}$ defined by ${h_\lambda }\left( z \right) := f\left( z \right) + \lambda z$, $\forall z \in \mathbb{R}$. Its first derivative is given by ${h_\lambda }'\left( z \right) = f'\left( z \right) + \lambda  \ge \lambda  - C_f$, $\forall z \in \mathbb{R}$. Thus, if $\lambda \ge C_f$, the mapping $h_\lambda$ is nondecreasing.}.

Now write $u_0 = \underline{u}$, and then given $u_k$, $k=0,1,2,\ldots$, inductively define $u_{k+1} \in H_0^1\left(U\right)$ to be the unique weak solution of the linear boundary-value problem\footnote{Combining the fact that $u_0=\underline{u} \in H^1\left(U\right)$ with Lemma \ref{lemma1.1} yields $f\left(u_0\right) +\lambda u_0 \in H^1\left(U\right)$. Hence, there exists a unique weak solution, say $u_1$, of $\left({\rm P}_1\right)$ such that $u_1\in H_0^1\left(U\right)$. Inductively, for $\left({\rm P}_{k+1}\right)$, combining the fact that $u_k \in H_0^1\left(U\right)$ and Lemma \ref{lemma1.1} gives us $f\left(u_k\right)+\lambda u_k \in H^1\left(U\right)$. Then there exists a unique weak solution $u_{k+1}\in H_0^1\left(U\right)$ of $\left({\rm P}_{k+1}\right)$.}
\begin{equation}
\label{1.11}
\left({\rm P}_{k+1} \right) \hspace{2mm} \left\{ \begin{split}
 - \Delta {u_{k + 1}} + \lambda {u_{k + 1}} &= f\left( {{u_k}} \right) + \lambda {u_k}, & \mbox{ in } U,\\
{u_{k + 1}} &= 0, & \mbox{ on } \partial U.
\end{split} \right.
\end{equation}
2. We claim
\begin{align}
\label{1.12}
\underline{u} = u_0 \le u_1 \le \ldots \le u_k\le \ldots \mbox{ a.e. in } U.
\end{align}
To confirm this, first note from \eqref{1.11} for $k=0$, i.e., $\left({\rm P}_1\right)$, that 
\begin{align}
\label{1.13}
\int_U {\left( {D{u_1} \cdot Dv + \lambda {u_1}v} \right)dx}  = \int_U {\left( {f\left( {{u_0}} \right) + \lambda {u_0}} \right)vdx} ,\hspace{2mm}\forall v \in H_0^1\left( U \right).
\end{align}
Subtracting \eqref{1.13} from \eqref{1.4}, recall $u_0=\underline{u}$, yields
\begin{align}
\int_U {D\left( {{u_0} - {u_1}} \right) \cdot Dvdx}  \le \int_U {\lambda \left( {{u_1} - {u_0}} \right)vdx} ,\hspace{2mm}\forall v \in H_0^1\left( U \right), \hspace{1mm} v\ge 0 \mbox{ a.e.}.
\end{align}
Set
\begin{align}
v: = {\left( {{u_0} - {u_1}} \right)^ + } \in H_0^1\left( U \right),\hspace{2mm} v \ge 0 \mbox{ a.e.},
\end{align}
we find
\begin{align}
\int_U {\left[ {D\left( {{u_0} - {u_1}} \right) \cdot D{{\left( {{u_0} - {u_1}} \right)}^ + } + \lambda \left( {{u_0} - {u_1}} \right){{\left( {{u_0} - {u_1}} \right)}^ + }} \right]dx}  \le 0.
\end{align}
But, by Lemma \ref{lemma1.2},
\begin{equation}
D{\left( {{u_0} - {u_1}} \right)^ + } = \left\{ \begin{split}
&D\left( {{u_0} - {u_1}} \right) & \mbox{ a.e. on } \left\{ {{u_0} \ge {u_1}} \right\},\\
&0 & \mbox{ a.e. on } \left\{ {{u_0} \le {u_1}} \right\}.
\end{split} \right.
\end{equation}
Consequently,
\begin{align}
\int_{\left\{ {{u_0} \ge {u_1}} \right\}} {\left( {{{\left| {D\left( {{u_0} - {u_1}} \right)} \right|}^2} + \lambda {{\left( {{u_0} - {u_1}} \right)}^2}} \right)dx}  \le 0,
\end{align}
so that $u_0 \le u_1$ a.e. in $U$. 

Now assume inductively that 
\begin{align}
\label{1.19}
u_{k-1} \le u_k \mbox{ a.e. in }U.
\end{align}
From \eqref{1.11}, we find, for $\left({\rm P}_{k+1}\right)$ and $\left({\rm P}_k\right)$, respectively,
\begin{align}
\label{1.20}
\int_U {\left( {D{u_{k + 1}} \cdot Dv + \lambda {u_{k + 1}}v} \right)dx}  &= \int_U {\left( {f\left( {{u_k}} \right) + \lambda {u_k}} \right)vdx} ,\\
\int_U {\left( {D{u_k} \cdot Dv + \lambda {u_k}v} \right)dx}  &= \int_U {\left( {f\left( {{u_{k - 1}}} \right) + \lambda {u_{k - 1}}} \right)vdx} ,
\end{align}
for all $v \in H_0^1\left( U \right)$. 

Subtract the last two equalities, we obtain
\begin{align}
\int_U {\left[ {D\left( {{u_k} - {u_{k + 1}}} \right) \cdot Dv + \lambda \left( {{u_k} - {u_{k + 1}}} \right)v} \right]dx}  = \int_U {\left[ {f\left( {{u_{k - 1}}} \right) - f\left( {{u_k}} \right) + \lambda \left( {{u_{k - 1}} - {u_k}} \right)} \right]vdx} ,
\end{align}
for all $v \in H_0^1\left( U \right)$. Then set $v: = {\left( {{u_k} - {u_{k + 1}}} \right)^ + } \in H_0^1\left( U \right)$, $v \ge 0 \mbox{ a.e.}$, we find
\begin{align}
\label{1.23}
& \int_U {\left[ {D\left( {{u_k} - {u_{k + 1}}} \right) \cdot D{{\left( {{u_k} - {u_{k + 1}}} \right)}^ + } + \lambda \left( {{u_k} - {u_{k + 1}}} \right){{\left( {{u_k} - {u_{k + 1}}} \right)}^ + }} \right]dx} \nonumber \\
 =&\ \int_U {\left[ {f\left( {{u_{k - 1}}} \right) - f\left( {{u_k}} \right) + \lambda \left( {{u_{k - 1}} - {u_k}} \right)} \right]{{\left( {{u_k} - {u_{k + 1}}} \right)}^ + }dx} .
\end{align}
Lemma \ref{lemma1.2} gives us
\begin{equation}
D{\left( {{u_k} - {u_{k + 1}}} \right)^ + } = \left\{ \begin{split}
& D\left( {{u_k} - {u_{k + 1}}} \right) & \mbox{ a.e. on } \left\{ {{u_k} \ge {u_{k + 1}}} \right\},\\
& 0 & \mbox{ a.e. on } \left\{ {{u_k} \le {u_{k + 1}}} \right\}.
\end{split} \right.
\end{equation}
Thus, \eqref{1.23} becomes
\begin{align}
& \int_{\left\{ {{u_k} \ge {u_{k + 1}}} \right\}} {\left( {{{\left| {D\left( {{u_k} - {u_{k + 1}}} \right)} \right|}^2} + \lambda {{\left( {{u_k} - {u_{k + 1}}} \right)}^2}} \right)dx} \\
 =&\ \int_U {\left( {f\left( {{u_{k - 1}}} \right) + \lambda {u_{k - 1}} - f\left( {{u_k}} \right) - \lambda {u_k}} \right){{\left( {{u_k} - {u_{k + 1}}} \right)}^ + }dx} \\
 =&\ \int_U {\left( {{h_\lambda }\left( {{u_{k - 1}}} \right) - {h_\lambda }\left( {{u_k}} \right)} \right){{\left( {{u_k} - {u_{k + 1}}} \right)}^ + }dx}  \le 0,
\end{align}
the last inequality holding in view of \eqref{1.19} and \eqref{1.10}. Therefore, $u_k\le u_{k+1}$ a.e. in $U$, as asserted.

3. Next we show 
\begin{align}
\label{1.28}
u_k \le \overline{u} \mbox{ a.e. in } U, \hspace{2mm} \forall k\in \mathbb{N}.
\end{align}
Statement \eqref{1.28} is valid for $k=0$ by hypothesis \eqref{1.8}. Assume now for induction that for some $k\in \mathbb{N}$,
\begin{align}
\label{1.29}
u_k \le \overline{u} \mbox{ a.e. in } U.
\end{align}
Then subtracting \eqref{1.3} from \eqref{1.20}, we obtain
\begin{align*}
\int_U {\left( {D\left( {{u_{k + 1}} - \overline u } \right) \cdot Dv + \lambda {u_{k + 1}}v} \right)dx}  \le \int_U {\left( {f\left( {{u_k}} \right) + \lambda {u_k} - f\left( {\overline u } \right)} \right)vdx} ,\hspace{2mm}\forall v \in H_0^1\left( U \right),\hspace{1mm}v \ge 0 \mbox{ a.e.},
\end{align*}
and thus
\begin{align}
& \int_U {\left( {D\left( {{u_{k + 1}} - \overline u } \right) \cdot Dv + \lambda \left( {{u_{k + 1}} - \overline u } \right)v} \right)dx}  \\
\le &\ \int_U {\left( {f\left( {{u_k}} \right) + \lambda {u_k} - f\left( {\overline u } \right) - \lambda \overline u } \right)vdx} \\
 =&\ \int_U {\left( {{h_\lambda }\left( {{u_k}} \right) - {h_\lambda }\left( {\overline u } \right)} \right)vdx}\le 0 ,\hspace{2mm} \forall v \in H_0^1\left( U \right) ,\hspace{1mm}v \ge 0 \mbox{ a.e.},
\end{align}
where the last inequality is deduced from \eqref{1.29}, \eqref{1.10}, and the positivity of $v$.

Taking $v:= \left(u_{k+1} -\overline{u}\right) ^+$, we find
\begin{align}
\int_{\left\{ {{u_{k + 1}} \ge \overline u } \right\}} {\left( {{{\left| {D\left( {{u_{k + 1}} - \overline u } \right)} \right|}^2} + \lambda {{\left( {{u_{k + 1}} - \overline u } \right)}^2}} \right)dx}  \le 0.
\end{align}
Thus, $u_{k+1}\le \overline{u}$ a.e. in $U$. By the principle of mathematical induction, \eqref{1.28} holds. 

4. In light of \eqref{1.12} and \eqref{1.28}, we have\footnote{As a consequence, $\left| {{u_k}} \right| \le \max \left\{ {\left| {\underline u } \right|,\left| {\overline u } \right|} \right\}$, $\forall k \in \mathbb{N}$, and thus
\begin{align}
{\left\| {{u_k}} \right\|_{{L^2}\left( U \right)}} \le {\left\| {\max \left\{ {\left| {\underline u } \right|,\left| {\overline u } \right|} \right\}} \right\|_{{L^2}\left( U \right)}},\hspace{2mm}\forall k \in \mathbb{N}.
\end{align}}
\begin{align}
\label{1.34}
\underline{u}\le \ldots \le u_k\le u_{k+1} \le \ldots \overline{u} \mbox{ a.e. in } U.
\end{align}
Therefore
\begin{align}
u\left( x \right): = \mathop {\lim }\limits_{k \to \infty } {u_k}\left( x \right)
\end{align}
exists for a.e. $x \in U$. Furthermore, we have
\begin{align}
u_k \to  u \mbox{ in } L^2\left(U\right),
\end{align}
as guaranteed by the Dominated Convergence Theorem and \eqref{1.34}.

Finally, we have
\begin{align}
{\left\| {f\left( {{u_k}} \right)} \right\|_{{L^2}\left( U \right)}} \le {\left\| {f\left( {{u_k}} \right) - f\left( 0 \right)} \right\|_{{L^2}\left( U \right)}} + {\left\| {f\left( 0 \right)} \right\|_{{L^2}\left( U \right)}}  \le {C_f}{\left\| {{u_k}} \right\|_{{L^2}\left( U \right)}} + \left| {f\left( 0 \right)} \right|\mbox{vol}{\left( U \right)^{\frac{1}{2}}}.
\end{align}
Since we have ${\left\| {f\left( {{u_k}} \right)} \right\|_{{L^2}\left( U \right)}} \le C\left( {{{\left\| {{u_k}} \right\|}_{{L^2}\left( U \right)}} + 1} \right)$ where the constant $C$ is given by 
\begin{align}
C := \max \left\{ {{C_f},\left| {f\left( 0 \right)} \right|\mbox{vol}{{\left( U \right)}^{\frac{1}{2}}}} \right\}, 
\end{align}
we deduce from \eqref{1.11} that ${\sup _k}{\left\| {{u_k}} \right\|_{H_0^1\left( U \right)}} < \infty $. Indeed, substituting $v = u_{k+1}\in H_0^1\left(U\right)$ into \eqref{1.20} yields
\begin{align}
\int_U {\left( {{{\left| {D{u_{k + 1}}} \right|}^2} + \lambda {{\left| {{u_{k + 1}}} \right|}^2}} \right)dx}  = \int_U {\left( {f\left( {{u_k}} \right) + \lambda {u_k}} \right){u_{k + 1}}dx} ,\hspace{2mm}\forall k \in \mathbb{N}.
\end{align}
Thus,
\begin{align}
\min \left\{ {1,\lambda } \right\}{\left\| {{u_{k + 1}}} \right\|_{H_0^1\left( U \right)}} & \le \int_U {\left( {{{\left| {D{u_{k + 1}}} \right|}^2} + \lambda {{\left| {{u_{k + 1}}} \right|}^2}} \right)dx} \\
& = \int_U {\left( {f\left( {{u_k}} \right) + \lambda {u_k}} \right){u_{k + 1}}dx} \\
& \le {\left\| {f\left( {{u_k}} \right)} \right\|_{{L^2}\left( U \right)}}{\left\| {{u_{k + 1}}} \right\|_{{L^2}\left( U \right)}} + \lambda {\left\| {{u_k}} \right\|_{{L^2}\left( U \right)}}{\left\| {{u_{k + 1}}} \right\|_{{L^2}\left( U \right)}}\\
& \le C\left( {{{\left\| {{u_k}} \right\|}_{{L^2}\left( U \right)}} + 1} \right){\left\| {{u_{k + 1}}} \right\|_{{L^2}\left( U \right)}} + \lambda {\left\| {{u_k}} \right\|_{{L^2}\left( U \right)}}{\left\| {{u_{k + 1}}} \right\|_{{L^2}\left( U \right)}}\\
& \le \left( {\lambda  + C} \right)\left\| {\max \left\{ {\left| {\underline u } \right|,\left| {\overline u } \right|} \right\}} \right\|_{{L^2}\left( U \right)}^2 + C{\left\| {\max \left\{ {\left| {\underline u } \right|,\left| {\overline u } \right|} \right\}} \right\|_{{L^2}\left( U \right)}},
\end{align}
for all $k\in \mathbb{N}$. Since this bound is independent of $k$, we deduce that ${\sup _{k\in \mathbb{N}}}{\left\| {{u_k}} \right\|_{H_0^1\left( U \right)}} < \infty $. Hence there is a subsequence $\left\{ {{u_{{k_j}}}} \right\}_{j = 1}^\infty $ which converges weakly in $H_0^1\left(U\right)$ to $u\in H_0^1\left(U\right)$. 

5. We at last verify that $u$ is a weak solution of problem \eqref{1.1}. For this, fix $v\in H_0^1\left(U\right)$. Then from \eqref{1.11}  we find
\begin{align}
\int_U {\left( {D{u_{{k_{j + 1}}}} \cdot Dv + \lambda {u_{{k_{j + 1}}}}v} \right)dx}  = \int_U {\left( {f\left( {{u_{{k_j}}}} \right) + \lambda {u_{{k_j}}}} \right)vdx} .
\end{align}
Let $j\to \infty$: 
\begin{align}
\int_U {\left( {Du \cdot Dv + \lambda uv} \right)dx}  = \int_U {\left( {f\left( u \right) + \lambda u} \right)vdx} .
\end{align}
Canceling the term involving $\lambda$, we at last confirm that
\begin{align}
\int_U {Du \cdot Dvdx}  = \int_U {f\left( u \right)vdx} ,
\end{align}
as desired.
\end{proof}
This proof illustrates the use of integration by parts, rather than the maximum principle, to establish comparisons between sub- and supersolutions. 



%We now consider the second approach.
%*****
\section{Problems}
\begin{prob}[Exercise 6, \cite{1}, p. 574]
Assume $f:\mathbb{R}\to \mathbb{R}$ is Lipschitz continuous, bounded, with $f\left(0\right) =0$ and $f'\left(0\right) \ge \lambda _1$, $\lambda _1$ denoting the principal eigenvalue for $-\Delta$ on $H_0^1\left(U\right)$. Use the method of sub- and supersolutions to show there exists a weak solution $u$ of 
\begin{equation}
\left\{ \begin{split}
- \Delta u &= f\left( u \right),& \mbox{ in } U,\\
u &= 0,& \mbox{ on } \partial U,\\
u &> 0,& \mbox{ in } U.
\end{split} \right.
\end{equation}
\end{prob}

\begin{prob}[Exercise 7, \cite{1}, p. 579]
Assume that $\underline{u}$, $\overline{u}$ are smooth sub- and supersolutions of the boundary-value problem \eqref{1.11}. Use the maximum principle to verify directly
\begin{align}
\underline{u} =u_0 \le u_1 \le \ldots \le u_k \le \ldots \overline{u} ,
\end{align}
where the $\left\{ {{u_k}} \right\}_{k = 0}^\infty $ are defined in the proof of Theorem \ref{theorem1.1}.
\end{prob}

\begin{proof}[Solution]
First of all, we need to assume in addition that $U$ is an open set of class $C^2$\footnote{This implies the smooth sub- and supersolutions also belongs to $C\left(\overline{U}\right)$, i.e., $\underline u ,\overline u  \in {C^2}\left( U \right) \cap C\left( {\overline U } \right)$. Thus, we can apply weak maximum principle as in this proof.} and $f\in H^m\left(U\right)$ with $m>\frac{N}{2}$. Then, using Theorem \ref{theorem3.5}, the weak solution $u_k$ of $\left({\rm P}_k\right)$ in Step 1 in the above proof then satisfies $u_k\in C^2\left(\overline{U}\right)$, for all $k\in \mathbb{Z}^+$. Now subtracting the PDE in \eqref{1.11} w.r.t. $\left({\rm P}_0\right)$ to \eqref{1.6} yields
\begin{align}
\label{2.3}
\Delta \left( {{u_1} - {u_0}} \right) \le \lambda \left( {{u_1} - {u_0}} \right),\mbox{ in } U.
\end{align}
Since $u_0=\underline{u} \le 0$ on $\partial U$ and $u_1 = 0$ on $\partial U$, applying Theorem 2, ii) (Weak maximum principle for $c\ge 0$), \cite{1}, p. 346 gives us
\begin{align}
\mathop {\min }\limits_{\overline U } \left( {{u_1} - {u_0}} \right) \ge  - \mathop {\max }\limits_{\partial U} {\left( {{u_1} - {u_0}} \right)^ - } = 0,
\end{align}
i.e., $u_1\ge u_0$ in $U$. 

At the $k^{\rm th}$ step, subtracting the PDEs in $\left({\rm P}_{k+1}\right)$ and $\left({\rm P}_k\right)$ yields
\begin{align}
\Delta \left( {{u_{k + 1}} - {u_k}} \right) + \lambda \left( {{u_k} - {u_{k + 1}}} \right) &= f\left( {{u_{k - 1}}} \right) - f\left( {{u_k}} \right) + \lambda \left( {{u_{k - 1}} - {u_k}} \right)\\
 &= {h_\lambda }\left( {{u_{k - 1}}} \right) - {h_\lambda }\left( {{u_k}} \right) \le 0,
\end{align}
since $h_\lambda$ is nondecreasing and $u_{k-1}\le u_k$ obtained in the previous step. Note that $u_{k+1}-u_k =0$ on $\partial U$, applying weak maximum principle similarly yields $u_{k+1}\le u_k$ in $U$. The Step 3 in the above proof is handled by the maximum principle similarly.
\end{proof}
Instead of applying directly the weak maximum principle, the following alternative proof uses a property of subharmonicity.

\begin{proof}[Alternative proof] 
It suffices to prove $u_0\le u_1$, the rest of Step 2 and 3 of the first proof is handled similarly. We assume that $U$ is an open \textit{connected} set of class $C^2$ for simplicity. After obtaining $u_k \in C^2\left(\overline{U}\right)$ for all $k\in \mathbb{Z}^+$, we assume for the contrary that $M:={\max _U}\left( {{u_0} - {u_1}} \right) > 0$. Define 
\begin{align}
F: = \left\{ {x \in U;{u_0} - {u_1} = M} \right\} ,
\end{align}
the set $F$ is nonempty and relatively closed in $U$. Take $x_0 \in F$, i.e., $u_0\left(x_0\right)-u_1\left(x_0\right) =M >0$, there exists a ball $B\left(x_0,r\right) \subset U$ such that $u_0-u_1>0$ due to the smoothness of $u_0$ and $u_1$. Then \eqref{2.3} gives $\Delta \left( {{u_0} - {u_1}} \right) \ge \lambda \left( {{u_0} - {u_1}} \right) > 0$ in $B\left(x_0,r\right)$. Hence, ${\left. {\left({u_0} - {u_1}\right)} \right|_{B\left( {{x_0},r} \right)}}$ is a subharmonic function which attains a global maximum at $x_0 \in B\left(x_0,r\right)$. The maximum principle for subharmonic function implies that $u_0 -u_1 =M$ in $B\left(x_0,r\right)$. In particular, this implies that $F$ is open, thus $F =U$. This contradicts with  the smoothness of $U$ and the fact that $u_0-u_1\le 0$ in $\partial U$.
\end{proof}

\section{Appendices}
The following results are used in the proof of Theorem \ref{theorem1.1}. We include them here, without proofs, for completeness.

\subsection{Two Properties of Weak Differentiation}
The following lemmas are needed in the proof of the main theorem.
\begin{lemma}\label{lemma1.1}
If $f\in L_{\rm loc}^1 \left(U\right)$ has weak partial derivative $\partial _i f\in L_{\rm loc}^1 \left(U\right)$ and $\psi \in C^\infty \left(U\right)$, then $\psi f$ is weakly differentiable with respect to $x_i$ and 
\begin{align}
\label{2.1}
{\partial _i}\left( {\psi f} \right) = {\partial _i}\psi f + \psi {\partial _i}f.
\end{align}
\end{lemma}

\begin{proof}
Let $\phi \in C_c^\infty \left(U\right)$ be any test function. Then $\psi \phi \in C_c^\infty \left(U\right)$ and the weak differentiability of $f$ implies that
\begin{align}
\int_U {f{\partial _i}\left( {\psi \phi } \right)dx}  =  - \int_U {{\partial _i}f\psi \phi dx} .
\end{align}
Expanding ${\partial _i}\left( {\psi \phi } \right) = \psi {\partial _i}\phi  + {\partial _i}\psi \phi $ in this equation and rearranging the result, we get
\begin{align}
\int_U {\psi f{\partial _i}\phi dx}  =  - \int_U {\left( {f{\partial _i}\psi  + \psi {\partial _i}f} \right)\phi dx} ,\hspace{2mm} \forall \phi  \in C_c^\infty \left( U \right).
\end{align}
Thus, $\psi f$ is weakly differentiable with respect to $x_i$ and its weak derivative is given by \eqref{2.1}.
\end{proof}

\begin{lemma}\label{lemma1.2}
Let $u\in H^1\left(U\right)$. Then $u^+ \in H^1\left(U\right)$ and its weak derivative is given by
\begin{equation}
D{u^ + }: = \left\{ \begin{split}
& Du, & \mbox{ a.e. on } \left\{ {u > 0} \right\},\\
& 0, & \mbox{ a.e. on } \left\{ {u \le 0} \right\}.
\end{split} \right.
\end{equation}
\end{lemma}
This lemma is a direct consequence of the following proposition. 
\begin{prop}\label{prop2.1}
If $u\in L_{\rm loc}^1 \left(U\right)$ has the weak derivative $\partial _i u\in L_{\rm loc}^1\left(U\right)$, then $\left| u \right| \in L_{loc}^1\left( U \right)$ is weakly differentiable and 
\begin{equation}
\label{2.5*}
{\partial _i}\left| u \right| = \left\{ \begin{split}
{\partial _i}u,& & \mbox{ if } u > 0,\\
0,& & \mbox{ if } u = 0,\\
 - {\partial _i}u,& & \mbox{ if } u < 0.
\end{split} \right.
\end{equation}
\end{prop}

\begin{proof}
Let ${f^\varepsilon }\left( t \right) = \sqrt {{t^2} + {\varepsilon ^2}}$. Since $f^\varepsilon$ is $C^1$ and globally Lipschitz, $f^\varepsilon \left(u\right)$ is weakly differentiable, and
\begin{align}
\int_U {{f^\varepsilon }\left( u \right){\partial _i}\phi dx}  =  - \int_U {\frac{{u{\partial _i}u}}{{\sqrt {{u^2} + {\varepsilon ^2}} }}\phi dx} ,\hspace{2mm} \forall \phi  \in C_c^\infty \left( U \right).
\end{align}
Taking the limit of this equation as $\varepsilon \to 0^+$ and using the Dominated Convergence Theorem \ref{theorem2.1}, we conclude that
\begin{align}
\int_U {\left| u \right|{\partial _i}\phi dx}  =  - \int_U {{\partial _i}\left| u \right|\phi dx} ,\hspace{2mm} \forall \phi  \in C_c^\infty \left( U \right),
\end{align}
where ${{\partial _i}\left| u \right|}$ is given by \eqref{2.5*}.
\end{proof}
Since the positive part of $u$ is given by ${u^ + }: = \frac{1}{2}\left( {\left| u \right| + u} \right)$, Proposition \ref{prop2.1} implies Lemma \ref{lemma1.1} directly.

\subsection{Dominated Convergence Theorem}
\begin{theorem}[Dominated Convergence Theorem]\label{theorem2.1}
Assume the functions $\left\{ {{f_k}} \right\}_{k = 1}^\infty$ are integrable and $f_k \to f$ a.e. Suppose also $\left| {{f_k}} \right| \le g$ a.e., for some summable function $g$. Then
\begin{align}
\int_{{\mathbb{R}^n}} {{f_k}dx}  \to \int_{{\mathbb{R}^n}} {fdx} .
\end{align}
\end{theorem}

\subsection{Lax-Milgram Theorem}
The Lax-Milgram theorem is a fairly simple abstract principle from linear functional analysis, which provides in certain circumstances the existence and uniqueness of a weak solution to some boundary-value problems. 

Assume that $H$ is a real Hilbert space, with norm $\left\|  \cdot  \right\|$ and inner product $\left(\cdot,\cdot\right)$, we let $\left\langle { \cdot , \cdot } \right\rangle $ denote the pairing of $H$ with its dual space. 
\begin{theorem}[Lax-Milgram Theorem]
Assume that $B:H\times H\to \mathbb{R}$ is a bilinear mapping, for which there exists constants $\alpha,\beta >0$ such that
\begin{align}
\left| {B\left[ {u,v} \right]} \right| \le \alpha \left\| u \right\|\left\| v \right\|, \hspace{2mm}\forall u,v \in H,
\end{align}
and 
\begin{align}
\beta {\left\| u \right\|^2} \le B\left[ {u,v} \right], \hspace{2mm}\forall u \in H.
\end{align}
Finally, let $f:H\to \mathbb{R}$ be a bounded linear functional on $H$.

Then there exists a unique element $u\in H$ such that
\begin{align}
B\left[ {u,v} \right] = \left\langle {f,v} \right\rangle , \hspace{2mm}\forall v \in H.
\end{align}
\end{theorem}

\subsection{Weak Convergence}
Let $X$ denote a real Banach space.
\begin{definition}[Weak convergence\footnote{See \cite{1}, Sec. D.4, p. 723.}]
We say a sequence $\left\{ {{u_k}} \right\}_{k = 1}^\infty  \subset X$ \emph{converges weakly} to $u\in X$, written $u_k\rightharpoonup u$, if $\left\langle {{u^*},{u_k}} \right\rangle  \to \left\langle {{u^*},u} \right\rangle $ for each bounded linear functional $u^* \in X^*$.
\end{definition}

\begin{theorem}[Weak compactness] 
Let $X$ be a reflexive Banach space and suppose the sequence $\left\{ {{u_k}} \right\}_{k = 1}^\infty  \subset X$ is bounded. Then there exists a subsequence $\left\{ {{u_{{k_j}}}} \right\}_{j = 1}^\infty  \subset \left\{ {{u_k}} \right\}_{k = 1}^\infty $ and $u\in X$ such that $u_{k_j}\rightharpoonup u$. 
\end{theorem}
In other words, bounded sequences in a reflexive Banach space are weakly precompact. In particular, \textit{a bounded sequence in a Hilbert space contains a weakly convergent subsequence}.



\subsection{Homogeneous Dirichlet Problem for the PDE $-\Delta u +\lambda u=f$, with $\lambda >0$}
This section is an obvious modification to the homogeneous Dirichlet problem for the Laplacian $-\Delta u +u=f$ presented in \cite{2}, p. 291.

Let $U \subset \mathbb{R}^N$ be an open bounded set. We are looking for a function $u: \overline{U}\to \mathbb{R}$ satisfying 
\begin{equation}
\label{2.5}
\left\{ \begin{split}
 - \Delta u + \lambda u &= f, & \mbox{ in } U,\\
u &= 0, & \mbox{ on } \partial U,
\end{split} \right.
\end{equation}
where $\lambda >0$, and $f$ is a given function on $U$.
\begin{definition}
A \emph{weak solution} of \eqref{2.5} is a function $u\in H_0^1\left(U\right)$ satisfying
\begin{align}
\int_U {\left( {Du \cdot Dv + \lambda uv} \right)dx}  = \int_U {fvdx} ,\hspace{2mm} \forall v \in H_0^1\left( U \right).
\end{align}
\end{definition}
We now focus on the existence and uniqueness of a weak solution of \eqref{2.5}.
\begin{theorem}[Dirichlet's principle]
Given any $f\in L^2\left(U\right)$, there exists a unique weak solution $u\in H_0^1\left(U\right)$ of \eqref{2.5}. Furthermore, $u$ is obtained by 
\begin{align}
\mathop {\min }\limits_{v \in H_0^1\left( U \right)} \left\{ {\frac{1}{2}\int_U {\left( {{{\left| {Dv} \right|}^2} + \lambda {{\left| v \right|}^2}} \right)dx}  - \int_U {fvdx} } \right\}.
\end{align}
\end{theorem}

\begin{proof}
Apply Lax-Milgram in the Hilbert space $H=H_0^1\left(U\right)$ with the bilinear form
\begin{align}
B\left[ {u,v} \right]: = \int_U {\left( {Du \cdot Dv + \lambda uv} \right)dx} ,\hspace{2mm}\forall u,v \in H_0^1\left( U \right),
\end{align}
and the linear functional $\phi :v \mapsto \int_U {fvdx}$, $\forall v \in H_0^1\left( U \right)$. 
\end{proof}
The following theorem gives more regularity on the weak solution.
\begin{theorem}[Regularity for the Dirichlet problem, see \cite{2}, p. 298]\label{theorem3.5}
Let $U$ be an open set of class $C^2$ with $\partial U$ bounded (or else $U=\mathbb{R}_+^N$). Let $f\in L^2\left(U\right)$ and let $u\in H_0^1\left(U\right)$ satisfy
\begin{align}
\int_U {\left( {Du \cdot D\varphi  + \lambda u\varphi } \right)dx}  = \int_U {f\varphi dx} ,\hspace{2mm}\forall \varphi  \in H_0^1\left( U \right).
\end{align}
Then $u\in H^2\left(U\right)$ and ${\left\| u \right\|_{{H^2}}} \le C{\left\| f \right\|_{{L^2}}}$, where $C$ is a constant depending only on $U$. Furthermore, if $U$ is of class $C^{m+2}$ and $f\in H^m\left(U\right)$, then 
\begin{align}
u\in H^{m+2} \left(U\right) \mbox{ and } {\left\| u \right\|_{{H^{m + 2}}}} \le C{\left\| f \right\|_{{H^m}}} .
\end{align}
In particular, if $f\in H^m\left(U\right)$ with $m>\frac{N}{2}$, then $u\in C^2\left(\overline{U}\right)$. Finally, if $U$ is of class $C^\infty$ and if $f\in C^\infty \left(\overline{U}\right)$, then $u\in C^\infty \left(\overline{U}\right)$.
\end{theorem}



%\section*{Appendix A. Supplementary notes}



%\section*{Acknowledgment}

%\printbibliography

%\bibliographystyle{siam}
%\bibliography{MYBIB}
%\Addresses
\newpage
\begin{thebibliography}{999}
\bibitem {1} Lawrence C. Evans. \textit{Partial Differential Equations}, 2e. Graduate Studies in Mathematics, Volume 19, AMS.
\bibitem {2} Haim Brezis. \textit{Functional Analysis, Sobolev Spaces and Partial Differential Equations}. Springer.
\bibitem {3} John K. Hunter. \textit{Notes on Partial Differential Equations}. 2014.
\end{thebibliography}
\end{document}