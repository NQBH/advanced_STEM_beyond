\documentclass[12pt]{article}
\usepackage[utf8]{inputenc}
\usepackage[vietnamese]{babel}
\usepackage{amsmath, amssymb}
\usepackage{listings}
\usepackage{enumitem}

\title{Bài Toán 2: Đếm số phân hoạch $p(n,k)$}
\author{Đồ Án Phân Hoạch Số Nguyên}
\date{}

\begin{document}

\maketitle

\section*{Phát biểu bài toán}
Cho hai số nguyên dương $n, k \in \mathbb{N}$. Đếm số phân hoạch của $n$ sao cho phần tử lớn nhất trong mỗi phân hoạch không vượt quá $k$. Ký hiệu hàm đếm là $p(n, k)$.

\textbf{Ví dụ:} $p(5,3)$ = 5, gồm các phân hoạch:
\[
(3,2), (3,1,1), (2,2,1), (2,1,1,1), (1,1,1,1,1)
\]

\section*{Công thức đệ quy}
Công thức đệ quy:
\[
p(n,k) = 
\begin{cases}
1 & \text{if } n = 0 \\
0 & \text{if } n < 0 \text{ or } k = 0 \\
p(n-k,k) + p(n,k-1) & \text{otherwise}
\end{cases}
\]
\textbf{Giải thích:}
\begin{itemize}
    \item $p(n-k, k)$ là số phân hoạch của $n$ có ít nhất một phần tử bằng $k$
    \item $p(n, k-1)$ là số phân hoạch của $n$ có tất cả phần tử nhỏ hơn $k$
\end{itemize}

\section*{Thuật toán (Quy hoạch động)}
Khởi tạo mảng hai chiều $dp[n+1][k+1]$, với:
\begin{itemize}
    \item $dp[i][j]$ lưu giá trị $p(i, j)$
    \item Gán $dp[0][j] = 1$ với mọi $j \geq 0$
    \item Duyệt $i$ từ 1 đến $n$, $j$ từ 1 đến $k$, cập nhật:
    \[
    dp[i][j] = dp[i-j][j] + dp[i][j-1] \quad \text{nếu } i \geq j
    \]
    ngược lại thì:
    \[
    dp[i][j] = dp[i][j-1]
    \]
\end{itemize}

\section*{Chú thích các biến}
\begin{itemize}[itemsep=0pt]
    \item \texttt{n, k} – đầu vào bài toán
    \item \texttt{dp[i][j]} – số phân hoạch của $i$ với phần tử lớn nhất $\leq j$
\end{itemize}

\section*{So sánh}
Có thể so sánh $p(n)$ (tổng phân hoạch của $n$) với:
\[
p(n) = \sum_{k=1}^{n} p(n,k)
\]

\end{document}
