\documentclass{article}
\usepackage[backend=biber,natbib=true,style=alphabetic,maxbibnames=50]{biblatex}
\addbibresource{/home/nqbh/reference/bib.bib}
\usepackage[utf8]{vietnam}
\usepackage{tocloft}
\renewcommand{\cftsecleader}{\cftdotfill{\cftdotsep}}
\usepackage[colorlinks=true,linkcolor=blue,urlcolor=red,citecolor=magenta]{hyperref}
\usepackage{amsmath,amssymb,amsthm,enumitem,float,graphicx,mathtools,tikz}
\usetikzlibrary{angles,calc,intersections,matrix,patterns,quotes,shadings}
\allowdisplaybreaks
\newtheorem{assumption}{Assumption}
\newtheorem{baitoan}{}
\newtheorem{cauhoi}{Câu hỏi}
\newtheorem{conjecture}{Conjecture}
\newtheorem{corollary}{Corollary}
\newtheorem{dangtoan}{Dạng toán}
\newtheorem{definition}{Definition}
\newtheorem{dinhluat}{Định luật}
\newtheorem{dinhly}{Định lý}
\newtheorem{dinhnghia}{Định nghĩa}
\newtheorem{example}{Example}
\newtheorem{ghichu}{Ghi chú}
\newtheorem{hequa}{Hệ quả}
\newtheorem{hypothesis}{Hypothesis}
\newtheorem{lemma}{Lemma}
\newtheorem{luuy}{Lưu ý}
\newtheorem{nhanxet}{Nhận xét}
\newtheorem{notation}{Notation}
\newtheorem{note}{Note}
\newtheorem{principle}{Principle}
\newtheorem{problem}{Problem}
\newtheorem{proposition}{Proposition}
\newtheorem{question}{Question}
\newtheorem{remark}{Remark}
\newtheorem{theorem}{Theorem}
\newtheorem{vidu}{Ví dụ}
\usepackage[left=1cm,right=1cm,top=5mm,bottom=5mm,footskip=4mm]{geometry}
\def\labelitemii{$\circ$}
\DeclareRobustCommand{\divby}{%
	\mathrel{\vbox{\baselineskip.65ex\lineskiplimit0pt\hbox{.}\hbox{.}\hbox{.}}}%
}
\def\labelitemii{$\circ$}
\setlist[itemize]{leftmargin=*}
\setlist[enumerate]{leftmargin=*}

\title{The Elements of Style}
\author{William Strunk, Jr. \and E. B. White}
\date{\today}

\begin{document}
\maketitle
\tableofcontents

%------------------------------------------------------------------------------%

\section*{Foreword}

%------------------------------------------------------------------------------%

\section*{Introduction}

%------------------------------------------------------------------------------%

\section{Elementary Rules of Usage}

\subsection{Form the possessive singular of nouns by adding 's.}

%------------------------------------------------------------------------------%

\subsection{In a series of 3 or more terms with a single conjunction, use a comma after each term except the last.}

%------------------------------------------------------------------------------%

\subsection{Enclose parenthetic expressions between commas.}

%------------------------------------------------------------------------------%

\subsection{Place a comma before a conjunction introducing an independent clause.}

%------------------------------------------------------------------------------%

\subsection{Do not join independent clauses with a comma.}

%------------------------------------------------------------------------------%

\subsection{Do not break sentences in 2.}

%------------------------------------------------------------------------------%

\subsection{Use a colon after an independent clause to introduce a list of particulars, an appositive, an amplification, or an illustrative quotation.}

%------------------------------------------------------------------------------%

\subsection{Use a dash to set off an abrupt break or interruption \& to announce a long appositive or summary.}

%------------------------------------------------------------------------------%

\subsection{The number of the subject determines the number of the verb.}

%------------------------------------------------------------------------------%

\subsection{Use the proper case of pronoun.}

%------------------------------------------------------------------------------%

\subsection{A participial phrase at the beginning of a sentence must refer to the grammatical subject.}

%------------------------------------------------------------------------------%

\section{Elementary Principles of Composition}

\subsection{Choose a suitable design \& hold to it.}
``A basic structural design underlies every kind of writing. Writers will in part follow this design, in part deviate from it, according to their skills, their needs, \& the unexpected events that accompany the act of composition. Writing, to be effective, must follow closely the thoughts of the writer, but not necessarily in the order in which those thoughts occur. This calls for a scheme of procedure. In some cases, the best design is no design, as with a love letter, which is simply an outpouring, or with a casual essay, which is a ramble. But in most cases, planning must be a deliberate prelude to writing. The 1st principle of composition, therefore, is to foresee or determine the shape of what is to come \& pursue that shape.

A sonnet is built on a 14-line frame, each line containing 5 feet. Hence, sonneteers know exactly where they are headed, although they may not know how to get there. Most forms of composition are less clearly defined, more flexible, but all have skeletons to which the writer will bring the flesh \& the blood. The more clearly the writer perceives the shape, the better are the chances of success.'' -- \cite[p. 29]{Strunk_White2019}

%------------------------------------------------------------------------------%

\subsection{Make the paragraph the unit of composition: 1 paragraph to each topic.}
``The paragraph is a convenient unit; it serves all forms of literary work. As long as it holds together, a paragraph may be of any length -- a single, short sentence or a passage of great duration.

If the subject on which you are writing is of slight extent, or if you intend to treat it briefly, there may be no need to divide it into topics. Thus, a brief description, a brief book review, a brief account of a single incident, a narrative merely outlining an action, the setting forth of a single idea -- any 1 of these is best written in a single paragraph. After the paragraph has been written, examine it to see whether division will improve it.

Ordinarily, however, a subject requires division into topics, each of which should be dealt with in a paragraph. The object of treating each topic in a paragraph by itself is, of course, to aid the reader. The beginning of each paragraph is a signal that a new step in the development of the subject has been reached.

As a rule, single sentences should not be written or printed as paragraphs. An exception may be made of sentences of transition, indicating the relation between the parts of an exposition or argument.

In dialogue, each speech, even if only a single word, is usually a paragraph by itself; i.e., a new paragraph begins with each change of speaker. The application of this rule when dialogue \& narrative are combined is best learned from examples in well-edited works of fiction. Sometimes a writer, seeking to create an effect of rapid talk or for some other reason, will elect not to set off each speech in a separate paragraph \& instead will run speeches together. The common practice, however, \& the one that serves best in most instances, is to give each speech a paragraph of its own.

As a rule, begin each paragraph either with a sentence that suggests the topic or with a sentence that helps the transition. If a paragraph forms part of a larger composition, its relation to what precedes, or its function as a part of the whole, may need to be expressed. This can sometimes be done by a mere word or phrase (\textit{again, therefore, for the same reason}) in the 1st sentence. Sometimes, however, it is expedient to get into the topic slowly, by way of a sentence or 2 of introduction or transition.

In narration \& description, the paragraph sometimes begins with a concise, comprehensive statement serving to hold together the details that follows.
\begin{quotation}\it
	The breeze served us admirably.
	
	The campaign opened with a series of reverses.
	
	The next 10 or 12 pages were filled with a curious set of entries.
\end{quotation}
But when this device, or any device, is too often used, it becomes a mannerism. More commonly, the opening sentence simply indicates by its subject the direction the paragraph is to take.
\begin{quotation}\it
	At length I thought I might return toward the stockade.
	
	He picked up the heavy lamp from the table \& began to explore.
	
	Another flight of steps, \& they emerged on the roof.
\end{quotation}
In animated narrative, the paragraphs are likely to be short \& without any semblance of a topic sentence, the writer rushing headlong, event following event in rapid succession. The break between such paragraphs merely serves the purpose of a rhetorical pause, throwing into prominence some detail of the action.

In general, remember that paragraphing calls for a good eye as well as a logical mind. Enormous blocks of print look formidable to readers, who are often reluctant to tackle them. Therefore, breaking long paragraphs in 2, even if it is not necessary to do so for sense, meaning, or logical development, is often a visual help. But remember, too, that firing off many short paragraphs in quick succession can be distracting. Paragraph breaks used only for show read like the writing of commerce or of display advertising. Moderation \& a sense of order should be the main considerations in paragraphing.'' -- \cite[pp. 30--31]{Strunk_White2019}

%------------------------------------------------------------------------------%

\subsection{Use the active voice.}
``The active voice is usually more direct \& vigorous than the passive:
\begin{example}
	I shall always remember my 1st visit to Boston.
\end{example}
This is much better than
\begin{example}
	My 1st visit to Boston will always be remembered by me.
\end{example}
The latter sentence is less direct, less bold, \& less concise. If the writer tries to make it more concise by omitting ``by me,''
\begin{example}
	My 1st visit to Boston will always be remembered,
\end{example}
it becomes indefinite: is it the writer or some undisclosed person or the world at large that will always remember this visit?

This rule does not, of course, mean that the writer should entirely discard the passive voice, which is frequently convenient \& sometimes necessary.
\begin{example}
	The dramatists of the Restoration are little esteemed today.
	
	Modern readers have little esteem for the dramatists of the Restoration.
\end{example}
The 1st would be the preferred form in a paragraph on the dramatists of the Restoration, the 2nd in a paragraph on the tastes of modern readers. The need to make a particular word the subject of the sentence will often, as in these examples, determine which voice is to be used.

The habitual use of the active voice, however, makes for forcible writing. This is true not only in narrative concerned principally with action but in writing of any kind. Many a tame sentence of description or exposition can be made lively \& emphatic by substituting a transitive in the active voice for some such perfunctory expression as \textit{there is} or \textit{could be heard}.
\begin{example}
	There were a great number of dead leaves lying on the ground. $\to$ Dead leaves covered the ground.
	
	At dawn the crowing of a rooster could be heard. $\to$ The cock's crow came with dawn.
	
	The reason he left college was that his health became impaired. $\to$ Failing health compelled him to leave college.
	
	It was not long before she was very sorry that she had said what she had. $\to$ She soon repented her words.
\end{example}
Note, in the examples above, that when a sentence is made stronger, it usually becomes shorter. Thus, brevity is a by-product of vigor.'' -- \cite[p. 32]{Strunk_White2019}

%------------------------------------------------------------------------------%

\subsection{Put statements in positive form.}
``Make definite assertions. Avoid tame, colorless, hesitating, noncommittal language. Use the word \textit{not} as a means of denial or in antithesis, never as a means of evasion.
\begin{example}
	He was not very often on time. $\to$ He usually came late.
	
	She did not think that studying Latin was a sensible way to use one's time. $\to$ She thought the study of Latin a waste of time.
	
	\emph{The Taming of the Shrew} is rather weak in spots. Shakespeare does not portray Katharine as a very admirable character, nor does Bianca remain long in memory as an important character in Shakespeare's works. $\to$ The women in \emph{The Taming of the Shrew} are unattractive. Katharine is disagreeable, Bianca insignificant.
\end{example}
The last example, before correction, is indefinite as well as negative. The corrected version, consequently, is simply a guess at the writer's intention.

All 3 examples show the weakness inherent in the word \textit{not}. Consciously or unconsciously, the reader is dissatisfied with being told only what is not; the reader wishes to be told what is. Hence, as a rule, it is better to express even a negative in positive form.
\begin{example}
	not honest $\to$ dishonest, not important $\to$ trifling, did not remember $\to$ forgot, did not pay any attention to $\to$ ignored, did not have much confidence in $\to$ distrusted.
\end{example}
Placing negative \& positive in opposition makes for a stronger structure.
\begin{example}
	Not charity, but simple justice.
	
	Not that I loved Caesar less, but that I loved Rome more.
	
	Ask not what your country can do for you -- ask what you can do for your country.
\end{example}
Negative words other than \textit{not} are usually strong.
\begin{example}
	Her loveliness I never knew
	
	Until she smiled on me.
\end{example}
Statements qualified with unnecessary auxiliaries or conditionals sound irresolute.
\begin{example}
	If you would let us know the time of your arrival, we would be happy to arrange your transportation from the airport. $\to$ If you will let us know the time of your arrival, we shall be happy to arrange your transportation from the airport.
	
	Applicants can make a good impression by being neat \& punctual. $\to$  Applicants will make a good impression if they are neat \& punctual.
	
	Plath may be ranked among those modem poets who died young. $\to$ Plath was one of those modern poets who died young.
\end{example}
If your every sentence admits a doubt, your writing will lack authority. Save the auxiliaries \textit{would, should, could, may, might}, \& \textit{can} for situations involving real uncertainty.'' -- \cite[pp. 33--34]{Strunk_White2019}

%------------------------------------------------------------------------------%

\subsection{Use definite, specific, concrete  language.}
``Prefer the specific to the general, the definite to the vague, the concrete to the abstract.
\begin{example}
	A period of unfavorable weather set in. $\to$ It rained every day for a week.
	
	He showed satisfaction as he took possession of his well-earned reward. $\to$ He grinned as he pocketed the coin.
\end{example}
If those who have studied the art of writing are in accord on any 1 point, it is this: the surest way to arouse \& hold the readers attention is by being specific, definite, \& concrete. The greatest writers -- Homer, Dante, Shakespeare -- are effective largely because they deal in particulars \& report the details that matter. Their words call up pictures.

Jean Stafford, to cite a more modern author, demonstrates in her short story ``In the Zoo'' how prose is made vivid by the use of words that evoke images \& sensations:
\begin{example}
	$\ldots$ Daisy \& I in time found asylum in a small menagerie down by the railroad tracks. It belonged to a gentle alcoholic ne'er-do- well, who did nothing all day long but drink bathtub gin in rickeys \&  play solitaire \&  smile to himself \&  talk to his animals. He had a little, stunted red vixen \&  a deodorized skunk, a parrot from Tahiti that spoke Parisian French, a woebegone coyote, \&  two capuchin monkeys, so serious \&  humanized, so small \&  sad \&  sweet, \&  so religious-looking with their tonsured heads that it was impossible not to think their gibberish was really an ordered language with a grammar that someday some philologist would understand.
	
	Gran knew about our visits to Mr. Murphy \&  she did not object, for it gave her keen pleasure to excoriate him when we came home. His vice was not a matter of guesswork; it was an established fact that he was half-seas over from dawn till midnight. ``With the black Irish,'' said Gran, ``the taste for drink is taken in with the mother's milk \&  is never mastered. Oh, I know all about those promises to join the temperance movement \&  not to touch another drop. The way to Hell is paved with good intentions.'' -- Excerpt from ``In the Zoo'' from Bad Characters by Jean Stafford.
\end{example}
If the experiences of Walter Mitty, of Molly Bloom, of Rabbit Angstrom have seemed for the moment real to countless readers, if in reading Faulkner we have almost the sense of inhabiting Yoknapatawpha County during the decline of the South, it is because the details used are definite, the terms concrete. It is not that every detail is given -- that would be impossible, as well as to no purpose -- but that all the significant details are given, \&  with such accuracy \&  vigor that readers, in imagination, can project themselves into the scene.

In exposition \&  in argument, the writer must likewise never lose hold of the concrete; \&  even when dealing with general principles, the writer must furnish particular instances of their application.

In his \textit{Philosophy of Style}, Herbert Spencer gives 2 sentences to illustrate how the vague \&  general can be turned into the vivid \&  particular:
\begin{example}
	In proportion as the manners, customs, \&  amusements of a nation are cruel \&  barbarous, the regulations of their penal code will be severe. $\to$ In proportion as men delight in battles, bullfights, \&  combats of gladiators, will they punish by hanging, burning, \&  the rack.
\end{example}
To show what happens when strong writing is deprived of its vigor, George Orwell once took a passage from the Bible \&  drained it of its blood. On the left, below, is Orwell’s translation; on the right, the verse from Ecclesiastes (King James Version).
\begin{example}
	Objective consideration of contemporary phenomena compels the conclusion that success or failure in competitive activities exhibits no tendency to be commensurate with innate capacity, but that a considerable element of the unpredictable must inevitably be taken into account. $\to$ I returned, \&  saw under the sun, that the race is not to the swift, nor the battle to the strong, neither yet bread to the wise, nor yet riches to men of understanding, nor yet favor to men of skill; but time \&  chance happeneth to them all.'' -- \cite[pp. 35--36]{Strunk_White2019}
\end{example}

%------------------------------------------------------------------------------%

\subsection{Omit needless words.}
``Vigorous writing is concise. A sentence should contain no unnecessary words, a paragraph no unnecessary sentences, for the same reason that a drawing should have no unnecessary lines \&  a machine no unnecessary parts. This requires not that the writer make all sentences short, or avoid all detail \&  treat subjects only in outline, but that every word tell.

Many expressions in common use violate this principle.
\begin{example}
	the question as to whether $\to$ whether (the question whether), there is no doubt but that $\to$ no doubt (doubtless), used for fuel purposes $\to$ used for fuel, he is a man who $\to$ he, in a hasty manner $\to$ hastily, this is a subject that $\to$ this subject, Her story is a strange one. $\to$ Her story is strange. the reason why is that $\to$ because.
\end{example}
\textit{The fact that} is an especially debilitating expression. It should be revised out of every sentence in which it occurs.
\begin{example}
	owing to the fact that $\to$ since (because), in spite of the fact that $\to$ though (although), call your attention to the fact that $\to$ remind you (notify you), I was unaware of the fact that $\to$ I was unaware that (did not know), the fact that he had not succeeded $\to$ his failure, the fact that I had arrived $\to$ my arrival.
\end{example}
See also the words \textit{case, character, nature} in Chap. IV. \textit{Who is, which was}, \&  the like are often superfluous.
\begin{example}
	His cousin, who is a member of the same firm $\to$ His cousin, a member of the same firm
	
	Trafalgar, which was Nelson's last battle $\to$ Trafalgar, Nelson’s last battle.
\end{example}
As the active voice is more concise than the passive, \&  a positive statement more concise than a negative one, many of the examples given under Rules 14 \&  15 illustrate this rule as well.

A common way to fall into wordiness is to present a single complex idea, step by step, in a series of sentences that might to advantage be combined into one.
\begin{example}
	Macbeth was very ambitious. This led him to wish to become king of Scotland. The witches told him that this wish of his would come true. The king of Scotland at this time was Duncan. Encouraged by his wife, Macbeth murdered Duncan. He was thus enabled to succeed Duncan as king. (51 words)
	
	$to$ Encouraged by his wife, Macbeth achieved his ambition \&  realized the prediction of the witches by murdering Duncan \&  becoming king of Scotland in his place. (26 words)'' -- \cite[pp. 37--38]{Strunk_White2019}
\end{example}

%------------------------------------------------------------------------------%

\subsection{Avoid a succession of loose sentences.}
``This rule refers especially to loose sentences of a particular type: those consisting of 2 clauses, the 2nd introduced by a conjunction or relative. A writer may err by making sentences too compact \& periodic. An occasional loose sentence prevents the style from becoming too formal \& gives the reader a certain relief. Consequently, loose sentences are common in easy, unstudied writing. The danger is that there may be too many of them.

An unskilled writer will sometimes construct a whole paragraph of sentences of this kind, using as connectives \textit{and, but}, and, less frequently, \textit{who, which, when, where}, \& \textit{while}, these last in nonrestrictive senses. (See Rule 3.)
\begin{example}
	The 3rd concert of the subscription series was given last evening, \& a large audience was in attendance. Mr. Edward Appleton was the soloist, \& the Boston Symphony Orchestra furnished the instrumental music. The former showed himself to be an artist of the 1st rank, while the latter proved itself fully deserving of its high reputation. The interest aroused by the series has been very gratifying to the Committee, \& it is planned to give a similar series annually hereafter. The 4th concert will be given on Tuesday, May 10, when an equally attractive program will be presented.
\end{example}
Apart from its triteness \& emptiness, the paragraph above is bad because of the structure of its sentences, with their mechanical symmetry \& singsong. Compare these sentences from the chapter ``What I Believe'' in E. M. Forster's \textit{2 Cheers for Democracy}:
\begin{example}
	I believe in aristocracy, though -- if that is the right word, \& if a democrat may use it. Not an aristocracy of power, based upon rank \& influence, but an aristocracy of the sensitive, the considerate \& the plucky. Its members are to be found in all nations \& classes, \& all through the ages, \& there is a secret understanding between them when they meet. They represent the true human tradition, the 1 permanent victory of our queer race over cruelty \& chaos. Thousands of them perish in obscurity, a few are great names. They are sensitive for others as well as for themselves, they are considerate without being fussy, their pluck is not swankiness but the power to endure, \& they can take a joke.
\end{example}
A writer who has written a series of loose sentences should recast enough of them to remove the monotony, replacing them with simple sentences, sentences of 2 clauses joined by a semicolon, periodic sentences of 2 clauses, or sentences (loose or periodic) of 3 clauses -- whichever best represent the real relations of the thought.'' -- \cite[pp. 39--40]{Strunk_White2019}

%------------------------------------------------------------------------------%

\subsection{Express coordinate ideas in similar form.}
``This principle, that of parallel construction, requires that expressions similar in content \& function be outwardly similar. The likeness of form enables the reader to recognize more readily the likeness of content \& function. The familiar Beatitudes exemplify the virtue of parallel construction.
\begin{quotation}
	Blessed are the poor in spirit: for theirs is the kingdom of heaven.
	
	Blessed are they that mourn: for they shall be comforted.
	
	Blessed are the meek: for they shall inherit the earth.
	
	Blessed are they which do hunger \& thirst after righteousness: for they shall be filled.
\end{quotation}
The unskilled writer often violates this principle, mistakenly believing in the value of constantly varying the form of expression. When repeating a statement to emphasize it, the writer may need to vary its form. Otherwise, the writer should follow the principle of parallel construction.
\begin{example}
	Formerly, science was taught by the textbook method, while now the laboratory method is employed. $\to$ Formerly, science was taught by the textbook method; now it is taught by the laboratory method.
\end{example}
The lefthand version gives the impression that the writer is undecided or timid, apparently unable or afraid to choose one form of expression \& hold to it. The righthand version shows that the writer has at least made a choice \& abided by it.

By this principle, an article or a preposition applying to all the members of a series must either be used only before the first term or else be repeated before each term.
\begin{example}
	The French, the Italians, Spanish, \& Portuguese $\to$ The French, the Italians, the Spanish, \& the Portuguese
	
	In spring, summer, or in winter $\to$ In spring, summer, or winter (In spring, in summer, or in winter).
\end{example}
Some words require a particular preposition in certain idiomatic uses. When such words are joined in a compound construction, all the appropriate prepositions must be included, unless they are the same.
\begin{example}
	His speech was marked by disagreement and scorn for his opponent's position. $\to$ His speech was marked by disagreement with and scorn for his opponent's position.
\end{example}
Correlative expressions (\textit{both, \&; not, but; not only, but also; either, or; 1st, 2nd, 3rd}; \& the like) should be followed by the same grammatical construction. Many violations of this rule can be corrected by rearranging the sentence.
\begin{example}
	It was both a long ceremony and very tedious. $\to$ The ceremony was both long and tedious.
	
	A time for not words but action. $\to$ A time not for words but for action.
	
	Either you must grant his request or incur his ill will. $\to$ You must either grant his request or incur his ill will.
	
	My objections are, 1st, the injustice of the measure; 2nd, that it is unconstitutional. $\to$ My objections are, 1st, that the measure is unjust; 2nd, that it is unconstitutional.
\end{example}
It may be asked, what if you need to express a rather large number of similar ideas -- say, 20? Must you write 20 consecutive sentences of the same pattern? On closer examination, you will probably find that the difficulty is imaginary -- that these 20 ideas can be classified in groups, \& that you need apply the principle only within each group. Otherwise, it is best to avoid the difficulty by putting statements in the form of a table.'' -- \cite[pp. 41--42]{Strunk_White2019}

%------------------------------------------------------------------------------%

\subsection{Keep related words together.}

%------------------------------------------------------------------------------%

\subsection{In summaries, keep to 1 tense.}

%------------------------------------------------------------------------------%

\subsection{Place the emphatic words of a sentence at the end.}

%------------------------------------------------------------------------------%

\section{A Few Matters of Form}

%------------------------------------------------------------------------------%

\section{Words \& Expressions Commonly Misused}

%------------------------------------------------------------------------------%

\section{An Approach to Style (With a List of Reminders)}

\subsection{Place yourself in the background.}
``Write in a way that draws the reader's attention to the sense \& substance of the writing, rather than to the mood \& temper of the author. If the writing is solid \& good, the mood \& temper of the writer will eventually be revealed \& not at the expense of the work. Therefore, the 1st piece of advice is this: to achieve style, begin by affecting none -- i.e., place yourself in the background. A careful \& honest writer does not need to worry about style. As you become proficient in the use of language, your style will emerge, because you yourself will emerge, \& when this happens you will find it increasingly easy to break through the barriers that separate you from other minds, other hearts -- which is, of course, the purpose of writing, as well as its principal reward. Fortunately, the act of composition, or creation, disciplines the mind; writing is 1 way to go about thinking, \& the practice \& habit of writing not only drain the mind but supply it, too.'' -- \cite[p. 78]{Strunk_White2019}

%------------------------------------------------------------------------------%

\subsection{Write in a way that comes naturally.}
``Write in a way that comes easily \& naturally to you, using words \& phrases that come readily to hand. But do not assume that becaues you have acted naturally your product is without flaw.

The use of language begins with imitation. The infant imitates the sounds made by its parents; the child imitates 1st the spoken language, then the stuff of books. The imitative life continues long after the writer is secure in the language, for it is almost impossible to avoid imitating what one admires. Never imitate consciously, but do not worry about being an imitator; take pains instead to admire what is good. Then when you write in a way that comes naturally, you will echo the halloos that bear repeating.'' -- \cite[p. 79]{Strunk_White2019}

%------------------------------------------------------------------------------%

\subsection{Work from a suitable design.}
``Before beginning to compose something, gauge the nature \& extent of the enterprise \& work from a suitable design. (See Chap. II, Rule 12.) Design informs even the simplest structure, whether of brick \& steel or of prose. You raise a pup tent from 1 sort of vision, a cathedral from another. This does not mean that you must sit with a blueprint always in front of you, merely that you had best anticipate what you are getting into. To compose a laundry list, you can work directly from the pile of soiled garments, ticking them off 1 by 1. By to write a biography, you will need at least a rough scheme; you cannot plunge in blindly \& start ticking off fact after fact about your subject, lest you miss the forest for the trees \& there be no end to your labors.

Sometimes, of course, impulse \& emotion are more compelling than design. If you are deeply troubled \& are composing a letter appealing for mercy or for love, you had best not attempt to organize your emotions; the prose will have a better chance if the emotions are left in disarray -- which you'll probably have to do anyway, since feelings do not usually lend themselves to rearrangement. But even the kind of writing that is essentially adventurous \& impetuous will on examination be found to have a secret plan: Columbus didn't just sail, he sailed west, \& the New World took shape from this simple \&, we now think, sensible design.'' -- \cite[p. 80]{Strunk_White2019}

%------------------------------------------------------------------------------%

\subsection{Write with nouns \& verbs.}
``Write with nouns \& verbs, not with adjectives \& adverbs. The adjective hasn't been built that can pull a weak or inaccurate noun out of a tight place. This is not to disparage adjectives \& adverbs; they are indispensable parts of speech. Occasionally they surprise us with their power, as in
\begin{quotation}\it
	Up the airy mountain,
	
	Down the rushy glen,
	
	We daren't go a-hunting
	
	For fear of little men $\ldots$
\end{quotation}
The nouns \textit{mountain} \& \textit{glen} are accurate enough, but had the mountain not become airy, the glen rushy, William Ailing-ham might never have got off the ground with this poem. In general, however, it is nouns \& verbs, not their assistants, that give good writing its toughness \& color.'' -- \cite[p. 81]{Strunk_White2019}

%------------------------------------------------------------------------------%

\subsection{Revise \& rewrite.}
``Revising is part of writing. Few writers are so expert that they can produce what they are after on the 1st try. Quite often you will discover, on examining the completed work, that there are serious flaws in the arrangement of the material, calling for transpositions. When this is the case, a word processor can save you time \& labor as you rearrange the manuscript. You can select material on your screen \& move it to a more appropriate spot, or, if you cannot find the right spot, you can move the material to the end of the manuscript until you decide whether to delete it. Some writers find that working with a printed copy of the manuscript helps them to visualize the process of change; others prefer to revise entirely on screen. Above all, do not be afraid to experiment with what you have written. Save both the original \& the revised versions; you can always use the computer to restore the manuscript to its original condition, should that course seem best. Remember, it is no sign of weakness or defeat that your manuscript ends up in need of major surgery. This is a common occurrence in all writing, \& among the best writers.'' -- \cite[p. 82]{Strunk_White2019}

%------------------------------------------------------------------------------%

\subsection{Do not overwrite.}
``Rich, ornate prose is hard to digest, generally unwholesome, \& sometimes nauseating. If the sickly-sweet word, the overblown phrase are your natural form of expression, as is sometimes the case, you will have to compensate for it by a show of vigor, \& by writing something as meritorious as the Songs of Songs, which is Solomon's.

When writing with a computer, you must guard against wordiness. The click \& flow of a word processor can be seductive, \& you may find yourself adding a few unnecessary words or even a whole passage just to experience the pleasure of running your fingers over the keyboard \& watching your words appear on the screen. It is always a good idea to reread your writing later \& ruthlessly delete the excess.'' -- \cite[p. 83]{Strunk_White2019}

%------------------------------------------------------------------------------%

\subsection{Do not overstate.}
``When you overstate, readers will be instantly on guard, \& everything that has preceded your overstatement as well as everything that follows it will be suspect in their minds because they have lost confidence in your judgment or your poise. Overstatement is 1 of the common faults. A single overstatement, wherever or however it occurs, diminishes the whole, \& a single carefree superlative has the power to destroy, for readers, the object of your enthusiasm.'' -- \cite[p. 84]{Strunk_White2019}

%------------------------------------------------------------------------------%

\subsection{Avoid the use of qualifiers.}
``\textit{Rather, very, little, pretty} -- these are the leeches that infest the pond of prose, sucking the blood of words. The constant use of the adjective \textit{little} (except to indicate size) is particularly debilitating; we should all try to do a little better, we should all be very watchful of this rule, for it is a rather important one, \& we are pretty sure to violate it now \& then.'' -- \cite[p. 85]{Strunk_White2019}

%------------------------------------------------------------------------------%

\subsection{Do not affect a breezy manner.}
``The volume of writing is enormous, these days, \& much of it has a sort of windiness about it, almost as though the author were in a state of euphoria. ``Spontaneous me,'' say Whitman, \&, in his innocence, let loose the hordes of uninspired scribblers who would 1 day confuse spontaneity with genius.

The breezy style is often the work of an egocentric, the person who imagines that everything that comes to mind is of general interest \& that uninhibited prose creates high spirits \& carries the day. Open any alumni magazine, turn to the class notes, \& you are quite likely to encounter old Spontaneous Me at work -- an aging collegian who writes something like this:
\begin{quotation}\it
	Well, guys, here I am again dishing the dirt about your disorderly classmates, after passing a week in the Big Apple trying to catch the Columbia hoops tilt \& then a cab-ride from hell through the West Side casbah. \& speaking of news, howzabout tossing a few primo items this way?
\end{quotation}
This is an extreme example, but the same wind blows, at lesser velocities, across vast expanses of journalistic prose. The author in this case has managed in 2 sentences to commit most of the unpardonable sins: he obviously has nothing to say, he is showing off \& directing the attention of the reader to himself, he is using slang with neither provocation nor ingenuity, he adopts a patronizing air by throwing in the word \textit{primo}, he is humorless (though full of fun), dull, \& empty. He has not done his work. Compare his opening remarks with the following -- a plunge directly into the news:
\begin{quotation}\it
	Clyde Crawford, who stroked the varsity shell in 1958, is swinging an oar again after a lapse of 40 years. Clyde resigned last spring as executive sales manager of the Indiana Flotex Company \& is now a gondolier in Venice.
\end{quotation}
This, although conventional, is compact, informative, unpretentious. The writer has dug up an item of news \& presented it in a straightforward manner. What the 1st writer tried to accomplish by cutting rhetorical capers \& by breeziness, the 2nd writer managed to achieve by good reporting, by keeping a tight rein on his material, \& by staying out of the act.'' -- \cite[p. 87]{Strunk_White2019}

%------------------------------------------------------------------------------%

\subsection{Use orthodox spelling.}
``In ordinary composition, use orthodox spelling. Do not write \textit{nite} for \textit{night, thru} for \textit{through, pleez} for \textit{please}, unless you plan to introduce a complete system of simplified spelling \& are prepared to take the consequences.

In the original edition of \textit{The Elements of Style}, there was a chapter on spelling. In it, the author had this to say:
\begin{quotation}\it
	The spelling of English words is not fixed \& invariable, nor does it depend on any other authority than general agreement. At the present day there is practically unanimous agreement as to the spelling of most words $\ldots$ At any given moment, however, a relatively small number of words may be spelled in more than 1 way. Gradually, as a rule, 1 of these forms comes to be generally preferred, \& the less customary form comes to look obsolete \& is discarded. From time to time new forms, mostly simplifications, are introduced by innovators, \& either win their place or die of neglect.
	
	The practical objection to unaccepted \& oversimplified spellings is the disfavor with which they are received by the reader. They distract his attention \& exhaust his patience. He reads the form though automatically, without thought of its needless complexity; he reads the abbreviation tho \& mentally supplies the missing letters, at the cost of a fraction of his attention. The writer has defeated his own purposed.
\end{quotation}
The language manages somehow to keep pace with events. A word that has taken hold in our century is \textit{thru-way}; it was born of necessity \& is apparently here to stay. In combination with \textit{way, thru} is more serviceable than \textit{through}; it is a high-speed word for readers who are going 65. \textit{Throughway} would be too long to fit on a road sign, too slow to serve the speeding eye. It is conceivable that because of our thruways, \textit{through} will eventually become \textit{thru} -- after many more thousands of miles of travel.'' -- \cite[p. 88]{Strunk_White2019}

%------------------------------------------------------------------------------%

\subsection{Do not explain too much.}
``It is seldom advisable to tell all. Be sparing, e.g., in the use of adverbs after ``he said,'' ``she replied,'' \& the like: ``he said consolingly''; ``she replied grumblingly.'' Let the conversation itself disclose the speaker's manner of condition. Dialogue heavily weighted with adverbs after the attributive verb is cluttery \& annoying. Inexperienced writers not only overwork their adverbs but load their attributives with explanatory verbs: ``he consoled,'' ``she congratulated.'' They do this, apparently, in the belief that the word \textit{said} is always in need of support, or because they have been told to do it by experts in the art of bad writing.'' -- \cite[p. 89]{Strunk_White2019}

%------------------------------------------------------------------------------%

\subsection{Do not construct awkward adverbs.}
``Adverbs are easy to build. Take an adjective or a participle, add \textit{-ly}, \& behold! you have an adverb. But you'd probably be better off without it. Do not write \textit{tangledly}. The word itself is a tangle. Do not even write \textit{tiredly}. Nobody says \textit{tangledly} \& not many people say \textit{tiredly}. Words that are not used orally are seldom the ones to put on paper.
\begin{example}
	He climbed tiredly to bed. $\to$ He climbed wearily to bed.
	
	The lamp cord lay tangledly beneath her chair. $\to$ The lamp cord lay in tangles beneath her chair.
\end{example}
Do not dress words up by adding \textit{-ly} to them, as though putting a hat on a horse.
\begin{example}
	overly $\to$ over, muchly $\to$ much, thusly $\to$ thus.'' -- \cite[p. 90]{Strunk_White2019}
\end{example}


%------------------------------------------------------------------------------%

\subsection{Make sure the reader knows who is speaking.}
``Dialogue is a total loss unless you indicate who the speaker is. In long dialogue passages containing no attributives, the reader may become lost \& be compelled to go back \& reread in order to puzzle the thing out. Obscurity is an imposition on the reader, to say nothing of its damage to the work.

In dialogue, make sure that your attributives do not awkwardly interrupt a spoken sentence. Place them where the break would come naturally in speech -- i.e., where the speaker would pause for emphasis, or take a breath. The best test for locating an attributive is to speak the sentence aloud.
\begin{example}
	``Now, my boy, we shall see,'' he said, ``how well you have learned your lesson.'' $\to$ ``Now, my boy,'' he said, ``we shall see how well you have learned your lesson.''
	
	``What's more, they would never,'' she added, ``consent to the plan.'' $\to$  ``What's more,'' she added, ``they would never consent to the plan.'''' -- \cite[p. 91]{Strunk_White2019}
\end{example}

%------------------------------------------------------------------------------%

\subsection{Avoid fancy words.}
``Avoid the elaborate, the pretentious, the coy, \& the cute. Do not be tempted by a 20-dollar word when there is a 10-center handy, ready \& able. Anglo-Saxon is a livelier tongue than Latin, so use Anglo-Saxon words. In this, as in so many matters pertaining to style, one's ear must be one's guide: \textit{gut} is a lustier noun than \textit{intestine}, but the 2 words are not interchangeable, because \textit{gut} is often inappropriate, being too coarse for the context. Never call a stomach a tummy without good reason.

If you admire fancy words, if every sky is \textit{beauteous}, every blonde \textit{curvaceous}, every intelligent child prodigious, if you are tickled by \textit{discombobulate}, you will have a bad time with Reminder 14. What is wrong, you ask, with \textit{beauteous?} No one knows, for sure. There is nothing wrong, really, with any word -- all are good, but some are better than others. A matter of ear, a matter of reading the books that sharpen the ear.

The line between the fancy \& the plain, between the atrocious \& the felicitous, is sometimes alarmingly fine. The opening phrase of the Gettysburg address is close to the line, at least by our standards today, \& Mr. Lincoln, knowingly or unknowingly, was flirting with disaster when he wrote ``4 score \& 7 years ago.'' The President could have got into his sentence with plain ``87'' -- a saving of 2 words \& less of a strain on the listeners' powers of multiplication. But Lincoln's ear must have told him to go ahead with 4 score \& 7. By doing so, he achieved cadence while skirting the edge of fanciness. Suppose he had blundered over the line \& written, ``In the year of our Lord seventeen hundred \& seventy-six.'' His speech would have sustained a heavy blow. Or suppose he had settle for ``87.'' In that case he would have got into his introductory sentence too quickly; the timing would have been bad.

The question of ear is vital. Only the writer whose ear is reliable is in a position to use bad grammar deliberately; this writer knows for sure when a colloquialism is better than formal phrasing \& is able to sustain the work at a level of good taste. So cock your ear. Years ago, students were warned not to end a sentence with a preposition; time, of course, has softened that rigid decree. Not only is the preposition acceptable at the end, sometimes it is more effective in that spot than anywhere else. ``A claw hammer, not an ax, was the tool he murdered her with.'' This is preferable to ``A claw hammer, not an ax, was the tool with which he murdered her.'' Why? Because it sounds more violent, more like murder. A matter of ear.

\& would you write ``The worst tennis player around here is I'' or ``The The worst tennis player around here is me''? The 1st is good grammar, the 2nd is good judgment -- although the \textit{me} might not do in all contexts.

The split infinitive is another trick of rhetoric in which the ear must be quicker than the handbook. Some infinitives seem to improve on being split, just as a stick of round stovewood does. ``I cannot bring myself to really like the fellow.'' The sentence is relaxed, the meaning is clear, the violation is harmless \& scarcely perceptible. Put the other way, the sentence becomes stiff, needlessly formal. A matter of ear.

There are times when the ear not only guides us through difficult situations but also saves us from minor or major embarrassments of prose. The ear, e.g., must decide when to omit \textit{that} from a sentence, when to retain it. ``She knew she could do it'' is preferable to ``She knew that she could do it'' -- simpler \& just as clear. Bu tin many cases the \textit{that} is needed. ``He felt that his big nose, which was sunburned, made him look ridiculous.'' Omit the \textit{that} \& you have ``He felt his big nose $\ldots$'''' -- \cite[p. 93]{Strunk_White2019}

%------------------------------------------------------------------------------%

\subsection{Do not use dialect unless your ear is good.}
``Do not attempt to use dialect unless you are a devoted student of the tongue you hope to reproduce. If you use dialect, be consistent. The reader will become impatient or confused upon finding 2 or more versions of the same word or expression. In dialect it is necessary to spell phonetically, or at least ingeniously, to capture unusual inflections. Take, e.g., the word \textit{once}. It often appears in dialect writing as \textit{oncet}, but \textit{oncet} looks as though it should be pronounced ``onset.'' A better spelling would be \textit{wunst}. But if you write it \textit{oncet} once, write it that way throughout. The best dialect writers, by \& large, are economical of their talents; they use the minimum, not the maximum, of deviation from the norm, thus sparing their readers as well as convincing them.'' -- \cite[p. 94]{Strunk_White2019}

%------------------------------------------------------------------------------%

\subsection{Be clear.}
``Clarity is not the prize in writing, nor it is always the principal mark of a good style. There are occasions when obscurity serves a literary yearning, if not a literary purpose, \& there are writers whose mien is more overcast than clear. But since writing is communication, clarity can only be a virtue. \& although there is no substitute for merit in writing, clarity comes closest to being one. Even to a writer who is being intentionally obscure or wild of tongue we can say, ``be obscure clearly! Be wild of tongue in a way we can understand!'' Even to writers of market letters, telling us (but not telling us) which securities are promising, we can say, ``Be cagey plainly! Be elliptical in a straightforward fashion!''

Clarity, clarity, clarity. When you become hopelessly mired in a sentence, it is best to start fresh; do not try to fight your way through against the terrible odds of syntax. Usually what is wrong is that the construction has become too involved at some point; the sentence needs to be broken apart \& replaced by 2 or more shorter sentences.

Muddiness is not merely a disturber of prose, it is also destroyer of life, of hope: death on the highway caused by a badly worded road sign, heartbreak among lovers caused by a misplaced phrase in a well-intentioned letter, anguish of a traveler expecting to be met at a railroad station \& not being met because of a slipshod telegram. Think of the tragedies that are rooted in ambiguity, \& be clear! When you say something, make sure you have said it. The chances of your having said it are only fair.'' -- \cite[p. 95]{Strunk_White2019}

%------------------------------------------------------------------------------%

\subsection{Do not inject opinion.}
``Unless there is a good reason for its being there, do not inject opinion into a piece of writing. We all have opinions about almost everything, \& the temptation to toss them in is great. To air one's views gratuitously, however, is to imply that the demand for them is brisk, which may not be the case, \& which, in any event, may not be relevant to the discussion. Opinions scattered indiscriminately about leave the mark of egotism on a work. Similarly, to air one's views at an improper time may be in bad taste. If you have received a letter inviting you to speak at the dedication of a new cat hospital, \& you have cats, your reply, declining the invitation, does not necessarily have to cover the full range of your emotions. You must make it clear that you will not attend, but you do not have to let fly at cats. The writer of the letter asked a civil question; attack cats, then, only if you can do so with good humor, good taste, \& in such a way that your answer will be courteous as well as responsive. Since you are out of sympathy with cats, you may quite properly give this as a reason for not appearing at the dedicatory ceremonies of a cat hospital. But bear in mind that your opinion of cats was not sought, only your services as a speaker. Try to keep things straight.'' -- \cite[p. 96]{Strunk_White2019}

%------------------------------------------------------------------------------%

\subsection{Use figures of speech sparingly.}
``The simile is a common device \& a useful one, but similes coming in rapid fire, one right on top of another, are more distracting than illuminating. Readers need time to catch their breath; they can't be expected to compare everything with something else, \& no relief in sight.

When you use metaphor, do not mix it up. I.e., don't start by calling something a swordfish \& end by calling it an hourglass.'' -- \cite[p. 97]{Strunk_White2019}

%------------------------------------------------------------------------------%

\subsection{Do not take shortcuts at the cost of clarity.}
``Do not use initials for the names of organizations or movements unless you are certain the initials will be readily understood. Write things out. Not everyone knows that MADD means Mothers Against Drunk Driving, \& even if everyone did, there are babies being born every minute who will someday encounter the name for the 1st time. They deserve to see the words, not simply the initials. A good rule is to start your article by writing out names in full, \& then, later, when your readers have got their bearings, to shorten them.

Many shortcuts are self-defeating; they waste the reader's time instead of conserving it. There are all sorts of rhetorical stratagems \& devices that attract writers who hope to be pithy, but most of them are simply bothersome. The longest way round is usually the shortest home, \& the one truly reliable shortcut in writing is to choose words that are strong \& surefooted to carry readers on their way.'' -- \cite[p. 98]{Strunk_White2019}

%------------------------------------------------------------------------------%

\subsection{Avoid foreign languages.}
``The writer will occasionally find it convenient or necessary to borrow from other languages. Some writers, however, from sheer exuberance or a desire to show off, sprinkle their work liberally with foreign expressions, with no regard for the reader's comfort. It is a bad habit. Write in English.'' -- \cite[p. 99]{Strunk_White2019}

%------------------------------------------------------------------------------%

\subsection{Prefer the standard to the offbeat.}
``Young writers will be drawn at every turn toward eccentricities in language. They will hear the beat of new vocabularies, the exciting rhythms of special segments of their society, each speaking a language of its own. All of us come under the spell of these unsettling drums; the problem for beginners is to listen to them, learn the words, feel the vibrations, \& not be carried away.

Youths invariably speak to other youths in a tongue of their own devising: they renovate the language with a wild vigor, as they would a basement apartment. By the time this paragraph sees print, \textit{psyched, nerd, ripoff, dude, geek}, \& \textit{funky} will be the words of yesteryear, \& we will be fielding more recent ones that have come bouncing into our speech -- some of them into our dictionary as well. A new word is always up for survival. Many do survive. Others grow stale \& disappear. Most are, at least in their infancy, more approximate to conversation than to composition.

Today, the language of advertising enjoys an enormous circulation. With its deliberate infractions of grammatical rules \& its crossbreeding of the parts of speech, it profoundly influences the tongues \& pens of children \& adults. Your new kitchen range is so revolutionary it \textit{obsoletes} all other ranges. Your counter top is beautiful because it is \textit{accessorized} with gold-plated faucets. Your cigarette tastes good \textit{like} a cigarette should. \&, \textit{like the man says}, you will want to try one. You will also, in all probability, want to try writing that way, using that language. You do so at your peril, for it is the language of mutilation.

Advertisers are quite understandably interested in what they call ``attention getting.'' The man photographed must have lost an eye or grown a pink beard, or he must have 3 arms or be sitting wrong-end-to on a horse. This technique is proper in its place, which is the world of selling, but the young writer had best not adopt the device of mutilation in ordinary composition, whose purpose is to engage, not paralyze, the readers senses. Buy the gold-plated faucets if you will, but do not accessorize your prose. To use the language well, do not begin by hacking it to bits; accept the whole body of it, cherish its classic form, its variety, \& its richness.

Another segment of society that has constructed a language of its own is business. People in business say that toner cartridges are \textit{in short supply}, that they have \textit{updated} the next shipment of these cartridges, \& that they will \textit{finalize} their recommendations at the next meeting of the board. They are speaking a language familiar \& dear to them. Its portentous nouns \& verbs invest ordinary events with high adventure; executives walk among toner cartridges, caparisoned like knights. We should tolerate them -- every person of spirit wants to ride a white horse. The only question is whether business vocabulary is helpful to ordinary prose. Usually, the same ideas can be expressed less formidably, if one makes the effort. A good many of the special words of business seem designed more to express the user's dreams than to express a precise meaning. Not all such words, of course, can be dismissed summarily; indeed, no word in the language can be dismissed offhand by anyone who has a healthy curiosity. \textit{Update} isn't a bad word; in the right setting it is useful. In the wrong setting, though, it is destructive, \& the trouble with adopting coinages too quickly is that they will bedevil one by insinuating themselves where they do not belong. This may sound like rhetorical snobbery, or plain stuffiness; but you will discover, in the course of your work, that the setting of a word is just as restrictive as the setting of a jewel. The general rule here is to prefer the standard. \textit{Finalize}, for instance, is not standard; it is special, \& it is a peculiarly fuzzy \& silly word. Does it mean ``terminate,'' or does it mean ``put into final form''? One can't be sure, really, what it means, \& one gets the impression that the person using it doesn't know, either, \& doesn't want to know.

The special vocabularies of the law, of the military, of government are familiar to most of us. Even the world of criticism has a modest pouch of private words (\textit{luminous, taut}), whose only virtue is that they are exceptionally nimble \& can escape from the garden of meaning over the wall. Of these critical words, Wilcott Gibbs once wrote, ``$\ldots$ they are detached from the language \& inflated like little balloons.'' The young writer should learn to spot them -- words that at 1st glance seem freighted with delicious meaning but that soon burst in air, leaving nothing but a memory of bright sound.

The language is perpetually in flux: it is a living stream, shifting, changing, receiving new strength from a thousand tributaries, losing old forms in the backwaters of time. To suggest that a young writer not swim in the main stream of this turbulence would be foolish indeed, \& such is not the intent of these cautionary remarks. The intent is to suggest that in choosing between the formal \& the informal, the regular \& the offbeat, the general \& the special, the orthodox \& the heretical, the beginner err on the side of conservatism, on the side of established usage. No idiom is taboo, no accent forbidden; there is simply a better chance of doing well if the writer holds a steady course, enters the stream of English quietly, \& does not thrash about.

``But,'' you may ask, ``what if it comes natural to me to experiment rather than conform? What if I am a pioneer, or even a genius?'' Answer: then be one. But do not forget that what may seem like pioneering may be merely evasion, or laziness -- the disinclination to submit to discipline. Writing good standard English is no cinch, \& before you have managed it you will have encountered enough rough country to satisfy even the most adventurous spirit.

Style takes its final shape more from attitudes of mind than from principles of composition, for, as an elderly practitioner once remarked, ``Writing is an act of faith, not a trick of grammar.'' This moral observation would have no place in a rule book were it not that style \textit{is} the writer, \& therefore what you are, rather than what you know, will at last determine your style. If you write, you must believe -- in the truth \& worth of the scrawl, in the ability of the reader to receive \& decode the message. No one can write decently who is distrustful of the reader's intelligence, or whose attitude is patronizing.

Many references have been made in this book to ``the reader,'' who has been much in the news. It is now necessary to warn you that you concern for the reader must be pure: you must sympathize with the reader's plight (most readers are in trouble about half the time) but never seek to know the reader's wants. Your whole duty as a writer is to please \& satisfy yourself, \& the true writer always plays to an audience of one. Start sniffing the air, or glancing at the Trend Machine, \& you are as good as dead, although you may make a nice living.

Full of belief, sustained \& elevated by the power of purpose, armed with the rules of grammar, you are ready to exposure. At this point, you may well pattern yourself on the fully exposed cow of Robert Louis Stevenson's rhyme. This friendly \& commendable animal, you may recall, was ``blown by all the winds that pass\texttt{/}\& wet with all the showers.'' \& so must you as a young writer be. In our modern idiom, we should say that you must get wet all over. Mr. Stevenson, working in a plainer style, said it with felicity, \& suddenly 1 cow, out of so many, received the gift of immortality. Like the steadfast writer, she is at home in the wind \& the rain; \&, thanks to 1 moment of felicity, she will live on \& on \& one.'' -- \cite[pp. 101--103]{Strunk_White2019}

%------------------------------------------------------------------------------%

\section{Afterword}
``Will Strunk \& E. B. White were unique collaborators. Unlike Gilbert \& Sullivan, or Woodward \& Bernstein, they worked separately \& decades apart.

We have no way of knowing whether Prof. Strunk took particular notice of Elwyn Brooks White, a student of his at Cornell University in 1919. Neither teacher nor pupil could have realized that their names would be linked as they now are. Nor could they have imagined that 38 years after they met, White would take this little gem of a textbook that Strunk had written for his students, polish it, expand it, \& transform it into a classic.

E. B. White shared Strunk's sympathy for the reader. To Strunk's do's \& don'ts he added passages about the power of words \& the clear expression of thoughts \& feelings. To the nuts \& bolts of grammar he added a rhetorical dimension.

The editors of this edition have followed in White's footsteps, once again providing fresh examples \& modernizing usage where appropriate. \textit{The Elements of Style} is still a little book, small enough \& important enough to carry in your pocket, as I carry mine. It has helped me to write better. I believe it can do the same for you.

\textsc{Charles Osgood}'' -- \cite[p. 104]{Strunk_White2019}

%------------------------------------------------------------------------------%

\section{Glossary}

\begin{itemize}
	\item \textbf{adjectival modifier:} A word, phrase, or clause that acts as an adjective in qualifying the meaning of a noun or pronoun, \textit{Your} country; a \textit{turn-of-the-century} style; people \textit{who are always late}.
	\item \textbf{adjective:} A word that modifies, quantifies, or otherwise describes a noun or pronoun. \textit{Drizzly} November; midnight \textit{dreary}; \textit{only} requirement.
	\item \textbf{adverb:} A word that modifies or otherwise qualifies a verb, an adjective, or another adverb. Gestures \textit{gracefully}; \textit{exceptionally} quiet engine.
	\item $\ldots$
\end{itemize}

%------------------------------------------------------------------------------%

\printbibliography[heading=bibintoc]
	
\end{document}