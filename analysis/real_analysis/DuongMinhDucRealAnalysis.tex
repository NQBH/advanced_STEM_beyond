\documentclass[a4paper]{article}
\usepackage{longtable,float,hyperref,color,amsmath,amsxtra,amssymb,latexsym,amscd,amsthm,amsfonts,graphicx}
\numberwithin{equation}{section}
\allowdisplaybreaks
\usepackage{fancyhdr}
\pagestyle{fancy}
\fancyhf{}
\fancyhead[RE,LO]{\footnotesize \textsc \leftmark}
\cfoot{\thepage}
\renewcommand{\headrulewidth}{0.5pt}
\setcounter{tocdepth}{3}
\setcounter{secnumdepth}{3}
\usepackage{imakeidx}
\makeindex[columns=2, title=Alphabetical Index, 
           options= -s index.ist]
\title{\Huge Duong Minh Duc, Real Analysis}
\author{\textsc{Nguyen Quan Ba Hong}\footnote{Typer.}\\
{\small Students at Faculty of Math and Computer Science,}\\ 
{\small Ho Chi Minh University of Science, Vietnam} \\
{\small \texttt{email. nguyenquanbahong@gmail.com}}\\
{\small \texttt{blog. \url{www.nguyenquanbahong.com}} 
\footnote{Copyright \copyright\ 2016-2017 by Nguyen Quan Ba Hong, Student at Ho Chi Minh University of Science, Vietnam. This document may be copied freely for the purposes of education and non-commercial research. Visit my site \texttt{\url{www.nguyenquanbahong.com}} to get more.}}}
\begin{document}
\maketitle
\begin{abstract}
I retype and correct some errors in \cite{1}.
\end{abstract}
\newpage
\tableofcontents
\newpage
\section{$L^p$ Spaces}
\textbf{Definition 1.1.} Let $X$ be a nonempty set. Let $\mathfrak{M}$ be a nonempty family  of subsets in $X$ satisfying the following properties.
\begin{align}
\Omega  &\in \mathfrak{M}\\
\Omega \backslash A &\in \mathfrak{M},\hspace{0.2cm}\forall A \in \mathfrak{M}\\
\bigcup\limits_{n = 1}^\infty  {{A_n}}  &\in \mathfrak{M},\hspace{0.2cm}\forall \left\{ {{A_n}} \right\}_{n = 1}^\infty  \subset \mathfrak{M}
\end{align}
then we call $\mathfrak{M}$ a \textit{$\sigma$-algebra} in $X$.\\
\\
\textbf{Definition 1.2.} Let $\mu :\mathfrak{M} \to \left[ {0,\infty } \right]$ be a mapping satisfying the following properties
\begin{enumerate}
\item \textit{Countably Additive.} If $\left\{ {{A_n}} \right\}_{n = 1}^\infty $ is a sequence of disjoint sets in $\mathfrak{M}$ then
\begin{align}
\mu \left( {\bigcup\limits_{n = 1}^\infty  {{A_n}} } \right) = \sum\limits_{n = 1}^\infty  {\mu \left( {{A_n}} \right)} 
\end{align}
\item There exists $B$ in $\mathfrak{M}$ such that $\mu \left(B\right) <\infty$.
\end{enumerate}
Then we call $\mu$ a \textit{positive measure} in $X$.\\
\\
\textbf{Proposition 1.3.} \textit{There exists a $\sigma$-algebra and a positive measure $\mu$ in space $\mathbb{R}^n$ satisfying the following properties}
\begin{enumerate}
\item \textit{All open sets and closed sets in $\mathbb{R}^n$ belong to $\mathfrak{M}$.}
\item \textit{For every cell $\left[ {{a_1},{b_1}} \right] \times  \cdots  \times \left[ {{a_n},{b_n}} \right]$,}
\begin{align}
\mu \left( {\left[ {{a_1},{b_1}} \right] \times  \cdots  \times \left[ {{a_n},{b_n}} \right]} \right) &= \prod\limits_{i = 1}^n {\left( {{b_i} - {a_i}} \right)} \\
\mu \left( {\left( {{a_1},{b_1}} \right) \times  \cdots  \times \left( {{a_n},{b_n}} \right)} \right) &= \prod\limits_{i = 1}^n {\left( {{b_i} - {a_i}} \right)} 
\end{align}
\item \textit{$\mu$ is preserved through a translation transformation}
\begin{align}
\mu \left( {E + a} \right) = \mu \left( E \right),\hspace{0.2cm}\forall E \in \mathfrak{M},a \in {\mathbb{R}^n}
\end{align}
\item \textit{Through a homothetic transformation,}
\begin{align}
\mu \left( {cE} \right) = {\left| c \right|^n}\mu \left( E \right),\hspace{0.2cm}\forall E \in \mathfrak{M},c \in {\mathbb{R}^n}\backslash \left\{ 0 \right\}
\end{align}
\end{enumerate}
\textbf{Definition 1.4.} We call $\mathfrak{M}$ and $\mu$ \textit{Lebesgue $\sigma$-algebra} and \textit{Lebesgue measure} in $\mathbb{R}^n$, respectively.\\

From now on, we use $\mathfrak{M}$ and $\mu$ to denote \textit{Lebesgue $\sigma$-algebra} and \textit{Lebesgue measure} in $\mathbb{R}^n$, respectively.\\
\\
\textbf{Theorem 1.5.} \textit{Let $E$ be a measurable set in $\mathbb{R}^n$ for which $\mu \left(E\right) <\infty$, and $\varepsilon$ be a positive real number. Then there exist a compact set $K$ and an open set $V$ such that}
\begin{align}
K \subset E &\subset V\\
\mu \left( {V\backslash K} \right) &< \varepsilon 
\end{align}
\textbf{Theorem 1.6.} \textit{Let $K$ be a compact set and an open set $V$ in $\mathbb{R}^n$ such that $K\subset V$. Then there exists a continuous function $\varphi :{\mathbb{R}^n} \to \left[ {0,1} \right]$ such that}
\begin{align}
\varphi \left( x \right) = \left\{ {\begin{array}{*{20}{l}}
{1,\hspace{0.2cm}\forall x \in K}\\
{0,\hspace{0.2cm}\forall x \in {\mathbb{R}^n}\backslash V}
\end{array}} \right.
\end{align}
\textbf{Problem 1.7.} \textit{Let $E$ be a measurable set in $\mathbb{R}^n$ for which $\mu \left(E\right) <\infty$, and $\varepsilon$ be a positive real number. Then there exists a continuous function $\varphi :\mathbb{R}^n \to \left[0,1\right]$ such that}
\begin{align}
\mu \left( {\left\{ {x \in {R^n}:{\chi _E}\left( x \right) \ne \varphi \left( x \right)} \right\}} \right) < \varepsilon 
\end{align}
\textsc{Hint.} Use Theorem 1.5 and Theorem 1.6. \hfill $\square$\\
\\
\textbf{Definition 1.8.} Let $\Omega$ be a measurable set in $\mathbb{R}^n$, ${c_1}, \ldots ,{c_m}$ be $m$ real numbers, and ${A_1}, \ldots ,{A_m}$ be $m$ measurable sets contained in $\Omega$. Define
\begin{align}
s\left( x \right) = \sum\limits_{i = 1}^m {{c_i}{\chi _{{A_i}}}\left( x \right)} ,\hspace{0.2cm}\forall x \in \Omega 
\end{align}
Then we call $f$ a \textit{simple function} in $\Omega$.\\
\\
\textbf{Definition 1.9.} Let $\Omega$ be a measurable set in $\mathbb{R}^n$. Let $f$ be a mapping from $\Omega$ into $\left[-\infty,\infty\right]$. We call $f$ a \textit{measurable mapping} in $\Omega$ if ${f^{ - 1}}\left( {\left( {a,\infty } \right]} \right) \in \mathfrak{M}$ for all real number $a$.\\
\\
\textbf{Theorem 1.10.} \textit{Let $f$ be a measurable function in a measurable set $\Omega$. Then there exists a sequence of simple function $\left\{ {{t_n}} \right\}_{n = 1}^\infty $ in $\Omega$ such that}
\begin{align}
0 \le {s_1}\left( x \right) \le {s_2}\left( x \right) \le  \cdots  \le {s_n}\left( x \right) \le f\left( x \right),\hspace{0.2cm}\forall x \in \Omega ,\forall n \in {\mathbb{Z}_ + }
\end{align}
and
\begin{align}
\mathop {\lim }\limits_{n \to \infty } {s_n}\left( x \right) = f\left( x \right),\hspace{0.2cm}\forall x \in \Omega 
\end{align}
\textbf{Definition 1.11.} Let $E\in \mathfrak{M}$. Define
\begin{align}
\int_E {sd\mu }  = \sum\limits_{k = 1}^m {{c_k}\mu \left( {{A_k} \cap E} \right)} 
\end{align}
and call $\int_E {sd\mu } $ the \textit{integral} of $s$ in $E$. This integral can be $\infty$.\\
\\
\textbf{Definition 1.22.} Let $\Omega$ be a measurable set in $\mathbb{R}^n$, let $E\in \mathfrak{M}$, and $f$ be a measurable function from $\Omega$ into $\left[0,\infty\right]$. Define $\mathfrak{F}\left(f\right)$ be the family of all simple functions $s$ in $\Omega$ such that $0\le s\le f$, and define
\begin{align}
\int_E {fd\mu }  = \mathop {\sup }\limits_{s \in \mathfrak{F}\left( f \right)} \int_E {sd\mu } 
\end{align}
We call $\int_E {fd\mu } $ \textit{Lebesgue integral} of $f$ in $E$ with respect to measure $\mu$. This integral of $f$ can be $\infty$.\\
\\
\textbf{Problem 1.23.} \textit{Let $\Omega$ be a measurable set in $\mathbb{R}^n$, $E\in \Omega$, and $f$ be a measurable function from $\Omega$ into $\left[0,\infty\right]$. Suppose that $\mu\left(E\right)=0$. Prove that}
\begin{align}
\int_E {fd\mu }  = 0
\end{align}
\textsc{Hint.} Consider that $f$ is a simple function. \hfill $\square$\\
\\
\textbf{Problem 1.24.} \textit{Let $\Omega$ be a measurable set in $\mathbb{R}^n$ and $f$ be a measurable function from $\Omega$ into $\left[0,\infty\right]$. Suppose}
\begin{align}
\int_\Omega {fd\mu }  < \infty 
\end{align}
\textit{Prove that}
\begin{align}
\mu \left( {\left\{ {x \in \Omega :f\left( x \right) = \infty } \right\}} \right) = 0
\end{align}
\textsc{Hint.} Let $\alpha \in \left(0,\infty\right)$. Define
\begin{align}
B = \left\{ {x \in \Omega :f\left( x \right) \ge \alpha } \right\}
\end{align}
Prove
\begin{align}
\int_\Omega  {fd\mu }  &\ge \int_B {fd\mu } \\
& \ge \int_B {\alpha {\chi _B}d\mu } \\
& = \alpha \mu \left( B \right)
\end{align}
Try it. \hfill $\square$\\
\\
\textbf{Definition 1.25.} Let $\Omega$ be a measurable set in $\mathbb{R}^n$, $E \in \mathfrak{M}$, and $f$ be a real measurable function in $\Omega$. Then $\left| f \right|$ is a function from $\Omega$ into $\left[0,\infty\right)$. Suppose 
\begin{align}
\int_\Omega  {\left| f \right|d\mu }  < \infty 
\end{align}
Define
\begin{align}
{f^ + }\left( x \right) &= \max \left\{ {f\left( x \right),0} \right\}\\
{f^ - }\left( x \right) &= \max \left\{ { - f\left( x \right),0} \right\}
\end{align}
and
\begin{align}
\int_E {fd\mu }  = \int_E {{f^ + }d\mu }  - \int_E {{f^ - }d\mu } 
\end{align}
We call $\int_E {fd\mu } $ the \textit{Lebesgue integral} of $f$ in $E$ with respect to $\mu$. This integral of $f$ is a real number.\\
\\
\textbf{Theorem 1.26 (Monotone Convergence Theorem).} \textit{Let $\Omega$ be a measurable set in $\mathbb{R}^n$ and $\left\{ {{f_n}} \right\}_{n = 1}^\infty $ be a sequence of measurable mappings from $\Omega$ into $\left[0,\infty\right]$, and $f$ be a mapping from $X$ into $\left[0,\infty\right]$. Suppose that}
\begin{align}
{f_1}\left( x \right) \le {f_2}\left( x \right) \le  \cdots  \le {f_n}\left( x \right) \le  \cdots ,\hspace{0.2cm}\forall x \in \Omega 
\end{align}
\textit{and}
\begin{align}
f\left( x \right) = \mathop {\lim }\limits_{n \to \infty } {f_n}\left( x \right),\hspace{0.2cm}\forall x \in \Omega 
\end{align}
\textit{Then}
\begin{align}
\int_X {fd\mu }  = \int_X {\mathop {\lim }\limits_{n \to \infty } {f_n}\left( x \right)d\mu }  = \mathop {\lim }\limits_{n \to \infty } \int_X {{f_n}d\mu } 
\end{align}
\textbf{Lemma 1.27 (Fatou's Lemma).} \textit{Let $\Omega$ be a measurable set in $\mathbb{R}^n$, $E\in \mathfrak{M}$, and $\left\{ {{g_n}} \right\}_{n = 1}^\infty $ be a sequence of measurable mappings from $\Omega$ into $\left[0,\infty\right]$. Then the following inequality holds}
\begin{align}
\int_E {\mathop {\lim \inf }\limits_{n \to \infty } {g_n}d\mu }  \le \mathop {\lim \inf }\limits_{n \to \infty } \int_E {{g_n}d\mu } 
\end{align}
\textbf{Theorem 1.28 (Lebesgue Dominated Convergence Theorem).} \textit{Let $\Omega$ be a measurable set in $\mathbb{R}^n$ and $\left\{ {{f_n}} \right\}_{n = 1}^\infty $  be a sequence of Lebesgue integrable functions in $\Omega$, and $f$ be a function in $X$. Suppose that there exists a Lebesgue integrable function $g$ in $\Omega$ such that}
\begin{align}
\left| {{f_n}\left( x \right)} \right| &\le g\left( x \right),\hspace{0.2cm}\forall x \in \Omega ,\forall n \in {Z_ + }\\
f\left( x \right) &= \mathop {\lim }\limits_{n \to \infty } {f_n}\left( x \right),\hspace{0.2cm}\forall x \in \Omega 
\end{align}
\textit{Then $f$ in Lebesgue integrable in $\Omega$ and}
\begin{align}
\int_\Omega  {fd\mu }  &= \mathop {\lim }\limits_{n \to \infty } \int_\Omega  {{f_n}d\mu } \\
\mathop {\lim }\limits_{n \to \infty } \int_\Omega  {\left| {{f_n} - f} \right|d\mu }  &= 0
\end{align}
\textbf{Problem 1.29.} \textit{Let $\Omega$ be a measurable set in $\mathbb{R}^n$, $f,g$ be two Lebesgue integrable funcitons in $\Omega$. Define}
\begin{align}
B = \left\{ {x \in \Omega :f\left( x \right) \ne g\left( x \right)} \right\}
\end{align}
\textit{Suppose $\mu \left(B\right)=0$. Let $E$ be a measurable set contained in $\Omega$. Prove that}
\begin{align}
\int_E {fd\mu }  = \int_E {gd\mu } 
\end{align}
\textsc{Hint.} Prove
\begin{align}
\int_E {fd\mu }  = \int_{E \cap B} {fd\mu }  + \int_{E\backslash B} {fd\mu } 
\end{align}
Try it. \hfill $\square$\\
\\
\textbf{Problem 1.30.} \textit{Let $\Omega$ be a measurable set in $\mathbb{R}^n$, and $f,g$ be two Lebesgue integrable functions in $\Omega$. Define}
\begin{align}
B = \left\{ {x \in \Omega :f\left( x \right) \ne g\left( x \right)} \right\}
\end{align}
\textit{Suppose that for all measurable set $E$ contained in $\Omega$}
\begin{align}
\int_E {fd\mu }  = \int_E {gd\mu } 
\end{align}
\textit{Prove that $\mu \left(B\right)=0$.}\\
\\
\textsc{Hint.} Define
\begin{align}
{C_n} &= \left\{ {x \in \Omega :f\left( x \right) - g\left( x \right) > \frac{1}{n}} \right\}\\
{D_n} &= \left\{ {x \in \Omega :g\left( x \right) - f\left( x \right) > \frac{1}{n}} \right\}
\end{align}
for all positive integers $n$. Notice that
\begin{align}
B = \left( {\bigcup\limits_{n = 1}^\infty  {{C_n}} } \right) \cup \left( {\bigcup\limits_{n = 1}^\infty  {{D_n}} } \right)
\end{align}
and
\begin{align}
\int_{{C_n}} {\left( {f - g} \right)d\mu }  &\ge \int_{{C_n}} {\frac{1}{n}d\mu } \\
& = \frac{1}{n}\mu \left( {{C_n}} \right)
\end{align}
Try it. \hfill $\square$\\
\\
\textbf{Definition 1.31.} Define $M\left(\Omega\right)$ be the set of all real measurable functions in $\Omega$. Let $f,g$ in $M\left(\Omega\right)$, we denote $f \sim g$ if
\begin{align}
\mu \left( {\left\{ {x \in \Omega :g\left( x \right) - f\left( x \right) \ne 0} \right\}} \right) = 0
\end{align}
\textbf{Problem 1.32.} \textit{Prove the relation $\sim$ is a equivalent relation in $M\left(\Omega\right)$.}\\
\\
\textsc{Hint.} Let $f,g,h$ in $M\left(\Omega\right)$, prove that
\begin{align}
f &\sim f\\
f \sim g &\Rightarrow g \sim f\\
\left( {f \sim g} \right) \wedge \left( {g \sim h} \right) &\Rightarrow f \sim h
\end{align}
\textbf{Definition 1.33.} Suppose that for all $x \in \Omega$, there exists a \textit{property} $P\left(x\right)$. We say that $P$ \textit{holds almost everywhere} (abbr. \textit{a.e.}) in $\Omega$ if 
\begin{align}
\mu \left( {\left\{ {x \in \Omega :P\left( x \right)\mbox{ fails}} \right\}} \right) = 0
\end{align}
\textbf{Definition 1.34.} Let $f \in M\left(\Omega\right)$. Define
\begin{align}
\widetilde f &= \left\{ {g \in M\left( \Omega  \right):g \sim f} \right\}\\
N\left( \Omega  \right) &= \left\{ {\widetilde h:h \in M\left( \Omega  \right)} \right\}
\end{align}
\textbf{Definition 1.35.} Let $\alpha $ be a real number, and $f,g \in M\left(\Omega\right)$. Put $h=f+g$ and $k=\alpha f$ and 
\begin{align}
\label{1.54}
\widetilde f + \widetilde g &= \widetilde h\\
\alpha \widetilde f &= \widetilde k \label{1.55}
\end{align}
\textbf{Problem 1.36.} \textit{Prove that $N\left(\Omega\right)$ is a vector space with defined addition and multiplication operators defined by \eqref{1.54} and \eqref{1.55}.}\\
\\
\textsc{Hint.} Prove that the addition and multiplication operators defined by \eqref{1.54} and \eqref{1.55} are well-defined as follows. Let $f_1,g_1\in M\left(\Omega\right)$. Put $h_1=f_1+g_1$, $k_1=\alpha f_1$. Check that $h \sim {h_1},k \sim {k_1}$. Then check that the defined $+,\cdot$ operators satisfy all laws of a vector space. \hfill $\square$\\
\\
\textbf{Problem 1.37.} \textit{Let $f,g \in M\left(\Omega\right)$ for which}
\begin{align}
f\left( x \right) \le g\left( x \right)\mbox{ a.e. on }\Omega 
\end{align}
\textit{Define}
\begin{align}
A = \left\{ {x \in \Omega :f\left( x \right) > g\left( x \right)} \right\}
\end{align}
and
\begin{align}
u\left( x \right) &= \left\{ {\begin{array}{*{20}{c}}
{f\left( x \right),\hspace{0.2cm}\forall x \in \Omega \backslash A}\\
{0,\hspace{0.2cm}\forall x \in A}
\end{array}} \right.\\
v\left( x \right) &= \left\{ {\begin{array}{*{20}{c}}
{g\left( x \right),\hspace{0.2cm}\forall x \in \Omega \backslash A}\\
{0,\hspace{0.2cm}\forall x \in A}
\end{array}} \right.
\end{align}
\textit{Prove that}
\begin{align}
u\left( x \right) &= f\left( x \right)\mbox{ a.e. on }\Omega \\
v\left( x \right) &= g\left( x \right)\mbox{ a.e. on }\Omega \\
v\left( x \right) &\le u\left( x \right),\hspace{0.2cm}\forall x \in \Omega 
\end{align}
\textbf{Definition 1.38.} Given $u,v \in N\left(\Omega\right)$, $f\in u$ and $g\in v$. We denote $u\le v$ if 
\begin{align}
\mu \left( {\left\{ {x \in \Omega :f\left( x \right) > g\left( x \right)} \right\}} \right) = 0
\end{align}
\textbf{Problem 1.39.} \textit{Prove that $u\le v$ is well-defined.}\\
\\
\textsc{Hint.} Given $h\in u$ and $k\in v$. Prove that
\begin{align}
\mu \left( {\left\{ {x \in \Omega :h\left( x \right) > k\left( x \right)} \right\}} \right) = 0
\end{align}
\textbf{Definition 1.40.} Let $\left\{ {{u_n}} \right\}_{n = 1}^\infty $ be a sequence in $N\left(\Omega\right)$ and $u\in N\left(\Omega\right)$. We say that $\left\{ {{u_n}} \right\}_{n = 1}^\infty $ \textit{converges to} $u$ in $\Omega$ if there exists $f \in u$ and $f_n\in u_n$ such that $\left\{ {{f_n}} \right\}_{n = 1}^\infty $ converges to $f$ a.e. on $\Omega$.\\
\\
\textbf{Problem 1.41.} \textit{Let $\left\{ {{u_n}} \right\}_{n = 1}^\infty $ be a sequence in $N\left(\Omega\right)$ converging to $u$ in $N\left(\Omega\right)$. Let $g \in u$ and $g_n \in u_n$. Prove that $\left\{ {{g_n}} \right\}_{n = 1}^\infty $ converges to $g$ a.e. on $\Omega$.}\\
\\
\textsc{Hint.} Define
\begin{align}
B &= \Omega \backslash \left\{ {x \in \Omega :\mathop {\lim }\limits_{n \to \infty } {f_n}\left( x \right) = f\left( x \right)} \right\}\\
{A_n} &= \left\{ {x \in \Omega :{f_n}\left( x \right) \ne {g_n}\left( x \right)} \right\}\\
A &= \left\{ {x \in \Omega :f\left( x \right) \ne g\left( x \right)} \right\}\\
D &= A \cup B \cup \left( {\bigcup\limits_{n = 1}^\infty  {{A_n}} } \right)
\end{align}
Prove $\mu\left(D\right)=0$ and $\left\{ {{g_n}} \right\}_{n = 0}^\infty $ converges to $g$ in $\Omega \backslash D$.\\
\\
\textbf{Definition 1.42.} Let $p \in \left[1,\infty\right)$, $\widetilde f \in {L^p}\left( \Omega  \right)$ and $g\in \widetilde{f}$. Define
\begin{align}
\int_\Omega  {\widetilde fdx}  = \int_\Omega  {gdx} 
\end{align}
We call $\int_\Omega  {\widetilde fdx} $ the \textit{integral} of $\widetilde{f}$.\\
\\
\textbf{Problem 1.43.} \textit{Prove that $\int_\Omega  {\widetilde fdx} $ is well-defined.}\\
\\
\textbf{Problem 1.44.} \textit{Suppose $\int_\Omega  {\left| {\widetilde f} \right|dx}  = 0$ and $\mu \left(\Omega\right)>0$. Prove that $\widetilde{f}=\widetilde{0}$.}\\
\\
\textsc{Hint.} Given $h\in \tilde{f}$, prove that $h\sim 0$. \hfill $\square$\\
\\
\textbf{Problem 1.45.} \textit{Let $\left\{ {{u_n}} \right\}_{n = 0}^\infty $ be a sequence in $N\left(\Omega\right)$ and $u$ in $N\left(\Omega\right)$.Suppose}
\begin{enumerate}
\item $\left\{ {{u_n}} \right\}_{n = 0}^\infty $ converges to $u$ in $\Omega$.
\item $0 \le {u_1} \le {u_2} \le  \cdots  \le {u_n} \le  \cdots$ \textit{in} $\Omega $.
\end{enumerate}
\textit{Prove that}
\begin{align}
\mathop {\lim }\limits_{n \to \infty } \int_\Omega  {{u_n}dx}  = \int_\Omega  {udx} 
\end{align}
\textsc{Hint.} Let $f_n\in u_n$. Define
\begin{align}
{B_n} = \left\{ {x \in \Omega :{f_n}\left( x \right) > {f_{n + 1}}\left( x \right)} \right\}
\end{align}
and
\begin{align}
B = \bigcup\limits_{n = 1}^\infty  {{B_n}} 
\end{align}
Prove that $\mu \left(B\right) =0$, then apply previous problems. \hfill $\square$\\
\\
\textbf{Definition 1.46.} Let $\left\{ {{u_n}} \right\}_{n = 0}^\infty $ be a sequence in $N\left(\Omega\right)$ and $f_n\in u_n$. Define
\begin{enumerate}
\item $\mathop {\lim \inf }\limits_{n \to \infty } {u_n}$ is the equivalent class of $\mathop {\lim \inf }\limits_{n \to \infty } {f_n}$.
\item $\mathop {\lim \sup }\limits_{n \to \infty } {u_n}$ is the equivalent class of $\mathop {\lim \sup }\limits_{n \to \infty } {f_n}$.
\end{enumerate}
\textbf{Problem 1.47.} \textit{Let $\left\{ {{u_n}} \right\}_{n = 0}^\infty $ be a sequence in $N\left(\Omega\right)$. Suppose $u_n\ge 0$ in $\Omega$ for all positive integers $n$. Prove that}
\begin{align}
\int_\Omega  {\mathop {\lim \inf }\limits_{n \to \infty } {u_n}dx}  \le \mathop {\lim \inf }\limits_{n \to \infty } \int_\Omega  {{u_n}dx} 
\end{align}
\textsc{Hint.} Proceed as Problem 1.45. \hfill $\square$\\
\\
\textbf{Problem 1.48.} \textit{Let $\left\{ {{u_n}} \right\}_{n = 0}^\infty $ be a sequence in $N\left(\Omega\right)$ and $u$ in $N\left(\Omega\right)$. Suppose that there exists a $v$ in $N\left(\Omega\right)$ such that}
\begin{enumerate}
\item $\left\{ {{u_n}} \right\}_{n = 0}^\infty $ converges to $u$ in $\Omega$.
\item $\left| {{u_n}} \right| \le v$ in $\Omega,\hspace{0.2cm} \forall n \in {\mathbb{Z}_ + }$.
\item $\int_\Omega  {vdx}  < \infty $.
\end{enumerate}
\textit{Prove that}
\begin{align}
\mathop {\lim }\limits_{n \to \infty } \int_\Omega  {{u_n}dx}  = \int_\Omega  {udx} 
\end{align}
\textit{and}
\begin{align}
\mathop {\lim }\limits_{n \to \infty } \int_\Omega  {\left| {{u_n} - u} \right|dx}  = 0
\end{align}
\textsc{Hint.} Proceed as Problem 1.45. \hfill $\square$\\
\\
\textbf{Definition 1.49.} Given $p\in \left[1,\infty \right)$. We denote by $L^p\left(\Omega\right)$ the set of all class of function $u$ in $N\left(\Omega\right)$ for which
\begin{align}
\int_\Omega  {{{\left| u \right|}^p}dx}  < \infty 
\end{align}
We define
\begin{align}
{\left\| u \right\|_p} = {\left( {\int_\Omega  {{{\left| u \right|}^p}dx} } \right)^{\frac{1}{p}}},\hspace{0.2cm}\forall u \in {L^p}\left( \Omega  \right)
\end{align}
\textbf{Definition 1.50.} We denote by $L^\infty \left(\Omega\right)$ the set of all class of functions $u$ in $N\left(\Omega\right)$ such that there exist a $f$ in $u$ and a real number $M$ such that
\begin{align}
\mu \left( {\left\{ {u:\left| {f\left( x \right)} \right| > M} \right\}} \right) = 0
\end{align}
We define
\begin{align}
{\left\| u \right\|_\infty } = \inf \left\{ {M \ge 0:\mu \left( {\left\{ {x \in \Omega :\left| {f\left( x \right)} \right| > M} \right\}} \right) = 0} \right\}
\end{align}
for all $u\in L^\infty \left(\Omega\right)$ and $f\in u$.\\
\\
\textbf{Problem 1.51.} \textit{Let $p \in \left[1,\infty\right]$, $u\in L^p\left(\Omega\right)$ and $f\in u$. Prove that}
\begin{align}
\mu \left( {\left\{ {x \in \Omega :f\left( x \right) = \infty } \right\}} \right) = 0
\end{align}
\textbf{Problem 1.52.} \textit{Prove that ${\left\|  \cdot  \right\|_\infty }$ is a norm in $L^{\infty}\left(\Omega\right)$.}\\
\\
\textbf{Theorem 1.53 (H\"{o}lder).} \textit{Let $p,q\in \left(1,\infty\right)$ such that}
\begin{align}
\frac{1}{p} + \frac{1}{q} = 1
\end{align}
\textit{and $f\in L^p\left(\Omega\right)$ and $g\in L^q\left(\Omega\right)$. Then the following inequality holds}
\begin{align}
\left| {\int_\Omega  {fgdx} } \right| \le {\left\| f \right\|_{{L^p}\left( \Omega  \right)}}{\left\| g \right\|_{{L^q}\left( \Omega  \right)}}
\end{align}
\textbf{Theorem 1.54 (Minkowski).} \textit{Let $p \in \left[1,\infty\right)$, $f,g \in L^p\left(E\right)$. Then}
\begin{align}
{\left\| {f + g} \right\|_{{L^p}\left( \Omega  \right)}} \le {\left\| f \right\|_{{L^p}\left( \Omega  \right)}} + {\left\| g \right\|_{{L^p}\left( \Omega  \right)}}
\end{align}
\textbf{Problem 1.55.} \textit{Given $p\in \left[1,\infty\right)$. Prove that $\left( {{L^p}\left( \Omega  \right),{{\left\|  \cdot  \right\|}_p}} \right)$ is a normed space.}\\
\\
\textbf{Problem 1.56.} \textit{Given $p\in \left[1,\infty\right)$ and $\left\{ {{u_n}} \right\}_{n = 1}^\infty$ in $L^p\left(\Omega\right)$, suppose that $\sum\limits_{n = 1}^\infty  {{u_n}} $ converges to $u$ in $L^p\left(\Omega\right)$. Prove that}
\begin{align}
{\left\| u \right\|_{{L^p}\left( \Omega  \right)}} \le \sum\limits_{n = 1}^\infty  {{{\left\| {{u_n}} \right\|}_{{L^p}\left( \Omega  \right)}}} 
\end{align}
\textsc{Hint.} Prove
\begin{align}
{\left\| {\sum\limits_{n = 1}^m {{u_n}} } \right\|_{{L^p}\left( \Omega  \right)}} \le \sum\limits_{n = 1}^m {{{\left\| {{u_n}} \right\|}_{{L^p}\left( \Omega  \right)}}} 
\end{align}
for all positive integers $m$. \hfill $\square$\\
\\
\textbf{Problem 1.57.} \textit{Let $p\in \left[1,\infty\right)$ and $\left\{ {{v_n}} \right\}_{n = 1}^\infty $ be a sequence in $L^p\left(\Omega\right)$ and $v\in L^p\left(\Omega\right)$. Suppose that there exist $g\in L^p\left(\Omega\right)$, $f\in u$ and $f_n\in u_n$ for all positive integers $n$, such that}
\begin{align}
\mathop {\lim }\limits_{n \to \infty } {f_n}\left( x \right) &= f\left( x \right),\hspace{0.2cm}\forall x \in \Omega \\
\left| {{f_n}\left( x \right)} \right| &\le g\left( x \right)
\end{align}
\textit{Prove that}
\begin{align}
\mathop {\lim }\limits_{n \to \infty } \int_\Omega  {{{\left| {{u_n} - u} \right|}^p}dx}  = 0
\end{align}
\textsc{Hint.} Use Lebesgue dominated convergence theorem to prove 
\begin{align}
\mathop {\lim }\limits_{n \to \infty } \int_\Omega  {{{\left| {{f_n} - f} \right|}^p}dx}  = 0
\end{align}
Try it. \hfill $\square$\\
\\
\textbf{Problem 1.58.} \textit{Let $p \in \left[1,\infty\right)$ and $\left\{ {{u_n}} \right\}_{n = 1}^\infty $ be a sequence in $L^p\left(\Omega\right)$. Suppose that}
\begin{align}
\sum\limits_{n = 1}^\infty  {{{\left\| {{u_n}} \right\|}_{{L^p}\left( \Omega  \right)}}}  \le 1
\end{align}
\textit{Let $f_n\in u_n$ for all positive integers $n$. Prove that $\sum\limits_{n = 1}^\infty  {{f_n}\left( x \right)}$ converges a.e. in $\Omega$, and there exists a $v\in L^p\left(\Omega\right)$ such that}
\begin{align}
\left| {\sum\limits_{n = 1}^m {{u_n}\left( x \right)} } \right| \le v,\hspace{0.2cm}\forall m \in {\mathbb{Z}_ + }
\end{align}
\textsc{Hint.} Define
\begin{align}
{g_m}\left( x \right) &= \sum\limits_{n = 1}^m {\left| {{f_n}\left( x \right)} \right|} \\
g\left( x \right) &= \sum\limits_{n = 1}^\infty  {\left| {{f_n}\left( x \right)} \right|} 
\end{align}
for all $x\in \Omega$.

Use Minkowski theorem to prove ${\left\| {{g_m}} \right\|_{{L^p}\left( \Omega  \right)}} \le 1$.

Notice that 
\begin{align}
g &= \mathop {\lim \inf }\limits_{m \to \infty } {g_m}\\
{g^p} &= \mathop {\lim \inf }\limits_{m \to \infty } g_m^p
\end{align}
Apply Fatou's lemma to prove ${\left\| g \right\|_{{L^p}\left( \Omega  \right)}} < \infty $.

Prove that $g\left(x\right)$ is a real number a.e. on $\Omega$. \hfill $\square$\\
\\
\textbf{Problem 1.59.} \textit{Let $p\in \left[1,\infty\right)$ and $\left\{ {{v_n}} \right\}_{n = 1}^\infty $ be a sequence converging to $v$ in $L^p\left(\Omega\right)$. Given $w_n\in v_n$ and $w\in v$. Prove that there exists subsequence $\left\{ {{w_{{n_k}}}} \right\}_{k = 1}^\infty $ of the sequence $\left\{ {{w_n}} \right\}_{n = 1}^\infty $ and $h$ such that}
\begin{align}
{w_{{n_k}}} &\to w\mbox{ as } k \to \infty \mbox{ a.e. in } \Omega \\
\int_\Omega  {{h^p}dx}  &< \infty \\
\left| {{w_{{n_k}}}\left( x \right)} \right| &\le h\left( x \right)\mbox{ a.e. in }\Omega 
\end{align}
\textsc{Hint.} Choose a subsequence $\left\{ {{w_{{n_k}}}} \right\}_{k = 1}^\infty $ of the sequence $\left\{ {{w_n}} \right\}_{n = 1}^\infty $ such that
\begin{align}
{\left\| {{w_{{n_{k + 1}}}} - {w_{{n_k}}}} \right\|_{{L^p}\left( \Omega  \right)}} \le \frac{1}{{{2^k}}},\hspace{0.2cm}\forall k \in {\mathbb{Z}_ + }
\end{align}
Apply Problem 1.57 for 
\begin{align}
{u_k} &= {v_{{n_{k + 1}}}} - {v_{{n_k}}}\\
{f_k} &= {w_{{n_{k + 1}}}} - {w_{{n_k}}}
\end{align}
Define $h = g + \left| {{w_{{n_1}}}} \right|$. Notice that
\begin{align}
{w_{{n_{k + 1}}}} = {w_{{n_1}}} + \sum\limits_{j = 1}^k {{f_j}} 
\end{align}
converges a.e. to $w$ in $\Omega$. Use Problem 1.57, prove
\begin{align}
\mathop {\lim }\limits_{k \to \infty } {\left\| {{w_{{n_k}}} - w} \right\|_{{L^p}\left( \Omega  \right)}} = 0
\end{align}
to deduce ${\left\| {w - z} \right\|_{{L^p}\left( \Omega  \right)}} = 0$. \hfill $\square$\\
\\
\textbf{Problem 1.60.} \textit{Given $p\in \left[1,\infty\right)$. Prove that $L^p\left(\Omega\right)$ is a Banach space.}\\
\\
\textsc{Hint.} Let $\left\{ {{v_n}} \right\}_{n = 1}^\infty $ be a Cauchy sequence in $L^p\left(\Omega\right)$. Choose a subsequence $\left\{ {{v_{{n_k}}}} \right\}_{k = 1}^\infty $ such that
\begin{align}
{\left\| {{v_{{n_{k + 1}}}} - {v_{{n_k}}}} \right\|_{{L^p}\left( \Omega  \right)}} \le \frac{1}{{{2^k}}},\forall k \in {\mathbb{Z}_ + }
\end{align}
\textsc{Hint.} Proceed as Problem 1.59. \hfill $\square$\\
\\
\textbf{Problem 1.61.} \textit{Prove that $L^\infty \left(\Omega\right)$ is a Banach space.}\\
\\
\textbf{Problem 1.62.} \textit{Prove that $L^2\left(\Omega\right)$ is a Hilbert space with the following scalar product}
\begin{align}
\left\langle {u,v} \right\rangle  = \int_\Omega  {uvdx} ,\hspace{0.2cm}\forall u,v \in {L^2}\left( \Omega  \right)
\end{align}
\textbf{Problem 1.63.} \textit{Given $p\in \left[1,\infty\right)$. Define}
\begin{align}
S = \left\{ {u \in {L^p}\left( \Omega  \right):\mbox{ there exists a simple functions }\in u} \right\}
\end{align}
\textit{Prove that $S$ is dense in $L^p\left(\Omega\right)$.}\\
\\
\textsc{Hint.} Given $v\in L^p\left(\Omega\right)$, prove that there exists a sequence $\left\{ {{s_n}} \right\}_{n = 1}^\infty $ in $S$ converging to $v$ in $L^p\left(\Omega\right)$. Consider the case $v\ge 0$. Use Theorem 1.10 and Problem 1.57. \hfill $\square$\\
\\
\textbf{Problem 1.64.} \textit{Given $p\in \left[1,\infty\right)$. Define}
\begin{align}
C = \left\{ {u \in {L^p}\left( \Omega  \right):\mbox{ there exists a continuous function } f \in u} \right\}
\end{align}
\textit{Prove that $C$ is dense in $L^p\left(\Omega\right)$.}\\
\\
\textsc{Hint.} Given $v\in S$. Use Problem 1.7 to prove that there exists a sequence $\left\{ {{f_n}} \right\}_{n = 1}^\infty $ in $C$ which converges to $v$ in $L^p\left(\Omega\right)$.\\
\\
\textbf{Definition 1.65.} Denote by $C_c\left(\mathbb{R}^n\right)$ the set of all real continuous function $f$ in $\mathbb{R}^n$ such that there exists a compact set $K_f$ containing the set
\begin{align}
\left\{ {x \in {\mathbb{R}^n}:f\left( x \right) \ne 0} \right\}
\end{align}
\textbf{Problem 1.66.} \textit{Given $p\in \left[1,\infty\right)$. Define}
\begin{align}
{C_c} = C \cap {C_c}\left( {{\mathbb{R}^n}} \right)
\end{align}
\textit{Prove that $C_c$ is dense in $L^p\left(\Omega\right)$.}\\
\\
\textsc{Hint.} Use Problem 1.7 to prove there exists a real continuous function $g_m$ from $\mathbb{R}^n$ into $\left[0,1\right]$ such that
\begin{align}
{g_m}\left( x \right) = \left\{ {\begin{array}{*{20}{l}}
{1,\mbox{ if }\left\| x \right\| \le m}\\
{0,\mbox{ if }\left\| x \right\| > m + 1}
\end{array}} \right.
\end{align}
Given $f\in C$. Define 
\begin{align}
f_m=g_mf
\end{align}
Apply Problem 1.57. \hfill $\square$\\
\\
\textbf{Theorem 1.67.} \textit{Let $p\in \left[1,\infty\right)$, $q\in \left(1,\infty\right]$ such that}
\begin{align}
\frac{1}{p} + \frac{1}{q} = 1
\end{align}
\textit{and $T$ be a continuous linear mapping from $L^p\left(\Omega\right)$ into $\mathbb{R}$. Then there exists a unique $u\in L^q\left(\Omega\right)$ for which}
\begin{align}
T\left( v \right) &= \int_\Omega  {uvdx} ,\hspace{0.2cm}\forall v \in {L^p}\left( \Omega  \right)\\
\left\| T \right\| &= {\left\| u \right\|_q}
\end{align}
\textbf{Theorem 1.68.} \textit{Let $p\in \left(1,\infty\right)$, $q\in \left(1,\infty\right)$ such that}
\begin{align}
\frac{1}{p} + \frac{1}{q} = 1
\end{align}
\textit{and $\left\{ {{u_n}} \right\}_{n = 1}^\infty $ be a bounded sequence in $L^p\left(\Omega\right)$. Then there exist a $u\in L^p\left(\Omega\right)$ and a subsequence $\left\{ {{u_{{n_k}}}} \right\}_{k = 1}^\infty $ of $\left\{ {{u_n}} \right\}_{n = 1}^\infty $ such that}
\begin{align}
\mathop {\lim }\limits_{k \to \infty } \int_\Omega  {{u_{{n_k}}}vdx}  = \int_\Omega  {uvdx} ,\hspace{0.2cm}\forall v \in {L^q}\left( \Omega  \right)
\end{align}
\textbf{Problem 1.69.} \textit{Let $u \in {L^1}\left( {{\mathbb{R}^n}} \right)$ and $\alpha$ be a nonzero real number. Given $f\in u$, define $g\left( x \right) = f\left( {\alpha x} \right)$ for all $x\in \mathbb{R}^n$. Prove that $g$ is Lebesgue integrable and}
\begin{align}
\int_{{\mathbb{R}^n}} {gd\mu }  = \frac{1}{{{{\left| \alpha  \right|}^n}}}\int_{{\mathbb{R}^n}} {fd\mu } 
\end{align}
\textsc{Hint.} Let $E$ be a measurable set in $\mathbb{R}^n$ such that $\mu\left(E\right)<\infty$. Consider the case that $f$ is the characteristic function of $E$. Prove that $g$ is the characteristic function of $\frac{1}{\alpha}E$. \hfill $\square$\\
\\
\textbf{Problem 1.70.} \textit{Let $u \in {L^1}\left( {{\mathbb{R}^n}} \right)$ and $a$ be a vector in $\mathbb{R}^n$. Given $f\in u$, define $g\left(y\right)=f\left(a-y\right)$ for all $y\in \mathbb{R}^n$. Prove that $g$ is Lebesgue integrable and}
\begin{align}
\int_{{\mathbb{R}^n}} {gd\mu }  = \int_{{\mathbb{R}^n}} {fd\mu } 
\end{align}
\textsc{Hint.} Let $E$ be a measurable set in $\mathbb{R}^n$ such that $\mu\left(E\right)<\infty$. Consider the case that $f$ is the characteristic function of $E$. Prove that $g$ is the characteristic function of $\left(a-E\right)$. \hfill $\square$\\
\\
\textbf{Problem 1.71.} \textit{Let $u\in L^1\left(\mathbb{R}^n\right)$ and $\varepsilon$ be a positive real number. Prove that there exists a positive real number $\delta$ such that for all measurable set $E$ satisfying $\mu\left(E\right)<\delta$, the following inequality holds}
\begin{align}
\int_E {\left| u \right|d\mu }  < \varepsilon 
\end{align}
\textsc{Hint.} Consider the following cases: $f$ is the characteristic function of $E$, $f$ is a simple function, $f$ is a nonnegative function. \hfill $\square$\\
\\
\textbf{Problem 1.72.} \textit{Let $E$ be a measurable set in $\mathbb{R}^n$ satisfying $\mu\left(E\right)<\infty$, and $r,s\in \left[1,\infty\right)$ such that $r<s$. Prove that $L^s\left(E\right)<L^r\left(E\right)$.}\\
\\
\textsc{Hint.} Define 
\begin{align}
p &= \frac{s}{r}\\
q &= \frac{1}{{1 - \frac{r}{s}}}
\end{align}
Let $u\in L^s\left(E\right)$, $f\in u$ and $g$ is the characteristic function of $E$. Applying H\"{o}lder inequality for ${\left| f \right|^r}$ and $g$. \hfill $\square$\\
\\
\textbf{Problem 1.73.} \textit{Let $E$ be a measurable set in $\mathbb{R}^n$, $p\in \left[1,\infty\right)$, $u\in L^p\left(E\right)$ and $f\in u$. Suppose}
\begin{align}
\label{1.122}
\int_E {fgd\mu }  = 0,\hspace{0.2cm}\forall g \in {C_c}\left( {{\mathbb{R}^n}} \right)
\end{align}
\textit{Prove that $f=0$ a.e. in $E$.}\\
\\
\textsc{Hint.} Use Problem 1.66 and H\"{o}lder inequality, prove \eqref{1.122} holds when $g$ is the characteristic function of a measurable set $f$ satisfying $\mu\left(F\right)<\infty$\\
\\
\textbf{Problem 1.74.} \textit{Let $E$ be a measurable set in $\mathbb{R}^n$, $p\in \left[1,\infty\right]$, $u\in L^p\left(E\right)$ and $f\in u$. Define}
\begin{align}
g\left( x \right) = \left\{ {\begin{array}{*{20}{c}}
{f\left( x \right),\mbox{ if }x \in E}\\
{0,\mbox{ if }x \in {\mathbb{R}^n}\backslash E}
\end{array}} \right.
\end{align}
Prove that ${\left| g \right|^p}$ is Lebesgue integrable. \hfill $\square$
\section{Convolution Product}
We will identify $\mathbb{R}^{m+n}$ with $\mathbb{R}^m\times \mathbb{R}^n$. Let $A$ be a subset in $\mathbb{R}^{m+n}$ and $f$ be a real function in $A$. For each $x\in \mathbb{R}^m$ and $y\in \mathbb{R}^n$, we define
\begin{align}
{A_x} &= \left\{ {y:\left( {x,y} \right) \in A} \right\}\\
{A^y} &= \left\{ {x:\left( {x,y} \right) \in A} \right\}\\
{f_x}\left( y \right) &= f\left( {x,y} \right),\mbox{ if }y \in {A_x}\\
{f^y}\left( x \right) &= f\left( {x,y} \right),\mbox{ if } x \in {A^y}
\end{align}
\textbf{Theorem 2.1.} \textit{Let $f$ be a Lebesgue measurable real function in $\mathbb{R}^{m+n}$. Then there exists measurable sets $A$ and $B$, which have zero measure in $\mathbb{R}^m$ and $\mathbb{R}^n$, such that}
\begin{enumerate}
\item \textit{$f_x$ is a measurable function in $\mathbb{R}^n$ for all $x\in \mathbb{R}^m\backslash A$.}
\item \textit{$f^y$ is a measurable function in $\mathbb{R}^m$ for all $y\in \mathbb{R}^n\backslash B$.}
\end{enumerate}
\textbf{Theorem 2.2 (Fubini).} \textit{Let $f$ be a nonnegative measurable function in $\mathbb{R}^{m+n}$. Then}
\begin{align}
\int_{{\mathbb{R}^{m + n}}} {fd{\mu _{m + n}}} & = \int_{{\mathbb{R}^m}} {\left( {\int_{{\mathbb{R}^n}} {{f_x}d{\mu _n}} } \right)d{\mu _m}} \\
& = \int_{{\mathbb{R}^n}} {\left( {\int_{{\mathbb{R}^m}} {{f^y}d{\mu _m}} } \right)d{\mu _n}} 
\end{align}
\textbf{Theorem 2.3 (Tonelli).} \textit{Let $g$ be a measurable real function in $\mathbb{R}^{m+n}$ such that}
\begin{align}
\int_{{\mathbb{R}^m}} {\left( {\int_{{\mathbb{R}^n}} {{{\left| g \right|}_x}d{\mu _n}} } \right)d{\mu _m}}  < \infty 
\end{align}
\textit{Then $g$ is Lebesgue integrable in $\mathbb{R}^{m+n}$.}\\
\\
The converse of Theorem 2.3 is only holds as follows.\\
\\
\textbf{Theorem 2.4 (Fubini).} \textit{Let $g$ be a Lebesgue integrable function in $\mathbb{R}^{m+n}$. Then}
\begin{enumerate}
\item $g_x$ is Lebesgue integrable in $\mathbb{R}^n$ for almost $x\in \mathbb{R}^n$.
\item $g_y$ is Lebesgue integrable in $\mathbb{R}^m$ for almost $y \in \mathbb{R}^n$.
\item The following equality holds
\begin{align}
\int_{{\mathbb{R}^{m + n}}} {gd{\mu _{m + n}}}  = \int_{{\mathbb{R}^m}} {\left( {\int_{{R^n}} {{g_x}d{\mu _n}} } \right)d{\mu _m}}  = \int_{{\mathbb{R}^n}} {\left( {\int_{{\mathbb{R}^m}} {{g^y}d{\mu _m}} } \right)d{\mu _n}} 
\end{align}
\end{enumerate}
\textbf{Problem 2.5.} \textit{Let $h$ be a measurable function in $\mathbb{R}^n$. Define $k\left(x,y\right)=h\left(y\right)$ for all $\left(x,y\right) \in \mathbb{R}^{n+n}$. Prove that $k$ is measurable in $\mathbb{R}^{n+n}$.}\\
\\
\textsc{Hint.} Use definitions. \hfill $\square$\\
\\
\textbf{Problem 2.6.} \textit{Let $h$ be a measurable function in $\mathbb{R}^n$. Define $k\left(x,y\right)=h\left(x-y\right)$ for all $\left(x,y\right) \in \mathbb{R}^{n+n}$. Prove that $k$ is measurable in $\mathbb{R}^{n+n}$.}\\
\\
\textsc{Hint.} Consider the case that $h$ is a continuous function. Then, use Problem 1.7. \hfill $\square$\\
\\
\textbf{Problem 2.7.} \textit{Let $f,g$ be two Lebesgue integrable functions in $\mathbb{R}^n$. Define $k\left(x,y\right)=f\left(y\right)g\left(x-y\right)$ for all $\left(x,y\right) \in \mathbb{R}^{n+n}$. Prove that $k$ is Lebesgue integrable in $\mathbb{R}^{n+n}$.}\\
\\
\textsc{Hint.} Prove that $k$ is measurable in $\mathbb{R}^{n+n}$. Use Fubini theorem to prove
\begin{align}
\int_{{\mathbb{R}^{n + n}}} {\left| {k\left( z \right)} \right|dz}  &= \int_{{\mathbb{R}^n}} {\left| {f\left( y \right)} \right|\left( {\int_{{\mathbb{R}^n}} {\left| {g\left( {x - y} \right)} \right|dx} } \right)dy} \\
 &= \left( {\int_{{\mathbb{R}^n}} {\left| {g\left( t \right)} \right|dt} } \right)\left( {\int_{{\mathbb{R}^n}} {\left| {f\left( y \right)} \right|dy} } \right)
\end{align}
\textbf{Problem 2.8.} \textit{Let $f,g$ be two Lebesgue integrable functions in $\mathbb{R}^n$. Prove that there exists a set $A$ such that $\mu \left(A\right)=0$ and the following integral is defined for all $x \in \mathbb{R}^n\backslash A$.}
\begin{align}
\int_{{\mathbb{R}^n}} {f\left( y \right)g\left( {x - y} \right)dy} ,\hspace{0.2cm}\forall x \in {\mathbb{R}^n}
\end{align}
\textsc{Hint.} Use Problem 1.24 and Fubini theorem. \hfill $\square$\\
\\
\textbf{Definition 2.9.} \textit{Let $f,g$ be two Lebesgue integrable functions in $\mathbb{R}^n$. The convolution product of $f$ and $g$ is a function defined by}
\begin{align}
f\star g\left( x \right) = \int_{{\mathbb{R}^n}} {f\left( y \right)g\left( {x - y} \right)dy} ,\hspace{0.2cm}\forall x \in {\mathbb{R}^n}
\end{align}
Then $f\star g$ is Lebesgue integrable and
\begin{align}
{\left\| {f\star g} \right\|_{{L^1}\left( {{\mathbb{R}^n}} \right)}} \le {\left\| f \right\|_{{L^1}\left( {{\mathbb{R}^n}} \right)}}{\left\| g \right\|_{{L^1}\left( {{\mathbb{R}^n}} \right)}}
\end{align}
\textbf{Definition 2.10.} Let $u,v\in L^1\left(\mathbb{R}^n\right)$. Given $f\in u$ and $g\in v$. We call the equivalent class of $f\star g$ the convolution product of $u$ and $v$, which is denoted by $u\star v$. Then 
\begin{align}
{\left\| {u\star v} \right\|_{{L^1}\left( {{\mathbb{R}^n}} \right)}} \le {\left\| u \right\|_{{L^1}\left( {{\mathbb{R}^n}} \right)}}{\left\| v \right\|_{{L^1}\left( {{\mathbb{R}^n}} \right)}}
\end{align}
\textbf{Problem 2.11.} \textit{Given $u,v\in L^1\left(\mathbb{R}^n\right)$. Prove that}
\begin{align}
u\star v=v\star u
\end{align}
\textsc{Hint.} Use Problem 1.70. \hfill $\square$\\
\\
\textbf{Problem 2.12.} \textit{Let $p\in \left(1,\infty\right)$, $u\in L^1\left(\mathbb{R}^n\right)$, and $v\in L^p\left(\mathbb{R}^n\right)$. Given $f\in u, g\in v$. Prove that $f\star g$ is defined a.e. and ${\left| {f\star g} \right|^p}$ is Lebesgue integrable in $\mathbb{R}^n$.}\\
\\
\textsc{Hint.} Consider $\left| f \right|\star {\left| g \right|^p}$. Define $q = \frac{p}{{p - 1}}$. Use H\"{o}lder inequality to prove
\begin{align}
&\int_{{\mathbb{R}^n}} {{{\left( {\int_{{\mathbb{R}^n}} {{{\left| {f\left( y \right)} \right|}^{\frac{1}{q}}}{{\left| {f\left( y \right)} \right|}^{\frac{1}{p}}}\left| {g\left( {x - y} \right)} \right|dy} } \right)}^p}dx} \\
& \le \int_{{\mathbb{R}^n}} {\left( {{{\left( {\int_{{\mathbb{R}^n}} {\left| {f\left( y \right)} \right|dy} } \right)}^{\frac{p}{q}}}\left( {\int_{{\mathbb{R}^n}} {\left| {f\left( y \right)} \right|{{\left| {g\left( {x - y} \right)} \right|}^p}dy} } \right)} \right)dx} \\
& \le {\left( {\int_{{\mathbb{R}^n}} {\left| {f\left( y \right)} \right|dy} } \right)^{\frac{p}{q}}}\int_{{\mathbb{R}^n}} {\left( {\int_{{R^n}} {\left| {f\left( y \right)} \right|{{\left| {g\left( {x - y} \right)} \right|}^p}dy} } \right)dx} \\
& \le {\left( {\int_{{\mathbb{R}^n}} {\left| {f\left( y \right)} \right|dy} } \right)^{\frac{p}{q}}}\left( {\int_{{\mathbb{R}^n}} {\left| {f\left( y \right)} \right|dy} } \right)\left( {\int_{{\mathbb{R}^n}} {{{\left| {g\left( y \right)} \right|}^p}dy} } \right)\\
 &\le {\left( {\int_{{\mathbb{R}^n}} {\left| {f\left( y \right)} \right|dy} } \right)^p}\left( {\int_{{\mathbb{R}^n}} {{{\left| {g\left( y \right)} \right|}^p}dy} } \right)\\
 &= \left\| f \right\|_{{L^1}\left( {{\mathbb{R}^n}} \right)}^p\left\| g \right\|_{{L^p}\left( {{\mathbb{R}^n}} \right)}^p
\end{align}
\textbf{Definition 2.13.} Given $p\in \left(1,\infty\right)$, $u\in L^1\left(\mathbb{R}^n\right)$, $v\in L^p\left(\mathbb{R}^n\right)$, $f\in u$ and $g\in v$. We call the equivalent class of $f\star g$ the \textit{convolution product} of $u$ and $v$, which is denoted by $u\star v$. The following inequality holds
\begin{align}
{\left\| {u\star v} \right\|_{{L^p}\left( {{\mathbb{R}^n}} \right)}} \le {\left\| u \right\|_{{L^1}\left( {{\mathbb{R}^n}} \right)}}{\left\| v \right\|_{{L^p}\left( {{\mathbb{R}^n}} \right)}}
\end{align}

Given $s = \left( {{s_1}, \ldots ,{s_n}} \right)$, we define
\begin{align}
\left| s \right| = \sum\limits_{i = 1}^n {\left| {{s_i}} \right|} 
\end{align}
and denote partial derivative $\frac{{{\partial ^s}f}}{{\partial x}}$ by $\frac{{{\partial ^{\left| s \right|}}f}}{{{\partial ^{{s_1}}}{x_1} \cdots {\partial ^{{s_n}}}{x_n}}}$. \\
\\
\textbf{Definition 2.14.} Let $r$ be a positive integer and $\Omega$ be an open set in $\mathbb{R}^n$. Define $C^r\left(\Omega\right)$ the set of real function in $\Omega$ such that all partial derivatives of order $s$ of $f$ exist and are continuous in $\Omega$ if $\left| s \right| \le r$.\\
\\
\textbf{Definition 2.15.} Let $r$ be a positive integer and $\Omega$ be an open set in $\mathbb{R}^n$. Define $C_c^r\left( \Omega  \right)$ is the set of all functions $f \in C^r\left(\Omega\right)$ such that there exists a compact set $K_f$ for which $f\left(x\right)=0$ for all $x \in {\mathbb{R}^n}\backslash {K_f}$. Then we define
\begin{align}
{C^\infty }\left( \Omega  \right) &= \bigcap\limits_{r = 1}^\infty  {{C^r}\left( \Omega  \right)} \\
C_c^\infty \left( \Omega  \right) &= \bigcap\limits_{r = 1}^\infty  {C_c^r\left( \Omega  \right)} 
\end{align}
\textbf{Problem 2.16.} \textit{Given $p \in \left[1,\infty\right)$, $u\in L^p\left(\mathbb{R}^n\right)$, $f\in u$ and $g\in C_c^r\left(\mathbb{R}^n\right)$. Prove $f\star g \in C^r\left(\mathbb{R}^n\right)$ and}
\begin{align}
\frac{{{\partial ^s}\left( {f\star g} \right)}}{{\partial x}} = f\star \frac{{{\partial ^s}g}}{{\partial x}},\hspace{0.2cm}\forall s,\left| s \right| \le r
\end{align}
\textsc{Hint.} Choose a compact set $K \subset \mathbb{R}^n$ such that $g\left(x\right)=0$ for all $x\in \mathbb{R}^n\backslash K$. Choose a positive real number $r_0$ such that $K\subset B\left(0,r_0\right)$. Put $e:=\left(1,0,\ldots,0\right)\in \mathbb{R}^n$. Let $t \in \left( { - 1,1} \right)\backslash \left\{ 0 \right\}$. Prove that
\begin{align}
g\left( {y + te} \right) - g\left( y \right) = 0,\hspace{0.2cm}\forall y \in {\mathbb{R}^n}\backslash B\left( {0,{r_0} + 2} \right)
\end{align}
Prove
\begin{align}
\frac{{f\star g\left( {x + te} \right) - f\star g\left( x \right)}}{t} = \int_{B\left( {0,r + 2} \right)} {f\left( y \right)\frac{{g\left( {x + te - y} \right) - g\left( {x - y} \right)}}{t}dy} 
\end{align}
Then use Lebesgue dominated convergence theorem to prove
\begin{align}
\mathop {\lim }\limits_{t \to 0} \frac{{f\star g\left( {x + te} \right) - f\star g\left( x \right)}}{t} &= \int_{B\left( {0,r + 2} \right)} {f\left( y \right)\frac{{\partial g}}{{\partial {x_1}}}\left( {x - y} \right)dy} \\
 &= \int_{{\mathbb{R}^n}} {f\left( y \right)\frac{{\partial g}}{{\partial {x_1}}}\left( {x - y} \right)dy} \\
 &= f\star \frac{{\partial g}}{{\partial {x_1}}}\left( x \right)
\end{align}
\textbf{Problem 2.17.} \textit{Prove that there exists a function $\rho \in C_c^\infty \left(\mathbb{R}^n\right)$ satisfying the following property}
\begin{align}
\rho \left( x \right) &\ge 0,\hspace{0.2cm}\forall x \in {\mathbb{R}^n}\\
\rho \left( x \right) &= 0,\hspace{0.2cm}\forall x \in {\mathbb{R}^n}\backslash B\left( {0,1} \right)\\
\int_{{R^n}} {\rho d\mu }  &= 1
\end{align}
\textsc{Hint.} Define
\begin{align}
\phi \left( t \right) = \left\{ {\begin{array}{*{20}{c}}
{{e^{\frac{1}{{{t^2} - 1}}}},\hspace{0.2cm}\forall t \in \mathbb{R},\left| t \right| < 1}\\
{0,\hspace{0.2cm}\forall t \in \mathbb{R},\left| t \right| \ge 1}
\end{array}} \right.
\end{align}
Prove that $\phi  \in {C^\infty }\left( \mathbb{R} \right)$ and
\begin{align}
{\phi ^{\left( m \right)}}\left( t \right) = \left\{ {\begin{array}{*{20}{c}}
{{e^{\frac{1}{{{t^2} - 1}}}}\sum\limits_{\alpha ,\beta } {{c_{m,\alpha ,\beta }}{t^\alpha }{{\left( {{t^2} - 1} \right)}^\beta }} ,\hspace{0.2cm}\forall t \in \mathbb{R},\left| t \right| < 1}\\
{0,\hspace{0.2cm}\forall t \in \mathbb{R},\left| t \right| \ge 1}
\end{array}} \right.
\end{align}
Define $\psi \left( x \right) = \phi \left( {{{\left| x \right|}^2}} \right)$ for all $x\in \mathbb{R}^n$ and
\begin{align}
c = \int_{{\mathbb{R}^n}} {\psi d\mu } 
\end{align}
Define 
\begin{align}
\label{2.38}
\rho \left( x \right) = \frac{1}{c}\psi \left( x \right),\hspace{0.2cm}\forall x \in {\mathbb{R}^n}
\end{align}
Check that $\rho$ defined by \eqref{2.38} satisfies all requirements. \hfill $\square$\\
\\
\textbf{Problem 2.18.} \textit{Let $\rho$ be defined by \eqref{2.38}. Define ${\rho _m}\left( x \right) = {m^n}\rho \left( {mx} \right)$ for all positive integer $m$ and for all $x \in \mathbb{R}^n$. Prove that}
\begin{align}
{\rho _m} &\in C_c^\infty \left( {{\mathbb{R}^n}} \right)\\
{\rho _m}\left( x \right) &\ge 0,\hspace{0.2cm}\forall x \in {\mathbb{R}^n}\\
{\rho _m}\left( x \right) &= 0,\hspace{0.2cm}\forall x \in {\mathbb{R}^n}\backslash B\left( {0,\frac{1}{m}} \right)\\
\int_{{\mathbb{R}^n}} {{\rho _m}d\mu }  &= 1
\end{align}
\textsc{Hint.} Use Problem 1.69. \hfill $\square$\\
\\
\textbf{Problem 2.19.} \textit{Let $f$ be a real continuous function in $\mathbb{R}^n$. Define}
\begin{align}
{f_m}\left( x \right) = \int_{{R^n}} {f\left( y \right){\rho _m}\left( {x - y} \right)dy} ,\hspace{0.2cm}\forall x \in {\mathbb{R}^n}
\end{align}
\textit{Let $r$ and $\epsilon$ be two positive real numbers. Prove that there exists a positive integer $N$ such that}
\begin{align}
\left| {f\left( x \right) - {f_m}\left( x \right)} \right| \le \varepsilon ,\hspace{0.2cm}\forall m \ge N,x \in B\left( {0,r} \right)
\end{align}
\textsc{Hint.} Choose $N$ such that
\begin{align}
\left| {f\left( x \right) - f\left( z \right)} \right| \le \varepsilon ,\hspace{0.2cm}\forall x,z \in B'\left( {0,r + 1} \right),\left| {x - z} \right| \le \frac{1}{N}
\end{align}
Prove
\begin{align}
f\left( x \right) - {f_m}\left( x \right) &= \int_{{\mathbb{R}^n}} {\left( {f\left( x \right) - f\left( {x - y} \right)} \right){\rho _m}\left( y \right)dy} \\
 &= \int_{B\left( {0,\frac{1}{m}} \right)} {\left( {f\left( x \right) - f\left( {x - y} \right)} \right){\rho _m}\left( y \right)dy} 
\end{align}
\textbf{Problem 2.20.} \textit{Given $p\in \left[1,\infty\right)$, $u\in L^p\left(\mathbb{R}^n\right)$ and $f\in u$. Prove that there exists a sequence $\left\{ {{f_m}} \right\}_{m = 1}^\infty $ in $C_c^\infty \left(\mathbb{R}^n\right)$ such that}
\begin{align}
\mathop {\lim }\limits_{m \to \infty } \int_{{\mathbb{R}^n}} {{{\left| {f - {f_m}} \right|}^p}d\mu }  = 0
\end{align}
\textsc{Hint.} Define
\begin{align}
{g_k}\left( x \right) = \left\{ {\begin{array}{*{20}{c}}
{f\left( x \right),\mbox{ if }\left| x \right| < k}\\
{0,\mbox{ if }\left| x \right| \ge k}
\end{array}} \right.
\end{align}
Use Lebesgue dominated convergence theorem to prove
\begin{align}
\mathop {\lim }\limits_{m \to \infty } \int_{{\mathbb{R}^n}} {{{\left| {f - {g_m}} \right|}^p}d\mu }  = 0
\end{align}
Then apply Problem 1.64, Problem 2.18 and Lebesgue dominated convergence theorem. \hfill $\square$
\section{Fourier Transform}
\textbf{Definition 3.1.} Let $f,g$ be two real Lebesgue integrable functions in $\mathbb{R}^n$. Define
\begin{align}
\int_{{\mathbb{R}^n}} {\left( {f + ig} \right)d\mu }  = \int_{{\mathbb{R}^n}} {fd\mu }  + i\int_{{\mathbb{R}^n}} {gd\mu } 
\end{align}
\textbf{Definition 3.2.} Let $f$ be a real Lebesgue integrable function in $\mathbb{R}$. Define
\begin{align}
\widehat f\left( t \right) = \int_{{\mathbb{R}^n}} {f\left( x \right){e^{ - 2\pi itx}}d\mu } ,\hspace{0.2cm}\forall t \in {\mathbb{R}^n}
\end{align}
We call $\widehat f$ the \textit{Fourier transform} of $f$.\\
\\
\textbf{Problem 3.3.} \textit{Prove that $\widehat{f}$ is continuous in $\mathbb{R}^n$.}\\
\\
\textsc{Hint.} Let $\left\{ {{t_m}} \right\}_{m = 1}^\infty $ be a sequence converging to $t$ in $\mathbb{R}^n$. Use Lebesgue dominated convergence theorem to prove that
\begin{align}
\mathop {\lim }\limits_{m \to \infty } \int_{{R^n}} {f\left( x \right)\left( {{e^{ - 2\pi i{t_m}x}} - {e^{ - 2\pi itx}}} \right)d\mu }  = 0
\end{align}
\textbf{Definition 3.4.} Let $u\in L^1\left(\mathbb{R}^n\right)$ and $f\in u$. We denote the equivalent class of $\widehat f$ by $\widehat u$. We call $\widehat u$ the Fourier transform of $u$.\\
\\
\textbf{Problem 3.5.} \textit{Let $f$ be a real Lebesgue integrable function in $\mathbb{R}^n$ and $z\in \mathbb{R}^n$. Define}
\begin{align}
{f_z}\left( x \right) = f\left( {x + z} \right),\hspace{0.2cm}\forall x \in {\mathbb{R}^n}
\end{align}
\textit{Prove that}
\begin{align}
{\widehat f_z}\left( t \right) = \widehat f\left( t \right){e^{ - 2\pi itz}},\hspace{0.2cm}\forall t \in {\mathbb{R}^n}
\end{align}
\textsc{Hint.} Use definition. \hfill $\square$\\
\\
\textbf{Problem 3.6.} \textit{Given $f\in C_c^\infty \left(\mathbb{R}^n\right)$, $\alpha =\left(\alpha _1,\ldots,\alpha _n\right) \in \mathbb{R}^n$. Define $g = \frac{{{\partial ^\alpha }f}}{{\partial x}}$. Prove that}
\begin{align}
\widehat g\left( t \right) = \widehat f\left( t \right)\prod\limits_{j = 1}^n {t_j^{{\alpha _j}}} ,\hspace{0.2cm}\forall t \in {\mathbb{R}^n}
\end{align}
\textsc{Hint.} Use definition and Lebesgue dominated convergence theorem. \hfill $\square$\\
\\
\textbf{Problem 3.7.} \textit{Given $f\in C_c^\infty \left(\mathbb{R}^n\right)$. Prove that}
\begin{align}
\mathop {\lim }\limits_{\left| t \right| \to \infty } \widehat f\left( t \right) = 0
\end{align}
\textsc{Hint.} Use Problem 1.66 and Problem 1.30. \hfill $\square$\\
\\
\textbf{Problem 3.8.} \textit{Denote by $C_0\left(\mathbb{R}^n\right)$ the set of all real continuous functions $h$ in $\mathbb{R}^n$ such that $h\left(t\right) \to 0$ as $\left| t \right| \to \infty $. Prove that $C_0\left(\mathbb{R}^n\right)$ is a normed space equipped the following norm}
\begin{align}
{\left\| h \right\|_\infty } = \sup \left\{ {\left| {h\left( t \right)} \right|:t \in {\mathbb{R}^n}} \right\}
\end{align}
\textbf{Problem 3.9.} \textit{Prove that the mapping $u\to \widehat u$ is a continuous linear mapping from $L^1\left(\mathbb{R}^n\right)$ into $C_0\left(\mathbb{R}^n\right)$.}\\
\\
\textsc{Hint.} Use previous problems. \hfill $\square$\\
\\
\textbf{Problem 3.10.} \textit{Let $f$ and $g$ be two real Lebesgue integrable functions in $\mathbb{R}^n$. Define $h=f\star g$ for all $x\in \mathbb{R}^n$. Prove that}
\begin{align}
\widehat h\left( t \right) = \widehat f\left( t \right)\widehat g\left( t \right),\hspace{0.2cm}\forall t \in {\mathbb{R}^n}
\end{align}
\textsc{Hint.} Prove that
\begin{align}
&\int_{{\mathbb{R}^n}} {\left( {\int_{{\mathbb{R}^n}} {f\left( y \right)g\left( {x - y} \right)dy} } \right){e^{ - 2\pi itx}}dx} \\
& = \int_{{\mathbb{R}^n}} {\left( {\int_{{\mathbb{R}^n}} {f\left( y \right){e^{ - 2\pi ity}}g\left( {x - y} \right)dy} } \right){e^{ - 2\pi it\left( {x - y} \right)}}dx} 
\end{align}
Use Fubini theorem. \hfill $\square$\\
\\\\
\textbf{Problem 3.11.} \textit{Let $f,g $ be two real Lebesgue integrable functions in $\mathbb{R}^n$. Prove that}
\begin{align}
\int_{{\mathbb{R}^n}} {\widehat gfd\mu }  = \int_{{\mathbb{R}^n}} {\widehat fgd\mu } 
\end{align}
\textsc{Hint.} Use Fubini theorem to prove
\begin{align}
\int_{{\mathbb{R}^n}} {\left( {\int_{{\mathbb{R}^n}} {f\left( x \right){e^{ - 2\pi itx}}dx} } \right)g\left( t \right)dt}  = \int_{{\mathbb{R}^n}} {\left( {\int_{{\mathbb{R}^n}} {g\left( t \right){e^{ - 2\pi itx}}dt} } \right)f\left( x \right)dx} 
\end{align}
Try it. \hfill $\square$\\
\\
\textbf{Theorem 3.12.} \textit{Suppose that $f$ and $\widehat{f}$ are Lebesgue integrable functions in $\mathbb{R}^n$. Then}
\begin{align}
f\left( x \right) = \int_{{\mathbb{R}^n}} {\widehat f\left( t \right){e^{2\pi itx}}dt}, \mbox{ a.e. in }{\mathbb{R}^n}
\end{align}
\vspace{1.5cm}
\begin{center}
\textsc{The End}
\end{center}
\newpage
\printindex
\newpage
\begin{thebibliography}{999}
\bibitem {1} Duong Minh Duc, \textit{Real Analysis}, Faculty of Math and Computer Science, Ho Chi Minh University of Science.
\end{thebibliography}
\end{document}