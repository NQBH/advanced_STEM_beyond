\documentclass{article}
\usepackage[backend=biber,natbib=true,style=alphabetic,maxbibnames=50]{biblatex}
\addbibresource{/home/nqbh/reference/bib.bib}
\usepackage[utf8]{vietnam}
\usepackage{tocloft}
\renewcommand{\cftsecleader}{\cftdotfill{\cftdotsep}}
\usepackage[colorlinks=true,linkcolor=blue,urlcolor=red,citecolor=magenta]{hyperref}
\usepackage{amsmath,amssymb,amsthm,enumitem,float,graphicx,mathtools,tikz}
\usetikzlibrary{angles,calc,intersections,matrix,patterns,quotes,shadings}
\allowdisplaybreaks
\newtheorem{assumption}{Assumption}
\newtheorem{baitoan}{}
\newtheorem{cauhoi}{Câu hỏi}
\newtheorem{conjecture}{Conjecture}
\newtheorem{corollary}{Corollary}
\newtheorem{dangtoan}{Dạng toán}
\newtheorem{definition}{Definition}
\newtheorem{dinhly}{Định lý}
\newtheorem{dinhnghia}{Định nghĩa}
\newtheorem{example}{Example}
\newtheorem{ghichu}{Ghi chú}
\newtheorem{hequa}{Hệ quả}
\newtheorem{hypothesis}{Hypothesis}
\newtheorem{lemma}{Lemma}
\newtheorem{luuy}{Lưu ý}
\newtheorem{nhanxet}{Nhận xét}
\newtheorem{notation}{Notation}
\newtheorem{note}{Note}
\newtheorem{principle}{Principle}
\newtheorem{problem}{Problem}
\newtheorem{proposition}{Proposition}
\newtheorem{question}{Question}
\newtheorem{remark}{Remark}
\newtheorem{theorem}{Theorem}
\newtheorem{vidu}{Ví dụ}
\usepackage[left=1cm,right=1cm,top=5mm,bottom=5mm,footskip=4mm]{geometry}
\def\labelitemii{$\circ$}
\DeclareRobustCommand{\divby}{%
	\mathrel{\vbox{\baselineskip.65ex\lineskiplimit0pt\hbox{.}\hbox{.}\hbox{.}}}%
}
\setlist[itemize]{leftmargin=*}
\setlist[enumerate]{leftmargin=*}

\title{Combinatorics -- Tổ Hợp}
\author{Nguyễn Quản Bá Hồng\footnote{A Scientist {\it\&} Creative Artist Wannabe. E-mail: {\tt nguyenquanbahong@gmail.com}. Bến Tre City, Việt Nam.}}
\date{\today}

\begin{document}
\maketitle
\begin{abstract}
	This text is a part of the series {\it Some Topics in Advanced STEM \& Beyond}:
	
	{\sc url}: \url{https://nqbh.github.io/advanced_STEM/}.
	
	Latest version:
	\begin{itemize}
		\item {\it Combinatorics -- Tổ Hợp}.
		
		PDF: {\sc url}: \url{.pdf}.
		
		\TeX: {\sc url}: \url{.tex}.
		\item {\it }.
		
		PDF: {\sc url}: \url{.pdf}.
		
		\TeX: {\sc url}: \url{.tex}.
	\end{itemize}
\end{abstract}
\tableofcontents

%------------------------------------------------------------------------------%

\section{Wikipedia's}

\subsection{Wikipedia{\tt/}extremal combinatorics}
``{\it Extremal combinatorics} is a field of mathematics, which is itself a part of mathematics. Extremal combinatorics studies how large or how small a collection of finite objects (numbers, graphs, vectors, sets, etc.) can be, if it has to satisfy certain restrictions.

Much of extremal combinatorics concerns \href{https://en.wikipedia.org/wiki/Class_(set_theory)}{classes} of sets; this is called {\it extremal set theory}. E.g., in an $n$-element set, what is the largest number of $k$-element subsets that can pairwise intersect one another? What is the largest number of subsets of which more contains any other? The latter question is answered by \href{https://en.wikipedia.org/wiki/Sperner%27s_theorem}{Sperner's theorem}, which gave rise to much of extremal set theory.

Another kind of example: How many people can be invited to a party where among each 3 people there are 2 who know each other \& 2 who don't know each other? \href{https://en.wikipedia.org/wiki/Ramsey_theory}{Ramsey theory} shows: at most 5 persons can attend such a party (see \href{https://en.wikipedia.org/wiki/Theorem_on_Friends_and_Strangers}{Theorem on Friends \& Strangers}). Or, suppose given a finite set of nonzero integers, \& are asked to mark as large a subset as possible of this set under the restriction that the sum of any 2 marked integers cannot be marked. It appears that (independent of what the given integers actually are) we can always mark at least $\frac{1}{3}$ of them.'' -- \href{https://en.wikipedia.org/wiki/Extremal_combinatorics}{Wikipedia{\tt/}extremal combinatorics}

%------------------------------------------------------------------------------%

\section{Miscellaneous}

%------------------------------------------------------------------------------%

\printbibliography[heading=bibintoc]
	
\end{document}