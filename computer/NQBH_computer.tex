\documentclass{article}
\usepackage[backend=biber,natbib=true,style=alphabetic,maxbibnames=50]{biblatex}
\addbibresource{/home/nqbh/reference/bib.bib}
\usepackage[utf8]{vietnam}
\usepackage{tocloft}
\renewcommand{\cftsecleader}{\cftdotfill{\cftdotsep}}
\usepackage[colorlinks=true,linkcolor=blue,urlcolor=red,citecolor=magenta]{hyperref}
\usepackage{amsmath,amssymb,amsthm,enumitem,float,graphicx,mathtools,tikz}
\usetikzlibrary{angles,calc,intersections,matrix,patterns,quotes,shadings}
\allowdisplaybreaks
\newtheorem{assumption}{Assumption}
\newtheorem{baitoan}{}
\newtheorem{cauhoi}{Câu hỏi}
\newtheorem{conjecture}{Conjecture}
\newtheorem{corollary}{Corollary}
\newtheorem{dangtoan}{Dạng toán}
\newtheorem{definition}{Definition}
\newtheorem{dinhly}{Định lý}
\newtheorem{dinhnghia}{Định nghĩa}
\newtheorem{example}{Example}
\newtheorem{ghichu}{Ghi chú}
\newtheorem{hequa}{Hệ quả}
\newtheorem{hypothesis}{Hypothesis}
\newtheorem{lemma}{Lemma}
\newtheorem{luuy}{Lưu ý}
\newtheorem{nhanxet}{Nhận xét}
\newtheorem{notation}{Notation}
\newtheorem{note}{Note}
\newtheorem{principle}{Principle}
\newtheorem{problem}{Problem}
\newtheorem{proposition}{Proposition}
\newtheorem{question}{Question}
\newtheorem{remark}{Remark}
\newtheorem{theorem}{Theorem}
\newtheorem{vidu}{Ví dụ}
\usepackage[left=1cm,right=1cm,top=5mm,bottom=5mm,footskip=4mm]{geometry}
\def\labelitemii{$\circ$}
\DeclareRobustCommand{\divby}{%
	\mathrel{\vbox{\baselineskip.65ex\lineskiplimit0pt\hbox{.}\hbox{.}\hbox{.}}}%
}
\setlist[itemize]{leftmargin=*}
\setlist[enumerate]{leftmargin=*}

\title{Computer -- Máy Tính}
\author{Nguyễn Quản Bá Hồng\footnote{A Scientist {\it\&} Creative Artist Wannabe. E-mail: {\tt nguyenquanbahong@gmail.com}. Bến Tre City, Việt Nam.}}
\date{\today}

\begin{document}
\maketitle
\begin{abstract}
	This text is a part of the series {\it Some Topics in Advanced STEM \& Beyond}:
	
	{\sc url}: \url{https://nqbh.github.io/advanced_STEM/}.
	
	Latest version:
	\begin{itemize}
		\item {\it Computer -- Máy Tính}.
		
		PDF: {\sc url}: \url{https://github.com/NQBH/advanced_STEM_beyond/blob/main/computer/NQBH_computer.pdf}.
		
		\TeX: {\sc url}: \url{https://github.com/NQBH/advanced_STEM_beyond/blob/main/computer/NQBH_computer.tex}.
	\end{itemize}
\end{abstract}
\tableofcontents

%------------------------------------------------------------------------------%

\section{Linux}
\textbf{\textsf{Resources -- Tài nguyên.}}
\begin{enumerate}
	\item \cite{Shotts2019}. {\sc William Shotts}. {\it The Linux Command Line: A Complete Introduction}.
\end{enumerate}
I used SUSE \& OpenSUSE in WIAS Berlin but I do not like it so I go back to Ubuntu.

%------------------------------------------------------------------------------%

\section{Programming}

\subsection{C{\tt/}C++}
\textbf{\textsf{Resources -- Tài nguyên.}}
\begin{enumerate}
	\item \cite{Ngoc_C}. {\sc Quách Tuấn Ngọc}. {\it Ngôn Ngữ Lập Trình C}.
	\item \cite{Ngoc_C++}. {\sc Quách Tuấn Ngọc}. {\it Ngôn Ngữ Lập Trình C++}.
	\item \cite{Stroustrup2013}. {\sc Bjarne Stroustrup}. {\it The C++ Programming Language}.
	\item \cite{Stroustrup2018}. {\sc Bjarne Stroustrup}. {\it A Tour of C++}.
\end{enumerate}

\subsection{Pascal}
\textbf{\textsf{Resources -- Tài nguyên.}}
\begin{enumerate}
	\item \cite{Ngoc_Pascal}. {\sc Quách Tuấn Ngọc}. {\it Ngôn Ngữ Lập Trình Pascal}.
	\item \cite{Ngoc_BT_Pascal}. {\sc Quách Tuấn Ngọc}. {\it Bài Tập Ngôn Ngữ Lập Trình Pascal}.
	\item \cite{Doanh_Tuan_Pascal}. {\sc Lê Văn Doanh, Trần Khắc Tuấn}. {\it101 Thuật Toán \& Chương Trình Bài Toán Khoa Học Kỹ Thuật \& Kinh Tế Bằng Ngôn Ngữ Turbo-Pascal}.
\end{enumerate}

\subsection{Python}
\textbf{\textsf{Resources -- Tài nguyên.}}
\begin{enumerate}
	\item \cite{Duc_200_BT_Python}. {\sc Nguyễn Tiến Đức}. {\it Tuyển Tập 200 Bài Tập Lập Trình Bằng Ngôn Ngữ Python}.
	\item \cite{Huy_sang_tao_thuat_toan_lap_trinh_tap_1}. {\sc Nguyễn Xuân Huy}. {\it Sáng Tạo Trong Thuật Toán \& Lập Trình. Tập 1}.
	\item \cite{Huy_sang_tao_thuat_toan_lap_trinh_tap_2}. {\sc Nguyễn Xuân Huy}. {\it Sáng Tạo Trong Thuật Toán \& Lập Trình. Tập 2}.
	\item \cite{Huy_sang_tao_thuat_toan_lap_trinh_tap_3}. {\sc Nguyễn Xuân Huy}. {\it Sáng Tạo Trong Thuật Toán \& Lập Trình. Tập 3}.
	\item \cite{Huy_sang_tao_thuat_toan_lap_trinh_tap_4}. {\sc Nguyễn Xuân Huy}. {\it Sáng Tạo Trong Thuật Toán \& Lập Trình. Tập 4}.
	\item \cite{Huy_sang_tao_thuat_toan_lap_trinh_tap_5}. {\sc Nguyễn Xuân Huy}. {\it Sáng Tạo Trong Thuật Toán \& Lập Trình. Tập 5}.
	\item \cite{Huy_sang_tao_thuat_toan_lap_trinh_tap_6}. {\sc Nguyễn Xuân Huy}. {\it Sáng Tạo Trong Thuật Toán \& Lập Trình. Tập 6}.
	\item \cite{Huy_sang_tao_thuat_toan_lap_trinh_tap_7}. {\sc Nguyễn Xuân Huy}. {\it Sáng Tạo Trong Thuật Toán \& Lập Trình. Tập 7}.
\end{enumerate}

%------------------------------------------------------------------------------%

\section{Software}

\subsection{FeNiCS}
\textbf{\textsf{Resources -- Tài nguyên.}}
\begin{enumerate}
	\item \cite{Dokken_Mitusch_Funke2020}. {\sc J\o rgen S. Dokken}. {\it Automatic shape derivatives for transient PDEs in FEniCS \& Firedrake}.
	\item \cite{Langtangen_Logg2016}. {\sc Hans Petter Langtangen, Anders Logg}. {\it Solving PDEs in Python}.
\end{enumerate}

\subsection{Firedrake}

\subsection{Fireshape}
\textbf{\textsf{Resources -- Tài nguyên.}}
\begin{enumerate}
	\item \cite{Paganini_Wechsung_Fireshape2020}. {\sc Alberto Paganini, Florian Wechsung}. {\it Fireshape Documentation, Release 0.0.1}.
	\item \cite{Paganini_Wechsung2020}. {\sc Alberto Paganini, Florian Wechsung}. {\it Fireshape: a shape optimization toolbox for Firedrake}.
\end{enumerate}

\subsection{Git}
\textbf{\textsf{Resources -- Tài nguyên.}}
\begin{enumerate}
	\item \cite{Chacon_Straub2014}. {\sc Scott Chacon, Ben Straub}. {\it Pro Git}.
\end{enumerate}

\subsection{Gmsh}
\textbf{\textsf{Resources -- Tài nguyên.}}
\begin{enumerate}
	\item \cite{Geuzaine_Remacle2009}. {\sc Christophe Geuzaine, Jean-Fran\c{c}ois Remacle}. {\it Gmsh: A 3D finite element mesh generator with built-in pre- \& post-processing facilities}.
\end{enumerate}

\subsection{OpenFOAM}
\textbf{\textsf{Resources -- Tài nguyên.}}
\begin{enumerate}
	\item There are 3 variants of OpenFOAM:
	\begin{enumerate}
		\item OpenFOAM.com: Commercial
		\item OpenFOAM.org: Open-source with a large community.
		\item Extended OpenFOAM
	\end{enumerate}
	\item \cite{Greenshields_Weller2022}. {\sc Christopher Greenshields, Henry Weller}. {\it Notes on Computational Fluid Dynamics: General Principles}.
	\item \cite{Towara_Naumann2013}. {\sc M. Towara, U. Naumann}. {\it A Discrete Adjoint Model for OpenFOAM}.
\end{enumerate}

\subsection{ParMooN}
\textbf{\textsf{Resources -- Tài nguyên.}}
\begin{enumerate}
	\item \cite{ParMooN2017}. {\sc Ulrich Wilbrandt, Clemens Bartsch, Naveed Ahmed, Volker John}. {\it ParMooN -- a modernized program package based on mapped finite elements}.
\end{enumerate}

\subsection{SU2}

\subsection{Sublime Text}
\textbf{\textsf{Resources -- Tài nguyên.}}
\begin{enumerate}
	\item \cite{Bos2014}. {\sc Wes Bos}. {\it Sublime Text Power User: A Complete Guide}.
	\item \cite{Peleg2014}. {\sc Dan Peleg}. {\it Mastering Sublime Text}
\end{enumerate}

%------------------------------------------------------------------------------%

\section{Miscellaneous}

%------------------------------------------------------------------------------%

\printbibliography[heading=bibintoc]
	
\end{document}