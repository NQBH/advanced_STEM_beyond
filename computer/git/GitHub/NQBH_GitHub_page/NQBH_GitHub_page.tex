\documentclass{article}
\usepackage[backend=biber,natbib=true,style=authoryear]{biblatex}
\addbibresource{/home/hong/1_NQBH/reference/bib.bib}
\usepackage[english,vietnamese]{babel}
\usepackage{tocloft}
\renewcommand{\cftsecleader}{\cftdotfill{\cftdotsep}}
\usepackage[colorlinks=true,linkcolor=blue,urlcolor=red,citecolor=magenta]{hyperref}
\usepackage{algorithm,algpseudocode,amsmath,amssymb,amsthm,float,graphicx,mathtools,multicol}
\allowdisplaybreaks
\numberwithin{equation}{section}
\newtheorem{assumption}{Assumption}[section]
\newtheorem{conjecture}{Conjecture}[section]
\newtheorem{corollary}{Corollary}[section]
\newtheorem{definition}{Definition}[section]
\newtheorem{example}{Example}[section]
\newtheorem{lemma}{Lemma}[section]
\newtheorem{notation}{Notation}[section]
\newtheorem{principle}{Principle}[section]
\newtheorem{problem}{Problem}[section]
\newtheorem{proposition}{Proposition}[section]
\newtheorem{question}{Question}[section]
\newtheorem{remark}{Remark}[section]
\newtheorem{theorem}{Theorem}[section]
\usepackage[left=0.5in,right=0.5in,top=1.5cm,bottom=1.5cm]{geometry}
\usepackage{fancyhdr}
\pagestyle{fancy}
\fancyhf{}
\lhead{\small \textsc{Sect.} ~\thesection}
\rhead{\small \nouppercase{\leftmark}}
\renewcommand{\sectionmark}[1]{\markboth{#1}{}}
\cfoot{\thepage}
\def\labelitemii{$\circ$}

\title{NQBH's GitHub Page}
\author{\selectlanguage{vietnamese} Nguyễn Quản Bá Hồng\footnote{Independent Researcher, Ben Tre City, Vietnam\\e-mail: \texttt{nguyenquanbahong@gmail.com}}}
\date{\today}

\begin{document}
\maketitle
\selectlanguage{english}
\begin{abstract}
	The construction of NQBH's website.
\end{abstract}

\tableofcontents
\selectlanguage{vietnamese}

%------------------------------------------------------------------------------%

\section{Build NQBH's GitHub Page}

\begin{question}
	What should I name my personal page\emph{\texttt{/}}website?
\end{question}

\begin{proof}[Answer]
	For authenticity\footnote{\textbf{authenticity} [n] [uncountable] the quality of being genuine or true, or based on fact.} reason, I should use my real name instead of some nickname, which sounds cool now but will turn into bullshit later in the long run, also non-unique. My ear will reject any kind of nickname like that -- too childish. Use my real name then. 1st of all, the option \textsc{nguyenquanbahong}, although unique (Who the hell has the same name as me?), seems too long which even makes me feel lazy to type every time I want to enter my website. \textsc{NQBH} seems to be a good compromise between brevity \& uniqueness. I do not think there are a lot of people whose their names share this abbreviation with me. Thus, I choose NQBH for the main part of my website name.
	
	\textit{Then what about the extension?} The commons are \textsc{.com, .edu, .vn}, etc. \textit{Which one should I choose?} Using method of elimination, I should not choose \textsc{.edu} since I will post some ``uneducated'' contents, however, with some education purposes, which will not yield any contradiction at all. Posting some uneducated contents under a website with \textsc{.edu} extension seems to contaminate\footnote{\textbf{contaminate} [v] \textbf{1.} [usually passive] to make a substance or place dirty or no longer pure by adding a substance that is dangerous or carries disease; \textbf{2.} \textbf{contaminate something} to influence people's ideas or attitudes in a bad way.} the prestigious label ``edu'' in various senses. How about \textsc{.com, .vn} extensions? These seem too common \& cost some money monthly\texttt{/}yearly, of course. Hence, I should use some free website builders\texttt{/}hosting with non-annoying extension. \textsc{GitHub}'s personal page seems a good choice for me.
\end{proof}
Next, I follow the following instruction: \href{https://jayrobwilliams.com/posts/2020/06/academic-website/}{Rob Williams\texttt{/}Building an Academic Website}.

\subsection{Getting Started}
``Most universities these days provide a free option, usually powered by \href{https://wordpress.org/}{WordPress} (both \href{https://sites.wustl.edu/}{WashU} \& \href{http://web.unc.edu/}{UNC} use WordPress for their respective offerings). While these sites are quick to set up \& come with the prestige of a \texttt{.edu} \textsc{url}, they have several drawbacks that have been extensively written on, e.g.,
\begin{itemize}
	\item University of California, Berkeley\texttt{/}Townsend Center for the Humanities\texttt{/}\href{https://townsendcenter.berkeley.edu/blog/personal-academic-webpages-how-tos-and-tips-better-site}{Rochelle Terman. \textit{Personal Academic Webpages: How-To's \& Tips for a Better Site}}.
	\item \href{https://martinlea.com/four-reasons-why-faculty-profile-pages-are-no-substitute-for-personal-academic-website/}{Martin Lea. \textit{Personal Academic Websites Versus Faculty Web Pages}}.
	\item \href{https://theacademicdesigner.com/2019/personal-academic-website-benefits/}{The Academic Designer\texttt{/}\textit{8 Benefits of a Personal Academic Website to Inspire You}}.
\end{itemize}
If you're a junior scholar, having your own personal webpage is even more important.
\begin{itemize}
	\item If (when) you move institutions, you'll lose your website.
	\item Even if you can export the contents of a WordPress site, there's no guarantee it will seamlessly integrate with another university's implementation.
	\item Even worse, you'll lose your search engine ranking since you'll be starting over from square one with a new \textsc{url}.
\end{itemize}
Even if you stay at the same institution for the rest of your career, you're at the mercy of IT \& your site may be taken down by a change to the hosting platform at some point in the future.''

``There are plenty of guides out there on how to create a personal website using tools like WordPress, \href{https://www.wix.com/}{Wix}, or \href{https://www.sites.google.com/}{Google Sites}\footnote{NQBH: I used Google Sites already. It is too slow to edit \& publish. So I will not use Google Site anymore.}. The free versions of these tools often come with ads, or at the least a message telling you which tool was used to create the website.'' Rob Williams ``use a \href{https://www.staticgen.com/about}{static site generator} that produces HTML from easy to edit \href{https://en.wikipedia.org/wiki/Markdown}{Markdown} files. Because the resulting site is static (it's just a collection of files with no interactively where users can, e.g., fill out \& submit forms) it can be hosted for free with \href{https://pages.github.com/}{GitHub Pages}. \href{http://svmiller.com/}{Steven Miller} has a nice rundown on all of the \href{http://svmiller.com/blog/2015/08/create-your-website-in-jekyll/#advantages}{advantages} of this approach.''

``This guide is intended for someone with a basic level of coding experience \& comfort with Markdown files.'' ``There are other guides to using static site generators to make academic websites, but they all assume a very high level of experience with the required tools \& the ability to conduct extensive troubleshooting on your own.''

\begin{quotation}
	``A brief aside on Git-speak: these periodic indended blocks will explain the terminology that git uses to help you understand what each Git command actually does.''
\end{quotation}
``2 of the most popular programs available for building static sites from Markdown files are \href{https://jekyllrb.com/}{Jekyll} \& \href{https://gohugo.io/}{Hugo}.'' ``There are plenty of differences under the hood, but the most important one for building an academic website is that Hugo integrates nicely with the \href{https://bookdown.org/yihui/blogdown/}{blogdown} R package, letting you write your website entirely in R.''

\begin{quotation}
	``Git-speak aside: the basic unit of GitHub is the repository. Repositories are just folders (directories, if you want to be pedantic), but Git keeps a record of the files in the folders. We'll start by making a repository on GitHub \& then later download that repository to our computer. In both cases, it's just a folder. The magic of \texttt{Git} is that we can link the 2 so that changes you make in your local repository (the one on your computer) will sync with the remote repository (the one on GitHub). When people (myself included) get lazy, they'll often shorten repository to `repo'.''
\end{quotation}
``\href{https://git-scm.com/}{Git} is a version control system designed to let teams of programmers collaborate on projects seamlessly. For us, it's just going to be the way that we upload files for our webpage to GitHub.''

\begin{quotation}
	``Git-speak aside: cloning a repository means creating a local repository (folder) on your computer that's connected to the remote repository (on GitHub). Cloning differs from downloading in that you are setting up a connection between the 2 folders so you can keep changes you make locally synced up with the remote repository (which is where GitHub will build your website from).''
\end{quotation}
The \texttt{git fork} \& \texttt{git clone} steps are straightforward.

\subsection{Previewing Website}
Once we upload our modified files to GitHub \& tell GitHub to turn them into a website, they're out there on the Internet for everyone to see. ``There's no need to broadcast all of those mistakes to the world, \& we can avoid this very easily by previewing our website locally. What this means is building the site from the various \texttt{.md} files, rendering it to \textsc{html}, \& then viewing it. We can do all of that on our computer without ever having to put it online.''

``To preview your website locally, you'll need to install Jekyll on your computer. The easiest way to do this is with \href{https://bundler.io/}{Bundler}. Bundler is a package manager for Ruby, which is the programming language that Jekyll is written in. This means that we need a full Ruby development environment to get Jekyll working to run our website locally.''

To install Bundler, run:
\begin{verbatim}
	hong@hong-Katana-GF76-11UC:~/1_NQBH/NQBH.github.io$ sudo gem install bundler
	Fetching bundler-2.3.12.gem
	Successfully installed bundler-2.3.12
	Parsing documentation for bundler-2.3.12
	Installing ri documentation for bundler-2.3.12
	Done installing documentation for bundler after 0 seconds
	1 gem installed
\end{verbatim}
``Next, we need to install any packages (called `gems' in Ruby) that Jekyll depends on. This is where Bundler shines by taking care of this whole process for us; it reads the \texttt{Gemfile} included with the source code \& install all required gems:
\begin{verbatim}
	$ bundler install
\end{verbatim}
If you want to see what's been installed, run \texttt{gem list} before \& after \texttt{bundle install}. If everything worked correctly, you can now launch your website! What we're going to do is start a webserver on your computer, which will let you access your website locally without having to put it ot on the Internet. We do this with
\begin{verbatim}
	bundle exec jekyll serve
\end{verbatim}
The \texttt{bundle exec} command is just a prefix that lets Ruby access all of the gems specified in the \texttt{gemfile}. The \texttt{jekyll serve} command builds your website \& starts a webserver so that you can view it locally. To access your website, open a browser \& go to \texttt{127.0.0.1:4000} or \texttt{localhost:4000}.''

``This is a special version of your site that's only accessible from your computer; no one else can see it! So this is the perfect place to play around, experiment, \& see how to make your site do what you want it to. This process is surprisingly easy. Make a change to a file, e.g., editing \verb|_pages/about.md| to introduce yourself, \& save the file. That's all you have to do; Jekyll will notice the change to the file \& automatically rebuild the site. All that's left to do is refresh your browser so you can see the changes!

Once you've made a couple changes to see how it works, you might want to turn off the webserver \& make lots of changes, then check out your handiwork. Or maybe you're just done working on your website for now. Either way, it's time to shut down the webserver. To do so, you can just close the terminal window, but you'll get a warning like this
\begin{verbatim}
	Do you want to terminate running processes on this window?
	Closing this window will terminate the running processes: jekyll serve --incremental, fsevent_watch
\end{verbatim}
To save yourself some time \& do this faster, simply press \texttt{Ctrl $+$ C}.''

Explicitly on my laptop:
\begin{verbatim}
	hong@hong-Katana-GF76-11UC:~/1_NQBH/NQBH.github.io$ sudo bundle install
	Don't run Bundler as root. Bundler can ask for sudo if it is needed, and
	installing your bundle as root will break this application for all non-root
	users on this machine.
	Fetching gem metadata from https://rubygems.org/..........
	Resolving dependencies....
	Using concurrent-ruby 1.1.10
	Using minitest 5.15.0
	Using faraday-httpclient 1.0.1
	Using thread_safe 0.3.6
	Using zeitwerk 2.5.4
	Using faraday-net_http_persistent 1.2.0
	Using faraday-patron 1.0.0
	Using faraday-rack 1.0.0
	Using faraday-retry 1.0.3
	Using ruby2_keywords 0.0.5
	Using forwardable-extended 2.6.0
	Using coffee-script-source 1.11.1
	Using rb-fsevent 0.11.1
	Using faraday-em_http 1.0.0
	Using liquid 4.0.3
	Using mercenary 0.3.6
	Using rouge 3.26.0
	Using safe_yaml 1.0.5
	Fetching racc 1.6.0
	Using faraday-em_synchrony 1.0.0
	Using public_suffix 4.0.7
	Using rubyzip 2.3.2
	Using jekyll-swiss 1.0.0
	Using unicode-display_width 1.8.0
	Fetching unf_ext 0.0.8.1
	Fetching commonmarker 0.23.4
	Using colorator 1.1.0
	Using faraday-excon 1.1.0
	Fetching http_parser.rb 0.8.0
	Using gemoji 3.0.1
	Fetching ffi 1.15.5
	Using jekyll-paginate 1.1.0
	Using bundler 2.3.12
	Fetching eventmachine 1.2.7
	Using multipart-post 2.1.1
	Using execjs 2.8.1
	Using i18n 0.9.5
	Using coffee-script 2.4.1
	Using tzinfo 1.2.9
	Using jekyll-coffeescript 1.1.1
	Using rexml 3.2.5
	Using addressable 2.8.0
	Using kramdown 2.3.2
	Using faraday-multipart 1.0.3
	Using kramdown-parser-gfm 1.1.0
	Using faraday-net_http 1.0.1
	Using activesupport 6.0.4.7
	Using faraday 1.10.0
	Using pathutil 0.16.2
	Using terminal-table 1.8.0
	Using sawyer 0.8.2
	Using octokit 4.22.0
	Using jekyll-gist 1.5.0
	Installing racc 1.6.0 with native extensions
	Installing commonmarker 0.23.4 with native extensions
	Installing http_parser.rb 0.8.0 with native extensions
	Installing eventmachine 1.2.7 with native extensions
	Installing unf_ext 0.0.8.1 with native extensions
	Installing ffi 1.15.5 with native extensions
	Gem::Ext::BuildError: ERROR: Failed to build gem native extension.
	
	current directory:
	/var/lib/gems/2.7.0/gems/commonmarker-0.23.4/ext/commonmarker
	/usr/bin/ruby2.7 -I /usr/lib/ruby/vendor_ruby -r
	./siteconf20220421-57640-1cwfdoz.rb extconf.rb
	mkmf.rb can't find header files for ruby at /usr/lib/ruby/include/ruby.h
	
	You might have to install separate package for the ruby development
	environment, ruby-dev or ruby-devel for example.
	
	extconf failed, exit code 1
	
	Gem files will remain installed in /var/lib/gems/2.7.0/gems/commonmarker-0.23.4
	for inspection.
	Results logged to
	/var/lib/gems/2.7.0/extensions/x86_64-linux/2.7.0/commonmarker-0.23.4/gem_make.out
	
	  /usr/lib/ruby/vendor_ruby/rubygems/ext/builder.rb:91:in `run'
	/usr/lib/ruby/vendor_ruby/rubygems/ext/ext_conf_builder.rb:48:in `block in
	build'
	  /usr/lib/ruby/2.7.0/tempfile.rb:291:in `open'
	  /usr/lib/ruby/vendor_ruby/rubygems/ext/ext_conf_builder.rb:28:in `build'
	  /usr/lib/ruby/vendor_ruby/rubygems/ext/builder.rb:157:in `build_extension'
	/usr/lib/ruby/vendor_ruby/rubygems/ext/builder.rb:191:in `block in
	build_extensions'
	  /usr/lib/ruby/vendor_ruby/rubygems/ext/builder.rb:188:in `each'
	  /usr/lib/ruby/vendor_ruby/rubygems/ext/builder.rb:188:in `build_extensions'
	  /usr/lib/ruby/vendor_ruby/rubygems/installer.rb:821:in `build_extensions'
	/var/lib/gems/2.7.0/gems/bundler-2.3.12/lib/bundler/rubygems_gem_installer.rb:71:in
	`build_extensions'
	/var/lib/gems/2.7.0/gems/bundler-2.3.12/lib/bundler/rubygems_gem_installer.rb:28:in
	`install'
	/var/lib/gems/2.7.0/gems/bundler-2.3.12/lib/bundler/source/rubygems.rb:204:in
	`install'
	/var/lib/gems/2.7.0/gems/bundler-2.3.12/lib/bundler/installer/gem_installer.rb:54:in
	`install'
	/var/lib/gems/2.7.0/gems/bundler-2.3.12/lib/bundler/installer/gem_installer.rb:16:in
	`install_from_spec'
	/var/lib/gems/2.7.0/gems/bundler-2.3.12/lib/bundler/installer/parallel_installer.rb:186:in
	`do_install'
	/var/lib/gems/2.7.0/gems/bundler-2.3.12/lib/bundler/installer/parallel_installer.rb:177:in
	`block in worker_pool'
	/var/lib/gems/2.7.0/gems/bundler-2.3.12/lib/bundler/worker.rb:62:in
	`apply_func'
	/var/lib/gems/2.7.0/gems/bundler-2.3.12/lib/bundler/worker.rb:57:in `block in
	process_queue'
	  /var/lib/gems/2.7.0/gems/bundler-2.3.12/lib/bundler/worker.rb:54:in `loop'
	/var/lib/gems/2.7.0/gems/bundler-2.3.12/lib/bundler/worker.rb:54:in
	`process_queue'
	/var/lib/gems/2.7.0/gems/bundler-2.3.12/lib/bundler/worker.rb:91:in `block (2
	levels) in create_threads'
	
	An error occurred while installing commonmarker (0.23.4), and Bundler
	cannot continue.
	
	In Gemfile:
	  github-pages was resolved to 226, which depends on
	    jekyll-commonmark-ghpages was resolved to 0.2.0, which depends on
	      jekyll-commonmark was resolved to 1.4.0, which depends on
	        commonmarker
	
	
	Gem::Ext::BuildError: ERROR: Failed to build gem native extension.
	
	    current directory: /var/lib/gems/2.7.0/gems/unf_ext-0.0.8.1/ext/unf_ext
	/usr/bin/ruby2.7 -I /usr/lib/ruby/vendor_ruby -r
	./siteconf20220421-57640-ehba7a.rb extconf.rb
	mkmf.rb can't find header files for ruby at /usr/lib/ruby/include/ruby.h
	
	You might have to install separate package for the ruby development
	environment, ruby-dev or ruby-devel for example.
	
	extconf failed, exit code 1
	
	Gem files will remain installed in /var/lib/gems/2.7.0/gems/unf_ext-0.0.8.1 for
	inspection.
	Results logged to
	/var/lib/gems/2.7.0/extensions/x86_64-linux/2.7.0/unf_ext-0.0.8.1/gem_make.out
	
	  /usr/lib/ruby/vendor_ruby/rubygems/ext/builder.rb:91:in `run'
	/usr/lib/ruby/vendor_ruby/rubygems/ext/ext_conf_builder.rb:48:in `block in
	build'
	  /usr/lib/ruby/2.7.0/tempfile.rb:291:in `open'
	  /usr/lib/ruby/vendor_ruby/rubygems/ext/ext_conf_builder.rb:28:in `build'
	  /usr/lib/ruby/vendor_ruby/rubygems/ext/builder.rb:157:in `build_extension'
	/usr/lib/ruby/vendor_ruby/rubygems/ext/builder.rb:191:in `block in
	build_extensions'
	  /usr/lib/ruby/vendor_ruby/rubygems/ext/builder.rb:188:in `each'
	  /usr/lib/ruby/vendor_ruby/rubygems/ext/builder.rb:188:in `build_extensions'
	  /usr/lib/ruby/vendor_ruby/rubygems/installer.rb:821:in `build_extensions'
	/var/lib/gems/2.7.0/gems/bundler-2.3.12/lib/bundler/rubygems_gem_installer.rb:71:in
	`build_extensions'
	/var/lib/gems/2.7.0/gems/bundler-2.3.12/lib/bundler/rubygems_gem_installer.rb:28:in
	`install'
	/var/lib/gems/2.7.0/gems/bundler-2.3.12/lib/bundler/source/rubygems.rb:204:in
	`install'
	/var/lib/gems/2.7.0/gems/bundler-2.3.12/lib/bundler/installer/gem_installer.rb:54:in
	`install'
	/var/lib/gems/2.7.0/gems/bundler-2.3.12/lib/bundler/installer/gem_installer.rb:16:in
	`install_from_spec'
	/var/lib/gems/2.7.0/gems/bundler-2.3.12/lib/bundler/installer/parallel_installer.rb:186:in
	`do_install'
	/var/lib/gems/2.7.0/gems/bundler-2.3.12/lib/bundler/installer/parallel_installer.rb:177:in
	`block in worker_pool'
	/var/lib/gems/2.7.0/gems/bundler-2.3.12/lib/bundler/worker.rb:62:in
	`apply_func'
	/var/lib/gems/2.7.0/gems/bundler-2.3.12/lib/bundler/worker.rb:57:in `block in
	process_queue'
	  /var/lib/gems/2.7.0/gems/bundler-2.3.12/lib/bundler/worker.rb:54:in `loop'
	/var/lib/gems/2.7.0/gems/bundler-2.3.12/lib/bundler/worker.rb:54:in
	`process_queue'
	/var/lib/gems/2.7.0/gems/bundler-2.3.12/lib/bundler/worker.rb:91:in `block (2
	levels) in create_threads'
	
	An error occurred while installing unf_ext (0.0.8.1), and Bundler
	cannot continue.
	
	In Gemfile:
	  github-pages was resolved to 226, which depends on
	    github-pages-health-check was resolved to 1.17.9, which depends on
	      dnsruby was resolved to 1.61.9, which depends on
	        simpleidn was resolved to 0.2.1, which depends on
	          unf was resolved to 0.1.4, which depends on
	            unf_ext
	
	
	Gem::Ext::BuildError: ERROR: Failed to build gem native extension.
	
	current directory:
	/var/lib/gems/2.7.0/gems/http_parser.rb-0.8.0/ext/ruby_http_parser
	/usr/bin/ruby2.7 -I /usr/lib/ruby/vendor_ruby -r
	./siteconf20220421-57640-zl5ph2.rb extconf.rb
	mkmf.rb can't find header files for ruby at /usr/lib/ruby/include/ruby.h
	
	You might have to install separate package for the ruby development
	environment, ruby-dev or ruby-devel for example.
	
	extconf failed, exit code 1
	
	Gem files will remain installed in /var/lib/gems/2.7.0/gems/http_parser.rb-0.8.0
	for inspection.
	Results logged to
	/var/lib/gems/2.7.0/extensions/x86_64-linux/2.7.0/http_parser.rb-0.8.0/gem_make.out
	
	  /usr/lib/ruby/vendor_ruby/rubygems/ext/builder.rb:91:in `run'
	/usr/lib/ruby/vendor_ruby/rubygems/ext/ext_conf_builder.rb:48:in `block in
	build'
	  /usr/lib/ruby/2.7.0/tempfile.rb:291:in `open'
	  /usr/lib/ruby/vendor_ruby/rubygems/ext/ext_conf_builder.rb:28:in `build'
	  /usr/lib/ruby/vendor_ruby/rubygems/ext/builder.rb:157:in `build_extension'
	/usr/lib/ruby/vendor_ruby/rubygems/ext/builder.rb:191:in `block in
	build_extensions'
	  /usr/lib/ruby/vendor_ruby/rubygems/ext/builder.rb:188:in `each'
	  /usr/lib/ruby/vendor_ruby/rubygems/ext/builder.rb:188:in `build_extensions'
	  /usr/lib/ruby/vendor_ruby/rubygems/installer.rb:821:in `build_extensions'
	/var/lib/gems/2.7.0/gems/bundler-2.3.12/lib/bundler/rubygems_gem_installer.rb:71:in
	`build_extensions'
	/var/lib/gems/2.7.0/gems/bundler-2.3.12/lib/bundler/rubygems_gem_installer.rb:28:in
	`install'
	/var/lib/gems/2.7.0/gems/bundler-2.3.12/lib/bundler/source/rubygems.rb:204:in
	`install'
	/var/lib/gems/2.7.0/gems/bundler-2.3.12/lib/bundler/installer/gem_installer.rb:54:in
	`install'
	/var/lib/gems/2.7.0/gems/bundler-2.3.12/lib/bundler/installer/gem_installer.rb:16:in
	`install_from_spec'
	/var/lib/gems/2.7.0/gems/bundler-2.3.12/lib/bundler/installer/parallel_installer.rb:186:in
	`do_install'
	/var/lib/gems/2.7.0/gems/bundler-2.3.12/lib/bundler/installer/parallel_installer.rb:177:in
	`block in worker_pool'
	/var/lib/gems/2.7.0/gems/bundler-2.3.12/lib/bundler/worker.rb:62:in
	`apply_func'
	/var/lib/gems/2.7.0/gems/bundler-2.3.12/lib/bundler/worker.rb:57:in `block in
	process_queue'
	  /var/lib/gems/2.7.0/gems/bundler-2.3.12/lib/bundler/worker.rb:54:in `loop'
	/var/lib/gems/2.7.0/gems/bundler-2.3.12/lib/bundler/worker.rb:54:in
	`process_queue'
	/var/lib/gems/2.7.0/gems/bundler-2.3.12/lib/bundler/worker.rb:91:in `block (2
	levels) in create_threads'
	
	An error occurred while installing http_parser.rb (0.8.0), and Bundler
	cannot continue.
	
	In Gemfile:
	  github-pages was resolved to 226, which depends on
	    jekyll-avatar was resolved to 0.7.0, which depends on
	      jekyll was resolved to 3.9.2, which depends on
	        em-websocket was resolved to 0.5.3, which depends on
	          http_parser.rb
	
	
	Gem::Ext::BuildError: ERROR: Failed to build gem native extension.
	
	    current directory: /var/lib/gems/2.7.0/gems/racc-1.6.0/ext/racc/cparse
	/usr/bin/ruby2.7 -I /usr/lib/ruby/vendor_ruby -r
	./siteconf20220421-57640-rk4k26.rb extconf.rb
	mkmf.rb can't find header files for ruby at /usr/lib/ruby/include/ruby.h
	
	You might have to install separate package for the ruby development
	environment, ruby-dev or ruby-devel for example.
	
	extconf failed, exit code 1
	
	Gem files will remain installed in /var/lib/gems/2.7.0/gems/racc-1.6.0 for
	inspection.
	Results logged to
	/var/lib/gems/2.7.0/extensions/x86_64-linux/2.7.0/racc-1.6.0/gem_make.out
	
	  /usr/lib/ruby/vendor_ruby/rubygems/ext/builder.rb:91:in `run'
	/usr/lib/ruby/vendor_ruby/rubygems/ext/ext_conf_builder.rb:48:in `block in
	build'
	  /usr/lib/ruby/2.7.0/tempfile.rb:291:in `open'
	  /usr/lib/ruby/vendor_ruby/rubygems/ext/ext_conf_builder.rb:28:in `build'
	  /usr/lib/ruby/vendor_ruby/rubygems/ext/builder.rb:157:in `build_extension'
	/usr/lib/ruby/vendor_ruby/rubygems/ext/builder.rb:191:in `block in
	build_extensions'
	  /usr/lib/ruby/vendor_ruby/rubygems/ext/builder.rb:188:in `each'
	  /usr/lib/ruby/vendor_ruby/rubygems/ext/builder.rb:188:in `build_extensions'
	  /usr/lib/ruby/vendor_ruby/rubygems/installer.rb:821:in `build_extensions'
	/var/lib/gems/2.7.0/gems/bundler-2.3.12/lib/bundler/rubygems_gem_installer.rb:71:in
	`build_extensions'
	/var/lib/gems/2.7.0/gems/bundler-2.3.12/lib/bundler/rubygems_gem_installer.rb:28:in
	`install'
	/var/lib/gems/2.7.0/gems/bundler-2.3.12/lib/bundler/source/rubygems.rb:204:in
	`install'
	/var/lib/gems/2.7.0/gems/bundler-2.3.12/lib/bundler/installer/gem_installer.rb:54:in
	`install'
	/var/lib/gems/2.7.0/gems/bundler-2.3.12/lib/bundler/installer/gem_installer.rb:16:in
	`install_from_spec'
	/var/lib/gems/2.7.0/gems/bundler-2.3.12/lib/bundler/installer/parallel_installer.rb:186:in
	`do_install'
	/var/lib/gems/2.7.0/gems/bundler-2.3.12/lib/bundler/installer/parallel_installer.rb:177:in
	`block in worker_pool'
	/var/lib/gems/2.7.0/gems/bundler-2.3.12/lib/bundler/worker.rb:62:in
	`apply_func'
	/var/lib/gems/2.7.0/gems/bundler-2.3.12/lib/bundler/worker.rb:57:in `block in
	process_queue'
	  /var/lib/gems/2.7.0/gems/bundler-2.3.12/lib/bundler/worker.rb:54:in `loop'
	/var/lib/gems/2.7.0/gems/bundler-2.3.12/lib/bundler/worker.rb:54:in
	`process_queue'
	/var/lib/gems/2.7.0/gems/bundler-2.3.12/lib/bundler/worker.rb:91:in `block (2
	levels) in create_threads'
	
	An error occurred while installing racc (1.6.0), and Bundler cannot
	continue.
	
	In Gemfile:
	  github-pages was resolved to 226, which depends on
	    jekyll-mentions was resolved to 1.6.0, which depends on
	      html-pipeline was resolved to 2.14.1, which depends on
	        nokogiri was resolved to 1.13.4, which depends on
	          racc
\end{verbatim}
To fix this error, I followed the answer of the question \href{https://stackoverflow.com/questions/70398847/an-error-occurred-while-installing-racc-1-6-0-and-bundler-cannot-continue}{Stack Overflow\texttt{/}An error occurred while installing racc (1.6.0), and Bundler cannot continue}:
\begin{enumerate}
	\item Update \texttt{ruby} to \href{https://www.ruby-lang.org/en/documentation/installation/}{the latest version 2.7.0p0} using:
	\begin{verbatim}
		sudo apt-get install ruby-full build-essential
	\end{verbatim}
	\item Rerun \texttt{bundle install} (no need \texttt{sudo}).
\end{enumerate}
Explicitly,
\begin{verbatim}
	hong@hong-Katana-GF76-11UC:~/1_NQBH/NQBH.github.io$ sudo apt-get install ruby-full build-essential
	Reading package lists... Done
	Building dependency tree... Done
	Reading state information... Done
	The following additional packages will be installed:
	  ri ruby-dev ruby2.7-dev ruby2.7-doc
	The following NEW packages will be installed:
	  build-essential ri ruby-dev ruby-full ruby2.7-dev ruby2.7-doc
	0 upgraded, 6 newly installed, 0 to remove and 0 not upgraded.
	Need to get 2.421 kB of archives.
	After this operation, 24,5 MB of additional disk space will be used.
	Do you want to continue? [Y/n] Y
	Get:1 http://vn.archive.ubuntu.com/ubuntu impish/main amd64 build-essential amd64 12.9ubuntu2 [4.678 B]
	Get:2 http://vn.archive.ubuntu.com/ubuntu impish-updates/main amd64 ruby2.7-doc all 2.7.4-1ubuntu3.1 [2.216 kB]
	Get:3 http://vn.archive.ubuntu.com/ubuntu impish/universe amd64 ri all 1:2.7+2build1 [4.418 B]
	Get:4 http://vn.archive.ubuntu.com/ubuntu impish-updates/main amd64 ruby2.7-dev amd64 2.7.4-1ubuntu3.1 [189 kB]
	Get:5 http://vn.archive.ubuntu.com/ubuntu impish/main amd64 ruby-dev amd64 1:2.7+2build1 [4.524 B]
	Get:6 http://vn.archive.ubuntu.com/ubuntu impish/universe amd64 ruby-full all 1:2.7+2build1 [2.586 B]
	Fetched 2.421 kB in 1s (4.291 kB/s)   
	Selecting previously unselected package build-essential.
	(Reading database ... 285417 files and directories currently installed.)
	Preparing to unpack .../0-build-essential_12.9ubuntu2_amd64.deb ...
	Unpacking build-essential (12.9ubuntu2) ...
	Selecting previously unselected package ruby2.7-doc.
	Preparing to unpack .../1-ruby2.7-doc_2.7.4-1ubuntu3.1_all.deb ...
	Unpacking ruby2.7-doc (2.7.4-1ubuntu3.1) ...
	Selecting previously unselected package ri.
	Preparing to unpack .../2-ri_1%3a2.7+2build1_all.deb ...
	Unpacking ri (1:2.7+2build1) ...
	Selecting previously unselected package ruby2.7-dev:amd64.
	Preparing to unpack .../3-ruby2.7-dev_2.7.4-1ubuntu3.1_amd64.deb ...
	Unpacking ruby2.7-dev:amd64 (2.7.4-1ubuntu3.1) ...
	Selecting previously unselected package ruby-dev:amd64.
	Preparing to unpack .../4-ruby-dev_1%3a2.7+2build1_amd64.deb ...
	Unpacking ruby-dev:amd64 (1:2.7+2build1) ...
	Selecting previously unselected package ruby-full.
	Preparing to unpack .../5-ruby-full_1%3a2.7+2build1_all.deb ...
	Unpacking ruby-full (1:2.7+2build1) ...
	Setting up ruby2.7-doc (2.7.4-1ubuntu3.1) ...
	Setting up ruby2.7-dev:amd64 (2.7.4-1ubuntu3.1) ...
	Setting up build-essential (12.9ubuntu2) ...
	Setting up ruby-dev:amd64 (1:2.7+2build1) ...
	Setting up ri (1:2.7+2build1) ...
	Setting up ruby-full (1:2.7+2build1) ...
\end{verbatim}
then rerun \texttt{bundle install}:
\begin{verbatim}
	hong@hong-Katana-GF76-11UC:~/1_NQBH/NQBH.github.io$ bundle install
	Fetching gem metadata from https://rubygems.org/..........
	Resolving dependencies....
	Using concurrent-ruby 1.1.10
	Using faraday-em_http 1.0.0
	Using thread_safe 0.3.6
	Using zeitwerk 2.5.4
	Using public_suffix 4.0.7
	Using bundler 2.3.12
	Using faraday-excon 1.1.0
	Using coffee-script-source 1.11.1
	Using colorator 1.1.0
	Using faraday-em_synchrony 1.0.0
	Using minitest 5.15.0
	Using execjs 2.8.1
	Following files may not be writable, so sudo is needed:
	  /usr/local/bin
	  /var/lib/gems/2.7.0
	  /var/lib/gems/2.7.0/build_info
	  /var/lib/gems/2.7.0/cache
	  /var/lib/gems/2.7.0/doc
	  /var/lib/gems/2.7.0/extensions
	  /var/lib/gems/2.7.0/gems
	  /var/lib/gems/2.7.0/plugins
	  /var/lib/gems/2.7.0/specifications
	Using multipart-post 2.1.1
	Using faraday-rack 1.0.0
	Using faraday-retry 1.0.3
	Using rb-fsevent 0.11.1
	Using mercenary 0.3.6
	Using rouge 3.26.0
	Using forwardable-extended 2.6.0
	Using gemoji 3.0.1
	Using faraday-net_http_persistent 1.2.0
	Using rubyzip 2.3.2
	Using unicode-display_width 1.8.0
	Using faraday-httpclient 1.0.1
	Using rexml 3.2.5
	Using safe_yaml 1.0.5
	Fetching unf_ext 0.0.8.1
	Fetching eventmachine 1.2.7
	Fetching commonmarker 0.23.4
	Using faraday-patron 1.0.0
	Using jekyll-swiss 1.0.0
	Using liquid 4.0.3
	Using pathutil 0.16.2
	Using i18n 0.9.5
	Using tzinfo 1.2.9
	Fetching ffi 1.15.5
	Using activesupport 6.0.4.7
	Using faraday-net_http 1.0.1
	Using faraday-multipart 1.0.3
	Using ruby2_keywords 0.0.5
	Using jekyll-paginate 1.1.0
	Using faraday 1.10.0
	Fetching racc 1.6.0
	Using terminal-table 1.8.0
	Using addressable 2.8.0
	Using kramdown 2.3.2
	Using sawyer 0.8.2
	Fetching http_parser.rb 0.8.0
	Using octokit 4.22.0
	Using coffee-script 2.4.1
	Using kramdown-parser-gfm 1.1.0
	Using jekyll-coffeescript 1.1.1
	Using jekyll-gist 1.5.0
	
	
	Your user account isn't allowed to install to the system RubyGems.
	  You can cancel this installation and run:
	
	      bundle config set --local path 'vendor/bundle'
	      bundle install
	
	  to install the gems into ./vendor/bundle/, or you can enter your password
	  and install the bundled gems to RubyGems using sudo.
	
	  Password: 
	Installing commonmarker 0.23.4 with native extensions
	Installing racc 1.6.0 with native extensions
	Installing http_parser.rb 0.8.0 with native extensions
	Installing eventmachine 1.2.7 with native extensions
	Installing unf_ext 0.0.8.1 with native extensions
	Installing ffi 1.15.5 with native extensions
	Fetching nokogiri 1.13.4 (x86_64-linux)
	Fetching unf 0.1.4
	Installing nokogiri 1.13.4 (x86_64-linux)
	Fetching html-pipeline 2.14.1
	Installing html-pipeline 2.14.1
	Installing unf 0.1.4
	Fetching simpleidn 0.2.1
	Installing simpleidn 0.2.1
	Fetching dnsruby 1.61.9
	Installing dnsruby 1.61.9
	Fetching jekyll-commonmark 1.4.0
	Installing jekyll-commonmark 1.4.0
	Fetching em-websocket 0.5.3
	Installing em-websocket 0.5.3
	Fetching ethon 0.15.0
	Fetching rb-inotify 0.10.1
	Installing rb-inotify 0.10.1
	Installing ethon 0.15.0
	Fetching sass-listen 4.0.0
	Fetching listen 3.7.1
	Fetching typhoeus 1.4.0
	Installing listen 3.7.1
	Installing sass-listen 4.0.0
	Installing typhoeus 1.4.0
	Fetching sass 3.7.4
	Fetching jekyll-watch 2.2.1
	Fetching github-pages-health-check 1.17.9
	Installing jekyll-watch 2.2.1
	Installing github-pages-health-check 1.17.9
	Installing sass 3.7.4
	Fetching jekyll-sass-converter 1.5.2
	Installing jekyll-sass-converter 1.5.2
	Fetching jekyll 3.9.2
	Installing jekyll 3.9.2
	Fetching jekyll-avatar 0.7.0
	Fetching jekyll-mentions 1.6.0
	Fetching jekyll-optional-front-matter 0.3.2
	Fetching jekyll-readme-index 0.3.0
	Fetching jemoji 0.12.0
	Fetching jekyll-remote-theme 0.4.3
	Fetching jekyll-seo-tag 2.8.0
	Fetching jekyll-sitemap 1.4.0
	Fetching jekyll-default-layout 0.1.4
	Fetching jekyll-feed 0.15.1
	Fetching jekyll-github-metadata 2.13.0
	Fetching jekyll-include-cache 0.2.1
	Fetching jekyll-titles-from-headings 0.5.3
	Fetching jekyll-commonmark-ghpages 0.2.0
	Fetching jekyll-relative-links 0.6.1
	Fetching jekyll-redirect-from 0.16.0
	Installing jekyll-optional-front-matter 0.3.2
	Installing jekyll-readme-index 0.3.0
	Installing jekyll-remote-theme 0.4.3
	Installing jekyll-default-layout 0.1.4
	Installing jemoji 0.12.0
	Installing jekyll-relative-links 0.6.1
	Installing jekyll-mentions 1.6.0
	Installing jekyll-include-cache 0.2.1
	Installing jekyll-redirect-from 0.16.0
	Installing jekyll-commonmark-ghpages 0.2.0
	Installing jekyll-titles-from-headings 0.5.3
	Installing jekyll-avatar 0.7.0
	Installing jekyll-feed 0.15.1
	Installing jekyll-github-metadata 2.13.0
	Installing jekyll-seo-tag 2.8.0
	Installing jekyll-sitemap 1.4.0
	Fetching hawkins 2.0.5
	Fetching jekyll-theme-midnight 0.2.0
	Fetching jekyll-theme-tactile 0.2.0
	Fetching jekyll-theme-dinky 0.2.0
	Fetching jekyll-theme-hacker 0.2.0
	Fetching minima 2.5.1
	Fetching jekyll-theme-leap-day 0.2.0
	Fetching jekyll-theme-merlot 0.2.0
	Fetching jekyll-theme-minimal 0.2.0
	Fetching jekyll-theme-primer 0.6.0
	Fetching jekyll-theme-slate 0.2.0
	Fetching jekyll-theme-architect 0.2.0
	Fetching jekyll-theme-cayman 0.2.0
	Fetching jekyll-theme-modernist 0.2.0
	Fetching jekyll-theme-time-machine 0.2.0
	Installing jekyll-theme-hacker 0.2.0
	Installing minima 2.5.1
	Installing jekyll-theme-time-machine 0.2.0
	Installing jekyll-theme-cayman 0.2.0
	Installing jekyll-theme-modernist 0.2.0
	Installing jekyll-theme-dinky 0.2.0
	Installing jekyll-theme-primer 0.6.0
	Installing jekyll-theme-tactile 0.2.0
	Installing jekyll-theme-architect 0.2.0
	Installing jekyll-theme-minimal 0.2.0
	Installing hawkins 2.0.5
	Installing jekyll-theme-slate 0.2.0
	Installing jekyll-theme-merlot 0.2.0
	Installing jekyll-theme-leap-day 0.2.0
	Installing jekyll-theme-midnight 0.2.0
	Fetching github-pages 226
	Installing github-pages 226
	Bundle complete! 4 Gemfile dependencies, 102 gems now installed.
	Use `bundle info [gemname]` to see where a bundled gem is installed.
	Post-install message from dnsruby:
	Installing dnsruby...
	  For issues and source code: https://github.com/alexdalitz/dnsruby
	  For general discussion (please tell us how you use dnsruby): https://groups.google.com/forum/#!forum/dnsruby
	Post-install message from sass:
	
	Ruby Sass has reached end-of-life and should no longer be used.
	
	* If you use Sass as a command-line tool, we recommend using Dart Sass, the new
	  primary implementation: https://sass-lang.com/install
	
	* If you use Sass as a plug-in for a Ruby web framework, we recommend using the
	  sassc gem: https://github.com/sass/sassc-ruby#readme
	
	* For more details, please refer to the Sass blog:
	  https://sass-lang.com/blog/posts/7828841
	
	Post-install message from html-pipeline:
	-------------------------------------------------
	Thank you for installing html-pipeline!
	You must bundle Filter gem dependencies.
	See html-pipeline README.md for more details.
	https://github.com/jch/html-pipeline#dependencies
	-------------------------------------------------
\end{verbatim}
``If you want to see what''s been installed, run \texttt{gem list} before and after \texttt{bundle install}.''
\begin{verbatim}
	hong@hong-Katana-GF76-11UC:~/1_NQBH/NQBH.github.io$ gem list
	
	*** LOCAL GEMS ***
	
	activesupport (6.0.4.7)
	addressable (2.8.0)
	benchmark (default: 0.1.0)
	bigdecimal (default: 2.0.0)
	bundler (2.3.12, default: 2.1.4)
	cgi (default: 0.1.0)
	coffee-script (2.4.1)
	coffee-script-source (1.11.1)
	colorator (1.1.0)
	commonmarker (0.23.4)
	concurrent-ruby (1.1.10)
	csv (default: 3.1.2)
	date (default: 3.0.0)
	dbm (default: 1.1.0)
	delegate (default: 0.1.0)
	did_you_mean (default: 1.4.0)
	dnsruby (1.61.9)
	em-websocket (0.5.3)
	etc (default: 1.1.0)
	ethon (0.15.0)
	eventmachine (1.2.7)
	execjs (2.8.1)
	faraday (1.10.0)
	faraday-em_http (1.0.0)
	faraday-em_synchrony (1.0.0)
	faraday-excon (1.1.0)
	faraday-httpclient (1.0.1)
	faraday-multipart (1.0.3)
	faraday-net_http (1.0.1)
	faraday-net_http_persistent (1.2.0)
	faraday-patron (1.0.0)
	faraday-rack (1.0.0)
	faraday-retry (1.0.3)
	fcntl (default: 1.0.0)
	ffi (1.15.5)
	fiddle (default: 1.0.0)
	fileutils (default: 1.4.1)
	forwardable (default: 1.3.1)
	forwardable-extended (2.6.0)
	gdbm (default: 2.1.0)
	gemoji (3.0.1)
	getoptlong (default: 0.1.0)
	github-pages (226)
	github-pages-health-check (1.17.9)
	hawkins (2.0.5)
	html-pipeline (2.14.1)
	http_parser.rb (0.8.0)
	i18n (0.9.5)
	io-console (default: 0.5.6)
	ipaddr (default: 1.2.2)
	irb (default: 1.2.6)
	jekyll (3.9.2)
	jekyll-avatar (0.7.0)
	jekyll-coffeescript (1.1.1)
	jekyll-commonmark (1.4.0)
	jekyll-commonmark-ghpages (0.2.0)
	jekyll-default-layout (0.1.4)
	jekyll-feed (0.15.1)
	jekyll-gist (1.5.0)
	jekyll-github-metadata (2.13.0)
	jekyll-include-cache (0.2.1)
	jekyll-mentions (1.6.0)
	jekyll-optional-front-matter (0.3.2)
	jekyll-paginate (1.1.0)
	jekyll-readme-index (0.3.0)
	jekyll-redirect-from (0.16.0)
	jekyll-relative-links (0.6.1)
	jekyll-remote-theme (0.4.3)
	jekyll-sass-converter (1.5.2)
	jekyll-seo-tag (2.8.0)
	jekyll-sitemap (1.4.0)
	jekyll-swiss (1.0.0)
	jekyll-theme-architect (0.2.0)
	jekyll-theme-cayman (0.2.0)
	jekyll-theme-dinky (0.2.0)
	jekyll-theme-hacker (0.2.0)
	jekyll-theme-leap-day (0.2.0)
	jekyll-theme-merlot (0.2.0)
	jekyll-theme-midnight (0.2.0)
	jekyll-theme-minimal (0.2.0)
	jekyll-theme-modernist (0.2.0)
	jekyll-theme-primer (0.6.0)
	jekyll-theme-slate (0.2.0)
	jekyll-theme-tactile (0.2.0)
	jekyll-theme-time-machine (0.2.0)
	jekyll-titles-from-headings (0.5.3)
	jekyll-watch (2.2.1)
	jemoji (0.12.0)
	json (default: 2.3.0)
	kramdown (2.3.2)
	kramdown-parser-gfm (1.1.0)
	liquid (4.0.3)
	listen (3.7.1)
	logger (default: 1.4.2)
	matrix (default: 0.2.0)
	mercenary (0.3.6)
	minima (2.5.1)
	minitest (5.15.0, 5.13.0)
	multipart-post (2.1.1)
	mutex_m (default: 0.1.0)
	net-pop (default: 0.1.0)
	net-smtp (default: 0.1.0)
	net-telnet (0.1.1)
	nokogiri (1.13.4 x86_64-linux)
	observer (default: 0.1.0)
	octokit (4.22.0)
	open3 (default: 0.1.0)
	openssl (default: 2.1.2)
	ostruct (default: 0.2.0)
	pathutil (0.16.2)
	power_assert (1.1.7)
	prime (default: 0.1.1)
	pstore (default: 0.1.0)
	psych (default: 3.1.0)
	public_suffix (4.0.7)
	racc (1.6.0, default: 1.4.16)
	rake (13.0.3)
	rb-fsevent (0.11.1)
	rb-inotify (0.10.1)
	rdoc (default: 6.2.1.1)
	readline (default: 0.0.2)
	readline-ext (default: 0.1.0)
	reline (default: 0.1.5)
	rexml (3.2.5, default: 3.2.3.1)
	rouge (3.26.0)
	rss (default: 0.2.8)
	ruby2_keywords (0.0.5)
	rubygems-update (3.2.5)
	rubyzip (2.3.2)
	safe_yaml (1.0.5)
	sass (3.7.4)
	sass-listen (4.0.0)
	sawyer (0.8.2)
	sdbm (default: 1.0.0)
	simpleidn (0.2.1)
	singleton (default: 0.1.0)
	stringio (default: 0.1.0)
	strscan (default: 1.0.3)
	terminal-table (1.8.0)
	test-unit (3.3.9)
	thread_safe (0.3.6)
	timeout (default: 0.1.0)
	tracer (default: 0.1.0)
	typhoeus (1.4.0)
	tzinfo (1.2.9)
	unf (0.1.4)
	unf_ext (0.0.8.1)
	unicode-display_width (1.8.0)
	uri (default: 0.10.0)
	webrick (default: 1.6.1)
	xmlrpc (0.3.0)
	yaml (default: 0.1.0)
	zeitwerk (2.5.4)
	zlib (default: 1.1.0)
\end{verbatim}

\subsection{Getting Online}
``$\ldots$ made some changes from the template, checked them out locally, \& you're ready to share your website with the world. This is a 2 step process. 1st we need to upload all of our modified files to the GitHub repo we forked from the template. Then we need to configure GitHub Pages to build \& deploy our website. Finally, if you want a custom domain name, we need to do some configuration outside of GitHub Pages to connect your domain name with your website.''

\subsubsection{Uploading changes to GitHub}
``To upload your changes to GitHub, we 1st have to make Git locally aware of them. We do this by committing the changes, then pushing them to the repo on GitHub.

\begin{quotation}
	``Git-speak aside: Git stores file histories as a series of changes or differences. A batch of changes (which can include changes in 1 or more files) is called a \emph{commit}. When you want to tell the remote repo (the one on GitHub) about changes you've made, you push a commit from the local repo to the remote one. Once you do this, GitHub looks at the differences \& modifies the files in the remote repo.''
\end{quotation}
Before we can commit the changes, we need to stage them. This just involved telling Git what changes we want to commit. To make our lives easier, let's check in on what changes we've made by \texttt{git status}. You should get results that look something similar to this:
\begin{verbatim}
	On branch master
	Your branch is up to date with 'origin/master'.
	
	Changes not staged for commit:
	  (use "git add <file>..." to update what will be committed)
	  (use "git checkout -- <file>..." to discard changes in working directory)
	
		modified:   _config.yml
	
	Untracked files:
	  (use "git add <file>..." to include in what will be committed)
	
		Gemfile.lock
	
	no changes added to commit (use "git add" and/or "git commit -a")
\end{verbatim}
We can ignore the 1st part for now. The 2nd part (\texttt{Changes not staged for commit}) will list any files that Git knows about that have changed. The 3rd part (\texttt{Untracked files}) includes files we haven't told Git about, so as far as it's concerned they don't exist.'' ``If you want to verify the changes you made, you'll want to \textit{diff} the file. Do this with
\begin{verbatim}
	git diff _config.yml
\end{verbatim}
[$\ldots$] Note that your output may or may not be color-coded depending on what type of system you're on \& your Git settings. Each line that begins with a \texttt{+} indicates an insertion \& each line that starts with a \texttt{-} is a deletion.'' The \texttt{git commit} \& \texttt{git push} steps are straightforward. ``Your repo on GitHub is the ``remote'' that your local git needs access to.''

\begin{remark}
	\begin{itemize}
		\item ``You can omit the \verb|-m "..."| with \verb|git commit|, but it will open up your system's default editor (likely vim, emacs, or nano), none of which are remotely beginner-friendly. Until you decide to learn 1 of them, \verb|git commit -m| is perfectly fine.
		\item This is the \emph{hash} for the commit; it's a 40 digit string that uniquely identifies the changes that you're committing. This is useful if you want to retrace your steps or undo things.
		\item If you've set up \href{https://help.github.com/en/github/authenticating-to-github/connecting-to-github-with-ssh}{SSH authentication} for GitHub, then you won't be prompted to enter your credentials.
		\item Ignore the ``Choose a theme'' button; it's for use with bare bones GitHub pages sites \& the \texttt{academicpages} template supplies all the components for its theme.''
	\end{itemize}
\end{remark}

\section{Customizing an Academic Website}
In this section, I followed \href{https://jayrobwilliams.com/posts/2020/07/customizing-website/}{Rob Williams\texttt{/}Customizing an Academic Website}.

``While \texttt{academicpages} is a great template, the accompanying documentation isn't \textit{particularly} useful if you want to make any changes that go beyond content into formatting. Thus, each new tweak I implement begins with something of a scavenger hunt.

Essentially, you need to track down there in the source code of your website a variable is originally defined, \& then edit it there. Luckily, RStudio makes this relative straightforward with its Find in Files function. You can access this special search from the Edit menu, or by pressing \textsc{Cmd $+$ Shift $+$ C} on MacOs or \texttt{Ctrl $+$ Shift $+$ C} otherwise. Once you've brought up the Find in Files dialog, enter the name of the variable you're looking for in the `Find' box \& your website's directory in the `Search in' box $\ldots$''

I just list below the names of modifications mentioned. I will add more details to these if I am care about them.

\subsection{Easier on The Eyes}

\subsection{Fixing Fancy Icons}

\subsection{Adding Some Color}

\subsection{Pushing Buttons}

\subsection{Going Forward}
``This is just a brief overview of the ways you can tweak your website from the base provided by the template. Let Google \& Stack Overflow be your guides. There will be some trial \& error, but the beauty of \texttt{git} is that even if you break something it's easy to roll back to changes to when everything was working.''

\section{Adding Content to an Academic Website}
In this section, I followed \href{https://jayrobwilliams.com/posts/2020/08/website-content/}{Rob Williams\texttt{/}Adding Content to an Academic Website}.

``However, adding new pages of tweaking the existing pages can be a little intimidating, \& I realized I should probably walk through how to do so. Luckily Jekyll's use of Markdown makes it really easy to add new content!''

\subsection{\#Content}
``Editing the welcome page for your site (\verb|_page/about.md|) is relatively straightforward. Things get a little trickier if you want to build an entirely new page to your website.''

\subsubsection{1st steps}
``1st things 1st, we need to create a file for the page itself. The main pages for your website are generated from \href{https://en.wikipedia.org/wiki/Markdown}{Makrdown} files contained in the \verb|_pages| directory. Create a new file called \texttt{software.md} in \verb|_pages|. Now, open it up in RStudio or your text editor of choice. If you've looked at the \texttt{.md} files for other pages, you'll notice that they all start with a similar block of text. This is a \href{https://en.wikipedia.org/wiki/YAML}{YAML} header that tells Jekyll the basic information needed to build the page. There are lots of different options you can include, but the only 2 you really need are the \texttt{permalink} for the page \& its \texttt{title}. Add the following to top of \texttt{software.md}:
\begin{verbatim}
	---
	permalink: /software/
	title: "Software"
	---
\end{verbatim}
Anything after that 2nd line of dashes will be translated into actual content on the page.''

\subsubsection{Fill it out}
``A couple of things to notice:
\begin{itemize}
	\item You can create headings using pound signs
	\begin{itemize}
		\item More pound signs produce smaller headings
	\end{itemize}
	\item You can create links using standard Markdown syntax, e.g., \texttt{[link text] (url)}
	\begin{itemize}
		\item If you're linking to a page generated from a source \texttt{.md} file in \verb|_pages|, just put a slash before the page name \& don't include any extension, e.g., \texttt{[software](/software)}
	\end{itemize}
	\item You can embed images by adding an exclamation point before the opening \texttt{[} in Markdown link syntax, e.g.,\\\texttt{![](/images/profile.png)}
	\item You can create code blocks like the ones on this page by enclosing text in triple backticks
	\begin{itemize}
		\item Put the name of the programming language after the opening backticks to activate \href{https://en.wikipedia.org/wiki/Syntax_highlighting}{syntax highlighting}
	\end{itemize}
	\item You can also embed raw HTML directly like I used to include 3 images next to one another.
\end{itemize}
These tools should be sufficient to let you build an awesome new page for your website. However, letting visitors actually get to your new page requires a little more work.''

\subsubsection{You can't get there from here}
``If you want to just add a link to your new page from an existing page, like the homepage, that's easy \& can be accomplished by adding a link to the Markdown source in \verb|_pages/about.md|. That's how I added my \href{https://jayrobwilliams.com/teaching-materials/}{teaching materials} page; it's just a link on my \href{https://jayrobwilliams.com/teaching}{teaching} page. But what about if you want your new page to be easily accessed from the fancy navigation bar at the top of the site?

To do that, we'll need to edit the files Jekyll uses to control navigation on the site. Open up \verb|_data/nagivation.yml| \& get ready to add our new page to the to menu. This is what it looks like in the template:
\begin{verbatim}
	# main links links
	main:
	  - title: "Publications"
	    url: /publications/
	
	  - title: "Talks"
	    url: /talks/    
	
	  - title: "Teaching"
	    url: /teaching/    
	    
	  - title: "Portfolio"
	    url: /portfolio/
	        
	  - title: "Blog Posts"
	    url: /year-archive/
	    
	  - title: "CV"
	    url: /cv/
	    
	  - title: "Guide"
	    url: /markdown/
\end{verbatim}
The order that items appear in top-to-bottom in this file is also the order they'll appear in left-to-right in the navigation bar. So decide where you want your new page to go, \& slot it in. This is what \verb|_data/navigation.yml| looks like for my website:
\begin{verbatim}
	# main links links
	main:
	  - title: "Publications"
	    url: /publications/
	    
	  - title: "Research"
	    url: /research/
	
	  - title: "Teaching"
	    url: /teaching/
	
	  - title: "Software"
	    url: /software/
	
	  - title: "Posts"
	    url: /posts/
	    
	  - title: "CV"
	    url: /cv/
\end{verbatim}
[$\ldots$] Removing elements from this file [\verb|_data/navigation.yml|] drops them from the navigation menu, so if there are any other pages in the template you don't plan to use, go ahead \& remove them now.'' Then \texttt{git push} all the changes.

\subsection{Uploading Files}
``1 of the advantages of using GitHub pages to host your website is that you don't have to use Dropbox to host \textsc{pdf}s of your working papers \& published articles, not to mention your CV. If you use Wix or WordPress, you may have to upload your files to Dropbox, \& then link to them on your site. This process has 3 major downsides:
\begin{enumerate}
	\item You have to update your website in 2 places to add or update a \textsc{pdf}
	\item Google Scholar will ignore Dropbox links, so you won't get a record of your scholarship online
	\item If someone clicks a Dropbox while viewing your site on their phone or tablet, it may take them to the Dropbox app or pop up a message about the app not being installed.
\end{enumerate}
\fbox{All of these are less than ideal.} Luckily, GitHub  Pages has the capability to addresses all 3 already built in. When you make an update to your website \& \texttt{git push} it to GitHub, \textit{all} tracked files get uploaded with it. This means it's super easy to upload your \textsc{pdf}s to your site \& link directly to them.''

\begin{example}
	``1st, copy the \textsc{pdf} into the \verb|files/pdf| directory in your site's directory. Next we need to tell git about this file, which we do with
	\begin{verbatim}
		git add files/pdf/working-paper.pdf
		git commit -m "add working paper"
		git push
	\end{verbatim}
	Don't forget to add a link to the paper somewhere on your research page so that visitors can access it. Here's an example of what that link might look like: \verb|[Working Paper](/files/pdf/working-paper.pdf)|. \& if you want to use the \href{https://jayrobwilliams.com/posts/2020/07/customizing-website/#pushing-buttons}{fancy button} from my post on \href{https://jayrobwilliams.com/posts/2020/07/customizing-website/}{customizing your site}, you would do this: \verb|[Working Paper](/files/pdf/working-paper.pdf){: .btn--research}|.
\end{example}

\subsection{Designing for Mobile}
``1 of the advantages of the \href{https://academicpages.github.io/}{academicpages} templates is that it is \href{https://en.wikipedia.org/wiki/Responsive_web_design}{responsive}, meaning that layouts change automatically with screen size to present content in the most efficient manner.'' ``When you're editing your website, it's a good idea to periodically check how it appears on a phone, as it's likely that a number of visitors to your site will view it on their phones.

To do so, you can use tools like Chrome's \href{https://developers.google.com/web/tools/chrome-devtools/device-mode}{device mode}, but this can be annoying \& doesn't perfectly capture the experience of navigating your site on a small touchscreen. The best way to do that is, unsurprisingly, to use your actual phone. However, this requires a slight tweak to our usual \verb|bundle exec jekyll serve| command. We need to add a \verb|--host| argument to the command, where the value of the argument is our computer's IP address. There are many ways to look this up, but here are 2 quick ones you can execute from the terminal:
\begin{itemize}
	\item On MacOS:
	\begin{verbatim}
		ifconfig en0 | grep inet | grep -v inet6 | awk '{print $2}'
	\end{verbatim}
	\item On Linux: \verb|hostname -I|.
\end{itemize}
What each of these will do is capture the \textit{local} IP address of the computer. Often this will be something like \texttt{192.168.1.x} or \texttt{10.0.0.x}. This won't let you access the site from outside your network over the Internet, but it \textit{will} let you access it locally on your own network. Once you've found your local IP address, you can serve your site on your local network, letting you view it on your phone or tablet. E.g., my IP address is \texttt{192.168.1.6}, so putting it all together I get:
\begin{verbatim}
	bundle exec jekyll serve --host 192.168.1.6
\end{verbatim}
This is quite a lot to type, \& your computer's local IP address can change occasionally, so you can't just keep putting in the same IP address each time. To save yourself some time by creating an \href{https://en.wikipedia.org/wiki/Alias_(command)}{alias} for the command. An alias is simply a way to refer to a longer command with a shorter label. To do this, you'll need to edit your \verb|.bash_profile| configuration file. The easiest way to do this is to run
\begin{verbatim}
	nano ~/.bash_profile
\end{verbatim}
This will open up your \verb|.bash_profile| in \href{https://en.wikipedia.org/wiki/GNU_nano}{nano}, a simple text editor. I've decided to call my aliased command \verb|server-site|, but you could call it anything you want. Scroll down to the end of the file \& add either
\begin{verbatim}
	alias serve-site="bundle exec jekyll serve --host=$(ifconfig en0 | grep inet | grep -v inet6 |
	    awk '{print $2}')"
\end{verbatim}
for MacOS or
\begin{verbatim}
	alias serve-site="bundle exec jekyll serve --host=$(hostname -I)"
\end{verbatim}
for Linux. Once you've added this line, save the file by pressing \texttt{Ctrl $+$ O} \& then \texttt{Enter} to use the existing filename, overwriting the old version of \verb|.bash_profile|. Then press \texttt{Ctrl $+$ X} to close nano. The last step is to tell your terminal about this new alias. You can accomplish this with
\begin{verbatim}
	source ~/.bash_profile
\end{verbatim}
regardless of whether you're on Linux or MacOS. Now whenever you want to check out your website on a mobile device, you can just navigate to your website's directory \& use the new \verb|serve-site| alias to launch it locally.''

\begin{remark}
	\begin{itemize}
		\item ``As a senior faculty member once pointed out to me, the search committee member who didn't fully read your application is most likely to pull up your website on their phone during a committee meeting.
		\item I'm assuming that you're using bash as your shell. If you're using a different shell, see this list \href{https://kb.iu.edu/d/abdy}{University Information Technology Services\texttt{/}Startup \& termination files used by the various Unix shells} for which configuration files you should be editing. Other shells may also define aliases in different ways.
		\item Feel free to use a different editor or use the \texttt{edit} command if you've set the default editor to your preferred editor.''
	\end{itemize}
\end{remark}

\section{NQBH.github.io}
Final result: \textsc{url}: \url{nqbh.github.io}.

\subsection{Hit Counter}
See, e.g., \href{https://stackoverflow.com/questions/57747640/how-to-implement-a-basic-page-view-counter-for-a-github-pages-powered-site}{StackOverflow\texttt{/}How to implement a basic page view counter for a GitHub-Pages-powered site?}

\subsubsection{\href{https://javascript.plainenglish.io/how-to-count-page-views-with-the-count-api-afc9369c1f8f}{JavaScript in Plain English\texttt{/}Mehdi Aoussiad. How to Count Page Views with The Count API}}

\begin{flushright}
	An easy way to count your page views with JavaScript.
\end{flushright}
``1 of the useful functionalities that you may need to have on your website is a counter that counts the views of your web pages. The easiest way to do that is through the \texttt{CountAPI} which gives you this amazing functionality.'' ``$\ldots$ learn how to count our page views using the \texttt{CountAPI} in JavaScript.''

\paragraph{The \texttt{CountAPI}.} ``This API allows you to make numeric counters. It's a very useful API to track the number of hits a page received. It also allows you to know the number of users that, e.g., clicked on a button. So it also counts events. The \href{https://countapi.xyz/}{\texttt{CountAPI}} provides a hit endpoint each time it is called, the counter will increase by 1. Check it out in the link below: \url{https://api.countapi.xyz/hit/namespace/key}. You should know that each counter is identified inside a \textit{namespace} with a \textit{key}. The \textit{namespace} should be unique, so you need to replace it with your website domain.''

\begin{example}
	``$\ldots$ you want to count \& display your web page views, the \texttt{CountAPI} allows you to do that by using their NPM package, JSONP, XHR, or jQuery. In the example below, we will use JSONP. You can check the other ways on \url{https://countapi.xyz/}. The 1st thing we do is creating our \textsc{html}:	
	\begin{verbatim}
		<body>
		
		  <h1>This page got <span id="visits"></span> views.</h1>
		
		</body>
	\end{verbatim}
	In the head tag, you will have to add a script CDN for the \texttt{CountAPI}. It will be like the following:	
	\begin{verbatim}
		<head>
		
		<script async src="https://api.countapi.xyz/hit/mysite.com/visits?  callback=callbackName"></script>
		
		</head>
	\end{verbatim}
	You will have to replace the \verb|mysite.com| with your own domain. Also, replace the \verb|callbackName| with any callback name that we will create below in JavaScript.
	
	Now we will go to JavaScript in order to create a callback that will get the number of visits \& replace it inside the element that has an ID \verb|visits|.	
	\begin{verbatim}
		function callbackName(response) {
		    document.getElementById('visits').innerText = response.value;
		}
	\end{verbatim}
	We will not call this function because it's already called on the CDN above.'' ``As you can see, this is the easiest way to count the views that your web page received.''
\end{example}
``The \texttt{CountAPI} makes it easier to count page views \& display them on the page. This API also has some more functionalities $\ldots$''.

\subsection{\href{https://countapi.xyz/}{\texttt{CountAPI}}}
``This API allows you to create simple numeric counters. IaaS, Integer as a Service. It goes down to:
\begin{itemize}
	\item Create a counter \& restrict its operations
	\item Reset the value of a counter
	\item Increment\texttt{/}decrement a counter.
\end{itemize}
All counters are accessible if you know the key \& there are not private counters (yet?). \textit{Want to track the number of hits a page had?} Sure. \textit{Want to know the number of users that clicked on the button ``Feed Cow''?} There you go.''

\paragraph{TL;DR.} ``Each counter is identified inside a \texttt{namespace} with a \texttt{key}. The namespace should be unique, so its recommend using your site's domain. In each namespace you can generate all the counters you may need.

The \texttt{hit} endpoint provides increment by 1 counters directly. Each time its requested the counter will increase by 1: \url{https://api.countapi.xyz/hit/namespace/key} $\Rightarrow$ \verb|200 { "value": 1234 }|.

\begin{example}
	``$\ldots$ want to display the number of pageviews a site received.	
	\begin{verbatim}
		<div id="visits">...</div>
	\end{verbatim}
	Remeber to change \verb|mysite.com| with your site's domain.
\end{example}

\subparagraph{Using countapi-js.}
\begin{verbatim}
	import countapi from 'countapi-js';
	
	countapi.visits().then((result) => {
	    console.log(result.value);
	});
\end{verbatim}

\subparagraph{Using JSONP.}
\begin{verbatim}
	<script>
	function cb(response) {
	    document.getElementById('visits').innerText = response.value;
	}
	</script>
	<script async src="https://api.countapi.xyz/hit/mysite.com/visits?callback=cb"></script>
\end{verbatim}

\subparagraph{Using XHR.}
\begin{verbatim}
	var xhr = new XMLHttpRequest();
	xhr.open("GET", "https://api.countapi.xyz/hit/mysite.com/visits");
	xhr.responseType = "json";
	xhr.onload = function() {
	    document.getElementById('visits').innerText = this.response.value;
	}
	xhr.send();
\end{verbatim}

\subparagraph{Using jQuery.}
\begin{verbatim}
	$.getJSON("https://api.countapi.xyz/hit/mysite.com/visits", function(response) {
	    $("#visits").text(response.value);
	});
\end{verbatim}

\subparagraph{Multiple pages.} ``If you want to have a counter for each individual page you can replace \texttt{visits} with a unique identifier for each page, i.e., \texttt{index, contact, item-1234}. Check \href{https://countapi.xyz/#format}{the right format} a key must have.

Alternatively, you can use some reserved words that are replaced server-side.

E.g., if a request is made from \texttt{https://mysite.com/example/page}:
\begin{itemize}
	\item \texttt{:HOST}: will be replaced with \texttt{mysite.com}
	\item \texttt{:PATHNAME}: will be replaced with \texttt{examplepage}.
\end{itemize}

\begin{remark}
	Reserved words are padded with dots if their length is $< 3$.
\end{remark}
So you could use something like:
\begin{verbatim}
	https://api.countapi.xyz/hit/mysite.com/:PATHNAME:
\end{verbatim}
Or even more generic (though not recommended):
\begin{verbatim}
	https://api.countapi.xyz/hit/:HOST:/:PATHNAME:
\end{verbatim}

\begin{remark}[Important]
	If you want to know the actual key used you can check the \verb|X-Path| header.''
\end{remark}

\subparagraph{Counting events.} ``You can use the API to count any kind of stuff, lets say:
\begin{verbatim}
	<button onclick="clicked()">Press Me!</button>
	<script>
	function clicked() {
	    var xhr = new XMLHttpRequest();
	    xhr.open("GET", "https://api.countapi.xyz/hit/mysite.com/awesomeclick");
	    xhr.responseType = "json";
	    xhr.onload = function() {
	        alert(`This button has been clicked ${this.response.value} times!`);
	    }
	    xhr.send();
	}
	</script>
\end{verbatim}

\paragraph{Libraries.} ``The official javascript promise based wrapper is available in \href{https://www.npmjs.com/package/countapi-js}{countapi-js} (v1.0.2).

\paragraph{Roadmap.} ``If this API starts getting some traction, I have some ideas in mind:
\begin{itemize}
	\item Being able to open a \href{https://developer.mozilla.org/en-US/docs/Web/API/Server-sent_events/Using_server-sent_events#Receiving_events_from_the_server}{SSE stream} \& receive updates in real time avoiding polling
	\item Enable batch creating\texttt{/}updating
	\item Unique counting?
	\item Float numbers?
	\item Generate a secret key to update\texttt{/}reset keys
\end{itemize}

\paragraph{FAQ.}
\begin{enumerate}
	\item ``\textit{Is the API free?} Completely free.
	\item \textit{Rate Limiting?} Key retrieving \& updating has \textbf{no} limits whatsoever. Key \textbf{creation} is limited to 20\texttt{/}IP\texttt{/}s.
	\item \textit{Can I delete a key?} You can't, just let the key expire. If you created a key with a wrong configuration you can always create another key.
	\item \textit{Will I blow up the \texttt{CountAPI} server?} \texttt{CountAPI} is using \href{https://redis.io/}{Redis} as database, a very fast key-value solution.''
\end{enumerate}

\paragraph{API.}

\subparagraph{Namespaces.} ``Namespaces are meant to avoid name collisions. You may specify a namespace during the creation of a key. Its recommend use the domain of the application as namespace to avoid collision with other websites. If the namespace is not specified the key is assigned to the \texttt{default} namespace. If your key resides in the default namespace you don't need to specify it.''

\subparagraph{Endpoint.} ``All requests support \href{https://developer.mozilla.org/en-US/docs/Web/HTTP/CORS}{cross-origin resource sharing} (CORS) \& SSL. You can use \href{https://en.wikipedia.org/wiki/JSONP}{JSONP} sending the callback parameter. JSONP requests will never fail, they will include the \textsc{http} code in the response. Also a $1\times 1$ GIF image is supported sending \texttt{?img}. Base API path: \url{https://api.countapi.xyz/}. In the case of a server failure, the API will send: $\Rightarrow$ \verb|500 { "error": "Error description" }|. For more technical details, see \href{https://countapi.xyz/#api}{\texttt{CountAPI}\texttt{/}API}.

\subsection{Flag Counters}
Instead of \texttt{CountAPI}, I used various flag counters, e.g., \href{https://flagcounter.com/}{Flag Counter.com}, \href{https://www.flagcounter.me/}{Flag Counter.me}, \href{https://flaghitcounter.com/}{Flag Hit Counter}, \href{https://www.worldflagcounter.com/}{World Flag Counter}. I will choose the best later but now I keep all of them. They are very similar though, only slightly different at layout \& default texts.

%------------------------------------------------------------------------------%

\selectlanguage{english}
\begin{thebibliography}{99}
	\bibitem[Williams2020]{Williams2020} Rob Williams. \href{https://jayrobwilliams.com/posts/2020/06/academic-website/}{\textit{Building an Academic Website}}. Jun 30, 2020.
\end{thebibliography}

%------------------------------------------------------------------------------%

\printbibliography[heading=bibintoc]
	
\end{document}