\documentclass{article}
\usepackage[backend=biber,natbib=true,style=alphabetic,maxbibnames=50]{biblatex}
\addbibresource{/home/nqbh/reference/bib.bib}
\usepackage[utf8]{vietnam}
\usepackage{tocloft}
\renewcommand{\cftsecleader}{\cftdotfill{\cftdotsep}}
\usepackage[colorlinks=true,linkcolor=blue,urlcolor=red,citecolor=magenta]{hyperref}
\usepackage{amsmath,amssymb,amsthm,enumitem,float,graphicx,mathtools,tikz}
\usetikzlibrary{angles,calc,intersections,matrix,patterns,quotes,shadings}
\allowdisplaybreaks
\newtheorem{assumption}{Assumption}
\newtheorem{baitoan}{}
\newtheorem{cauhoi}{Câu hỏi}
\newtheorem{conjecture}{Conjecture}
\newtheorem{corollary}{Corollary}
\newtheorem{dangtoan}{Dạng toán}
\newtheorem{definition}{Definition}
\newtheorem{dinhly}{Định lý}
\newtheorem{dinhnghia}{Định nghĩa}
\newtheorem{example}{Example}
\newtheorem{ghichu}{Ghi chú}
\newtheorem{hequa}{Hệ quả}
\newtheorem{hypothesis}{Hypothesis}
\newtheorem{lemma}{Lemma}
\newtheorem{luuy}{Lưu ý}
\newtheorem{nhanxet}{Nhận xét}
\newtheorem{notation}{Notation}
\newtheorem{note}{Note}
\newtheorem{principle}{Principle}
\newtheorem{problem}{Problem}
\newtheorem{proposition}{Proposition}
\newtheorem{question}{Question}
\newtheorem{remark}{Remark}
\newtheorem{theorem}{Theorem}
\newtheorem{vidu}{Ví dụ}
\usepackage[left=1cm,right=1cm,top=5mm,bottom=5mm,footskip=4mm]{geometry}
\def\labelitemii{$\circ$}
\DeclareRobustCommand{\divby}{%
	\mathrel{\vbox{\baselineskip.65ex\lineskiplimit0pt\hbox{.}\hbox{.}\hbox{.}}}%
}
\setlist[itemize]{leftmargin=*}
\setlist[enumerate]{leftmargin=*}

\title{Physics -- Vật Lý}
\author{Nguyễn Quản Bá Hồng\footnote{A Scientist {\it\&} Creative Artist Wannabe. E-mail: {\tt nguyenquanbahong@gmail.com}. Bến Tre City, Việt Nam.}}
\date{\today}

\begin{document}
\maketitle
\begin{abstract}
	This text is a part of the series {\it Some Topics in Advanced STEM \& Beyond}:
	
	{\sc url}: \url{https://nqbh.github.io/advanced_STEM/}.
	
	Latest version:
	\begin{itemize}
		\item {\it Physics -- Vật Lý}.
		
		PDF: {\sc url}: \url{https://github.com/NQBH/advanced_STEM_beyond/blob/main/physics/NQBH_physics.pdf}.
		
		\TeX: {\sc url}: \url{https://github.com/NQBH/advanced_STEM_beyond/blob/main/physics/NQBH_physics.tex}.
	\end{itemize}
\end{abstract}
\tableofcontents

%------------------------------------------------------------------------------%

\section{Wikipedia}

\subsection{Wikipedia{\tt/}advection}
``In the field of \href{https://en.wikipedia.org/wiki/Physics}{physics}, \href{https://en.wikipedia.org/wiki/Engineering}{engineering}, and \href{https://en.wikipedia.org/wiki/Earth_science}{earth sciences}, \textit{advection} is the \href{https://en.wikipedia.org/wiki/Transport}{transport} of a substance or quantity by bulk motion.

The properties of that substance are carried with it.

Generally the majority of the advected substance is a fluid.

The properties that are carried with the advected substance are \href{https://en.wikipedia.org/wiki/Conservation_of_energy}{conserved} properties e.g. \href{https://en.wikipedia.org/wiki/Energy}{energy}.

An example of advection is the transport of \href{https://en.wikipedia.org/wiki/Pollutant}{pollutants} or \href{https://en.wikipedia.org/wiki/Silt}{silt} in a \href{https://en.wikipedia.org/wiki/River}{river} by bulk water flow downstream.

Another commonly advected quantity is energy or \href{https://en.wikipedia.org/wiki/Enthalpy}{enthalpy}.

Here the fluid may be any material that contains thermal energy, e.g. \href{https://en.wikipedia.org/wiki/Water}{water} or \href{https://en.wikipedia.org/wiki/Air}{air}.

In general, any substance or conserved, \href{https://en.wikipedia.org/wiki/Intensive_and_extensive_properties}{extensive} quantity can be advected by a fluid that can hold or contain the quantity or substance.

%
During advection, a fluid transports some conserved quantity or material via bulk motion.

The fluid's motion is described mathematically as a \href{https://en.wikipedia.org/wiki/Vector_field}{vector field}, and the transported material is described by a \href{https://en.wikipedia.org/wiki/Scalar_field}{scalar field} showing its distribution over space.

Advection requires currents in the fluid, and so cannot happen in rigid solids.

It does not include transport of substances by \href{https://en.wikipedia.org/wiki/Molecular_diffusion}{molecular diffusion}.

%
Advection is sometimes confused with the more encompassing process of \href{https://en.wikipedia.org/wiki/Convection}{convection} which is the combination of advective transport and diffusive transport.

%
In \href{https://en.wikipedia.org/wiki/Meteorology}{meteorology} and \href{https://en.wikipedia.org/wiki/Physical_oceanography}{physical oceanography}, advection often refers to the transport of some property of the atmosphere or \href{https://en.wikipedia.org/wiki/Ocean}{ocean}, e.g., \href{https://en.wikipedia.org/wiki/Heat}{heat}, humidity (see \href{https://en.wikipedia.org/wiki/Water_vapor}{moisture}) or \href{https://en.wikipedia.org/wiki/Salinity}{salinity}.

Advection is important for the formation of \href{https://en.wikipedia.org/wiki/Orographic}{orographic} clouds and the precipitation of water from clouds, as part of the \href{https://en.wikipedia.org/wiki/Hydrological_cycle}{hydrological cycle}.

\subsubsection{Distinction between advection \& convection}
The term \textit{advection} often serves as a synonym for \href{https://en.wikipedia.org/wiki/Convection}{\textit{convection}}, and this correspondence of terms is used in the literature.

More technically, convection applies to the movement of a fluid (often due to density gradients created by thermal gradients), whereas advection is the movement of some material by the velocity of the fluid.

Thus, somewhat confusingly, it is technically correct to think of momentum being advected by the velocity field in the Navier-Stokes equations, although the resulting motion would be considered to be convection.

Because of the specific use of the term convection to indicate transport in association with thermal gradients, it is probably safer to use the term advection if one is uncertain about which terminology best describes their particular system.

\subsubsection{Meteorology}
In \href{https://en.wikipedia.org/wiki/Meteorology}{meteorology} and \href{https://en.wikipedia.org/wiki/Physical_oceanography}{physical oceanography}, advection often refers to the horizontal transport of some property of the atmosphere or \href{https://en.wikipedia.org/wiki/Ocean}{ocean}, e.g., \href{https://en.wikipedia.org/wiki/Heat}{heat}, humidity or salinity, and convection generally refers to vertical transport (vertical advection).

Advection is important for the formation of \href{https://en.wikipedia.org/wiki/Orographic_cloud}{orographic clouds} (terrain-forced convection) and the precipitation of water from clouds, as part of the \href{https://en.wikipedia.org/wiki/Hydrological_cycle}{hydrological cycle}.

\subsubsection{Other quantities}
The advection equation also applies if the quantity being advected is represented by a \href{https://en.wikipedia.org/wiki/Probability_density_function}{probability density function} at each point, although accounting for diffusion is more difficult.

\subsubsection{Mathematics of advection}
The \textit{advection equation} is the \href{https://en.wikipedia.org/wiki/Partial_differential_equation}{PDE} that governs the motion of a conserved \href{https://en.wikipedia.org/wiki/Scalar_field}{scalar field} as it is advected by a known \href{https://en.wikipedia.org/wiki/Velocity_field}{velocity vector field}.

It is derived using the scalar field's \href{https://en.wikipedia.org/wiki/Conservation_law}{conservation law}, together with \href{https://en.wikipedia.org/wiki/Gauss's_theorem}{Gauss's theorem}, and taking the \href{https://en.wikipedia.org/wiki/Infinitesimal}{infinitesimal} limit.

%
One easily visualized example of advection is the transport of ink dumped into a river.

As the river flows, ink will move downstream in a ``pulse'' via advection, as the water's movement itself transports the ink.

If added to a lake without significant bulk water flow, the ink would simply disperse outwards from its source in a \href{https://en.wikipedia.org/wiki/Diffusion}{diffusive} manner, which is not advection.

Note that as it moves downstream, the ``pulse'' of ink will also spread via diffusion.

The sum of these processes is called \href{https://en.wikipedia.org/wiki/Convection}{convection}.

\paragraph{The advection equation.} In Cartesian coordinates the advection \href{https://en.wikipedia.org/wiki/Operator_(mathematics)}{operator} is
\begin{align*}
	{\bf u}\cdot\nabla = u_x\partial_x + u_y\partial_y + u_z\partial_z,
\end{align*}
where ${\bf u} = (u_x,u_y,u_z)$ is the \href{https://en.wikipedia.org/wiki/Velocity_field}{velocity field}, and $\nabla$ is the \href{https://en.wikipedia.org/wiki/Del}{del} operator (note that \href{https://en.wikipedia.org/wiki/Cartesian_coordinate_system}{Cartesian coordinates} are used here).

%
The advection equation for a conserved quantity described by a \href{https://en.wikipedia.org/wiki/Scalar_field}{scalar field} $\psi$ is expressed mathematically by a \href{https://en.wikipedia.org/wiki/Continuity_equation}{continuity equation}:
\begin{align*}
	\partial_t\psi + \nabla\cdot\left(\psi{\bf u}\right) = 0,
\end{align*}
where $\nabla\cdot$  is the divergence operator and again ${\bf u}$ is the \href{https://en.wikipedia.org/wiki/Velocity_field}{velocity vector field}.

Frequently, it is assumed that the flow is \href{https://en.wikipedia.org/wiki/Incompressible_flow}{incompressible}, i.e., the \href{https://en.wikipedia.org/wiki/Velocity_field}{velocity field} satisfies 
\begin{align*}
	\nabla\cdot{\bf u} = 0.
\end{align*}
In this case, ${\bf u}$ is said to be \href{https://en.wikipedia.org/wiki/Solenoidal}{solenoidal}.

If this is so, the above equation can be rewritten as
\begin{align*}
	\partial_t\psi + {\bf u}\cdot\nabla\psi = 0.
\end{align*}
In particular, if the flow is steady, then
\begin{align*}
	{\bf u}\cdot\nabla\psi = 0,
\end{align*}
which shows that $\psi$ is constant along a streamline.

Hence, $\partial_t\psi = 0$, so $\psi$ does not vary in time.

%
If a vector quantity ${\bf a}$ (e.g., a \href{https://en.wikipedia.org/wiki/Magnetic_field}{magnetic field}) is being advected by the \href{https://en.wikipedia.org/wiki/Solenoidal}{solenoidal} \href{https://en.wikipedia.org/wiki/Velocity_field}{velocity field} ${\bf u}$, the advection equation above becomes: 
\begin{align*}
	\partial_t{\bf a} + ({\bf u}\cdot\nabla){\bf a} = 0.
\end{align*}
Here, ${\bf a}$ is a vector field instead of the scalar field $\psi$.

\paragraph{Solving the equation.} \textsf{A simulation of the advection equation where ${\bf u} = (\sin t,\cos t)$ is solenoidal.}

%
The advection equation is not simple to solve \href{https://en.wikipedia.org/wiki/Numerical_analysis}{numerically}: the system is a \href{https://en.wikipedia.org/wiki/Hyperbolic_partial_differential_equation}{hyperbolic partial differential equation}, and interest typically centers on \href{https://en.wikipedia.org/wiki/Continuous_function}{discontinuous} ``shock'' solutions (which are notoriously difficult for numerical schemes to handle).

%
Even with 1 space dimension and a constant \href{https://en.wikipedia.org/wiki/Velocity_field}{velocity field}, the system remains difficult to simulate.

The equation becomes
\begin{align*}
	\partial_t\psi + u_x\partial_x\psi = 0,
\end{align*}
where $\psi = \psi(t,x)$ is the scalar field being advected and $u_x$ is the $x$ component of the vector ${\bf u} = (u_x,0,0)$.

\paragraph{Treatment of the advection operator in the compressible NSEs.} According to Zang,[Zang, Thomas (1991). ``On the rotation and skew-symmetric forms for incompressible flow simulations''. Applied Numerical Mathematics. 7: 27--40. Bibcode:1991ApNM....7...27Z. doi:10.1016/0168-9274(91)90102-6.] numerical simulation can be aided by considering the \href{https://en.wikipedia.org/wiki/Skew-symmetric_matrix}{skew symmetric} form for the advection operator.
\begin{align*}
	\frac{1}{2}{\bf u}\cdot\nabla{\bf u} + \frac{1}{2}\nabla({\bf u}{\bf u}) \mbox{ where } \nabla({\bf u}{\bf u}) = \left[\nabla({\bf u}u_x),\nabla({\bf u}u_y),\nabla({\bf u}u_z)\right]
\end{align*}
and ${\bf u}$ is the same as above.

%
Since skew symmetry implies only imaginary eigenvalues, this form reduces the ``blow up'' and ``spectral blocking'' often experienced in numerical solutions with sharp discontinuities (see Boyd[Boyd, John P. (2000). \textit{Chebyshev and Fourier Spectral Methods} 2nd edition. Dover. p. 213.]).

%
Using \href{https://en.wikipedia.org/wiki/Vector_calculus_identities#Vector_dot_product}{vector calculus identities}, these operators can also be expressed in other ways, available in more software packages for more coordinate systems.
\begin{align*}
	{\bf u}\cdot\nabla{\bf u} &= \nabla\left(\frac{\|{\bf u}\|^2}{2}\right) + (\nabla\times{\bf u})\times{\bf u},\\
	\frac{1}{2}{\bf u}\cdot\nabla{\bf u} + \frac{1}{2}\nabla({\bf u}{\bf u}) &= \nabla\left(\frac{\|{\bf u}\|^2}{2}\right) + (\nabla\times{\bf u})\times{\bf u} + \frac{1}{2}{\bf u}(\nabla\cdot{\bf u}).
\end{align*}
This form also makes visible that the \href{https://en.wikipedia.org/wiki/Skew-symmetric_matrix}{skew symmetric} operator introduces error when the velocity field diverges.

Solving the advection equation by numerical methods is very challenging and there is a large scientific literature about this.'' -- \href{https://en.wikipedia.org/wiki/Advection}{Wikipedia{\tt/}advection}

%------------------------------------------------------------------------------%

\subsection{Wikipedia{\tt/}diffusion}
{\sf Some particles are \href{https://en.wikipedia.org/wiki/Dissolution_(chemistry)}{dissolved} in a glass of water. At 1st, the particles are all near 1 top corner of the glass. If the particles randomly move around (``diffuse'') in the water, they eventually become distributed randomly \& uniformly from an area of high concentration to an area of low, \& organized (diffusion continues, but with no net \href{https://en.wikipedia.org/wiki/Flux}{flux}).}

``{\it Diffusion} is the net movement of anything (e.g., atoms, ions, molecules, energy) generally from a region of higher \href{https://en.wikipedia.org/wiki/Concentration}{concentration} to a region of lower concentration. Diffusion is driven by a gradient in \href{https://en.wikipedia.org/wiki/Gibbs_free_energy}{Gibbs free energy} or \href{https://en.wikipedia.org/wiki/Chemical_potential}{chemical potential}. It is possible to diffuse ``uphill'' from a region of lower concentration to a region of higher concentration, as in \href{https://en.wikipedia.org/wiki/Spinodal_decomposition}{spinodal decomposition}. Diffusion is a stochastic process due to the inherent randomness of the diffusing entity \& can be used to model many real-life stochastic scenarios. Therefore, diffusion \& the corresponding mathematical models are used in several fields beyond physics, e.g. \href{https://en.wikipedia.org/wiki/Statistics}{statistics}, \href{https://en.wikipedia.org/wiki/Probability_theory}{probability theory}, \href{https://en.wikipedia.org/wiki/Information_theory}{information theory}, \href{https://en.wikipedia.org/wiki/Neural_networks}{neural networks}, \href{https://en.wikipedia.org/wiki/Finance}{finance}, \& \href{https://en.wikipedia.org/wiki/Marketing}{marketing}.

The concept of diffusion is widely used in many fields, including physics (\href{https://en.wikipedia.org/wiki/Molecular_diffusion}{particle diffusion}), chemistry, biology, sociology, economics, statistics, data science, \& finance (diffusion of people, ideas, data, \& price values). The central idea of diffusion, however, is common to all of these: a substance or collection undergoing diffusion spreads out from a point or location at which there is a higher concentration of that substance or collection.

A \href{https://en.wikipedia.org/wiki/Gradient}{gradient} is the change in the value of a quantity; e.g., concentration, \href{https://en.wikipedia.org/wiki/Pressure}{pressure}, or \href{https://en.wikipedia.org/wiki/Temperature}{temperature} with the change in another variable, usually \href{https://en.wikipedia.org/wiki/Distance}{distance}. A change in concentration over a distance is called a \href{https://en.wikipedia.org/wiki/Molecular_diffusion}{concentration gradient}, a change in pressure over a distance is called a \href{https://en.wikipedia.org/wiki/Pressure_gradient}{pressure gradient}, \& a change in temperature over a distance is called a \href{https://en.wikipedia.org/wiki/Temperature_gradient}{temperature gradient}.

The word {\it diffusion} derives from the Latin word, {\it diffundere}, which means ``to spread out''.

A distinguishing feature of diffusion is that it depends on particle \href{https://en.wikipedia.org/wiki/Random_walk}{random walk}, \& results in mixing or mass transport without requiring directed bulk motion. Bulk motion, or bulk flow, is the characteristic of \href{https://en.wikipedia.org/wiki/Advection}{advection}. The term \href{https://en.wikipedia.org/wiki/Convection}{convection} is used to describe the combination of both \href{https://en.wikipedia.org/wiki/Transport_phenomena}{transport phenomena}.

-- 1 đặc điểm nổi bật của sự khuếch tán là nó phụ thuộc vào bước đi ngẫu nhiên của hạt và dẫn đến sự trộn lẫn hoặc vận chuyển khối lượng mà không cần chuyển động khối có hướng. Chuyển động khối, hay dòng chảy khối, là đặc điểm của sự tiến bộ. Thuật ngữ đối lưu được sử dụng để mô tả sự kết hợp của cả hai hiện tượng vận chuyển.

In a diffusion process can be described by \href{https://en.wikipedia.org/wiki/Fick%27s_laws_of_diffusion}{Fick's laws}, it is called a normal diffusion (or Fickian diffusion), otherwise, it is called an \href{https://en.wikipedia.org/wiki/Anomalous_diffusion}{anomalous diffusion} (or non-Fickian diffusion).

When talking about the extent of diffusion, 2 length scales are used in 2 different scenarios:
\begin{enumerate}
	\item \href{https://en.wikipedia.org/wiki/Brownian_motion}{Brownian motion} of an \href{https://en.wikipedia.org/wiki/Impulse_response}{impulsive} point source (e.g., 1 single spray of perfume) -- the square root of the \href{https://en.wikipedia.org/wiki/Mean_squared_displacement}{mean squared displacement} from this point. In Fickian diffusion, this is $\sqrt{2nDt}$, where $n$ is the dimension of this Brownian motion;
	\item \href{https://en.wikipedia.org/wiki/Fick%27s_laws_of_diffusion#Example_solutions_and_generalization}{Constant concentration source} in 1D -- the diffusion length. In Fickian diffusion, this is $2\sqrt{Dt}$.
\end{enumerate}

\subsubsection{Diffusion vs. bulk flow}

\subsubsection{Diffusion in the context of different disciplines}


\subsubsection{History of diffusion in physics}

\subsubsection{Basic models of diffusion}

\subsubsection{Diffusion in physics}

\subsubsection{Random walk (random motion)}


'' -- \href{https://en.wikipedia.org/wiki/Diffusion}{Wikipedia{\tt/}diffusion}

%------------------------------------------------------------------------------%

\subsection{Wikipedia{\tt/}physics}
\href{https://en.wikipedia.org/wiki/Physics}{Wikipedia{\tt/}physics}

%------------------------------------------------------------------------------%

%------------------------------------------------------------------------------%

\section{Miscellaneous}

%------------------------------------------------------------------------------%

\printbibliography[heading=bibintoc]
	
\end{document}