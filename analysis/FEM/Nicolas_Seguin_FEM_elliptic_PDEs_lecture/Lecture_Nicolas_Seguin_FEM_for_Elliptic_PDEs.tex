\documentclass[11pt,a4paper,center,notitlepage]{article}
\usepackage[backend=biber]{biblatex}

% Use Natbib reference style
%\usepackage{natbib}
 %\bibliographystyle{abbrvnat}

%\usepackage[backend=biber,style=authoryear,natbib=true]{biblatex} % Use the bibtex backend with the authoryear citation style (which resembles APA)

\addbibresource{mybib.bib} % The filename of the bibliography
% For tabular
\usepackage{tabularx}
%\usepackage{arydshln,leftidx,mathtools}
%
%\setlength{\dashlinedash}{.4pt}
%\setlength{\dashlinegap}{.8pt}

\usepackage[autostyle=true]{csquotes} % Required to generate language-dependent quotes in the bibliography

\usepackage{algorithm}
\usepackage{algpseudocode}
\usepackage[utf8]{inputenc} 
%\usepackage[T1]{fontenc}
\usepackage[english]{babel} 
\usepackage{color}
\usepackage{textcomp,multicol,enumerate,amsmath,amssymb,amsthm,eufrak,latexsym,makeidx}
\newcommand{\vertiii}[1]{{\left\vert\kern-0.25ex\left\vert\kern-0.25ex\left\vert #1 
    \right\vert\kern-0.25ex\right\vert\kern-0.25ex\right\vert}}
% For insert figure
\usepackage{subfig}
\usepackage{graphicx,epstopdf}

%% Color Reference
\usepackage[usenames,dvipsnames,svgnames,table]{xcolor}
\usepackage[colorlinks=true,
            linkcolor=blue,
            urlcolor=gray,
            citecolor=magenta]{hyperref}
            
\allowdisplaybreaks
% No page break in Bibliography
\numberwithin{equation}{section}
\addto{\captionsenglish}{%
  \renewcommand{\bibname}{References}
}


\textwidth=16 cm
\textheight=22 cm
\topmargin= -1 cm
\oddsidemargin=0 cm
\evensidemargin=1 cm
\parindent=0.6 cm
\parskip=1.5 mm
\newtheorem{lemma}{Lemma}[section]
\newtheorem{corollary}{Corollary}[section]
\newtheorem{definition}{Definition}[section]
\newtheorem{prop}{Proposition}[section]
\newtheorem{theorem}{Theorem}[section]
\newtheorem{notation}{Notation}[section]
\newtheorem{remark}{Remark}[section]
\newtheorem{example}{Example}[section]
\newtheorem{ques}{Question}[section]
\newtheorem{sol}{Solution}[section]
\renewcommand{\thenotation}{}
\renewcommand{\thesection}{\arabic{section}}
\renewcommand{\thesubsection}
{\arabic{section}.\arabic{subsection}}
\pagestyle{plain}

% Reference
\newcommand{\er}{\eqref}

%\newtheorem{theorem}{Theorem}[section]
%\newtheorem{corollary}[theorem]{Corollary}
%\newtheorem{lemma}[theorem]{Lemma}
%\newtheorem{proposition}[theorem]{Proposition}

%\theoremstyle{definition}
%\newtheorem{definition}[theorem]{Definition}
%\newtheorem{remark}{Remark}
%\newtheorem{properties}{Properties}
%\newtheorem*{notation}{Notation}
%\newtheorem{counter}{Counter-example}
%\newtheorem{open}{Open problem}
%\newtheorem{conjecture}{Conjecture}


%% FONT commands
\newcommand{\txt}[1]{\;\text{ #1 }\;}%% Used in math only
\newcommand{\tbf}{\textbf}%% Bold face. Usage: \tbf{...}
\newcommand{\tit}{\textit}%% Italic
\newcommand{\tsc}{\textsc}%% Small caps
\newcommand{\trm}{\textrm}
\newcommand{\mbf}{\mathbf}%% Math bold
\newcommand{\mrm}{\mathrm}%% Math Roman
\newcommand{\bsym}{\boldsymbol}%% Bold math symbol
%%Macros for changing font size in math.
\newcommand{\scs}{\scriptstyle}%% as in subscript
\newcommand{\sss}{\scriptscriptstyle}%% as in sub-subscript
\newcommand{\txts}{\textstyle}
\newcommand{\dsps}{\displaystyle}
%%Macros for changing font size in text.
\newcommand{\fnz}{\footnotesize}
\newcommand{\scz}{\scriptsize}
%%\tiny<\scz<\fsz<\small<\large<\Large<\huge<\Huge
%%%%%%%%%%%%
%%%%%%%%%%%%
%% EQUATION commands
\newcommand{\be}{\begin{equation}}
\newcommand{\bel}[1]{\begin{equation}\label{#1}}
\newcommand{\ee}{\end{equation}}
%% This macro does not work with amstex.
\newcommand{\eqnl}[2]{\begin{equation}\label{#1}{#2}\end{equation}}
%%use not advisable; confusing
\newcommand{\barr}{\begin{eqnarray}}
\newcommand{\earr}{\end{eqnarray}}
\newcommand{\bars}{\begin{eqnarray*}}
\newcommand{\ears}{\end{eqnarray*}}
\newcommand{\nnu}{\nonumber \\}
%%%%%%%%%%%%%%%
%% Unnumbered THEOREM env.
%% New env. to be used for unnumbered theorem, lemma etc.
%%(but with specified name)

\newtheorem{subn}{\name}
\renewcommand{\thesubn}{}
\newcommand{\bsn}[1]{\def\name{#1}\begin{subn}}
\newcommand{\esn}{\end{subn}}
%%%%%%%%%%%%%%
%% NUMBERED THEOREM env.
%% Environments: theorem, lemma, corollary defintion and
%%related commands,
%% designed to provide consecutive numbering of these forms.


\newtheorem{sub}{\name}[section]
\newcommand{\dn}[1]{\def\name{#1}}

%%used in conjuction with sub or subn.

\newcommand{\bs}{\begin{sub}}
\newcommand{\es}{\end{sub}}
\newcommand{\bsl}[1]{\begin{sub}\label{#1}}
	
%% the above must be preceeded by \dn (name definition),
%% however this is superceded by the list of commands bth etc. below.
%%%%%%%%%%%%
%% NUMBERED THEOREM env. (cont.)
%% List of commands derived from 'sub' env. for theorem, lemma etc.
%% designed to provide consecutive numbering of these forms.
\newcommand{\bth}[1]{\def\name{Theorem}\begin{sub}\label{t:#1}}
\newcommand{\blemma}[1]{\def\name{Lemma}\begin{sub}\label{l:#1}}
\newcommand{\bcor}[1]{\def\name{Corollary}\begin{sub}\label{c:#1}}	
\newcommand{\bdef}[1]{\def\name{Definition}\begin{sub}\label{d:#1}}
\newcommand{\bprop}[1]{\def\name{Proposition}\begin{sub}\label{p:#1}}	
%% ARRAY commands.%%%%%%%%%%%%%%%%%%%%%%%%%%%%%%%%%%
%% RERERENCE commands.
%% \newcommand{\R}[1]{(\ref{#1})}

\newcommand{\R}{\eqref}
\newcommand{\re}{\eqref}
\newcommand{\rth}[1]{Theorem~\ref{t:#1}}
\newcommand{\rlemma}[1]{Lemma~\ref{l:#1}}
\newcommand{\rcor}[1]{Corollary~\ref{c:#1}}
\newcommand{\rdef}[1]{Definition~\ref{d:#1}}
\newcommand{\rprop}[1]{Proposition~\ref{p:#1}}
%%%%%%%%%%%
\newcommand{\BA}{\begin{array}}
\newcommand{\EA}{\end{array}}
\newcommand{\BAN}{\renewcommand{\arraystretch}{1.2}
\setlength{\arraycolsep}{2pt}\begin{array}}
\newcommand{\BAV}[2]{\renewcommand{\arraystretch}{#1}
\setlength{\arraycolsep}{#2}\begin{array}}
%Note: The first variable gives the amount of stretching:
%(#1) x default.
%For instance #1=1.2 means a 20% stretching.
%The second variable should be
%written for instance in the form  4pt ; here the default is 5pt
%\newcommand{\EAN}{\end{array}\setlength{\arraycolsep}{5pt}}
\newcommand{\BSA}{\begin{subarray}}
\newcommand{\ESA}{\end{subarray}}	
%Note: These are used in subscripts as well as superscripts.
%They work essentially like 'array'.

\newcommand{\BAL}{\begin{aligned}}
	\newcommand{\EAL}{\end{aligned}}
\newcommand{\BALG}{\begin{alignat}}
	\newcommand{\EALG}{\end{alignat}}
%% the abbrev. does not work with latex2e
\newcommand{\BALGN}{\begin{alignat*}}
	\newcommand{\EALGN}{\end{alignat*}}
%% the abbrev. does not work with latex2e
%% The 'aligned' environment must be placed inside an 'equation' env.
%% in the same way as the array.
%% One could use also the 'align' env. or the 'alignat' env.
%% However in this case each line is numbered, unless '\notag' is used.
%% The 'alignat'
%% has a slightly different format (the number of columns must be %%specified in advance)
%% but it has the advantage that the distance between columns
%%is at our disposition.
%% (The default would be zero distance.) Using 'alignat*' we can have %%the advantages
%% of alignat plus the situation where separate lines are not numbered.
%% However in this case there is no numbering at all
%%(unless we provide a tag).
%%%%%%%%%%
%% PROOF, REMARK etc.
\newcommand{\note}[1]{\noindent\textit{#1.}\hspace{2mm}}
\newcommand{\Proof}{\note{Proof}}
%\newcommand{\qed}{\hspace{10mm}\hfill $\square$}
%\newcommand{\qed}{\\${}$ \hfill $\square$}
\newcommand{\Remark}{\note{Remark}}
%%%%%%%% Style command.
\newcommand{\modin}{$\,$\\[-4mm] \indent}
%% To be used after \section in order to start new line with \indent.
%%%%%%%%%%%%
%% MATHEMATICAL symbols
\newcommand{\forevery}{\quad \forall}
\newcommand{\set}[1]{\{#1\}}
\newcommand{\setdef}[2]{\{\,#1:\,#2\,\}}
\newcommand{\setm}[2]{\{\,#1\mid #2\,\}}
%% Arrows
\newcommand{\mt}{\mapsto}
\newcommand{\lra}{\longrightarrow}
\newcommand{\lla}{\longleftarrow}
\newcommand{\llra}{\longleftrightarrow}
\newcommand{\Lra}{\Longrightarrow}
\newcommand{\Lla}{\Longleftarrow}
\newcommand{\Llra}{\Longleftrightarrow}
\newcommand{\warrow}{\rightharpoonup}

%% Brackets, delimiters
\newcommand{\paran}[1]{\left (#1 \right )}
%% adjustable parantheses
\newcommand{\sqbr}[1]{\left [#1 \right ]}
%% adjustable square brackets
\newcommand{\curlybr}[1]{\left \{#1 \right \}}
%% adjustable curly brackets
\newcommand{\abs}[1]{\left |#1\right |}

%% adjustable vertical delimiters
\newcommand{\norm}[1]{\left \|#1\right \|}

%% adjustable norm
\newcommand{\paranb}[1]{\big (#1 \big )}

%% non-adjustable parantheses (big)
\newcommand{\lsqbrb}[1]{\big [#1 \big ]}

%% non-adjustable square brackets (big)
\newcommand{\lcurlybrb}[1]{\big \{#1 \big \}}

%% non-adjustable curly brackets(big)
\newcommand{\absb}[1]{\big |#1\big |}

%% non-adjustable vertical delimiters(big)
\newcommand{\normb}[1]{\big \|#1\big \|}

%% non-adjustable norm (big)
\newcommand{	\paranB}[1]{\Big (#1 \Big )}

%% non-adjustable parantheses (Big)
\newcommand{\absB}[1]{\Big |#1\Big |}

%% non-adjustable vertical delimiters(Big)
\newcommand{\normB}[1]{\Big \|#1\Big \|}%% non-adjustable norm (Big)
\newcommand{\produal}[1]{\langle #1 \rangle}%% the pairing of X' and X
%%%%%%%%%%%%%%%%%
%% Adjustable parantheses etc. in a different DEFINITION format.
%\def\adp(#1){\left (#1 \right )}%% adjustable parantheses
%\def\adsb(#1){\left [#1\right ]}%% adjustable square brackets
%\def\adcb(#1){\left \{#1\right \}}%% adjustable curly brackets
%\def\abs|#1|{\left |#1\right |}%% adjustable vertical delimiters
%%%%%%%%%%%%%%%%
%% More mathematical symbols
\newcommand{\thkl}{\rule[-.5mm]{.3mm}{3mm}}
\newcommand{\thknorm}[1]{\thkl #1 \thkl\,}
\newcommand{\trinorm}[1]{|\!|\!| #1 |\!|\!|\,}
\newcommand{\bang}[1]{\langle #1 \rangle}%% angle bracket
\def\angb<#1>{\langle #1 \rangle}%% angle bracket
%% The two last lines yield the same result.
%% The second is used as follows: \angb<a,b>
\newcommand{\vstrut}[1]{\rule{0mm}{#1}}
\newcommand{\rec}[1]{\frac{1}{#1}}
%% OPERATOR names.
%% OPERATOR names.
\newcommand{\opname}[1]{\mbox{\rm #1}\,}
\newcommand{\supp}{\opname{supp}}
\newcommand{\dist}{\opname{dist}}
\newcommand{\myfrac}[2]{{\displaystyle \frac{#1}{#2} }}
\newcommand{\myint}[2]{{\displaystyle \int_{#1}^{#2}}}
\newcommand{\mysum}[2]{{\displaystyle \sum_{#1}^{#2}}}
\newcommand {\dint}{{\displaystyle \myint\!\!\myint}}%%%%%%%%%%
%%%%%%% SPACE commands
\newcommand{\q}{\quad}
\newcommand{\qq}{\qquad}
\newcommand{\hsp}[1]{\hspace{#1mm}}
\newcommand{\vsp}[1]{\vspace{#1mm}}
%%%%%%%%%%%
%% ABREVIATIONS
\newcommand{\ity}{\infty}
\newcommand{\prt}{\partial}
\newcommand{\sms}{\setminus}
\newcommand{\ems}{\emptyset}
\newcommand{\ti}{\times}
\newcommand{\pr}{^\prime}
\newcommand{\ppr}{^{\prime\prime}}
\newcommand{\tl}{\tilde}
\newcommand{\sbs}{\subset}
\newcommand{\sbeq}{\subseteq}
\newcommand{\nind}{\noindent}
\newcommand{\ind}{\indent}
\newcommand{\ovl}{\overline}
\newcommand{\unl}{\underline}
\newcommand{\nin}{\not\in}
\newcommand{\pfrac}[2]{\genfrac{(}{)}{}{}{#1}{#2}}

%% frac with parantheses.
%%%%%%%%%%%
%%%%%%%%%%%%%

%%Macros for Greek letters.
\def\ga{\alpha}     \def\gb{\beta}       \def\gg{\gamma}
\def\gc{\chi}       \def\gd{\delta}      \def\gep{\epsilon}
\def\gth{\theta}                         \def\vge{\varepsilon}
\def\gf{\phi}       \def\vgf{\phi}    \def\gh{\eta}
\def\gi{\iota}      \def\gk{\kappa}      \def\gl{\lambda}
\def\gm{\mu}        \def\gn{\nu}         \def\gp{\pi}
\def\vgp{\varpi}    \def\gr{\gd}        \def\vgr{\varrho}
\def\gs{\sigma}     \def\vgs{\varsigma}  \def\gt{\tau}
\def\gu{\upsilon}   \def\gv{\vartheta}   \def\gw{\omega}
\def\gx{\xi}        \def\gy{\psi}        \def\gz{\zeta}
\def\Gg{\Gamma}     \def\Gd{\Delta}      \def\Gf{\Phi}
\def\Gth{\Theta}
\def\Gl{\Lambda}    \def\Gs{\Sigma}      \def\Gp{\Pi}
\def\Gw{\Omega}     \def\Gx{\Xi}         \def\Gy{\Psi}

%%Macros for calligraphic letters.
\def\CS{{\mathcal S}}   \def\CM{{\mathcal M}}   \def\CN{{\mathcal N}}
\def\CR{{\mathcal R}}   \def\CO{{\mathcal O}}   \def\CP{{\mathcal P}}
\def\CA{{\mathcal A}}   \def\CB{{\mathcal B}}   \def\CC{{\mathcal C}}
\def\CD{{\mathcal D}}   \def\CE{{\mathcal E}}   \def\CF{{\mathcal F}}
\def\CG{{\mathcal G}}   \def\CH{{\mathcal H}}   \def\CI{{\mathcal I}}
\def\CJ{{\mathcal J}}   \def\CK{{\mathcal K}}   \def\CL{{\mathcal L}}
\def\CT{{\mathcal T}}   \def\CU{{\mathcal U}}   \def\CV{{\mathcal V}}
\def\CZ{{\mathcal Z}}   \def\CX{{\mathcal X}}   \def\CY{{\mathcal Y}}
\def\CW{{\mathcal W}} \def\CQ{{\mathcal Q}}
%%%%%
%%Macros for 'blackboard' letters (See (27) for display.)
\def\BBA {\mathbb A}   \def\BBb {\mathbb B}    \def\BBC {\mathbb C}
\def\BBD {\mathbb D}   \def\BBE {\mathbb E}    \def\BBF {\mathbb F}
\def\BBG {\mathbb G}   \def\BBH {\mathbb H}    \def\BBI {\mathbb I}
\def\BBJ {\mathbb J}   \def\BBK {\mathbb K}    \def\BBL {\mathbb L}
\def\BBM {\mathbb M}   \def\BBN {\mathbb N}    \def\BBO {\mathbb O}
\def\BBP {\mathbb P}   \def\BBR {\mathbb R}    \def\BBS {\mathbb S}
\def\BBT {\mathbb T}   \def\BBU {\mathbb U}    \def\BBV {\mathbb V}
\def\BBW {\mathbb W}   \def\BBX {\mathbb X}    \def\BBY {\mathbb Y}
\def\BBZ {\mathbb Z}

%%Macros for Ghotic (Fraktur) letters.
\def\GTA {\mathfrak A}   \def\GTB {\mathfrak B}    \def\GTC {\mathfrak C}
\def\GTD {\mathfrak D}   \def\GTE {\mathfrak E}    \def\GTF {\mathfrak F}
\def\GTG {\mathfrak G}   \def\GTH {\mathfrak H}    \def\GTI {\mathfrak I}
\def\GTJ {\mathfrak J}   \def\GTK {\mathfrak K}    \def\GTL {\mathfrak L}
\def\GTM {\mathfrak M}   \def\GTN {\mathfrak N}    \def\GTO {\mathfrak O}
\def\GTP {\mathfrak P}   \def\GTR {\mathfrak R}    \def\GTS {\mathfrak S}
\def\GTT {\mathfrak T}   \def\GTU {\mathfrak U}    \def\GTV {\mathfrak V}
\def\GTW {\mathfrak W}   \def\GTX {\mathfrak X}    \def\GTY {\mathfrak Y}
\def\GTZ {\mathfrak Z}   \def\GTQ {\mathfrak Q}
\def\sign{\mathrm{sign\,}}
\def\bdw{\prt\Gw\xspace}
\def\nabu{|\nabla u|}
\def\tr{\mathrm{tr\,}}
\def\gap{{\ga_+}}
\def\gan{{\ga_-}}

\def\N{\mathbb{N}}
\def\Z{\mathbb{Z}}
\def\Q{\mathbb{Q}}
\def\R{\mathbb{R}}


\def\Proof.{{\rm{Proof. }}}
\def\End{\hspace{1cm} $\Box$\\}


\renewcommand{\baselinestretch}{1.1}

\let\e=\varepsilon
\let\vp=\phi
\let\t=\tilde
\let\ol=\overline
\let\ul=\underline
\let\.=\cdot
\let\0=\emptyset
\let\mc=\mathcal
\def\ex{\exists\;}
\def\fa{\forall\;}
\def\se{\ \Leftarrow\ }
\def\solose{\ \Rightarrow\ }
\def\sse{\ \Leftrightarrow\ }
\def\meno{\,\backslash\,}
\def\pp{,\dots,}
\def\D{\mc{D}}
\def\O{\Omega}


\def\loc{\text{\rm loc}}
\def\diam{\text{\rm diam}}
\def\dist{\text{\rm dist}}
\def\dv{\text{\rm div}}
\def\sign{\text{\rm sign}}
\def\supp{\text{\rm supp}}
\def\tr{\text{\rm Tr}}
\def\vec{\text{\rm vec}}
\def\inter{\text{\rm int\,}}
\def\norma#1{\|#1\|_\infty}

\newcommand{\esssup}{\mathop{\rm ess{\,}sup}}
\newcommand{\essinf}{\mathop{\rm ess{\,}inf}}
\newcommand{\su}[2]{\genfrac{}{}{0pt}{}{#1}{#2}}

\def\eq#1{{\rm(\ref{eq:#1})}}
\def\thm#1{Theorem \ref{thm:#1}}
\def\seq#1{(#1_n)_{n\in\N}}
\def\limn{\lim_{n\to\infty}}


\def\PP{\mc{P}}
\def\pe{principal eigenvalue}
\def\MP{maximum principle}
\def\SMP{strong maximum principle}
\def\l{\lambda_1}

\def\bq{\begin{equation}}
\def\eq{\end{equation}}

\def\l{\label}

\newenvironment{formula}[1]{\begin{equation}\label{eq:#1}}	{\end{equation}\noindent}

\title{Lecture $\star$ Nicolas Seguin, The Finite Element Method for Elliptic Partial Differential Equations}
\author{Collector: Nguyen Quan Ba Hong\footnote{Master 2 student at UFR math\'ematiques, Universit\'e de Rennes 1, Beaulieu - B\^atiment 22 et 23, 263 avenue du G\'en\'eral Leclerc, 35042 Rennes CEDEX, France.\newline
E-mail: \texttt{nguyenquanbahong@gmail.com} \newline
Blog: \texttt{\url{www.nguyenquanbahong.com}} \newline 
Copyright \copyright\ 2016-2018 by Nguyen Quan Ba Hong. This document may be copied freely for the purposes of education and non-commercial research. Visit my site to get more.}}
\begin{document}
\maketitle
\begin{abstract}
This context is the lecture given by Prof. Nicolas Seguin in the course \textit{Finite Element Method}, in the Master 2 Fundamental Mathematics program 2018-2019. 
\end{abstract}
\textbf{Brief introduction.} ``This lecture is a numerical counterpart to \textit{Sobolev spaces \& elliptic equations}. In the first part, after some reminders on linear elliptic partial differential equations, the approximation of the associated solutions by the finite element methods is investigated. Their construction and their analysis are described in one and two dimensions. The second part of the lectures consists in defining a generic strategy for the implementation of the method based on the variational formulation. A program is written in \textsc{Matlab} (implementable with \textsc{Matlab} or \textsc{Octave}).''
%\textbf{Mathematics Subject Classification (2010):} 
%
%\noindent
%\textbf{Keywords:} \emph{}
\maketitle
\newpage
\tableofcontents
\newpage
\section{Introduction}
\subsection{Finite Differences for the Poisson Equation}
We look at the following 1-D problem with Dirichlet boundary conditions
\begin{equation}
\label{1.1}
\left\{ \begin{split}
 - u'' &= f, \mbox{ on } \left( {0,1} \right),\\
u\left( 0 \right) &= {u_l},\\
u\left( 1 \right) &= {u_r},
\end{split} \right.
\end{equation}
where $f$ is the source term, $u_l, u_r \in \mathbb{R}$. 

If we define $F\left( x \right) := \int_0^x {f\left( s \right)ds} $, then the unique solution of \eqref{1.1} can be written as
\begin{align}
\label{1.2}
u\left( x \right) =  - \int_0^x {F\left( s \right)ds}  + {u_l} + x\left( {{u_r} - {u_l} + \int_0^1 {F\left( s \right)ds} } \right), \hspace{2mm} \forall x\in \left[0,1\right]. 
\end{align}
\begin{proof}[Proof of \eqref{1.2}]
Integrating both sides of \eqref{1.1} yields
\begin{align}
\label{1.3}
  u'\left( x \right) = -\int_0^x {f\left( s \right)ds}  + u'\left( 0 \right) = -F\left( x \right) + u'\left( 0 \right), \hspace{2mm} \forall  x\in \left[0,1\right].
\end{align}
Integrating both sides of \eqref{1.3} and then using the left Dirichlet boundary condition give us
\begin{align}
\label{1.4}
u\left( x \right) =  - \int_0^x {F\left( s \right)ds} +u_l + u'\left( 0 \right)x, \hspace{2mm} \forall  x \in \left[ {0,1} \right].
\end{align}
Plugging $x=1$ into \eqref{1.4} yields 
\begin{align}
\label{1.5}
{u_r} =  - \int_0^1 {F\left( s \right)ds} +{u_l} + u'\left( 0 \right).
\end{align}
A combination of \eqref{1.4} and \eqref{1.5} implies \eqref{1.2}. 
\end{proof}
The analysis of convergence of numerical methods depends on the smoothness of the solution $u$, which is obtained from the smoothness of the source term $f$. 
\subsubsection{Finite Difference Method}
\begin{definition}[Discretization]\label{def1.1}
Let $N\in \mathbb{N}^*$\footnote{$\mathbb{N}^*$ denotes the set of positive integers.}, we define the $N$-\emph{space step} as $h:=\frac{1}{N+1}$, and a \emph{uniform discretization} or a \emph{uniform mesh} of $\left[0,1\right]$ as ${\left( {{x_i}} \right)_{0 \le i \le N + 1}}$, where ${x_i}: = ih$ for $i = 0, \ldots ,N + 1$.
\end{definition}
We want to approximate $u\left(x_i\right)$ by some number $u_i$, knowing $u_0 = u\left(0\right) =u_l$, and $u_{N+1} = u\left(1\right) =u_r$. Unknown: ${u_h}: = {\left( {{u_i}} \right)_{1 \le i \le N}} \in {\mathbb{R}^N}$.

\noindent
\textbf{Taylor expansion.} We have
\begin{align}
u\left( {{x_{i + 1}}} \right) &= u\left( {{x_i}} \right) + hu'\left( {{x_i}} \right) + \frac{{{h^2}}}{2}u''\left( {{x_i}} \right) + O\left( {{h^3}} \right),\\
u\left( {{x_{i - 1}}} \right) &= u\left( {{x_i}} \right) - hu'\left( {{x_i}} \right) + \frac{{{h^2}}}{2}u''\left( {{x_i}} \right) + O\left( {{h^3}} \right),
\end{align}
and so
\begin{align}
u''\left( {{x_i}} \right) = \frac{{u\left( {{x_{i + 1}}} \right) - 2u\left( {{x_i}} \right) + u\left( {{x_{i - 1}}} \right)}}{{{h^2}}} + O\left( h \right).
\end{align}
We deduce the difference formula
\begin{equation*}
\left({\rm FD} \right)\hspace{2mm}\left\{ \begin{split}
 - \frac{{{u_{i + 1}} - 2{u_i} + {u_{i - 1}}}}{{{h^2}}} &= f\left( {{x_i}} \right),\hspace{2mm}1 \le i \le N,\\
{u_0} &= {u_l},\\
{u_{N + 1}} &= {u_r},
\end{split} \right.
\end{equation*}
Denote $A_h$ the following tridiagonal matrix
\begin{align}
{A_h} = \frac{1}{{{h^2}}}\left[ {\begin{array}{*{20}{c}}
2&{ - 1}& \cdots &0\\
{ - 1}&2& \cdots &0\\
 \vdots & \ddots & \ddots & \vdots \\
0&0& \cdots &2
\end{array}} \right],
\end{align}
for all $h>0$, then 
\begin{align}
\left( {\rm FD} \right) \Leftrightarrow {A_h}{u_h} = {b_h},
\end{align}
where the vector $b_h \in \mathbb{R}^N$ is defined by
\begin{align}
{b_h}: = \left[ {\begin{array}{*{20}{c}}
{f\left( {{x_1}} \right) + \frac{1}{{{h^2}}}{u_l}}\\
{f\left( {{x_2}} \right)}\\
 \vdots \\
{f\left( {{x_N}} \right) + \frac{1}{{{h^2}}}{u_r}}
\end{array}} \right].
\end{align}
\begin{prop}
$A_h$ is symmetric, positive definite, and so invertible. Then there exists one and only one solution $u_h$ of $\left( {\rm FD}\right)$. 
\end{prop}
\noindent
\textsc{Question.} $u_h \to u$ as $h\to 0$?
\subsubsection{Consistency}
\begin{definition}[Error of consistency, discretization operator]\label{Definition1.1}
The error of consistency is defined as
\begin{align}
{\varepsilon _h}\left( u \right): = {A_h}{\Pi _h}\left( u \right) - {b_h},
\end{align}
where 
\begin{align}
{\Pi _h}\left( u \right): = \left[ {\begin{array}{*{20}{c}}
{u\left( {{x_1}} \right)}\\
 \vdots \\
{u\left( {{x_N}} \right)}
\end{array}} \right],
\end{align}
which is called the discretization operator. 
\end{definition}
We then have directly from Definition \ref{Definition1.1}
\begin{align}
{A_h}{\Pi _h}\left( u \right) = {b_h} + {\varepsilon _h}\left( u \right).
\end{align}
\begin{prop}[Consistency]
The scheme $\left( {\rm FD}\right)$ is consistent, i.e.,
\begin{align}
{\left\| {{\varepsilon _h}\left( u \right)} \right\|_\infty } \to 0 \mbox{ as } h \to 0,
\end{align}
where $u$ is the solution of the Poisson problem \eqref{1.1}. Furthermore,
\begin{align}
{\left\| {{\varepsilon _h}\left( u \right)} \right\|_\infty } \le C{h^2}, \hspace{2mm} \forall  h>0.
\end{align}
\end{prop}
\begin{proof}
Make more rigorous the Taylor expansion to obtain
\begin{align}
\label{1.17}
{\left\| {{\varepsilon _h}\left( u \right)} \right\|_\infty } \le \frac{{{h^2}}}{{12}}\mathop {\sup }\limits_{x \in \left[ {0,1} \right]} \left| {{u^{\left( 4 \right)}}\left( x \right)} \right|.
\end{align}
This completes our proof.
\end{proof}
\begin{remark}
This requires (lots of) smoothness of the solution.
\end{remark}
\subsubsection{Stability}
Given an error $\delta b_h$ on $b_h$, one has
\begin{align}
{A_h}\left( {{u_h} + \delta {u_h}} \right) = {b_h} + \delta {b_h}.
\end{align}
Thus,
\begin{align}
{A_h}\delta {u_h} = \delta {b_h}.
\end{align}
\textsc{Stability.} Control of $\delta u_h$ by $\delta b_h$. Here, $\delta {u_h} = A_h^{ - 1}\delta {b_h}$.
\begin{prop}[Stability]\label{prop1.3}
\begin{align}
\label{1.20}
\left\| {A_h^{ - 1}} \right\| \le \frac{1}{8}.
\end{align}
\end{prop}
As a consequence, one has
\begin{align}
{\left\| {\delta {u_h}} \right\|_\infty } \le \frac{1}{8}{\left\| {\delta {b_h}} \right\|_\infty }.
\end{align}
\subsubsection{Convergence}
One has
\begin{align}
{A_h}\left( {{u_h} - {\Pi _h}\left( u \right)} \right) =  - {\varepsilon _h}\left( u \right).
\end{align}
Combining \eqref{1.17} and \eqref{1.20} yields
\begin{align}
{\left\| {{u_h} - {\Pi _h}\left( u \right)} \right\|_\infty } \le \frac{{{h^2}}}{{96}}\mathop {\sup }\limits_{x \in \left[ {0,1} \right]} \left| {{u^{\left( 4 \right)}}\left( x \right)} \right|.
\end{align}
\begin{theorem}[Convergence]
If $u \in C^4 \left( \left[0,1\right]\right)$ then the scheme $\left( {\rm FD}\right)$ converges to $u$ with quadrature rate.
\end{theorem}
\begin{remark}
\begin{itemize}
\item Same analysis in $\mathbb{R}^n$, $n\ge 1$, for the Dirichlet problem for the Poisson's equation
\begin{equation}
\left\{ \begin{split}
 - \Delta u &= f \mbox{ on } \Omega : = {\left( {0,1} \right)^n},\\
{\left. u \right|_{\partial \Omega }} &= g.
\end{split} \right.
\end{equation}
\item Problem with more complex domain due to the discretization of $\partial \Omega$, we would like to use more complex discretization.
\item The solution has to be smooth (e.g., $C^4$ above), which is in contradiction with the theory of Lax-Milgram, i.e., $f \in {L^2}\left( \Omega  \right) \Rightarrow u \in {H^2}\left( \Omega  \right)$ and $f \in {H^{ - 1}}\left( \Omega  \right) \Rightarrow u \in {H^1}\left( \Omega  \right)$.
\end{itemize}
\end{remark}
\subsection{Analysis of Elliptic PDEs} 
\subsubsection{Sobolev Spaces}
\begin{definition}[Distribution] Given $\Omega \subset \mathbb{R}^n$, $T$ is a distribution on $\Omega$ if $T$ is a linear form on $C_c^\infty \left(\Omega\right)$, such that $\forall K \subset \Omega$ compact, $\exists k \in \mathbb{N}$, ${C_K}>0$ such that
\begin{align}
\forall \phi  \in \mathcal{D}\left( \Omega  \right),\hspace{2mm}{\rm supp}\ \phi \subset K \hspace{2mm},\left| {\left\langle {T,\phi } \right\rangle } \right| \le {C_K}\mathop {\max }\limits_{\left| \alpha  \right| \le k} {\left\| {{\partial ^\alpha }\phi } \right\|_\infty } .
\end{align}
And $k$ is called the order of $T$.
\end{definition}
\begin{example}
For any $f \in L_{\rm loc}^1\left( \Omega  \right)$, $T:\phi  \mapsto \int_\Omega  {f\phi dx} $ is a distribution.
\end{example}
\begin{definition}[Convergence in the sense of distribution]
We say $u_n$ converges to $u$ in the sense of distribution, denoted as $u_n \rightharpoonup u$ as $n\to \infty$ in $\mathcal{D}'\left(\Omega\right)$, if 
\begin{align}
\left\langle {{u_n},\phi } \right\rangle  \to \left\langle {u,\phi } \right\rangle \mbox{ as } n \to \infty ,\hspace{2mm}\forall \phi  \in \mathcal{D}\left( \Omega  \right): = C^\infty \left( \Omega  \right).
\end{align}
\end{definition}

\begin{definition}[Distributional derivative]
Given $u\in \mathcal{D}'\left(\Omega\right)$, for $1\le i \le n$, we denote $\frac{{\partial u}}{{\partial {x_i}}}$ the distribution defined by
\begin{align}
\left\langle {\frac{{\partial u}}{{\partial {x_i}}},\phi } \right\rangle  =  - \left\langle {u,\frac{{\partial \phi }}{{\partial {x_i}}}} \right\rangle ,\hspace{2mm}\forall \phi  \in \mathcal{D}\left( \Omega  \right).
\end{align}
More generally, for all $\alpha \in \mathbb{N}^n$, we denote $\partial ^\alpha u$ the distribution defined by
\begin{align}
\left\langle {{\partial ^\alpha }u,\phi } \right\rangle  = {\left( { - 1} \right)^{\left| \alpha  \right|}}\left\langle {u,{\partial ^\alpha }\phi } \right\rangle ,\hspace{2mm}\forall \phi  \in \mathcal{D}\left( \Omega  \right).
\end{align}
\end{definition}

\begin{definition}[Sobolev spaces]
\begin{align}
{H^1}\left( \Omega  \right): &= \left\{ {u \in {L^2}\left( \Omega  \right);\frac{{\partial u}}{{\partial {x_i}}} \in {L^2}\left( \Omega  \right),\hspace{2mm}1 \le i \le n} \right\},\\
{H^m}\left( \Omega  \right): &= \left\{ {u \in {L^2}\left( \Omega  \right);{\partial ^\alpha }u \in {L^2}\left( \Omega  \right),\hspace{2mm}\forall \alpha :\left| \alpha  \right| \le m} \right\},\\
{W^{m,p}}\left( \Omega  \right):& = \left\{ {u \in {L^p}\left( \Omega  \right);{\partial ^\alpha }u \in {L^p}\left( \Omega  \right),\hspace{2mm}\forall \alpha :\left| \alpha  \right| \le m} \right\},
\end{align}
and
\begin{align}
{\left\| u \right\|_{{H^m}\left( \Omega  \right)}}: &= {\left( {\sum\limits_{\left| \alpha  \right| \le m} {\left\| {{\partial ^\alpha }u} \right\|_{{L^2}\left( \Omega  \right)}^2} } \right)^{\frac{1}{2}}},\\
H_0^1\left( \Omega  \right): &= {\left. {\overline {\mathcal{D}\left( \Omega  \right)} } \right|_{{H^1}\left( \Omega  \right)}}.
\end{align}
\end{definition}

\begin{theorem}[Poincar\'e inequality] 
If $\Omega$ is bounded, $\exists {C_\Omega } > 0$ such that 
\begin{align}
{\left\| u \right\|_{{L^2}\left( \Omega  \right)}} \le {C_\Omega }{\left\| {\nabla u} \right\|_{{L^2}\left( \Omega  \right)}},\hspace{2mm}\forall u \in H_0^1\left( \Omega  \right).
\end{align}
\end{theorem}

\begin{definition}[Dual space of Sobolev spaces]
\begin{align}
{H^{ - 1}}\left( \Omega  \right): = \left\{ {u \in \mathcal{D}'\left( \Omega  \right);u = {f_0} + \sum\limits_{i = 1}^n {\frac{{\partial {f_i}}}{{\partial {x_i}}}} ,\hspace{2mm}{f_i} \in {L^2}\left( \Omega  \right),\hspace{2mm}0 \le i \le n} \right\}.
\end{align}
\end{definition}
Given $\Omega \subset \mathbb{R}^n$, $\partial \Omega$ is bounded and of class $C^1$, The trace operator ${\gamma _0}:\mathcal{D}\left( {\overline \Omega  } \right) \to {C^0}\left( {\partial \Omega } \right)$ can be extended to a linear continuous map from $H^1\left(\Omega\right)$ to $L^2\left(\partial \Omega\right)$. 
\begin{theorem}[Green's formula]
For all $u,v\in H^1\left(\Omega\right)$,
\begin{align}
\int_\Omega  {\frac{{\partial u}}{{\partial {x_i}}}vdx}  =  - \int_\Omega  {u\frac{{\partial v}}{{\partial {x_i}}}dx}  + \int_{\partial \Omega } {{\gamma _0}u \cdot {\gamma _0}v{{\rm n}_i}d\sigma } ,
\end{align}
where $\overrightarrow  {\rm n} = {\left( {{n_i}} \right)_{1 \le i \le n}}$ is the unit normal vector to $\partial \Omega$ going outside $\Omega$.
\end{theorem}

\begin{prop}
Let $u\in H^1\left(\Omega\right)$, one has
\begin{align}
u \in H_0^1\left( \Omega  \right) \Leftrightarrow \widetilde u\left( x \right) \in H_0^1\left( {{\mathbb{R}^n}} \right) \Leftrightarrow {\gamma _0}u = 0 \mbox{ a.e. }\partial \Omega ,
\end{align}
where 
\begin{equation}
\widetilde u\left( x \right): = \left\{ \begin{split}
& u\left( x \right), & \mbox{ if } x &\in \Omega ,\\
& 0, & \mbox{ if } x &\in {\mathbb{R}^n}\backslash \Omega ,
\end{split} \right.
\end{equation}
\end{prop}

\begin{theorem}
Let $\Omega \subset \mathbb{R}^n$ be bounded and of class $C^1$. The canonical injection from $H^1\left(\Omega\right)$ to $L^2\left(\Omega\right)$ is compact.
\end{theorem}

\subsubsection{Variational Formulation}
Let $\Omega \subset \mathbb{R}^N$ whose $\partial \Omega$ is of class $C^1$ and bounded. We search for the solution of the following problem
\begin{equation}
\label{1.39}
\left( {\rm P}\right) \hspace{2mm}\left\{ \begin{split}
 - \Delta u &= f, & \mbox{ in }\Omega ,\\
u &= 0, & \mbox{ on }\partial \Omega ,
\end{split} \right.
\end{equation}
where the source term $f$ is given.

The \textit{variational formulation} associated with $\left( {\rm P}\right)$ is defined as
%\begin{equation}
%\left( {\rm VF} \right) \hspace{2mm}\left\{ \begin{split}
%& \mbox{Find } u \in H_0^1\left( \Omega  \right) \mbox{ such that}\\
%& \int_\Omega  {\nabla u \cdot \nabla vdx}  = \int_\Omega  {fvdx} , \hspace{2mm}\forall v \in H_0^1\left( \Omega  \right),
%\end{split} \right.
%\end{equation}
\begin{align}
\left( {\rm VF} \right) \hspace{2mm} \mbox{Find }u \in H_0^1\left( \Omega  \right) \mbox{ such that }\int_\Omega  {\nabla u \cdot \nabla vdx}  = \int_\Omega  {fvdx} ,\hspace{2mm}\forall v \in H_0^1\left( \Omega  \right),
\end{align}
This gives a ``definition'' of solutions of $\left( {\rm P}\right)$. Moreover, under some smoothness assumptions, $\left( {\rm P} \right) \Leftrightarrow \left( {\rm VF} \right)$ by the Green's formula.
\begin{definition}[Variational formulation]
A \emph{variational formulation} is composed by
\begin{itemize}
\item $V$: a Hilbert space;
\item $a$: a bilinear continuous form on $V\times V$;
\item $l$: a linear continuous form on $V$;
\end{itemize}
and corresponds to a problem
\begin{align}
\left( {\rm VF} \right) \hspace{2mm}\mbox{Find } u\in V \mbox{ such that } a\left(u,v\right)=l\left(v\right), \hspace{2mm} \forall v\in V,
\end{align}
\end{definition}

\begin{theorem}[Lax-Milgram]\label{laxmilgram}
Given a variational formulation and $a$ is coercive, i.e., 
\begin{align}
\exists \alpha  > 0, \hspace{2mm}\forall u \in V, \hspace{2mm}a\left( {u,u} \right) \ge \alpha \left\| u \right\|_V^2,
\end{align}
Then, $\left( {\rm VF} \right)$ admits one and only one solution $u\in V$.
\end{theorem}

\begin{remark}
\begin{enumerate}
\item This applies to the problem $\left( {\rm P}\right)$, assuming $f\in H^{-1}\left(\Omega\right)$.
\item The coercivity of $a$ is obtained by the Poincar\'e's inequality.
\item The notion of variational formulation and Lax-Milgram theorem \ref{laxmilgram} can be generalized to more general elliptic PDEs, e.g.,
\begin{equation}
\left\{ \begin{split}
 - {\rm div}\left( {A\left( x \right)\nabla u} \right) + b\left( x \right) \cdot \nabla u + c\left( x \right)u &= f, & \mbox{ in } \Omega ,\\
u &= g, & \mbox{ on }\partial \Omega ,
\end{split} \right.
\end{equation}
with nonhomogeneous Dirichlet boundary condition, where $g \in {H^{\frac{1}{2}}}\left( {\partial \Omega } \right): = {\gamma _0}\left( {{H^1}\left( \Omega  \right)} \right)$. And this Dirichlet boundary conditions can be replace by the following Neumann boundary condition
\begin{align}
\frac{{\partial u}}{{\partial {\rm n}}} = 0,\mbox{ on }\partial \Omega ,
\end{align}
where $\bf n$ is the unit normal vector to $\partial \Omega$.
\item The Poisson's equation \eqref{1.39} can be interpreted as ``$u = {\left( { - \Delta } \right)^{ - 1}}f$''.
\item If $\Omega$ is bounded and of class $C^2$, $u$ is the solution of $\left( {\rm P}\right)$ and $f\in L^2\left(\Omega\right)$ then $u\in H^2\left(\Omega\right)$ and there exists a positive constant $C$ such that the following inequality holds
\begin{align}
{\left\| u \right\|_{{H^2}}} \le C{\left\| f \right\|_{{L^2}}},
\end{align}
whereas there is a priori non-control of the derivatives $\frac{{{\partial ^2}u}}{{\partial {x_i}\partial {x_j}}}$ for $i\ne j$.
\end{enumerate}
\end{remark}

\subsubsection{The FEM for the 1-D Dirichlet Problem}
We want to approximate the solution of $\left( {\rm P}\right)$ defined by \eqref{1.39}. To do this, we set $V = H_0^1\left( \Omega  \right)$ and
\begin{align}
a\left( {u,v} \right):&= \int_\Omega  {\nabla u \cdot \nabla vdx} ,\hspace{2mm}\forall u,v \in V,\\
l\left( v \right):&= \int_\Omega  {fvdx} ,\hspace{2mm}\forall v \in V,
\end{align}
Assume that we have a finite dimensional subspace $V_h \subset V$. The \textit{internal} (or \textit{conformal}) \textit{approximation} of $\left( {\rm VF} \right)$ is defined as 
\begin{align}
\left( {\rm VF}_h \right) \hspace{2mm} \mbox{ Find }{u_h} \in {V_h} \mbox{ such that } a\left( {{u_h},{v_h}} \right) = l\left( {{v_h}} \right),\hspace{2mm}\forall {v_h} \in {V_h},
\end{align}

\begin{lemma}
If $a$ is coercive on $V$, then $\left( {\rm VF}_h \right)$ admits one and only one solution. Moreover, $\left( {\rm VF}_h \right)$ is equivalent to a linear system to solve.
\end{lemma}

\begin{proof}
The existence and uniqueness of the solution, say $u_h\in V_h$, of $\left( {\rm VF}_h \right)$ is deduced by the Lax-Milgram theorem \ref{laxmilgram}. Since $V_h$ is finite dimensional vector space, we can introduce a basis of $V_h$ as ${\left( {{\phi _i}} \right)_{1 \le i \le N}}$ with $N= \dim V_h$. Write ${u_h}\left( x \right) = \sum\nolimits_{i = 1}^N {{u_i}{\phi _i}\left( x \right)} $, we arrive at a linear system $A_h U_h =F_h$ where ${A_h} := {\left( {a\left( {{\phi _i},{\phi _j}} \right)} \right)_{1 \le i,j \le N}}$, ${U_h}: = {\left( {{u_i}} \right)_{1 \le i \le N}}$, and ${F_h} := {\left( {l\left( {{\phi _i}} \right)} \right)_{1 \le i \le N}}$. Lastly, the coercivity of $a$ implies the positive definiteness of $A_h$.
\end{proof}

\begin{remark}
For finite differences, the invertibility of the matrix has to be proved for all problems and for all discretizations used. However, for Finite Elements (FEs, for short), it is direct.
\end{remark}

\begin{lemma}[C\'ea's lemma]
Under the same assumptions, the following inequality holds
\begin{align}
{\left\| {u - {u_h}} \right\|_V} \le \frac{M}{\alpha }\mathop {\inf }\limits_{{v_h} \in {V_h}} {\left\| {u - {v_h}} \right\|_V},
\end{align}
where $M$, $\alpha$ are the continuity and coercivity constants of $a$, respectively.
\end{lemma}

\begin{proof}
Let $w_h\in V_h$ be a test function. $\left( {\rm VF}_h \right)$ implies $a\left( {{u_h},{w_h}} \right) = l\left( {{w_h}} \right)$ while $\left({\rm VF}\right)$ gives us $a\left( {u,{w_h}} \right) = l\left( {{w_h}} \right)$ since $V_h\subset V$. Subtracting these yields $a\left( {u - {u_h},{w_h}} \right) = 0$. Put $v_h:=u_h-w_h\in V_h$, one has
\begin{align}
\alpha \left\| {u - {u_h}} \right\|_V^2 \le a\left( {u - {u_h},u - {u_h}} \right) = a\left( {u - {u_h},u - {v_h}} \right) \le M{\left\| {u - {u_h}} \right\|_V}{\left\| {u - {v_h}} \right\|_V},
\end{align}
which implies the desired inequality.
\end{proof}
In particular, we want that
\begin{itemize}
\item $d\left(u,V_h\right) \to 0$ as $h\to 0^+$ ``fast'', and
\item the linear system $A_h U_h=F_h$ is ``easy'' to solve.
\end{itemize}
Reconsider \eqref{1.1} with $u_l=u_r=0$ and its discretization given by \ref{def1.1} , and $f\in H^{-1}\left(0,1\right)$, we define $T_j^ - : = \left[ {{x_{j - 1}},{x_j}} \right]$, $T_j^ + : = \left[ {{x_j},{x_{j + 1}}} \right]$. Consider the following space
\begin{align}
{W_h}: = \left\{ {v \in C\left( {\left[ {0,1} \right]} \right);{{\left. v \right|}_{T_j^ + }} \in {\mathbb{P}_1},\hspace{2mm} j = 0, \ldots ,N} \right\},
\end{align}
Notice that ${W_h} \not\subset V: = H_0^1\left( {0,1} \right)$ but $W_h \subset H^1\left(0,1\right)$, and $\dim W_h = N+2$. We take
\begin{align}
\label{1.52}
{V_h}: = {W_h} \cap V = \left\{ {v \in {W_h};v\left( 0 \right) = v\left( 1 \right) = 0} \right\}, 
\end{align}
then $\dim V_h=N$. A basis\footnote{The elements of this basis are well-known as \textit{hat functions}.} of $V_h$, say  ${\left( {{\phi _j}} \right)_{1 \le j \le N}}$, is given by $\phi _i \left(x_j\right) =\delta _{ij}$ for all $i=1,\ldots ,N$, $j=0,\ldots,N+1$. A simple calculation give yields the following explicit formula of $\phi_j$'s as follows,
\begin{equation}
{\phi _j}\left( x \right) = \left\{ \begin{split}
& \frac{{x - {x_j}}}{h}, &\mbox{ in } T_j^ - ,\\
& \frac{{{x_{j + 1}} - x}}{h}, &\mbox{ in } T_j^ + ,\\
& 0, &\mbox{ elsewhere},
\end{split} \right.
\end{equation}
Consider the matrix $A_h$ whose the $ij$th element is defined by $a\left( {{\phi _i},{\phi _j}} \right) = \int_0^1 {{\phi _i}'{\phi _j}'dx}$, if $\left| {i - j} \right| > 1$ then $A_{ij} =0$ since  ${\rm supp}\ \phi _i \cap {\rm supp}\ \phi _j =\emptyset$. If $i=j$, $A_{ij}=\frac{2}{h}$. And if $\left| {i - j} \right| = 1$, $A_{ij}=-\frac{1}{h}$. Hence, 
\begin{align}
{A_h} = \frac{1}{h}\left[ {\begin{array}{*{20}{c}}
2&{ - 1}& \cdots &0\\
{ - 1}&2& \cdots &0\\
 \vdots & \ddots & \ddots & \vdots \\
0&0& \cdots &2
\end{array}} \right],
\end{align}
This matrix is sparse since it contains many zero terms.

\noindent
$\star$ \textbf{Convergence and Error Estimates.} 
\begin{definition}
Let $v\in H^1\left(0,1\right)$, we define 
\begin{align}
{\Pi _h}v\left( x \right): = \sum\limits_{j = 1}^{N + 1} {v\left( {{x_j}} \right){\phi _j}\left( x \right)}, \hspace{2mm} \forall x\in \left[0,1\right].
\end{align}
\end{definition}

\begin{lemma}
There exists a positive constant being independent of $h$ such that 
\begin{align}
{\left\| {{\Pi _h} v } \right\|_{{H^1}}} \le C{\left\| v \right\|_{{H^1}}},\hspace{2mm} \forall v \in {H^1}\left( {0,1} \right).
\end{align}
\end{lemma}

\begin{proof}
For $x\in T_j^+$, one has $\left( {{\Pi _h}v} \right)'\left( x \right) = {{\left( {v\left( {{x_{j + 1}}} \right) - v\left( {{x_j}} \right)} \right)} \mathord{\left/
 {\vphantom {{\left( {v\left( {{x_{j + 1}}} \right) - v\left( {{x_j}} \right)} \right)} h}} \right.
 \kern-\nulldelimiterspace} h}$ and thus
\begin{align}
\int_{T_j^ + } {{{\left( {\left( {{\Pi _h}v} \right)'\left( x \right)} \right)}^2}dx}  = \frac{{{{\left( {v\left( {{x_{j + 1}}} \right) - v\left( {{x_j}} \right)} \right)}^2}}}{h} = \frac{1}{h}{\left( {\int_{T_j^ + } {v'dx} } \right)^2} \le \int_{T_j^ + } {{{\left| {v'} \right|}^2}dx} .
\end{align}
Summing this inequality over the index $j$ yields ${\left\| {\left( {{\Pi _h}v} \right)'} \right\|_{{L^2}}} \le {\left\| {v'} \right\|_{{L^2}}}$.

Moreover, 
\begin{align}
{\left\| {{\Pi _h}v} \right\|_{{L^2}}} \le \mathop {\max }\limits_{\left[ {0,1} \right]} \left| {{\Pi _h}v} \right| \le \mathop {\max }\limits_{\left[ {0,1} \right]} \left| v \right| \le C{\left\| v \right\|_{{H^1}}}.
\end{align}
Combing these inequalities yields the desired result.
\end{proof}

\begin{lemma}
\begin{itemize}
\item[a)] There exists a positive constant $C$ which is independent of $h$ such that 
\begin{align}
{\left\| {v - {\Pi _h}v} \right\|_{{L^2}}} \le Ch{\left\| {v'} \right\|_{{L^2}}},\hspace{2mm} \forall v \in {H^1}\left( {0,1} \right).
\end{align}
\item[b)] There exists a positive constant $C$ which is independent of $h$ such that
\begin{align}
{\left\| {v - {\Pi _h}v} \right\|_{{L^2}}} \le C{h^2}{\left\| {v''} \right\|_{{L^2}}},\mbox{ and }\left\| {v' - \left( {{\Pi _h}v} \right)'} \right\|_{{L^2}} \le Ch{\left\| {v''} \right\|_{{L^2}}},\hspace{2mm}\forall v \in {H^2}\left( {0,1} \right).
\end{align}
\item[c)] One has 
\begin{align}
\mathop {\lim }\limits_{h \to 0} {\left\| {v' - \left( {{\Pi _h}v} \right)'} \right\|_{{L^2}}} = 0,\hspace{2mm} \forall v \in {H^1}\left( {0,1} \right).
\end{align}
\end{itemize}
\end{lemma}

\begin{theorem}\label{theorem1.6}
\begin{itemize}
\item[a)] Let $f\in H^{-1}\left(0,1\right)$, $u\in H_0^1\left(0,1\right)$ is the solution of $\left( {\rm VF}\right)$, $u_h\in V_h$ is the solution of $\left( {\rm VF}_h \right)$. Then
\begin{align}
\mathop {\lim }\limits_{h \to 0} {\left\| {u - {u_h}} \right\|_{{H^1}}} = 0.
\end{align}
\item[b)] Let $f\in L^2\left(0,1\right)$, then $u\in H^2\left(0,1\right)$ and
\begin{align}
{\left\| {u - {u_h}} \right\|_{{H^1}}} &\le Ch{\left\| f \right\|_{{L^2}}}, \\
{\left\| {u - {u_h}} \right\|_{{L^2}}} &\le C{h^2}{\left\| f \right\|_{{L^2}}},
\end{align}
\end{itemize}
\end{theorem}

\begin{proof}
The first inequality is deduced from 
\begin{align}
{\left\| {u - {u_h}} \right\|_{{H^1}}} \le {\left\| {u - {\Pi _h}u} \right\|_{{H^1}}} \to 0 \mbox{ as } h \to 0.
\end{align}
For the second inequality, let $f\in L^2\left(0,1\right)$, a well-known regularity result gives us $u\in H^2\left(0,1\right)$. And then
\begin{align}
{\left\| {u - {u_h}} \right\|_{{H^1}}} \le {\left\| {u - {\Pi _h}u} \right\|_{{H^1}}} \le Ch{\left\| {u''} \right\|_{{L^2}}} = Ch{\left\| f \right\|_{{L^2}}}.
\end{align}
For the last one, one has
\begin{align}
{\left\| {u - {u_h}} \right\|_{{L^2}}} = \mathop {\sup }\limits_{g \in {L^2}\left( {0,1} \right),g \ne 0} \frac{{\int_0^1 {\left( {u - {u_h}} \right)gdx} }}{{{{\left\| g \right\|}_{{L^2}}}}}.
\end{align}
For any $g\in L^2\left(0,1\right)$, we consider the following variational formulation associated with $g$,
\begin{align}
\left( {{\rm VF}_g} \right) \mbox{ Find } {\phi _g} \in H_0^1\left( {0,1} \right) \mbox{ such that } a\left( {{\phi _g},v} \right) = \left\langle {g,v} \right\rangle ,\hspace{2mm}\forall v \in H_0^1\left( {0,1} \right).
\end{align}
Then there exists a positive constant $C$ such that ${\left\| {{\phi _g}} \right\|_{{H^1}}} \le C{\left\| g \right\|_{{L^2}}}$. 

Take $v:=u-u_h$, one has $a\left( {u - {u_h},{\phi _g}} \right) = \int_0^1 {\left( {u - {u_h}} \right)gdx} $. Since ${\Pi _h}{\phi _g} \in {V_h}$, one also has $a\left( {u - {u_h},{\Pi _h}{\phi _g}} \right) = 0$. Hence,
\begin{align*}
\int_0^1 {\left( {u - {u_h}} \right)gdx}  &= a\left( {u - {u_h},{\phi _g} - {\Pi _h}{\phi _g}} \right) \le {\left\| {\left( {u - {u_h}} \right)'} \right\|_{{L^2}}}{\left\| {\left( {{\phi _g} - {\Pi _h}{\phi _g}} \right)'} \right\|_{{L^2}}}\\
& \le {\left\| {\left( {u - {u_h}} \right)'} \right\|_{{H^1}}}{\left\| {\left( {{\phi _g} - {\Pi _h}{\phi _g}} \right)'} \right\|_{{H^1}}} \le Ch{\left\| f \right\|_{{L^2}}} \cdot Ch{\left\| {{\phi _g}''} \right\|_{{L^2}}} \le C{h^2}{\left\| f \right\|_{{L^2}}}{\left\| g \right\|_{{L^2}}},
\end{align*}
which implies that ${\left\| {u - {u_h}} \right\|_{{L^2}}} \le C{h^2}{\left\| f \right\|_{{L^2}}}$. 
\end{proof}

\begin{remark}
When $f=0$, Theorem \ref{theorem1.6} implies that $u=u_h$. Moreover, this is analogous to Proposition \ref{prop1.3} in FDM.
\end{remark}

\subsubsection{Construction of $A_h$ and $b_h$}
Denote ${\mathcal{T}_h}: = \left\{ {\left[ {{x_i},{x_{i + 1}}} \right];i = 0, \ldots ,N} \right\}$, then the elements of the matrix $A_h$ can be written as
\begin{align}
{\left( {{A_h}} \right)_{ij}} = \sum\limits_{T \in {\mathcal{T}_h},T \subset {\rm supp}\ {\phi _i} \ \cap \ {\rm supp}\ {\phi _j}} {\int_T {{\phi _i}'{\phi _j}'dx} } .
\end{align}
So ${\left( {{A_h}} \right)_{ij}} \ne 0 \Leftrightarrow i = j - 1,j,j + 1$. 

Let $X_\alpha$, $\alpha=1,2$ be the vertices of $T$. Then $T$ is only defined by $\left(X_\alpha\right)_\alpha$ local or $\left(x_i\right)_i$ global. 
\begin{definition}[Elementary matrix]
The \emph{elementary matrix} related to $T$ is defined as 
\begin{align}
{m_T}: = {\left( {\int_T {{{\widetilde \phi }_\alpha }'{{\widetilde \phi }_\beta }'dx} } \right)_{\alpha ,\beta  = 1,2}} ,
\end{align}
where ${\widetilde \phi _\alpha },{\widetilde \phi _\beta }$ are the local elements of the basis of $V_h$, associated with $X_\alpha,X_\beta$. 

In the same way, 
\begin{align}
{b_T}: = {\left( {\int_T {f{{\widetilde \phi }_\beta }dx} } \right)_{\beta  = 1,2}} .
\end{align}
\end{definition}
\newpage
\begin{algorithm}[H]
\caption{Elementary matrices and RHS.}
\label{algorithm1}
\begin{algorithmic}[1]
\For {$T\in \mathcal{T}_h$}
    \For {$\beta=1,2$}
        \State Compute $\left(b_T\right)_\beta$;
        \For {$\alpha=1,2$}
            \State Compute $\left(m_T\right)_{\alpha,\beta}$;
        \EndFor
    \EndFor
\EndFor
\State $b_h=0$; $A_h=0$;
\For {$T\in \mathcal{T}_h$}
    \For {$\beta=1,2$}
        \State $i:=$ global index of $X_\beta$;
        \State ${\left( {{b_h}} \right)_i}: = {\left( {{b_h}} \right)_i} + {\left( {{b_T}} \right)_\beta }$;
        \For {$\alpha=1,2$}
            \State $j:=$ global index of $X_\alpha$;
            \State ${\left( {{A_h}} \right)_{ij}}: = {\left( {{A_h}} \right)_{ij}} + {\left( {{m_T}} \right)_{\alpha \beta }}$;
        \EndFor
    \EndFor
\EndFor
\end{algorithmic}
\end{algorithm}

\begin{remark}[Quick notes]
\begin{itemize}
\item The algorithm is implemented for loops on the elements $T\in \mathcal{T}_h$, not on the two indices $i, j$, which will cause double loops.
\item To compute $m_T$ and $b_T$, we
\begin{itemize}
\item[$\circ$] define a reference interval: $\widehat{T} =\left[0,1\right]$;
\item[$\circ$] define an affine function $F_T$ such that $F_T\left(\widehat{T}\right)=T$.
\item[$\circ$] Carry out all the computations over the reference interval $\widehat{T}$ and apply $F_T$ (use numerical integrations).
\end{itemize}
\end{itemize}
\end{remark}

\section{General Construction of the FEM}
In this section, we will approximate the solutions of some second-order elliptic PDEs for which their corresponding variational formulations are well-posed.
\begin{definition}[Triangulation]
Let $\Omega \in \mathbb{R}^n$ be a domain, a \emph{triangulation} (or \emph{mesh}) $\mathcal{T}_h$ of $\Omega$ is a partition of $\Omega$ into (closed) elements $K$'s such that 
\begin{align}
\overline \Omega   = \bigcup\limits_{K \in {\mathcal{T}_h}} K .
\end{align}
A \emph{space of polynomial approximation} can be defined by
\begin{align}
\label{1.73}
{P_K}: = \left\{ {v:K \to \mathbb{R};v \in {\mathbb{P}_m}\left( K \right)} \right\}, \hspace{2mm} \forall K \in {\mathcal{T}_h} ,
\end{align}
for some positive integer $m$.

A \emph{global space} $X_h$ is defined by
\begin{align}
{X_h}: = \left\{ {v:\Omega  \to \mathbb{R};{{\left. v \right|}_K} \in {P_K},\hspace{2mm} \forall K \in {\mathcal{T}_h}} \right\} .
\end{align}
\end{definition}

\begin{remark}
Define $V_h$ as
\begin{align}
{V_h}: = \left\{ {v:\Omega  \to \mathbb{R};{{\left. v \right|}_K} \in {P_K},\hspace{2mm} \forall K \in {\mathcal{T}_h},\hspace{2mm} {{\left. v \right|}_{\partial \Omega }} = 0} \right\},
\end{align}
which is analogous to its 1-D version \eqref{1.52}, one has $V_h \subset X_h$. Notice the difference between these spaces, $V_h$ is used to approximate functions in $H_0^1 \left(0,1\right)$ while $X_h$ is used to do that in $H^1\left(0,1\right)$. In addition, the elements of a basis of $X_h$ for small $h>0$ will have ``small'' supports.
\end{remark}

\subsection{Classical Families of FEM}
Here are two classical FEs:
\begin{itemize}
\item $n$-simplices Finite Element;
\item $n$-rectangular Finite Element;
\end{itemize}
\textit{Quick note.} Points of approximation $\equiv$ unknowns $\equiv$ degrees of freedom (\textit{dof}, for short).

If we use
\begin{itemize}
\item points of values, this will lead us to Lagrange Finite Element;
\item points values and directional derivatives, this will gives us Hermite Finite Element.
\end{itemize}
We assume that there exists an affine mapping $F_T$ such that $F_T\left(\widehat{K}\right) =K$, where $\widehat{K}$ is a reference element. This also suggest that $\Omega$ is a polyhedron.

\subsubsection{Some Basis}
We consider the following variational formulation 
\begin{align}
\left( {\rm VF}\right) \hspace{2mm} \mbox{Find }u \in V \mbox{ such that } a\left( {u,v} \right) = f\left( v \right), \hspace{2mm} \forall v \in V,
\end{align}
where $a$ is the bilinear form associated with some second-order elliptic PDEs for which Lax-Milgram theorem \ref{laxmilgram} can apply.

\noindent
\textit{Galerkin method.} For a given finite dimensional subspace $V_h \subset V$, we associate $\left( {\rm VF}\right)$ with the following variational formulation
\begin{align}
\left( {{\rm VF}_h} \right) \hspace{2mm} \mbox{Find } {u_h} \in {V_h} \mbox{ such that } a\left( {{u_h},{v_h}} \right) = f\left( {{v_h}} \right), \hspace{2mm} \forall {v_h} \in {V_h}.
\end{align}

\begin{remark}
If $a$ is symmetric, $\left( {\rm VF}\right)$ and $\left( {\rm VF}_h \right)$ are equivalent to the problem of minimizing the energy
\begin{align}
J\left( u \right): = \frac{1}{2}a\left( {u,u} \right) - f\left( u \right) .
\end{align}
In this case, Galerkin method is also called Ritz method.
\end{remark}
\noindent
$\star$ \textbf{Construction of $\mathcal{T}_h$.} 

\noindent
$\left(\rm FEM_1\right)$:
\begin{itemize}
\item[$\left(\mathcal{T}_1\right)$] $\overline \Omega   = \bigcup\nolimits_{K \in {\mathcal{T}_h}} K $;
\item[$\left(\mathcal{T}_2\right)$] $\forall K\in \mathcal{T}_h$, $K$ is closed, $\mathop K\limits^ o   \ne \emptyset $;
\item[$\left(\mathcal{T}_3\right)$] $\forall K_1,K_2 \in \mathcal{T}_h$, $K_1\ne K_2$, $\mathop {{K_1}}\limits^o  \cap \mathop {{K_2}}\limits^o  = \emptyset $;
\item[$\left(\mathcal{T}_4\right)$] $\forall K\in \mathcal{T}_h$, $\partial K$ is Lipschitz continuous.
\end{itemize}
$\star$ \textbf{Construction of $X_h$.} To construct $X_h$, we need the following lemma.
\begin{lemma}[J\^en\^ome Dronion]\label{lemma1.5}
If $\partial \Omega$ is bounded and Lipschitz continuous, then 
\begin{align}
H_0^1\left( \Omega  \right) = \left\{ {u \in {H^1}\left( \Omega  \right);{\gamma _0}u = 0} \right\} .
\end{align}
\end{lemma}

\begin{theorem}
Assume ${P_K} \subset {H^1}\left( {\mathop K\limits^o } \right)$\footnote{The choice \eqref{1.73} is not used here, $P_K$ is assumed to be more general.}, $\forall K\in \mathcal{T}_h$, $X_h \subset C^0 \left(\overline{\Omega}\right)$. Then $X_h \subset H^1\left(\Omega\right)$ and 
\begin{align}
{X_{0h}}: = \left\{ {{v_h} \in {X_h};{\gamma _0}{v_h} = 0} \right\} \subset H_0^1\left( \Omega  \right) .
\end{align}
\end{theorem}
\begin{proof}
Let $v\in X_h$. Clearly, $v\in L^2\left(\Omega\right)$. We want to find functions $w_i \in L^2\left(\Omega\right)$, $i=1,\ldots,n$ such that
\begin{align}
\int_\Omega  {{w_i}\phi dx}  =  - \int_\Omega  {v{\partial _i}\phi dx} ,\hspace{2mm} \forall \phi  \in \mathcal{D}\left( \Omega  \right).
\end{align}
Let's take ${\left. {{w_i}} \right|_K} = {\left. {{\partial _i}v} \right|_K} \in {L^2}\left( K \right)$. Green's formula helps us arrive at
\begin{align}
\int_K {{{\left. {{\partial _i}v} \right|}_K}\phi dx}  =  - \int_K {{{\left. v \right|}_K}{\partial _i}\phi dx}  + \int_{\partial K} {{\gamma _0}{{\left. v \right|}_K}{\gamma _0}\phi {{\left( {{\nu _K}} \right)}_i}d\sigma } ,\hspace{2mm}\forall K \in {\mathcal{T}_h}, \hspace{1mm} \forall \phi  \in \mathcal{D}\left( \Omega  \right).
\end{align}
Summing this equality over $K\in \mathcal{T}_h$ yields
\begin{align}
\label{1.82}
\int_\Omega  {{w_i}\phi dx}  =  - \int_\Omega  {v{\partial _i}\phi dx}  + \sum\limits_{K \in {\mathcal{T}_h}} {\int_{\partial K} {{\gamma _0}{{\left. v \right|}_K}{\gamma _0}\phi {{\left( {{\nu _K}} \right)}_i}d\sigma } } ,\hspace{2mm} \forall \phi  \in \mathcal{D}\left( \Omega  \right).
\end{align}
Either $\partial K \subset \partial \Omega$ we have $\gamma _0 \phi =0$, or we have two contributions of opposite signs. Thus, \eqref{1.82} is equivalent to
\begin{align}
\left\langle {{w_i},\phi } \right\rangle  =  - \left\langle {v,{\partial _i}\phi } \right\rangle ,\hspace{2mm}\forall \phi  \in \mathcal{D}\left( \Omega  \right).
\end{align}
Therefore, $X_h\subset H^1\left(\Omega\right)$. The second inclusion is a corollary of Lemma \ref{lemma1.5}.
\end{proof}

\begin{theorem}
Assume ${P_K} \subset {H^2}\left( {\mathop K\limits^o } \right)$\footnote{The choice \eqref{1.73} is also not used here, $P_K$ is assumed to be more general.}, $\forall K\in \mathcal{T}_h$, $X_h \subset C^1 \left(\overline{\Omega}\right)$. Then $X_h \subset H^2\left(\Omega\right)$, and
\begin{align}
{X_{0h}}: &= \left\{ {{v_h} \in {X_h};{\gamma _0}{v_h} = 0} \right\} \subset {H^2}\left( \Omega  \right) \cap H_0^1\left( \Omega  \right),\\
{X_{00h}}: &= \left\{ {{v_h} \in {X_{0h}};{\gamma _0}\left( {{\partial _\nu }v} \right) = 0} \right\} \subset H_0^2\left( \Omega  \right).
\end{align}
\end{theorem}

\subsection{Examples of Finite Elements and Finite Element Spaces}
For a moment, we do not consider any boundary conditions. Assume that $\Omega \subset \mathbb{R}^n$ is a polyhedron, we reuse the notions of triangulation $\mathcal{T}_h$, reference element $\widehat{K}$, affine mapping $F_K$ as above.
\subsubsection{$n$-Simplices of Type $\left(k\right)$}
The $n$-simplices of type $\left(k\right)$ are the most classical ones: ``very'' general, ``easy'' to implement, and apply to ``many'' cases.

Denote $\mathbb{P}_k$ the spaces of polynomial of degree less than or equal to $k$ in $\mathbb{R}^n$, one can verify that $\dim \mathbb{P}_k =\frac{\left(n+k\right)!}{n!k!}$. Consider
\begin{align}
p \in {\mathbb{P}_k}:{\mathbb{R}^n} &\to \mathbb{R}\\
x &\mapsto p\left( x \right): = \sum\limits_{\left| \alpha  \right| \le k} {{\gamma _\alpha }{x^\alpha }} ,
\end{align}
where $\alpha \in \mathbb{N}^n$ is the multi-index notation.
\begin{definition}[$n$-simplex in $\mathbb{R}^n$]
The $n$-simplex $K$ is defined as the convex hull of $\left(n+1\right)$-points $\left(a_j\right)_{j=1,\ldots,n+1}$, $a_j\in \mathbb{R}^n$. 
\end{definition}
Then, 
\begin{align}
\mathop K\limits^o  \ne \emptyset  \Leftrightarrow A: = \left[ {\begin{array}{*{20}{c}}
{{a_{1,1}}}& \cdots &{{a_{1,n + 1}}}\\
 \vdots & \ddots & \vdots \\
{{a_{n,1}}}& \ddots &{{a_{n,n + 1}}}\\
1& \cdots &1
\end{array}} \right] \mbox{ is not singular} ,
\end{align}
where ${a_j} = \left[a_{1j},\ldots,a_{nj}\right]^T$, $j=1,\ldots,n+1$. 

More explicitly, the $n$-simplex $K$ can be represented as 
\begin{align}
K: = \left\{ {x \in {\mathbb{R}^n};x = \sum\limits_{j = 1}^{n + 1} {{\lambda _j}{a_j}} ,\sum\limits_{j = 1}^{n + 1} {{\lambda _j}}  = 1,0 \le {\lambda _j} \le 1,j = 1, \ldots ,n + 1} \right\},
\end{align}
i.e., each point $x\in K$ can be written as a convex combination of the vertices $a_j$'s. The $\left(n+1\right)$-tuple $\lambda _j$'s is the barycentric coordinates of the corresponding vector $x\in \mathbb{R}^n$. Denote $\Lambda =\left[\lambda _1,\ldots\lambda _{n+1}\right]^T$, one has $A\Lambda  = \left[ {\begin{array}{*{20}{c}}
x\\
1
\end{array}} \right]$ . 

Moreover, if $B=A^{-1}$, then 
\begin{align}
{\lambda _i}\left( x \right) = \sum\limits_{j = 1}^n {{b_{ij}}{x_j}}  + {b_{i,n + 1}}.
\end{align}
Note that $\lambda _i$ is an affine function in the variable $x$, i.e., $\lambda _i\in \mathbb{P}_1\left(K\right)$, and 
\begin{align}
{\lambda _i}\left( {{a_j}} \right) = {\delta _{ij}},\hspace{2mm} 1 \le i,j \le n + 1.
\end{align}
$\star$ \textit{Type $\left(1\right)$ ($k=1$).} Any $p\in \mathbb{P}_1\left(K\right)$ is uniquely defined by its values at the vertices $\left(a_j\right)_{j=1,\ldots,n+1}$, 
\begin{align}
p\left( x \right) = \sum\limits_{i = 1}^{n + 1} {p\left( {{a_j}} \right){\lambda _j}\left( x \right)} .
\end{align}
These FEs are also called $\mathbb{P}_1$ Lagrange FEs. Let $K$ be the $n$-simplex, $P_K:=\mathbb{P}_1$, then the set of dof of $K$ is defined by
\begin{align}
{\Sigma _K}: = \left\{ {p\left( {{a_j}} \right);j = 1, \ldots ,n + 1} \right\} .
\end{align}
$\star$ \textit{Type $\left(2\right)$ ($k=2$).} Let $\overline {{a_{ij}}} : = \frac{1}{2}\left( {{a_i} + {a_j}} \right)\in \partial K$ be the middle of the edge whose vertices are $a_i$ and $a_j$.
\begin{lemma}
One has for all $1\le i,j,k\le n+1$,
\begin{align}
{\lambda _k}\left( {\overline {{a_{ij}}} } \right) = \frac{1}{2}\left( {{\delta _{ki}} + {\delta _{kj}}} \right),
\end{align}
and
\begin{align}
p\left( x \right) = \sum\limits_{i = 1}^{n + 1} {p\left( {{a_i}} \right){\lambda _i}\left( x \right)\left( {2{\lambda _i}\left( x \right) - 1} \right)}  + 4\sum\limits_{1 \le i < j \le n + 1} {p\left( {\overline {{a_{ij}}} } \right){\lambda _i}\left( x \right){\lambda _j}\left( x \right)} ,\hspace{2mm} \forall p \in {\mathbb{P}_2}\left( K \right).
\end{align}
\end{lemma}
These FEs are called $\mathbb{P}_2$ Lagrange FEs. Let $K$ be the $n$-simplex, $P_K:=\mathbb{P}_2$, $\dim \mathbb{P}_2=\frac{1}{2}\left(n+1\right)\left(n+2\right)$, the set of dof of $K$ is defined by 
\begin{align}
{\Sigma _K}: = \left\{ {p\left( {{a_j}} \right),p\left( {\overline {{a_{ij}}} } \right);1 \le i,j \le n + 1} \right\}.
\end{align}
The polynomial $p\in \mathbb{P}_2$ can be rewritten in terms of elements of $\Sigma _K$ as
\begin{align}
p\left( x \right) = \sum\limits_{\rm{dof } \in {\Sigma _K}} {{\phi _{\rm dof}}\left( x \right)\mbox{dof}} .
\end{align}

\begin{theorem}
For all positive integer $k$, $p\in \mathbb{P}_k$ is uniquely determined by its values on the set 
\begin{align}
{\Sigma _K}: = \left\{ {p\left( x \right);x = \sum\limits_{j = 1}^{n + 1} {{l_j}{a_j}} ,\hspace{2mm}\sum\limits_{j = 1}^{n + 1} {{l_j}}  = 1,\hspace{2mm} {l_j} \in \left\{ {0,\frac{1}{n}, \ldots ,1} \right\}} \right\}.
\end{align}
\end{theorem}
\noindent
$\star$ \textbf{From Simplices to Triangulation.} Recall $\overline \Omega   = \bigcup\nolimits_{K \in {\mathcal{T}_h}} K $, we need to our setting to be in $C^0\left(\overline{\Omega}\right)$ to be able to go from ${H^1}\left( {\mathop K\limits^o } \right)$ to $H^1\left(\Omega\right)$. Consider the following additional assumption on the triangulation $\mathcal{T}_h$,
\begin{itemize}
\item[$\left(\mathcal{T}_5\right)$] Any face of a $n$-simplex $K_1 \in \mathcal{T}_h$ is either a face of another simplex $K_2\in \mathcal{T}_h$ or a subset of $\partial \Omega$.
\end{itemize}
$\left(\rm FEM_2\right)$ The spaces $P_K$'s are spaces of polynomials and the global space of approximation $X_h$ is defined by 
\begin{align}
{X_h}: = \left\{ {v \in {C^0}\left( {\overline \Omega  } \right);{{\left. v \right|}_K} \in {P_K}} \right\}.
\end{align}
Hence, for all $K\in \mathcal{T}_h$, the spaces $P_K$'s should correspond to a polynomial space of the same order.
\begin{definition}[Global set of degree of freedoms]
The \emph{global set of dof} is defined by
\begin{align}
{\Sigma _h}: = \bigcup\limits_{K \in {\mathcal{T}_h}} {{\Sigma _K}} .
\end{align}
\end{definition}

\begin{theorem}
Let $X_h$ be the FE space for the $n$-simplices of degree $k\ge 1$. Then ${X_h} \subset {C^0}\left( {\overline \Omega  } \right) \cap {H^1}\left( \Omega  \right)$.
\end{theorem}
\noindent 
$\left(\rm FEM_3\right)$ $X_h$ admits a basis of elements whose supports are ``small''.

After renumeration, let us write 
\begin{align}
{\Sigma _h} = \left\{ {p\left( {{b_l}} \right);1 \le l \le M} \right\}.
\end{align}

\begin{definition}[Canonical basis]
The canonical basis of $X_h$ is defined by $w_l\in X_h$, $w_l\left(b_l'\right)=\delta _{ll'}$, $1\le l,l'\le M$.
\end{definition}

\begin{lemma}
\begin{align}
{\rm supp }\ {w_l} = \bigcup\limits_{{b_l} \in K \in {\mathcal{T}_h}} K ,\hspace{2mm} \forall l = 1, \ldots ,M.
\end{align}
\end{lemma}

\subsubsection{$n$-Rectangles of Type $\left(k\right)$}
Given $n\ge 2$, we denote by $\mathbb{Q}_k$ the space of polynomials of degree $\le k$ for each variable, independently. More explicitly, for any $p\in \mathbb{Q}_k$, $p$ can be represented as
\begin{align}
p\left( x \right) = \sum\limits_{\left\{ {\alpha ;{\alpha _i} \le k,1 \le i \le n} \right\}} {{\gamma _\alpha }{x^\alpha }}  = \sum\limits_{\left\{ {\alpha ;{\alpha _i} \le k,1 \le i \le n} \right\}} {{\gamma _\alpha }x_1^{{\alpha _1}} \cdots x_n^{{\alpha _n}}} .
\end{align}
For example, $k=1$, $p\left( x \right) = {\gamma _{00}} + {\gamma _{10}}{x_1} + {\gamma _{01}}{x_2} + {\gamma _{11}}{x_1}{x_2} \in {\mathbb{Q}_1}$. Moreover, $\mathbb{Q}_1 \subsetneq \mathbb{P}_2$.

In general, 
\begin{align}
\mathbb{P}_k \subsetneq \mathbb{Q}_k \subsetneq \mathbb{P}_{nk}.
\end{align}

\begin{lemma}
$\dim \mathbb{Q}_k= \left(k+1\right)^n$.
\end{lemma}
To go from the reference element $\widehat{K}$ to $K$, we define a diagonal affine mapping $F_K$ such that $F_K\left(\widehat{K}\right)=K$ and $F_K\left(x\right)=B_Kx +b_k$, where $B_K$ is a diagonal matrix, and $b_k\in \mathbb{R}^n$.
\begin{prop}
Any polynomial $p\in \mathbb{Q}_k \left(\widehat{K}\right)$ is uniquely determined by its value on the set
\begin{align}
{M_{\widehat K}}: = \left\{ {x = \left( {\frac{{{i_1}}}{k}, \ldots ,\frac{{{i_n}}}{k}} \right) \in \widehat K;{i_j} \in \left\{ {0, \ldots ,k} \right\},1 \le j \le n} \right\},
\end{align}
and then its set of dof is given by
\begin{align}
{\Sigma _{\widehat K}}: = \left\{ {p\left( x \right);x \in {M_{\widehat K}}} \right\}.
\end{align}
\end{prop}
\noindent
$\star$ \textbf{From $n$-Rectangles to the Triangulation/Rectangulation.} Same as before.

\subsubsection{Hermite FE on $n$-Simplices}
In addition to point values, we will use the values of the partial derivatives.
\begin{theorem}
Let $K$ be the $n$-simplex whose vertices are $a_i$, $i=1,\ldots,n+1$. Let 
\begin{align}
\overline {{a_{ijk}}} : = \frac{{{a_i} + {a_j} + {a_k}}}{3}, \hspace{2mm} 1 \le i,j,k \le n + 1.
\end{align}
Then a polynomial $p\in \mathbb{P}_3$ is uniquely determined by its values and the values of its partial derivatives at the vertices $a_i$'s and its values at the points $\overline{a_{ijk}}$'s.
\end{theorem}

\begin{proof}
Check that $\dim \mathbb{P}_3 ={\rm card}\ \Sigma _K$ where
\begin{align}
{\Sigma _K}: = \left\{ {p\left( {{a_j}} \right),p\left( {\overline {{a_{ijk}}} } \right),{\partial _l}p\left( {{a_i}} \right);1 \le i \le n + 1, \hspace{1mm}1 \le i < j < k \le n + 1, \hspace{1mm}1 \le l \le n} \right\}.
\end{align}
Check that for any $p\in \mathbb{P}_3$, we have
\begin{align}
p =&\ \sum\limits_{i = 1}^{n + 1} {\left( { - 2\lambda _i^3 + 3\lambda _i^2 - 7{\lambda _i}\sum\limits_{j < k,j \ne i,k \ne i} {{\lambda _j}{\lambda _k}} } \right)p\left( {{a_i}} \right)}  + 27\sum\limits_{i < j < k} {{\lambda _i}{\lambda _j}{\lambda _k}p\left( {\overline {{a_{ijk}}} } \right)} \\
& + \sum\limits_{i \ne j} {{\lambda _i}{\lambda _j}\left( {2{\lambda _i} + {\lambda _j} - 1} \right)Dp\left( {{a_i}} \right)\left( {{a_j} - {a_i}} \right)} .
\end{align}
This completes the proof.
\end{proof}

\section{General Properties}
\begin{definition}[Finite element]
A \emph{finite element} is a tuple $\left(K,P,\Sigma\right)$ where
\begin{enumerate}
\item $K$ is a closed subset of $\mathbb{R}^n$, $K\ne \emptyset$, and $\partial K$ Lipschitz-continuous (polyhedra).
\item $P$ is a space of real-valued functions defined on $K$ (e.g., polynomials).
\item $\Sigma$ is a finite set of linearly independent linear functions $\left( {{\phi _i}} \right)_{i = 1, \ldots ,N}$ defined on $P$. We assume that $\Sigma$ is $P$-unisolvent: for all  $\alpha \in \mathbb{R}^N$, there exists a unique $p\in P$ such that $\phi _i\left(p\right) =\alpha _i$, $i=1,\ldots,N$. Therefore, there exist $p_i \in P$, $i=1,\ldots,N$ such that $\phi _i\left(p_j\right) =\delta _{ij}$, $1\le i,j\le N$,
\begin{align}
p\left( x \right) = \sum\limits_{i = 1}^N {{\varphi _i}\left( p \right){p_i}\left( x \right)} ,\hspace{2mm}\forall p \in P,\hspace{1mm}\forall x \in K.
\end{align}
\end{enumerate} 
\end{definition}
We have $\dim P =\mbox{card} \Sigma$, and
\begin{equation}
\mbox{Duality: } \hspace{2mm}\left\{ \begin{split}
&{{\left( {{p_i}} \right)}_i}: & \mbox{ basis of } \left( {K,P,\Sigma } \right),\\
&{{\left( {{\phi _i}} \right)}_i}: & \mbox{ dof of } \left( {K,P,\Sigma } \right).
\end{split} \right.
\end{equation}

\begin{definition}[$P$-interpolation operator]
The \emph{$P$-interpolation operator} $\Pi$ of a smooth function $v$ is defined as 
\begin{align}
\Pi v: = \sum\limits_{i = 1}^N {{\phi _i}\left( v \right){p_i}} .
\end{align}
\end{definition}
Equivalently, we have $\Pi v \in P$ and ${\phi _i}\left( {\Pi v} \right) = \overline {{\phi _i}} \left( v \right)$, $\forall 1 \le i \le N$. 

\noindent
\textit{Note.} $\phi _i : P\to \mathbb{R}$, $\overline{\phi _i}: C^s \left(K\right) \to \mathbb{R}$ where $s$: $P\subset C^s$.
\begin{remark}
``Smooth''? $v\in C^s$, where $s$ is the maximal order of derivatives of $p$ which appears in the $\left(\phi _i\right)_i$. Moreover, $\mbox{Dom } \Pi = C^s \left(K\right)$, and $\Pi p=p$, $\forall p\in P$.
\end{remark}

\begin{prop}
Each FE presented before forms an affine family. Let $\left( {\widehat K,\widehat P,\widehat \Sigma } \right)$ be the reference element. For all $K\in \mathcal{T}_h$, there exist a unique  affine and invertible mapping $F_K$ such that
\begin{align}
{F_K}\left( {\widehat K} \right) = K,\hspace{1mm} {p_K} = \widehat {{p_K}} \circ F_K^{ - 1} .
\end{align}
\end{prop}
$F_K\left(\widehat{a}_i\right)=a_i$, where $\left(a_i\right)$ and $\left(\widehat{a}_i\right)$ are the points where the dof are computed (as for Hermite FE).

\begin{corollary}
The interpolant operator and the affine mapping commute. 
\begin{align}
\left( {{\Pi _K}v} \right) \circ {F_K}\left( v \right) = {\Pi _{\widehat K}}\left( {v \circ {F_K}} \right)\left( x \right),\hspace{2mm}\forall v \in {C^s}\left( K \right),\hspace{1mm}\forall x \in \widehat K.
\end{align}
\end{corollary}
In the notion of affine equivalence of $\left( {\widehat K,\widehat P,\widehat \Sigma } \right)$ and $\left(K,P,\Sigma\right)$, it is easier to use this commutation property instead of the correspondence of the $\left(\widehat{\phi_i}\right) _i$, and $\left(\phi _i\right)_i$.

\subsection{The Global Setting}
$\overline \Omega   = \bigcup\nolimits_{K \in {\mathcal{T}_h}} K$ , ${X_h}: = \left\{ {v \in {C^0}\left( {\overline \Omega  } \right);{{\left. v \right|}_K} \in {\mathbb{P}_k}} \right\}$, and ${\Sigma _h}: = \left\{ {{{\left( {{\varphi _{i,K}}} \right)}_{i = 1, \ldots ,N;K \in {\mathcal{T}_h}}}} \right\}$. 

In $\Sigma _h$, no repetition of the dof associated with points in ${K_1} \cap {K_2} \ne \emptyset $, so we change the notation as $\left(\phi _{h,i}\right) _{i=1,\ldots,M}$. Set
\begin{align}
{\Pi _h}\left( v \right) := \sum\limits_{j = 1}^M {{\varphi _{h,i}}\left( v \right){p_{h,i}}} ,
\end{align}
where the $p_{h,i}$ are defined by the $\left(p_{K,i}\right)_{i=1,\ldots,N}$ and extended by zero.

\section{Analysis of the FEM}
We want to look at the convergence of the FEM and obtain error estimates. Let $u$ be the solution of $\left({\rm VF}\right)$ in $V$, $u_h$ be the solution of $\left({\rm VF}_h\right)$ in $V_h$, where $V_h\subset V$, $\dim V_h <+\infty$, $V_h \subset X_h$. 
\begin{lemma}[C\'ea's lemma]
\begin{align}
{\left\| {u - {u_h}} \right\|_V} \le \frac{M}{\alpha }\mathop {\inf }\limits_{{v_h} \in {V_h}} {\left\| {u - {v_h}} \right\|_V}.
\end{align}
\end{lemma}
\noindent
\textit{Idea here}: Use ${v_h}: = {\Pi _h}u \in {V_h}$ to deduce
\begin{align}
{\left\| {u - {u_h}} \right\|_V} \le \frac{M}{\alpha }{\left\| {u - {\Pi _h}u} \right\|_V}.
\end{align}
``Just'' a problem of interpolation.

\begin{lemma}
\begin{align}
{\left\| {u - {\Pi _h}u} \right\|_{{H^1}\left( \Omega  \right)}} = {\left( {\sum\limits_{K \in {\mathcal{T}_h}} {\left\| {u - {\Pi _h}u} \right\|_{{H^1}\left( \Omega  \right)}^2} } \right)^{\frac{1}{2}}} .
\end{align}
\end{lemma}
As a result, local interpolation results are sufficient.

In the following, we have in mind the problem 
\begin{align}
\mbox{Find }u \in H_0^1\left( \Omega  \right) \mbox{ s.t. }- \Delta u = f \mbox{ in } \Omega ,
\end{align}
and Lagrange FE. The error in $K$ depends on the shape of $K$. 

Let $\widehat{K}$ be the reference element, i.e.,
\begin{align}
\widehat K: = \left\{ {x \in {\mathbb{R}^n};{x_i} \ge 0,\hspace{2mm}\forall i = 1, \ldots ,n,\hspace{2mm}\sum\limits_{i = 1}^n {{x_i}}  \le 1} \right\}.
\end{align}

\begin{prop}
Let $F_K$ be the affine mapping such that $F_K\left(\widehat{K}\right) =K$ and denote ${F_K}\left( {\widehat x} \right) = {B_K}\widehat x + {a_0}$, $\forall \widehat x \in \widehat K$. Then we have the following properties on matrix $B_K$.
\begin{enumerate}
\item $\left| {\det {B_K}} \right| = n!\left| K \right|$ (where $\left| K \right|$ : surface when $n=2$, volume when $n=3$);
\item ${\left\| {{B_K}} \right\|_2} \le {{{h_K}} \mathord{\left/
 {\vphantom {{{h_K}} {{\rho _{\widehat K}}}}} \right.
 \kern-\nulldelimiterspace} {{\rho _{\widehat K}}}}$;
\item ${\left\| {B_K^{ - 1}} \right\|_2} \le {{{h_{\widehat K}}} \mathord{\left/
 {\vphantom {{{h_{\widehat K}}} {{\rho _K}}}} \right.
 \kern-\nulldelimiterspace} {{\rho _K}}}$;
\end{enumerate}
where $h_K$ and $h_{\widehat{K}}$ are the diameters of $K$ and $\widehat{K}$, $\rho _K$ and $\rho _{\widehat{K}}$ are the diameter of the largest ball included in $K$ and $\widehat{K}$.
\end{prop}

\begin{remark}
\begin{itemize}
\item If $h_K =const$ and $\rho \to 0$ then $K$ becomes flat. More generally, it becomes flat when $\frac{{{h_K}}}{{{\rho _K}}} \to  + \infty $.
\item If $K'$ is homoketic to $K$ then
\begin{align}
\frac{{{h_K}}}{{{\rho _K}}} = \frac{{{h_{K'}}}}{{{\rho _{K'}}}} .
\end{align}
\end{itemize}
\end{remark}

\begin{theorem}
Let $v: K\to \mathbb{R}$ then $v\in H^m\left(K\right)$ iff $\widehat v = v \circ {F_K} \in {H^m}\left( {\widehat K} \right)$. 

Moreover, for all $0\le k\le m$,
\begin{align}
{\left| v \right|_{{H^k}\left( K \right)}} &\le \frac{{C{{\left| K \right|}^{\frac{1}{2}}}}}{{\rho _K^k}}{\left| {\widehat v} \right|_{{H^k}\left( {\widehat K} \right)}},\\
{\left| {\widehat v} \right|_{{H^k}\left( {\widehat K} \right)}} &\le \frac{{Ch_K^k}}{{{{\left| K \right|}^{\frac{1}{2}}}}}{\left| v \right|_{{H^k}\left( K \right)}},
\end{align}
where 
\begin{align}
{\left| v \right|_{{H^k}\left( K \right)}}: = {\left( {\sum\limits_{\left| \alpha  \right| = k} {\left\| {{\partial ^\alpha }v} \right\|_{{L^2}\left( \Omega  \right)}^2} } \right)^{\frac{1}{2}}} .
\end{align}
\end{theorem}

\begin{proof}
For $m=1$, $k=0$ ($L^2$),
\begin{align}
\left\| v \right\|_{{L^2}\left( K \right)}^2 = \int_K {{{\left| {v\left( x \right)} \right|}^2}dx}  = \int_{\widehat K} {{{\left| {\widehat v\left( {\widehat x} \right)} \right|}^2}\left| {\det {B_K}} \right|d\widehat x}  \le C\left\| {\widehat v} \right\|_{{L^2}\left( {\widehat K} \right)}^2.
\end{align}
For $k=1$, estimate ${\left\| {\nabla v} \right\|_{{L^2}}}$: $\nabla \widehat v = B_K^T\nabla v \circ {F_K}$. Since $\widehat v = v \circ {F_K}$ (assume $a_0=0$, where ${F_K}\left( {\widehat x} \right) = {B_K}\widehat x + {a_0}$),
\begin{align}
\widehat v\left( {\widehat x} \right) = v\left( {\sum\limits_j {{b_{1j}}\widehat {{x_j}}} , \ldots ,\sum\limits_j {{b_{nj}}\widehat {{x_j}}} } \right),\\
\frac{{\partial \widehat v}}{{\partial \widehat {{x_i}}}}\left( {\widehat x} \right) = {b_{1i}}\frac{{\partial v}}{{\partial {x_1}}}\left( {{F_K}\left( {\widehat x} \right)} \right) +  \cdots  + {b_{ni}}\frac{{\partial v}}{{\partial {x_n}}}\left( {{F_K}\left( {\widehat x} \right)} \right).
\end{align}
Hence, $\nabla \widehat v\left( {\widehat x} \right) = B_K^T\nabla v\left( {{F_K}\left( {\widehat x} \right)} \right)$, and
\begin{align}
{\left\| {\nabla \widehat v} \right\|_{{L^2}\left( {\widehat K} \right)}} \le {\left\| {B_K^T} \right\|_2}{\left\| {\widehat {\nabla v}} \right\|_{{L^2}\left( {\widehat K} \right)}} ,
\end{align}
where $\widehat {\nabla v}: = \nabla v \circ {F_K}$.
\begin{align}
\left\| {\nabla \widehat v} \right\|_{{L^2}\left( {\widehat K} \right)}^2 \le \frac{{h_K^2}}{{\rho _K^2}}\left\| {\widehat {\nabla v}} \right\|_{{L^2}\left( {\widehat K} \right)}^2 \le \frac{1}{{\left| K \right|}}\frac{{Ch_K^2}}{{\rho _{\widehat K}^2}}\left\| {\nabla v} \right\|_{{L^2}\left( K \right)}^2,
\end{align}
where the last inequality is obtained from the previous inequality for $k=0$.

The other inequality is obtained by changing $K$ and $\widehat{K}$.
\end{proof}

\begin{definition}
Let $\widehat{v}$ be a smooth function on $\widehat{K}$, we introduce
\begin{align}
{\Pi _{\widehat K}}\widehat v\left( {\widehat x} \right): = \sum\limits_{i = 1}^{{\rm card }{\Sigma _K}} {\widehat v\left( {{a_i}} \right){p_i}\left( {\widehat x} \right)} ,\hspace{2mm} \forall \widehat{x}\in \widehat{K}.
\end{align}
Let $v$ be a smooth function on $\overline{\Omega}$, we introduce
\begin{align}
{\Pi _h}v\left( x \right): = \sum\limits_{\alpha  \in {\rm dof}} {v\left( \alpha  \right){p_{h,i}}\left( x \right)} ,\hspace{2mm}\forall x \in \overline \Omega  .
\end{align}
\end{definition}

\begin{theorem}[Deny-Lion]
Let $U$ be an open bounded set of $\mathbb{R}^n$, with $\partial U$ Lipschitz-continuous, and $k\in \mathbb{N}$. There exists a constant $C>0$ such that
\begin{align}
\mathop {\inf }\limits_{\pi  \in {\mathbb{P}_k}\left( U \right)} {\left\| {u - \pi } \right\|_{{H^{k + 1}}\left( U \right)}} \le C{\left| u \right|_{{H^{k + 1}}\left( U \right)}},\hspace{2mm}\forall u \in {H^{k + 1}}\left( U \right).
\end{align}
\end{theorem}
In the case $k=0$, we obtain the Poincar\'e-Wirtinger inequality
\begin{align}
{\left\| {u - \frac{1}{{\left| U \right|}}\int_U {u\left( x \right)dx} } \right\|_{{H^1}}} \le C{\left\| u \right\|_{{H^1}}}, \hspace{2mm} \forall u\in H^1\left(U\right).
\end{align}

\begin{proof}
For all $\alpha \in \mathbb{N}^n$, $\left| \alpha  \right| \le k$, $f_\alpha$ be a linear form defined on $H^{k+1}\left(U\right)$ such that
\begin{align}
{f_\alpha }\left( u \right): = \int_U {{\partial ^\alpha }udx} , \hspace{2mm} \forall u \in {H^{k + 1}}\left( U \right).
\end{align}
Define 
\begin{align}
\mathcal{F}:{\mathbb{P}_k} &\to {\mathbb{R}^M}\\
\pi  &\mapsto {\left( {{f_\alpha }\left( x \right)} \right)_{\left| \alpha  \right| \le k}},
\end{align}
then $\mathcal{F}$ is bijective (injectivity actually is sufficient). Indeed, suppose for the contrary, let $\pi \in \mathbb{P}_k$ such that $\pi \ne 0$, $\mathcal{F}\left(\pi\right) =0$. Let $\alpha _0\in \mathbb{N}^n$ such that the monomial $x^{\alpha _0}$ is nonzero and all the monomial $x^\alpha$ with $\left| \alpha  \right| > \left| {{\alpha _0}} \right|$ are null. But $f_{\alpha _0} \left(\pi\right)$, contradiction. Therefore, $\mathcal{F}$ is bijective. 

Let us check that there exists a constant $C>0$ such that
\begin{align}
{\left\| u \right\|_{{H^{k + 1}}}} \le C\left( {{{\left| u \right|}_{k + 1}} + \sum\limits_{\left| \alpha  \right| \le k} {\left| {{f_\alpha }\left( x \right)} \right|} } \right),\hspace{2mm} \forall u \in {H^{k + 1}}\left( U \right),
\end{align}
where
\begin{align}
{\left| u \right|_{k + 1}}: = {\left( {\sum\limits_{\left| \alpha  \right| = k + 1} {\left\| {{\partial ^\alpha }u} \right\|_{{L^2}}^2} } \right)^{\frac{1}{2}}},\hspace{2mm} \forall u \in {H^{k + 1}}\left( U \right).
\end{align}
By contradiction, there exists a sequence ${\left( {{u_n}} \right)_n} \subset {H^{k + 1}}\left( U \right)$ such that ${\left\| u \right\|_{{H^{k + 1}}}} = 1$, and
\begin{align}
{\left| {{u_n}} \right|_{k + 1}} + \sum\limits_{\left| \alpha  \right| \le k} {\left| {{f_\alpha }\left( {{u_n}} \right)} \right|}  \le \frac{1}{n}.
\end{align}
Then there exists a subsequence ${\left( {{u_{\varphi \left( n \right)}}} \right)_n}$ such that $u_{\varphi \left(n\right)} \rightharpoonup u$ in $H^{k+1}\left(U\right)$, and $u_{\varphi \left(n\right)} \to u$ in $H^k\left(U\right)$ as $n\to \infty$. 

Moreover, for all $\alpha \in \mathbb{N}^n$, $\left| \alpha  \right| = k + 1$, $\partial ^\alpha u_{\varphi \left(n\right)} \rightharpoonup 0 = \partial ^\alpha  u$ as $n\to \infty$, thus $u\in \mathbb{P}_k$. Also, ${f_\alpha }\left( {{u_{\varphi \left( n \right)}}} \right) \to 0$ as $n \to \infty$, $\forall \alpha$, $\left| \alpha  \right| \le k$. Therefore, $\mathcal{F} \left(u\right)=0 \Rightarrow u=0$. This is impossible since ${\left\| u \right\|_{{H^{k + 1}}}} = 1$. For all $u\in H^{k+1}\left(U\right)$, by surjectivity of $\mathcal{F}$, there exists a $\widetilde \pi \in \mathbb{P}_k$ such that $\mathcal{F}\left(u\right) =\mathcal{F}\left(\widetilde \pi\right)$. Hence, for all $\alpha \in \mathbb{N}^n$, $\left| \alpha  \right| \le k$, $f_\alpha \left(u-\widetilde \pi\right) =0$.
\begin{align}
\mathop {\inf }\limits_{\pi  \in {\mathbb{P}_k}} {\left\| {u - \pi } \right\|_{{H^{k + 1}}\left( U \right)}} \le {\left\| {u - \widetilde \pi } \right\|_{{H^{k + 1}}\left( U \right)}} \le C\left( {{{\left| {u - \widetilde \pi } \right|}_{k + 1}} + \sum\limits_{\left| \alpha  \right| \le k} {{f_\alpha }\left( {u - \widetilde \pi } \right)} } \right) \le C{\left| u \right|_{k + 1}},
\end{align}
since ${\left| {\widetilde \pi } \right|_{k + 1}} = 0$ (because $\widetilde \pi \in \mathbb{P}_k$).
\end{proof}
\noindent
\textit{Note.} $U$ has to be convex. 
\begin{remark}
\begin{itemize}
\item ${\rm card }\left\{ {\alpha  \in {\mathbb{N}^n};\left| \alpha  \right| \le k} \right\} = \dim {P_k} = :M$.
\item This theorem can be rephrased by considering the quotient space $H^{k+1}/\mathbb{P}_k$, and
\begin{align}
\mathop V\limits^o : = \left\{ {w \in {H^{k + 1}}\left( U \right);w - v \in {\mathbb{P}_k}} \right\}.
\end{align}
\end{itemize}
\end{remark}

\begin{theorem}[Bramble-Hilbert]
Let $U$ be an open convex subset of $\mathbb{R}^n$, $\partial U$ bounded and Lipschitz-continuous. Let $\Phi$ be a linear continuous mapping from $H^{k+1}\left(U\right)$ to a Banach space $E$. If $\Phi =0$ on $\mathbb{P}_k$, then there exists a constant $C>0$ such that 
\begin{align}
{\left\| {\Phi u} \right\|_E} \le C{\left| u \right|_{k + 1}},\hspace{2mm}\forall u \in {H^{k + 1}}\left( U \right).
\end{align}
\end{theorem}

\begin{proof}
Let $u\in H^{k+1}\left(U\right)$, $\pi \in \mathbb{P}_k$, ${\left\| {\Phi u} \right\|_E} = {\left\| {\Phi \left( {u - \pi } \right)} \right\|_E}$, then 
\begin{align}
\left\| {\Phi u} \right\| \le \left\| \Phi  \right\|{\left\| {u - \pi } \right\|_{{H^{k + 1}}}} .
\end{align}
Since this is true for all $\pi$,
\begin{align}
{\left\| {\Phi u} \right\|_E} \le C\mathop {\inf }\limits_{\pi  \in {\mathbb{P}_k}} {\left\| {u - \pi } \right\|_{{H^{k + 1}}}} \le C{\left| u \right|_{k + 1}} ,
\end{align}
where the last inequality is deduced from Deny-Lions theorem.
\end{proof}
Now we can take $\Phi :u \mapsto u - {\Pi _K}u$.

\begin{prop}
There exists a constant $C>0$ such that  $\forall K \in {\mathcal{T}_h},\forall v \in {H^{k + 1}}\left( K \right)$:
\begin{align}
{\left| {v - {\Pi _K}v} \right|_{{H^k}\left( K \right)}} &\le \frac{{Ch_K^{k + 1}}}{{\rho _K^k}}{\left| v \right|_{{H^{k + 1}}\left( K \right)}},\\
{\left| v \right|_k} &\le \frac{{C{{\left| K \right|}^{\frac{1}{2}}}}}{{\rho _K^{{h_K}}}}{\left| {\widehat v} \right|_k},\\
{\left| {\widehat v} \right|_k} &\le \frac{{C{h_K}}}{{{{\left| K \right|}^{\frac{1}{2}}}}},
\end{align}
where $h_K$ is ${\rm diam } K$, $\rho _K$ is the diameter of the largest ball included in $K$.
\end{prop}

\begin{proof}
Define
\begin{align} 
\Phi :{H^{k + 1}}\left( {\widehat K} \right) &\to {H^k}\left( {\widehat K} \right)\\
\widehat v &\mapsto \widehat v - {\Pi _{\widehat K}}\widehat v,
\end{align}
then $\Phi \pi  = 0,\forall \pi  \in {P_k}$, and $\Phi$ is linear continuous.

\textbf{Bramble-Hilbert theorem}: $\exists C > 0$, $\forall \widehat v \in {H^{k + 1}}\left( {\widehat K} \right)$, ${\left\| {\widehat v - {\Pi _{\widehat K}}\widehat v} \right\|_{{H^k}}} \le C{\left| {\widehat v} \right|_{{H^{k + 1}}}}$, for any $K \in \mathcal{T}_h$, let $F_K$ the associated affine mapping s.t. $F_K\left(\widehat{K}\right)=K$.

For all $v\in H^{k+1}\left(K\right)$, we denote $\widehat v = v \circ {F_K}$, then
\begin{align}
{\left| {v - {\Pi _K}v} \right|_{{H^k}\left( K \right)}} \le \frac{{C{{\left| K \right|}^{\frac{1}{2}}}}}{{\rho _K^k}}{\left| {\widehat v - {\Pi _{\widehat K}}\widehat v} \right|_{{H^k}\left( {\widehat K} \right)}} \le \frac{{C{{\left| K \right|}^{\frac{1}{2}}}}}{{\rho _K^k}}{\left| {\widehat v} \right|_{{H^{k + 1}}\left( {\widehat K} \right)}} \le \frac{{Ch_K^{k + 1}}}{{\rho _K^k}}{\left| v \right|_{{H^{k + 1}}\left( K \right)}}.
\end{align}
This completes the proof.
\end{proof}
${\mathbb{P}_k} \subset {H^1}\left( K \right)$, $\forall K$ and ${X_h} \subset {C^0}\left( \Omega  \right) \Rightarrow {X_h} \subset {H^1}\left( \Omega  \right)$. 

\noindent
$\star$ \textbf{From the local estimate to the global estimate.} 
\begin{align}
\forall v \in {H^{k + 1}}\left( \Omega  \right),\hspace{2mm} {\left| {v - {\Pi _h}v} \right|_{{H^k}\left( \Omega  \right)}} \le C\sigma _h^k\mathop {\sup }\limits_{K \in {\mathcal{T}_h}} {h_K}{\left| v \right|_{{H^{k + 1}}\left( \Omega  \right)}},
\end{align}
where
\begin{align}
{\sigma _h} = \mathop {\sup }\limits_{K \in {\mathcal{T}_h}} \frac{{{h_K}}}{{{\rho _K}}} .
\end{align}
$\to$ Need to have ${H^{k + 1}}\left( \Omega  \right) \subset {C^0}\left( {\overline \Omega  } \right)$, i.e., $k+1\ge \frac{d}{2}$ ($d=1$: $k\ge 1$, $d=2,3$: $k\ge 1$).

\begin{definition}
A family of meshes $\left(\mathcal{T}_h\right)_{h\ge 0}$ is called \emph{regular} if there exists $C>0$ s.t. $\sigma _h\le C$, $\forall h>0$.
\end{definition}
$\to$ To avoid flat triangles.

\begin{lemma}
A family of meshes is regular $ \Leftrightarrow \exists C > 0$, $\forall h > 0$, $\forall K \in {\mathcal{T}_h}$, $\left| K \right| \ge Ch_K^n$.
\end{lemma}

\begin{theorem}
Let $\left(\mathcal{T}_h\right)_{h\ge 0}$ be a regular family of meshes and define $h := {\sup _{K \in {\mathcal{T}_h}}}{h_K}$ for any $\mathcal{T}_h$. Assume $f$ in $\left({\rm VF}\right)$ to be in $L^2 \left(\Omega\right)$ (then $u \in H_0^1\left( \Omega  \right) \cap {H^2}\left( \Omega  \right)$). Consider $u_h$ given by Lagrange FE of degree $k$,
\begin{align}
{\left\| {u - {u_h}} \right\|_{{H^1}\left( \Omega  \right)}} &\le Ch{\left| u \right|_{{H^2}\left( \Omega  \right)}},\\
{\left\| {u - {u_h}} \right\|_{{L^2}\left( \Omega  \right)}} &\le C{h^2}{\left| u \right|_{{H^2}\left( \Omega  \right)}}.
\end{align}
\end{theorem}

\begin{remark}
The error estimates do not depend on $k$, $\to$ only order 1.
\end{remark}

\begin{proof}[Proof of the last inequality]
Aubin-Nitsche Trick. Look at the adjoint problem; For any $v\in L^2\left(\Omega\right)$, find $\varphi _v\in H_0^1\left(\Omega\right)$ s.t. 
\begin{align}
\left( {\rm AVF} \right) \hspace{2mm} a\left( {w,{\varphi _v}} \right) = {\left( {v,w} \right)_{{L^2}}}, \hspace{2mm} \forall w \in H_0^1\left( \Omega  \right).
\end{align}
then $\varphi _v\in H^2\left(\Omega\right)$, ${\left\| {{\varphi _v}} \right\|_{{H^2}}} \le C{\left\| v \right\|_{{L^2}}}$. Define error $e_h:=u-u_h$ and take $v=w=e_h$ in $\left({\rm AVF}\right)$: $a\left( {{e_h},{\varphi _{{e_h}}}} \right) = \left\| {{e_h}} \right\|_{{L^2}}^2$. Therefore, $\left\| {{e_h}} \right\|_{{L^2}}^2 = a\left( {{e_h},{\varphi _{{e_h}}} - {\Pi _h}{\varphi _{{e_h}}}} \right)$ because $e_h$ is $a$-orthogonal to $V_h$: $a\left(e_h,v_h\right)=0$, $\forall v_h\in V_h$.
\begin{align}
\left\| {{e_h}} \right\|_{{L^2}}^2 &\le \left\| a \right\|{\left\| {{e_h}} \right\|_{{H^1}}}{\left\| {{\varphi _{{e_h}}} - {\Pi _h}{\varphi _{{e_h}}}} \right\|_{{H^1}}}\\
& \le \left\| a \right\|{\left\| {{e_h}} \right\|_{{H^1}}}h{\left\| {{\varphi _{{e_h}}}} \right\|_{{H^2}}} \mbox{ (interpolation result)}\\
& \le \left\| a \right\|{\left\| {{e_h}} \right\|_{{H^1}}}h{\left\| {{e_h}} \right\|_{{L^2}}}\mbox{ (elliptic smoothness)}.
\end{align}
Thus, ${\left\| {{e_h}} \right\|_{{L^2}}} \le C{h^2}{\left| u \right|_{{H^2}}}$ (first error estimate). 
\end{proof}

\section{Practical Part}
$-\Delta u +u=f$ and BCs. PDE $\to$ VF. We take test function $v=0$ on $\Gamma _D$ because no dof on $\Gamma _D$.
\begin{align}
u &\in \left\{ {w \in {H^1};{{\left. w \right|}_{{\Gamma _D}}} = {u_D}} \right\} = V + {U_D},\\
v &\in \left\{ {w \in {H^1};{{\left. w \right|}_{{\Gamma _D}}} = 0} \right\} = V,
\end{align}
where $U_D\in H^1$ and ${\left. {{U_D}} \right|_{{\Gamma _D}}} = {u_D}$ (Dirichlet boundary condition). 

\textbf{Discretization.} $\Omega  \to {\Omega _h} = \bigcup\nolimits_{K \in {\mathcal{T}_h}} K $, and $V \to {V_h} = \mbox{span}\left\{ {{\varphi _1}, \ldots ,{\varphi _N}} \right\}$: a space of finite dimension.
\begin{align*}
\left( {\rm VF}_h \right) \hspace{2mm} {u_h} \in {V_h} + {U_D}, \hspace{2mm} \forall {v_h} \in {V_h}:\int_{{\Omega _h}} {\nabla {u_h} \cdot \nabla {v_h}}  + \int_{{\Omega _h}} {{u_h}{v_h}}  = \int_{{\Omega _h}} {{f_h}{v_h}}  + \int_{{\Gamma _{N,h}}} {{g_{N,h}}{v_h}} .
\end{align*}
${u_h} = {U_D} + \sum\limits_{J = 1}^N {{u_J}{\varphi _J}} $, take $v=\varphi _I$, $I=1,\ldots,N$.
\begin{align*}
\sum\limits_{J = 1}^N {{u_J}\left( {\int_{{\Omega _h}} {\nabla {\varphi _J} \cdot \nabla {\varphi _I}}  + \int_{{\Omega _h}} {{\varphi _J}{\varphi _I}} } \right)}  = \int_{{\Omega _h}} {f{\varphi _I}}  + \int_{{\Gamma _N}} {{g_N}{\varphi _I}}  - \int_{{\Omega _h}} {\nabla {u_D} \cdot \nabla {\varphi _I}}  - \int_{{\Omega _h}} {{u_D}{\varphi _I}} .
\end{align*}
obtain $AX=B$. 

${H^1}\mathop  \to \limits^{\rm F.E} {V_h}$. The F.E is conform if $V_h\subset H^1$: $\mathbb{P}_1$ Lagrange F.E. $H^1$ conform and $\mathbb{P}_1$ Lagrange F.E. nonconform. 

$\mathbb{P}^k$ F.E., $\to$ Nodes $no_I$, $I=1,\ldots,M$, $\varphi _I$ defined on $no_I$: $\varphi _I \in V_h$, $\varphi _I\left(no_J\right) =\delta _{IJ}$.
\begin{align}
\int_{{\Omega _h}} {{\varphi _I}{\varphi _J}}  = \sum\limits_{K \in {\mathcal{T}_h}} {\int_K {{\varphi _I}{\varphi _J}} }  = \sum\limits_{K \in {\mathcal{T}_h}, \hspace{1mm} n{o_I} \in K,\hspace{1mm} n{o_J} \in K} {\int_K {{\varphi _I}{\varphi _J}} } .
\end{align}
$\exists i,j \in \left\{ {1,2,3} \right\}$, ${\varphi _I} = \varphi _i^K$, ${\varphi _J} = \varphi _j^K$, $\varphi _i^K \in {\mathbb{P}^k}\left( K \right)$, $\varphi _i^K\left( {no_j^K} \right) = {\delta _{ij}}$.

The integral ${\int_K {{\varphi _I}{\varphi _J}} }$ can be calculated from the knowledge of $\left( {\int_K {\varphi _i^K\varphi _j^K} } \right)_{i,j = 1,2,3}$ (elementary matrix related to $K$).

\noindent
\textbf{FE Calculate of $A$ and $B$.} Calculation of the elementary matrices and the elementary RHS. $\exists !{F_K} \in {P_1} \times {P_1}$ s.t. ${F_K}\left( {{{\widehat {no}}_i}} \right) = no_i^K$.
\begin{align}
\int_K {\varphi _i^K\left( x \right)\varphi _j^K\left( x \right)dx} & = \int_{\widehat K} {\underbrace {\varphi _i^K\left( {{F_K}\left( {\widehat x} \right)} \right)}_{{{\widehat \varphi }_i}\left( {\widehat x} \right)}\underbrace {\varphi _j^K\left( {{F_K}\left( {\widehat x} \right)} \right)}_{{{\widehat \varphi }_j}\left( {\widehat x} \right)}\left| {J{F_K}\left( {\widehat x} \right)} \right|d\widehat x} \\
& = \int_{\widehat K} {{{\widehat \varphi }_i}\left( {\widehat x} \right){{\widehat \varphi }_j}\left( {\widehat x} \right)\left| {J{F_K}\left( {\widehat x} \right)} \right|d\widehat x} .
\end{align}
More precise:
\begin{align}
{F_K}\left( {\widehat x} \right) = \sum\limits_{i = 1}^3 {no_i^K{{\widehat \psi }_i}\left( {\widehat x} \right)} .
\end{align}
Isoparametric F.E. ${\widehat \psi _i} = {\widehat \varphi _i}$. Non-isoparametric F.E. ${\left( {{{\widehat \varphi }_i}} \right)_i} \ne {\left( {{{\widehat \psi }_i}} \right)_i}$.
\begin{align}
{\widehat \varphi _1}\left( {\begin{array}{*{20}{c}}
{{{\widehat x}_1}}\\
{{{\widehat x}_2}}
\end{array}} \right) = {\widehat x_1}, \hspace{2mm} {\widehat \varphi _2}\left( {\begin{array}{*{20}{c}}
{{{\widehat x}_1}}\\
{{{\widehat x}_2}}
\end{array}} \right) = {\widehat x_2},\hspace{2mm} {\widehat \varphi _1}\left( {\begin{array}{*{20}{c}}
{{{\widehat x}_1}}\\
{{{\widehat x}_2}}
\end{array}} \right) = 1 - {\widehat x_1} - {\widehat x_2} .
\end{align}
We have $\varphi _i^K \circ {F_K} = {\widehat \varphi _i}$, $i=1,2,3$. 

Integration over $\widehat{K}$: use Gauss quadrature formula for the reference triangle.

Some useful identities:
\begin{align}
\int_0^1 {{x^k}{{\left( {1 - x} \right)}^k}dx}  &= \frac{{k!l!}}{{\left( {k + l + 1} \right)!}},\\
\int_K {{{\left( {\lambda _1^K} \right)}^{{k_1}}}{{\left( {\lambda _2^K} \right)}^{{k_2}}}}  &= 2\left| K \right|\frac{{{k_1}!{k_2}!}}{{\left( {{k_1} + {k_2} + 2} \right)!}},\\
\nabla {\lambda _1} \cdot \nabla {\lambda _1} &= \frac{1}{{4{{\left| K \right|}^2}}}\left\| {no_2^Kno_3^K} \right\|,\\
\nabla {\lambda _2} \cdot \nabla {\lambda _3} &=  - \frac{1}{{4{{\left| K \right|}^2}}}\overrightarrow {n{o_1}n{o_2}}  \cdot \overrightarrow {n{o_1}n{o_3}} .
\end{align}


%\section*{Acknowledgment}

\printbibliography
%\bibliographystyle{siam}
%\bibliography{MYBIB}
%\Addresses
\end{document}