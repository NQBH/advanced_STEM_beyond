\documentclass{article}
\usepackage[backend=biber,natbib=true,style=alphabetic,maxbibnames=50]{biblatex}
\addbibresource{/home/nqbh/reference/bib.bib}
\usepackage[utf8]{vietnam}
\usepackage{tocloft}
\renewcommand{\cftsecleader}{\cftdotfill{\cftdotsep}}
\usepackage[colorlinks=true,linkcolor=blue,urlcolor=red,citecolor=magenta]{hyperref}
\usepackage{amsmath,amssymb,amsthm,enumitem,float,graphicx,mathtools,tikz}
\usetikzlibrary{angles,calc,intersections,matrix,patterns,quotes,shadings}
\allowdisplaybreaks
\newtheorem{assumption}{Assumption}
\newtheorem{baitoan}{}
\newtheorem{cauhoi}{Câu hỏi}
\newtheorem{conjecture}{Conjecture}
\newtheorem{corollary}{Corollary}
\newtheorem{dangtoan}{Dạng toán}
\newtheorem{definition}{Definition}
\newtheorem{dinhly}{Định lý}
\newtheorem{dinhnghia}{Định nghĩa}
\newtheorem{example}{Example}
\newtheorem{ghichu}{Ghi chú}
\newtheorem{hequa}{Hệ quả}
\newtheorem{hypothesis}{Hypothesis}
\newtheorem{lemma}{Lemma}
\newtheorem{luuy}{Lưu ý}
\newtheorem{nhanxet}{Nhận xét}
\newtheorem{notation}{Notation}
\newtheorem{note}{Note}
\newtheorem{principle}{Principle}
\newtheorem{problem}{Problem}
\newtheorem{proposition}{Proposition}
\newtheorem{question}{Question}
\newtheorem{remark}{Remark}
\newtheorem{theorem}{Theorem}
\newtheorem{vidu}{Ví dụ}
\usepackage[left=1cm,right=1cm,top=5mm,bottom=5mm,footskip=4mm]{geometry}
\def\labelitemii{$\circ$}
\DeclareRobustCommand{\divby}{%
	\mathrel{\vbox{\baselineskip.65ex\lineskiplimit0pt\hbox{.}\hbox{.}\hbox{.}}}%
}
\setlist[itemize]{leftmargin=*}
\setlist[enumerate]{leftmargin=*}

\title{Lecture: Probability Theory {\it\&} Statistics Theory\\Bài Giảng: Lý Thuyết Xác Suất {\it\&} Lý Thuyết Thống Kê}
\author{Nguyễn Quản Bá Hồng\footnote{A scientist- {\it\&} creative artist wannabe, a mathematics {\it\&} computer science lecturer of Department of Artificial Intelligence {\it\&} Data Science (AIDS), School of Technology (SOT), UMT Trường Đại Học Quản Lý {\it\&} Công Nghệ Thành Phố Hồ Chí Minh, \url{https://www.umt.edu.vn}, Việt Nam.\\E-mail: {\sf nguyenquanbahong@gmail.com} {\it\&} {\sf hong.nguyenquanba@umt.edu.vn}. Website: \url{https://nqbh.github.io/}. GitHub: \url{https://github.com/NQBH}.}}
\date{\today}

\begin{document}
\maketitle
\begin{abstract}
	This text is a part of the series {\it Some Topics in Advanced STEM \& Beyond}:
	
	{\sc url}: \url{https://nqbh.github.io/advanced_STEM/}.
	
	Latest version:
	\begin{itemize}
		\item {\it Lecture: Probability Theory \& Statistics Theory -- Bài Giảng: Lý Thuyết Xác Suất \& Lý Thuyết Thống Kê}.
		
		PDF: {\sc url}: \url{https://github.com/NQBH/advanced_STEM_beyond/blob/main/probability_statistics/lecture/NQBH_probability_statistics_lecture.pdf}.
		
		\TeX: {\sc url}: \url{https://github.com/NQBH/advanced_STEM_beyond/blob/main/probability_statistics/lecture/NQBH_probability_statistics_lecture.tex}.
		\item {\it }.
		
		PDF: {\sc url}: \url{.pdf}.
		
		\TeX: {\sc url}: \url{.tex}.
	\end{itemize}
\end{abstract}
\tableofcontents

%------------------------------------------------------------------------------%

\section{Basic Probability Theory -- Lý Thuyết Xác Suất Cơ Bản}

\subsection{Measure space \& probability space -- Không gian đo \& không gian xác suất}
2 giả định về không gian mẫu $\Omega$:
\begin{enumerate}
	\item Không gian mẫu $\Omega$ là đồng xác suất.
	\item Không gian mẫu $\Omega$ là không đồng xác suất.
\end{enumerate}

%------------------------------------------------------------------------------%

\subsection{Random variable -- Biến ngẫu nhiên}
Đặt $(X = a)$ chỉ biến cố ``$X$ lấy giá trị $a$''.

%------------------------------------------------------------------------------%

\subsection{Discrete random variable -- Biến ngẫu nhiên rời rạc}
Các biến ngẫu nhiên rời rạc trong lý thuyết xác suất thường được biểu diễn bằng các bảng phân phối xác suất:
\begin{table}[H]
	\centering
	\begin{tabular}{|c|c|c|c|c|}
		\hline
		$X$ & $x_1$ & $x_2$ & $\ldots$ & $x_n$ \\
		\hline
		$P(X = x_i)$ & $p_1$ & $p_2$ & $\ldots$ & $p_n$ \\
		\hline
	\end{tabular}
\end{table}
Xét không gian xác suất $(\Omega,\mathfrak{M},P)$ \& ánh xạ $X:\Omega\to\mathbb{R}$.

\begin{dinhnghia}
	$X$ được gọi là 1 {\rm biến ngẫu nhiên} trên $\Omega$ khi
	\begin{equation*}
		(X\le x)\coloneqq\{\omega\in\Omega;X(\omega)\le x\} = X^{-1}((-\infty,x])\in\mathfrak{M},\ \forall x\in\mathbb{R}.
	\end{equation*}
	Khi đó, hàm $F_X:\mathbb{R}\to\mathbb{R}$ xác định bởi
	\begin{equation}
		\label{cdf}
		\tag{cdf}
		F_X(x)\coloneqq P(X\le x),\ \forall x\in\mathbb{R},
	\end{equation}
	được gọi là {\rm hàm phân phối tích lũy (cumulative distribution function, abbr., cdf)} của biến ngẫu nhiên $X$.
\end{dinhnghia}

%------------------------------------------------------------------------------%

\subsection{Continuous random variable -- Biến ngẫu nhiên liên tục}

%------------------------------------------------------------------------------%

\subsection{Limit theorems -- Các định lý giới hạn}

%------------------------------------------------------------------------------%

\section{Basic Statistics Theory -- Lý Thuyết Thống Kê Cơ Bản}

\subsection{Data description -- Mô tả dữ liệu}

%------------------------------------------------------------------------------%

\subsection{Sample theory -- Lý thuyết mẫu}

%------------------------------------------------------------------------------%

\subsection{Estimation theory -- Lý thuyết ước lượng}

%------------------------------------------------------------------------------%

\subsection{Testing theory -- Lý thuyết kiểm định}

%------------------------------------------------------------------------------%

\subsection{Linear regression model -- Mô hình hồi quy tuyến tính}

%------------------------------------------------------------------------------%

\section{Miscellaneous}

%------------------------------------------------------------------------------%

\printbibliography[heading=bibintoc]
	
\end{document}