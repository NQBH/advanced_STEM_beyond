\documentclass[oneside]{book}
\usepackage[backend=biber,natbib=true,style=authoryear]{biblatex}
\addbibresource{/home/nguyen/1_NQBH/reference/bib.bib}
\usepackage[utf8]{inputenc}
\usepackage{graphicx}
\usepackage[colorlinks=true,linkcolor=blue,urlcolor=red,citecolor=magenta]{hyperref}
\usepackage{amsmath,amssymb,amsthm,mathtools}
\allowdisplaybreaks
\usepackage{tcolorbox}
\numberwithin{equation}{section}
\newtheorem{assumption}{Assumption}[section]
\newtheorem{lemma}{Lemma}[section]
\newtheorem{corollary}{Corollary}[section]
\newtheorem{definition}{Definition}[section]
\newtheorem{proposition}{Proposition}[section]
\newtheorem{theorem}{Theorem}[section]
\newtheorem{notation}{Notation}[section]
\newtheorem{remark}{Remark}[section]
\newtheorem{example}{Example}[section]
\newtheorem{question}{Question}[section]
\newtheorem{problem}{Problem}[section]
\newtheorem{conjecture}{Conjecture}[section]
\usepackage[left=0.5in,right=0.5in,top=1.5cm,bottom=1.5cm]{geometry}
\usepackage{fancyhdr}
\pagestyle{fancy}
\fancyhf{}
\addtolength{\headheight}{0pt}% obsolete
\lhead{\scriptsize \chaptername~\thechapter}
\rhead{\scriptsize \nouppercase{\leftmark}} %\nouppercase !
\renewcommand{\chaptermark}[1]{\markboth{#1}{}}
\cfoot{\thepage}
\def\labelitemii{$\circ$}
\usepackage{enumitem}

\title{Formal Computation Report}
\author{Nguyen Quan Ba Hong}
\date{\today}

\begin{document}
\maketitle
\setcounter{secnumdepth}{6}
\setcounter{tocdepth}{6}
\tableofcontents

%------------------------------------------------------------------------------%

\chapter{Materials}

\section{Notation}
For a scalar function $f:\Omega\subset\mathbb{R}^N\to\mathbb{R}$, denote the transpose of its gradient by
\begin{align*}
    \nabla^\top f\coloneqq(\nabla f)^\top = \begin{bmatrix}
        \partial_{x_1}f & \cdots & \partial_{x_N}f
    \end{bmatrix}\in\operatorname{Mat}_\mathbb{R}(1,N).
\end{align*}
For a vector field ${\bf u}:\Omega\subset\mathbb{R}^N\to\mathbb{R}^M$, denote the its Jacobian matrix and its gradient by
\begin{align*}
    D{\bf u} = \begin{bmatrix}
        \partial_{x_1}{\bf u} & \cdots & \partial_{x_N}{\bf u}
    \end{bmatrix} = \begin{bmatrix}
        \nabla^\top u_1\\\vdots\\ \nabla^\top u_m
    \end{bmatrix} &= \begin{bmatrix}
        \partial_{x_1}u_1 & \cdots & \partial_{x_N}u_1\\\vdots & \ddots & \vdots\\ \partial_{x_1}u_M & \cdots & \partial_{x_N}u_M
    \end{bmatrix} = \left(\partial_{x_j}u_i\right)_{i,j=1}^{N,M}\in\operatorname{Mat}_\mathbb{R}(M,N),\\
    \nabla{\bf u} = \begin{bmatrix}
        \nabla u_1 & \cdots & \nabla u_m
    \end{bmatrix} = \begin{bmatrix}
        \partial_{x_1}{\bf u}^\top\\\vdots\\ \partial_{x_N}{\bf u}^\top
    \end{bmatrix} &= \begin{bmatrix}
    \partial_{x_1}u_1 & \cdots & \partial_{x_1}u_M\\\vdots & \ddots & \vdots\\ \partial_{x_N}u_1 & \cdots & \partial_{x_N}u_M
\end{bmatrix} = \left(\partial_{x_i}u_j\right)_{i,j=1}^{N,M}\in\operatorname{Mat}_\mathbb{R}(N,M).
\end{align*}
Note that $\nabla{\bf u} = (D{\bf u})^\top$, or $D{\bf u} = \nabla^\top{\bf u}\coloneqq(\nabla{\bf u})^\top$.

\subsection{Differentiability in Banach spaces}
See \cite[Chap. 2, Sect. 2.6]{Troltzsch2010}.

\begin{definition}[1st variation]
    Let $(U,\|\cdot\|_U)$ and $(V,\|\cdot\|_V)$ be Banach spaces, and $\mathcal{U}$ be a nonempty open subset of $U$, $u\in\mathcal{U}$ and $h\in U$, and $F:\mathcal{U}\subset U\to V$ be given. If the limit
    \begin{align*}
        \delta F(u,h)\coloneqq\lim_{t\downarrow 0} \frac{1}{t}\left(F(u + th) - F(u)\right)
    \end{align*}
    exists in $V$, then it is called the \emph{directional derivative} of $F$ at $u$ in the direction $h$.
    
    If this limit exists for all $h\in U$, then the mapping $h\mapsto\delta F(u,h)$ is termed the \emph{1st variation} of $F$ at $u$.
\end{definition}

\section{Integration by parts formulas}

\subsection{Divergence theorem}

\begin{theorem}[Divergence/Gauss-Green]
    \label{theorem: divergence}
    Given a bounded smooth domain $D$ in $\mathbb{R}^N$ and a Lipschitzian domain $\Omega$ in $D$, then
    \begin{align}
        \label{div}
        \tag{div}
        \forall\boldsymbol{\varphi}\in C^1\left(\overline{D},\mathbb{R}^N\right),\ \int_\Omega \nabla\cdot\boldsymbol{\varphi}{\rm d}x = \int_\Gamma \boldsymbol{\varphi}\cdot{\bf n}{\rm d}\Gamma,
    \end{align}
    where ${\bf n}$ denotes the outward unit normal field.
\end{theorem}

\begin{lemma}[Integration by parts]
    If $\Omega$ is a bounded $C^1$ open set in $\mathbb{R}^N$ with boundary $\Gamma\coloneqq\partial\Omega$ and $v\in C^1(\overline{\Omega})$, $f\in C^1(\overline{\Omega})$, then
    \begin{align}
        \label{ibp}
        \tag{ibp}
        \int_\Omega \nabla f\cdot{\bf v}{\rm d}{\bf x} = -\int_\Omega f\nabla\cdot{\bf v}{\rm d}{\bf x} + \int_\Gamma f{\bf v}\cdot{\bf n}{\rm d}\Gamma.
    \end{align}
\end{lemma}

\begin{proof}
    Use integration by parts formula for each component:
    \begin{align*}
        \int_\Omega \nabla f\cdot{\bf v}{\rm d}{\bf x} &= \int_\Omega \sum_{i=1}^N \partial_{x_i}fv_i{\rm d}{\bf x} = \sum_{i=1}^N \int_\Omega \partial_{x_i}fv_i{\rm d}{\bf x} = \sum_{i=1}^N -\int_\Omega f\partial_{x_i}v_i{\rm d}{\bf x} + \int_\Gamma fv_in_i{\rm d}\Gamma\\
        &= -\int_\Omega f\sum_{i=1}^N \partial_{x_i}v_i{\rm d}{\bf x} + \int_\Gamma f\sum_{i=1}^N v_in_i{\rm d}\Gamma = -\int_\Omega f\nabla\cdot{\bf v}{\rm d}{\bf x} + \int_\Gamma f{\bf v}\cdot{\bf n}{\rm d}\Gamma.
    \end{align*}
\end{proof}

\subsection{Green's identities for scalar functions}

\begin{lemma}[Green's identities for scalar functions]
    If $\Omega$ is a bounded $C^1$ open set in $\mathbb{R}^N$ with boundary $\Gamma\coloneqq\partial\Omega$ and $u,v\in C^2(\overline{\Omega})$, then
    \begin{align}
        \int_\Omega \Delta uv{\rm d}{\bf x} &= -\int_\Omega \nabla u\cdot\nabla v{\rm d}{\bf x} + \int_\Gamma \partial_{\bf n}uv{\rm d}\Gamma,\label{Green's 1st identity}\tag{1st-Green}\\
        \int_\Omega \Delta uv{\rm d}{\bf x} &= \int_\Omega u\Delta v{\rm d}{\bf x} + \int_\Gamma \partial_{\bf n}uv - u\partial_{\bf n}v{\rm d}\Gamma.\label{Green's 2nd identity}\tag{2nd-Green}
    \end{align}
\end{lemma}
About notation, $\Delta uv$ means $(\Delta u)v$, not $\Delta(uv)$; also, $\partial_{\bf n}uv$ means $(\partial_{\bf n}u)v$, not $\partial_{\bf n}(uv)$.

%
Note that Green's 1st identity \eqref{Green's 1st identity} is a special case of the more general identity derived from the divergence theorem \ref{theorem: divergence} by substituting $\boldsymbol{\varphi} = \varphi{\bf f}$ into \eqref{Green's 1st identity}:
\begin{align}
    \label{divergence theorem's consequence}
    \int_\Omega \left(\varphi\nabla\cdot{\bf f} + \nabla\varphi\cdot{\bf f}\right){\rm d}{\bf x} = \int_\Gamma \varphi{\bf f}\cdot{\bf n}{\rm d}\Gamma.
\end{align}

\subsection{Green's identities for vector fields}
\begin{lemma}[Green's identities for vector fields]
    If $\Omega$ is a bounded $C^1$ open set in $\mathbb{R}^N$ with boundary $\Gamma\coloneqq\partial\Omega$ and ${\bf u},{\bf v}\in C^2(\overline{\Omega},\mathbb{R}^N)$, then
    \begin{align}
        \int_\Omega \Delta{\bf u}\cdot{\bf v}{\rm d}{\bf x} &= -\int_\Omega \nabla{\bf u}:\nabla{\bf v}{\rm d}{\bf x} + \int_\Gamma \partial_{\bf n}{\bf u}\cdot{\bf v}{\rm d}\Gamma,\label{Green's 1st identity vector}\tag{1st-Green-vec}\\
        \int_\Omega \Delta{\bf u}\cdot{\bf v}{\rm d}{\bf x} &= \int_\Omega {\bf u}\cdot\Delta{\bf v}{\rm d}{\bf x} + \int_\Gamma \left(\partial_{\bf n}{\bf u}\cdot{\bf v} - \partial_{\bf n}{\bf v}\cdot{\bf u}\right){\rm d}\Gamma.\label{Green's 2nd identity vector}\tag{2nd-Green-vec}
    \end{align}
\end{lemma}

\begin{proof}
    Use Green's identities for scalar functions to obtain the former:
    \begin{align*}
        \int_\Omega \Delta{\bf u}\cdot{\bf v}{\rm d}{\bf x} &= \int_\Omega \sum_{i=1}^N \Delta u_iv_i{\rm d}{\bf x} = \sum_{i=1}^N \int_\Omega \Delta u_iv_i{\rm d}{\bf x} = \sum_{i=1}^N -\int_\Omega \nabla u_i\cdot\nabla v_i{\rm d}{\bf x} + \int_\Gamma \partial_{\bf n}u_iv_i{\rm d}\Gamma\\
        &= -\int_\Omega \sum_{i=1}^N \nabla u_i\cdot\nabla v_i{\rm d}{\bf x} + \int_\Gamma \sum_{i=1}^N \partial_{\bf n}u_iv_i{\rm d}\Gamma = -\int_\Omega \nabla{\bf u}:\nabla{\bf v}{\rm d}{\bf x} + \int_\Gamma \partial_{\bf n}{\bf u}\cdot{\bf v}{\rm d}\Gamma.
    \end{align*}
    The latter is obtained by subtracting the former with itself after interchanging the roles of ${\bf u}$ and ${\bf v}$.
\end{proof}

\subsection{Integration by parts involving matrices/2nd-order tensors}
For a matrix/2nd-order tensor $A = (A_{ij})_{i,j=1}^{M,N} = (A_{i \cdot})_{i=1}^M = ((A_{\cdot j})_{j=1}^N)^\top\in C^1(\overline{\Omega};\mathbb{R}^{M\times N})$ (where $A_{i \cdot}$ and $A_{\cdot j}$ is the $i$th row and $j$th column of $A$, respectively), its divergence is defined by
\begin{align*}
    \nabla\cdot A = \left(\nabla\cdot A_{\cdot j}\right)_{j=1}^N = \left(\sum_{i=1}^M \partial_{x_i}A_{ij}\right)_{j=1}^N.
\end{align*}

\begin{lemma}[Integration by parts involving 2nd-order tensors]
    If $\Omega$ is a bounded $C^1$ open set in $\mathbb{R}^N$ with boundary $\Gamma\coloneqq\partial\Omega$ and $A\in C^1(\overline{\Omega};\mathbb{R}^{N^2})$, ${\bf v}\in C^1(\overline{\Omega},\mathbb{R}^N)$, $f\in C^1(\overline{\Omega})$, then
    \begin{align}
        \int_\Omega \left(\nabla\cdot A\right)\cdot{\bf v}{\rm d}{\bf x} &= -\int_\Omega A:\nabla{\bf v}{\rm d}{\bf x} + \int_\Gamma {\bf n}^\top A{\bf v}{\rm d}\Gamma,\label{integration by parts for matrix 1}\tag{ibp-mat1}\\
        \int_\Omega \nabla\cdot(fA)\cdot{\bf v}{\rm d}{\bf x} &= -\int_\Omega fA:\nabla{\bf v}{\rm d}{\bf x} + \int_\Gamma f{\bf n}^\top A{\bf v}{\rm d}\Gamma,\label{integration by parts for matrix 2}\tag{ibp-mat2}
    \end{align}
    where ${\bf x}^\top A{\bf y}\coloneqq\sum_{i=1}^N\sum_{j=1}^N x_iA_{ij}y_j$ for all ${\bf x},{\bf y}\in\mathbb{R}^N$.
\end{lemma}

\begin{proof}
    Use integration by parts formula to obtain the 1st identity:
    \begin{align*}
        \int_\Omega (\nabla\cdot A)\cdot{\bf v}{\rm d}{\bf x} &= \int_\Omega \left(\sum_{i=1}^N \partial_{x_i}A_{ij}\right)_{j=1}^N\cdot{\bf v}{\rm d}{\bf x} = \int_\Omega \sum_{j=1}^N\sum_{i=1}^N \partial_{x_i}A_{ij}v_j{\rm d}{\bf x} = \sum_{i=1}^N\sum_{j=1}^N \int_\Omega \partial_{x_i}A_{ij}v_j{\rm d}{\bf x}\\
        &= \sum_{i=1}^N\sum_{j=1}^N -\int_\Omega A_{ij}\partial_{x_i}v_j{\rm d}{\bf x} + \int_\Gamma n_iA_{ij}v_j{\rm d}\Gamma = -\int_\Omega \sum_{i=1}^N\sum_{j=1}^N A_{ij}\partial_{x_i}v_j{\rm d}{\bf x} + \int_\Gamma \sum_{i=1}^N\sum_{j=1}^N n_iA_{ij}v_j{\rm d}\Gamma\\
        &= -\int_\Omega A:\nabla{\bf v}{\rm d}{\bf x} + \int_\Gamma {\bf n}^\top A{\bf v}{\rm d}\Gamma,
    \end{align*}
    and the 2nd one is obtained by applying the 1st one for $\widetilde{A}\coloneqq fA$.
\end{proof}

\begin{remark}
    For any $A\in\operatorname{Mat}_N(\mathbb{R})$, the bilinear form
    \begin{align*}
        B_A:\mathbb{R}^N\times\mathbb{R}^N&\to\mathbb{R}\\
        ({\bf x},{\bf y})&\mapsto B_A({\bf x},{\bf y})\coloneqq{\bf x}^\top A{\bf y} = \sum_{i=1}^N\sum_{j=1}^N x_iA_{ij}y_j,
    \end{align*}
    can be rewritten in other representations as follows:
    \begin{align*}
        B_A({\bf x},{\bf y}) = {\bf x}^\top A{\bf y} = (A{\bf y})\cdot{\bf x} = \left(A^\top{\bf x}\right)^\top{\bf y} = \left(A^\top{\bf x}\right)\cdot{\bf y} = ({\bf x}\otimes{\bf y}):A = ({\bf x}{\bf y}^\top):A.
    \end{align*}
\end{remark}

\section{Shape calculus}
\paragraph*{Hadamard semiderivative and velocity method.} The next theorem relates the Hadamard semiderivative and the semiderivative obtained by the velocity method for real functions defined on $\mathbb{R}^N$.

\begin{theorem}
    Let $f:N_X\to\mathbb{R}$ be a real function defined in a neighborhood $N_X$ of a point $X$ in $\mathbb{R}^N$. Then $f$ is Hadamard semidifferentiable at $(X,v)$ iff there exists $\tau > 0$ s.t. for all velocity fields $V:[0,\tau]\to\mathbb{R}$ satisfying the following assumptions:
    \begin{itemize}
        \item[(a)] $(V_1)$ $\forall x\in\mathbb{R}^N$, $V(\cdot,x)\in C([0,\tau];\mathbb{R}^N)$;
        \item[(b)] $(V_2)$ $\exists c > 0$, $\forall x,y\in\mathbb{R}^N$, $\|V(\cdot,y) - V(\cdot,x)\|_{C([0,\tau];\mathbb{R}^N)}\le c|y - x|$;
        \item[(c)] the limit \textbf{(2.24)}
        \begin{align*}
            df(X;v)\coloneqq\lim_{\forall V,\,V(0) = v,\,t\downarrow 0} \frac{f\left(T_t(V)(X)\right) - f(X)}{t}
        \end{align*}
        exists and is independent of the choice of $V$ satisfying (a) and (b), where
        \begin{align*}
            T_t(V)(X) &\coloneqq x(t;X),\\
            \frac{dx}{dt}(t;X) &= V\left(t,x(t;X)\right),\ 0 < t < \tau,\ x(0;X) = X.
        \end{align*}
    \end{itemize}
\end{theorem}

\subsection{1st-order shape semiderivatives and derivatives}

\subsubsection{Eulerian and Hadamard semiderivatives}
We give the definitions in the constrained case under assumptions (5.5) in Chap. 4 that can be specialized to the unconstrained case ($D = \mathbb{R}^N$).\footnote{In the unconstrained case, everything reduces to the conditions (V) on the velocity field given by (4.2) in Chap. 4: there exists $\tau = \tau(V) > 0$ s.t. \textbf{(3.1)}
    \begin{align*}
        (V)\ \forall x\in\mathbb{R}^N,\ V(\cdot,x)\in C([0,\tau];\mathbb{R}^N),\ \exists c > 0,\ \forall x,y\in\mathbb{R}^N,\ \|V(\cdot,y) - V(\cdot,x)\|_{C\left([0,\tau];\mathbb{R}^N\right)}\le c|y - x|.
\end{align*}}

Consider a velocity field \textbf{(3.2)}
\begin{align*}
    V:[0,\tau]\times\overline{D}\to\mathbb{R}^N
\end{align*}
verifying condition $({\rm V1}_D)$ and the equivalent form $({\rm V2}_C)$ of condition $({\rm V2}_D)$\footnote{We replace condition $({\rm V2}_D)$ in (5.5) of Chap. 4
    \begin{align*}
        ({\rm V2}_D)\ \forall x\in\overline{D},\ \forall t\in[0,\tau],\ \pm V(t,x)\in T_{\overline{D}}(x)
    \end{align*}
    by its equivalent form $({\rm V2}_C)$ by using Theorem 5.2 in Chap. 4.} for some $\tau = \tau(V) > 0$: \textbf{(3.3)}
\begin{align*}
    &({\rm V1}_D)\ \forall x\in\overline{D},\ V(\cdot,x)\in C\left([0,\tau];\mathbb{R}^N\right),\ \exists c > 0,\ \forall x,y\in\overline{D},\ \|V(\cdot,y) - V(\cdot,x)\|_{C\left([0,\tau];\mathbb{R}^N\right)}\le c|y - x|,\\
    &({\rm V2}_C)\ \forall t\in[0,\tau],\ \forall x\in\overline{D},\ V(t,x)\in L_D(x) = \{-C_D(x)\}\cap C_D(x),
\end{align*}
where $L_D(x) = C_D(x)\cap\{-C_D(x)\}$ is the \textit{linear tangent space} to $D$ at the point $x\in\partial D$ and $C_D(x)$ is Clarke's tangent cone to $\overline{D}$ at $x$ (Theorem 5.2 in Chap. 4).

Refer to the set of conditions (3.2)-(3.3) as condition $({\rm V}_D)$.

%
Introduce the following linear subspace of $\operatorname{Lip}(D,\mathbb{R}^N)$: \textbf{(3.4)}
\begin{align*}
    \operatorname{Lip}_L(D,\mathbb{R}^N)\coloneqq\left\{\theta\in\operatorname{Lip}(D,\mathbb{R}^N);\forall X\in\partial D,\ \theta(X)\in L_D(X)\right\}.
\end{align*}
Under the action of a velocity field $V$ satisfying conditions (3.2)-(3.3), a set $\Omega$ in $\overline{D}$ is transformed into a new domain, \textbf{(3.5)}
\begin{align*}
    \Omega_t(V)\coloneqq T_t(V)(\Omega) = \left\{T_t(V)(X);\forall X\in\Omega\right\},
\end{align*}
also contained in $\overline{D}$.

The transformation $T_t:\overline{D}\to\overline{D}$ associated with the velocity field $x\mapsto V(t)(x)\coloneqq V(t,x):\overline{D}\to\mathbb{R}^N$ is given by \textbf{(3.6)}
\begin{align*}
    T_t(X)\coloneqq x(t,X),\ t\ge 0,\ X\in\overline{D},
\end{align*}
where $x(\cdot,X)$ is solution of the differential equation \textbf{(3.7)}
\begin{align*}
    \frac{dx}{dt}(t,X) = V\left(t,x(t,X)\right),\ t\ge 0,\ x(0,X) = X.
\end{align*}
Recall Definition 3.1 of Sect. 3 in Chap. 4 of a shape functional.

\begin{definition}
    Given a nonempty subset $D$ of $\mathbb{R}^N$, consider the set $\mathcal{P}(D) = \{\Omega;\Omega\subset D\}$ of subsets of $D$. The set $D$ will be referred to as the underlying \emph{holdall} or \emph{universe}. A \emph{shape functional} is a map \textbf{(3.8)}
    \begin{align*}
        J:\mathcal{A}\to E
    \end{align*}
    from some \emph{admissible family} $\mathcal{A}$ of sets in $\mathcal{P}(D)$ into a topological space $E$.
\end{definition}
E.g., $\mathcal{A}$ could be the set $\mathcal{X}(\Omega) = \{F(\Omega);\forall F\in\mathcal{F}(\Theta)\}$.

\begin{definition}
    Let $J$ be a real-valued shape functional.
    \begin{itemize}
        \item[(i)] Let $V$ be a velocity field satisfying conditions (3.2)-(3.3). $J$ has an \emph{Eulerian semiderivative} at $\Omega$ in the direction $V$ if the limit \textbf{(3.9)}
        \begin{align*}
            \lim_{t\downarrow 0} \frac{J\left(\Omega_t(V)\right) - J(\Omega)}{t}
        \end{align*}
        exists. It will be denoted by $dJ(\Omega;V)$. For simplicity, when $V(t,x) = \theta(x)$, $\theta\in\operatorname{Lip}_L(D,\mathbb{R}^N)$ an autonomous velocity field, we shall write $dJ(\Omega;\theta)$.
        \item[(ii)] Let $\Theta$ be a topological vector subspace of $\operatorname{Lip}_L(D,\mathbb{R}^N)$ satisfying conditions (3.2)-(3.3). $J$ has a \emph{Hadamard semiderivative} at $\Omega$ in the direction $\theta\in\Theta$ if \textbf{(3.10)}
        \begin{align*}
            \lim_{V\in C^0([0,\tau];\Theta),\,V(0) = \theta,\,t\downarrow 0} \frac{J\left(\Omega_t(V)\right) - J(\Omega)}{t}
        \end{align*}
        exists, depends only on $\theta$, and is independent of the choice of $V$ satisfying
        conditions (3.2)-(3.3). It will be denoted by $d_HJ(\Omega;\theta)$. If $d_HJ(\Omega;\theta)$ exists, $J$ has an Eulerian semiderivative at $\Omega$ in the direction $V(t,x) = \theta(x)$ and
        \begin{align*}
            d_HJ(\Omega;\theta) = dJ(\Omega;V(0)) = dJ(\Omega;V).
        \end{align*}
        \item[(iii)] Let $\Theta$ be a topological vector subspace of $\operatorname{Lip}_L(D,\mathbb{R}^N)$ satisfying conditions (3.2)-(3.3). $J$ has a \emph{Hadamard derivative} at $\Omega$ w.r.t. $\Theta$ if it has a Hadamard semiderivative at $\Omega$ in all directions $\theta\in\Theta$ and the map \textbf{(3.11)}
        \begin{align*}
            \theta\mapsto dJ(\Omega;\theta):\Theta\to\mathbb{R}
        \end{align*}
        is linear and continuous. The map (3.11) will be denoted $G(\Omega)$ and referred to as the \emph{gradient} of $J$ in the topological dual $\Theta'$ of $\Theta$.
    \end{itemize}
\end{definition}
The definition of an Eulerian semiderivative is quite general.

E.g., it readily applies to shape functionals defined on closed submanifolds $D$ of $\mathbb{R}^N$.

It includes cases where $dJ(\Omega;V)$ is dependent not only on $V(0)$ but also on $V(t)$ in a neighborhood of $t = 0$.

We shall see that this will not occur under some continuity assumption on the map $V\mapsto dJ(\Omega;V)$.

When $dJ(\Omega;V)$ depends only on $V(0)$, the analysis can be specialized to autonomous vector fields $V$ and the semiderivative can be related to the gradient of $J$.

\begin{example}
    For any measurable subset $\Omega$ of $\mathbb{R}^N$, consider the volume function \textbf{(3.12)}
    \begin{align*}
        J(\Omega) = \int_\Omega {\rm d}x.
    \end{align*}
    For $\Omega$ with finite volume and $V$ in $C([0,\tau];C_0^1(\mathbb{R}^N,\mathbb{R}^N))$, consider the transformations $T_t(\Omega)$ of $\Omega$:
    \begin{align*}
        J\left(T_t(\Omega)\right) = \int_{T_t(\Omega)} {\rm d}x = \int_\Omega \left|\det(DT_t)\right|{\rm d}x = \int_\Omega \det(DT_t){\rm d}x
    \end{align*}
    for $t$ small since $\det(DT_0) = \det I = 1$.
    
    This yields the Eulerian semiderivative \textbf{(3.13)}
    \begin{align*}
        dJ(\Omega;V) = \int_\Omega \nabla\cdot V(0){\rm d}x.
    \end{align*}
    By definition it depends only on $V(0)$ and hence $J$ has a Hadamard semiderivative \textbf{(3.14)}
    \begin{align*}
        d_HJ(\Omega;V(0)) = \int_\Omega \nabla\cdot V(0){\rm d}x
    \end{align*}
    and even a gradient $G(\Omega)$ for $\Theta = C_0^1(\mathbb{R}^N,\mathbb{R}^N)$.
\end{example}
The following simple continuity condition can also be used to obtain the Hadamard semidifferentiability.

\begin{theorem}
    Let $\Theta$ be a Banach subspace of $\operatorname{Lip}_L(D,\mathbb{R}^N)$ satisfying conditions (3.2)-(3.3), $J$ be a real-valued shape functional, and $\Omega$ be a subset of $\mathbb{R}^N$.
    \begin{itemize}
        \item[(i)] Given $\theta\in\Theta$, if \textbf{(3.15)}
        \begin{align*}
            \forall V\in C\left([0,\tau];\Theta\right) \mbox{ s.t. } V(0) = \theta,\ dJ(\Omega;V) \mbox{ exists},
        \end{align*}
        and if the map \textbf{(3.16)}
        \begin{align*}
            V\mapsto dJ(\Omega;V): C\left([0,\tau];\Theta\right)\to\mathbb{R}
        \end{align*}
        is continuous for the subspace of $V$'s s.t. $V(0) = \theta$, then $J$ is Hadamard semidifferentiable at $\Omega$ in the direction $\theta$ w.r.t. $\Theta$ and \textbf{(3.17)}
        \begin{align*}
            \forall V\in C\left([0,\tau];\Theta\right),\ V(0) = \theta,\ dJ(\Omega;V) = d_HJ(\Omega;\theta).
        \end{align*}
        \item[(ii)] If for all $V$ in $C([0,\tau];\Theta)$, $dJ(\Omega;V)$ exists and the map \textbf{(3.18)}
        \begin{align*}
            V\mapsto dJ(\Omega;V):C\left([0,\tau];\Theta\right)\to\mathbb{R}
        \end{align*}
        is continuous, then $J$ is Hadamard semidifferentiable at $\Omega$ in the direction $V(0)$ w.r.t. $\Theta$ and
        \begin{align*}
            \forall V\in C\left([0,\tau];\Theta\right),\ dJ(\Omega;V) = dJ(\Omega;V(0)) = d_HJ(\Omega;V(0)).
        \end{align*}
    \end{itemize}
\end{theorem}

For simplicity, the last theorem was given only for a Banach space $\Theta$.

This includes the spaces $\mathcal{C}_0^k(\mathbb{R}^N)$, $\mathcal{C}^k(\overline{\mathbb{R}^N})$, $\mathcal{C}^{0,1}(\overline{\mathbb{R}^N})$, and $\mathcal{B}^k(\mathbb{R}^N,\mathbb{R}^N)$ considered in Chaps. 3--4.

However, its conclusions are not limited to Banach spaces.

Other constructions can be used as illustrated below in the unconstrained case.

Consider velocity fields in \textbf{(3.20)}
\begin{align*}
    \vec{\mathcal{V}}^{m,k}\coloneqq\lim_{\vec{K}} \left\{V_K^{m,k};\forall K \mbox{ compact in } \mathbb{R}^N\right\},
\end{align*}
where for $m\ge 0$, \textbf{(3.21)}
\begin{align*}
    V_K^{m,k}\coloneqq C^m\left([0,\tau];\mathcal{D}^k(K,\mathbb{R}^N)\cap\operatorname{Lip}(\mathbb{R}^N,\mathbb{R}^N)\right)
\end{align*}
and $\lim\limits_\rightarrow$ denotes the inductive limit set w.r.t. $K$ endowed with its natural inductive limit topology.

For autonomous fields, this construction reduces to \textbf{(3.22)-(3.23)}
\begin{align*}
    \vec{\mathcal{V}}^k\coloneqq\lim_{\vec{K}} \left\{V_K^k;\forall K \mbox{ compact in } \mathbb{R}^N\right\},
\end{align*}
\begin{equation*}
    V_K^k\coloneqq\left\{\begin{split}
        &\mathcal{D}^0(K,\mathbb{R}^N)\cap\operatorname{Lip}(K,\mathbb{R}^N), && k = 0,\\
        &\mathcal{D}^k(K,\mathbb{R}^N), &&1\le k\le\infty.
    \end{split}\right.
\end{equation*}
For $k\ge 1$, $\vec{\mathcal{V}}^k = \mathcal{D}^k(\mathbb{R}^N,\mathbb{R}^N)$.

In all cases the conditions (3.1) of the unconstrained case $D = \mathbb{R}^N$ are verified.

\begin{theorem}
    Let $J$ be a real-valued shape functional, $\Omega$ be a subset of $\mathbb{R}^N$, and $k\ge 0$ be an integer.
    \begin{itemize}
        \item[(i)] Given $\theta\in\vec{\mathcal{V}}^k$, assume that \textbf{(3.24)}
        \begin{align*}
            \forall V\in\vec{\mathcal{V}}^{0,k},\ V(0) = \theta,\ dJ(\Omega;V) \mbox{ exists}
        \end{align*}
        and that the map \textbf{(3.25)}
        \begin{align*}
            V\mapsto dJ(\Omega;V):\vec{\mathcal{V}}^{0,k}\to\mathbb{R}
        \end{align*}
        is continuous for all $V$'s such that $V(0) = \theta$. Then $J$ is Hadamard semidifferentiable in $\Omega$ in the direction $\theta$ w.r.t. $\mathcal{V}_k$ and \textbf{(3.26)}
        \begin{align*}
            \forall V\in\vec{\mathcal{V}}^{0,k},\ V(0) = \theta,\ d_HJ(\Omega;\theta) = dJ(\Omega;V) = dJ(\Omega;V(0)).
        \end{align*}
        \item[(ii)] Assume that for all $V\in\vec{\mathcal{V}}^{0,k}$, $dJ(\Omega;V)$ exists and that the map \textbf{(3.27)}
        \begin{align*}
            V\mapsto dJ(\Omega;V):\vec{\mathcal{V}}^{0,k}\to\mathbb{R}
        \end{align*}
        is continuous. Then $J$ is Hadamard semidifferentiable in $\Omega$ in the direction $V(0)$ w.r.t. $\mathcal{V}^k$ and \textbf{(3.28)}
        \begin{align*}
            \forall V\in\vec{\mathcal{V}}^{0,k},\ d_HJ(\Omega;V(0)) = dJ(\Omega;V) = dJ(\Omega;V(0)).
        \end{align*}
    \end{itemize}
\end{theorem}

\subsubsection{Hadamard semidifferentiability \& Courant metric continuity}
We now relate the Hadamard semidifferentiability of a shape functional with the Courant metric continuity studied in Sect. 6 of Chap. 4.

Recall the generic metric spaces $\mathcal{F}(\Theta)$ of Micheletti with metric $d$ and the quotient group $\mathcal{F}(\Theta)/\mathcal{G}$ with the Courant metric $d_\mathcal{G}$ in (2.31) of Chap. 3.

\begin{theorem}
    Let $\Theta$ be equal to $C_0^{k+1}(\mathbb{R}^N,\mathbb{R}^N)$, $C^{k+1}(\overline{\mathbb{R}^N},\mathbb{R}^N)$, or $C^{k,1}(\overline{\mathbb{R}^N},\mathbb{R}^N)$, $k\ge 0$. Assume that $\mathcal{G} = \mathcal{G}(\Omega)$ for $\Omega$ closed or open and crack-free. If a shape functional $J$ is Hadamard semidifferentiable for all $\theta\in\Theta$, then it is continuous w.r.t. the Courant metric $d_{\mathcal{G}(\Omega)}$.
\end{theorem}

\subsubsection{Perturbations of the Identity \& Gateaux \& Fréchet derivatives}
Assume that $\mathcal{G} = \mathcal{G}(\Omega)$ for $\Omega$ closed or open and crack-free and that there exists a ball $B_\varepsilon$ of radius $\varepsilon > 0$ in $\Theta$ and a constant $c > 0$ s.t. \textbf{(3.29)}
\begin{align*}
    \forall f\in B_\varepsilon,\ \left\|[I + f]^{-1} - I\right\|_\Theta\le c\|f\|_\Theta.
\end{align*}
Except for $C^0(\overline{\mathbb{R}^N},\mathbb{R}^N)$ and $\mathcal{B}^0(\mathbb{R}^N,\mathbb{R}^N)$, this is true in all Banach spaces $\Theta\subset C^{0,1}(\overline{\mathbb{R}^N},\mathbb{R}^N)$ considered in Chaps. 3 and 4 (cf. Theorem 2.14 in Sect. 2.5.2 and Theorem 2.17 in Sect. 2.6.2 of Chap. 3).

Hence the maps
\begin{align*}
    f\mapsto I + f\mapsto[I + f]:B_\varepsilon\subset\Theta\to\mathcal{F}(\Theta)\to\mathcal{F}(\Theta)/\mathcal{G}(\Omega)
\end{align*}
are well-defined and continuous in $f = 0$ since
\begin{align*}
    d_{\mathcal{G}}(I,I + f)\le d(I,I + f)\le\|f\|_\Theta + \left\|[I + f]^{-1} - I\right\|_\Theta\le(1 + c)\|f\|_\Theta.
\end{align*}
For a shape functional $J$, the map
\begin{align*}
    [I + f]\mapsto J_\Omega(f)\coloneqq J\left([I + f](\Omega)\right):\mathcal{F}(\Theta)/\mathcal{G}(\Omega)\to\mathbb{R}
\end{align*}
is well-defined since $J$ is invariant on $\mathcal{G}(\Omega)$ and the map
\begin{align*}
    f\mapsto J_\Omega(f)\coloneqq J\left([I + f](\Omega)\right):\Theta\to\mathbb{R}
\end{align*}
is continuous in $f = 0$ if $J$ is continuous in $\Omega$ for the Courant metric on $\mathcal{F}(\Theta)/\mathcal{G}(\Omega)$.

%
The following definitions are now the extension of the standard definitions of Sect. 2 for a function $J_\Omega(f)$ defined on the ball $B_\varepsilon$ in the normed vector space $\Theta$ to topological vector spaces.

They parallel the ones of Definition 3.2.

\begin{definition}
    Let $J$ be a real-valued shape functional and $\Theta$ be a topological vector subspace of $\operatorname{Lip}(\mathbb{R}^N,\mathbb{R}^N)$. For $f\in\Theta$, denote $[I + f](\Omega)$ by $\Omega_f$.
    \begin{itemize}
        \item[(i)] $J_\Omega$ is said to have a \emph{Gateaux semiderivative} at $f$ in the direction $\theta\in\Theta$ if the following limit exists and is finite: \textbf{(3.30)}
        \begin{align*}
            dJ_\Omega(f;\theta)\coloneqq\lim_{t\downarrow 0} \frac{J\left([I + f + t\theta](\Omega)\right) - J\left([I + f](\Omega)\right)}{t}.
        \end{align*}
        \item[(ii)] $J_\Omega$ is said to be \emph{Gateaux differentiable} at $f$ if it has a \emph{Gateaux semiderivative} at $f$ in all directions $\theta\in\Theta$ and the map \textbf{(3.31)}
        \begin{align*}
            \theta\mapsto dJ_\Omega(f;\theta):\Theta\to\mathbb{R}
        \end{align*}
        is linear and continuous. The map (3.31) is denoted $\nabla J_\Omega(f)$ and referred to as the gradient of $J_\Omega$ in the topological dual $\Theta'$ of $\Theta$.
        \item[(iii)] If, in addition, $\Theta$ is a normed vector space, we say that $J$ is \emph{Fréchet differentiable} at $f$ if $J$ is Gateaux differentiable at $f$ and \textbf{(3.32)}
        \begin{align*}
            \lim_{\|\theta\|_\Theta\to 0} \frac{\left|J\left([I + f + \theta](\Omega)\right) - J\left([I + f](\Omega)\right) - \langle\nabla J_\Omega(f),\theta\rangle_\Theta\right|}{\|\theta\|_\Theta} = 0.
        \end{align*}
    \end{itemize}
\end{definition}
The semiderivatives of $J$ and $J_\Omega$ are related.

\begin{theorem}
    Let $J$ be a real-valued shape functional, let $f\in\Theta$, and let $I + f\in\mathcal{F}(\Theta)$.
    \begin{itemize}
        \item[(i)] Assume that $J_\Omega$ has a Gateaux semiderivative at $f$ in the direction $\theta\in\Theta$; then $J$ has an Eulerian semiderivative at $\Omega_f$ in the direction $V_\theta^f$ and \textbf{(3.33)}
        \begin{align*}
            dJ_\Omega(f;\theta) = dJ(\Omega_f;V_\theta^f),\ V_\theta^f(t)\coloneqq\theta\circ[I + f + t\theta]^{-1}
        \end{align*}
        for $t$ sufficiently small.
        \item[(ii)] If $J$ has a Hadamard semiderivative at $\Omega_f$ in the direction $\theta\circ[I + f]^{-1}$, then $J_\Omega$ has a Gateaux semiderivative at $f$ in the direction $\theta$ and \textbf{(3.34)}
        \begin{align*}
            dJ_\Omega(f;\theta) = d_HJ\left(\Omega_f;\theta\circ[I + f]^{-1}\right).
        \end{align*}
        Conversely, if $J_\Omega$ has a Gateaux semiderivative at $f$ in the direction $\theta\circ[I + f]$, then $J$ has a Hadamard semiderivative at $\Omega_f$ in the direction $\theta$. If either $dJ_\Omega(f;\theta)$ or $d_HJ(\Omega_f;\theta)$ is linear and continuous w.r.t. all $\theta$ in $\Theta$, so is the other and \textbf{(3.35)}
        \begin{align*}
            \forall\theta\in\Theta,\ \langle\nabla J_\Omega(f),\theta\rangle_\Theta = \langle G(\Omega_f),\theta\circ[I + f]^{-1}\rangle_\Theta,\ \langle G(\Omega_f),\theta\rangle_\Theta = \langle\nabla J_\Omega(f),\theta\circ[I + f]\rangle_\Theta.
        \end{align*}
    \end{itemize}
\end{theorem}
We have standard sufficient conditions for the Fréchet differentiability from Theorem 2.3.

\begin{theorem}
    Let $J$ be a real-valued shape functional. Let $\Omega$ be a subset of $\mathbb{R}^N$ and $\Theta$ be $C_0^{k+1}(\mathbb{R}^N,\mathbb{R}^N)$, $C^{k+1}(\overline{\mathbb{R}^N},\mathbb{R}^N)$, $\mathcal{C}^{k,1}(\overline{\mathbb{R}^N})$, or $\mathcal{B}^{k+1}(\mathbb{R}^N,\mathbb{R}^N)$, $k\ge 0$. If $J_\Omega$ is Gateaux differentiable for all $f$ in $B_\varepsilon$ and the map
    \begin{align*}
        f\mapsto\nabla J_\Omega(f):B_\varepsilon\to\Theta'
    \end{align*}
    is continuous in $f = 0$, then $J_\Omega$ is Fréchet differentiable in $f = 0$.
\end{theorem}

\subsubsection{Shape gradient \& structure theorem}
In view of the previous discussion we now specialize to autonomous vector fields $V$ to further study the properties and the structure of $dJ(\Omega;V)$.

For simplicity we also specialize to the unconstrained case in $\mathbb{R}^N$.

The constrained case yields similar results but is technically more involved (cf. M. C. Delfour and J.-P. Zolésio [14] for $D$ open in $\mathbb{R}^N$).

%
The choice of a \textit{shape gradient} depends on the choice of the topological vector subspace $\Theta$ of $\operatorname{Lip}(\mathbb{R}^N,\mathbb{R}^N)$.

We choose to work in the classical framework of the Theory of Distributions (cf. L. Schwartz [3]) with $\Theta = \mathcal{D}(\mathbb{R}^N,\mathbb{R}^N)$, the space of all infinitely differentiable transformations $\theta$ of $\mathbb{R}^N$ with compact support.

For these velocity fields $V$, conditions (3.1) are satisfied.

\begin{definition}
    Let $J$ be a real-valued shape functional. Let $\Omega$ be a subset of $\mathbb{R}^N$.
    \begin{itemize}
        \item[(i)] The function $J$ is said to be \emph{shape differentiable} at $\Omega$ if it is differentiable at $\Omega$ for all $\theta$ in $\mathcal{D}(\mathbb{R}^N,\mathbb{R}^N)$.
        \item[(ii)] The map (3.11) defines a vector distribution $G(\Omega)$ in $\mathcal{D}(\mathbb{R}^N,\mathbb{R}^N)'$, which will be referred to as the \emph{shape gradient} of $J$ at $\Omega$.
        \item[(iii)] When, for some finite $k\ge 0$, $G(\Omega)$ is continuous for the $\mathcal{D}^k(\mathbb{R}^N,\mathbb{R}^N)$-topology, we say that the shape gradient $G(\Omega)$ is of order $k$.
    \end{itemize}
\end{definition}
The next theorem gives additional properties of shape differentiable functions.

\textbf{Notation 3.1.} Associate with a subset $A$ of $\mathbb{R}^N$ and an integer $k\ge 0$ the set
\begin{align*}
    L_A^k\coloneqq\left\{V\in\mathcal{D}^k(\mathbb{R}^N,\mathbb{R}^N);\forall x\in A,\ V(x)\in L_A(x)\right\},
\end{align*}
where $L_A(x) = \{-C_A(x)\}\cap C_A(x)$ and $C_A(x)$ is given by (5.27) in Chap. 4.

\begin{theorem}[Structure theorem]
    Let $J$ be a real-valued shape functional. Assume that $J$ has a shape gradient $G(\Omega)$ for some $\Omega\subset\mathbb{R}^N$ with boundary $\Gamma$.
    \begin{itemize}
        \item[(i)] The support of the shape gradient $G(\Omega)$ is contained in $\Gamma$.
        \item[(ii)] If $\Omega$ is open or closed in $\mathbb{R}^N$ and the shape gradient is of order $k$ for some $k\ge 0$, then there exists $[G(\Omega)]$ in $\left(\mathcal{D}^k/L_\Omega^k\right)'$ s.t. for all $V$ in $\mathcal{D}^k\coloneqq\mathcal{D}^k(\mathbb{R}^N,\mathbb{R}^N)$ \textbf{(3.36)}
        \begin{align*}
            dJ(\Omega;V) = \langle[G(\Omega)],q_LV\rangle_{\mathcal{D}^k/L_\Omega^k},
        \end{align*}
        where $q_L:\mathcal{D}^k\to\mathcal{D}^k/L_\Omega^k$ is the canonical quotient surjection. Moreover \textbf{(3.37)}
        \begin{align*}
            G(\Omega) = (q_L)^*[G(\Omega)],
        \end{align*}
        where $(q_L)^*$ denotes the transpose of the linear map $q_L$.
    \end{itemize}
\end{theorem}

\begin{remark}
    When the boundary $\Gamma$ of $\Omega$ is compact and $J$ is shape differentiable at $\Omega$, the distribution $G(\Omega)$ is of finite order.
    
    Once this is known, the conclusions of Theorem 3.6 (ii) apply with $k$ equal to the order of $G(\Omega)$.
    
    Hence $G(\Omega)$ will belong to a Hilbert space $H^{-s}(\mathbb{R}^N)$ for some $s\ge 0$.
\end{remark}
The quotient space is very much related to a trace on the boundary $\Gamma$, and when the boundary $\Gamma$ is sufficiently smooth we can indeed make that identification.

\begin{corollary}
    Assume that the assumptions of Theorem 3.6 are satisfied for an open domain $\Omega$, that the order of $G(\Omega)$ is $k\ge 0$, and that the boundary $\Gamma$ of $\Omega$ is $C^{k+1}$. Then for all $x$ in $\Gamma$, $L_\Omega(x)$ is an $(N - 1)$-dimensional hyperplane to $\Omega$ at $x$ and there exists a unique outward unit normal ${\bf n}(x)$ which belongs to $C^k(\Gamma;\mathbb{R}^N)$. As a result, the kernel of the map \textbf{(3.38)}
    \begin{align*}
        V\mapsto\gamma_\Gamma(V)\cdot{\bf n}:\mathcal{D}^k(\mathbb{R}^N,\mathbb{R}^N)\to C^k(\Gamma)
    \end{align*}
    coincides with $L_\Omega^k$, where $\gamma_\Gamma:\mathcal{D}^k(\mathbb{R}^N,\mathbb{R}^N)\to C^k(\Gamma,\mathbb{R}^N)$ is the trace of $V$ on $\Gamma$. Moreover, the map $p_L(V)$ \textbf{(3.39)}
    \begin{align*}
        q_L(V)\mapsto p_L\left(q_L(V)\right)\coloneqq\gamma_\Gamma(V)\cdot{\bf n}:\mathcal{D}^k/L_\Omega^k\to C^k(\Gamma)
    \end{align*}
    is a well-defined isomorphism. In particular, there exists a scalar distribution $g(\Gamma)$ in $\mathbb{R}^N$ with support in $\Gamma$ s.t. $g(\Gamma)\in C^k(\Gamma)'$ and for all $V$ in $\mathcal{D}^k(\mathbb{R}^N,\mathbb{R}^N)$ \textbf{(3.40)}
    \begin{align*}
        dJ(\Omega;V) = \langle g(\Gamma),\gamma_\Gamma(V)\cdot{\bf n}\rangle_{C^k(\Gamma)}
    \end{align*}
    and \textbf{(3.41)}
    \begin{align*}
        G(\Omega) = {}^*(q_L)[G(\Omega)],\ [G(\Omega)] = {}^*(p_L)g(\Gamma).
    \end{align*}
    When $g(\Gamma)\in L^1(\Gamma)$ \textbf{(3.42)}
    \begin{align*}
        dJ(\Omega;V) = \int_\Gamma gV\cdot{\bf n}{\rm d}\Gamma \mbox{ and } G = \gamma_\Gamma^*(g{\bf n}),
    \end{align*}
    where $\gamma_\Gamma$ is the trace operator on $\Gamma$.
\end{corollary}

\begin{remark}
    In 1907, J. Hadamard [1] used displacements along the normal to the boundary $\Gamma$ of a $C^\infty$-domain (as in Sect. 3.3.1 of Chap. 4) to compute the derivative of the 1st eigenvalue of the clamped plate.
    
    Theorem 3.6 and its corollary are generalizations to arbitrary shape functionals of that property to open or closed domains with an arbitrary boundary.
    
    The structure theorem for shape functionals on open domains with a $C^{k+1}$-boundary is due to J.-P. Zolésio [12] in 1979 and not to Hadamard even if the formula (3.42) is often called the \emph{Hadamard formula}.
\end{remark}

\begin{example}
    For any measurable subset $\Omega$ of $\mathbb{R}^N$, consider the volume shape function (3.12) of Example 3.1:
    \begin{align*}
        J(\Omega) = \int_\Omega {\rm d}x.
    \end{align*}
    For $\Omega$ with finite volume and $V$ in $\mathcal{D}^1(\mathbb{R}^N,\mathbb{R}^N)$, we have seen in Example 3.1 that \textbf{(3.43)}
    \begin{align*}
        dJ(\Omega;V) = \int_\Omega \nabla\cdot V{\rm d}x = \int_{\mathbb{R}^N} \chi_\Omega\nabla\cdot V{\rm d}x,
    \end{align*}
    and this is formula (3.40) with $k = 1$. For an open domain $\Omega$ with a $C^1$ compact boundary $\Gamma$, \textbf{(3.44)}
    \begin{align*}
        dJ(\Omega;V) = \int_\Gamma V\cdot{\bf n}{\rm d}\Gamma,
    \end{align*}
    which is also continuous w.r.t. $V$ in $\mathcal{D}^0(\mathbb{R}^N,\mathbb{R}^N)$.
    
    Here the smoothness of the boundary decreases the order of the distribution $G(\Omega)$.
    
    This raises the question of the characterization of the family of all subsets $\Omega$ of $\mathbb{R}^N$ for which the map \textbf{(3.45)}
    \begin{align*}
        V\mapsto\int_\Omega \nabla\cdot V{\rm d}x:\mathcal{D}^1(\mathbb{R}^N,\mathbb{R}^N)\to\mathbb{R}
    \end{align*}
    can be continuously extended to $\mathcal{D}^0(\mathbb{R}^N,\mathbb{R}^N)$.
    
    But this is the family of \emph{locally finite perimeter sets}: sets $\Omega$ whose characteristic function belongs to $\operatorname{BV}_{\rm loc}(\mathbb{R}^N)$.
\end{example}

\subsection{Elements of shape calculus}

\subsubsection{Basic formula for domain integrals}
The simplest examples of domain functions are given by \textit{volume integrals} over a bounded open domain $\Omega$ in $\mathbb{R}^N$.

They use a basic formula in connection with the family of transformations $\{T_t;0\le t\le\tau\}$.

Assume that condition (3.1) is satisfied by the velocity field $\{V(t);0\le t\le\tau\}$.

Further assume that $V\in C^0([0,\tau];C_{\rm loc}^1(\mathbb{R}^N,\mathbb{R}^N))$ and that $\tau > 0$ is s.t. the \textit{Jacobian} $J_t$ is strictly positive:
\begin{align*}
    \forall t\in[0,\tau],\ J_t(X)\coloneqq\det DT_t(X) > 0,\ (DT_t)_{ij} = \partial_jT_i, 
\end{align*}
where $DT_t(X)$ is the \textit{Jacobian matrix} of the transformation $T_t = T_t(V)$ associated with the velocity vector field $V$.

Given a function $\varphi$ in $W_{\rm loc}^{1,1}(\mathbb{R}^N)$, consider for $0\le t\le\tau$ the volume integral \textbf{(4.1)}
\begin{align*}
    J\left(\Omega_t(V)\right)\coloneqq\int_{\Omega_t(V)} \varphi{\rm d}x,
\end{align*}
where $\Omega_t(V)\coloneqq T_t(V)(\Omega)$.

By the change of variables formula \textbf{(4.2)}
\begin{align*}
    J\left(\Omega_t(V)\right) = \int_{\Omega_t(V)} \varphi{\rm d}x = \int_\Omega \varphi\circ T_tJ_t{\rm d}x,
\end{align*}
and the following formulae and results are easy to check.

\begin{theorem}
    Let $\varphi$ be a function in $W_{\rm loc}^{1,1}(\mathbb{R}^N)$. Assume that the vector field $V = \{V(t);0\le t\le\tau\}$ satisfies condition $(V)$.
    \begin{itemize}
        \item[(i)] For each $t\in[0,\tau]$ the map
        \begin{align*}
            \varphi\mapsto\varphi\circ T_t:W_{\rm loc}^{1,1}\left(\mathbb{R}^N\right)\to W_{\rm loc}^{1,1}\left(\mathbb{R}^N\right)
        \end{align*}
        and its inverse are both locally Lipschitzian and
        \begin{align*}
            \nabla\left(\varphi\circ T_t\right) = (DT_t)^\top\nabla\varphi\circ T_t.
        \end{align*}
        \item[(ii)] If $V\in C^0([0,\tau];C_{\rm loc}^1(\mathbb{R}^N,\mathbb{R}^N))$, then the map
        \begin{align*}
            t\mapsto\varphi\circ T_t:[0,\tau]\to W_{\rm loc}^{1,1}\left(\mathbb{R}^N\right)
        \end{align*}
        is well-defined and for each $t$ \textbf{(4.3)}
        \begin{align*}
            \frac{d}{dt}\varphi\circ T_t = \left(\nabla\varphi\cdot V(t)\right)\circ T_t\in L_{\rm loc}^1(\mathbb{R}^N).
        \end{align*}
        Hence the function
        \begin{align*}
            t\mapsto\varphi\circ T_t \mbox{ belongs to } C^1\left([0,\tau];L_{\rm loc}^1(\mathbb{R}^N)\right)\cap C^0\left([0,\tau];W_{\rm loc}^{1,1}(\mathbb{R}^N)\right).
        \end{align*}
        \item[(iii)] If $V\in C^0([0,\tau];C_{\rm loc}^1(\mathbb{R}^N,\mathbb{R}^N))$, then the map
        \begin{align*}
            t\mapsto J_t:[0,\tau]\to C_{\rm loc}^0\left(\mathbb{R}^N\right)
        \end{align*}
        is differentiable and \textbf{(4.4)}
        \begin{align*}
            \frac{dJ_t}{dt} = \left[\nabla\cdot V(t)\right]\circ T_tJ_t\in C_{\rm loc}^0(\mathbb{R}^N).
        \end{align*}
        Hence the map $t\mapsto J_t$ belongs to $C^1([0,\tau];C_{\rm loc}^0(\mathbb{R}^N))$.
    \end{itemize}
\end{theorem}
Indeed it is easy to check that
\begin{align*}
    \frac{d}{dt}DT_t(X) &= DV\left(t,T_t(X)\right)DT_t(X),\ DT_0(X) = I,\\
    \frac{d}{dt}\det DT_t(X) &= \operatorname{tr}DV\left(t,T_t(X)\right)\det DT_t(X)\\
    \Rightarrow\frac{d}{dt}\det DT_t(X) &= \nabla\cdot V\left(t,T_t(X)\right)\det DT_t(X),\ \det DT_0(X) = 1,
\end{align*}
and (4.4) follows directly by definition of $J_t(X)$.

%
From (4.2), (4.3), and (4.4)
\begin{align*}
    dJ(\Omega;V) &= \left.\frac{d}{dt}J\left(\Omega_t(V)\right)\right|_{t = 0} = \int_\Omega \nabla\varphi\cdot V(0) + \varphi\nabla\cdot V(0){\rm d}x\\
    \Rightarrow dJ(\Omega;V) &= \int_\Omega \nabla\cdot\left(\varphi V(0)\right){\rm d}x.
\end{align*}
If $\Omega$ has a Lipschitzian boundary, then by Stokes's theorem
\begin{align*}
    dJ(\Omega;V) = \int_\Gamma \varphi V(0)\cdot{\bf n}{\rm d}\Gamma.
\end{align*}

\begin{theorem}
    Assume that there exists $\tau > 0$ s.t. the velocity field $V(t)$ satisfies conditions $(V)$ and $V\in C^0([0,\tau];C_{\rm loc}^1(\mathbb{R}^N,\mathbb{R}^N))$. Given a function $\varphi\in C(0,\tau;W_{\rm loc}^{1,1}(\mathbb{R}^N))\cap C^1(0,\tau;L_{\rm loc}^1(\mathbb{R}^N))$ and a bounded measurable domain $\Omega$ with boundary $\Gamma$, the semiderivative of the function \textbf{(4.5)}
    \begin{align*}
        \boxed{J_V(t)\coloneqq\int_{\Omega_t(V)} \varphi(t){\rm d}x}
    \end{align*}
    at $t = 0$ is given by \textbf{(4.6)}
    \begin{align*}
        \boxed{dJ_V(0) = \int_\Omega \varphi'(0) + \nabla\cdot\left(\varphi(0)V(0)\right){\rm d}x,}
    \end{align*}
    where $\varphi(0)(x)\coloneqq\varphi(0,x)$ and $\varphi'(0)(x)\coloneqq\partial_t\varphi(0,x)$. If, in addition, $\Omega$ is an open domain with a Lipschitzian boundary $\Gamma$, then \textbf{(4.7)}
    \begin{align*}
        \boxed{dJ_V(0) = \int_\Omega \varphi'(0){\rm d}x + \int_\Gamma \varphi(0)V(0)\cdot{\bf n}{\rm d}\Gamma.}
    \end{align*}
\end{theorem}

\subsubsection{Basic formula for boundary integrals}
Given $\psi$ in $H_{\rm loc}^2(\mathbb{R}^N)$, consider for some bounded open Lipschitzian domain $\Omega$ in $\mathbb{R}^N$ the shape functional \textbf{(4.8)}
\begin{align*}
    J(\Omega)\coloneqq\int_\Gamma \psi{\rm d}\Gamma.
\end{align*}
This integral is invariant w.r.t. a homeomorphism which maps $\Omega$ onto itself (and hence $\Gamma$ onto itself).

Given the velocity field $V$ and $t\ge 0$, consider the expression
\begin{align*}
    J\left(\Omega_t(V)\right)\coloneqq\int_{\Gamma_t(V)} \psi{\rm d}\Gamma_t.
\end{align*}
Using the change of variables $T_t(V)$ and the material introduced in (3.13) of Sect. 3.2 and (5.30) of Sect. 5 in Chap. 2, this integral can be brought back from $\Gamma_t$ to $\Gamma$: \textbf{(4.9)}
\begin{align*}
    J\left(\Omega_t(V)\right)\coloneqq\int_{\Gamma_t} \psi{\rm d}\Gamma_t = \int_\Gamma \psi\circ T_t\omega_t{\rm d}\Gamma,
\end{align*}
where the density $\omega_t$ is given as \textbf{(4.10)}
\begin{align*}
    \omega_t = \left|M\left(DT_t\right){\bf n}\right|,
\end{align*}
${\bf n}$ is the outward normal field on $\Gamma$, and $M(DT_t)$ is the cofactor matrix of $DT_t$, i.e. \textbf{(4.11)}
\begin{align*}
    M\left(DT_t\right) = J_t\left(DT_t\right)^{-\top}\Rightarrow\omega_t = J_t|\left(DT_t\right)^{-\top}{\bf n}|.
\end{align*}
It can easily be checked from (4.10) and (4.11) that $t\mapsto\omega_t$ is differentiable in $C^0(\Gamma)$ and that the limit \textbf{(4.12)}
\begin{align*}
    \omega' = \lim_{t\downarrow 0} \frac{1}{t}\left(\omega_t - \omega\right) = \nabla\cdot V(0) - DV(0){\bf n}\cdot{\bf n}
\end{align*}
in the $C^0(\Gamma)$-norm is linear and continuous w.r.t. $V(0)$ in the $C_{\rm loc}^1(\mathbb{R}^N,\mathbb{R}^N)$ Fréchet topology.

Hence \textbf{(4.13)}
\begin{align*}
    dJ(\Omega;V) = \int_\Gamma \nabla\psi\cdot V(0) + \psi\left(\nabla\cdot V(0) - DV(0){\bf n}\cdot{\bf n}\right){\rm d}\Gamma.
\end{align*}
From Corollary 1 to the structure Theorem 3.6, $dJ(\Omega;V)$ depends only on the normal component ${\bf v}\cdot{\bf n}$ of the velocity field $V(0)$ on $\Gamma$: \textbf{(4.14)}
\begin{align*}
    {\bf v}_{\bf n}\coloneqq{\bf v}\cdot{\bf n},\ {\bf v}\coloneqq V(0)|_\Gamma
\end{align*}
through Hadamard's formula (3.40).

In view of this property any other velocity field with the same smoothness and normal component on $\Gamma$ will yield the same limit.

Given $k > 0$ consider the \textit{tubular neighborhood} \textbf{(4.15)}
\begin{align*}
    S_k(\Gamma)\coloneqq\left\{x\in\mathbb{R}^N;\left|b(x)\right| < k\right\}
\end{align*}
of $\Gamma$ in $\mathbb{R}^N$ for the oriented distance function $b = b_\Omega$ associated with $\Omega$.

Assuming that $\Gamma$ is compact and of class $C^2$, there exists $h > 0$ s.t. $b\in C^2(S_{2h}(\Gamma))$.

Let $\varphi\in\mathcal{D}(\mathbb{R}^N)$ be s.t. $\varphi = 1$ in $S_h(\Gamma)$ and $\varphi = 0$ outside of $S_{2h}(\Gamma)$.

Consider the velocity field
\begin{align*}
    W(t)\coloneqq(V(0)\cdot\nabla b)\nabla b\varphi.
\end{align*}
Clearly, the normal component of $W(0)$ on $\Gamma$ coincides with ${\bf v}_{\bf n}$.

Moreover, in $S_h(\Gamma)$
\begin{align*}
    \nabla\psi\cdot W &= \nabla\psi\cdot\nabla bV(0)\cdot\nabla b\Rightarrow\nabla\psi\cdot W|_\Gamma = \nabla\psi\cdot{\bf n}V(0)\cdot{\bf n} = \partial_{\bf n}\psi{\bf v}_{\bf n},\\
    DW &= V(0)\cdot\nabla bD^2b + \nabla b(\nabla(V(0)\cdot\nabla b))^\top,\\
    \nabla\cdot W &= V(0)\cdot\nabla b\Delta b + \nabla b\cdot\nabla(V(0)\cdot\nabla b),\\
    DW\nabla b\cdot\nabla b &= V(0)\cdot\nabla bD^2b\nabla b\cdot\nabla b + \nabla(V(0)\cdot\nabla b)\cdot\nabla b = \nabla(V(0)\cdot\nabla b)\cdot\nabla b,\\
    \nabla\cdot W - DW\nabla b\cdot\nabla b &= V(0)\cdot\nabla b\Delta b\\
    \Rightarrow\nabla\cdot W - DW\nabla b\cdot\nabla b|_\Gamma &= V(0)\cdot{\bf n}H = H{\bf v}_{\bf n}
\end{align*}
since $\nabla b|_\Gamma = {\bf n}$, $D^2b\nabla b = 0$, and $H = \Delta b$ is the \textit{additive curvature}, i.e., the sum of the $N - 1$ curvatures of $\Gamma$ or $N - 1$ times the mean curvature $\overline{H}$.

Finally, \textbf{(4.16)}
\begin{align*}
    dJ(\Omega;V) = \int_\Gamma \left(\partial_{\bf n}\psi + \psi H\right){\bf v}_{\bf n}{\rm d}\Gamma.
\end{align*}

\begin{theorem}
    Let $\Gamma$ be the boundary of a bounded open subset $\Omega$ of $\mathbb{R}^N$ of class $C^2$ and $\psi$ be an element of $C^1([0,\tau];H_{\rm loc}^2(\mathbb{R}^N))$. Assume that $V\in C^0([0,\tau];C_{\rm loc}^1(\mathbb{R}^N,\mathbb{R}^N))$. Consider the function
    \begin{align*}
        \boxed{J_V(t)\coloneqq\int_{\Gamma_t(V)} \psi(t){\rm d}\Gamma_t.}
    \end{align*}
    Then the derivative of $J_V(t)$ w.r.t. $t$ in $t = 0$ is given by the expression \textbf{(4.17)}
    \begin{align*}
        \boxed{dJ_V(0) = \int_\Gamma \psi'(0) + \left(\partial_{\bf n}\psi + H\psi\right)V(0)\cdot{\bf n}{\rm d}\Gamma = \int_\Gamma \psi'(0) + \nabla\psi\cdot V(0) + \psi\left(\nabla\cdot V(0) - DV(0){\bf n}\cdot{\bf n}\right){\rm d}\Gamma,}
    \end{align*}
    where $\psi'(0)(x)\coloneqq\partial_t\psi(0,x)$.
\end{theorem}
Note that, as in the case of the volume integral, we have 2 formulae.

Hence we have the following identity: \textbf{(4.18)}
\begin{align*}
    \int_\Gamma \left(\partial_{\bf n}\psi + H\psi\right)V(0)\cdot{\bf n}{\rm d}\Gamma = \int_\Gamma \nabla\psi\cdot V(0) + \psi\left(\nabla\cdot V(0) - DV(0){\bf n}\cdot{\bf n}\right){\rm d}\Gamma.
\end{align*}

\subsubsection{Examples of shape derivatives}

\paragraph{Volume of $\Omega$.} Consider the volume shape functional
\begin{align*}
    J(\Omega) = \int_\Omega {\rm d}x.
\end{align*}
This shape functional is used as a constraint on the domain in several examples of shape optimization problems.

We get \textbf{(4.19)}
\begin{align*}
    dJ(\Omega;V) = \int_\Omega \nabla\cdot V(0){\rm d}x,
\end{align*}
and, if $\Gamma$ is Lipschitzian, \textbf{(4.20)}
\begin{align*}
    dJ(\Omega;V) = \int_\Gamma V(0)\cdot{\bf n}{\rm d}\Gamma.
\end{align*}
A sufficient condition on the field $V(0)$ to preserve the volume is $\nabla\cdot V(0) = 0$ in $\Omega$ and, if $\Gamma$ is Lipschitzian, $V(0)\cdot{\bf n} = 0$ on $\Gamma$.

\paragraph{Surface area of $\Gamma$.} Consider the (shape) area function
\begin{align*}
    J(\Omega) = \int_\Gamma {\rm d}\Gamma.
\end{align*}
Assuming that $\Gamma$ is of class $C^2$ we get from (4.17) \textbf{(4.21)}
\begin{align*}
    dJ(\Omega;V) = \int_\Gamma HV(0)\cdot{\bf n}{\rm d}\Gamma,
\end{align*}
where $H = \Delta b_\Omega$ is the additive curvature.

The condition to keep the surface of $\Gamma$ constant is that $V(0)\cdot{\bf n}$ be orthogonal (in $L^2(\Gamma)$) to $H$.

\paragraph{$H^1(\Omega)$-norm.} Given $\phi$ and $\psi$ in $H_{\rm loc}^2(\mathbb{R}^N)$, consider the shape functional
\begin{align*}
    J(\Omega) = \int_\Omega \nabla\phi\cdot\nabla\psi{\rm d}x.
\end{align*}
By using the change of variables $T_t(V)$, $\Omega_t = \Omega_t(V) = T_t(V)(\Omega)$ and \textbf{(4.22)}
\begin{align*}
    \int_{\Omega_t} \nabla\phi\cdot\nabla\psi{\rm d}x = \int_\Omega \left[A(V)(t)\nabla(\phi\circ T_t)\right]\cdot\nabla\left(\psi\circ T_t\right){\rm d}x,
\end{align*}
where $A(V)$ is the following matrix associated with the field $V$: \textbf{(4.23)}
\begin{align*}
    A(V)(t) = J(t)(DT_t)^{-1}(DT_t)^{-\top}
\end{align*}
and $J(t) = \det(DT_t)$.

Expression (4.23) is easily obtained from the identity \textbf{(4.24)}
\begin{align*}
    (\nabla\phi)\circ T_t = (DT_t)^{-\top}\nabla(\phi\circ T_t).
\end{align*}
If $V\in C^0([0,\tau];C_{\rm loc}^k(\mathbb{R}^N,\mathbb{R}^N))$, $k\ge 1$, $T$ and $T^{-1}$ belong to $C^1([0,\tau];C_{\rm loc}^k(\mathbb{R}^N,\mathbb{R}^N))$ and $A(V)$ to $C^1([0,\tau],C_{\rm loc}^{k-1}(\mathbb{R}^N,\mathbb{R}^{N^2}))$.

Then if $\phi$, $\psi$ belongs to $H_{\rm loc}^2(\mathbb{R}^N)$, we get $t\mapsto\phi\circ T_t$, which is differentiable in $H_{\rm loc}^1(\mathbb{R}^N)$, with $\partial_t\phi\circ T_t|_{t = 0} = \nabla\phi\cdot V(0)$, which belongs to $H_{\rm loc}^1(\mathbb{R}^N)$.

Finally, we obtain \textbf{(4.25)}
\begin{align*}
    dJ(\Omega;V) = \int_\Omega [A'(V)\nabla \phi]\cdot\nabla\psi{\rm d}x + \int_\Omega \left[\nabla(\nabla\phi\cdot V(0))\cdot\nabla\psi + \nabla\phi\cdot\nabla(\nabla\psi\cdot V(0))\right]{\rm d}x,
\end{align*}
where $A'(V)$ is the derivative in the $C_{\rm loc}^{k-1}(\mathbb{R}^N,\mathbb{R}^{N^2})$-norm \textbf{(4.26)}
\begin{align*}
    A'(V)\coloneqq\frac{\partial}{\partial t}A(V)(t)|_{t = 0} = \nabla\cdot V(0)I - 2\varepsilon\left(V(0)\right)
\end{align*}
and $\varepsilon(V(0))$ is the symmetrized Jacobian matrix (the strain tensor associated with the field $V(0)$ in elasticity) \textbf{(4.27)}
\begin{align*}
    \varepsilon\left(V(0)\right) = \frac{1}{2}\left(DV(0)^\top + DV(0)\right).
\end{align*}
We finally obtain the \textit{volume expression} \textbf{(4.28)}
\begin{align*}
    dJ(\Omega;V) = \int_\Omega \left[\nabla\cdot V(0)I - 2\varepsilon\left(V(0)\right)\right]\nabla\phi\cdot\nabla\psi + \left[\nabla(\nabla\phi\cdot V(0))\cdot\nabla\psi + \nabla\phi\cdot\nabla(\nabla\psi\cdot V(0))\right]{\rm d}x.
\end{align*}
When $\Gamma$ is a $C^1$-submanifold, $\phi,\psi\in H_{\rm loc}^2(\mathbb{R}^N)$, we obtain the simpler \textit{boundary expression} by directly using formula (4.17): \textbf{(4.29)}
\begin{align*}
    dJ(\Omega;V) = \int_\Gamma \nabla\phi\cdot\nabla\psi V(0)\cdot{\bf n}{\rm d}\Gamma.
\end{align*}
This simple example nicely illustrates the notion of \textbf{density gradient} $g$.

From expression (4.28) it was obvious that the mapping $V\mapsto dJ(\Omega;V)$ was well-defined, linear, and continuous on $C^1([0,\tau];C_{\rm loc}^1(\mathbb{R}^N,\mathbb{R}^N))$.

By the structure theorem and Hadamard's formula we knew that $dJ$ could be written in the form
\begin{align*}
    \int_\Gamma gV(0)\cdot{\bf n}{\rm d}\Gamma.
\end{align*}
But from (4.29) we know that $g$, which is an element of $\mathcal{D}^1(\Gamma)'$, is an element of $W^{1/2,1}(\Gamma)$ given by \textbf{(4.30)}
\begin{align*}
    g = \nabla\phi\cdot\nabla\psi \mbox{ (traces on }\Gamma).
\end{align*}
The direct calculation of g from expression (4.24) would have been very fastidious.

\paragraph{Normal Derivative.} Let $\Gamma$ be of class $C^2$ and $\phi\in H_{\rm loc}^2(\mathbb{R}^N)$ be given.

Consider the following shape function:
\begin{align*}
    J(\Omega)\coloneqq\int_\Gamma \left|\partial_{\bf n}\phi\right|^2{\rm d}\Gamma = \int_\Gamma \left|\nabla\phi\cdot{\bf n}\right|^2{\rm d}\Gamma.
\end{align*}
By the change of variables formula we get, with $\Omega_t = T_t(V)(\Omega)$ and $\Gamma_t = T_t(V)(\Gamma)$, \textbf{(4.31)}
\begin{align*}
    J(\Omega_t)\coloneqq\int_{\Gamma_t} \left|\nabla\phi\cdot{\bf n}_t\right|^2{\rm d}\Gamma_t = \int_\Gamma \left[(DT_t)^{-\top}\nabla(\phi\circ T_t)\cdot({\bf n}_t\circ T_t)\right]^2\omega_t{\rm d}\Gamma,
\end{align*}
where ${\bf n}_t\circ T_t$ is the transported normal field ${\bf n}_t$ from $\Gamma_t$ onto $\Gamma$.

The derivative can be obtained by using formula (4.17) of Theorem 4.3 and 1 of the above 2 expressions.

However, the 1st expression first requires the construction of an extension ${\bf N}_t$ of the normal ${\bf n}_t$ in a neighborhood of $\Gamma$.

In both cases the following result will be useful.

\begin{theorem}
    Let $k\ge 1$ be an integer. Given a velocity field $V(t)$ satisfying condition $(V)$ s.t. $V\in C([0,\tau];C_{\rm loc}^k(\mathbb{R}^N,\mathbb{R}^N))$, then \textbf{(4.32)}
    \begin{align*}
        {\bf n}_t\circ T_t = \frac{(DT_t)^{-\top}{\bf n}}{\left|(DT_t)^{-\top}{\bf n}\right|} = \frac{M(DT_t){\bf n}}{\left|M(DT_t){\bf n}\right|},
    \end{align*}
    where ${\bf n}$ and ${\bf n}_t$ are the respective outward normals to $\Omega$ and $\Omega_t$ on $\Gamma$ and $\Gamma_t$ and $M(DT_t)$ is the cofactor's matrix of $DT_t$.
\end{theorem}
Recalling the expression for $\omega_t$
\begin{align*}
    \omega_t = \left|M(DT_t){\bf n}\right| = J(t)\left|(DT_t)^{-\top}{\bf n}\right|,
\end{align*}
our boundary integral becomes \textbf{(4.34)}
\begin{align*}
    \int_{\Gamma_t} \left|\nabla\phi\cdot{\bf n}_t\right|^2{\rm d}\Gamma_t = \int_\Gamma \left|[A(t)\nabla(\phi\circ T_t)]\cdot{\bf n}\right|^2\omega_t^{-1}{\rm d}\Gamma.
\end{align*}
Using expression (4.26) of $A'(V)$ and expression (4.12) of $\omega'$ we get
\begin{align*}
    dJ(\Omega;V) &= 2\int_\Gamma \partial_{\bf n}\phi\left[A'(V)\nabla\phi\cdot{\bf n} + \nabla\left(\nabla\phi\cdot V(0)\right)\cdot{\bf n}\right] - \left|\partial_{\bf n}\phi\right|^2\omega'{\rm d}\Gamma\\
    &= 2\int_\Gamma \partial_{\bf n}\phi\left\{\left[\nabla\cdot V(0)I - DV(0) - DV(0)^\top\right]\nabla\phi\cdot{\bf n} + \nabla\left(\nabla\phi\cdot V(0)\right)\cdot{\bf n}\right\} - \left|\partial_{\bf n}\phi\right|^2\left(\nabla\cdot V(0) - DV(0){\bf n}\cdot{\bf n}\right){\rm d}\Gamma\\
    &= \int_\Gamma 2\partial_{\bf n}\phi\left(\nabla\cdot V(0)\partial_{\bf n}\phi - DV(0)\nabla\phi\cdot{\bf n} + D^2\phi V(0)\cdot{\bf n}\right) - \left|\partial_{\bf n}\phi\right|^2\left(\nabla\cdot V(0) - DV(0){\bf n}\cdot{\bf n}\right){\rm d}\Gamma\\
    &= \int_\Gamma \left|\partial_{\bf n}\phi\right|^2\left(\nabla\cdot V(0) - DV(0){\bf n}\cdot{\bf n}\right) + 2\partial_{\bf n}\phi\left(DV(0){\bf n}\cdot{\bf n}\partial_{\bf n}\phi - DV(0)\nabla\phi\cdot{\bf n} + D^2\phi V(0)\cdot{\bf n}\right){\rm d}\Gamma.
\end{align*}
This formula can be somewhat simplified by using identity (4.18) with
\begin{align*}
    \psi &= \left|\nabla\phi\cdot\nabla b\right|^2\Rightarrow\psi|_\Gamma = \left|\partial_{\bf n}\phi\right|^2,\\
    \nabla\psi &= 2\nabla\phi\cdot\nabla b\nabla(\nabla\phi\cdot\nabla b) = 2\nabla\phi\cdot\nabla b\left[D^2\phi\nabla b + D^2b\nabla\psi\right]\\
    \Rightarrow\nabla\psi|_\Gamma &= 2\partial_{\bf n}\phi\left[D^2\phi{\bf n} + D^2b\nabla\phi\right]\\
    \Rightarrow\nabla\psi\cdot V(0)|_\Gamma &= 2\partial_{\bf n}\phi\left[D^2\phi{\bf n} + D^2b\nabla\phi\right]\cdot V(0)\\
    \Rightarrow\partial_{\bf n}\psi &= \nabla\psi\cdot\nabla b|_\Gamma = 2\partial_{\bf n}\phi\left[D^2\phi{\bf n} + D^2b\nabla\phi\right]\cdot\nabla b = 2\partial_{\bf n}\phi D^2\phi{\bf n}\cdot{\bf n}.
\end{align*}
We obtain \textbf{(4.35)}
\begin{align*}
    \int_\Gamma \left(2\partial_{\bf n}\phi D^2\phi{\bf n}\cdot{\bf n} + H\left|\partial_{\bf n}\phi\right|^2\right)V(0)\cdot{\bf n}{\rm d}\Gamma = \int_\Gamma 2\partial_{\bf n}\phi\left[D^2\phi{\bf n} + D^2b\nabla\phi\right]\cdot V(0) + \left|\partial_{\bf n}\phi\right|^2\left(\nabla\cdot V(0) - DV(0){\bf n}\cdot{\bf n}\right){\rm d}\Gamma,
\end{align*}
and hence \textbf{(4.36)}
\begin{align*}
    dJ(\Omega;V) = \int_\Gamma \left(2\partial_{\bf n}\phi D^2\phi{\bf n}\cdot{\bf n} + H\left|\partial_{\bf n}\phi\right|^2\right)V(0)\cdot{\bf n} + 2\partial_{\bf n}\phi\left(\partial_{\bf n}\phi DV(0){\bf n}\cdot{\bf n} - \nabla\phi\cdot\left(DV(0)^\top{\bf n} + D^2bV(0)\right)\right){\rm d}\Gamma.
\end{align*}
This formula can be more readily obtained from the first expression (4.31) and the extension \textbf{(4.37)}
\begin{align*}
    {\bf N}_t = \frac{(DT_t)^{-\top}\nabla b}{\left|(DT_t)^{-\top}\nabla b\right|}\circ T_t^{-1}
\end{align*}
of the normal ${\bf n}_t$.

To compute ${\bf N}'$, decompose ${\bf N}_t$ as follows:
\begin{align*}
    {\bf N}_t &= f(t)\circ T_t^{-1},\ f(t) = \frac{g(t)}{\sqrt{g(t)\cdot g(t)}},\ g(t) = (DT_t)^{-\top}\nabla b,\\
    {\bf N}' &= f' - Df(0)V(0),\ g' = -DV(0)^\top\nabla b,\ f(0) = \nabla b,\\
    f' &= \frac{g'|g(0)| - g(0)\cdot g'\frac{g(0)}{|g(0)|}}{|g(0)|^2} = g' - g'\cdot\nabla b\nabla b = DV(0)\nabla b\cdot\nabla b\nabla b - DV(0)^\top\nabla b.
\end{align*}
So finally \textbf{(4.38)}
\begin{align*}
    {\bf N}'|_\Gamma = (DV(0){\bf n}\cdot{\bf n}){\bf n} - DV(0)^\top{\bf n} - D^2bV(0)
\end{align*}
and
\begin{align*}
    \left.\frac{\partial}{\partial t}\left|\nabla\phi\cdot{\bf N}_t\right|^2\right|_{t = 0} = 2\partial_{\bf n}\phi\nabla\phi\cdot{\bf N}' = 2\partial_{\bf n}\phi\nabla\phi\cdot\left[(DV(0){\bf n}\cdot{\bf n}){\bf n} - DV(0)^\top{\bf n} - D^2bV(0)\right].
\end{align*}
The 1st part of the integral (4.36) depends explicitly on the normal component of $V(0)$.

Yet we know from the structure theorem that for this function, the shape derivative depends only on the normal component of $V(0)$.

To make this explicit, it is necessary to introduce some elements of tangential calculus.

\subsection{Elements of tangential calculus}
Let $\Omega$ be an open domain of class $C^2$ in $\mathbb{R}^N$ with compact boundary $\Gamma$.

Therefore, there exists $h > 0$ s.t. $b = b_\Omega\in C^2(S_{2h}(\Gamma))$.

The \textit{projection} of a point $x$ onto $\Gamma$ is given by
\begin{align*}
    p(x)\coloneqq x - b(x)\nabla b(x),
\end{align*}
and the \textit{orthogonal projection operator} of a vector onto the tangent plane $T_{p(x)}\Gamma$ is given by
\begin{align*}
    P(x)\coloneqq I - \nabla b(x)\nabla b(x)^\top.
\end{align*}
Notice that, as a transformation of $T_{p(x)}\Gamma$,
\begin{align*}
    P(x):T_{p(x)}\Gamma\to T_{p(x)}\Gamma,\ P(x) = I - \nabla b(x)\nabla b(x)^\top
\end{align*}
is the identity transformation on $T_{p(x)}\Gamma$.

In fact $P(x)$ coincides with the \textit{1st fundamental form}.

Similarly $D^2b$ can be considered as a transformation of $T_{p(x)}$ since $D^2b(x){\bf n}(x) = D^2b(x)\nabla b(x) = 0$.

We shall show in Sect. 5.6 that
\begin{align*}
    D^2b(x):T_{p(x)}\to T_{p(x)}
\end{align*}
coincides with the 2nd fundamental form of $\Gamma$.

Similarly $D^2b(x)^2$ coincides with the \textit{3rd fundamental form}.

Finally
\begin{align*}
    Dp(x) = I - \nabla b(x)\nabla b(x)^\top - bD^2b(x) \mbox{ and } Dp|_\Gamma = P.
\end{align*}

\subsubsection{Intrinsic definition of the tangential gradient}
The classical way to define the tangential gradient of a scalar function $f:\Gamma\to\mathbb{R}$ is through an appropriately smooth extension $F$ of $f$ in a neighborhood of $\Gamma$ using the fact that the resulting expression on $\Gamma$ is independent of the choice of the extension $F$.

In this section an equivalent direct intrinsic  definition is given in terms of the extension $f\circ p$ of the function $f$.

This is the basis of a simple differential calculus on $\Gamma$ which uses the Euclidean differential calculus in the ambient neighborhood of $\Gamma$.

%
Given $f\in C^1(\Gamma)$, let $F\in C^1(S_{2h}(\Gamma))$ be a $C^1$-extension of $f$.

Define
\begin{align*}
    g(F)\coloneqq\nabla F|_\Gamma - \partial_{\bf n}F{\bf n} \mbox{ on } \Gamma.
\end{align*}
This is the orthogonal projection $P(x)\nabla F(x)$ of $\nabla F(x)$ onto the tangent plane $T_x\Gamma$ to $\Gamma$ at $x$.

To get something intrinsic, $g(F)$ must be independent of the choice of $F$.

It is sufficient to show that $g(F) = 0$ for $f = 0$.

But $F = f = 0$ on $\Gamma$ and the tangential component of $\nabla F$ is 0 on $\Gamma$, and
\begin{align*}
    \nabla F|_\Gamma = \partial_{\bf n}F{\bf n}\Rightarrow g(F) = \nabla F|_\Gamma - \partial_{\bf n}F{\bf n} = 0.
\end{align*}

\begin{definition}[General extension]
    Assume that $\Gamma$ is compact and that there exists $h > 0$ s.t. $b_\Omega\in C^2(S_{2h}(\Gamma))$. Given an extension $F\in C^1(S_{2h}(\Gamma))$ of $f\in C^1(\Gamma)$, the \emph{tangential gradient} of $f$ in a point of $\Gamma$ is defined as
    \begin{align*}
        \nabla_\Gamma f\coloneqq\nabla F|_\Gamma - \partial_{\bf n}F{\bf n}.
    \end{align*}
\end{definition}
The notation is quite natural.

The subscript $\Gamma$ of $\nabla_\Gamma f$ indicates that the gradient is w.r.t. the variable $x$ in the submanifold $\Gamma$.

\begin{theorem}
    Assume that $\Gamma$ is compact and that there exists $h > 0$ s.t. $b_\Omega\in C^2(S_{2h}(\Gamma))$ and that $f\in C^1(\Gamma)$. Then
    \begin{itemize}
        \item[(i)] $\nabla_\Gamma f = (P\nabla F)|_\Gamma$ and ${\bf n}\cdot\nabla_\Gamma f = \nabla b\cdot\nabla_\Gamma f = 0$;
        \item[(ii)] $\nabla(f\circ p) = [I - bD^2b]\nabla_\Gamma f\circ p$ and $\nabla(f\circ p)|_\Gamma = \nabla_\Gamma f$.
    \end{itemize}
\end{theorem}

In view of part (ii) of the theorem $f\circ p$ plays the role of a \textit{canonical extension} of a map $f:\Gamma\to\mathbb{R}$ to a neighborhood $S_{2h}(\Gamma)$ of $\Gamma$ and its gradient is tangent to the level sets of $b$.

This suggests the use of the following definition of tangential gradient, which will be the clue to the tangential differential calculus.

\begin{definition}[Canonical extension]
    Under the assumptions of Definition 5.1 on $b_\Omega$, associate with $f\in C^1(\Gamma)$ \textbf{(5.1)}
    \begin{align*}
        \nabla_\Gamma f\coloneqq\nabla(f\circ p)|_\Gamma.
    \end{align*}
\end{definition}

\begin{remark}
    This definition naturally extends to nonempty sets $A$ s.t. $d_A^2$ belongs to $C^{1,1}(S_{2h}(A))$, since the projection onto $A$,
    \begin{align*}
        p_A(x) = x - \frac{1}{2}\nabla d_A^2(x),
    \end{align*}
    is $C^{0,1}$.
    
    They are the sets of positive reach introduced by Federer.
    
    They include convex sets and submanifolds of codimension larger than or equal to 1.
\end{remark}

\begin{theorem}
    Under the assumption of Definition 5.1 on $b_\Omega$ for $f\in C^1(\Gamma)$ the following hold:
    \begin{itemize}
        \item[(i)] $\nabla b\cdot\nabla(f\circ p) = 0$ in $S_h(\Gamma)$ and ${\bf n}\cdot\nabla_\Gamma f = 0$ on $\Gamma$.
        \item[(ii)] $\nabla F|_\Gamma - \partial_{\bf n}F{\bf n} = (P\nabla F)|_\Gamma = \nabla_\Gamma f$ and $\nabla(f\circ p) = [I - bD^2b]\nabla_\Gamma f\circ p$ in $\Gamma$.
    \end{itemize}
\end{theorem}

\subsubsection{1st-order derivatives}
The \textit{tangential Jacobian matrix} of a vector function $v\in C^1(\Gamma)^M$, $M\ge 1$, is defined in the same way as the gradient: \textbf{(5.2)}
\begin{align*}
    D_\Gamma v\coloneqq D(v\circ p)|_\Gamma \mbox{ or } (D_\Gamma v)_{ij} = (\nabla_\Gamma v_i)_j.
\end{align*}
If $v^\top = (v_1,\ldots,v_M)$, then
\begin{align*}
    (D_\Gamma v)^\top = \left(\nabla_\Gamma v_1,\ldots,\nabla_\Gamma v_M\right),
\end{align*}
where $\nabla_\Gamma v_i$ is a column vector.

From the previous theorems about the tangential gradient we can recover the definition from an extension $V\in C^1(S_{2h}(\Gamma))^M$ of $v$:
\begin{align*}
    (D_\Gamma v)^\top = (P\nabla V_1,\ldots,P\nabla V_M)|_\Gamma = (I - \nabla b\nabla b^\top)(DV)^\top|_\Gamma = (DV)^\top|_\Gamma - \nabla b(DV\nabla b)^\top|_\Gamma,
\end{align*}
and \textbf{(5.3)}
\begin{align*}
    D_\Gamma v = DV|_\Gamma - DV{\bf n}{\bf n}^\top = (DVP)|_\Gamma.
\end{align*}
Also,
\begin{align*}
    (D(v\circ p))^\top = [I - bD^2b]\left(\nabla_\Gamma v_1,\ldots,\nabla_\Gamma v_M\right)\circ p = [I - bD^2b](D_\Gamma v)^\top\circ p,
\end{align*}
and we have for the extension \textbf{(5.4)}
\begin{align*}
    D(v\circ p) = D_\Gamma v\circ p[I - bD^2b].
\end{align*}
Note that \textbf{(5.5)}
\begin{align*}
    D(v\circ p)\nabla b = 0 \mbox{ and } D_\Gamma v{\bf n} = 0.
\end{align*}
For a vector function $v\in C^1(\Gamma)^N$ define the \textit{tangential divergence} as \textbf{(5.6)}
\begin{align*}
    \operatorname{div}_\Gamma v\coloneqq\nabla\cdot(v\circ p)|_\Gamma,
\end{align*}
and it is easy to show that
\begin{align*}
    \operatorname{div}_\Gamma v &= \nabla\cdot(v\circ p)|_\Gamma = \operatorname{tr}D(v\circ p)|_\Gamma = \operatorname{tr}D_\Gamma v,\\
    \operatorname{div}_\Gamma v &= \operatorname{tr}\left[DV|_\Gamma - DV{\bf n}{\bf n}^\top\right] = \nabla\cdot V|_\Gamma - DV{\bf n}\cdot{\bf n}.
\end{align*}
The \textit{tangential linear strain tensor} of linear elasticity is given by \textbf{(5.7)-(5.9)}
\begin{align*}
    \varepsilon_\Gamma(v) &\coloneqq\frac{1}{2}\left(D_\Gamma v + (D_\Gamma v)^\top\right),\\
    \varepsilon(v\circ p) &= \frac{1}{2}\left(D(v\circ p) + D(v\circ p)^\top\right) = \varepsilon_\Gamma(v)\circ p - \frac{b}{2}\left[D_\Gamma v\circ pD^2b + D^2b(D_\Gamma v)^\top\circ p\right]\\
    \Rightarrow\varepsilon_\Gamma(v) &= \varepsilon(v\circ p)|_\Gamma.
\end{align*}
The tangential Jacobian matrix of the normal ${\bf n}$ is especially interesting since, from Theorem 8.4 (i) in Chap. 7, ${\bf n}\circ p = \nabla b = \nabla b\circ p$.

As a result \textbf{(5.10)}
\begin{align*}
    D_\Gamma({\bf n}) = D^2b|_\Gamma = D_\Gamma({\bf n})^\top\Rightarrow\varepsilon_\Gamma({\bf n}) = D^2b|_\Gamma.
\end{align*}
The \textit{tangential vectorial divergence} of a matrix or tensor function $A$ is defined as \textbf{(5.11)}
\begin{align*}
    (\tilde{\operatorname{div}}_\Gamma A)_i\coloneqq\operatorname{div}_\Gamma(A_{i,\cdot}).
\end{align*}

\subsubsection{2nd-order derivatives}
Assume that $\Omega$ is of class $C^3$.

The simplest 2nd-order derivative is the \textit{Laplace-Beltrami operator} of a function $f\in C^2(\Gamma)$, which is defined as \textbf{(5.12)}
\begin{align*}
    \Delta_\Gamma f\coloneqq\operatorname{div}_\Gamma\left(\nabla_\Gamma f\right).
\end{align*}
Recall from Theorem 5.2 the identity
\begin{align*}
    \nabla(f\circ p) = [I - bD^2b]\nabla_\Gamma f\circ p.
\end{align*}
Then
\begin{align*}
    \nabla\cdot(\nabla_\Gamma f\circ p) &= \nabla\cdot(\nabla(f\circ p)) + \nabla\cdot\left(bD^2b\nabla_\Gamma f\circ p\right),\\
    \nabla\cdot(\nabla_\Gamma f\circ p) &= \Delta(f\circ p) + b\nabla\cdot(D^2b\nabla_\Gamma f\circ p) + \left(D^2b\nabla_\Gamma f\circ p\right)\cdot\nabla b,\\
    \nabla\cdot(\nabla_\Gamma f\circ p) &= \Delta(f\circ p) + b\nabla\cdot\left(D^2b\nabla_\Gamma f\circ p\right),
\end{align*}
and by taking restrictions to $\Gamma$,
\begin{align*}
    \Delta_\Gamma f = \Delta(f\circ p)|_\Gamma.
\end{align*}
The \textit{tangential Hessian matrix} of 2nd-order derivatives is defined as \textbf{(5.13)}
\begin{align*}
    D_\Gamma^2 f\coloneqq D_\Gamma(\nabla_\Gamma f).
\end{align*}
Here the curvatures of the submanifold begin to appear.

The Hessian matrix is not symmetrical and does not coincide with the restriction of the Hessian matrix of the canonical extension.

Specifically,
\begin{align*}
    D^2(f\circ p) &= D\left(\nabla(f\circ p)\right) = D\left([I - bD^2b]\nabla_\Gamma f\circ p\right) = D\left(\nabla_\Gamma f\circ p\right) - D\left(bD^2b\nabla_\Gamma f\circ p\right)\\
    &= D(\nabla_\Gamma f\circ p) - bD(D^2b\nabla_\Gamma f\circ p) - D^2b\nabla_\Gamma f\circ p\nabla b^\top,\\
    D^2(f\circ p)|_\Gamma &= D_\Gamma(\nabla_\Gamma f) - D^2b\nabla_\Gamma f\nabla b^\top = D_\Gamma^2f - D^2b\nabla_\Gamma f{\bf n}^\top.
\end{align*}
Of course, since $D^2(f\circ p)$ is symmetrical we also have
\begin{align*}
    D^2(f\circ p) &= \left(D\left(\nabla(f\circ p)\right)\right)^\top = \left(D\left([I - bD^2b]\nabla_\Gamma f\circ p\right)\right)^\top = \left(D\left(\nabla_\Gamma f\circ p\right)\right)^\top - \left(D\left(bD^2b\nabla_\Gamma f\circ p\right)\right)^\top\\
    &= \left(D(\nabla_\Gamma f\circ p)\right)^\top - b(D(D^2b\nabla_\Gamma f\circ p))^\top - \nabla b(D^2b\nabla_\Gamma f\circ p)^\top,\\
    D^2(f\circ p)|_\Gamma &= (D_\Gamma(\nabla_\Gamma f))^\top - \nabla b(D^2b\nabla_\Gamma f)^\top = (D_\Gamma^2f)^\top - {\bf n}(D^2b\nabla_\Gamma f)^\top.
\end{align*}
As a final result we have the following identity: \textbf{(5.14)}
\begin{align*}
    D_\Gamma^2f - (D^2b\nabla_\Gamma f){\bf n}^\top = D^2(f\circ p)|_\Gamma = (D_\Gamma^2f)^\top - {\bf n}(D^2b\nabla_\Gamma f)^\top,
\end{align*}
since by definition $D_\Gamma^2f^\top = (D_\Gamma^2f)^\top$.

So the Hessian and its transpose differ by terms that contain a first-order derivative as is well known in differential geometry.

Note that $D^2b\nabla_\Gamma f = -(D_\Gamma(\nabla_\Gamma f))^\top{\bf n} = -(D_\Gamma^2)^\top f{\bf n}$ and that we also can write \textbf{(5.15)}
\begin{align*}
    D_\Gamma^2f + (D_\Gamma^2f)^\top[{\bf n}{\bf n}^\top] = D^2(f\circ p)|_\Gamma = (D_\Gamma^2f)^\top + [{\bf n}{\bf n}^\top]D_\Gamma^2f\Rightarrow PD_\Gamma^2f = (PD_\Gamma^2f)^\top.
\end{align*}

\subsubsection{A few useful formulae \& the chain rule}
Associate with $F\in C^1(S_{2h}(\Gamma))$ and $V\in C^1(S_{2h}(\Gamma))^N$ \textbf{(5.16)}
\begin{align*}
    f\coloneqq F|_\Gamma,\ {\bf v}\coloneqq V|_\Gamma,\ {\bf v}_{\bf n}\coloneqq{\bf v}\cdot{\bf n},\ {\bf v}_\Gamma\coloneqq{\bf v} - {\bf v}_{\bf n}{\bf n}.
\end{align*}
where ${\bf v}_\Gamma$ and ${\bf v}_{\bf n}$ are the respective tangential part and the normal component of ${\bf v}$.

In view of the previous definitions the following identities are easy to check: \textbf{(5.17)-(5.20)}
\begin{align*}
    \nabla F|_\Gamma &= \nabla_\Gamma f + \partial_{\bf n}F{\bf n},\\
    DV|_\Gamma &= D_\Gamma{\bf v} + DV{\bf n}{\bf n}^\top,\\
    \nabla\cdot V|_\Gamma &= \operatorname{div}_\Gamma{\bf v} + DV{\bf n}\cdot{\bf n},\\
    D_\Gamma{\bf v}{\bf n} &= 0,\ D^2b{\bf n} = 0.
\end{align*}
Decomposing ${\bf v}$ into its tangential part and its normal component, \textbf{(5.21)-(5.23)}
\begin{align*}
    D_\Gamma{\bf v} &= D_\Gamma{\bf v}_\Gamma + {\bf v}_{\bf n}D^2b + {\bf n}(\nabla_\Gamma{\bf v}_{\bf n})^\top,\\
    \operatorname{div}_\Gamma{\bf v} &= \operatorname{div}_\Gamma{\bf v}_\Gamma + \Delta b{\bf v}_{\bf n} = \operatorname{div}_\Gamma{\bf v}_\Gamma + H{\bf v}_{\bf n},\\
    \nabla_\Gamma{\bf v}_{\bf n} &= (D_\Gamma{\bf v})^\top{\bf n} + D^2b{\bf v}_\Gamma.
\end{align*}
Given $f\in C^1(\Gamma)$ and $g\in C^1(\Gamma;\Gamma)$, consider the canonical extensions $f\circ p\in C^1(S_{2h}(\Gamma))$ and $g\circ p\in C^1(S_{2h}(\Gamma);\mathbb{R}^N)$ and the gradient of the composition \textbf{(5.24)}
\begin{align*}
    \nabla(f\circ p\circ g\circ p) &= (D(g\circ p))^\top\nabla(f\circ p)\circ g\circ p\\
    \Rightarrow\nabla_\Gamma(f\circ g) &= (D_\Gamma g)^\top\nabla_\Gamma f\circ g,
\end{align*}
and for a vector-valued function ${\bf v}\in C^1(\Gamma;\mathbb{R}^N)$ \textbf{(5.25)}
\begin{align*}
    D_\Gamma({\bf v}\circ g) = D_\Gamma{\bf v}\circ gD_\Gamma g.
\end{align*}

\subsubsection{The Stokes \& Green formulae}
1 interesting application of the shape calculus in connection with the tangential calculus is the tangential Stokes formula.

Given ${\bf v}\in C^1(\Gamma)^N$, consider Stokes formula in $\mathbb{R}^N$ for the vector function ${\bf v}\circ p$:
\begin{align*}
    \int_\omega \nabla\cdot({\bf v}\circ p){\rm d}x = \int_\Gamma {\bf v}\cdot{\bf n}{\rm d}\Gamma.
\end{align*}
Given an autonomous velocity field $V$, differentiate both sides of Stokes's formula w.r.t. $t$,
\begin{align*}
    \int_{\Omega_t(V)} \nabla\cdot(v\circ p){\rm d}x = \int_{\Gamma_t(V)} (v\circ p)\cdot{\bf n}_t = \int_{\Gamma_t(V)} (v\circ p)\cdot{\bf N}_t{\rm }d\Gamma_t,
\end{align*}
where ${\bf N}_t$ is the extension (4.37) of ${\bf n}_t$.

This gives the new identity
\begin{align*}
    \int_\Gamma \nabla\cdot(v\circ p)V\cdot{\bf n}{\rm d}\Gamma = \int_\Gamma {\bf v}\cdot{\bf N}' + \left\{\partial_{\bf n}[({\bf v}\circ p)\cdot\nabla b] + H{\bf v}\cdot{\bf n}\right\}V\cdot{\bf n}{\rm d}\Gamma.
\end{align*}
Now choose the velocity field $V = \nabla b\psi$, where $\psi\in\mathcal{D}(S_{2h}(\Gamma))$ is chosen in such a way that $\psi = 1$ on $S_h(\Gamma)$.

Using expression (4.38) for ${\bf N}'$,
\begin{align*}
    V\cdot{\bf n} &= {\bf n}\cdot{\bf n} = 1,\ \nabla\cdot({\bf v}\circ p)|_\Gamma = \operatorname{div}_\Gamma{\bf v},\\
    {\bf N}'|_\Gamma &= DV{\bf n}\cdot{\bf n}{\bf n} - (DV)^\top{\bf n} - D^2bV = D^2b\nabla b\cdot\nabla b\nabla b - D^2b\nabla b - D^2b\nabla b = 0,\\
    \nabla(({\bf v}\circ p)\cdot\nabla b) &= D^2b{\bf v}\circ p + (D({\bf v}\circ p)^\top)\nabla b,\\
    \partial_{\bf n}[({\bf v}\circ p)\cdot\nabla b] &= [D^2b{\bf v} + (D_\Gamma{\bf v})^\top{\bf n}]\cdot{\bf n} = {\bf v}\cdot D^2b\nabla b + {\bf n}\cdot D_\Gamma{\bf v}{\bf n} = 0. 
\end{align*}
Finally we get the \textit{tangential Stokes formula} with $H = \Delta b$: \textbf{(5.26)}
\begin{align*}
    \int_\Gamma \operatorname{div}_\Gamma{\bf v}{\rm d}\Gamma = \int_\Gamma H{\bf v}\cdot{\bf n}{\rm d}\Gamma.
\end{align*}
For a function $f\in C^1(\Gamma)$ and a vector ${\bf v}\in C^1(\Gamma)^N$, the above formula also yields the \textit{tangential Green's formula} \textbf{(5.27)}
\begin{align*}
    \int_\Gamma f\operatorname{div}_\Gamma{\bf v} + \nabla_\Gamma f\cdot{\bf v}{\rm d}\Gamma = \int_\Gamma Hf{\bf v}\cdot{\bf n}{\rm d}\Gamma.
\end{align*}

%
Coming back to formula (4.36) for $dJ(\Omega;V(0))$ of the boundary integral of the square of the normal derivative in Sect. 4.3.3, the tangential calculus is now used on the term
\begin{align*}
    \int_\Gamma &2\partial_{\bf n}\phi\left[\partial_{\bf n}\phi DV(0){\bf n}\cdot{\bf n} - \nabla\phi\cdot\left(DV(0)^\top{\bf n} + D^2bV(0)\right)\right]{\rm d}\Gamma\\
    T\coloneqq &\partial_{\bf n}\phi DV(0){\bf n}\cdot{\bf n} - \nabla\phi\cdot\left(DV(0)^\top{\bf n} + D^2bV(0)\right).
\end{align*}
From identity (5.18) with ${\bf v} = V(0)|_\Gamma$ and identity (5.23)
\begin{align*}
    DV(0)^\top{\bf n} &= (D_\Gamma{\bf v})^\top{\bf n} + DV(0){\bf n}\cdot{\bf n}{\bf n},\\
    \nabla\phi\cdot DV(0)^\top{\bf n} &= \nabla\phi\cdot(D_\Gamma{\bf v})^\top{\bf n} + \partial_{\bf n}\phi DV(0){\bf n}\cdot{\bf n},\\
    \Rightarrow T &= -\nabla\phi\cdot\left[(D_\Gamma{\bf v})^\top{\bf n} + D^2b{\bf v}_\Gamma\right] = -\nabla\phi\cdot\nabla_\Gamma{\bf v}_{\bf n} = -\nabla_\Gamma\phi\cdot\nabla_\Gamma{\bf v}_{\bf n}.
\end{align*}
Therefore by using the tangential Stokes formula (5.26),
\begin{align*}
    \int_\Gamma 2\partial_{\bf n}\phi T{\rm d}\Gamma &= -\int_\Gamma 2\partial_{\bf n}\phi\nabla_\Gamma\phi\cdot\nabla_\Gamma{\bf v}_{\bf n}{\rm d}\Gamma = \int_\Gamma -2\operatorname{div}_\Gamma\left(\partial_{\bf n}\phi{\bf v}_{\bf n}\nabla_\Gamma\phi\right) + 2\operatorname{div}_\Gamma\left(\partial_{\bf n}\nabla_\Gamma\phi\right){\bf v}_{\bf n}{\rm d}\Gamma\\
    &= \int_\Gamma -2H\partial_{\bf n}\phi{\bf v}_{\bf n}\nabla_\Gamma\phi\cdot{\bf n} + 2\operatorname{div}_\Gamma\left(\partial_{\bf n}\phi\nabla_\Gamma\phi\right){\bf v}_{\bf n}{\rm d}\Gamma = \int_\Gamma 2\operatorname{div}_\Gamma\left(\partial_{\bf n}\phi\nabla_\Gamma\phi\right){\bf v}_{\bf n}{\rm d}\Gamma\\
    &= \int_\Gamma 2\operatorname{div}_\Gamma\left(\partial_{\bf n}\phi\nabla_\Gamma\phi\right)V(0)\cdot{\bf n}{\rm d}\Gamma.
\end{align*}
Substituting into expression (4.36), we finally get the explicit formula in terms of $V(0)\cdot{\bf n}$ as predicted by the structure theorem \textbf{(5.29)}
\begin{align*}
    dJ(\Omega;V) = \int_\Gamma \left[2\partial_{\bf n}\phi D^2\phi{\bf n}\cdot{\bf n} + H\left|\partial_{\bf n}\phi\right|^2 + 2\operatorname{div}_\Gamma\left(\partial_{\bf n}\phi\nabla_\Gamma\phi\right)\right]V(0)\cdot{\bf n}{\rm d}\Gamma.
\end{align*}

\subsection{2nd-order semiderivative \& shape Hessian*}

%------------------------------------------------------------------------------%

\chapter{Shape Optimization for Stationary Stokes Equations}

\section{Stationary Stokes equations: existence, uniqueness, \& regularity}
We consider the following stationary Stokes equations:
\begin{equation}
    \label{stationary Stokes}
    \tag{S}
    \left\{\begin{split}
        -\nu\Delta{\bf u} + \nabla p &= {\bf f} &&\mbox{ in } \Omega,\\
        \nabla\cdot{\bf u} &= f_{\rm div} &&\mbox{ in } \Omega,\\
        \gamma_0{\bf u} = {\bf u}_\Gamma \mbox{ i.e., } {\bf u} &= {\bf u}_\Gamma &&\mbox{ on } \Gamma,
    \end{split}\right.
\end{equation}
where $\gamma_0\in\mathcal{L}(H^1(\Omega),L^2(\Gamma))$ is the trace operator s.t. $\gamma_0u =$ the restriction of $u$ to $\Gamma$ for every function $u\in H^1(\Omega)$ (see, e.g., \cite[p. 6]{Temam2000}).

\begin{theorem}[Case $f_{\rm div} = 0$, ${\bf u}_\Gamma = {\bf 0}$]
    \begin{itemize}
        \item[(i)] (Existence) For any open set $\Omega\subset\mathbb{R}^N$ which is bounded in some direction, and for every ${\bf f}\in{\bf H}^{-1}(\Omega)$, the problem
        \begin{align}
            \label{Temam2000 (2.6)}
            {\bf u} \mbox{ belongs to } V \mbox{ and satisfies } \nu(({\bf u},{\bf v})) = \langle{\bf f},{\bf v}\rangle_{V^\star,V},\ \forall{\bf v}\in V
        \end{align}
        has a unique solution ${\bf u}$.
        
        Moreover, there exists a function $p\in L_{\rm loc}^2(\Omega)$ s.t. the following 2 statements are satisfied:
        \begin{itemize}
            \item[(a)] there exists $p\in L^2(\Omega)$ s.t. $-\nu\Delta{\bf u} + \nabla p = {\bf f}$ in the distribution sense in $\Omega$;
            \item[(b)] $\nabla\cdot{\bf u} = 0$ in the distribution sense in $\Omega$.
        \end{itemize}    
        If $\Omega$ is an open bounded set of class $C^2$, then $p\in L^2(\Omega)$ and (a), (b), and $\gamma_0{\bf u} = {\bf 0}$ are satisfied by ${\bf u}$ and $p$.
        \item[(ii)] (A variational property) The solution ${\bf u}$ of \eqref{Temam2000 (2.6)} is also the unique element of $V$ s.t.
        \begin{align*}
            E({\bf u})\le E({\bf v}),\ \forall{\bf v}\in V, \mbox{ where } E({\bf v})\coloneqq\nu\|{\bf v}\|^2 - 2\langle{\bf f},{\bf v}\rangle_{V^\star,V}.
        \end{align*}
    \end{itemize}    
\end{theorem}

\begin{theorem}[Non-homogeneous Stokes: Existence]
    \begin{itemize}
        \item[(i)] Let $\Omega$ be an open bounded set of class $C^2$ in $\mathbb{R}^N$. Let there be given ${\bf f}\in{\bf H}^{-1}(\Omega)$, $f_{\rm div}\in L^2(\Omega)$, ${\bf u}_\Gamma\in{\bf H}^{1/2}(\Gamma)$, s.t.
        \begin{align*}
            \int_\Omega f_{\rm div}{\rm d}{\bf x} = \int_\Gamma {\bf u}_\Gamma\cdot{\bf n}{\rm d}\Gamma.
        \end{align*}
        Then there exists ${\bf u}\in{\bf H}^1(\Omega)$, $p\in L^2(\Omega)$, which are solution of the non-homogeneous Stokes problem \eqref{stationary Stokes}, ${\bf u}$ is unique and $p$ is unique up to the addition of a constant.
        \item[(ii)] Let $\Omega$ be a Lipschitz open bounded set in $\mathbb{R}^N$. Let there be given ${\bf f}\in{\bf H}^{-1}(\Omega)$, $f_{\rm div}\in L^2(\Omega)$, ${\bf u}_\Gamma$ as the trace of a function ${\bf u}_{\Gamma,0}\in{\bf H}^1(\Omega)$, s.t.
        \begin{align*}
            \int_\Omega f_{\rm div}{\rm d}{\bf x} = \int_\Omega \nabla\cdot{\bf u}_{\Gamma,0}{\rm d}{\bf x}.
        \end{align*}
        Then there exists ${\bf u}\in{\bf H}^1(\Omega)$, $p\in L^2(\Omega)$, which are solution of the non-homogeneous Stokes problem
        \begin{equation*}
            \left\{\begin{split}
                -\nu\Delta{\bf u} + \nabla p &= {\bf f} &&\mbox{ in } \Omega,\\
                \nabla\cdot{\bf u} &= f_{\rm div} &&\mbox{ in } \Omega,\\
                {\bf u} - {\bf u}_{\Gamma,0}&\in{\bf H}_0^1(\Omega).
            \end{split}\right.
        \end{equation*}
        ${\bf u}$ is unique and $p$ is unique up to the addition of a constant.
    \end{itemize}
\end{theorem}

\begin{theorem}[Non-homogeneous Stokes: regularity]
    \begin{itemize}
        \item[(i)] Let $\Omega$ be an open bounded set of class $C^r$, $r = \max(m + 2,2)$, $m$ integer $> 0$. Let us suppose that ${\bf u}\in{\bf W}^{2,\alpha}(\Omega)$, $p\in W^{1,\alpha}(\Omega)$, $1 < \alpha < +\infty$, are solutions of the generalized Stokes problem
        \begin{equation*}
            \left\{\begin{split}
                -\nu\Delta{\bf u} + \nabla p &= {\bf f} &&\mbox{ in } \Omega,\\
                \nabla\cdot{\bf u} &= f_{\rm div} &&\mbox{ in } \Omega,\\
                \gamma_0{\bf u} = {\bf u}_\Gamma \mbox{ i.e., } {\bf u} &= {\bf u}_\Gamma &&\mbox{ on } \Gamma,
            \end{split}\right.
        \end{equation*}
        If ${\bf u}\in{\bf W}^{m,\alpha}(\Omega)$, $f_{\rm div}\in W^{m+1,\alpha}(\Omega)$ and ${\bf u}_\Gamma\in{\bf W}^{m + 2 - 1/\alpha,\alpha}(\Gamma)$,\footnote{$W^{m + 2 - 1/\alpha,\alpha}(\Gamma) = \gamma_0W^{m+2,\alpha}(\Omega)$ and is equipped with the image norm
        \begin{align*}
            \|\psi\|_{W^{m + 2 - 1/\alpha,\alpha}(\Gamma)} = \inf_{\gamma_0{\bf u} = \psi} \|{\bf u}\|_{{\bf W}^{m+2,\alpha}(\Omega)}.
        \end{align*}} then ${\bf u}\in{\bf W}^{m+2,\alpha}(\Omega),\ p\in W^{m+1,\alpha}(\Omega)$ and there exists a constant $c_0(\alpha,\nu,m,\Omega)$ s.t.
        \begin{align}
            \label{Temam2000 (2.46)}
            \|{\bf u}\|_{{\bf W}^{m+2,\alpha}(\Omega)} + \|p\|_{W^{m+1,\alpha}(\Omega)/\mathbb{R}}\le c_0\left\{\|{\bf f}\|_{{\bf W}^{m,\alpha}(\Omega)} + \|f_{\rm div}\|_{W^{m+1,\alpha}(\Omega)} + \|{\bf u}_\Gamma\|_{{\bf W}^{m + 2 - 1/\alpha,\alpha}(\Gamma)} + d_\alpha\|{\bf u}\|_{{\bf L}^\alpha(\Omega)}\right\},
        \end{align}
        where $d_\alpha = 0$ for $\alpha\ge 2$, $d_\alpha = 1$ for $1 < \alpha < 2$.
    \end{itemize}
\end{theorem}

\begin{proposition}
    Let $\Omega$ be an open set of $\mathbb{R}^N$, $N = 2$ or 3, of class $C^r$, $r = \max(m + 2,2)$, $m$ integer $\ge -1$, and let ${\bf f}\in{\bf W}^{m,\alpha}(\Omega)$, $f_{\rm div}\in W^{m+1,\alpha}(\Omega)$, ${\bf u}_\Gamma\in{\bf W}^{m + 2 - 1/\alpha,\alpha}(\Gamma)$ be given satisfying the compatibility condition
    \begin{align*}
        \int_\Omega f_{\rm div}{\rm d}{\bf x} = \int_\Gamma {\bf u}_\Gamma\cdot{\bf n}{\rm d}\Gamma.
    \end{align*}
    Then there exist unique functions ${\bf u}$ and $p$ ($p$ is unique up to a constant) which are solutions of \eqref{stationary Stokes} and satisfy ${\bf u}\in{\bf W}^{m+2,\alpha}(\Omega)$, $p\in{\bf W}^{m+1,\alpha}(\Omega)$, and \eqref{Temam2000 (2.46)} with $d_\alpha = 0$ for any $\alpha$, $1 < \alpha < \infty$.
\end{proposition}

\section{Cost functionals for \eqref{stationary Stokes}}
We consider the following general cost functional for \eqref{stationary Stokes} is given by
\begin{align}
    \label{cost functional for stationary Stokes}
    \tag{cost-S}
    J({\bf u},p,\Omega)\coloneqq\int_\Omega J_\Omega({\bf x},{\bf u},\nabla{\bf u},p){\rm d}{\bf x} + \int_\Gamma J_\Gamma({\bf x},{\bf u},\nabla{\bf u},p,{\bf n},{\bf t}){\rm d}\Gamma.
\end{align}

\section{Lagrangian \& extended Lagrangian for \eqref{stationary Stokes}}
To derive the adjoint equations for \eqref{stationary Stokes}, we 1st introduce the following Lagrangian (see, e.g., \cite{Troltzsch2010}):
\begin{align}
    \label{Lagrangian for stationary Stokes}
    \tag{$L$-S}
    L({\bf u},p,\Omega,{\bf v},q)\coloneqq J({\bf u},p,\Omega) - \int_\Omega (-\nu\Delta{\bf u} + \nabla p - {\bf f})\cdot{\bf v} + q(\nabla\cdot{\bf u} - f_{\rm div}){\rm d}{\bf x},
\end{align}
and the following extended Lagrangian:
\begin{align}
    \mathcal{L}({\bf u},p,\Omega,{\bf v},q,{\bf v}_\Gamma)\coloneqq&\,L({\bf u},p,\Omega,{\bf v},q) - \int_\Gamma ({\bf u} - {\bf u}_\Gamma)\cdot{\bf v}_\Gamma{\rm d}\Gamma\nonumber\\
    =&\, J({\bf u},p,\Omega) - \int_\Omega (-\nu\Delta{\bf u} + \nabla p - {\bf f})\cdot{\bf v} + q(\nabla\cdot{\bf u} - f_{\rm div}){\rm d}{\bf x} - \int_\Gamma ({\bf u} - {\bf u}_\Gamma)\cdot{\bf v}_\Gamma{\rm d}\Gamma,\label{extended Lagrangian for stationary Stokes}\tag{$\mathcal{L}$-S}
\end{align}
where ${\bf u},q,{\bf u}_\Gamma$ are Lagrange multipliers.

We also introduce the following ``mixed'' Lagrangian:
\begin{align}
    \label{mixed Lagrangian for stationary Stokes}
    \tag{$L_{\mathcal{L}}$-S}
    L_{\mathcal{L}}({\bf u},p,\Omega,{\bf v},q,{\bf v}_\Gamma)\coloneqq&\, L({\bf u},p,\Omega,{\bf v},q) - \delta_{\mathcal{L}}\int_\Gamma ({\bf u} - {\bf u}_\Gamma)\cdot{\bf v}_\Gamma{\rm d}\Gamma\\
    =&\, J({\bf u},p,\Omega) - \int_\Omega (-\nu\Delta{\bf u} + \nabla p - {\bf f})\cdot{\bf v} + q(\nabla\cdot{\bf u} - f_{\rm div}){\rm d}{\bf x} - \delta_{\mathcal{L}}\int_\Gamma ({\bf u} - {\bf u}_\Gamma)\cdot{\bf v}_\Gamma{\rm d}\Gamma, \mbox{ where } \delta_{\mathcal{L}}\in\{0,1\}.
\end{align}

\section{Shape optimization problems}
Here are 3 different shape optimization problems associated with \eqref{cost functional for stationary Stokes}, \eqref{Lagrangian for stationary Stokes}, and \eqref{extended Lagrangian for stationary Stokes}, respectively:
\begin{align*}
    &\min_{\Omega\in\mathcal{O}_{\rm ad}} J({\bf u},p,\Omega) \mbox{ s.t. } ({\bf u},p) \mbox{ solves \eqref{stationary Stokes}},\\
    &\min_{\Omega\in\mathcal{O}_{\rm ad}} L({\bf u},p,\Omega,{\bf v},q) \mbox{ s.t. } ({\bf u},p) \mbox{ satisfies } {\bf u} = {\bf u}_\Gamma \mbox{ on } \Gamma,\\
    &\min_{\Omega\in\mathcal{O}_{\rm ad}} \mathcal{L}({\bf u},p,\Omega,{\bf v},q,{\bf v}_{\rm bc}) \mbox{ with } ({\bf u},p) \mbox{ unconstrained},
\end{align*}
and
\begin{equation*}
    \min_{\Omega\in\mathcal{O}_{\rm ad}} L_{\mathcal{L}}({\bf u},p,\Omega,{\bf v},q,{\bf v}_\Gamma)\ \left\{\begin{split}
        &\mbox{s.t. } ({\bf u},p) \mbox{ satisfies } {\bf u} = {\bf u}_\Gamma \mbox{ on } \Gamma &&\mbox{ if } \delta_{\mathcal{L}} = 0,\\
        &\mbox{with } ({\bf u},p) \mbox{ unconstrained} &&\mbox{ if } \delta_{\mathcal{L}} = 1.
    \end{split}\right.
\end{equation*}
Choose the adjoint variables/Lagrangian multipliers $({\bf v},q,{\bf v}_\Gamma)$ s.t.
\begin{align*}
    D_{\bf u}L_{\mathcal{L}}({\bf u},p,\Omega,{\bf v},q,{\bf v}_\Gamma)\tilde{\bf u} + D_pL_{\mathcal{L}}({\bf u},p,\Omega,{\bf v},q,{\bf v}_\Gamma)\tilde{p} = 0,\ \forall({\bf u},p,\Omega,\tilde{\bf u},\tilde{p}).
\end{align*}
Expand this more explicitly for all $({\bf u},p,\Omega,\tilde{\bf u},\tilde{p})$:
\begin{align*}
    &\int_\Omega D_{\bf u}\left(J_\Omega({\bf x},{\bf u},\nabla{\bf u},p)\right)\tilde{\bf u} + D_p\left(J_\Omega({\bf x},{\bf u},\nabla{\bf u},p)\right)\tilde{p}{\rm d}{\bf x} + \int_\Gamma D_{\bf u}\left(J_\Gamma({\bf x},{\bf u},\nabla{\bf u},p,{\bf n},{\bf t})\right)\tilde{\bf u} + D_p\left(J_\Gamma({\bf x},{\bf u},\nabla{\bf u},p,{\bf n},{\bf t})\right)\tilde{p}{\rm d}\Gamma\\
    &-\int_\Omega (-\nu\Delta\tilde{\bf u} + \nabla\tilde{p})\cdot{\bf v} + q\nabla\cdot\tilde{\bf u}{\rm d}{\bf x} - \delta_{\mathcal{L}}\int_\Gamma \tilde{\bf u}\cdot{\bf v}_\Gamma{\rm d}\Gamma = 0,
\end{align*}
and more explicitly:
\begin{align*}
    &\int_\Omega D_{\bf u}J_\Omega({\bf x},{\bf u},\nabla{\bf u},p)\tilde{\bf u} + D_{\nabla{\bf u}}J_\Omega({\bf x},{\bf u},\nabla{\bf u},p)\nabla\tilde{\bf u} + D_pJ_\Omega({\bf x},{\bf u},\nabla{\bf u},p)\tilde{p}{\rm d}{\bf x}\\
    &+ \int_\Gamma D_{\bf u}J_\Gamma({\bf x},{\bf u},\nabla{\bf u},p,{\bf n},{\bf t})\tilde{\bf u} + D_{\nabla{\bf u}}J_\Gamma({\bf x},{\bf u},\nabla{\bf u},p,{\bf n},{\bf t})\nabla\tilde{\bf u} + D_pJ_\Gamma({\bf x},{\bf u},\nabla{\bf u},p,{\bf n},{\bf t})\tilde{p}{\rm d}\Gamma\\
    &-\int_\Omega (-\nu\Delta\tilde{\bf u} + \nabla\tilde{p})\cdot{\bf v} + q\nabla\cdot\tilde{\bf u}{\rm d}{\bf x} - \delta_{\mathcal{L}}\int_\Gamma \tilde{\bf u}\cdot{\bf v}_\Gamma{\rm d}\Gamma = 0,\ \forall({\bf u},p,\Omega,\tilde{\bf u},\tilde{p}).
\end{align*}
Integrate by parts:
\begin{enumerate}
    \item Term $D_{\nabla{\bf u}}J_\Omega({\bf x},{\bf u},\nabla{\bf u},p)\nabla\tilde{\bf u}$:
    \begin{align*}
        &\int_\Omega D_{\nabla{\bf u}} J_\Omega({\bf x},{\bf u},\nabla{\bf u},p)\nabla\tilde{\bf u}{\rm d}{\bf x} = \int_\Omega \nabla_{\nabla{\bf u}}J_\Omega({\bf x},{\bf u},\nabla{\bf u},p):\nabla\tilde{\bf u}{\rm d}{\bf x} = \int_\Omega \sum_{i=1}^N\sum_{j=1}^N \partial_{\partial_{x_i}u_j}J_\Omega({\bf x},{\bf u},\nabla{\bf u},p)\partial_{x_i}\tilde{u}_j{\rm d}{\bf x}\\
        =&\, \sum_{i=1}^N\sum_{j=1}^N \int_\Omega \partial_{\partial_{x_i}u_j}J_\Omega({\bf x},{\bf u},\nabla{\bf u},p)\partial_{x_i}\tilde{u}_j{\rm d}{\bf x} = \sum_{i=1}^N\sum_{j=1}^N -\int_\Omega \partial_{x_i}\partial_{\partial_{x_i}u_j}J_\Omega({\bf x},{\bf u},\nabla{\bf u},p)\tilde{u}_j{\rm d}{\bf x} + \int_\Gamma n_i\partial_{\partial_{x_i}u_j}J_\Omega({\bf x},{\bf u},\nabla{\bf u},p)\tilde{u}_j{\rm d}\Gamma\\
        =&\, -\int_\Omega \sum_{j=1}^N \tilde{u}_j\sum_{i=1}^N \partial_{x_i}\partial_{\partial_{x_i}u_j}J_\Omega({\bf x},{\bf u},\nabla{\bf u},p){\rm d}{\bf x} + \int_\Gamma \sum_{i=1}^N\sum_{j=1}^N n_i\partial_{\partial_{x_i}u_j}J_\Omega({\bf x},{\bf u},\nabla{\bf u},p)\tilde{u}_j{\rm d}\Gamma\\
        =&\, -\int_\Omega \sum_{j=1}^N \tilde{u}_j\nabla\cdot\left(\nabla_{\nabla u_j}J_\Omega({\bf x},{\bf u},\nabla{\bf u},p)\right){\rm d}{\bf x} + \int_\Gamma {\bf n}^\top\nabla_{\nabla{\bf u}}J_\Omega({\bf x},{\bf u},\nabla{\bf u},p)\tilde{\bf u}{\rm d}\Gamma\\
        =&\, -\int_\Omega \nabla\cdot\left(\nabla_{\nabla{\bf u}}J_\Omega({\bf x},{\bf u},\nabla{\bf u},p)\right)\cdot\tilde{\bf u}{\rm d}{\bf x} + \int_\Gamma {\bf n}^\top\nabla_{\nabla{\bf u}}J_\Omega({\bf x},{\bf u},\nabla{\bf u},p)\tilde{\bf u}{\rm d}\Gamma,
    \end{align*}
    where $\nabla_{\nabla{\bf u}} f(\nabla{\bf u})\coloneqq\left(\partial_{\partial_{x_i}u_j}f(\nabla{\bf u})\right)_{i,j=1}^N$ for any scalar function $f$.
    \item Term $\nu\Delta\tilde{\bf u}$: Use \eqref{Green's 2nd identity vector} to obtain:
    \begin{align*}
        \int_\Omega \nu\Delta\tilde{\bf u}\cdot{\bf v}{\rm d}{\bf x} = \int_\Omega \tilde{\bf u}\cdot\Delta{\bf v}{\rm d}{\bf x} + \int_\Gamma (\partial_{\bf n}\tilde{\bf u}\cdot{\bf v} - \partial_{\bf n}{\bf v}\cdot\tilde{\bf u}){\rm d}\Gamma.
    \end{align*}
    \item Term $-\nabla\tilde{p}\cdot{\bf v}$: Use \eqref{ibp} to obtain:
    \begin{align*}
        -\int_\Omega \nabla\tilde{p}\cdot{\bf v}{\rm d}{\bf x} = \int_\Omega \tilde{p}\nabla\cdot{\bf v}{\rm d}{\bf x} - \int_\Gamma \tilde{p}{\bf v}\cdot{\bf n}{\rm d}\Gamma.
    \end{align*}
    \item Term $-q\nabla\cdot\tilde{\bf u}$: Use \eqref{ibp} to obtain:
    \begin{align*}
        -\int_\Omega q\nabla\cdot\tilde{\bf u}{\rm d}{\bf x} = \int_\Omega \nabla q\cdot\tilde{\bf u}{\rm d}{\bf x} - \int_\Gamma q\tilde{\bf u}\cdot{\bf n}{\rm d}\Gamma.
    \end{align*}
\end{enumerate}
Gather terms:
\begin{align}
    &\int_\Omega \left[\nabla_{\bf u}J_\Omega({\bf x},{\bf u},\nabla{\bf u},p) - \nabla\cdot(\nabla_{\nabla{\bf u}}J_\Omega({\bf x},{\bf u},\nabla{\bf u},p)) + \Delta{\bf v} + \nabla q\right]\cdot\tilde{\bf u} + \tilde{p}\left[\partial_pJ_\Omega({\bf x},{\bf u},\nabla{\bf u},p) + \nabla\cdot{\bf v}\right]{\rm d}{\bf x}\nonumber\\
    &+ \int_\Gamma \left[\nabla_{\nabla{\bf u}}J_\Omega({\bf x},{\bf u},\nabla{\bf u},p)\cdot{\bf n} + \nabla_{\bf u}J_\Gamma({\bf x},{\bf u},\nabla{\bf u},p,{\bf n},{\bf t}) - \partial_{\bf n}{\bf v} - q{\bf n} - \delta_{\mathcal{L}}{\bf v}_\Gamma\right]\cdot\tilde{\bf u} + \tilde{p}\left[\partial_pJ_\Gamma({\bf x},{\bf u},\nabla{\bf u},p,{\bf n},{\bf t}) - {\bf v}\cdot{\bf n}\right]{\rm d}\Gamma\nonumber\\
    &+ \int_\Gamma \nabla_{\nabla{\bf u}}J_\Gamma({\bf x},{\bf u},\nabla{\bf u},p,{\bf n},{\bf t}):\nabla\tilde{\bf u} + \partial_{\bf n}\tilde{\bf u}\cdot{\bf v}{\rm d}\Gamma = 0,\ \forall({\bf u},p,\Omega,\tilde{\bf u},\tilde{p}).\label{Euler-Lagrange equation for stationary Stokes}\tag{EuLa-S}
\end{align}
Since this equation holds for all variations $(\tilde{\bf u},\tilde{p})$, consider the following 2 cases:
\begin{itemize}
    \item \textbf{Case $\delta_{\mathcal{L}} = 0$.} This means to ``activate'' the boundary-condition constraint ${\bf u} = {\bf u}_\Gamma$ on $\Gamma$\footnote{I.e., $\Gamma_{\rm D}^{\bf u}\equiv\Gamma_{\rm nv}^{\bf u}\equiv\Gamma$ in this case. For the definitions of these notations, see the chapter for a general framework.}, so it will not be penalized by the Lagrangian $L$. Note that if ${\bf u} + \tilde{\bf u} = {\bf u}_\Gamma$ on $\Gamma$ also, then the variation $\tilde{\bf u}$ equals ${\bf 0}$. Hence, in this case, we only consider $\tilde{\bf u}$ s.t. $\tilde{\bf u}|_\Gamma = {\bf 0}$.
    
    We now deduce the adjoint equations of \eqref{stationary Stokes} from \eqref{Euler-Lagrange equation for stationary Stokes} as follows:
    \begin{itemize}
        \item Choose $\tilde{\bf u} = {\bf 0}$ in $\overline{\Omega}$, \eqref{Euler-Lagrange equation for stationary Stokes} then becomes
        \begin{align*}
            \int_\Omega \left[\partial_pJ_\Omega({\bf x},{\bf u},\nabla{\bf u},p) + \nabla\cdot{\bf v}\right]\tilde{p}{\rm d}{\bf x} + \int_\Gamma \tilde{p}\left[\partial_pJ_\Gamma({\bf x},{\bf u},\nabla{\bf u},p,{\bf n},{\bf t}) - {\bf v}\cdot{\bf n}\right]{\rm d}\Gamma = 0,\ \forall({\bf u},p,\Omega,\tilde{p}).
        \end{align*}
        Then choose $\tilde{p}$ varying s.t. $\tilde{p}|_\Gamma = 0$, then the last equality yields
        \begin{align*}
            \int_\Omega \left[\partial_pJ_\Omega({\bf x},{\bf u},\nabla{\bf u},p) + \nabla\cdot{\bf v}\right]\tilde{p}{\rm d}{\bf x} = 0,\ \forall({\bf u},p,\Omega,\tilde{p}) \mbox{ s.t. } \tilde{p}|_\Gamma = 0.
        \end{align*}
        Hence, $({\bf v},q)$ satisfies
        \begin{align}
            \label{stationary Stokes: domain integrand variation p}
            \boxed{\nabla\cdot{\bf v} = -\partial_pJ_\Omega({\bf x},{\bf u},\nabla{\bf u},p) \mbox{ in } \Omega.}
        \end{align}
        Plug it back in, obtain
        \begin{align*}
            \int_\Gamma \tilde{p}\left[\partial_pJ_\Gamma({\bf x},{\bf u},\nabla{\bf u},p,{\bf n},{\bf t}) - {\bf v}\cdot{\bf n}\right]{\rm d}\Gamma = 0,\ \forall({\bf u},p,\Omega,\tilde{p}).
        \end{align*}
        Thus, $({\bf v},q)$ satisfies
        \begin{align}
            \label{stationary Stokes: boundary integrand variation p}
            \boxed{{\bf v}\cdot{\bf n} = \partial_pJ_\Gamma({\bf x},{\bf u},\nabla{\bf u},p,{\bf n},{\bf t}) \mbox{ on } \Gamma.}
        \end{align}
        \item Assume that $({\bf v},q)$ satisfies \eqref{stationary Stokes: domain integrand variation p} and \eqref{stationary Stokes: boundary integrand variation p}, \eqref{Euler-Lagrange equation for stationary Stokes} then becomes
        \begin{align*}
            &\int_\Omega \left[\nabla_{\bf u}J_\Omega({\bf x},{\bf u},\nabla{\bf u},p) - \nabla\cdot(\nabla_{\nabla{\bf u}}J_\Omega({\bf x},{\bf u},\nabla{\bf u},p)) + \Delta{\bf v} + \nabla q\right]\cdot\tilde{\bf u} {\rm d}{\bf x}\nonumber\\
            &+ \int_\Gamma \nabla_{\nabla{\bf u}}J_\Gamma({\bf x},{\bf u},\nabla{\bf u},p,{\bf n},{\bf t}):\nabla\tilde{\bf u} + \partial_{\bf n}\tilde{\bf u}\cdot{\bf v}{\rm d}\Gamma = 0,\ \forall({\bf u},p,\Omega,\tilde{\bf u},\tilde{p}) \mbox{ s.t. } \tilde{\bf u}|_\Gamma = {\bf 0}.
        \end{align*}
        Choose $\tilde{\bf u}$ varying s.t. $\tilde{\bf u}_\Gamma = {\bf 0}$ and $\nabla\tilde{\bf u}|_\Gamma = {\bf 0}_{N\times N}$, the last equality yields
        \begin{align*}
            \int_\Omega \left[\nabla_{\bf u}J_\Omega({\bf x},{\bf u},\nabla{\bf u},p) - \nabla\cdot(\nabla_{\nabla{\bf u}}J_\Omega({\bf x},{\bf u},\nabla{\bf u},p)) + \Delta{\bf v} + \nabla q\right]\cdot\tilde{\bf u} {\rm d}{\bf x} = 0,\ \forall({\bf u},p,\Omega,\tilde{\bf u}) \mbox{ s.t. } \tilde{\bf u}|_\Gamma = {\bf 0},\ \nabla\tilde{\bf u}|_\Gamma = {\bf 0}_{N\times N}.
        \end{align*}
        Hence, $({\bf v},q)$ satisfies
        \begin{align}
            \label{stationary Stokes: domain integrand variation u}
            \boxed{\Delta{\bf v} + \nabla q = -\nabla_{\bf u}J_\Omega({\bf x},{\bf u},\nabla{\bf u},p) + \nabla\cdot(\nabla_{\nabla{\bf u}}J_\Omega({\bf x},{\bf u},\nabla{\bf u},p)) \mbox{ in } \Omega.}
        \end{align}
        Plug it back in, obtain
        \begin{align*}
            \int_\Gamma \nabla_{\nabla{\bf u}}J_\Gamma({\bf x},{\bf u},\nabla{\bf u},p,{\bf n},{\bf t}):\nabla\tilde{\bf u} + \partial_{\bf n}\tilde{\bf u}\cdot{\bf v}{\rm d}\Gamma = 0,\ \forall({\bf u},p,\Omega,\tilde{\bf u}) \mbox{ s.t. } \tilde{\bf u}|_\Gamma = {\bf 0}.
        \end{align*}
        Note that
        \begin{align*}
            \partial_{\bf n}\tilde{\bf u}\cdot{\bf v} = \sum_{i=1}^N \partial_{\bf n}\tilde{u}_iv_i = \sum_{i=1}^N v_i\nabla\tilde{u}_i\cdot{\bf n} = \sum_{i=1}^N\sum_{j=1}^N v_i\partial_{x_j}\tilde{u}_in_j = \sum_{i=1}^N\sum_{j=1}^N (n_iv_j)\partial_{x_i}\tilde{u}_j = ({\bf n}\otimes{\bf v}):\nabla\tilde{\bf u} = ({\bf n}{\bf v}^\top):\nabla\tilde{\bf u},
        \end{align*}
        hence the last integral equality can be rewritten as follows:
        \begin{align*}
            \int_\Gamma \left[\nabla_{\nabla{\bf u}}J_\Gamma({\bf x},{\bf u},\nabla{\bf u},p,{\bf n},{\bf t}) + {\bf n}\otimes{\bf v}\right]:\nabla\tilde{\bf u}{\rm d}\Gamma = 0,\ \forall({\bf u},p,\Omega,\tilde{\bf u}) \mbox{ s.t. } \tilde{\bf u}|_\Gamma = {\bf 0}.
        \end{align*}
        Thus, ${\bf v}$ satisfies
        \begin{align*}
            \boxed{{\bf n}\otimes{\bf v} = -\nabla_{\nabla{\bf u}}J_\Gamma({\bf x},{\bf u},\nabla{\bf u},p,{\bf n},{\bf t}) \mbox{ on } \Gamma.}
        \end{align*}
    \end{itemize}
    Gather all boxed equations just derived, we obtain the following ``pre-adjoint'' equations for \eqref{stationary Stokes}:
    \begin{equation}
        \label{pre-adjoint of stationary Stokes}
        \tag{pre-adj-S}
        \boxed{\left\{\begin{split}
            \Delta{\bf v} + \nabla q &= -\nabla_{\bf u}J_\Omega({\bf x},{\bf u},\nabla{\bf u},p) + \nabla\cdot(\nabla_{\nabla{\bf u}}J_\Omega({\bf x},{\bf u},\nabla{\bf u},p)) &&\mbox{ in } \Omega,\\
            \nabla\cdot{\bf v} &= -\partial_pJ_\Omega({\bf x},{\bf u},\nabla{\bf u},p) &&\mbox{ in } \Omega,\\
            {\bf v}\cdot{\bf n} &= \partial_pJ_\Gamma({\bf x},{\bf u},\nabla{\bf u},p,{\bf n},{\bf t}) &&\mbox{ on } \Gamma,\\
            {\bf n}\otimes{\bf v} &= -\nabla_{\nabla{\bf u}}J_\Gamma({\bf x},{\bf u},\nabla{\bf u},p,{\bf n},{\bf t}) &&\mbox{ on } \Gamma.
        \end{split}\right.}
    \end{equation}

    \begin{remark}
        It should be emphasized that the boundary conditions of \eqref{pre-adjoint of stationary Stokes} seems to be overdetermined and an a priori assumption on $J_\Gamma$ should be made. Indeed, the last equation of \eqref{pre-adjoint of stationary Stokes} reads
        \begin{align}
            \label{last equation of pre-adjoint of stationary Stokes}
            n_iv_j = -\partial_{\partial_{x_i}u_j}J_\Gamma({\bf x},{\bf u},\nabla{\bf u},p,{\bf n},{\bf t}) \mbox{ on } \Gamma,\ \forall i,j = 1,\ldots,N,
        \end{align}
        hence
        \begin{align*}
            {\bf v}\cdot{\bf n} = \sum_{i=1}^N v_in_i = \sum_{i=1}^N -\partial_{\partial_{x_i}u_i}J_\Gamma({\bf x},{\bf u},\nabla{\bf u},p,{\bf n},{\bf t}),
        \end{align*}
        and thus $J_\Gamma$ must satisfy
        \begin{align*}
            \boxed{-\operatorname{tr}(\nabla_{\nabla{\bf u}}J_\Gamma({\bf x},{\bf u},\nabla{\bf u},p,{\bf n},{\bf t})) = \partial_pJ_\Gamma({\bf x},{\bf u},\nabla{\bf u},p,{\bf n},{\bf t}).}
        \end{align*}
        Another consequence of \eqref{last equation of pre-adjoint of stationary Stokes} is as follows:
        \begin{align*}
            &\forall{\bf x}\in\Gamma,\ \forall i = 1,\ldots,N:\ n_i({\bf x}) = 0\Rightarrow\partial_{\partial_{x_i}u_j}J_\Gamma({\bf x},{\bf u},\nabla{\bf u},p,{\bf n},{\bf t}) = 0,\ \forall j = 1,\ldots,N,\\
            &\forall{\bf x}\in\Gamma,\ \forall i,j = 1,\ldots,N:\ \partial_{\partial_{x_i}u_j}J_\Gamma({\bf x},{\bf u},\nabla{\bf u},p,{\bf n},{\bf t}) = 0\Rightarrow((n_i({\bf x}) = 0)\vee(v_j({\bf x}) = 0)).
        \end{align*}
        At any point ${\bf x}\in\Gamma$ s.t. $n_i({\bf x})\ne 0$ for all $i = 1,\ldots,N$, the following equation holds there:
        \begin{align*}
            v_j = -\frac{1}{n_i}\partial_{\partial_{x_i}u_j}J_\Gamma({\bf x},{\bf u},\nabla{\bf u},p,{\bf n},{\bf t}) = -\frac{1}{n_j}\partial_{\partial_{x_j}u_j}J_\Gamma({\bf x},{\bf u},\nabla{\bf u},p,{\bf n},{\bf t}),\ \forall i,j = 1,\ldots,N,
        \end{align*}
        Hence, if we denote $\hat{j}_{\Gamma,j}({\bf x})\coloneqq\partial_{\partial_{x_j}u_j}J_\Gamma({\bf x},{\bf u},\nabla{\bf u},p,{\bf n},{\bf t})$ for all $j = 1,\ldots,N$ and $\hat{\bf j}_\Gamma({\bf x}) = [\hat{j}_{\Gamma,1}({\bf x}),\ldots,\hat{j}_{\Gamma,N}({\bf x})]^\top$, then
        \begin{align*}
            \partial_{\partial_{x_i}u_j}J_\Gamma({\bf x},{\bf u},\nabla{\bf u},p,{\bf n},{\bf t}) = \frac{n_i({\bf x})}{n_j({\bf x})}\partial_{\partial_{x_j}u_j}J_\Gamma({\bf x},{\bf u},\nabla{\bf u},p,{\bf n},{\bf t}) = \frac{n_i({\bf x})}{n_j({\bf }x)}\hat{j}_{\Gamma,j}({\bf x}),\ \forall i,j = 1,\ldots,N,
        \end{align*}
        and thus
        \begin{align*}
            \nabla_{\nabla{\bf u}}J_\Gamma({\bf x},{\bf u},\nabla{\bf u},p,{\bf n},{\bf t}) = \left(\frac{n_i({\bf x})}{n_j({\bf x})}\hat{j}_{\Gamma,j}({\bf x})\right)_{i,j=1}^N,\ \forall{\bf x}\in\Gamma \mbox{ s.t. } n_j({\bf x})\ne 0,\ \forall j = 1,\ldots,N.
        \end{align*}
        Note that if $n_j({\bf x}) = 0$, then $\partial_{\partial_{x_j}u_i}J_\Gamma({\bf x},{\bf u},\nabla{\bf u},p,{\bf n},{\bf t}) = 0$ for all $i = 1,\ldots,N$, and in particular, $\partial_{\partial_{x_j}u_j}J_\Gamma({\bf x},{\bf u},\nabla{\bf u},p,{\bf n},{\bf t}) = 0$, i.e., $\hat{j}_{\Gamma,j}({\bf x}) = 0$
        
        At last, we also need an assumption of our general cost function \eqref{cost functional for stationary Stokes} for which the divergence theorem for ${\bf v}$ will hold, i.e.:
        \begin{align*}
            \int_\Omega \nabla\cdot{\bf v}{\rm d}{\bf x} = \int_\Gamma {\bf v}\cdot{\bf n}{\rm d}\Gamma.
        \end{align*}
        Combining this with \eqref{pre-adjoint of stationary Stokes}, we obtain
        \begin{align*}
            -\int_\Omega \partial_pJ_\Omega({\bf x},{\bf u},\nabla{\bf u},p){\rm d}{\bf x} = \int_\Gamma \partial_pJ_\Gamma({\bf x},{\bf u},\nabla{\bf u},p,{\bf n},{\bf t}){\rm d}\Gamma.
        \end{align*}
    \end{remark}    
    In summary, the cost function \eqref{cost functional for stationary Stokes} must satisfy the following assumption
    
    \begin{assumption}[Assumptions on \eqref{cost functional for stationary Stokes}]
        \label{assumptions on cost functional for stationary Stokes}
        \begin{itemize}
            \item[(i)] $-\operatorname{tr}(\nabla_{\nabla{\bf u}}J_\Gamma({\bf x},{\bf u},\nabla{\bf u},p,{\bf n},{\bf t})) = \partial_pJ_\Gamma({\bf x},{\bf u},\nabla{\bf u},p,{\bf n},{\bf t})$.
            \item[(ii)] $\forall{\bf x}\in\Gamma$, $\forall i = 1,\ldots,N$: $n_i({\bf x}) = 0\Rightarrow\partial_{\partial_{x_i}u_j}J_\Gamma({\bf x},{\bf u},\nabla{\bf u},p,{\bf n},{\bf t}) = 0$, $\forall j = 1,\ldots,N$.
            \item[(iii)] There exists a vector function $\hat{\bf j}_\Gamma({\bf x}) = [\hat{j}_{\Gamma,1}({\bf x}),\ldots,\hat{j}_{\Gamma,N}({\bf x})]^\top$ s.t.
            \begin{align*}
                \nabla_{\nabla{\bf u}}J_\Gamma({\bf x},{\bf u},\nabla{\bf u},p,{\bf n},{\bf t}) = \left(\frac{n_i({\bf x})}{n_j({\bf x})}\hat{j}_{\Gamma,j}({\bf x})\right)_{i,j=1}^N,\ \forall{\bf x}\in\Gamma \mbox{ s.t. } n_j({\bf x})\ne 0,\ \forall j = 1,\ldots,N.
            \end{align*}
            \item[(iv)] (Compatibility)
            \begin{align*}
                -\int_\Omega \partial_pJ_\Omega({\bf x},{\bf u},\nabla{\bf u},p){\rm d}{\bf x} = \int_\Gamma \partial_pJ_\Gamma({\bf x},{\bf u},\nabla{\bf u},p,{\bf n},{\bf t}){\rm d}\Gamma.
            \end{align*}
        \end{itemize}
    \end{assumption}
    Under Assumption \ref{assumptions on cost functional for stationary Stokes}, \eqref{pre-adjoint of stationary Stokes} becomes the following adjoint equations of \eqref{stationary Stokes}:
    \begin{equation}
        \label{adjoint of stationary Stokes}
        \tag{adj-S}
        \boxed{\left\{\begin{split}
                \Delta{\bf v} + \nabla q &= -\nabla_{\bf u}J_\Omega({\bf x},{\bf u},\nabla{\bf u},p) + \nabla\cdot(\nabla_{\nabla{\bf u}}J_\Omega({\bf x},{\bf u},\nabla{\bf u},p)) &&\mbox{ in } \Omega,\\
                \nabla\cdot{\bf v} &= -\partial_pJ_\Omega({\bf x},{\bf u},\nabla{\bf u},p) &&\mbox{ in } \Omega,\\
                v_i({\bf x}) &= \left\{\begin{split}
                    -\frac{1}{n_i({\bf x})}\partial_{\partial_{x_i}u_i}J_\Gamma({\bf x},{\bf u},\nabla{\bf u},p,{\bf n},{\bf t})& &&\mbox{ if } n_i({\bf x})\ne 0,\\
                    0& &&\mbox{ if } n_i({\bf x}) = 0,
                \end{split}\right.\ \forall i = 1,\ldots,N, &&\mbox{ on } \Gamma.
            \end{split}\right.}
    \end{equation}
    \item \textbf{Case $\delta_{\mathcal{L}} = 1$.} This means to ``deactivate'' the boundary-condition constraint ${\bf u} = {\bf u}_\Gamma$ on $\Gamma$, so it will be
penalized by the extended Lagrangian $\mathcal{L}$.
    
    We now deduce the adjoint equations of \eqref{stationary Stokes} from \eqref{Euler-Lagrange equation for stationary Stokes} as follows:
    \begin{itemize}
        \item Choose $\tilde{\bf u} = {\bf 0}$ in $\overline{\Omega}$, \eqref{Euler-Lagrange equation for stationary Stokes} then becomes
        \begin{align*}
            \int_\Omega \tilde{p}\left[\partial_pJ_\Omega({\bf x},{\bf u},\nabla{\bf u},p) + \nabla\cdot{\bf v}\right]{\rm d}{\bf x} + \int_\Gamma  \tilde{p}\left[\partial_pJ_\Gamma({\bf x},{\bf u},\nabla{\bf u},p,{\bf n},{\bf t}) - {\bf v}\cdot{\bf n}\right]{\rm d}\Gamma = 0,\ \forall({\bf u},p,\Omega,\tilde{p}).
        \end{align*}
        Then choose $\tilde{p}$ varying s.t. $\tilde{p}|_\Gamma = 0$, then the last inequality yields
        \begin{align*}
            \int_\Omega \tilde{p}\left[\partial_pJ_\Omega({\bf x},{\bf u},\nabla{\bf u},p) + \nabla\cdot{\bf v}\right]{\rm d}{\bf x} = 0,\ \forall({\bf u},p,\Omega,\tilde{p}) \mbox{ s.t. } \tilde{p}|_\Gamma = 0.
        \end{align*}
        Hence, $({\bf v},q)$ satisfies
        \begin{align}
            \label{extended domain integrand variation p: stationary Stokes}
            \boxed{\nabla\cdot{\bf v} = -\partial_pJ_\Omega({\bf x},{\bf u},\nabla{\bf u},p) \mbox{ in } \Omega.}
        \end{align}
        Plug it back in, obtain
        \begin{align*}
            \int_\Gamma  \tilde{p}\left[\partial_pJ_\Gamma({\bf x},{\bf u},\nabla{\bf u},p,{\bf n},{\bf t}) - {\bf v}\cdot{\bf n}\right]{\rm d}\Gamma = 0,\ \forall({\bf u},p,\Omega,\tilde{p}).
        \end{align*}
        and thus $({\bf v},q)$ also satisfies
        \begin{align}
            \label{extended boundary integrand variation p: stationary Stokes}
            \boxed{{\bf v}\cdot{\bf n} = \partial_pJ_\Gamma({\bf x},{\bf u},\nabla{\bf u},p,{\bf n},{\bf t}) \mbox{ on } \Gamma.}
        \end{align}
        \item Assume that $({\bf v},q)$ satisfies \eqref{extended domain integrand variation p: stationary Stokes} and \eqref{extended boundary integrand variation p: stationary Stokes}, \eqref{Euler-Lagrange equation for stationary Stokes} then becomes
        \begin{align*}
            &\int_\Omega \left[\nabla_{\bf u}J_\Omega({\bf x},{\bf u},\nabla{\bf u},p) - \nabla\cdot(\nabla_{\nabla{\bf u}}J_\Omega({\bf x},{\bf u},\nabla{\bf u},p)) + \Delta{\bf v} + \nabla q\right]\cdot\tilde{\bf u}{\rm d}{\bf x}\nonumber\\
            &+ \int_\Gamma \left[\nabla_{\nabla{\bf u}}J_\Omega({\bf x},{\bf u},\nabla{\bf u},p)\cdot{\bf n} + \nabla_{\bf u}J_\Gamma({\bf x},{\bf u},\nabla{\bf u},p,{\bf n},{\bf t}) - \partial_{\bf n}{\bf v} - q{\bf n} - {\bf v}_\Gamma\right]\cdot\tilde{\bf u}{\rm d}\Gamma\nonumber\\
            &+ \int_\Gamma \nabla_{\nabla{\bf u}}J_\Gamma({\bf x},{\bf u},\nabla{\bf u},p,{\bf n},{\bf t}):\nabla\tilde{\bf u} + \partial_{\bf n}\tilde{\bf u}\cdot{\bf v}{\rm d}\Gamma = 0,\ \forall({\bf u},p,\Omega,\tilde{\bf u},\tilde{p}).
        \end{align*}
        Choose $\tilde{\bf u}$ varying s.t. $\tilde{\bf u}|_\Gamma = {\bf 0}$ and $\nabla\tilde{\bf u}|_\Gamma = {\bf 0}_{N\times N}$, the last equality yields
        \begin{align*}
            \int_\Omega \left[\nabla_{\bf u}J_\Omega({\bf x},{\bf u},\nabla{\bf u},p) - \nabla\cdot(\nabla_{\nabla{\bf u}}J_\Omega({\bf x},{\bf u},\nabla{\bf u},p)) + \Delta{\bf v} + \nabla q\right]\cdot\tilde{\bf u}{\rm d}{\bf x} = 0,\ \forall({\bf u},p,\Omega,\tilde{\bf u}) \mbox{ s.t. } \tilde{\bf u}|_\Gamma = {\bf 0} \mbox{ and } \nabla\tilde{\bf u}|_\Gamma = {\bf 0}_{N\times N}.
        \end{align*}
        Hence, $({\bf v},q)$ satisfies
        \begin{align*}
            \boxed{\Delta{\bf v} + \nabla q = -\nabla_{\bf u}J_\Omega({\bf x},{\bf u},\nabla{\bf u},p) + \nabla\cdot(\nabla_{\nabla{\bf u}}J_\Omega({\bf x},{\bf u},\nabla{\bf u},p)) \mbox{ in } \Omega.}
        \end{align*}
        Plug it back in, obtain
        \begin{align*}
            &\int_\Gamma \left[\nabla_{\nabla{\bf u}}J_\Omega({\bf x},{\bf u},\nabla{\bf u},p)\cdot{\bf n} + \nabla_{\bf u}J_\Gamma({\bf x},{\bf u},\nabla{\bf u},p,{\bf n},{\bf t}) - \partial_{\bf n}{\bf v} - q{\bf n} - {\bf v}_\Gamma\right]\cdot\tilde{\bf u}{\rm d}\Gamma\nonumber\\
            &+ \int_\Gamma \nabla_{\nabla{\bf u}}J_\Gamma({\bf x},{\bf u},\nabla{\bf u},p,{\bf n},{\bf t}):\nabla\tilde{\bf u} + \partial_{\bf n}\tilde{\bf u}\cdot{\bf v}{\rm d}\Gamma = 0,\ \forall({\bf u},p,\Omega,\tilde{\bf u},\tilde{p}).
        \end{align*}
        Choose $\tilde{\bf u}$ varying s.t. $\tilde{\bf u}|_\Gamma = {\bf 0}$, the last equality then becomes
        \begin{align*}
            \int_\Gamma \left[\nabla_{\nabla{\bf u}}J_\Gamma({\bf x},{\bf u},\nabla{\bf u},p,{\bf n},{\bf t}) + {\bf n}\otimes{\bf v}\right]:\nabla\tilde{\bf u}{\rm d}\Gamma = 0,\ \forall({\bf u},p,\Omega,\tilde{\bf u}) \mbox{ s.t. } \tilde{\bf u}|_\Gamma = {\bf 0}.
        \end{align*}
        Thus, ${\bf v}$ satisfies
        \begin{align*}
            \boxed{{\bf n}\otimes{\bf v} = -\nabla_{\nabla{\bf u}}J_\Gamma({\bf x},{\bf u},\nabla{\bf u},p,{\bf n},{\bf t}) \mbox{ on } \Gamma.}
        \end{align*}
        Plug it back in again, obtain
        \begin{align*}
            \int_\Gamma \left[\nabla_{\nabla{\bf u}}J_\Omega({\bf x},{\bf u},\nabla{\bf u},p)\cdot{\bf n} + \nabla_{\bf u}J_\Gamma({\bf x},{\bf u},\nabla{\bf u},p,{\bf n},{\bf t}) - \partial_{\bf n}{\bf v} - q{\bf n} - {\bf v}_\Gamma\right]\cdot\tilde{\bf u}{\rm d}\Gamma = 0,\ \forall({\bf u},p,\Omega,\tilde{\bf u}).
        \end{align*}
        Hence, $({\bf v},q)$ satisfies
        \begin{align*}
            \boxed{\partial_{\bf n}{\bf v} + q{\bf n} + \delta_{\mathcal{L}}{\bf v}_\Gamma = \nabla_{\nabla{\bf u}}J_\Omega({\bf x},{\bf u},\nabla{\bf u},p)\cdot{\bf n} + \nabla_{\bf u}J_\Gamma({\bf x},{\bf u},\nabla{\bf u},p,{\bf n},{\bf t}) \mbox{ on } \Gamma.}
        \end{align*}
    \end{itemize}
    Gather all boxed equations just derived, we obtain the following ``pre-adjoint'' equations for \eqref{stationary Stokes}:
    \begin{equation}
        \label{exteded pre-adjoint of stationary Stokes}
        \tag{ex-pre-adj-S}
        \boxed{\left\{\begin{split}
                \Delta{\bf v} + \nabla q &= -\nabla_{\bf u}J_\Omega({\bf x},{\bf u},\nabla{\bf u},p) + \nabla\cdot(\nabla_{\nabla{\bf u}}J_\Omega({\bf x},{\bf u},\nabla{\bf u},p)) &&\mbox{ in } \Omega,\\
                \nabla\cdot{\bf v} &= -\partial_pJ_\Omega({\bf x},{\bf u},\nabla{\bf u},p) &&\mbox{ in } \Omega,\\
                {\bf v}\cdot{\bf n} &= \partial_pJ_\Gamma({\bf x},{\bf u},\nabla{\bf u},p,{\bf n},{\bf t}) &&\mbox{ on } \Gamma,\\
                {\bf n}\otimes{\bf v} &= -\nabla_{\nabla{\bf u}}J_\Gamma({\bf x},{\bf u},\nabla{\bf u},p,{\bf n},{\bf t}) &&\mbox{ on } \Gamma,\\
                \partial_{\bf n}{\bf v} + q{\bf n} + {\bf v}_\Gamma &= \nabla_{\nabla{\bf u}}J_\Omega({\bf x},{\bf u},\nabla{\bf u},p)\cdot{\bf n} + \nabla_{\bf u}J_\Gamma({\bf x},{\bf u},\nabla{\bf u},p,{\bf n},{\bf t}) &&\mbox{ on } \Gamma.
            \end{split}\right.}
    \end{equation}
    Similar as the previous case, under Assumption \ref{assumptions on cost functional for stationary Stokes}, we obtain the following adjoint equations of \eqref{stationary Stokes}:
    \begin{equation}
        \label{exteded adjoint of stationary Stokes}
        \tag{ex-adj-S}
        \boxed{\left\{\begin{split}
            \Delta{\bf v} + \nabla q &= -\nabla_{\bf u}J_\Omega({\bf x},{\bf u},\nabla{\bf u},p) + \nabla\cdot(\nabla_{\nabla{\bf u}}J_\Omega({\bf x},{\bf u},\nabla{\bf u},p)) &&\mbox{ in } \Omega,\\
            \nabla\cdot{\bf v} &= -\partial_pJ_\Omega({\bf x},{\bf u},\nabla{\bf u},p) &&\mbox{ in } \Omega,\\
            v_i({\bf x}) &= \left\{\begin{split}
                -\frac{1}{n_i({\bf x})}\partial_{\partial_{x_i}u_i}J_\Gamma({\bf x},{\bf u},\nabla{\bf u},p,{\bf n},{\bf t})& &&\mbox{ if } n_i({\bf x})\ne 0,\\
                0& &&\mbox{ if } n_i({\bf x}) = 0,
            \end{split}\right.\ \forall i = 1,\ldots,N, &&\mbox{ on } \Gamma,\\
        {\bf v}_\Gamma &= \nabla_{\nabla{\bf u}}J_\Omega({\bf x},{\bf u},\nabla{\bf u},p)\cdot{\bf n} + \nabla_{\bf u}J_\Omega({\bf x},{\bf u},\nabla{\bf u},p,{\bf n},{\bf t}) - \partial_{\bf n}{\bf v} - q{\bf n} &&\mbox{ on } \Gamma.
        \end{split}\right.}
    \end{equation}
\end{itemize}

\begin{remark}
    It seems that using standard Lagrangian is enough for \eqref{stationary Stokes}, there is no need to use the extended Lagrangian.
\end{remark}

\section{Shape derivatives of \eqref{stationary Stokes}-constrained \eqref{cost functional for stationary Stokes}}
To calculate the shape derivatives of \eqref{cost functional for stationary Stokes} under the constraint state equation \eqref{stationary Stokes}, we consider the following \textit{perturbed cost functional}
\begin{align}
    \label{perturbed cost functional for stationary Stokes}
    J({\bf u}_t,p_t,\Omega_t)\coloneqq\int_{\Omega_t} J_\Omega({\bf x},{\bf u}_t,\nabla{\bf u}_t,p_t){\rm d}{\bf x} + \int_{\Gamma_t} J_\Gamma({\bf x},{\bf u}_t,\nabla{\bf u}_t,p_t,{\bf n}_t,{\bf t}_t){\rm d}\Gamma_t,
\end{align}
where $({\bf u}_t,p_t)$ denotes the strong/classical solution of \eqref{stationary Stokes} on the perturbed domain $\Omega_t\coloneqq T_t(V)(\Omega)$, i.e.:
\begin{equation}
    \label{perturbed stationary Stokes}
    \tag{ptb-S}
    \left\{\begin{split}
        -\nu\Delta{\bf u}_t + \nabla p_t &= {\bf f} &&\mbox{ in } \Omega_t,\\
        \nabla\cdot{\bf u}_t &= f_{\rm div} &&\mbox{ in } \Omega_t,\\
        \gamma_0{\bf u}_t = {\bf u}_\Gamma \mbox{ i.e., } {\bf u}_t &= {\bf u}_\Gamma &&\mbox{ on } \Gamma_t,
    \end{split}\right.
\end{equation}
where $\Gamma_t\coloneqq\partial\Omega_t$.

Now subtracting \eqref{perturbed stationary Stokes} to \eqref{stationary Stokes} side by side, and taking $\lim_{t\downarrow 0}$, obtain:

%------------------------------------------------------------------------------%

\chapter{Shape Optimization for Stationary Incompressible Viscous Navier-Stokes Equations}

\section{A general framework}
In this chapter, we will design a framework for shape optimization for the PDEs of the following form (with assumption that the density $\rho = \mbox{const}$):
\begin{equation}
    \label{general stationary fluid dynamics PDEs}
    \tag{gfld}
    \left\{\begin{split}
        {\bf P}({\bf x},{\bf u},\nabla{\bf u},\Delta{\bf u},p,\nabla p) &= {\bf f}({\bf x},{\bf u},\nabla{\bf u},p) &&\mbox{ in } \Omega,\\
        -\nabla\cdot{\bf u} &= f_{\rm div}({\bf x},{\bf u},\nabla{\bf u},p) &&\mbox{ in } \Omega,\\
        {\bf Q}({\bf x},{\bf u},\nabla{\bf u},p,{\bf n},{\bf t}) &= {\bf f}_{\rm bc}({\bf x}) &&\mbox{ on } \Gamma,
    \end{split}\right.
\end{equation}
where ${\bf P}(\cdot,\ldots,\cdot) = (P_1,\ldots,P_N)(\cdot,\ldots,\cdot)$, ${\bf f}(\cdot,\ldots,\cdot) = (f_1,\ldots,f_N)(\cdot,\ldots,\cdot)$ and ${\bf Q}(\cdot,\ldots,\cdot)$ denote the main PDE (e.g., here, NSEs), the source terms, and the set of boundary conditions, respectively.

\subsection{Weak formulation of \eqref{general stationary fluid dynamics PDEs}}
Test both sides of the 1st equation of \eqref{general stationary fluid dynamics PDEs} with a test function ${\bf v}$ and those of the 2nd one with a test function $q$ over $\Omega$:
\begin{equation}
    \label{weak formulation of general stationary fluid dynamics PDEs}
    \tag{wf-gfld}
    \left\{\begin{split}
        \int_\Omega {\bf P}({\bf x},{\bf u},\nabla{\bf u},\Delta{\bf u},p,\nabla p)\cdot{\bf v}{\rm d}{\bf x} &= \int_\Omega {\bf f}({\bf x},{\bf u},\nabla{\bf u},p)\cdot{\bf v}{\rm d}{\bf x},\\
        -\int_\Omega q\nabla\cdot{\bf u}{\rm d}{\bf x} &= \int_\Omega qf_{\rm div}({\bf x},{\bf u},\nabla{\bf u},p){\rm d}{\bf x},
    \end{split}\right.
\end{equation}
and then integrate by parts all the 2nd-order terms w.r.t. ${\bf u}$ and all the 1st-order terms w.r.t. $p$ in the 1st equation of \eqref{weak formulation of general stationary fluid dynamics PDEs}.

Plugging the boundary conditions (i.e., the 3rd equation of \eqref{general stationary fluid dynamics PDEs}) into the equations just obtained to embed them into the weak formulation.

\subsection{Cost functionals}
A general cost functional for \eqref{general stationary fluid dynamics PDEs} is given by
\begin{align}
    \label{cost functional of general stationary fluid dynamics PDEs}
    \tag{cost-gfld}
    J({\bf u},p,\Omega)\coloneqq\int_\Omega J_\Omega({\bf x},{\bf u},\nabla{\bf u},p){\rm d}{\bf x} + \int_\Gamma J_\Gamma({\bf x},{\bf u},\nabla{\bf u},p,{\bf n},{\bf t}){\rm d}\Gamma.
\end{align}

\subsection{Lagrangian \& extended Lagrangian}
To derive the adjoint equations for \eqref{general stationary fluid dynamics PDEs}, 1st introduce the following Lagrangian (see, e.g., \cite{Troltzsch2010}):
\begin{align}
    \label{Lagrangian for general stationary fluid dynamics PDEs}
    \tag{$L$-gfld}
    L({\bf u},p,\Omega,{\bf v},q)\coloneqq J({\bf u},p,\Omega) + \int_\Omega -\left({\bf P}({\bf x},{\bf u},\nabla{\bf u},\Delta{\bf u},p,\nabla p) - {\bf f}({\bf x},{\bf u},\nabla{\bf u},p)\right)\cdot{\bf v} + q\left(\nabla\cdot{\bf u} + f_{\rm div}({\bf x},{\bf u},\nabla{\bf u},p)\right){\rm d}{\bf x},
\end{align}
and the following extended Lagrangian:
\begin{align}
    \label{extended Lagrangian for general stationary fluid dynamics PDEs}
    \tag{$\mathcal{L}$-gfld}
    \mathcal{L}({\bf u},p,\Omega,{\bf v},q,{\bf v}_{\rm bc})\coloneqq&\, L({\bf u},p,\Omega,{\bf v},q)
    - \int_\Gamma \left({\bf Q}({\bf x},{\bf u},\nabla{\bf u},p,{\bf n},{\bf t}) - {\bf f}_{\rm bc}({\bf x})\right)\cdot{\bf v}_{\rm bc}{\rm d}\Gamma\\
    =&\, J({\bf u},p,\Omega) + \int_\Omega -\left({\bf P}({\bf x},{\bf u},\nabla{\bf u},\Delta{\bf u},p,\nabla p) - {\bf f}({\bf x},{\bf u},\nabla{\bf u},p)\right)\cdot{\bf v} + q\left(\nabla\cdot{\bf u} + f_{\rm div}({\bf x},{\bf u},\nabla{\bf u},p)\right){\rm d}{\bf x}\nonumber\\
    &- \int_\Gamma \left({\bf Q}({\bf x},{\bf u},\nabla{\bf u},p,{\bf n},{\bf t}) - {\bf f}_{\rm bc}({\bf x})\right)\cdot{\bf v}_{\rm bc}{\rm d}\Gamma,\nonumber
\end{align}
where ${\bf v}$, $q$, ${\bf v}_{\rm bc}$ are Lagrange multipliers.

\subsection{Shape optimization problems}
Here are 3 different shape optimization problems associated with \eqref{cost functional of general stationary fluid dynamics PDEs}, \eqref{Lagrangian for general stationary fluid dynamics PDEs}, and \eqref{extended Lagrangian for general stationary fluid dynamics PDEs}, respectively:
\begin{align*}
    &\min_{\Omega\in\mathcal{O}_{\rm ad}} J({\bf u},p,\Omega) \mbox{ s.t. } ({\bf u},p) \mbox{ solves \eqref{general stationary fluid dynamics PDEs}},\\
    &\min_{\Omega\in\mathcal{O}_{\rm ad}} L({\bf u},p,\Omega,{\bf v},q) \mbox{ s.t. } ({\bf u},p) \mbox{ satisfies } {\bf Q}({\bf x},{\bf u},\nabla{\bf u},p,{\bf n},{\bf t}) = {\bf f}_{\rm bc}({\bf x}) \mbox{ on } \Gamma,\\
    &\min_{\Omega\in\mathcal{O}_{\rm ad}} \mathcal{L}({\bf u},p,\Omega,{\bf v},q,{\bf v}_{\rm bc}) \mbox{ with } ({\bf u},p) \mbox{ unconstrained}.
\end{align*}
In the 2nd optimization problem, the main PDEs (i.e., equations in $\Omega$) are already penalized by implicitly embedded into the Lagrangian \eqref{Lagrangian for general stationary fluid dynamics PDEs}, meanwhile in the 3rd one, both main PDEs and boundary conditions are penalized by implicitly embedded into the extended Lagrangian \eqref{extended Lagrangian for general stationary fluid dynamics PDEs}.
\begin{remark}
    Remind that this is just a formal framework. For a rigorous one, the correct function spaces of both state and adjoint variables need inserting into these optimization problems.
\end{remark}
Here $({\bf u},p)$, $({\bf v},q)$ (also ${\bf v}_{\rm bc}$ for the last shape optimization problem with the extended Lagrangian), and $\Omega$ are the \textit{state variables, adjoint variables}, and \textit{control variable} for these shape optimization problems.

\subsection{Adjoint equations of \eqref{general stationary fluid dynamics PDEs}}
A natural question arises:
\begin{question}
    Should the Lagrangian or the extended Lagrangian be used to derive the adjoint equations?
\end{question}
To answer this question, introduce the following ``mixed'' Lagrangian:
\begin{align}
    \label{mixed Lagrangian for general stationary fluid dynamics PDEs}
    \tag{$L_{\mathcal{L}}$-gfld}
    L_{\mathcal{L}}({\bf u},p,\Omega,{\bf v},q,{\bf v}_{\rm bc})\coloneqq&\, L({\bf u},p,\Omega,{\bf v},q)
    - \delta_{\mathcal{L}}\int_\Gamma \left({\bf Q}({\bf x},{\bf u},\nabla{\bf u},p,{\bf n},{\bf t}) - {\bf f}_{\rm bc}({\bf x})\right)\cdot{\bf v}_{\rm bc}{\rm d}\Gamma\\
    =&\, J({\bf u},p,\Omega) + \int_\Omega -\left({\bf P}({\bf x},{\bf u},\nabla{\bf u},\Delta{\bf u},p,\nabla p) - {\bf f}({\bf x},{\bf u},\nabla{\bf u},p)\right)\cdot{\bf v} + q\left(\nabla\cdot{\bf u} + f_{\rm div}({\bf x},{\bf u},\nabla{\bf u},p)\right){\rm d}{\bf x}\nonumber\\
    &- \delta_{\mathcal{L}}\int_\Gamma \left({\bf Q}({\bf x},{\bf u},\nabla{\bf u},p,{\bf n},{\bf t}) - {\bf f}_{\rm bc}({\bf x})\right)\cdot{\bf v}_{\rm bc}{\rm d}\Gamma, \mbox{ where } \delta_{\mathcal{L}}\in\{0,1\}\nonumber.
\end{align}

\begin{remark}
    No need to assume the ``switch'' between Lagrangian and extended Lagrangian is a real number, i.e. $\delta_{\mathcal{L}}\in\mathbb{R}$, since the scaling is already embedded in the Lagrange multiplier ${\bf v}_{\rm bc}$.
\end{remark}
Hence,
\begin{equation*}
    L_{\mathcal{L}}({\bf u},p,\Omega,{\bf v},q,{\bf v}_{\rm bc}) = \left\{\begin{split}
        &L({\bf u},p,\Omega,{\bf v},q) &&\mbox{ if } \delta_{\mathcal{L}} = 0,\\
        &\mathcal{L}({\bf u},p,\Omega,{\bf v},q,{\bf v}_{\rm bc}) &&\mbox{ if } \delta_{\mathcal{L}} = 1.
    \end{split}\right.
\end{equation*}
Then the shape optimization problem reads
\begin{equation*}
    \min_{\Omega\in\mathcal{O}_{\rm ad}} L_{\mathcal{L}}({\bf u},p,\Omega,{\bf v},q,{\bf v}_{\rm bc})\ \left\{\begin{split}
        &\mbox{s.t. } ({\bf u},p) \mbox{ satisfies } {\bf Q}({\bf x},{\bf u},\nabla{\bf u},p,{\bf n},{\bf t}) = {\bf f}_{\rm bc}({\bf x}) \mbox{ on } \Gamma &&\mbox{ if } \delta_{\mathcal{L}} = 0,\\
        &\mbox{with } ({\bf u},p) \mbox{ unconstrained} &&\mbox{ if } \delta_{\mathcal{L}} = 1.
    \end{split}\right.
\end{equation*}
Next, choose the Lagrangian multiplier $({\bf v},q,{\bf v}_{\rm bc})$ s.t. the variation of the chosen Lagrangian (simple/extended/mixed) w.r.t. state variables vanishes, i.e.:
\begin{align*}
    \delta_{({\bf u},p)} L_{\mathcal{L}}({\bf u},p,\Omega,{\bf v},q,{\bf v}_{\rm bc};\tilde{\bf u},\tilde{p}) = 0,\ \forall(\tilde{\bf u},\tilde{p}),
\end{align*}
where
\begin{align*}
    &\delta_{({\bf u},p)} L_{\mathcal{L}}({\bf u},p,\Omega,{\bf v},q,{\bf v}_{\rm bc};\tilde{\bf u},\tilde{p})\\
   \coloneqq&\, \lim_{t\downarrow 0} \frac{1}{t}\left(L_{\mathcal{L}}({\bf u} + t\tilde{\bf u},p + t\tilde{p},\Omega,{\bf v},q,{\bf v}_{\rm bc}) - L_{\mathcal{L}}({\bf u},p,\Omega,{\bf v},q,{\bf v}_{\rm bc})\right)\\
    =&\, \lim_{t\downarrow 0} \frac{1}{t}\left(L_{\mathcal{L}}({\bf u} + t\tilde{\bf u},p + t\tilde{p},\Omega,{\bf v},q,{\bf v}_{\rm bc}) - L_{\mathcal{L}}({\bf u},p + t\tilde{p},\Omega,{\bf v},q,{\bf v}_{\rm bc})\right) + \lim_{t\downarrow 0} \frac{1}{t}\left(L_{\mathcal{L}}({\bf u},p + t\tilde{p},\Omega,{\bf v},q,{\bf v}_{\rm bc}) - L_{\mathcal{L}}({\bf u},p,\Omega,{\bf v},q,{\bf v}_{\rm bc})\right)\\
    =&\, \lim_{t\downarrow 0} \delta_{\bf u} L_{\mathcal{L}}({\bf u},p + t\tilde{p},\Omega,{\bf v},q,{\bf v}_{\rm bc};\tilde{\bf u}) + \delta_p L_{\mathcal{L}}({\bf u},p,\Omega,{\bf v},q,{\bf v}_{\rm bc};\tilde{p}) = \delta_{\bf u} L_{\mathcal{L}}({\bf u},p,\Omega,{\bf v},q,{\bf v}_{\rm bc};\tilde{\bf u}) + \delta_p L_{\mathcal{L}}({\bf u},p,\Omega,{\bf v},q,{\bf v}_{\rm bc};\tilde{p}).
\end{align*}
Provided Gateaux/Fr\'echet differentiability of $L_{\mathcal{L}}$ is guaranteed, in the case $\delta_{\mathcal{L}} = 1$, since the state variables $({\bf u},p)$ is now formally unconstrained \texttt{[added corrected function spaces for its rigorous counterpart]}, the derivative of the Lagrangian w.r.t. $({\bf u},p)$ has to vanish at any optimal point, say e.g. $({\bf u}^\star,p^\star,\Omega^\star)$, i.e.,
\begin{align*}
    D_{({\bf u},p)} L_{\mathcal{L}}({\bf u}^\star,p^\star,\Omega^\star,{\bf v},q,{\bf v}_{\rm bc})(\tilde{\bf u},\tilde{p}) = 0,\ \forall(\tilde{\bf u},\tilde{p}).
\end{align*}
Combine this with
\begin{align*}
    D_{({\bf u},p)} L_{\mathcal{L}}({\bf u},p,\Omega,{\bf v},q,{\bf v}_{\rm bc})(\tilde{\bf u},\tilde{p}) = D_{\bf u}L_{\mathcal{L}} ({\bf u},p,\Omega,{\bf v},q,{\bf v}_{\rm bc})\tilde{\bf u} + D_pL_{\mathcal{L}} ({\bf u},p,\Omega,{\bf v},q,{\bf v}_{\rm bc})\tilde{p},\ \forall(\tilde{\bf u},\tilde{p}),
\end{align*}
then obtain
\begin{align*}
    D_{\bf u}L_{\mathcal{L}} ({\bf u}^\star,p^\star,\Omega^\star,{\bf v},q,{\bf v}_{\rm bc})\tilde{\bf u} + D_pL_{\mathcal{L}} ({\bf u}^\star,p^\star,\Omega^\star,{\bf v},q,{\bf v}_{\rm bc})\tilde{p} = 0,\ \forall(\tilde{\bf u},\tilde{p}).
\end{align*}
Motivated by this stationary equation, choose the adjoint variables/Lagrange multipliers $({\bf v},q,{\bf v}_{\rm bc})$ s.t. 
\begin{align*}
    \boxed{D_{\bf u} L_{\mathcal{L}}({\bf u},p,\Omega,{\bf v},q,{\bf v}_{\rm bc})\tilde{\bf u} + D_pL_{\mathcal{L}}({\bf u},p,\Omega,{\bf v},q,{\bf v}_{\rm bc})\tilde{p} = 0,\ \forall({\bf u},p,\Omega,\tilde{\bf u},\tilde{p}).}
\end{align*}
Expand this more explicitly for all $({\bf u},p,\Omega,\tilde{\bf u},\tilde{p})$:
\begin{align*}
    &\int_\Omega D_{\bf u}\left(J_\Omega({\bf x},{\bf u},\nabla{\bf u},p)\right)\tilde{\bf u} + D_p\left(J_\Omega({\bf x},{\bf u},\nabla{\bf u},p)\right)\tilde{p}{\rm d}{\bf x} + \int_\Gamma D_{\bf u}\left(J_\Gamma({\bf x},{\bf u},\nabla{\bf u},p,{\bf n},{\bf t})\right)\tilde{\bf u} + D_p\left(J_\Gamma({\bf x},{\bf u},\nabla{\bf u},p,{\bf n},{\bf t})\right)\tilde{p}{\rm d}\Gamma\\
    &+ \int_\Omega -D_{\bf u}\left({\bf P}({\bf x},{\bf u},\nabla{\bf u},\Delta{\bf u},p,\nabla p) - {\bf f}({\bf x},{\bf u},\nabla{\bf u},p)\right)\tilde{\bf u}\cdot{\bf v} - D_p\left({\bf P}({\bf x},{\bf u},\nabla{\bf u},\Delta{\bf u},p,\nabla p) - {\bf f}({\bf x},{\bf u},\nabla{\bf u},p)\right)\tilde{p}\cdot{\bf v}\\
    &\hspace{1cm} + qD_{\bf u}\left(\nabla\cdot{\bf u} + f_{\rm div}({\bf x},{\bf u},\nabla{\bf u},p)\right)\tilde{\bf u} + qD_p\left(\nabla\cdot{\bf u} + f_{\rm div}({\bf x},{\bf u},\nabla{\bf u},p)\right)\tilde{p}{\rm d}{\bf x}\\
    &- \delta_{\mathcal{L}}\int_\Gamma D_{\bf u}\left({\bf Q}({\bf x},{\bf u},\nabla{\bf u},p,{\bf n},{\bf t}) - {\bf f}_{\rm bc}({\bf x})\right)\tilde{\bf u}\cdot{\bf v}_{\rm bc} + D_p\left({\bf Q}({\bf x},{\bf u},\nabla{\bf u},p,{\bf n},{\bf t}) - {\bf f}_{\rm bc}({\bf x})\right)\tilde{p}\cdot{\bf v}_{\rm bc}{\rm d}\Gamma,
\end{align*}
and more explicitly:
\begin{align*}
    &\int_\Omega D_{\bf u}J_\Omega({\bf x},{\bf u},\nabla{\bf u},p)\tilde{\bf u} + D_{\nabla{\bf u}}J_\Omega({\bf x},{\bf u},\nabla{\bf u},p)\nabla\tilde{\bf u} + D_pJ_\Omega({\bf x},{\bf u},\nabla{\bf u},p)\tilde{p}{\rm d}{\bf x}\\
    &+ \int_\Gamma D_{\bf u}J_\Gamma({\bf x},{\bf u},\nabla{\bf u},p,{\bf n},{\bf t})\tilde{\bf u} + D_{\nabla{\bf u}}J_\Gamma({\bf x},{\bf u},\nabla{\bf u},p,{\bf n},{\bf t})\nabla\tilde{\bf u} + D_pJ_\Gamma({\bf x},{\bf u},\nabla{\bf u},p,{\bf n},{\bf t})\tilde{p}{\rm d}\Gamma\\
    &+ \int_\Omega -D_{\bf u}{\bf P}({\bf x},{\bf u},\nabla{\bf u},\Delta{\bf u},p,\nabla p)\tilde{\bf u}\cdot{\bf v} - D_{\nabla{\bf u}}{\bf P}({\bf x},{\bf u},\nabla{\bf u},\Delta{\bf u},p,\nabla p)\nabla\tilde{\bf u}\cdot{\bf v} - D_{\Delta{\bf u}}{\bf P}({\bf x},{\bf u},\nabla{\bf u},\Delta{\bf u},p,\nabla p)\Delta\tilde{\bf u}\cdot{\bf v}\\
    &\hspace{1cm} + D_{\bf u}{\bf f}({\bf x},{\bf u},\nabla{\bf u},p)\tilde{\bf u}\cdot{\bf v} + D_{\nabla{\bf u}}{\bf f}({\bf x},{\bf u},\nabla{\bf u},p)\nabla\tilde{\bf u}\cdot{\bf v}\\
    &\hspace{1cm} - D_p{\bf P}({\bf x},{\bf u},\nabla{\bf u},\Delta{\bf u},p,\nabla p)\tilde{p}\cdot{\bf v} - D_{\nabla p}{\bf P}({\bf x},{\bf u},\nabla{\bf u},\Delta{\bf u},p,\nabla p)\nabla\tilde{p}\cdot{\bf v} + D_p{\bf f}({\bf x},{\bf u},\nabla{\bf u},p)\tilde{p}\cdot{\bf v}\\
    &\hspace{1cm} + q\nabla\cdot\tilde{\bf u} + qD_{\bf u}f_{\rm div}({\bf x},{\bf u},\nabla{\bf u},p)\tilde{\bf u} + qD_{\nabla{\bf u}}f_{\rm div}({\bf x},{\bf u},\nabla{\bf u},p)\nabla\tilde{\bf u} + qD_pf_{\rm div}({\bf x},{\bf u},\nabla{\bf u},p)\tilde{p}{\rm d}{\bf x}\\
    &- \delta_{\mathcal{L}}\int_\Gamma D_{\bf u}{\bf Q}({\bf x},{\bf u},\nabla{\bf u},p,{\bf n},{\bf t})\tilde{\bf u}\cdot{\bf v}_{\rm bc} + D_{\nabla{\bf u}}{\bf Q}({\bf x},{\bf u},\nabla{\bf u},p,{\bf n},{\bf t})\nabla\tilde{\bf u}\cdot{\bf v}_{\rm bc} + D_p{\bf Q}({\bf x},{\bf u},\nabla{\bf u},p,{\bf n},{\bf t})\tilde{p}\cdot{\bf v}_{\rm bc}{\rm d}\Gamma = 0,\ \forall({\bf u},p,\Omega,\tilde{\bf u},\tilde{p}).
\end{align*}

\begin{question}
    Integrate by parts which terms?
    
    \textsc{Answer.} All the terms which are the domain integrals containing derivatives of the variations of state variables to transfer their derivatives to the adjoint variables. Roughly speaking:
    \begin{align*}
        \int_\Omega \left\{\nabla\tilde{\bf u},\Delta\tilde{\bf u},\nabla\tilde{p}\right\}\cdot\left\{{\bf v},q\right\}{\rm d}{\bf x}\xrightarrow{\rm i.b.p.}\int_\Omega \left\{\tilde{\bf u},\tilde{p}\right\}\cdot\left\{\nabla{\bf v},\Delta{\bf v},\nabla q\right\}{\rm d}{\bf x} + \mbox{ boundary integrals } \int_\Gamma \cdots{\rm d}\Gamma.
    \end{align*}
    The domain integrals on the r.h.s., after gathered appropriately, will yield the adjoint PDEs, meanwhile the boundary integrals, also after gathered appropriately, will yield the adjoint boundary conditions.
\end{question}
\begin{align*}
    &\int_\Omega D_{\bf u}J_\Omega({\bf x},{\bf u},\nabla{\bf u},p)\tilde{\bf u} + {\color{red}D_{\nabla{\bf u}}J_\Omega({\bf x},{\bf u},\nabla{\bf u},p)\nabla\tilde{\bf u}} + D_pJ_\Omega({\bf x},{\bf u},\nabla{\bf u},p)\tilde{p}{\rm d}{\bf x}\\
    &+ \int_\Gamma D_{\bf u}J_\Gamma({\bf x},{\bf u},\nabla{\bf u},p,{\bf n},{\bf t})\tilde{\bf u} + D_{\nabla{\bf u}}J_\Gamma({\bf x},{\bf u},\nabla{\bf u},p,{\bf n},{\bf t})\nabla\tilde{\bf u} + D_pJ_\Gamma({\bf x},{\bf u},\nabla{\bf u},p,{\bf n},{\bf t})\tilde{p}{\rm d}\Gamma\\
    &+ \int_\Omega -D_{\bf u}{\bf P}({\bf x},{\bf u},\nabla{\bf u},\Delta{\bf u},p,\nabla p)\tilde{\bf u}\cdot{\bf v} - {\color{red}D_{\nabla{\bf u}}{\bf P}({\bf x},{\bf u},\nabla{\bf u},\Delta{\bf u},p,\nabla p)\nabla\tilde{\bf u}\cdot{\bf v}} -  {\color{red}D_{\Delta{\bf u}}{\bf P}({\bf x},{\bf u},\nabla{\bf u},\Delta{\bf u},p,\nabla p)\Delta\tilde{\bf u}\cdot{\bf v}}\\
    &\hspace{1cm} + D_{\bf u}{\bf f}({\bf x},{\bf u},\nabla{\bf u},p)\tilde{\bf u}\cdot{\bf v} +  {\color{red}D_{\nabla{\bf u}}{\bf f}({\bf x},{\bf u},\nabla{\bf u},p)\nabla\tilde{\bf u}\cdot{\bf v}}\\
    &\hspace{1cm} - D_p{\bf P}({\bf x},{\bf u},\nabla{\bf u},\Delta{\bf u},p,\nabla p)\tilde{p}\cdot{\bf v} -  {\color{red}D_{\nabla p}{\bf P}({\bf x},{\bf u},\nabla{\bf u},\Delta{\bf u},p,\nabla p)\nabla\tilde{p}\cdot{\bf v}} + D_p{\bf f}({\bf x},{\bf u},\nabla{\bf u},p)\tilde{p}\cdot{\bf v}\\
    &\hspace{1cm} +  {\color{red}q\nabla\cdot\tilde{\bf u}} + qD_{\bf u}f_{\rm div}({\bf x},{\bf u},\nabla{\bf u},p)\tilde{\bf u} +  {\color{red}qD_{\nabla{\bf u}}f_{\rm div}({\bf x},{\bf u},\nabla{\bf u},p)\nabla\tilde{\bf u}} + qD_pf_{\rm div}({\bf x},{\bf u},\nabla{\bf u},p)\tilde{p}{\rm d}{\bf x}\\
    &- \delta_{\mathcal{L}}\int_\Gamma D_{\bf u}{\bf Q}({\bf x},{\bf u},\nabla{\bf u},p,{\bf n},{\bf t})\tilde{\bf u}\cdot{\bf v}_{\rm bc} + D_{\nabla{\bf u}}{\bf Q}({\bf x},{\bf u},\nabla{\bf u},p,{\bf n},{\bf t})\nabla\tilde{\bf u}\cdot{\bf v}_{\rm bc} + D_p{\bf Q}({\bf x},{\bf u},\nabla{\bf u},p,{\bf n},{\bf t})\tilde{p}\cdot{\bf v}_{\rm bc}{\rm d}\Gamma = 0,\ \forall({\bf u},p,\Omega,\tilde{\bf u},\tilde{p}).
\end{align*}
Integrate by parts:
\begin{enumerate}[leftmargin=0in]
    \item Term $D_{\nabla{\bf u}}J_\Omega({\bf x},{\bf u},\nabla{\bf u},p)\nabla\tilde{\bf u}$:
    \begin{align*}
        &\int_\Omega D_{\nabla{\bf u}} J_\Omega({\bf x},{\bf u},\nabla{\bf u},p)\nabla\tilde{\bf u}{\rm d}{\bf x} = \int_\Omega \nabla_{\nabla{\bf u}}J_\Omega({\bf x},{\bf u},\nabla{\bf u},p):\nabla\tilde{\bf u}{\rm d}{\bf x} = \int_\Omega \sum_{i=1}^N\sum_{j=1}^N \partial_{\partial_{x_i}u_j}J_\Omega({\bf x},{\bf u},\nabla{\bf u},p)\partial_{x_i}\tilde{u}_j{\rm d}{\bf x}\\
        =&\, \sum_{i=1}^N\sum_{j=1}^N \int_\Omega \partial_{\partial_{x_i}u_j}J_\Omega({\bf x},{\bf u},\nabla{\bf u},p)\partial_{x_i}\tilde{u}_j{\rm d}{\bf x} = \sum_{i=1}^N\sum_{j=1}^N -\int_\Omega \partial_{x_i}\partial_{\partial_{x_i}u_j}J_\Omega({\bf x},{\bf u},\nabla{\bf u},p)\tilde{u}_j{\rm d}{\bf x} + \int_\Gamma n_i\partial_{\partial_{x_i}u_j}J_\Omega({\bf x},{\bf u},\nabla{\bf u},p)\tilde{u}_j{\rm d}\Gamma\\
        =&\, -\int_\Omega \sum_{j=1}^N \tilde{u}_j\sum_{i=1}^N \partial_{x_i}\partial_{\partial_{x_i}u_j}J_\Omega({\bf x},{\bf u},\nabla{\bf u},p){\rm d}{\bf x} + \int_\Gamma \sum_{i=1}^N\sum_{j=1}^N n_i\partial_{\partial_{x_i}u_j}J_\Omega({\bf x},{\bf u},\nabla{\bf u},p)\tilde{u}_j{\rm d}\Gamma\\
        =&\, -\int_\Omega \sum_{j=1}^N \tilde{u}_j\nabla\cdot\left(\nabla_{\nabla u_j}J_\Omega({\bf x},{\bf u},\nabla{\bf u},p)\right){\rm d}{\bf x} + \int_\Gamma {\bf n}^\top\nabla_{\nabla{\bf u}}J_\Omega({\bf x},{\bf u},\nabla{\bf u},p)\tilde{\bf u}{\rm d}\Gamma\\
        =&\, -\int_\Omega \nabla\cdot\left(\nabla_{\nabla{\bf u}}J_\Omega({\bf x},{\bf u},\nabla{\bf u},p)\right)\cdot\tilde{\bf u}{\rm d}{\bf x} + \int_\Gamma {\bf n}^\top\nabla_{\nabla{\bf u}}J_\Omega({\bf x},{\bf u},\nabla{\bf u},p)\tilde{\bf u}{\rm d}\Gamma,
    \end{align*}
    where $\nabla_{\nabla{\bf u}} f(\nabla{\bf u})\coloneqq\left(\partial_{\partial_{x_i}u_j}f(\nabla{\bf u})\right)_{i,j=1}^N$ for any scalar function $f$.
    \item Term $- D_{\nabla{\bf u}}{\bf P}({\bf x},{\bf u},\nabla{\bf u},\Delta{\bf u},p,\nabla p)\nabla\tilde{\bf u}\cdot{\bf v}$:
    \begin{align*}
        &-\int_\Omega D_{\nabla{\bf u}}{\bf P}({\bf x},{\bf u},\nabla{\bf u},\Delta{\bf u},p,\nabla p)\nabla\tilde{\bf u}\cdot{\bf v}{\rm d}{\bf x} = -\int_\Omega \left(\nabla_{\nabla{\bf u}}P_k({\bf x},{\bf u},\nabla{\bf u},\Delta{\bf u},p,\nabla p):\nabla\tilde{\bf u}\right)_{k=1}^N\cdot{\bf v}{\rm d}{\bf x}\\
        =&\, -\int_\Omega \sum_{k=1}^N \nabla_{\nabla{\bf u}}P_k({\bf x},{\bf u},\nabla{\bf u},\Delta{\bf u},p,\nabla p):\nabla\tilde{\bf u}v_k{\rm d}{\bf x} = -\int_\Omega \sum_{k=1}^N\sum_{i=1}^N\sum_{j=1}^N \partial_{\partial_{x_i}u_j}P_k({\bf x},{\bf u},\nabla{\bf u},\Delta{\bf u},p,\nabla p)\partial_{x_i}\tilde{u}_jv_k{\rm d}{\bf x}\\
        =&\, -\sum_{i=1}^N\sum_{j=1}^N\sum_{k=1}^N \int_\Omega \partial_{\partial_{x_i}u_j}P_k({\bf x},{\bf u},\nabla{\bf u},\Delta{\bf u},p,\nabla p)\partial_{x_i}\tilde{u}_jv_k{\rm d}{\bf x}\\
        =&\, \sum_{i=1}^N\sum_{j=1}^N\sum_{k=1}^N \int_\Omega \partial_{x_i}\partial_{\partial_{x_i}u_j}P_k({\bf x},{\bf u},\nabla{\bf u},\Delta{\bf u},p,\nabla p)\tilde{u}_jv_k + \partial_{\partial_{x_i}u_j}P_k({\bf x},{\bf u},\nabla{\bf u},\Delta{\bf u},p,\nabla p)\tilde{u}_j\partial_{x_i}v_k{\rm d}{\bf x}\\
        &\hspace{15mm} -\int_\Gamma n_i\partial_{\partial_{x_i}u_j}P_k({\bf x},{\bf u},\nabla{\bf u},\Delta{\bf u},p,\nabla p)\tilde{u}_jv_k{\rm d}\Gamma\\
        =&\, \int_\Omega \sum_{j=1}^N \tilde{u}_j\sum_{k=1}^N v_k\sum_{i=1}^N \partial_{x_i}\partial_{\partial_{x_i}u_j}P_k({\bf x},{\bf u},\nabla{\bf u},\Delta{\bf u},p,\nabla p) + \sum_{j=1}^N \tilde{u}_j\sum_{k=1}^N\sum_{i=1}^N \partial_{\partial_{x_i}u_j}P_k({\bf x},{\bf u},\nabla{\bf u},\Delta{\bf u},p,\nabla p)\partial_{x_i}v_k{\rm d}{\bf x}\\
        & -\int_\Gamma \sum_{j=1}^N \tilde{u}_j\sum_{k=1}^N v_k\sum_{i=1}^N n_i\partial_{\partial_{x_i}u_j}P_k({\bf x},{\bf u},\nabla{\bf u},\Delta{\bf u},p,\nabla p){\rm d}\Gamma\\
        =&\, \int_\Omega \sum_{j=1}^N \tilde{u}_j\sum_{k=1}^N v_k\nabla\cdot\left(\nabla_{\nabla u_j}P_k({\bf x},{\bf u},\nabla{\bf u},\Delta{\bf u},p,\nabla p)\right) + \sum_{j=1}^N \tilde{u}_j\sum_{k=1}^N \nabla_{\nabla u_j}P_k({\bf x},{\bf u},\nabla{\bf u},\Delta{\bf u},p,\nabla p)\cdot\nabla v_k{\rm d}{\bf x}\\
        & -\int_\Gamma \sum_{j=1}^N \tilde{u}_j\sum_{k=1}^N v_k\nabla_{\nabla u_j}P_k({\bf x},{\bf u},\nabla{\bf u},\Delta{\bf u},p,\nabla p)\cdot{\bf n}{\rm d}\Gamma\\
        =&\, \int_\Omega \sum_{j=1}^N \tilde{u}_j\nabla\cdot\left(\nabla_{\nabla u_j}{\bf P}({\bf x},{\bf u},\nabla{\bf u},\Delta{\bf u},p,\nabla p)\right)\cdot{\bf v} + \sum_{j=1}^N \tilde{u}_j\nabla_{\nabla u_j}{\bf P}({\bf x},{\bf u},\nabla{\bf u},\Delta{\bf u},p,\nabla p):\nabla{\bf v}{\rm d}{\bf x}\\
        &- \int_\Gamma \sum_{j=1}^N \tilde{u}_j\left(\nabla_{\nabla u_j}{\bf P}({\bf x},{\bf u},\nabla{\bf u},\Delta{\bf u},p,\nabla p)\cdot{\bf n}\right)\cdot{\bf v}{\rm d}\Gamma\\
        =&\, \int_\Omega \left(\nabla\cdot\left(\nabla_{\nabla{\bf u}}{\bf P}({\bf x},{\bf u},\nabla{\bf u},\Delta{\bf u},p,\nabla p)\right)\cdot{\bf v}\right)\cdot\tilde{\bf u} + \left(\nabla_{\nabla{\bf u}}{\bf P}({\bf x},{\bf u},\nabla{\bf u},\Delta{\bf u},p,\nabla p):\nabla{\bf v}\right)\cdot\tilde{\bf u}{\rm d}{\bf x}\\
        &- \int_\Gamma \left(\left(\nabla_{\nabla{\bf u}}{\bf P}({\bf x},{\bf u},\nabla{\bf u},\Delta{\bf u},p,\nabla p)\cdot{\bf n}\right)\cdot{\bf v}\right)\cdot\tilde{\bf u}{\rm d}\Gamma.
    \end{align*}
    \item Term $D_{\Delta{\bf u}}{\bf P}({\bf x},{\bf u},\nabla{\bf u},\Delta{\bf u},p,\nabla p)\Delta\tilde{\bf u}\cdot{\bf v}$:
    \begin{align*}
        &-\int_\Omega D_{\Delta{\bf u}}{\bf P}({\bf x},{\bf u},\nabla{\bf u},\Delta{\bf u},p,\nabla p)\Delta\tilde{\bf u}\cdot{\bf v}{\rm d}{\bf x} = -\int_\Omega \left(\partial_{\Delta u_j}P_i({\bf x},{\bf u},\nabla{\bf u},\Delta{\bf u},p,\nabla p)\right)_{i,j=1}^N\Delta\tilde{\bf u}\cdot{\bf v}{\rm d}{\bf x}\\
        =&\, -\int_\Omega \left(\sum_{j=1}^N \partial_{\Delta u_j}P_i({\bf x},{\bf u},\nabla{\bf u},\Delta{\bf u},p,\nabla p)\Delta\tilde{u}_j\right)_{j=1}^N\cdot{\bf v}{\rm d}{\bf x} = -\int_\Omega \sum_{i=1}^N\sum_{j=1}^N \partial_{\Delta u_j}P_i({\bf x},{\bf u},\nabla{\bf u},\Delta{\bf u},p,\nabla p)\Delta\tilde{u}_jv_i{\rm d}{\bf x}\\
        =&\, -\sum_{i=1}^N\sum_{j=1}^N \int_\Omega \partial_{\Delta u_j}P_i({\bf x},{\bf u},\nabla{\bf u},\Delta{\bf u},p,\nabla p)\Delta\tilde{u}_jv_i{\rm d}{\bf x}\\
        =&\, -\sum_{i=1}^N\sum_{j=1}^N \int_\Omega \tilde{u}_j\Delta\left(\partial_{\Delta u_j}P_i({\bf x},{\bf u},\nabla{\bf u},\Delta{\bf u},p,\nabla p)v_i\right){\rm d}{\bf x}\\
        &+ \int_\Gamma \partial_{\bf n}\tilde{u}_j\partial_{\Delta u_j}P_i({\bf x},{\bf u},\nabla{\bf u},\Delta{\bf u},p,\nabla p)v_i - \tilde{u}_j\partial_{\bf n}\left(\partial_{\Delta u_j}P_i({\bf x},{\bf u},\nabla{\bf u},\Delta{\bf u},p,\nabla p)v_i\right){\rm d}\Gamma\\
        =&\, -\sum_{i=1}^N\sum_{j=1}^N \int_\Omega \tilde{u}_j\Delta\partial_{\Delta u_j}P_i({\bf x},{\bf u},\nabla{\bf u},\Delta{\bf u},p,\nabla p)v_i + \tilde{u}_j\partial_{\Delta u_j}P_i({\bf x},{\bf u},\nabla{\bf u},\Delta{\bf u},p,\nabla p)\Delta v_i{\rm d}{\bf x}\\
        &+ \int_\Gamma \partial_{\bf n}\tilde{u}_j\partial_{\Delta u_j}P_i({\bf x},{\bf u},\nabla{\bf u},\Delta{\bf u},p,\nabla p)v_i - \tilde{u}_j\partial_{\bf n}\partial_{\Delta u_j}P_i({\bf x},{\bf u},\nabla{\bf u},\Delta{\bf u},p,\nabla p)v_i - \tilde{u}_j\partial_{\Delta u_j}P_i({\bf x},{\bf u},\nabla{\bf u},\Delta{\bf u},p,\nabla p)\partial_{\bf n}v_i{\rm d}\Gamma\\
        =&\, -\int_\Omega \sum_{i=1}^N\sum_{j=1}^N v_i\Delta\partial_{\Delta u_j}P_i({\bf x},{\bf u},\nabla{\bf u},\Delta{\bf u},p,\nabla p)\tilde{u}_j + \Delta v_i\partial_{\Delta u_j}P_i({\bf x},{\bf u},\nabla{\bf u},\Delta{\bf u},p,\nabla p)\tilde{u}_j{\rm d}{\bf x}\\
        &+ \int_\Gamma \sum_{i=1}^N\sum_{j=1}^N v_i\partial_{\Delta u_j}P_i({\bf x},{\bf u},\nabla{\bf u},\Delta{\bf u},p,\nabla p)\partial_{\bf n}\tilde{u}_j - \sum_{i=1}^N\sum_{j=1}^N v_i\partial_{\bf n}\partial_{\Delta u_j}P_i({\bf x},{\bf u},\nabla{\bf u},\Delta{\bf u},p,\nabla p)\tilde{u}_j\\
        &\hspace{1cm} - \sum_{i=1}^N\sum_{j=1}^N \partial_{\bf n}v_i\partial_{\Delta u_j}P_i({\bf x},{\bf u},\nabla{\bf u},\Delta{\bf u},p,\nabla p)\tilde{u}_j{\rm d}\Gamma\\
        =&\, -\int_\Omega {\bf v}^\top\Delta D_{\Delta{\bf u}}{\bf P}({\bf x},{\bf u},\nabla{\bf u},\Delta{\bf u},p,\nabla p)\tilde{\bf u} + \Delta{\bf v}^\top D_{\Delta{\bf u}}{\bf P}({\bf x},{\bf u},\nabla{\bf u},\Delta{\bf u},p,\nabla p)\tilde{\bf u}{\rm d}{\bf x}\\
        &+ \int_\Gamma {\bf v}^\top D_{\Delta{\bf u}}{\bf P}({\bf x},{\bf u},\nabla{\bf u},\Delta{\bf u},p,\nabla p)\partial_{\bf n}\tilde{\bf u} - {\bf v}^\top\partial_{\bf n}D_{\Delta{\bf u}}{\bf P}({\bf x},{\bf u},\nabla{\bf u},\Delta{\bf u},p,\nabla p)\tilde{\bf u} - \partial_{\bf n}{\bf v}^\top D_{\Delta{\bf u}}{\bf P}({\bf x},{\bf u},\nabla{\bf u},\Delta{\bf u},p,\nabla p)\tilde{\bf u}{\rm d}\Gamma.
    \end{align*}
    \item Term $D_{\nabla{\bf u}}{\bf f}({\bf x},{\bf u},\nabla{\bf u},p)\nabla\tilde{\bf u}\cdot{\bf v}$:
    \begin{align*}
        &\int_\Omega D_{\nabla{\bf u}}{\bf f}({\bf x},{\bf u},\nabla{\bf u},p)\nabla\tilde{\bf u}\cdot{\bf v}{\rm d}{\bf x} = \int_\Omega \left(\nabla_{\nabla{\bf u}}f_k({\bf x},{\bf u},\nabla{\bf u},p):\nabla\tilde{\bf u}\right)_{k=1}^N\cdot{\bf v}{\rm d}{\bf x} = \int_\Omega \sum_{k=1}^N \nabla_{\nabla{\bf u}}f_k({\bf x},{\bf u},\nabla{\bf u},p):\nabla\tilde{\bf u}v_k{\rm d}{\bf x}\\
        =&\, \int_\Omega \sum_{k=1}^N\sum_{i=1}^N\sum_{j=1}^N \partial_{\partial_{x_i}u_j}f_k({\bf x},{\bf u},\nabla{\bf u},p)\partial_{x_i}\tilde{u}_jv_k{\rm d}{\bf x} = \sum_{i=1}^N\sum_{j=1}^N\sum_{k=1}^N \int_\Omega \partial_{\partial_{x_i}u_j}f_k({\bf x},{\bf u},\nabla{\bf u},p)\partial_{x_i}\tilde{u}_jv_k{\rm d}{\bf x}\\
        =&\, \sum_{i=1}^N\sum_{j=1}^N\sum_{k=1}^N -\int_\Omega \partial_{x_i}\partial_{\partial_{x_i}u_j}f_k({\bf x},{\bf u},\nabla{\bf u},p)\tilde{u}_jv_k + \partial_{\partial_{x_i}u_j}f_k({\bf x},{\bf u},\nabla{\bf u},p)\tilde{u}_j\partial_{x_i}v_k{\rm d}{\bf x} + \int_\Gamma n_i\partial_{\partial_{x_i}u_j}f_k({\bf x},{\bf u},\nabla{\bf u},p)\tilde{u}_jv_k{\rm d}\Gamma\\
        =&\, -\int_\Omega \sum_{j=1}^N \tilde{u}_j\sum_{k=1}^N v_k\sum_{i=1}^N \partial_{x_i}\partial_{\partial_{x_i}u_j}f_k({\bf x},{\bf u},\nabla{\bf u},p) + \sum_{j=1}^N \tilde{u}_j\sum_{k=1}^N\sum_{i=1}^N \partial_{\partial_{x_i}u_j}f_k({\bf x},{\bf u},\nabla{\bf u},p)\partial_{x_i}v_k{\rm d}{\bf x}\\
        &+ \int_\Gamma \sum_{j=1}^N \tilde{u}_j\sum_{k=1}^N v_k\sum_{i=1}^N n_i\partial_{\partial_{x_i}u_j}f_k({\bf x},{\bf u},\nabla{\bf u},p){\rm d}\Gamma\\
        =&\, - \int_\Omega \sum_{j=1}^N \tilde{u}_j\sum_{k=1}^N v_k\nabla\cdot\left(\nabla_{\nabla u_j}f_k({\bf x},{\bf u},\nabla{\bf u},p)\right) + \sum_{j=1}^N \tilde{u}_j\sum_{k=1}^N \nabla_{\nabla u_j}f_k({\bf x},{\bf u},\nabla{\bf u},p)\cdot\nabla v_k{\rm d}{\bf x}\\
        &+ \int_\Gamma \sum_{j=1}^N \tilde{u}_j\sum_{k=1}^N v_k\nabla_{\nabla u_j}f_k({\bf x},{\bf u},\nabla{\bf u},p)\cdot{\bf n}{\rm d}\Gamma\\
        =&\, -\int_\Omega \sum_{j=1}^N \tilde{u}_j\nabla\cdot\left(\nabla_{\nabla u_j}{\bf f}({\bf x},{\bf u},\nabla{\bf u},p)\right)\cdot{\bf v} + \sum_{j=1}^N \tilde{u}_j\nabla_{\nabla u_j}{\bf f}({\bf x},{\bf u},\nabla{\bf u},p):\nabla{\bf v}{\rm d}{\bf x} + \int_\Gamma \sum_{j=1}^N \tilde{u}_j\left(\nabla_{\nabla u_j}{\bf f}({\bf x},{\bf u},\nabla{\bf u},p)\cdot{\bf n}\right)\cdot{\bf v}{\rm d}\Gamma\\
        =&\, -\int_\Omega \left(\nabla\cdot\left(\nabla_{\nabla{\bf u}}{\bf f}({\bf x},{\bf u},\nabla{\bf u},p)\right)\cdot{\bf v}\right)\cdot\tilde{\bf u} + \left(\nabla_{\nabla{\bf u}}{\bf f}({\bf x},{\bf u},\nabla{\bf u},p):\nabla{\bf v}\right)\cdot\tilde{\bf u}{\rm d}{\bf x} + \int_\Gamma \left(\left(\nabla_{\nabla{\bf u}}{\bf f}({\bf x},{\bf u},\nabla{\bf u},p)\cdot{\bf n}\right)\cdot{\bf v}\right)\cdot\tilde{\bf u}{\rm d}\Gamma.
    \end{align*}
    \item Term $-D_{\nabla p}{\bf P}({\bf x},{\bf u},\nabla{\bf u},\Delta{\bf u},p,\nabla p)\nabla\tilde{p}\cdot{\bf v}$:
    \begin{align*}
        &-\int_\Omega D_{\nabla p}{\bf P}({\bf x},{\bf u},\nabla{\bf u},\Delta{\bf u},p,\nabla p)\nabla\tilde{p}\cdot{\bf v}{\rm d}{\bf x} = -\int_\Omega \left(\sum_{j=1}^N \partial_{\partial_{x_j}p}P_i({\bf x},{\bf u},\nabla{\bf u},\Delta{\bf u},p,\nabla p)\partial_{x_j}\tilde{p}\right)_{i=1}^N\cdot{\bf v}{\rm d}{\bf x}\\
        =&\, - \int_\Omega \sum_{i=1}^N\sum_{j=1}^N v_i\partial_{\partial_{x_j}p}P_i({\bf x},{\bf u},\nabla{\bf u},\Delta{\bf u},p,\nabla p)\partial_{x_j}\tilde{p}{\rm d}{\bf x} = - \sum_{i=1}^N\sum_{j=1}^N \int_\Omega v_i\partial_{\partial_{x_j}p}P_i({\bf x},{\bf u},\nabla{\bf u},\Delta{\bf u},p,\nabla p)\partial_{x_j}\tilde{p}{\rm d}{\bf x}\\
        =&\, \sum_{i=1}^N\sum_{j=1}^N \int_\Omega \partial_{x_j}v_i\partial_{\partial_{x_j}p}P_i({\bf x},{\bf u},\nabla{\bf u},\Delta{\bf u},p,\nabla p)\tilde{p} + v_i\partial_{x_j}\partial_{\partial_{x_j}p}P_i({\bf x},{\bf u},\nabla{\bf u},\Delta{\bf u},p,\nabla p)\tilde{p}{\rm d}{\bf x} - \int_\Gamma v_i\partial_{\partial_{x_j}p}P_i({\bf x},{\bf u},\nabla{\bf u},\Delta{\bf u},p,\nabla p)\tilde{p}n_j{\rm d}\Gamma\\
        =&\, \int_\Omega \tilde{p}\sum_{i=1}^N\sum_{j=1}^N \partial_{x_i}v_j\partial_{\partial_{x_i}p}P_j({\bf x},{\bf u},\nabla{\bf u},\Delta{\bf u},p,\nabla p) + \tilde{p}\sum_{i=1}^N v_i\sum_{j=1}^N \partial_{x_j}\partial_{\partial_{x_j}p}P_i({\bf x},{\bf u},\nabla{\bf u},\Delta{\bf u},p,\nabla p){\rm d}{\bf x}\\
        &- \int_\Gamma \tilde{p}\sum_{i=1}^N\sum_{j=1}^N n_i\partial_{\partial_{x_j}p}P_j({\bf x},{\bf u},\nabla{\bf u},\Delta{\bf u},p,\nabla p)v_j{\rm d}\Gamma\\
        =&\, \int_\Omega \tilde{p}\nabla_{\nabla p}{\bf P}({\bf x},{\bf u},\nabla{\bf u},\Delta{\bf u},p,\nabla p):\nabla{\bf v} + \tilde{p}\sum_{i=1}^N v_i\nabla\cdot\left(\nabla_{\nabla p}P_i({\bf x},{\bf u},\nabla{\bf u},\Delta{\bf u},p,\nabla p)\right){\rm d}{\bf x} - \int_\Gamma \tilde{p}{\bf n}^\top\nabla_{\nabla p}{\bf P}({\bf x},{\bf u},\nabla{\bf u},\Delta{\bf u},p,\nabla p){\bf v}{\rm d}\Gamma\\
        =&\, \int_\Omega \tilde{p}\nabla_{\nabla p}{\bf P}({\bf x},{\bf u},\nabla{\bf u},\Delta{\bf u},p,\nabla p):\nabla{\bf v} + \tilde{p}\nabla\cdot\left(\nabla_{\nabla p}{\bf P}({\bf x},{\bf u},\nabla{\bf u},\Delta{\bf u},p,\nabla p)\right)\cdot{\bf v}{\rm d}{\bf x} - \int_\Gamma \tilde{p}{\bf n}^\top\nabla_{\nabla p}{\bf P}({\bf x},{\bf u},\nabla{\bf u},\Delta{\bf u},p,\nabla p){\bf v}{\rm d}\Gamma.
    \end{align*}
    \item Term $q\nabla\cdot\tilde{\bf u}$: Use \eqref{ibp},
    \begin{align*}
        \int_\Omega q\nabla\cdot\tilde{\bf u}{\rm d}{\bf x} = -\int_\Omega \nabla q\cdot\tilde{\bf u}{\rm d}{\bf x} + \int_\Gamma q\tilde{\bf u}\cdot{\bf n}{\rm d}\Gamma.
    \end{align*}
    \item Term $qD_{\nabla{\bf u}}f_{\rm div}({\bf x},{\bf u},\nabla{\bf u},p)\nabla\tilde{\bf u}$:
    \begin{align*}
        &\int_\Omega qD_{\nabla{\bf u}}f_{\rm div}({\bf x},{\bf u},\nabla{\bf u},p)\nabla\tilde{\bf u}{\rm d}{\bf x} = \int_\Omega q\nabla_{\nabla{\bf u}}f_{\rm div}({\bf x},{\bf u},\nabla{\bf u},p):\nabla\tilde{\bf u}{\rm d}{\bf x}\\
        =&\, \int_\Omega q\sum_{i=1}^N\sum_{j=1}^N \partial_{\partial_{x_i}u_j}f_{\rm div}({\bf x},{\bf u},\nabla{\bf u},p)\partial_{x_i}\tilde{u}_j{\rm d}{\bf x} = \sum_{i=1}^N\sum_{j=1}^N \int_\Omega q\partial_{\partial_{x_i}u_j}f_{\rm div}({\bf x},{\bf u},\nabla{\bf u},p)\partial_{x_i}\tilde{u}_j{\rm d}{\bf x}\\
        =&\, \sum_{i=1}^N\sum_{j=1}^N -\int_\Omega \partial_{x_i}q\partial_{\partial_{x_i}u_j}f_{\rm div}({\bf x},{\bf u},\nabla{\bf u},p)\tilde{u}_j + q\partial_{x_i}\partial_{\partial_{x_i}u_j}f_{\rm div}({\bf x},{\bf u},\nabla{\bf u},p)\tilde{u}_j{\rm d}{\bf x} + \int_\Gamma qn_i\partial_{\partial_{x_i}u_j}f_{\rm div}({\bf x},{\bf u},\nabla{\bf u},p)\tilde{u}_j{\rm d}\Gamma\\
        =&\, -\int_\Omega \sum_{i=1}^N\sum_{j=1}^N \partial_{x_i}q\partial_{\partial_{x_i}u_j}f_{\rm div}({\bf x},{\bf u},\nabla{\bf u},p)\tilde{u}_j + q\sum_{j=1}^N \tilde{u}_j\sum_{i=1}^N \partial_{x_i}\partial_{\partial_{x_i}u_j}f_{\rm div}({\bf x},{\bf u},\nabla{\bf u},p){\rm d}{\bf x}\\
        &+ \int_\Gamma q\sum_{i=1}^N\sum_{j=1}^N n_i\partial_{\partial_{x_i}u_j}f_{\rm div}({\bf x},{\bf u},\nabla{\bf u},p)\tilde{u}_j{\rm d}\Gamma\\
        =&\, -\int_\Omega \nabla^\top q\nabla_{\nabla{\bf u}}f_{\rm div}({\bf x},{\bf u},\nabla{\bf u},p)\tilde{\bf u} + q\sum_{j=1}^N \tilde{u}_j\nabla\cdot\left(\nabla_{\nabla u_j}f_{\rm div}({\bf x},{\bf u},\nabla{\bf u},p)\right){\rm d}{\bf x} + \int_\Gamma q{\bf n}^\top\nabla_{\nabla{\bf u}}f_{\rm div}({\bf x},{\bf u},\nabla{\bf u},p)\tilde{\bf u}{\rm d}\Gamma\\
        =&\, -\int_\Omega \nabla^\top q\nabla_{\nabla{\bf u}}f_{\rm div}({\bf x},{\bf u},\nabla{\bf u},p)\tilde{\bf u} + q\left(\nabla\cdot\left(\nabla_{\nabla{\bf u}}f_{\rm div}({\bf x},{\bf u},\nabla{\bf u},p)\right)\right)\cdot\tilde{\bf u}{\rm d}{\bf x} + \int_\Gamma q{\bf n}^\top\nabla_{\nabla{\bf u}}f_{\rm div}({\bf x},{\bf u},\nabla{\bf u},p)\tilde{\bf u}{\rm d}\Gamma.
    \end{align*}
\end{enumerate}
Gather terms:
\begin{equation}
    \label{Euler-Lagrange for general stationary fluid dynamics PDEs}
    \tag{EuLa-gfld}
    \left.\begin{split}
        &\int_\Omega \left[\nabla_{\bf u}J_\Omega({\bf x},{\bf u},\nabla{\bf u},p) - \nabla\cdot\left(\nabla_{\nabla{\bf u}}J_\Omega({\bf x},{\bf u},\nabla{\bf u},p)\right) - \nabla_{\bf u}{\bf P}({\bf x},{\bf u},\nabla{\bf u},\Delta{\bf u},p,\nabla p){\bf v} + \nabla\cdot\left(\nabla_{\nabla{\bf u}}{\bf P}({\bf x},{\bf u},\nabla{\bf u},\Delta{\bf u},p,\nabla p)\right)\cdot{\bf v}\right.\\
        &\hspace{1cm} + \nabla_{\nabla{\bf u}}{\bf P}({\bf x},{\bf u},\nabla{\bf u},\Delta{\bf u},p,\nabla p):\nabla{\bf v} - \Delta\nabla_{\Delta{\bf u}}{\bf P}({\bf x},{\bf u},\nabla{\bf u},\Delta{\bf u},p,\nabla p){\bf v} - \nabla_{\Delta{\bf u}}{\bf P}({\bf x},{\bf u},\nabla{\bf u},\Delta{\bf u},p,\nabla p)\Delta{\bf v}\\
        &\hspace{1cm} + \nabla_{\bf u}{\bf f}({\bf x},{\bf u},\nabla{\bf u},p){\bf v} - \nabla\cdot\left(\nabla_{\nabla{\bf u}}{\bf f}({\bf x},{\bf u},\nabla{\bf u},p)\right)\cdot{\bf v} - \nabla_{\nabla{\bf u}}{\bf f}({\bf x},{\bf u},\nabla{\bf u},p):\nabla{\bf v} - \nabla q + q\nabla_{\bf u}f_{\rm div}({\bf x},{\bf u},\nabla{\bf u},p)\\
        &\hspace{1cm} \left.- \nabla_{\nabla{\bf u}}f_{\rm div}({\bf x},{\bf u},\nabla{\bf u},p)\cdot\nabla q - q\nabla\cdot\left(\nabla_{\nabla{\bf u}}f_{\rm div}({\bf x},{\bf u},\nabla{\bf u},p)\right)\right]\cdot\tilde{\bf u}{\rm d}{\bf x}\\
        &+ \int_\Omega \tilde{p}\left[D_pJ_\Omega({\bf x},{\bf u},\nabla{\bf u},p) - D_p{\bf P}({\bf x},{\bf u},\nabla{\bf u},\Delta{\bf u},p,\nabla p)\cdot{\bf v} + \nabla_{\nabla p}{\bf P}({\bf x},{\bf u},\nabla{\bf u},\Delta{\bf u},p,\nabla p):\nabla{\bf v}\right.\\
        &\hspace{1cm} \left.+ \nabla\cdot\left(\nabla_{\nabla p}{\bf P}({\bf x},{\bf u},\nabla{\bf u},\Delta{\bf u},p,\nabla p)\right)\cdot{\bf v} + D_p{\bf f}({\bf x},{\bf u},\nabla{\bf u},p)\cdot{\bf v} + qD_pf_{\rm div}({\bf x},{\bf u},\nabla{\bf u},p)\right]{\rm d}{\bf x}\\
        &+ \int_\Gamma \left[\nabla_{\nabla{\bf u}}J_\Omega({\bf x},{\bf u},\nabla{\bf u},p)\cdot{\bf n} + \nabla_{\bf u}J_\Gamma({\bf x},{\bf u},\nabla{\bf u},p,{\bf n},{\bf t}) - \left(\nabla_{\nabla{\bf u}}{\bf P}({\bf x},{\bf u},\nabla{\bf u},\Delta{\bf u},p,\nabla p)\cdot{\bf n}\right)\cdot{\bf v}\right.\\
        &\hspace{1cm} - \partial_{\bf n}\nabla_{\Delta{\bf u}}{\bf P}({\bf x},{\bf u},\nabla{\bf u},\Delta{\bf u},p,\nabla p){\bf v} - \nabla_{\Delta{\bf u}}{\bf P}({\bf x},{\bf u},\nabla{\bf u},\Delta{\bf u},p,\nabla p)\partial_{\bf n}{\bf v} + \left(\nabla_{\nabla{\bf u}}{\bf f}({\bf x},{\bf u},\nabla{\bf u},p)\cdot{\bf n}\right)\cdot{\bf v} + q{\bf n}\\
        &\hspace{1cm} \left.+ q\nabla_{\nabla{\bf u}}f_{\rm div}({\bf x},{\bf u},\nabla{\bf u},p)\cdot{\bf n} - \delta_{\mathcal{L}}\nabla_{\bf u}{\bf Q}({\bf x},{\bf u},\nabla{\bf u},p,{\bf n},{\bf t}){\bf v}_{\rm bc}\right]\cdot\tilde{\bf u}{\rm d}\Gamma\\
        &+ \int_\Gamma \tilde{p}\left[D_pJ_\Gamma({\bf x},{\bf u},\nabla{\bf u},p,{\bf n},{\bf t}) - {\bf n}^\top\nabla_{\nabla p}{\bf P}({\bf x},{\bf u},\nabla{\bf u},\Delta{\bf u},p,\nabla p){\bf v} - \delta_{\mathcal{L}}D_p{\bf Q}({\bf x},{\bf u},\nabla{\bf u},p,{\bf n},{\bf t})\cdot{\bf v}_{\rm bc}\right]{\rm d}\Gamma\\
        &+ \int_\Gamma \nabla_{\nabla{\bf u}}J_\Gamma({\bf x},{\bf u},\nabla{\bf u},p,{\bf n},{\bf t}):\nabla\tilde{\bf u} + {\bf v}^\top D_{\Delta{\bf u}}{\bf P}({\bf x},{\bf u},\nabla{\bf u},\Delta{\bf u},p,\nabla p)\partial_{\bf n}\tilde{\bf u}\\
        &\hspace{1cm} - \delta_{\mathcal{L}}D_{\nabla{\bf u}}{\bf Q}({\bf x},{\bf u},\nabla{\bf u},p,{\bf n},{\bf t})\nabla\tilde{\bf u}\cdot{\bf v}_{\rm bc}{\rm d}\Gamma = 0,\ \forall({\bf u},p,\Omega,\tilde{\bf u},\tilde{p}).
    \end{split}\right.
\end{equation}
Since this equation holds for all variations $(\tilde{\bf u},\tilde{p})$, consider the following 2 cases:
\begin{itemize}[leftmargin=0in]
    \item \textbf{Case $\delta_{\mathcal{L}} = 0$.} This means to ``activate'' the boundary-condition constraint ${\bf Q}({\bf x},{\bf u},\nabla{\bf u},p,{\bf n},{\bf t}) = {\bf f}_{\rm bc}({\bf x})$ on $\Gamma$, so it will not be penalized by the Lagrangian $L$. To simplify further the last equation, we define $\Gamma_{\rm v}^{\bf u}$ and $\Gamma_{\rm v}^p$ as the ``varying'' components w.r.t. ${\bf u}$ and $p$ of $\Gamma$, respectively, i.e.,
    \begin{align*}
        \Gamma_{\rm nv}^{\bf u} &\coloneqq\left\{{\bf x}\in\Gamma;\left({\bf Q}({\bf x},{\bf u} + \tilde{\bf u},\nabla{\bf u} + \nabla\tilde{\bf u},p + \tilde{p},{\bf n},{\bf t}) = {\bf Q}({\bf x},{\bf u},\nabla{\bf u},p,{\bf n},{\bf t})\right)\Rightarrow\tilde{\bf u} = {\bf 0}\right\},\\
        \Gamma_{\rm nv}^p &\coloneqq\left\{{\bf x}\in\Gamma;\left({\bf Q}({\bf x},{\bf u} + \tilde{\bf u},\nabla{\bf u} + \nabla\tilde{\bf u},p + \tilde{p},{\bf n},{\bf t}) = {\bf Q}({\bf x},{\bf u},\nabla{\bf u},p,{\bf n},{\bf t})\right)\Rightarrow\tilde{p} = 0\right\},\\
        \Gamma_{\rm v}^{\bf u} &\coloneqq\Gamma\backslash\Gamma_{\rm nv}^{\bf u},\ \Gamma_{\rm v}^p\coloneqq\Gamma\backslash\Gamma_{\rm nv}^p.
    \end{align*}
    E.g., the Dirichlet components of $\Gamma$ w.r.t. ${\bf u}$ and $p$, denoted by $\Gamma_{\rm D}^{\bf u}$ and $\Gamma_{\rm D}^p$, respectively:
    \begin{equation*}
        \left\{\begin{split}
            {\bf u} &= {\bf u}_{\rm D},&&\mbox{ on } \Gamma_{\rm D}^{\bf u},\\
            p &= p_{\rm D},&&\mbox{ on } \Gamma_{\rm D}^p.
        \end{split}\right.
    \end{equation*}
    belong to the ``non-variation'' components just defined of $\Gamma$: $\Gamma_{\rm D}^{\bf u}\subset\Gamma_{\rm nv}^{\bf u}$ and $\Gamma_{\rm D}^p\subset\Gamma_{\rm nv}^p$.
    
    To see the general structure, we rewrite \eqref{Euler-Lagrange for general stationary fluid dynamics PDEs} as follows:
    \begin{equation}
        \label{brief Euler-Lagrange for general stationary fluid dynamics PDEs}
        \tag{brief-EuLa-gfld}
        \left.\begin{split}
            &\int_\Omega {\bf F}_\Omega^{\tilde{\bf u}}({\bf x},{\bf u},\nabla{\bf u},\Delta{\bf u},p,\nabla p,{\bf v},\nabla{\bf v},\Delta{\bf v},q,\nabla q)\cdot\tilde{\bf u}{\rm d}{\bf x} + \int_\Omega F_\Omega^{\tilde{p}}({\bf x},{\bf u},\nabla{\bf u},\Delta{\bf u},p,\nabla p,{\bf v},\nabla{\bf v},q)\tilde{p}{\rm d}{\bf x}\\
            &\hspace{5mm}+ \int_\Gamma {\bf F}_\Gamma^{\tilde{\bf u}}({\bf x},{\bf u},\nabla{\bf u},\Delta{\bf u},p,\nabla p,{\bf v},\nabla{\bf v},q,{\bf n},{\bf t})\cdot\tilde{\bf u}{\rm d}\Gamma + \int_\Gamma F_\Gamma^{\tilde{p}}({\bf x},{\bf u},\nabla{\bf u},\Delta{\bf u},p,\nabla p,{\bf v},{\bf n},{\bf t})\tilde{p}{\rm d}\Gamma\\
            &\hspace{5mm}+ \int_\Gamma F_\Gamma^{\nabla\tilde{\bf u}}({\bf x},{\bf u},\nabla{\bf u},\Delta{\bf u},p,\nabla p,{\bf v},{\bf n},{\bf t},\nabla\tilde{\bf u}){\rm d}\Gamma = 0,\ \forall({\bf u},p,\Omega,\tilde{\bf u},\tilde{p}),
        \end{split}\right.
    \end{equation}
    where
    \begin{align*}
        &{\bf F}_\Omega^{\tilde{\bf u}}({\bf x},{\bf u},\nabla{\bf u},\Delta{\bf u},p,\nabla p,{\bf v},\nabla{\bf v},\Delta{\bf v},q,\nabla q)\\
        &\hspace{5mm}\coloneqq\nabla_{\bf u}J_\Omega({\bf x},{\bf u},\nabla{\bf u},p) - \nabla\cdot\left(\nabla_{\nabla{\bf u}}J_\Omega({\bf x},{\bf u},\nabla{\bf u},p)\right) - \nabla_{\bf u}{\bf P}({\bf x},{\bf u},\nabla{\bf u},\Delta{\bf u},p,\nabla p){\bf v} + \nabla\cdot\left(\nabla_{\nabla{\bf u}}{\bf P}({\bf x},{\bf u},\nabla{\bf u},\Delta{\bf u},p,\nabla p)\right)\cdot{\bf v}\\
        &\hspace{1cm}+ \nabla_{\nabla{\bf u}}{\bf P}({\bf x},{\bf u},\nabla{\bf u},\Delta{\bf u},p,\nabla p):\nabla{\bf v} - \Delta\nabla_{\Delta{\bf u}}{\bf P}({\bf x},{\bf u},\nabla{\bf u},\Delta{\bf u},p,\nabla p){\bf v} - \nabla_{\Delta{\bf u}}{\bf P}({\bf x},{\bf u},\nabla{\bf u},\Delta{\bf u},p,\nabla p)\Delta{\bf v}\\
        &\hspace{1cm}+ \nabla_{\bf u}{\bf f}({\bf x},{\bf u},\nabla{\bf u},p){\bf v} - \nabla\cdot\left(\nabla_{\nabla{\bf u}}{\bf f}({\bf x},{\bf u},\nabla{\bf u},p)\right)\cdot{\bf v} - \nabla_{\nabla{\bf u}}{\bf f}({\bf x},{\bf u},\nabla{\bf u},p):\nabla{\bf v} - \nabla q + q\nabla_{\bf u}f_{\rm div}({\bf x},{\bf u},\nabla{\bf u},p)\\
        &\hspace{1cm}- \nabla_{\nabla{\bf u}}f_{\rm div}({\bf x},{\bf u},\nabla{\bf u},p)\cdot\nabla q - q\nabla\cdot\left(\nabla_{\nabla{\bf u}}f_{\rm div}({\bf x},{\bf u},\nabla{\bf u},p)\right),\\
        &F_\Omega^{\tilde{p}}({\bf x},{\bf u},\nabla{\bf u},\Delta{\bf u},p,\nabla p,{\bf v},\nabla{\bf v},q)\\
        &\hspace{5mm}\coloneqq D_pJ_\Omega({\bf x},{\bf u},\nabla{\bf u},p) - D_p{\bf P}({\bf x},{\bf u},\nabla{\bf u},\Delta{\bf u},p,\nabla p)\cdot{\bf v} + \nabla_{\nabla p}{\bf P}({\bf x},{\bf u},\nabla{\bf u},\Delta{\bf u},p,\nabla p):\nabla{\bf v}\\
        &\hspace{1cm} + \nabla\cdot\left(\nabla_{\nabla p}{\bf P}({\bf x},{\bf u},\nabla{\bf u},\Delta{\bf u},p,\nabla p)\right)\cdot{\bf v} + D_p{\bf f}({\bf x},{\bf u},\nabla{\bf u},p)\cdot{\bf v} + qD_pf_{\rm div}({\bf x},{\bf u},\nabla{\bf u},p),\\
        &{\bf F}_\Gamma^{\tilde{\bf u}}({\bf x},{\bf u},\nabla{\bf u},\Delta{\bf u},p,\nabla p,{\bf v},\nabla{\bf v},q,{\bf n},{\bf t})\\
        &\hspace{5mm}\coloneqq\nabla_{\nabla{\bf u}}J_\Omega({\bf x},{\bf u},\nabla{\bf u},p)\cdot{\bf n} + \nabla_{\bf u}J_\Gamma({\bf x},{\bf u},\nabla{\bf u},p,{\bf n},{\bf t}) - \left(\nabla_{\nabla{\bf u}}{\bf P}({\bf x},{\bf u},\nabla{\bf u},\Delta{\bf u},p,\nabla p)\cdot{\bf n}\right)\cdot{\bf v}\\
        &\hspace{1cm} - \partial_{\bf n}\nabla_{\Delta{\bf u}}{\bf P}({\bf x},{\bf u},\nabla{\bf u},\Delta{\bf u},p,\nabla p){\bf v} - \nabla_{\Delta{\bf u}}{\bf P}({\bf x},{\bf u},\nabla{\bf u},\Delta{\bf u},p,\nabla p)\partial_{\bf n}{\bf v} + \left(\nabla_{\nabla{\bf u}}{\bf f}({\bf x},{\bf u},\nabla{\bf u},p)\cdot{\bf n}\right)\cdot{\bf v} + q{\bf n}\\
        &\hspace{1cm} + q\nabla_{\nabla{\bf u}}f_{\rm div}({\bf x},{\bf u},\nabla{\bf u},p)\cdot{\bf n},\\
        &F_\Gamma^{\tilde{p}}({\bf x},{\bf u},\nabla{\bf u},\Delta{\bf u},p,\nabla p,{\bf v},{\bf n},{\bf t})\coloneqq D_pJ_\Gamma({\bf x},{\bf u},\nabla{\bf u},p,{\bf n},{\bf t}) - {\bf n}^\top\nabla_{\nabla p}{\bf P}({\bf x},{\bf u},\nabla{\bf u},\Delta{\bf u},p,\nabla p){\bf v},\\
        &F_\Gamma^{\nabla\tilde{\bf u}}({\bf x},{\bf u},\nabla{\bf u},\Delta{\bf u},p,\nabla p,{\bf v},{\bf n},{\bf t},\nabla\tilde{\bf u})\coloneqq\nabla_{\nabla{\bf u}}J_\Gamma({\bf x},{\bf u},\nabla{\bf u},p,{\bf n},{\bf t}):\nabla\tilde{\bf u} + {\bf v}^\top D_{\Delta{\bf u}}{\bf P}({\bf x},{\bf u},\nabla{\bf u},\Delta{\bf u},p,\nabla p)\partial_{\bf n}\tilde{\bf u},
    \end{align*}
    for all $({\bf x},{\bf u},p,{\bf v},q,{\bf n},{\bf t},\tilde{\bf u},\tilde{p})$ s.t.
    \begin{align*}
        {\bf Q}({\bf x},{\bf u},\nabla{\bf u},p,{\bf n},{\bf t}) = {\bf Q}({\bf x},{\bf u} + \tilde{\bf u},\nabla{\bf u} + \nabla\tilde{\bf u},p + \tilde{p},{\bf n},{\bf t}) = {\bf f}_{\rm bc}({\bf x}) \mbox{ on } \Gamma.
    \end{align*}    
    We now deduce the adjoint equations of \eqref{general stationary fluid dynamics PDEs} from \eqref{brief Euler-Lagrange for general stationary fluid dynamics PDEs} as follows:
    \begin{itemize}
        \item Choose $\tilde{\bf u} = {\bf 0}$ in $\overline{\Omega}$, \eqref{brief Euler-Lagrange for general stationary fluid dynamics PDEs} then becomes
        \begin{align*}
            \int_\Omega F_\Omega^{\tilde{p}}({\bf x},{\bf u},\nabla{\bf u},\Delta{\bf u},p,\nabla p,{\bf v},\nabla{\bf v},q)\tilde{p}{\rm d}{\bf x} + \int_\Gamma F_\Gamma^{\tilde{p}}({\bf x},{\bf u},\nabla{\bf u},\Delta{\bf u},p,\nabla p,{\bf v},{\bf n},{\bf t})\tilde{p}{\rm d}\Gamma = 0,\ \forall({\bf u},p,\Omega,\tilde{p}).
        \end{align*}
        Then choose $\tilde{p}$ varying s.t. $\tilde{p}|_\Gamma = 0$, then the last equality yields
        \begin{align*}
            \int_\Omega F_\Omega^{\tilde{p}}({\bf x},{\bf u},\nabla{\bf u},\Delta{\bf u},p,\nabla p,{\bf v},\nabla{\bf v},q)\tilde{p}{\rm d}{\bf x} = 0,\ \forall({\bf u},p,\Omega,\tilde{p}) \mbox{ s.t. } \tilde{p}|_\Gamma = 0.
        \end{align*}
        Hence, $({\bf v},q)$ satisfies
        \begin{align}
            \label{domain integrand variation p}
            \boxed{F_\Omega^{\tilde{p}}({\bf x},{\bf u},\nabla{\bf u},\Delta{\bf u},p,\nabla p,{\bf v},\nabla{\bf v},q) = 0 \mbox{ in } \Omega.}
        \end{align}
        Plug it back in, obtain
        \begin{align*}
            \int_\Gamma F_\Gamma^{\tilde{p}}({\bf x},{\bf u},\nabla{\bf u},\Delta{\bf u},p,\nabla p,{\bf v},{\bf n},{\bf t})\tilde{p}{\rm d}\Gamma = 0,\ \forall({\bf u},p,\Omega,\tilde{p}).
        \end{align*}
        Note that $\tilde{p}|_{\Gamma_{\rm nv}^p} = 0$, the last equality yields
        \begin{align*}
            \int_{\Gamma_{\rm v}^p} F_\Gamma^{\tilde{p}}({\bf x},{\bf u},\nabla{\bf u},\Delta{\bf u},p,\nabla p,{\bf v},{\bf n},{\bf t})\tilde{p}{\rm d}\Gamma = 0,\ \forall({\bf u},p,\Omega,\tilde{p}).
        \end{align*}
        Thus, $({\bf v},q)$ satisfies
        \begin{align}
            \label{boundary integrand variation p}
            \boxed{F_\Gamma^{\tilde{p}}({\bf x},{\bf u},\nabla{\bf u},\Delta{\bf u},p,\nabla p,{\bf v},{\bf n},{\bf t}) = 0 \mbox{ on } \Gamma_{\rm v}^p.}
        \end{align}
        \item Assume that $({\bf v},q)$ satisfies \eqref{domain integrand variation p} and \eqref{boundary integrand variation p}, \eqref{brief Euler-Lagrange for general stationary fluid dynamics PDEs} then becomes
        \begin{align*}
            &\int_\Omega {\bf F}_\Omega^{\tilde{\bf u}}({\bf x},{\bf u},\nabla{\bf u},\Delta{\bf u},p,\nabla p,{\bf v},\nabla{\bf v},\Delta{\bf v},q,\nabla q)\cdot\tilde{\bf u}{\rm d}{\bf x} + \int_\Gamma {\bf F}_\Gamma^{\tilde{\bf u}}({\bf x},{\bf u},\nabla{\bf u},\Delta{\bf u},p,\nabla p,{\bf v},\nabla{\bf v},q,{\bf n},{\bf t})\cdot\tilde{\bf u}{\rm d}\Gamma\\
            &\hspace{5mm}+ \int_\Gamma F_\Gamma^{\nabla\tilde{\bf u}}({\bf x},{\bf u},\nabla{\bf u},\Delta{\bf u},p,\nabla p,{\bf v},{\bf n},{\bf t},\nabla\tilde{\bf u}){\rm d}\Gamma = 0,\ \forall({\bf u},p,\Omega,\tilde{\bf u}),
        \end{align*}
        Choose $\tilde{\bf u}$ varying s.t. $\tilde{\bf u}|_\Gamma = {\bf 0}$ and $\nabla\tilde{\bf u}|_\Gamma = {\bf 0}_{N\times N}$, the last equality yields
        \begin{align*}
            \int_\Omega {\bf F}_\Omega^{\tilde{\bf u}}({\bf x},{\bf u},\nabla{\bf u},\Delta{\bf u},p,\nabla p,{\bf v},\nabla{\bf v},\Delta{\bf v},q,\nabla q)\cdot\tilde{\bf u}{\rm d}{\bf x} = 0,\ \forall({\bf u},p,\Omega,\tilde{\bf u}) \mbox{ s.t. } \tilde{\bf u}|_\Gamma = {\bf 0} \mbox{ and } \nabla\tilde{\bf u}|_\Gamma = {\bf 0}_{N\times N}.
        \end{align*}
        Hence, $({\bf v},q)$ satisfies
        \begin{align}
            \label{domain integrand variation u}
            \boxed{{\bf F}_\Omega^{\tilde{\bf u}}({\bf x},{\bf u},\nabla{\bf u},\Delta{\bf u},p,\nabla p,{\bf v},\nabla{\bf v},\Delta{\bf v},q,\nabla q) = {\bf 0} \mbox{ in } \Omega.}
        \end{align}
        Plug it back in, obtain
        \begin{align*}
            \int_\Gamma {\bf F}_\Gamma^{\tilde{\bf u}}({\bf x},{\bf u},\nabla{\bf u},\Delta{\bf u},p,\nabla p,{\bf v},\nabla{\bf v},q,{\bf n},{\bf t})\cdot\tilde{\bf u}{\rm d}\Gamma + \int_\Gamma F_\Gamma^{\nabla\tilde{\bf u}}({\bf x},{\bf u},\nabla{\bf u},\Delta{\bf u},p,\nabla p,{\bf v},{\bf n},{\bf t},\nabla\tilde{\bf u}){\rm d}\Gamma = 0,\ \forall({\bf u},p,\Omega,\tilde{\bf u}),
        \end{align*}
        Note that $\tilde{\bf u}|_{\Gamma_{\rm nv}^{\bf u}} = {\bf 0}$, the last equality yields
        \begin{align*}
            \int_{\Gamma_{\rm v}^{\bf u}} {\bf F}_\Gamma^{\tilde{\bf u}}({\bf x},{\bf u},\nabla{\bf u},\Delta{\bf u},p,\nabla p,{\bf v},\nabla{\bf v},q,{\bf n},{\bf t})\cdot\tilde{\bf u}{\rm d}\Gamma + \int_\Gamma F_\Gamma^{\nabla\tilde{\bf u}}({\bf x},{\bf u},\nabla{\bf u},\Delta{\bf u},p,\nabla p,{\bf v},{\bf n},{\bf t},\nabla\tilde{\bf u}){\rm d}\Gamma = 0,\ \forall({\bf u},p,\Omega,\tilde{\bf u}).
        \end{align*}
        To simplify the last equality further, we need the explicit formula of ${\bf P}$ and ${\bf Q}$.
    \end{itemize}
    Conclude:
    \begin{equation}
        \label{adjoint general stationary fluid dynamics PDEs}
        \tag{adj-gfld}
        \boxed{\left\{\begin{split}
            &-\nabla_{\Delta{\bf u}}{\bf P}({\bf x},{\bf u},\nabla{\bf u},\Delta{\bf u},p,\nabla p)\Delta{\bf v} + \left(\nabla_{\nabla{\bf u}}{\bf P}({\bf x},{\bf u},\nabla{\bf u},\Delta{\bf u},p,\nabla p) - \nabla_{\nabla{\bf u}}{\bf f}({\bf x},{\bf u},\nabla{\bf u},p)\right):\nabla{\bf v}\\
            &\hspace{1cm}- \left[\nabla_{\bf u}{\bf P}({\bf x},{\bf u},\nabla{\bf u},\Delta{\bf u},p,\nabla p) + \Delta\nabla_{\Delta{\bf u}}{\bf P}({\bf x},{\bf u},\nabla{\bf u},\Delta{\bf u},p,\nabla p) - \nabla_{\bf u}{\bf f}({\bf x},{\bf u},\nabla{\bf u},p)\right]{\bf v}\\
            &\hspace{1cm}+ \left[\nabla\cdot\left(\nabla_{\nabla{\bf u}}{\bf P}({\bf x},{\bf u},\nabla{\bf u},\Delta{\bf u},p,\nabla p)\right) - \nabla\cdot\left(\nabla_{\nabla{\bf u}}{\bf f}({\bf x},{\bf u},\nabla{\bf u},p)\right)\right]\cdot{\bf v} - \nabla q - \nabla_{\nabla{\bf u}}f_{\rm div}({\bf x},{\bf u},\nabla{\bf u},p)\cdot\nabla q\\
            &\hspace{1cm}+ q\left[-\nabla\cdot\left(\nabla_{\nabla{\bf u}}f_{\rm div}({\bf x},{\bf u},\nabla{\bf u},p)\right) + \nabla_{\bf u}f_{\rm div}({\bf x},{\bf u},\nabla{\bf u},p)\right] = \nabla\cdot\left(\nabla_{\nabla{\bf u}}J_\Omega({\bf x},{\bf u},\nabla{\bf u},p)\right) - \nabla_{\bf u}J_\Omega({\bf x},{\bf u},\nabla{\bf u},p) \mbox{ in } \Omega,\\
            &\nabla_{\nabla p}{\bf P}({\bf x},{\bf u},\nabla{\bf u},\Delta{\bf u},p,\nabla p):\nabla{\bf v} + \left[-D_p{\bf P}({\bf x},{\bf u},\nabla{\bf u},\Delta{\bf u},p,\nabla p) + \nabla\cdot\left(\nabla_{\nabla p}{\bf P}({\bf x},{\bf u},\nabla{\bf u},\Delta{\bf u},p,\nabla p)\right) + D_p{\bf f}({\bf x},{\bf u},\nabla{\bf u},p)\right]\cdot{\bf v}\\
            &\hspace{1cm} = - D_pJ_\Omega ({\bf x},{\bf u},\nabla{\bf u},p) - qD_pf_{\rm div}({\bf x},{\bf u},\nabla{\bf u},p) \mbox{ in } \Omega,\\
            &{\bf Q}({\bf x},{\bf u},\nabla{\bf u},p,{\bf n},{\bf t}) = {\bf f}_{\rm bc}({\bf x}) \mbox{ on } \Gamma,\\
            &{\bf n}^\top\nabla_{\nabla p}{\bf P}({\bf x},{\bf u},\nabla{\bf u},\Delta{\bf u},p,\nabla p){\bf v} = D_pJ_\Gamma({\bf x},{\bf u},\nabla{\bf u},p,{\bf n},{\bf t}) \mbox{ on } \Gamma_{\rm v}^p,\\
            &\int_{\Gamma_{\rm v}^{\bf u}} \left[\nabla_{\nabla{\bf u}}J_\Omega({\bf x},{\bf u},\nabla{\bf u},p)\cdot{\bf n} + \nabla_{\bf u}J_\Gamma({\bf x},{\bf u},\nabla{\bf u},p,{\bf n},{\bf t}) - \left(\nabla_{\nabla{\bf u}}{\bf P}({\bf x},{\bf u},\nabla{\bf u},\Delta{\bf u},p,\nabla p)\cdot{\bf n}\right)\cdot{\bf v}\right.\\
            &\hspace{1cm} - \partial_{\bf n}\nabla_{\Delta{\bf u}}{\bf P}({\bf x},{\bf u},\nabla{\bf u},\Delta{\bf u},p,\nabla p){\bf v} - \nabla_{\Delta{\bf u}}{\bf P}({\bf x},{\bf u},\nabla{\bf u},\Delta{\bf u},p,\nabla p)\partial_{\bf n}{\bf v} + \left(\nabla_{\nabla{\bf u}}{\bf f}({\bf x},{\bf u},\nabla{\bf u},p)\cdot{\bf n}\right)\cdot{\bf v} + q{\bf n}\\
            &\hspace{1cm} \left.+ q\nabla_{\nabla{\bf u}}f_{\rm div}({\bf x},{\bf u},\nabla{\bf u},p)\cdot{\bf n}\right]\cdot\tilde{\bf u}{\rm d}\Gamma\\
            &\hspace{5mm}+ \int_\Gamma \nabla_{\nabla{\bf u}}J_\Gamma({\bf x},{\bf u},\nabla{\bf u},p,{\bf n},{\bf t}):\nabla\tilde{\bf u} + {\bf v}^\top D_{\Delta{\bf u}}{\bf P}({\bf x},{\bf u},\nabla{\bf u},\Delta{\bf u},p,\nabla p)\partial_{\bf n}\tilde{\bf u}{\rm d}\Gamma = 0,\ \forall({\bf u},p,\Omega,\tilde{\bf u}).
        \end{split}\right.}
    \end{equation}

    \begin{remark}
        The last integral equation in \eqref{adjoint general stationary fluid dynamics PDEs} seems ``overdetermined''. Indeed, choose $\tilde{\bf u}$ varying s.t. $\tilde{\bf u}|_\Gamma = {\bf 0}$, then $({\bf v},q)$ satisfies
        \begin{align*}
            \int_\Gamma \nabla_{\nabla{\bf u}}J_\Gamma({\bf x},{\bf u},\nabla{\bf u},p,{\bf n},{\bf t}):\nabla\tilde{\bf u} + {\bf v}^\top D_{\Delta{\bf u}}{\bf P}({\bf x},{\bf u},\nabla{\bf u},\Delta{\bf u},p,\nabla p)\partial_{\bf n}\tilde{\bf u}{\rm d}\Gamma = 0,\ \forall({\bf u},p,\Omega,\tilde{\bf u}) \mbox{ s.t. } \tilde{\bf u}|_\Gamma = {\bf 0}.
        \end{align*}
        The l.h.s. of this equals
        \begin{align*}
            &\int_\Gamma \sum_{i=1}^N\sum_{j=1}^N \partial_{\partial_{x_i}u_j}J_\Gamma({\bf x},{\bf u},\nabla{\bf u},p,{\bf n},{\bf t})\partial_{x_i}\tilde{u}_j + \sum_{i=1}^N\sum_{j=1}^N v_i\partial_{\Delta u_j}P_i({\bf x},{\bf u},\nabla{\bf u},\Delta{\bf u},p,\nabla p)\partial_{\bf n}\tilde{u}_j{\rm d}\Gamma\\
            =&\, \int_\Gamma \sum_{i=1}^N\sum_{j=1}^N \partial_{\partial_{x_i}u_j}J_\Gamma({\bf x},{\bf u},\nabla{\bf u},p,{\bf n},{\bf t})\partial_{x_i}\tilde{u}_j + \sum_{i=1}^N\sum_{j=1}^N v_i\partial_{\Delta u_j}P_i({\bf x},{\bf u},\nabla{\bf u},\Delta{\bf u},p,\nabla p)\sum_{k=1}^N n_k\partial_{x_k}\tilde{u}_j{\rm d}\Gamma\\
            =&\, \int_\Gamma \sum_{i=1}^N\sum_{j=1}^N \partial_{\partial_{x_i}u_j}J_\Gamma({\bf x},{\bf u},\nabla{\bf u},p,{\bf n},{\bf t})\partial_{x_i}\tilde{u}_j + \sum_{k=1}^N\sum_{j=1}^N v_k\partial_{\Delta u_j}P_k({\bf x},{\bf u},\nabla{\bf u},\Delta{\bf u},p,\nabla p)\sum_{i=1}^N n_i\partial_{x_i}\tilde{u}_j{\rm d}\Gamma\\
            =&\, \int_\Gamma \sum_{i=1}^N\sum_{j=1}^N \left[\partial_{\partial_{x_i}u_j}J_\Gamma({\bf x},{\bf u},\nabla{\bf u},p,{\bf n},{\bf t}) + \sum_{k=1}^N v_k\partial_{\Delta u_j}P_k({\bf x},{\bf u},\nabla{\bf u},\Delta{\bf u},p,\nabla p)n_i\right]\partial_{x_i}\tilde{u}_j{\rm d}\Gamma,
        \end{align*}
        hence
        \begin{align*}
            \int_\Gamma \sum_{i=1}^N\sum_{j=1}^N \left[\partial_{\partial_{x_i}u_j}J_\Gamma({\bf x},{\bf u},\nabla{\bf u},p,{\bf n},{\bf t}) + \sum_{k=1}^N v_k\partial_{\Delta u_j}P_k({\bf x},{\bf u},\nabla{\bf u},\Delta{\bf u},p,\nabla p)n_i\right]\partial_{x_i}\tilde{u}_j{\rm d}\Gamma = 0,\ \forall({\bf u},p,\Omega,\tilde{\bf u}) \mbox{ s.t. } \tilde{\bf u}|_\Gamma = {\bf 0}.
        \end{align*}
        This implies that
        \begin{align*}
            \partial_{\partial_{x_i}u_j}J_\Gamma({\bf x},{\bf u},\nabla{\bf u},p,{\bf n},{\bf t}) + \sum_{k=1}^N v_k\partial_{\Delta u_j}P_k({\bf x},{\bf u},\nabla{\bf u},\Delta{\bf u},p,\nabla p)n_i = 0,\ \forall i,j = 1,\ldots,N.
        \end{align*}
        To determine ${\bf v}$, we only need to solve one of the following $N$ linear equations:
        \begin{align*}
            \forall j = 1,\ldots,N,\ j^{\rm th}\mbox{ linear system}:\ 
            \sum_{k=1}^N v_k\partial_{\Delta u_j}P_k({\bf x},{\bf u},\nabla{\bf u},\Delta{\bf u},p,\nabla p)n_i = -\partial_{\partial_{x_i}u_j}J_\Gamma({\bf x},{\bf u},\nabla{\bf u},p,{\bf n},{\bf t}),\ \forall i = 1,\ldots,N.
        \end{align*}
    \end{remark}    
    \item \textbf{Case $\delta_{\mathcal{L}} = 1$.} This means to ``deactivate'' the boundary-condition constraint ${\bf Q}({\bf x},{\bf u},\nabla{\bf u},p,{\bf n},{\bf t}) = {\bf f}_{\rm bc}({\bf x})$ on $\Gamma$, so it will be penalized by the extended Lagrangian $\mathcal{L}$.
    
    Again, to see the general structure, we rewrite \eqref{Euler-Lagrange for general stationary fluid dynamics PDEs} as follows:
    \begin{equation}
        \label{brief extended Euler-Lagrange for general stationary fluid dynamics PDEs}
        \tag{brief-exEuLa-gfld}
        \left.\begin{split}
            &\int_\Omega \boldsymbol{\mathcal{F}}_\Omega^{\tilde{\bf u}}({\bf x},{\bf u},\nabla{\bf u},\Delta{\bf u},p,\nabla p,{\bf v},\nabla{\bf v},\Delta{\bf v},q,\nabla q)\cdot\tilde{\bf u}{\rm d}{\bf x} + \int_\Omega \mathcal{F}_\Omega^{\tilde{p}}({\bf x},{\bf u},\nabla{\bf u},\Delta{\bf u},p,\nabla p,{\bf v},\nabla{\bf v},q)\tilde{p}{\rm d}{\bf x}\\
            &\hspace{5mm}+ \int_\Gamma \boldsymbol{\mathcal{F}}_\Gamma^{\tilde{\bf u}}({\bf x},{\bf u},\nabla{\bf u},\Delta{\bf u},p,\nabla p,{\bf v},\nabla{\bf v},q,{\bf v}_{\rm bc},{\bf n},{\bf t})\cdot\tilde{\bf u}{\rm d}\Gamma + \int_\Gamma \mathcal{F}_\Gamma^{\tilde{p}}({\bf x},{\bf u},\nabla{\bf u},\Delta{\bf u},p,\nabla p,{\bf v},{\bf v}_{\rm bc},{\bf n},{\bf t})\tilde{p}{\rm d}\Gamma\\
            &\hspace{5mm}+ \int_\Gamma \mathcal{F}_\Gamma^{\nabla\tilde{\bf u}}({\bf x},{\bf u},\nabla{\bf u},\Delta{\bf u},p,\nabla p,{\bf v},{\bf v}_{\rm bc},{\bf n},{\bf t},\nabla\tilde{\bf u}){\rm d}\Gamma = 0,\ \forall({\bf u},p,\Omega,\tilde{\bf u},\tilde{p}),
        \end{split}\right.
    \end{equation}
    where
    \begin{align*}
        &\boldsymbol{\mathcal{F}}_\Omega^{\tilde{\bf u}}({\bf x},{\bf u},\nabla{\bf u},\Delta{\bf u},p,\nabla p,{\bf v},\nabla{\bf v},\Delta{\bf v},q,\nabla q)\\
        &\hspace{5mm}\coloneqq\nabla_{\bf u}J_\Omega({\bf x},{\bf u},\nabla{\bf u},p) - \nabla\cdot\left(\nabla_{\nabla{\bf u}}J_\Omega({\bf x},{\bf u},\nabla{\bf u},p)\right) - \nabla_{\bf u}{\bf P}({\bf x},{\bf u},\nabla{\bf u},\Delta{\bf u},p,\nabla p){\bf v} + \nabla\cdot\left(\nabla_{\nabla{\bf u}}{\bf P}({\bf x},{\bf u},\nabla{\bf u},\Delta{\bf u},p,\nabla p)\right)\cdot{\bf v}\\
        &\hspace{1cm}+ \nabla_{\nabla{\bf u}}{\bf P}({\bf x},{\bf u},\nabla{\bf u},\Delta{\bf u},p,\nabla p):\nabla{\bf v} - \Delta\nabla_{\Delta{\bf u}}{\bf P}({\bf x},{\bf u},\nabla{\bf u},\Delta{\bf u},p,\nabla p){\bf v} - \nabla_{\Delta{\bf u}}{\bf P}({\bf x},{\bf u},\nabla{\bf u},\Delta{\bf u},p,\nabla p)\Delta{\bf v}\\
        &\hspace{1cm}+ \nabla_{\bf u}{\bf f}({\bf x},{\bf u},\nabla{\bf u},p){\bf v} - \nabla\cdot\left(\nabla_{\nabla{\bf u}}{\bf f}({\bf x},{\bf u},\nabla{\bf u},p)\right)\cdot{\bf v} - \nabla_{\nabla{\bf u}}{\bf f}({\bf x},{\bf u},\nabla{\bf u},p):\nabla{\bf v} - \nabla q + q\nabla_{\bf u}f_{\rm div}({\bf x},{\bf u},\nabla{\bf u},p)\\
        &\hspace{1cm}- \nabla_{\nabla{\bf u}}f_{\rm div}({\bf x},{\bf u},\nabla{\bf u},p)\cdot\nabla q - q\nabla\cdot\left(\nabla_{\nabla{\bf u}}f_{\rm div}({\bf x},{\bf u},\nabla{\bf u},p)\right),\\
        &\mathcal{F}_\Omega^{\tilde{p}}({\bf x},{\bf u},\nabla{\bf u},\Delta{\bf u},p,\nabla p,{\bf v},\nabla{\bf v},q)\\
        &\hspace{5mm}\coloneqq D_pJ_\Omega({\bf x},{\bf u},\nabla{\bf u},p) - D_p{\bf P}({\bf x},{\bf u},\nabla{\bf u},\Delta{\bf u},p,\nabla p)\cdot{\bf v} + \nabla_{\nabla p}{\bf P}({\bf x},{\bf u},\nabla{\bf u},\Delta{\bf u},p,\nabla p):\nabla{\bf v}\\
        &\hspace{1cm} + \nabla\cdot\left(\nabla_{\nabla p}{\bf P}({\bf x},{\bf u},\nabla{\bf u},\Delta{\bf u},p,\nabla p)\right)\cdot{\bf v} + D_p{\bf f}({\bf x},{\bf u},\nabla{\bf u},p)\cdot{\bf v} + qD_pf_{\rm div}({\bf x},{\bf u},\nabla{\bf u},p),\\
        &\boldsymbol{\mathcal{F}}_\Gamma^{\tilde{\bf u}}({\bf x},{\bf u},\nabla{\bf u},\Delta{\bf u},p,\nabla p,{\bf v},\nabla{\bf v},q,{\bf v}_{\rm bc},{\bf n},{\bf t})\\
        &\hspace{5mm}\coloneqq\nabla_{\nabla{\bf u}}J_\Omega({\bf x},{\bf u},\nabla{\bf u},p)\cdot{\bf n} + \nabla_{\bf u}J_\Gamma({\bf x},{\bf u},\nabla{\bf u},p,{\bf n},{\bf t}) - \left(\nabla_{\nabla{\bf u}}{\bf P}({\bf x},{\bf u},\nabla{\bf u},\Delta{\bf u},p,\nabla p)\cdot{\bf n}\right)\cdot{\bf v}\\
        &\hspace{1cm} - \partial_{\bf n}\nabla_{\Delta{\bf u}}{\bf P}({\bf x},{\bf u},\nabla{\bf u},\Delta{\bf u},p,\nabla p){\bf v} - \nabla_{\Delta{\bf u}}{\bf P}({\bf x},{\bf u},\nabla{\bf u},\Delta{\bf u},p,\nabla p)\partial_{\bf n}{\bf v} + \left(\nabla_{\nabla{\bf u}}{\bf f}({\bf x},{\bf u},\nabla{\bf u},p)\cdot{\bf n}\right)\cdot{\bf v} + q{\bf n}\\
        &\hspace{1cm} + q\nabla_{\nabla{\bf u}}f_{\rm div}({\bf x},{\bf u},\nabla{\bf u},p)\cdot{\bf n} - \nabla_{\bf u}{\bf Q}({\bf x},{\bf u},\nabla{\bf u},p,{\bf n},{\bf t}){\bf v}_{\rm bc},\\
        &\mathcal{F}_\Gamma^{\tilde{p}}({\bf x},{\bf u},\nabla{\bf u},\Delta{\bf u},p,\nabla p,{\bf v},{\bf v}_{\rm bc},{\bf n},{\bf t})\coloneqq D_pJ_\Gamma({\bf x},{\bf u},\nabla{\bf u},p,{\bf n},{\bf t}) - {\bf n}^\top\nabla_{\nabla p}{\bf P}({\bf x},{\bf u},\nabla{\bf u},\Delta{\bf u},p,\nabla p){\bf v} - D_p{\bf Q}({\bf x},{\bf u},\nabla{\bf u},p,{\bf n},{\bf t})\cdot{\bf v}_{\rm bc},\\
        &\mathcal{F}_\Gamma^{\nabla\tilde{\bf u}}({\bf x},{\bf u},\nabla{\bf u},\Delta{\bf u},p,\nabla p,{\bf v},{\bf v}_{\rm bc},{\bf n},{\bf t},\nabla\tilde{\bf u})\\
        &\hspace{5mm}\coloneqq\nabla_{\nabla{\bf u}}J_\Gamma({\bf x},{\bf u},\nabla{\bf u},p,{\bf n},{\bf t}):\nabla\tilde{\bf u} + {\bf v}^\top D_{\Delta{\bf u}}{\bf P}({\bf x},{\bf u},\nabla{\bf u},\Delta{\bf u},p,\nabla p)\partial_{\bf n}\tilde{\bf u} - D_{\nabla{\bf u}}{\bf Q}({\bf x},{\bf u},\nabla{\bf u},p,{\bf n},{\bf t})\nabla\tilde{\bf u}\cdot{\bf v}_{\rm bc},
    \end{align*}
    for all $({\bf x},{\bf u},p,{\bf v},q,{\bf n},{\bf t},\tilde{\bf u},\tilde{p})$.
    
    We now deduce the adjoint equations of \eqref{general stationary fluid dynamics PDEs} from \eqref{brief extended Euler-Lagrange for general stationary fluid dynamics PDEs} as follows:
    \begin{itemize}
        \item Choose $\tilde{\bf u} = {\bf 0}$ in $\overline{\Omega}$, \eqref{brief extended Euler-Lagrange for general stationary fluid dynamics PDEs} then becomes
        \begin{align*}
            \int_\Omega \mathcal{F}_\Omega^{\tilde{p}}({\bf x},{\bf u},\nabla{\bf u},\Delta{\bf u},p,\nabla p,{\bf v},\nabla{\bf v},q)\tilde{p}{\rm d}{\bf x} + \int_\Gamma \mathcal{F}_\Gamma^{\tilde{p}}({\bf x},{\bf u},\nabla{\bf u},\Delta{\bf u},p,\nabla p,{\bf v},{\bf v}_{\rm bc},{\bf n},{\bf t})\tilde{p}{\rm d}\Gamma = 0,\ \forall({\bf u},p,\Omega,\tilde{p}).
        \end{align*}
        Then choose $\tilde{p}$ varying s.t. $\tilde{p}|_\Gamma = 0$, then the last equality yields
        \begin{align*}
            \int_\Omega \mathcal{F}_\Omega^{\tilde{p}}({\bf x},{\bf u},\nabla{\bf u},\Delta{\bf u},p,\nabla p,{\bf v},\nabla{\bf v},q)\tilde{p}{\rm d}{\bf x} = 0,\ \forall({\bf u},p,\Omega,\tilde{p}) \mbox{ s.t. } \tilde{p}|_\Gamma = 0.
        \end{align*}
        Hence, $({\bf v},q)$ satisfies
        \begin{align}
            \label{extended domain integrand variation p}
            \boxed{\mathcal{F}_\Omega^{\tilde{p}}({\bf x},{\bf u},\nabla{\bf u},\Delta{\bf u},p,\nabla p,{\bf v},\nabla{\bf v},q) = 0 \mbox{ in } \Omega.}
        \end{align}
        Plug it back in, obtain
        \begin{align*}
            \int_\Gamma \mathcal{F}_\Gamma^{\tilde{p}}({\bf x},{\bf u},\nabla{\bf u},\Delta{\bf u},p,\nabla p,{\bf v},{\bf v}_{\rm bc},{\bf n},{\bf t})\tilde{p}{\rm d}\Gamma = 0,\ \forall({\bf u},p,\Omega,\tilde{p}).
        \end{align*}
        Unlike the previous case, $\tilde{p}$ can vary on the whole of $\Gamma$, hence the last equality implies that $({\bf v},q)$ satisfies
        \begin{align}
            \label{extended boundary integrand variation p}
            \boxed{\mathcal{F}_\Gamma^{\tilde{p}}({\bf x},{\bf u},\nabla{\bf u},\Delta{\bf u},p,\nabla p,{\bf v},{\bf v}_{\rm bc},{\bf n},{\bf t}) = 0 \mbox{ on } \Gamma.}
        \end{align}
        \item Assume that $({\bf v},q)$ satisfies \eqref{extended domain integrand variation p} and \eqref{extended boundary integrand variation p}, \eqref{brief extended Euler-Lagrange for general stationary fluid dynamics PDEs} then becomes
        \begin{align*}
            &\int_\Omega \boldsymbol{\mathcal{F}}_\Omega^{\tilde{\bf u}}({\bf x},{\bf u},\nabla{\bf u},\Delta{\bf u},p,\nabla p,{\bf v},\nabla{\bf v},\Delta{\bf v},q,\nabla q)\cdot\tilde{\bf u}{\rm d}{\bf x} + \int_\Gamma \boldsymbol{\mathcal{F}}_\Gamma^{\tilde{\bf u}}({\bf x},{\bf u},\nabla{\bf u},\Delta{\bf u},p,\nabla p,{\bf v},\nabla{\bf v},q,{\bf v}_{\rm bc},{\bf n},{\bf t})\cdot\tilde{\bf u}{\rm d}\Gamma\\
            &\hspace{5mm}+ \int_\Gamma \mathcal{F}_\Gamma^{\nabla\tilde{\bf u}}({\bf x},{\bf u},\nabla{\bf u},\Delta{\bf u},p,\nabla p,{\bf v},{\bf v}_{\rm bc},{\bf n},{\bf t},\nabla\tilde{\bf u}){\rm d}\Gamma = 0,\ \forall({\bf u},p,\Omega,\tilde{\bf u}).
        \end{align*}
        Choose $\tilde{\bf u}$ varying s.t. $\tilde{\bf u}|_\Gamma = {\bf 0}$ and $\nabla\tilde{\bf u}|_\Gamma = {\bf 0}_{N\times N}$, the last equality yields
        \begin{align*}
            \int_\Omega \boldsymbol{\mathcal{F}}_\Omega^{\tilde{\bf u}}({\bf x},{\bf u},\nabla{\bf u},\Delta{\bf u},p,\nabla p,{\bf v},\nabla{\bf v},\Delta{\bf v},q,\nabla q)\cdot\tilde{\bf u}{\rm d}{\bf x} = 0,\ \forall({\bf u},p,\Omega,\tilde{\bf u}) \mbox{ s.t. } \tilde{\bf u}|_\Gamma = {\bf 0} \mbox{ and } \nabla\tilde{\bf u}|_\Gamma = {\bf 0}_{N\times N}.
        \end{align*}
        Hence, $({\bf v},q)$ satisfies
        \begin{align*}
            \boxed{\boldsymbol{\mathcal{F}}_\Omega^{\tilde{\bf u}}({\bf x},{\bf u},\nabla{\bf u},\Delta{\bf u},p,\nabla p,{\bf v},\nabla{\bf v},\Delta{\bf v},q,\nabla q) = {\bf 0} \mbox{ in } \Omega.}
        \end{align*}
        Plug it back in, obtain
        \begin{align*}
            \int_\Gamma \boldsymbol{\mathcal{F}}_\Gamma^{\tilde{\bf u}}({\bf x},{\bf u},\nabla{\bf u},\Delta{\bf u},p,\nabla p,{\bf v},\nabla{\bf v},q,{\bf v}_{\rm bc},{\bf n},{\bf t})\cdot\tilde{\bf u}{\rm d}\Gamma + \int_\Gamma \mathcal{F}_\Gamma^{\nabla\tilde{\bf u}}({\bf x},{\bf u},\nabla{\bf u},\Delta{\bf u},p,\nabla p,{\bf v},{\bf v}_{\rm bc},{\bf n},{\bf t},\nabla\tilde{\bf u}){\rm d}\Gamma = 0,\ \forall({\bf u},p,\Omega,\tilde{\bf u}),
        \end{align*}
        Choose $\tilde{\bf u}$ varying s.t. $\tilde{\bf u}|_\Gamma = {\bf 0}$, the last equality then becomes
        \begin{align*}
            \int_\Gamma \mathcal{F}_\Gamma^{\nabla\tilde{\bf u}}({\bf x},{\bf u},\nabla{\bf u},\Delta{\bf u},p,\nabla p,{\bf v},{\bf v}_{\rm bc},{\bf n},{\bf t},\nabla\tilde{\bf u}){\rm d}\Gamma = 0,\ \forall({\bf u},p,\Omega,\tilde{\bf u}) \mbox{ s.t. } \tilde{\bf u}|_\Gamma = {\bf 0}.
        \end{align*}    
        The l.h.s. equals
        \begin{align*}
            &\int_\Gamma \mathcal{F}_\Gamma^{\nabla\tilde{\bf u}}({\bf x},{\bf u},\nabla{\bf u},\Delta{\bf u},p,\nabla p,{\bf v},{\bf v}_{\rm bc},{\bf n},{\bf t},\nabla\tilde{\bf u}){\rm d}\Gamma\\
            =&\, \int_\Gamma \nabla_{\nabla{\bf u}}J_\Gamma({\bf x},{\bf u},\nabla{\bf u},p,{\bf n},{\bf t}):\nabla\tilde{\bf u} + {\bf v}^\top D_{\Delta{\bf u}}{\bf P}({\bf x},{\bf u},\nabla{\bf u},\Delta{\bf u},p,\nabla p)\partial_{\bf n}\tilde{\bf u} - D_{\nabla{\bf u}}{\bf Q}({\bf x},{\bf u},\nabla{\bf u},p,{\bf n},{\bf t})\nabla\tilde{\bf u}\cdot{\bf v}_{\rm bc}{\rm d}\Gamma\\
            =&\, \int_\Gamma \sum_{i=1}^N\sum_{j=1}^N \partial_{\partial_{x_i}u_j}J_\Gamma({\bf x},{\bf u},\nabla{\bf u},p,{\bf n},{\bf t})\partial_{x_i}\tilde{u}_j + \sum_{i=1}^N\sum_{j=1}^N v_i\partial_{\Delta u_j}P_i({\bf x},{\bf u},\nabla{\bf u},\Delta{\bf u},p,\nabla p)\partial_{\bf n}\tilde{u}_j\\
            &\hspace{1cm}- \sum_{k=1}^N\sum_{i=1}^N\sum_{j=1}^N \partial_{\partial_{x_i}u_j}Q_k({\bf x},{\bf u},\nabla{\bf u},p,{\bf n},{\bf t})\partial_{x_i}\tilde{u}_jv_{{\rm bc},k}{\rm d}\Gamma\\
            =&\, \int_\Gamma \sum_{i=1}^N\sum_{j=1}^N \partial_{\partial_{x_i}u_j}J_\Gamma({\bf x},{\bf u},\nabla{\bf u},p,{\bf n},{\bf t})\partial_{x_i}\tilde{u}_j + \sum_{i=1}^N\sum_{j=1}^N v_i\partial_{\Delta u_j}P_i({\bf x},{\bf u},\nabla{\bf u},\Delta{\bf u},p,\nabla p)\sum_{k=1}^N n_k\partial_{x_k}\tilde{u}_j\\
            &\hspace{1cm}- \sum_{i=1}^N\sum_{j=1}^N \partial_{x_i}\tilde{u}_j\sum_{k=1}^N\partial_{\partial_{x_i}u_j}Q_k({\bf x},{\bf u},\nabla{\bf u},p,{\bf n},{\bf t})v_{{\rm bc},k}{\rm d}\Gamma\\
            =&\, \int_\Gamma \sum_{i=1}^N\sum_{j=1}^N \partial_{\partial_{x_i}u_j}J_\Gamma({\bf x},{\bf u},\nabla{\bf u},p,{\bf n},{\bf t})\partial_{x_i}\tilde{u}_j + \sum_{k=1}^N\sum_{j=1}^N v_k\partial_{\Delta u_j}P_k({\bf x},{\bf u},\nabla{\bf u},\Delta{\bf u},p,\nabla p)\sum_{i=1}^N n_i\partial_{x_i}\tilde{u}_j\\
            &\hspace{1cm}- \sum_{i=1}^N\sum_{j=1}^N \partial_{x_i}\tilde{u}_j\sum_{k=1}^N\partial_{\partial_{x_i}u_j}Q_k({\bf x},{\bf u},\nabla{\bf u},p,{\bf n},{\bf t})v_{{\rm bc},k}{\rm d}\Gamma\\
            =&\, \int_\Gamma \sum_{i=1}^N\sum_{j=1}^N \left[\partial_{\partial_{x_i}u_j}J_\Gamma({\bf x},{\bf u},\nabla{\bf u},p,{\bf n},{\bf t}) + \sum_{k=1}^N v_k\partial_{\Delta u_j}P_k({\bf x},{\bf u},\nabla{\bf u},\Delta{\bf u},p,\nabla p)n_i\right.\\
            &\hspace{2cm}\left. - \sum_{k=1}^N \partial_{\partial_{x_i}u_j}Q_k({\bf x},{\bf u},\nabla{\bf u},p,{\bf n},{\bf t})v_{{\rm bc},k}\right]\partial_{x_i}\tilde{u}_j{\rm d}\Gamma,
        \end{align*}
        hence
        \begin{align*}
            &\int_\Gamma \sum_{i=1}^N\sum_{j=1}^N \left[\partial_{\partial_{x_i}u_j}J_\Gamma({\bf x},{\bf u},\nabla{\bf u},p,{\bf n},{\bf t}) + \sum_{k=1}^N v_k\partial_{\Delta u_j}P_k({\bf x},{\bf u},\nabla{\bf u},\Delta{\bf u},p,\nabla p)n_i\right.\\
            &\hspace{2cm}\left. - \sum_{k=1}^N \partial_{\partial_{x_i}u_j}Q_k({\bf x},{\bf u},\nabla{\bf u},p,{\bf n},{\bf t})v_{{\rm bc},k}\right]\partial_{x_i}\tilde{u}_j{\rm d}\Gamma = 0,\ \forall({\bf u},p,\Omega,\tilde{u}) \mbox{ s.t. } \tilde{\bf u}|_\Gamma = {\bf 0}.
        \end{align*}
        This implies that
        \begin{align*}
            \partial_{\partial_{x_i}u_j}J_\Gamma({\bf x},{\bf u},\nabla{\bf u},p,{\bf n},{\bf t}) + \sum_{k=1}^N v_k\partial_{\Delta u_j}P_k({\bf x},{\bf u},\nabla{\bf u},\Delta{\bf u},p,\nabla p)n_i - \sum_{k=1}^N \partial_{\partial_{x_i}u_j}Q_k({\bf x},{\bf u},\nabla{\bf u},p,{\bf n},{\bf t})v_{{\rm bc},k} = 0,\ \forall i,j = 1,\ldots,N,
        \end{align*}
        or equivalently,
        \begin{align*}
            \boxed{\sum_{k=1}^N v_k\partial_{\Delta u_j}P_k({\bf x},{\bf u},\nabla{\bf u},\Delta{\bf u},p,\nabla p)n_i - \partial_{\partial_{x_i}u_j}Q_k({\bf x},{\bf u},\nabla{\bf u},p,{\bf n},{\bf t})v_{{\rm bc},k} = -\partial_{\partial_{x_i}u_j}J_\Gamma({\bf x},{\bf u},\nabla{\bf u},p,{\bf n},{\bf t}),\ \forall i,j = 1,\ldots,N.}
        \end{align*}
        Assume that the last equality holds, then plug it back in to obtain:
        \begin{align*}
            \int_\Gamma \boldsymbol{\mathcal{F}}_\Gamma^{\tilde{\bf u}}({\bf x},{\bf u},\nabla{\bf u},\Delta{\bf u},p,\nabla p,{\bf v},\nabla{\bf v},q,{\bf v}_{\rm bc},{\bf n},{\bf t})\cdot\tilde{\bf u}{\rm d}\Gamma = 0,\ \forall({\bf u},p,\Omega,\tilde{\bf u}),
        \end{align*}
        hence $({\bf v},q,{\bf v}_{\rm bc})$ satisfies
        \begin{align*}
            \boxed{\boldsymbol{\mathcal{F}}_\Gamma^{\tilde{\bf u}}({\bf x},{\bf u},\nabla{\bf u},\Delta{\bf u},p,\nabla p,{\bf v},\nabla{\bf v},q,{\bf v}_{\rm bc},{\bf n},{\bf t}) = {\bf 0} \mbox{ on } \Gamma.}
        \end{align*}
    \end{itemize}
    Conclude:
    \begin{equation}
        \label{extended adjoint general stationary fluid dynamics PDEs}
        \tag{ex-adj-gfld}
        \boxed{\left\{\begin{split}
                &-\nabla_{\Delta{\bf u}}{\bf P}({\bf x},{\bf u},\nabla{\bf u},\Delta{\bf u},p,\nabla p)\Delta{\bf v} + \left(\nabla_{\nabla{\bf u}}{\bf P}({\bf x},{\bf u},\nabla{\bf u},\Delta{\bf u},p,\nabla p) - \nabla_{\nabla{\bf u}}{\bf f}({\bf x},{\bf u},\nabla{\bf u},p)\right):\nabla{\bf v}\\
                &\hspace{1cm}- \left[\nabla_{\bf u}{\bf P}({\bf x},{\bf u},\nabla{\bf u},\Delta{\bf u},p,\nabla p) + \Delta\nabla_{\Delta{\bf u}}{\bf P}({\bf x},{\bf u},\nabla{\bf u},\Delta{\bf u},p,\nabla p) - \nabla_{\bf u}{\bf f}({\bf x},{\bf u},\nabla{\bf u},p)\right]{\bf v}\\
                &\hspace{1cm}+ \left[\nabla\cdot\left(\nabla_{\nabla{\bf u}}{\bf P}({\bf x},{\bf u},\nabla{\bf u},\Delta{\bf u},p,\nabla p)\right) - \nabla\cdot\left(\nabla_{\nabla{\bf u}}{\bf f}({\bf x},{\bf u},\nabla{\bf u},p)\right)\right]\cdot{\bf v} - \nabla q - \nabla_{\nabla{\bf u}}f_{\rm div}({\bf x},{\bf u},\nabla{\bf u},p)\cdot\nabla q\\
                &\hspace{1cm}+ q\left[-\nabla\cdot\left(\nabla_{\nabla{\bf u}}f_{\rm div}({\bf x},{\bf u},\nabla{\bf u},p)\right) + \nabla_{\bf u}f_{\rm div}({\bf x},{\bf u},\nabla{\bf u},p)\right] = \nabla\cdot\left(\nabla_{\nabla{\bf u}}J_\Omega({\bf x},{\bf u},\nabla{\bf u},p)\right) - \nabla_{\bf u}J_\Omega({\bf x},{\bf u},\nabla{\bf u},p) \mbox{ in } \Omega,\\
                &\nabla_{\nabla p}{\bf P}({\bf x},{\bf u},\nabla{\bf u},\Delta{\bf u},p,\nabla p):\nabla{\bf v} + \left[-D_p{\bf P}({\bf x},{\bf u},\nabla{\bf u},\Delta{\bf u},p,\nabla p) + \nabla\cdot\left(\nabla_{\nabla p}{\bf P}({\bf x},{\bf u},\nabla{\bf u},\Delta{\bf u},p,\nabla p)\right) + D_p{\bf f}({\bf x},{\bf u},\nabla{\bf u},p)\right]\cdot{\bf v}\\
                &\hspace{1cm} = - D_pJ_\Omega ({\bf x},{\bf u},\nabla{\bf u},p) - qD_pf_{\rm div}({\bf x},{\bf u},\nabla{\bf u},p) \mbox{ in } \Omega,\\
                &- \nabla_{\Delta{\bf u}}{\bf P}({\bf x},{\bf u},\nabla{\bf u},\Delta{\bf u},p,\nabla p)\partial_{\bf n}{\bf v} + \left[\left(-\nabla_{\nabla{\bf u}}{\bf P}({\bf x},{\bf u},\nabla{\bf u},\Delta{\bf u},p,\nabla p) + \nabla_{\nabla{\bf u}}{\bf f}({\bf x},{\bf u},\nabla{\bf u},p)\right)\cdot{\bf n}\right]\cdot{\bf v}\\
                &\hspace{1cm}- \partial_{\bf n}\nabla_{\Delta{\bf u}}{\bf P}({\bf x},{\bf u},\nabla{\bf u},\Delta{\bf u},p,\nabla p){\bf v} + q{\bf n} - \nabla_{\bf u}{\bf Q}({\bf x},{\bf u},\nabla{\bf u},p,{\bf n},{\bf t}){\bf v}_{\rm bc}\\
                &\hspace{1cm}= - \nabla_{\nabla{\bf u}}J_\Omega({\bf x},{\bf u},\nabla{\bf u},p)\cdot{\bf n} - \nabla_{\bf u}J_\Gamma({\bf x},{\bf u},\nabla{\bf u},p,{\bf n},{\bf t}) - q\nabla_{\nabla{\bf u}}f_{\rm div}({\bf x},{\bf u},\nabla{\bf u},p)\cdot{\bf n} \mbox{ on } \Gamma,\\
                &{\bf n}^\top\nabla_{\nabla p}{\bf P}({\bf x},{\bf u},\nabla{\bf u},\Delta{\bf u},p,\nabla p){\bf v} + D_p{\bf Q}({\bf x},{\bf u},\nabla{\bf u},p,{\bf n},{\bf t})\cdot{\bf v}_{\rm bc} = D_pJ_\Gamma({\bf x},{\bf u},\nabla{\bf u},p,{\bf n},{\bf t}) \mbox{ on } \Gamma,\\
                &\sum_{k=1}^N v_k\partial_{\Delta u_j}P_k({\bf x},{\bf u},\nabla{\bf u},\Delta{\bf u},p,\nabla p)n_i - \partial_{\partial_{x_i}u_j}Q_k({\bf x},{\bf u},\nabla{\bf u},p,{\bf n},{\bf t})v_{{\rm bc},k} = -\partial_{\partial_{x_i}u_j}J_\Gamma({\bf x},{\bf u},\nabla{\bf u},p,{\bf n},{\bf t}),\ \forall i,j = 1,\ldots,N.
            \end{split}\right.}
    \end{equation}
\end{itemize}

\begin{remark}
    The last equation in \eqref{extended adjoint general stationary fluid dynamics PDEs} can be simplified further when the explicit form of ${\bf P}$ and ${\bf Q}$ are given.
\end{remark}

\subsection{Weak formulation of adjoint equations of \eqref{general stationary fluid dynamics PDEs}*}
Test the 1st 2 equations of \eqref{adjoint general stationary fluid dynamics PDEs} with ${\bf w}$ and $r$, respectively, over $\Omega$:
\begin{equation*}
    \left\{\begin{split}
        &\int_\Omega {\color{red}\nabla_{\Delta{\bf u}}{\bf P}({\bf x},{\bf u},\nabla{\bf u},\Delta{\bf u},p,\nabla p)\Delta{\bf v}\cdot{\bf w}} + \left(\nabla_{\nabla{\bf u}}{\bf P}({\bf x},{\bf u},\nabla{\bf u},\Delta{\bf u},p,\nabla p) - \nabla_{\nabla{\bf u}}{\bf f}({\bf x},{\bf u},\nabla{\bf u},p)\right):\nabla{\bf v}\cdot{\bf w}\\
        &\hspace{1cm}- \left[\nabla_{\bf u}{\bf P}({\bf x},{\bf u},\nabla{\bf u},\Delta{\bf u},p,\nabla p) + \Delta\nabla_{\Delta{\bf u}}{\bf P}({\bf x},{\bf u},\nabla{\bf u},\Delta{\bf u},p,\nabla p) - \nabla_{\bf u}{\bf f}({\bf x},{\bf u},\nabla{\bf u},p)\right]{\bf v}\cdot{\bf w}\\
        &\hspace{1cm}+ \left[\left[\nabla\cdot\left(\nabla_{\nabla{\bf u}}{\bf P}({\bf x},{\bf u},\nabla{\bf u},\Delta{\bf u},p,\nabla p)\right) - \nabla\cdot\left(\nabla_{\nabla{\bf u}}{\bf f}({\bf x},{\bf u},\nabla{\bf u},p)\right)\right]\cdot{\bf v}\right]\cdot{\bf w} - {\color{red}\nabla q\cdot{\bf w}} + {\color{red}\left(\nabla_{\nabla{\bf u}}f_{\rm div}({\bf x},{\bf u},\nabla{\bf u},p)\cdot\nabla q\right)\cdot{\bf w}}\\
        &\hspace{1cm}+ q\left[\nabla\cdot\left(\nabla_{\nabla{\bf u}}f_{\rm div}({\bf x},{\bf u},\nabla{\bf u},p)\right) - \nabla_{\bf u}f_{\rm div}({\bf x},{\bf u},\nabla{\bf u},p)\right]\cdot{\bf w}{\rm d}{\bf x}\\
        &\hspace{5mm} = \int_\Omega \nabla\cdot\left(\nabla_{\nabla{\bf u}}J_\Omega({\bf x},{\bf u},\nabla{\bf u},p)\right)\cdot{\bf w} - \nabla_{\bf u}J_\Omega({\bf x},{\bf u},\nabla{\bf u},p)\cdot{\bf w}{\rm d}{\bf x},\\
        &\int_\Omega r\nabla_{\nabla p}{\bf P}({\bf x},{\bf u},\nabla{\bf u},\Delta{\bf u},p,\nabla p):\nabla{\bf v}\\
        &\hspace{1cm} + r\left[-D_p{\bf P}({\bf x},{\bf u},\nabla{\bf u},\Delta{\bf u},p,\nabla p) + \nabla\cdot\left(\nabla_{\nabla p}{\bf P}({\bf x},{\bf u},\nabla{\bf u},\Delta{\bf u},p,\nabla p)\right) + D_p{\bf f}({\bf x},{\bf u},\nabla{\bf u},p)\right]\cdot{\bf v}{\rm d}{\bf x}\\
        &\hspace{5mm} = \int_\Omega - rD_pJ_\Omega ({\bf x},{\bf u},\nabla{\bf u},p) + qrD_pf_{\rm div}({\bf x},{\bf u},\nabla{\bf u},p){\rm d}{\bf x}.
    \end{split}\right.
\end{equation*}
Integrate by parts:
\begin{enumerate}[leftmargin=0in]
    \item Term $\nabla_{\Delta{\bf u}}{\bf P}({\bf x},{\bf u},\nabla{\bf u},\Delta{\bf u},p,\nabla p)\Delta{\bf v}\cdot{\bf w}$:
    \begin{align*}
        &\int_\Omega -\nabla_{\Delta{\bf u}}{\bf P}({\bf x},{\bf u},\nabla{\bf u},\Delta{\bf u},p,\nabla p)\Delta{\bf v}\cdot{\bf w}{\rm d}{\bf x} = \int_\Omega \left(\sum_{j=1}^N \partial_{\Delta u_i}P_j({\bf x},{\bf u},\nabla{\bf u},\Delta{\bf u},p,\nabla p)\Delta v_j\right)_{i=1}^N\cdot{\bf w}{\rm d}{\bf x}\\
        =&\, \int_\Omega \sum_{i=1}^N\sum_{j=1}^N w_i\partial_{\Delta u_i}P_j({\bf x},{\bf u},\nabla{\bf u},\Delta{\bf u},p,\nabla p)\Delta v_j{\rm d}{\bf x} = \sum_{i=1}^N\sum_{j=1}^N \int_\Omega w_i\partial_{\Delta u_i}P_j({\bf x},{\bf u},\nabla{\bf u},\Delta{\bf u},p,\nabla p)\Delta v_j{\rm d}{\bf x}\\
        =&\, \sum_{i=1}^N\sum_{j=1}^N -\int_\Omega \nabla v_j\cdot\nabla w_i\partial_{\Delta u_i}P_j({\bf x},{\bf u},\nabla{\bf u},\Delta{\bf u},p,\nabla p) + \nabla v_j\cdot\nabla\partial_{\Delta u_i}P_j({\bf x},{\bf u},\nabla{\bf u},\Delta{\bf u},p,\nabla p)w_i{\rm d}{\bf x}\\
        &\hspace{15mm}+ \int_\Gamma \partial_{\bf n}v_jw_i\partial_{\Delta u_i}P_j({\bf x},{\bf u},\nabla{\bf u},\Delta{\bf u},p,\nabla p){\rm d}\Gamma\\
        =&\, -\int_\Omega \sum_{j=1}^N \nabla v_j\cdot\sum_{i=1}^N \nabla w_i\partial_{\Delta u_i}P_j({\bf x},{\bf u},\nabla{\bf u},\Delta{\bf u},p,\nabla p) + \sum_{j=1}^N \nabla v_j\cdot\sum_{i=1}^N w_i\nabla\partial_{\Delta u_i}P_j({\bf x},{\bf u},\nabla{\bf u},\Delta{\bf u},p,\nabla p){\rm d}{\bf x}\\
        &+ \int_\Gamma \sum_{i=1}^N\sum_{j=1}^N w_i\partial_{\Delta u_i}P_j({\bf x},{\bf u},\nabla{\bf u},\Delta{\bf u},p,\nabla p)\partial_{\bf n}v_j{\rm d}\Gamma\\
        =&\, -\int_\Omega \sum_{j=1}^N \nabla v_j\cdot\left(\nabla_{\Delta{\bf u}}P_j({\bf x},{\bf u},\nabla{\bf u},\Delta{\bf u},p,\nabla p)\cdot\nabla{\bf w}\right) + \sum_{j=1}^N \nabla v_j\cdot\left(\nabla\nabla_{\Delta{\bf u}}P_j({\bf x},{\bf u},\nabla{\bf u},\Delta{\bf u},p,\nabla p)\cdot{\bf w}\right){\rm d}{\bf x}\\
        &+ \int_\Gamma {\bf w}^\top\nabla_{\Delta{\bf u}}{\bf P}({\bf x},{\bf u},\nabla{\bf u},\Delta{\bf u},p,\nabla p)\partial_{\bf n}{\bf v}{\rm d}\Gamma\\
        =&\, -\int_\Omega \nabla{\bf v}:\left(\nabla_{\Delta{\bf u}}{\bf P}({\bf x},{\bf u},\nabla{\bf u},\Delta{\bf u},p,\nabla p)\cdot\nabla{\bf w}\right) + \nabla{\bf v}:\left(\nabla\nabla_{\Delta{\bf u}}{\bf P}({\bf x},{\bf u},\nabla{\bf u},\Delta{\bf u},p,\nabla p)\cdot{\bf w}\right){\rm d}{\bf x}\\
        &+ \int_\Gamma {\bf w}^\top\nabla_{\Delta{\bf u}}{\bf P}({\bf x},{\bf u},\nabla{\bf u},\Delta{\bf u},p,\nabla p)\partial_{\bf n}{\bf v}{\rm d}\Gamma.
    \end{align*}
    \item Term $-\nabla q\cdot{\bf w}$:
    \begin{align*}
        -\int_\Omega \nabla q\cdot{\bf w}{\rm d}{\bf x} = \int_\Omega q\nabla\cdot{\bf w}{\rm d}{\bf x} - \int_\Gamma q{\bf w}\cdot{\bf n}{\rm d}\Gamma.
    \end{align*}
    \item Term $\left(\nabla_{\nabla{\bf u}}f_{\rm div}({\bf x},{\bf u},\nabla{\bf u},p)\cdot\nabla q\right)\cdot{\bf w}$:
    \begin{align*}
        &\int_\Omega \left(\nabla_{\nabla{\bf u}}f_{\rm div}({\bf x},{\bf u},\nabla{\bf u},p)\cdot\nabla q\right)\cdot{\bf w}{\rm d}{\bf x} = \int_\Omega \nabla^\top q\nabla_{\nabla{\bf u}}f_{\rm div}({\bf x},{\bf u},\nabla{\bf u},p){\bf w}{\rm d}{\bf x}\\
        =&\, \int_\Omega \sum_{i=1}^N\sum_{j=1}^N \partial_{x_i}q\partial_{\partial_{x_i}u_j}f_{\rm div}({\bf x},{\bf u},\nabla{\bf u},p)w_j{\rm d}{\bf x} = \sum_{i=1}^N\sum_{j=1}^N \int_\Omega \partial_{x_i}q\partial_{\partial_{x_i}u_j}f_{\rm div}({\bf x},{\bf u},\nabla{\bf u},p)w_j{\rm d}{\bf x}\\
        =&\, \sum_{i=1}^N\sum_{j=1}^N -\int_\Omega q\partial_{x_i}\partial_{\partial_{x_i}u_j}f_{\rm div}({\bf x},{\bf u},\nabla{\bf u},p)w_j + q\partial_{\partial_{x_i}u_j}f_{\rm div}({\bf x},{\bf u},\nabla{\bf u},p)\partial_{x_i}w_j{\rm d}{\bf x} + \int_\Gamma qn_i\partial_{\partial_{x_i}u_j}f_{\rm div}({\bf x},{\bf u},\nabla{\bf u},p)w_j{\rm d}\Gamma\\
        =&\, -\int_\Omega q\sum_{j=1}^N w_j\sum_{i=1}^N \partial_{x_i}\partial_{\partial_{x_i}u_j}f_{\rm div}({\bf x},{\bf u},\nabla{\bf u},p) + q\sum_{i=1}^N\sum_{j=1}^N \partial_{\partial_{x_i}u_j}f_{\rm div}({\bf x},{\bf u},\nabla{\bf u},p)\partial_{x_i}w_j{\rm d}{\bf x}\\
        &+ \int_\Gamma q\sum_{i=1}^N\sum_{j=1}^N n_i\partial_{\partial_{x_i}u_j}f_{\rm div}({\bf x},{\bf u},\nabla{\bf u},p)w_j{\rm d}\Gamma\\
        =&\, -\int_\Omega q\sum_{j=1}^N w_j\nabla\cdot\left(\nabla_{\nabla u_j}f_{\rm div}({\bf x},{\bf u},\nabla{\bf u},p)\right) + q\nabla_{\nabla{\bf u}}f_{\rm div}({\bf x},{\bf u},\nabla{\bf u},p):\nabla{\bf w}{\rm d}{\bf x} + \int_\Gamma q{\bf n}^\top\nabla_{\nabla{\bf u}}f_{\rm div}({\bf x},{\bf u},\nabla{\bf u},p){\bf w}{\rm d}\Gamma\\
        =&\, -\int_\Omega q\nabla\cdot\left(\nabla_{\nabla{\bf u}}f_{\rm div}({\bf x},{\bf u},\nabla{\bf u},p)\right)\cdot{\bf w} + q\nabla_{\nabla{\bf u}}f_{\rm div}({\bf x},{\bf u},\nabla{\bf u},p):\nabla{\bf w}{\rm d}{\bf x} + \int_\Gamma q{\bf n}^\top\nabla_{\nabla{\bf u}}f_{\rm div}({\bf x},{\bf u},\nabla{\bf u},p){\bf w}{\rm d}\Gamma.
    \end{align*}
\end{enumerate}
Plug in to obtain:
\begin{align*}
    &\int_\Omega -\nabla{\bf v}:\left(\nabla_{\Delta{\bf u}}{\bf P}({\bf x},{\bf u},\nabla{\bf u},\Delta{\bf u},p,\nabla p)\cdot\nabla{\bf w}\right) + \nabla{\bf v}:\left(\nabla\nabla_{\Delta{\bf u}}{\bf P}({\bf x},{\bf u},\nabla{\bf u},\Delta{\bf u},p,\nabla p)\cdot{\bf w}\right)\\
    &\hspace{1cm} + \left(\nabla_{\nabla{\bf u}}{\bf P}({\bf x},{\bf u},\nabla{\bf u},\Delta{\bf u},p,\nabla p) + \nabla_{\nabla{\bf u}}{\bf f}({\bf x},{\bf u},\nabla{\bf u},p)\right):\nabla{\bf v}\cdot{\bf w}\\
    &\hspace{1cm} - \left[\nabla_{\bf u}{\bf P}({\bf x},{\bf u},\nabla{\bf u},\Delta{\bf u},p,\nabla p) + \Delta\nabla_{\Delta{\bf u}}{\bf P}({\bf x},{\bf u},\nabla{\bf u},\Delta{\bf u},p,\nabla p) - \nabla_{\bf u}{\bf f}({\bf x},{\bf u},\nabla{\bf u},p)\right]{\bf v}\cdot{\bf w}\\
    &\hspace{1cm}+ \left[\left[\nabla\cdot\left(\nabla_{\nabla{\bf u}}{\bf P}({\bf x},{\bf u},\nabla{\bf u},\Delta{\bf u},p,\nabla p)\right) - \nabla\cdot\left(\nabla_{\nabla{\bf u}}{\bf f}({\bf x},{\bf u},\nabla{\bf u},p)\right)\right]\cdot{\bf v}\right]\cdot{\bf w} + q\nabla\cdot{\bf w} - q\nabla\cdot\left(\nabla_{\nabla{\bf u}}f_{\rm div}({\bf x},{\bf u},\nabla{\bf u},p)\right)\cdot{\bf w}\\
    &\hspace{1cm}+ q\nabla_{\nabla{\bf u}}f_{\rm div}({\bf x},{\bf u},\nabla{\bf u},p):\nabla{\bf w} + q\left[\nabla\cdot\left(\nabla_{\nabla{\bf u}}f_{\rm div}({\bf x},{\bf u},\nabla{\bf u},p)\right) - \nabla_{\bf u}f_{\rm div}({\bf x},{\bf u},\nabla{\bf u},p)\right]\cdot{\bf w}{\rm d}{\bf x}\\
    &+ \int_\Gamma {\bf w}^\top\nabla_{\Delta{\bf u}}{\bf P}({\bf x},{\bf u},\nabla{\bf u},\Delta{\bf u},p,\nabla p)\partial_{\bf n}{\bf v} - q{\bf w}\cdot{\bf n} + q{\bf n}^\top\nabla_{\nabla{\bf u}}f_{\rm div}({\bf x},{\bf u},\nabla{\bf u},p){\bf w}{\rm d}\Gamma\\
    &\hspace{1cm} = \int_\Omega \nabla\cdot\left(\nabla_{\nabla{\bf u}}J_\Omega({\bf x},{\bf u},\nabla{\bf u},p)\right)\cdot{\bf w} - \nabla_{\bf u}J_\Omega({\bf x},{\bf u},\nabla{\bf u},p)\cdot{\bf w}{\rm d}{\bf x}.
\end{align*}
Note that
\begin{align*}
    &\int_{\Gamma\backslash\Gamma_{\rm D}^{\bf u}} {\bf w}^\top\nabla_{\Delta{\bf u}}{\bf P}({\bf x},{\bf u},\nabla{\bf u},\Delta{\bf u},p,\nabla p)\partial_{\bf n}{\bf v} - q{\bf w}\cdot{\bf n} + q{\bf n}^\top\nabla_{\nabla{\bf u}}f_{\rm div}({\bf x},{\bf u},\nabla{\bf u},p){\bf w}{\rm d}\Gamma\\
    =&\, \int_{\Gamma\backslash\Gamma_{\rm D}^{\bf u}} \left[\nabla_{\nabla{\bf u}}J_\Omega({\bf x},{\bf u},\nabla{\bf u},p)\cdot{\bf n} + \nabla_{\bf u}J_\Gamma({\bf x},{\bf u},\nabla{\bf u},p,{\bf n},{\bf t}) - \left(\nabla_{\nabla{\bf u}}{\bf P}({\bf x},{\bf u},\nabla{\bf u},\Delta{\bf u},p,\nabla p)\cdot{\bf n}\right)\cdot{\bf v}\right.\\
    &\hspace{1cm} \left.- \partial_{\bf n}\nabla_{\Delta{\bf u}}{\bf P}({\bf x},{\bf u},\nabla{\bf u},\Delta{\bf u},p,\nabla p){\bf v} + \left(\nabla_{\nabla{\bf u}}{\bf f}({\bf x},{\bf u},\nabla{\bf u},p)\cdot{\bf n}\right)\cdot{\bf v} - \delta_{\mathcal{L}}\nabla_{\bf u}{\bf Q}({\bf x},{\bf u},\nabla{\bf u},p,{\bf n},{\bf t}){\bf v}_{\rm bc}\right]\cdot{\bf w}{\rm d}\Gamma\\
    &+ \int_{\Gamma\backslash\Gamma_{\rm D}^p} r\left[D_pJ_\Gamma({\bf x},{\bf u},\nabla{\bf u},p,{\bf n},{\bf t}) - {\bf n}^\top\nabla_{\nabla p}{\bf P}({\bf x},{\bf u},\nabla{\bf u},\Delta{\bf u},p,\nabla p){\bf v} - \delta_{\mathcal{L}}D_p{\bf Q}({\bf x},{\bf u},\nabla{\bf u},p,{\bf n},{\bf t})\cdot{\bf v}_{\rm bc}\right]{\rm d}\Gamma\\
    &+ \int_\Gamma \nabla_{\nabla{\bf u}}J_\Gamma({\bf x},{\bf u},\nabla{\bf u},p,{\bf n},{\bf t}):\nabla{\bf w} + {\bf v}^\top D_{\Delta{\bf u}}{\bf P}({\bf x},{\bf u},\nabla{\bf u},\Delta{\bf u},p,\nabla p)\partial_{\bf n}{\bf w} - \delta_{\mathcal{L}}D_{\nabla{\bf u}}{\bf Q}({\bf x},{\bf u},\nabla{\bf u},p,{\bf n},{\bf t})\nabla{\bf w}\cdot{\bf v}_{\rm bc}{\rm d}\Gamma,
\end{align*}
Thus, obtain the following weak formulation of \eqref{adjoint general stationary fluid dynamics PDEs}:
\begin{equation}
    \label{weak form adjoint general stationary fluid dynamics PDEs}
    \tag{wf-adj-gfld}
    \left\{\begin{split}
        &\int_\Omega -\nabla{\bf v}:\left(\nabla_{\Delta{\bf u}}{\bf P}({\bf x},{\bf u},\nabla{\bf u},\Delta{\bf u},p,\nabla p)\cdot\nabla{\bf w}\right) + \nabla{\bf v}:\left(\nabla\nabla_{\Delta{\bf u}}{\bf P}({\bf x},{\bf u},\nabla{\bf u},\Delta{\bf u},p,\nabla p)\cdot{\bf w}\right)\\
        &\hspace{1cm} + \left(\nabla_{\nabla{\bf u}}{\bf P}({\bf x},{\bf u},\nabla{\bf u},\Delta{\bf u},p,\nabla p) - \nabla_{\nabla{\bf u}}{\bf f}({\bf x},{\bf u},\nabla{\bf u},p)\right):\nabla{\bf v}\cdot{\bf w}\\
        &\hspace{1cm} - \left[\nabla_{\bf u}{\bf P}({\bf x},{\bf u},\nabla{\bf u},\Delta{\bf u},p,\nabla p) + \Delta\nabla_{\Delta{\bf u}}{\bf P}({\bf x},{\bf u},\nabla{\bf u},\Delta{\bf u},p,\nabla p) - \nabla_{\bf u}{\bf f}({\bf x},{\bf u},\nabla{\bf u},p)\right]{\bf v}\cdot{\bf w}\\
        &\hspace{1cm}+ \left[\left[\nabla\cdot\left(\nabla_{\nabla{\bf u}}{\bf P}({\bf x},{\bf u},\nabla{\bf u},\Delta{\bf u},p,\nabla p)\right) - \nabla\cdot\left(\nabla_{\nabla{\bf u}}{\bf f}({\bf x},{\bf u},\nabla{\bf u},p)\right)\right]\cdot{\bf v}\right]\cdot{\bf w} + q\nabla\cdot{\bf w} - q\nabla\cdot\left(\nabla_{\nabla{\bf u}}f_{\rm div}({\bf x},{\bf u},\nabla{\bf u},p)\right)\cdot{\bf w}\\
        &\hspace{1cm}+ q\nabla_{\nabla{\bf u}}f_{\rm div}({\bf x},{\bf u},\nabla{\bf u},p):\nabla{\bf w} + q\left[\nabla\cdot\left(\nabla_{\nabla{\bf u}}f_{\rm div}({\bf x},{\bf u},\nabla{\bf u},p)\right) - \nabla_{\bf u}f_{\rm div}({\bf x},{\bf u},\nabla{\bf u},p)\right]\cdot{\bf w}{\rm d}{\bf x}\\
        &\hspace{5mm}+ \int_\Gamma {\bf w}^\top\nabla_{\Delta{\bf u}}{\bf P}({\bf x},{\bf u},\nabla{\bf u},\Delta{\bf u},p,\nabla p)\partial_{\bf n}{\bf v} - q{\bf w}\cdot{\bf n} + q{\bf n}^\top\nabla_{\nabla{\bf u}}f_{\rm div}({\bf x},{\bf u},\nabla{\bf u},p){\bf w}{\rm d}\Gamma\\
        &\hspace{1cm} = \int_\Omega \nabla\cdot\left(\nabla_{\nabla{\bf u}}J_\Omega({\bf x},{\bf u},\nabla{\bf u},p)\right)\cdot{\bf w} - \nabla_{\bf u}J_\Omega({\bf x},{\bf u},\nabla{\bf u},p)\cdot{\bf w}{\rm d}{\bf x},\\
        &\int_\Omega r\nabla_{\nabla p}{\bf P}({\bf x},{\bf u},\nabla{\bf u},\Delta{\bf u},p,\nabla p):\nabla{\bf v}\\
        &\hspace{1cm} + r\left[-D_p{\bf P}({\bf x},{\bf u},\nabla{\bf u},\Delta{\bf u},p,\nabla p) + \nabla\cdot\left(\nabla_{\nabla p}{\bf P}({\bf x},{\bf u},\nabla{\bf u},\Delta{\bf u},p,\nabla p)\right) + D_p{\bf f}({\bf x},{\bf u},\nabla{\bf u},p)\right]\cdot{\bf v}{\rm d}{\bf x}\\
        &\hspace{1cm} = \int_\Omega qrD_pf_{\rm div}({\bf x},{\bf u},\nabla{\bf u},p) - rD_pJ_\Omega({\bf x},{\bf u},\nabla{\bf u},p){\rm d}{\bf x}.
    \end{split}\right.
\end{equation}



\subsection{Shape derivatives of \eqref{general stationary fluid dynamics PDEs}-constrained \eqref{cost functional of general stationary fluid dynamics PDEs}}
To calculate the shape derivatives of \eqref{cost functional of general stationary fluid dynamics PDEs} under the constraint state equation \eqref{general stationary fluid dynamics PDEs}, consider the \textit{perturbed cost functional}:
\begin{align}
    \label{perturbed cost functional of general stationary fluid dynamics PDEs}
    \tag{ptb-cost-gfld}
    J({\bf u}_t,p_t,\Omega_t) := \int_{\Omega_t} J_\Omega({\bf x},{\bf u}_t,\nabla{\bf u}_t,p_t){\rm d}{\bf x} + \int_{\Gamma_t} J_\Gamma({\bf x},{\bf u}_t,\nabla{\bf u}_t,p_t,{\bf n}_t,{\bf t}_t){\rm d}\Gamma_t,
\end{align}
where $({\bf u}_t,p_t)$ denotes the strong/classical solution (if exist and unique) of \eqref{general stationary fluid dynamics PDEs} on the perturbed domain $\Omega_t := T_t(V)(\Omega)$, i.e.:
\begin{equation}
    \label{perturbed general stationary fluid dynamics PDEs}
    \tag{ptb-gfld}
    \left\{\begin{split}
        {\bf P}({\bf x},{\bf u}_t,\nabla{\bf u}_t,\Delta{\bf u}_t,p_t,\nabla p_t) &= {\bf f}({\bf x},{\bf u}_t,\nabla{\bf u}_t,p_t) &&\mbox{ in } \Omega_t,\\
        -\nabla\cdot{\bf u}_t &= f_{\rm div}({\bf x},{\bf u}_t,\nabla{\bf u}_t,p_t) &&\mbox{ in } \Omega_t,\\
        {\bf Q}({\bf x},{\bf u}_t,\nabla{\bf u}_t,p_t,{\bf n}_t,{\bf t}_t) &= {\bf f}_{\rm bc}({\bf x}) &&\mbox{ on } \Gamma_t,
    \end{split}\right.
\end{equation}
where $\Gamma_t := \partial\Omega_t$.

Define:
\begin{itemize}
    \item Local shape derivative:
    \begin{align*}
        {\bf u}'({\bf x};V) := \lim_{t\downarrow 0} \frac{{\bf u}_t({\bf x}) - {\bf u}({\bf x})}{t},\ p'({\bf x};V) := \lim_{t\downarrow 0} \frac{p_t({\bf x}) - p({\bf x})}{t},\ \forall{\bf x}\in D.
    \end{align*}
    \item Material derivative:
    \begin{align*}
        {\rm d}{\bf u}({\bf x};V) := \lim_{t\downarrow 0} \frac{{\bf u}_t({\bf x}_t) - {\bf u}({\bf x})}{t},\ {\rm d}p({\bf x};V) := \lim_{t\downarrow 0} \frac{p_t({\bf x}_t) - p({\bf x})}{t}, \mbox{ where } {\bf x}_t := T_t(V)({\bf x}),\ \forall{\bf x}\in D.
    \end{align*}
\end{itemize}
Now subtracting \eqref{perturbed general stationary fluid dynamics PDEs} to \eqref{general stationary fluid dynamics PDEs} side by side, taking $\lim_{t\downarrow 0}$, we obtain:
\begin{equation}
    \label{PDEs local shape derivative for general stationary fluid dynamics PDEs}
    \left\{\begin{split}
        &D_{\bf u}{\bf P}({\bf x},{\bf u},\nabla{\bf u},\Delta{\bf u},p,\nabla p){\bf u}'({\bf x};V) + D_{\nabla{\bf u}}{\bf P}({\bf x},{\bf u},\nabla{\bf u},\Delta{\bf u},p,\nabla p)\nabla{\bf u}'({\bf x};V) + D_{\Delta{\bf u}}{\bf P}({\bf x},{\bf u},\nabla{\bf u},\Delta{\bf u},p,\nabla p)\Delta{\bf u}'({\bf x};V)\\
        &\hspace{5mm}+ D_p{\bf P}({\bf x},{\bf u},\nabla{\bf u},\Delta{\bf u},p,\nabla p)p'({\bf x};V) + D_{\nabla p}{\bf P}({\bf x},{\bf u},\nabla{\bf u},\Delta{\bf u},p,\nabla p)\nabla p'({\bf x};V)\\
        &\hspace{1cm}= D_{\bf u}{\bf f}({\bf x},{\bf u},\nabla{\bf u},p){\bf u}'({\bf x};V) + D_{\nabla{\bf u}}{\bf f}({\bf x},{\bf u},\nabla{\bf u},p)\nabla{\bf u}'({\bf x};V) + D_p{\bf f}({\bf x},{\bf u},\nabla{\bf u},p)p'({\bf x};V) \mbox{ in } \Omega,\\
        &-\nabla\cdot{\bf u}'({\bf x};V) = D_{\bf u}f_{\rm div}({\bf x},{\bf u},\nabla{\bf u},p){\bf u}'({\bf x};V) + D_{\nabla{\bf u}}f_{\rm div}({\bf x},{\bf u},\nabla{\bf u},p)\nabla{\bf u}'({\bf x};V) + D_pf_{\rm div}({\bf x},{\bf u},\nabla{\bf u},p)p'({\bf x};V) \mbox{ in } \Omega,\\
        &D_{\bf u}{\bf Q}({\bf x},{\bf u},\nabla{\bf u},p,{\bf n},{\bf t}){\bf u}'({\bf x};V) + D_{\nabla{\bf u}}{\bf Q}({\bf x},{\bf u},\nabla{\bf u},p,{\bf n},{\bf t})\nabla{\bf u}'({\bf x};V) + D_p{\bf Q}({\bf x},{\bf u},\nabla{\bf u},p,{\bf n},{\bf t})p'({\bf x};V)\\
        &\hspace{5mm}+ D_{\bf n}{\bf Q}({\bf x},{\bf u},\nabla{\bf u},p,{\bf n},{\bf t}){\bf n}'({\bf x};V) + D_{\bf t}{\bf Q}({\bf x},{\bf u},\nabla{\bf u},p,{\bf n},{\bf t}){\bf t}'({\bf x};V) = {\bf 0} \mbox{ on } \Gamma.
    \end{split}\right.
\end{equation}
Now start to compute the 1st-order shape derivative for \eqref{cost functional of general stationary fluid dynamics PDEs}. Applying the domain and boundary formulas for the domain and boundary integrals, respectively, yields the following ``4 combinations'' (2 combinations for each of domain and boundary integrals, but only 2 of 4 presented here for brevity):
\begin{align*}
    dJ({\bf u},p,\Omega;V) =&\, \int_\Omega J_\Omega'({\bf x},{\bf u},\nabla{\bf u},p;V) + \nabla\cdot\left(J_\Omega({\bf x},{\bf u},\nabla{\bf u},p)V(0)\right){\rm d}{\bf x}\\
    &+ \int_\Gamma J_\Gamma'({\bf x},{\bf u},\nabla{\bf u},p,{\bf n},{\bf t};V) + \nabla\left(J_\Gamma({\bf x},{\bf u},\nabla{\bf u},p,{\bf n},{\bf t})\right)\cdot V(0) + J_\Gamma({\bf x},{\bf u},\nabla{\bf u},p,{\bf n},{\bf t})\left(\nabla\cdot V(0) - DV(0){\bf n}\cdot{\bf n}\right){\rm d}\Gamma\\
    =&\, \int_\Omega J_\Omega'({\bf x},{\bf u},\nabla{\bf u},p;V){\rm d}{\bf x} + \int_\Gamma J_\Omega({\bf x},{\bf u},\nabla{\bf u},p)V(0)\cdot{\bf n}{\rm d}\Gamma\\
    &+ \int_\Gamma J_\Gamma'({\bf x},{\bf u},\nabla{\bf u},p,{\bf n},{\bf t};V) + \left[\partial_{\bf n}(J_\Gamma({\bf x},{\bf u},\nabla{\bf u},p,{\bf n},{\bf t})) + HJ_\Gamma({\bf x},{\bf u},\nabla{\bf u},p,{\bf n},{\bf t})\right]V(0)\cdot{\bf n}{\rm d}\Gamma.
\end{align*}
We now compute these explicitly.
\begin{enumerate}[leftmargin=0in]
    \item \textit{1st representation of shape derivative.}
    \begin{align*}
        &dJ({\bf u},p,\Omega;V)\\
        =&\, \int_\Omega J_\Omega'({\bf x},{\bf u},\nabla{\bf u},p;V) + \nabla\cdot\left(J_\Omega({\bf x},{\bf u},\nabla{\bf u},p)V(0)\right){\rm d}{\bf x}\\
        &+ \int_\Gamma J_\Gamma'({\bf x},{\bf u},\nabla{\bf u},p,{\bf n},{\bf t};V) + \nabla\left(J_\Gamma({\bf x},{\bf u},\nabla{\bf u},p,{\bf n},{\bf t})\right)\cdot V(0) + J_\Gamma({\bf x},{\bf u},\nabla{\bf u},p,{\bf n},{\bf t})\left(\nabla\cdot V(0) - DV(0){\bf n}\cdot{\bf n}\right){\rm d}\Gamma\\
        =&\, \int_\Omega D_{\bf u}J_\Omega({\bf x},{\bf u},\nabla{\bf u},p){\bf u}'({\bf x};V) + D_{\nabla{\bf u}}J_\Omega({\bf x},{\bf u},\nabla{\bf u},p)\nabla{\bf u}'({\bf x};V) + \partial_pJ_\Omega({\bf x},{\bf u},\nabla{\bf u},p)p'({\bf x};V) + \nabla\cdot\left(J_\Omega({\bf x},{\bf u},\nabla{\bf u},p)V(0)\right){\rm d}{\bf x}\\
        &+ \int_\Gamma D_{\bf u}J_\Gamma({\bf x},{\bf u},\nabla{\bf u},p,{\bf n},{\bf t}){\bf u}'({\bf x};V) + D_{\nabla{\bf u}}J_\Gamma({\bf x},{\bf u},\nabla{\bf u},p,{\bf n},{\bf t})\nabla{\bf u}'({\bf x};V) + \partial_pJ_\Gamma({\bf x},{\bf u},\nabla{\bf u},p,{\bf n},{\bf t})p'({\bf x};V)\\
        &\hspace{1cm}+ D_{\bf n}J_\Gamma({\bf x},{\bf u},\nabla{\bf u},p,{\bf n},{\bf t}){\bf n}'({\bf x};V) + D_{\bf t}J_\Gamma({\bf x},{\bf u},\nabla{\bf u},p,{\bf n},{\bf t}){\bf t}'({\bf x};V)\\
        &\hspace{1cm}+ \left[DJ_\Gamma({\bf x},{\bf u},\nabla{\bf u},p,{\bf n},{\bf t}) + D_{\bf u}J_\Gamma({\bf x},{\bf u},\nabla{\bf u},p,{\bf n},{\bf t})D{\bf u} + D_{\nabla{\bf u}}J_\Gamma({\bf x},{\bf u},\nabla{\bf u},p,{\bf n},{\bf t})D\nabla{\bf u} + \partial_pJ_\Gamma({\bf x},{\bf u},\nabla{\bf u},p,{\bf n},{\bf t})Dp\right.\\
        &\hspace{15mm} \left.+ D_{\bf n}J_\Gamma({\bf x},{\bf u},\nabla{\bf u},p,{\bf n},{\bf t})D{\bf n} + D_{\bf t}J_\Gamma({\bf x},{\bf u},\nabla{\bf u},p,{\bf n},{\bf t})D{\bf t}\right]V(0)\\
        &\hspace{1cm}+ J_\Gamma({\bf x},{\bf u},\nabla{\bf u},p,{\bf n},{\bf t})\left(\nabla\cdot V(0) - DV(0){\bf n}\cdot{\bf n}\right){\rm d}\Gamma\\
        =&\, \int_\Omega \nabla_{\bf u}J_\Omega({\bf x},{\bf u},\nabla{\bf u},p)\cdot{\bf u}'({\bf x};V) + \nabla_{\nabla{\bf u}}J_\Omega({\bf x},{\bf u},\nabla{\bf u},p):\nabla{\bf u}'({\bf x};V) + \partial_pJ_\Omega({\bf x},{\bf u},\nabla{\bf u},p)p'({\bf x};V) + \nabla\cdot\left(J_\Omega({\bf x},{\bf u},\nabla{\bf u},p)V(0)\right){\rm d}{\bf x}\\
        &+ \int_\Gamma \nabla_{\bf u}J_\Gamma({\bf x},{\bf u},\nabla{\bf u},p,{\bf n},{\bf t})\cdot{\bf u}'({\bf x};V) + \nabla_{\nabla{\bf u}}J_\Gamma({\bf x},{\bf u},\nabla{\bf u},p,{\bf n},{\bf t}):\nabla{\bf u}'({\bf x};V) + \partial_pJ_\Gamma({\bf x},{\bf u},\nabla{\bf u},p,{\bf n},{\bf t})p'({\bf x};V)\\
        &\hspace{1cm}+ \nabla_{\bf n}J_\Gamma({\bf x},{\bf u},\nabla{\bf u},p,{\bf n},{\bf t})\cdot{\bf n}'({\bf x};V) + \nabla_{\bf t}J_\Gamma({\bf x},{\bf u},\nabla{\bf u},p,{\bf n},{\bf t}):{\bf t}'({\bf x};V)\\
        &\hspace{1cm}+ \left[\nabla J_\Gamma({\bf x},{\bf u},\nabla{\bf u},p,{\bf n},{\bf t}) + \nabla_{\bf u}J_\Gamma({\bf x},{\bf u},\nabla{\bf u},p,{\bf n},{\bf t})\cdot\nabla{\bf u} + \nabla\nabla{\bf u}:\nabla_{\nabla{\bf u}}J_\Gamma({\bf x},{\bf u},\nabla{\bf u},p,{\bf n},{\bf t}) + \partial_pJ_\Gamma({\bf x},{\bf u},\nabla{\bf u},p,{\bf n},{\bf t})\nabla p\right.\\
        &\hspace{15mm} \left.+ \nabla{\bf n}\nabla_{\bf n}J_\Gamma({\bf x},{\bf u},\nabla{\bf u},p,{\bf n},{\bf t}) + \nabla{\bf t}:\nabla_{\bf t}J_\Gamma({\bf x},{\bf u},\nabla{\bf u},p,{\bf n},{\bf t})\right]\cdot V(0)\\
        &\hspace{1cm}+ J_\Gamma({\bf x},{\bf u},\nabla{\bf u},p,{\bf n},{\bf t})\left(\nabla\cdot V(0) - DV(0){\bf n}\cdot{\bf n}\right){\rm d}\Gamma,
    \end{align*}
    where we have expanded $\nabla\left(J_\Gamma({\bf x},{\bf u},\nabla{\bf u},p,{\bf n},{\bf t})\right)$ as follows:
    \begin{align*}
        &J_\Gamma({\bf x},{\bf u},\nabla{\bf u},p,{\bf n},{\bf t})\\
        =&\, J(x_1,\ldots,x_N,u_1,\ldots,u_N,\partial_{x_1}u_1,\ldots,\partial_{x_N}u_1,\ldots,\partial_{x_1}u_N,\ldots,\partial_{x_N}u_N,p,n_1,\ldots,n_N,t_{1,1},\ldots,t_{1,N},\ldots,t_{N-1,1},\ldots,t_{N-1,N}),\\
        &\partial_{x_k}\left(J_\Gamma({\bf x},{\bf u},\nabla{\bf u},p,{\bf n},{\bf t})\right)\\
        =&\, \partial_{x_k}J_\Gamma({\bf x},{\bf u},\nabla{\bf u},p,{\bf n},{\bf t}) + \sum_{i=1}^N \partial_{u_i}J_\Gamma({\bf x},{\bf u},\nabla{\bf u},p,{\bf n},{\bf t})\partial_{x_k}u_i\\
        &+ \sum_{i=1}^N\sum_{j=1}^N \partial_{\partial_{x_i}u_j}J_\Gamma({\bf x},{\bf u},\nabla{\bf u},p,{\bf n},{\bf t})\partial_{x_k}\partial_{x_i}u_j + \partial_pJ_\Gamma({\bf x},{\bf u},\nabla{\bf u},p,{\bf n},{\bf t})\partial_{x_k}p + \sum_{i=1}^N \partial_{n_i}J_\Gamma({\bf x},{\bf u},\nabla{\bf u},p,{\bf n},{\bf t})\partial_{x_k}n_i\\
        &+ \sum_{i=1}^{N-1}\sum_{j=1}^N \partial_{t_{i,j}}J_\Gamma({\bf x},{\bf u},\nabla{\bf u},p,{\bf n},{\bf t})\partial_{x_k}t_{i,j}\\
        =&\, \partial_{x_k}J_\Gamma({\bf x},{\bf u},\nabla{\bf u},p,{\bf n},{\bf t}) + \nabla_{\bf u}J_\Gamma({\bf x},{\bf u},\nabla{\bf u},p,{\bf n},{\bf t})\cdot\partial_{x_k}{\bf u} + \nabla_{\nabla{\bf u}}J_\Gamma({\bf x},{\bf u},\nabla{\bf u},p,{\bf n},{\bf t}):\partial_{x_k}\nabla{\bf u} + \partial_pJ_\Gamma({\bf x},{\bf u},\nabla{\bf u},p,{\bf n},{\bf t})\partial_{x_k}p\\
        &+ \nabla_{\bf n}J_\Gamma({\bf x},{\bf u},\nabla{\bf u},p,{\bf n},{\bf t})\cdot\partial_{x_k}{\bf n} + \sum_{i=1}^{N-1} \nabla_{{\bf t}_i}J_\Gamma({\bf x},{\bf u},\nabla{\bf u},p,{\bf n},{\bf t})\cdot\partial_{x_k}{\bf t}_i\\
        =&\, \partial_{x_k}J_\Gamma({\bf x},{\bf u},\nabla{\bf u},p,{\bf n},{\bf t}) + \nabla_{\bf u}J_\Gamma({\bf x},{\bf u},\nabla{\bf u},p,{\bf n},{\bf t})\cdot\partial_{x_k}{\bf u} + \nabla_{\nabla{\bf u}}J_\Gamma({\bf x},{\bf u},\nabla{\bf u},p,{\bf n},{\bf t}):\partial_{x_k}\nabla{\bf u} + \partial_pJ_\Gamma({\bf x},{\bf u},\nabla{\bf u},p,{\bf n},{\bf t})\partial_{x_k}p\\
        &+ \nabla_{\bf n}J_\Gamma({\bf x},{\bf u},\nabla{\bf u},p,{\bf n},{\bf t})\cdot\partial_{x_k}{\bf n} + \nabla_{\bf t}J_\Gamma({\bf x},{\bf u},\nabla{\bf u},p,{\bf n},{\bf t}):\partial_{x_k}{\bf t},
    \end{align*}
    hence
    \begin{align*}
        \nabla\left(J_\Gamma({\bf x},{\bf u},\nabla{\bf u},p,{\bf n},{\bf t})\right) =&\, \nabla J_\Gamma({\bf x},{\bf u},\nabla{\bf u},p,{\bf n},{\bf t}) + \nabla{\bf u}\nabla_{\bf u}J_\Gamma({\bf x},{\bf u},\nabla{\bf u},p,{\bf n},{\bf t}) + \nabla\nabla{\bf u}:\nabla_{\nabla{\bf u}}J_\Gamma({\bf x},{\bf u},\nabla{\bf u},p,{\bf n},{\bf t})\\
        &+ \partial_pJ_\Gamma({\bf x},{\bf u},\nabla{\bf u},p,{\bf n},{\bf t})\nabla p + \nabla{\bf n}\nabla_{\bf n}J_\Gamma({\bf x},{\bf u},\nabla{\bf u},p,{\bf n},{\bf t}) + \nabla{\bf t}:\nabla_{\bf t}J_\Gamma({\bf x},{\bf u},\nabla{\bf u},p,{\bf n},{\bf t}).
    \end{align*}
    Integrate by parts the term $\nabla_{\nabla{\bf u}}J_\Omega({\bf x},{\bf u},\nabla{\bf u},p):\nabla{\bf u}'({\bf x};V)$ in $dJ({\bf u},p,\Omega;V)$:
    \begin{align*}
        &\int_\Omega \nabla_{\nabla{\bf u}}J_\Omega({\bf x},{\bf u},\nabla{\bf u},p):\nabla{\bf u}'({\bf x};V){\rm d}{\bf x} = \int_\Omega \sum_{i=1}^N\sum_{j=1}^N \partial_{\partial_{x_i}u_j}J_\Omega({\bf x},{\bf u},\nabla{\bf u},p)\partial_{x_i}u_j'({\bf x};V){\rm d}{\bf x}\\
        =&\, \sum_{i=1}^N\sum_{j=1}^N \int_\Omega \partial_{\partial_{x_i}u_j}J_\Omega({\bf x},{\bf u},\nabla{\bf u},p)\partial_{x_i}u_j'({\bf x};V){\rm d}{\bf x}\\
        =&\, \sum_{i=1}^N\sum_{j=1}^N -\int_\Omega \partial_{x_i}\partial_{\partial_{x_i}u_j}J_\Omega({\bf x},{\bf u},\nabla{\bf u},p)u_j'({\bf x};V){\rm d}{\bf x} + \int_\Gamma n_i\partial_{\partial_{x_i}u_j}J_\Omega({\bf x},{\bf u},\nabla{\bf u},p)u_j'({\bf x};V){\rm d}\Gamma\\
        =&-\int_\Omega \sum_{i=1}^N\sum_{j=1}^N \partial_{x_i}\partial_{\partial_{x_i}u_j}J_\Omega({\bf x},{\bf u},\nabla{\bf u},p)u_j'({\bf x};V){\rm d}{\bf x} + \int_\Gamma \sum_{i=1}^N\sum_{j=1}^N n_i\partial_{\partial_{x_i}u_j}J_\Omega({\bf x},{\bf u},\nabla{\bf u},p)u_j'({\bf x};V){\rm d}\Gamma\\
        =&-\int_\Omega \sum_{j=1}^N u_j'({\bf x};V)\sum_{i=1}^N \partial_{x_i}\partial_{\partial_{x_i}u_j}J_\Omega({\bf x},{\bf u},\nabla{\bf u},p){\rm d}{\bf x} + \int_\Gamma {\bf n}^\top\nabla_{\nabla{\bf u}}J_\Omega({\bf x},{\bf u},\nabla{\bf u},p){\bf u}'({\bf x};V){\rm d}\Gamma\\
        =&-\int_\Omega \sum_{j=1}^N u_j'({\bf x};V)\nabla\cdot\left(\nabla_{\nabla u_j}J_\Omega({\bf x},{\bf u},\nabla{\bf u},p)\right){\rm d}{\bf x} + \int_\Gamma {\bf n}^\top\nabla_{\nabla{\bf u}}J_\Omega({\bf x},{\bf u},\nabla{\bf u},p){\bf u}'({\bf x};V){\rm d}\Gamma\\
        =&-\int_\Omega \nabla\cdot\left(\nabla_{\nabla{\bf u}}J_\Omega({\bf x},{\bf u},\nabla{\bf u},p)\right)\cdot{\bf u}'({\bf x};V){\rm d}{\bf x} + \int_\Gamma
        {\bf n}^\top\nabla_{\nabla{\bf u}}J_\Omega({\bf x},{\bf u},\nabla{\bf u},p){\bf u}'({\bf x};V){\rm d}\Gamma.
    \end{align*}
    Then
    \begin{align*}
        &dJ({\bf u},p,\Omega;V)\\
        =&\, \int_\Omega \left[\nabla_{\bf u}J_\Omega({\bf x},{\bf u},\nabla{\bf u},p) - \nabla\cdot(\nabla_{\nabla{\bf u}}J_\Omega({\bf x},{\bf u},\nabla{\bf u},p))\right]\cdot{\bf u}'({\bf x};V) + \partial_pJ_\Omega({\bf x},{\bf u},\nabla{\bf u},p)p'({\bf x};V) + \nabla\cdot\left(J_\Omega({\bf x},{\bf u},\nabla{\bf u},p)V(0)\right){\rm d}{\bf x}\\
        &+ \int_\Gamma {\bf n}^\top\nabla_{\nabla{\bf u}}J_\Omega({\bf x},{\bf u},\nabla{\bf u},p){\bf u}'({\bf x};V) + \nabla_{\bf u}J_\Gamma({\bf x},{\bf u},\nabla{\bf u},p,{\bf n},{\bf t})\cdot{\bf u}'({\bf x};V) + \nabla_{\nabla{\bf u}}J_\Gamma({\bf x},{\bf u},\nabla{\bf u},p,{\bf n},{\bf t}):\nabla{\bf u}'({\bf x};V)\\
        &\hspace{1cm}+ \partial_pJ_\Gamma({\bf x},{\bf u},\nabla{\bf u},p,{\bf n},{\bf t})p'({\bf x};V) + \nabla_{\bf n}J_\Gamma({\bf x},{\bf u},\nabla{\bf u},p,{\bf n},{\bf t})\cdot{\bf n}'({\bf x};V) + \nabla_{\bf t}J_\Gamma({\bf x},{\bf u},\nabla{\bf u},p,{\bf n},{\bf t}):{\bf t}'({\bf x};V)\\
        &\hspace{1cm}+ \left[\nabla J_\Gamma({\bf x},{\bf u},\nabla{\bf u},p,{\bf n},{\bf t}) + \nabla_{\bf u}J_\Gamma({\bf x},{\bf u},\nabla{\bf u},p,{\bf n},{\bf t})\cdot\nabla{\bf u} + \nabla\nabla{\bf u}:\nabla_{\nabla{\bf u}}J_\Gamma({\bf x},{\bf u},\nabla{\bf u},p,{\bf n},{\bf t}) + \partial_pJ_\Gamma({\bf x},{\bf u},\nabla{\bf u},p,{\bf n},{\bf t})\nabla p\right.\\
        &\hspace{15mm} \left.+ \nabla{\bf n}\nabla_{\bf n}J_\Gamma({\bf x},{\bf u},\nabla{\bf u},p,{\bf n},{\bf t}) + \nabla{\bf t}:\nabla_{\bf t}J_\Gamma({\bf x},{\bf u},\nabla{\bf u},p,{\bf n},{\bf t})\right]\cdot V(0)\\
        &\hspace{1cm}+ J_\Gamma({\bf x},{\bf u},\nabla{\bf u},p,{\bf n},{\bf t})\left(\nabla\cdot V(0) - DV(0){\bf n}\cdot{\bf n}\right){\rm d}\Gamma,
    \end{align*}
    Test \eqref{PDEs local shape derivative for general stationary fluid dynamics PDEs} with the adjoint variable $({\bf v},q)$, obtain
    \begin{equation*}
        \left\{\begin{split}
            &\int_\Omega D_{\bf u}{\bf P}({\bf x},{\bf u},\nabla{\bf u},\Delta{\bf u},p,\nabla p){\bf u}'({\bf x};V)\cdot{\bf v} + {\color{red}D_{\nabla{\bf u}}{\bf P}({\bf x},{\bf u},\nabla{\bf u},\Delta{\bf u},p,\nabla p)\nabla{\bf u}'({\bf x};V)\cdot{\bf v}} + {\color{red}D_{\Delta{\bf u}}{\bf P}({\bf x},{\bf u},\nabla{\bf u},\Delta{\bf u},p,\nabla p)\Delta{\bf u}'({\bf x};V)\cdot{\bf v}}\\
            &\hspace{5mm}+ D_p{\bf P}({\bf x},{\bf u},\nabla{\bf u},\Delta{\bf u},p,\nabla p)p'({\bf x};V)\cdot{\bf v} + {\color{red}D_{\nabla p}{\bf P}({\bf x},{\bf u},\nabla{\bf u},\Delta{\bf u},p,\nabla p)\nabla p'({\bf x};V)\cdot{\bf v}}{\rm d}{\bf x}\\
            &\hspace{1cm}= \int_\Omega D_{\bf u}{\bf f}({\bf x},{\bf u},\nabla{\bf u},p){\bf u}'({\bf x};V)\cdot{\bf v} + {\color{red}D_{\nabla{\bf u}}{\bf f}({\bf x},{\bf u},\nabla{\bf u},p)\nabla{\bf u}'({\bf x};V)\cdot{\bf v}} + D_p{\bf f}({\bf x},{\bf u},\nabla{\bf u},p)p'({\bf x};V)\cdot{\bf v}{\rm d}{\bf x},\\
            &\int_\Omega {\color{red}-q\nabla\cdot{\bf u}'({\bf x};V)}{\rm d}{\bf x} = \int_\Omega qD_{\bf u}f_{\rm div}({\bf x},{\bf u},\nabla{\bf u},p){\bf u}'({\bf x};V) + {\color{red}qD_{\nabla{\bf u}}f_{\rm div}({\bf x},{\bf u},\nabla{\bf u},p)\nabla{\bf u}'({\bf x};V)} + qD_pf_{\rm div}({\bf x},{\bf u},\nabla{\bf u},p)p'({\bf x};V){\rm d}{\bf x},
        \end{split}\right.
    \end{equation*}
    Integrate by parts:
    \begin{enumerate}[leftmargin=0in]
        \item Term $D_{\nabla{\bf u}}{\bf P}({\bf x},{\bf u},\nabla{\bf u},\Delta{\bf u},p,\nabla p)\nabla{\bf u}'({\bf x};V)\cdot{\bf v}$: Use the result for term 2 before with $\tilde{\bf u} := {\bf u}'({\bf x};V)$:
        \begin{align*}
            &\int_\Omega D_{\nabla{\bf u}}{\bf P}({\bf x},{\bf u},\nabla{\bf u},\Delta{\bf u},p,\nabla p)\nabla{\bf u}'({\bf x};V)\cdot{\bf v}{\rm d}{\bf x}\\
            =&\, -\int_\Omega \left(\nabla\cdot\left(\nabla_{\nabla{\bf u}}{\bf P}({\bf x},{\bf u},\nabla{\bf u},\Delta{\bf u},p,\nabla p)\right)\cdot{\bf v}\right)\cdot{\bf u}'({\bf x};V) + \left(\nabla_{\nabla{\bf u}}{\bf P}({\bf x},{\bf u},\nabla{\bf u},\Delta{\bf u},p,\nabla p):\nabla{\bf v}\right)\cdot{\bf u}'({\bf x};V){\rm d}{\bf x}\\
            &+ \int_\Gamma \left(\left(\nabla_{\nabla{\bf u}}{\bf P}({\bf x},{\bf u},\nabla{\bf u},\Delta{\bf u},p,\nabla p)\cdot{\bf n}\right)\cdot{\bf v}\right)\cdot{\bf u}'({\bf x};V){\rm d}\Gamma.
        \end{align*}
        \item Term $D_{\Delta{\bf u}}{\bf P}({\bf x},{\bf u},\nabla{\bf u},\Delta{\bf u},p,\nabla p)\Delta{\bf u}'({\bf x};V)\cdot{\bf v}$: Use the result for term 3 before with $\tilde{\bf u} := {\bf u}'({\bf x};V)$:
        \begin{align*}
            &\int_\Omega D_{\Delta{\bf u}}{\bf P}({\bf x},{\bf u},\nabla{\bf u},\Delta{\bf u},p,\nabla p)\Delta{\bf u}'({\bf x};V)\cdot{\bf v}{\rm d}{\bf x}\\
            =&\, \int_\Omega {\bf v}^\top\Delta D_{\Delta{\bf u}}{\bf P}({\bf x},{\bf u},\nabla{\bf u},\Delta{\bf u},p,\nabla p){\bf u}'({\bf x};V) + \Delta{\bf v}^\top D_{\Delta{\bf u}}{\bf P}({\bf x},{\bf u},\nabla{\bf u},\Delta{\bf u},p,\nabla p){\bf u}'({\bf x};V){\rm d}{\bf x}\\
            &- \int_\Gamma {\bf v}^\top D_{\Delta{\bf u}}{\bf P}({\bf x},{\bf u},\nabla{\bf u},\Delta{\bf u},p,\nabla p)\partial_{\bf n}{\bf u}'({\bf x};V) - {\bf v}^\top\partial_{\bf n}D_{\Delta{\bf u}}{\bf P}({\bf x},{\bf u},\nabla{\bf u},\Delta{\bf u},p,\nabla p){\bf u}'({\bf x};V)\\
            &\hspace{1cm}- \partial_{\bf n}{\bf v}^\top D_{\Delta{\bf u}}{\bf P}({\bf x},{\bf u},\nabla{\bf u},\Delta{\bf u},p,\nabla p){\bf u}'({\bf x};V){\rm d}\Gamma.
        \end{align*}
        \item Term $D_{\nabla p}{\bf P}({\bf x},{\bf u},\nabla{\bf u},\Delta{\bf u},p,\nabla p)\nabla p'({\bf x};V)\cdot{\bf v}$: Use the result for term 5 before with $\tilde{\bf u} := {\bf u}'({\bf x};V)$:
        \begin{align*}
            &\int_\Omega D_{\nabla p}{\bf P}({\bf x},{\bf u},\nabla{\bf u},\Delta{\bf u},p,\nabla p)\nabla p'({\bf x};V)\cdot{\bf v}{\rm d}{\bf x}\\
            =&\, -\int_\Omega p'({\bf x};V)\nabla_{\nabla p}{\bf P}({\bf x},{\bf u},\nabla{\bf u},\Delta{\bf u},p,\nabla p):\nabla{\bf v} + p'({\bf x};V)\nabla\cdot\left(\nabla_{\nabla p}{\bf P}({\bf x},{\bf u},\nabla{\bf u},\Delta{\bf u},p,\nabla p)\right)\cdot{\bf v}{\rm d}{\bf x}\\
            &+ \int_\Gamma p'({\bf x};V){\bf n}^\top\nabla_{\nabla p}{\bf P}({\bf x},{\bf u},\nabla{\bf u},\Delta{\bf u},p,\nabla p){\bf v}{\rm d}\Gamma.
        \end{align*}
        \item Term $D_{\nabla{\bf u}}{\bf f}({\bf x},{\bf u},\nabla{\bf u},p)\nabla{\bf u}'({\bf x};V)\cdot{\bf v}$: Use the result for term 4 before with $\tilde{\bf u} := {\bf u}'({\bf x};V)$:
        \begin{align*}
            \int_\Omega D_{\nabla{\bf u}}{\bf f}({\bf x},{\bf u},\nabla{\bf u},p)\nabla{\bf u}'({\bf x};V)\cdot{\bf v}{\rm d}{\bf x} =& -\int_\Omega \left(\nabla\cdot\left(\nabla_{\nabla{\bf u}}{\bf f}({\bf x},{\bf u},\nabla{\bf u},p)\right)\cdot{\bf v}\right)\cdot{\bf u}'({\bf x};V) + \left(\nabla_{\nabla{\bf u}}{\bf f}({\bf x},{\bf u},\nabla{\bf u},p):\nabla{\bf v}\right)\cdot{\bf u}'({\bf x};V){\rm d}{\bf x}\\
            &+ \int_\Gamma \left(\left(\nabla_{\nabla{\bf u}}{\bf f}({\bf x},{\bf u},\nabla{\bf u},p)\cdot{\bf n}\right)\cdot{\bf v}\right)\cdot{\bf u}'({\bf x};V){\rm d}\Gamma.
        \end{align*}
        \item Term $q\nabla\cdot{\bf u}'({\bf x};V)$: Use the result for term 6 before with $\tilde{\bf u} := {\bf u}'({\bf x};V)$:
        \begin{align*}
            \int_\Omega -q\nabla\cdot{\bf u}'({\bf x};V){\rm d}{\bf x} = \int_\Omega \nabla q\cdot{\bf u}'({\bf x};V){\rm d}{\bf x} - \int_\Gamma q{\bf u}'({\bf x};V)\cdot{\bf n}{\rm d}\Gamma.
        \end{align*}
        \item Term $qD_{\nabla{\bf u}}f_{\rm div}({\bf x},{\bf u},\nabla{\bf u},p)\nabla{\bf u}'({\bf x};V)$: Use the result for term 7 before with $\tilde{\bf u} := {\bf u}'({\bf x};V)$:
        \begin{align*}
            \int_\Omega qD_{\nabla{\bf u}}f_{\rm div}({\bf x},{\bf u},\nabla{\bf u},p)\nabla{\bf u}'({\bf x};V){\rm d}{\bf x} =& -\int_\Omega \nabla^\top q\nabla_{\nabla{\bf u}}f_{\rm div}({\bf x},{\bf u},\nabla{\bf u},p){\bf u}'({\bf x};V) + q\left(\nabla\cdot\left(\nabla_{\nabla{\bf u}}f_{\rm div}({\bf x},{\bf u},\nabla{\bf u},p)\right)\right)\cdot{\bf u}'({\bf x};V){\rm d}{\bf x}\\
            &+ \int_\Gamma q{\bf n}^\top\nabla_{\nabla{\bf u}}f_{\rm div}({\bf x},{\bf u},\nabla{\bf u},p){\bf u}'({\bf x};V){\rm d}\Gamma.
        \end{align*}
    \end{enumerate}
    Plug in, obtain then
    \begin{equation*}
        \left\{\begin{split}
            &\int_\Omega \left[\nabla_{\Delta{\bf u}}{\bf P}({\bf x},{\bf u},\nabla{\bf u},\Delta{\bf u},p,\nabla p)\Delta{\bf v} - \left(\nabla_{\nabla{\bf u}}{\bf P}({\bf x},{\bf u},\nabla{\bf u},\Delta{\bf u},p,\nabla p) - \nabla_{\nabla{\bf u}}{\bf f}({\bf x},{\bf u},\nabla{\bf u},p)\right):\nabla{\bf v}\right.\\
            &\hspace{1cm}+ \left[\nabla_{\bf u}{\bf P}({\bf x},{\bf u},\nabla{\bf u},\Delta{\bf u},p,\nabla p) + \Delta\nabla_{\Delta{\bf u}}{\bf P}({\bf x},{\bf u},\nabla{\bf u},\Delta{\bf u},p,\nabla p) - \nabla_{\bf u}{\bf f}({\bf x},{\bf u},\nabla{\bf u},p)\right]{\bf v}\\
            &\hspace{1cm}\left.- \left[\nabla\cdot\left(\nabla_{\nabla{\bf u}}{\bf P}({\bf x},{\bf u},\nabla{\bf u},\Delta{\bf u},p,\nabla p)\right) - \nabla\cdot\left(\nabla_{\nabla{\bf u}}{\bf f}({\bf x},{\bf u},\nabla{\bf u},p)\right)\right]\cdot{\bf v}\right]\cdot{\bf u}'({\bf x};V){\rm d}{\bf x}\\
            &\hspace{5mm}+ \int_\Omega \left[\left[D_p{\bf P}({\bf x},{\bf u},\nabla{\bf u},\Delta{\bf u},p,\nabla p) - \nabla\cdot\left(\nabla_{\nabla p}{\bf P}({\bf x},{\bf u},\nabla{\bf u},\Delta{\bf u},p,\nabla p)\right) - D_p{\bf f}({\bf x},{\bf u},\nabla{\bf u},p)\right]\cdot{\bf v}\right.\\
            &\hspace{15mm}\left.-\nabla_{\nabla p}{\bf P}({\bf x},{\bf u},\nabla{\bf u},\Delta{\bf u},p,\nabla p):\nabla{\bf v}\right]p'({\bf x};V){\rm d}{\bf x}\\
            &\hspace{5mm}+ \int_\Gamma \left[\left(\nabla_{\nabla{\bf u}}{\bf P}({\bf x},{\bf u},\nabla{\bf u},\Delta{\bf u},p,\nabla p)\cdot{\bf n}\right)\cdot{\bf v} + \partial_{\bf n}\nabla_{\Delta{\bf u}}{\bf P}({\bf x},{\bf u},\nabla{\bf u},\Delta{\bf u},p,\nabla p){\bf v} + \nabla_{\Delta{\bf u}}{\bf P}({\bf x},{\bf u},\nabla{\bf u},\Delta{\bf u},p,\nabla p)\partial_{\bf n}{\bf v}\right.\\
            &\hspace{1cm}\left.- \left(\nabla_{\nabla{\bf u}}{\bf f}({\bf x},{\bf u},\nabla{\bf u},p)\cdot{\bf n}\right)\cdot{\bf v}\right]\cdot{\bf u}'({\bf x};V){\rm d}\Gamma\\
            &\hspace{5mm}+ \int_\Gamma -{\bf v}^\top D_{\Delta{\bf u}}{\bf P}({\bf x},{\bf u},\nabla{\bf u},\Delta{\bf u},p,\nabla p)\partial_{\bf n}{\bf u}'({\bf x};V) + p'({\bf x};V){\bf n}^\top\nabla_{\nabla p}{\bf P}({\bf x},{\bf u},\nabla{\bf u},\Delta{\bf u},p,\nabla p){\bf v}{\rm d}\Gamma = 0,\\
            &\int_\Omega \left[\nabla q - q\nabla_{\bf u}f_{\rm div}({\bf x},{\bf u},\nabla{\bf u},p) + D_{\nabla{\bf u}}f_{\rm div}({\bf x},{\bf u},\nabla{\bf u},p)\nabla q + q\nabla\cdot\left(\nabla_{\nabla{\bf u}}f_{\rm div}({\bf x},{\bf u},\nabla{\bf u},p)\right)\right]\cdot{\bf u}'({\bf x};V){\rm d}{\bf x}\\
            &\hspace{5mm}- \int_\Omega qD_pf_{\rm div}({\bf x},{\bf u},\nabla{\bf u},p)p'({\bf x};V){\rm d}{\bf x} - \int_\Gamma \left[qn + qD_{\nabla{\bf u}}f_{\rm div}({\bf x},{\bf u},\nabla{\bf u},p){\bf n}\right]\cdot{\bf u}'({\bf x};V){\rm d}\Gamma = 0.
        \end{split}\right.
    \end{equation*}
    Add them together, obtain
    \begin{align*}
        &\int_\Omega \left[\nabla_{\Delta{\bf u}}{\bf P}({\bf x},{\bf u},\nabla{\bf u},\Delta{\bf u},p,\nabla p)\Delta{\bf v} - \left(\nabla_{\nabla{\bf u}}{\bf P}({\bf x},{\bf u},\nabla{\bf u},\Delta{\bf u},p,\nabla p) - \nabla_{\nabla{\bf u}}{\bf f}({\bf x},{\bf u},\nabla{\bf u},p)\right):\nabla{\bf v}\right.\\
        &\hspace{1cm}+ \left[\nabla_{\bf u}{\bf P}({\bf x},{\bf u},\nabla{\bf u},\Delta{\bf u},p,\nabla p) + \Delta\nabla_{\Delta{\bf u}}{\bf P}({\bf x},{\bf u},\nabla{\bf u},\Delta{\bf u},p,\nabla p) - \nabla_{\bf u}{\bf f}({\bf x},{\bf u},\nabla{\bf u},p)\right]{\bf v}\\
        &\hspace{1cm}\left.- \left[\nabla\cdot\left(\nabla_{\nabla{\bf u}}{\bf P}({\bf x},{\bf u},\nabla{\bf u},\Delta{\bf u},p,\nabla p)\right) - \nabla\cdot\left(\nabla_{\nabla{\bf u}}{\bf f}({\bf x},{\bf u},\nabla{\bf u},p)\right)\right]\cdot{\bf v}\right.\\
        &\hspace{1cm}\left.+ \nabla q - q\nabla_{\bf u}f_{\rm div}({\bf x},{\bf u},\nabla{\bf u},p) + D_{\nabla{\bf u}}f_{\rm div}({\bf x},{\bf u},\nabla{\bf u},p)\nabla q + q\nabla\cdot\left(\nabla_{\nabla{\bf u}}f_{\rm div}({\bf x},{\bf u},\nabla{\bf u},p)\right)\right]\cdot{\bf u}'({\bf x};V){\rm d}{\bf x}\\
        &+ \int_\Omega \left[\left[D_p{\bf P}({\bf x},{\bf u},\nabla{\bf u},\Delta{\bf u},p,\nabla p) - \nabla\cdot\left(\nabla_{\nabla p}{\bf P}({\bf x},{\bf u},\nabla{\bf u},\Delta{\bf u},p,\nabla p)\right) - D_p{\bf f}({\bf x},{\bf u},\nabla{\bf u},p)\right]\cdot{\bf v}\right.\\
        &\hspace{1cm}\left.-\nabla_{\nabla p}{\bf P}({\bf x},{\bf u},\nabla{\bf u},\Delta{\bf u},p,\nabla p):\nabla{\bf v} - qD_pf_{\rm div}({\bf x},{\bf u},\nabla{\bf u},p)\right]p'({\bf x};V){\rm d}{\bf x}\\
        &+ \int_\Gamma \left[\left(\nabla_{\nabla{\bf u}}{\bf P}({\bf x},{\bf u},\nabla{\bf u},\Delta{\bf u},p,\nabla p)\cdot{\bf n}\right)\cdot{\bf v} + \partial_{\bf n}\nabla_{\Delta{\bf u}}{\bf P}({\bf x},{\bf u},\nabla{\bf u},\Delta{\bf u},p,\nabla p){\bf v} + \nabla_{\Delta{\bf u}}{\bf P}({\bf x},{\bf u},\nabla{\bf u},\Delta{\bf u},p,\nabla p)\partial_{\bf n}{\bf v}\right.\\
        &\hspace{1cm}\left.- \left(\nabla_{\nabla{\bf u}}{\bf f}({\bf x},{\bf u},\nabla{\bf u},p)\cdot{\bf n}\right)\cdot{\bf v} - qn - qD_{\nabla{\bf u}}f_{\rm div}({\bf x},{\bf u},\nabla{\bf u},p){\bf n}\right]\cdot{\bf u}'({\bf x};V){\rm d}\Gamma\\
        &+ \int_\Gamma -{\bf v}^\top D_{\Delta{\bf u}}{\bf P}({\bf x},{\bf u},\nabla{\bf u},\Delta{\bf u},p,\nabla p)\partial_{\bf n}{\bf u}'({\bf x};V) + p'({\bf x};V){\bf n}^\top\nabla_{\nabla p}{\bf P}({\bf x},{\bf u},\nabla{\bf u},\Delta{\bf u},p,\nabla p){\bf v}{\rm d}\Gamma = 0.
    \end{align*}
    Combine this with \eqref{adjoint general stationary fluid dynamics PDEs}, obtain
    \begin{align*}
        &\int_\Omega \left[\nabla_{\bf u}J_\Omega({\bf x},{\bf u},\nabla{\bf u},p) - \nabla\cdot\left(\nabla_{\nabla{\bf u}}J_\Omega({\bf x},{\bf u},\nabla{\bf u},p)\right)\right]\cdot{\bf u}'({\bf x};V) + D_pJ_\Omega({\bf x},{\bf u},\nabla{\bf u},p)p'({\bf x};V){\rm d}{\bf x}\\
        &+ \int_\Gamma \left[-\nabla_{\bf u}{\bf Q}({\bf x},{\bf u},\nabla{\bf u},p,{\bf n},{\bf t}){\bf v}_{\rm bc} + \nabla_{\nabla{\bf u}}J_\Omega({\bf x},{\bf u},\nabla{\bf u},p)\cdot{\bf n} + \nabla_{\bf u}J_\Gamma({\bf x},{\bf u},\nabla{\bf u},p,{\bf n},{\bf t})\right]\cdot{\bf u}'({\bf x};V){\rm d}\Gamma\\
        &+ \int_\Gamma -{\bf v}^\top D_{\Delta{\bf u}}{\bf P}({\bf x},{\bf u},\nabla{\bf u},\Delta{\bf u},p,\nabla p)\partial_{\bf n}{\bf u}'({\bf x};V) + p'({\bf x};V){\bf n}^\top\nabla_{\nabla p}{\bf P}({\bf x},{\bf u},\nabla{\bf u},\Delta{\bf u},p,\nabla p){\bf v}{\rm d}\Gamma = 0.
    \end{align*}
    Combine this equality with the formula of shape derivative $dJ({\bf u},p,\Omega)$, obtain:
    \begin{align*}
        dJ({\bf u},p,\Omega;V) =&\, \int_\Omega \nabla\cdot\left(J_\Omega({\bf x},{\bf u},\nabla{\bf u},p)V(0)\right){\rm d}{\bf x}\\
        &+ \int_\Gamma \nabla_{\nabla{\bf u}}J_\Gamma({\bf x},{\bf u},\nabla{\bf u},p,{\bf n},{\bf t}):\nabla{\bf u}'({\bf x};V) + \partial_pJ_\Gamma({\bf x},{\bf u},\nabla{\bf u},p,{\bf n},{\bf t})p'({\bf x};V) + \nabla_{\bf n}J_\Gamma({\bf x},{\bf u},\nabla{\bf u},p,{\bf n},{\bf t})\cdot{\bf n}'({\bf x};V)\\
        &\hspace{1cm}+ \nabla_{\bf t}J_\Gamma({\bf x},{\bf u},\nabla{\bf u},p,{\bf n},{\bf t}):{\bf t}'({\bf x};V)\\
        &\hspace{1cm}+ \left[\nabla J_\Gamma({\bf x},{\bf u},\nabla{\bf u},p,{\bf n},{\bf t}) + \nabla_{\bf u}J_\Gamma({\bf x},{\bf u},\nabla{\bf u},p,{\bf n},{\bf t})\cdot\nabla{\bf u} + \nabla\nabla{\bf u}:\nabla_{\nabla{\bf u}}J_\Gamma({\bf x},{\bf u},\nabla{\bf u},p,{\bf n},{\bf t})\right.\\
        &\hspace{15mm} \left.+ \partial_pJ_\Gamma({\bf x},{\bf u},\nabla{\bf u},p,{\bf n},{\bf t})\nabla p + \nabla{\bf n}\nabla_{\bf n}J_\Gamma({\bf x},{\bf u},\nabla{\bf u},p,{\bf n},{\bf t}) + \nabla{\bf t}:\nabla_{\bf t}J_\Gamma({\bf x},{\bf u},\nabla{\bf u},p,{\bf n},{\bf t})\right]\cdot V(0)\\
        &\hspace{1cm}+ J_\Gamma({\bf x},{\bf u},\nabla{\bf u},p,{\bf n},{\bf t})\left(\nabla\cdot V(0) - DV(0){\bf n}\cdot{\bf n}\right) + \left(\nabla_{\bf u}{\bf Q}({\bf x},{\bf u},\nabla{\bf u},p,{\bf n},{\bf t}){\bf v}_{\rm bc}\right)\cdot{\bf u}'({\bf x};V)\\
        &\hspace{1cm}+ {\bf v}^\top D_{\Delta{\bf u}}{\bf P}({\bf x},{\bf u},\nabla{\bf u},\Delta{\bf u},p,\nabla p)\partial_{\bf n}{\bf u}'({\bf x};V) - p'({\bf x};V){\bf n}^\top\nabla_{\nabla p}{\bf P}({\bf x},{\bf u},\nabla{\bf u},\Delta{\bf u},p,\nabla p){\bf v}{\rm d}\Gamma.
    \end{align*}
    To eliminate $({\bf u}'({\bf x};V),p({\bf x};V))$ in boundary integrals, we need the explicit formula of ${\bf Q}({\bf x},{\bf u},\nabla{\bf u},p,{\bf n},{\bf t})$ (but not ${\bf f}_{\rm bc}({\bf x})$).
    
    \item \textit{2nd representation of shape derivative.} Use the same technique:
    \begin{align*}
        &dJ({\bf u},p,\Omega;V)\\
        =&\, \int_\Omega J_\Omega'({\bf x},{\bf u},\nabla{\bf u},p;V){\rm d}{\bf x}\\
        &+ \int_\Gamma J_\Omega({\bf x},{\bf u},\nabla{\bf u},p)V(0)\cdot{\bf n} + J_\Gamma'({\bf x},{\bf u},\nabla{\bf u},p,{\bf n},{\bf t};V) + \left[\partial_{\bf n}(J_\Gamma({\bf x},{\bf u},\nabla{\bf u},p,{\bf n},{\bf t})) + HJ_\Gamma({\bf x},{\bf u},\nabla{\bf u},p,{\bf n},{\bf t})\right]V(0)\cdot{\bf n}{\rm d}\Gamma\\
        =&\, \int_\Omega D_{\bf u}J_\Omega({\bf x},{\bf u},\nabla{\bf u},p){\bf u}'({\bf x};V) + D_{\nabla{\bf u}}J_\Omega({\bf x},{\bf u},\nabla{\bf u},p)\nabla{\bf u}'({\bf x};V) + \partial_pJ_\Omega({\bf x},{\bf u},\nabla{\bf u},p)p'({\bf x};V){\rm d}{\bf x}\\
        &+ \int_\Gamma J_\Omega({\bf x},{\bf u},\nabla{\bf u},p)V(0)\cdot{\bf n} + D_{\bf u}J_\Gamma({\bf x},{\bf u},\nabla{\bf u},p,{\bf n},{\bf t}){\bf u}'({\bf x};V) + D_{\nabla{\bf u}}J_\Gamma({\bf x},{\bf u},\nabla{\bf u},p,{\bf n},{\bf t})\nabla{\bf u}'({\bf x};V)\\
        &\hspace{1cm}+ \partial_pJ_\Gamma({\bf x},{\bf u},\nabla{\bf u},p,{\bf n},{\bf t})p'({\bf x};V) + D_{\bf n}J_\Gamma({\bf x},{\bf u},\nabla{\bf u},p,{\bf n},{\bf t}){\bf n}'({\bf x};V) + D_{\bf t}J_\Gamma({\bf x},{\bf u},\nabla{\bf u},p,{\bf n},{\bf t}){\bf t}'({\bf x};V)\\
        &\hspace{1cm}+ \left[DJ_\Gamma({\bf x},{\bf u},\nabla{\bf u},p,{\bf n},{\bf t}) + D_{\bf u}J_\Gamma({\bf x},{\bf u},\nabla{\bf u},p,{\bf n},{\bf t})D{\bf u} + D_{\nabla{\bf u}}J_\Gamma({\bf x},{\bf u},\nabla{\bf u},p,{\bf n},{\bf t})D\nabla{\bf u} + \partial_pJ_\Gamma({\bf x},{\bf u},\nabla{\bf u},p,{\bf n},{\bf t})Dp\right.\\
        &\hspace{15mm} \left.+ D_{\bf n}J_\Gamma({\bf x},{\bf u},\nabla{\bf u},p,{\bf n},{\bf t})D{\bf n} + D_{\bf t}J_\Gamma({\bf x},{\bf u},\nabla{\bf u},p,{\bf n},{\bf t})D{\bf t}\right]\cdot{\bf n}V(0)\cdot{\bf n} + HJ_\Gamma({\bf x},{\bf u},\nabla{\bf u},p,{\bf n},{\bf t})V(0)\cdot{\bf n}{\rm d}\Gamma\\
        =&\, \int_\Omega \nabla_{\bf u}J_\Omega({\bf x},{\bf u},\nabla{\bf u},p)\cdot{\bf u}'({\bf x};V) + \nabla_{\nabla{\bf u}}J_\Omega({\bf x},{\bf u},\nabla{\bf u},p):\nabla{\bf u}'({\bf x};V) + \partial_pJ_\Omega({\bf x},{\bf u},\nabla{\bf u},p)p'({\bf x};V){\rm d}{\bf x}\\
        &+ \int_\Gamma J_\Omega({\bf x},{\bf u},\nabla{\bf u},p)V(0)\cdot{\bf n} + \nabla_{\bf u}J_\Gamma({\bf x},{\bf u},\nabla{\bf u},p,{\bf n},{\bf t})\cdot{\bf u}'({\bf x};V) + \nabla_{\nabla{\bf u}}J_\Gamma({\bf x},{\bf u},\nabla{\bf u},p,{\bf n},{\bf t}):\nabla{\bf u}'({\bf x};V)\\
        &\hspace{1cm}+ \partial_pJ_\Gamma({\bf x},{\bf u},\nabla{\bf u},p,{\bf n},{\bf t})p'({\bf x};V) + \nabla_{\bf n}J_\Gamma({\bf x},{\bf u},\nabla{\bf u},p,{\bf n},{\bf t})\cdot{\bf n}'({\bf x};V) + \nabla_{\bf t}J_\Gamma({\bf x},{\bf u},\nabla{\bf u},p,{\bf n},{\bf t}):{\bf t}'({\bf x};V)\\
        &\hspace{1cm}+ \left[\nabla J_\Gamma({\bf x},{\bf u},\nabla{\bf u},p,{\bf n},{\bf t}) + \nabla_{\bf u}J_\Gamma({\bf x},{\bf u},\nabla{\bf u},p,{\bf n},{\bf t})\cdot\nabla{\bf u} + \nabla\nabla{\bf u}:\nabla_{\nabla{\bf u}}J_\Gamma({\bf x},{\bf u},\nabla{\bf u},p,{\bf n},{\bf t}) + \partial_pJ_\Gamma({\bf x},{\bf u},\nabla{\bf u},p,{\bf n},{\bf t})\nabla p\right.\\
        &\hspace{15mm} \left.+ \nabla{\bf n}\nabla_{\bf n}J_\Gamma({\bf x},{\bf u},\nabla{\bf u},p,{\bf n},{\bf t}) + \nabla{\bf t}:\nabla_{\bf t}J_\Gamma({\bf x},{\bf u},\nabla{\bf u},p,{\bf n},{\bf t})\right]\cdot{\bf n}V(0)\cdot{\bf n} + HJ_\Gamma({\bf x},{\bf u},\nabla{\bf u},p,{\bf n},{\bf t})V(0)\cdot{\bf n}{\rm d}\Gamma\\
        =&\, \int_\Omega \left[\nabla_{\bf u}J_\Omega({\bf x},{\bf u},\nabla{\bf u},p) - \nabla\cdot(\nabla_{\nabla{\bf u}}J_\Omega({\bf x},{\bf u},\nabla{\bf u},p))\right]\cdot{\bf u}'({\bf x};V) + \partial_pJ_\Omega({\bf x},{\bf u},\nabla{\bf u},p)p'({\bf x};V){\rm d}{\bf x}\\
        &+ \int_\Gamma J_\Omega({\bf x},{\bf u},\nabla{\bf u},p)V(0)\cdot{\bf n} + {\bf n}^\top\nabla_{\nabla{\bf u}}J_\Omega({\bf x},{\bf u},\nabla{\bf u},p){\bf u}'({\bf x};V) + \nabla_{\bf u}J_\Gamma({\bf x},{\bf u},\nabla{\bf u},p,{\bf n},{\bf t})\cdot{\bf u}'({\bf x};V)\\
        &\hspace{1cm}+ \nabla_{\nabla{\bf u}}J_\Gamma({\bf x},{\bf u},\nabla{\bf u},p,{\bf n},{\bf t}):\nabla{\bf u}'({\bf x};V) + \partial_pJ_\Gamma({\bf x},{\bf u},\nabla{\bf u},p,{\bf n},{\bf t})p'({\bf x};V) + \nabla_{\bf n}J_\Gamma({\bf x},{\bf u},\nabla{\bf u},p,{\bf n},{\bf t})\cdot{\bf n}'({\bf x};V)\\
        &\hspace{1cm}+ \nabla_{\bf t}J_\Gamma({\bf x},{\bf u},\nabla{\bf u},p,{\bf n},{\bf t}):{\bf t}'({\bf x};V)\\
        &\hspace{1cm}+ \left[\nabla J_\Gamma({\bf x},{\bf u},\nabla{\bf u},p,{\bf n},{\bf t}) + \nabla_{\bf u}J_\Gamma({\bf x},{\bf u},\nabla{\bf u},p,{\bf n},{\bf t})\cdot\nabla{\bf u} + \nabla\nabla{\bf u}:\nabla_{\nabla{\bf u}}J_\Gamma({\bf x},{\bf u},\nabla{\bf u},p,{\bf n},{\bf t}) + \partial_pJ_\Gamma({\bf x},{\bf u},\nabla{\bf u},p,{\bf n},{\bf t})\nabla p\right.\\
        &\hspace{15mm} \left.+ \nabla{\bf n}\nabla_{\bf n}J_\Gamma({\bf x},{\bf u},\nabla{\bf u},p,{\bf n},{\bf t}) + \nabla{\bf t}:\nabla_{\bf t}J_\Gamma({\bf x},{\bf u},\nabla{\bf u},p,{\bf n},{\bf t})\right]\cdot{\bf n}V(0)\cdot{\bf n} + HJ_\Gamma({\bf x},{\bf u},\nabla{\bf u},p,{\bf n},{\bf t})V(0)\cdot{\bf n}{\rm d}\Gamma\\
        =&\, \int_\Gamma J_\Omega({\bf x},{\bf u},\nabla{\bf u},p)V(0)\cdot{\bf n} + \nabla_{\nabla{\bf u}}J_\Gamma({\bf x},{\bf u},\nabla{\bf u},p,{\bf n},{\bf t}):\nabla{\bf u}'({\bf x};V) + \partial_pJ_\Gamma({\bf x},{\bf u},\nabla{\bf u},p,{\bf n},{\bf t})p'({\bf x};V)\\
        &\hspace{1cm}+ \nabla_{\bf n}J_\Gamma({\bf x},{\bf u},\nabla{\bf u},p,{\bf n},{\bf t})\cdot{\bf n}'({\bf x};V) + \nabla_{\bf t}J_\Gamma({\bf x},{\bf u},\nabla{\bf u},p,{\bf n},{\bf t}):{\bf t}'({\bf x};V)\\
        &\hspace{1cm}+ \left[\nabla J_\Gamma({\bf x},{\bf u},\nabla{\bf u},p,{\bf n},{\bf t}) + \nabla_{\bf u}J_\Gamma({\bf x},{\bf u},\nabla{\bf u},p,{\bf n},{\bf t})\cdot\nabla{\bf u} + \nabla\nabla{\bf u}:\nabla_{\nabla{\bf u}}J_\Gamma({\bf x},{\bf u},\nabla{\bf u},p,{\bf n},{\bf t}) + \partial_pJ_\Gamma({\bf x},{\bf u},\nabla{\bf u},p,{\bf n},{\bf t})\nabla p\right.\\
        &\hspace{15mm} \left.+ \nabla{\bf n}\nabla_{\bf n}J_\Gamma({\bf x},{\bf u},\nabla{\bf u},p,{\bf n},{\bf t}) + \nabla{\bf t}:\nabla_{\bf t}J_\Gamma({\bf x},{\bf u},\nabla{\bf u},p,{\bf n},{\bf t})\right]\cdot{\bf n}V(0)\cdot{\bf n} + HJ_\Gamma({\bf x},{\bf u},\nabla{\bf u},p,{\bf n},{\bf t})V(0)\cdot{\bf n}\\
        &\hspace{1cm}+ (\nabla_{\bf u}{\bf Q}({\bf x},{\bf u},\nabla{\bf u},p,{\bf n},{\bf t}){\bf v}_{\rm bc})\cdot{\bf u}'({\bf x};V) + {\bf v}^\top D_{\Delta{\bf u}}{\bf P}({\bf x},{\bf u},\nabla{\bf u},\Delta{\bf u},p,\nabla p)\partial_{\bf n}{\bf u}'({\bf x};V)\\
        &\hspace{1cm}- p'({\bf x};V){\bf n}^\top\nabla_{\nabla p}{\bf P}({\bf x},{\bf u},\nabla{\bf u},\Delta{\bf u},p,\nabla p){\bf v}{\rm d}\Gamma.
    \end{align*}
    To eliminate $({\bf u}'({\bf x};V),p({\bf x};V))$ in boundary integrals, we need the explicit formula of ${\bf Q}({\bf x},{\bf u},\nabla{\bf u},p,{\bf n},{\bf t})$ (but not ${\bf f}_{\rm bc}({\bf x})$).
\end{enumerate}
Conclude: The 1st-order shape derivative of \eqref{cost functional of general stationary fluid dynamics PDEs} under the state constraint \eqref{general stationary fluid dynamics PDEs} s given by either of the following representations:
\begin{equation*}
    \boxed{\left.\begin{split}
        &dJ({\bf u},p,\Omega;V)\\
        =&\, \int_\Omega \nabla\cdot\left(J_\Omega({\bf x},{\bf u},\nabla{\bf u},p)V(0)\right){\rm d}{\bf x}\\
        &+ \int_\Gamma \nabla_{\nabla{\bf u}}J_\Gamma({\bf x},{\bf u},\nabla{\bf u},p,{\bf n},{\bf t}):\nabla{\bf u}'({\bf x};V) + \partial_pJ_\Gamma({\bf x},{\bf u},\nabla{\bf u},p,{\bf n},{\bf t})p'({\bf x};V) + \nabla_{\bf n}J_\Gamma({\bf x},{\bf u},\nabla{\bf u},p,{\bf n},{\bf t})\cdot{\bf n}'({\bf x};V)\\
        &\hspace{1cm}+ \nabla_{\bf t}J_\Gamma({\bf x},{\bf u},\nabla{\bf u},p,{\bf n},{\bf t}):{\bf t}'({\bf x};V) + \nabla\left(J_\Gamma({\bf x},{\bf u},\nabla{\bf u},p,{\bf n},{\bf t})\right)\cdot V(0)\\
        &\hspace{1cm}+ J_\Gamma({\bf x},{\bf u},\nabla{\bf u},p,{\bf n},{\bf t})\left(\nabla\cdot V(0) - DV(0){\bf n}\cdot{\bf n}\right) + \left(\nabla_{\bf u}{\bf Q}({\bf x},{\bf u},\nabla{\bf u},p,{\bf n},{\bf t}){\bf v}_{\rm bc}\right)\cdot{\bf u}'({\bf x};V)\\
        &\hspace{1cm}+ {\bf v}^\top D_{\Delta{\bf u}}{\bf P}({\bf x},{\bf u},\nabla{\bf u},\Delta{\bf u},p,\nabla p)\partial_{\bf n}{\bf u}'({\bf x};V) - p'({\bf x};V){\bf n}^\top\nabla_{\nabla p}{\bf P}({\bf x},{\bf u},\nabla{\bf u},\Delta{\bf u},p,\nabla p){\bf v}{\rm d}\Gamma\\
        =&\, \int_\Omega \nabla\cdot\left(J_\Omega({\bf x},{\bf u},\nabla{\bf u},p)V(0)\right){\rm d}{\bf x}\\
        &+ \int_\Gamma \nabla_{\nabla{\bf u}}J_\Gamma({\bf x},{\bf u},\nabla{\bf u},p,{\bf n},{\bf t}):\nabla{\bf u}'({\bf x};V) + \partial_pJ_\Gamma({\bf x},{\bf u},\nabla{\bf u},p,{\bf n},{\bf t})p'({\bf x};V)\\
        &\hspace{1cm}+ \nabla_{\bf n}J_\Gamma({\bf x},{\bf u},\nabla{\bf u},p,{\bf n},{\bf t})\cdot{\bf n}'({\bf x};V) + \nabla_{\bf t}J_\Gamma({\bf x},{\bf u},\nabla{\bf u},p,{\bf n},{\bf t}):{\bf t}'({\bf x};V)\\
        &\hspace{1cm}+ \left[\nabla J_\Gamma({\bf x},{\bf u},\nabla{\bf u},p,{\bf n},{\bf t}) + \nabla_{\bf u}J_\Gamma({\bf x},{\bf u},\nabla{\bf u},p,{\bf n},{\bf t})\cdot\nabla{\bf u} + \nabla\nabla{\bf u}:\nabla_{\nabla{\bf u}}J_\Gamma({\bf x},{\bf u},\nabla{\bf u},p,{\bf n},{\bf t})\right.\\
        &\hspace{15mm} \left.+ \partial_pJ_\Gamma({\bf x},{\bf u},\nabla{\bf u},p,{\bf n},{\bf t})\nabla p + \nabla{\bf n}\nabla_{\bf n}J_\Gamma({\bf x},{\bf u},\nabla{\bf u},p,{\bf n},{\bf t}) + \nabla{\bf t}:\nabla_{\bf t}J_\Gamma({\bf x},{\bf u},\nabla{\bf u},p,{\bf n},{\bf t})\right]\cdot V(0)\\
        &\hspace{1cm}+ J_\Gamma({\bf x},{\bf u},\nabla{\bf u},p,{\bf n},{\bf t})\left(\nabla\cdot V(0) - DV(0){\bf n}\cdot{\bf n}\right) + \left(\nabla_{\bf u}{\bf Q}({\bf x},{\bf u},\nabla{\bf u},p,{\bf n},{\bf t}){\bf v}_{\rm bc}\right)\cdot{\bf u}'({\bf x};V)\\
        &\hspace{1cm}+ {\bf v}^\top D_{\Delta{\bf u}}{\bf P}({\bf x},{\bf u},\nabla{\bf u},\Delta{\bf u},p,\nabla p)\partial_{\bf n}{\bf u}'({\bf x};V) - p'({\bf x};V){\bf n}^\top\nabla_{\nabla p}{\bf P}({\bf x},{\bf u},\nabla{\bf u},\Delta{\bf u},p,\nabla p){\bf v}{\rm d}\Gamma\\
        =&\, \int_\Gamma J_\Omega({\bf x},{\bf u},\nabla{\bf u},p)V(0)\cdot{\bf n} + \nabla_{\nabla{\bf u}}J_\Gamma({\bf x},{\bf u},\nabla{\bf u},p,{\bf n},{\bf t}):\nabla{\bf u}'({\bf x};V) + \partial_pJ_\Gamma({\bf x},{\bf u},\nabla{\bf u},p,{\bf n},{\bf t})p'({\bf x};V)\\
        &\hspace{1cm}+ \nabla_{\bf n}J_\Gamma({\bf x},{\bf u},\nabla{\bf u},p,{\bf n},{\bf t})\cdot{\bf n}'({\bf x};V) + \nabla_{\bf t}J_\Gamma({\bf x},{\bf u},\nabla{\bf u},p,{\bf n},{\bf t}):{\bf t}'({\bf x};V)\\
        &\hspace{1cm}+ \partial_{\bf n}\left(J_\Gamma({\bf x},{\bf u},\nabla{\bf u},p,{\bf n},{\bf t})\right)V(0)\cdot{\bf n} + HJ_\Gamma({\bf x},{\bf u},\nabla{\bf u},p,{\bf n},{\bf t})V(0)\cdot{\bf n}\\
        &\hspace{1cm}+ (\nabla_{\bf u}{\bf Q}({\bf x},{\bf u},\nabla{\bf u},p,{\bf n},{\bf t}){\bf v}_{\rm bc})\cdot{\bf u}'({\bf x};V) + {\bf v}^\top D_{\Delta{\bf u}}{\bf P}({\bf x},{\bf u},\nabla{\bf u},\Delta{\bf u},p,\nabla p)\partial_{\bf n}{\bf u}'({\bf x};V)\\
        &\hspace{1cm}- p'({\bf x};V){\bf n}^\top\nabla_{\nabla p}{\bf P}({\bf x},{\bf u},\nabla{\bf u},\Delta{\bf u},p,\nabla p){\bf v}{\rm d}\Gamma\\
        =&\, \int_\Gamma J_\Omega({\bf x},{\bf u},\nabla{\bf u},p)V(0)\cdot{\bf n} + \nabla_{\nabla{\bf u}}J_\Gamma({\bf x},{\bf u},\nabla{\bf u},p,{\bf n},{\bf t}):\nabla{\bf u}'({\bf x};V) + \partial_pJ_\Gamma({\bf x},{\bf u},\nabla{\bf u},p,{\bf n},{\bf t})p'({\bf x};V)\\
        &\hspace{1cm}+ \nabla_{\bf n}J_\Gamma({\bf x},{\bf u},\nabla{\bf u},p,{\bf n},{\bf t})\cdot{\bf n}'({\bf x};V) + \nabla_{\bf t}J_\Gamma({\bf x},{\bf u},\nabla{\bf u},p,{\bf n},{\bf t}):{\bf t}'({\bf x};V)\\
        &\hspace{1cm}+ \left[\nabla J_\Gamma({\bf x},{\bf u},\nabla{\bf u},p,{\bf n},{\bf t}) + \nabla_{\bf u}J_\Gamma({\bf x},{\bf u},\nabla{\bf u},p,{\bf n},{\bf t})\cdot\nabla{\bf u} + \nabla\nabla{\bf u}:\nabla_{\nabla{\bf u}}J_\Gamma({\bf x},{\bf u},\nabla{\bf u},p,{\bf n},{\bf t}) + \partial_pJ_\Gamma({\bf x},{\bf u},\nabla{\bf u},p,{\bf n},{\bf t})\nabla p\right.\\
        &\hspace{15mm} \left.+ \nabla{\bf n}\nabla_{\bf n}J_\Gamma({\bf x},{\bf u},\nabla{\bf u},p,{\bf n},{\bf t}) + \nabla{\bf t}:\nabla_{\bf t}J_\Gamma({\bf x},{\bf u},\nabla{\bf u},p,{\bf n},{\bf t})\right]\cdot{\bf n}V(0)\cdot{\bf n} + HJ_\Gamma({\bf x},{\bf u},\nabla{\bf u},p,{\bf n},{\bf t})V(0)\cdot{\bf n}\\
        &\hspace{1cm}+ (\nabla_{\bf u}{\bf Q}({\bf x},{\bf u},\nabla{\bf u},p,{\bf n},{\bf t}){\bf v}_{\rm bc})\cdot{\bf u}'({\bf x};V) + {\bf v}^\top D_{\Delta{\bf u}}{\bf P}({\bf x},{\bf u},\nabla{\bf u},\Delta{\bf u},p,\nabla p)\partial_{\bf n}{\bf u}'({\bf x};V)\\
        &\hspace{1cm}- p'({\bf x};V){\bf n}^\top\nabla_{\nabla p}{\bf P}({\bf x},{\bf u},\nabla{\bf u},\Delta{\bf u},p,\nabla p){\bf v}{\rm d}\Gamma
    \end{split}\right.}
\end{equation*}

\section{Stationary incompressible viscous Navier-Stokes equations}
Consider the following stationary incompressible viscous Navier-Stokes equations:
\begin{equation}
    \label{stationary incompressible viscous NSEs}
    \tag{NS}
    \left\{\begin{split}
        -\nabla\cdot\left(\nu({\bf x},{\bf u},\nabla{\bf u},p)\nabla{\bf u}\right) + ({\bf u}\cdot\nabla){\bf u} + \nabla p &= {\bf f}({\bf x},{\bf u},\nabla{\bf u},p) &&\mbox{ in } \Omega,\\
        \nabla\cdot{\bf u} &= f_{\rm div}({\bf x},{\bf u},\nabla{\bf u},p) &&\mbox{ in } \Omega.
    \end{split}\right.
\end{equation}

\section{Weak formulations for stationary incompressible viscous Navier-Stokes equations}
Test both sides of the 1st equation of \eqref{general stationary incompressible NSEs} with a test function ${\bf v}$ and those of the 2nd one with a test function $q$ over $\Omega$:
\begin{equation}
    \label{tested general stationary incompressible NSEs}
    \tag{test-gsincNS}
    \left\{\begin{split}
        \int_\Omega -\nabla\cdot\left(\nu({\bf x},{\bf u},\nabla{\bf u},p)\nabla{\bf u}\right)\cdot{\bf v} + \left({\bf u}\cdot\nabla\right){\bf u}\cdot{\bf v} + \nabla p\cdot{\bf v}{\rm d}{\bf x} &= \int_\Omega {\bf f}({\bf x},{\bf u},\nabla{\bf u},p)\cdot{\bf v}{\rm d}{\bf x},\\
        \int_\Omega q\nabla\cdot{\bf u}{\rm d}{\bf x} &= \int_\Omega qf_{\rm div}({\bf x},{\bf u},\nabla{\bf u},p){\rm d}{\bf x}.
    \end{split}\right.
\end{equation}
Apply \eqref{integration by parts for matrix 2} for the 1st term in the l.h.s. of the 1st equation of \eqref{tested general stationary incompressible NSEs}:
\begin{align*}
    \int_\Omega -\nabla\cdot\left(\nu({\bf x},{\bf u},\nabla{\bf u},p)\nabla{\bf u}\right)\cdot{\bf v}{\rm d}{\bf x} = \int_\Omega \nu({\bf x},{\bf u},\nabla{\bf u},p)\nabla{\bf u}:\nabla{\bf v}{\rm d}{\bf x} - \int_\Gamma \nu({\bf x},{\bf u},\nabla{\bf u},p){\bf n}^\top\nabla{\bf u}{\bf v}{\rm d}\Gamma.
\end{align*}
Apply \eqref{ibp} for the 3rd term in the l.h.s. of the 1st equation of \eqref{tested general stationary incompressible NSEs}:
\begin{align*}
    \int_\Omega \nabla p\cdot{\bf v}{\rm d}{\bf x} = -\int_\Omega p\nabla\cdot{\bf v}{\rm d}{\bf x} + \int_\Gamma p{\bf v}\cdot{\bf n}{\rm d}\Gamma.
\end{align*} 
Keep the 2nd equation of \eqref{tested general stationary incompressible NSEs}, then it becomes
\begin{equation}
    \label{weak formulation of general stationary incompressible NSEs}
    \tag{wf-gsincNS}
    \left\{\begin{split}
        \int_\Omega \nu({\bf x},{\bf u},\nabla{\bf u},p)\nabla{\bf u}:\nabla{\bf v} + ({\bf u}\cdot\nabla){\bf u}\cdot{\bf v} - p\nabla\cdot{\bf v} - {\bf f}({\bf x},{\bf u},\nabla{\bf u},p)\cdot{\bf v}{\rm d}{\bf x} + \int_\Gamma p{\bf v}\cdot{\bf n} - \nu({\bf x},{\bf u},\nabla{\bf u},p){\bf n}^\top\nabla{\bf u}{\bf v}{\rm d}\Gamma &= 0,\\
        \int_\Omega q\nabla\cdot{\bf u} - qf_{\rm div}({\bf x},{\bf u},\nabla{\bf u},p){\rm d}{\bf x} &= 0.
    \end{split}\right.    
\end{equation}

\section{A general cost functionals \& its associated optimization problem}
The optimization problem associated with the cost functional \eqref{cost functional of general stationary fluid dynamics PDEs} can be formulated as follows:

Find $\Omega$ over a class of admissible domain $\mathcal{O}_{\rm ad}$ s.t. the cost functional \eqref{cost functional of general stationary fluid dynamics PDEs} is minimized subject to \eqref{general stationary incompressible NSEs}, i.e.:
\begin{align}
    \label{optimization problem for general stationary incompressible NSEs}
    \tag{opt-gsincNS}
    \min_{\Omega\in\mathcal{O}_{\rm ad}} J({\bf u},p,\Omega) \mbox{ s.t. } ({\bf u},p) \mbox{ solves } \eqref{general stationary incompressible NSEs}.
\end{align}

%------------------------------------------------------------------------------%

\printbibliography[heading=bibintoc]
\end{document}