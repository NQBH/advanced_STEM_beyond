\documentclass{article}
\usepackage[backend=biber,natbib=true,style=alphabetic,maxbibnames=50]{biblatex}
\addbibresource{/home/nqbh/reference/bib.bib}
\usepackage[utf8]{vietnam}
\usepackage{tocloft}
\renewcommand{\cftsecleader}{\cftdotfill{\cftdotsep}}
\usepackage[colorlinks=true,linkcolor=blue,urlcolor=red,citecolor=magenta]{hyperref}
\usepackage{amsmath,amssymb,amsthm,enumitem,float,graphicx,mathtools,tikz}
\usetikzlibrary{angles,calc,intersections,matrix,patterns,quotes,shadings}
\allowdisplaybreaks
\newtheorem{assumption}{Assumption}
\newtheorem{baitoan}{}
\newtheorem{cauhoi}{Câu hỏi}
\newtheorem{conjecture}{Conjecture}
\newtheorem{corollary}{Corollary}
\newtheorem{dangtoan}{Dạng toán}
\newtheorem{definition}{Definition}
\newtheorem{dinhly}{Định lý}
\newtheorem{dinhnghia}{Định nghĩa}
\newtheorem{example}{Example}
\newtheorem{ghichu}{Ghi chú}
\newtheorem{hequa}{Hệ quả}
\newtheorem{hypothesis}{Hypothesis}
\newtheorem{lemma}{Lemma}
\newtheorem{luuy}{Lưu ý}
\newtheorem{nhanxet}{Nhận xét}
\newtheorem{notation}{Notation}
\newtheorem{note}{Note}
\newtheorem{principle}{Principle}
\newtheorem{problem}{Problem}
\newtheorem{proposition}{Proposition}
\newtheorem{question}{Question}
\newtheorem{remark}{Remark}
\newtheorem{theorem}{Theorem}
\newtheorem{vidu}{Ví dụ}
\usepackage[left=1cm,right=1cm,top=5mm,bottom=5mm,footskip=4mm]{geometry}
\def\labelitemii{$\circ$}
\DeclareRobustCommand{\divby}{%
	\mathrel{\vbox{\baselineskip.65ex\lineskiplimit0pt\hbox{.}\hbox{.}\hbox{.}}}%
}
\setlist[itemize]{leftmargin=*}
\setlist[enumerate]{leftmargin=*}

\title{Computer Music -- Âm Nhạc Máy Tính}
\author{Nguyễn Quản Bá Hồng\footnote{A Scientist {\it\&} Creative Artist Wannabe. E-mail: {\tt nguyenquanbahong@gmail.com}. Bến Tre City, Việt Nam.}}
\date{\today}

\begin{document}
\maketitle
\begin{abstract}
	This text is a part of the series {\it Some Topics in Advanced STEM \& Beyond}:
	
	{\sc url}: \url{https://nqbh.github.io/advanced_STEM/}.
	
	Latest version:
	\begin{itemize}
		\item {\it Computer Music -- Âm Nhạc Máy Tính}.
		
		PDF: {\sc url}: \url{.pdf}.
		
		\TeX: {\sc url}: \url{.tex}.
		\item {\it }.
		
		PDF: {\sc url}: \url{.pdf}.
		
		\TeX: {\sc url}: \url{.tex}.
	\end{itemize}
\end{abstract}
\tableofcontents

%------------------------------------------------------------------------------%

\section{}

%------------------------------------------------------------------------------%

%------------------------------------------------------------------------------%

\section{Wikipedia}

\subsection{Wikipedia{\tt/}computer music}
``{\it Computer music} is application of \href{https://en.wikipedia.org/wiki/Computing_technology}{computing technology} in \href{https://en.wikipedia.org/wiki/Musical_composition}{music composition}, to help human composers create new music or to have computers independently create music, e.g. with \href{https://en.wikipedia.org/wiki/Algorithmic_composition}{algorithmic composition} programs. it includes theory \& application of new \& existing computer software technologies \& basic aspects of music, e.g. \href{https://en.wikipedia.org/wiki/Sound_synthesis}{sound synthesis}, \href{https://en.wikipedia.org/wiki/Digital_signal_processing}{digital signal processing}, \href{https://en.wikipedia.org/wiki/Sound_design}{sound design}, sonic diffusion, \href{https://en.wikipedia.org/wiki/Acoustics}{acoustics}, \href{https://en.wikipedia.org/wiki/Electrical_engineering}{electrical engineering}, \& \href{https://en.wikipedia.org/wiki/Psychoacoustics}{psychoacoustics}. Field of computer music can trace its roots back to origins of \href{https://en.wikipedia.org/wiki/Electronic_music}{electric music}, \& 1st experiments \& innovations with electronic instruments at turn of 20th century.

\subsubsection{History}

\subsubsection{Advances}

\subsubsection{Research}

\subsubsection{Machine improvisation}

\subsubsection{Live coding}

'' -- \href{https://en.wikipedia.org/wiki/Computer_music}{Wikipedia{\tt/}computer music}

%------------------------------------------------------------------------------%

\subsection{Wikipedia{\tt/}transcription (music)}
``{\sf A {\sc J. S. Bach} keyboard piece transcribed for guitar.} In music, {\it transcription} is practice of \href{https://en.wikipedia.org/wiki/Musical_notation}{notating} a piece or a sound which was previously unnotated \&{\tt/}or unpopular as a written music, e.g., a \href{https://en.wikipedia.org/wiki/Jazz_improvisation}{jazz improvisation} or a \href{https://en.wikipedia.org/wiki/Video_game_soundtrack}{video game soundtrack}. When a musician is tasked with creating \href{https://en.wikipedia.org/wiki/Sheet_music}{sheet music} from a recording \& they write down notes that make up piece in \href{https://en.wikipedia.org/wiki/Music_notation}{music notation}, it is said: they created a {\it musical transcription} of that recording. Transcription may also mean rewriting a piece of music, either solo or \href{https://en.wikipedia.org/wiki/Musical_ensemble}{ensemble}, for another instrument or other instruments than which it was originally intended. \href{https://en.wikipedia.org/wiki/Beethoven_Symphonies_(Liszt)}{Beethoven Symphonies} transcribed for solo piano by \href{https://en.wikipedia.org/wiki/Franz_Liszt}{Franz Liszt} are an example. Transcription in this sense is sometimes called \href{https://en.wikipedia.org/wiki/Arrangement}{\it arrangement}, although strictly speaking transcriptions are faithful adaptations, whereas arrangements change significant aspects of original piece.

Further examples of music transcription include \href{https://en.wikipedia.org/wiki/Ethnomusicology}{ethnomusicological} notation of \href{https://en.wikipedia.org/wiki/Oral_tradition}{oral traditions} of folk music, e.g. Béla Bartók's and Ralph Vaughan Williams' collections of national folk music of Hungary \& England resp. French composer Olivier Messiaen transcribed \href{https://en.wikipedia.org/wiki/Bird_song}{birdsong} in wild, \& incorporated it into many of his compositions, e.g. his \href{https://en.wikipedia.org/wiki/Catalogue_d%27oiseaux}{Catalogue d'oiseaux} for solo piano. Transcription of this nature involves scale degree recognition \& harmonic analysis, both of which transcriber will need \href{https://en.wikipedia.org/wiki/Relative_pitch}{relative} or \href{https://en.wikipedia.org/wiki/Perfect_pitch}{perfect pitch} to perform. 

In popular music \& rock, there are 2 forms of transcription. Individual performers copy a note-for-note guitar solo or other melodic line. As well, music publishers transcribe entire recordings of guitar solos \& bass lines \& sell sheet music in bound books. Music publishers also publish PVG (piano{\tt/}vocal{\tt/}guitar) transcriptions of popular music, where melody line is transcribed, \& then accompaniment on recording is arranged as a piano part. Guitar aspect of PVG label is achieved through guitar chords written above melody. Lyrics are also included below melody.

\subsubsection{Adaptation}

\subsubsection{Transcription aids}

\subsubsection{Automatic music transcription}

'' -- \href{https://en.wikipedia.org/wiki/Transcription_(music)}{Wikipedia{\tt/}transcription (music)}

%------------------------------------------------------------------------------%

\section{Miscellaneous}

%------------------------------------------------------------------------------%

\printbibliography[heading=bibintoc]
	
\end{document}