\documentclass[12pt]{article}
\usepackage[utf8]{inputenc}
\usepackage[vietnamese]{babel}
\usepackage{amsmath, amssymb}
\usepackage{listings}
\usepackage{enumitem}
\usepackage{geometry}
\geometry{margin=1in}

\title{Bài Toán 10: Thuật Toán BFS Trên Đồ Thị Tổng Quát}
\author{Đồ Án 5.1: Breadth-first Search}
\date{}

\begin{document}

\maketitle

\section*{Phát biểu bài toán}
Cho một đồ thị tổng quát $G = (V, E)$, không giả định đơn, đa cung hay hướng/vô hướng.  
Viết thuật toán Breadth-First Search (BFS) có khả năng hoạt động chính xác trên mọi loại đồ thị.

\section*{Giả thuyết}
\begin{itemize}
    \item Đồ thị có thể có nhiều thành phần liên thông
    \item Đỉnh có thể có self-loop (tức là $(v,v) \in E$)
    \item Có thể có nhiều cạnh nối giữa một cặp đỉnh
    \item Có thể là đồ thị hữu hướng hoặc vô hướng
    \item Có thể rỗng
\end{itemize}

\section*{Ý tưởng}
\begin{itemize}
    \item Sử dụng BFS tiêu chuẩn cho từng thành phần liên thông
    \item Duyệt BFS từ mọi đỉnh chưa được thăm để bao phủ toàn bộ đồ thị
    \item Sử dụng mảng \texttt{visited[]} để tránh lặp chu kỳ, đa cung hoặc self-loop
\end{itemize}

\section*{Thuật toán BFS tổng quát (pseudocode)}
\begin{verbatim}
General_BFS(G):
    n ← số lượng đỉnh
    visited[v] ← False ∀ v ∈ V
    for mỗi đỉnh u từ 0 đến n-1:
        if not visited[u]:
            Q ← hàng đợi mới
            enqueue(Q, u)
            visited[u] ← True
            while Q không rỗng:
                v ← dequeue(Q)
                xử lý v
                for mỗi đỉnh w ∈ adj[v]:
                    if not visited[w]:
                        visited[w] ← True
                        enqueue(Q, w)
\end{verbatim}

\section*{Chú thích các biến số}
\begin{itemize}
    \item \texttt{G}: đồ thị tổng quát, có thể có hướng, đa cung, self-loop
    \item \texttt{adj[v]}: danh sách các đỉnh kề với $v$ (có thể trùng)
    \item \texttt{visited[v]}: boolean kiểm tra đã duyệt đỉnh $v$
    \item \texttt{Q}: hàng đợi BFS dùng cho từng thành phần liên thông
    \item \texttt{res}: danh sách kết quả BFS từ từng component
\end{itemize}

\section*{Xử lý bài toán}
\begin{itemize}
    \item Với self-loop: $(v,v)$ không ảnh hưởng nếu đã kiểm tra \texttt{visited[v]} đúng cách
    \item Với multigraph: dù có nhiều cạnh trùng, BFS chỉ duyệt 1 lần
    \item Với đồ thị không liên thông: dùng BFS trên từng thành phần
\end{itemize}

\end{document}
