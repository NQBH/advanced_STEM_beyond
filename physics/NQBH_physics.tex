\documentclass{article}
\usepackage[backend=biber,natbib=true,style=alphabetic,maxbibnames=50]{biblatex}
\addbibresource{/home/nqbh/reference/bib.bib}
\usepackage[utf8]{vietnam}
\usepackage{tocloft}
\renewcommand{\cftsecleader}{\cftdotfill{\cftdotsep}}
\usepackage[colorlinks=true,linkcolor=blue,urlcolor=red,citecolor=magenta]{hyperref}
\usepackage{amsmath,amssymb,amsthm,enumitem,float,graphicx,mathtools,tikz}
\usetikzlibrary{angles,calc,intersections,matrix,patterns,quotes,shadings}
\allowdisplaybreaks
\newtheorem{assumption}{Assumption}
\newtheorem{baitoan}{}
\newtheorem{cauhoi}{Câu hỏi}
\newtheorem{conjecture}{Conjecture}
\newtheorem{corollary}{Corollary}
\newtheorem{dangtoan}{Dạng toán}
\newtheorem{definition}{Definition}
\newtheorem{dinhly}{Định lý}
\newtheorem{dinhnghia}{Định nghĩa}
\newtheorem{example}{Example}
\newtheorem{ghichu}{Ghi chú}
\newtheorem{hequa}{Hệ quả}
\newtheorem{hypothesis}{Hypothesis}
\newtheorem{lemma}{Lemma}
\newtheorem{luuy}{Lưu ý}
\newtheorem{nhanxet}{Nhận xét}
\newtheorem{notation}{Notation}
\newtheorem{note}{Note}
\newtheorem{principle}{Principle}
\newtheorem{problem}{Problem}
\newtheorem{proposition}{Proposition}
\newtheorem{question}{Question}
\newtheorem{remark}{Remark}
\newtheorem{theorem}{Theorem}
\newtheorem{vidu}{Ví dụ}
\usepackage[left=1cm,right=1cm,top=5mm,bottom=5mm,footskip=4mm]{geometry}
\def\labelitemii{$\circ$}
\DeclareRobustCommand{\divby}{%
	\mathrel{\vbox{\baselineskip.65ex\lineskiplimit0pt\hbox{.}\hbox{.}\hbox{.}}}%
}
\setlist[itemize]{leftmargin=*}
\setlist[enumerate]{leftmargin=*}

\title{Physics -- Vật Lý}
\author{Nguyễn Quản Bá Hồng\footnote{A Scientist {\it\&} Creative Artist Wannabe. E-mail: {\tt nguyenquanbahong@gmail.com}. Bến Tre City, Việt Nam.}}
\date{\today}

\begin{document}
\maketitle
\begin{abstract}
	This text is a part of the series {\it Some Topics in Advanced STEM \& Beyond}:
	
	{\sc url}: \url{https://nqbh.github.io/advanced_STEM/}.
	
	Latest version:
	\begin{itemize}
		\item {\it Physics -- Vật Lý}.
		
		PDF: {\sc url}: \url{https://github.com/NQBH/advanced_STEM_beyond/blob/main/physics/NQBH_physics.pdf}.
		
		\TeX: {\sc url}: \url{https://github.com/NQBH/advanced_STEM_beyond/blob/main/physics/NQBH_physics.tex}.
	\end{itemize}
\end{abstract}
\tableofcontents

%------------------------------------------------------------------------------%

\section{Wikipedia}

\subsection{Wikipedia{\tt/}advection}
``In the field of \href{https://en.wikipedia.org/wiki/Physics}{physics}, \href{https://en.wikipedia.org/wiki/Engineering}{engineering}, \& \href{https://en.wikipedia.org/wiki/Earth_science}{earth sciences}, {\it advection} is the \href{https://en.wikipedia.org/wiki/Transport}{transport} of a substance or quantity by bulk motion.

The properties of that substance are carried with it.

Generally the majority of the advected substance is a fluid.

The properties that are carried with the advected substance are \href{https://en.wikipedia.org/wiki/Conservation_of_energy}{conserved} properties e.g. \href{https://en.wikipedia.org/wiki/Energy}{energy}.

An example of advection is the transport of \href{https://en.wikipedia.org/wiki/Pollutant}{pollutants} or \href{https://en.wikipedia.org/wiki/Silt}{silt} in a \href{https://en.wikipedia.org/wiki/River}{river} by bulk water flow downstream.

Another commonly advected quantity is energy or \href{https://en.wikipedia.org/wiki/Enthalpy}{enthalpy}.

Here the fluid may be any material that contains thermal energy, e.g. \href{https://en.wikipedia.org/wiki/Water}{water} or \href{https://en.wikipedia.org/wiki/Air}{air}.

In general, any substance or conserved, \href{https://en.wikipedia.org/wiki/Intensive_and_extensive_properties}{extensive} quantity can be advected by a fluid that can hold or contain the quantity or substance.

%
During advection, a fluid transports some conserved quantity or material via bulk motion.

The fluid's motion is described mathematically as a \href{https://en.wikipedia.org/wiki/Vector_field}{vector field}, \& the transported material is described by a \href{https://en.wikipedia.org/wiki/Scalar_field}{scalar field} showing its distribution over space.

Advection requires currents in the fluid, \& so cannot happen in rigid solids.

It does not include transport of substances by \href{https://en.wikipedia.org/wiki/Molecular_diffusion}{molecular diffusion}.

%
Advection is sometimes confused with the more encompassing process of \href{https://en.wikipedia.org/wiki/Convection}{convection} which is the combination of advective transport \& diffusive transport.

%
In \href{https://en.wikipedia.org/wiki/Meteorology}{meteorology} \& \href{https://en.wikipedia.org/wiki/Physical_oceanography}{physical oceanography}, advection often refers to the transport of some property of the atmosphere or \href{https://en.wikipedia.org/wiki/Ocean}{ocean}, e.g., \href{https://en.wikipedia.org/wiki/Heat}{heat}, humidity (see \href{https://en.wikipedia.org/wiki/Water_vapor}{moisture}) or \href{https://en.wikipedia.org/wiki/Salinity}{salinity}.

Advection is important for the formation of \href{https://en.wikipedia.org/wiki/Orographic}{orographic} clouds \& the precipitation of water from clouds, as part of the \href{https://en.wikipedia.org/wiki/Hydrological_cycle}{hydrological cycle}.

\subsubsection{Distinction between advection \& convection}
The term {\it advection} often serves as a synonym for \href{https://en.wikipedia.org/wiki/Convection}{{\it convection}}, \& this correspondence of terms is used in the literature.

More technically, convection applies to the movement of a fluid (often due to density gradients created by thermal gradients), whereas advection is the movement of some material by the velocity of the fluid.

Thus, somewhat confusingly, it is technically correct to think of momentum being advected by the velocity field in the Navier-Stokes equations, although the resulting motion would be considered to be convection.

Because of the specific use of the term convection to indicate transport in association with thermal gradients, it is probably safer to use the term advection if one is uncertain about which terminology best describes their particular system.

\subsubsection{Meteorology}
In \href{https://en.wikipedia.org/wiki/Meteorology}{meteorology} \& \href{https://en.wikipedia.org/wiki/Physical_oceanography}{physical oceanography}, advection often refers to the horizontal transport of some property of the atmosphere or \href{https://en.wikipedia.org/wiki/Ocean}{ocean}, e.g., \href{https://en.wikipedia.org/wiki/Heat}{heat}, humidity or salinity, \& convection generally refers to vertical transport (vertical advection).

Advection is important for the formation of \href{https://en.wikipedia.org/wiki/Orographic_cloud}{orographic clouds} (terrain-forced convection) \& the precipitation of water from clouds, as part of the \href{https://en.wikipedia.org/wiki/Hydrological_cycle}{hydrological cycle}.

\subsubsection{Other quantities}
The advection equation also applies if the quantity being advected is represented by a \href{https://en.wikipedia.org/wiki/Probability_density_function}{probability density function} at each point, although accounting for diffusion is more difficult.

\subsubsection{Mathematics of advection}
The {\it advection equation} is the \href{https://en.wikipedia.org/wiki/Partial_differential_equation}{PDE} that governs the motion of a conserved \href{https://en.wikipedia.org/wiki/Scalar_field}{scalar field} as it is advected by a known \href{https://en.wikipedia.org/wiki/Velocity_field}{velocity vector field}.

It is derived using the scalar field's \href{https://en.wikipedia.org/wiki/Conservation_law}{conservation law}, together with \href{https://en.wikipedia.org/wiki/Gauss's_theorem}{Gauss's theorem}, \& taking the \href{https://en.wikipedia.org/wiki/Infinitesimal}{infinitesimal} limit.

%
One easily visualized example of advection is the transport of ink dumped into a river.

As the river flows, ink will move downstream in a ``pulse'' via advection, as the water's movement itself transports the ink.

If added to a lake without significant bulk water flow, the ink would simply disperse outwards from its source in a \href{https://en.wikipedia.org/wiki/Diffusion}{diffusive} manner, which is not advection.

Note that as it moves downstream, the ``pulse'' of ink will also spread via diffusion.

The sum of these processes is called \href{https://en.wikipedia.org/wiki/Convection}{convection}.

\paragraph{The advection equation.} In Cartesian coordinates the advection \href{https://en.wikipedia.org/wiki/Operator_(mathematics)}{operator} is
\begin{align*}
	{\bf u}\cdot\nabla = u_x\partial_x + u_y\partial_y + u_z\partial_z,
\end{align*}
where ${\bf u} = (u_x,u_y,u_z)$ is the \href{https://en.wikipedia.org/wiki/Velocity_field}{velocity field}, \& $\nabla$ is the \href{https://en.wikipedia.org/wiki/Del}{del} operator (note that \href{https://en.wikipedia.org/wiki/Cartesian_coordinate_system}{Cartesian coordinates} are used here).

%
The advection equation for a conserved quantity described by a \href{https://en.wikipedia.org/wiki/Scalar_field}{scalar field} $\psi$ is expressed mathematically by a \href{https://en.wikipedia.org/wiki/Continuity_equation}{continuity equation}:
\begin{align*}
	\partial_t\psi + \nabla\cdot\left(\psi{\bf u}\right) = 0,
\end{align*}
where $\nabla\cdot$  is the divergence operator \& again ${\bf u}$ is the \href{https://en.wikipedia.org/wiki/Velocity_field}{velocity vector field}.

Frequently, it is assumed that the flow is \href{https://en.wikipedia.org/wiki/Incompressible_flow}{incompressible}, i.e., the \href{https://en.wikipedia.org/wiki/Velocity_field}{velocity field} satisfies 
\begin{align*}
	\nabla\cdot{\bf u} = 0.
\end{align*}
In this case, ${\bf u}$ is said to be \href{https://en.wikipedia.org/wiki/Solenoidal}{solenoidal}.

If this is so, the above equation can be rewritten as
\begin{align*}
	\partial_t\psi + {\bf u}\cdot\nabla\psi = 0.
\end{align*}
In particular, if the flow is steady, then
\begin{align*}
	{\bf u}\cdot\nabla\psi = 0,
\end{align*}
which shows that $\psi$ is constant along a streamline.

Hence, $\partial_t\psi = 0$, so $\psi$ does not vary in time.

%
If a vector quantity ${\bf a}$ (e.g., a \href{https://en.wikipedia.org/wiki/Magnetic_field}{magnetic field}) is being advected by the \href{https://en.wikipedia.org/wiki/Solenoidal}{solenoidal} \href{https://en.wikipedia.org/wiki/Velocity_field}{velocity field} ${\bf u}$, the advection equation above becomes: 
\begin{align*}
	\partial_t{\bf a} + ({\bf u}\cdot\nabla){\bf a} = 0.
\end{align*}
Here, ${\bf a}$ is a vector field instead of the scalar field $\psi$.

\paragraph{Solving the equation.} {\sf A simulation of the advection equation where ${\bf u} = (\sin t,\cos t)$ is solenoidal.}

%
The advection equation is not simple to solve \href{https://en.wikipedia.org/wiki/Numerical_analysis}{numerically}: the system is a \href{https://en.wikipedia.org/wiki/Hyperbolic_partial_differential_equation}{hyperbolic partial differential equation}, \& interest typically centers on \href{https://en.wikipedia.org/wiki/Continuous_function}{discontinuous} ``shock'' solutions (which are notoriously difficult for numerical schemes to handle).

%
Even with 1 space dimension \& a constant \href{https://en.wikipedia.org/wiki/Velocity_field}{velocity field}, the system remains difficult to simulate.

The equation becomes
\begin{align*}
	\partial_t\psi + u_x\partial_x\psi = 0,
\end{align*}
where $\psi = \psi(t,x)$ is the scalar field being advected \& $u_x$ is the $x$ component of the vector ${\bf u} = (u_x,0,0)$.

\paragraph{Treatment of the advection operator in the compressible NSEs.} According to Zang,[Zang, Thomas (1991). ``On the rotation \& skew-symmetric forms for incompressible flow simulations''. Applied Numerical Mathematics. 7: 27--40. Bibcode:1991ApNM....7...27Z. doi:10.1016/0168-9274(91)90102-6.] numerical simulation can be aided by considering the \href{https://en.wikipedia.org/wiki/Skew-symmetric_matrix}{skew symmetric} form for the advection operator.
\begin{align*}
	\frac{1}{2}{\bf u}\cdot\nabla{\bf u} + \frac{1}{2}\nabla({\bf u}{\bf u}) \mbox{ where } \nabla({\bf u}{\bf u}) = \left[\nabla({\bf u}u_x),\nabla({\bf u}u_y),\nabla({\bf u}u_z)\right]
\end{align*}
and ${\bf u}$ is the same as above.

%
Since skew symmetry implies only imaginary eigenvalues, this form reduces the ``blow up'' \& ``spectral blocking'' often experienced in numerical solutions with sharp discontinuities (see Boyd[Boyd, John P. (2000). {\it Chebyshev \& Fourier Spectral Methods} 2nd edition. Dover. p. 213.]).

%
Using \href{https://en.wikipedia.org/wiki/Vector_calculus_identities#Vector_dot_product}{vector calculus identities}, these operators can also be expressed in other ways, available in more software packages for more coordinate systems.
\begin{align*}
	{\bf u}\cdot\nabla{\bf u} &= \nabla\left(\frac{\|{\bf u}\|^2}{2}\right) + (\nabla\times{\bf u})\times{\bf u},\\
	\frac{1}{2}{\bf u}\cdot\nabla{\bf u} + \frac{1}{2}\nabla({\bf u}{\bf u}) &= \nabla\left(\frac{\|{\bf u}\|^2}{2}\right) + (\nabla\times{\bf u})\times{\bf u} + \frac{1}{2}{\bf u}(\nabla\cdot{\bf u}).
\end{align*}
This form also makes visible that the \href{https://en.wikipedia.org/wiki/Skew-symmetric_matrix}{skew symmetric} operator introduces error when the velocity field diverges.

Solving the advection equation by numerical methods is very challenging \& there is a large scientific literature about this.'' -- \href{https://en.wikipedia.org/wiki/Advection}{Wikipedia{\tt/}advection}

%------------------------------------------------------------------------------%

\subsection{Wikipedia{\tt/}computational physics}
``{\it Computational physics} is the study \& implementation of numerical analysis to solve problems in physics. Historically, computational physics was the 1st application of modern computers in science, \& is now a subset of \href{https://en.wikipedia.org/wiki/Computational_science}{computational science}. It is sometimes regarded as a subdiscipline (or offshoot) of \href{https://en.wikipedia.org/wiki/Theoretical_physics}{theoretical physics}, but others consider it an intermediate branch between theoretical \& \href{https://en.wikipedia.org/wiki/Experimental_physics}{experimental physics} -- an area of study which supplements both theory \& experiment.

\subsubsection{Overview}
In physics, different \href{https://en.wikipedia.org/wiki/Theory}{theories} based on mathematical models provide very precise predictions on how system behave. Unfortunately, it is often the case that solving the mathematical model for a particular system in order to produce a useful prediction is not feasible. This can occur, e.g., when the solution does not have a \href{https://en.wikipedia.org/wiki/Closed-form_expression}{closed-form expression}, or is too complicated. In such cases, numerical approximations are required. Computational physics is the subject that deals with these numerical approximations: the approximation of the solution is written as a finite (\& typically large) number of simple mathematical operations (\href{https://en.wikipedia.org/wiki/Algorithm}{algorithm}), \& a computer is used to perform these operations \& compute an approximated solution \& respective \href{https://en.wikipedia.org/wiki/Approximation_error}{approximation error}.

\paragraph{Status in physics.} {\sf A representation of the multidisciplinary nature of computational physics both as an overlap of physics, applied mathematics, \& computer science \& as a bridge among them.} There is a debate about the status of computation within the scientific method. Sometimes it is regarded as more akin to theoretical physics; some others regard computer simulation as ``\href{https://en.wikipedia.org/wiki/Computer_experiment}{computer experiments}'', yet still others consider it an intermediate or different branch between theoretical \& \href{https://en.wikipedia.org/wiki/Experimental_physics}{experimental physics}, a 3rd way that supplements theory \& experiment. While computers can be used in experiments for the measurement \& recording (\& storage) of data, this clearly does not constitute a computational approach.

\subsubsection{Challenges in computational physics}
Computational physics problems are in general very difficult to solve exactly. This is due to several (mathematical) reasons: lack of algebraic \&{\tt/}or analytic solvability, \href{https://en.wikipedia.org/wiki/Complexity}{complexity}, \& chaos. E.g., even apparently simple problems, e.g., calculating the \href{https://en.wikipedia.org/wiki/Wavefunction}{wavefunction} of an electron orbiting an atom in a strong \href{https://en.wikipedia.org/wiki/Electric_field}{electric field} (\href{https://en.wikipedia.org/wiki/Stark_effect}{Stark effect}, may require great effort to formulate a practical algorithm (if one can be found); other cruder or brute-force techniques, e.g. \href{https://en.wikipedia.org/wiki/Graphical_method}{graphical methods} or \href{https://en.wikipedia.org/wiki/Root_finding}{root finding}, may be required. On the more advanced side, mathematical \href{https://en.wikipedia.org/wiki/Perturbation_theory}{perturbation theory} is also sometimes used (a working is shown for this particular \href{https://en.wikipedia.org/wiki/Perturbation_theory#Example_of_degenerate_perturbation_theory_%E2%80%93_Stark_effect_in_resonant_rotating_wave}{example}). In addition, the \href{https://en.wikipedia.org/wiki/Computational_cost}{computational cost} \& \href{https://en.wikipedia.org/wiki/Computational_complexity_theory}{computational complexity} for \href{https://en.wikipedia.org/wiki/Many-body_problem}{many-body problems} (\& their \href{https://en.wikipedia.org/wiki/N-body_problem}{classical counterparts}) tend to grow quickly. A macroscopic system typically has a size of the order of $10^{23}$ constituent particles, so it is somewhat of a problem. Solving quantum mechanical problems is generally of \href{https://en.wikipedia.org/wiki/EXP}{exponential order} in the size of the system \& for classical $N$-body it is of order $N$-squared. Finally, many physical systems are inherently nonlinear at best, \& at worst \href{https://en.wikipedia.org/wiki/Chaos_theory}{chaotic}: i.e., it can be difficult to ensure any \href{https://en.wikipedia.org/wiki/Numerical_error}{numerical errors} do not grow to the point of rendering the `solution' useless.

\subsubsection{Methods \& algorithms}
Because computational physics uses a broad class of problems, it is generally divided amongst the different mathematical problems it numerically solves, or the methods it apples. Between them, one can consider:
\begin{itemize}
	\item \href{https://en.wikipedia.org/wiki/Root-finding_algorithm}{root finding} (using e.g. \href{https://en.wikipedia.org/wiki/Newton%27s_method}{Newton--Raphson method})
	\item \href{https://en.wikipedia.org/wiki/System_of_linear_equations}{system of linear equations} (using e.g. \href{https://en.wikipedia.org/wiki/LU_decomposition}{LU decomposition})
	\item \href{https://en.wikipedia.org/wiki/Ordinary_differential_equation}{ODEs} (using e.g. \href{https://en.wikipedia.org/wiki/Runge%E2%80%93Kutta_methods}{Runge--Kutta methods})
	\item \href{https://en.wikipedia.org/wiki/Integral}{integration} (using e.g. \href{https://en.wikipedia.org/wiki/Romberg%27s_method}{Romberg method} \& \href{https://en.wikipedia.org/wiki/Monte_Carlo_integration}{Monte Carlo integration})
	\item PDEs (using e.g. FDM \& \href{https://en.wikipedia.org/wiki/Relaxation_(iterative_method)}{relaxation} method)
	\item \href{https://en.wikipedia.org/wiki/Matrix_eigenvalue_problem}{matrix eigenvalue problem} (using e.g. \href{https://en.wikipedia.org/wiki/Jacobi_eigenvalue_algorithm}{Jacobi eigenvalue algorithm} \& \href{https://en.wikipedia.org/wiki/Power_iteration}{power iteration})
\end{itemize}
All these methods (\& several others) are used to calculate physical properties of the modeled systems.

Computational physics also borrows a number of ideas from \href{https://en.wikipedia.org/wiki/Computational_chemistry}{computational chemistry} -- e.g., the \href{https://en.wikipedia.org/wiki/Density_functional_theory}{density functional theory} used by computational solid state physicists to calculate properties of solids is basically the same as that used by chemists to calculate the properties of molecules.

Furthermore, computational physics encompasses the \href{https://en.wikipedia.org/wiki/Performance_tuning}{tunning} of the \href{https://en.wikipedia.org/wiki/Self-tuning#Examples}{software}{\tt/}\href{https://en.wikipedia.org/wiki/Category:Computer_hardware_tuning}{hardware structure} to solve the problems (as the problems usually can be very large, in \href{https://en.wikipedia.org/wiki/High_performance_computing}{processing power need} or in \href{https://en.wikipedia.org/wiki/High-throughput_computing}{memory requests}).

\subsubsection{Divisions}
It is possible to find a corresponding computational branch for every major field in physics:
\begin{enumerate}
	\item \href{https://en.wikipedia.org/wiki/Computational_mechanics}{Computational mechanics} consists of CFDs, computational \href{https://en.wikipedia.org/wiki/Solid_mechanics}{solid mechanics} \& computational \href{https://en.wikipedia.org/wiki/Contact_mechanics}{contact mechanics}.
	\item \href{https://en.wikipedia.org/wiki/Computational_electrodynamics}{Computational electrodynamics} is the process of modeling the interaction of \href{https://en.wikipedia.org/wiki/Electromagnetic_fields}{electromagnetic fields} with physical objects \& the environment. 1 subfield at the confluence between CFD \& electromagnetic modeling is \href{https://en.wikipedia.org/wiki/Computational_magnetohydrodynamics}{computational magnetohydrodynamics}.
	\item \href{https://en.wikipedia.org/wiki/Computational_chemistry}{Computational chemistry} is a rapidly growing field that was developed due to the \href{https://en.wikipedia.org/wiki/Quantum_many-body_problem}{quantum many-body problem}.
	\item Computational \href{https://en.wikipedia.org/wiki/Solid_state_physics}{solid state physics} is a very important division of computational physics dealing directly with \href{https://en.wikipedia.org/wiki/Material_science}{material science}.
	\item Computational \href{https://en.wikipedia.org/wiki/Statistical_mechanics}{statistical mechanics} is a field related to computational \href{https://en.wikipedia.org/wiki/Condensed_matter}{condensed matter} which deals with the simulation of models \& theories (e.g. \href{https://en.wikipedia.org/wiki/Percolation}{percolation} \& \href{https://en.wikipedia.org/wiki/Spin_model}{spin models}) that are difficult to solve otherwise.
	\item Computational \href{https://en.wikipedia.org/wiki/Statistical_physics}{statistial physics} make heavy use of Monte Carlo--like methods. More broadly, (particularly through the use of \href{https://en.wikipedia.org/wiki/Agent_based_modeling}{agent based modeling} \& \href{https://en.wikipedia.org/wiki/Cellular_automata}{cellular automata}) it also concerns itself with (\& finds application in, through the use of its techniques) in the \href{https://en.wikipedia.org/wiki/Social_sciences}{social sciences}, \href{https://en.wikipedia.org/wiki/Network_theory}{network theory}, \& mathematical models for the propagation of disease (most notably, the \href{https://en.wikipedia.org/wiki/Compartmental_models_in_epidemiology#SIR_Model_on_Networks}{SIR Model}) \& the \href{https://en.wikipedia.org/wiki/Wildfire_modeling}{spread of forest fires}.
	\item \href{https://en.wikipedia.org/wiki/Numerical_relativity}{Numerical relativity} is a (relatively) new field interested in finding numerical solutions to the field equations of both \href{https://en.wikipedia.org/wiki/Special_relativity}{special relativity} \& \href{https://en.wikipedia.org/wiki/General_relativity}{general relativity}.
	\item \href{https://en.wikipedia.org/wiki/Computational_particle_physics}{Computational particle physics} deals with problems motivated by particle physics.
	\item \href{https://en.wikipedia.org/wiki/Computational_astrophysics}{Computational astrophysics} is the application of these techniques \& methods to astrophysical problems \& phenomena.
	\item \href{https://en.wikipedia.org/wiki/Computational_biophysics}{Computational biophysics} is a branch of biophysics \& \href{https://en.wikipedia.org/wiki/Computational_biology}{computational biology} itself, applying methods of computer science \& physics to large complex biological problems.
\end{enumerate}

\subsubsection{Applications}
Due to the broad class of problems computational of physics deals, it is an essential component of modern research in different areas of physics, namely: \href{https://en.wikipedia.org/wiki/Accelerator_physics}{accelerator physics}, \href{https://en.wikipedia.org/wiki/Astrophysics}{astrophysics}, \href{https://en.wikipedia.org/wiki/General_theory_of_relativity}{general theory of relativity} (through \href{https://en.wikipedia.org/wiki/Numerical_relativity}{numerical relativity}), \href{https://en.wikipedia.org/wiki/Fluid_mechanics}{fluid mechanics} (CFDs), \href{https://en.wikipedia.org/wiki/Lattice_field_theory}{lattice field theory}{\tt/}\href{https://en.wikipedia.org/wiki/Lattice_gauge_theory}{lattice gauge theory} (especially \href{https://en.wikipedia.org/wiki/Lattice_QCD}{lattice quantum chromodynamics}), \href{https://en.wikipedia.org/wiki/Plasma_physics}{plasma physics} (see \href{https://en.wikipedia.org/wiki/Plasma_modeling}{plasma modeling}), simulating physical systems (using e.g. \href{https://en.wikipedia.org/wiki/Molecular_dynamics}{molecular dynamics}), \href{https://en.wikipedia.org/wiki/Nuclear_engineering_computer_codes}{nuclear engineering computer codes}, \href{https://en.wikipedia.org/wiki/Protein_structure_prediction}{protein structure prediction}, \href{https://en.wikipedia.org/wiki/Weather_prediction}{weather prediction}, \href{https://en.wikipedia.org/wiki/Solid_state_physics}{solid state physics}, \href{https://en.wikipedia.org/wiki/Soft_condensed_matter}{soft condensed matter} physics, hypervelocity impact physics etc.

Computational solid state physics, e.g., uses \href{https://en.wikipedia.org/wiki/Density_functional_theory}{density functional theory} to calculate properties of solids, a method similar to that used by chemists to study molecules. Other quantities of interest in solid state physics, e.g. electronic band structure, magnetic properties \& charge densities can be calculated by this \& several methods, including the \href{https://en.wikipedia.org/wiki/Luttinger-Kohn_model}{Lutting-Kohn}{\tt/}\href{https://en.wikipedia.org/wiki/K.p_method}{k.p method} \& \href{https://en.wikipedia.org/wiki/Ab-initio}{ab-initio} methods.

On top of advanced physics software, there are also a myriad of tools of analytics available for beginning students of physics e.g. PASCO Capstone software.'' -- \href{https://en.wikipedia.org/wiki/Computational_physics}{Wikipedia{\tt/}computational physics}

%------------------------------------------------------------------------------%

\subsection{Wikipedia{\tt/}diffusion}
{\sf Some particles are \href{https://en.wikipedia.org/wiki/Dissolution_(chemistry)}{dissolved} in a glass of water. At 1st, the particles are all near 1 top corner of the glass. If the particles randomly move around (``diffuse'') in the water, they eventually become distributed randomly \& uniformly from an area of high concentration to an area of low, \& organized (diffusion continues, but with no net \href{https://en.wikipedia.org/wiki/Flux}{flux}).}

``{\it Diffusion} is the net movement of anything (e.g., atoms, ions, molecules, energy) generally from a region of higher \href{https://en.wikipedia.org/wiki/Concentration}{concentration} to a region of lower concentration. Diffusion is driven by a gradient in \href{https://en.wikipedia.org/wiki/Gibbs_free_energy}{Gibbs free energy} or \href{https://en.wikipedia.org/wiki/Chemical_potential}{chemical potential}. It is possible to diffuse ``uphill'' from a region of lower concentration to a region of higher concentration, as in \href{https://en.wikipedia.org/wiki/Spinodal_decomposition}{spinodal decomposition}. Diffusion is a stochastic process due to the inherent randomness of the diffusing entity \& can be used to model many real-life stochastic scenarios. Therefore, diffusion \& the corresponding mathematical models are used in several fields beyond physics, e.g. \href{https://en.wikipedia.org/wiki/Statistics}{statistics}, \href{https://en.wikipedia.org/wiki/Probability_theory}{probability theory}, \href{https://en.wikipedia.org/wiki/Information_theory}{information theory}, \href{https://en.wikipedia.org/wiki/Neural_networks}{neural networks}, \href{https://en.wikipedia.org/wiki/Finance}{finance}, \& \href{https://en.wikipedia.org/wiki/Marketing}{marketing}.

The concept of diffusion is widely used in many fields, including physics (\href{https://en.wikipedia.org/wiki/Molecular_diffusion}{particle diffusion}), chemistry, biology, sociology, economics, statistics, data science, \& finance (diffusion of people, ideas, data, \& price values). The central idea of diffusion, however, is common to all of these: a substance or collection undergoing diffusion spreads out from a point or location at which there is a higher concentration of that substance or collection.

A \href{https://en.wikipedia.org/wiki/Gradient}{gradient} is the change in the value of a quantity; e.g., concentration, \href{https://en.wikipedia.org/wiki/Pressure}{pressure}, or \href{https://en.wikipedia.org/wiki/Temperature}{temperature} with the change in another variable, usually \href{https://en.wikipedia.org/wiki/Distance}{distance}. A change in concentration over a distance is called a \href{https://en.wikipedia.org/wiki/Molecular_diffusion}{concentration gradient}, a change in pressure over a distance is called a \href{https://en.wikipedia.org/wiki/Pressure_gradient}{pressure gradient}, \& a change in temperature over a distance is called a \href{https://en.wikipedia.org/wiki/Temperature_gradient}{temperature gradient}.

The word {\it diffusion} derives from the Latin word, {\it diffundere}, which means ``to spread out''.

A distinguishing feature of diffusion is that it depends on particle \href{https://en.wikipedia.org/wiki/Random_walk}{random walk}, \& results in mixing or mass transport without requiring directed bulk motion. Bulk motion, or bulk flow, is the characteristic of \href{https://en.wikipedia.org/wiki/Advection}{advection}. The term \href{https://en.wikipedia.org/wiki/Convection}{convection} is used to describe the combination of both \href{https://en.wikipedia.org/wiki/Transport_phenomena}{transport phenomena}.

-- 1 đặc điểm nổi bật của sự khuếch tán là nó phụ thuộc vào bước đi ngẫu nhiên của hạt và dẫn đến sự trộn lẫn hoặc vận chuyển khối lượng mà không cần chuyển động khối có hướng. Chuyển động khối, hay dòng chảy khối, là đặc điểm của sự tiến bộ. Thuật ngữ đối lưu được sử dụng để mô tả sự kết hợp của cả hai hiện tượng vận chuyển.

In a diffusion process can be described by \href{https://en.wikipedia.org/wiki/Fick%27s_laws_of_diffusion}{Fick's laws}, it is called a normal diffusion (or Fickian diffusion), otherwise, it is called an \href{https://en.wikipedia.org/wiki/Anomalous_diffusion}{anomalous diffusion} (or non-Fickian diffusion).

When talking about the extent of diffusion, 2 length scales are used in 2 different scenarios:
\begin{enumerate}
	\item \href{https://en.wikipedia.org/wiki/Brownian_motion}{Brownian motion} of an \href{https://en.wikipedia.org/wiki/Impulse_response}{impulsive} point source (e.g., 1 single spray of perfume) -- the square root of the \href{https://en.wikipedia.org/wiki/Mean_squared_displacement}{mean squared displacement} from this point. In Fickian diffusion, this is $\sqrt{2nDt}$, where $n$ is the dimension of this Brownian motion;
	\item \href{https://en.wikipedia.org/wiki/Fick%27s_laws_of_diffusion#Example_solutions_and_generalization}{Constant concentration source} in 1D -- the diffusion length. In Fickian diffusion, this is $2\sqrt{Dt}$.
\end{enumerate}
{\sf Diffusion from a microscopic \& a macroscopic point of view. Initially, there are \href{https://en.wikipedia.org/wiki/Solute}{solute} molecules on the left side of a barrier \& none on the right. The barrier is removed, \& the solute diffuses to fill the whole container. Top: A single molecule moves around randomly. Middle: With more molecules, there is a statistical trend that the solute fills the container more \& more uniformly. Bottom: With an enormous number of solute molecules, all randomness is gone: The solute appears to move smoothly \& deterministically from high-concentration areas to low-concentration areas. There is no microscopic \href{https://en.wikipedia.org/wiki/Force}{force} pushing molecules rightward, but there {\it appears} to be one in the bottom panel. This apparent force is called an \href{https://en.wikipedia.org/wiki/Entropic_force}{entropic force}.}

\subsubsection{Diffusion vs. bulk flow}
``Bulk flow'' is the movement{\tt/}flow of an entire body due to a pressure gradient (e.g., water coming out of a tap). ``Diffusion'' is the gradual movement{\tt/}dispersion of concentration within a body with no net movement of matter. An example of a process where both \href{https://en.wikipedia.org/wiki/Mass_flow_(life_sciences)}{bulk motion} \& diffusion occur is human breathing.

1st, there is a ``bulk flow'' process. The \href{https://en.wikipedia.org/wiki/Lungs}{lungs} are located in the \href{https://en.wikipedia.org/wiki/Thoracic_cavity}{thoracic cavity}, which expands as the 1st step in external respiration. This expansion leads to an increase in volume of the \href{https://en.wikipedia.org/wiki/Pulmonary_alveolus}{alveoli} in the lungs, which causes a decrease in pressure in the alveoli. This creates a pressure gradient between the \href{https://en.wikipedia.org/wiki/Air}{air} outside the body at relatively high pressure \& the alveoli\footnote{phế nang.} at relatively low pressure. The air moves down the pressure gradient through the airways of the lungs \& into the alveoli until the pressure of the air \& that in the alveoli are equal, i.e., the movement of air by bulk flow stops once there is no longer a pressure gradient.

2nd, there is a ``diffusion'' process. The air arriving in the alveoli has a higher concentration of oxygen than the ``stale'' air in the alveoli. The increase in oxygen concentration creates a concentration gradient for oxygen between the air in the alveoli \& the blood in the \href{https://en.wikipedia.org/wiki/Capillaries}{capillaries} that surround the alveoli. Oxygen then moves by diffusion, down the concentration gradient, into the blood. The other consequence of the air arriving in alveoli is that the concentration of \href{https://en.wikipedia.org/wiki/Carbon_dioxide}{carbon dioxide} $\rm CO_2$ in the alveoli decreases. This creates a concentration gradient for carbon dioxide to diffuse from the blood into the alveoli, as fresh air has a very low concentration of carbon dioxide compared to the \href{https://en.wikipedia.org/wiki/Blood}{blood} in the body.

3rd, there is another ``bulk flow'' process. The pumping action of the \href{https://en.wikipedia.org/wiki/Heart}{heart} then transports the blood around the body. As the left ventricle of the heart contracts, the volume decreases, which increases the pressure in the ventricle. This creates a pressure gradient between the heart \& the capillaries, \& blood moves through \href{https://en.wikipedia.org/wiki/Blood_vessel}{blood vessels} by bulk flow down the pressure gradient.

\subsubsection{Diffusion in the context of different disciplines}
{\sf Diffusion furnaces used for \href{https://en.wikipedia.org/wiki/Thermal_oxidation}{thermal oxidation}}. ``There are 2 ways to introduce the notion of {\it diffusion}: either a \href{https://en.wiktionary.org/wiki/phenomenon}{phenomenological approach} starting with \href{https://en.wikipedia.org/wiki/Fick%27s_laws_of_diffusion}{Fick's laws of diffusion} \& their mathematical consequences, or a physical \& atomistic one, by considering the {\it\href{https://en.wikipedia.org/wiki/Random_walk}{random walk} of the diffusion particles}.

In the phenomenological approach, {\it diffusion is the movement of a substance from a region of high concentration to a region of low concentration without bulk motion}. According to Fick's laws, the diffusion \href{https://en.wikipedia.org/wiki/Flux#Flux_as_flow_rate_per_unit_area}{flux} is proportional to the negative gradient $-\nabla u$ of concentrations. It goes from regions of higher concentration to regions of lower concentration. Sometime later, various generalizations of Fick's laws were developed in the frame of \href{https://en.wikipedia.org/wiki/Thermodynamics}{thermodynamics} \& \href{https://en.wikipedia.org/wiki/Non-equilibrium_thermodynamics}{non-equilibrium thermodynamics}.

From the {\it atomistic point of view}, diffusion is considered as a result of the random walk of the diffusion particles. In \href{https://en.wikipedia.org/wiki/Molecular_diffusion}{molecular diffusion}, the moving molecules in a gas, liquid, or solid are self-propelled by kinetic energy. Random walk of small particles in suspension in a fluid was discovered in 1827 by \href{https://en.wikipedia.org/wiki/Robert_Brown_(botanist,_born_1773)}{\sc Robert Brown}, who found that minute particle suspended in a liquid medium \& just large enough to be visible under an optical microscope exhibit a rapid \& continually irregular motion of particles known as Brownian movement. The theory of the \href{https://en.wikipedia.org/wiki/Brownian_motion}{Brownian motion} \& the atomistic backgrounds of diffusion were developed by \href{https://en.wikipedia.org/wiki/Albert_Einstein}{\sc Albert Einstein}. The concept of diffusion is typically applied to any subject matter involving random walks in \href{https://en.wikipedia.org/wiki/Statistical_ensemble_(mathematical_physics)}{ensembles} of individuals.

In chemistry \& \href{https://en.wikipedia.org/wiki/Materials_science}{materials science}, diffusion also refers to the movement of fluid molecules in porous solids. Different types of diffusion are distinguished in porous solids. \href{https://en.wikipedia.org/wiki/Molecular_diffusion}{Molecular diffusion} occurs when the collision with another molecule is more likely than the collision with the pore walls. Under such conditions, the diffusivity is similar to that in a non-confined space \& is proportional to the mean free path. \href{https://en.wikipedia.org/wiki/Knudsen_diffusion}{Knudsen diffusion} occurs when the pore diameter is comparable to or smaller than the mean free path of the molecule diffusing through the pore. Under this condition, the collision with the pore walls becomes gradually more likely \& the diffusivity is lower. Finally there is configurational diffusion, which happens if the molecules have comparable size to that of the pore. Under this condition, the diffusivity is much lower compared to molecular diffusion \& small differences in the kinetic diameter of the molecule cause large differences in \href{mass diffusivity}.

\href{https://en.wikipedia.org/wiki/Biologist}{Biologists} often use the terms ``net movement'' or ``net diffusion'' to describe the movement of ions or molecules by diffusion. E.g., oxygen can diffuse through cell membranes so long as there is a higher concentration of oxygen outside the cell. However, because the movement of molecules is random, occasionally oxygen molecules move out of the cell (against the concentration gradient). Because there are more oxygen molecules outside the cell, the probability that oxygen molecules will enter the cell is higher than the probability that oxygen molecules will leave the cell. Therefore, the ``net'' movement of oxygen molecules (the difference between the number of molecules either entering or leaving the cell) is into the cell. I.e., there is a {\it net movement} of oxygen molecules down the concentration gradient.

\subsubsection{History of diffusion in physics}
In the scope of time, diffusion in solids was used long before the theory of diffusion was created. E.g., \href{https://en.wikipedia.org/wiki/Pliny_the_Elder}{\sc Pliny the Elder} had previously described the \href{https://en.wikipedia.org/wiki/Cementation_process}{cementation process}, which produces steel from the element \href{https://en.wikipedia.org/wiki/Iron}{iron} (Fe) through carbon diffusion. Another example is well known for many centuries, the diffusion of colors of \href{https://en.wikipedia.org/wiki/Stained_glass}{stained glass} or \href{https://en.wikipedia.org/wiki/Earthenware}{earthenwave} \& \href{https://en.wikipedia.org/wiki/Chinese_ceramics}{Chinese ceramics}.

In modern science, the 1st systematic experimental study of diffusion was performed by \href{https://en.wikipedia.org/wiki/Thomas_Graham_(chemist)}{\sc Thomas Graham}. He studied diffusion in gases, \& the main phenomenon was described by him in 1831--1833:
\begin{quote}
	``$\ldots$ gases of different nature, when brought into contact, do not arrange themselves according to their density, the heaviest undermost, \& the lighter uppermost, but they spontaneously diffuse, mutually \& equally, through each other, \& so remain in the intimate state of mixture for any length of time.''
\end{quote}
The measurements of Graham contributed to \href{https://en.wikipedia.org/wiki/James_Clerk_Maxwell}{\sc James Clerk Maxwell} deriving, in 1867, the coefficient of diffusion for $\rm CO_2$ in the air. The error rate is $< 5\%$.

In 1855, \href{https://en.wikipedia.org/wiki/Adolf_Fick}{\sc Adolf Fick}, the 26-year-old anatomy demonstrator from Z\"urich, proposed \href{https://en.wikipedia.org/wiki/Fick%27s_laws_of_diffusion}{Fick's law of diffusion}. He used Graham's research, stating his goal as ``the development of a fundamental law, for the operation of diffusion in a single element of space''. He asserted a deep analogy between diffusion \& conduction of heat or electricity, creating a formalism similar to \href{https://en.wikipedia.org/wiki/Thermal_conduction}{Fourier's law for heat conduction} (1822) \& \href{https://en.wikipedia.org/wiki/Ohm%27s_law}{Ohm's law} for electric current (1827).

\href{https://en.wikipedia.org/wiki/Robert_Boyle}{\sc Robert Boyle} demonstrated diffusion in solids in the 17th century by penetration of zinc Zn into a copper Cu coin. Nevertheless, diffusion in solids was not systematically studied until the 2nd part of the 19th century. \href{https://en.wikipedia.org/wiki/William_Chandler_Roberts-Austen}{\sc William Chandler Roberts-Austen}, the well-known British metallurgist \& former assistant of {\sc Thomas Graham} studied systematically solid state diffusion on the example of gold in lead in 1896:
\begin{quote}
	``$\ldots$ My long connection with Graham's researches made it almost a duty to attempt to extend his work on liquid diffusion to metals.''
\end{quote}
In 1858, \href{https://en.wikipedia.org/wiki/Rudolf_Clausius}{Rudolf Clausius} introduced the concept of the \href{https://en.wikipedia.org/wiki/Mean_free_path}{mean free path}. In the same year, \href{https://en.wikipedia.org/wiki/James_Clerk_Maxwell}{\sc James Clerk Maxwell} developed the 1st atomistic theory of transport processes in gases. The modern atomistic theory of diffusion \& Brownian motion was developed by \href{https://en.wikipedia.org/wiki/Albert_Einstein}{\sc Albert Einstein}, \href{https://en.wikipedia.org/wiki/Marian_Smoluchowski}{Marian Smoluchowski}, \& \href{https://en.wikipedia.org/wiki/Jean-Baptiste_Perrin}{Jean-Baptise Perrin}. \href{https://en.wikipedia.org/wiki/Ludwig_Boltzmann}{\sc Ludwig Boltzmann}, in the development of the atomistic backgrounds of the macroscopic \href{https://en.wikipedia.org/wiki/Transport_phenomena}{transport processes}, introduced the \href{https://en.wikipedia.org/wiki/Boltzmann_equation}{Boltzmann equation}, which has served mathematics \& physics with a source of transport process ideas \& concerns for $> 140$ years.

In 1920--1921, \href{https://en.wikipedia.org/wiki/George_de_Hevesy}{\sc George de Hevesy} measured \href{https://en.wikipedia.org/wiki/Self-diffusion}{self-diffusion} using \href{https://en.wikipedia.org/wiki/Radioisotope}{raioisotopes}. He studied self-diffusion of radioactive isotopes of lead in the liquid \& solid lead.

\href{https://en.wikipedia.org/wiki/Yakov_Frenkel}{\sc Yakov Frenkel} (sometimes, Jakov{\tt/}Jacob Frenkel) proposed, \& elaborated in 1926, the idea of diffusion in crystals through local defects (vacancies \& \href{https://en.wikipedia.org/wiki/Interstitial_defect}{interstitial} atoms). He concluded, the diffusion process in condensed matter is an ensemble of elementary jumps \& quasichemical interactions of particles \& defects. He introduced several mechanisms of diffusion \& found rate constants from experimental data.

Sometime later, \href{https://en.wikipedia.org/wiki/Carl_Wagner}{\sc Carl Wagner} \& \href{https://en.wikipedia.org/wiki/Walter_H._Schottky}{\sc Walter H. Schottky} developed Frenkel's ideas about mechanisms of diffusion further. Presently, it is universally recognized that atomic defects are necessary to mediate diffusion in crystals.

\href{https://en.wikipedia.org/wiki/Henry_Eyring_(chemist)}{\sc Henry Eyring}, with co-authors, applied his theory of \href{https://en.wikipedia.org/wiki/Transition_state_theory}{absolute reaction rates} to Frenkel's quasichemical model of diffusion. The analogy between \href{https://en.wikipedia.org/wiki/Chemical_kinetics}{reaction kinetics} \& diffusion leads to various nonlinear versions of Fick's law.

\subsubsection{Basic models of diffusion}

\subsubsection{Diffusion in physics}

\subsubsection{Random walk (random motion)}

'' -- \href{https://en.wikipedia.org/wiki/Diffusion}{Wikipedia{\tt/}diffusion}

%------------------------------------------------------------------------------%

\subsection{Wikipedia{\tt/}dynamics (mechanics)}
``\textit{Dynamics} is the \href{https://en.wikipedia.org/wiki/Branch_(academia)#Physics}{branch} of \href{https://en.wikipedia.org/wiki/Classical_mechanics}{classical mechanics} that is concerned with the study of \href{https://en.wikipedia.org/wiki/Force_(physics)}{force} \& their effects on \href{https://en.wikipedia.org/wiki/Motion_(physics)}{motion}. \href{https://en.wikipedia.org/wiki/Isaac_Newton}{Isaac Newton} was the 1st to formulate the fundamental \href{https://en.wikipedia.org/wiki/Physical_law}{physical laws} that govern dynamics in classical non-relativistic physics, especially his \href{https://en.wikipedia.org/wiki/Second_law_of_motion}{2nd law of motion}.'' -- \href{https://en.wikipedia.org/wiki/Dynamics_(mechanics)}{Wikipedia{\tt/}dynammics (mechanics)}

\subsubsection{Principles}
``Generally speaking, researchers involved in dynamics study how a physical system might develop or alter over time \& study the causes of those changes. In addition, Newton established the fundamental physical laws which govern dynamics in physics. By studying his system of mechanics, dynamics can be understood. In particular, dynamics is mostly related to Newton's 2nd law of motion. However, all 3 laws of motion are taken into account because these are interrelated in any given observation or experiment.'' -- \href{https://en.wikipedia.org/wiki/Dynamics_(mechanics)#Principles}{Wikipedia{\tt/}dynamics (mechanics){\tt/}principles}

\subsubsection{Linear \& rotational dynamics}
``The study of dynamics falls under 2 categories: linear \& rotational. Linear dynamics pertains to objects moving in a line \& involves such quantities as \href{https://en.wikipedia.org/wiki/Force}{force}, \href{https://en.wikipedia.org/wiki/Mass}{mass}{\tt/}\href{https://en.wikipedia.org/wiki/Inertia#Mass_and_inertia}{inertia}, \href{https://en.wikipedia.org/wiki/Displacement_(vector)}{displacement} (in units of distance), \href{https://en.wikipedia.org/wiki/Velocity}{velocity} (distance per unit time), \href{https://en.wikipedia.org/wiki/Acceleration}{acceleration} (distance per unit of time squared) \& \href{https://en.wikipedia.org/wiki/Momentum}{momentum} (mass times unit of velocity). Rotational dynamics pertains to obejcts that are rotating or moving in a curved path \& involves such quantities as \href{https://en.wikipedia.org/wiki/Torque}{torque}, \href{https://en.wikipedia.org/wiki/Moment_of_inertia}{moment of inertia}{\tt/}\href{https://en.wikipedia.org/wiki/Rotational_inertia}{rotational inertia}, \href{https://en.wikipedia.org/wiki/Angular_displacement}{angular displacement} (in radians or less often, degrees), \href{https://en.wikipedia.org/wiki/Angular_velocity}{angular velocity} (radians per unit time), \href{https://en.wikipedia.org/wiki/Angular_acceleration}{angular acceleration} (radians per unit of time squared) \& \href{https://en.wikipedia.org/wiki/Angular_momentum}{angular momentum} (moment of inertia times unit of angular velocity). Very often, objects exhibit linear \& rotational motion.

For classical \href{https://en.wikipedia.org/wiki/Electromagnetism}{electromagnetism}, \href{https://en.wikipedia.org/wiki/Maxwell%27s_equations}{Maxwell's equations} describe the kinematics. The dynamics of classical systems involving both mechanics \& electromagnetism are described by the combination of Newton's laws, Maxwell's equations, \& the \href{https://en.wikipedia.org/wiki/Lorentz_force}{Lorentz force}.'' -- \href{https://en.wikipedia.org/wiki/Dynamics_(mechanics)#Linear_and_rotational_dynamics}{Wikipedia{\tt/}dynamics (mechanics){\tt/}linear \& rotational dynamics}

\subsubsection{Force}
``Main article: \href{https://en.wikipedia.org/wiki/Force}{Wikipedia{\tt/}force}. From Newton, force can be defined as an exertion or \href{https://en.wikipedia.org/wiki/Pressure}{pressure} which can cause an object to \href{https://en.wikipedia.org/wiki/Accelerate}{accelerate}. The concept of force is used to describe an influence which causes a \href{https://en.wikipedia.org/wiki/Free_body}{free body} (object) to accelerate. It can be a push or a pull, which causes an object to change direction, have new \href{https://en.wikipedia.org/wiki/Velocity}{velocity}, or to \href{https://en.wikipedia.org/wiki/Deformation_(mechanics)}{deform} temporarily or permanently. Generally speaking, force causes an object's \href{https://en.wikipedia.org/wiki/Motion_(physics)}{state of motion} to change.'' -- \href{https://en.wikipedia.org/wiki/Dynamics_(mechanics)#Force}{Wikipedia{\tt/}dynamics (mechanics){\tt/}force}

\subsubsection{Newton's laws}
``Main article: \href{https://en.wikipedia.org/wiki/Newton%27s_laws_of_motion}{Wikipedia{\tt/}Newton's laws of motion}. Newton described force as the ability to cause a mass to accelerate. His 3 laws can be summarized as follows:
\begin{enumerate}
	\item \textit{1st law}: if there is no net force on an object, then its velocity is constant: either the object is at rest (if its velocity is equal to zero), or it moves with constant speed in a single direction.
	\item \textit{2nd law}: The rate of change of linear momentum ${\bf P}$ of an object is equal to the net force ${\bf F}_{\rm net}$, i.e., $\frac{{\rm d}{\bf P}}{{\rm d}t} = {\bf F}_{\rm net}$.
	\item \textit{3rd law}: When a 1st body exerts a force ${\bf F}_1$ on a 2nd body, the 2nd body simultaneously exerts a force ${\bf F}_2 = -{\bf F}_1$ on the 1st body. I.e., ${\bf F}_1$ \& ${\bf F}_2$ are equal in magnitude \& opposite in direction.
\end{enumerate}
Newton's laws of motion are valid only in an \href{https://en.wikipedia.org/wiki/Inertial_frame_of_reference}{inertial frame of reference}.'' -- \href{https://en.wikipedia.org/wiki/Dynamics_(mechanics)}{Wikipedia{\tt/}dynamics (mechanics)}

%------------------------------------------------------------------------------%

\subsection{Wikipedia{\tt/}fluid mechanics}
\textit{Fluid mechanics} is the branch of \href{https://en.wikipedia.org/wiki/Physics}{physics} concerned with the mechanics of \href{https://en.wikipedia.org/wiki/Fluid}{fluids} (\href{https://en.wikipedia.org/wiki/Liquid}{liquids}, \href{https://en.wikipedia.org/wiki/Gas}{gases}, and \href{https://en.wikipedia.org/wiki/Plasma_(physics)}{plasmas}) and the \href{https://en.wikipedia.org/wiki/Force}{forces} on them.[White, Frank M. (2011). Fluid Mechanics (7th ed.). McGraw-Hill. ISBN 978-0-07-352934-9.]:3

It has applications in a wide range of disciplines, including \href{https://en.wikipedia.org/wiki/Mechanical_engineering}{mechanical}, \href{https://en.wikipedia.org/wiki/Civil_engineering}{civil}, \href{https://en.wikipedia.org/wiki/Chemical_engineering}{chemical} and \href{https://en.wikipedia.org/wiki/Biomedical_engineering}{biomedical engineering}, \href{https://en.wikipedia.org/wiki/Geophysics}{geophysics}, \href{https://en.wikipedia.org/wiki/Oceanography}{oceanography}, \href{https://en.wikipedia.org/wiki/Meteorology}{meteorology}, \href{https://en.wikipedia.org/wiki/Astrophysics}{astrophysics}, and \href{https://en.wikipedia.org/wiki/Biology}{biology}.

%
It can be divided into \href{https://en.wikipedia.org/wiki/Fluid_statics}{fluid statics}, the study of fluids at rest; and \href{https://en.wikipedia.org/wiki/Fluid_dynamics}{fluid dynamics}, the study of the effect of forces on fluid motion.[White, Frank M. (2011). Fluid Mechanics (7th ed.). McGraw-Hill. ISBN 978-0-07-352934-9.]:3

It is a branch of \href{https://en.wikipedia.org/wiki/Continuum_mechanics}{continuum mechanics}, a subject which models matter without using the information that it is made out of atoms; i.e., it models matter from a \textit{macroscopic} viewpoint rather than from \textit{microscopic}.

Fluid mechanics, especially fluid dynamics, is an active field of research, typically mathematically complex.

Many problems are partly or wholly unsolved and are best addressed by \href{https://en.wikipedia.org/wiki/Numerical_methods}{numerical methods}, typically using computers.

A modern discipline, called \href{https://en.wikipedia.org/wiki/Computational_fluid_dynamics}{computational fluid dynamics} (CFD), is devoted to this approach.[2]

\href{https://en.wikipedia.org/wiki/Particle_image_velocimetry}{Particle image velocimetry}, an experimental method for visualizing and analyzing fluid flow, also takes advantage of the highly visual nature of fluid flow.

\subsubsection{Brief history}
Main article: \href{https://en.wikipedia.org/wiki/History_of_fluid_mechanics}{History of fluid mechanics}.

%
The study of fluid mechanics goes back at least to the days of \href{https://en.wikipedia.org/wiki/Ancient_Greece}{ancient Greece}, when \href{https://en.wikipedia.org/wiki/Archimedes}{Archimedes} investigated fluid statics and \href{https://en.wikipedia.org/wiki/Buoyancy}{buoyancy} and formulated his famous law known now as the \href{https://en.wikipedia.org/wiki/Archimedes'_principle}{Archimedes' principle}, which was published in his work \href{https://en.wikipedia.org/wiki/On_Floating_Bodies}{On Floating Bodies} - generally considered to be the 1st major work on fluid mechanics.

Rapid advancement in fluid mechanics began with \href{https://en.wikipedia.org/wiki/Leonardo_da_Vinci}{Leonardo da Vinci} (observations and experiments), \href{https://en.wikipedia.org/wiki/Evangelista_Torricelli}{Evangelista Torricelli} (invented the \href{https://en.wikipedia.org/wiki/Barometer}{barometer}), \href{https://en.wikipedia.org/wiki/Isaac_Newton}{Isaac Newton} (investigated \href{https://en.wikipedia.org/wiki/Viscosity}{viscosity}) and \href{https://en.wikipedia.org/wiki/Blaise_Pascal}{Blaise Pascal} (researched \href{https://en.wikipedia.org/wiki/Hydrostatics}{hydrostatics}, formulated \href{https://en.wikipedia.org/wiki/Pascal's_law}{Pascal's law}), and was continued by \href{https://en.wikipedia.org/wiki/Daniel_Bernoulli}{Daniel Bernoulli} with the introduction of mathematical fluid dynamics in \textit{Hydrodynamica} (1739).

%
Inviscid flow was further analyzed by various mathematicians (\href{https://en.wikipedia.org/wiki/Jean_le_Rond_d'Alembert}{Jean le Rond d'Alembert}, \href{https://en.wikipedia.org/wiki/Joseph_Louis_Lagrange}{Joseph Louis Lagrange}, \href{https://en.wikipedia.org/wiki/Pierre-Simon_Laplace}{Pierre-Simon Laplace}, \href{https://en.wikipedia.org/wiki/Sim%C3%A9on_Denis_Poisson}{Siméon Denis Poisson}) and viscous flow was explored by a multitude of \href{https://en.wikipedia.org/wiki/Engineers}{engineers} including \href{https://en.wikipedia.org/wiki/Jean_L%C3%A9onard_Marie_Poiseuille}{Jean Léonard Marie Poiseuille} and \href{https://en.wikipedia.org/wiki/Gotthilf_Hagen}{Gotthilf Hagen}.

Further mathematical justification was provided by \href{https://en.wikipedia.org/wiki/Claude-Louis_Navier}{Claude-Louis Navier} and \href{https://en.wikipedia.org/wiki/George_Gabriel_Stokes}{George Gabriel Stokes} in the \href{https://en.wikipedia.org/wiki/Navier-Stokes_equations}{Navier-Stokes equations}, and \href{https://en.wikipedia.org/wiki/Boundary_layers}{boundary layers} were investigated (\href{https://en.wikipedia.org/wiki/Ludwig_Prandtl}{Ludwig Prandtl}, \href{https://en.wikipedia.org/wiki/Theodore_von_K%C3%A1rm%C3%A1n}{Theodore von Kármán}), while various scientists such as \href{https://en.wikipedia.org/wiki/Osborne_Reynolds}{Osborne Reynolds}, \href{https://en.wikipedia.org/wiki/Andrey_Kolmogorov}{Andrey Kolmogorov}, and \href{https://en.wikipedia.org/wiki/Geoffrey_Ingram_Taylor}{Geoffrey Ingram Taylor} advanced the understanding of fluid viscosity and \href{https://en.wikipedia.org/wiki/Turbulence}{turbulence}.

\subsubsection{Main branches}

\begin{enumerate}
	\item {\bf Fluid statics.} Main article: \href{https://en.wikipedia.org/wiki/Fluid_statics}{Fluid statics}.
	
	%
	\href{https://en.wikipedia.org/wiki/Fluid_statics}{Fluid statics} or \textit{hydrostatics} is the branch of fluid mechanics that studies \href{https://en.wikipedia.org/wiki/Fluid}{fluids} at rest.
	
	It embraces the study of the conditions under which fluids are at rest in stable equilibrium; and is contrasted with fluid dynamics, the study of fluids in motion.
	
	Hydrostatics offers physical explanations for many phenomena of everyday life, such as why atmospheric pressure changes with altitude, why wood and oil float on water, and why the surface of water is always level whatever the shape of its container.
	
	Hydrostatics is fundamental to hydraulics, the engineering of equipment for storing, transporting and using fluids.
	
	It is also relevant to some aspects of \href{https://en.wikipedia.org/wiki/Geophysics}{geophysics} and \href{https://en.wikipedia.org/wiki/Astrophysics}{astrophysics} (e.g., in understanding \href{https://en.wikipedia.org/wiki/Plate_tectonics}{plate tectonics} and anomalies in the \href{https://en.wikipedia.org/wiki/Gravity_of_Earth}{Earth's gravitational field}), to \href{https://en.wikipedia.org/wiki/Meteorology}{meteorology}, to \href{https://en.wikipedia.org/wiki/Medicine}{medicine} (in the context of \href{https://en.wikipedia.org/wiki/Blood_pressure}{blood pressure}), and many other fields.
	\item {\bf Fluid dynamics.} Main article: \href{https://en.wikipedia.org/wiki/Fluid_dynamics}{Fluid dynamics}.
	
	%
	\href{https://en.wikipedia.org/wiki/Fluid_dynamics}{Fluid dynamics} is a subdiscipline of fluid mechanics that deals with \textit{fluid flow} - the science of liquids and gases in motion.[3]
	
	Fluid dynamics offers a systematic structure - which underlies these \href{https://en.wikipedia.org/wiki/Practical_disciplines}{practical disciplines} - that embraces empirical and semi-empirical laws derived from \href{https://en.wikipedia.org/wiki/Flow_measurement}{flow measurement} and used to solve practical problems.
	
	The solution to a fluid dynamics problem typically involves calculating various properties of the fluid, such as \href{https://en.wikipedia.org/wiki/Velocity}{velocity}, \href{https://en.wikipedia.org/wiki/Pressure}{pressure}, \href{https://en.wikipedia.org/wiki/Density}{density}, and \href{https://en.wikipedia.org/wiki/Temperature}{temperature}, as functions of space and time.
	
	It has several subdisciplines itself, including \href{https://en.wikipedia.org/wiki/Aerodynamics}{aerodynamics}[4][5][6][7] (the study of air and other gases in motion) and \textit{hydrodynamics}[8][9] (the study of liquids in motion).
	
	Fluid dynamics has a wide range of applications, including calculating \href{https://en.wikipedia.org/wiki/Force}{forces} and \href{https://en.wikipedia.org/wiki/Moment_(physics)}{movements} on \href{https://en.wikipedia.org/wiki/Aircraft}{aircraft}, determining the \href{https://en.wikipedia.org/wiki/Mass_flow_rate}{mass flow rate} of \href{https://en.wikipedia.org/wiki/Petroleum}{petroleum} through pipelines, predicting evolving \href{https://en.wikipedia.org/wiki/Weather}{weather} patterns, understanding \href{https://en.wikipedia.org/wiki/Nebula}{nebulae} in \href{https://en.wikipedia.org/wiki/Interstellar_space}{interstellar space} and modeling \href{https://en.wikipedia.org/wiki/Explosions}{explosions}.
	
	Some fluid-dynamical principles are used in \href{https://en.wikipedia.org/wiki/Traffic_engineering_(transportation)}{traffic engineering} and crowd dynamics.
\end{enumerate}

\subsubsection{Relationship to continuum mechanics}
Fluid mechanics is a subdiscipline of \href{https://en.wikipedia.org/wiki/Continuum_mechanics}{continuum mechanics}, as illustrated in the following table.
\begin{itemize}
	\item \textbf{\href{https://en.wikipedia.org/wiki/Continuum_mechanics}{Continuum mechanics}.} The study of the physics of continuous materials
	\begin{itemize}
		\item \textbf{\href{https://en.wikipedia.org/wiki/Solid_mechanics}{Solid mechanics}.} The study of the physics of continuous materials with a defined rest shape.
		\begin{itemize}
			\item \textbf{\href{https://en.wikipedia.org/wiki/Elasticity_(physics)}{Elasticity}.} Describes materials that return to their rest shape after applied \href{https://en.wikipedia.org/wiki/Stress_(physics)}{stresses} are removed.
			\item \textbf{\href{https://en.wikipedia.org/wiki/Plasticity_(physics)}.} Describes materials that permanently deform after a sufficient applied stress.
			\begin{itemize}
				\item \textbf{\href{https://en.wikipedia.org/wiki/Rheology}.} The study of materials with both solid and fluid characteristics.
			\end{itemize}
		\end{itemize}
		\item \textbf{Fluid mechanics.} The study of the physics of continuous materials which deform when subjected to a force.
		\begin{itemize}
			\item \href{https://en.wikipedia.org/wiki/Non-Newtonian_fluid}{Non-Newtonian fluids} do not undergo strain rates proportional to the applied shear stress.
			\begin{itemize}
				\item \textbf{\href{https://en.wikipedia.org/wiki/Rheology}.} The study of materials with both solid and fluid characteristics.
			\end{itemize}
			\item \href{https://en.wikipedia.org/wiki/Newtonian_fluid}{Newtonian fluids} undergo strain rates proportional to the applied shear stress.
		\end{itemize}
	\end{itemize}
\end{itemize}
In a mechanical view, a fluid is a substance that does not support \href{https://en.wikipedia.org/wiki/Shear_stress}{shear stress}; that is why a fluid at rest has the shape of its containing vessel.

A fluid at rest has no shear stress.

\subsubsection{Assumptions}
\textsf{Balance for some integrated fluid quantity in a \href{https://en.wikipedia.org/wiki/Control_volume}{control volume} enclosed by a \href{https://en.wikipedia.org/wiki/Control_surface_(fluid_dynamics)}{control surface}.}

%
The assumptions inherent to a fluid mechanical treatment of a physical system can be expressed in terms of mathematical equations.

Fundamentally, every fluid mechanical system is assumed to obey:
\begin{itemize}
	\item \href{https://en.wikipedia.org/wiki/Conservation_of_mass}{Conservation of mass}
	\item \href{https://en.wikipedia.org/wiki/Conservation_of_energy}{Conservation of energy}
	\item \href{https://en.wikipedia.org/wiki/Conservation_of_momentum}{Conservation of momentum}
	\item The continuum assumption.
\end{itemize}
E.g., the assumption that mass is conserved means that for any fixed \href{https://en.wikipedia.org/wiki/Control_volume}{control volume} (e.g., a spherical volume) - enclosed by a \href{https://en.wikipedia.org/wiki/Control_surface_(fluid_dynamics)}{control surface} - the \href{https://en.wikipedia.org/wiki/Derivative}{rate of change} of the mass contained in that volume is equal to the rate at which mass is passing through the surface from \textit{outside} to \textit{inside}, minus the rate at which mass is passing from \textit{inside} to \textit{outside}.

This can be expressed as an \href{https://en.wikipedia.org/wiki/Continuity_equation#Integral_form}{equation in integral form} over the control volume.[Batchelor, George K. (1967). \textit{An Introduction to Fluid Dynamics}. Cambridge University Press. p. 74. ISBN 0-521-66396-2.]:74

%
The \textit{continuum assumption} is an idealization of \href{https://en.wikipedia.org/wiki/Continuum_mechanics}{continuum mechanics} under which fluids can be treated as \href{https://en.wikipedia.org/wiki/Continuous_function}{continuous}, even though, on a microscopic scale, they are composed of \href{https://en.wikipedia.org/wiki/Molecules}{molecules}.

Under the continuum assumption, macroscopic (observed/measurable) properties such as density, pressure, temperature, and bulk velocity are taken to be well-defined at ``infinitesimal'' volume elements - small in comparison to the characteristic length scale of the system, but large in comparison to molecular length scale.

Fluid properties can vary continuously from 1 volume element to another and are average values of the molecular properties.

The continuum hypothesis can lead to inaccurate results in applications like supersonic speed flows, or molecular flows on nano scale.[11]

Those problems for which the continuum hypothesis fails can be solved using \href{https://en.wikipedia.org/wiki/Statistical_mechanics}{statistical mechanics}.

To determine whether or not the continuum hypothesis applies, the \href{https://en.wikipedia.org/wiki/Knudsen_number}{Knudsen number}, defined as the ratio of the molecular \href{https://en.wikipedia.org/wiki/Mean_free_path}{mean free path} to the characteristic length \href{https://en.wikipedia.org/wiki/Scale_(ratio)}{scale}, is evaluated.

Problems with Knudsen numbers below 0.1 can be evaluated using the continuum hypothesis, but molecular approach (statistical mechanics) can be applied to find the fluid motion for larger Knudsen numbers.

\subsubsection{NSEs}
Main article: \href{https://en.wikipedia.org/wiki/Navier-Stokes_equations}{Navier-Stokes equations}.

%
The \textit{Navier-Stokes equations} (named after \href{https://en.wikipedia.org/wiki/Claude-Louis_Navier}{Claude-Louis Navier} and \href{https://en.wikipedia.org/wiki/George_Gabriel_Stokes}{George Gabriel Stokes}) are \href{https://en.wikipedia.org/wiki/Differential_equations}{differential equations} that describe the force balance at a given point within a fluid.

For an \href{https://en.wikipedia.org/wiki/Incompressible_fluid}{incompressible fluid} with vector velocity field ${\bf u}$, the Navier-Stokes equations are[12][13][14][15]
\begin{align*}
	\partial_t{\bf u} + ({\bf u}\cdot\nabla){\bf u} = -\frac{1}{\rho}\nabla P + \nu\Delta{\bf u}.
\end{align*}
These differential equations are the analogues for deformable materials to Newton's equations of motion for particles - the Navier-Stokes equations describe changes in \href{https://en.wikipedia.org/wiki/Momentum}{momentum} (\href{https://en.wikipedia.org/wiki/Force}{force}) in response to \href{https://en.wikipedia.org/wiki/Pressure}{pressure} $P$ and viscosity, parameterized by the \href{https://en.wikipedia.org/wiki/Kinematic_viscosity}{kinematic viscosity} $\nu$ here.

Occasionally, \href{https://en.wikipedia.org/wiki/Body_force}{body forces}, such as the gravitational force or Lorentz force are added to the equations.

%
Solutions of the Navier-Stokes equations for a given physical problem must be sought with the help of \href{https://en.wikipedia.org/wiki/Calculus}{calculus}.

In practical terms, only the simplest cases can be solved exactly in this way.

These cases generally involve non-turbulent, steady flow in which the \href{https://en.wikipedia.org/wiki/Reynolds_number}{Reynolds number} is small.

For more complex cases, especially those involving \href{https://en.wikipedia.org/wiki/Turbulence}{turbulence}, such as global weather systems, aerodynamics, hydrodynamics and many more, solutions of the Navier-Stokes equations can currently only be found with the help of computers.

This branch of science is called \href{https://en.wikipedia.org/wiki/Computational_fluid_dynamics}{computational fluid dynamics}.[16][17][18][19][20]

\subsubsection{Inviscid and viscous fluids}
An \textit{inviscid fluid} has no \href{https://en.wikipedia.org/wiki/Viscosity}{viscosity}, $\nu = 0$.

In practice, an inviscid flow is an \href{https://en.wikipedia.org/wiki/Ideal_fluid}{idealization}, one that facilitates mathematical treatment.

In fact, purely inviscid flows are only known to be realized in the case of \href{https://en.wikipedia.org/wiki/Superfluidity}{superfluidity}.

Otherwise, fluids are generally \textit{viscous}, a property that is often most important within a \href{https://en.wikipedia.org/wiki/Boundary_layer}{boundary layer} near a solid surface,[21] where the flow must match onto the \href{https://en.wikipedia.org/wiki/No-slip_condition}{no-slip condition} at the solid.

In some cases, the mathematics of a fluid mechanical system can be treated by assuming that the fluid outside of boundary layers is inviscid, and then \href{https://en.wikipedia.org/wiki/Method_of_matched_asymptotic_expansions}{matching} its solution onto that for a thin \href{https://en.wikipedia.org/wiki/Laminar_flow}{laminar} boundary layer.

%
For fluid flow over a porous boundary, the fluid velocity can be discontinuous between the free fluid and the fluid in the porous media (this is related to the Beavers and Joseph condition).

Further, it is useful at low \href{https://en.wikipedia.org/wiki/Speed_of_sound}{subsonic} speeds to assume that gas is \href{https://en.wikipedia.org/wiki/Incompressible_fluid}{incompressible} - i.e., the density of the gas does not change even though the speed and \href{https://en.wikipedia.org/wiki/Static_pressure}{static pressure} change.

\subsubsection{Newtonian vs. non-Newtonian fluids}
A \textit{Newtonian fluid} (named after \href{https://en.wikipedia.org/wiki/Isaac_Newton}{Isaac Newton}) is defined to be a \href{https://en.wikipedia.org/wiki/Fluid}{fluid} whose \href{https://en.wikipedia.org/wiki/Shear_stress}{shear stress} is linearly proportional to the velocity gradient in the direction \href{https://en.wikipedia.org/wiki/Perpendicular}{perpendicular} to the plane of shear.

This definition means regardless of the forces acting on a fluid, it \textit{continues to flow}.

E.g., water is a Newtonian fluid, because it continues to display fluid properties no matter how much it is stirred or mixed.

A slightly less rigorous definition is that the \href{https://en.wikipedia.org/wiki/Drag_(physics)}{drag} of a small object being moved slowly through the fluid is proportional to the force applied to the object.

(Compare \href{https://en.wikipedia.org/wiki/Friction}{friction}).

Important fluids, like water as well as most gases, behave - to good approximation - as a Newtonian fluid under normal conditions on Earth.[Batchelor, George K. (1967). \textit{An Introduction to Fluid Dynamics}. Cambridge University Press. p. 74. ISBN 0-521-66396-2.]:145

%
By contrast, stirring a \href{https://en.wikipedia.org/wiki/Non-Newtonian_fluid}{non-Newtonian fluid} can leave a ``hole'' behind.

This will gradually fill up over time - this behavior is seen in materials such as pudding, \href{https://en.wikipedia.org/wiki/Non-newtonian_fluid#Oobleck}{oobleck}, or \href{https://en.wikipedia.org/wiki/Sand}{sand} (although sand isn't strictly a fluid).

Alternatively, stirring a non-Newtonian fluid can cause the viscosity to decrease, so the fluid appears ``thinner'' (this is seen in non-drip \href{https://en.wikipedia.org/wiki/Paint}{paints}).

There are many types of non-Newtonian fluids, as they are defined to be something that fails to obey a particular property - e.g., most fluids with long molecular chains can react in a non-Newtonian manner.[Batchelor, George K. (1967). \textit{An Introduction to Fluid Dynamics}. Cambridge University Press. p. 74. ISBN 0-521-66396-2.]:145

\paragraph{Equations for a Newtonian fluid.}
Main article: \href{https://en.wikipedia.org/wiki/Newtonian_fluid}{Newtonian fluid}.

%
The constant of proportionality between the viscous stress tensor and the velocity gradient is known as the \href{https://en.wikipedia.org/wiki/Viscosity}{viscosity}.

A simple equation to describe incompressible Newtonian fluid behavior is
\begin{align*}
	\tau = -\mu\frac{dv}{dy},
\end{align*}
where
\begin{itemize}
	\item $\tau$ is the shear stress exerted by the fluid (``\href{https://en.wikipedia.org/wiki/Drag_(physics)}{drag}'')
	\item $\mu$ is the fluid viscosity - a constant of proportionality
	\item $\frac{dv}{dy}$ is the velocity gradient perpendicular to the direction of shear.
\end{itemize}
For a Newtonian fluid, the viscosity, by definition, depends only on \href{https://en.wikipedia.org/wiki/Temperature}{temperature} and \href{https://en.wikipedia.org/wiki/Pressure}{pressure}, not on the forces acting upon it.

If the fluid is \href{https://en.wikipedia.org/wiki/Incompressible_fluid}{incompressible} the equation governing the viscous stress (in \href{https://en.wikipedia.org/wiki/Cartesian_coordinate_system}{Cartesian coordinates}) is
\begin{align*}
	\tau_{ij} = \mu\left(\partial_{x_j}v_i + \partial_{x_i}v_j\right)
\end{align*}
where
\begin{itemize}
	\item $\tau_{ij}$ is the shear stress on the $i$th face of a fluid element in the $j$th direction
	\item $v_i$ is the velocity in the $i$th direction
	\item $x_j$ is the $j$th direction coordinate.
\end{itemize}
If the fluid is not incompressible the general form for the viscous stress in a Newtonian fluid is
\begin{align*}
	\tau_{ij} = \mu\left(\partial_{x_j}v_i + \partial_{x_i}v_j - \frac{2}{3}\delta_{ij}\nabla\cdot{\bf v}\right) + \kappa\delta_{ij}\nabla\cdot{\bf v}
\end{align*}
where $\kappa$ is the 2nd viscosity coefficient (or bulk viscosity).

If a fluid does not obey this relation, it is termed a \href{https://en.wikipedia.org/wiki/Non-Newtonian_fluid}{non-Newtonian fluid}, of which there are several types.

Non-Newtonian fluids can be either plastic, Bingham plastic, pseudoplastic, dilatant, thixotropic, rheopectic, viscoelastic.

%
In some applications, another rough broad division among fluids is made: ideal and non-ideal fluids.

An \textit{ideal fluid} is non-viscous and offers no resistance whatsoever to a shearing force.

An ideal fluid really does not exist, but in some calculations, the assumption is justifiable.

1 example of this is the flow far from solid surfaces.

In many cases, the viscous effects are concentrated near the solid boundaries (such as in boundary layers) while in regions of the flow field far away from the boundaries the viscous effects can be neglected and the fluid there is treated as it were inviscid (ideal flow).

When the viscosity is neglected, the term containing the viscous stress tensor $\tau$ in the Navier-Stokes equation vanishes.

The equation reduced in this form is called the \href{https://en.wikipedia.org/wiki/Euler_equations_(fluid_dynamics)}{Euler equation}.

%------------------------------------------------------------------------------%

\subsection{Wikipedia{\tt/}external flow}
In \href{https://en.wikipedia.org/wiki/Fluid_mechanics}{fluid mechanics}, \textit{external flow} is such a flow that \href{https://en.wikipedia.org/wiki/Boundary_layer}{boundary layers} develop freely, without constraints imposed by adjacent surfaces.

Accordingly, there will always exist a region of the flow outside the boundary layer in which velocity, temperature, and/or \href{https://en.wikipedia.org/wiki/Concentration_gradient}{concentration gradients} are negligible.

It can be defined as the flow of a fluid around a body that is completely submerged in it.

%
An example includes fluid motion over a flat plate (inclined or parallel to the free stream velocity) and flow over curved surfaces such as a sphere, cylinder, \href{https://en.wikipedia.org/wiki/Airfoil}{airfoil}, or \href{https://en.wikipedia.org/wiki/Turbine_blade}{turbine blade}, air flowing around an airplane and water flowing around the submarines.'' -- \href{https://en.wikipedia.org/wiki/External_flow}{Wikipedia{\tt/}external flow}

%------------------------------------------------------------------------------%

\subsection{Wikipedia{\tt/}Gauss's principle of least constraint}
``The \textit{principle of least constraint} is 1 \href{https://en.wikipedia.org/wiki/Variational_principle}{variational formulation} of \href{https://en.wikipedia.org/wiki/Classical_mechanics}{classical mechanics} enunciated by \href{https://en.wikipedia.org/wiki/Carl_Friedrich_Gauss}{Carl Friedrich Gauss} in 1829, equivalent to all other formulations of analytical mechanics. Intuitively, it says that the acceleration of a \href{https://en.wikipedia.org/wiki/Constraint_(classical_mechanics)}{constrained} \href{https://en.wikipedia.org/wiki/Physical_system}{physical system} will be as similar as possible to that of the corresponding unconstrained system.'' -- \href{https://en.wikipedia.org/wiki/Gauss%27s_principle_of_least_constraint}{Wikipedia{\tt/}Gauss's principle of least constraint}

\subsubsection{Statement}
``The principle of least constraint is a \href{https://en.wikipedia.org/wiki/Least_squares}{least squares} principle stating that the true accelerations of a mechanical system of $n$ masses is the minimum of the quantity
\begin{align*}
	Z\coloneqq\sum_{j=1}^n m_j\left|\ddot{\bf r}_j - \frac{{\bf F}_j}{m_j}\right|^2,
\end{align*}
where the $j$th particle has \href{https://en.wikipedia.org/wiki/Mass}{mass} $m_j$, \href{https://en.wikipedia.org/wiki/Position_vector}{position vector} ${\bf r}_j$, \& applied non-constraint force ${\bf F}_j$ acting on the mass.

The notation $\dot{\bf r}$ indicates \href{https://en.wikipedia.org/wiki/Time_derivative}{time derivative} of a vector function ${\bf r}(t)$, i.e., position. The corresponding \href{https://en.wikipedia.org/wiki/Acceleration}{accelerations} $\ddot{\bf r}_j$ satisfy the imposed constraints, which in general depends on the current state of the system, $\{{\bf r}_j(t),\dot{\bf r}_j(t)\}$.

It is recalled the fact that due to active ${\bf F}_j$ \& reactive (constraint) ${\bf F}_{{\bf c}j}$ forces being applied, with resultant ${\bf R} = \sum_{j=1}^n {\bf F}_j + {\bf F}_{{\bf c}j}$, a system will experience an acceleration $\ddot{\bf r}_j = \sum_{j=1}^n \frac{{\bf F}_j}{m_j} + \frac{{\bf F}_{{\bf c}j}}{m_j} = \sum_{j=1}^n {\bf a}_j + {\bf a}_{{\bf c}j}$.

\paragraph{Connections to other formulations.} Gauss's principle is equivalent to \href{https://en.wikipedia.org/wiki/D%27Alembert%27s_principle}{D'Alembert's principle}. The principle of least constraint is qualitatively similar to \href{https://en.wikipedia.org/wiki/Hamilton%27s_principle}{Hamilton's principle}, which states that the true path taken by a mechanical system is an extremum of the \href{https://en.wikipedia.org/wiki/Action_(physics)}{action}. However, Gauss's principle is a true (local) \textit{minimal} principle, whereas the other is an \textit{extremal} principle.'' -- \href{https://en.wikipedia.org/wiki/Gauss%27s_principle_of_least_constraint#Statement}{Wikipedia{\tt/}Gauss's principle of least constraint{\tt/}statement}

\subsubsection{Hertz's principle of least curvature}
``Hertz's principle of least curvature is a special case of Gauss's principle, restricted by the 2 conditions that there are no externally applied forces, no interactions (which can usually be expressed as a \href{https://en.wikipedia.org/wiki/Potential_energy}{potential energy}), \& all masses are equal. W.l.o.g., the masses may be set equal to 1. Under these conditions, Gauss's minimized quantity can be written $Z = \sum_{j=1}^n |\ddot{\bf r}_j|^2$. The \href{https://en.wikipedia.org/wiki/Kinetic_energy}{kinetic energy} $T$ is also conserved under these conditions $T\coloneqq\frac{1}{2}\sum_{j=1}^n |\dot{\bf r}_j|^2$. Since the \href{https://en.wikipedia.org/wiki/Line_element}{line element} ${\rm d}s^2$ in the $3n$-dimensional space of the coordinates is defined ${\rm d}s^2\coloneqq\sum_{j=1}^n |{\rm d}{\bf r}_j|^2$, the \href{https://en.wikipedia.org/wiki/Conservation_of_energy}{conservation of energy} may also be written $\left(\frac{{\rm d}s}{{\rm d}t}\right)^2 = 2T$. Dividing $Z$ by $2T$ yields another minimal quantity $K\coloneqq\sum_{j=1}^n \left|\frac{{\rm d}^2{\bf r}_j}{{\rm d}s^2}\right|^2$. Since $\sqrt{K}$ is the local \href{https://en.wikipedia.org/wiki/Curvature}{curvature} of the trajectory in the $3n$-dimensional space of the coordinates, minimization of $K$ is equivalent to finding the trajectory of least curvature (a \href{https://en.wikipedia.org/wiki/Geodesic}{geodesic}) that is consistent with the constraints. Hertz's principle is also a special case of \href{https://en.wikipedia.org/wiki/Carl_Gustav_Jakob_Jacobi}{Jacobi}'s formulation of \href{https://en.wikipedia.org/wiki/Maupertuis%27_principle}{the least-action principle}.'' -- \href{https://en.wikipedia.org/wiki/Gauss's_principle_of_least_constraint}{Wikipedia{\tt/}Gauss's principle of least constraint}

%------------------------------------------------------------------------------%

\subsection{Wikipedia{\tt/}mathematical physics}
``{\it Mathematical physics} refers to the development of mathematical methods for application to problems in physics. The \href{https://en.wikipedia.org/wiki/Journal_of_Mathematical_Physics}{\it Journal of Mathematical Physics} defines the field as ``the application of mathematics to problems in physics \& the development of mathematical methods suitable for such applications \& for the formulation of physical theories''. An alternative definition would also include those mathematics that are inspired by physics, known as \href{https://en.wikipedia.org/wiki/Physical_mathematics}{physical mathematics}.

\subsubsection{Scope}
There are several distinct branches of mathematical physics, \& these roughly correspond to particular historical parts of our world.
\begin{enumerate}
	\item {\bf Classical mechanics.} Main articles: \href{https://en.wikipedia.org/wiki/Lagrangian_mechanics}{Wikipedia{\tt/}Lagrangian mechanics}, \href{https://en.wikipedia.org/wiki/Hamiltonian_mechanics}{Wikipedia{\tt/}Hamiltonian mechanics}. Applying the techniques of mathematical physics to classical mechanics typically involves the rigorous, abstract, \& advanced reformulation of Newtonian mechanics in terms of Lagrangian mechanics \& Hamiltonian mechanics (including both approaches in the presence of constraints). Both formulations are embodied in \href{https://en.wikipedia.org/wiki/Analytical_mechanics}{analytical mechanics} \& lead to an understanding of the deep interplay between the notions of \href{https://en.wikipedia.org/wiki/Symmetry_(physics)}{symmetry} \& \href{https://en.wikipedia.org/wiki/Conservation_law}{conserved quantities} during the dynamical evolution of mechanical system, as embodied within the most elementary formulation of \href{https://en.wikipedia.org/wiki/Noether%27s_theorem}{Noether's theorem}. These approaches \& ideas have been extended to other areas of physics, e.g. \href{https://en.wikipedia.org/wiki/Statistical_mechanics}{statistical mechanics}, \href{https://en.wikipedia.org/wiki/Continuum_mechanics}{continuum mechanics}, \href{https://en.wikipedia.org/wiki/Classical_field_theory}{classical field theory}, \& \href{https://en.wikipedia.org/wiki/Quantum_field_theory}{quantum field theory}. Moreover, they have provided multiple examples \& ideas in \href{https://en.wikipedia.org/wiki/Differential_geometry}{differential geometry} (e.g., several notions in \href{https://en.wikipedia.org/wiki/Symplectic_geometry}{symplectic geometry} \& \href{https://en.wikipedia.org/wiki/Vector_bundle}{vector bundles}).
	\item {\bf PDEs.} Within mathematics proper, the theory of PDE, \href{https://en.wikipedia.org/wiki/Variational_calculus}{variational calculus}, \href{https://en.wikipedia.org/wiki/Fourier_analysis}{Fourier analysis}, \href{https://en.wikipedia.org/wiki/Potential_theory}{potential theory}, \& \href{https://en.wikipedia.org/wiki/Vector_analysis}{vector analysis} are perhaps most closely associated with mathematical physics. These fields were developed intensively from the 2nd half of the 18th century (by, e.g., \href{https://en.wikipedia.org/wiki/Jean_le_Rond_d%27Alembert}{\sc D'Alembert}, \href{https://en.wikipedia.org/wiki/Leonhard_Euler}{\sc Euler}, \& \href{https://en.wikipedia.org/wiki/Joseph-Louis_Lagrange}{\sc Lagrange}) until the 1930s. Physical applications of these developments include \href{https://en.wikipedia.org/wiki/Hydrodynamics}{hydrodynamics}, \href{https://en.wikipedia.org/wiki/Celestial_mechanics}{celestial mechanics}, \href{https://en.wikipedia.org/wiki/Continuum_mechanics}{continuum mechanics}, \href{https://en.wikipedia.org/wiki/Elasticity_theory}{elasticity theory}, \href{https://en.wikipedia.org/wiki/Acoustics}{acoustics}, \href{https://en.wikipedia.org/wiki/Thermodynamics}{thermodynamics}, \href{https://en.wikipedia.org/wiki/Electricity}{electricity}, \href{https://en.wikipedia.org/wiki/Magnetism}{magnetism}, \& \href{https://en.wikipedia.org/wiki/Aerodynamics}{aerodynamics}. {\sf An example of mathematical physics: solutions of \href{https://en.wikipedia.org/wiki/Schr%C3%B6dinger%27s_equation}{Schr\"odinger's equation} for \href{https://en.wikipedia.org/wiki/Quantum_harmonic_oscillator}{quantum harmonic oscillators} with their \href{https://en.wikipedia.org/wiki/Probability_amplitude}{amplitude}.}
	\item {\bf Quantum theory.} Main article: \href{https://en.wikipedia.org/wiki/Quantum_mechanics}{Wikipedia{\tt/}quantum mechanics}. The theory of \href{https://en.wikipedia.org/wiki/Atomic_spectra}{atomic spectra} (\&, later, \href{https://en.wikipedia.org/wiki/Quantum_mechanics}{quantum mechanics}) developed almost concurrently with some parts of the mathematical fields of linear algebra, the \href{https://en.wikipedia.org/wiki/Spectral_theory}{spectral theory} of \href{https://en.wikipedia.org/wiki/Operator_(mathematics)}{operators}, \href{https://en.wikipedia.org/wiki/Operator_algebra}{operator algebras} \&, more broadly, \href{https://en.wikipedia.org/wiki/Functional_analysis}{functional analysis}. Nonrelativistic quantum mechanics includes \href{https://en.wikipedia.org/wiki/Schr%C3%B6dinger}{Schr\"odinger} operators, \& it has connections to \href{https://en.wikipedia.org/wiki/Atomic,_molecular,_and_optical_physics}{atomic \& molecular physics}. \href{https://en.wikipedia.org/wiki/Quantum_information}{Quantum information} theory is another subspecialty.
	\item {\bf Relativity \& quantum relativistic theories.} \href{https://en.wikipedia.org/wiki/Theory_of_relativity}{Wikipedia{\tt/}theory of relativity}, \href{https://en.wikipedia.org/wiki/Quantum_field_theory}{Wikipedia{\tt/}quantum field theory}. The \href{https://en.wikipedia.org/wiki/Special_relativity}{special} \& \href{https://en.wikipedia.org/wiki/General_relativity}{general} theories of relativity require a rather different type of mathematics. This was \href{https://en.wikipedia.org/wiki/Group_theory}{group theory}, which played an important role in both \href{https://en.wikipedia.org/wiki/Quantum_field_theory}{quantum field theory} \& differential geometry. This was, however, gradually supplemented by \href{https://en.wikipedia.org/wiki/Topology}{topology} \& functional analysis in the mathematical description of \href{https://en.wikipedia.org/wiki/Physical_cosmology}{cosmological} as well as \href{https://en.wikipedia.org/wiki/Quantum_field_theory}{quantum field theory} phenomena. In the mathematical description of these physical areas, some concepts in \href{https://en.wikipedia.org/wiki/Homological_algebra}{homological algebra} \& \href{https://en.wikipedia.org/wiki/Category_theory}{category theory} are also important.
	\item {\bf Statistical mechanics.} Main article: \href{https://en.wikipedia.org/wiki/Statistical_mechanics}{Wikipedia{\tt/}statistical mechanics}. Statistical mechanics forms a separate field, which includes the theory of \href{https://en.wikipedia.org/wiki/Phase_transition}{phase transitions}. It relies upon the Hamiltonian mechanics (for its quantum versions) \& it is closely related with the more mathematical \href{https://en.wikipedia.org/wiki/Ergodic_theory}{ergodic theory} \& some parts of \href{https://en.wikipedia.org/wiki/Probability_theory}{probability theory}. There are increasing interactions between \href{https://en.wikipedia.org/wiki/Combinatorics_and_physics}{combinatorics \& physics}, in particular statistical physics.
\end{enumerate}

\subsubsection{Usage}
{\sf Relationship between mathematics \& physics: Math $\to$ [Mathematical Physics, Ontology] $\to$ Physics.} The usage of the term ``mathematical physics'' is sometimes \href{https://en.wikipedia.org/wiki/Idiosyncratic}{idiosyncratic} (có phong cách riêng). Certain parts of mathematics that initially arose from the development of physics are not, in fact, considered parts of mathematical physics, while other closely related fields are. E.g., \href{https://en.wikipedia.org/wiki/Ordinary_differential_equation}{ODEs} \& \href{https://en.wikipedia.org/wiki/Symplectic_geometry}{symplectic geometry} are generally viewed as purely mathematical disciplines, whereas \href{https://en.wikipedia.org/wiki/Dynamical_system}{dynamical systems} \& \href{https://en.wikipedia.org/wiki/Hamiltonian_mechanics}{Hamiltonian mechanics} belong to mathematical physics. \href{https://en.wikipedia.org/wiki/John_Herapath}{\sc John Herapath} used the term for the title of his 1847 text on ``mathematical principles of natural philosophy'', the scope at that time being ``the causes of heat, gaseous elasticity, gravitation, \& other great phenomena of nature''.
\begin{enumerate}
	\item {\bf Mathematical vs. theoretical physics.} The term ``mathematical physics'' is sometimes used to denote research aimed at studying \& solving problems in physics or \href{https://en.wikipedia.org/wiki/Thought_experiment}{thought experiments} within a mathematically \href{https://en.wikipedia.org/wiki/Mathematical_rigour}{rigorous} framework. In this sense, mathematical physics covers a very broad academic realm distinguished only by the blending of some mathematical aspect \& theoretical physics aspect. Although related to \href{https://en.wikipedia.org/wiki/Theoretical_physics}{theoretical physics}, mathematical physics in this sense emphasizes the mathematical rigor of the similar type as found in mathematics.
	
	On the other hand, theoretical physics emphasizes the links to observations \& \href{https://en.wikipedia.org/wiki/Experimental_physics}{experimental physics}, which often requires theoretical physicists (\& mathematical physicists in the more general sense) to use \href{https://en.wikipedia.org/wiki/Heuristic}{heuristic}, \href{https://en.wikipedia.org/wiki/Intuition_(knowledge)}{intuitive}, or approximate arguments. Such arguments are not considered rigorous by mathematicians.
	
	Such mathematical physicists primarily expand \& elucidate (làm sáng tỏ) physics \href{https://en.wikipedia.org/wiki/Theories}{theories}. Because of the required level of mathematical rigor, these researches often deal with questions that theoretical physicists have considered to be already solved. However, they can sometimes show that the previous solution was incomplete, incorrect, or simply too naive. E.g., issues about attempts to infer the 2nd law of \href{https://en.wikipedia.org/wiki/Thermodynamics}{thermodynamics} from \href{https://en.wikipedia.org/wiki/Statistical_mechanics}{statistical mechanics}. Other examples concern the subtleties involved with synchronization procedures in special \& general relativity (\href{https://en.wikipedia.org/wiki/Sagnac_effect}{Sagnac effect} \& \href{https://en.wikipedia.org/wiki/Einstein_synchronisation}{Einstein synchronization}).
	
	The effort to put physical theories on a mathematically rigorous footing not only developed physics but also has influenced developments of some mathematical areas. E.g., the development of quantum mechanics \& some aspects of \href{https://en.wikipedia.org/wiki/Functional_analysis}{functional analysis} parallel each other in many ways. The mathematical study of \href{https://en.wikipedia.org/wiki/Quantum_mechanics}{quantum mechanics}, \href{https://en.wikipedia.org/wiki/Quantum_field_theory}{quantum field theory}, \& \href{https://en.wikipedia.org/wiki/Quantum_statistical_mechanics}{quantum statistical mechanics} has motivated results in \href{https://en.wikipedia.org/wiki/Operator_algebra}{operator algebras}. The attempt to construct a rigorous mathematical formulation of quantum field theory has also brought about some progress in fields such as \href{https://en.wikipedia.org/wiki/Representation_theory}{representation theory}.
\end{enumerate}

\subsubsection{Prominent mathematical physicists}

\begin{enumerate}
	\item {\bf Before Newton.}
	\item {\bf Newtonian physics \& post Newtonian.}
	\item {\bf Relativistic.}
	\item {\bf Quantum.}
	\item {\bf List of prominent contributors to mathematical physics in the 20th century.}
\end{enumerate}
'' -- \href{https://en.wikipedia.org/wiki/Mathematical_physics}{Wikipedia{\tt/}mathematical physics}

%------------------------------------------------------------------------------%

\subsection{Wikipedia{\tt/}numerical methods in fluid mechanics}
``\href{https://en.wikipedia.org/wiki/Fluid_motion}{Fluid motion} is governed by the \href{https://en.wikipedia.org/wiki/Navier-Stokes_equations}{Navier-Stokes equations}, a set of coupled and nonlinear partial differential equations derived from the basic laws of conservation of \href{https://en.wikipedia.org/wiki/Mass}{mass}, \href{https://en.wikipedia.org/wiki/Momentum}{momentum} and \href{https://en.wikipedia.org/wiki/Energy}{energy}.

The unknowns are usually the \href{https://en.wikipedia.org/wiki/Flow_velocity}{flow velocity}, the \href{https://en.wikipedia.org/wiki/Pressure}{pressure} and \href{https://en.wikipedia.org/wiki/Density}{density} and \href{https://en.wikipedia.org/wiki/Temperature}{temperature}.

The \href{https://en.wikipedia.org/wiki/Analytical_solution}{analytical solution} of this equation is impossible hence scientists resort to laboratory experiments in such situations. The answers delivered are, however, usually qualitatively different since dynamical and geometric similitude are difficult to enforce simultaneously between the lab experiment and the \href{https://en.wikipedia.org/wiki/Prototype}{prototype}.

Furthermore, the design and construction of these experiments can be difficult (and costly), particularly for stratified rotating flows.

\href{https://en.wikipedia.org/wiki/Computational_fluid_dynamics}{Computational fluid dynamics} (CFD) is an additional tool in the arsenal of scientists.

In its early days CFD was often controversial, as it involved additional approximation to the governing equations and raised additional (legitimate) issues.

Nowadays CFD is an established discipline alongside theoretical and experimental methods.

This position is in large part due to the exponential growth of computer power which has allowed us to tackle ever larger and more complex problems.

\subsubsection{Discretization}
The central process in CFD is the process of \href{https://en.wikipedia.org/wiki/Discretization}{discretization}, i.e. the process of taking differential equations with an infinite number of \href{https://en.wikipedia.org/wiki/Degrees_of_freedom}{degrees of freedom}, and reducing it to a system of finite degrees of freedom.

Hence, instead of determining the solution everywhere and for all times, we will be satisfied with its calculation at a finite number of locations and at specified time intervals.

The \href{https://en.wikipedia.org/wiki/Partial_differential_equations}{PDEs} are then reduced to a system of algebraic equations that can be solved on a computer.

Errors creep in during the discretization process.

The nature and characteristics of the errors must be controlled in order to ensure that:
\begin{itemize}
	\item we are solving the correct equations (consistency property)
	\item that the error can be decreased as we increase the number of degrees of freedom (stability and convergence).
\end{itemize}
Once these 2 criteria are established, the power of computing machines can be leveraged to solve the problem in a numerically reliable fashion.

Various discretization schemes have been developed to cope with a variety of issues.

The most notable for our purposes are: \href{https://en.wikipedia.org/wiki/Finite_difference_methods}{finite difference methods}, finite volume methods, \href{https://en.wikipedia.org/wiki/Finite_element_methods}{finite element methods}, and \href{https://en.wikipedia.org/wiki/Spectral_methods}{spectral methods}.

\subsubsection{FDM}
Finite difference replace the infinitesimal limiting process of derivative calculation:
\begin{align*}
	\lim_{\Delta x\to 0} f'(x) = \frac{f(x + \Delta x) - f(x)}{\Delta x},
\end{align*}
with a finite limiting process, i.e.,
\begin{align*}
	f'(x) = \frac{f(x + \Delta x) - f(x)}{\Delta x} + \mathcal{O}(\Delta x).
\end{align*}
The term $\mathcal{O}(\Delta x)$ gives an indication of the magnitude of the error as a function of the mesh spacing.

In this instance, the error is halved if the grid spacing, $\Delta x$ is halved, and we say that this is a 1st order method.

Most FDM used in practice are at least 2nd order accurate except in very special circumstances.

Finite Difference method is still the most popular numerical method for solution of PDEs because of their simplicity, efficiency and low computational cost.

Their major drawback is in their geometric inflexibility which complicates their applications to general complex domains.

These can be alleviated by the use of either mapping techniques and/or masking to fit the computational mesh to the computational domain. 

\subsubsection{FEM}
The finite element method was designed to deal with problem with complicated computational regions.

The PDE is 1st recast into a variational form which essentially forces the mean error to be small everywhere.

The discretization step proceeds by dividing the computational domain into elements of triangular or rectangular shape.

The solution within each element is interpolated with a polynomial of usually low order.

Again, the unknowns are the solution at the collocation points.

The CFD community adopted the FEM in the 1980s when reliable methods for dealing with advection dominated problems were devised.

\subsubsection{Spectral method}
Both finite element and finite difference methods are low order methods, usually of 2nd - 4th order, and have local approximation property.

By local we mean that a particular collocation point is affected by a limited number of points around it.

In contrast, spectral method have global approximation property.

The interpolation functions, either polynomials or trigonomic functions are global in nature.

Their main benefits is in the rate of convergence which depends on the smoothness of the solution (i.e. how many continuous derivatives does it admit).

For infinitely smooth solution, the error decreases exponentially, i.e. faster than algebraic.

Spectral methods are mostly used in the computations of homogeneous turbulence, and require relatively simple geometries.

Atmospheric model have also adopted spectral methods because of their convergence properties and the regular spherical shape of their computational domain.

\subsubsection{FVM}
Finite volume methods are primarily used in \href{https://en.wikipedia.org/wiki/Aerodynamics}{aerodynamics} applications where strong shocks and discontinuities in the solution occur.

Finite volume method solves an integral form of the governing equations so that local continuity property do not have to hold.

\subsubsection{Computational cost}
The \href{https://en.wikipedia.org/wiki/CPU}{CPU} time to solve the system of equations differs substantially from method to method.

Finite differences are usually the cheapest on a per grid point basis followed by the finite element method and spectral method.

However, a per grid point basis comparison is a little like comparing apple and oranges.

Spectral methods deliver more accuracy on a per grid point basis than either FEM or FDM.

The comparison is more meaningful if the question is recast as ``\textit{what is the computational cost to achieve a given error tolerance?}''.

The problem becomes 1 of defining the error measure which is a complicated task in general situations.

\subsubsection{Forward Euler approximation}
\begin{align*}
	\frac{u^{n + 1} - u^n}{\Delta t}\approx\kappa u^n.
\end{align*}
Equation is an explicit approximation to the original differential equation since no information about the unknown function at the future time $(n + 1)t$ has been used on the RHS of the equation.

In order to derive the error committed in the approximation we rely again on Taylor series.

\subsubsection{Backward difference}
This is an example of an implicit method since the unknown $u(n + 1)$ has been used in evaluating the slope of the solution on the RHS; this is not a problem to solve for $u(n + 1)$ in this scalar and linear case.

For more complicated situations like a nonlinear RHS or a system of equations, a nonlinear system of equations may have to be inverted.'' -- \href{https://en.wikipedia.org/wiki/Numerical_methods_in_fluid_mechanics}{Wikipedia{\tt/}numerical methods in fluid mechanics}

%------------------------------------------------------------------------------%

\subsection{Wikipedia{\tt/}physics}
``

'' -- \href{https://en.wikipedia.org/wiki/Physics}{Wikipedia{\tt/}physics}

%------------------------------------------------------------------------------%

\subsection{Wikipedia{\tt/}hydrostatics}
{\sf Table of Hydraulics \& Hydrostatics, from the 1728 Cyclop\ae dia.}

%
{\it Fluid statics} or {\it hydrostatics} is the branch of \href{https://en.wikipedia.org/wiki/Fluid_mechanics}{fluid mechanics} that studies ``\href{https://en.wikipedia.org/wiki/Fluid}{fluids} at rest \& the pressure in a fluid or exerted by a fluid on an immersed body''.[``Hydrostatics''. Merriam-Webster. Retrieved Sep 11, 2018.]

%
It encompasses the study of the conditions under which fluids are at rest in \href{https://en.wikipedia.org/wiki/Mechanical_equilibrium}{stable equilibrium} as opposed to \href{https://en.wikipedia.org/wiki/Fluid_dynamics}{fluid dynamics}, the study of fluids in motion.

Hydrostatics are categorized as a part of the fluid statics, which is the study of all fluids, incompressible or not, at rest.

%
Hydrostatics is fundamental to \href{https://en.wikipedia.org/wiki/Hydraulics}{hydraulics}, the engineering of equipment for storing, transporting \& using fluids.

It is also relevant to \href{https://en.wikipedia.org/wiki/Geophysics}{geophysics} \& \href{https://en.wikipedia.org/wiki/Astrophysics}{astrophysics} (e.g., in understanding \href{https://en.wikipedia.org/wiki/Plate_tectonics}{plate tectonics} \& the anomalies of the \href{https://en.wikipedia.org/wiki/Gravity_of_Earth}{Earth's gravitational field}), to \href{https://en.wikipedia.org/wiki/Meteorology}{meteorology}, to \href{https://en.wikipedia.org/wiki/Medicine}{medicine} (in the context of \href{https://en.wikipedia.org/wiki/Blood_pressure}{blood pressure}), \& many other fields.

%
Hydrostatics offers physical explanations for many phenomena of everyday life, such as why \href{https://en.wikipedia.org/wiki/Atmospheric_pressure}{atmospheric pressure} changes with \href{https://en.wikipedia.org/wiki/Altitude}{altitude}, why wood \& oil float on water, \& why the surface of still water is always level.

\subsubsection{History}
Some principles of hydrostatics have been known in an empirical \& intuitive sense since antiquity, by the builders of boats, \href{https://en.wikipedia.org/wiki/Cistern}{cisterns}, \href{https://en.wikipedia.org/wiki/Aqueduct_(water_supply)}{aqueducts} \& \href{https://en.wikipedia.org/wiki/Fountain}{fountains}.

\href{https://en.wikipedia.org/wiki/Archimedes}{Archimedes} is credited with the discovery of \href{https://en.wikipedia.org/wiki/Archimedes'_Principle}{Archimedes' Principle}, which relates the \href{https://en.wikipedia.org/wiki/Buoyancy}{buoyancy} force on an object that is submerged in a fluid to the weight of fluid displaced by the object.

The \href{https://en.wikipedia.org/wiki/Roman_Empire}{Roman} engineer \href{https://en.wikipedia.org/wiki/Vitruvius}{Vitruvius} warned readers about \href{https://en.wikipedia.org/wiki/Lead}{lead} pipes bursting under hydrostatic pressure.[Marcus Vitruvius Pollio (ca. 15 BCE), ``The Ten Books of Architecture'', Book VIII, Chapter 6. At the University of Chicago's Penelope site. Accessed on 2013-02-25.]

%
The concept of pressure \& the way it is transmitted by fluids was formulated by the \href{https://en.wikipedia.org/wiki/France}{French} \href{https://en.wikipedia.org/wiki/Mathematician}{mathematician} \& \href{https://en.wikipedia.org/wiki/Philosopher}{philosopher} \href{https://en.wikipedia.org/wiki/Blaise_Pascal}{Blaise Pascal} in 1647.

\paragraph{Hydrostatics in ancient Greece \& Rome.}

\subparagraph{Pythagorean Cup.} Main article: \href{https://en.wikipedia.org/wiki/Pythagorean_cup}{Pythagorean cup}.

%
The ``fair cup'' or \href{https://en.wikipedia.org/wiki/Pythagorean_cup}{Pythagorean cup}, which dates from about the 6th century BC, is a hydraulic technology whose invention is credited to the Greek mathematician \& geometer Pythagoras.

It was used as a learning tool.

%
The cup consists of a line carved into the interior of the cup, \& a small vertical pipe in the center of the cup that leads to the bottom.

The height of this pipe is the same as the line carved into the interior of the cup.

The cup may be filled to the line without any fluid passing into the pipe in the center of the cup.

However, when the amount of fluid exceeds this fill line, fluid will overflow into the pipe in the center of the cup.

Due to the drag that molecules exert on one another, the cup will be emptied.

\subparagraph{Heron's fountain.} Main article: \href{https://en.wikipedia.org/wiki/Heron's_fountain}{Heron's fountain}.

%
\href{https://en.wikipedia.org/wiki/Heron's_fountain}{Heron's fountain} is a device invented by \href{https://en.wikipedia.org/wiki/Heron_of_Alexandria}{Heron of Alexandria} that consists of a jet of fluid being fed by a reservoir of fluid.

The fountain is constructed in such a way that the height of the jet exceeds the height of the fluid in the reservoir, apparently in violation of principles of hydrostatic pressure.

The device consisted of an opening \& 2 containers arranged one above the other.

The intermediate pot, which was sealed, was filled with fluid, \& several \href{https://en.wikipedia.org/wiki/Cannula}{cannula} (a small tube for transferring fluid between vessels) connecting the various vessels.

Trapped air inside the vessels induces a jet of water out of a nozzle, emptying all water from the intermediate reservoir. 

\paragraph{Pascal's contribution in hydrostatics.} Main article: \href{https://en.wikipedia.org/wiki/Pascal's_Law}{Pascal's Law}.

%
Pascal made contributions to developments in both hydrostatics \& hydrodynamics.

\href{https://en.wikipedia.org/wiki/Pascal's_Law}{Pascal's Law} is a fundamental principle of fluid mechanics that states that any pressure applied to the surface of a fluid is transmitted uniformly throughout the fluid in all directions, in such a way that initial variations in pressure are not changed.

\subsubsection{Pressure in fluids at rest}
Due to the fundamental nature of fluids, a fluid cannot remain at rest under the presence of a \href{https://en.wikipedia.org/wiki/Shear_stress}{shear stress}.

However, fluids can exert \href{https://en.wikipedia.org/wiki/Pressure}{pressure} \href{https://en.wikipedia.org/wiki/Surface_normal}{normal} to any contacting surface.

If a point in the fluid is thought of as an infinitesimally small cube, then it follows from the principles of equilibrium that the pressure on every side of this unit of fluid must be equal.

If this were not the case, the fluid would move in the direction of the resulting force.

Thus, the pressure on a fluid at rest is \href{https://en.wikipedia.org/wiki/Isotropic}{isotropic}; i.e., it acts with equal magnitude in all directions.

This characteristic allows fluids to transmit force through the length of pipes or tubes; i.e., a force applied to a fluid in a pipe is transmitted, via the fluid, to the other end of the pipe.

This principle was 1st formulated, in a slightly extended form, by Blaise Pascal, \& is now called \href{https://en.wikipedia.org/wiki/Pascal's_law}{Pascal's law}.

\paragraph{Hydro pressure.} See also: \href{https://en.wikipedia.org/wiki/Vertical_pressure_variation}{Vertical pressure variation}.

%
In a fluid at rest, all frictional \& inertial stresses vanish \& the state of stress of the system is called {\it hydrostatic}.

When this condition of $V = 0$ is applied to the \href{https://en.wikipedia.org/wiki/Navier-Stokes_equations}{Navier-Stokes equations}, the gradient of pressure becomes a function of body forces only.

For a \href{https://en.wikipedia.org/wiki/Barotropic_fluid}{barotropic fluid} in a conservative force field like a gravitational force field, the pressure exerted by a fluid at equilibrium becomes a function of force exerted by gravity.

%
The hydrostatic pressure can be determined from a control volume analysis of an infinitesimally small cube of fluid.

Since pressure is defined as the force exerted on a test area ($p = \frac{F}{A}$, with $p$: pressure, $F$: force normal to area $A$, $A$: area), \& the only force acting on any such small cube of fluid is the weight of the fluid column above it, hydrostatic pressure can be calculated according to the following formula:
\begin{align*}
	p(z) - p(z_0) = \frac{1}{A}\int_{z_0}^z {\rm d}z'\iint_A {\rm d}x'{\rm d}y'\rho(z')g(z') = \int_{z_0}^z {\rm d}z'\rho(z')g(z'),
\end{align*}
where:
\begin{itemize}
	\item $p$ is the hydrostatic pressure (Pa),
	\item $\rho$ is the fluid \href{https://en.wikipedia.org/wiki/Density}{density} ($\rm kg/m^3$),
	\item $g$ is \href{https://en.wikipedia.org/wiki/Gravity}{gravitational} acceleration ($\rm m/s^2$),
	\item $A$ is the test area ($m^2$),
	\item $z$ is the height (parallel to the direction of gravity) of the test area (m),
	\item $z_0$ is the height of the \href{https://en.wikipedia.org/wiki/Pressure_measurement#Absolute,_gauge_and_differential_pressures_-_zero_reference}{zero reference point of the pressure} (m).
\end{itemize}
For water \& other liquids, this integral can be simplified significantly for many practical applications, based on the following 2 assumptions: Since many liquids can be considered \href{https://en.wikipedia.org/wiki/Incompressible}{incompressible}, a reasonable good estimation can be made from assuming a constant density throughout the liquid.

(The same assumption cannot be made within a gaseous environment.)

Also, since the height $h$ of the fluid column between $z$ \& $z_0$ is often reasonably small compared to the radius of the Earth, one can neglect the variation of $g$.

Under these circumstances, the integral is simplified into the formula:
\begin{align*}
	p - p_0 = \rho gh,
\end{align*}
where $h$ is the height $z - z_0$ of the liquid column between the test volume \& the zero reference point of the pressure.

This formula is often called \href{https://en.wikipedia.org/wiki/Simon_Stevin}{Stevin's} law.[3][4]

Note that this reference point should lie at or below the surface of the liquid.

Otherwise, one has to split the integral into 2 (or more) terms with the constant $\rho_{\rm liquid}$ \& $\rho(z')$ above.

E.g., the \href{https://en.wikipedia.org/wiki/Pressure_measurement#Absolute,_gauge_and_differential_pressures_-_zero_reference}{absolute pressure} compared to vacuum is:
\begin{align*}
	p = \rho gH + p_{\rm atm},
\end{align*}
where $H$ is the total height of the liquid column above the test area to the surface, \& $p_{\rm atm}$ is the \href{https://en.wikipedia.org/wiki/Atmospheric_pressure}{atmospheric pressure}, i.e., the pressure calculated from the remaining integral over the air column from the liquid surface to infinity.

This can easily be visualized using a \href{https://en.wikipedia.org/wiki/Pressure_prism}{pressure prism}.

%
Hydrostatic pressure has been used in the preservation of foods in a process called \href{https://en.wikipedia.org/wiki/Pascalization}{pascalization}.[5]

\paragraph{Medicine.} In medicine, hydrostatic pressure in \href{https://en.wikipedia.org/wiki/Blood_vessel}{blood vessels} is the pressure of the blood against the wall.

It is the opposing force to \href{https://en.wikipedia.org/wiki/Oncotic_pressure}{oncotic pressure}.

\paragraph{Atmospheric pressure.} \href{https://en.wikipedia.org/wiki/Statistical_mechanics}{Statistical mechanics} shows that, for a pure \href{https://en.wikipedia.org/wiki/Ideal_gas}{ideal gas} of constant temperature, $T$, its pressure, $p$ will vary with height, $h$, as: 
\begin{align*}
	p(h) = p(0)e^{-\frac{Mgh}{kT}},
\end{align*}
where:
\begin{itemize}
	\item $g$ is the \href{https://en.wikipedia.org/wiki/Standard_gravity}{acceleration due to gravity}
	\item $T$ is the \href{https://en.wikipedia.org/wiki/Absolute_temperature}{absolute temperature}
	\item $k$ is \href{https://en.wikipedia.org/wiki/Boltzmann_constant}{Boltzmann constant}
	\item $M$ is the mass of a single \href{https://en.wikipedia.org/wiki/Molecule}{molecule} of gas
	\item $p$ is the pressure
	\item $h$ is the height
\end{itemize}
This is known as the \href{https://en.wikipedia.org/wiki/Barometric_formula}{barometric formula}, \& maybe derived from assuming the pressure is \href{https://en.wikipedia.org/wiki/Hydrostatic_pressure}{hydrostatic}.

%
If there are multiple types of molecules in the gas, the \href{https://en.wikipedia.org/wiki/Partial_pressure}{partial pressure} of each type will be given by this equation.

Under most conditions, the distribution of each species of gas is independent of the other species.

\paragraph{Buoyancy.} Main article: \href{https://en.wikipedia.org/wiki/Buoyancy}{Buoyancy}.

%
Any body of arbitrary shape which is immersed, partly or fully, in a fluid will experience the action of a net force in the opposite direction of the local pressure gradient.

If this pressure gradient arises from gravity, the net force is in the vertical direction opposite that of the gravitational force.

This vertical force is termed buoyancy or buoyant force \& is equal in magnitude, but opposite in direction, to the weight of the displaced fluid.

Mathematically,
\begin{align*}
	F = \rho gV,
\end{align*}
where $\rho$ is the density of the fluid, $g$ is the acceleration due to gravity, \& $V$ is the volume of fluid directly above the curved surface.[Fox, Robert; McDonald, Alan; Pritchard, Philip (2012). {\it Fluid Mechanics} (8 ed.). John Wiley \& Sons. pp. 76--83. ISBN 978-1-118-02641-0.]

In the case of a \href{https://en.wikipedia.org/wiki/Ship}{ship}, e.g., its weight is balanced by pressure forces from the surrounding water, allowing it to float.

If more cargo is loaded onto the ship, it would sink more into the water - displacing more water \& thus receive a higher buoyant force to balance the increased weight.

%
Discovery of the principle of buoyancy is attributed to \href{https://en.wikipedia.org/wiki/Archimedes}{Archimedes}. 

\paragraph{Hydrostatic force on submerged surfaces.} The horizontal \& vertical components of the hydrostatic force acting on a submerged surface are given by the following:[6]
\begin{align*}
	F_{\rm h} &= p_cA,\\
	F_{\rm v} &= \rho gV,
\end{align*}
where:
\begin{itemize}
	\item $p_c$ is the pressure at the centroid of the vertical projection of the submerged surface
	\item $A$ is the area of the same vertical projection of the surface
	\item $\rho$ is the density of the fluid
	\item $g$ is the acceleration due to gravity
	\item $V$ is the volume of fluid directly above the curved surface
\end{itemize}

\subsubsection{Liquids (fluids with free surfaces)}
Liquids can have free surfaces at which they interface with gases, or with a \href{https://en.wikipedia.org/wiki/Vacuum}{vacuum}.

In general, the lack of the ability to sustain a \href{https://en.wikipedia.org/wiki/Shear_stress}{shear stress} entails that free surfaces rapidly adjust towards an equilibrium.

However, on small length scales, there is an important balancing force from \href{https://en.wikipedia.org/wiki/Surface_tension}{surface tension}.

\paragraph{Capillary action.} When liquids are constrained in vessels whose dimensions are small, compared to the relevant length scales, \href{https://en.wikipedia.org/wiki/Surface_tension}{surface tension} effects become important leading to the formation of a \href{https://en.wikipedia.org/wiki/Meniscus_(liquid)}{meniscus} through \href{https://en.wikipedia.org/wiki/Capillary_action}{capillary action}.

This capillary action has profound consequences for biological systems as it is part of 1 of the 2 driving mechanisms of the flow of water in \href{https://en.wikipedia.org/wiki/Plant}{plant} \href{https://en.wikipedia.org/wiki/Xylem}{xylem}, the \href{https://en.wikipedia.org/wiki/Transpirational_pull}{transpirational pull}.

\paragraph{Hanging drops.} Without surface tension, \href{https://en.wikipedia.org/wiki/Drop_(liquid)}{drops} would not be able to form.

The dimensions \& stability of drops are determined by surface tension.

The drop's surface tension is directly proportional to the cohesion property of the fluid.'' -- \href{https://en.wikipedia.org/wiki/Hydrostatics}{Wikipedia{\tt/}Hydrostatics}

%------------------------------------------------------------------------------%

\subsection{Wikipedia{\tt/}stationary-action principle}
``The \textit{stationary-action principle} -- also known as the \textit{principle of least action} -- is a \href{https://en.wikipedia.org/wiki/Variational_principle}{variational principle} that, when applied to the \href{https://en.wikipedia.org/wiki/Action_(physics)}{\textit{action}} of a \href{https://en.wikipedia.org/wiki/Mechanics}{mechanical} system, yields the \href{https://en.wikipedia.org/wiki/Equations_of_motion}{equations of motion} for that system. The principle states that the trajectories (i.e., the solutions of the equations of motions) are \href{https://en.wikipedia.org/wiki/Stationary_point}{\textit{stationary points}} of the system's \textit{action functional}. The term ``least action'' is a historical misnomer since the principle has no minimality requirement: the value of the action functional need not be minimal (even locally) on the trajectories. Least action refers to the absolute value of the action functional being minimized.

The principle can be used to derive \href{https://en.wikipedia.org/wiki/Newtonian_mechanics}{Newtonian}, \href{https://en.wikipedia.org/wiki/Lagrangian_mechanics}{Lagrangian}, \& \href{https://en.wikipedia.org/wiki/Hamiltonian_mechanics}{Hamiltonian} \href{https://en.wikipedia.org/wiki/Equations_of_motion}{equations of motion}, \& even \href{https://en.wikipedia.org/wiki/General_relativity}{general relativity} (see \href{https://en.wikipedia.org/wiki/Einstein%E2%80%93Hilbert_action}{Einstein--Hilbert action}). In relativity, a different action must be minimized or maximized.

The classical mechanics \& electromagnetic expressions are a consequence of quantum mechanics. The stationary action method helped in the development of quantum mechanics. In 1933, the physicist \href{https://en.wikipedia.org/wiki/Paul_Dirac}{Paul Dirac} demonstrated how this principle can be used in quantum calculations by discerning the \href{https://en.wikipedia.org/wiki/Path_integral_formulation#Quantum_action_principle}{quantum mechanical underpinning} of the principle in the \href{https://en.wikipedia.org/wiki/Interference_(wave_propagation)#Quantum_interference}{quantum interference} of amplitudes. Subsequently \href{https://en.wikipedia.org/wiki/Julian_Schwinger}{Julian Schwinger} \& \href{https://en.wikipedia.org/wiki/Richard_Feynman}{Richard Feynman} independently applied this principle in quantum electrodynamics.

The principle remains central in \href{https://en.wikipedia.org/wiki/Modern_physics}{modern physics} \& mathematics, being applied in \href{https://en.wikipedia.org/wiki/Thermodynamics}{thermodynamics}, \href{https://en.wikipedia.org/wiki/Fluid_mechanics}{fluid mechanics}, the \href{https://en.wikipedia.org/wiki/Theory_of_relativity}{theory of relativity}, \href{https://en.wikipedia.org/wiki/Quantum_mechanics}{quantum mechanics}, \href{https://en.wikipedia.org/wiki/Particle_physics}{particle physics}, \& \href{https://en.wikipedia.org/wiki/String_theory}{string theory} \& is a focus of modern mathematical investigation in \href{https://en.wikipedia.org/wiki/Morse_theory}{Morse theory}. \href{https://en.wikipedia.org/wiki/Maupertuis%27_principle}{Maupertuis' principle} \& \href{https://en.wikipedia.org/wiki/Hamilton%27s_principle}{Hamilton's principle} exemplify the principle of stationary action.

The action principle is preceded by earlier ideas in \href{https://en.wikipedia.org/wiki/Optics}{optics}. In \href{https://en.wikipedia.org/wiki/Ancient_Greece}{ancient Greece}, \href{https://en.wikipedia.org/wiki/Euclid}{Euclid} wrote in his \textit{Catoptrica} that, for the path of light reflecting from a mirror, the \href{https://en.wikipedia.org/wiki/Angle_of_incidence_(optics)}{angle of incidence} equals the \href{https://en.wikipedia.org/wiki/Angle_of_reflection}{angle of reflection}. \href{https://en.wikipedia.org/wiki/Hero_of_Alexandria}{Hero of Alexandria} later showed that this path was the shortest length \& least time.

Scholars often credit \href{Pierre Louis Maupertuis} for formulating the principle of least action because he wrote about it in 1744 \& 1746. However, \href{https://en.wikipedia.org/wiki/Leonhard_Euler}{Leonhard Euler} discussed the principle in 1744, \& evidence shows that \href{https://en.wikipedia.org/wiki/Gottfried_Leibniz}{Gottfried Leibniz} preceded both by 39 years.'' -- \href{https://en.wikipedia.org/wiki/Stationary-action_principle}{Wikipedia{\tt/}stationary-action principle}

\subsubsection{General statement}
\textsf{Fig. As the system evolves, ${\bf q}$ traces a path through \href{https://en.wikipedia.org/wiki/Configuration_space_(physics)}{configuration space} (only some are shown). The path taken by the system (red) has a stationary action ($\delta S = 0$) under small changes in the configuration of the system ($\delta{\bf q}$).}

``The \href{https://en.wikipedia.org/wiki/Action_(physics)}{\textit{action}}, denoted $\mathcal{S}$, of a physical system is defined as the \href{https://en.wikipedia.org/wiki/Integral_(mathematics)}{integral} of the \href{https://en.wikipedia.org/wiki/Lagrangian_mechanics}{Lagrangian} $L$ between 2 instants of \href{https://en.wikipedia.org/wiki/Time_in_physics}{time} $t_1$ \& $t_2$ -- technically a \href{https://en.wikipedia.org/wiki/Functional_(mathematics)}{functional} of the $N$ \href{https://en.wikipedia.org/wiki/Generalized_coordinates}{generalized coordinates} ${\bf q} = (q_1,\ldots,q_n)$ which are functions of time \& define the \href{https://en.wikipedia.org/wiki/Configuration_space_(physics)}{configuration} of the system:
\begin{align*}
	{\bf q}:\mathbb{R}&\to\mathbb{R}^N,\\
	\mathcal{S}[{\bf q},t_1,t_2] &= \int_{t_1}^{t_2} L({\bf q}(t),\dot{\bf q}(t),t)\,{\rm d}t,
\end{align*}
where the dot denotes the \href{https://en.wikipedia.org/wiki/Time_derivative}{time derivative}, \& $t$ is time. Mathematically the principle is $\delta\mathcal{S} = 0$, where $\delta$ (lowercase Greek \href{https://en.wikipedia.org/wiki/Delta_(letter)}{delta}) means a \textit{small} change. In words this reads:
\begin{quotation}
	\textit{The path taken by the system between times $t_1$ \& $t_2$ \& configurations $q_1$ \& $q_2$ is the one for which the \textbf{action} is \textbf{stationary (no change)} to \textbf{1st order}.}
\end{quotation}
Stationary action is not always a minimum, despite the historical name of least action. It is a minimum principle for sufficiently short, finite segments in the path.

In applications the statement \& definition of action are taken together: $\delta\int_{t_1}^{t_2} L({\bf q},\dot{\bf q},t)\,{\rm d}t = 0$. The action \& Lagrangian both contain the dynamics of the system for all times. The term ``path'' is simply refers to a curve traced out by the system in terms of the coordinates in the \href{https://en.wikipedia.org/wiki/Configuration_space_(physics)}{configuration space}, i.e., the curve ${\bf q}(t)$, parametrized by time (see also \href{https://en.wikipedia.org/wiki/Parametric_equation}{parametric equation} for this concept).'' -- \href{https://en.wikipedia.org/wiki/Stationary-action_principle#General_statement}{Wikipedia{\tt/}stationary-action principle{\tt/}general statement}

\subsubsection{Origins, statements, \& controversy}

\begin{enumerate}
	\item {\sc Fermat.} ``Main article: \href{https://en.wikipedia.org/wiki/Fermat%27s_principle}{Wikipedia{\tt/}Fermat's principle}. In the 1600s, \href{https://en.wikipedia.org/wiki/Pierre_de_Fermat}{Pierre de Fermat} postulated that \textit{``light travels between 2 given points along the path of shortest time,''} which is known as the \textit{principle of least time} or \href{https://en.wikipedia.org/wiki/Fermat%27s_principle}{Fermat's principle}.
	\item {\sc Maupertuis.} Main article: \href{https://en.wikipedia.org/wiki/Maupertuis_principle}{Maupertuis principle}. Credit for the formulation of the \textit{principle of least action} is commonly given to \href{https://en.wikipedia.org/wiki/Pierre_Louis_Maupertuis}{Pierre Louis Maupertuis}, who felt that ``Nature is thrifty in all its actions'', \& applied the principle broadly:
	\begin{quotation}
		``The laws of movement \& of rest deduced from this principle being precisely the same as those observed in nature, we can admire the application of it to all phenomena. The movement of animals, the vegetative growth of plants $\ldots$ are only its consequences; \& the spectacle of the universe becomes so much the grander, so much more beautiful, the worthier of its Author, when one knows that a small number of laws, most wisely established, suffice for all movements.'' -- Pierre Louis Maupertuis
	\end{quotation}
	This notion of Maupertuis, although somewhat deterministic today, does capture much of the essence of mechanics.
	
	In application to physics, Maupertuis suggested that the quantity to be minimized was the product of the duration (time) of movement within a system by the ``\href{https://en.wikipedia.org/wiki/Vis_viva}{vis viva}'', \fbox{\textbf{Maupertuis' principle}: $\delta\int 2T(t)\,{\rm d}t = 0$}, which is the integral of twice what we now call the \href{https://en.wikipedia.org/wiki/Kinetic_energy}{kinetic eneergy} $T$ of the system.
	\item {\sc Euler.} \href{https://en.wikipedia.org/wiki/Leonhard_Euler}{Leonhard Euler} gave a formulation of the action principle in 1744, in very recognizable terms, in the \textit{Additamentum 2} to his \textit{Methodus Inveniendi Lineas Curvas Maximi Minive Proprietate Gaudentes}. Beginning with the 2nd paragraph:
	\begin{quotation}
		Let the mass of the projectile be $M$, \& let its speed be $v$ while being moved over an infinitesimal distance ${\rm d}s$. The body will have a momentum $Mv$ that, when multiplied by the distance ${\rm d}s$, will give $Mv{\rm d}s$, the momentum of the body integrated over the distance ${\rm d}s$. Now I assert that the curve thus described by the body to be the curve (from among all other curves connecting the same endpoints) that minimizes $\int Mv\,{\rm d}s$ or, provided that $M$ is constant along the path, $M\int v\,{\rm d}s$.'' -- Leonhard Euler
	\end{quotation}
	As Euler states, $\int Mv\,{\rm d}s$ is the integral of the momentum over distance traveled, which, in modern notation, equals the abbreviated or \href{https://en.wikipedia.org/wiki/Reduced_action}{reduced action} \fbox{\textbf{Euler's principle} $\delta\int p\,{\rm d}q = 0$.} Thus, Euler made an equivalent \& (apparently) independent statement of the variational principle in the same year as Maupertuis, albeit slightly later. Curiously, Euler did not claim any priority, as the following episode shows.
	\item {\bf Disputed priority.} Maupertuis' priority was disputed in 1751 by the mathematician \href{https://en.wikipedia.org/wiki/Samuel_K%C3%B6nig}{Samuel K\"onig}, who claimed that it had been invented by \href{https://en.wikipedia.org/wiki/Gottfried_Leibniz}{Gottfried Leibniz} in 1707. Although similar to many of Leibniz's arguments, the principle itself has not been documented in Leibniz's works. K\"onig himself showed a \textit{copy} of a 1707 letter from Leibniz to \href{https://en.wikipedia.org/wiki/Jacob_Hermann_(mathematician)}{Jacob Hermann} with the principle, but the \textit{original} letter has been lost. In contentious proceedings, K\"onig was accused of forgery\footnote{\textbf{forgery} [n] (plural \textbf{forgeries}) \textbf{1.} [uncountable] the crime of copying money, documents, etc. in order to cheat people; \textbf{2.} [countable] something, e.g. a document, piece of paper money, etc., that has been copied in order to cheat people, \textsc{synonym}: \textbf{fake}.}, \& even the \href{https://en.wikipedia.org/wiki/Frederick_the_Great}{King of Prussia} entered the debate, defending Maupertuis (the head of his Academy), while \href{https://en.wikipedia.org/wiki/Voltaire}{Voltaire} defended K\"onig.
	
	Euler, rather than claiming priority, was a staunch defender of Maupertuis, \& Euler himself prosecuted K\"onig for forgery before the Berlin Academy on Apr 13, 1752. The claims of forgery were re-examined 150 years later, \& archival work by C.I. Gerhardt in 1898 \& W. Kabitz in 1913 uncovered other copies of the letter, \& 3 others cited by K\"onig, in the \href{https://en.wikipedia.org/wiki/Bernoulli_family}{Bernoulli} archives.'' -- \href{https://en.wikipedia.org/wiki/Stationary-action_principle#Origins,_statements,_and_controversy}{Wikipedia{\tt/}stationary-action principle{\tt/}origins, statements, \& controversy}
\end{enumerate}

\subsubsection{Further development}
``Euler continued to write on the topic; in his \textit{R\'eflexions sur quelques loix g\'en\'erales de la nature} (1748), he called action ``effort''. His expression corresponds to modern \href{https://en.wikipedia.org/wiki/Potential_energy}{potential energy}, \& his statement of least action says that the total potential energy of a system of bodies at rest is minimized, a principle of modern statics.
\begin{enumerate}
	\item {\sc Lagrange \& Hamilton.} Main article: \href{https://en.wikipedia.org/wiki/Hamilton%27s_principle}{Wikipedia{\tt/}Hamilton's principle}. Much of the calculus of variations was stated by \href{https://en.wikipedia.org/wiki/Joseph-Louis_Lagrange}{Joseph-Louis Lagrange} in 1760 \& he proceeded to apply this to problems in dynamics. In \textit{M\'ecanique analytique} (1788) Lagrange derived the general \href{https://en.wikipedia.org/wiki/Lagrangian_equations_of_motion}{equations of motion} of a mechanical body. \href{https://en.wikipedia.org/wiki/William_Rowan_Hamilton}{William Rowan Hamilton} in 1834 \& 1835 applied the variational principle to the classical \href{https://en.wikipedia.org/wiki/Lagrangian_mechanics}{Lagrangian} \href{https://en.wikipedia.org/wiki/Function_(mathematics)}{function} $L = T - V$ to obtain the \href{https://en.wikipedia.org/wiki/Euler%E2%80%93Lagrange_equations}{Euler--Lagrange equations} in their present form.
	\item {\sc Jacobi, Morse, \& Caratheodory.} In 1842, \href{https://en.wikipedia.org/wiki/Carl_Gustav_Jacobi}{Carl Gustav Jacobi} tackled the problem of whether the variational principle always found minima as opposed to other \href{https://en.wikipedia.org/wiki/Stationary_points}{stationary points} (maxima or stationary \href{https://en.wikipedia.org/wiki/Saddle_points}{saddle points}); most of his work focused on \href{https://en.wikipedia.org/wiki/Geodesics}{geodesics} on 2D surfaces. The 1st clear general statements were given by \href{https://en.wikipedia.org/wiki/Marston_Morse}{Marston Morse} in the 1920s \& 1930s, leading to what is now known as \href{https://en.wikipedia.org/wiki/Morse_theory}{Morse theory}. E.g., Morse showed that the number of \href{https://en.wikipedia.org/wiki/Conjugate_points}{conjugate points} in a trajectory equaled the number of negative eigenvalues in the 2nd variation of the Lagrangian. A particularly elegant derivation of the Euler--Lagrange equation was formulated by \href{https://en.wikipedia.org/wiki/Constantin_Caratheodory}{Constantin Caratheodory} \& published by him in 1935.
	\item {\sc Gauss \& Hertz.} Other extremal principles of \href{https://en.wikipedia.org/wiki/Classical_mechanics}{classical mechanics} have been formulated, e.g. \href{https://en.wikipedia.org/wiki/Gauss%27s_principle_of_least_constraint}{Gauss's principle of least constraint} \& its corollary, \href{https://en.wikipedia.org/wiki/Hertz%27s_principle_of_least_curvature}{Hertz's principle of least curvature}.'' -- \href{https://en.wikipedia.org/wiki/Stationary-action_principle#Further_development}{Wikipedia{\tt/}stationary-action principle{\tt/}further development}
\end{enumerate}

\subsubsection{Disputes about possible teleological aspects}
``The mathematical equivalence of the \href{https://en.wikipedia.org/wiki/Differential_equation}{differential} \href{https://en.wikipedia.org/wiki/Equations_of_motion}{equations of motion} \& their \href{https://en.wikipedia.org/wiki/Integral_equation}{integral} counterpart has important philosophical implications. The differential equations are statements about quantities localized to a single point in space or single moment of time. E.g., \href{https://en.wikipedia.org/wiki/Newton%27s_laws_of_motion}{Newton's 2nd law} ${\bf F} = m{\bf a}$ states that the \textit{instantaneous} force ${\bf F}$ applied to a mass $m$ produces an acceleration ${\bf a}$ at the same \textit{instant}. By contrast, the action principle is not localized to a point; rather, it involves integrals over an interval of time \& (for fields) an extended region of space. Moreover, in the usual formulation of \href{https://en.wikipedia.org/wiki/Classical_physics}{classical} action principles, the initial \& final states of the system are fixed, e.g.,
\begin{quotation}
	\textit{Given that the particle begins at position $x_1$ at time $t_1$ \& ends at position $x_2$ at time $t_2$, the physical trajectory that connects these 2 endpoints is an \href{https://en.wikipedia.org/wiki/Extremum}{extremum} of the action integral}.
\end{quotation}
In particular, the fixing of the \textit{final} state has been interpreted as giving the action principle a \href{https://en.wikipedia.org/wiki/Teleology}{teleological character} which has been controversial historically. However, according to W. Yourgrau \& S. Mandelstam, \textit{the teleological approach $\ldots$ presupposes that the variational principles themselves have mathematical characteristics which they \emph{de facto} do not possess}. In addition, some critics maintain this apparent \href{https://en.wikipedia.org/wiki/Teleology}{teleology} occurs because of the way in which the question was asked. By specifying some but not all aspects of both the initial \& final conditions (the positions but not the velocities) we are making some inferences about the initial conditions from the final conditions, \& it is this ``backward'' inference that can be seen as a teleological explanation. Teleology can also be overcome if we consider the classical description as a limiting case of the \href{https://en.wikipedia.org/wiki/Quantum_mechanics}{quantum} formalism of \href{https://en.wikipedia.org/wiki/Path_integral_formulation}{path integration}, in which stationary paths are obtained as a result of interference of amplitudes along all possible paths.

The short story \href{https://en.wikipedia.org/wiki/Story_of_Your_Life}{Story of Your Life} by the speculative fiction writer \href{https://en.wikipedia.org/wiki/Ted_Chiang}{Ted Chiang} contains visual depictions of \href{https://en.wikipedia.org/wiki/Fermat%27s_Principle}{Fermat's Principle} along with a discussion of its teleological dimension. \href{https://en.wikipedia.org/wiki/Keith_Devlin}{Keith Devlin}'s \textit{The Math Instinct} contains a chapter, ``Elvis the Welsh Corgi Who Can Do Calculus'' that discusses the calculus ``embedded'' in some animals as they solve the ``least time'' problem in actual situations.'' -- \href{https://en.wikipedia.org/wiki/Stationary-action_principle}{Wikipedia{\tt/}stationary-action principle}

%------------------------------------------------------------------------------%

\subsection{Wikipedia{\tt/}universal (metaphysics)}
``In \href{https://en.wikipedia.org/wiki/Metaphysics}{metaphysics}, a {\it universal} is what particular things have in common, namely characteristics or qualities. In other words, universals are repeatable or recurrent entities that can be instantiated or exemplified by many particular things. E.g., suppose there are 2 chairs in a room, each of which is green. These 2 chairs share the quality ``\href{https://en.wiktionary.org/wiki/chairness}{chairness}'', as well as ``greenness'' or the quality of being green; in other words, they share 2 ``universals''. There are 3 major kinds of qualities or characteristics: \href{https://en.wikipedia.org/wiki/Type_(metaphysics)}{types or kinds} (e.g. mammal), \href{https://en.wikipedia.org/wiki/Property_(metaphysics)}{properties} (e.g., short, strong), \& \href{https://en.wikipedia.org/wiki/Relation_(metaphysics)}{relations} (e.g., father of, next to). These are all different types of universals.

Paradigmatically, universals are \href{https://en.wikipedia.org/wiki/Abstract_(philosophy)}{\it abstract} (e.g., humanity), whereas \href{https://en.wikipedia.org/wiki/Particular}{particulars} are \href{https://en.wikipedia.org/wiki/Concrete_(philosophy)}{\it concrete} (e.g., the personhood of Socrates). However, universals are not necessarily abstract \& particulars are not necessarily concrete. E.g., one might hold that numbers are particular yet abstract objects. Likewise, some philosophers, e.g., \href{https://en.wikipedia.org/wiki/David_Malet_Armstrong}{\sc D. M. Armstrong}, consider universals to be concrete.

Most do not consider \href{https://en.wikipedia.org/wiki/Class_(philosophy)}{classes} to be universals, although some prominent philosophers do, e.g. {\sc John Bigelow}.

\subsubsection{Problem of universals}
Main article: \href{https://en.wikipedia.org/wiki/Problem_of_universals}{Wikipedia{\tt/}problem of universals}. The \href{https://en.wikipedia.org/wiki/Problem_of_universals}{Wikipedia{\tt/}problem of universals} is an ancient problem in metaphysics on the existence of universals. The problem arises from attempts to account for the phenomenon of \href{https://en.wikipedia.org/wiki/Similarity_(philosophy)}{similarity} or attribute agreement among things. E.g., \href{https://en.wikipedia.org/wiki/Grass}{grass} \& \href{https://en.wikipedia.org/wiki/Granny_Smith}{{\sc Granny Smith} apples} are similar or agree in attribute, namely in having the attribute of greenness. The issue is how to account for this sort of agreement in attribute among things.

There are many philosophical positions regarding universals. Taking ``\href{https://en.wikipedia.org/wiki/Beauty}{beauty}'' as an example, 4 positions are:
\begin{itemize}
	\item \href{https://en.wikipedia.org/wiki/Idealism}{Idealism}: beauty is a property constructed in the mind, so it exists only in descriptions of things.
	\item \href{https://en.wikipedia.org/wiki/Platonic_realism}{Platonic extreme realism}: beauty is a property that exists in an ideal form independently of any mind or thing.
	\item \href{https://en.wikipedia.org/wiki/Aristotle%27s_theory_of_universals}{Aristotelian moderate realism} or \href{https://en.wikipedia.org/wiki/Conceptualism}{conceptualism}: beauty is a property of things ({\it fundamentum in re}) that the mind abstracts from these beautiful things.
	\item \href{https://en.wikipedia.org/wiki/Nominalism}{Nominalism}: there are no universals, only individuals.
\end{itemize}
Taking a broader view, the main positions are generally considered classifiable as: \href{https://en.wikipedia.org/wiki/Philosophical_realism}{extreme realism}, \href{https://en.wikipedia.org/wiki/Nominalism}{nominalism} (sometimes simply named ``anti-realism'' with regard to universals), \href{https://en.wikipedia.org/wiki/Moderate_realism}{moderate realism}, \& \href{https://en.wikipedia.org/wiki/Idealism}{idealism}. Extreme Realists posit the existence of independent, abstract universals to account for attribute agreement. Nominalists deny that universals exist, claiming that they are not necessary to explain attribute agreement. Conceptualists posit that universals exist only in the \href{https://en.wikipedia.org/wiki/Philosophy_of_mind}{mind}, or when conceptualized, denying the independent existence of universals, but accepting they have a {\it fundamentum in re}. Complications which arise include the implications of language use \& the complexity of relating language to \href{https://en.wikipedia.org/wiki/Ontology}{ontology}.

\subsubsection{Particular}
Main article: \href{https://en.wikipedia.org/wiki/Particular}{Wikipedia{\tt/}particular}. A universal may have instances, known as its {\it particulars}. E.g., the type {\it dog} (or {\it doghood}) is a universal, as are the property {\it red} (or {\it redness}) \& the relation {\it betweenness} (or {\it being between}). Any particular dog, red thing, or object that is between other things is not a unniversal, however, but is an {\it instance} of a universal. I.e., a universal type ({\it doghood}), property ({\it redness}), or relation ({\it betweenness}) \href{https://en.wikipedia.org/wiki/Substance_theory#Inherence}{inheres} in a particular object (a specific dog, red thing, or object between other things).

\subsubsection{Platonic realism}
\href{https://en.wikipedia.org/wiki/Platonic_realism}{Platonic realism} holds universals to be the \href{https://en.wikipedia.org/wiki/Referent}{referents} of general items, e.g. the \href{https://en.wikipedia.org/wiki/Abstraction}{\it abstract}, nonphysical, non-mental entities to which words e.g. ``sameness'', ``circularity'', \& ``beauty'' refer. Particulars are the referents of proper names, e.g. ``Phaedo'', or of definite descriptions that identify single objects, e.g. the phrase, ``that person over there''. Other metaphysical theories may use the terminology of universals to describe physical entities.

Plato's examples of what we might today call universals included mathematical \& geometrical ideas such as a circle \& natural numbers as universals. Plato's views on universals did, however, vary across several different discussions. In some cases, {\sc Plato} spoke as if the perfect circle functioned as the \href{https://en.wikipedia.org/wiki/Substantial_form}{form} or blueprint for all copies \& for the word definition of {\it circle}. In other discussions, {\sc Plato} describes particulars as ``participating'' in the associated universal.

Contemporary realists agree with the thesis that universals are multiply-exemplifiable entities. E.g., \href{https://en.wikipedia.org/wiki/D._M._Armstrong}{\sc D. M. Armstrong}, {\sc Nicholas Wolterstorff, Reinhardt Grossmann, Michael Loux}.

\subsubsection{Nominalism}
Nominalists hold that universals are not real mind-independent entities but either merely concepts (sometimes called ``conceptualism'') or merely names. Nominalists typically argue that properties are abstract particulars (like tropes) rather than universals. \href{https://en.wikipedia.org/wiki/JP_Moreland}{\sc JP Moreland} distinguishes between ``extreme'' \& ``moderate'' nominalism. Examples of nominalists include Buddhist logicians \& \href{https://en.wikipedia.org/wiki/Apoha}{apoha} theorists, the medieval philosophers \href{https://en.wikipedia.org/wiki/Roscelin_of_Compi%C3%A8gne}{\sc Roscelin of Compi\`egne} \& \href{https://en.wikipedia.org/wiki/William_of_Ockham}{William of Ockham} \& contemporary philosophers \href{https://en.wikipedia.org/wiki/W._V._O._Quine}{\sc W. V. O. Quine}, \href{https://en.wikipedia.org/wiki/Wilfred_Sellars}{\sc Wilfred Sellars}, \href{https://en.wikipedia.org/wiki/D._C._Williams}{D. C. Williams}, \& \href{https://en.wikipedia.org/wiki/Keith_Campbell_(philosopher)}{Keith Campbell}.

\subsubsection{Ness-ity-hood principle}
The {\it ness-ity-hood principle} is used mainly by English-speaking philosophers to generate convenient, concise names for universals or \href{https://en.wikipedia.org/wiki/Property_(philosophy)}{properties}. According to the Ness-Ity-Hood Principle, a name for any universal may be formed by taking the name of the \href{https://en.wikipedia.org/wiki/Predicate_(grammar)}{predicate} \& adding the suffix ``ness'', ``ity'', or ``hood''. E.g., the universal that is distinctive of left-handers may be formed by taking the predicate ``left-handed'' \& adding ``ness', which yields the name ``left-handedness''. The principle is most helpful in cases where there is not an established or standard name of the universal in ordinary English usage: What is the name of the universal distinctive of chairs? ``Chair'' in English is used not only as a subject (as in ``The chair is broken''), but also as a predicate (as in ``That is a chair''). So to generate a name for the universal distinctive of chairs, take the predicate ``chair'' \& add ``ness'', which yields ``chairness''.

'' -- \href{https://en.wikipedia.org/wiki/Universal_(metaphysics)}{Wikipedia{\tt/}universal (metaphysics)}

%------------------------------------------------------------------------------%

\section{Miscellaneous}

%------------------------------------------------------------------------------%

\printbibliography[heading=bibintoc]
	
\end{document}