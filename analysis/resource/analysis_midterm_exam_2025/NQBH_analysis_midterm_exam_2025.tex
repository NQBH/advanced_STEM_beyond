\documentclass[12pt]{article}
\usepackage[backend=biber,natbib=true,style=alphabetic,maxbibnames=50]{biblatex}
\addbibresource{/home/nqbh/reference/bib.bib}
\usepackage[utf8]{vietnam}
\usepackage{tocloft}
\renewcommand{\cftsecleader}{\cftdotfill{\cftdotsep}}
\usepackage[colorlinks=true,linkcolor=blue,urlcolor=red,citecolor=magenta]{hyperref}
\usepackage{amsmath,amssymb,amsthm,enumitem,float,graphicx,mathtools,tikz}
\usetikzlibrary{angles,calc,intersections,matrix,patterns,quotes,shadings}
\allowdisplaybreaks
\newtheorem{assumption}{Assumption}
\newtheorem{baitoan}{Bài}
\newtheorem{cauhoi}{Câu hỏi}
\newtheorem{conjecture}{Conjecture}
\newtheorem{corollary}{Corollary}
\newtheorem{dangtoan}{Dạng toán}
\newtheorem{definition}{Definition}
\newtheorem{dinhly}{Định lý}
\newtheorem{dinhnghia}{Định nghĩa}
\newtheorem{example}{Example}
\newtheorem{ghichu}{Ghi chú}
\newtheorem{hequa}{Hệ quả}
\newtheorem{hypothesis}{Hypothesis}
\newtheorem{lemma}{Lemma}
\newtheorem{luuy}{Lưu ý}
\newtheorem{nhanxet}{Nhận xét}
\newtheorem{notation}{Notation}
\newtheorem{note}{Note}
\newtheorem{principle}{Principle}
\newtheorem{problem}{Problem}
\newtheorem{proposition}{Proposition}
\newtheorem{question}{Question}
\newtheorem{remark}{Remark}
\newtheorem{theorem}{Theorem}
\newtheorem{vidu}{Ví dụ}
\usepackage[margin=2cm]{geometry}
\def\labelitemii{$\circ$}
\DeclareRobustCommand{\divby}{%
	\mathrel{\vbox{\baselineskip.65ex\lineskiplimit0pt\hbox{.}\hbox{.}\hbox{.}}}%
}
\setlist[itemize]{leftmargin=*}
\setlist[enumerate]{leftmargin=*}

\title{Đề Thi Giữa Kỳ Nhập Môn Giải Tích Hè 2025\\
	Introduction to Mathematical Analysis Summer 2025}
\date{\today}

\begin{document}
\maketitle
\noindent{\bf Yêu cầu.}
\begin{enumerate}
	\item Được phép sử dụng tài liệu giấy không giới hạn.
	\item Cấm sử dụng thiết bị điện tử, AIs. Nếu phát hiện 0 điểm ngay lần đầu tiên \& thu bài (không có cảnh cáo).
	\item Làm theo yêu cầu. Viết code giấy, nếu sai cú pháp, vẫn chấm nội dung.
	\item Nếu làm không được, viết định nghĩa sẽ được +0.25 điểm.
	\item Không làm được ý trước, vẫn có thể sử dụng kết quả các ý trước để làm ý sau của bài toán.
    \item Thời gian thi: 2 hours.
\end{enumerate}

\begin{baitoan}[Giới hạn của dãy số]
	(a) {\rm(1 điểm)} Tính giới hạn của dãy số $\{u_n\}_{n=1}^\infty$ được xác định bởi $u_n = e^{-\frac{1}{\sqrt{n}}}$. (b) {\rm(0.5 điểm)} Chứng minh $\lim_{n\to\infty} u_n = l$ bằng ngôn ngữ $\varepsilon$. (c) {\rm(1 điểm)} Tìm công thức, viết thuật toán \& chương trình {\sf C{\tt/}C++, Python} để tính chỉ số tối ưu
	\begin{equation*}
		N_\varepsilon^{\rm opt} = \min\{N(\varepsilon);|u_n - l| < \varepsilon,\ \forall n\ge N_\varepsilon\},\ \forall\varepsilon\in(0,\infty).
	\end{equation*}
\end{baitoan}

\begin{baitoan}[Giới hạn của hàm số]
	(a) {\rm(1 điểm)} Tính giới hạn của hàm số $\lim_{x\to x_0} f(x)$ với $f(x) = \dfrac{2x^4 - 6x^3 + x^2 + 3}{x - 1}$ tại $x = 1$. (b) {\rm(0.5 điểm)} Chứng minh $\lim_{x\to x_0} f(x) = l$ với giới hạn $l$ tìm được ở câu (a) bằng ngôn ngữ $\varepsilon$-$\delta$. (c) {\rm(1 điểm)} Tìm công thức, viết thuật toán \& chương trình {\sf C{\tt/}C++, Python} để tính $\delta_\varepsilon^{\rm opt}$ tối ưu
	\begin{equation*}
		\delta_\varepsilon^{\rm opt}(x_0)\coloneqq\min\{\delta(x_0,\varepsilon);|x - x_0| < \delta(x_0,\varepsilon)\Rightarrow|f(x) - l| < \varepsilon\},\ \forall\varepsilon\in(0,\infty).
	\end{equation*}
\end{baitoan}

\begin{baitoan}[Đạo hàm \& numerical differentiation]
	(a) {\rm(1 điểm)} Tính đạo hàm $f'(x_0)$ với hàm $f:\mathbb{R}\to\mathbb{R}$, $f(x) = \cosh(x^2 - 3x + 1)$. (b) {\rm(1 điểm)}  Tìm đạo hàm của hàm $f(x) = \cosh(x^2 - 3x + 1)$ tại điểm $x = 2$ bằng định nghĩa. (c) {\rm(1 điểm)} Xấp xỉ đạo hàm bằng $3$ công thức Newton forward, Newton backward, \& Stirling.
\end{baitoan}

\begin{baitoan}[Tích phân \& numerical integration]
	(a) {\rm(1 điểm)} Tính nguyên hàm $\int  2^{-x}\tanh2^{1 - x}\,{\rm d}x$. (b) {\rm(1 điểm)} Tính tích phân $\int_{-1}^1 \dfrac{{\rm d}x}{\sqrt{(x + 2)(3 - x)}}\,{\rm d}x$. (c) {\rm(1.5 điểm)} Xấp xỉ tích phân
    \begin{equation*}
        \int_{-1}^1 (x + 2)\sin(x^2 + 4x - 6)\,{\rm d}x,
    \end{equation*}
    bằng trapezoidal rule:
    \begin{align*}
        \int_a^b f(x)\,{\rm d}x &= \frac{b - a}{2}(f(a) + f(b)) - \frac{(b - a)^3}{12}f''(\xi)\mbox{ for some }\xi\in(a,b)\\
        &\approx\frac{b - a}{2}(f(a) + f(b)),
    \end{align*}
    \& đánh giá sai số.
\end{baitoan}

\begin{baitoan}[Tổng hợp kiến thức]
    {\rm(3 điểm)} Cho 1 dãy số $\{a_n\}_{n=1}^\infty$ với số hạng được xác định bởi
    \begin{equation*}
        a_n = f(n) + g'(n) + \int_{a(n)}^{b(n)} h(x)\,{\rm d}x,\ \forall n\in\mathbb{N}^\star.
    \end{equation*}
    Tìm điều kiện để dãy số: (a) hội tụ. (b) bị chặn. (c) ...
\end{baitoan}

%------------------------------------------------------------------------------%

\printbibliography[heading=bibintoc]

\end{document}