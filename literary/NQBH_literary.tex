\documentclass{article}
\usepackage[backend=biber,natbib=true,style=alphabetic,maxbibnames=50]{biblatex}
\addbibresource{/home/nqbh/reference/bib.bib}
\usepackage[utf8]{vietnam}
\usepackage{tocloft}
\renewcommand{\cftsecleader}{\cftdotfill{\cftdotsep}}
\usepackage[colorlinks=true,linkcolor=blue,urlcolor=red,citecolor=magenta]{hyperref}
\usepackage{amsmath,amssymb,amsthm,enumitem,float,graphicx,mathtools,soul,tikz}
\usetikzlibrary{angles,calc,intersections,matrix,patterns,quotes,shadings}
\allowdisplaybreaks
\newtheorem{assumption}{Assumption}
\newtheorem{baitoan}{}
\newtheorem{cauhoi}{Câu hỏi}
\newtheorem{conjecture}{Conjecture}
\newtheorem{corollary}{Corollary}
\newtheorem{dangtoan}{Dạng toán}
\newtheorem{definition}{Definition}
\newtheorem{dinhly}{Định lý}
\newtheorem{dinhnghia}{Định nghĩa}
\newtheorem{example}{Example}
\newtheorem{ghichu}{Ghi chú}
\newtheorem{hequa}{Hệ quả}
\newtheorem{hypothesis}{Hypothesis}
\newtheorem{lemma}{Lemma}
\newtheorem{luuy}{Lưu ý}
\newtheorem{nhanxet}{Nhận xét}
\newtheorem{notation}{Notation}
\newtheorem{note}{Note}
\newtheorem{principle}{Principle}
\newtheorem{problem}{Problem}
\newtheorem{proposition}{Proposition}
\newtheorem{question}{Question}
\newtheorem{remark}{Remark}
\newtheorem{theorem}{Theorem}
\newtheorem{vidu}{Ví dụ}
\usepackage[left=1cm,right=1cm,top=5mm,bottom=5mm,footskip=4mm]{geometry}
\def\labelitemii{$\circ$}
\DeclareRobustCommand{\divby}{%
	\mathrel{\vbox{\baselineskip.65ex\lineskiplimit0pt\hbox{.}\hbox{.}\hbox{.}}}%
}
\setlist[itemize]{leftmargin=*}
\setlist[enumerate]{leftmargin=*}

\title{Literary -- Văn Chương}
\author{Nguyễn Quản Bá Hồng\footnote{A Scientist {\it\&} Creative Artist Wannabe. E-mail: {\tt nguyenquanbahong@gmail.com}. Bến Tre City, Việt Nam.}}
\date{\today}

\begin{document}
\maketitle
\begin{abstract}
	This text is a part of the series {\it Some Topics in Advanced STEM \& Beyond}:
	
	{\sc url}: \url{https://nqbh.github.io/advanced_STEM/}.
	
	Latest version:
	\begin{itemize}
		\item {\it Literary -- Văn Chương}.
		
		PDF: {\sc url}: \url{https://github.com/NQBH/advanced_STEM_beyond/blob/main/literary/NQBH_literary.pdf}.
		
		\TeX: {\sc url}: \url{https://github.com/NQBH/advanced_STEM_beyond/blob/main/literary/NQBH_literary.tex}.
	\end{itemize}
\end{abstract}
\tableofcontents

%------------------------------------------------------------------------------%

\section{\href{https://www.facebook.com/EducatedMindsPage}{Facebook/Educated Minds}}

\begin{abstract}
	\textbf{mind} [n] a beautiful servant, a dangerous master.
\end{abstract}

\begin{enumerate}
	\item I don't lose people, people lose me. - Understand this
	\item Unplanned moments are always better than planned ones.
	\item ``\textit{Life isn't always fair}.
	
	\textit{Some people are born into better environments}.
	
	\textit{Some people have better genetics}.
	
	\textit{Some are in the right place at the right time}.
	
	\textit{If you're trying to change your life, all of this is irrelevant}.
	
	\textit{All that matters is that you accept where you are, figure out where you want to be, and then do what you can, today and everyday, to hold your head high and keep moving forward}.'' - Lori Deschene
	\item \textit{3 Easy ways to die}:
	\begin{itemize}
		\item Puff a cigarette daily.
		
		You will die 10 years early.
		\item Drink alcohol daily.
		
		You will die 30 years early.
		\item Love someone who doesn't love you back.
		
		You will die daily.
	\end{itemize}
	\item \textsc{Successful adulting}: Not stabbing someone when you really want to.
	\item \textbf{Check yourself}
	
	Sometimes you are the toxic person.
	
	Sometimes you are the mean, negative person you're looking to push away.
	
	Sometimes the problem is you.
	
	And that doesn't make you less worthy.
	
	Keep on growing.
	
	Keep on checking yourself.
	
	Keep on motivating yourself.
	
	Mistakes are opportunities.
	
	Look at them, own them. Grow from them and move on.
	
	Do better.
	
	Be better.
	
	You're human.
	
	It's okay.
	\item ``\textit{You stop explaining yourself when you realize people only understand from their level of perception.}'' - Jim Carrey
	\item ``It's impossible,'' said pride.
	
	``It's risky,'' said experience.
	
	``It's pointless,'' said reason.
	
	``Give it a try,'' whispered the heart.
	\item Relationship status:
	\begin{itemize}
		\item Single
		\item Taken
		\item IDK, cursed or something
	\end{itemize}
	\item Talk with people who make you see the world differently.
	\item ``\textit{It's fucked up how people get judged for being real and how people get loved for being fake.}'' - 2Pac
	\item Don't treat people as bad as they are, treat them as good as you are.
	\item People who tolerate me on a daily basis$\ldots$ they're the real heroes.
	\item Biggest lesson learned this year is probably to not give so much of yourself to people who will not do the same for you.
	\item Real friends talk shit to your face, and say nice things behind your back.
	\item Not everyone likes me, but not everyone matters.
	\item Learn to be along.
	
	Not everyone will stay forever.
	\item I want my daughter to be kind but I also want her to know that she can throat punch someone if she needs to.
	\item How many girls can proudly say, \textit{I have original eyebrows}.
	\item If we date, there will be moments when \textit{I will just stare at you and smile}, know that in those moments I'm appreciating everything about you.
	\item Rule number 1: Fuck what they think.
	\item Sad truth: some of your friends secretly hate you.
	\item Dear best friend,
	
	I love you more daily.
	
	I wish you could see yourself the way I see you and I wish you could love yourself the way I love you.
	
	And above all, I wish your life is everything you deserve because, in my opinion, you deserve the world.
	
	I will stand by you forever.
	
	My heart will always belong to you.
	\item According to research, people who are always late are usually more creative.
	\item If a woman says `do what you want'
	
	Do not do what you want.
	
	Stand still, do not blink, do not answer, don't even breathe, just play dead.
	\item Don't be afraid of losing people.
	
	Be afraid of losing yourself by trying to please everyone around you.
	\item Just because I \textit{can't} sing, doesn't mean I \textit{won't} sing.
	\item \textsc{Be brave}.
	
	Even if you're not, pretend to be.
	\item ``\textit{Life is like riding a bicycle. To keep your balance, you must keep moving.}'' - Albert Einstein, in a letter to his son Eduard, Feb 5, 1930
	\item My attitude is a result of your actions, so if you don't like my attitude, blame yourself!
	\item The oldest sibling is usually the most well behaved.
	\item Do you have a friend who is far away from you but is still your bestfriend?
	\item You are the \textit{books} you read, the \textit{movies} you watch, the \textit{music} you listen to, the \textit{people} you spend time with, the \textit{conversations} you engage in.
	
	Choose wisely what you feed your mind.
	\item Every now and then, a person with no agenda, no ulterior motive and no self-interest will take pleasure in helping you to succeed, grow and live your purpose.
	
	This person will operate in love, will seek no praise and will want nothing in return.
	
	This person is a gift.
	\item I hope to arrive to my death \textit{late, in love} and a little drunk.
	\item Don't confuse my personality with my attitude.
	
	My personality is who I am.
	
	My attitude depends on who you are.
	\item Everything is temporary; emotions, thoughts, people and scenery.
	
	Do not become attached, just flow with it.
	\item Both facing poverty, but the daughter still sees her \textit{dad} as a king and he sees her as his whole world.
	
	This is what money can't buy.
	\item Who else loves that earthy smell when rain hits the ground?
	\item Current attitude towards work
	\begin{itemize}
		\item Fuck this job, I'm out!
		\item Lol just kidding, I've got bills to pay.
	\end{itemize}
	\item People come and go in your life, but the right ones will always stay.
	\item Talking to you, laughing with you, being with you, changes my whole mood.
	\item When you meet me, you think I'm quiet.
	
	When you talk to me, you wish I was quiet.
	
	When you know me, you get scared when I am quiet.
	\item I like looking at you when you are not paying attention because I think that is when you are the most beautiful.
	
	Because you are not trying, you are just being you and to to me, that is better than any picture you or anyone else could ever take.
	\item Don't be too confident when someone tells you they like you.
	
	The real question is, \textit{until when?}
	
	Because just like seasons, people \textit{change}.
	\item Love is the answer.
	\item Easy to make 10 friends in a year.
	
	But 1 friend for 10 years is special.
	\item No one is coming to save you.
	
	This life is 100\% your \textit{responsibility}.
	\item Never \textit{hide} your bad side to make someone stay, \textit{show} your bad side and see who can stay.
	\item \textit{My goal} is to build a life I don't need a vacation from.
	\item Sometimes you don't get closure.
	
	You just move on.
	\item Work hard, stay disciplined and be patient.
	
	You time will come.
	\item If overthinking situations burned calories, I'd be dead.
	\item Karma comes after everyone eventually.
	
	You can't get away with screwing people over your whole life, I don't care who you are.
	
	What goes around comes around.
	
	That's how it works.
	
	Sooner or later the universe will serve you the revenge that you deserve.
	\item ``\textit{Having a soft heart in a cruel world is courage, not weakness.}'' - Katherine Henson
	\item When someone tells me to do something that I was already planning on doing, well now I am not doing it.
	\item ``\textit{She is rare because she is real}.'' - Mark Anthony
	\item If you expect the world to be fair with you because you are fair, you're fooling yourself.
	
	That's like expecting the lion not to eat you because you didn't eat him.
	\item Win in your mind and you will win in your reality.
	\item Love all, trust few.
	
	Not everything is real \& not everyone is true.
	\item \textsc{Suggest a book} for others to read.
	\item \textsc{We were all humans} until race disconnected us, religion separated us, politics divided us and wealth classified us.
	\item ``\textit{She likes you?}
	
	\textit{How do you know?}''
	
	``\textit{The eyes, Chico}.
	
	\textit{They never lie}.''
	\item \textbf{I'll always encourage the reckless texts confessing your feelings.}
	
	The kind where you throw your phone after hitting Send.
	
	I'll always encourage the horribly straightforward conversations at 3am when conversations get deep and you can't always put how you feel into words.
	\item True love has no expiry date.
	\item To people who reply late: If I wanted a reply after 2 days then I would have sent the message via a pigeon.
	\item There's difference between being liked and being valued.
	
	A lot of people like you.
	
	Not many value you.
	
	Be valued.
	\item The same boiling water that softens the potato hardens the egg.
	
	It's about what you're made of, not the circumstances.
	\item Most of my misunderstandings are because of my tone of voice.
	
	People think I'm angry even if I'm just explaining my point.
	\item ``\textit{Yesterday I was clever, so I wanted to change the world}.
	
	\textit{Today I am wise, so I am changing myself.}'' - Rumi
	\item I literally love being at home.
	
	In my own space.
	
	Comfortable.
	
	Not surrounded by people.
	\item I love places that make you realize how tiny you and \textsc{Your} problems are.
	\item Things we can't buy: Time, Happiness, Friends, Dreams, Hope, Love, Health.
	\item When you pray for someone, you are offering them the purest kind of love.
	\item Cheating is a choice, not a mistake.
	
	Loyalty is a responsibility, not a choice.
	\item A time will come in your life when some people will regret why they treated you wrong.
	
	Trust me, it will definitely come.
	\item The ego wants quantity but the soul wants quality.
	\item Yes I'm old school.
	
	I have good manners.
	
	I show others respect and I will always help those who need me.
	
	It's not because I'm old fashioned, it's because I was raised properly.
	\item You are rich, when you are content and happy with what you have.
	\item 1 of my greatest accomplishments is keeping my mouth shut even when I'm pissed off.
	\item Family is anyone who loves you unconditionally.
	\item Saying what you feel.
	
	It's not being rude.
	
	It's called being real.
	\item Deep conversation with the right people are priceless.
	\item Always cherish the girl who is with you when you have nothing.
	\item I need a coffee, a vacation, and a bag full of cash.
	
	That's all.
	\item It's nice when someone remembers small details about you.
	\item God removes people from your life because he heard conversations that you didn't hear.
	\item What's the most important thing you've done this year?
	
	\textit{Survived}$\ldots$
	\item Before you judge me, make sure you're perfect.
	
	If you're not, then shut up.
	\item What is salary?
	
	Something which comes like Tortoise and goes like Rabbit.
	\item You get what you focus on, so focus on what you want.
	\item The wrong one will find you in peace and leave you in pieces, but the right one will find you in pieces and lead you to peace.
	\item If you love someone, let them sleep.
	\item I'm not picky.
	
	I just know what I want.
	\item The best thing about the worst time of your life is that you get to see the \textit{true colors} of everyone.
	\item ``\textit{Exterior beauty, without the depth of a kind soul is merely decoration}.'' - Vanessa Quintero
	\item Every master was once a beginner.
	\item The older I get the more I appreciate being home doing absolutely nothing.
	\item I still watch cartoons and I don't care how old I am.
	\item - Wooo! It's Friday! You excited?!
	
	- Any excitement for `Friday' only proves that you're an unwitting slave to this social construct we call `the week'.
	
	Friday is nothing but a false hope, designed to distract you from noticing the machine silently oppressing you.
	
	You're just a hamster on a wheel linda, too excited by `Friday' to notice you're in a cage.
	
	- This is why nobody likes you Steve.
	\item No matter how nice your pictures are or how real your quotes are$\ldots$ there are some people who will never hit the like button just because it's you.
	\item The smartest thing a woman can ever learn, is to never need a man.
	\item If you find someone who makes you smile, who checks up on you often to see if you're okay, who watches out for you and wants the very best for you, \textbf{don't let them go}.
	
	Keep them close and don't take them for granted.
	
	People like that are hard to find.
	\item ``\textit{If my eyes could show my soul, everyone would cry when they saw me smile}.'' - Kurt Cobain
	\item People who are always Online, are the loneliest people.
	\item Sometimes to go far you must go alone.
	\item Obstacles do not block the path.
	
	They are the path.
	\item Never argue with someone who believes their own lies.
	\item Do you have a friend who is far away from you but is still your bestfriend?
	\item ``\textit{Go for someone who is proud to have you}.'' - Frank Ocean
	\item ``\textit{I have absolutely no desire to fit in}.'' - Coach Tim
	\item \textbf{Askhole} [n] A person who constantly asks for your advice, yet always does the exact the opposite of what you told them.
	\item Sometimes you have to keep your good news to yourself.
	
	Everybody is not genuinely happy for you.
	\item Fall in love with someone who wants you, who waits for you, who understands you even in the madness: someone who helps you, and guides you, someone who is your support, your hope.
	
	Fall in love with someone who talks with you after a fight.
	\item Sometimes I feel like giving up, and then I remember I have a lot of motherfuckers to prove wrong.
	\item I like people I can have comfortable silences with.
	\item I like people with whom I can have comfortable silences.
	\item We're all bad in someone's story.
	\item Dear music, thanks for always clearing my head, healing my heart, and lifting my spirit.
	\item I don't care what anyone thinks of me.
	
	Except dogs, I want dogs to like me.
	\item Be Honest: Are you missing someone right now?
	\item The number 1 mistake many women make is wasting years of their life waiting for a man to change.
	\item I had to forgive a person who wasn't even sorry$\ldots$ that's strength.
	\item Let them hate.
	
	Just make sure they spell your name right.
	\item The best 6 doctors: Sunshine. Water. Rest. Air. Exercise.
	\item ``\textit{To all the people who are loving and kind to me.}
	
	\textit{Thank you for all the sunshine you bring into my life}.'' - Brigitte Nicole
	\item It's easy to take off your clothes and have sex.
	
	People do it all the time.
	
	But opening up your soul to someone, letting them into your spirit, thoughts, fears, future, hopes, dreams$\ldots$ that is being naked.
	\item I truly appreciate kindness.
	
	I appreciate people checking up on me.
	
	I appreciate a quick message, I appreciate those who ask if I'm okay, I appreciate every single person in my life who has tried to brighten my days.
	
	\textit{It's the little things that matter the most}.
	\item Do you have someone with whom you have completed more than 5 years of friendships?
	\item Life is more beautiful when you meet the right hairdresser.
	\item Yes I am a girl, and I'm not interested in makeup.
	
	Yes, We exist!
	\item Have you ever been too nice and ended up in a situation that could've been avoided if you just would've been an asshole.
	\item ``\textit{Holding a grudge doesn't make you strong; it makes you bitter.}
	
	\textit{Forgiving doesn't make you weak; it sets you free}.'' - Dave Willis
	\item ``\textit{Sometimes you have to love people from a distance}.'' - Robert Tew
	\item The world is not full of assholes, but they are strategically placed so that you'll come across one every day.
	
	Every. Day.
	\item ``\textit{The most important thing in communication is hearing what isn't said}.'' - Peter F. Drucker
	\item \textsc{Know your worth}: You must find the courage to leave the table if respect is no longer being served.
	\item The best conversations are from 12--3am.
	\item No one wants to support you at the start of your journey but once they see your success they all want to be friends again.
	\item I'm naturally irritated when I 1st wake up.
	
	You have to give me a few minutes to adjust.
	\item Always love your friends from your heart not from your Mood or Need.
	\item Real is rare and fake is everywhere.
	\item Life is weird.
	
	1st you want to grow up, then you want to be a kid again.
	\item You know too much psychology when you can't get mad because you understand everyone's reasons for doing everything.
	\item It's blessing to have a friend with same level of dirty mind.
	\item Have you ever looked at someone and hoped they stay in your life forever?
	\item Shit happens.
	
	Every day.
	
	To everyone.
	
	The difference is in how people deal with it.
	\item I either keep it all inside, or say exactly how I feel with no filter.
	
	There is no in between.
	\item I don't have time to google Lyrics.
	
	I sing what I hear.
	\item If a cat crosses road it doesn't mean bad luck.
	
	It means that the cat is going somewhere.
	\item Weak people revenge.
	
	Strong people forgive.
	
	Intelligent people ignore.
	\item ``I want you back'' - Me talking to the money that I spent for nothing
	\item A person becomes 10 times more attractive not by their looks but by their acts of kindness, love, respect, honesty, and loyalty they show.
	\item Lucky are those who find \textbf{a true loyal friend} in this fake world.
	\item Stop following the crowd$\ldots$
	
	They are lost.
	\item Everyone has a friend during each stage of life.
	
	But only lucky ones have the same friend in all stages of life.
	\item You can say sorry.
	
	But you can't change the story.
	\item ``\textit{The hardest decisions in life is whether to ``walk away'' or ``try harder''.}'' - Ziad K. Abdelnour
	\item Sometimes \textit{home} isn't 4 walls, it's 2 eyes and a heartbeat.
	\item Every pain gives a lesson and every lesson changes a person.
	\item If you're funny, you're automatically more attractive.
	
	Beauty fades but \textit{sarcasm is forever}.
	\item ``\textit{Soon you realize that many people will love the idea of you but will lack the maturity to handle the reality of you}.'' - Reyna Biddy
	\item A moment of silence for those who hate us but can't unfriend us because they afraid of not knowing what's happening in our lives.
	\item ``\textit{There will be very painful moments in your life that will change your entire world in a matter of minutes.}
	
	\textit{These moments will change you}.
	
	\textit{Let them make you \textsc{stronger, smarter}, and \textsc{kinder}.}
	
	\textit{But don't you go and become someone that you're not}.
	
	\textit{Cry}.
	
	\textit{Scream if you have to}.
	
	\textit{Then you straighten out that crown and keep moving}.'' - Erin van Vuren
	\item I just want to
	\begin{itemize}
		\item take photos
		\item drink coffee
		\item travel the world
		\item meet new people
		\item be happy
	\end{itemize}
	\item Re-set, Re-adjust, Re-start, Re-focus$\ldots$
	
	As many times as you need to.
	\item If you have just 3 tickets for world tour.
	
	Choose that 2 friends with whom you want to travel the world.
	\item Learn when to be aggressive and when to be patient.
	\item \textsc{Educate yourself}.
	
	When a question about a certain topic pops up, google it.
	
	Watch movies and documentaries.
	
	When something sparks your interest, read about it.
	
	Read read read.
	
	Study, learn, stimulate your brain.
	
	Don't just rely on the school system, educate that beautiful mind of yours.
	\item And then it happens$\ldots$
	
	1 day you wake up and you're in this place.
	
	You're in this place where everything feels right.
	
	You heart is calm.
	
	Your soul is lit.
	
	Your thoughts are positive.
	
	Your vision is clear.
	
	You're at peace, at peace with where you've been, at peace with what you've been though and at peace with where you're headed.
	\item I never cared about the material things someone could give me.
	
	I care about time, attention, honesty, loyalty, and effort.
	
	Those gifts mean more than anything money could buy.
	\item A year changes you a lot.
	\item With great power comes great electricity bill.
	\item Some talk to you in their free time, and some free their time to talk to you.
	
	Learn the difference.
	\item ``Dead people receive more flowers than the living ones because regret is stronger than gratitude.'' - Anne Frank
	\item - Where did you find that?
	
	I've been search for it everywhere.
	
	- I created it [happiness] myself.
	\item No more expectations, just gonna go with the flow and whatever happens, happens.
	\item Be the reason someone feels, seen, heard, and supported.
	\item Karma said: If you break someone's heart and they still talk to you with the same excitement and respect, believe me, they really love you.
	\item Birds born in a cage think flying is an illness.
	\item An English professor wrote the words:
	
	``A woman without her man is nothing'' on the chalkboard and asked the students to punctuate it correctly.
	
	All of the makes in the class wrote: ``A woman, without her man, is nothing.''
	
	All of the females in the class wrote: ``A woman: without her, man is nothing.''
	
	Punctuation is powerful.
	\item ``To be is to do.'' - Socrates
	
	``To do is to be.'' - Kant
	
	``Do be do be do'' - Scooby Doo
	\item Be a good person, but don't waste time to prove it.
	\item Friends come and go like the waves of the ocean, but true ones stay like an octopus on your face.
	\item People come and go in your life, but the right ones will always stay.
	\item Have you ever \textit{lost all respect} for someone?
	
	Like, you don't hate them but you don't feel the need to associate yourself or say anything to them anymore.
	\item Sometimes in life we just need a hug$\ldots$ no words, no advice, just a bug to make you feel you matter.
	\item Raise your hand if you need a break from life and go on a trip far away from society and worries.
	\item Normal People: 1 episode per night
	
	Me: 1 season per night
	\item Unplanned trips are the best because planned trips never happen.
	\item You know what's beautiful?
	
	\textit{A real conversation}.
	\item ``\textit{Mondays are fine.}
	
	\textit{It's your life that sucks}.'' - R. Gervais
	\item You love her because her skin is white and soft?
	
	It's not Love, it's Dove.
	\item Every girl has male best friend in her life with whom she can share everything, without any fear.
	\item Not responding is also a response.
	\item Stupid is Knowing the truth, seeing the truth, but still believing the lies.
	\item Who is your 2 am friend?
	\item Life is too short to be normal.
	
	Stay weird.
	\item Boys who can cook are husband materials.
	\item That 1 girl who looks shy and innocent from outside, but she is big naughty from inside.
	\item - Hello, I'm here for a job interview.
	
	- Great, and do you have experience?
	
	- Yes, this is my 20th interview.
	\item \textbf{Don't complicate life}
	\begin{itemize}
		\item Missing somebody? \textbf{Call}
		\item Wanna meet up? \textbf{Invite}
		\item Wanna be understood? \textbf{Explain}
		\item Have questions? \textbf{Ask}
		\item Don't like something? \textbf{Say it}
		\item Like something? \textbf{State it}
		\item Want something? \textbf{Ask for it}
		\item Love someone? \textbf{Tell them}
	\end{itemize}
	Keep your life \textsc{simple}.
	\item A Daughter is 1 of the most beautiful gifts the world has to give.
	\item You don't always need a logical reason for doing everything in your life.
	
	Do it because you want to; because it's fun; because it makes you happy.
	\item Never forget who helped you out while everyone else was making excuses.
	\item Silence is better than unnecessary drama.
	\item Getting married after 30 is still beautiful.
	
	Starting a family after 35 is still possible.
	
	Buying your 1st House after 40 is still a boss move.
	
	Don't let people rush you with their timelines.
	\item I try to be a nice person, but sometimes my mouth doesn't cooperate.
	\item 1 of my dream is to go on a night camping trip with my favorite people.
	\item If everybody likes you, you have a serious problem.
	\item Jobs fill your pocket.
	
	Adventures fill your soul.
	\item Everything in life is temporary.
	
	So if things are going good, enjoy it because it won't last forever.
	
	And if things are going bad, don't worry.
	
	It can't last forever either.
	\item ``\textit{I can respect any person who can put their ego aside and say, I made a mistake, ``I apologize, and I am correcting the behavior''}.'' - sylvester mcnutt III
	\item Happiness is$\ldots$ Not setting alarm for the next day.
	\item Laughter is the best medicine, but if you're laughing for no reason you need medicine.
	\item Remember that you are \textsc{Water}.
	
	Cry. Cleanse. Flow. Let Go.
	
	Remember that you are \textsc{Fire}
	
	Burn. Tame. Adapt. Ignite.
	
	Remember that you are \textsc{Air}
	
	Observe. Breath. Focus. Decide.
	
	Remember that you are \textsc{Earth}
	
	Ground. Give. Build. Heal.
	
	Remember that you are \textsc{Spirit}
	
	Connect. Listen. Know. Be Still.
	\item Stay low key.
	
	Not everyone needs to know everything about you.
	\item I'm in love with this quote:
	
	``\textit{When you get what you want, that's God's direction, when you don't get what you want, that's God's protection}.''
	\item Once you have matured, you realize silence is more important than proving a point.
	\item - Morning: Tired.
	
	- Afternoon: Dying for a rest.
	
	- Night: Can't sleep.
	\item Sometimes, the nicest people you meet are covered in tattoos, and sometimes the most judgmental people you meet go to church on Sundays.
	\item ``\textit{You don't have a soul.}
	
	\textit{You are a soul}.
	
	\textit{You have a body}.'' - C. S. Lewis
	\item Choose people who choose you.
	\item ``\textit{The problem with the world is that the intelligent people are full of doubts and the stupid ones are full of confidence}.'' - Charles Bukowski
	\item The older I get, the more I realized I don't want to be around drama, conflict or stress.
	
	I want a cozy home, good food, and to be surrounded by happy people.
	\item Don't \textbf{become} who hurt you.
	\item A beautiful face will age and a perfect body will change, but a beautiful soul will always be a beautiful soul.
	\item All good things are wild \& free.
	\item If you focus on the hurt, you will continue to suffer.
	
	If you focus on the lesson, you will continue to grow.
	\item Blanket on: too hot
	
	Blanket off: too cold
	
	1 leg out: perfect
	\item ``\textit{Understand me}.
	
	\textit{I'm not like an ordinary world}.
	
	\textit{I have my madness, I live in another dimension and I do not have time for things that have no soul}.'' - Charles Bukowski
	\item Communicate.
	
	Even when it's uncomfortable or uneasy.
	
	1 of the best ways to heal, is simply getting everything out.
	\item Accepting your nose is the 1st step to self love.
	\item Stop looking for a partner.
	
	Focus on your goals and rebuilding your life.
	
	The right person will eventually find their way to you.
	\item Do (more) things that make you forget to check your phone.
	\item Hello, my name is human.
	
	And I came down from the stars.
	\item I suffer from that disorder where I speak the truth and it pissed people off.
	\item ``\textit{She fell}.
	
	\textit{She crashed}.
	
	\textit{She broke}.
	
	\textit{She cried}.
	
	\textit{She crawled}.
	
	\textit{She hurt}.
	
	\textit{She surrender}.
	
	\textit{And then$\ldots$ She rose again}.'' - Nausicaa Twila
	\item There are 2 reasons why we don't trust people.
	
	1st, we don't know them.
	
	2nd, we know them.
	\item ``\textit{I didn't wish him all the best, because that would be a lie, but I didn't wish him all the worst either}.
	
	\textit{I simply wished him whatever it was he deserved}.
	
	\textit{Now whether he deserved good things or bad things, was none of my business - that was between him, and karma}.'' - CiCi B.
	\item ``\textit{The child who is not embraced by the village will burn it down to feel its warmth}.'' - African Proverb
	\item Every time I trust somebody, they show me why I shouldn't.
	\item In a room full of art I'd still stare at you.
	\item \textsc{A serious life tip}:
	
	Even the nicest people have their limits.
	
	Don't push them too far and don't try to reach those limits, because the nicest people can also be the scariest assholes once they've had enough.
	\item ``\textit{Life has taught me that you can't control someone's loyalty}.
	
	\textit{No matter how good you are to them, doesn't mean they'll treat you the same}.
	
	\textit{No matter how much they mean to you, doesn't mean they'll value you the same}.
	
	\textit{Sometimes the people you love the most, turn out to be the people you can trust the least}.'' - Trent Shalton
	\item Love is Temporary.
	
	Food is Permanent.
	\item When you become really close to someone, you can hear their voice in your head when you read their texts.
	\item Ever locked yourself in the bathroom, cried, washed your face, and came back like nothing happened?
	\item Don't you just wanna sit on the roof of your house with someone, watch the night sky, and talk about how crazy life is.
	\item I lost many friends just because I stop texting them 1st.
	\item \textbf{How to be happy}
	
	Ignore people who think they know more about you than you do.
	\item \textbf{Did you know?}
	
	Every day, the heart creates enough energy to drive a truck 20 miles.
	
	In a lifetime, that is equivalent to driving to the moon and back.
	
	So, when you tell someone you love them ``to the moon and back'' you're essentially saying you will love them with all the blood your heart pumps your whole life, which I think is equally as meaningful.
	\item \textit{Fall in love} with someone who wants you, who waits for you, who understands you.
	
	Someone who helps you, and guides you, someone who is your support, your hope.
	
	Fall in love with someone who talks with you after a fight.
	
	Fall in love with someone who misses you and wants to be with you.
	
	Do not fall in love only with a body or with a face; or with the idea of being in love.
	\item Never fuck with someone who is not afraid to be alone.
	
	You will lose every single time.
	\item ``\textit{People who are spiritually minded tend to suffer from anxiety and depression more}.
	
	\textit{You know why?}
	
	\textit{Because their eyes are open to a world that is in need of repair}.
	
	\textit{They literally have an increased ability to feel the emotions of people around them}.'' - Osho
	\item When something good happens, \textbf{travel} to celebrate.
	
	If something bad happens, \textbf{travel} to forget it.
	
	If nothing happens, \textbf{travel} to make something happen.
	\item \textsc{Positive life}
	
	If you can't find true love, work hard, make money and enjoy your single life in peace.
	
	Nobody has ever died from being Single, but so many have died for being with the wrong partner$\ldots$
	
	Life is too short to be wasting your time with the wrong person.
	\item Good girls love bad boys
	\item When you see your Ex posting status about true love, loyalty \& emotions.
	\item Be good to people for no reason.
	\item Fill your life with adventures, not things.
	
	Have stories to tell, not stuff to show.
	\item Did you know?
	
	Chocolate comes from Cocoa, which is a tree.
	
	That makes it a plant$\ldots$
	
	So chocolate is a salad.
	\item Mr. Bean taught me 1 thing in life.
	
	Enjoy your own company instead of expecting someone else to make you happy.
	\item Girls be like$\ldots$ I don't have anything to wear.
	\item A person who trusts no one now, once trusted someone too much.
	\item What's a queen without her king?
	
	Well, historically speaking, more powerful.
	\item Maybe if we tell people the \textit{brain} is an \textit{app}, they'll start using it.
	\item If you don't do stupid things while you're young, you'll have nothing to laugh about when you're old.
	\item ``\textit{If you're pretty, you're pretty}.
	
	\textit{But the only way to be beautiful is \textbf{to be loving}}.
	
	\textit{Otherwise, it's just ``Congratulations about your face.''}.'' - John Mayer
	\item When life puts you in tough situations, don't say ``Why me'' say ``Try me''.
	\item If you kick me when I'm down, you better pray I don't get up.
	\item You learn nothing from life if you think you're right all the time.
	\item \textbf{A Perfect Man}
	\begin{itemize}
		\item wakes up at 5 am everyday
		\item exercises everyday
		\item makes his own bed
		\item cleans his room
		\item works sincerely
		\item does not touch alcohol
		\item helps in the kitchen
		\item does not indulge in night life
		\item is always punctual
		\item prays daily
		\item reads
		\item hits the bed at 9 pm sharp.
	\end{itemize}
	Where do you find such Perfect Men?
	
	Answer: Jail.
	\item Being a music lover with a terrible voice is the saddest part of my life.
	\item 98\% of my problems would be solved if I stopped overthinking things and calmed the hell down.
	\item ``\textit{Everyone you meet always asks if you have a \textsc{Career}, are \textsc{Married} or own a \textsc{House} as if life was some kind of grocery list}.
	
	\textit{But no one ever asks you if you are \textsc{Happy}}.'' - Heath Ledger
	\item Birthplace: Earth
	
	Race: Human
	
	Politics: Freedom
	
	Religion: Love
	\item \textbf{Signs of maturity}:
	\begin{itemize}
		\item[1.] Small talks no longer excite you.
		\item[2.] Sleep is better than a Friday night out.
		\item[3.] You forgive more.
		\item[4.] You become more open-minded.
		\item[5.] You respect differences.
		\item[6.] You don't force love.
		\item[7.] You accept heartaches.
		\item[8.] You don't judge easily.
		\item[9.] You sometimes prefer to be silent than to engage in a nonsense fight.
		\item[10.] Your happiness don't depend on people but on your inner self.
	\end{itemize}
	\item What's coming is better than what's gone.
	\item If you don't like me please don't pretend that you do.
	
	Ever.
	\item Imagine being loved the way you love.
	\item Someone once asked me: ``Why do you love music so much?''.
	
	I replied: Because it's the only thing that stays when everything and everyone is gone.
	\item ``\textit{Just because you don't share it on social media, doesn't mean you're not up to big things}.
	
	\textit{Live it and stay low key}.
	
	\textit{Privacy is everything}. - Denzel Washington
	\item Deleting your story after the target audience saw it.
	\item In life, what you really want will never come easy.
	\item If you're not losing friends, you're not growing up.
	\item To that friend who has never been tired in Listening to my dramas.
	
	Thank you I owe you a lot!
	\item You act like it's you against the world but it's really just you against yourself.
	\item What if you met the right person but at the wrong time.
	\item We all have that 1 friend who eats a lot but never gets fat.
	\item \textit{\textbf{Stop being offended} by a Facebook post, by a piece of art, by people displaying affection, or by what someone said to you}.
	
	\textit{\textbf{Be offended} by war, poverty, greed and injustice}. - Sue Fitzmaurice
	\item Have less.
	
	Be more.
	
	Do more.
	\item \textbf{Brother} [n] (bruhth-er) The person who is there when you need him; someone who picks you up when you fall; a person who sticks up for you when no one else will; a brother is always a friend.
	\item Be Honest: Do you still love your ex?
	\item ``\textit{I think people spend \textbf{too much time staring at screens} and not enough time drinking wine, tongue kissing, and dancing under the moon}.'' - Rachel Wolchin
	\item The only way to win with a toxic person is not to play.
	\item \textbf{Keep going.}
	
	No matter how stuck you feel.
	
	No matter how bad things are right now.
	
	No matter how many days you've spent crying.
	
	No matter how hopeless and depressed you feel.
	
	No matter how many days you've spent wishing things were different.
	
	I promise you won't feel this way forever.
	
	\textbf{Keep going}.
	\item \textbf{Earthing}: The process of absorbing earths free flowing electrons from it's surface through the soles of ones feet.
	\item \textbf{How to save your heart?}
	
	\textsc{Should}:
	\begin{itemize}
		\item Never expect.
		\item Never demand.
		\item Never assume.
	\end{itemize}
	\item \textsc{Know}:
	\begin{itemize}
		\item Your limits.
		\item Where you stand.
		\item Your role.
	\end{itemize}
	\item \textsc{Don't}:
	\begin{itemize}
		\item Get affected.
		\item Get jealous.
		\item Get paranoid.
	\end{itemize}
	\item \textsc{Just}: Go with the flow and stay happy.
	\item Thins are not always what they seem.
	
	The deer isn't crossing the road.
	
	The road is crossing the forest.
	\item Surround yourself with people who push you to do and be better.
	
	No drama or negativity.
	
	Just higher goals and higher motivation.
	
	Good times and positive energy.
	
	No jealousy or hate.
	
	Simply bringing out the absolute best in each other.
	\item Move in silence, only speak when it's time to say \textsc{Checkmate}.
	\item Once you start loving someone, it's very hard to stop$\ldots$
	\item Home is not a place, it's a feeling.
	\item Do you know that awesome feeling when you get into bed, fall right to sleep, stay asleep all night, and wake up feeling refreshed?
	
	Me neither.
	\item So much of our happiness depends on how we choose to look at the world.
	\item It was never the way he looked.
	
	Always the way she was.
	
	I could've fallen in love with my eyes closed.
	\item Robin William's friends: ``He was always happy.
	
	Everyone adored him.''
	
	Kate Spade's Dad: ``I just talked to her an hour before and she was planning a trip.
	
	She was just like her brand - happy, cheerful and full of color.''
	
	Anthony Bourdain's best friend: ``He loved his life and had this extraordinary ability to just connect with people.''
	
	So, let me say this really loud so the people in the back of the room can hear me$\ldots$
	
	\textsc{Sometimes you need to check on those who seem the strongest}.
	\item ``\textit{There is a misconception that Buddhism is a religion \& that you worship Buddha}.
	
	\textit{Buddhism is a practice, like Yoga, you can be a Christian \& practice Buddhism}.
	
	\textit{I met a Catholic priest who lives in a Buddhist Monastery in France}.
	
	\textit{He told me that Buddhism makes him a better Christian}.
	
	\textit{I love that}.'' - Thich Nhat Hanh
	\item Never judge people by their past.
	
	People learn.
	
	People change.
	
	People move on.
	\item Dear Girls:
	
	Before going into a relationship with any guy in this century.
	
	Take the boy's picture \& post it on Facebook, Twitter, \& Instagram with caption: ``\textit{Whose boyfriend is this?}''
	
	If no one claims him in a week than you're free to date him.
	\item Be careful who you push away$\ldots$
	
	Some of us don't come back.
	\item If it \textit{excites you} and \textit{scares you} at the same time, it might be a good thing to try.
	\item Classy is when you have a lot to say but you choose to remain silent in front of fools.
	\item Wearing unbranded cheap clothes doesn't mean you're poor.
	
	Remember: You have a family to feed.
	
	Not a community to impress.
	\item You're lucky if you've found a person who never gets tired of understanding your nonsense attitude.
	\item \textbf{I miss you}.
	
	I miss your voice.
	
	I miss your smile.
	
	I miss your smell.
	
	I miss your hugs.
	
	I miss your jokes.
	
	I miss how you made me feel.
	
	\textbf{I miss your everything}$\ldots$
	\item Tell my mistakes to me not to others.
	
	Because my mistakes are to be corrected by me, not by others.
	\item Stop editing your pics, what if you go missing?
	
	How can we find you if you look like Angelina Jolie on Instagram \& potato in real life.
	\item When you build in silence, people don't know what to attack.
	\item Sometimes you gotta play the tool, to fool the fool, who thinks they're fooling you.
	\item I'm actually a very nice person, until you piss me off.
	\item Make sure you check your spelling and grammar before post anything.
	
	Because there are so many jobless English professor on social media.
	\item I wish I met some people earlier, some a little later and some never at all.
	\item Being good is a mistake nowadays, people think you're Fake.
	\item To my best friends,
	
	1 day, all of us will get separated from each other.
	
	We will miss our conversations of everything and nothing and the dreams we had.
	
	Days, months and years will pass until this contact becomes rare.
	
	1 day, our children will see our pictures and ask: ``Who are these people?''
	
	And we will smile with invisible tears because a heart is touched with a strong word and you will say:
	
	``It was them that I had the best days of my life with.''
	\item I've only met 3 or 4 people that understand me.
	
	Everyone else assumes I'm either angry, sarcastic or just an asshole.
	\item Never forget 2 people in your life.
	\begin{itemize}
		\item[1.] The person who lost everything just to make you win. (Your Father)
		\item[2.] The person who was with you in every pain. (Your Mother)
	\end{itemize}
	\item Stand up for what you believe in, even if it means standing alone.
	\item Romeo died because of Juliet, Jack died because of Rose, stay single if you want to live.
	\item Train your mind to stay calm in every situation.
	\item \textit{Go for it}.
	
	Whether it ends good or bad, it was an experience.
	\item Never attribute to malice what can be explained by ignorance.
	
	People are far more stupid than they are evil.
	\item Me risking my job, career and future to get an extra 15 minutes of sleep.
	\item Mean World Syndrome is a phenomenon whereby violence-related content of mass media makes viewers believe that the world is more dangerous than it actually is.
	\item Find someone who enjoys your Madness.
	
	Not an Idiot who forces you to be normal.
	\item People will come and go in life, but the person in the mirror will be there forever: So be good to yourself.
	\item Old saying - Think before you speak.
	
	New saying - Google, before you post.
	\item I think \textit{1 of the greatest feelings} in the world is when someone openly tells you how much you mean to them.
	
	Stuff like that is so rare.
	\item I don't care if I'm selfish.
	
	After putting people 1st for the longest time and being disappointed I deserve to do whatever makes me feel happy.
	\item My 3 wishes:
	\begin{itemize}
		\item[1.] To earn money without working.
		\item[2.] To love without being hurt.
		\item[3.] To eat without getting fat.
	\end{itemize}
	\item I just wanna sit outside at night and talk about life with someone.
	\item 1 of the best feelings in the world is when you hug someone you love and they hug you back even tighter.
	\item \textbf{Petrichor} [n] the smell of earth after rain.
	\item Post an unpopular opinion you have and only comment \textit{agree} or \textit{disagree} under others.
	
	\textit{No arguing}.
	\item Don't consider my kindness as my weakness.
	
	The best in me is sleeping, not dead.
	\item It has been said that everlasting friends go long periods of time without speaking and never question the friendship.
	
	These friends pick up like they just spoke yesterday, regardless of how long it has been or how far away they live, and they don't hold grudges.
	
	They understand that life is busy, and you will always love them.
	\item \textbf{A beautiful day begins with a beautiful mindset.}
	
	When you wake up, take a second to think about what privilege it is to simply be alive and healthy.
	
	The moment you start acting like \textit{life is a blessing}, it will start to feel like one.
	\item In the end, we all just want \textit{someone that chooses us}.
	
	Over everyone else.
	
	Under any circumstances.
	\item When a man realizes that his woman is a teammate not an opponent, that's when his life changes.
	
	Don't break her down.
	
	Build her up and let her help you win.
	\item Not all storms come to disrupt your life, some come to clear your path.
	\item Be strong enough to stand alone, be yourself enough to stand apart, but be wise enough to stand together when the time comes.
	\item Some husbands hold their wife's hand in malls because if they have her hand, she'll go for shopping.
	
	It looks ``Romantic'' but it's actually ``Economic''.
	\item \textit{You are so missed and loved}.
	
	There isn't a day that goes by that I don't think about you.
	
	Although I can't see you, you are always in my heart.
	
	\textit{Rest in Peace}.
	\item Don't ignore the effort of a person who tries to keep in touch.
	
	It's not all the time someone cares.
	\item Be kind to unkind people, they need it the most.
	\item Never hurt that person whose only intention was to see you happy.
	\item Silence is the best reply to a fool.
	\item Don't love too deeply until you're sure that the other person loves you with the same depth.
	
	Because the depth of your love today is the depth of your wound tomorrow.
	\item ``\textit{I have no special talent}.
	
	\textit{I am only passionately curious}.'' - Albert Einstein
	\item \textbf{Be happy} in front of people who don't like you - it kills them.
	\item Spend time with people who are good for your mental health.
	\item A real man knows that 1 woman is enough.
	\item You can't go back and change the beginning, but you can start where you are and change the ending.
	\item Not everyone is meant to be in your future.
	
	Some people are just passing through to teach lessons in life.
	\item If all religions teach peace, why can't all religions achieve peace?
	\item \textit{Angels} exist but sometimes they don't have wings and are called \textit{friends}.
	\item Veni. Vidi. Amavi.
	
	We came. We saw. We \textit{loved}.
	\item I'm innocent but my best friend taught me about dirty stuffs.
	\item To all the people who make me happy: thank you for being in my life!!!
	\item Stuck between ``I need to save money'' and ``You only live once''.
	\item ``\textit{People will teach you how to love by not loving you back}.
	
	\textit{People will teach you how to forgive by not apologizing}.
	
	\textit{People will teach you kindness by their judgment}.
	
	\textit{People will teach you how to grow by remaining stagnant}.
	
	\textit{Pay attention when you're going through painful and mysterious times}.
	
	\textit{Listen to the wisdom life is trying to teach you}. - Meredith Marple
	\item \textit{Keep your distance} from people who will never admit they are wrong and always try to make you feel like it's all your fault.
	\item I have no time to battle egos and small minds.
	\item If you aren't happy single, you won't be happy taken.
	
	Happiness comes from within, NOT from men.
	\item A religious person will do what he is told$\ldots$ no matter what is right$\ldots$ whereas a spiritual person will do what is right$\ldots$ no matter what he is told.
	\item Leadership is not about being the best.
	
	Leadership is about making everyone else better.
	\item ``\textit{How amazing it is to find someone who wants to hear about all the things that go on in your head}.'' - Nina LaCour
	\item I think as you grow older you look for very different things in people.
	
	\textbf{Honesty. Loyalty. Integrity.}
	
	But most of all, you look for someone who will stand right by your side when the walls start crumbing and the fires rage within.
	
	They are right there, and at that moment, you know they've got you. - The better man project
	\item Sometime those who don't socialize much aren't actually anti-social.
	
	They just have no tolerance for drama, stupidity, and fake people.
	\item I like being at home in my own little world.
	
	The real world has too many assholes.
	\item Does anyone else tell their pets, ``I'll be back soon'' when they leave the house or is that just me?
	\item \textbf{If a door closes, quit banging on it.}
	
	Whatever was behind it was not meant for you.
	
	Consider that perhaps the door was closed because you're worth so much more than what was on the other side.
	\item Not everyone will understand your journey.
	
	That's okay.
	
	You're here to live your life, not to make everyone understand.
	\item I asked an old man, ``\textit{Even after 95 years, you still call your wife Darling, Honey and Love. What's the secret?}''
	
	Old man: ``\textit{I forgot her name 10 years ago and I'm scare to ask her}.''
	\item Whatever you do, good or bad, people will always have something negative to say about you and that's life.
	\item Me: \textit{I want to travel}.
	
	Bank account: \textit{Where? To work?}
	\item Can I just be a little kid again?
	
	No stress, no worries, just fun.
	\item \textbf{Never beg for love.}
	
	\textbf{Never beg} someone to be with you.
	
	\textbf{Never beg} someone to come back to stay.
	
	\textbf{Never beg} for attention, commitment, affection, time and effort.
	
	You should never have to ask to feel wanted.
	
	Begging is demanding and degrading.
	
	If someone doesn't willingly give you these things, with their arms wide open, they aren't worth it.
	
	No one, under any circumstances, is ever worth begging for.
	\item The world is gonna judge you no matter what you do, so live your life the way you want to.
	\item ``\textit{You could give some people a drop of water, and they'd still appreciate you}.
	
	\textit{You could give other people the entire ocean, and they'd still take you for granted}. - Yasmin Mogahed
	\item Silence isn't empty, it's full of answers.
	\item That moment when your best friend sends you a voice message \& you're running everywhere to search earphones cause you don't trust that idiot.
	\item When someone calls me ``lazy'': Thank you. I know.
	\item The best thing about being single is you can have crush on anyone, anywhere, anytime.
	\item 1 of the best things about getting older: knowing someone is an asshole before they even speak.
	\item Don't wait until you are rich to be happy.
	
	Happiness is free!
	\item The fact that jellyfish have survived for 650 million years without brains, gives hope to many people.
	\item \textit{The most beautiful things in life are not things}.
	
	They are people, places, memories and pictures.
	
	They're feelings and moments, and smiles, and laughter.
	\item \textit{Psychology says}: When you focus on problem, you will have more problems.
	
	When you focus on possibilities, you will have more opportunities.
	\item ``\textit{Don't let anyone tell you that your independence is the reason for you being single}.
	
	\textit{Your strength as a woman isn't the cause for your loneliness}.
	
	\textit{You're alone because you're rather not entertain a weak man}.'' - r.h. Sin
	\item ``\textit{Look 1st for honesty in a smile, not beauty}.'' - J.H. Hard
	\item ``\textit{Beware of false knowledge; it is more dangerous than ignorance}.'' - George Bernard Shaw
	\item Only you and you alone can change your situation.
	
	Don't blame it on anything or anyone.
	\item \textit{Parked car conversations} are low key therapy sessions.
	\item \textit{So don't hurt her, don't change her, don't analyze and don't expect more than she can give}.
	
	\textit{Smile when she makes you happy, let her know when she makes you mad, and miss her when she's not there}.'' - Bob Marley
	\item Don't wait for things to get easier, simpler, better.
	
	Life will always be complicated.
	
	Learn to be happy right now.
	
	Otherwise, you'll run out of time.
	\item Car rides by yourself with loud music are good for the soul.
	\item ``\textit{How beautiful it is to find someone who asks for nothing but your company}.'' - Brigitte Nicole
	\item The best things in life aren't things.
	\item 4 things you can't get back:
	\begin{itemize}
		\item The \textit{stone} after it's \textit{thrown}.
		\item The \textit{word} after it's \textit{said}.
		\item The \textit{occasion} after it's \textit{missed}.
		\item The \textit{time} after it's \textit{gone}.
	\end{itemize}
	\item ``Someone told me the other day that he felt bad for single people because they are lonely all the time.
	
	I told him that's not true I'm single and I don't feel lonely.
	
	I take myself out to eat, I buy myself clothes.
	
	I have great times by myself.
	
	Once you know how to take care of yourself company becomes an option and not a necessity.'' - Keanu Reeves
	\item To care for those who once cared for us is 1 of the highest honors.
	\item She was simple like quantum physics.
	\item If you feel like you're losing everything, remember that trees lose their leaves every year and they still stand tall and wait for better days to come.
	\item Fuck your shoe collection.
	
	Show me your book collection.
	\item Sometimes God sends an ex back into your life to see if you are still stupid.
	\item You don't need to be liked by everyone, not everyone has a good taste.
	\item ``\textit{To me, Nature is a place where you retreat whenever you feel exhausted or sad}.
	
	\textit{I'm more happy when I'm surrounded by the sound of birds than sound of people}.'' - Keanu Reeves
	\item When you are angry, be silent.
	\item Sometimes I just agree with people so that they can stop talking.
	\item If ``Plan A'' didn't work.
	
	The alphabet has 25 more letters!
	
	Stay Cool.
	\item It is important to have friends who are proud of you when you get a new job or learn to bake or do big things, but it is also important to have friends who are proud of you when you \textit{get out of bed and take a shower}.
	\item When you die and hear your wife saying ``I've never made him sad.''
	\item What makes you different makes you beautiful. - Book of Prosperity
	\item Small steps in the right direction are better than big ones in the wrong direction.
	\item ``\textit{$\ldots$ and oh, how they \textbf{love the woman who won't take any shit} from anyone, until they realize that she won't take any shit from them, either}.'' - CiCi B.
	\item 1 day, you will be at the place you always wanted to be.
	
	Don't stop believing.
	\item You can't force people to stay in your life.
	
	Stay is a choice, so be thankful to the people who chose you.
	\item Stay strong, make them wonder how you are still smiling.
	\item ``\textit{My life isn't perfect but I'm thankful for everything I have}.'' - Note from Life
	\item Only once you have carried your own water, will you learn the value of every drop.
	\item ``Nothing stronger than a broken man rebuilding himself.'' - Via (Dark Secrets)
	\item If you fail, never give up because F.A.I.L means ``First Attempt In Learning''; End is not the end, in fact E.N.D. means ``Effort Never Dies''.
	
	If you get No as an answer, remember N.O. means ``Next Opportunity''.
	\item Don't wait until you've reached your goal to be proud of yourself, be proud of every step you take toward teaching that goal.
	\item Family 1st no matter what.
	\item When my circle got smaller, my vision got clearer.
	
	There's strength in loyalty, not numbers.
	\item People who do not understand your silence will never understand your words.
	\item Behind every successful man there is a woman. - Fact
	
	Because women don't follow unsuccessful men. - Universal Truth
	\item \textit{What's your relationship status?}
	
	\textsc{Single}
	
	\textsc{In a relationship}
	
	\textsc{Crushing}
	
	\textsc{Married}
	
	\textsc{It's complicated}
	
	\textsc{Broken hearted}
	\item You will never speak to anyone more than you speak to yourself in your head, be kind to yourself.
	\item Find happiness in simple things.
	\item Happiness is not about getting all you want, it is about enjoying all you have.
	\item ``\textit{That woman has loved me skinny, she's loved me fat}.
	
	\textit{She's love me bald, she's loved me hairy}.
	
	\textit{That woman, I know, loves me}.
	
	\textit{So, I'm a lucky man}. - Tom Hanks about his wife, Rita Wilson
	\item \textbf{Be afraid of the quiet ones.}
	
	They are the ones who actually think.
	\item Never stop being a good person because of bad people.
	\item \textit{Good sense of humor}, \textit{dirty mind} and \textit{beautiful heart}.
	
	Deadly combination.
	\item I just want to
	\begin{itemize}
		\item take photos
		\item drink coffee
		\item travel the world
		\item meet new people
		\item be happy
	\end{itemize}
	\item \textit{Admit it}, sometimes you say your true feeling through jokes.
	\item ``\textit{1 man's \emph{I'm not ready} is another man's \textbf{I knew the second I saw her}.}'' - Meredith Marple
	\item Give$\ldots$ Yourself$\ldots$ Time.
	\item A relationship should be about helping each other deal with the stress the world brings.
	
	Not adding unnecessary stress to each others lives.
	\item I am not open to many people.
	
	I'm usually quiet and I don't really like attention.
	
	So if I like you enough to show you the real me, you must be very special.
	\item Why am I the type of person that still believes someone's a good person even when they've shown me in every way that they're not?
	\item No matter how good a person you are, you are evil in someone's story.
	\item ``\textit{I was raised to treat the janitor with the same respect as the CEO}.'' - Tom Hardy
	\item Next time a stranger talks to me when I am alone, I will just look at them shocked and just whisper quietly ``You can see me?''
	\item Sometimes I wish I was a bird, so I could fly over certain people and shit on their heads.
	\item \textsc{Listen} and \textsc{Silent} are spelled with the same letters.
	
	Think about it.
	\item I'm not bossy!
	
	I have skills$\ldots$
	
	leadership skills!!
	
	Understand?
	\item Don't be afraid to be open-minded.
	
	Your brain isn't going to fall out.
	\item Just because I give you advice, it doesn't mean I know more than you.
	
	It just means I've done more stupid shit.
	\item When you're laughing so hard \& you try to stop, but you look at the person and laugh again.
	\item I hold in a lot.
	
	When I'm in pain, I don't like to worry other people.
	
	No matter how hard I cry, or how much somebody asks, my answers will be, I'm fine.
	
	Even if it's not true.
	
	Do you do this too?
	\item I hate when people ask me ``\textit{Why are you so quiet?}''.
	
	Because I am.
	
	That's how I function.
	
	I don't ask others ``\textit{Why are you so noisy?}
	
	\textit{Why do you talk so much?}''
	
	It's rude.
	\item If you're not amazed by the stars on a clear night, then we won't work.
	\item ``The 2 most important days in your life are the day you are born and the day you find out why.'' - Mark Twain (1835--1910)
	\item Me: \textit{I'm blessed, I have someone who loves me \& fights for me}.
	
	Friend: \textit{Who?}
	
	Me: \textit{My white blood cells}$\ldots$
	\item ``\textit{Sometimes life has a cruel sense of humor, giving you the thing you always wanted at the worst time possible}.'' - Lisa Kleypas
	\item Loyal people get hurt the most.
	
	True or False?
	\item ``\textit{We all have inner demons to fight}.
	
	\textit{We call these demons `fear', and `hatred', and `anger'}.
	
	\textit{If you don't conquer them, then a life of a hundred years$\ldots$ is a tragedy}.
	
	\textit{If you do, a life of a single day can be a triumph}.'' - Yip Man
	\item Sometimes beauty depends on phone quality
	\item Sometimes people pretend you're a bad person so they don't feel guilty about the things they did to you.
	\item Me driving home from work knowing I'm only going home to eat and sleep so I can do it all again tomorrow.
	\item I said I'm able to work under pressure, not that I will die for your company.
	\item There's a message in the way a person treats you.
	
	\textbf{Just listen$\ldots$}
	\item Find someone who will \textit{love your soul} more than your body.
	\item They laugh at me because I'm different I laugh at them because they are all same.
	\item ``\textit{When a person tells you that you hurt them, you don't get to decide that you didn't}.'' - Louis C.K.
	\item Don't tell anyone ``I hate you'' directly, say ``You are the Monday of my life''.
	\item What the fuck$\ldots$!
	
	Psychotherapist is 1 word! 1 word!!! [Psycho the rapist]
	\item Imagine if instead of putting tombstones, we planted trees on grave, we would never have to worry about deforestation.
	\item Who else loves that earthy smell when rain hits the ground?
	\item ``\textit{If we fully understood the power of sexual energy we would refrain from casual sex}.
	
	\textit{Sex is sacred and to be shared with authentic purity of both partners}.
	
	\textit{Sexual energy is intense and can heal the universe through the vibration when 2 people join together and share their souls together}.
	
	\textit{DNA is exchanged during sex}.
	
	\textit{You imprint yourself on another}.
	
	\textit{Mindful sex is important}.'' - Diane L
	\item Half the world is starving.
	
	The other half is trying to lose weight.
	\item \textbf{Life is the most difficult exam.}
	
	Many people fail because they try to copy others, not realizing that everyone has a different question paper.
	\item Live a healthy lifestyle.
	
	Choose drinks with minimal amounts of sugar.
	\item I see humans but no humanity.
	\item ``\textit{I don't love studying}.
	
	\textit{I hate studying}.
	
	\textit{I like learning}.
	
	\textit{Learning is beautiful}.'' - Natalie Portman
	\item \textbf{Do things} that make you forget to check your phone.
	\item Note to self: stop missing people who don't even waste a second thinking about you.
	\item Dirty minded friends make life so much funnier.
	\item Yes, I overthink but I also \textit{over-love}$\ldots$
	\item It's never too late for a new beginning in your life.
	\item Do you ever think about going somewhere nobody knows you and starting a new life?
	\item I don't have a short temper, I just have a quick reaction to bullshit.
	\item Things take time.
	
	So just be patient.
	\item You learn nothing from life if you think you're right all the time.
	\item Constantly torn between ``if it's meant to be, it will be'' and ``if you want it, go and get it.''
	\item If you are not doing what you love, you are wasting your time.
	\item ``\textit{The problem is people are being hated when they are real, and are being loved when they are fake}.'' - Bob Marley
	\item Always love your friends from your heart, not from your mood or need.
	\item Followers will never know how hard the leader tries to create path.
	\item 1 day, you will be at the place you always wanted to be.
	
	Don't stop believing.
	\item 6 months from now you can be in a completely different space, mentally, spiritually and financially.
	
	Keep working and believing in yourself.
	\item Life's too short to tolerate shit that doesn't make you happy.
	\item The best sign of a \textit{happy relationship} is no sign of it on social media.
	\item Dear Beach, I think of you all the time$\ldots$
	\item If you have a friend with whom you can speak and open your heart anytime.
	
	Trust me you're blessed.
	\item \textbf{Hugs are actually so underrated}, especially those hugs that are so tight you can literally feel the other person's heartbeat and for a moment everything feels so calm and safe like nothing can hurt you.
	\item I never make the same mistake twice, I make it like 5 or 6 times, just to be sure.
	\item Don't chase, don't beg, don't be desperate, don't stress, just relax.
	
	When you relax, it will come to you.
	
	\textbf{Make you wants want you.}
	\item A \textbf{mistake} that makes you \textbf{humble} is better than an \textbf{achievement} that makes you \textbf{arrogant}.
	\item 1 loyal friend is worth more than 1000 fake ones.
	\item Dear life, I've always been strong ever since.
	
	But can you please be easy on me sometimes?
	\item ``\textit{I've learned that home isn't a place, it's a feeling}.'' - Cecelia Ahern
	\item A soulmate is a friend who never leaves.
	\item Don't believe everything you think.
	\item 1 of the best pleasures in life$\ldots$ is to read a book in total silence.
	\item A good teacher is like a candle - it consumes itself to light the way for others.
	\item To have someone understand your mind is a different kind of intimacy.
	\item Listen, smile, agree, and then do whatever the fuck you were gonna do anyway.
	\item 1 of the best lessons you can learn in life is to master how to \textbf{remain calm.}
	\item ``\textit{When it hurts - observe}.
	
	\textit{Life is trying to teach you something}.'' - Anita Krizzan
	\item Take being called crazy as a Compliment.
	
	It means you've found the courage to be yourself when so many others have not.
	\item If a friendship lasts longer than 7 years, psychologists say it would probably last a lifetime.
	\item It's better to wait long than to marry wrong.
	\item ``\textit{If tomorrow, women woke up and decided they really like their bodies, just think how many industries would go out of business}.'' - Dr. Gail Dines
	\item Having someone who can handle all your moods is such a blessing.
	\item I'm at a point in my life where I just want to be very quiet.
	\item Some people act like they are trying to help you.
	\item \textit{You know what's attractive?}
	
	Seeing people change for that 1 person.
	
	Whether it's cutting down their drinking habits or to stop doing drugs and maybe even something small like to stop swearing.
	
	It just proves that when it comes down to it, they would do whatever it takes to see that person happy.
	
	Stopping habits is a hard thing to do and to see people actually doing that for someone's love is really attractive to me.
	\item Sisters are the 2nd version of Mother.
	
	She will give you advice, protect you and scold you, but will never tolerate if someone hurts you.
	\item If you have at least 1 person genuinely supporting you$\ldots$ \textbf{you're blessed.}
	\item \textit{12 things to remember.}
	\begin{itemize}
		\item[1.] The past cannot be changed.
		\item[2.] Opinions don't define your reality.
		\item[3.] Everyone's journey is different.
		\item[4.] Things always get better with time.
		\item[5.] Judgments are a confession of character.
		\item[6.] Overthinking will lead to sadness.
		\item[7.] Happiness is found within.
		\item[8.] Positive thoughts create positive things.
		\item[9.] Smiles are contagious.
		\item[10.] Kindness is free.
		\item[11.] You only fail if you quit.
		\item[12.] What goes around comes around.
	\end{itemize}
	\item I wish my life had background music so I could understand what the hell is going on.
	\item Sometimes the people with the greatest potential often take the longest to find their path because their sensitivity is a double edged sword - it lives at the heart of their brilliance, but it also makes them more susceptible to life's pains.
	
	Good thing we aren't being penalized for handing in our purpose late.
	
	The soul doesn't know a thing about deadlines.
	\item May your vibes shift the whole damn frequency of the room when you walk in.
	\item Scientists have found that babies sometimes cry at night to prevent their parents from making another baby.
	\item A strong woman will automatically stop trying if she feels unwanted.
	
	She won't fix it or beg, she'll just walk away.
	\item I found the key to happiness.
	
	Stay away from idiots.
	\item We are energy and energy never dies, it just changes form.
	\item As you get older, you'll realize that a \$300 watch and a \$30 watch both tell the same time.
	
	A Miachel Kors wallet and a Forever 21 wallet hold the same amount of money.
	
	A \$300,000 house and a \$100,000 house host the same loneliness.
	
	A Ford will also drive you as far as a Bentley.
	
	True happiness is not found in materialistic things.
	
	It comes from the love an laughter found with each other.
	
	Stay humble$\ldots$ the holes dug for us in the ground are all the same size.
	\item Once you feel you're avoided by someone, never disturb them again.
	\item ``\textit{There's an herb for every system, every organ, every gland, and every tissue of our body}.
	
	\textit{Mother nature has put medicine in our food}.'' - Bob Marley
	\item Her intuition was her favorite superpower.
	\item Facebook friends: 600
	
	Real friends: 20
	
	Hard time friends: 2
	\item If you're single
	\begin{itemize}
		\item[1.] You're either talking to someone
		\item[2.] Stuck on an ex
		\item[3.] Chasing someone who's taken
		\item[4.] Ignoring someone who wants you
		\item[5.] Too happy on your own
	\end{itemize}
	\item Can a boy and girl be a best friend?
	\item You don't always need a plan.
	
	Sometimes you just need to breathe.
	
	Trust.
	
	Let go, and see what happens.
	\item Everything depends on the way \textsc{You} see!
	\item Some people are so Fake that if you look behind their necks you'll see: \texttt{Made in China}.
	\item ``\textit{The coolest people I've ever met have the most colorful pasts, they've lived lives of risk, made bad choices, learned lessons, explored, and they're not afraid of being real}.
	
	\textit{Tattered tapestries woven of similar threads, they've my kind of people}.
	
	\textit{My favorite shades of crazy}.'' - Stephen L. Lizotte
	\item ``\textit{Take a \textbf{lover} who looks at you like maybe you are \textbf{magic}.}'' - Frida Kahlo
	\item A Japanese legend says that if you \textbf{can't sleep at night} it's because you're awake in someone else's dream.
	\item Here's what's cool:
	\begin{itemize}
		\item[1.] Say ``thank you.''
		\item[2.] Apologizing when you're wrong.
		\item[3.] Showing up on time.
		\item[4.] Being nice to strangers.
		\item[5.] Listening without interrupting.
		\item[6.] Admitting you were wrong.
		\item[7.] Following your dreams.
		\item[8.] Being a mentor.
		\item[9.] Learning and using people's names.
		\item[10.] Holding doors open for others.
	\end{itemize}
	\item \textit{Some say I'm too sensitive}, but truth is I just feel too much.
	
	Every word, every action and every energy goes straight to my heart.
	\item My favorite thing is when people remember little things I told them.
	
	Like seriously?
	
	You actually listened to me thank you.
	\item This is Jane.
	
	Jane is in a relationship.
	
	Jane doesn't post on social media about how much she loves her partner.
	
	She tells him this in person.
	
	She doesn't mention every little insignificant thing they do.
	
	Jane knows that nobody gives a damn.
	
	Jane is smart.
	
	Be like Jane.
	\item No one is more real than your mom.
	
	She'll talk shit on your face but pray for you behind your back.
	\item \textsf{A picture taken somewhere on the Indian highway: 2 elephants reach out in a brief moment of love and bonding before being taken away from each other for a lifetime of serving man.}
	\item I've always been someone who looks `\textbf{too deep}' into something or someone.
	
	That's because I realized from a young age that there's always more than what meets the eye.
	\item ``\textit{Lie your life for you not for anyone else}.
	
	\textit{Don't let the fear of being judged, rejected or disliked stop you from being yourself}.'' - Sonya Parker
	\item ``\textit{I don't trust someone who is nice to me but rude to the waiter, because they would treat me the same way if I were in that position}.'' - Muhammad Ali
	\item 2 constant issues of my life:
	\begin{itemize}
		\item[1.] I have nothing to wear.
		\item[2.] I have no space to keep my clothes.
	\end{itemize}
	\item Maybe you don't see people looking at you because you aren't looking at them.
	
	Maybe you don't hear all the good things people say about you because you're too focused on the bad.
	
	\textbf{Maybe you're a lot more} wonderful, beautiful and special than you ever give yourself credit for.
	\item Beautiful people are bad at math.
	\item A mistake which makes you Humble is much better than an Achievement that makes you Arrogant.
	\item ``\textit{Jobs fill your pocket}.
	
	\textit{Adventures fill your soul}.'' - Jaime Lyn Beatty
	\item It's time to start living the life you've imagined.
	\item My mind is more talkative than my mouth.
	\item Why do people call?
	
	Why can't they just text!?!
	\item ``\textit{You think you've see her naked because she took her clothes off?}
	
	\textit{Tell me about her dreams}.
	
	\textit{Tell me what breaks her heart}.
	
	\textit{What is she passionate about, and what makes her cry?}
	
	\textit{Tell me about her childhood}.
	
	\textit{Better yet, tell me 1 story about her that you're not in}.
	
	\textit{You've seen her skin, and you've touched her body}.
	
	\textit{But you still know as much about her as a book you once found, but never got around to opening}.'' - Dominic Matthew Jackson
	\item I don't know how to be anything other than intense.
	
	I don't know how to experience without feeling too much and thinking too much.
	
	I don't know how to sit still and quiet my mind and just be.
	
	I am always searching, always questioning, struggling to find the meaning in everything.
	
	I am passionate and I am deep, and even if I am misunderstood, I am finally okay with that.
	\item And when he asked what it was that he could give to her that she'd never had before, her answer was so simple.
	
	\textbf{Consistency}, she said.
	
	``If you want to give me something that no man has ever given to me, then don't give me mixed signals nor mixed emotions that leave me wondering.
	
	I'm tired of wondering.
	
	If you're going to be here with me, then be here, if you ever feel the need to leave me, then stay gone.
	
	\textbf{All I need at this point from someone is consistency.}''
	\item ``\textit{I'm starved for connection, not attention}.'' - Bridgett Devoue
	\item ``If you have been brutally broken but still have the courage to be gentle to other living beings, then you're a badass with a heart of an angel.'' - Keanu Reeves
	\item Nothing teaches better than this trio: the fears, the tears, the years, \textbf{\{the golden trio\}}.
	\item Ask a woman what kind of a man she wants then sit and listen how she explains characteristics of non living things.
	\item \textit{7 reasons you should be thankful for having a \textbf{big sister}}
	\begin{itemize}
		\item[1.] She's your go-to person for everything.
		\item[2.] You may hate each other at times.
		
		But your love is unimaginable.
		\item[3.] You know she will always protect you with everything she has.
		\item[4.] She's your best secret keeper.
		\item[5.] Her wardrobe is your wardrobe.
		\item[6.] She has seen the craziest side of you but still loves you.
		\item[7.] Nothing can replace the beautiful bond you 2 share.
	\end{itemize}
	\item To my best friends,
	
	1 day, all of us will get separated from each other.
	
	We will miss our conversations of everything and nothing and the dreams we had.
	
	Days, months and years will pass until this contact becomes rare.
	
	1 day, our children will see our pictures and ask: ``\textit{Who are these people?}''.
	
	And we will smile with invisible tears because a heart is touched with a strong word and you will say:
	
	``\textit{It was them that I had the best days of my life with}.''
	\item The true mark of maturity is when somebody hurts you, and you try to \textbf{understand their situation} instead of hurting them back.
	\item Dear stress: let's break up.
	\item ``\textit{Expectation is the root of all heartache}.'' - William Shakespeare
	\item Fake people have an image to maintain.
	
	Real people just don't give a fuck.
	\item You want to come into my life, the door is open.
	
	You want to get out of my life, the door is open.
	
	Just 1 request$\ldots$ \textbf{don't stand at the door}, you're blocking the traffic.
	\item ``\textit{So, if you are too tired to speak, sit next to me for I, too, am fluent in silence}.'' - R. Arnold
	\item Fall in love with the person who enjoys your madness.
	
	Not an idiot who forces you to be normal.
	\item Don't be afraid to love again.
	
	Not everyone is like your ex.
	\item I hate remembering good times I had with people who ended up being really shitty to me.
	\item The most beautiful part to loving a guarded girl is this: when she lets you in, it's not because she needs you.
	
	She stopped needing people a long time ago.
	
	It's because she wants you.
	
	And that - that is the purest love of all.
	\item You don't always need a plan.
	
	Sometimes you just need to breathe, trust, let go, and see what happens.
	\item What consumes your mind, controls your mind.
	\item I have 3 different personalities:
	\begin{itemize}
		\item the one where I'm out-going and loud,
		\item the one where I'm shy and quiet as fuck,
		\item the one where I hate everyone and every little thing bothers me.
	\end{itemize}
	\item People are the pretties when they talk about something they really love, with passion in their eyes.
	\item Be happy.
	
	Not because everything is good, but because you can see the good side of everything.
	\item I'm 100\% OK with having like 3 friends because people suck.
	\item \textbf{Brother}: a person who is there when you need him; someone who picks you up when you fall; a person who sticks up for you when no one else will;
	
	\textbf{a brother is always a friend.}
	\item I don't care if I'm selfish.
	
	After putting people 1st for the longest time and being disappointed I deserve to do whatever makes me feel happy.
	\item Sometimes, having coffee with your best friend is all the therapy you need.
	\item Even the smallest lie can break the biggest trust.
	\item Because at the end of the day you're the person I want to come home to.
	
	You're the person I want to tell how my day went.
	
	You're the person I want to share my happiness, sadness, frustration, and success with.
	\item Don't expect to get what you give.
	
	Not everyone has a heart like you.
	\item You don't really know someone until you say \textsc{No} to them.
	\item So you want to be happy?
	
	Then stop letting the smallest things ruin your whole entire day.
	
	If you're bored with your daily routine, do something unexpected.
	
	Stop complaining about how alone you are when you're surrounded by people who actually care about you.
	
	Forget all the drama and let go of all the grudges you've been holding.
	
	Stop wasting time lingering over all that you could have, should have and would have done.
	
	Stop spending your days thinking of how much better you could do; stop longing for something that has been and always will be out of your reach.
	
	Just live the days as they come.
	
	Wake up every morning and smile at the wonderful day that awaits you.
	
	Take a risk for once.
	
	Let yourself be happy, because you deserve it.
	\item ``\textit{1 day you will wake up and there won't be any more time to do the things you've always wanted}.
	
	\textbf{\textit{Do it now}.}'' - Paulo Coelho
	\item Fall in love with the person who enjoys your madness.
	
	Not an idiot who forces you to be normal.
	\item Always be in love with a soul, not a face.
	\item And if they want to leave, hold the door open for them.
	\item Never lie to girls with big foreheads.
	
	They never forget easily.
	
	They're hiding 64GB drive in there.
	\item ``\textit{Before you diagnose yourself which \textbf{depression} or \textbf{low self-esteem}, 1st make sure that you are not, in fact, just \textbf{surrounded by assholes}.}'' - William Gibson
	\item I need to spend a good night \textit{laughing too loud} and \textit{eating too much} with my best friends.
	\item ``\textit{Travel and tell no one}.
	
	\textit{Live a true love story and tell no one}.
	
	\textit{Live happily and tell no one}.
	
	\textit{People ruin beautiful things}.'' - Kahlil Gibran
	\item Psychology says people who love black color have the most colorful minds.
	\item Some day we will find what we are looking for.
	
	Or maybe not, maybe we'll find something much greater than that.
	\item Stop looking for a partner.
	
	Focus on your goals and rebuilding your life.
	
	The right person will eventually find their way to you.
	\item Without rain nothing grows.
	
	Learn to embrace the storms in your life.
	\item Raise your hand: if you have a bad habit of laughing at serious moments.
	\item Never reply when you are angry.
	
	Never make a promise when you are happy.
	
	Never make a decision when you are sad.
	\item Some people will only ``love you'' as much as they can use you.
	
	Their ``loyalty'' ends where the benefits stop.
	\item Brain Cell $\sim$ The Universe
	
	What if we're living in a brain cell of another creature?
	\item If you try and fail, congratulations because most people don't even try.
	\item I'm not impressed by money, followers, degrees and titles.
	
	I'm impressed by kindness, integrity, humility and generosity.
	\item Instead of wiping away your tears, wipe away the people who made you cry!
	\item Girls, if a guy
	\begin{itemize}
		\item remembers your birthday
		\item knows what you enjoy
		\item saves your pictures
		\item understands your family \& friends.
	\end{itemize}
	This guy is not your man.
	
	This guy is Mark Zuckerberg.
	\item Lucky are those who find a true loyal friend in this fake world.
	\item I am at the stage of my life where I don't force someone to stay in life.
	
	If you wanna stay you can stay.
	
	If you wanna leave you can leave.
	\item I feel like water solves all problems.
	
	Want to lose weight?
	
	Drink water.
	
	Clear face?
	
	Use water.
	
	Tired of a person?
	
	Drown them.
	\item Do you ever listen to a song and remember exactly what life was like when you 1st heard it?
	\item It is not the \textit{lie} that bothers me.
	
	It's the \textit{insult to my intelligence} that I find offensive.
	\item ``\textit{Okay I know that there are terrible terrible people out there but listen:}
	
	\textit{I also know that there are people who stop and smile at tiny plants growing out of sidewalk cracks, people who laugh so loud they snort, people who compliment others randomly, people who take pictures of their friends because they love seeing their friends happy, people who ramble about things that they're passionate about, people who blush and stutter, people who are kind, people who are warm, people who love and love and love and love}.'' - queens-bees
	\item You know what your problem is?
	
	You get attached, fast.
	And once you're attached to someone, you do everything you can to please them and make them happy.
	
	It's never been about what you want, it's always everyone's needs before your own.
	
	You give out too many chances to people, who quite frankly, do not deserve them.
	
	They take advantage of you, and you become a pushover.
	
	But you're okay with that, because they're in your life and that's all you ever really wanted.
	
	And even if they screw you over, you'll still be there for them.
	
	Because that's you, that who you are.
	
	Once you get attached to someone, they capture your heart and they always have a place there.
	
	And that is why it's so hard for you to let him go.
	\item Strength does not come from winning.
	
	Your struggles develop your strengths.
	
	When you go through hardships and decide not to surrender, that is strength.
	\item Sometimes, the girl who is always there for everyone else, needs someone to be there for her.
	\item \textsc{Thought} for the day: A person who feels appreciated will always do more than is expected.
	\item It's easy to make 15 friends in 1 year.
	
	But keeping 1 friend for 15 years is special.
	\item \textsc{Value Yourself!}
	
	Losing your love is okay.
	
	Losing your best friend is okay.
	
	But losing yourself in the process of getting them back to you is not okay.
	
	Value yourself, You are important.
	\item ``\textit{Once a year, go somewhere you're never been before}.'' - Dalai Lama
	\item Everybody isn't your friend.
	
	Just because they hang out around you and laugh with you doesn't mean they're there for you.
	
	Just because they say they got your back, doesn't mean they won't stab you in it.
	
	People pretend well.
	
	Jealousy sometimes doesn't live far.
	
	So know your circle.
	
	At the end of the day, real situations expose fake people, so pay attention.
	\item Be yourself.
	
	People don't have to like you, and you don't have to care.
	\item I've found the key to happiness.
	
	Stay away from assholes.
	\item ``\textit{I often think that the night is more alive than the day}.'' - Vincent Van Gogh
	\item Life is too short to be \textit{normal}.
	
	Stay \textit{weird}.
	\item You can tell a lot about a person by what's on their playlist.
	\item I'd rather be alone, than around chaos and confusion.
	
	\textbf{Silence beats drama any day.}
	\item ``Have the courage to be disliked.'' - Bruce Lee
	\item I think weekends are made in China: they don't last long.
	\item When you're trying to love people but you're also an introvert and have boundaries.
	\item Your brother will never say he loves you but he loves you more than anyone else in this world.
	\item I honestly love being around positive people.
	
	You're not judged, there's no drama, everyone just wants to relax and have a nice time.
	\item When you truly don't care what the fuck anyone thinks of you, you have reached a dangerously awesome level of freedom$\ldots$!!
	\item So many years of education yet nobody ever taught us how to love ourselves and why it's so important.
	\item You have to train your mind to be stronger than your emotions or else you'll lose yourself every time.
	\item Girls are so extra dramatic when they get mad.
	
	Everything is ``Bye Delete my number'' ``I'm done. I hate you. You won't hear from me no more. Have a blessed life. Leave me alone'' and then they sit there waiting for a reply.
	\item Stop explaining yourself to other people.
	
	You owe no one any explanation of what you do.
	
	Your life is yours, not theirs.
	\item ``\textit{Great minds discuss ideas}.
	
	\textit{Average minds discuss events}.
	
	\textit{Small minds discuss people}.'' - Eleanor Roosevelt
	\item Sir William Golding regarding women:
	\begin{quotation}
		I think women are foolish to pretend they are equal to men, they are far superior and always have been.
		
		Whatever you give a woman, she will make it greater.
		
		If you give her sperm, she'll give you a baby.
		
		If you give her groceries, she'll give you a meal.
		
		If you give her a smile, she'll give you heart heart.
		
		She multiplies and enlarges what is given to her.
		
		So, if you give her any crap, be ready to receive a ton of shit.
	\end{quotation}
	\item I truly appreciate kindness.
	
	I appreciate a quick message.
	
	I appreciate those who ask if I'm okay.
	
	I appreciate people checking up on me.
	
	I appreciate every single person in my life who has tried to brighten my days.
	
	\textit{It's the little things that matter the most}.
	\item An arrow can only be shot by pulling it backward.
	
	When life is dragging you back with difficulties, it means it's going to launch you into something great.
	
	So just focus, and keep aiming.
	\item Too much perfection is a mistake.
	\item \textbf{Left handers:} are creative, smart and have a big heart.
	
	Who do you know is a lefty?
	\item Time won't make you forget, it will make you grow and understand things.
	\item ``\textit{Stop thinking that just because you have a college degree it makes you smart}.
	
	\textit{I know a lot of people who have a driver's license but they can't drive for shit}.'' - Ray Velcoro
	\item Sometimes the hardest battle is against yourself.
	\item Being raised right doesn't mean you don't drink, party, smoke or use curse words.
	
	Being raised right is how you treat people, your manners and respect.
	\item A group of Engineering students and their teacher were given free airplane tickets to go on a holiday.
	
	Once on the plane the Captain announced that they were on the plane the students had built.
	
	Everyone freaked out and rushed out of the plane, except for the teacher who stayed there with calm.
	
	When the flight attendant asked why he hadn't left he responded ``I know the abilities of my students quite well, this shit won't even start''.
	\item If you have a friend with whom you can speak and open your heart anytime$\ldots$
	
	Trust me, you're blessed.
	\item Shout out to the women who fix another woman's crown without telling the world it was crooked.
	\item I respect people with real intentions who tell me the truth.
	\item Never be afraid to say what you feel.
	\item I'll choose honesty over perfection every single time.
	\item Enjoy every moment with your partner$\ldots$
	
	Because you're gonna block each other 1 day.
	\item Find a woman with a \textit{brain}.
	
	They all have \textit{vaginas}.
	\item Even the strongest feelings expire when ignore and taken for granted.
	\item \textit{The best advice}$\ldots$
	
	Save money every week!
	
	It doesn't matter how much.
	
	Just save!
	
	Listen to your parent's advice$\ldots$ at the end of the day they are the only ones who want the best for you.
	
	Choose your friends wisely as you are the product of your environment.
	
	Learn to be alone and independent.
	
	It's a skill few master.
	
	Educate yourself - read, read, read.
	
	Be healthy \& look after your body.
	
	Don't wait for someone to love you; learn to love \textsc{Yourself} 1st.
	
	You'll be OK.
	\item I need a rest.
	
	A long, long, long, long, long, long long rest.
	\item I am strong, but I am tired.
	\item Always remember, someone's effort is a reflection of their interest in you.
	\item ``\textit{Education is not the learning of facts but the training of the mind to think}.'' - Albert Einstein
	\item In life, what you really want; will never come easy.
	\item Always hope but never expect.
	\item Never let your friends feel lonely disturb them all the time.
	\item The problem with closed minded people is their mouth is always open.
	\item Please don't be an asshole to me, because then I may have to be an asshole to you, and I'm way better at being an asshole than you are.
	\item A well read woman is a dangerous creature.
	\item Even devil, ghosts, monsters, and witches are less dangerous than fake human beings.
	\item Never change to be accepted by others.
	
	Stay weird.
	\item Take the risk or lose the chance.
	\item 1 day, someone will love you the way you deserve to be loved and you won't have to fight for it.
	\item The best revenge is no revenge.
	
	Move on.
	
	Be happy.
	\item It's not that I can't see what they see.
	
	It's that I see what they can't.
	\item Respect people's privacy$\ldots$
	
	Don't ask them why they aren't married yet, when they going to have a baby, why they are separated or divorced, why they left their job, why they like to watch movies alone, why they cry listening to someone's sad story, why they don't go out more$\ldots$
	
	Unless they decide to share it with you, it's none of your business$\ldots$!
	\item A woman is \textbf{unstoppable} after she realizes she deserves better.
	\item Time has a way of showing us what really matters in life.
	\item A new study links drinking coffee to a longer lifespan.
	
	Awesome!
	
	I'm gonna be immortal.
	\item People who have been single for too long are the hardest to love.
	
	They have become so used to being single, independent and self-sufficient that it takes something extraordinary to convince them that they need you in their life.
	\item You don't have to be crazy to be my friend.
	
	\textit{I'll train you}.
	\item When you're wrong admit it.
	
	When you're right be quiet.
	\item A good wife can bring balance to your life.
	\item Twinkle twinkle little star.
	
	Singles are the superstar.
	\item ``\textit{There are rare people who will show up at the right time, help you through the hard times and stay into your best time$\ldots$}
	
	\textit{Those are the keepers}.'' - Nausicaa Twila
	\item When a storm is coming, all other birds seek shelter.
	
	The Eagle alone, avoids the storm by flying above it.
	
	So, in the storms of life$\ldots$
	
	May your heart soar like an Eagle.
	\item People say a lot.
	
	So, I watch what they do.
	\item Life is not about being rich, being popular, being highly educated or being perfect.
	
	It's about being real, humble, and kind.
	\item Isn't it scary, knowing that any time could be the last time you talk to someone?
	
	Always keep that in mind.
	\item I am too old or worry about who likes me and who dislikes me.
	
	I have more important things to do.
	
	If you love me, I love you.
	
	If you support me, I support you.
	
	If you hate me, I don't care.
	
	Life goes on with or without you.
	\item They say good things take time.
	
	That's why I'm always late.
	\item Silent people have the loudest minds.
	\item ``\textit{I have a theory that as long as you have 1 good friend, \textbf{1 real friend}, you can get through anything}.'' - Dana Reinhardt
	\item Your truest friends are the people who don't walk out the door when life gets hard.
	
	They actually pour some coffee and pull up a chair.
	\item Life is short.
	
	Spend it with people who make you laugh and feel loved.
	\item Nothing is more attractive than loyalty.
	\item He said, ``there are only 2 days in the year that nothing can be done.
	
	One is called \textit{yesterday} and the other is called \textit{tomorrow}, so today is the right day to love, believe, do and mostly live.'' - Dalai Lama
	\item Put me in a room with the same people who talk shit about me and watch how friendly they become.
	\item Give yourself enough respect to walk away from someone who doesn't see your worth.
	\item No one owns th water.
	
	No one owns the land.
	
	No one owns the oceans.
	
	No one owns the sand.
	
	These are given by our Mother:
	
	\textit{The Planet provides for free}.
	
	Only by the hands of the greedy, does the Earth require a fee.
	\item A wise man can always be found alone.
	
	A weak man can always be found in a crowd.
	\item We are all in the same game, just different levels, dealing with the same hell, just different devils.
	\item Tell me your deepest fantasy.
	
	8--12 hours of uninterrupted sleep.
	\item You know I'm comfortable with you when:
	\begin{itemize}
		\item[1.] I can be weird with you.
		\item[2.] I sing whatever song comes into my mind.
		\item[3.] I say what's on my mind.
		\item[4.] I talk nonsense.
	\end{itemize}
	\item Never expect.
	
	Never assume.
	
	Never ask.
	
	And Never demand.
	
	Just let it be, if it's meant to be, it will happen.
	\item Usually, the people with the best advice are the ones with most problems.
	\item You can't change how people feel about you, so don't try.
	
	Just live your life and be happy.
	\item Ladies, Please stop wasting your time looking for Mr. Right
	
	Just find Mr. Left and drag that idiot to the right.
	\item \textit{People who have been single for too long are the hardest to love}, because they have become so used to being single, independent, and self-sufficient that it takes something extraordinary to convince them that they need you in their life.
	\item Before you ask why someone hates you, ask yourself why you give a fuck.
	\item Friends who do not have a lot of pictures together have a lot of memories together!
	\item People who correct your grammar during a conversation always die alone.
	\item ``\textit{Why worry?}
	
	\textit{If you have done the very best you can}.
	
	\textit{Worrying won't make it any better}.
	
	\textit{If you want to be successful, respect 1 rule}.
	
	\textbf{\textit{Never let failure take control of you}.}
	
	\textit{Everybody has gone through something that has changed them in a way that they could never go back to person they once were}.
	
	\textit{Relationships are like electric currents, \textbf{wrong connection will give you shock throughout your life.}}
	
	\textbf{\textit{But the right one will light up your life}.}'' - Leonardo Dicaprio
	\item Everyone you meet is fighting a battle you know nothing about.
	
	Be kind.
	
	Always.
	\item Mood: wanna move to another city and start a new life.
	\item Don't be afraid to love again, not everyone is like your ex.
	\item \textbf{Stay single} until you meet the person that will take care of you the most.
	\item Once in your life you'll come across \textbf{a special person} that makes you happy, supports you, and makes you a better person.
	
	\textbf{Don't let them go.}
	\item If a friendship lasts longer than 7 years, psychologists say it will last a lifetime.
	\item ``\textit{If there's just 1 piece of advice I can give you, it's this when there's something you really want, fight for it, don't give up no matter how hopeless it seems}.
	
	\textit{And when you've lost hope, ask yourself if 10 years from now, you're gonna wish you gave it just 1 more shot, because they best things in life, they don't come free}.'' - Shonda Rhimes
	\item Love is temporary but friendship is forever$\ldots$
	\item \textbf{Always help someone.}
	
	You might be the only one that does.
	\item ``\textit{Stay kind}.
	
	\textit{It makes you beautiful}.'' - Najwa Zebian
	\item ``\textit{When she's important to you, you won't make excuses}.
	
	\textit{You'll make sacrifices}.'' - R. Solo
	\item Friendship isn't about who you've known the longest, it's about who walked into your life, said ``\textit{I'm here for you}'' \textit{and proved it}.
	\item  May all the negative energy trying to bring you down come to an end.
	
	May the dark thoughts, the overthinking, and the doubt exit your mind right now.
	
	May clarity replace confusion.
	
	May hope replace fear.
	
	May blessings and success fill your life.
	
	May the light of your spirit shine so bright that nothing can dim your glow.
	
	Shine one.
	\item Common sense is not a gift, it's a punishment.
	
	Because you have to deal with everyone who doesn't have it.
	\item A wise woman once said ``fuck this shit'' and she lived happily ever after.
	\item \textit{Challenge}: When a negative thought enters your mind, think 3 positive ones.
	
	Train yourself to flip the script.
	
	Accept the challenge!
	\item Strangers can become best friends just as easy as best friends can become strangers.
	\item Hold on to every genuine person you find.
	
	This generation has people driven by \textit{ego}, \textit{money} and \textit{status}.
	
	As a result good souls are ruined daily.
	
	Keep your head up and be conscious of the energy you give out and connect with.
	\item The quality of your thinking determines the quality of your life.
	\item A relationship means that you come together to \textbf{make each other better.}
	
	Believe in each other.
	
	Support each other.
	
	Build each other.
	
	\textbf{Be their peace, not their problem.}
	\item May everything I post confuse you until you learn to mind your business.
	\item People should seriously stop expecting normal from me$\ldots$
	
	We all know it's never gonna happen!
	\item Choose a partner who is good for you.
	
	Not good for your parents.
	
	Not good for your image.
	
	Not good for your bank account.
	
	Choose someone who's going to make your life emotionally fulfilling.
	\item Not all classrooms have 4 walls.
	\item Finding friends with the same mental disorder is \textit{priceless}.
	\item If you stay, stay forever.
	
	If you go, do it today.
	
	If you change, change for the better.
	
	And if you talk, make sure you mean what you say.
	\item Hugs are actually so underrated, especially those hugs that are so tight you can literally feel the other person's heartbeat and for a moment everything feels so calm and safe like nothing can hurt you.
	\item ``\textit{Hell is empty and all the devils are here}.'' - William Shakespeare
	\item Your relationship doesn't need to make sense to anyone, except you and your partner.
	
	It's a relationship.
	
	Not a community project.
	\item Never say ``That won't happen to me.''
	
	Life has a funny way of proving us wrong.
	\item When I put my earphones on, I enter the world of my own.
	\item I love people who gossip behind my back.
	
	That's exactly where they belong.
	
	Behind my back.
	\item Psychology says if you can't sleep at night it's all because you slept in the afternoon.
	
	Not everything is about love.
	\item You often feel tired, not because you've done too much, but because you've done too little of what sparks a light in you.
	\item Once in a lifetime we meet someone who changes everything.
	\item Being alone has a power that very few people can handle.
	\item Having a real friend is \textit{a blessing}.
	\item ``\textit{Do you know what happens when you decide to stop worrying about what other people might think of you?}
	
	\textit{You get to dance}.
	
	\textit{You get to sing}.
	
	\textit{You get to laugh loudly, paint, write, and create}.
	
	\textit{You get to be yourself}.
	
	\textit{And you know what?}
	
	\textit{Some people won't like you}.
	
	\textit{Some will laugh or mock or point out flaws$\ldots$ but it just won't bother you all that much}.'' - Doe Zantamata
	\item Your brother will never say he loves you but he loves you more than anyone else in this world.
	\item If it involves mountain, breakfast food, coffee or campfires, I'm in.
	\item \textit{Don't be afraid of being different}, be afraid of being the same as everyone else.
	\item Have you ever given someone else a motivational speech while you were hurting on the inside?
	
	That's \textbf{strength}!
	\item There's always 3 best friends.
	
	The wild one, the one who got that IDGAF attitude, and the cute, innocent one.
	\item Without rain nothing grows.
	
	Learn to embrace the storms in your life.
	\item You don't always need a logical reason for doing everything in your life.
	
	Do it because you want to; because it's fun; because it makes you happy.
	\item Marry a girl who loves food more than expensive gifts.
	\item ``\textit{No amount of security is worth the suffering of a mediocre life chained to a routine that has killed your dreams}.'' - Maya Mendoza
	\item If you fix a time with your best friend to meet at 6 p.m and he is there at 6 p.m, then he is not your best friend.
	\item ``\textit{How to tell somebody is genuinely interested in you}:
	
	\textit{If you removed all of your effort from the equation would any communication remain between you?}
	
	\textit{If not, there is nothing there, and you deserve better}.'' - Beau Taplin
	\item ``\textit{Working hard for something we don't care about is called stress;}
	
	\textit{Working hard for something we love is called passion}.'' - Simon Sinek
	\item Life is not about the quantity of friends you have.
	
	It is about the quality of friends you have.
	\item I am a \textbf{strong} person.
	
	But once in a while, I need someone to hug me and tell me, ``Everything is going to be okay.''
	\item Tell the truth and run$\ldots$
	\item There is honestly no reason to lie to me.
	
	I'm too understanding.
	
	I get it.
	
	I get life
	
	I know that shit happens.
	
	Just be straight up with me.
	\item I love the smell of freshly brewed coffee in the morning.
	
	And I love the sound of no one talking to me while I drink it.
	\item Never stop working on yourself.
	\item It's amazing when your humor is exactly the same as someone else's and you both just spend the whole day laughing at everything you both say.
	\item Not every friend request is a friend request.
	
	Some are surveillance cameras.
	\item ``\textit{Magic happens when you don't give up, even though you want to}.
	
	\textit{The universe always falls in love with a stubborn heart}.'' - JmStorm
	\item Sometimes I delete my own posts because I'm not the same person I was 4 minutes ago.
	\item Someday someone will break you so badly that you'll become unbreakable.
	\item The right one will know all your weaknesses and never use them against you.
	\item ``\textit{No Girl will choose 6 pack over 6 cars}$\ldots$
	
	\textit{So stop going to the gym and Go to Work}. - Robert Mugabe (Former PM of Zimbabwe)
	\item ``\textit{When you find people who not only tolerate your quirks but celebrate them with glad cries of ``Me too!'' be sure to cherish them}.
	
	\textit{Because \textbf{those weirdos are your tribe}.}'' - A.J. Downey
	\item When you're happy, you enjoy the music but when you're sad, you understand the lyrics.
	\item When a Man makes money, he feels like he wants more women, but a Woman makes money she feels like she doesn't need a man.
	\item Dear Music, Thanks for always clearing my head, healing my heart, and lifting my spirits.
	\item I have these 2 neighbors, and they're married, and they gotta be in their late 30s, and I'm making dinner, and I look out the window, and they're running around outside in their pajamas and bare feet with water pistols soaking each other and laughing so loud it made me realize \textbf{I'm wasting so much time trying to make relationships perfect} when all that's really needed is someone who will laugh with me for the rest of my life.'' - on-cloud-mine
	\item ``\textit{Since I've been spending my time in nature with animals I found out that, we are the wild savages and they are the ones who are actually civilized}.'' - Jim Carrey
	\item The world is full of monsters with friendly faces and angels full of scars.
	\item Solve the problem or leave the problem.
	
	But do not sleep with the problem.
	\item When someone appears in your dreams, it's because that person misses you.
	\item I like poems
	
	museums
	
	real conversation
	
	genuine people
	
	Friday nights
	
	fantasies
	
	I like stories
	
	intimacy
	
	I like soul.
	\item \textsc{Decide}
	
	``\textit{So, do it}.
	
	\textit{Decide}.
	
	\textit{Is this the life you want to live?}
	
	\textit{Is this the person you want to love?}
	
	\textit{Is this the best you can be?}
	
	\textit{Can you be stronger?}
	
	\textit{Kinder?}
	
	\textit{More Compassionate?}
	
	\textit{Decide}.
	
	\textit{Breathe in}.
	
	\textit{Breathe out and decide}.'' - Meredith Grey
	\item Be alone.
	
	Eat alone.
	
	Take yourself on dates.
	
	Sleep alone.
	
	Take the time to understand and love who you are.
	\item Girls with glasses are damn cute.
	\item A satisfied life is better than successful life.
	
	Because our success is measured by others, by our satisfaction is measured by our own \textit{soul}, \textit{mind} and \textit{heart}.
	\item Damaged people are \textsc{Strong} because they know how to survive.
	\item 3 most beautiful things in the world:
	\begin{itemize}
		\item[1.] Morning Sleep
		\item[2.] Afternoon Sleep
		\item[3.] Night Sleep
	\end{itemize}
	\item Everyone needs someone who will call and say, ``Get dressed, we're going on an adventure.''
	\item Life is short.
	
	Make sure you spend as much time as possible on the Internet arguing with strangers about politics.
	\item \textsc{Simple formula for living}
	\begin{itemize}
		\item Live beneath your means.
		\item Return everything you borrow.
		\item Stop blaming other people.
		\item Admit it when you make mistake.
		\item Give clothes not worn to charity.
		\item Do something nice and try not to get caught.
		\item Listen more; talk less.
		\item Everyday take a 30 min. walk.
		\item Strive for excellence, not perfection.
		\item Be on time.
		
		Don't make excuses.
		\item Don't argue.
		
		Get organized.
		\item Be kind to unkind people.
		\item Let someone cut ahead of you in line.
		\item Take time to be alone.
		\item Cultivate good manners.
		\item Be humble.
		\item Realize and accept that life isn't fair.
		\item Know when to keep your mouth shut.
		\item Go an entire day without criticizing anyone.
		\item Learn from the past.
		
		Plan for the future.
		\item Live in the present.
		\item Don't sweat the small stuff.
		\item It's all small stuff.
	\end{itemize}
	\item I would do it again!
	\item Listen to your best friends, sometimes they know you more than you know yourself.
	\item Notice the people who make an effort to stay in your life.
	\item Success is not always what you see.
	\item I don't care what other people think of me I enjoy my life with my own rules.
	\item Which pill would you take?
	\begin{itemize}
		\item \textsc{Red}: Love \& Happiness
		\item \textsc{Blue}: Money \& Power
	\end{itemize}
	\item Appreciate those who don't give up on you.
	\item Do not correct a fool or he will hate you.
	
	Correct a wise man and he will appreciate you.
	\item There comes a time in life, when you walk away from all the drama and people who create it.
	
	Surround yourself with people who make you laugh, forget the bad, \textbf{and focus on the good.}
	
	Love the people who treat you right.
	
	Pray for the ones who don't.
	
	Life is too short to by anything but happy.
	
	Falling down is part of life, getting back up is living.
	\item Don't be a parrot, in life, be an Eagle.
	
	A parrot talks way too much but can't fly high but an eagle is silent \& has the power to touch the sky.
	\item \textbf{The Full circle of life}: That moment when you realize the beginning and the end is same.
	\item Patience is when you're supposed to get mad, but you choose to understand.
	\item The goal is to laugh forever with someone you take seriously.
	\item You are a prisoner of your own mind.
	\item ``\textit{I think the saddest people always try their hardest to make people happy}.
	
	\textit{Because they know what it's like to feel \textbf{absolutely worthless} and they don't want anybody else to feel like that}.'' - Robin Williams (1951--2014)
	\item Someone who takes the time to listen when we are at our lowest is rare and should be valued.
	\item If you lose a friend because you're honest, it wasn't a good friend.
	\item You know what's sexy?
	
	A real conversation.
	\item Behind every strong, independent woman lies a broken little girl who had to learn how to get back up and to never depend on anyone.
	\item \textbf{Trust} [v] Something which I don't do anymore.
	\item Unexpected friendships are the best ones.
	\item The most common cause of stress nowadays is dealing with idiots.
	\item Your truest friends are the people who don't walk out the door when life gets hard.
	
	They actually pour some coffee and pull up a chair.
	\item Great Leaders don't tell you what to do.
	
	They show you how it's done.
	\item Caring about what people think of you is useless.
	
	Most people don't even know what they think of themselves.
	\item In the blink of an eye, everything can change.
	
	So \textbf{forgive often and love with all your heart.}
	
	You may never know when you may not have that chance again.
	\item We get so worried about \textit{being pretty}.
	
	Let's be pretty kind.
	
	Pretty funny.
	
	Pretty smart.
	
	Pretty strong.
	\item Never think that what you have to offer is insignificant.
	
	There will always be someone out there who needs what you have to give.
	\item Amazing Life-Hack: If you sleep till lunch time then you can save the Breakfast Money.
	\item Men, if you can't control your woman$\ldots$ then you've found the right one.
	\item Believe in yourself!
	\item Dear Heart, please stop getting involved in everything.
	
	Your job is to pump blood, that's it.
	\item Food will never break your heart.
	\item Your age doesn't define your maturity, your grades don't define your intelligence, and rumors don't define who you are.
	\item I don't go crazy.
	
	I am crazy, I just go normal from time to time.
	\item Before you marry someone, wait.
	
	Wait till you see them when they're angry.
	
	Are they abusive?
	
	Are they quiet?
	
	How do they deal with their anger?
	
	Because your children can't choose who gets to be their parents, but you can.
	
	Do it for them.
	\item ``\textit{You don't have to be positive all the time}.
	
	\textit{It's perfectly okay to feel sad, angry, annoyed, frustrated, scared, or anxious}.
	
	\textit{Having feeling doesn't make you a `negative person'}.
	
	\textit{It makes you human}.'' - Lori Deschene
	\item Brain: \textit{Be patient}
	
	Heart: \textit{Until when?}
	\item Trust yourself.
	
	You've survived a lot, and you'll survive whatever is coming.
	\item You know You're an adult when you get excited to just go home.
	\item Real eyes realize real lies.
	\item Sometimes you need bad things to happen to inspire you to change and grow.
	\item If your life were a book, what would the title be?
	\item If you remember anything of me, after I leave this world, remember that I loved even when it was foolish.
	
	That I cared even when it was unwanted.
	
	When my body is gone, remember my heart.
	\item When you fully trust someone without any doubt, you finally get 1 of 2 results:
	
	\textit{A person for life} or \textit{A lesson for life}$\ldots$
	\item I'm the kind of person who doesn't talk to others until they talk to me 1st.
	\item People wait all week for Friday, all year for summer, all life for happiness.
	\item No matter how many times a snake sheds its skin it will always be a snake.
	
	Remember that before allowing certain people back into your life.
	\item \textit{Good morning my wonderful friends}.
	
	May your day begin with a smile on your face, love in your heart and happiness within your soul.
	\item \textbf{The perfect man} [n] An unknown creature available only in books and movies.
	\item Sometimes a person needs to hear those 3 words$\ldots$
	
	No, not those words.
	
	These$\ldots$
	
	\textit{I've got you}.
	\item Kindness is not an act, it's a lifestyle.
	\item The most precious gift you can give someone is the gift of your time and attention.
	\item Just like seasons, people change.
	\item ``\textit{Straight roads do not make skillful drivers}.'' - Paulo Coelho
	\item If you have at least 1 person genuinely supporting you, you're blessed.
	\item I hate when people use their zodiac to justify their shitty behavior like, ``Sorry, I can't help it, I'm a Scorpio.''
	
	No Susan, you're just an asshole.
	\item Sometimes you just need \textbf{an adventure} to cleanse the bitter taste of life from your soul.
	\item The biggest lesson I learned this year is to not force anything; conversations, friendships, relationships, attention, love.
	
	Anything forced is just not worth fighting for, whatever flow flows, what crashes crashes.
	
	It is what it is.
	\item You make me laugh even when I am not in the mood to smile.
	\item If you cannot be the poet, be the poem.
	\item \st{Single}
	
	\st{Taken}
	
	Building an empire,
	
	Finding myself,
	
	Healing myself,
	
	Loving myself,
	
	Being passionate,
	
	Getting fit,
	
	Growing friendships,
	
	Meeting new people \& making memories.
	\item If friendship lasts longer than 10 years, you are no longer just friends.
	
	You are family.
	\item Give your heart to dogs.
	
	They will never break it.
	\item Ordinary is never beautiful.
	
	To be beautiful, You must be weird, different, strange.
	\item You will never know the value of a moment until it becomes a memory.
	\item ``\textit{People think I'm anti-social because I don't joint their conversations}.
	
	\textit{The truth is, I don't give a damn what they're talking about most of the time}.'' - Health Ledger
	\item \textbf{People} will always notice the change in your attitude towards them, but they will never notice it's their behavior that made your change.
	\item I asked God to send me a man who will always love me.
	
	So he gave me a son.
	\item Consider how hard it is to change yourself and you'll understand what little chance you have in trying to change others.
	\item Confuse them with your silence.
	
	Shock them with your actions.
	\item If I'm wrong, educate me.
	
	Don't belittle me.
	\item Chap. 1: ``If it is important enough to you, you will find a way.
	
	If it is not, you will find an excuse.''
	\item Kindness is a language the blind can see and the deaf can hear.
	\item I am definitely not the same person I was when this year started.
	\item People who stay in the car to listen to music a little bit longer are my kind of people.
	\item I don't know how people can fake an entire relationship.
	
	I can't even fake a hello to someone I don't like.
	\item Apparently when you treat people like they treat you, they get upset.
	\item Stay strong, make them wonder how you're still smiling.
	\item Fall in love with the person who enjoys your Madness.
	
	Not an idiot who forces you to be normal.
	\item Say what you feel.
	
	It's not being rude, it's called being \textit{real}.
	\item If you repeat a lie often enough, it becomes \st{truth} politics.
	\item Note to self:
	
	It is entirely possible to be a kind, loving person who refuses to tolerate bullshit.
	
	In fact, it's not only possible, it's absolutely necessary. - Nanea Hoffman
	\item \textit{That's when I realized what a true friend was}.
	
	\textit{Someone who would always love you - the imperfect you, the confused you, the wrong you - because that is what people are supposed to do}. - r. j. l.
	\item Always be in love with a soul, not a face.
	\item The reason I like staying up late so much is because between the hours of 1am to 5am, the world is quiet and no one experts anything from me.
	
	I could stare at my wall for 4 hours and there would be no consequences.
	
	It's so silent and calm.
	
	I love it.
	\item I don't believe in forever but I do believe in growing old with someone you love.
	\item Find someone who will love your soul more than your body.
	\item ``\textit{My mom said something}.
	
	\textit{``You can lie down for people to walk on you and they will still complain that you're not flat enough.''}
	
	\textit{Live you life}.'' - Mature Gambino
	\item ``\textit{Remember that sometimes not getting what you want is a wonderful stroke of luck}.'' - Dalai Lama
	\item Do you know what awesome feeling when you get into bed, fall right to sleep, stay asleep all night, and wake up feeling refreshed?
	
	Me neither.
	\item 1 day or day 1.
	
	You decide.
	\item Hugging is the silent way of saying ``You matter to me''.
	\item The most common cause of stress nowadays is dealing with idiots.
	\item So many years of education yet nobody ever taught us how to love ourselves and why it's so important.
	\item Sometimes my heart needs more time to accept what my mind already knows.
	\item We've been friends for so long I can't remember which 1 of us is the bad influence.
	\item The biggest communication problem is we do not listen to understand.
	
	We listen to reply.
	\item Dear bestie,
	
	No matter how many friends I have, no matter how much I talk to them and spend time with them, always remember that no one can replace you.
	
	You were, are and will always be irreplaceable.
	
	You will always have a special place in my heart.
	
	Someone yours till eternity.
	\item Sometimes the best therapy is a long drive and good music.
	\item \textit{Go for it}.
	
	Whether it ends good or bad, it was an experience.
	\item When you \textit{forgive}, you \textit{heal}.
	
	When you \textit{let go}, you \textit{grow}.
	\item The older I get, the more I realize I don't want to be around drama, conflict or stress.
	
	I want a cozy home, good food, and to be surrounded by happy people.
	\item A Saint was asked - ``What is Anger$\ldots$?''
	
	He gave a beautiful answer - ``It is a punishment we give to ourselves, for somebody else's mistake!''
	\item Find someone who wants to invest in you, learn from your, see you win, support your visions and fall in love with you daily.
	\item ``\textit{How amazing it is to find someone who wants to hear about all the things that go on in your head}.'' - Nina LaCour
	\item Travel.
	
	Money can always return.
	
	Time will not.
	\item Such a great feeling when someone just genuinely wants to talk to you and wants to know how your life is going.
	\item ``\textit{I think people spend \textbf{too much time staring at screens} and not enough time drinking wine, tongue kissing, and dancing under the moon}.'' - Rachel Wolchin
	\item Dear life, I've always been strong even since.
	
	But can you please be easy on me sometimes?
	\item \textsc{Understanding is the nature of Love}:
	
	``\textit{Understanding someone's suffering is the best gift you can give another person}.
	
	\textit{Understanding is love's other name}.
	
	\textit{If you don't understand, you can't love}.'' - Thich Nhat Hanh
	\item ``\textit{Don't be the reason someone feels insecure}.
	
	\textit{Be the reason someone feels seen, heard, and supported by the whole universe}.'' - Cleo Wade
	\item Never let an old flame burn you twice.
	\item May your vibes shift the whole damn frequency of the room when you walk in.
	\item ``\textit{Stay away from negative people}.
	
	\textit{They have a problem for every solution}.'' - Albert Einstein
	\item ``\textit{I know people who graduated college at 21 and didn't get a salary job until they were 27, I know people who graduated at 25 and already had a salary job}.
	
	\textit{I know people who have children and are single, I know people who are married and had to wait 8--10 years to be parents}.
	
	\textit{I know people who are in a relationship and love someone else, I know people who love each other and aren't together, there are people waiting to love and be loved}.
	
	\textit{My point is, everything in life happens according to our time, our lock}.
	
	\textit{You may look at your friends and some may seem to be ahead or behind you, but they're not, they're living according to the pace of their clock, so be patient}.
	
	\textit{You're not falling behind, it's just not your time}$\ldots$'' - Julissa Loaiza
	\item Some days it's hard to find motivation$\ldots$
	
	Some days motivation finds you!
	\item ``\textit{So don't hurt her, don't change her, don't analyze and don't expect more than she can give}.
	
	\textit{Smile when she makes you happy, let her know when you makes you mad, and miss her when she's not there}.'' - Bob Marley
	\item ``\textit{Sometimes you have to love people from a distance}.'' - Robert Tew
	\item People who make me laugh until it hurts are my favorite kinds of people.
	\item When they call you crazy, remember that great ideas don't come from average minds.
	\item Life would be so much easier if we just said what we fucking feel.
	\item Forgive him, but let him go.
	\item I'm not going to complain about you not texting me back, if I'm not worth your attention then OK that's when I start to lose interest.
	\item Come lay with me for hours so we can talk thousands of nothings while it means Millions of somethings.
	\item Before you judge me, make sure you're perfect.
	
	If you're not, then shut up.
	\item ``\textit{Your heart will never lie to you, that's your mind's job}.'' - Medusa
	\item People keep telling me the right guy will come along, but I think mine got hit by a bus or something.
	\item Note to self: stop missing people who don't even waste a 2nd thinking about you.
	\item Learn to wait.
	
	There's always time for everything.
	\item You want to change your life?
	
	Change the way you think.
	\item Respect is for those who deserve it, not for those who demand it.
	\item My Super Power?
	
	I can look you dead in the face while you're talking and not hear a damn thing you said.
	\item Why is there no maximum wage?
	\item You're always 1 decision away from a totally different life.
	\item People don't care for you when you are alone.
	
	They just care for you when they are alone.
	\item Be your own driver for the race called Life.
	
	Never expect anything, Expectations ruin your Life.
	
	Just work hard and believe in yourself to achieve what your goals are.
	\item Don't work 8 hours for a company then go home \& not work on your own goals.
	
	You're not tired, \textit{you're uninspired}.
	\item Every girl has male best friend in her life with whom she can share everything, without any fear.
	\item Psychology says: Go with the choice that scares you the most, because that's the one that's going to help you grow.
	\item If you want to know where your \textit{heart} is, look where your \textit{mind} goes when it wanders.
	\item \textbf{The truth?}
	
	\textbf{I like you.}
	
	\textbf{A lot.}
	
	You make me happy.
	
	You're smart.
	
	You're different.
	
	You're a little crazy, and awkward, and your smile alone can make my day.
	\item Don't settle for being someone's sometimes.
	\item People who have done you wrong, will forever think your posts are about them.
	\item Choose your battles wisely because if you fight them all you'll be too tired to win the really important ones.
	\item ``\textit{If you tell the \textbf{truth}, you don't have to \textbf{remember} anything}.'' - Mark Twain
	\item 1 of the best feelings is finally losing your attachment to somebody that isn't good for you.
	\item The things that excite you are not random.
	
	They are connected to your purpose.
	
	Follow them.
	\item Never let your fear decide your future.
	\item The better you become, the better you attract.
	\item ``\textit{Great minds discuss ideas}.
	
	\textit{Average minds discuss events}.
	
	\textit{Small minds discuss people}.'' - Eleanor Roosevelt
	\item \textit{I wanna go on a roadtrip someday}.
	
	Alone or with someone I love.
	
	I wanna get away.
	
	Explore places.
	
	Sleep in the car.
	
	Stop a lot just to admire the view.
	
	Visit museums and try out coffee shops.
	
	Listen to my favorite albums while driving.
	
	Have a polaroid camera.
	
	Take pretty pictures of the sunrise.
	
	Take pictures of myself.
	
	Run through a forest.
	
	Chase fog.
	
	Chase the sun.
	
	Spend hours on a field making flower crowns.
	
	Feel the wind in my hairs.
	
	Buy souvenirs.
	
	Meet people.
	
	Take time to observe.
	
	\textit{I wanna make memories}.
	
	\textit{I wanna feel alive}.
	\item I had to forgive a person who wasn't even sorry$\ldots$ \textbf{that's strength.}
	\item If you don't like where you are, move.
	
	You are not a tree!
	\item I'm not heartless, I just learned to use my heart less.
	\item In the end, we all just want someone that chooses us$\ldots$
	
	Over everyone else, under any circumstances.
	\item Me: Alexa, can you check my bank balance and tell me which Apple product I can afford$\ldots$
	
	Alexa: Apple juice$\ldots$
	\item ``\textit{Live your life for you not for anyone else}.
	
	\textit{Don't let the fear of being judged, rejected or disliked stop you from being yourself}.'' - Sonya Parker
	\item ``\textit{No great mind has ever existed without a touch of madness}.'' - Aristotle
	\item As your sister, always remember$\ldots$
	
	I loved you yesterday.
	
	I love you today.
	
	I always have.
	
	\textbf{I always will.}
	\item \textit{My biggest problem?}
	
	I notice everything.
	\item When you truly don't care what the fuck anyone thinks of you, you have reached a dangerously awesome level of freedom$\ldots$!!
	\item ``\textit{Never wish them pain}.
	
	\textit{That's not who you are}.
	
	\textit{If they caused you pain, they must have pain inside}.
	
	\textit{Wish them healing}.'' - Najwa Zebian
	\item A good laugh and a long sleep are the 2 best cures for anything.
	\item Underestimate me.
	
	That'll be fun.
	\item \textbf{Love} [n] Giving someone the power to destroy you, and trusting them not to.
	\item \textit{Sadness is}$\ldots$ not meeting your best friend for a long time.
	\item ``\textit{Inner peace is the new success}.'' - Idil Ahmed
	\item ``\textit{You don't have a soul}.
	
	\textit{You are a soul}.
	
	\textit{You have a body}.'' - C. S. Lewis
	\item ``\textit{Art is how we decorate space, music is how we decorate time.}'' - Jean-Michel Basquiat
	\item People who show you new music are important.
	\item Personality cannot be photoshopped.
	\item It's rare to meet someone with a mind that's just as beautiful as their face.
	\item Words are like keys.
	
	If you choose them right, they can open any heart and shut any mouth. - ideaspot
	\item Never be ashamed of how much you love, or how quickly you fall.
	
	Love fully, love completely, but most importantly, love naturally - and don't you ever apologize for it.
	
	Don't ever be sorry for loving the way your heart knows how.
	\item \textbf{Positive life.}
	
	If you can't find true love, work hard, make money and enjoy your single life in peace.
	
	Nobody has ever died from being Single, but so many have died for being with the wrong partner$\ldots$
	
	Life is too short to be wasting your time with the wrong person.
	\item \textit{Who are you?}
	
	Answer without: Name, Job, Things you did, Friends.
	\item Now close your eyes, and please understand that you are still young, and the universe is endless, and somehow, everything will be okay.
	\item Just because you are right, does not mean, I am wrong.
	
	You just haven't seen life from my side.
	\item \textbf{Don't} be impressed by:
	\begin{itemize}
		\item[1.] Money
		\item[2.] Followers
		\item[3.] Degrees
		\item[4.] Titles
	\end{itemize}
	\textbf{Do} be impressed by:
	\begin{itemize}
		\item[1.] Generosity
		\item[2.] Integrity
		\item[3.] Humility
		\item[4.] Kindness
	\end{itemize}
	\item Fall in love with souls, not faces.
	\item \textit{A real man can't stand seeing his woman hurt}.
	
	He's very careful with his actions and decisions because he can never be the reason for her pain.
	\item \textbf{Petrichor} [n] the smell of earth after rain.
	\item Car rides by yourself with loud music are good for the soul.
	\item \textit{I didn't change}.
	
	I just see things differently now.
	\item People ask me, ``\textit{why are you single?}
	
	\textit{You're attractive, intelligent, caring and creative}.''
	
	I reply, ``\textbf{\textit{I'm over-qualified}.}''
	\item Drink Water.
	
	Meditate.
	
	Moisturize your skin.
	
	Eat some juicy Fruits.
	
	Take long showers.
	
	Stretch and Breathe.
	
	Read Good Books.
	
	Start a Business.
	
	Mind your own Business.
	
	Love unconditionally.
	\item People don't always say: \textit{I love you}.
	
	Sometimes it sounds like:
	
	\textit{Be safe}.
	
	\textit{Did you eat?}
	
	\textit{Call me when you get home}.
	
	\textit{I made you this}.
	\item Don't wait until you've reached your goal to be proud of yourself, be proud of every step you take toward reaching that goal.
	\item Even the smallest lie can break the biggest trust.
	\item You are not best friends if you didn't hate each other when you 1st met.
	\item If you lose someone, but find yourself, you won.
	\item - Do you keep money in the bank or at home?
	
	- In my memories.
	\item Once you have matured, you realize silence is more important than proving a point.
	\item Deleting your story after the target audience saw it.
	\item No regrets in life.
	
	Just lessons learned.
	\item Judge me when you are perfect.
	\item ``\textit{She is rare because she is real}.'' - Mark Anthony
	\item The guys in pictures are waiters and chef.
	
	They work hard to cook delicious food for us but that they hardly get time to eat properly.
	
	They have to sacrifice to make us feel good.
	
	Don't shout at them if they are little late.
	
	Be kind \& show some respect to them.
	
	It costs nothing.
	\item ``\textit{You can master many languages but if you can't understand someone's silence, you fail}.'' - Suri Singh
	\item You don't always need a logical reason for doing everything in your life.
	
	Do it because you want to; because it's fun; because it makes you happy.
	\item Don't change yourself to win someone's heart.
	
	Stay true and you will find someone who \textit{likes you for being you}.
	\item Before you fake your lifestyle on Facebook, at least block the people that know you in person.
	\item I love it when someone's laugh is funnier than the joke.
	\item Whether you're right or wrong, you will be criticized any way.
	\item I think I'm afraid to be happy because whenever I get too happy, something bad always happens.
	\item Imagine being loved the way you love.
	\item Loyal people still exist.
	\item A beautiful face will age and a perfect body will change, but a beautiful soul will always be a beautiful soul.
	\item Never be defined by your past.
	
	It was just a lesson, not a life sentence.
	\item He told you you're the only flower in his garden and you believe it, have you ever seen a garden with 1 flower?
	\item I'm multitasking$\ldots$
	
	I can listen, ignore and forget at the same time.
	\item I'm a good person.
	
	But don't give me a reason to show you my evil side.
	\item Someone asked me, ``Who hurt you?''.
	
	I replied, ``My own expectation.''
	\item When a man opens the door of his car for his wife.
	
	You can be sure of 1 thing, either the car is new or the wife is.
	\item \textbf{Adulting}: Fucking bullshit, would not recommend.
	\item I care.
	
	I always care.
	
	This is my problem.
	\item The right ones, will do anything for you.
	\item Never forget who was there for you when no one else was.
	\item When you know they're lying but you still keep listening to them.
	\item Do everything with \textit{a good heart} and expect nothing in return.
	\item We all have that friend who can talk for 12 hours non stop.
	\item Don't break up.
	
	Fix the problem.
	
	Start the romance again.
	
	Go on dates again.
	
	Work on winning each other over again.
	
	This is why there are so many failed relationships.
	
	If you love each other and are best friends then breaking up is not the answer.
	\item People say a lot.
	
	So, I watch what they do.
	\item 1 of the best lessons you can learn in life is to master how to remain calm.
	\item Very sadly, when you meet old classmates, many of them want to know what you are doing now, not because they care, but because they want to assess if they are more successful than you in order to know whether to respect you or to look down on you.
	\item - Male privilege is wearing the same outfit multiple times to events while girls can't wear the same dress twice no matter how cute it is.
	
	- There isn't a single straight man on earth that cares if you wear the same cute dress twice.
	
	The negative comments will come from other women.
	\item Psychology says never lie to a girl because she already knows the truth before she asks.
	\item If you want to stay in a small cabin in the woods away from humans.
	\item Life of singles: Eat $+$ music $+$ sleep.
	
	Repeat$\ldots$!!
	\item I'm not lazy.
	
	I'm on Energy saving mode.
	\item If you didn't come from a healthy family, make sure a healthy family comes from you.
	\item We try to hide our feelings but we forget that our eyes speak.
	\item The eye only sees what the mind is prepared to comprehend.
	\item I will \textit{win} not immediately but definitely.
	\item It's time to start living the life you've imagined.
	\item I was trying to lose weight.
	
	I saw cake.
	
	Cake saw me.
	
	Cake disappeared.
	\item Sometimes the chains that prevent us from being free are more mental than physical.
	\item A clear rejection is always better than a fake promise.
	\item Some people will only ``love you'' as much as they can use you.
	
	Their ``loyalty'' ends where the benefits stop.
	\item Open-minded people don't care to be right, they care to understand.
	
	There's never a right or wrong answer.
	
	Everything is about understanding.
	\item Tired but I have goals.
	\item You don't always need a plan.
	
	Sometimes you just need to breathe, trust, let go, and see what happens.
	\item Don't put the key to your happiness in someone else's pocket.
	\item When you are angry, be silent.
	\item L.I.F.E.
	
	Living Isn't Fucking Easy.
	\item When you are a good person, you don't lose people, they lose you.
	\item Girls love their father so much.
	
	Because there is at least 1 man in the world who will never hurt her.
	\item \textit{True lover!}
	
	The only one who sings for me.
	
	The only one who gives me love bites.
	
	The only one who keeps coming back even when I chase them away.
	
	- Mosquito
	\item If you don't fight for what you want, don't cry for what you lost.
	\item No one meets anyone by accident.
	\item My wife is my strength.
	
	All the other Women are my weakness.
	\item When you both mad at each other \& waiting for one to apologize 1st.
	\item You can't force people to stay in your life.
	
	Staying is a choice, so be thankful to the people who chose you.
	\item We don't know what tomorrow will bring.
	
	So don't stay mad for too long.
	
	Learn to forgive and love with all your heart.
	
	Don't worry about the people who don't like you.
	
	Enjoy the ones who love you.
	\item - I heard something about you.
	
	I don't know if it's true.
	
	- It's true.
	\item Once you start loving someone.
	
	It's very hard to stop.
	\item If you're funny, you're automatically more attractive.
	
	Beauty fades but \textit{sarcasm is forever}.
	\item \textbf{Cherophobia}: The fear of happiness because they believe that every time they feel too happy, something bad comes and ruins it.
	\item \textbf{Happiness} is the new rich.
	
	\textbf{Inner peace} is the new success.
	
	\textbf{Health} is the new wealth.
	
	\textbf{Kindness} is the new cool.
	\item Never save things for special occasions.
	
	Being alive is the only special occasion there is.
	\item The person who is rude in behavior actually cares the most$\ldots$!!
	\item Of course I changed, I realized that I deserve so much better.
	\item - You know.
	
	People treat me like a god.
	
	- How?
	
	- They ignore my existence unless they need something from me.
	\item I'm jealous of people who can go to sleep seconds after closing their eyes.
	
	How do they do that?
	\item My kind of attitude.
	
	I'm not good at making revenge but I can ignore your existence like I didn't meet you in my whole life.
	\item - Let's go out today.
	
	- I'm broke.
	
	- Let's go somewhere cheap.
	
	- No.
	
	- Why?
	
	- I'm broke but I'm classy.
	\item It's better to wait long than to marry wrong.
	\item Imagine finding both love and friendship in 1 person.
	\item If your phone is \textsc{Samsung}, it means your storage is full$\ldots$
	\item I used to think I was introverted because I enjoyed being alone but it turns out I really liked being at peace with myself and my surroundings \& I am extremely extroverted with people who bring me comfort and happiness.
	\item At night: I can't sleep.
	
	In the morning: I can't wake up.
	\item To myself, and this year, don't let your heart rule your mind.
	
	Let your mind settle things with your heart.
	
	And most importantly, love yourself 1st.
	\item Who else loves the sound of heavy rain and thunder on a dark night?
	
	I find it so peaceful.
	\item Sometimes, no matter how nice you are, how kind you are, how caring you are, how loving you are, it just isn't enough for some people.
	\item I am no longer available for things that make me feel like shit.
	\item ``\textit{Mature by mind, Kid at heart}.
	
	\textit{Rude from outside, Caring from inside}.
	
	\textit{Best personality ever}.'' - Scrawled Stories
	\item Crying without tears is painful.
	\item Music removes the pain.
	\item The more I get to know people, the more I realize why Noah only let animals on the boat.
	\item ``\textit{I always felt like \textbf{I'm not from this generation}, I just live in it}.
	
	\textit{Because the way my mindset differs from the majority, you'd think \textbf{I come from a different dimension}.}
	
	\textit{That's why I keep things to myself because \textbf{a lot of people won't understand me}.}'' - Keanu Reeves
	\item And if I asked you to name all the things that you love, how long would it take for you to name yourself?
	\item Be careful who you push away$\ldots$
	
	Some of us don't come back.
	\item ``\textit{Shine bright like a diamond}.'' - Rihanna
	
	``\textit{Diamond don't shine idiot, they reflect}.'' - Albert Einstein
	\item Honest feelings and bad timing make the most painful combination.
	\item Mean World Syndrome is a phenomenon whereby violence-related content of mass media makes viewers believe that the world is more dangerous than it actually is.
	\item Why insult someone when you can say something nice in a very sarcastic tone.
	\item Sometimes the best therapy is a long drive, good music and good company.
	\item Why am I the type of person that still believes someone's a good person even when they've shown me in every way that they're not?
	\item My normal in life is simple, you treat me good and I'll definitely treat you better.
	\item ``\textit{1 of my biggest mistakes in life is thinking people will show me the same love I've shown them}.'' - Heath Ledger
	\item I just wanna sit outside at night and talk about life with someone.
	\item You weren't born to just play bills and die.
	\item Short hair means single.
	\item Love someone who is humble, kind, and empathetic.
	
	Not only with you, but with a beggar on the street, or a stranger in the supermarket.
	
	\textbf{Common courtesy} is important.
	
	\textbf{Compassion} is important.
	
	\textbf{Kindness} is important.
	\item 1 of my dream is to go on a night camping trip with my favorite people.
	\item It's nice when someone remembers small details about you.
	\item I am not open to many people.
	
	I'm usually quiet and I don't really like attention.
	
	So if I like you enough to show you the real me, you must be very special.
	\item Friends who say ``I love you'' before hanging up the phone are so important.
	\item I've got a good heart, but this mouth$\ldots$
	\item \textit{Your friends are a reflection of your own personality}.
	
	Intelligent people tend to have less friends than the average person.
	
	The smarter you are, the more selective you become.
	\item If you marry the right person, everyday is a Valenline's day.
	
	Marry the wrong person, everyday is Martyrs Day.
	
	Marry the lazy person, everyday is Labor Day.
	
	Marry the rich person, everyday is New Year's Day.
	
	Marry an immature person, everyday would seem like Children's Day.
	
	Marry a cheater or liar, everyday will become April Fool's Day.
	
	And if you don't get married everyday is Independence Day!
	\item Me: I want to travel.
	
	Bank account: You can afford to walk down the stairs.
	\item Your comfort zone will kill you.
	\item You see a person's \textit{true colors} when you are no longer beneficial to their life.
	\item I want money not a job.
	\item Talking to you, laughing with you, being with you, changes the whole mood.
	\item You can't go back and change the beginning, but you can start where you are and change the ending.
	\item Being both soft and strong is a combination very few have mastered.
	\item There is a purpose for everyone you meet.
	
	Some people come into your life to test you, some to teach you, some to use you, and some to bring out the very best in you.
	\item People with dirty mind have good heart.
	\item It's time to just be happy.
	
	Being sad, angry, and overthinking isn't worth it anymore.
	
	Just let things flow.
	
	Be positive.
	\item We do not see things as they are, we see things as we are.
	\item Isn't it funny how day by day nothing changes, but when you look back everything is different?
	\item I have no time to battle egos and small minds.
	\item Every pain gives a lesson and every lesson changes a person.
	\item Don't believe everything you think.
	\item Yes, I am a nice person, but if you cross the line too many times, everything can change very quickly.
	\item ``\textit{The No. 1 reason people fail in life is because they listen to their friends, family, and neighbors}.'' - Napoleon Hill
	\item Q: \textit{Do you hate people?}
	
	A: ``\textit{I don't hate them}$\ldots$
	
	\textit{I just feel better when they are not around}.'' - Charles Bukowski
	\item \textit{Is there anything she can't handle?}
	
	She's been broken.
	
	She's been knocked down.
	
	She's been defeated.
	
	She's felt pain that most couldn't handle.
	
	She looks fear in the face; year after year, day after day, but yet, she never runs.
	
	She never hides.
	
	And she \textsc{always} finds a way to get back up.
	
	She's unbreakable.
	
	She's a warrior.
	
	She's you.
	\item ``\textit{Don't live the same year 75 times and call it a life}.'' - Robin S. Sharma
	\item The best view comes after the hardest climb.
	\item I don't trust \textit{words}, I trust \textit{actions}.
	\item To have someone understand your mind is a different kind of intimacy.
	\item Stop looking for a partner.
	
	Focus on your goals and rebuilding your life.
	
	The right person will eventually find their way to you.
	\item \textit{3 Rules for a lasting relationship}:
	\begin{itemize}
		\item[1.] Never make your partner feel unwanted.
		\item[2.] No matter how hard things get, never cheat.
		\item[3.] Always have your partner's back through the good and bad times.
	\end{itemize}
	\item I'm actually a very nice person, until you piss me off.
	\item She chooses wisely who she allows in her life.
	
	Not because she is better than anyone, but because she remembers what happened when she wasn't careful and allowed just anyone in her circle.
	
	A time when she trusted easily and naively believed that most people had the same heart as her.
	
	This doesn't mean that she always gets it right, that every now and then she isn't fooled.
	
	But it does mean that she can spot the fakers quicker and let the people who are not good for her go without any hesitation.
	\item To my family and friends nearby and faraway, I want you to know that no matter what is going on$\ldots$
	
	\textbf{I always love you}.
	\item I want rich people problems.
	
	Like where to park my yacht$\ldots$
	\item She was simple like quantum physics.
	\item Here's what's cool:
	\begin{itemize}
		\item[1.] Saying ``thank you''.
		\item[2.] Apologizing when you're wrong.
		\item[3.] Showing up on time.
		\item[4.] Being nice to strangers.
		\item[5.] Listening without interrupting.
		\item[6.] Admitting you were wrong.
		\item[7.] Following your dreams.
		\item[8.] Being a mentor.
		\item[9.] Learning and using people's names.
		\item[10.] Holding doors open for others.
	\end{itemize}
	\item ``\textit{A little bit of attention and kindness can totally change a whole life, and a lack of that can do the same}.'' - Adrien Brody
	\item - We had a surprise test today
	
	- And?
	
	- I was really surprised.
	\item Would you give a second chance to someone who has hurt you once?
	\item If all religions teach peace, why can't all religions achieve peace?
	\item Dear music, thanks for always clearing my head, healing my heart, and lifting my spirit.
	\item I am not open to many people.
	
	I'm usually quiet and I don't really like attention.
	
	So if I like you enough to show you the real me, you must be very special.
	\item I'm antisocial, yet social$\ldots$
	
	I don't speak 1st, but when someone speaks to me I will speak to them.
	
	Some days I'm really annoying and talkative, other days I'm like a turtle, in my shell like nah today isn't my day for socializing.
	\item To that 1 soul reading this.
	
	I know you're tired, you're fed up, you're so close to breaking but there's strength within you, even when you feel weak.
	
	Keep fighting.
	\item ``\textit{I have seen beauty in people who were called ugly and I've seen the devil in the most angelic faces}.'' - Conny Cernik
	\item ``\textit{I always loved butterflies because they remind us that it's never too late to \textbf{transform ourselves}.}''- Drew Barrymore
	\item And when you choose a life partner, you're choosing a lot of things, including your parenting partner and someone who will deeply influence your children, your eating companion for about 20,000 meals, your travel companion for about 100 vacations, your primary leisure time and retirement friend, you career therapist, and someone whose day you'll hear about 18,000 times.
	
	Intense shit.
	\item \textit{So you want to be happy?}
	
	Then stop letting the smallest things ruin your whole entire day.
	
	If you're bored with your daily routine, do something unexpected.
	
	Stop complaining about how alone you are when you're surrounded by people who actually care about you.
	
	Forget all the drama and let go of all the grudges you've been holding.
	
	Stop wasting time lingering over all that you could have, should have and would have done.
	
	Stop spending your days thinking of how much better you could do; stop longing for something that has been and always will be out of your reach.
	
	Just live the days as they come.
	
	Wake up every morning and smile at the wonderful day that awaits you.
	
	Take a risk for once.
	
	Let yourself be happy, \textit{because you deserve it}.
	\item Every bad situation will have something positive.
	
	Even a dead clock shows correct time twice a day.
	\item Whatever you hear about me please believe it.
	
	I no longer have time to explain myself.
	
	You can also add some if you want.
	\item My life, my choices.
	
	My mistakes, my lessons, not your business.
	\item ``\textit{The art of knowing is knowing what to ignore}.'' - Rumi
	\item ``\textit{You know what breaks me?}
	
	\textit{When someone is visibly excited about a feeling or an idea or a hope or a risk taken, and they tell you about it but preface it with: ``Sorry, this is dumb but-''}.
	
	\textit{Don't do that}.
	
	\textit{I don't know who came here before me, who conditioned you to think you had to apologize or feel obtuse}.
	
	\textit{But not here}.
	
	\textit{Dream so big it's silly}.
	
	\textit{Laugh so hard it's obnoxious}.
	
	\textit{Love so much it's impossible}.
	
	\textit{And don't you ever feel unintelligent}.
	
	\textit{And don't you ever apologize}.
	
	\textit{And don't you ever shrink so you can squeeze yourself into small places and small minds}.
	
	\textit{Grow}.
	
	\textit{It's a big world}.
	
	\textit{There's room}.
	
	\textit{You fit}.
	
	\textit{I promise}.'' - Owen Lindley
	\item Me complaining that I have no social life when in reality I love staying home and not talk to anyone for several days in a row.
	\item The most beautiful part to loving a guarded girl is this: when she lets you in, it's not because she needs you.
	
	She stopped needing people a long time ago.
	
	It's because she wants you.
	
	And that - that is the purest love of all.
	\item Have you ever looked at someone and hoped they stay in your life forever?
	\item \textbf{Stay true to yourself.}
	
	``\textit{Don't worry about what people think of you or about the way they try to make you feel}.
	
	\textit{If people want to see you as a good person, they will}.
	
	\textit{If they want to see you as a bad person, absolutely nothing you do will stop them}.
	
	\textit{Ironically, the more you try to show them your good intentions, the more reason you give them to knock you down if they are committed to misunderstanding you}.
	
	\textit{Keep your head up high and be confident in what you do}.
	
	\textit{Be confident in your intentions and keep your eyes ahead instead of wasting your time on those who want to drag you back}.
	
	\textit{Because you can't change people's views, you have to believe that true change for yourself comes from within you, not from anyone else}.'' - Najwa Zebian
	\item You will see in the world what you carry in your heart.
	\item I saw a guy today at Starbucks.
	
	He had no smartphone, tablet or laptop.
	
	He just sat there drinking his coffee.
	
	\textit{Like a psychopath}.
	\item \textbf{Ambitchous} [a] The desire to become a better bitch.
	\item Nobody has your back like your Mama$\ldots$
	
	Love her while she's still alive.
	\item 1 thing society must understand: Marrying late is better than marrying wrong.
	\item Use the pain in your past as \textsc{Fuel}.
	
	Fuel that will drive you straight to a better future.
	\item ``\textit{Energy is the currency of the universe}.
	
	\textit{When you `pay' attention to something, you but that experience}.
	
	\textit{So when you allow your consciousness to focus on someone or something that annoys you, you feed it your energy, and it reciprocates the experience of being annoyed}.
	
	\textit{Be selective in your focus because your attention feeds the energy of it and keeps it alive}.
	
	\textit{Not just within you, but in the collective consciousness as well}.'' - Emily Maroutian
	\item \textbf{Introvert inclusion}:
	
	Sometimes we want to be left alone.
	
	Sometimes we want to be included.
	
	Most of the time we want to be included with the option to be left alone.
	\item Your self respect has to be stronger than your feelings.
	\item I hold in a lot.
	
	When I'm in pain, I don't like to worry other people.
	
	No matter how hard I cry, or how much somebody asks, my answers will be, \textit{I'm fine}.
	
	Even if it's not true.
	
	Do you do this too?
	\item Don't consider my kindness as my weakness.
	
	The best in me is sleeping, not dead.
	\item 1 text from right person can change your whole mood.
	\item Don't marry.
	
	Save money \& travel.
	\item Distance doesn't separate people.
	
	\textit{Silence does}.
	\item Best way to earn respect is$\ldots$ by treating others with respect.
	\item Sometimes \textit{home} isn't 4 walls, it's 2 eyes and a heartbeat.
	\item My 3 wishes:
	\begin{itemize}
		\item[1.] To earn money without working.
		\item[2.] To love without being hurt.
		\item[3.] To eat without getting fat.
	\end{itemize}
	\item Don't love too deeply until you're sure that the other person loves you with the same depth.
	
	Because the depth of your love today is the depth of your wound tomorrow.
	\item ``\textit{It's nice to be told you're beautiful or hot or whatever, but I'd love to hear someone say that I make things easier, that they're happy I exist, they don't know what they'd do without me, I'm strong, that they hope we never lose each other, that I have something to offer}.
	
	\textit{Compliments don't always have to be about appearance}.'' - kirbyshayll
	\item Everyone has a friend during each stage of life.
	
	But only lucky ones have the same friend in all stages of life.
	\item Find happiness in simple things.
	\item Being broke is part of the journey.
	
	Staying broke is a fucking choice.
	\item So many plans.
	
	So little money.
	\item I've only met 3 or 4 people that understand me.
	
	Everyone else assumes I'm either angry, sarcastic or just an asshole.
	\item \textit{Bad new is}: You cannot make people like, love, understand, validate, accept or be nice to you.
	
	You can't control them either.
	
	\textit{Good news is}: It doesn't matter.
	\item Sometimes in life we just need a hug$\ldots$ no words, no advice, just a hug to make you feel you matter.
	\item No one is able to make the female a queen except her father$\ldots$!!
	\item The Japanese say you have 3 faces.
	\begin{itemize}
		\item The 1st face, you show to the world.
		\item The 2nd face, you show to your close friends, and your family.
		\item The 3rd face, you never show anyone.
		
		It is the truest reflection of who you are.
	\end{itemize}
	\item ``Why are you wearing that outfit again?''
	
	Because I paid for it and I have a washing machine!!
	\item ``\textit{I drink to make other people more interesting}.'' - Ernest Hemingway
	\item A good teacher is like a candle - it consumes itself to light the way for others.
	\item A relationship should be about helping each other deal with the stress the world brings.
	
	Not adding unnecessary stress to each others lives.
	\item ``\textit{When a person tells you that you hurt them, you don't get to decide that you didn't}.'' - Louis C.K.
	\item I love when someone figures out the small things that you like by paying \textit{such close attention} to you instead of you having to tell them.
	\item \textit{Fact 1}: Reading can make you a better conversationalist.
	
	\textit{Fact 2}: Neighbors will never complain that you are reading too loud.
	
	\textit{Fact 3}: Knowledge by osmosis has not yet been perfected, so you'd better read.
	
	\textit{Fact 4}: Books have stopped bullets.
	
	Reading could save your life.
	
	\textit{Fact 5}: Dinosaurs did not read.
	
	Look what happened to them.
	\item When we think of \textit{meant to be}, we automatically assume forever.
	
	But maybe it isn't supposed to last forever.
	
	Maybe it's just someone who is in your life to teach you something.
	
	Maybe the forever is not the person, but \textit{what we gain from them}.
	\item Sometimes I just agree with people so that they can stop talking.
	\item ``\textit{If you're pretty, you're pretty}.
	
	\textit{But the only way to be beautiful is \textbf{to be loving}.}
	
	\textit{Otherwise, it's just ``Congratulations about your face''}.'' - John Mayer
	\item Notice the people who are happy for your happiness, and sad for your sadness.
	
	They're the ones who \textit{deserve special places in your heart}.
	\item Don't tell your daughter that when a boy is mean or rude to her it's because he has a crush on her.
	
	Don't teach her that abuse is a sign of love.
	\item Maybe I am different than most, but I do not see people in my life as part of a game.
	
	If I am your friend I am loyal until my last breath and if I love you it is until my last beat.
	\item \textit{Help others} - Even when you know they can't help you back.
	\item When a woman replies with ``OK'' as a message.
	
	Go back through the last 200 messages you sent and find your mistake$\ldots$
	\item Being popular on Facebook is like sitting at the cool table in the cafeteria at a mental hospital.
	\item Once you start staying at home, it becomes addiction.
	\item I'd rather be alone, than around chaos and confusion.
	
	\textit{Silence beats drama any day}.
	\item Let them be wrong about you.
	
	There is nothing to prove.
	\item A woman's loyalty is tested when her man has nothing.
	
	A man's loyalty is tested when he has everything.
	\item The biggest lesson I have learned this year is that no one is really your friend or truly loves you until they've seen every dark shadow inside you$\ldots$
	
	\textit{And stayed}.
	\item Never give up on something you really meant.
	
	It's difficult to wait, but it's more difficult to regret.
	\item ``\textit{I've learned a lot this year}$\ldots$
	
	\textit{I learned that things don't always turn out the way you planned, or the way you think they should}.
	
	\textit{And I've learned that there are things that go wrong that don't always get fixed or get put back together the way they were before}.
	
	\textit{ I've learned that some broken things stay broken, and I've learned that you can get through bad times and keep looking for better ones, as long as you have people who love you}.'' - Jennifer Weiner
	\item \textit{Communicate}.
	
	Even when it's uncomfortable or uneasy.
	
	1 of the best ways to heal, is simply getting everything out.
	\item I don't know how to be anything other than intense.
	
	I don't know how to experience without feeling too much and thinking too much.
	
	I don't know how to sit still and quiet my mind and just be.
	
	I am always searching, always questioning, struggling to find the meaning in everything.
	
	I am passionate and I am deep, and even if I am misunderstood, I am finally okay with that.
	\item \textit{How to be happy}: Ignore people who think they know more about you than you do.
	\item Follow your brain.
	
	Your heart is an idiot.
	\item How can people shoot tiktok videos in public places.
	
	Dude I can't even take selfie when someone is looking at me$\ldots$
	\item Don't be afraid to be open-minded.
	
	Your brain isn't going to fall out.
	\item Listen, smile, agree, and then do whatever the fuck you were gonna do anyway.
	\item No one in this world is pure and perfect.
	
	If you avoid people for their mistakes, you will be alone in this world.
	
	So judge less and love more.
	\item A boy was in a taxi eating chocolate, then he took another one, then a man next to him said ``Do you know that will damage your teeth''.
	
	The boy replied: ``My grandfather lived 132 years''.
	
	The man asked: ``Was it because of eating chocolate?''
	
	The boy replied: ``No. He was always minding his own business.''
	\item ``\textit{Go for someone who is proud to have you}.'' - Frank Ocean
	\item We all know a girl who can find out everything about 1 person within an hour.
	\item ``\textit{Travel and tell no one}.
	
	\textit{Live a true love story and tell no one}.
	
	\textit{Live happily and tell no one}.
	
	\textit{People ruin beautiful things}.'' - Kahlil Gibran
	\item You're lucky if you've found a person who never gets tired of understanding your nonsense attitude.
	\item \textit{You know what's attractive?}
	
	Seeing people change for that 1 person.
	
	Whether it's cutting down their drinking habits or to stop doing drugs and maybe even something small like to stop swearing.
	
	It just proves that when it comes down to it, they would do whatever it takes to see that person happy.
	
	Stopping habits is a hard thing to do and to see people actually doing that for someone's love is really attractive to me.
	\item I'm in love with this quote:
	\begin{quotation}
		``\textit{When you get what you want, that's God direction, when you don't get what you want, that's God's protection}.''
	\end{quotation}
	\item Everyone wants to be the sun to brighten up someone's life, but why not be the moon, to shine on someone's darkest hour?
	\item I lost many friends just because I stop texting them 1st.
	\item \textit{I miss you}.
	
	I miss your voice.
	
	I miss your smile.
	
	I miss your smell.
	
	I miss your hugs.
	
	I miss your jokes.
	
	I miss how you made me feel.
	
	\textit{I miss your everything}$\ldots$
	\item It costs \$0.00 to treat someone \textit{with respect}.
	\item Funny thing is when you start feeling happy alone, everyone else wants to be with you.
	\item I love people who are direct even if you have a difference of opinions with them.
	
	At least you know where they stand and they don't play games.
	\item When you pray for someone, you are offering them the purest kind of love.
	\item \textit{A beautiful day beings with a beautiful mindset}.
	
	When you wake up, take a second to think about what privilege it is to simply be alive and healthy.
	
	The moment you start acting like \textit{life is a blessing}, it will start to feel like one.
	\item If you kick me when I'm down, you better pray I don't get up.
	\item Always end the day with a positive thought.
	
	No matter how hard today was, tomorrow is full of possibilities.
	\item So many men think women want money, cars and gifts.
	
	But the right woman wants a man's time, effort, passion, honesty, loyalty, smile, and him choosing to put her as his priority.
	\item Everyone talks about mother's love but no one talks about a father's sacrifice.
	\item \textit{Hugs are actually so underrated}, especially those hugs that are so tight you can literally feel the other person's heartbeat and for a moment everything feels so calm and safe like nothing can hurt you.
	\item Surrender your worries to God, and you will find strength.
	\item If you're not amazed by the stars on a clear night, then we won't work.
	\item - Do you think I'm weird?
	
	- Yeah, but so what?
	
	Everybody's weird.
	\item You can meet somebody tomorrow who has better intentions for you than someone you're known forever.
	
	Time means nothing, character does.
	\item Alphabet:
	\begin{itemize}
		\item \textbf{A}ppreciate
		\item \textbf{B}uild something
		\item \textbf{C}onnect
		\item \textbf{D}o what is difficult
		\item \textbf{E}xplore
		
		\textbf{E}xpress your gratitude
		\item \textbf{F}orgive
		\item \textbf{G}ather
		\item \textbf{H}onor
		\item \textbf{I}gnore the skeptics
		\item \textbf{J}ust be
		\item \textbf{K}now you are loved
		\item \textbf{L}isten
		\item \textbf{M}ake
		\item \textbf{N}ourish body and soul
		\item \textbf{O}bserve
		\item \textbf{P}lant a seed
		\item \textbf{Q}uestion
		\item \textbf{R}ead
		\item \textbf{S}tretch
		\item \textbf{T}ry something new
		\item \textbf{U}nplug
		\item \textbf{V}ote
		\item \textbf{W}onder
		\item Say \textbf{Y}es to adventure
		\item Get enough \textbf{Z}zzzz
	\end{itemize}
	\item I had to forgive a person who wasn't even sorry$\ldots$ that's strength.
	\item Be with someone who puts down their phone and listens to how your day went.
	\item ``\textit{Have the courage to be disliked}.'' - Bruce Lee
	\item It's not about who's real to your face.
	
	It's about who stays loyal behind your back.
	\item Give respect.
	
	Take respect.
	\item Sorry we're looking for someone aged 22-26 with 30 I years of experience.
	\item I would rather be known in life as an honest sinner, than a lying hypocrite.
	\item Should you ever find yourself the victim of other people's bitterness, jealousy, lies, and insecurities.
	
	Don't be mad.
	
	\textit{Remember things could be worse}.
	
	You could be them.
	\item ``\textit{The women whom I love and admire for their \textbf{strength and grace} did not get that way because shit worked out}.
	
	\textit{They got that way because shit went wrong, and they handled it}.
	
	\textit{They handled it in a thousand different ways, on a thousand different days, but they handle it}.
	
	\textbf{\textit{Those women are my superheros}.}'' - Elizabeth Gilbert
	\item ``\textit{Take a \textbf{lover} who looks at you like maybe you are \textbf{magic}.}'' - Frida Kahlo
	\item A kid asks his dad, ``\textit{What's a man?}''.
	
	The dad says, ``\textit{A man is someone who is responsible and cares for their family}.''
	
	The kid says, ``\textit{I hope 1 day I can be a man just like mom!}''.
	\item I tried to be nice but sometimes my mouth doesn't co-operate.
	\item Be mature enough to accept rejections and failures.
	\item Females never listen properly.
	
	Wife: ``\textit{I lost my keys again!}''
	
	Husband: ``\textit{It's in your jeans}.''
	
	Wife: ``\textit{Don't drag my family into this}.''
	\item Have less.
	
	Be more.
	
	Do more.
	\item The true mark of maturity is when somebody hurts you and you try to understand their situation instead of trying to hurt them back.
	\item Even the strongest feelings expire when ignored and taken for granted.
	\item Legends are those who still talks to their \textsc{Ex} as a friend.
	\item If it doesn't open, it's not your door.
	\item You can tell a lot about a person by what's on their playlist.
	\item You do not need to text your friend every single day to still be friends.
	
	I have friends who check out on me every 6 months and I know they are just doing their thing and we understand that it's all good.
	
	Someone's life doesn't have to resolve around you for them to still love you.
	\item I don't care if I'm selfish.
	
	After putting people 1st for the longest time and being disappointed I deserve to do whatever makes me feel happy.
	\item If speaking kindly to plants helps them grow.
	
	Imagine what speaking kindly to humans can do.
	\item ``\textit{If I am lost, find me but do not ask me to come back just yet}.
	
	\textit{Sit with me in this lost place and maybe you will understand why I come here too often, what draws me to my neverland}.
	
	\textit{Find me, but bring me back when I am ready}.
	
	\textit{Maybe you will get to know me a little better}.
	
	\textit{Maybe we can get lost together}.'' - The Dreamer
	\item \textbf{Travel.}
	
	\textbf{Make memories.}
	
	\textbf{Have adventures.}
	
	Because I guarantee when you're on your death bed, you won't think about that flashy car you bought or the 20 pairs of designer shoes you owned.
	
	But you will think about that time you got lost in your favorite city.
	
	The nights spent falling in love under the stars, the mornings you woke up and watched the sunrise and all the beautiful people you met along the way.
	
	You'll think of the moment that made you truly feel alive.
	
	And at the very end, those memories will be only possessions that you own.
	\item Me: ``\textit{I'm sad}.''
	
	Friend: ``\textit{Don't be sad}.''
	
	Me: ``\textit{My goodness, what an idea}.
	
	\textit{Why didn't I think of that?}''
	\item A person becomes 10 times more attractive not by their looks but by their acts of kindness, love, respect, honesty, and loyalty they show.
	\item You can say sorry.
	
	But you can't change the story.
	\item If you need a break from life and go on a trip far away from society and worries.
	\item I love the friends that I haven't seen for days, weeks, months or even years and the bond is still strong as ever.
	\item Always remember, someone's effort is a reflection of their interest in you.
	\item Relatives from father's side are unless!
	\item Reading can seriously damage your ignorance.
	\item I need to spend a good night \textit{laughing too loud} and \textit{eating too much} with my best friends.
	\item There's nothing more comfortable than sleeping while someone gets ready for work.
	\item ``\textbf{\textit{Start over, my darling}.}
	
	\textit{Be brave enough to find the life you want and courageous enough to chase it}.
	
	\textit{Then start over and love yourself the way you were always meant to}.'' - Madalyn Beck
	\item She blocked me, I called her, I begged her to unblock me she unlock me, now I block her.
	
	\textsc{Ego} satisfied.
	\item Be you.
	
	The world will adjust.
	\item ``\textit{Many people think that being spiritual means being positive, but being spiritual means being conscious and aware}.
	
	\textit{To become conscious is a much different thing than to become positive}.
	
	\textit{To become conscious and aware, we must become authentic}.
	
	\textit{Authenticity includes both positive and negative}.'' - Teal Swan
	\item Always remember that everything happens for a reason.
	
	It might not make sense now, but at the right time it will.
	\item ``\textit{I often think that the night is more alive than the day}.'' - Vincent Van Gogh
	\item Unless it's mad, passionate, extraordinary love, it's a waste of your time.
	
	There are too many mediocre things in life.
	
	Love shouldn't be 1 of them. - Dreams for an Insomniac
	\item I like my bed more than I like most people.
	\item Don't rush on anything, when the time is right, it'll happen. - Book of Serenity
	\item ``\textit{I loved her not for the way she danced with my angels but for the way the sound of her name could silence my demons}.'' - Christopher Poindexter
	\item Second chances have never been a problem with me.
	
	I tend to give about 7 or 8 before I realize I'm an idiot.
	\item Wrong is wrong, even if everyone is doing it.
	
	Right is right, even if no one is doing it.
	\item ``\textit{Don't wait for everything to be perfect before you decide to enjoy your life}.'' - Joyce Meyer
	\item Obama retired at 55, Trump started at 70.
	
	Sydney is 3 hours ahead of Perth, but that doesn't make Perth slow.
	
	Someone graduated at the age of 22, but waited 5 years before securing a good job.
	
	Someone became a CEO at 25 and died at 50.
	
	While another became a CEO at 50 and lived to 90 years.
	
	Someone is still single, while someone else got married.
	
	Everyone in this world works based on their time zone.
	
	People around you might seem to be ahead of you and some might seem to be behind you.
	
	But everyone is running their own race, in their own time.
	
	Do not envy them and do not mock them.
	
	They are in their time zone, and you are in yours.
	
	Life is about waiting for the right moment to act.
	
	So relax.
	
	You're not early.
	
	You're not late.
	
	You are very much on time.
	\item Some people will never like you.
	
	Because your spirit irritates their demons.
	\item The lesson:
	\begin{itemize}
		\item not everyone you love will stay,
		\item not everyone you trust will be loyal,
		\item some people only exist as examples of what to avoid.
	\end{itemize}
	\item A fool shows off to get glory.
	
	A wise person stays quiet to find peace.
	\item My mind is more talkative than my mouth.
	\item ``\textit{Stop chasing the wrong one}.
	
	\textit{The right one won't run}.'' - Alfa
	\item Sometimes$\ldots$ when you give a fuck.
	
	That fuck, fucks you up.
	\item I don't trust words anymore.
	
	I only trust actions.
	
	People can pretend to do a lot without being serious about it.
	\item Love is what makes you smile when you're tired.
	\item ``\textit{For too much of my life I've apologized when I wasn't wrong, all to make a situation better}.
	
	\textit{I'm not going to be that person anymore}.'' - Samantha King
	\item Her intuition was her favorite superpower.
	\item Sometimes all you need is a hug from the right person and all the stress just melts away.
	\item Just remember that all the shit someone puts you through, sooner or later finds its way back to them.
	\item I think Panda is my spirit animal.
	
	Lazy, dark circles and always hungry.
	\item Ya wanna know what a real relationship goal is?
	
	Going against all odds in a relationship choosing to make it work regardless of the circumstances, trials, and tribulations cause that's the story of how most of our grandparents made it to 50+ years of marriage.
	
	You don't give up.
	\item Be forgiving.
	
	Be understanding$\ldots$ but don't be a fool.
	\item ``\textit{Don't let anyone tell you that your independence is the reason for you being single}.
	
	\textit{Your strength as a women isn't the cause for your loneliness}.
	
	\textit{You're alone because you'd rather not entertain a weak man}.'' - r.h. Sin
	\item You stop attracting certain people when you heal the parts of you that once needed them.
	\item If we date, there will be moments when \textit{I will just stare at you and smile}, know that in those moments I'm appreciating everything about you.
	\item \textit{True love} isn't found.
	
	It's built.
	\item When someone enters your room switches on light and leaves without switching it off.
	
	Most irritating thing.
	\item I no longer force things.
	
	What flows, flows.
	
	What crashes, crashes.
	
	I only have space and energy for the things that are meant for me.
	\item Someday, we'll forget the hurt, the reason we cried and who caused us pain.
	
	We will finally realize that the secret of being free is not revenge, but letting things unfold in their own way and own time.
	
	After all, what matters is not the 1st, but the last chapter of our life which shows how well we ran the race.
	
	So smile, laugh, forgive, believe and love all over again.
	\item ``\textit{The most dangerous heart disease: strong memory}.'' - Nizar Qabbani
	\item Things take time.
	
	So just be patient.
	\item ``\textit{Strong people don't put other down}.
	
	\textit{They lift them up}.'' - Darth Vader philanthropist
	\item The people in your life should be a source of reducing stress, not causing more of it.
	\item ``\textit{being an introvert is wanting to be invited but not wanting to go anywhere}.
	
	\textit{Being lonely at home but not wanting anyone in your space unless you really like them}.
	
	\textit{And even if you really like them, you want them to go home soon}.'' - Daniel Radcliffe
	\item Understanding is an art, and not everyone is an artist.
	\item Mind says: Move on.
	
	Heart says: Hold on.
	\item I respect people who tell me the truth, no matter how hard it is.
	\item Either stay real or stay away.
	\item ``\textit{Soon you realize that many people will love the idea of you but will lack the maturity to handle the reality of you}.'' - Reyna Biddy
	\item In a society that profits from your self doubt, liking yourself is a rebellious fact.
	\item You'll never find another me.
	
	Not sure if that's a good thing or a bad thing, but it's the truth.
	\item Be strong enough to stand alone, be yourself enough to stand apart, but be wise enough to stand together when the time comes.
	\item A relationship isn't always 50/50.
	
	Some days, a person will struggle.
	
	You suck it up and pick up that 80/20 because they need you.
	
	\textsc{That's love!}
	\item After all, life goes on.
	\item I'll always encourage the reckless texts confessing your feelings.
	
	The kind where you throw your phone after hitting Send.
	
	I'll always encourage the horribly straightforward conversations at 3am when conversations get deep and you can't always put how you feel into words.
	\item The most valuable gift you can receive is an honest friend.
	\item Nothing is free$\ldots$
	
	If you stay in my house, you have to work.
	\item I've always been someone who looks `\textit{too deep}' into something or someone.
	
	That's because I realized from a young age that there's always more than what meets the eye.
	\item ``\textit{You will never understand the damage you did to someone until the same thing is done to you}.
	
	\textit{That's why I'm here}.'' - Karma
	\item Don't be afraid of being outnumbered.
	
	A lion walks along while the sheep flock together.
	\item When no one believes in you, there will be your mom still supporting you.
	\item A friend is someone who listens to your bullshit, tells you that it is bullshit and listens some more.
	\item If 2 girls are together, their main topic would be about ``Boys''.
	\item If she has a job, her own car, pays her bills and lives comfortably, understand that she wants loyalty not your money.
	
	She can finance herself!
	\item My posts will confuse you.
	
	You'll think I'm in love, later on you'll think I'm single and at 1 moment you'll think I'm broken so only 1 thing I want to tell you now is: ``\textit{Mind your own business}''.
	\item Small step in the right direction are better than big ones in the wrong direction.
	\item When you lie on your resume but still get the job.
	\item If they act like they can live without you, \textit{help them do it}.
	\item As you are shifting, you will begin to realize that you are not the same person you need to be.
	
	The things you used to tolerate have now become intolerable.
	
	Where you one remained quiet, you are now speaking your truth.
	
	Where you once battled and argued, you are now choosing to remain silent.
	
	You are beginning to understand the value of your voice and there are some situations that no longer deserve your time, energy, and focus.
	\item Fake people have an image to maintain.
	
	Real people just don't give a fuck.
	\item Happy women's day to Alexa and Siri$\ldots$ the only women who listen to men.
	\item Just because I don't react, doesn't mean I didn't notice.
	\item I reward loyalty with loyalty and disloyalty with distance.
	\item Never let your feelings get too deep, people can change anytime.
	\item And in the end all I learned was how to be strong alone.
	\item No matter how good you are, you can always be replaced.
	\item Don't be a slave to your emotions.
	
	Control them.
	\item Give me 1 song.
	
	1 song only, that means a lot to you.
	
	I will listen to it.
	
	Then I will know you much better.
	\item Some say I'm too sensitive, but truth is I just feel too much.
	
	Every word, every action and every energy goes straight to my heart.
	\item ``\textit{The world will not be destroyed by those who do evil, but by those who watch them without doing anything}.'' - Albert Einstein
	\item I'm cold as ice.
	
	But in the right hands, I'll melt.
	\item I lost myself trying to please everyone else, now I'm losing everyone while trying to find myself.
	\item I love music.
	
	For me, music is morning coffee.
	
	It's mood medicine.
	
	It's pure magic.
	
	A good song is like a good meal I just want to inhale it and then share a bite with someone else.
	\item If someone stays by your side through your worst times, they're the ones who deserve to be with you through your best times.
	\item Sometimes you gotta accept that some things will never go back to how they used to me.
	\item ``\textit{You could give some people a drop of water, and they'd still appreciate you}.
	
	\textit{You could give other people the entire ocean, and they'd still take you for granted}.'' - Yasmin Mogahed
	\item The wiser you get, the less your speak.
	\item It's important to make friendships that are deeper than gossiping, drinking, smoking, and going out.
	
	Make friends who can go get breakfast with, make friends you can cry with, make friends who will support your life goals and believe in you.
	\item Your truest friends are the people who don't walk out the door when life gets hard.
	
	They actually pour some coffee and pull up a chair.
	\item Find 3 hobbies:
	\begin{itemize}
		\item one to make you money,
		\item one to keep you in shape,
		\item and one to keep you creative.
	\end{itemize}
	\item Enjoy every moment you have.
	
	Because in life, there are no rewinds$\ldots$
	\item True love is$\ldots$ getting fat together$\ldots$
	\item Be Honest: Do you still love your ex?
	\item If you've never lost \textit{your mind}, you've never followed \textit{your heart}.
	\item Smile.
	
	It confuses people.
	\item Choose people who choose you.
	\item ``\textit{What you seek is seeking you}.'' - Rumi
	\item I have a therapist.
	
	Her name is music.
	\item Who am I to judge another, when I myself walk imperfectly.
	\item Don't \textit{become} who hurt you.\hfill$\square$
\end{enumerate}

%------------------------------------------------------------------------------%

%------------------------------------------------------------------------------%

\section{{\sc Brian W. Kernighan, P. J. Plauger}. The Elements of Programming Style}

\subsection{Wikipedia{\tt/}The Elements of Programming Style}
``{\it The Elements of Programming Style}, by \href{https://en.wikipedia.org/wiki/Brian_W._Kernighan}{Brian W. Kernighan} \& \href{https://en.wikipedia.org/wiki/P._J._Plauger}{P. J. Plauger}, is a study of \href{https://en.wikipedia.org/wiki/Programming_style}{programming style}, advocating the notion that computer programs should be written not only to satisfy the compiler or personal programming ``style'', but also for ``readability'' by humans, specially \href{https://en.wikipedia.org/wiki/Software_maintenance}{software maintenance} engineers, \href{https://en.wikipedia.org/wiki/Programmers}{programmers}, \& \href{https://en.wikipedia.org/wiki/Technical_writers}{technical writers}. It was originally published in 1974.

The book pays explicit homage\footnote{{\it homage (to somebody{\tt/}something)} something that is said or done to show respect for somebody.}, in title \& tone, to \href{https://en.wikipedia.org/wiki/The_Elements_of_Style}{The Elements of Style}, by Strunk \& White \& is considered a practical template promoting \href{https://en.wikipedia.org/wiki/Edsger_Dijkstra}{Edsger Dijkstra's structured programming} discussions. It has been influential \& has spawned a series of similar texts tailored to individual languages, such as {\it The Elements of C Programming Style, The Elements of C\# Style, The Elements of Java(TM) Style, The Elements of MATLAB Style}, etc.

The book is built on short examples from actual, published programs in programming textbooks. This results in a practical treatment rather than an abstract or academic discussion. The style is diplomatic \& generally sympathetic in its criticism, \& unabashedly honest as well -- some of the examples with which it finds fault are from the authors's own work (1 example in the 2nd edition is from the 1st edition).'' -- \href{https://en.wikipedia.org/wiki/The_Elements_of_Programming_Style}{Wikipedia{\tt/}The Elements of Programming Style}

\subsubsection{Lessons}
``Its lessons are summarized at the end of each section in \href{https://en.wikipedia.org/wiki/Aphorism}{pithy maxims}, such as ``Let the machine do the dirty work'':
\begin{enumerate}
	\item Write clearly -- don't be too clever.
	\item Say what you mean, simply \& directly.
	\item Use library functions whenever feasible.
	\item Avoid too many temporary variables.
	\item Write clearly -- don't sacrifice clarity for efficiency.
	\item Let the machine do the dirty work.
	\item Replace repetitive expressions by calls to common functions.
	\item Parenthesize to avoid ambiguity.
	\item Choose variable names that don't be confused.
	\item Avoid unnecessary branches.
	\item If a logical expression is hard to understand, try transforming it.
	\item Choose a data representation that makes the program simple.
	\item Write 1st in easy-to-understand pseudo language; then translate into whatever language you have to use.
	\item Modularize. Use procedures \& functions.
	\item Avoid gotos completely if you can keep the program readable.
	\item Don't patch bad code -- rewrite t.
	\item Write \& test a big program in small pieces.
	\item Use recursive procedures for recursively-defined data structures.
	\item Test input for plausibility \& validity.
	\item Make sure input doesn't violate the limits of the program.
	\item Terminate input by end-of-file marker, not by count.
	\item Identify bad input; recover if possible.
	\item Make input easy to prepare \& output self-explanatory.
	\item Use uniform input formats.
	\item Make input easy to proofread.
	\item Use self-identifying input. Allow defaults. Echo both on output.
	\item Make sure all variables are initialized before use.
	\item Don't stop at 1 bug.
	\item Use debugging compilers.
	\item Watch out for off-by-1 errors.
	\item Take care to branch the right way on equality.
	\item Be careful if a loop exits to the same place from the middle \& the bottom.
	\item Make sure you code does ``nothing'' gracefully\footnote{1. in an attractive way that shows control; showing a smooth, attractive form; 2. in a polite \& kind way, especially in a difficult situation.}.
	\item Test programs at their boundary values.
	\item Check some answers by hand.
	\item {\tt10.0} times {\tt0.1} is hardly ever {\tt1.0}.
	\item {\tt7/8} is zero while {\tt7.0/8.0} is not zero.
	\item Don't compare floating point numbers solely for equality.
	\item Make it right before you make it faster.
	\item Make it fail-safe before you make it faster.
	\item Make it clear before you make it faster.
	\item Don't sacrifice clarity for small gains in efficiency.
	\item Let your compiler do the simple optimizations.
	\item Don't strain to reuse code; reorganize instead.
	\item Make sure special cases are truly special.
	\item Keep it simple to make it faster.
	\item Don't diddle code to make it faster -- find a better algorithm.
	\item Instrument your programs. Measure before making efficiency changes.
	\item Make sure comments \& code agree.
	\item Don't just echo the code with comments -- make every comment count.
	\item Don't comment bad code -- rewrite it.
	\item Use variable names that mean something.
	\item Use statement labels that mean something.
	\item Format a program to help the reader understand it.
	\item Document your data layouts.
	\item Don't over-comment.
\end{enumerate}
Modern readers may find it a shortcoming that its examples use older \href{https://en.wikipedia.org/wiki/Procedural_programming_languages}{procedural programming languages} (\href{https://en.wikipedia.org/wiki/Fortran}{Fortran} \& \href{https://en.wikipedia.org/wiki/PL/I}{PL{\tt/}I}) that are quite different from those popular today. Few of today's popular languages had been invented when this book was written. However, many of the book's points that generally concern stylistic \& structural issues transcend \& details of particular languages.

\href{https://en.wikipedia.org/wiki/Kilobaud_Microcomputing}{Kilobaud Microcomputing} stated that ``If you intend to write programs to be used by other people, then you should read this book. If you expect to become a professional programmer, this book is mandatory reading.''

\subsection{Software Quotes{\tt/}P. J. Plauger}

\begin{enumerate}
	\item Make your programs read from top to bottom.
	\item Let the machine do the dirty work.
	\item Where there are 2 bugs, there is likely to be a 3rd.
	\item Choose a data representation that makes the program simple.
	\item Take care to branch the right way on equality.
	\item Let the data structure the program.
	\item Test input for validity \& plausibility.
	\item Make sure your code `does nothing' gracefully.
	\item Don't patch bad code -- rewrite it.
	\item His major concern is: ``The principle of 1 Right Place -- there should be 1 Right Place to look for any nontrivial piece of code, \& 1 Right Place to make a likely maintenance change.''
	\item Don't stop with your 1st draft.
	\item The more dogmatic\footnote{being certain that your beliefs are right \& that others should accept them, without paying attention to evidence or other opinions.} you are about applying a design method, the fever real-life problems you are going to solve.
	\item Make it right before you make it fast. Make it clear before you make it fast. Keep it right when you make it faster.
	\item People who preach software design as a disciplined activity spend considerable energy making us all feel guilty. We can never be structured enough or object-oriented enough to achieve nirvana in this lifetime. We all truck around a kind of original sin from having learned Basic at an impressionable age. But my bet is that most of us are better designers than the purists will ever acknowledge.
	
	-- Những người rao giảng thiết kế phần mềm như một hoạt động có kỷ luật đã tiêu tốn rất nhiều công sức khiến tất cả chúng ta đều cảm thấy tội lỗi. Chúng ta không bao giờ có thể có đủ cấu trúc hoặc đủ hướng đến đối tượng để đạt được niết bàn trong cuộc đời này. Tất cả chúng ta đều mắc phải một loại tội lỗi nguyên thủy do đã học ngôn ngữ Basic (Cơ bản) ở độ tuổi dễ bị ảnh hưởng. Nhưng tôi cá rằng hầu hết chúng ta đều là những nhà thiết kế giỏi hơn những gì những người theo chủ nghĩa thuần túy thừa nhận.
\end{enumerate}

%------------------------------------------------------------------------------%

\section{{\sc Terence Tao}}

\subsection{\href{https://terrytao.wordpress.com/career-advice/}{Career Advice}}

\begin{quotation}
	\textit{Advice is what we ask for when we already know the answer but wish we didn't}. - \href{http://en.wikipedia.org/wiki/Erica_Jong}{Erica Jong}
\end{quotation}
Here is Terence Tao's collection of various pieces of advice on academic career issues in mathematics, roughly arranged by the stage of career at which the advice is most pertinent (though of course some of the advice pertains to multiple stages).

\begin{remark}[Disclaimer]
	The advice here is very generic in nature; Terence Tao does not pretend to have any sort of ``silver bullet'' that will solve all career issues.
	
	You will of course need to evaluate many factors, contexts, and needs specific to your own situation, as well as employing a healthy dose of common sense, before making any important career decisions.
	
	Terence Tao would in particular recommend \href{https://terrytao.wordpress.com/career-advice/talk-to-your-advisor/}{discussing such decisions with your advisor} if you have one, as he or she will be familiar with your situation and will likely be able to provide pertinent advice.
	
	Also, it should be clear that most of this advice is targeted towards academic careers in mathematics; of course, there are many other career options available besides this, but Terence Tao has no particularly informed advice to offer for such alternatives.
\end{remark}

\subsubsection{\href{https://terrytao.wordpress.com/career-advice/talk-to-your-advisor/}{Talk to your advisor}}

\begin{quotation}
	\textit{It is the province of knowledge to speak and it is the privilege of wisdom to listen}. - \href{https://en.wikipedia.org/wiki/Oliver_Wendell_Holmes_Sr.}{Oliver Wendell Holmes}, ``The Poet at the Breakfast Table''
\end{quotation}
Your advisor is 1 of the best sources of guidance you have; not only in directly assisting you with your \textit{research topic}, but in directing you (both explicitly and implicitly) to \textit{relevant researchers, conferences, publications, open problems, folklore}, or \textit{other pieces of good mathematics}.

Your advisor also knows your situation well and can give career advice which is tailored to your specific strengths and weaknesses (unlike the \href{https://terrytao.wordpress.com/career-advice/}{generic} advice in these pages).

%
If things get to the point that you are actively avoiding your advisor (or vice versa), that is a very bad sign. In particular, you should be aware of your advisor's schedule, and conversely your advisor should be aware of when you will be available in the department, and what you are currently working on.

%
For similar reasons, you should give your advisor some advance warning if you want to take a long period of time away from your studies.

%
If your advisor is unavailable, you should regularly discuss mathematical issues with at least 1 other mathematician instead, preferably an experienced one.

[Also, it is not uncommon for a student to have both a formal advisor, who handles all the official paperwork, and an informal advisor, with which you discuss research and career issues.]

%
Of course, you should not rely \textit{purely} on your advisor; you also need to \href{https://terrytao.wordpress.com/career-advice/take-the-initiative/}{take the initiative} when it comes to your mathematical career.

\subsubsection{\href{https://terrytao.wordpress.com/career-advice/take-the-initiative/}{Take the initiative}}

\begin{quotation}
	\textit{The best teacher is the one who suggests rather than dogmatizes, and inspires his listener with the wish to teach himself}. - \href{https://en.wikipedia.org/wiki/Edward_Bulwer-Lytton}{Edward Bulwer-Lytton}
\end{quotation}
While you should talk to your advisor, you should not be completely reliant on him or her; after all, you are going to have to do mathematics primarily on your own once you graduate!

%
If you feel like you want to learn something, do something, or write something, you do not have to clear it with your advisor - just go ahead and do it (though in some cases other priorities, such as writing your thesis, may be temporarily more important, and you should of course keep your advisor updated as to what you are doing mathematically).

Research your library or the internet, talk with other graduate students or faculty, read papers and books on your own (both \href{https://terrytao.wordpress.com/career-advice/learn-and-relearn-your-field/}{in your field} and in \href{https://terrytao.wordpress.com/career-advice/don't-be-afraid-to-learn-things-outside-your-field/}{nearby fields}), \href{https://terrytao.wordpress.com/career-advice/attend-talks-and-conferences-even-those-not-directly-related-to-your-work/}{attend conferences}, and so forth. (See also ``\href{https://terrytao.wordpress.com/career-advice/ask-yourself-dumb-questions-and-answer-them/}{ask yourself dumb questions}''.)

%
1 specific suggestion Terence Tao has is to subscribe (either \href{http://www.arxiv.org/help/rss}{by RSS}, or \href{http://arxiv.org/help/subscribe}{by email}) to be notified of new papers which appear on the \href{http://www.arxiv.org/}{arXiv} in the subject areas that you are interested in.

%
In a somewhat related spirit, while it is certainly acceptable to have mathematical role models, one should not try to mimic them too slavishly; you need to \fbox{\href{https://terrytao.wordpress.com/advice-on-writing-papers/write-in-your-own-voice/}{develop your own personal style}}, \fbox{\textit{exploiting your own strengths}} and \textit{mitigating your own weaknesses}, which will not be identical to those of your role models.

\textit{Ultimately, it is \fbox{better to follow the mathematics than to follow a mathematician}.}

\subsubsection{\href{https://terrytao.wordpress.com/career-advice/learn-and-relearn-your-field/}{Learn \& relearn your field}}

\begin{quotation}
	\it
	Even fairly good students, when they have obtained the solution of the problem and written down neatly the argument, shut their books and look for something else.
	
	Doing so, they miss an important and instructive phase of the work. $\ldots$
	
	A good teacher should understand and impress on his students the view that no problem whatever is completely exhausted.
	
	1 of the 1st and foremost duties of the teacher is not to give his students the impression that mathematical problems have little connection with each other, and no connection at all with anything else.
	
	We have a natural opportunity to investigate the connections of a problem when looking back at its solution. - \href{http://en.wikipedia.org/wiki/George_P%C3%B3lya}{\emph{George Pólya}}, ``\href{http://en.wikipedia.org/wiki/How_to_Solve_It}{How to Solve It}''
\end{quotation}
Learning never really stops in this business, even in your chosen specialty; e.g. Terence Tao is still learning surprising things about basic harmonic analysis, more than 10 years after writing his thesis in the topic.

%
Just because you know a statement and proof of Fundamental Lemma X, you should not take that lemma for granted; instead, you should dig deeper until you really understand what the lemma is all about:
\begin{itemize}
	\item \textit{Can you find alternate proofs?}
	\item \textit{If you know 2 proofs of the lemma, do you know to what extent the proofs are equivalent?}
	
	\textit{Do they generalize in different ways?}
	
	\textit{What themes do the proofs have in common?}
	
	\textit{What are the other relative strengths and weaknesses of the 2 proofs?}
	\item \textit{Do you know why each of the hypotheses are necessary?}
	\item \textit{What kind of generalizations are known/conjectured/heuristic?}
	\item \textit{Are there weaker and simpler versions which can suffice for some applications?}
	\item \textit{What are some model examples demonstrating that lemma in action?}
	\item \textit{When is it a good idea to use the lemma, and when isn't it?}
	\item \textit{What kind of problems can it solve, and what kind of problems are beyond its ability to assist with?}
	\item \textit{Are there analogues of that lemma in other areas of mathematics?}
	\item \textit{Does the lemma fit into a wider paradigm or program?}
\end{itemize}
It is particularly useful to lecture on your field, or \href{https://terrytao.wordpress.com/career-advice/write-down-what-youve-done/}{write lecture notes or other expository material}, even if it is just for your own personal use.

You will eventually be able to internalize even very difficult results using efficient mental shorthand; this not only allows you to use these results effortlessly, and \href{https://terrytao.wordpress.com/career-advice/continually-aim-just-beyond-your-current-range/}{improve your own ability} in the field, but also \textit{frees up mental space to learn even more material}.

%
Another useful way to learn more about one's field is to take a \fbox{key paper} in that field, and perform a citation search on that paper (i.e. search for other papers that cite the key paper).

There are many tools for citation searches nowadays; e.g., \href{http://www.ams.org/mathscinet/search.html}{MathSciNet} offers this functionality, and even a general-purpose web search engine can often give useful ``hits'' that one might not have previously been aware of.

\subsubsection{\href{https://terrytao.wordpress.com/career-advice/write-down-what-youve-done/}{Write down what you've done}}

\begin{quotation}
	\textit{Every composer knows the anguish and despair occasioned by forgetting ideas which one had no time to write down}. - \href{https://en.wikipedia.org/wiki/Hector_Berlioz}{Hector Berlioz}
\end{quotation}
There were many occasions early in his career when Terence Tao read, heard about, or stumbled upon some neat mathematical trick or argument, and thought he understood it well enough that he didn't need to write it down; and then, say 6 months later, when he actually needed to recall that trick, he couldn't reconstruct it at all.

Eventually he resolved to write down (preferably on a computer) a sketch of any interesting argument he came across - not necessarily at a publication level of quality, but detailed enough that he could then safely forget about the details, and readily recover the argument from the sketch whenever the need arises.

%
Terence Tao recommend that you do this also, as it serves several useful purposes:
\begin{enumerate}
	\item It makes the argument permanently available to you in the future, and may eventually be helpful in your later research papers, lecture notes, teaching, or research proposals.
	\item It gives you practice in mathematical writing, both at the technical level (e.g. in learning how to use TeX) and at an expository or pedagogical level.
	\item It tests whether you have really understood the argument on more than just a superficial level.
	\item It frees up mental space; you no longer have to remember the exact details of the argument, and so can devote your memory to learning newer topics.
\end{enumerate}
Once you have written up such a sketch, you might consider \href{https://terrytao.wordpress.com/career-advice/make-your-work-available/}{making it available} (e.g. on your web site), even if it does not rise to the level of originality and depth required for a publishable paper.

%
For somewhat similar reasons, if you have an incomplete (or otherwise unsatisfactory) argument for a problem that you are working on, and you are \href{https://terrytao.wordpress.com/career-advice/use-the-wastebasket/}{planning to abandon it}, you may still wish to write an informal sketch of it just for yourself (giving barely enough details to allow you to readily reconstruct the whole thing later on), and store it somewhere on your computer, just in case you find you have need for it some time in the future.

(Of course, Terence Tao also recommends some way of backing up your computer files, e.g. by using a cloud-based file storage system such as Dropbox.)

%------------------------------------------------------------------------------%

\section{{\sc William Strunk Jr., E. B. White}. The Elements of Style}
\textbf{\textsf{Resources -- Tài nguyên.}}
\begin{enumerate}
	\item \cite{Strunk_element_style}. {\sc William Strunk Jr.} {\it The Elements of Style}.
	\item \cite{Strunk_White_element_style}. {\sc William Strunk Jr., E. B. White}. {\it The Elements of Style}.
\end{enumerate}

\subsection{Wikipedia{\tt/}The Elements of Style}
``{\it The Elements of Style} (also called {\it Strunk \& White}) is a \href{https://en.wikipedia.org/wiki/Style_guide}{style guide} for formal grammar used in \href{https://en.wikipedia.org/wiki/American_English}{American English} writing. The 1st publishing was written by \href{https://en.wikipedia.org/wiki/William_Strunk_Jr.} n 1918, \& published by \href{https://en.wikipedia.org/wiki/Harcourt_(publisher)}{Harcourt} in 1920, comprising 8 ``elementary rules of usage,'' 10 ``elementary principles of composition,'' ``a few matters of form,'' a list of 49 ``words \& expressions commonly misused,'' \& a list of 57 ``words often misspelled.'' Writer \& editor \href{https://en.wikipedia.org/wiki/E._B._White}{E. B. White} greatly enlarged \& revised the book for publication by \href{https://en.wikipedia.org/wiki/Macmillan_Publishers}{Macmillan} in 1959. That was the 1st edition of the book, which \href{https://en.wikipedia.org/wiki/Time_(magazine)}{Time} recognized in 2011 as 1 of the 100 best \& most influenced non-fiction books written in English since 1923. American wit Dorothy Parker said, regarding the book:
\begin{quote}
	``If you have any young friends who aspire to become writers, the 2nd-greatest favor you can do them is to present them with copies of {\it The Elements of Style}. The 1st greatest, of course, is to shoot them now, while they're happy.'' -- \href{https://en.wikipedia.org/wiki/The_Elements_of_Style}{Wikipedia{\tt/}Elements of Style}
\end{quote}

\subsubsection{Content}
See \href{https://en.wikipedia.org/wiki/The_Elements_of_Style}{Wikipedia{\tt/}The Elements of Style}. ``Strunk concentrated on the cultivation of good writing \& composition; the original 1918 edition exhorted writers to ``omit needless words'', use the \href{https://en.wikipedia.org/wiki/Active_voice}{active voice}, \& employ \href{https://en.wikipedia.org/wiki/Parallelism_(grammar)}{parallelism} appropriately.'' [$\ldots$] ``The 3rd edition of {\it The Elements of Style} (1979) features 54 points: a list of common word-usage errors; 11 rules of punctuation \& grammar; 11 principles of writing; 11 matters of form; \&, in Chap. V, 21 reminders for better style. The final reminder, the 21st, ``Prefer the standard to the offbeat\footnote{{\bf offbeat} [a] [usually before noun] ({\it informal}) different from what most people expect, {\sc synonym}: {\bf unconventional}.}'', is thematically integral\footnote{{\bf integral} [a] {\bf 1.} being an essential part of something; {\bf 2.} [usually before noun] included as part of something, rather than supplied separately; {\bf 3.} [usually before noun] having all the parts that are necessary for something to be complete.} to the subject of {\it The Elements of Style}, yet does stand as a discrete\footnote{{\bf discrete} [a] ({\it formal or specialist}) independent of other things of the same type, {\sc synonym}: {\bf separate}.} essay about writing lucid\footnote{{\bf lucid} [a] {\bf 1.} clearly expressed; easy to understand, {\sc synonym}: clear; {\bf 2.} able to think clearly, especially when somebody cannot usually do this.} prose\footnote{{\bf prose} [n] [uncountable] writing that is not poetry.}. To write well, White advises writers to have the proper\footnote{{\bf proper} [a] {\bf 1.} [only before noun] ({\it especially British English}) right, appropriate or correct; according to the rules, {\sc opposite}: {\bf improper}; {\bf 2.} [only before noun] {\it British English}) considered to be real \& of a good enough standard; {\bf 3.} socially \& morally acceptable, {\sc opposite}: {\bf improper}; {\bf 4.} [after noun] according to the most exact meaning of the word; {\bf 5. proper to somebody{\tt/}something} belonging to a particular type of person or thing; natural in a particular situation or place.} mind-set, that they write to please themselves, \& that they aim for ``1 moment of felicity\footnote{{\bf felicity} [n] {\bf 1.} [uncountable] great happiness; {\bf 2.} [uncountable] the quality of being well chosen or suitable; {\bf 3. felicities} [plural] well-chosen or successful features, especially in a speech or piece of writing.}'', a phrase by \href{https://en.wikipedia.org/wiki/Robert_Louis_Stevenson}{Robert Louis Stevenson}. Thus Strunk's 1918 recommendation:
\begin{quotation}
	``Vigorous\footnote{{\bf vigorous} [a] {\bf 1.} involving physical strength, effort or energy; {\bf 2.} done with determination, energy or enthusiasm; {\bf 3.} strong \& healthy.} writing is concise\footnote{{\bf concise} [a] giving only the information that is necessary \& important, using few words.}. A sentence should contain no unnecessary words, a paragraph no unnecessary sentences, for the same reason that a drawing should have no unnecessary lines \& a machine no unnecessary parts. This requires not that the writer make all his sentences short, or that he avoid all detail \& treat his subjects only in outline, but that he make every word tell.'' -- ``Elementary Principles of Composition'', {\it The Element of Style} \cite{Strunk_element_style}''
\end{quotation}
[$\ldots$] ``The 4th edition of {\it The Elements of Style} (2000), published 54 years after Strunk's death, omits his stylistic\footnote{{\bf stylistic} [a] [only before noun] connected with the style that a writer, artist or musician uses.} advice about masculine\footnote{{\bf masculine} [a] {\bf 1.} having the qualities or appearance considered to be typical of men; connected with or like men; {\bf 2.} (in some languages) belonging to a class of nouns, pronouns or adjectives that have masculine gender, not feminine or neuter.} pronouns: ``unless the antecedent\footnote{{\bf antecedent} [n] a thing or an event that exists or comes before something else \& has an influence on it; [a] existing or coming before something else, \& having an influence on it.} is or must be feminine''. In its place, the following sentence has been added: ``many writers find the use of the generic {\it he} or {\it his} to rename indefinite antecedents limiting or offensive.'' Further, the retitled entry ``They. He or she'', in Chap. IV: {\it Misused Words \& Expressions}, advises the writer to avoid an ``unintentional emphasis on the masculine''.'' -- \href{https://en.wikipedia.org/wiki/The_Elements_of_Style#Content}{Wikipedia{\tt/}The Element of Style{\tt/}content}

\subsubsection{Reception}
``{\it The Elements of Style} was listed as 1 of the 100 best \& most influential\footnote{{\bf influential} [a] having a lot of influence on the way that somebody{\tt/}something behaves or develops, or on the way that somebody thinks.} books written in English since 1923 by {\it Time} in its 2011 list. Upon its release, Charles Poor, writing for \href{https://en.wikipedia.org/wiki/The_New_York_Times}{{\it The New York Times}}, called it ``a splendid\footnote{{\bf splendid} [a] ({\it especially British English}) {\bf 1.} very impressive; very beautiful; {\bf 2.} ({\it old-fashioned}) excellent; very good, {\sc synonym}: great.} trophy for all who are interested in reading \& writing.'' American poet \href{https://en.wikipedia.org/wiki/Dorothy_Parker}{Dorothy Parker} has, regarding the book, said:
\begin{quotation}
	``If you have any young friends who aspire to become writers, the 2nd-greatest favor you can do them is to present them with copies of {\it The Elements of Style}. The 1st-greatest, of course, is to shoot them now, while they're happy.''
\end{quotation}
Criticism\footnote{{\bf criticism} [n] {\bf 1.} [uncountable, countable] the act of expressing disapproval of somebody{\tt/}something \& opinions about their faults or bad qualities; a statement showing disapproval; {\bf 2.} [uncountable] the work or activity of analyzing \& making fair, careful judgments about somebody{\tt/}something, especially books, music, etc.} of {\it Strunk \& White} has largely focused on claims that it has a \href{https://en.wikipedia.org/wiki/Linguistic_prescriptivism}{prescriptivist}\footnote{{\bf prescriptive} [a] {\bf 1.} telling people what should be done or how something should be done; {\bf 2.} ({\it linguistics}) telling people how a language should be used, rather than describing how it is used, {\sc opposite}: {\bf descriptive}.} nature, or that it has become a general \href{https://en.wikipedia.org/wiki/Anachronism}{anachronism}\footnote{{\bf anachronism} [n] {\bf 1.} [countable] a person, a custom or an idea that seems old-fashioned \& does not belong to the present; {\bf 2.} [countable, uncountable] something that is placed, e.g., in a book or play, in the wrong period of history; the fact of placing something in the wrong period of history.} in the face of modern English usage.

In criticizing {\it The Elements of Style}, \href{https://en.wikipedia.org/wiki/Geoffrey_Pullum}{Geoffrey Pullum}, professor of \href{https://en.wikipedia.org/wiki/Linguistics}{linguistics} at the \href{https://en.wikipedia.org/wiki/University_of_Edinburgh}{University of Edinburgh}, \& co-author of \href{https://en.wikipedia.org/wiki/The_Cambridge_Grammar_of_the_English_Language}{{\it The Cambridge Grammar of the English Language}} (2002), said that:
\begin{quotation}
	``The book's toxic mix of \href{https://en.wikipedia.org/wiki/Linguistic_purism}{purism}\footnote{{\bf purism} [n] [uncountable] the belief that things should be done in the traditional way \& that there are correct forms in languages, art, etc. that should be followed.}, \href{https://en.wikipedia.org/wiki/Atavism}{atavism}, \& personal \href{https://en.wikipedia.org/wiki/Eccentricity_(behavior)}{eccentricity}\footnote{{\bf eccentricity} [n] {\bf 1.} [uncountable] behavior that people think is strange or unusual; the quality of being unusual \& different from other people; {\bf 2.} [countable, usually plural] an unusual act or habit.} is not underpinned\footnote{{\bf underpin} [v] to support or form the basis of something.} by a proper grounding\footnote{{\bf grounding} [n] [singular, uncountable] knowledge \& understanding of the basic parts of a subject; a basis for something.} in English grammar. It is often so misguided that the authors appear not to notice their own egregious\footnote{{\bf egregious} [a] ({\it formal}) extremely bad.} flouting\footnote{{\bf flout} [v] {\bf flout something} to show that you have no respect for a law, etc. by openly not obeying it, {\sc synonym}: {\bf defy}.} of its own rules $\ldots$ It's sad. Several generations of college students learned their grammar from the uninformed\footnote{{\bf uninformed} [a] having or showing a lack of knowledge or information about something, {\sc opposite}: informed.} bossiness\footnote{{\bf bossiness} [n] [uncountable] ({\it disapproving}) bossy behavior.} of {\it Strunk \& White}, \& the result is a nation of educated people who know they feel vaguely\footnote{{\bf vaguely} [adv] {\bf 1.} in a way that is not detailed or exact; {\bf 2.} slightly.} anxious\footnote{{\bf anxious} [a] {\bf 1. anxious (about something)} feeling worried or nervous; {\bf 2.} wanting something very much.} \& insecure\footnote{{\bf insecure} [a] {\bf 1.} not confident, especially about yourself or your abilities, {\sc opposite}: {\bf secure}; {\bf 2.} not safe or protected, {\sc opposite}: {\bf secure}.} whenever they write {\it however} or {\it than me} or {\it was} or {\it which}, but can't tell you why.''
\end{quotation}
Pullum has argued, e.g., that the authors misunderstood what constitutes the \href{https://en.wikipedia.org/wiki/English_passive_voice}{passive voice}\footnote{NQBH: Personally, I prefer the passive voice to the active one.}, \& he criticized their proscription\footnote{{\bf proscription} [n] [countable, uncountable] ({\it formal}) {\bf proscription (against{\tt/}on something)} the act of saying officially that something is banned; the stat of being banned.} of established \& unproblematic\footnote{{\bf unproblematic} [a] not having or causing problems, {\sc opposite}: {\bf problematic}.} English usages, e.g. the \href{https://en.wikipedia.org/wiki/Split_infinitive}{split infinitive} \& the use of {\it which} in a restrictive \href{https://en.wikipedia.org/wiki/English_relative_clause#That_or_which}{relative clause}. On \href{https://en.wikipedia.org/wiki/Language_Log}{Language Log}, a blog about language written by \href{https://en.wikipedia.org/wiki/Linguists}{linguists}, he further criticized {\it The Elements of Style} for promoting \href{https://en.wikipedia.org/wiki/Linguistic_prescriptivism}{linguistic precriptivism} \& \href{https://en.wikipedia.org/wiki/Hypercorrection}{hypercorrection} among \href{https://en.wikipedia.org/wiki/Anglophones}{Anglophones}, \& called it ``the book that ate American's brain''.

\href{https://en.wikipedia.org/wiki/The_Boston_Globe}{{\it The Boston Globe}}'s review described {\it The Elements of Style Illustrated} (2005), with illustrations by Maira Kalman, as an ``aging zombie of a book $\ldots$ a hodgepodge\footnote{{\bf hodgepodge} [n] ({\it North American English}) (also {\bf hotchpotch}, {\it especially in British English}) [singular] ({\it informal}) a number of things mixed together without any particular order or reason.}, its now-antiquated\footnote{{\bf antiquated} [a] ({\it usually disapproving}) (of things or ideas) old-fashioned \& no longer suitable for modern conditions, {\sc synonym}: {\bf outdated}.} \href{https://en.wikipedia.org/wiki/Pet_peeve}{pet peeves} jostling for\footnote{{\bf jostle for} [phrasal verb] {\bf jostle for something} to compete strongly \& with force for something.} space with 1970s taboos\footnote{{\bf taboo} [n] {\bf 1. taboo (against{\tt/}on something)} a cultural or religious custom that does not allow people to do, use or talk about a particular thing; {\bf 2. taboo (against{\tt/}on something)} a general agreement not to do something or talk about something.} \& 1990s computer advice''.

Nevertheless, many contemporary\footnote{{\bf contemporary} [a] {\bf 1.} belonging to the present time, {\sc synonym} {\bf modern}; {\bf 2.} (especially of people \& society) belonging to the same time as somebody{\tt/}something else.} authors still recommend it highly. Their praise\footnote{{\bf praise} [v] {\bf 1.} to express your approval or admiration for somebody{\tt/}something; {\bf 2. praise God} to express your thanks to or your respect for God.} tends to focus on its characterization\footnote{{\bf characterization} [n] [uncountable, countable] {\bf 1. characterization (of something)} the process of discovering or describing the qualities or features of something; the result of this process; {\bf 2.} the way in which the characters in a story, play or film are made to seem real.} of \fbox{good writing \& how to achieve it}, grammar being just 1 element of that purpose. In \href{https://en.wikipedia.org/wiki/On_Writing:_A_Memoir_of_the_Craft}{On writing} (2000, p. 11), \href{https://en.wikipedia.org/wiki/Stephen_King}{Stephen King} writes:
\begin{quotation}
	``There is little or no detectable \href{https://en.wikipedia.org/wiki/Bullshit}{bullshit} in that book. (Of course, it's short; at 85 pages it's much shorter than this one.) I'll tell you right now that every aspiring writer should read {\it The Elements of Style}. Rule 17 in the chapter titled {\it Principles of Composition} is `Omit needless words.' I will try to do that here.''
\end{quotation}
In 2011, Tim Skern remarked that {\it The Elements of Style} ``remains the best book available on writing good English.''

In 2013, \href{https://en.wikipedia.org/wiki/Nevile_Gwynne}{Nevile Gwynne} reproduced {\it The Elements of Style} in his work \href{https://en.wikipedia.org/wiki/Gwynne%27s_Grammar}{{\it Gwynne's Grammar}}. Britt Peterson of the \href{https://en.wikipedia.org/wiki/Boston_Globe}{{\it Boston Globe}} wrote that his inclusion of the book was a ``curious\footnote{{\bf curious} [a] {\bf 1.} having a strong desire to know about something; {\bf 2.} strange \& unusual.} addition''.

In 2016, the Open Syllabus Project lists {\it The Elements of Style} as the most frequently assigned text in US academic \href{https://en.wikipedia.org/wiki/Syllabus}{syllabuses}, based on an analysis of 933,635 texts appearing in over 1 million syllabuses.'' -- \href{https://en.wikipedia.org/wiki/The_Elements_of_Style#Reception}{Wikipedia{\tt/}The Elements of Style{\tt/}reception}

``The 1st writer I watched at work was my stepfather, E. B. White.\footnote{Sự ảnh hưởng, đặc biệt đến nhân cách \& việc lựa chọn nghề nghiệp, của những hình mẫu đầu tiên mà ta, 1 cách tình cờ hay được số phận sắp đặt, gặp gỡ trong cuộc đời.} Each Tuesday morning, he would close his study door \& sit down to write the ``Notes \& Comment'' page for {\it The New Yorker}. The task was familiar to him -- he was required to file a few hundred words of editorial\footnote{{\bf editorial} [a] [usually before noun] connected with the task of preparing something e.g. a newspaper, a book, or a television or radio programme, to be published or broadcast; [n] an important article in a journal or a newspaper, that expresses the editor's opinion about an issue.} of personal commentary on some topic in or out of the news that week -- but the sounds of his typewriter\footnote{{\bf typewriter} [n] a machine that produces writing similar to print. It has keys that you press to make metal letters or signs hit a piece of paper through a long, narrow piece of cloth covered with ink ($=$ colored liquid).} \footnote{NQBH: I like the term ``typewriter'' in any literary scene., which sounds traditional \& sexy, opposite to personal notebooks{\tt/}laptop now: modern \& robust.} from his room came in hesitant\footnote{{\bf hesitant} [a] slow to speak or act because you feel uncertain, embarrassed or unwilling.} bursts\footnote{{\bf burst} [v] {\bf 1.} [intransitive, transitive] to break open or apart, especially because of pressure from inside; to make something break in this way; {\bf 2.} [intransitive] {\bf $+$ adv.{\tt/}prep.} to go or come from somewhere suddenly; {\bf burst into something} [phrasal verb] to start producing something suddenly \& with great force; [n] a short period of a particular activity or strong emotion that often starts suddenly.}, with long silences in between. Hours went by. Summoned at last for lunch, he was silent \& preoccupied\footnote{{\bf preoccupied} [a] thinking \&{\tt/}or worrying continuously about something so that you do not pay attention to other things.}, \& soon excused himself to get back to the job. When the copy went off at last, in the afternoon RFD pouch\footnote{{\bf pouch} [n] {\bf 1.} a small bag, usually made of leather, \& often carried in a pocket or attached to a belt; {\bf 2.} a large bag for carrying letters, especially official ones; {\bf 3.} a pocket of skin on the stomach of some female marsupial animals, e.g. kangaroos, in which they carry their young; {\bf 4.} a pocket of skin in the cheeks of some animals, e.g. hamsters, in which they store food.} -- we were in Maine, a day's mail away from New York -- he rarely seemed satisfied. \fbox{``It isn't good enough.''}\footnote{``The quest for perfection can never end.''} he said sometimes, \fbox{``I wish it were better.''}

\fbox{Writing is hard}, even for authors who do it all the time. Less frequent practitioners -- the job applicant; the business executive with an annual report to get out; the high school senior with a Faulkner assignment; the graduate-school student with her thesis proposal; the writer of a letter of condolence\footnote{{\bf condolence} [n] [countable, usually plural, uncountable] sympathy that you feel for somebody when a person in their family or that they know well has died; an expression of this sympathy.} -- often get stuck in an awkward\footnote{{\bf awkward} [a] {\bf 1.} embarrassed; making you feel embarrassed; {\bf 2.} difficult to deal with, {\sc synonym}: {\bf difficult}; {\bf 3.} not convenient; {\bf 4.} difficult because of its shape or design; {\bf 5.} not moving in an easy way; not comfortable or elegant.} passage or find a muddle\footnote{{\bf muddle} [v] ({\it especially British English}) {\bf 1.} to put things in the wrong order or mix them up; {\bf 2.} muddle somebody (up) to confuse somebody; {\bf 3.} muddle somebody{\tt/}something (up)$|$ {\bf muddle A (up) with B} to confuse 1 person or thing with another, {\sc synonym}: {\bf mix up}.} on their screens, \& then blame themselves. What should be easy \& flowing looks tangled\footnote{{\bf tangled} [a] {\bf 1.} twisted together in an untidy way; {\bf 2.} complicated, \& not easy to understand.} or feeble\footnote{{\bf feeble} [a] {\bf 1.} very weak; {\bf 2.} not effective; not showing energy or effort.} or overblown\footnote{{\bf overblown} [a] {\bf 1.} that is made to seem larger, more impressive or more important than it really is, {\sc synonym}: {\bf exaggerated}; {\bf 2.} (of flowers) past the best, most beautiful stage.} -- not what was meant at all. \fbox{What's wrong with me}, each one thinks. \fbox{Why can't I get this right?}''

[$\ldots$] White knew that a compendium\footnote{{\bf compendium} [n] (plural {\bf compendia, compendiums}) a collection of facts, drawings \& photographs on a particular subject, especially in a book.} of specific tips -- about singular \& plural verbs, parentheses, the ``that'' -- ``which'' scuffle\footnote{{\bf scuffle} [n] {\bf scuffle (with somebody) $|$ scuffle (between A \& B)} a short \& not very violent fight or struggle; [v] {\bf 1.} [intransitive] {\bf scuffle (with somebody)} (of 2 or more people) to fight or struggle with each other for a short time, in a way that is not very serious; {\bf 2.} [intransitive] {\bf $+$ adv.{\tt/}prep.} to move quickly making a quiet rubbing noise.}, \& many others -- could clear up a recalcitrant\footnote{{\bf recalcitrant} [a] ({\it formal}) unwilling to obey rules or follow instructions; difficult to control.} sentence or subclause when quickly reconsulted\footnote{{\bf consult} [v] {\bf 1.} [transitive, intransitive] to discuss something with somebody to get their permission for something, or to help you make a decision; {\bf 2.} [transitive, intransitive] to go to somebody for information or advice, especially an expert e.g. a doctor or lawyer; {\bf 3.} [transitive] {\bf consult something} to look in or at something to get information, {\sc synonym}: {\bf refer to something}.}, \& that the larger principles needed to be kept in plain sight, like a wall sampler.

How simple they look, set down here in White's last chapter: ``\fbox{Write in a way that comes naturally},'' ``\fbox{Revise \& rewrite},'' ``\fbox{Do not explain too much},'' \& the rest; above all, the cleansing\footnote{{\bf cleanse} [v] {\bf 1.} [transitive, intransitive] {\bf cleanse (something)} to clean your skin or a wound; {\bf 2.} [transitive] {\bf cleanse somebody (of{\tt/}from something}) ({\it literary}) to take away somebody's guilty feelings or sin.}, clarion\footnote{{\bf clarion} [n] {\bf 1.} a medieval trumpet with clear shrill tones; {\bf 2.} the sound of or as if of a clarion' [a] brilliantly clear; loud \& clear.} ``Be clear.'' How often I have turned to them, in the book or in my mind, while trying to start or unblock or revise some piece of my own writing! They help -- they really do. They work. They are the way.

E. B. White's prose is celebrated for its ease\footnote{{\bf ease} [n] [uncountable] {\bf 1.} lack of difficulty or effort, {\sc opposite}: {\bf difficulty}; {\bf 2.} the state of feeling relaxed or comfortable, without anxiety, problems or pain.} \& clarity\footnote{{\bf clarity} [n] [uncountable] {\bf 1.} the quality of being expressed clearly; {\bf 2.} the ability to think about or understand something clearly; {\bf 3.} if a picture, substance or sound has clarity, you can see or hear it very clearly, or see through it easily.} -- just think of {\it Charlotte's Web} -- but maintaining this standard required endless attention. When the new issue of {\it The New Yorker} turned up in Maine, I sometimes saw him reading his ``Comment'' piece over to himself, with only a slightly different expression than the one he'd worn on the day it went off. Well, O.K., he seemed to be saying. \fbox{At least I got the elements right.}

This edition has been modestly\footnote{{\bf modest} [a] {\bf 1.} fairly limited or small in amout; {\bf 2.} not expensive, rich or impressive; {\bf 3.} (of people, especially women, or their clothes) not showing too much of the body; not intended to attract attention, especially in a sexual way; {\bf 4.} ({\it approving}) not talking much about your own abilities or possessions.} updated, with word processors \& air conditioners making their 1st appearance among White's references, \& with a light redistribution of genders to permit a feminine pronoun or female farmer to take their places among the males who once innocently\footnote{{\bf innocent} [a] {\bf 1.} not guilty of a crime, etc.; not having done something wrong, {\sc opposite}: {\bf guilty}; {\bf 2.} [only before noun] suffering harm or being killed because of a crime, war, etc. although not directly involved in it; {\bf 3.} having little experience of evil or unpleasant things, or of sexual matters; {\bf 4.} not intended to cause harm or upset somebody, {\sc synonym}: {\bf harmless}.} served him.'' [$\ldots$] ``What is not here is anything about E-mail -- the rules-free, lower-case flow that cheerfully keeps us in touch these days. E-mail is conversation, \& it may be replacing the sweet \& endless talking we once sustained\footnote{{\bf sustain} [v] {\bf 1. sustain somebody{\tt/}something} to provide enough of what somebody{\tt/}something needs in order to live or exist; {\bf 2.} to make something continue for some time without becoming less, {\sc synonym}: {\bf maintain}; {\bf 3. sustain something} ({\it formal}) to experience something bad, {\sc synonym}: {\bf suffer}; {\bf 4. sustain something} to provide evidence to support an opinion, a theory, etc., {\sc synonym}: {\bf uphold}; {\bf 5. sustain something} ({\it law}) to decide that a claim, etc. is valid, {\sc synonym}: {\bf uphold}.} (\& tucked away\footnote{{\bf tuck away} [phrasal verb] {\bf tuck something $\leftrightarrow$ away} {\bf 1. be tucked away} to be located in a quiet place, where not many people go; {\bf 2.} to hide something somewhere or keep it in a safe place; {\bf 3.} ({\it British English, informal}) to eat a lot of food.}) within the informal letter. But we are all writers \& readers as well as communicators, with \fbox{the need at times to please \& satisfy ourselves} (as White put it) with the \fbox{clear \& almost perfect thought}.'' -- \cite[{\it Foreword} by Roger Angell]{Strunk_White_element_style}

``I [E. B. White] passed the course, graduated from the university, \& \fbox{forgot the book but not the professor}.'' [$\ldots$]

``{\it The Elements of Style}, when I [E. B. White] reexamined it in 1957, seemed to me to contain \fbox{rich deposits\footnote{{\bf deposit} [n] {\bf 1.} a layer of a substance that has been left somewhere, especially by a river or flood, or is found at the bottom of a liquid; {\bf 2.} a layer of a substance that has formed naturally underground; {\bf 3.} [usually singular] {\bf a deposit (on something)} a sum of money that is given as the 1st part of a larger payment; {\bf 4.} (in the British political system) the amount of money that a candidate in an election to Parliament has to pay, \& that is returned if they get enough votes.} of gold}. It was Will Strunk's {\it parvum opus}\footnote{{\bf parvum opus} [from Latin] [n] a little work, a small but meaningful work of an artist or writer.}, his attempt to cut the vast tangle\footnote{{\bf tangle} [n] {\bf 1.} a twisted mass of threads, hair, etc. that cannot be easily separated; {\bf 2.} a lack of order; a confused state; {\bf 3.} ({\it informal}) a disagreement or fight; [v] [transitive, intransitive] {\bf tangle (something) up} to twist something into an untidy mass; to become twisted in this way.} of English rhetoric\footnote{{\bf rhetoric} [n] [uncountable] {\bf 1.} ({\it often disapproving} speech or writing that is intended to influence people, but that is not completely honest or sincere; {\bf 2.} the skill of using language in speech or writing in a special way that influences or entertains people.)} down to size \& write its rules \& principles on the head of a pin\footnote{{\bf pin} [n] {\bf 1.} a short thin piece of stiff wire with a sharp point at 1 end \& a round head at the other, used to hold or attach things; {\bf 2.} a short piece of metal or other material, used to hold things together; {\bf 3.} a piece of metal with a sharp point, worn for decoration; {\bf 4.} 1 of the metal parts that stick out of an electric plug \& fit into a socket; [v] {\bf pin something ($+$ adv.{\tt/}prep.)} to attach something onto another thing or join things together with a pin, etc.; {\bf pin something down} [phrasal verb] to explain or understand something exactly.}. Will himself had hung the tag ``little'' on the book; he referred to it sardonically\footnote{{\bf sardonically} [adv] ({\it disapproving}) in a way that shows that you think that you are better than other people \& do not take them seriously, {\sc synonym}: {\bf mockingly}.} \& with secret pride as ``the {\it little book},'' always giving the word ``little'' a special twist, as though he were putting a spin on a ball. In its original form, it was a 43 page summation of the case for cleanliness, accuracy\footnote{{\bf accuracy} [n] {\bf 1.} [uncountable] the state of being exact or correct, {\sc opposite}: {\bf inaccuracy}; {\bf 2.} [uncountable, countable] ({\it specialist}) the degree to which the result of a measurement or calculation matches the correct value or a standard, {\sc opposite}: {\bf inaccuracy}.}, \& brevity\footnote{{\bf brevity} [n] [uncountable] {\bf 1.} the quality of using few words when speaking or writing; {\bf 2. brevity (of something)} the fact of lasting a short time.} in the use of English. Today, 52 years later, its vigor\footnote{{\bf vigor} [n] [uncountable] {\bf 1.} effort, energy, \& enthusiasm; {\bf 2. vigor (of something)} physical strength; good health.} is unimpaired\footnote{{\bf unimpaired} [a] ({\it formal}) not damaged or made less good, {\sc opposite}: {\bf impaired}.}, \& for sheer\footnote{{\bf sheer} [a] {\bf 1.} [only before noun] used to emphasize the size, degree or amount of something; nothing but; {\bf 2.} very steep.} pith\footnote{{\bf pith} [n] [uncountable] {\bf 1.} a soft dry white substance inside the skin of oranges \& some other fruits; {\bf 2.} the essential or most important part of something.} I think it probably sets a record that is not likely to be broken. Even after I got through tampering with\footnote{{\bf tamper with} [phrasal verb] {\bf tamper with something} to make changes to something without permission, especially in order to damage it, {\sc synonym}: interfere with.} it, it was still a tiny thing, \fbox{a barely tarnished\footnote{{\bf tarnished} [v] {\bf 1.} [intransitive, transitive] if mental tarnishes or something tarnishes it, it no longer looks bright \& shiny; {\bf 2.} [transitive, often passive] to damage the good opinion people have of somebody{\tt/}something, {\sc synonym}: {\bf taint}; [n] [singular, uncountable] a thin layer on the surface of a metal that makes it look darker \& less bright.} gem\footnote{{\bf gem} [n] {\bf 1.} (also less frequent {\bf gemstone}) a precious stone that has been cut \& polished \& is used in jewellery, {\sc synonym}: {\bf jewel, precious stone}; {\bf 2.} a person, place or thing that is especially good.}}. 7 rules of usage, 11 principles of composition\footnote{{\bf composition} [n] {\bf 1.} [uncountable] the different parts that something is made of; the way in which the different parts are organized; {\bf 2.} [countable] a piece of music or a poem; {\bf 3.} [uncountable] the act of writing a piece of music or a poem; {\bf 4.} [uncountable] ({\it art}) the arrangement of people of objects in a painting, photograph or scene of a film.}, a few matters of form, \& a list of words \& expressions commonly misused -- that was the sum \& substance\footnote{{\bf substance} [n] {\bf 1.} a type of solid, liquid or gas that has particular qualities; {\bf 2.} [countable] a drug or chemical, especially an illegal one, that has a particular effect on the mind or body; {\bf 3.} [uncountable] the most important or main part of something; {\bf 4.} [uncountable] ({\it formal}) importance; {\bf 5.} [uncountable] the quality of being based on facts or the truth.} of Prof. Strunk's work. Somewhat audaciously\footnote{{\bf audaciously} [adv] ({\it formal}) in a way that shows you are willing to take risks or to do something that shocks people.}, \& in an attempt to give my publisher his money's worth, I [E. B. White] added a chapter called ``An Approach to Style,'' setting forth my own prejudices\footnote{{\bf prejudice} [n] [uncountable, countable] an unreasonable dislike of a person, group, etc., especially when it is based on their race, religion, sex, etc.}, my notions of error, my articles of faith. This chapter (Chap. V) is addressed particularly to those who feel that English prose composition is not only a necessary skill but a sensible pursuit as well -- a way to spend one's days. I think Prof. Strunk would not object to that.''

[$\ldots$] ``I have now completed a 3rd revision. Chap. IV has been refurbished\footnote{{\bf refurbish} [v] {\bf refurbish something} to clean \& decorate a room, building, etc. in order to make it more attractive, more useful, etc.} with words \& expressions of a recent vintage\footnote{{\bf vintage} [n] {\bf 1.} the wine that was produced in a particular year or place; the year in which it was produced; {\bf 2.} [usually singular] the period or season of gathering grapes for making wine; [a] [only before noun] {\bf 1. vintage} wine is of very good quality \& has been stored for several years; {\bf 2.} (British English) (of a vehicle) made between 1919 \& 1930 \& admired for its style \& interest; {\bf 3.} typical of a period in the past \& of high quality; the best work of the particular person; {\bf 4. vintage year} a particular good \& successful year.}; 4 rules of usage have been added to Chap. I. Fresh examples have been added to some of the rules \& principles, amplification\footnote{{\bf amplification} [n] [uncountable] {\bf 1. amplification (of something)} the process of increasing the amplitude of an electrical signal; {\bf 2.} (biochemistry) {\bf amplification (of something)} the process by which many copies of something, e.g. a gene, are made; {\bf 3. amplification (of something)} the action of making something greater or easier to notice; {\bf 4.} the action of adding details to a story, statement, etc.; details added to a story, statement, etc.} has reared\footnote{{\bf rear} [v] {\bf 1. rear somebody{\tt/}something} [often passive] to care for young children or animals until they are fully grown, {\sc synonym}: {\bf raise}; {\bf 2. rear something} to breed or keep animals or birds, e.g. on a farm; {\bf something rears its head} [idiom] (of something unpleasant) to appear or happen; [n] (usually {\bf the rear}) [singular] the back part of something; [a] [only before noun] at or near the back of something.} its head in a few places in the text where I felt an assault\footnote{{\bf assault} [n] {\bf 1.} [uncountable, countable] the crime of attacking somebody physically; in law, {\bf assault} is an act that threatens physical harm to somebody, whether or not actual harm is done: {\it to commit}{\tt/}{\it be charged with assault}; {\bf 2.} [countable] (by an army, etc.) the act of attacking somebody{\tt/}something, {\sc synonym}: {\bf attack}; {\bf 3.} [countable, usually singular, uncountable] an act of criticizing or attacking somebody{\tt/}something severely; [v] {\bf assault somebody} to attack somebody physically.} could successfully be made on the bastions\footnote{{\bf bastion} [n] {\bf 1.} ({\it formal}) a group of people or a system that protects a way of life or a belief when it seems that it may disappear; {\bf 2.} a place that military forces are defending.} of its brevity, \& in general the book has received a thorough overhaul\footnote{{\bf overhaul} [n] an examination of a machine or system, including doing repairs on it or making changes to it; [v] {\bf 1. overhaul something} to examine every part of a machine, system, etc. \& make any necessary changes or repairs; {\bf 2. overhaul somebody} to come from behind a person you are competing against in a race \& go past them, {\sc synonym}: {\bf overtake}.} -- to correct errors, delete bewhiskered\footnote{{\bf bewhiskered} [a] {\bf 1.} having whiskers; bearded; {\bf 2.} ancient, as a witticism, expression, etc.; pass\'e; hoary.} entries, \& enliven\footnote{{\bf enliven} [v] ({\it formal}) {\bf enliven something} to make something more interesting or more fun.} the argument.

Prof. Strunk was a positive man. His book contains rules of grammar phrased as direct orders. In the main I [E. B. White] have not tried to soften his commands, or modify his pronouncements\footnote{{\bf pronouncement} [n] a formal public statement.}, or remove the special objects of his scorn\footnote{{\bf scorn} [n] [uncountable] a strong feeling that somebody{\tt/}something is stupid or not good enough, usually shown by the way you speak, {\sc synonym}: {\bf contempt}; [v] {\bf 1. scorn somebody{\tt/}something} to feel or show that you think somebody{\tt/}something is stupid \& you do not respect them or it, {\sc synonym}: {\bf dismiss}; {\bf 2.} ({\it formal}) to refuse to have or do something because you are too proud.}. I have tried, instead, to preserve\footnote{{\bf preserve} [v] {\bf 1. preserve something} to keep a particular quality or feature; {\bf 2.} to keep something safe from harm, in good condition or in its original state; {\bf 3.} to prevent something from decaying, by treating it in a particular way; [n] [singular] an activity, job or interest that is thought to be suitable for 1 particular person or group of people.} the flavor\footnote{{\bf flavor} [n] {\bf 1.} [uncountable] {\bf flavor (of something)} how food or drink tastes, {\sc synonym}: {\bf taste}; {\bf 2.} [countable] a particular type of taste; {\bf 3.} [singular] a particular quality or atmosphere; {\bf 4.} [singular] {\bf a{\tt/}the flavor of something} an idea of what something is like.} of his discontent\footnote{{\bf discontent} [n] (also {\bf discontentment}) {\bf 1.} [uncountable] a feeling of being unhappy because you are not satisfied with a particular situation, {\sc synonym}: {\bf dissatisfaction}; {\bf 2.} [countable] {\bf discontent (of somebody)} a thing that makes you feel unhappy \& not satisfied with a particular situation, {\sc synonym}: {\bf dissatisfaction}.} while slightly enlarging the scope of the discussion. {\it The Elements of Style} does not pretend\footnote{{\bf pretend} [v] {\bf 1.} to behave in a particular way, in order to make other people believe something that is not true; {\bf 2.} (usually used in negative sentences \& questions) to claim to be, do or have something, especially when this is not true.} to survey\footnote{{\bf survey} [n] {\bf 1. survey} (of somebody{\tt/}something) an investigation of the opinions, behavior, etc. of a particular group of people, which is usually done by asking them questions; {\bf 2.} an act of examining \& recording the measurements, features, etc. of an area of land in order to make a map or plan of it; {\bf 3. survey (of something)} a general study, view or description of something; [v] {\bf 1. survey somebody{\tt/}something} to investigate the opinions or behavior of a group of people by asking them a series of questions; {\bf 2. survey something} to study \& give a general description of something; {\bf 3. survey something} to measure \& record the features of an area of land, e.g. in order to make a map or in preparation for building; {\bf 4. survey something} to look carefully at the whole of something, especially in order to get a general impression of it, {\sc synonym}: {\bf inspect}.} the whole field. Rather it proposes\footnote{{\bf propose} [v] {\bf 1.} to suggest a plan or an idea for people to consider \& decide on; {\bf 2.} to suggest an explanation of something for people to consider.} to give in brief space the principal\footnote{{\bf principal} [a] [only before noun] main; most important.} requirements of plain\footnote{{\bf plain} [a] {\bf 1.} easy to see or understand, {\sc synonym}: {\bf clear}; {\bf 2.} [only before noun] expressed in a clear \& simple way, without using technical language; {\bf 3.} not trying to deceive anyone; honest \& direct; {\bf 4.} not decorated or complicated; simple; in computing, {\bf plain text} is data representing text that is not written in code or using special formatting \& can be read, displayed or printed without much processing: {\it Mathematical formulae are an example of content that cannot be represented satisfactorily via plain text.}; {\bf 5.} without marks or a pattern on it; {\bf 6.} [only before noun] (used for emphasis) simple; nothing but. {\sc synonym}: {\bf sheer}.} English style. It concentrates\footnote{{\bf concentrate} [v] {\bf 1.} [transitive, often passive] {\bf concentrate something $+$ adv.{\tt/}prep.} to bring something together in 1 place; {\bf 2.} [intransitive, transitive] to give all your attention to something \& not think about anything else; {\bf 3.} [transitive] {\bf concentrate something} to increase the strength of a substance by reducing its volume, e.g. by boiling it; {\bf concentrate on something} [phrasal verb] to spend more time doing 1 particular thing than others; [n] [countable, uncountable] {\bf concentrate (of something)} a substance that is made stronger because water or other substances have been removed.} on fundamentals\footnote{{\bf fundamentals} [n] [plural] {\bf fundamentals (of something)} the basic \& most important parts of something.}: the rules of usage \& principles of composition most commonly violated\footnote{{\bf violet} [v] {\bf 1. violate something} to go against or refuse to obey a law, an agreement, etc.; {\bf 2. violate something} to not treat something with respect.}.

The reader will soon discover that these rules \& principles are in the form of sharp commands, Sergeant\footnote{{\bf sergeant} [n] (abbr., {\bf Sergt, Sgt}) {\bf 1.} a member of 1 of the middle ranks in the army \& the air force, below an officer; {\bf 2.} (in a UK) a police officer just below the rank of an inspector; {\bf 3.} (in the US) a police officer just below the rank of a lieutenant or caption.} Strunk snapping\footnote{{\bf snap} [v] {\it break} {\bf 1.} [transitive, intransitive] to break something suddenly with a sharp noise; to be broken in this way; {\it take photograph} {\bf 2.} [transitive, intransitive] ({\it informal}) to take a photograph; {\it open}{\tt/}{\it close}{\tt/}{\it move into position} {\bf 3.} [intransitive, transitive] to move, or to move something, into a particular position quickly, especially with a sudden sharp noise; {\it speak impatiently} {\bf 4.} [transitive, intransitive] to speak or say something in an impatient, usually angry, voice; {\it of animal} {\bf 5.} [intransitive] {\bf snap (at somebody{\tt/}something)} to try to bite somebody{\tt/}something, {\sc synonym}: {\bf nip}; {\it lose control} {\bf 6.} [intransitive] to suddenly be unable to control your feelings any longer because the situation has become too difficult; {\it fasten clothing} {\bf 7.} [intransitive, transitive] {\bf snap (something)} ({\it North American English}) to fasten a piece of clothing with a snap; {\it in American football} {\bf 8.} [transitive] {\bf snap something} ({\it sport}) (in American football) to start play by passing the ball back between your legs.} orders to his platoon\footnote{{\bf platoon} [n] a small group of soldiers that is part of a company \& commanded by a lieutenant.}. ``Do not join independent clauses with a comma.'' (Rule 5.) ``Do not break sentences in 2.'' (Rule 6.) ``Use the active voice.'' (Rule 14.) ``Omit\footnote{{\bf omit} [v] {\bf 1.} to not include something{\tt/}somebody, either deliberately or because you have forgotten it{\tt/}them, {\sc synonym}: {\bf leave somebody{\tt/}something out (of something)}; {\bf 2. omit to do something} to not do or fail to do something.} needless\footnote{{\bf needless} [a] (of something bad) not necessary; that could be avoided, {\sc synonym}: unnecessary.} words.'' (Rule 17.) ``Avoid a succession\footnote{{\bf succession} [n] {\bf 1.} [countable, usually singular] a number of things or people that follow each other in time or order, {\sc synonym}: {\bf series}; {\bf 2.} [uncountable] the act of taking over an official position or title; {\bf 3.} [uncountable] the right to take over an official position or title, especially to become the king or queen of a country.} of loose\footnote{{\bf loose} [a] {\bf 1.} not firmly fixed where it should be; that can become separated from something; {\bf 2.} not tightly packed together; not solid or hard; {\bf 3.} not strictly organized or controlled; {\bf 4.} not exact; not very careful; {\bf 5.} (of clothes) not fitting closely, {\sc opposite}: {\bf tight}; {\bf 6.} not tied together; not held in position by anything or contained in anything; {\bf 7.} ({\it medical}) (of body waste) having too much liquid in it.} sentences.'' (Rule 18.) ``In summaries, keep to 1 tense.'' (Rule 21.) Each rule or principle is followed by a short hortatory\footnote{{\bf hortatory} [a] trying to strongly encourage or persuade someone to do something.} essay, \& usually the exhortation\footnote{{\bf exhortation} [n] [countable, uncountable] ({\it formal}) {\bf exhortation (to do something)} an act of trying very hard to persuade somebody to do something.} is followed by, or interlarded\footnote{{\bf interlard} [v] (used with object) {\bf 1.} to diversify by adding or interjecting something unique, striking, or contrasting (usually followed by {\it with}); {\bf 2.} (of things) to be intermixed in.} with, examples in parallel columns -- the true vs. the false, the right vs. the wrong, the timid\footnote{{\bf timid} [a] shy \& nervous; not brave.} vs. the bold, the ragged\footnote{{\bf ragged} [a] {\bf 1.} (of clothes) old \& torn, {\sc synonym}: {\bf shabby}; {\bf 2.} (of people) wearing old or torn clothes; {\bf 3.} having an outline, an edge or a surface that is not straight or even; {\bf 4.} not smooth or regular; not showing control or careful preparation; {\bf 5.} ({\it informal}) very tired, especially after physical effort.} vs. the trim\footnote{{\bf trim} [v] {\bf 1. trim something} to make something neater, smaller, better, etc., by cutting parts from it; {\bf 2.} to cut away unnecessary parts from something; {\bf 3.} [usually passive] {\bf trim something (with something)} to decorate something, especially around its edges.}. From every line there peers out at me the puckish\footnote{{\bf puckish} [a] [usually before noun] ({\it literary}) enjoying playing tricks on other people, {\sc synonym}: {\bf mischievous}.} face of my professor, his short hair parted neatly\footnote{{\bf neat} [a] {\bf 1.} in good order; carefully done or arranged; {\bf 2.} simple but clever; {\bf 3.} containing or made out of just 1 substance; not mixed with anything else.} in the middle \& combed down over his forehead, his eyes blinking incessantly\footnote{{\bf incessantly} [adv] ({\it usually disapproving}) without stopping, {\sc synonym}: {\bf constantly}.} behind steel-rimmed spectacles\footnote{{\bf spectacle} [n] {\bf 1.} [countable, uncountable] {\bf spectacle (of something)} a performance or an event that is very impressive \& exciting to look at; {\bf 2.} [singular] {\bf spectacle (of something)} an unusual, embarrassing or sad sight or situation that attracts a lot of attention; {\bf 3.} ({\bf spectacles}) [plural] [{\it formal}] $=$ {\bf glass}.} as though he had just emerged into strong light, his lips nibbling each other like nervous horses, his smile shuttling to \& fro under a carefully edged mustache.

``Omit needless words!'' cries the author on p. 23, \& into that imperative\footnote{{\bf imperative} [n] a thing that is very important \& needs immediate attention or action; [a] [not usually before noun] very important \& needing immediate attention or action, {\sc synonym}: {\bf vital}.} Will Strunk \fbox{really put his heart \& soul}. In the days when I was sitting in his class, he omitted so many needless words, \& omitted them so forcibly\footnote{{\bf forcibly} [adv] {\bf 1.} in a way that involves the use of physical force; {\bf 2.} in a way that makes something very clear.} \& with such eagerness\footnote{{\bf eager} [a] very interested \& excited by something that is going to happen or about something that you want to do, {\sc synonym}: {\bf keen}.} \& obvious relish\footnote{{\bf relish} [v] to get great pleasure from something; to want very much to do or have something, {\sc synonym}: {\bf enjoy}; [n] {\bf 1.} [uncountable] great pleasure; {\bf 2.} [uncountable, countable] a cold, thick, spicy sauce made from fruit \& vegetables that have been boiled, that is served with meat, cheese, etc.}, that he often seemed in the position of having shortchanged\footnote{{\bf short-change} [v] [often passive] {\bf 1. short-change somebody} to give back less than the correct amount of money to somebody who has paid for something with more than the exact price; {\bf 2. short-change somebody} to treat somebody unfairly by not giving them what they have earned or deserve.} himself -- a man left with nothing more to say yet with time to fill, a radio prophet who had outdistanced\footnote{{\bf outdistance} [v] {\bf outdistance somebody{\tt/}something} to leave somebody{\tt/}something behind by going faster, further, etc.; to be better than somebody{\tt/}something, {\sc synonym}: {\bf outstrip}.} the clock. Will Strunk got out of this predicament\footnote{{\bf predicament} [n] a difficult or an unpleasant situation, especially one where it is difficult to know what to do, {\sc synonym}: {\bf quandary}.} by a simple trick: he uttered\footnote{{\bf utter} [v] {\bf utter something} to make a sound with your voice; to say something.} every sentence 3 times. When he delivered his oration\footnote{{\bf oration} [n] ({\it formal}) a formal speech made on a public occasion, especially as part of a ceremony.} on brevity to the class, he leaned forward over his desk, grasped his coat lapels\footnote{{\bf lapel} [n] 1 of the 2 front parts of the top of a coat or jacket that are joined to the collar \& are folded back.} in his hands, \&, in a husky\footnote{{\bf husky} [a] {\bf 1.} (of a person of their voice) sounding deep, quiet \& rough, sometimes in an attractive way; {\bf 2.} ({\it North American English}) with a large, strong body; [n] (North American English also {\bf huskie}) a large strong dog with thick hair, used for pulling sledges across snow.}, conspiratorial\footnote{{\bf conspiratorial} [a] {\bf 1.} connected with, or making you think of, a conspiracy ($=$ a secret plan to do something illegal); {\bf 2.} (of a person's behavior) suggesting that a secret is being shared.} voice, said, ``Rule 17. Omit needless words! Omit needless words! Omit needless word!''

He was a memorable\footnote{{\bf memorable} [a] special, good or unusual \& therefore worth remembering; easy to remember.} man, friendly \& funny. Under the remembered sting of his kindly lash\footnote{{\bf lash} [v] {\bf 1.} [intransitive, transitive] to hit somebody{\tt/}something with great force, {\sc synonym}: {\bf pound}; {\bf 2.} [transitive] {\bf lash somebody{\tt/}something} to hit a person or an animal with a whip, rope, stick, etc., {\sc synonym}: {\bf beat}.}, I have been trying to omit needless words since 1919, \& although there are still many words that cry for omission \& the huge task will never be accomplished, it is exciting to me to reread to masterly Strunkian elaboration\footnote{{\bf elaboration} [n] [uncountable, countable] {\bf 1.} the act of explaining or describing something in a more detailed way; {\bf 2.} the process of developing a plan, an idea, etc. \& making it complicated or detailed; {\bf 3. elaboration (of something)} ({\it biology}) the production of a substance or structure from elements or simpler constituents in a natural process.} of this noble\footnote{{\bf noble} [a] {\bf 1.} belonging to a family of high social rank, {\sc synonym}: {\bf aristocratic}; {\bf 2.} having or showing fine personal qualities that people admire, e.g. courage, honesty \& care for others; [n] a person who comes from a family of high social rank; a member of the nobility, {\sc synonym}: {\bf aristocratic}.} theme\footnote{{\bf theme} [n] the subject of a talk, piece of writing, exhibition, etc.; an idea that keeps returning in a piece of research or a work of art or literature.}. It goes:
\begin{quotation}
	{\it Vigorous writing is concise. A sentence should contain no unnecessary words, a paragraph no unnecessary sentences, for the same reason that a drawing should have no unnecessary lines \& a machine no unnecessary parts. This requires not that the writer make all sentences short or avoid all detail \& treat subjects only in outline, but that every word tell.}
\end{quotation}
There you have a short, valuable essay on the nature \& beauty of brevity -- 59 words that could change the world. Having recovered from his adventure in prolixity\footnote{{\bf prolixity} [n] [uncountable] ({\it formal}) the fact of using too many words \& therefore creating a piece of writing, a speech, etc., that is boring.} (59 words were a lot of words in the tight world of William Strunk Jr.), the professor proceeds to give a few quick lessons in pruning\footnote{{\bf pruning} [n] [uncountable] {\bf 1.} the activity of cutting off some of the branches from a tree, bush, etc. so that it will grow better \& stronger; {\bf 2.} the act of making something smaller by removing parts; the act of cutting out parts of something.}. Students learn to cut the dead-wood from ``this is a subject that,'' reducing it to ``this subject,'' a saving of 3 words. They learn to trim\footnote{{\bf trim} [v] {\bf 1. trim something} to make something neater, smaller, better, etc., by cutting parts from it; {\bf 2.} to cut away unnecessary parts from something; {\bf 3.} [usually passive] {\bf trim something (with something)} to decorate something, especially around its edges.} ``used for fuel purposes'' down to ``used for fuel.'' They learn that they are being chatterboxes\footnote{{\bf chatterbox} [n] ({\it informal}) a person who talks a lot, especially a child.} when they say ``the question as to whether'' \& that they should just say ``whether'' -- a saving of 4 words out of a possible 5.

The professor devotes\footnote{{\bf devote} [v] {\bf devote yourself to somebody{\tt/}something} to give most of your time, energy or attention to somebody{\tt/}something, {\sc synonym}: {\bf dedicate}; {\bf devote something to something}: to give an amount of time, attention or resources to something.} a special paragraph to the vile\footnote{{\bf vile} [a] {\bf 1.} ({\it informal}) extremely unpleasant or bad, {\sc synonym}: {\bf disgusting}; {\bf 2.} ({\it formal}) morally bad; completely unacceptable, {\sc synonym}: {\bf wicked}.} expression {\it the fact that}, a phrase that causes him to quiver\footnote{{\bf quiver} [v] to shake slightly; to make a slight movement, {\sc synonym}: {\bf tremble}; [n] {\bf 1.} an emotion that has an effect on your body; a slight movement in part of your body; {\bf 2.} a case for carrying arrows.} with revulsion\footnote{{\bf revulsion} [n] [uncountable, singular] ({\it formal}) a strong feeling of horror, {\sc synonym}: {\bf disgust, repugnance}.}. The expression, he says, should be ``revised out of every sentence in which it occurs.'' But a shadow\footnote{{\bf shadow} [n] {\bf 1.} [countable] the dark area or shape produced by somebody{\tt/}something coming between light \& a surface; {\bf 2.} [uncountable] ({\bf shadows} [plural]) darkness, especially that produced by somebody{\tt/}something coming between light \& a surface; {\bf 3.} [singular] the strong (usually bad) influence of somebody{\tt/}something.} of gloom\footnote{{\bf gloom} [n] {\bf 1.} [uncountable, singular] a feeling of being sad \& without hope, {\sc synonym}: {\bf depression}; {\bf 2.} [uncountable] ({\it literary}) almost total darkness.} seems to hang over the page, \& you feel that he knows how hopeless his cause is. I suppose I have written {\it the fact that} a thousand times in the heat of composition, revised it out maybe 500 times in the cool aftermath\footnote{{\bf aftermath} [n] [usually singular] the situation that exists as a result of an important (\& usually unpleasant) event, especially a war, an accident, etc.}. To be batting only .500 this late in the season, to fail half the time to connect with this fat pitch, saddens me, for it seems a betrayal of the man who showed me how to swing\footnote{{\bf swing} [v] {\bf 1.} [intransitive, transitive] to change to make somebody{\tt/}something change from 1 opinion or mood to another; {\bf 2.} [intransitive, transitive] to turn or change direction suddenly; to make something do this; {\bf 3.} [intransitive, transitive] to move backwards or forwards or from side to side while hanging from a fixed point; to make something do this; {\bf 4.} [intransitive, transitive] to move or make something move with a wide curved movement; [n] a change from 1 opinion or situation to another; the amount by which something changes.} at it \& made the swinging seem worthwhile.

I treasure\footnote{{\bf treasure} [n] {\bf 1.} [uncountable] a collection of valuable things e.g. gold, silver \& jewelery; {\bf 2.} [countable, usually plural] a highly valued object; {\bf 3.} [singular] a person who is much loved or valued; [v] {\bf treasure something} to have or keep something that you love \& that is extremely valuable to you, {\sc synonym}: {\bf cherish}.} {\it The Elements of Style} for its sharp\footnote{{\bf sharp} [a] {\bf 1.} [usually before noun] (especially of a change in something) sudden \& fast; {\bf 2.} [usually before noun] (especially of a difference in something) clear \& definite; {\bf 3.} (especially of something that can cut or make a hole in something) having a fine edge or point, {\sc opposite}: {\bf blunt}; {\bf 4.} (of a person or what they say) critical or severe; {\bf 5.} (of a physical feeling or an emotion) very strong \& sudden, often like being cut or wounded, {\sc synonym}: {\bf intense}; {\bf 6.} changing direction suddenly; {\bf 7.} (of people or their minds or eyes) quick to notice or understand things or to react.} advice, but I treasure it even more for the \fbox{audacity}\footnote{{\bf audacity} [n] [uncountable] behavior that is brave but likely to shock or offend people, {\sc synonym}: {\bf nerve}.} \& self-confidence\footnote{{\bf self-confidence} [n] [uncountable] confidence in yourself \& your abilities, {\sc synonym}: {\bf self-assurance, confidence}.} of its author. \fbox{Will knew where he stood.} He was so sure of where he stood, \& made his position so clear \& so plausible, that his peculiar\footnote{{\bf peculiar} [a] belonging to or connected with 1 particular place, situation, person, etc., \& not others.} stance\footnote{{\bf stance} [n] the opinions that somebody has about something \& expresses publicly, {\sc synonym}: {\bf position}.} has continued to invigorate\footnote{{\bf invigorate} [v] {\bf 1. invigorate somebody} to make somebody feel healthy \& full of energy; {\bf 2. invigorate something} to make a situation, an organization, etc. efficient \& successful.} me -- \&, I am sure, thousands of other ex-students -- during the years that have intervened\footnote{{\bf intervene} [v] {\bf 1.} [intransitive] to become involved in a situation in order to improve it or stop it from getting worse; {\bf 2.} [intransitive] to happen in the time between events; {\bf 3.} [intransitive] to exist or be found in the space between things; {\bf 4.} [intransitive] to happen in a way  that delays something or prevents it from happening.} since our 1st encounter\footnote{{\bf encounter} [v] {\bf 1. encounter something} to experience something, especially something unpleasant or difficult, while you are trying to do something else, {\sc synonym}: {\bf run into something}; {\bf 2. encounter something{\tt/}somebody} to discover or experience something, or meet somebody, especially something{\tt/}somebody new, unusual or unexpected, {\sc synonym}: {\bf come across somebody{\tt/}something}; [n] a meeting, especially one that is sudden or unexpected.}. He had a number of likes \& dislikes that were almost as whimsical\footnote{{\bf whimsical} [a] unusual \& not serious in a way that is either funny or annoying.} as the choice of a necktie, yet he made them seem utterly\footnote{{\bf utter} [a] [only before noun] used to emphasize how complete something is, {\sc synonym}: {\bf total}; [v] {\bf utter something} to make a sound with your voice; to say something.} convincing. He disliked the word {\it forceful}\footnote{{\bf forceful} [a] {\bf 1.} (of people) expressing opinion firmly \& clearly in a way that persuades other people to believe them, {\sc synonym}: {\bf assertive}; {\bf 2.} (of opinions, etc.) expressed firmly \& clearly so that other people believe them; {\bf 3.} using force; {\bf 4.} (of action) strong \& effective.} \& advised us to use {\it forcible}\footnote{{\bf forcible} [a] [only before noun] involving the use of physical force.} instead. He felt that the word {\it clever}\footnote{{\bf clever} [a] {\bf 1.} (especially British English) quick at learning \& understanding things, {\sc synonym}: {\bf intelligent}; {\bf 2. clever (at something{\tt/}doing somethign)} (especially British English) skillful; {\bf 3.} showing intelligence or skill, e.g. in the design of an object, in an idea or somebody's actions.} was greatly overused: ``It is best restricted to ingenuity\footnote{{\bf ingenuity} [n] [uncountable] the ability to invent things or solve problems in clever new ways, {\sc synonym}: {\bf inventiveness}.} displayed in small matters.'' He despised\footnote{{\bf despise} [v] (not used in the progressive tenses) to dislike \& have no respect for somebody{\tt/}something.} the expression {\it student body}, which he termed gruesome\footnote{{\bf gruesome} [a] very unpleasant \& filling you with horror, usually because it is connected with death or injury.}, \& made a special trip downtown to the {\it Alumni News} office 1 day to protest\footnote{{\bf protest} [n] [uncountable, countable] the expression of strong disagreement with or opposition to something; a statement or an action that shows this.} the expression \& suggest that {\it studentry} be substituted\footnote{{\bf substitute} [v] [intransitive, transitive] to take the place of somebody{\tt/}something else; to use somebody{\tt/}something instead of somebody{\tt/}something else; [n] a person or thing that you use or have instead of the usual one.} -- a coinage\footnote{{\bf coinage} [n] {\bf 1.} [uncountable] the coins used in a particular place or at a particular time; coins of a particular type; {\bf 2.} [countable, uncountable] a word or phrase that has been invented recently; the process of inventing a word or phrase.} of his own, which he felt was similar to {\it citizenry}\footnote{{\bf citizenry} [n] [singular $+$ singular or plural verb] ({\it formal}) all the citizens of a particular town, country, etc.}. I am told that the {\it News} editor was so charmed by the visit, if not by the word, that he ordered the student body buried, never to rise again. {\it Studentry} has taken its place. It's not much of an improvement, but it does sound less cadaverous\footnote{{\bf cadaverous} [a] ({\it literary}) (of a person) extremely pale, thin \& looking ill.}, \& it made Will Strunk quite happy.

Some years ago, when the heir\footnote{{\bf heir} [n] {\bf 1.} a person who has the legal right to receive somebody's property, money or title when that person dies; {\bf 2.} a person who is thought to continue the work or a tradition started by somebody else.} to the throne of England was a child, I noticed a headline in the {\it Times} about Bonnie Prince Charlie: ``CHARLES' TONSILS OOUT.'' Immediately Rule 1 leapt to mind.
\begin{quotation}
	{\bf 1.} Form the possessive singular of nouns by adding {\it 's}. Follow this rule whatever the final consonant\footnote{{\bf consonant} [n] {\bf 1.} (phonetics) a speech sound made by completely or partly stopping the flow of air being breathed out through the mouth; {\bf 2.} a letter of the alphabet that represents a consonant sound.}. Thus write, {\it Charles's friend, Burns's poems, the witch's malice\footnote{{\bf malice} [n] [uncountable] a desire to harm somebody caused by a feeling of hate.}}.
\end{quotation}
Clearly, Will Strunk had foreseen\footnote{{\bf foreseen} [v] to know about something before it happens.}, as far back as 1918, the dangerous tonsillectomy\footnote{{\bf tonsillectomy} [n] ({\it medical}) a medical operation to remove the tonsils.} of a prince, in which the surgeon removes the tonsils \& the {\it Times} copy desk removes the final {\it s}. He started his book with it. I commend Rule 1 to the {\it Times}, \& I trust that Charles's throat, not Charles' throat, is in fine shape today.

Style rules of this sort are, of course, somewhat a matter of individual preference\footnote{{\bf preference} [n] {\bf 1.} [countable, usually singular, uncountable] a greater interest in or desire for somebody{\tt/}something than somebody{\tt/}something else; {\bf 2.} [countable] a thing that is liked better or best.}, \& even the established rules of grammar are open to challenge. Prof. Strunk, although 1 of the most inflexible\footnote{{\bf inflexible} [a] {\bf 1.} ({\it disapproving}) that cannot be changed or made more suitable for a particular situation, {\sc synonym}: {\bf rigid}; {\bf 2.} ({\it disapproving}) (of people or organizations) unwilling to change their opinions, decision or behavior.} \& choosy\footnote{{\bf choosy} [a] ({\it informal}) careful in choosing; difficult to please, {\sc synonym}: {\bf fussy, picky}.} of men, was quick to acknowledge\footnote{{\bf acknowledge} [v]  {\bf 1.} to accept that something is true or exists; {\bf 2.} to accept that somebody{\tt/}something has a particular quality, importance or status, {\sc synonym}: {\bf recognize}; {\bf 3. acknowledge somebody{\tt/}something} to publicly express thanks fo help or inspiration; {\bf 4. acknowledge something} to tell somebody that you have received something that they sent to you.} the fallacy\footnote{{\bf fallacy} [n] {\bf 1.} [countable] a false idea that many people believe is true; {\bf 2.} [uncountable, countable] a false way of thinking about something.} of inflexibility \& the danger of doctrine\footnote{{\bf doctrine} [n] {\bf 1.} [countable, uncountable] {\bf doctrine (of something)} a belief or principle, or set of beliefs or principles, held by a religion, a political party or a legal system; {\bf 2.} ({\bf Doctrine}) [countable] (US) a statement of government policy, especially foreign policy.}. ``It is an old observation,'' he wrote, ``that the best writers sometimes disregard\footnote{{\bf disregard} [v] {\bf disregard something} to not consider something; to treat something as unimportant, {\sc synonym}: {\bf ignore}.} the rules of rhetoric\footnote{{\bf rhetoric} [n] [uncountable] {\bf 1.} ({\it often disapproving}) speech or writing that is intended to influence people, but that is not completely honest or sincere; {\bf 2.} the skill of using language in speech or writing in a special way that influences or entertains people.}. \texttt{[stop translating here]} When they do so, however, the reader will usually find in the sentence some compensating merit, attained at the cost of the violation. Unless he is certain of doing as well, he will probably do best to follow the rules.''

It is encouraging to see how perfectly a book, even a dusty rule book, perpetuates \& extends the spirit of a man. Will Strunk loved the clear, the brief, the bold, \& his book is clear, brief, bold. Boldness is perhaps its chief distinguishing mark. On p. 26, explaining 1 of his parallels, he says, ``The lefthand version gives the impression that the writer is undecided or timid, apparently unable or afraid to choose 1 form of expression \& hold to it.'' \& his original Rule 11 was ``Make definite assertions.'' That was Will all over. He scorned the vague, the tame, the colorless, the irresolute. He felt it was worse to be irresolute than to be wrong. I remember a day in class when he leaned far forward, in his characteristic pose -- the pose of a man about to impart a secret -- \& croaked, ``If you don't know how to pronounce a word, say it loud! If you don't know how to pronounce a word, say it loud!'' This comical piece of advice struck me as sound at the time, \& I still respect it.\fbox{ Why compound ignorance with inaudibility?} \fbox{Why run \& hide?}

All through {\it The Elements of Style} one finds evidence of the author's deep sympathy for the reader. Will felt that the reader was in serious trouble most of the time, floundering in a swamp, \& that it was the duty of anyone attempting to write English to drain this swamp quickly \& get the reader up on dry ground, or at least to throw a rope. In revising the text, I have tried to hold steadily in mind this belief of his, this concern for the bewildered reader.

In the English classes of today, ``the little book'' is surrounded by longer, lower textbooks -- books with permissive steering \& automatic transitions. Perhaps the book has become something of a curiosity. To me, it still seems to maintain its original poise, standing, in a drafty time, erect, resolute, \& assured. I still find the Strunkian wisdom a comfort, the Strunkian humor a delight, \& the Strunkian attitude forward right-\&-wrong a blessing undisguised.'' -- \cite[Introduction (by E. B. White)]{Strunk_White_element_style}

%------------------------------------------------------------------------------%

\subsection*{Foreword}
``This classic work, still used by college students as a guide to succinct \& clear writing, was formulated by a university English teacher, {\sc William Strunk, Jr.}, for the benefit of his students. The little work's great longevity \& continuing relevance is a credit to the man who conceived it.

William Strunk, Jr. was the oldest of 4 children born \& raised in Cincinnati, Ohio, by his parents William \& Ella Gerretson Strunk. he took his bachelor's degree from the University of Cincinnati with a Bachelor of Arts in 1890. After achieving his degree he was employed by Rose Polytechnical Institute to teach mathematics from 1890--1891. From there he went on to teach at Cornell University while also earning his PhD, which he took in 1896. Following his PhD, he traveled to France for the academic year of 1898--99 where he spent time at the University of Paris: the Sorbonne \& the Coll\`ege de France where he studied philosophy $\Phi$ \& morphology\footnote{1. (biology) the form \& structure of animals \& plants, studied as a science; 2. (linguistics) the forms of words, studied as a branch of linguistics.}.

Upon his return to the United States, Strunk began teaching English at his alma mater, Cornell University. He taught there for 46 years \& was elected to Phi Beta Kappa $\phi\beta\kappa$, America's most prestigious honor society in the liberal arts, before his retirement. During his time teaching Strunk specialized in both English \& non-English literature.

He published several books, the 1st being {\it The Elements of Style} in 1918. In 1922, Strunk published {\it English Metres} which concentrated on the study of the poetic metrical form. Following the success of {\it The Elements of Style}, Strunk revised it with Edward A. Tenney in 1935, \& renamed it {\it The Element \& Practice of Composition}. As well as being an educator, Strunk served as a literary consultant for the film studio Metro-Goldwyn-Mayer from 1935--1936 on the set of the 1936 production of {\it Romeo \& Juliet}. William Strunk Jr. retired from teaching in 1937 \& died a few years later in 1945.

{\it The Elements of Style} was written \& privately published for his Cornell students in 1918. It was a guide to writing \& editing with the intention ``to lighten the task of instructor \& student by concentrating attention $\ldots$ on a few essentials, the rules of usage \& principles of composition most commonly violated.'' On the part of the instructor, it allowed him to simply refer to the rule which was broken when grading rather than having to explain it. For the student it became a guide or how-to for writing essays.

The short introduction to ``style'' included grammar insights such as how to form a possessive singular noun \& does \& don'ts such as, ``do use the active voice'' but ``don't break 1 sentence into 2.'' Strunk even had the brilliant foresight to add a section on commonly misspelt words.

In 1957, E.B. White, as a previous student of Strunk, praised what had become known as ``the little book''. He was then commissioned to revise \& update the book by Macmillian \& Company. That book, commonly known as ``Strunk \& White'' went on to sell over 2 million copies \& is still used in colleges \& universities today. This edition contains just Strunk's original 1918 advice to his students.''

%------------------------------------------------------------------------------%

\subsection*{Introduction}
``This book aims to give a brief space the principal requirements of plain English style. It aims to lighten the task of instructor \& student by concentrating attention (in Chaps. II \& III) on a few essentials, the rules of usage \& principles of composition most commonly violated. In accordance with this plan it lays down 3 rules for the use of the common, instead of a score or more, \& 1 for the use of the semicolon, in the belief that these 4 rules provide for all the internal punctuation that is required by 19 sentences out of 20. Similarly, it gives in Chap. III only those principles of the paragraph \& the sentence which are of the widest application. The book thus covers only a small portion of the field of English style. The experience of its writer has been that once past the essentials, students profit most by individual instruction based on the problems of their own work, \& that each instructors has his own body of theory, which he may prefer to that offered by any textbook.

The numbers of the sections may be used as references in correcting manuscript.

The writer's colleagues in the Department of English in Cornell University have greatly helped him in the preparation of his manuscript. Mr. George McLane Wood has kindly consented to the inclusion under Rule 10 of some material from his Suggestions to Authors.The following books are recommended for reference or further study: in connection with Chaps. II \& IV:
\begin{enumerate}
	\item F. Howard Collins. {\it Author \& Printer} (Henry Frowde).
	\item Chicago University Press. {\it Manual of Style}.
	\item T. L. De Vinne. {\it Correct Composition (The Century Company)}.
	\item Horace Hart. {\it Rules for Compositors \& Printers} (Oxford University Press).
	\item George McLane Wood. {\it Extracts from the Style-Book of the Government Printing Office} (United States Geological Survey).
\end{enumerate}
in connection with Chaps. III \& V
\begin{enumerate}
	\item {\it The King's English} (Oxford University Press).
	\item Sir Arthur Quiller-Couch. {\it The Art of Writing} (Putnam), especially the chapter, {\it Interlude on Jargon}.
	\item George McLane Wood. {\it Suggestions to Authors} (United States Geological Survey).
	\item John Lesslie Hall. {\it English Usage} (Scott, Foresman \& Co.).
	\item James P. Kelley. {\it Workmanship in Words} (Little, Brown \& Co.).
\end{enumerate}
In these will be found full discussions of many points here briefly treated \& an abundant store of illustrations to supplement those given in this book.

It is an old observation that the best writers sometimes disregard the rules of rhetoric. When they do so, however, the reader will usually find in the sentence some compensating merit, attained at the cost of the violation. Unless he is certain of doing as well, he will probably do best to follow the rules. After he has learned, by their guidance, to write plain English adequate for everyday uses, let him look, for the secrets of style, to the study of the masters of literature.''

%------------------------------------------------------------------------------%

%------------------------------------------------------------------------------%

\section*{Amazon/reviews}
\begin{quotation}
	`The Elements of Style' (1918), by William Strunk, Jr., is an American English writing style guide.
	
	It is the best-known, most influential prescriptive treatment of English grammar \& usage, \& often is required reading \& usage in U.S. high school \& university composition classes.
	
	This edition of `The Elements of Style' details 8 elementary rules of usage, 10 elementary principles of composition, ``a few matters of form'', \& a list of commonly misused words \& expressions.
\end{quotation}

\begin{quotation}
	``{\it $\ldots$ a marvelous \& timeless little book$\ldots$ Here, succinctly, elegantly \& without fuss are the essentials of writing clear, correct English}.'' - John Clare, {\it The Telegraph}
\end{quotation}

%------------------------------------------------------------------------------%

\section*{Forword}

The 1st writer Roger Angell watched at work was my stepfather, E. B. White.

Each Tuesday morning, he would close his study door \& sit down to write the ``Notes \& Comment'' page for {\it The New Yorker}.

The task was familiar with him - he was required to file a few hundred words of editorial or personal commentary on some topic in or out of the news that week - but the sounds of his typewriter from his room came in hesitant bursts, with long silences in between.

Hours went by.

Summoned at last for lunch, he was silent \& preoccupied, \& soon excused himself to get back to the job.

When the copy went off at last, in the afternoon RFD pouch - we were in Maine, a day's mail away from New York - he rarely seemed satisfied.

``It isn't good enough,'' he said sometimes.

``I wish it were better.''

%
Writing is hard, even for authors who do it all the time.

Less frequent practitioners - the job applicant; the business executive with an annual report to get out; the high school senior with a Faulkner assignment; the graduate-school student with her thesis proposal; the writer of a letter of condolence - often get stuck in an awkward passage of find a muddle on their screens, \& then blame themselves.

What should be easy \& flowing looks tangled or feeable or overblown - not what was meant at all.

What's wrong with me, each one thinks.

Why can't I get this right?

%
It was this recurring question, put to himself, that must have inspired White to revive \& add to a textbook by an English professor of his, Will Strunk Jr., that he had 1st read in college, \& to get it published.

The result, this quiet book, has been in print for 40 years, \& has offered more than 10 million writers a helping hand.

White knew that a compendium of specific tips - about singular \& plural verbs, parentheses, that ``that'' - ``which'' scuffle, \& many others - could clear up a recalcitrant sentence or subclause when quickly reconsulted, \& that the larger principles needed to be kept in plain sight, like a wall sampler.

%
How simple they look, set down here in White's last chapter: ``Write in a way that comes naturally,'' ``Revise \& rewrite,'' ``Do not explain too much,'' \& the rest; above all, the cleansing, clarion ``Be clear.''

How often Roger Angell has turned to them, in the book or in my mind, while trying to start or unblock or revise some piece of my own writing!

They help - they really do.

They work.

They are the way.

%
E. B. White's prose is celebrated for its ease \& clarity - just think of {\it Charlotte's Web} - but maintaining this standard required endless attention.

When the new issue of {\it The New Yorker} turned up in Maine, Roger Angell sometimes saw him reading his ``Comment'' piece over to himself, with only a slightly different expression than the one he'd worn on the day it went off.

Well, O.K., he seemed to be saying.

{\it At least I got the elements right}.

%
This edition has been modestly updated, with word processors \& air conditioners making their 1st appearance among White's references, \& with a light redistribution of genders to permit a feminine pronoun or female farmer to take their places among the males who once innocently served him.

Sylvia Plath has knocked Keats out of the box, \& Roger Angell notices that ``America'' has become ``this country'' in a sample text, to forestall a subsequent \& possibly demeaning ``she'' in the same paragraph.

What is not here is anything about E-mail - the rules-free, lower-case flow that cheerfully keeps us in touch these days.

E-mail is conversation, \& it may be replacing the sweet \& endless talking we once sustained (and tucked away) within the informal letter.

But we are all writers \& readers as well as communicators, with the need at times to please \& satisfy ourselves (as White put it) with the clear \& almost perfect thought.

\begin{flushright}
	Roger Angell
\end{flushright}

%------------------------------------------------------------------------------%

\subsection*{Introduction}

At the close of the 1st World War, when E. B. White was a student at Cornell, E. B. White took a course called {\it English 8}.

My professor was William Strunk Jr.

A textbook required for the course was a slim volume called {\it The Elements of Style}, whose authors was the professor himself.

The year was 1919.

The book was known on the campus in those days as ``the little book,'' with the stress on the word ``little.''

It had been privately printed by the author.

%
E. B. White passed the course, graduated from the university, \& forgot the book but not the professor.

Some 38 years later, the book bobbed up again in my life when Macmillan commissioned me to revise it for the college market \& the general trade.

Meantime, Prof. Strunk had died.

%
{\it The Elements of Style}, when E. B. White reexamined it in 1957, seemed to me to contain rich deposits of gold.

It was Will Strunk's {\it parvum opus}, his attempt to cut the vast tangle of English rhetoric down to size \& write its rules \& principles on the head of a pint.

Will himself had hung the tag ``little'' on the book; he referred to it sardonically \& with secret pride as ``the {\it little} book,'' always giving the word ``little'' a special twist, as though he were putting a spin on a ball.

In its original form, it was a 43 page summation of the case for cleanliness, accuracy, \& brevity in the use of English.

Today, 52 years later, its vigor is unimpaired, \& for sheer pith E. B. White thinks it probably sets a record that is not likely to be broken.

Even after E. B. White got through tampering with it, it was still a tiny thing, a barely tarnished gem.

7 rules of usage, 11 principles of composition, a few matters of form, \& a list of words \& expressions commonly misused - that was the sum \& substance of Prof. Strunk's work.

Somewhat audaciously, \& in an attempt to give my publisher his money's worth, E. B. White added a chapter called ``An Approach to Style,'' setting forth my own prejudices, my notions of error, my articles of faith.

This chapter (Chap. V) is addressed particularly to those who feel that English prose composition is not only a necessary skills but a sensible pursuit as well - a way to spend one's days.

E. B. White thinks Prof. Strunk would not object to that.

%
A 2nd edition of the book was published in 1972.

E. B. White has now completed a 3rd revision.

Chap. IV has been refurbished with words \& expressions of a recent vintage; 4 rules of usage have been added to Chap. I.

Fresh examples have been added to some of the rules \& principles, amplification has reared its head in a few places in the text where E. B. White felt an assault could successfully be made on the bastions of its brevity, \& in general the book has received a thorough overhaul - to correct errors, delete bewhiskered entries, \& enliven the argument.

%
Prof. Strunk was a positive man.

His book contains rules of grammar phrased as direct orders.

In the main E. B. White has not tried to soften his commands, or modify his pronouncements, or remove the special objects of his scorn.

E. B. White has tried, instead, to preserve the flavor of his discontent while slightly enlarging the scope of the discussion.

{\it The Element of Style} does not pretend to survey the whole field.

Rather it proposes to give in brief space the principal requirements of plain English style.

It concentrates on fundamentals: the rules of usage \& principles of composition most commonly violated.

%
The reader will soon discover that these rules \& principles are in the form of sharp commands, Sergeant Strunk snapping orders to his platoon.

``Do not joint independent clauses with a comma.'' (Rule 5.)

``Do not break sentences in 2.'' (Rule 6.)

``Use the active voice.'' (Rule 14.)

``Omit needless words.'' (Rule 17.)

``Avoid a succession of loose sentences.'' (Rule 18.)

``In summaries, keep to 1 tense.'' (Rule 21.)

Each rule or principle is followed by a short hortatory essay, \& usually the exhortation is followed by, or interlarded with, examples in parallel columns - the true vs. the false, the right vs. the wrong, the timid vs. the bold, the ragged vs. the trim.

From every line there peers out at me the puckish face of my professor, his short hair pared neatly in the middle \& combed down over his forehead, his eyes blinking incessantly behind steel-rimmed spectacles as though he had just emerged into strong light, his lips nibbling each other like nervous horses, his smile shuttling to \& fro under a carefully edged mustache.

%
``Omit needless words!'' cries the author on p. 24, \& into that imperative Will Strunk really put his heart \& soul.

In the days when E. B. White was sitting in his class, he omitted so many needless words, \& omitted them so forcibly \& with such eagerness \& obvious relish, that he often seemed in the position of having shortchanged himself - a man left with nothing more to say yet with time to fill, a radio prophet who had out-distanced the clock.

Will Strunk got out of this predicament by a simple trick: he uttered every sentence 3 times.

When he delivered his oration on brevity to the class, he leaned forward over his desk, grasped his coat lapels in his hands, and, in a husky, conspiratorial voice, said, ``Rule 17. Omit needless words! Omit needless words! Omit needless words!''

%
He was a memorable man, friendly \& funny.

Under the remembered string of his kindly lash, E. B. White has been trying to omit needless words since 1919, \& although there are still many words that cry for omission \& the huge task will never be accomplished, it is exciting to me to reread the masterly Strunkian elaboration of this noble theme.

It goes:

\begin{quotation}
	\it
	Vigorous writing is concise. A sentence should contain no unnecessary words, a paragraph no unnecessary sentences, for the same reason that a drawing should have no unnecessary lines \& a machine no unnecessary parts.
	
	This requires not that the writer make all sentences short or avoid all detail \& treat subjects only in outline, but that every word tell.
\end{quotation}
There you have a short, valuable essay on the nature \& beauty of brevity - 59 words that could change the world.

Having recovered from his adventure in prolixity (59 words were a lot of words in the tight world of William Strunk Jr.), the professor proceeds to give a few quick lessons in pruning.

Students learn to cut the dead-wood from ``this is a subject that,'' reducing it to ``this subject,'' a saving of 3 words.

They learn to trim ``used for fuel purposes'' down to ``used for fuel.''

They learn that they are being chatterboxes when they say ``the question as to whether'' \& they should just day ``whether'' - a saving of 4 words out of a possible 5.

%
The professor devotes a special paragraph to the vile expression {\it the fact that}, a phrase that causes him to quiver with revulsion.

The expression, he says, should be ``revised out of every sentence in which it occurs.''

But a shadow of gloom seems to hang over the page, \& you feel that he knows how hopeless his cause is.

E. B. White supposes E. B. White has written {\it the fact that} a thousand times in the heat of composition, revised it out maybe 500 times in the cool aftermath.

To be batting only .500 this late in the season, to fail half the time to connect with this fat pitch, saddens me, for it seems a betrayal of the man who showed me how to swing at it \& made the swinging seem worthwhile.

%
E. B. White treasures {\it The Elements of Style} for its shape advice, but E. B. White treasures it even more for the audacity \& self-confidence of its author.

Will knew where he stood.

He was so sure of where he stood, \& made his position so clear \& so plausible, that his peculiar stance has continued to invigorate me - and, E. B. White is sure, thousands of other ex-students - during the years that have intervened since our 1st encounter.

He had a number of likes \& dislikes that were almost as whimsical as the choice of a necktie, yet he made them seem utterly convincing.

He disliked the word {\it forceful} \& advised us to use {\it forcible} instead.

He felt that the word {\it clever} was greatly overused: ``It is best restricted to ingenuity displayed in small matters.''

He despised the expression {\it student body}, which he termed gruesome, \& made a special trip downtown to the {\it Alumni News} office 1 day to protest the expression \& suggest that {\it studentry} be substituted - a coinage of is own, which he felt was similar to {\it citizenry}.

E. B. White is told that the {\it News} editor was so charmed by the visit, if not by the word, that he ordered the student body buried, never to rise again.

{\it Studentry} has taken its place.

It's not much of an improvement, but it does sound less cadaverous, \& it made Will Strunk quite happy.

%
Some years ago, when the heir to the throne of England was a child, E. B. White noticed a headline in the {\it Times} about Bonnie Prince Charlie: ``CHARLES' TONSILS OUT.''

Immediately Rule 1 leapt to mind.
\begin{enumerate}
	\item Form the possessive singular of nuns by adding {\it 's}.
	
	Follow this rule whatever the final consonant.
	
	Thus write,
	\begin{example}
		Charles's friend
		
		Burns's poems
		
		the witch's malice
	\end{example}
\end{enumerate}
Clearly, Will Strunk had foreseen, as far back as 1918, the dangerous tonsillectomy of a prince, in which the surgeon removes the tonsils \& the {\it Times} copy desk removes the final {\it s}.

He started his book with it.

E. B. White recommends Rule 1 to the {\it Times}, \& E. B. White trusts that Charles's throat, not Charles' throat, is in fine shape today.

%
Style rules of this sort are, of course, somewhat a matter of individual preference, \& even the established rules of grammar are open to challenge.

Professor Strunk, although 1 of the most inflexible \& choosy of men, was quick to acknowledge the fallacy of inflexibility \& the danger of doctrine.
\begin{quotation}
	``{\it It is an old observation},'' he wrote, ``{\it that the best writers sometimes disregard the rules of rhetoric. When they do so, however, the reader will usually find in the sentence some compensating merit, attained at the cost of the violation. Unless he is certain of doing as well, he will probably do best to follow the rules}.''
\end{quotation}
It is encouraging to see how perfectly a book, even a dusty rule book, perpetuates \& extends the spirit of a man.

Will Strunk loved the clear, the brief, the bold, \& his book is clear, brief, bold.

Boldness is perhaps its chief distinguishing mark.

On p. 26, explaining 1 of his parallels, he says,
\begin{quotation}
	``{\it The lefthand version gives the impression that the writer is undecided or timid, apparently unable or afraid to choose 1 form of expression \& hold to it}.''
\end{quotation}
\& his original Rule 11 was ``Make definite assertions.''

That was Will all over.

He scorned the vague, the tame, the colorless, the irresolute.

He felt it was worse to be irresolute than to be wrong.

E. B. White remembers a day in class when he leaned far forward, in this characteristic pose - the pose of a man about to impart a secret - \& croaked,
\begin{quotation}
	``{\it If you don't know how to pronounce a word, say it loud! If you don't know how to pronounce a word, say it loud!}''
\end{quotation}
This comical piece of advice struck me as sound at the time, \& E. B. White still respects it.

{\it Why compound ignorance with inaudibility?}

{\it Why run \& hide?}

%
All through {\it The Elements of Style} one finds evidences of the author's deep sympathy for the reader.

Will felt that the reader was in serious trouble most of the time, floundering in a swamp, \& that it was the duty of anyone attempting to write English to drain this swamp quickly \& get the reader up on dry ground, or at least to throw a rope.

In revising the text, E. B. White has tried to hold steadily in mind this belief of his, this concern for the bewildered reader.

%
In the English classes of today, ``the little book'' is surrounded be longer, lower textbooks - books with permissive steering \& automatic transitions.

Perhaps the book has become something of a curiosity.

To E. B. White, it still seems to maintain its original poise, standing, in a drafty time, erect, resolute, \& assured.

E. B. White still finds the Strunkian wisdom a comfort, the Strunkian humor a delight, \& the Strunkian attitude toward right-and-wrong a blessing undisguised.

\begin{flushright}
	E. B. White
\end{flushright}

%------------------------------------------------------------------------------%

\subsection{Elementary Rules of Usage}

\subsubsection{Form the possessive singular of nouns by adding 's.}
Follow this rule whatever the final consonant.

Thus write,
\begin{example}
	Charles's friend
	
	Burns's poems
	
	the witch's malice
\end{example}
Exceptions are the possessive of ancient proper names in {\it -es} an {\it -is}, the possessive {\it Jesus'}, \& such forms as {\it for conscience' sake, for righteousness' sake}.

But such forms as {\it Achilles' heel, Moses' laws, Isis' temple} are commonly replaced by
\begin{example}
	the laws of Moses
	
	the temple of Isis
\end{example}
The pronominal possessives {\it hers, its, theirs, yours}, \& {\it ours} have no apostrophe.

Indefinite pronouns, however, use the apostrophe to show possession.
\begin{example}
	one's rights
	
	somebody else's umbrella
\end{example}
A common error is to write {\it it's} for {\it its}, or vice versa.

The 1st is a contraction, meaning ``it is.''

The 2nd is a possessive.
\begin{example}
	It's wise dog that scratches its own fleas.
\end{example}

%------------------------------------------------------------------------------%

\subsubsection{In a series of 3 or more terms with a single conjunction, use a comma after each term except the last}
Thus write,
\begin{example}
	red, white, \& blue
	
	gold, silver, or copper
	
	He opened the letter, read it, \& made a note of its contents.
\end{example}
This comma is often referred to as the ``serial'' comma.

In the names of business firms the last comma is usually omitted.

Follow the usage of the individual firm.
\begin{example}
	Little, Brown \& Company
	
	Donaldson, Lufkin \& Jenrette
\end{example}

%------------------------------------------------------------------------------%

\subsubsection{Enclose parenthetic expressions between commas}
\begin{example}
	The best way to see a country, unless you are pressed for time, is to travel on foot.
\end{example}
This rule is difficult to apply; it is frequently hard to decide whether a single word, e.g., {\it however}, or a brief phrase is or is not parenthetic.

If the interruption to the flow of the sentence is but slight, the commas may be safely omitted.

But whether the interruption is slight or considerable, never omit 1 comma \& leave the other.

There is no defense for such punctuation as
\begin{example}
	Marjories husband, Colonel Nelson paid us a visit yesterday.
\end{example}
or
\begin{example}
	My brother you will be pleased to hear, is now in perfect health.
\end{example}
Dates usually contain parenthetic words or figures.

Punctuate as follows:
\begin{example}
	February to July, 1992
	
	April 6, 1986
	
	Wednesday, November 14, 1990
\end{example}
Note that it is customary to omit the comma in
\begin{example}
	6 April 1988
\end{example}
The last form is an excellent way to write a date; the figures are separated by a word \& are, for that reason, quickly grasped.

%
A name or a title in direct address is parenthetic.
\begin{example}
	If, Sir, you refuse, I cannot predict what will happen.
	
	Well, Susan, this is a fine mess you are in.
\end{example}
The abbreviation {\it etc., i.e.}, \& {\it e.g.}, the abbreviations for academic degrees, \& titles that follow a name are parenthetic \& should be punctuated accordingly.
\begin{example}
	Letters, packages, etc., should go here.
	
	Horace Fulsome, Ph.D., presided.
	
	Rachel Simonds, Attorney
	
	The Reverend Harry Lang, S.J.
\end{example}
No comma, however, should separate a noun from a restrictive term of identification.
\begin{example}
	Billy the Kid
	
	The novelist Jane Austen
	
	William the Conqueror
	
	The poet Sappho
\end{example}
Although {\it Junior}, with its abbreviation {\it Jr.}, has commonly been regarded as parenthetic, logic suggests that it is, in fact, restrictive \& therefore not in need of a comma.
\begin{example}
	James Wright Jr.
\end{example}
Nonrestrictive relative clauses are parenthetic, as are similar clauses introduced by conjunctions indicating time or place.

Commas are therefore needed.

A nonrestrictive clauses is one that does not serve to identify or define the antecedent noun.
\begin{example}
	The audience, which had at 1st been indifferent, became more \& more interested.
	
	In 1769, when Napoleon was born, Corsica had but recently been acquired by France.
	
	Nether Stowey, where Coleridge wrote The Rime of the Ancient Mariner, is a few miles from Bridgewater.
\end{example}
In these sentences, the clauses introduced by {\it which, when}, \& {\it where} are nonrestrictive; they do not limit or define, they merely add something.

In the 1st example, the clause introduced by {\it which} does not serve to tell which of several possible audiences is meant; the reader presumably knows that already.

The clause adds, parenthetically, a statement supplementing that in the main clause.

Each of the 3 sentences is a combination of 2 statements that might have been made independently.
\begin{example}
	The audience was at 1st indifferent. Later it became more \& more interested.
	
	Napoleon was born in 1769. At that time Corsica had but recently been acquired by France.
	
	Coleridge wrote The Rime of the Ancient Mariner at Nether Stowey. Nether Stowey is a few miles from Bridgewater.
\end{example}
Restrictive clauses, by contrast, are not parenthetic \& not set off by commas.

Thus.
\begin{example}
	People who live in glass houses shouldn't throw stones.
\end{example}
Here the clause introduced by {\it who} does serve to tell which people are meant; the sentence, unlike the sentences above, cannot be split into 2 independent statements.

The same principle of comma use applies to participial phrases \& to appositives.
\begin{example}
	People sitting in the rear couldn't hear, \emph{(restrictive)}
	
	Uncle Bert, being slightly deaf, moved forward, \emph{(non-restrictive)}
	
	My cousin Bob is a talented harpist, \emph{(restrictive)}
	
	Our oldest daughter, Mary, sings, \emph{nonrestrictive}
\end{example}
When the main clause of a sentence is preceded by a phrase or a subordinate clause, use a comma to set off these elements.
\begin{example}
	Partly by hard fighting, partly by diplomatic skill, they enlarged their dominions to the east \& rose to royal rank with the possession of Sicily.
\end{example}

%------------------------------------------------------------------------------%

\subsubsection{Place a comma before a conjunction introducing an independent clause}
\begin{example}
	The early records of the city have disappeared, \& the story of its 1st years can no longer be reconstructed.
	
	The situation is perilous, but there is still 1 chance of escape.
\end{example}
2-part sentences of which the 2nd member is introduced by as (in the sense of ``because''), {\it for, or, nor}, or {\it while} (in the sense of ``and at the same time'') likewise require a comma before the conjunction.

%
If a dependent clause, or an introductory phrase requiring to be set off by a comma, precedes the 2nd independent clause, no comma is needed after the conjunction.
\begin{example}
	The situation is perilous, but if we are prepared to act promptly, there is still 1 chance of escape.
\end{example}
When the subject is the same for both clauses \& is expressed only once, a comma is useful if the connective is {\it but}.

When the connective is {\it and}, the comma should be omitted if the relation between the 2 statements is close or immediate.
\begin{example}
	I have heard the arguments, but am still unconvinced.
	
	He has had several year's experience \& is thoroughly competent.
\end{example}

%------------------------------------------------------------------------------%

\subsubsection{Do not join independent clauses with a comma}
If 2 or more clauses grammatically complete \& not joined by a conjunction are to form a single compound sentence, the proper mark of punctuation is a semicolon.
\begin{example}
	Mary Shelley's works are entertaining; they are full of engaging ideas.
	
	It is nearly half past 5; we cannot reach town before dark.
\end{example}
It is, of course, equally correct to write each of these as 2 sentences, replacing the semicolons with periods.
\begin{example}
	Mary Shelley's works are entertaining. They are full of engaging ideas.
	
	It is nearly half past 5. We cannot reach town before dark.
\end{example}
If a conjunction is inserted, the proper mark is a comma. (Rule 4.)
\begin{example}
	Mary Shelley's works are entertaining, for they are full of engaging ideas.
	
	It is nearly half past 5, \& we cannot reach town before dark.
\end{example}
A comparison of the 3 forms given above will show clearly the advantage of the 1st.

It is, at least in the examples given, better than the 2nd form because it suggests the close relationship between the 2 statements in a way that the 2nd does not attempt, \& better than the 3rd because it is briefer \& therefore more forcible.

Indeed, this simple method of indicating relationship between statements is 1 of the most useful devices of composition.

The relationship, as above, is commonly 1 of cause \& consequence.

%
Note that if the 2nd clause is preceded by an adverb, e.g., {\it accordingly, besides, then, therefore}, or {\it thus}, \& not by a conjunction, the semicolon is still required.
\begin{example}
	I had never been in the place before; besides, it was dark as a tomb.
\end{example}
An exception to the semicolon rule is worth noting here.

A comma is preferable when the clauses are very short \& alike in form, or when the tone of the sentence is easy \& conversational.
\begin{example}
	Man proposes, God disposes.
	
	The gates swung apart, the bridge fell, the portcullis was drawn up.
	
	I hardly knew him, he was so changed.
	
	Here today, gone tomorrow.
\end{example}

%------------------------------------------------------------------------------%

\subsubsection{Do not break sentences in 2}
In other words, do not use periods for commas.
\begin{example}
	I met them on a Cunard liner many years ago. Coming home from Liverpool to New York.
	
	She was an interesting talker. A woman who had traveled all over the world \& lived in half of dozen countries.
\end{example}
In both these examples, the 1st period should be replaced by a comma \& the following word begun with a small letter.

%
It is permissible to make an emphatic word or expression serve the purpose of a sentence \& to punctuate it accordingly:
\begin{example}
	Again \& again he called out. No reply.
\end{example}
The writer must, however, be certain that the emphasis is warranted, lest a clipped sentence seem merely a blunder in syntax or in punctuation.

Generally speaking, the place for broken sentences is in dialogue, when a character happens to speak in a clipped or fragmentary way.

%
Rules 3, 4, 5, \& 6 cover the most important principles that govern punctuation.

They should be so thoroughly mastered that their application becomes 2nd nature.

%------------------------------------------------------------------------------%

\subsubsection{Use a colon after an independent clause to introduce a list of particulars, an appositive, an amplification, or an illustrative quotation}
A colon tells the reader that what follows is closely related to the preceding clause.

The colon has more effect than the comma, less power to separate than the semicolon, \& more formality than the dash.

It usually follows an independent clause \& should not separate a verb from its complement or a preposition from its object.

The examples in the lefthand column, below, are wrong; they should be rewritten as in the righthand column.
\begin{example}
	Your dedicated whittler requires: a knife, a piece of wood, \& a back porch.
	
	Understanding is that penetrating quality of knowledge that grows from: theory, practice, conviction, assertion, error, \& humiliation.
	
	Your dedicated whittler requires 3 props: a knife, a piece of wood, \& a back porch.
	
	Understanding is that penetrating quality of knowledge that grows from theory, practice, conviction, assertion, error, \& humiliation.
\end{example}
Join 2 independent clauses with a colon if the 2nd interprets or amplifies the 1st.
\begin{example}
	But even so, there was a directness \& dispatch about animal burial: there was no stopover in the undertaker's foul parlor, no wreath or spray.
\end{example}
A colon may introduce a quotation that supports or contributes to the preceding clause.
\begin{example}
	The squalor of the streets reminded her of a line from Oscar Wilde: ``We are all in the gutter, but some of us are looking at the star.''
\end{example}
The colon also has certain functions of form: to follow the salutation of a former letter, to separate hour from minute in a notation of time, \& to separate the title of work from its subtitle or a Bible chapter from a verse.
\begin{example}
	Dear Mr. Montague:
	
	departs at 10:48 P.M.
	
	Practical Calligraphy: An Introduction to Italic Script
\end{example}

%------------------------------------------------------------------------------%

\subsubsection{Use a dash to set off an abrupt break or interruption \& to announce a long appositive or summary}
A dash is a mark of separation stronger than a comma, less formal than a colon, \& more relaxed than parentheses.
\begin{example}
	His 1st thought on getting out of bed - if he had any thought at all - was to get back in again.
	
	The rear axle began to make a noise - a grinding, chattering, teeth-gritting rasp.
	
	The increasing reluctance of the sun to rise, the extra nip in the breeze, the patter of shed leaves dropping - all the evidences of fall drifting into winter were clearer each day.
\end{example}
Use a dash only when a more common mark of punctuation seems inadequate.
\begin{example}
	Her father's suspicions proved well-founded - it was not Edward she cared for - it was San Francisco.
	
	$\to$ Her father's suspicions proved well-founded. It was not Edward she cared for, it was San Francisco.
	
	Violence - the kind you see on television - is not honestly violent - there lies its harm.
	
	$\to$ Violence, the kind you see on television, is not honestly violent. There lies its harm.
\end{example}

%------------------------------------------------------------------------------%

\subsubsection{The number of the subject determines the number of the verb}
Words that intervene between subject \& verb do not affect the number of the verb.
\begin{example}
	The bittersweet flavor of youth - its trials, its joys, its adventures, its challenges - are not soon forgotten.
	
	$\to$ The bittersweet flavor of youth - its trials, its joys, its adventures, its challenges - is not soon forgotten.
\end{example}
A common blunder is the use of a singular verb form in a relative clause following ``one of$\ldots$'' or a similar expression when the relative is the subject.
\begin{example}
	1 of the the ablest scientists who has attacked this problem.
	
	$\to$ 1 of the ablest scientists who have attached this problem.
	
	1 of those people who is never ready on time $\to$ 1 of those people who are never ready on time
\end{example}
Use a singular verb form after {\it each, either, everyone, everybody, neither, nobody, someone}.
\begin{example}
	Everybody thinks he has a unique sense of humor.
	
	Although both clocks strike cheerfully, neither keeps good time.
\end{example}
With {\it none}, use the singular verb when the word means ``no one'' or ``not one.''
\begin{example}
	None of us are perfect.
	
	None of us is perfect.
\end{example}
A plural verb is commonly used when {\it none} suggests more than 1 thing or person.
\begin{example}
	None are so fallible as those who are sure they're right.
\end{example}
A compound subject formed of 2 or more nouns joined by {\it and} almost always requires a plural verb.
\begin{example}
	The walrus \& the carpenter were walking close at hand.
\end{example}
But certain compounds, often cliches, are so inseparable they are considered a unit \& so take a singular verb, as do compound subjects qualified by {\it each} or {\it every}.
\begin{example}
	The long \& the short of it is$\ldots$
	
	Bread \& butter was all she served.
	
	Give \& take is essential to a happy household.
	
	Every window, picture, \& mirror was smashed.
\end{example}
A singular subject remains singular even if other nouns are connected to it by {\it with, as well as, in addition to, except, together with}, \& {\it no less than}.
\begin{example}
	His speech as well as his manner is objectionable.
\end{example}
A linking verb agrees with the number of its subject.
\begin{example}
	What is wanted is a few more pairs of hands.
	
	The trouble with truth is its many varieties.
\end{example}
Some nouns that appear to be plural are usually construed as singular \& given a singular verb.
\begin{example}
	Politics is an art, not a science.
	
	The Republican Headquarters is on this side of the tracks.
\end{example}
But
\begin{example}
	The general's quarters are across the river.
\end{example}
In these cases the writer must simply learn the idioms.

The contents of a book is singular.

The contents of a jar may be either singular or plural, depending on what's in the jar - jam or marbles.

%------------------------------------------------------------------------------%

\subsubsection{Use the proper case of pronoun}
The personal pronouns, as well as the pronoun {\it who}, change form as they function as subject or object.
\begin{example}
	Will Jane or he be hired, do you think?
	
	The culprit, it turned out, was he.
	
	We heavy eaters would rather walk than ride.
	
	Who knocks?
	
	Give this work to whoever looks idle.
\end{example}
In the last example, {\it whoever} is the subject of {\it look idle}; the object of the preposition {\it to} is the entire clause {\it whoever looks idle}.

When {\it who} introduces a subordinate clause, its case depends on its function in that clause.
\begin{example}
	Virgil Soames is the candidate whom we think will win.
	
	$\to$ Virgil Soames is the candidate who we think will win. [We think \emph{he} will win.]
	
	Virgil Soames is the candidate who we hope to elect.
	
	$\to$ Virgil Soames is the candidate whom we hope to elect. [We hope to elect \emph{him}.]
\end{example}
A pronoun in a comparison is nominative if it is the subject of a stated or understood verb.
\begin{example}
	Sandy writes better than I. (Than I write.)
\end{example}
In general, avoid ``understood'' verbs by supplying them.
\begin{example}
	I think Horace admires Jessica more than I.
	
	$\to$ I think Horace admires Jessica more than I do.
	
	Polly loves cake more than me.
	
	$\to$ Polly loves cake more than she loves me.
\end{example}
The objective case is correct in the following examples.
\begin{example}
	The ranger offered Shirley \& him some advice on campsites.
	
	They came to meet the Baldwins \& us.
	
	Let's talk it over between us, then, you \& me.
	
	Whom should I ask?
\end{example}

\begin{example}
	A group of us taxpayers protested.
\end{example}
{\it Us} in the last example is in apposition to taxpayers, the object of the proposition {\it of}.

The wording, although grammatically defensible, is rarely apt.

``A group of us protested as taxpayers'' is better, if not exactly equivalent.

%
Use the simple personal pronoun as a subject.
\begin{example}
	Blake \& myself stayed home.
	
	$\to$ Blake \& I stayed home.
	
	Howard \& yourself brought the lunch, I thought.
	
	$\to$ Howard \& you brought the lunch, I thought.
\end{example}
The possessive case of pronouns is used to show ownership.

It has 2 forms: the adjectival modifier, {\it your} hat, \& the noun form, a hat {\it of yours}.
\begin{example}
	The dog has buried 1 of your gloves \& 1 of mine in the flower bed.
\end{example}
Gerunds usually require the possessive case.
\begin{example}
	Mother objected to our driving on the icy roads.
\end{example}
A present participle as a verbal, on the other hand, takes the objective case.
\begin{example}
	They heard him singing in the shower.
\end{example}
The difference between a verbal participle \& a gerund is not always obvious, but note what is really said in each of the following.
\begin{example}
	Do you mind me asking a question?
	
	Do you mind my asking a question?
\end{example}
In the 1st sentence, the queried objection is to {\it me}, as opposed to other members of the group, asking a question.

In the 2nd example, the issue is whether a question may be asked at all.

%------------------------------------------------------------------------------%

\subsubsection{A participial phrase at the beginning of a sentence must refer to the grammatical subject}
\begin{example}
	Walking slowly down the road, he saw a woman accompanied by 2 children.
\end{example}
The word {\it walking} refers to the subject to the sentence, not to the woman.

To make it refer to the woman, the writer must recast the sentence.
\begin{example}
	He saw a woman, accompanied by 2 children, walking slowly down the road.
\end{example}
Participial phrases preceded by a conjunction or by a preposition, nouns in apposition, adjectives, \& adjective phrases come under the same rule if they begin the sentence.
\begin{example}
	On arriving in Chicago, his friends met him at the station.
	
	$\to$ On arriving in Chicago, he was met at the station by his friends.
	
	A soldier of proved valor, they entrusted him with the defense of the city.
	
	$\to$ A soldier of proved valor, he was entrusted with the defense of the city.
	
	Young \& inexperienced, the task seemed easy to me.
	
	$\to$ Young \& inexperienced, I thought the task easy.
	
	Without a friend to counsel him, the temptation proved irresistible.
	
	$\to$ Without a friend to counsel him, he found the temptation irresistible.
\end{example}
Sentences violating Rule 11 are often ludicrous:
\begin{example}
	Being in a dilapidated condition, I was able to buy the house very cheap.
	
	Wondering irresolutely what to do next, the clock struck 12.
\end{example}

%------------------------------------------------------------------------------%

\subsection{Elementary Principles of Composition}

\subsubsection{Choose a suitable design \& hold to it.}
``A basic structural design underlies every kind of writing. Writers will in part follow this design, in part deviate from it, according to their skills, their needs, \& the unexpected events that accompany the act of composition. Writing, to be effective, must follow closely the thoughts of the writer, but not necessarily in the order in which those thoughts occur. This calls for a scheme of procedure. In some cases, the best design is no design, as with a love letter, which is simply an outpouring, or with a casual essay, which is a ramble. But in most cases, planning must be a deliberate prelude to writing. The 1st principle of composition, therefore, is to foresee or determine the shape of what is to come \& pursue that shape.

A sonnet is built on a 14-line frame, each line containing 5 feet. Hence, sonneteers know exactly where they are headed, although they may not know how to get there. Most forms of composition are less clearly defined, more flexible, but all have skeletons to which the writer will bring the flesh \& the blood. The more clearly the writer perceives the shape, the better are the chances of success.'' -- \cite[p. 29]{Strunk_White_element_style}

%------------------------------------------------------------------------------%

\subsubsection{Make the paragraph the unit of composition: 1 paragraph to each topic.}
``The paragraph is a convenient unit; it serves all forms of literary work. As long as it holds together, a paragraph may be of any length -- a single, short sentence or a passage of great duration.

If the subject on which you are writing is of slight extent, or if you intend to treat it briefly, there may be no need to divide it into topics. Thus, a brief description, a brief book review, a brief account of a single incident, a narrative merely outlining an action, the setting forth of a single idea -- any 1 of these is best written in a single paragraph. After the paragraph has been written, examine it to see whether division will improve it.

Ordinarily, however, a subject requires division into topics, each of which should be dealt with in a paragraph. The object of treating each topic in a paragraph by itself is, of course, to aid the reader. The beginning of each paragraph is a signal that a new step in the development of the subject has been reached.

As a rule, single sentences should not be written or printed as paragraphs. An exception may be made of sentences of transition, indicating the relation between the parts of an exposition or argument.

In dialogue, each speech, even if only a single word, is usually a paragraph by itself; i.e., a new paragraph begins with each change of speaker. The application of this rule when dialogue \& narrative are combined is best learned from examples in well-edited works of fiction. Sometimes a writer, seeking to create an effect of rapid talk or for some other reason, will elect not to set off each speech in a separate paragraph \& instead will run speeches together. The common practice, however, \& the one that serves best in most instances, is to give each speech a paragraph of its own.

As a rule, begin each paragraph either with a sentence that suggests the topic or with a sentence that helps the transition. If a paragraph forms part of a larger composition, its relation to what precedes, or its function as a part of the whole, may need to be expressed. This can sometimes be done by a mere word or phrase ({\it again, therefore, for the same reason}) in the 1st sentence. Sometimes, however, it is expedient to get into the topic slowly, by way of a sentence or 2 of introduction or transition.

In narration \& description, the paragraph sometimes begins with a concise, comprehensive statement serving to hold together the details that follows.
\begin{quotation}\it
	The breeze served us admirably.
	
	The campaign opened with a series of reverses.
	
	The next 10 or 12 pages were filled with a curious set of entries.
\end{quotation}
But when this device, or any device, is too often used, it becomes a mannerism. More commonly, the opening sentence simply indicates by its subject the direction the paragraph is to take.
\begin{quotation}\it
	At length I thought I might return toward the stockade.
	
	He picked up the heavy lamp from the table \& began to explore.
	
	Another flight of steps, \& they emerged on the roof.
\end{quotation}
In animated narrative, the paragraphs are likely to be short \& without any semblance of a topic sentence, the writer rushing headlong, event following event in rapid succession. The break between such paragraphs merely serves the purpose of a rhetorical pause, throwing into prominence some detail of the action.

In general, remember that paragraphing calls for a good eye as well as a logical mind. Enormous blocks of print look formidable to readers, who are often reluctant to tackle them. Therefore, breaking long paragraphs in 2, even if it is not necessary to do so for sense, meaning, or logical development, is often a visual help. But remember, too, that firing off many short paragraphs in quick succession can be distracting. Paragraph breaks used only for show read like the writing of commerce or of display advertising. Moderation \& a sense of order should be the main considerations in paragraphing.'' -- \cite[pp. 30--31]{Strunk_White_element_style}

%------------------------------------------------------------------------------%

\subsubsection{Use the active voice.}
``The active voice is usually more direct \& vigorous than the passive:
\begin{example}
	I shall always remember my 1st visit to Boston.
\end{example}
This is much better than
\begin{example}
	My 1st visit to Boston will always be remembered by me.
\end{example}
The latter sentence is less direct, less bold, \& less concise. If the writer tries to make it more concise by omitting ``by me,''
\begin{example}
	My 1st visit to Boston will always be remembered,
\end{example}
it becomes indefinite: is it the writer or some undisclosed person or the world at large that will always remember this visit?

This rule does not, of course, mean that the writer should entirely discard the passive voice, which is frequently convenient \& sometimes necessary.
\begin{example}
	The dramatists of the Restoration are little esteemed today.
	
	Modern readers have little esteem for the dramatists of the Restoration.
\end{example}
The 1st would be the preferred form in a paragraph on the dramatists of the Restoration, the 2nd in a paragraph on the tastes of modern readers. The need to make a particular word the subject of the sentence will often, as in these examples, determine which voice is to be used.

The habitual use of the active voice, however, makes for forcible writing. This is true not only in narrative concerned principally with action but in writing of any kind. Many a tame sentence of description or exposition can be made lively \& emphatic by substituting a transitive in the active voice for some such perfunctory expression as {\it there is} or {\it could be heard}.
\begin{example}
	There were a great number of dead leaves lying on the ground. $\to$ Dead leaves covered the ground.
	
	At dawn the crowing of a rooster could be heard. $\to$ The cock's crow came with dawn.
	
	The reason he left college was that his health became impaired. $\to$ Failing health compelled him to leave college.
	
	It was not long before she was very sorry that she had said what she had. $\to$ She soon repented her words.
\end{example}
Note, in the examples above, that when a sentence is made stronger, it usually becomes shorter. Thus, brevity is a by-product of vigor.'' -- \cite[p. 32]{Strunk_White_element_style}

%------------------------------------------------------------------------------%

\subsubsection{Put statements in positive form.}
``Make definite assertions. Avoid tame, colorless, hesitating, noncommittal language. Use the word {\it not} as a means of denial or in antithesis, never as a means of evasion.
\begin{example}
	He was not very often on time. $\to$ He usually came late.
	
	She did not think that studying Latin was a sensible way to use one's time. $\to$ She thought the study of Latin a waste of time.
	
	\emph{The Taming of the Shrew} is rather weak in spots. Shakespeare does not portray Katharine as a very admirable character, nor does Bianca remain long in memory as an important character in Shakespeare's works. $\to$ The women in \emph{The Taming of the Shrew} are unattractive. Katharine is disagreeable, Bianca insignificant.
\end{example}
The last example, before correction, is indefinite as well as negative. The corrected version, consequently, is simply a guess at the writer's intention.

All 3 examples show the weakness inherent in the word {\it not}. Consciously or unconsciously, the reader is dissatisfied with being told only what is not; the reader wishes to be told what is. Hence, as a rule, it is better to express even a negative in positive form.
\begin{example}
	not honest $\to$ dishonest, not important $\to$ trifling, did not remember $\to$ forgot, did not pay any attention to $\to$ ignored, did not have much confidence in $\to$ distrusted.
\end{example}
Placing negative \& positive in opposition makes for a stronger structure.
\begin{example}
	Not charity, but simple justice.
	
	Not that I loved Caesar less, but that I loved Rome more.
	
	Ask not what your country can do for you -- ask what you can do for your country.
\end{example}
Negative words other than {\it not} are usually strong.
\begin{example}
	Her loveliness I never knew
	
	Until she smiled on me.
\end{example}
Statements qualified with unnecessary auxiliaries or conditionals sound irresolute.
\begin{example}
	If you would let us know the time of your arrival, we would be happy to arrange your transportation from the airport. $\to$ If you will let us know the time of your arrival, we shall be happy to arrange your transportation from the airport.
	
	Applicants can make a good impression by being neat \& punctual. $\to$  Applicants will make a good impression if they are neat \& punctual.
	
	Plath may be ranked among those modem poets who died young. $\to$ Plath was one of those modern poets who died young.
\end{example}
If your every sentence admits a doubt, your writing will lack authority. Save the auxiliaries {\it would, should, could, may, might}, \& {\it can} for situations involving real uncertainty.'' -- \cite[pp. 33--34]{Strunk_White_element_style}

%------------------------------------------------------------------------------%

\subsubsection{Use definite, specific, concrete  language.}
``Prefer the specific to the general, the definite to the vague, the concrete to the abstract.
\begin{example}
	A period of unfavorable weather set in. $\to$ It rained every day for a week.
	
	He showed satisfaction as he took possession of his well-earned reward. $\to$ He grinned as he pocketed the coin.
\end{example}
If those who have studied the art of writing are in accord on any 1 point, it is this: the surest way to arouse \& hold the readers attention is by being specific, definite, \& concrete. The greatest writers -- Homer, Dante, Shakespeare -- are effective largely because they deal in particulars \& report the details that matter. Their words call up pictures.

Jean Stafford, to cite a more modern author, demonstrates in her short story ``In the Zoo'' how prose is made vivid by the use of words that evoke images \& sensations:
\begin{example}
	$\ldots$ Daisy \& I in time found asylum in a small menagerie down by the railroad tracks. It belonged to a gentle alcoholic ne'er-do- well, who did nothing all day long but drink bathtub gin in rickeys \&  play solitaire \&  smile to himself \&  talk to his animals. He had a little, stunted red vixen \&  a deodorized skunk, a parrot from Tahiti that spoke Parisian French, a woebegone coyote, \&  two capuchin monkeys, so serious \&  humanized, so small \&  sad \&  sweet, \&  so religious-looking with their tonsured heads that it was impossible not to think their gibberish was really an ordered language with a grammar that someday some philologist would understand.
	
	Gran knew about our visits to Mr. Murphy \&  she did not object, for it gave her keen pleasure to excoriate him when we came home. His vice was not a matter of guesswork; it was an established fact that he was half-seas over from dawn till midnight. ``With the black Irish,'' said Gran, ``the taste for drink is taken in with the mother's milk \&  is never mastered. Oh, I know all about those promises to join the temperance movement \&  not to touch another drop. The way to Hell is paved with good intentions.'' -- Excerpt from ``In the Zoo'' from Bad Characters by Jean Stafford.
\end{example}
If the experiences of Walter Mitty, of Molly Bloom, of Rabbit Angstrom have seemed for the moment real to countless readers, if in reading Faulkner we have almost the sense of inhabiting Yoknapatawpha County during the decline of the South, it is because the details used are definite, the terms concrete. It is not that every detail is given -- that would be impossible, as well as to no purpose -- but that all the significant details are given, \&  with such accuracy \&  vigor that readers, in imagination, can project themselves into the scene.

In exposition \&  in argument, the writer must likewise never lose hold of the concrete; \&  even when dealing with general principles, the writer must furnish particular instances of their application.

In his {\it Philosophy of Style}, Herbert Spencer gives 2 sentences to illustrate how the vague \&  general can be turned into the vivid \&  particular:
\begin{example}
	In proportion as the manners, customs, \&  amusements of a nation are cruel \&  barbarous, the regulations of their penal code will be severe. $\to$ In proportion as men delight in battles, bullfights, \&  combats of gladiators, will they punish by hanging, burning, \&  the rack.
\end{example}
To show what happens when strong writing is deprived of its vigor, George Orwell once took a passage from the Bible \&  drained it of its blood. On the left, below, is Orwell’s translation; on the right, the verse from Ecclesiastes (King James Version).
\begin{example}
	Objective consideration of contemporary phenomena compels the conclusion that success or failure in competitive activities exhibits no tendency to be commensurate with innate capacity, but that a considerable element of the unpredictable must inevitably be taken into account. $\to$ I returned, \&  saw under the sun, that the race is not to the swift, nor the battle to the strong, neither yet bread to the wise, nor yet riches to men of understanding, nor yet favor to men of skill; but time \&  chance happeneth to them all.'' -- \cite[pp. 35--36]{Strunk_White_element_style}
\end{example}

%------------------------------------------------------------------------------%

\subsubsection{Omit needless words.}
``Vigorous writing is concise. A sentence should contain no unnecessary words, a paragraph no unnecessary sentences, for the same reason that a drawing should have no unnecessary lines \&  a machine no unnecessary parts. This requires not that the writer make all sentences short, or avoid all detail \&  treat subjects only in outline, but that every word tell.

Many expressions in common use violate this principle.
\begin{example}
	the question as to whether $\to$ whether (the question whether), there is no doubt but that $\to$ no doubt (doubtless), used for fuel purposes $\to$ used for fuel, he is a man who $\to$ he, in a hasty manner $\to$ hastily, this is a subject that $\to$ this subject, Her story is a strange one. $\to$ Her story is strange. the reason why is that $\to$ because.
\end{example}
{\it The fact that} is an especially debilitating expression. It should be revised out of every sentence in which it occurs.
\begin{example}
	owing to the fact that $\to$ since (because), in spite of the fact that $\to$ though (although), call your attention to the fact that $\to$ remind you (notify you), I was unaware of the fact that $\to$ I was unaware that (did not know), the fact that he had not succeeded $\to$ his failure, the fact that I had arrived $\to$ my arrival.
\end{example}
See also the words {\it case, character, nature} in Chap. IV. {\it Who is, which was}, \&  the like are often superfluous.
\begin{example}
	His cousin, who is a member of the same firm $\to$ His cousin, a member of the same firm
	
	Trafalgar, which was Nelson's last battle $\to$ Trafalgar, Nelson’s last battle.
\end{example}
As the active voice is more concise than the passive, \&  a positive statement more concise than a negative one, many of the examples given under Rules 14 \&  15 illustrate this rule as well.

A common way to fall into wordiness is to present a single complex idea, step by step, in a series of sentences that might to advantage be combined into one.
\begin{example}
	Macbeth was very ambitious. This led him to wish to become king of Scotland. The witches told him that this wish of his would come true. The king of Scotland at this time was Duncan. Encouraged by his wife, Macbeth murdered Duncan. He was thus enabled to succeed Duncan as king. (51 words)
	
	$to$ Encouraged by his wife, Macbeth achieved his ambition \&  realized the prediction of the witches by murdering Duncan \&  becoming king of Scotland in his place. (26 words)'' -- \cite[pp. 37--38]{Strunk_White_element_style}
\end{example}

%------------------------------------------------------------------------------%

\subsubsection{Avoid a succession of loose sentences.}
``This rule refers especially to loose sentences of a particular type: those consisting of 2 clauses, the 2nd introduced by a conjunction or relative. A writer may err by making sentences too compact \& periodic. An occasional loose sentence prevents the style from becoming too formal \& gives the reader a certain relief. Consequently, loose sentences are common in easy, unstudied writing. The danger is that there may be too many of them.

An unskilled writer will sometimes construct a whole paragraph of sentences of this kind, using as connectives {\it and, but}, and, less frequently, {\it who, which, when, where}, \& {\it while}, these last in nonrestrictive senses. (See Rule 3.)
\begin{example}
	The 3rd concert of the subscription series was given last evening, \& a large audience was in attendance. Mr. Edward Appleton was the soloist, \& the Boston Symphony Orchestra furnished the instrumental music. The former showed himself to be an artist of the 1st rank, while the latter proved itself fully deserving of its high reputation. The interest aroused by the series has been very gratifying to the Committee, \& it is planned to give a similar series annually hereafter. The 4th concert will be given on Tuesday, May 10, when an equally attractive program will be presented.
\end{example}
Apart from its triteness \& emptiness, the paragraph above is bad because of the structure of its sentences, with their mechanical symmetry \& singsong. Compare these sentences from the chapter ``What I Believe'' in E. M. Forster's {\it 2 Cheers for Democracy}:
\begin{example}
	I believe in aristocracy, though -- if that is the right word, \& if a democrat may use it. Not an aristocracy of power, based upon rank \& influence, but an aristocracy of the sensitive, the considerate \& the plucky. Its members are to be found in all nations \& classes, \& all through the ages, \& there is a secret understanding between them when they meet. They represent the true human tradition, the 1 permanent victory of our queer race over cruelty \& chaos. Thousands of them perish in obscurity, a few are great names. They are sensitive for others as well as for themselves, they are considerate without being fussy, their pluck is not swankiness but the power to endure, \& they can take a joke.
\end{example}
A writer who has written a series of loose sentences should recast enough of them to remove the monotony, replacing them with simple sentences, sentences of 2 clauses joined by a semicolon, periodic sentences of 2 clauses, or sentences (loose or periodic) of 3 clauses -- whichever best represent the real relations of the thought.'' -- \cite[pp. 39--40]{Strunk_White_element_style}

%------------------------------------------------------------------------------%

\subsubsection{Express coordinate ideas in similar form.}
``This principle, that of parallel construction, requires that expressions similar in content \& function be outwardly similar. The likeness of form enables the reader to recognize more readily the likeness of content \& function. The familiar Beatitudes exemplify the virtue of parallel construction.
\begin{quotation}
	Blessed are the poor in spirit: for theirs is the kingdom of heaven.
	
	Blessed are they that mourn: for they shall be comforted.
	
	Blessed are the meek: for they shall inherit the earth.
	
	Blessed are they which do hunger \& thirst after righteousness: for they shall be filled.
\end{quotation}
The unskilled writer often violates this principle, mistakenly believing in the value of constantly varying the form of expression. When repeating a statement to emphasize it, the writer may need to vary its form. Otherwise, the writer should follow the principle of parallel construction.
\begin{example}
	Formerly, science was taught by the textbook method, while now the laboratory method is employed. $\to$ Formerly, science was taught by the textbook method; now it is taught by the laboratory method.
\end{example}
The lefthand version gives the impression that the writer is undecided or timid, apparently unable or afraid to choose one form of expression \& hold to it. The righthand version shows that the writer has at least made a choice \& abided by it.

By this principle, an article or a preposition applying to all the members of a series must either be used only before the first term or else be repeated before each term.
\begin{example}
	The French, the Italians, Spanish, \& Portuguese $\to$ The French, the Italians, the Spanish, \& the Portuguese
	
	In spring, summer, or in winter $\to$ In spring, summer, or winter (In spring, in summer, or in winter).
\end{example}
Some words require a particular preposition in certain idiomatic uses. When such words are joined in a compound construction, all the appropriate prepositions must be included, unless they are the same.
\begin{example}
	His speech was marked by disagreement \& scorn for his opponent's position. $\to$ His speech was marked by disagreement with \& scorn for his opponent's position.
\end{example}
Correlative expressions ({\it both, \&; not, but; not only, but also; either, or; 1st, 2nd, 3rd}; \& the like) should be followed by the same grammatical construction. Many violations of this rule can be corrected by rearranging the sentence.
\begin{example}
	It was both a long ceremony \& very tedious. $\to$ The ceremony was both long \& tedious.
	
	A time for not words but action. $\to$ A time not for words but for action.
	
	Either you must grant his request or incur his ill will. $\to$ You must either grant his request or incur his ill will.
	
	My objections are, 1st, the injustice of the measure; 2nd, that it is unconstitutional. $\to$ My objections are, 1st, that the measure is unjust; 2nd, that it is unconstitutional.
\end{example}
It may be asked, what if you need to express a rather large number of similar ideas -- say, 20? Must you write 20 consecutive sentences of the same pattern? On closer examination, you will probably find that the difficulty is imaginary -- that these 20 ideas can be classified in groups, \& that you need apply the principle only within each group. Otherwise, it is best to avoid the difficulty by putting statements in the form of a table.'' -- \cite[pp. 41--42]{Strunk_White_element_style}

%------------------------------------------------------------------------------%

\subsubsection{Keep related words together.}
The position of the words in a sentence is the principal means of showing their relationship.

Confusion \& ambiguity result when words are badly placed.

The writer must, therefore, bring together the words \& groups of words that are related in thought \& keep apart those that are not so related.
\begin{example}
	He noticed a large stain in the rug that was right in the center.
	
	$\to$ He noticed a large stain right in the center of the rug.
	
	You can call your mother in London \& tell her all about George's taking you out to dinner for just 2 dollars.
	
	$\to$ For just 2 dollars you can call your mother in London \& tell her all about George's taking you out to dinner.
	
	New York's 1st commercial human-sperm bank opened Friday with semen samples from 18 men frozen in a stainless steel tank.
	
	$\to$ New York's 1st commercial human-sperm bank opened Friday when semen samples were taken from 18 men.
	
	The samples were then frozen \& stored in a stainless steel tank.
\end{example}
In the lefthand version of the 1st example, the reader has no way of knowing whether the stain was in the center of the rug or the rug was in the center of the room.

In the lefthand version of the 2nd example, the reader may well wonder which cost 2 dollars - the phone call or the dinner.

In the lefthand version of the 3rd example, the reader's heart goes out to those 18 poor fellows frozen in a steel tank.

%
The subject of a sentence \& the principal verb should not, as a rule, be separated by a phrase or clause that can be transferred to the beginning.
\begin{example}
	Toni Morrison, in \emph{Beloved}, writes about characters who have escaped from slavery but are haunted by its heritage.
	
	$\to$ In \emph{Beloved}, Toni Morrison writes about characters who have escaped from slavery but are haunted by its heritage.
	
	A dog, if you fail to discipline him, becomes a household pest.
	
	$\to$ Unless disciplined, a dog becomes a household pest.
\end{example}
Interposing a phrase or a clause, as in the lefthand examples above, interrupts the flow of the main clause.

This interruption, however, is not usually bothersome when the flow is checked only by a relative clause or by an expression in apposition.

Sometimes, in periodic sentences, the interruption is a deliberate device for creating suspense. (See example under Rule 22.)

%
The relative pronoun should come, it most instances, immediately after its antecedent.
\begin{example}
	There was a stir in the audience that suggested disapproval.
	
	$\to$ A stir that suggested disapproval swept the audience.
	
	He wrote 3 articles about his adventures in Spain, which were published in \emph{Harper's Magazine}.
	
	$\to$ He published 3 articles in \emph{Harper's Magazine} about his adventures in Spain.
	
	This is a portrait of Benjamin Harrison, who became President in 1889.
	
	He was the grandson of William Henry Harrison.
	
	$\to$ This is a portrait of Benjamin Harrison, grandson of William Henry Harrison, who became President in 1889.
\end{example}
If the antecedent consists of a group of words, the relative comes at the end of the group, unless this would cause ambiguity.
\begin{example}
	The Superintendent of the Chicago Division, who
\end{example}
No ambiguity results from the above.

But
\begin{example}
	A proposal to amend the Sherman Act, which has been variously judged
\end{example}
leaves the reader wondering whether it is the proposal or the Act that has been variously judged.

The relative clause must be moved forward, to read, ``A proposal, which has been variously judged, to amend the Sherman Act$\ldots$''.

Similarly
\begin{example}
	The grandson of William Henry Harrison, who $\to$ William Henry Harrison's grandson, Benjamin Harrison, who
\end{example}
A noun in apposition may come between antecedent \& relative, because in such a combination no real ambiguity can arise.
\begin{example}
	The Duke of York, his brother, who was regarded with hostility by the Whigs
\end{example}
Modifiers should come, if possible, next to the words they modify.

If several expressions modify the same word, they should be ranged so that no wrong relation is suggested.
\begin{example}
	All the members were not present.
	
	$\to$ Not all the members were present.
	
	She only found 2 mistakes.
	
	$\to$ She found only 2 mistakes.
	
	The director said he hoped all members would give generously to the Fund at a meeting of the committee yesterday.
	
	$\to$ At a meeting of the committee yesterday, the director said he hoped all members would give generously to the Fund.
	
	Major R. E. Joyce will give a lecture on Tuesday evening in Bailey Hall, to which the public is invited on ``My Experiences in Mesopotamia'' at 8:00 P.M.
	
	$\to$ On Tuesday evening at 8, Major R. E. Joyce will give a lecture in Bailey Hall on ``My Experiences in Mesopotamia.''
	
	The public is invited.
\end{example}
Note, in the last lefthand example, how swiftly meaning departs when words are wrongly juxtaposed.

%------------------------------------------------------------------------------%

\subsubsection{In summaries, keep to 1 tense.}
In summarizing the action of a drama, use the present tense.

In summarizing a poem, story, or novel, also use the present, though you may use the past if it seems more natural to do so.

If the summary is in the present tense, antecedent action should be expressed by the perfect; if in the past, by the past perfect.
\begin{example}
	Chance prevents Friar John from delivering Friar Lawrence's letter to Romeo.
	
	Meanwhile, owing to her father's arbitrary change of the day set for her wedding, Juliet has been compelled to drink the potion on Tuesday night, with the result that Balthasar informs Romeo of her supposed death before Friar Lawrence learns of the nondelivery of the letter.
\end{example}
But whichever tense is used in the summary, a past tense in indirect discourse or in indirect question remains unchanged.
\begin{example}
	The Friar confessed that it was he who married them.
\end{example}
Apart from the exceptions noted, the writer should use the same tense throughout.

Shifting from 1 tense to another gives the appearance of uncertainty \& irresolution.

%
In presenting the statements or the thought of someone else, as in summarizing an essay or reporting a speech, do not overwork such expressions as ``he said,'' ``she stated,'' ``the speaker added,'' ``the speaker then went on to say,'' ``the author also thinks.''

Indicate clearly at the outset, once for all, that what follows is summary, \& then waste no words in repeating the notification.

%
In notebooks, in newspapers, in handbooks of literature, summaries of 1 kind or another be indispensable, \& for children in primary schools retelling a story in their own words is a useful exercise.

But in the criticism or interpretation of literature, be careful to avoid dropping into summary.

It may be necessary to devote 1 or 2 sentences to indicating the subject, or the opening situation, of the work being discussed, or to cite numerous details to illustrate its qualities.

But you should aim at writing an orderly discussion supported by evidence, not a summary with occasional comment.

Similarly, if the scope of the discussion includes a number of works, as a rule it is better not to take them up singly in chronological order but to aim from the beginning at establishing general conclusions.

%------------------------------------------------------------------------------%

\subsubsection{Place the emphatic words of a sentence at the end.}
The proper place in the sentence for the word or group of words that the writer desires to make most prominent is usually the end.
\begin{example}
	Humanity has hardly advanced in fortitude since that time, though it has advanced in many other ways.
	
	$\to$ Since that time, humanity has advanced in many ways, but it has hardly advanced in fortitude.
	
	This steel is principally used for making razors, because of its hardness.
	
	$\to$ Because of its hardness, this steel is used principally or making razors.
\end{example}
The word or group of words entitled to this position of prominence is usually the logical predicate - i.e., the {\it new} element in the sentence, as it is in the 2nd example.

The effectiveness of the periodic sentence arises from the prominence it gives to the main statement.
\begin{example}
	4 centuries ago, Christopher Columbus, 1 of the Italian mariners whom the decline of their own republics had put at the service of the world \& of adventure, seeking for Spain a westward passage to the Indies to offset the achievement of Portuguese discoverers, lighted on America.
	
	With these hopes \& in this belief I would urge you, laying aside all hindrance, thrusting away all private aims, to devote yourself unswervingly to the vigorous \& successful prosecution of this war.
\end{example}
The other prominent position in the sentence is the beginning.

Any element in the sentence other than the subject becomes emphatic when placed 1st.
\begin{example}
	Deceit or treachery she could never forgive.
	
	Vast \& rude, fretted by the action of nearly 3000 years, the fragments of this architecture may often seem, at 1st sight, like works of nature.
	
	Home is the sailor.
\end{example}
A subject coming 1st in its sentence may be emphatic, but hardly by its position alone.

In the sentence
\begin{example}
	Great kings worshiped at his shrine
\end{example}
the emphasis upon {\it kings} arises largely from its meaning \& from the context.

To receive special emphasis, the subject of a sentence must take the position of the predicate.
\begin{example}
	Through the middle of the valley flowed a winding stream.
\end{example}
The principle that the proper place for what is to be made most prominent is the end applies equally to the words of a sentence, to the sentences of a paragraph, \& to the paragraphs of a composition.

%------------------------------------------------------------------------------%

\subsection{A Few Matters of Form}

\begin{enumerate}
	\item {\bf Colloquialisms.} If you use a colloquialism or a slang word or phrase, simply use it; do not draw attention to it by enclosing it in quotation marks.
	
	To do so is to put on airs, as through you were inviting the reader to join you in a select society of those who know better.
	\item {\bf Exclamations.} Do not attempt to emphasize simple statements by using a mark of exclamation.
	\begin{example}
		It was wonderful show!
		
		$\to$ It was a wonderful show.
	\end{example}
	The exclamation mark is to be reserved for use after true exclamations or commands.
	\begin{example}
		What a wonderful show!
		
		Halt!
	\end{example}
	\item {\bf Headings.} If a manuscript is to be submitted for publication, leave plenty of space at the top of page 1.
	
	The editor will need this space to write directions to the compositor.
	
	Place the heading, or title, at least a 4th of the way down the page.
	
	Leave a blank line, or its equivalent in space, after the heading.
	
	On succeeding pages, begin near the top, but not so near as to give a crowded appearance.
	
	Omit the period after a title or heading.
	
	A question mark or an exclamation point may be used if the heading calls for it.
	\item {\bf Hyphen.} When 2 or more words are combined to form a compound adjective, a hyphen is usually required.
	\begin{example}
		``He belonged to the leisure class \& enjoyed leisure-class pursuits.''
		
		``She entered her boat in the round-the-island race.''
	\end{example}
	Do not use a hyphen between words that can better be written as 1 word: {\it water-fowl, waterfowl}.
	
	Common sense will aid you in the decision, but a dictionary is more reliable.
	
	The steady evolution of the language seems to favor union: 2 words eventually become one, usually after a period of hyphenation.
	\begin{example}
		bed chamber $\to$ bed-chamber $\to$ bedchamber
		
		wild life $\to$ wild-life $\to$ wildlife
		
		bell boy $\to$ bell-boy $\to$ bellboy
	\end{example}
	The hyphen can play tricks on the unwary, as it did it Chattanooga when 2 newspapers merged - the {\it News} \& the {\it Free Press}.
	
	Someone introduced a hyphen into the merger, \& the paper became {\it The Chattanooga News-Free Press}, which sounds as though the paper were news-free, or devoid of news.
	
	Obviously, we ask too much of a hyphen when we ask it to cast its spell over words it does not adjoint.
	\item {\bf Margins.} Keep righthand \& lefthand margins roughly the same width.
	
	\begin{remark}[Exception]
		If a great deal of annotating or editing is anticipated, the lefthand margin should be roomy enough to accommodate this work.
	\end{remark}
	\item {\bf Numerals.} Do not spell out dates or other serial numbers.
	
	Write them in figures or in Roman notation, as appropriate.
	\begin{example}
		August 9, 1988
		
		Part XII
		
		Rule 3
		
		352d Infantry
	\end{example}
	
	\begin{remark}[Exception]
		When they occur in dialogue, most dates \& numbers are best spelled out.
		\begin{example}
			``I arrived home on August ninth.''
			
			``In the year 1990, I turned twenty-one.''
			
			``Read Chapter Twelve.''
		\end{example}
	\end{remark}
	\item {\bf Parentheses.} A sentence containing an expression in parentheses is punctuated outside the last mark of parenthesis exactly as if the parenthetical expression were absent.
	
	The expression within the marks is punctuated as if it stood by itself, except that the final stop is omitted unless it is question mark or an exclamation point.
	\begin{example}
		I went to her house yesterday (my 3rd attempt to see her), but she had left town.
		
		He declares (and why should we doubt his good faith?) that he is now certain of success.
	\end{example}
	(When a wholly detached expression or sentence is parenthesized, the final stop comes before the last mark of parenthesis.)
	\item {\bf Quotations.} Formal quotations cited as documentary evidence are introduced by a colon \& enclosed in quotation marks.
	\begin{example}
		The United States Coast Pilot has this to say of the place: ``Bracy Cove, 0.5 mile eastward of Bear Island, is exposed to southeast winds, has a rocky \& uneven bottom, \& is unfit for anchorage.''
	\end{example}
	A quotation grammatically in apposition or the direct object of a verb is preceded by a comma \& enclosed in quotation marks.
	\begin{example}
		I am reminded of the advice of my neighbor, ``Never worry about your heart till it stops beating.''
		
		Mark Twain says, ``{\bf A classic is something that everybody wants to have read \& nobody wants to read.}''
	\end{example}
	When a quotation is followed by an attributive phrase, the comma is enclosed within the quotation marks.
	\begin{example}
		``I can't attend,'' she said.
	\end{example}
	Typographical usage dictates that the comma be inside the marks, though logically it often seems not to belong there.
	\begin{example}
		``The Fish,'' ``Poetry,'' \& ``The Monkeys'' are in Marianne Moore's Selected Poems.
	\end{example}
	When quotations of an entire line, or more, of either verse or prose are to be distinguished typographically from text matter, as are the quotations in this book, begin on a fresh line \& indent.
	
	Quotation marks should not be used unless they appear in the original, as in dialogue.
	\begin{example}
		Wordworth's enthusiasm for the French Revolution was at 1st unbounded:
		
		Bliss was it in that dawn to be alive,
		
		But to be young was very heaven!
	\end{example}
	Quotations introduced by {\it that} are indirect discourse \& not enclosed in quotation marks.
	\begin{example}
		Keats declares that beauty is truth, truth beauty.
		
		Dickinson states that a coffin is a small domain.
	\end{example}
	Proverbial expressions \& familiar phrases of literary origin require no quotation marks.
	\begin{example}
		These are the times that try men's souls.
		
		He lives far from the madding crowd.
	\end{example}
	\item {\bf References.} In scholarly work requiring exact references, abbreviate titles that occur frequently, giving the full forms in an alphabetical list at the end.
	
	As a general practice, give the references in parentheses or in footnotes, not in the body of the sentence.
	
	Omit the words {\it act, scene, line, book, volume, page}, except when referring to only 1 of them.
	
	Punctuate as indicated below.
	\begin{example}
		in the 2nd scene of the 3rd act $\to$ in III.ii (Better still, simply insert m.ii in parenthesis at the proper place in the sentence.)
		
		After the killing of Polonius, Hamlet is placed under guard (IV.ii.14).
		
		2 Samuel i: 17--27
		
		Othello II.iii. 264--267, III.iii. 155--161
	\end{example}
	\item {\bf Syllabication.} When a word must be divided at the end of a line, consult a dictionary to learn the syllables between which division should be made.
	
	The student will do well to examine the syllable division in a number of pages of any carefully printed book.
	\item {\bf Titles.} For the titles of literary works, scholarly usage prefers italics with capitalized initials.
	
	The usage of editors \& publishers varies, some using italics with capitalized initials, others using Roman with capitalized initials \& with or without quotation marks.
	
	Use italics (indicated in manuscript by underscoring) except in writing for a periodical that follows a different practice.
	
	Omit initial {\it A} or {\it The} from titles when you place the possessive before them.
	\begin{example}
		\emph{A} Tale of 2 Cities; \emph{Dickens's} Tale of 2 Cities.
		
		The Age of Innocence; \emph{Wharton's} Age of Innocence.
	\end{example}
\end{enumerate}

%------------------------------------------------------------------------------%

\subsection{Words \& Expressions Commonly Misused}
Many of the words \& expressions listed here are not so much bad English as bad style, the commonplaces of careless writing.

As illustrated under {\it Feature}, the proper correction is likely to be not the replacement of 1 word or set of words by another but the {\it replacement of vague generality by definite statement}.

%
The shape of our language is not rigid; in questions of usage we have no lawgiver whose word is final.

Students whose curiosity is aroused by the interpretations that follow, or whose doubts are raised, will wish to pursue their investigations further.

Books useful in such pursuits are {\it Merriam Webster's Collegiate Dictionary}, 10th Edition; {\it The American Heritage Dictionary of the English Language}, 3rd Edition; {\it Webster's 3rd New International Dictionary; The New Fowler's Modern English Usage}, 3rd Edition, edited by R. W. Burchfield; {\it Modern American Usage: A Guide} by Wilson Follett \& Erik Wensberg; \& {\it The Careful Writer} by Theodore M. Bernstein.
\begin{enumerate}
	\item {\bf Aggravate. Irritate.} The 1st means ``to add to'' an already troublesome or vexing matter or condition.
	
	The 2nd means ``to vex'' or ``to annoy'' or ``to chafe.''
	\item {\bf All right.} Idiomatic in familiar speech as a detached phrase in the sense ``Agreed,'' or ``Go ahead,'' or ``O.K.''
	
	Properly written as 2 words - {\it all right}.
	\item {\bf Allude.} Do not confuse with {\it elude}.
	
	You {\it allude} to a book; you {\it elude} a pursuer.
	
	Note, too, that {\it allude} is not synonymous with {\it refer}.
	
	An allusion is an indirect mention, a reference is a specific one.
	\item {\bf Allusion.} Easily confused with {\it illusion}.
	
	The 1st means ``an indirect reference''; the 2nd means ``an unreal image'' or ``a false impression.''
	\item {\bf Alternate. Alternative.} The words are not always interchangeable as nouns or adjectives.
	
	The 1st means every other one in a series; the 2nd, 1 of 2 possibilities.
	
	As the other one of a series of 2, an {\it alternate} may stand for ``a substitute,'' but an {\it alternative}, although used in a similar sense, connotes a matter of choice that is never present with {\it alternate}.
	\begin{example}
		As the flooded road left them no alternative, they took the alternate route.
	\end{example}
	\item {\bf Among. Between.} When more than 2 things or persons are involved, {\it among} is usually called for: ``The money was divided among the 4 players.''
	
	When, however, more than 2 are involved but each is considered  individually, {\it between} is preferred: ``an agreement between the 6 heirs.''
	\item {\bf And/or.} A device, or shortcut, that damages a sentence \& often leads to confusion or ambiguity.
	\begin{example}
		1st of all, would an honor system successfully cut down on the amount of stealing and/or cheating?
		
		$\to$ 1st of all, would an honor system reduce the incidence of stealing or cheating or both?
	\end{example}
	\item {\bf Anticipate.} Use {\it expect} in the sense of simple expectation.
	\begin{example}
		I anticipated that he would look older.
		
		$\to$ I expected that he would look older.
		
		My brother anticipated the upturn in the market.
		
		$\to$ My brother expected the upturn in the market.
	\end{example}
	In the 2nd example, the word {\it anticipated} is ambiguous.
	
	It could mean simply that the brother believed the upturn would occur, or it could mean that he acted in advance of the expected upturn - by buying stock, perhaps.
	\item {\bf Anybody.} In the sense of ``any person,'' not to be written as 2 words.
	
	{\it Anybody.} In the sense of ``any person,'' not to be written as 2 words.
	
	{\it Any body} means ``any corpse,'' or ``any human form,'' or ``any group.''
	
	The rule holds equally for {\it everybody, nobody}, \& {\it somebody}.
	\item {\bf Anyone.} In the sense of ``anybody,'' written as 1 word.
	
	{\it Any one} means ``any single person'' or ``any single thing.''
	\item {\bf As good or better than.} Expressions of this type should be corrected by rearranging the sentences.
	\begin{example}
		My opinion is as good or better than his.
		
		$\to$ My opinion is as good as his, or better (if not better).
	\end{example}
	\item {\bf As to whether.} {\it Whether} is sufficient.
	\item {\bf As yet.} {\it Yet} nearly always is as good, if not better.
	\begin{example}
		No agreement has been reached as yet.
		
		$\to$ No agreement has yet been reached.
	\end{example}
	The chief exception is at the beginning of a sentence, where {\it yet} means something different.
	\begin{example}
		Yet (\emph{or} despite everything) he has not succeeded.
		
		As yet (\emph{or} so far) he has not succeeded.
	\end{example}
	\item {\bf Being.} Not appropriate after {\it regard$\ldots$} as.
	\begin{example}
		He is regarded as being the best dancer in the club.
		
		$\to$ He is regarded as the best dancer in the club.
	\end{example}
	\item {\bf But.} Unnecessary after {\it doubt} \& {\it help}.
	\begin{example}
		I have no doubt but that $\to$ I have no doubt that
		
		He could not help but see that $\to$ He could not help seeing that
	\end{example}
	The too-frequent use of {\it but} as a conjunction leads to the fault discussed under Rule 18.
	
	A loose sentence formed with {\it but} can usually be converted into a periodic sentence formed with {\it although}.
	
	Particularly awkward is one {\it but} closely following another, thus making a contrast to a contrast, or a reservation to a reservation.
	
	This is easily corrected by rearrangement.
	\begin{example}
		Our country had vast resources but seemed almost wholly unprepared for war.
		
		But within a year it had created an army of 4 million.
		
		$\to$ Our country seemed almost wholly unprepared for war, but it had vast resources.
		
		Within a year it had created an army of 4 million.
	\end{example}
	\item {\bf Can.} Means ``am (is, are) able.''
	
	Not to be used as a substitute for {\it may}.
	\item {\bf Care less.} The dismissive ``I couldn't care less'' is often used with the shortened ``not'' mistakenly (and mysteriously) omitted: ``I could care less.''
	
	The error destroys the meaning of the sentence \& is careless indeed.
	\item {\bf Case.} Often unnecessary.
	\begin{example}
		In many cases, the rooms lacked air conditioning.
		
		$\to$ Many of the rooms lacked air conditioning.
		
		It has rarely been the case that any mistake has been made.
		
		$\to$ Few mistakes have been made.
	\end{example}
	\item {\bf Certainly.} Used indiscriminately by some speakers, much as others use {\it very}, in an attempt to intensify any \& every statement.
	
	A mannerism of this kind, bad in speech, is even worse in writing.
	\item {\bf Character.} Often simply redundant, used from a mere habit of wordiness.
	\begin{example}
		acts of a hostile character $\to$ hostile acts
	\end{example}
	\item {\bf Claim.} [v] With object-noun, means ``lay claim to.''
	
	May be used with a dependent clause if this sense is clearly intended: ``She claimed that she was the sole heir.''
	
	(But even here {\it claimed to be} would be better.)
	
	Not to be used as a substitute for {\it declare, maintain}, or {\it charge}.
	\begin{example}
		He claimed he knew how.
		
		$\to$ He declared he knew how.
	\end{example}
	\item {\bf Clever.} Note that the word means 1 thing when applied to people, another when applied to horses.
	
	A clever horse is a good-natured one, not an ingenious one.
	\item {\bf Compare.} To {\it compare to} is to point out or imply resemblances between objects regarded as essentially of a different order; to {\it compare with} is mainly to point out differences between objects regarded as essentially of the same order.
	
	Thus, life has been {\it compared to} a pilgrimage, {\it to} a drama, {\it to} a battle; Congress may be {\it compared with} the British Parliament.
	
	Paris has been {\it compared to} ancient Athens; it may be {\it compared with} modern London.
	\item {\bf Comprise.} Literally, ``embrace'': A zoo comprises mammals, reptiles, \& birds (because it ``embraces,'' or ``includes,'' them).
	
	But animals do not comprise (``embrace'') a zoo - they constitute a zoo.
	\item {\bf Consider.} Not followed by {\it as} when it means ``believe to be.''
	\begin{example}
		I consider him as competent.
		
		$\to$ I consider him competent.
	\end{example}
	When {\it considered} means ``examined'' or ``discussed,'' it is followed by {\it as}:
	\begin{example}
		The lecturer considered Eisenhower 1st as soldier \& 2nd as administrator.
	\end{example}
	\item {\bf Contact.} As a transitive verb, the word is vague \& self-important.
	
	Do not {\it contact} people; get in touch with them, look them up, phone them, find them, or meet them.
	\item {\bf Cope.} An intransitive verb used with {\it with}.
	
	In formal writing, one doesn't ``cope,'' one ``copes with'' something or somebody.
	\begin{example}
		I knew they'd cope. (jocular)
		
		$\to$ I knew they would cope with the situation.
	\end{example}
	\item {\bf Currently.} In the sense of {\it now} with a verb in the present tense, {\it currently} is usually redundant; emphasis is better achieved through a more precise reference to time.
	\begin{example}
		We are currently reviewing your application.
		
		$\to$ We are at this moment reviewing your application.
	\end{example}
	\item {\bf Data.} Like {\it strata, phenomena}, \& {\it media}, {\it data} is a plural \& is best used with a plural verb.
	
	The word, however, is slowly gaining acceptance as a singular.
	\begin{example}
		The data is misleading.
		
		$\to$ These data are misleading.
	\end{example}
	\item {\bf Different than.} Here logic supports established usage: one thing differs {\it from} another, hence, {\it different from}.
	
	Or, {\it other than, unlike}.
	\item {\bf Disinterested.} Means ``impartial.''
	
	Do not confuse it {\it with uninterested}, which means ``not interested in.''
	\begin{example}
		Let a disinterested person judge our dispute, (an impartial person)
		
		This man is obviously uninterested in our dispute, (couldn't care less)
	\end{example}
	\item {\bf Divided into.} Not to be misused for {\it composed of}.
	
	The line is sometimes difficult to draw; doubtless plays are divided into acts, but poems are composed of stanzas.
	
	An apple, halved, is divided into sections, but an apple is composed of seeds, flesh, \& skin.
	\item {\bf Due to.} Loosely used for {\it through, because of}, or {\it owning to}, in adverbial phrases.
	\begin{example}
		He lost the 1st game due to carelessness.
		
		$\to$ He lost the 1st game because of carelessness.
	\end{example}
	In correct use, synonymous with {\it attributable to}: ``The accident was due to bad weather''; ``losses due to preventable fires.''
	\item {\bf Each \& every one.} Pitchman's jargon.
	
	Avoid, except in dialogue.
	\begin{example}
		It should be a lesson to each \& every one of us.
		
		$\to$ It should be a lesson to every one of us (to us all).
	\end{example}
	\item {\bf Effect.} As a noun, means ``result''; as a verb, means ``to bring about,'' ``to accomplish'' (not to be confused with {\it affect}, which means ``to influence'').
	
	As a noun, often loosely used in perfunctory writing about fashions, music, painting, \& other arts: ``a Southwestern effect''; ``effects in pale green''; ``very delicate effects''; ``subtle effects''; ``a charming effect was produced.''
	
	The writer who has a definite meaning to express will not take refuge in such vagueness.
	\item {\bf Enormity.} Use only in the sense of ``monstrous wickedness.''
	
	Misleading, if not wrong, when used to express bigness.
	\item {\bf Enthuse.} An annoying verb growing out of the noun {\it enthusiasm}.
	
	Not recommended.
	\begin{example}
		She was enthused about her new car.
		
		$\to$ She was enthusiastic about her new car.
		
		She enthused about her new car.
		
		$\to$ She talked enthusiastically (expressed enthusiasm) about her new car.
	\end{example}
	\item {\bf Etc.} Literally, ``and other things''; sometimes loosely used to mean ``and other persons.''
	
	The phrase is equivalent to {\it \& the rest, \& so forth}, \& hence is not to be used if 1 of these would be insufficient - i.e., if the reader would be left in doubt as to any important particulars.
	
	Least open to objection when it represents the last terms of a list already given almost in full, or immaterial words at the end of a quotation.
	
	%
	At the end of a list introduced by {\it such as, for example}, or any similar expression, {\it etc.} is incorrect.
	
	In formal writing, {\it etc.} is a misfit.
	
	An item important enough to call for {\it etc.} is probably important enough to be named.
	\item {\bf Fact.} Use this word only of matters capable of direct verification, not of matters of judgment.
	
	That a particular event happened on a given date \& that lead melts at a certain temperature are facts.
	
	But such conclusions as that Napoleon was the greatest of modern generals or that the climate of California is delightful, however defensible they may be, are not properly called facts.
	\item {\bf Facility.} Why must jails, hospitals, \& schools suddenly become ``facilities''?
	\begin{example}
		Parents complained bitterly about the fire hazard in the wooden facility.
		
		$\to$ Parents complained bitterly about the fire hazard in the wooden schoolhouse.
		
		He has been appointed warden of the new facility.
		
		$\to$ He has been appointed warden of the new prison.
	\end{example}
	\item {\bf Factor.} A hackneyed word; the expressions of which it is a part can usually be replaced by something more direct \& idiomatic.
	\begin{example}
		Her superior training was the great factor in her winning the match.
		
		$\to$ She won the match by being better trained.
		
		Air power is becoming an increasingly important factor in deciding battles.
		
		$\to$ Air power is playing a larger \& larger part in deciding battles.
	\end{example}
	\item {\bf Farther. Further.} The 2 words are commonly interchanged, but there is a distinction worth observing: {\it farther} serves best as a distance word, {\it further} as a time or quantity word.
	
	You chase a ball {\it farther} than the other fellow; you pursue a subject {\it further}.
	\item {\bf Feature.} Another hackneyed word; like {\it factor}, it usually adds nothing to the sentence in which it occurs.
	\begin{example}
		A feature of the entertainment especially worthy of mention was the singing of Allison Jones.
		$\to$ (Better use the same number of words to tell what Allison Jones sang \& how she sang it.)
	\end{example}
	As a verb, in the sense of ``offer as a special attraction,'' it is to be avoided.
	\item {\bf Finalize.} A pompous, ambiguous verb.
	
	(See Chap. V, Reminder 21.)
	\item {\bf Fix.} Colloquial in America for {\it arrange, prepare, mend}.
	
	The usage is well established.
	
	But bear in mind that this verb is from {\it figere:} ``to make firm,'' ``to place definitely.''
	
	These are the preferred meanings of the word.
	\item {\bf Flammable.} An oddity, chiefly useful in saving lives.
	
	The common word meaning ``combustible'' is {\it inflammable}.
	
	But some people are thrown off by the {\it in-} \& think {\it inflammable} means ``not combustible.''
	
	For this reason, trucks carrying gasoline or explosives are now marked FLAMMABLE.
	
	Unless you are operating such a truck \& hence are concerned with the safety of children \& illiterates, use {\it inflammable}.
	\item {\bf Folk.} A collective noun, equivalent to {\it people}.
	
	Use the singular form only.
	
	{\it Folks}, in the sense of ``parents,'' ``family,'' ``those present,'' is colloquial \& foo folksy for formal writing.
	\begin{example}
		Her folks arrived by the afternoon train.
		
		$\to$ Her father \& mother arrived by the afternoon train.
	\end{example}
	\item {\bf Fortuitous.} Limited to what happens by chance.
	
	Not to be used for {\it fortunate} or {\it lucky}.
	\item {\bf Get.} The colloquial {\it have got} for {\it have} should not be used in writing.
	
	The preferable form of the participle is {\it got}, not {\it gotten}.
	\begin{example}
		He has not got any sense.
		
		$\to$ He has no sense.
		
		They returned without having gotten any.
		
		$\to$ They returned without having got any.
	\end{example}
	\item {\bf Gratuitous.} Means ``unearned,'' or ``unwarranted.''
	\begin{example}
		The insult seemed gratuitous, (undeserved)
	\end{example}
	\item {\bf He is a man who.} A common type of redundant expression; see Rule 17.
	\begin{example}
		He is a man who is very ambitious.
		
		$\to$ He is very ambitious.
		
		Vermont is a state that attracts visitors because of its winter sports.
		
		$\to$ Vermont attracts visitors because of its winter sports.
	\end{example}
	\item {\bf Hopefully.} This once-useful adverb meaning ``with hope'' has been distorted \& is now widely used to mean ``I hope'' or ``it is to be hoped.''
	
	Such use is not merely wrong, it is silly.
	
	To say, ``Hopefully I'll leave on the noon plane'' is to talk nonsense.
	
	Do you mean you'll leave on the noon plane in a hopeful frame of mind?
	
	Or do you mean you hope you'll leave on the noon plane?
	
	Whichever you mean, you haven't said it clearly.
	
	Although the word in its new, free-floating capacity may be pleasurable \& even useful to many, it offends the ear of many others, who do not like to see words dulled or eroded, particularly when the erosion leads to ambiguity, softness, or nonsense.
	\item {\bf However.} Avoid starting a sentence with {\it however} when the meaning is ``nevertheless.''
	
	The word usually serves better when not in 1st position.
	\begin{example}
		The roads were almost impassable.
		
		However, we at least succeeded in reaching camp.
		
		$\to$ The roads were almost impassable.
		
		At last, however, we succeeded in reaching camp.
	\end{example}
	When {\it however} comes 1st, it means ``in whatever way'' or ``to whatever extent.''
	\begin{example}
		However you advise him, he will probably do as he thinks best.
		
		However discouraging the prospect, they never lost heart.
	\end{example}
	\item {\bf Illusion.} See allusion.
	\item {\bf Imply. Infer.} Not interchangeable.
	
	Something implied is something suggested or indicated, though not expressed.
	
	Something inferred is something deduced from evidence at hand.
	\begin{example}
		Farming implies early rising.
		
		Since she was a a farmer, we inferred that she got up early.
	\end{example}
	\item {\bf Importantly.} Avoid by rephrasing.
	\begin{example}
		More importantly, he paid for the damages.
		
		$\to$ What's more, he paid for the damages.
		
		With the breeze freshening, he altered course to pass inside the island.
		
		More importantly, as things turned out, he tucked in a reef.
		
		$\to$ With the breeze freshening, he altered course to pass inside the island.
		
		More important, as things turned out, he tucked in a reef.
	\end{example}
	\item {\bf In regard to.} Often wrongly written {\it in regards to}.
	
	But {\it as regards} is correct, \& means the same thing.
	\item {\bf In the last analysis.} A bankrupt expression.
	\item {\bf Inside of. Inside.} The {\it of} following {\it inside} is correct in the adverbial meaning ``in less than.''
	
	In other meanings, {\it of} is unnecessary.
	\begin{example}
		Inside of 5 minutes I'll be inside the bank.
	\end{example}
	\item {\bf Insightful.} The word is a suspicious overstatement for ``perceptive.''
	
	If it is to be used at all, it should be used for instances of remarkably penetrating vision.
	
	Usually, it crops up merely to inflate the commonplace.
	\begin{example}
		That was an insightful remark you made.
		
		$\to$ That was a perceptive remark you made.
	\end{example}
	\item {\bf In terms of.} A piece of padding usually best omitted.
	\begin{example}
		The job was unattractive in terms of salary.
		
		$\to$ The salary made the job unattractive.
	\end{example}
	\item {\bf Interesting.} An unconvincing word; avoid it as a means of introduction.
	
	Instead of announcing that what you are about to tell is interesting, make it so.
	\begin{example}
		An interesting story is told of $\to$ (Tell the story without preamble.)
		
		In connection with the forthcoming visit of Mr. B. to America, it is interesting to recall that he
		
		$\to$ Mr. B., who will soon visit America
	\end{example}
	Also to be avoided in introduction is the word {\it funny}.
	
	Nothing becomes funny by being labeled so.
	\item {\bf Irregardless.} Should be {\it regardless}.
	
	The error results from failure to see the negative in {\it -less} \& from a desire to get it in as a prefix, suggested by such words as {\it irregular, irresponsible}, and, perhaps especially, {\it irrespective}.
	\item {\bf -ize.} Do not coin verbs by adding this tempting suffix.
	
	Many good \& useful verbs do end in {\it -ize}: {\it summarize, fraternize, harmonize, fertilize}.
	
	But there is a growing list of abominations: {\it containerize, prioritize, finalize}, to name 3.
	
	Be suspicious of {\it -ize}; let your ear \& your eye guide you.
	
	Never tack {\it -ize} onto a noun to create a verb.
	
	Usually you will discover that a useful verb already exists.
	
	Why say ``utilize'' when there is the simple, unpretentious word {\it use}?
	\item {\bf Kind of.} Except in familiar style, not to be used as a substitute for {\it rather} or {\it something like}.
	
	Restrict it to its literal sense: ``Amber is kind of fossil resin''; ``I dislike that kind of publicity.''
	
	The same holds true for {\it sort of}.
	\item {\bf Lay.} A transitive verb.
	
	Except in slang (``Let it lay''), do not misuse it for the intransitive verb {\it lie}.
	
	The hen, or the play, {\it lays} an egg; the llama {\it lies} down.
	
	The playwright went home \& {\it lay} down.
	\begin{example}
		lie, lay, lain, lying
		
		lay, laid, laid, laying
	\end{example}
	\item {\bf Leave.} Not to be misused for {\it let}.
	\begin{example}
		Leave it stand the way it is.
		
		$\to$ Let it stand the way it is.
		
		Leave go of that rope!
		
		$\to$ Let go of that rope!
	\end{example}
	\item {\bf Less.} Should not be misused {\it for fewer}.
	\begin{example}
		They had less workers than in the previous campaign.
		
		$\to$ They had fewer workers than in the previous campaign.
	\end{example}
	{\it Less} refers to quantity, {\it fewer} to number.
	
	``His troubles are less than mine'' means ``His troubles are not so great as mine.''
	
	``His troubles are fewer than mine'' means ``His troubles are not so numerous as mine.''
	\item {\bf Like.} Not to be used for the conjunction {\it as}.
	
	{\it Like} governs nouns \& pronouns; before phrases \& clauses the equivalent word is as.
	\begin{example}
		We spent the evening like in the old days.
		
		$\to$ We spent the evening as in the old days.
		
		Chlo\"e smells good, like a baby should.
		
		$\to$ Chlo\"e smells good, as a baby should.
	\end{example}
	The use of {\it like} for {\it as} has its defenders; they argue that any usage that achieves currency becomes valid automatically.
	
	This, they say, is the way the language is formed.
	
	It is \& it isn't.
	
	An expression sometimes enjoys a vogue, much as an article of apparel does.
	
	{\it Like} has long been widely misused by the illiterate; lately it has been taken up by the knowing \& the well-informed, who find it catchy, or liberating, \& who use it as though they were slumming.
	
	If every word or device that achieved currency were immediately authenticated, simply on the ground of popularity, the language would be as chaotic as a ball game with no foul lines.
	
	For the students, perhaps the most useful thing to know about {\it like} is that most carefully edited publications regard its use before phrases \& clauses as simple error.
	\item {\bf Line. Along these lines.} {\it Line} in the sense of ``course of procedure, conduct, thought'' is allowable but has been so overworked, particularly in the phrase {\it along these lines}, that a writer who aims at freshness or originality had better discard it entirely.
	\begin{example}
		Mr. B. also spoke along the same lines.
		
		$\to$ Mr. B. also spoke to the same effect.
		
		She is studying along the line of French literature.
		
		$\to$ She is studying French literature.
	\end{example}
	{\bf Literal. Literally.} Often incorrectly used in support of exaggeration or violent metaphor.
	\begin{example}
		a literal flood of abuse $\to$ a flood of abuse
		
		literally dead with fatigue $\to$ almost dead with fatigue
	\end{example}
	\item {\bf Loan.} A noun. As a verb, prefer {\it lend}.
	\begin{example}
		Lend me your ears.
		
		the loan of your ears
	\end{example}
	\item {\bf Meaningful.} A bankrupt adjective.
	
	Choose another, or rephrase.
	\begin{example}
		His was a meaningful contribution.
		
		$\to$ His contribution counted heavily.
		
		We are instituting many meaningful changes in the curriculum.
		
		$\to$ We are improving the curriculum in many ways.
	\end{example}
	\item {\bf Memento.} Often incorrectly written {\it momento}.
	\item {\bf Most.} Not to be used for {\it almost} in formal composition.
	\begin{example}
		most everybody $\to$ almost everybody
		
		most all the time $\to$ almost all the time
	\end{example}
	{\bf Nature.} Often simply redundant, used like {\it character}.
	\begin{example}
		acts of a hostile nature $\to$ hostile acts
	\end{example}
	{\it Nature} should be avoided in such vague expressions as ``a lover of nature,'' ``poems about nature.''
	
	Unless more specific statements follow, the reader cannot tell whether the poems have to do with natural scenery, rural life, the sunset, the untracked wilderness, or the habits of squirrels.
	\item {\bf Nauseous. Nauseated.} The 1st means ``sickening to contemplate''; the 2nd means ``sick at the stomach.''
	
	Do not, therefore, say, ``I feel nauseous,'' unless you are sure you have that effect on others.
	\item {\bf Nice.} A shaggy, all-purpose word, to be used sparingly in formal composition.
	
	``I had a nice time.''
	
	``It was nice weather.''
	
	``She was so nice to her mother.''
	
	The meanings are indistinct.
	
	{\it Nice} is most useful in the sense of ``precise'' or ``dedicate'': ``a nice distinction.''
	\item {\bf Nor.} Often used wrongly for {\it or} after negative expressions.
	\begin{example}
		He cannot eat nor sleep.
		
		$\to$ He cannot eat or sleep./He can neither eat nor sleep./He cannot eat nor can he sleep.
	\end{example}
	\item {\bf Noun used as verb.} Many nouns have lately been pressed into service as verbs.
	
	Not all are bad, but all are suspect.
	\begin{example}
		Be prepared for kisses when you gift your girlfriend with this merry scent.
		
		$\to$ Be prepared for kisses when you give your girlfriend this merry scent.
		
		The candidate hosted a dinner for 50 of her workers.
		
		$\to$ The candidate gave a dinner for 50 of her workers.
		
		The meeting was chaired by Mr. Oglethorp.
		
		$\to$ Mr. Oglethorp was chair of the meeting.
		
		She headquarters in Newark.
		
		$\to$ She has headquarters in Newark.
		
		The theater troupe debuted last fall.
		
		$\to$ The theater troupe made its debut last fall.
	\end{example}
	\item {\bf Offputting. Ongoing.} Newfound adjectives, to be avoided because they are inexact \& clumsy.
	
	{\it Ongoing} is a mix of ``continuing'' \& ``active'' \& is usually superfluous.
	\begin{example}
		He devoted all his spare time to the ongoing program for aid to the elderly.
		
		$\to$ He devoted all his spare time to the program for aid to the elderly.
	\end{example}
	{\it Offputting} might mean ``objectionable,'' ``disconcerting,'' ``distasteful.''
	
	Select instead a word whose meaning is clear.
	
	As a simple test, transform the participles to verbs.
	
	It is possible to {\it upset} something.
	
	But to {\it offput}?
	
	To {\it ongo}?
	\item {\bf One.} In the sense of ``a person,'' not to be followed by {\it his} or {\it her}.
	\begin{example}
		One must watch his step.
		
		$\to$ One must watch one's step. (You must watch your step.)
	\end{example}
	\item {\bf One of the most.} Avoid this feeable formula.
	
	``1 of the most exciting developments of modern science is $\ldots$''; ``Switzerland is 1 of the most beautiful countries of Europe.''
	
	There is nothing wrong with the grammar; the formula is simply threadbare.
	\item {\bf -oriented.} A clumsy, pretentious device, much in vogue.
	
	Find a better way of indicating orientation or alignment or direction.
	\begin{example}
		It was a manufacturing-oriented company.
		
		$\to$ It was a company chiefly concerned with manufacturing.
		
		Many of the skits are situation-oriented.
		
		$\to$ Many of the skits rely on situation.
	\end{example}
	\item {\bf Partially.} Not always interchangeable with {\it partly}.
	
	Best used in the sense of ``to a certain degree,'' when speaking of a condition or state: ``I'm partially resigned to it.''
	
	{\it Partly} carries the idea of a part as distinct from the whole - usually a physical object.
	\begin{example}
		The log was partially submerged.
		
		$\to$ The log was partly submerged.
		
		She was partially in \& partially out.
		
		$\to$ She was partly in \& partly out./She was part in, part out.
	\end{example}
	\item {\bf Participle for verbal noun.}
	\begin{example}
		There was little prospect of the Senate accepting even this compromise.
		
		$\to$ There was little prospect of the Senate's accepting even this compromise.
	\end{example}
	In the lefthand column, {\it accepting} is a present participle; in the righthand column, it is a verbal noun (gerund).
	
	The construction shown in the lefthand column is occasionally found, \& has its defenders.
	
	Yet it is easy to see that the 2nd sentence has to do not with a prospect of the Senate but with a prospect of accepting.
	
	%
	Any sentence in which the use of the possessive is awkward or impossible should of course be recast.
	\begin{example}
		In the event of a reconsideration of the whole matters becoming necessary.
		
		$\to$ If it should become necessary to reconsider the whole matter.
		
		There was great dissatisfaction with the decision of the arbitrators being favorable to the company.
		
		$\to$ There was great dissatisfaction with the arbitrators' decision in favor of the company.
	\end{example}
	\item {\bf People.} A word with many meanings.
	
	({\it The American Heritage Dictionary}, 3rd Edition, gives 9.)
	
	{\it The people} is a political term, not to be confused with {\it the public}.
	
	From the people comes political support or opposition; from the public comes artistic appreciation or commercial patronage.
	
	%
	The word {\it people} is best not used with words of number, in place of {\it persons}.
	
	If of ``6 people'' 5 went away, how many people would be left?
	
	Answer: 1 people.
	\item {\bf Personalize.} A pretentious word, often carrying bad advice.
	
	Do not {\it personalize} your prose; simply make it good \& keep it clean.
	
	See Chap. V, Reminder 1.
	\begin{example}
		a highly personalized affair $\to$ a highly personal affair
		
		Personalize your stationery. $\to$ Design a letterhead.
	\end{example}
	\item {\bf Personally.} Often unnecessary.
	\begin{example}
		Personally, I thought it was a good book.
		
		$\to$ I thought it a good book.
	\end{example}
	\item {\bf Possess.} Often used because to the writer it sounds more impressive than {\it have} or {\it own}.
	
	Such usage is not incorrect but is to be guarded against.
	\begin{example}
		She possessed great courage.
		
		$\to$ She had great courage (was very brave).
		
		He was the fortunate possessor of $\to$ He was lucky enough to own
	\end{example}
	\item {\bf Presently.} Has 2 meanings: ``in a short while'' \& ``currently.''
	
	Because of this ambiguity it is best restricted to the 1st meaning: ``She'll be here presently'' (``soon,'' or ``in a short time'').
	\item {\bf Prestigious.} Often an adjective of last resort.
	
	{\it It's in the dictionary, but that doesn't mean you have to use it}.
	\item {\bf Refer.} See {\it allude}.
	\item {\bf Regretful.} Sometimes carelessly used for {\it regrettable}: ``The mixup was due to a regretful breakdown in communications.''
	\item {\bf Relate.} Not to be used intransitively to suggest rapport.
	\begin{example}
		I relate well to Janet.
		
		$\to$ Janet \& I see things the same way./Janet \& I have a lot in common.
	\end{example}
	\item {\bf Respective. Respectively.} These words may usually be omitted with advantage.
	\begin{example}
		Works of fiction are listed under the names of their respective authors.
		
		$\to$ Works of fiction are listed under the names of their authors.
		
		The mile run \& the 2-mile run were won by Jones \& Cummings respectively.
		
		$\to$ The mile run was won by Jones, the 2-mile run by Cummings.
	\end{example}
	\item {\bf Secondly, thirdly, etc.} Unless you are prepared to begin {\it with 1stly} \& defend it (which will be difficult), do not prettify numbers with {\it -ly}.
	
	Modern usage prefers {\it second, third}, \& so on.
	\item {\bf Shall. Will.} In formal writing, the future tense requires {\it shall} for the 1st person, {\it will} for the 2nd \& 3rd.
	
	The formula to express the speaker's belief regarding a future action or state is {\it I shall}; {\it I will} expresses determination or consent.
	
	A swimmer in distress cries, ``I shall drown; no one will save me!''
	
	A suicide puts it the other way: ``I will drown; no one shall save me!''
	
	In relaxed speech, however, the words {\it shall} \& {\it will} are seldom used precisely; our ear guides us or fails to guide us, as the case may be, \& we are quite likely to drown when we want to survive \& survive when we want to drown.
	\item {\bf So.} Avoid, in writing, the use of so as an intensifier: ``so good''; ``so warm''; ``so delightful.''
	\item {\bf Sort of.} See {\it kind of}.
	\item {\bf Split infinitive.} There is precedent from the 14th century down for interposing an adverb between {\it to} \& the infinitive it governs, but the construction should be avoided unless the writer wishes to place unusual stress on the adverb.
	\begin{example}
		to diligently inquire $\to$ to inquire diligently
	\end{example}
	For another side to the split infinitive, see Chap. V, Reminder 14.
	\item {\bf State.} Not to be used as a mere substitute for {\it say, remark}.
	
	Restrict it to the sense of ``express fully or clearly'': ``He refused to state his objections.''
	\item {\bf Student body.} 9 times out of 10 a needles \& awkward expression, meaning no more than the simple word {\it students}.
	\begin{example}
		a member of the student body $\to$ a student
		
		popular with the student body $\to$ liked by the students
	\end{example}
	\item {\bf Than.} Any sentence with {\it than} (to express comparison) should be examined to make sure to essential words are missing.
	\begin{example}
		I'm probably closer to my mother than my father. (Ambiguous.)
		
		$\to$ I'm probably closer to my mother than to my father./I'm probably closer to my mother than my father is.
		
		It looked more like a cormorant than a heron.
		
		$\to$ It looked more like a cormorant than like a heron.
	\end{example}
	\item {\bf Thanking you in advance.} This sounds as if the writer meant, ``It will not be worth my while to write to you again.''
	
	In making your request, write ``Will you please,'' or ``I shall be obliged.''
	
	Then, later, if you feel moved to do so, or if the circumstances call for it, write a letter of acknowledgment.
	\item {\bf That. Which.} {\it That} is the defining, or restrictive, pronoun, {\it which} the nondefining, or nonrestrictive. (See Rule 3.)
	\begin{example}
		The lawn mower that is broken is in the garage. (Tells which one.)
		
		The lawn mower, which is broken, is in the garage. (Adds a fact about the only mower in question.)
	\end{example}
	The use of {\it which} for {\it that} is common in written \& spoken language (``Let us now go even unto Bethlehem, \& see this thing which is come to pass.'').
	
	Occasionally {\it which} seems preferable to {\it that}, as in the sentence from the Bible.
	
	But it would be a convenience to all if these 2 pronouns were used with precision.
	
	Careful writers, watchful for small conveniences, go {\it which}-hunting, remove the defining {\it whiches}, \& by so doing improve their work.
	\item {\bf The foreseeable future.} A cliche, \& a fuzzy one.
	
	How much of the future is foreseeable?
	
	10 minutes?
	
	10 years?
	
	Any of it?
	
	By whom is it foreseeable?
	
	Seers?
	
	Experts?
	
	Everybody?
	\item {\bf The truth} is$\ldots$ {\bf The fact} is$\ldots$ A bad beginning for a sentence.
	
	If you feel you are possessed of the truth, or of the fact, simply state it.
	
	Do not give it advance billing.
	\item {\bf They. He or She.} Do not use {\it they} when the antecedent is a distributive expression such as {\it each, each one, everybody, everyone, many a man}.
	
	Use the singular pronoun.
	\begin{example}
		Every one of us knows they are fallible.
		
		$\to$ Everyone in the community, whether they are a member of the Association or not, is invited to attend.
		
		$\to$ Everyone in the community, whether he is a member of the Association or not, is invited to attend.
	\end{example}
	A similar fault is the use of the plural pronoun with the antecedent {\it anybody, somebody, someone}, the intention being either to avoid the awkward {\it he or she} or to avoid committing oneself to one or the other.
	
	Some bashful speakers even say, ``A friend of mine told me that they$\ldots$''
	
	%
	The use of {\it he} as a pronoun for nouns embracing both genders is a simple, practical convention rooted in the beginnings of the English language.
	
	Currently, however, many writers find the use of the generic {\it he} or {\it his} to rename indefinite antecedents limiting or offensive.
	
	Substituting {\it he or she} in its place is the logical thing to do if it works.
	
	But it often doesn't work, if only because repetition makes it sound boring or silly.
	
	%
	Consider these strategies to avoid an awkward overuse of {\it he or she} or an unintentional emphasis on the masculine:
	
	Use the plural rather than the singular.
	\begin{example}
		The writer must address his readers' concerns.
		
		$\to$ Writers must address their readers' concerns.
	\end{example}
	Eliminate the pronoun altogether.
	\begin{example}
		The writer must address his readers' concerns.
		
		$\to$ The writer must address reader's concerns.
	\end{example}
	Substitute the 2nd person for the 3rd person.
	\begin{example}
		The writer must address his readers' concerns.
		
		$\to$ As a writer, you must address your readers' concerns.
	\end{example}
	No one need fear to use {\it he} if common sense supports it.
	
	If you think {\it she} is a handy substitute for {\it he}, try it \& see what happens.
	
	Alternatively, put all controversial nouns in the plural \& avoid the choice of sex altogether, although you may find your prose sounding general \& diffuse as a result.
	\item {\bf This.} The pronoun {\it this}, referring to the complete sense of a preceding sentence or clause, can't always carry the load \& so may produce an imprecise statement.
	\begin{example}
		Visiting dignitaries watched yesterday as ground was broken for the new high-energy physics laboratory with a blowout safety wall.
		
		This is the 1st visible evidence of the university's plans for modernization \& expansion.
		
		$\to$ Visiting dignitaries watched yesterday as ground was broken for the new high-energy physics laboratory with a blowout safety wall.
		
		The ceremony afforded the 1st visible evidence of the university's plans for modernization \& expansion.
	\end{example}
	In the lefthand example above, {\it this} does not immediately make clear what the 1st visible evidence is.
	\item {\bf Thrust.} This showy noun, suggestive of power, hinting of sex, is the darling of executives, politicos, \& speech-writers.
	
	Use it sparingly.
	
	Save it for specific application.
	\begin{example}
		Our reorganization plan has a tremendous thrust.
		
		$\to$ The piston has a 5-inch thrust.
		
		The thrust of his letter was that he was working more hours than he'd bargained for.
		
		$\to$ The point he made in his letter was that he was working more hours than he'd bargained for.
	\end{example}
	\item {\bf Tortuous. Torturous.} A winding road is {\it tortuous}, a painful ordeal is {\it torturous}.
	
	Both words carry the idea of ``twist,'' the twist having been a form of torture.
	\item {\bf Transpire.} Not to be used in the sense of ``happen,'' ``come to pass.''
	
	Many writers so use it (usually when groping toward imagined elegance), but their usage finds little support in the Latin ``breathe across or through.''
	
	It is correct, however, in the sense of ``become known.''
	
	``Eventually, the grim account of his villainy transpired'' (literally, ``leaked through or out'').
	\item {\bf Try.} Takes the infinitive: ``try to mend it,'' not ``try \& mend it.''
	
	Students of the language will argue that {\it try and} has won through \& become idiom.
	
	Indeed it has, \& it is relaxed \& acceptable.
	
	But {\it try to} is precise, \& when you are writing formal prose, try \& write {\it try to}.
	\item {\bf Type.} Not a synonym for {\it kind of}.
	
	The examples below are common vulgarisms.
	\begin{example}
		that type employee $\to$ that kind of employee
		
		I dislike that type publicity.
		
		$\to$ I dislike that kind of publicity.
		
		small, home-type hotels $\to$ small, homelike hotels
		
		a new type plane $\to$ a plane of a new design (new kind)
	\end{example}
	\item {\bf Unique.} Means ``without like or equal.''
	
	Hence, there can be no degrees of uniqueness.
	\begin{example}
		It was the most unique coffee maker on the market.
		
		$\to$ It was a unique coffee maker.
		
		The balancing act was very unique.
		
		$\to$ The balancing act was unique.
		
		Of all the spiders, the one that lives in a bubble under water is the most unique.
		
		$\to$ Among spiders, the one that lives in a bubble under water is unique.
	\end{example}
	\item {\bf Utilize.} Prefer {\it use}.
	\begin{example}
		I utilized the facilities.
		
		$\to$ I used the toilet.
		
		He utilized the dishwasher.
		
		$\to$ He used the dishwasher.
	\end{example}
	\item {\bf Verbal.} Sometimes means ``word for word'' \& in this sense may refer to something expressed in writing.
	
	{\it Oral} (from Latin {\it os}, ``mouth'') limits the meaning to what is transmitted by speech.
	
	{\it Oral agreement} is more precise than {\it verbal agreement}.
	\item {\bf Very.} Use this word sparingly.
	
	Where emphasis is necessary, use words strong in themselves.
	\item {\bf While.} Avoid the indiscriminate use of this word for {\it and, but}, \& {\it although}.
	
	Many writers use it frequently as a substitute for {\it and} or {\it but}, either from a mere desire to vary the connective or from doubt about which of the 2 connectives is more appropriate.
	
	In this use it is best replaced by a semicolon.
	\begin{example}
		The office \& salesrooms are on the ground floor, while the rest of the building is used for manufacturing.
		
		$\to$ The office \& salesrooms are on the ground floor; the rest of the building is used for manufacturing.
	\end{example}
	Its use as a virtual equivalent {\it of although} is allowable in sentences where this leads to no ambiguity or absurdity.
	\begin{example}
		While I admire his energy, I wish it were employed in a better cause.
	\end{example}
	This is entirely correct, as shown by the paraphrase
	\begin{example}
		I admire his energy; at the same time, I wish it were employed in a better cause.
	\end{example}
	Compare:
	\begin{example}
		While the temperature reaches 90 or 95 degrees in the daytime, the nights are often chilly.
	\end{example}
	The paraphrase shows why the use of {\it while} is incorrect:
	\begin{example}
		The temperature reaches 90 or 95 degrees in the daytime; at the same time the nights are often chilly.
	\end{example}
	In general, the writer will do well to use {\it while} only with strict literalness, in the sense of ``during the time that.''
	\item {\bf -wise.} Not to be used indiscriminately as a pseudosuffix: {\it taxwise, pricewise, marriagewise, prosewise, saltwater taffy-wise}.
	
	Chiefly useful when it means ``in the manner of'': {\it clockwise}.
	
	There is not a noun in the language to which {\it -wise} cannot be added if the spirit moves one to add it.
	
	The sober writer will abstain from the use of this wild additive.
	\item {\bf Worth while.} Overworked as a term of vague approval \& (with {\it not}) of disapproval.
	
	Strictly applicable only to actions: ``Is it worth while to telegraph?''
	\begin{example}
		His books are not worth while.
		
		$\to$ His books are not worth reading (are not worth one's while to read; do not repay reading).
	\end{example}
	The adjective {\it worthwhile} (1 word) is acceptable but emaciated.
	
	Use a stronger word.
	\begin{example}
		a worthwhile project $\to$ a promising (useful, valuable, exciting) project
	\end{example}
	\item {\bf Would.} Commonly used to express habitual or repeated action.
	
	(``He would get up early \& prepare his own breakfast before he went to work.'')
	
	But when the idea of habit or repetition is expressed, in such phrases as {\it once a year, every day, each Sunday}, the past tense, without {\it would}, is usually sufficient, and, from its brevity, more emphatic.
	\begin{example}
		Once a year he would visit the old mansion.
		
		$\to$ Once a year he visited the old mansion.
	\end{example}
	In narrative writing, always indicate the transition from the general to the particular - i.e., from sentences that merely state a general habit to those that express the action of a specific day or period.
	
	Failure to indicate the change will cause confusion.
	\begin{example}
		Townsend would get up early \& prepare his own breakfast.
		
		If the day was cold, he filled the stove \& had a warm fire burning before he left the house.
		
		On his way to the garbage, he noticed that there were footprints in the new-fallen snow on the porch.
	\end{example}
\end{enumerate}

%------------------------------------------------------------------------------%

\subsection{An Approach to Style (With a List of Reminders)}
Up to this point, the book has been concerned with what is correct, or acceptable, in the use of English.

In this final chapter, we approach style in its broader meaning: style in the sense of what is distinguished \& distinguishing.

Here we leave solid ground.

Who can confidently say what ignites a certain combination of words, causing them to explode in the mind?

Who knows why certain notes in music are capable of stirring the listener deeply, though the same notes slightly rearranged are impotent?

These are high mysteries, \& this chapter is a mystery story, thinly disguised.

There is no satisfactory explanation of style, no infallible guide to good writing, no assurance that a person who thinks clearly will be able to write clearly, no key that unlocks the door, no inflexible rule by which writers may shape their course.

Writers will often find themselves steering by stars that are disturbingly in motion.

%
The preceding chapters contain instructions drawn from established English usage; this one contains advice drawn from a writer's experience of writing.

Since the book is a rule book, these cautionary remarks, these subtly dangerous hints, are presented in the form of rules, but they are, in essence, mere gentle reminders: they state what most of us know \& at times forget.

%
Style is an increment in writing.

When we speak of Fitzgerald's style, we don't mean his command of the relative pronoun, we mean th sound his words make on paper.

All writers, by the way they use the language, reveal something of their spirits, their habits, their capacities, \& their biases.

This is inevitable as well as enjoyable.

All writing is communication; creative writing is communication through revelation - it is the Self escaping into the open.

No writer long remains incognito.

%
If you doubt that style is something of a mystery, try rewriting a familiar sentence \& see what happens.

Any much-quoted sentence will do.

Suppose we take ``These are the times that try men's souls.''

Here we have 8 short, easy words, forming a simple declarative sentence.

The sentence contains no flashy ingredient such as ``Damn the torpedoes!'' \& the words, as you see, are ordinary.

Yet in that arrangement, they have shown great durability; the sentence is into its 3rd century.

Now compare a few variations:
\begin{example}
	Times like these try men's souls.
	
	How trying it is to live in these times!
	
	These are trying times for men's souls.
	
	Soulwise, these are trying times.
\end{example}
It seems unlikely that Thomas Paine could have made his sentiment stick if he had couched it in any of these forms.

By why not?

No fault of grammar can be detected in them, \& in every case the meaning is clear.

Each version is correct, \& each, for some reason that we can't readily put our finger on, is marked for oblivion.

We could, of course, talk about ``rhythm'' \& ``cadence,'' but the talk would be vague \& unconvincing.

We could declare {\it soulwise} to be a silly word,    inappropriate to the occasion; but even that won't do - it does not answer the main question.

Are we even sure {\it soulwise} is silly?

If {\it otherwise} is a serviceable word, what's the matter with {\it soulwise}?

%
Here is another sentence, this one by a later Tom.

It is not a famous sentence, although its author (Thomas Wolfe) is well known.

``Quick are the mouths of earth, \& quick the teeth that fed upon this loveliness.''

The sentence would not take a prize for clarity, \& rhetorically it is at the opposite pole from ``These are the times.''

Try it in a different form, without the inversions:
\begin{example}
	The mouths of earth are quick, \& the teeth that fed upon this loveliness are quick, too.
\end{example}
The author's meaning is still intact, but not his overpowering emotion.

What was poetical \& sensuous has become prosy \& wooden; instead of the secret sounds of beauty, we are left with the simple crunch of mastication.

(Whether Mr. Wolfe was guilty of overwriting is, of course, another question - one that is not pertinent here.)

%
With some writers, style not only reveals the spirit of the man but reveals his identity, as surely as would his fingerprints.

Here, following, are 2 brief passages from the works of 2 American novelists.

The subject in each case is languor.

In both, the words used are ordinary, \& there is nothing eccentric about the construction.
\begin{example}
	He did not still feel weak, he was merely luxuriating in that supremely gutful lassitude of convalescence in which time, hurry, doing, did not exist, the accumulating seconds \& minutes \& hours to which in its well state the body is slave both waking \& sleeping, now reversed \& time now the lip-server \& mendicant to the body's pleasure instead of the body thrall to time's headlong course.
	\\
	
	Manuel drank his brandy.
	
	He felt sleepy himself.
	
	It was too hot to go out into the town.
	
	Besides there was nothing to do.
	
	He wanted to see Zurito.
	
	He would go to sleep while he waited.
\end{example}
Anyone acquainted with Faulkner \& Hemingway will have recognized them in these passages \& perceived which was which.

How different are their languors!

%
Or take 2 American poets, stopping at evening.

One stops by woods, the other by laughing flesh.
\begin{example}
	My little horse must think it queer
	
	To stop without a farmhouse near
	
	Between the woods \& frozen lake
	
	The darkest evening of the year.
	\\
	
	I have perceived that to be with those I like is enough,
	
	To stop in company with the rest at evening is enough,
	
	To be surrounded by beautiful, curious, breathing,
	
	laughing flesh is enough$\ldots$
\end{example}
Because of the characteristic styles, there is little question about identity here, \& if the situations were reversed, with Whitman stopping by woods \& Frost by laughing flesh (not one of his regularly scheduled stops), the reader would know who was who.

%
Young writer often suppose that style is a garnish for the meat of prose, a sauce by which a dull dish is made palatable.

Style has no such separate entity; it is nondetachable, unfilterable.

The beginner should approach style warily, realizing that it is an expression of self, \& should turn resolutely away from all devices that are popularly believed to indicate style - all mannerisms, tricks, adornments.

The approach to style is by way of plainness, simplicity, orderliness, sincerity.

%
Writing is, for most, laborious \& slow.

The mind travels faster than the pen; consequently, writing becomes a question of learning to make occasional wing shots, bringing down the bird of thought as it flashes by.

A writer is a gunner, sometimes waiting in the blind for something to come in, sometimes roaming the countryside hoping to scare something up.

Like other gunners, the writer must cultivate patience, working many covers to bring down 1 partridge.

Here, following, are some suggestions \& cautionary hints that may help the beginner find the way to a satisfactory style.

\subsubsection{Place yourself in the background.}
``Write in a way that draws the reader's attention to the sense \& substance of the writing, rather than to the mood \& temper of the author. If the writing is solid \& good, the mood \& temper of the writer will eventually be revealed \& not at the expense of the work. Therefore, the 1st piece of advice is this: to achieve style, begin by affecting none -- i.e., place yourself in the background. A careful \& honest writer does not need to worry about style. As you become proficient in the use of language, your style will emerge, because you yourself will emerge, \& when this happens you will find it increasingly easy to break through the barriers that separate you from other minds, other hearts -- which is, of course, the purpose of writing, as well as its principal reward. Fortunately, the act of composition, or creation, disciplines the mind; writing is 1 way to go about thinking, \& the practice \& habit of writing not only drain the mind but supply it, too.'' -- \cite[p. 78]{Strunk_White_element_style}

%------------------------------------------------------------------------------%

\subsubsection{Write in a way that comes naturally.}
``Write in a way that comes easily \& naturally to you, using words \& phrases that come readily to hand. But do not assume that becaues you have acted naturally your product is without flaw.

The use of language begins with imitation. The infant imitates the sounds made by its parents; the child imitates 1st the spoken language, then the stuff of books. The imitative life continues long after the writer is secure in the language, for it is almost impossible to avoid imitating what one admires. Never imitate consciously, but do not worry about being an imitator; take pains instead to admire what is good. Then when you write in a way that comes naturally, you will echo the halloos that bear repeating.'' -- \cite[p. 79]{Strunk_White_element_style}

%------------------------------------------------------------------------------%

\subsubsection{Work from a suitable design.}
``Before beginning to compose something, gauge the nature \& extent of the enterprise \& work from a suitable design. (See Chap. II, Rule 12.) Design informs even the simplest structure, whether of brick \& steel or of prose. You raise a pup tent from 1 sort of vision, a cathedral from another. This does not mean that you must sit with a blueprint always in front of you, merely that you had best anticipate what you are getting into. To compose a laundry list, you can work directly from the pile of soiled garments, ticking them off 1 by 1. By to write a biography, you will need at least a rough scheme; you cannot plunge in blindly \& start ticking off fact after fact about your subject, lest you miss the forest for the trees \& there be no end to your labors.

Sometimes, of course, impulse \& emotion are more compelling than design. If you are deeply troubled \& are composing a letter appealing for mercy or for love, you had best not attempt to organize your emotions; the prose will have a better chance if the emotions are left in disarray -- which you'll probably have to do anyway, since feelings do not usually lend themselves to rearrangement. But even the kind of writing that is essentially adventurous \& impetuous will on examination be found to have a secret plan: Columbus didn't just sail, he sailed west, \& the New World took shape from this simple \&, we now think, sensible design.'' -- \cite[p. 80]{Strunk_White_element_style}

%------------------------------------------------------------------------------%

\subsubsection{Write with nouns \& verbs.}
``Write with nouns \& verbs, not with adjectives \& adverbs. The adjective hasn't been built that can pull a weak or inaccurate noun out of a tight place. This is not to disparage adjectives \& adverbs; they are indispensable parts of speech. Occasionally they surprise us with their power, as in
\begin{quotation}\it
	Up the airy mountain,
	
	Down the rushy glen,
	
	We daren't go a-hunting
	
	For fear of little men $\ldots$
\end{quotation}
The nouns {\it mountain} \& {\it glen} are accurate enough, but had the mountain not become airy, the glen rushy, William Ailing-ham might never have got off the ground with this poem. In general, however, it is nouns \& verbs, not their assistants, that give good writing its toughness \& color.'' -- \cite[p. 81]{Strunk_White_element_style}

%------------------------------------------------------------------------------%

\subsubsection{Revise \& rewrite.}
``Revising is part of writing. Few writers are so expert that they can produce what they are after on the 1st try. Quite often you will discover, on examining the completed work, that there are serious flaws in the arrangement of the material, calling for transpositions. When this is the case, a word processor can save you time \& labor as you rearrange the manuscript. You can select material on your screen \& move it to a more appropriate spot, or, if you cannot find the right spot, you can move the material to the end of the manuscript until you decide whether to delete it. Some writers find that working with a printed copy of the manuscript helps them to visualize the process of change; others prefer to revise entirely on screen. Above all, do not be afraid to experiment with what you have written. Save both the original \& the revised versions; you can always use the computer to restore the manuscript to its original condition, should that course seem best. Remember, it is no sign of weakness or defeat that your manuscript ends up in need of major surgery. This is a common occurrence in all writing, \& among the best writers.'' -- \cite[p. 82]{Strunk_White_element_style}

%------------------------------------------------------------------------------%

\subsubsection{Do not overwrite.}
``Rich, ornate prose is hard to digest, generally unwholesome, \& sometimes nauseating. If the sickly-sweet word, the overblown phrase are your natural form of expression, as is sometimes the case, you will have to compensate for it by a show of vigor, \& by writing something as meritorious as the Songs of Songs, which is Solomon's.

When writing with a computer, you must guard against wordiness. The click \& flow of a word processor can be seductive, \& you may find yourself adding a few unnecessary words or even a whole passage just to experience the pleasure of running your fingers over the keyboard \& watching your words appear on the screen. It is always a good idea to reread your writing later \& ruthlessly delete the excess.'' -- \cite[p. 83]{Strunk_White_element_style}

%------------------------------------------------------------------------------%

\subsubsection{Do not overstate.}
``When you overstate, readers will be instantly on guard, \& everything that has preceded your overstatement as well as everything that follows it will be suspect in their minds because they have lost confidence in your judgment or your poise. Overstatement is 1 of the common faults. A single overstatement, wherever or however it occurs, diminishes the whole, \& a single carefree superlative has the power to destroy, for readers, the object of your enthusiasm.'' -- \cite[p. 84]{Strunk_White_element_style}

%------------------------------------------------------------------------------%

\subsubsection{Avoid the use of qualifiers.}
``{\it Rather, very, little, pretty} -- these are the leeches that infest the pond of prose, sucking the blood of words. The constant use of the adjective {\it little} (except to indicate size) is particularly debilitating; we should all try to do a little better, we should all be very watchful of this rule, for it is a rather important one, \& we are pretty sure to violate it now \& then.'' -- \cite[p. 85]{Strunk_White_element_style}

%------------------------------------------------------------------------------%

\subsubsection{Do not affect a breezy manner.}
``The volume of writing is enormous, these days, \& much of it has a sort of windiness about it, almost as though the author were in a state of euphoria. ``Spontaneous me,'' say Whitman, \&, in his innocence, let loose the hordes of uninspired scribblers who would 1 day confuse spontaneity with genius.

The breezy style is often the work of an egocentric, the person who imagines that everything that comes to mind is of general interest \& that uninhibited prose creates high spirits \& carries the day. Open any alumni magazine, turn to the class notes, \& you are quite likely to encounter old Spontaneous Me at work -- an aging collegian who writes something like this:
\begin{quotation}\it
	Well, guys, here I am again dishing the dirt about your disorderly classmates, after passing a week in the Big Apple trying to catch the Columbia hoops tilt \& then a cab-ride from hell through the West Side casbah. \& speaking of news, howzabout tossing a few primo items this way?
\end{quotation}
This is an extreme example, but the same wind blows, at lesser velocities, across vast expanses of journalistic prose. The author in this case has managed in 2 sentences to commit most of the unpardonable sins: he obviously has nothing to say, he is showing off \& directing the attention of the reader to himself, he is using slang with neither provocation nor ingenuity, he adopts a patronizing air by throwing in the word {\it primo}, he is humorless (though full of fun), dull, \& empty. He has not done his work. Compare his opening remarks with the following -- a plunge directly into the news:
\begin{quotation}\it
	Clyde Crawford, who stroked the varsity shell in 1958, is swinging an oar again after a lapse of 40 years. Clyde resigned last spring as executive sales manager of the Indiana Flotex Company \& is now a gondolier in Venice.
\end{quotation}
This, although conventional, is compact, informative, unpretentious. The writer has dug up an item of news \& presented it in a straightforward manner. What the 1st writer tried to accomplish by cutting rhetorical capers \& by breeziness, the 2nd writer managed to achieve by good reporting, by keeping a tight rein on his material, \& by staying out of the act.'' -- \cite[p. 87]{Strunk_White_element_style}

%------------------------------------------------------------------------------%

\subsubsection{Use orthodox spelling.}
``In ordinary composition, use orthodox spelling. Do not write {\it nite} for {\it night, thru} for {\it through, pleez} for {\it please}, unless you plan to introduce a complete system of simplified spelling \& are prepared to take the consequences.

In the original edition of {\it The Elements of Style}, there was a chapter on spelling. In it, the author had this to say:
\begin{quotation}\it
	The spelling of English words is not fixed \& invariable, nor does it depend on any other authority than general agreement. At the present day there is practically unanimous agreement as to the spelling of most words $\ldots$ At any given moment, however, a relatively small number of words may be spelled in more than 1 way. Gradually, as a rule, 1 of these forms comes to be generally preferred, \& the less customary form comes to look obsolete \& is discarded. From time to time new forms, mostly simplifications, are introduced by innovators, \& either win their place or die of neglect.
	
	The practical objection to unaccepted \& oversimplified spellings is the disfavor with which they are received by the reader. They distract his attention \& exhaust his patience. He reads the form though automatically, without thought of its needless complexity; he reads the abbreviation tho \& mentally supplies the missing letters, at the cost of a fraction of his attention. The writer has defeated his own purposed.
\end{quotation}
The language manages somehow to keep pace with events. A word that has taken hold in our century is {\it thru-way}; it was born of necessity \& is apparently here to stay. In combination with {\it way, thru} is more serviceable than {\it through}; it is a high-speed word for readers who are going 65. {\it Throughway} would be too long to fit on a road sign, too slow to serve the speeding eye. It is conceivable that because of our thruways, {\it through} will eventually become {\it thru} -- after many more thousands of miles of travel.'' -- \cite[p. 88]{Strunk_White_element_style}

%------------------------------------------------------------------------------%

\subsubsection{Do not explain too much.}
``It is seldom advisable to tell all. Be sparing, e.g., in the use of adverbs after ``he said,'' ``she replied,'' \& the like: ``he said consolingly''; ``she replied grumblingly.'' Let the conversation itself disclose the speaker's manner of condition. Dialogue heavily weighted with adverbs after the attributive verb is cluttery \& annoying. Inexperienced writers not only overwork their adverbs but load their attributives with explanatory verbs: ``he consoled,'' ``she congratulated.'' They do this, apparently, in the belief that the word {\it said} is always in need of support, or because they have been told to do it by experts in the art of bad writing.'' -- \cite[p. 89]{Strunk_White_element_style}

%------------------------------------------------------------------------------%

\subsubsection{Do not construct awkward adverbs.}
``Adverbs are easy to build. Take an adjective or a participle, add {\it -ly}, \& behold! you have an adverb. But you'd probably be better off without it. Do not write {\it tangledly}. The word itself is a tangle. Do not even write {\it tiredly}. Nobody says {\it tangledly} \& not many people say {\it tiredly}. Words that are not used orally are seldom the ones to put on paper.
\begin{example}
	He climbed tiredly to bed. $\to$ He climbed wearily to bed.
	
	The lamp cord lay tangledly beneath her chair. $\to$ The lamp cord lay in tangles beneath her chair.
\end{example}
Do not dress words up by adding {\it -ly} to them, as though putting a hat on a horse.
\begin{example}
	overly $\to$ over, muchly $\to$ much, thusly $\to$ thus.'' -- \cite[p. 90]{Strunk_White_element_style}
\end{example}


%------------------------------------------------------------------------------%

\subsubsection{Make sure the reader knows who is speaking.}
``Dialogue is a total loss unless you indicate who the speaker is. In long dialogue passages containing no attributives, the reader may become lost \& be compelled to go back \& reread in order to puzzle the thing out. Obscurity is an imposition on the reader, to say nothing of its damage to the work.

In dialogue, make sure that your attributives do not awkwardly interrupt a spoken sentence. Place them where the break would come naturally in speech -- i.e., where the speaker would pause for emphasis, or take a breath. The best test for locating an attributive is to speak the sentence aloud.
\begin{example}
	``Now, my boy, we shall see,'' he said, ``how well you have learned your lesson.'' $\to$ ``Now, my boy,'' he said, ``we shall see how well you have learned your lesson.''
	
	``What's more, they would never,'' she added, ``consent to the plan.'' $\to$  ``What's more,'' she added, ``they would never consent to the plan.'''' -- \cite[p. 91]{Strunk_White_element_style}
\end{example}

%------------------------------------------------------------------------------%

\subsubsection{Avoid fancy words.}
``Avoid the elaborate, the pretentious, the coy, \& the cute. Do not be tempted by a 20-dollar word when there is a 10-center handy, ready \& able. Anglo-Saxon is a livelier tongue than Latin, so use Anglo-Saxon words. In this, as in so many matters pertaining to style, one's ear must be one's guide: {\it gut} is a lustier noun than {\it intestine}, but the 2 words are not interchangeable, because {\it gut} is often inappropriate, being too coarse for the context. Never call a stomach a tummy without good reason.

If you admire fancy words, if every sky is {\it beauteous}, every blonde {\it curvaceous}, every intelligent child prodigious, if you are tickled by {\it discombobulate}, you will have a bad time with Reminder 14. What is wrong, you ask, with {\it beauteous?} No one knows, for sure. There is nothing wrong, really, with any word -- all are good, but some are better than others. A matter of ear, a matter of reading the books that sharpen the ear.

The line between the fancy \& the plain, between the atrocious \& the felicitous, is sometimes alarmingly fine. The opening phrase of the Gettysburg address is close to the line, at least by our standards today, \& Mr. Lincoln, knowingly or unknowingly, was flirting with disaster when he wrote ``4 score \& 7 years ago.'' The President could have got into his sentence with plain ``87'' -- a saving of 2 words \& less of a strain on the listeners' powers of multiplication. But Lincoln's ear must have told him to go ahead with 4 score \& 7. By doing so, he achieved cadence while skirting the edge of fanciness. Suppose he had blundered over the line \& written, ``In the year of our Lord seventeen hundred \& seventy-six.'' His speech would have sustained a heavy blow. Or suppose he had settle for ``87.'' In that case he would have got into his introductory sentence too quickly; the timing would have been bad.

The question of ear is vital. Only the writer whose ear is reliable is in a position to use bad grammar deliberately; this writer knows for sure when a colloquialism is better than formal phrasing \& is able to sustain the work at a level of good taste. So cock your ear. Years ago, students were warned not to end a sentence with a preposition; time, of course, has softened that rigid decree. Not only is the preposition acceptable at the end, sometimes it is more effective in that spot than anywhere else. ``A claw hammer, not an ax, was the tool he murdered her with.'' This is preferable to ``A claw hammer, not an ax, was the tool with which he murdered her.'' Why? Because it sounds more violent, more like murder. A matter of ear.

\& would you write ``The worst tennis player around here is I'' or ``The The worst tennis player around here is me''? The 1st is good grammar, the 2nd is good judgment -- although the {\it me} might not do in all contexts.

The split infinitive is another trick of rhetoric in which the ear must be quicker than the handbook. Some infinitives seem to improve on being split, just as a stick of round stovewood does. ``I cannot bring myself to really like the fellow.'' The sentence is relaxed, the meaning is clear, the violation is harmless \& scarcely perceptible. Put the other way, the sentence becomes stiff, needlessly formal. A matter of ear.

There are times when the ear not only guides us through difficult situations but also saves us from minor or major embarrassments of prose. The ear, e.g., must decide when to omit {\it that} from a sentence, when to retain it. ``She knew she could do it'' is preferable to ``She knew that she could do it'' -- simpler \& just as clear. Bu tin many cases the {\it that} is needed. ``He felt that his big nose, which was sunburned, made him look ridiculous.'' Omit the {\it that} \& you have ``He felt his big nose $\ldots$'''' -- \cite[p. 93]{Strunk_White_element_style}

%------------------------------------------------------------------------------%

\subsubsection{Do not use dialect unless your ear is good.}
``Do not attempt to use dialect unless you are a devoted student of the tongue you hope to reproduce. If you use dialect, be consistent. The reader will become impatient or confused upon finding 2 or more versions of the same word or expression. In dialect it is necessary to spell phonetically, or at least ingeniously, to capture unusual inflections. Take, e.g., the word {\it once}. It often appears in dialect writing as {\it oncet}, but {\it oncet} looks as though it should be pronounced ``onset.'' A better spelling would be {\it wunst}. But if you write it {\it oncet} once, write it that way throughout. The best dialect writers, by \& large, are economical of their talents; they use the minimum, not the maximum, of deviation from the norm, thus sparing their readers as well as convincing them.'' -- \cite[p. 94]{Strunk_White_element_style}

%------------------------------------------------------------------------------%

\subsubsection{Be clear.}
``Clarity is not the prize in writing, nor it is always the principal mark of a good style. There are occasions when obscurity serves a literary yearning, if not a literary purpose, \& there are writers whose mien is more overcast than clear. But since writing is communication, clarity can only be a virtue. \& although there is no substitute for merit in writing, clarity comes closest to being one. Even to a writer who is being intentionally obscure or wild of tongue we can say, ``be obscure clearly! Be wild of tongue in a way we can understand!'' Even to writers of market letters, telling us (but not telling us) which securities are promising, we can say, ``Be cagey plainly! Be elliptical in a straightforward fashion!''

Clarity, clarity, clarity. When you become hopelessly mired in a sentence, it is best to start fresh; do not try to fight your way through against the terrible odds of syntax. Usually what is wrong is that the construction has become too involved at some point; the sentence needs to be broken apart \& replaced by 2 or more shorter sentences.

Muddiness is not merely a disturber of prose, it is also destroyer of life, of hope: death on the highway caused by a badly worded road sign, heartbreak among lovers caused by a misplaced phrase in a well-intentioned letter, anguish of a traveler expecting to be met at a railroad station \& not being met because of a slipshod telegram. Think of the tragedies that are rooted in ambiguity, \& be clear! When you say something, make sure you have said it. The chances of your having said it are only fair.'' -- \cite[p. 95]{Strunk_White_element_style}

%------------------------------------------------------------------------------%

\subsubsection{Do not inject opinion.}
``Unless there is a good reason for its being there, do not inject opinion into a piece of writing. We all have opinions about almost everything, \& the temptation to toss them in is great. To air one's views gratuitously, however, is to imply that the demand for them is brisk, which may not be the case, \& which, in any event, may not be relevant to the discussion. Opinions scattered indiscriminately about leave the mark of egotism on a work. Similarly, to air one's views at an improper time may be in bad taste. If you have received a letter inviting you to speak at the dedication of a new cat hospital, \& you have cats, your reply, declining the invitation, does not necessarily have to cover the full range of your emotions. You must make it clear that you will not attend, but you do not have to let fly at cats. The writer of the letter asked a civil question; attack cats, then, only if you can do so with good humor, good taste, \& in such a way that your answer will be courteous as well as responsive. Since you are out of sympathy with cats, you may quite properly give this as a reason for not appearing at the dedicatory ceremonies of a cat hospital. But bear in mind that your opinion of cats was not sought, only your services as a speaker. Try to keep things straight.'' -- \cite[p. 96]{Strunk_White_element_style}

%------------------------------------------------------------------------------%

\subsubsection{Use figures of speech sparingly.}
``The simile is a common device \& a useful one, but similes coming in rapid fire, one right on top of another, are more distracting than illuminating. Readers need time to catch their breath; they can't be expected to compare everything with something else, \& no relief in sight.

When you use metaphor, do not mix it up. I.e., don't start by calling something a swordfish \& end by calling it an hourglass.'' -- \cite[p. 97]{Strunk_White_element_style}

%------------------------------------------------------------------------------%

\subsubsection{Do not take shortcuts at the cost of clarity.}
``Do not use initials for the names of organizations or movements unless you are certain the initials will be readily understood. Write things out. Not everyone knows that MADD means Mothers Against Drunk Driving, \& even if everyone did, there are babies being born every minute who will someday encounter the name for the 1st time. They deserve to see the words, not simply the initials. A good rule is to start your article by writing out names in full, \& then, later, when your readers have got their bearings, to shorten them.

Many shortcuts are self-defeating; they waste the reader's time instead of conserving it. There are all sorts of rhetorical stratagems \& devices that attract writers who hope to be pithy, but most of them are simply bothersome. The longest way round is usually the shortest home, \& the one truly reliable shortcut in writing is to choose words that are strong \& surefooted to carry readers on their way.'' -- \cite[p. 98]{Strunk_White_element_style}

%------------------------------------------------------------------------------%

\subsubsection{Avoid foreign languages.}
``The writer will occasionally find it convenient or necessary to borrow from other languages. Some writers, however, from sheer exuberance or a desire to show off, sprinkle their work liberally with foreign expressions, with no regard for the reader's comfort. It is a bad habit. Write in English.'' -- \cite[p. 99]{Strunk_White_element_style}

%------------------------------------------------------------------------------%

\subsubsection{Prefer the standard to the offbeat.}
``Young writers will be drawn at every turn toward eccentricities in language. They will hear the beat of new vocabularies, the exciting rhythms of special segments of their society, each speaking a language of its own. All of us come under the spell of these unsettling drums; the problem for beginners is to listen to them, learn the words, feel the vibrations, \& not be carried away.

Youths invariably speak to other youths in a tongue of their own devising: they renovate the language with a wild vigor, as they would a basement apartment. By the time this paragraph sees print, {\it psyched, nerd, ripoff, dude, geek}, \& {\it funky} will be the words of yesteryear, \& we will be fielding more recent ones that have come bouncing into our speech -- some of them into our dictionary as well. A new word is always up for survival. Many do survive. Others grow stale \& disappear. Most are, at least in their infancy, more approximate to conversation than to composition.

Today, the language of advertising enjoys an enormous circulation. With its deliberate infractions of grammatical rules \& its crossbreeding of the parts of speech, it profoundly influences the tongues \& pens of children \& adults. Your new kitchen range is so revolutionary it {\it obsoletes} all other ranges. Your counter top is beautiful because it is {\it accessorized} with gold-plated faucets. Your cigarette tastes good {\it like} a cigarette should. \&, {\it like the man says}, you will want to try one. You will also, in all probability, want to try writing that way, using that language. You do so at your peril, for it is the language of mutilation.

Advertisers are quite understandably interested in what they call ``attention getting.'' The man photographed must have lost an eye or grown a pink beard, or he must have 3 arms or be sitting wrong-end-to on a horse. This technique is proper in its place, which is the world of selling, but the young writer had best not adopt the device of mutilation in ordinary composition, whose purpose is to engage, not paralyze, the readers senses. Buy the gold-plated faucets if you will, but do not accessorize your prose. To use the language well, do not begin by hacking it to bits; accept the whole body of it, cherish its classic form, its variety, \& its richness.

Another segment of society that has constructed a language of its own is business. People in business say that toner cartridges are {\it in short supply}, that they have {\it updated} the next shipment of these cartridges, \& that they will {\it finalize} their recommendations at the next meeting of the board. They are speaking a language familiar \& dear to them. Its portentous nouns \& verbs invest ordinary events with high adventure; executives walk among toner cartridges, caparisoned like knights. We should tolerate them -- every person of spirit wants to ride a white horse. The only question is whether business vocabulary is helpful to ordinary prose. Usually, the same ideas can be expressed less formidably, if one makes the effort. A good many of the special words of business seem designed more to express the user's dreams than to express a precise meaning. Not all such words, of course, can be dismissed summarily; indeed, no word in the language can be dismissed offhand by anyone who has a healthy curiosity. {\it Update} isn't a bad word; in the right setting it is useful. In the wrong setting, though, it is destructive, \& the trouble with adopting coinages too quickly is that they will bedevil one by insinuating themselves where they do not belong. This may sound like rhetorical snobbery, or plain stuffiness; but you will discover, in the course of your work, that the setting of a word is just as restrictive as the setting of a jewel. The general rule here is to prefer the standard. {\it Finalize}, for instance, is not standard; it is special, \& it is a peculiarly fuzzy \& silly word. Does it mean ``terminate,'' or does it mean ``put into final form''? One can't be sure, really, what it means, \& one gets the impression that the person using it doesn't know, either, \& doesn't want to know.

The special vocabularies of the law, of the military, of government are familiar to most of us. Even the world of criticism has a modest pouch of private words ({\it luminous, taut}), whose only virtue is that they are exceptionally nimble \& can escape from the garden of meaning over the wall. Of these critical words, Wilcott Gibbs once wrote, ``$\ldots$ they are detached from the language \& inflated like little balloons.'' The young writer should learn to spot them -- words that at 1st glance seem freighted with delicious meaning but that soon burst in air, leaving nothing but a memory of bright sound.

The language is perpetually in flux: it is a living stream, shifting, changing, receiving new strength from a thousand tributaries, losing old forms in the backwaters of time. To suggest that a young writer not swim in the main stream of this turbulence would be foolish indeed, \& such is not the intent of these cautionary remarks. The intent is to suggest that in choosing between the formal \& the informal, the regular \& the offbeat, the general \& the special, the orthodox \& the heretical, the beginner err on the side of conservatism, on the side of established usage. No idiom is taboo, no accent forbidden; there is simply a better chance of doing well if the writer holds a steady course, enters the stream of English quietly, \& does not thrash about.

``But,'' you may ask, ``what if it comes natural to me to experiment rather than conform? What if I am a pioneer, or even a genius?'' Answer: then be one. But do not forget that what may seem like pioneering may be merely evasion, or laziness -- the disinclination to submit to discipline. Writing good standard English is no cinch, \& before you have managed it you will have encountered enough rough country to satisfy even the most adventurous spirit.

Style takes its final shape more from attitudes of mind than from principles of composition, for, as an elderly practitioner once remarked, ``Writing is an act of faith, not a trick of grammar.'' This moral observation would have no place in a rule book were it not that style {\it is} the writer, \& therefore what you are, rather than what you know, will at last determine your style. If you write, you must believe -- in the truth \& worth of the scrawl, in the ability of the reader to receive \& decode the message. No one can write decently who is distrustful of the reader's intelligence, or whose attitude is patronizing.

Many references have been made in this book to ``the reader,'' who has been much in the news. It is now necessary to warn you that you concern for the reader must be pure: you must sympathize with the reader's plight (most readers are in trouble about half the time) but never seek to know the reader's wants. Your whole duty as a writer is to please \& satisfy yourself, \& the true writer always plays to an audience of one. Start sniffing the air, or glancing at the Trend Machine, \& you are as good as dead, although you may make a nice living.

Full of belief, sustained \& elevated by the power of purpose, armed with the rules of grammar, you are ready to exposure. At this point, you may well pattern yourself on the fully exposed cow of Robert Louis Stevenson's rhyme. This friendly \& commendable animal, you may recall, was ``blown by all the winds that pass{\tt/}\& wet with all the showers.'' \& so must you as a young writer be. In our modern idiom, we should say that you must get wet all over. Mr. Stevenson, working in a plainer style, said it with felicity, \& suddenly 1 cow, out of so many, received the gift of immortality. Like the steadfast writer, she is at home in the wind \& the rain; \&, thanks to 1 moment of felicity, she will live on \& on \& one.'' -- \cite[pp. 101--103]{Strunk_White_element_style}

%------------------------------------------------------------------------------%

\subsection{Afterword}
``Will Strunk \& E. B. White were unique collaborators. Unlike Gilbert \& Sullivan, or Woodward \& Bernstein, they worked separately \& decades apart.

We have no way of knowing whether Prof. Strunk took particular notice of Elwyn Brooks White, a student of his at Cornell University in 1919. Neither teacher nor pupil could have realized that their names would be linked as they now are. Nor could they have imagined that 38 years after they met, White would take this little gem of a textbook that Strunk had written for his students, polish it, expand it, \& transform it into a classic.

E. B. White shared Strunk's sympathy for the reader. To Strunk's do's \& don'ts he added passages about the power of words \& the clear expression of thoughts \& feelings. To the nuts \& bolts of grammar he added a rhetorical dimension.

The editors of this edition have followed in White's footsteps, once again providing fresh examples \& modernizing usage where appropriate. {\it The Elements of Style} is still a little book, small enough \& important enough to carry in your pocket, as I carry mine. It has helped me to write better. I believe it can do the same for you.

{\sc Charles Osgood}'' -- \cite[p. 104]{Strunk_White_element_style}

%------------------------------------------------------------------------------%

\subsection{Glossary}

\begin{enumerate}
	\item {\bf adjectival modifier.} A word, phrase, or clause that acts as an adjective in qualifying the meaning of a noun or pronoun, {\it Your} country; a {\it turn-of-the-century} style; people {\it who are always late}.
	\item {\bf adjective.} A word that modifies, quantifies, or otherwise describes a noun or pronoun.
	
	{\it Drizzly} November; midnight {\it dreary}; {\it only} requirement.
	\item {\bf adverb.} A word that modifies or otherwise qualifies a verb, an adjective, or another adverb.
	
	Gestures {\it gracefully}; {\it exceptionally} quiet engine.
	\item {\bf adverbial phrase.} A phrase that functions as an adverb. (See {\it phrase}.)
	
	Landon laughs {\it with abandon}.
	\item {\bf agreement.} The correspondence of a verb with its subject in person \& number (Karen {\it goes} to Cal Tech; her sisters {\it go} to UCLA), \& of a pronoun with its antecedent in person, number, \& gender (Ass soon as Karen finished the exam, {\it she} picked up {\it her} books \& left the room).
	\item {\bf antecedent.} The noun to which a pronoun refers.
	
	A pronoun \& its antecedent must agree in person, number, \& gender.
	
	Michael \& {\it his} teammates moved off campus.
	\item {\bf appositive.} A noun or noun phrase that renames or adds identifying information to a noun it immediately follows.
	
	His brother, {\it an accountant with Arthur Andersen}, was recently promoted.
	\item {\bf articles.} The words {\it a, an}, \& {\it the}, which signal or introduce nouns.
	
	The definite article {\it the} refers to a particular item: {\it the} report.
	
	The indefinite articles {\it a} \& {\it an} refer to a general item or one not already mentioned: {\it an} apple.
	\item {\bf auxiliary verb.} A verb that combines with the main verb to show differences in tense, person, \& voice.
	
	The most common auxiliaries are forms of {\it be, do}, \& {\it have}.
	
	I {\it am} going; we {\it did} not go; they {\it have} gone. (See also {\it modal auxiliaries}.)
	\item {\bf case.} The form of a noun or pronoun that reflects its grammatical function in a sentence as subject ({\it they}), object ({\it them}), or possessor ({\it their}).
	
	{\it She} gave {\it her} employees a raise that pleased {\it them} greatly.
	\item {\bf clause.} A group of related words that contains a subject \& predicate.
	
	{\it Moths swarm} around a burning candle.
	
	While {\it she was taking} the test, {\it Karen muttered} to herself.
	\item {\bf colloquialism.} A word or expression appropriate to informal conversation but not usually suitable for academic or business writing.
	
	They wanted to {\it get even} (instead of they wanted to {\it retaliate}).
	\item {\bf complement.} A word or phrase (especially a noun or adjective) that completes the predicate.
	\begin{itemize}
		\item {\bf Subject complements} complete linking verbs \& rename or describe the subject: Martha is my {\it neighbor}.
		
		She seems {\it shy}.
		\item {\bf Object complements} complete transitive verbs by describing or renaming the direct object: They found the play {\it exciting}.
		
		Robert considers Mary {\it a wonderful wife}.
	\end{itemize}
	\item {\bf compound sentence.} 2 or more independent clauses joined by a coordinating conjunction, a correlative conjunction, or a semicolon.
	
	{\it Caesar conquered Gaul}, but {\it Alexander the Great conquered the world}.
	\item {\bf compound subject.} 2 or more simple subjects joined by a coordinating or correlative conjunction.
	
	{\it Hemingway \& Fitzgerald} had little in common.
	\item {\bf conjunction.} A word that joins words, phrases, clauses, or sentences.
	
	The coordinating conjunctions, {\it and, but, or, nor, yet, so, for}, join grammatically equivalent elements.
	
	Correlative conjunctions ({\it both, and; either, or; neither, nor}) join the same kinds of elements.
	\item {\bf contraction.} A shortened form of a word or group of words: {\it can't} for cannot; {\it they're} for they are.
	\item {\bf correlative expression.} See {\it conjunction}.
	\item {\bf dependent clause.} A group of words that includes a subject \& verb but is subordinate to an independent clause in a sentence.
	
	Dependent clauses begin with either a subordinating conjunction, e.g., {\it if, because, since}, or a relative pronoun, e.g., {\it who, which, that}.
	
	{\it When it gets dark}, we'll find a restaurant {\it that has music}.
	\item {\bf direct object.} A noun or pronoun that receives the action of a transitive verb.
	
	Pearson publishes {\it books}.
	\item {\bf gerund.} The {\it -ing} form of a verb that functions as a noun: {\it Hiking} is good exercise.
	
	She was praised for her {\it playing}.
	\item {\bf indefinite pronoun.} A pronoun that refers to an unspecified person ({\it anybody}) or thing ({\it something}).
	\item {\bf independent clause.} A group of words with a subject \& verb that can stand alone as a sentence.
	
	{\it Raccoons steal food}.
	\item {\bf indirect object.} A noun or pronoun that indicates to whom or for whom, to what or for what the action of a transitive verb is performed.
	
	I asked {\it her} a question.
	
	Ed gave {\it the door} a kick.
	\item {\bf infinitive/split infinitive.} In the present tense, a verb phrase consisting of {\it to} followed by the base form of the verb ({\it to write}).
	
	A split infinitive occurs when 1 or more words separate {\it to} \& the verb ({\it to boldly go}).
	\item {\bf intransitive verb.} A verb that does not take a direct object.
	
	His {\it nerve failed}.
	\item {\bf linking verb.} A verb that joins the subject of a sentence to its complement.
	
	Prof. Chapman {\it is a} philosophy teacher.
	
	They {\it were} ecstatic.
	\item {\bf loose sentence.} A sentence that begins with the main idea \& then attaches modifiers, qualifiers, \& additional details: He was determined to succeed, with or without the promotion he was hoping for \& in spite of the difficulties he was confronting at every turn.
	\item {\bf main clause.} An independent clause, which can stand alone as a grammatically complete sentence.
	
	Grammarians quibble.
	\item {\bf modal auxiliaries.} Any of the verbs that combine with the main verb to express obligation ({\it must}), necessity ({\it should}), permission ({\it may}), probability ({\it might}), possibility ({\it could}), ability ({\it can}), or tentativeness ({\it would}).
	
	{\it Mary might} wash the car.
	\item {\bf modifier.} A word or phrase that qualifies, describes, or limits the meaning of a word, phrase, or clause.
	
	{\it Frayed} ribbon, {\it dancing} flowers, {\it worldly} wisdom.
	\item {\bf nominative pronoun.} A pronoun that functions as a subject or subject complement: {\it I, we, you, he, she, it, they, who}.
	\item {\bf nonrestrictive modifier.} A phrase or clause that does not limit or restrict the essential meaning of the element it modifies.
	
	My youngest niece, {\it who lives in Ann Arbor}, is a magazine editor.
	\item {\bf noun.} A word that names a person, place, thing, or idea.
	
	Most nouns have a plural form \& a possessive form.
	
	{\it Carol}; the {\it park}; the {\it cup}; {\it democracy}.
	\item {\bf number.} A feature of nouns, pronouns, \& a few verbs, referring to singular or plural.
	
	A subject \& its corresponding verb must be consistent in number; a pronoun should agree in number with its antecedent.
	
	A {\it solo flute plays}; 2 {\it oboes join} in.
	\item {\bf object.} The noun or pronoun that completes a prepositional phrase or the meaning of a transitive verb.
	
	(See also {\it direct object, indirect object}, \& {\it preposition}.)
	
	Frost offered {\it his audience a poetic performance} they would likely never forget.
	\item {\bf participial phrase.} A present or past participle with accompanying modifiers, objects, or complements.
	
	The buzzards, {\it circling with sinister determination}, squawked loudly.
	\item {\bf participle.} A verb that functions as an adjective.
	
	Present participles end in {\it -ing} ({\it brimming}); past participles typically end in {\it -d} or {\it -ed} ({\it injured}) or {\it -en} ({\it broken}) but may appear in other forms ({\it brought, been, gone}).
	\item {\bf periodic sentence.} A sentence that expresses the main idea at the end.
	
	With or without their parents' consent, \& whether or not they receive the assignment relocation they requested, {\it they are determined to get married}.
	\item {\bf phrase.} A group of related words that functions as a unit but lacks a subject, a verb, or both.
	
	{\it Without the resources to continue}.
	\item {\bf possessive.} The case of nouns \& pronouns that indicates ownership or possession ({\it Harold's, ours, mine}).
	\item {\bf predicate.} The verb \& its related words in a clause or sentence.
	
	The predicate expresses what the subject does, experiences, or is.
	
	{\it Birds fly}.
	
	{\it The partygoers celebrated wildly for a long time}.
	\item {\bf preposition.} A word that relates its object (a noun, pronoun, or {\it -ing} verb form) to another word in the sentence.
	
	She is the leader {\it of} our group.
	
	We opened the door {\it by} picking the lock.
	
	She went {\it out} the window.
	\item {\bf prepositional phrase.} A group of words consisting of a preposition, its object, \& any of the object's modifiers.
	
	Georgia {\it on my mind}.
	\item {\bf principal verb.} The predicating verb in a main clause or sentence.
	\item {\bf pronominal possessive.} Possessive pronouns, e.g., {\it hers, its}, \& {\it theirs}.
	\item {\bf proper noun.} The name of a particular person ({\it Frank Sinatra}), place ({\it Boston}), or thing ({\it Moby Dick}).
	
	Proper nouns are capitalized.
	
	Common nouns name classes of people ({\it singers}), places ({\it cities}), or things ({\it books}) \& are not capitalized.
	\item {\bf relative clause.} A clause introduced by a relative pronoun, e.g., {\it who, which, that}, or by a relative adverb, e.g., {\it where, when, why}.
	\item {\bf relative pronoun.} A pronoun that connects a dependent clause to a main clause in a sentence: {\it who, whom, whose, which, that, what, whoever, whomever, whichever}, \& {\it whatever}.
	\item {\bf restrictive term, element, clause.} A phrase or clause that limits the essential meaning of the sentence element it modifies or identifies.
	
	Professional athletes {\it who perform exceptionally} should earn stratospheric salaries.
	
	Since there are no commas before \& after the italicized clause, the italicized clause is restrictive \& suggests that only those athletes who perform exceptionally are entitled to such salaries.
	
	If commas were added before {\it who} \& after {\it exceptionally}, the clause would be nonrestrictive \& would suggest that {\it all} professional athletes should receive stratospheric salaries.
	\item {\bf sentence fragment.} A group of words that is not grammatically a complete sentence but is punctuated as one: {\it Because it mattered greatly}.
	\item {\bf subject.} The noun or pronoun that indicates what a sentence is about, \& which the principal verb of a sentence elaborates.
	
	{\it The new Steven Spielberg movie} is a box office hit.
	\item {\bf subordinate clause.} A clause dependent on the main clause in a sentence.
	
	{\it After we finish our work}, we will go out for dinner.
	\item {\bf syntax.} The order or arrangement of words in a sentence.
	
	Syntax may exhibit parallelism ({\it I came, I saw, I conquered}), inversion ({\it Whose woods these are I think I know}), or other formal characteristics.
	\item {\bf tense.} The time of a verb's action or state of being, e.g., past, present, or future.
	
	{\it Saw, see, will see}.
	\item {\bf transition.} A word or group of words that aids coherence in writing by showing the connections between ideas.
	
	William Carlos Williams was influenced by the poetry of Walt Whitman.
	
	{\it Moreover}, Williams's emphasis on the present \& the immediacy of the ordinary represented a rejection of the poetic stance \& style of his contemporary T. S. Eliot.
	
	{\it In addition}, William's poetry$\ldots$
	\item {\bf transitive verb.} A verb that requires a direct object to complete its meaning: They {\it washed} their new car.
	
	An {\it intransitive verb} does not require an object to complete its meaning: The audience {\it laughed}.
	
	Many verbs can be both: The wind {\it blew} furiously.
	
	My car {\it blew} a gasket.
	\item {\bf verb.} A word or group of words that expresses the action or indicates the state of being of the subject.
	
	\fbox{Verbs {\it activate} sentences.}
	\item {\bf verbal.} A verb form that functions in a sentence as a noun, an adjective, or an adverb rather than as a principal verb.
	
	{\it Thinking} can be fun.
	
	An {\it embroidered} handkerchief.
	
	(See {\it also gerund, infinitive}, \& {\it participle}.)
	\item {\bf voice.} The attribute of a verb that indicates whether its subject is active (Janet {\it played} the guitar) or passive (The guitar {\it was played} by Janet).\hfill$\square$
\end{enumerate}

%------------------------------------------------------------------------------%

\section{{\sc David Foster Wallace}. This Is Water: Some Thoughts, Delivered on a Significant Occasion, about Living a Compassionate Life}
\textbf{\textsf{Resources -- Tài nguyên.}}
\begin{enumerate}
	\item \cite{Wallace_water}. {\sc David Foster Wallace}. {\it This Is Water: Some Thoughts, Delivered on a Significant Occasion, about Living a Compassionate Life}
\end{enumerate}
``This Is Water

There are these 2 young fish swimming along \& they happen to meet an older fish swimming the other way, who nods at them \& says, ``Morning, boys, How's the water?''

\& the 2 young fish swim on for a bit, \& then eventually 1 of them looks at the other \& goes, ``What the hell is water?''

This is a standard requirement of US commencement speeches, the deployment of didactic little parable-ish stories.

The story thing turns out to be 1 of the better, less bullshitty conventions of the genre $\ldots$ but if you're worried that I plan to present myself here as the wise old fish explaining what water is to you younger fish, please don't be.

I am not the wise old fish.

The immediate point of the fish story is merely that the most obvious, ubiquitous, important realities are often the ones that are hardest to see \& talk about.

Stated as an English sentence, of course, this is just a banal platitude -- but the fact is that, in the day to day trenches of adult existence, banal platitudes can have a life-or-death importance.

Or so I wish to suggest to you on this dry \& lovely morning.

Of course the main requirement of speeches like this is that I'm supposed to talk about your liberal arts education's meaning, to try to explain why the degree you're about to receive has actual human value instead of just a material payoff.

So let's talk about the single most pervasive clich\'e in the commencement speech genre, which is that a liberal arts education is not so much about filling you up with knowledge as it is about, quote, ``teaching you how to think.''

If you're like me as a college student, you've never liked hearing this, \& you tend to feel a bit insulted by the claim that you've needed anybody to teach you how to think, since the fact that you even got admitted to a college this good seems like proof that you already know how to think.

But I'm going to posit to you that the liberal arts clich\'e turns out not to be insulting at all, because the really significant education in thinking that we are supposed to get in a place like this isn't really about the capacity to think, but rather about the choice of what to think about.

If your complete freedom of choice regarding what to think about seems too obvious to waste time talking about, I'd ask you to think about fish \& water, \& to bracket, for just a few minutes, your skepticism about the value of the totally obvious.

Here's another didactic little story.

There are these 2 guys sitting together in a bar in the remote Alaskan wilderness.

1 of the guys is religious, the other's an atheist, \& they're arguing about the existence of God with that special intensity that comes after abut the 4th beer.

\& the atheist says, ``Look, it's not like I don't have actual reasons for not believing in God.

It's not like I haven't ever experimented with the whole God-\&-prayer thing.

Just last month, I got caught off away from the camp in that terrible blizzard, \& I couldn't see a thing, \& I was totally lost, \& it was 50 below, \& so I did, I tried it: I fell to my knees in the snow \& cried out, `God, if there is a God, I'm lost in this blizzard, \& I'm gonna die if you don't help me!''

\& now, in the bar, the religious guy looks at the atheist all puzzled: ``Well then, you must believe now,'' he says. ``After all, here you are, alive.''

The atheist rolls his eyes like the religious guy is a total simp: ``No, man, all that happened was that a couple Eskimos just happened to come wandering by, \& they showed me the way back to the camp.''

It's easy to run this story through a kind of standard liberal arts analysis: The exact same experience can mean 2 completely different things to 2 different people, given those people's 2 different belief templates \& 2 different ways of constructing meaning from experience.

Because we prize tolerance \& diversity of belief, nowhere in our liberal arts analysis do we want to claim that one guy's interpretation is true \& the other guy's is false or bad.

Which is fine, except we also never end up talking about just where these individual templates \& beliefs come from, meaning, where they come from {\it inside} the 2 guys.

As if a person's most basic orientation toward the world \& the meaning of his experience were somehow automatically hardwired, like height or shoe size, or absorbed from the culture, like language.

As if how we construct meaning were not actually a matter of personal, intentional choice, of conscious decision.

Plus, there's the matter of arrogance.

The nonreligious guy is so totally, obnoxiously confident in his dismissal of the possibility that the Eskimos had anything to do with his prayer for help.

True, there are plenty of religious people who seem arrogantly certain of their own interpretations, too.

They're probably even more repulsive than atheists, at least to most of us here, but the fact is that religious dogmatists' problem is exactly the same as the story's atheist's -- arrogance, blind certainty, a closed-mindedness that's like an imprisonment so complete that the prisoner doesn't even know he's locked up.

The point here is that I think this is 1 part of what the liberal arts mantra of ``teaching me how to think'' is really supposed to mean: To be just a little less arrogant, to have some ``critical awareness'' abut myself \& my certainties $\ldots$ because a huge percentage of the stuff that I tend to be automatically certain of is, it turns out, totally wrong \& deluded.

I have learned this the hard way, as I predict you graduates will, too.

Here's 1 example of the utter wrongness of something I tend to be automatically sure of.

Everything in my own immediate experience supports my deep belief that I am the absolute center of the universe, the realest, most vivid \& important person in existence.

We rarely think about this sort of natural, basic self-centeredness, because it's so socially repulsive, but it's pretty much the same for all of us, deep down.

It is our default setting, hardwired into our boards at birth.

Think about it: There is no experience you've had that you were not at the absolute center of.

The world as you experience it is there in front of you, or behind you, to the left or right of you, on your TV, on your monitor, or whatever.

Other people's thoughts \& feelings have to be communicated to you somehow, but your own are so immediate, urgent, {\it real}.

You get the idea.

But please don't worry that I'm getting ready to preach to you about compassion or other-directedness or all the so-called ``virtues.''

This is not a matter of virtue -- It's a matter of my choosing to do the work of somehow altering or getting free of my natural, hardwired default setting, which is to be deeply \& literally self-centered, \& to see \& interpret everything through this lens of self.

People who {\it can} adjust their natural default setting this way are often described as being, quote, ``well-adjusted,'' which I suggest to you is not an accidental term.

Given the academic setting here, an obvious question is how much of this work of adjusting our default setting involves actual knowledge or intellect.

The answer, not surprisingly, is that it depends on what kind of knowledge we're talking about.

Probably the most dangerous thing about an academic education, at least in my own case, is that it enables my tendency to over-intellectualize stuff, to get lost in abstract thinking instead of simply paying attention to what's going on in front of me.

Instead of paying attention to what's going on {\it inside} me.

As I'm sure you guys know by now, it is extremely difficult to stay alert \& attentive instead of getting hypnotized by the constant monologue inside your head.

What you don't yet know are the stakes of this struggle.

In the 20 years since my own graduation, I have come gradually to understand these stakes, \& to see that the liberal arts clich\'e about ``teaching you how to think'' was actually shorthand for a very deep \& important truth.

``Learning how to think'' really means learning how to exercise some control over {\it how} \& {\it what} you think.

It means being conscious \& aware enough to {\it choose} what you pay attention to \& to {\it choose} how you construct meaning from experience.

Because if you cannot or will not exercise this kind of choice in adult life, you will be totally hosed.

Think of the old clich\'e about the mind being ``an excellent servant but a terrible master.''

This, like many clich\'es, so lame \& banal on the surface, actually expresses a great \& terrible truth.

It is not the least bit coincidental that adults who commit suicide with firearms nearly always shoot themselves in $\ldots$ the {\it head}.

\& the truth is that most of these suicides are actually dead long before they pull the trigger.

\& I submit that this is what the real, no-shit value of your liberal arts education is supposed to be about: How to keep from going through your comfortable, prosperous, respectable adult life dead, unconscious, a slave to your head \& to your natural default setting of being uniquely, completely, imperially alone, day in \& day out.

That may sound like hyperbole, or abstract nonsense.

So let's get concrete.

The plain fact is that you graduating seniors do not yet have any clue what ``day in, day out'' really means.

There happen to be whole large parts of adult American life that nobody talks about in commencement speeches.

1 such part involves \fbox{boredom, routine, \& petty frustration.}

The parents \& older folks here will know all too well what I am talking about.

By way of example, let's say it's an average adult day, \& you get up in the morning, go to your challenging, white-collar college-graduate job, \& you work hard for 9--10 hours, \& at the end of the day you are tired, \& you're stressed out, \& all you want is to go home \& have a good supper \& maybe unwind for a couple hours \& then hit the rack early because you have to get up the next day \& do it all again.

But then you remember there's no food at home -- you haven't had time to shop this week because of your challenging job -- \& so now after work you have to get in your car \& drive to the supermarket.

It's the end of the workday, \& the traffic's very bad, so getting to the store takes way longer than it should, \& when you finally get there, the supermarket is very crowded, because of course it's the time of day when all the other people with jobs also try to squeeze in some grocery shopping, \& the store is hideously, fluorescently lit, \& infused with soul-killing Muzak or corporate pop, \& it's pretty much the last place you want to be, but you can't just get in \& quickly out.

You have to wander all over the huge, overlit store's crowded aisles to find the stuff you want, \& you have to maneuver your junky cart through all these other tired, hurried people with carts, \& of course there are also the glacially slow old people \& the spacey people \& the ADHD kids who all block the aisle, \& you have to grit your teeth \& try to be polite as you ask them to let you by, \& eventually, finally, you get all your supper supplies, except now it turns out there aren't enough checkout lanes open even though it's the end-of-the-day rush, so the checkout line is incredibly long.

Which is stupid \& infuriating, but you can't take your fury out on the frantic lady working the register, who is overworked at a job whose daily tedium \& meaninglessness surpass the imagination of any of us here at a prestigious college $\ldots$ but anyway, you finally get to the checkout line's front, \& you pay for your food, \& wait to get your check or card authenticated by a machine, \& you get told to ``Have a nice day'' in a voice that is the absolute voice of {\it death}.

\& then you have to take your creepy flimsy plastic bags of groceries in your cart with the one crazy wheel that pulls maddeningly to the left, all the way out through the crowded, bumpy, littery parking lot, \& try to load the bags in your car in such a way that everything doesn't roll out of the bags \& roll around in the trunk on the way home, \& then you have to drive all the way home through slow, heavy, SUV-intensive rush-hour traffic, et cetera, et cetera.

Everyone here has done this, of course -- but it hasn't yet been part of your graduates' actual life routine, day after week after month after year.

But it will be, \& many more dreary, annoying, seemingly meaningless routines besides $\ldots$

Except that's not the point.

The point is that petty, frustrating crap like this is exactly where the work of choosing comes in.

Because the traffic jams \& crowded aisles \& long checkout lines give me time to think, \& if I don't make a conscious decision about how to think \& what to pay attention to, I'm gonna be pissed \& miserable every time I have to food-shop, because my natural default setting is that situations like this are really all about {\it me}, about my hungriness \& my fatigue \& my desire to just get home, \& it's going to seem, for all the world, like everybody else is just {\it in my way}, \& who the fuck are all these people in my way?

\& look at how repulsive most of them are \& how stupid \& cow-like \& dead-eyed \& nonhuman they seem here in the checkout line, or at how annoying \& rude it is that people are talking loudly on cell phones in the middle of the line, \& look at how deeply unfair this is: I've worked hard all day \& I am starved \& tired \& I can't even get home to eat \& unwind because of all these stupid goddamn {\it people}.

Or, of course, if I'm in a more socially conscious, liberal arts form of my default setting, I can spend time in the end-of-the-day traffic jam being angry \& disgusted at all the huge, stupid, lane-blocking SUVs \& Hummers \& V-12 pickup trucks burning their wasteful, selfish, 40-gallon tanks of gas, \& I can dwell on the fact that the patriotic or religious bumper stickers always seem to be on the biggest, most disgustingly selfish vehicles driven by the ugliest, most inconsiderate \& aggressive drivers, who are usually talking on cell phones as they cut people off in order to get just 20 stupid feet ahead in the traffic jam, \& I can think about how our children's children will despise us for wasting all the future's fuel \& probably screwing up the climate, \& how spoiled \& stupid \& selfish \& disgusting we all are, \& how it all {\it sucks}, \& so on \& so forth $\ldots$

Look, if I choose to think this way, fine, lots of us do -- except that thinking this way tends to be so easy \& automatic it doesn't {\it have} to be a choice.

Thinking this way is my natural default setting.

It's the automatic, unconscious way that I experience the boring, frustrating, crowded parts of adult life when I'm operating on the automatic, unconscious belief that I am the center of the world \& that my immediate needs \& feelings are what should determine the world's priorities.

The thing is that there are obviously different ways to think about these kinds of situations.

In this traffic, all these vehicles stuck \& idling in my way: It's not impossible that some of these people in SUVs have been in horrible auto accidents in the past \& now find driving so traumatic that their therapist has all but ordered them to get a huge, heavy SUV so they can feel safe enough to drive; or that the Hummer that just cut me off is maybe being driven by a father whose little child is hurt or sick in the seat next to him, \& he's trying to rush to the hospital, \& he is in a way bigger, more legitimate hurry than I am -- it is actually {\it I} who aim in {\it his} way.

Or I can choose to force myself to consider the likelihood that everyone else in the supermarket's checkout line is probably just as bored \& frustrated as I am, \& that some of these people actually have much harder, more tedious or painful lives than I do, overall.

\& so on.

Again, please don't think that I'm giving you moral advice, or that I'm saying you are ``supposed to'' think this way, or that anyone expects you to just automatically do it, because it's hard, it takes will \& mental effort, \& if you're like me, some days you won't be able to do it, or else you just flat-out won't want to.

But most days, if you're aware enough to give yourself a choice, you can choose to look differently at this fat, dead-eyed, over-made-up lady who just screamed at her kid in the checkout line -- maybe she's not usually like this; maybe she's been up 3 straight nights holding the hand of her husband, who's dying of bone cancer, or maybe this very lady is the low-wage clerk at the motor vehicles department who just yesterday helped your spouse resolve a nightmarish red-tape problem through some small act of bureaucratic kindness.

Of course, none of this is likely, but it's also not impossible -- it just depends what you want to consider.

If you're automatically sure that you know what reality is \& who \& what is really important -- if you want to operate on your default setting -- then you, like me, probably will not consider possibilities that aren't pointless \& annoying.

But if you've really learned how to think, how to pay attention, then you will know you have other options.

It will actually be within your power to experience a crowded, hot, slow, consumer-hell-type situation as not only meaningful, but sacred, on fire with the same force that lit the stars -- compassion, love, the subsurface unity of all things.

Not that that mystical stuff's necessarily true: The only thing that's capital-T True is that you get to {\it decide} how you're going to try to see it.

This, I submit, is the freedom of real education, of learning how to be well-adjusted: You get to consciously decide what has meaning \& what doesn't.

You get to decide what to worship $\ldots$

Because here's something else that's true.

In the day-to-day trenches of adult life, there is actually no such thing as atheism.

There is no such thing as not worshiping.

Everybody worships.

The only choice we get is {\it what} to worship.

\& an outstanding reason for choosing some sort of god or spiritual-type thing to worship -- be it J.C. Or Allah, be it Yahweh or the Wiccan mother-goddess or the 4 Noble Truths or some infrangible set of ethical principles -- is that pretty much anything else you worship will eat you alive.

If you worship money \& things -- if they are where you tap real meaning in life -- then you will never have enough.

Never feel you have enough.

It's the truth.

Worship your own body \& beauty \& sexual allure \& you will always feel ugly, \& when time \& age start showing, you will die a million deaths before they finally plant you.

On 1 level we all know this stuff already -- it's been codified as myths, proverbs, clich\'es, bromides, epigrams, parables: the skeleton of every great story.

Worship your intellect, being seen as smart -- you will end up feeling stupid, a fraud, always on the verge of being found out.

\& so on.

Look, the insidious thing about these forms of worship is not that they're evil or sinful; it is that they are {\it unconscious}.

They are default settings.

They're the kind of worship you just gradually slip into, day after day, getting more \& more selective about what you see \& how you measure value without ever being fully aware that that's what you're doing.

\& the so-called ``real world'' will not discourage you from operating on your default settings, because the so-called ``real world'' of men \& money \& power hums along quite nicely on the fuel of fear \& contempt \& frustration \& craving \& the worship of self.

Our own present culture has harnessed these forces in ways that have yielded extraordinary wealth \& comfort \& personal freedom.

The freedom all to be lords of our tiny skull-sized kingdoms, alone at the center of all creation.

This kind of freedom has much to recommend it.

But of course there are all different kinds of freedom, \& the kind that is most precious you will not hear much talked about in the great outside world of winning \& achieving \& displaying.

The really important kind of freedom involves attention, \& awareness, \& discipline, \& effort, \& being able truly to care about other people \& to sacrifice for them, over \& over, in myriad petty little unsexy ways, every day.

That is real freedom.

That is being taught how to think.

The alternative is unconsciousness, the default setting, the ``rat race'' -- the constant, gnawing sense of having had \& lost some infinite thing.

I know that this stuff probably doesn't sound fun \& breezy or grandly inspirational the way a commencement speech's central stuff should sound.

What it is, so far as I can see, is the truth, with a whole lot of rhetorical bullshit pared away.

Obviously, you can think of it whatever you wish.

But please don't dismiss it as some finger-wagging Dr. Laura sermon.

None of this is about morality, or religion, or dogma, or big fancy questions of life after death.

The capital-T Truth is about life {\it before} death.

It is about making it to 30, or maybe even 50, without wanting to shoot yourself in the head.

It is about the real value of a real education, which has nothing to do with grades or degrees \& everything to do with simple awareness -- awareness of what is so real \& essential, so hidden in plain sight all around us, that we have to keep reminding ourselves over \& over:

``This is water.''

``This is water.''

``These Eskimos might be much more than they seem.''

It is unimaginably hard to do this -- to live consciously, adultly, day in \& day out.

Which means yet another clich\'e is true: Your education really {\it is} the job of a lifetime, \& it commences -- now.

I wish you way more than luck.

{\sc David Foster Wallace} wrote the acclaimed novels {\it Infinite Jest} \& {\it The Broom of the System} \& the story collections {\it Oblivion, Brief Interviews with Hideous Men}, \& {\it Girl with Curious Hair}. His nonfiction includes the essay collections {\it Consider the Lobster} \& {\it A supposedly Fun Thing I'll Never Do  Again}, \& the full-length work {\it Everything \& More}. He died in 2008.'' -- \cite{Wallace_water}

%------------------------------------------------------------------------------%


%------------------------------------------------------------------------------%

\section{{\sc William Zinsser}. On Well Writing}

%------------------------------------------------------------------------------%

\section{Medium}

\begin{itemize}
	\item Official website: \url{https://medium.com/}.
	\item {\sf Trick.} To access Medium website in Firefox: {\sf Settings} $\to$ Search DNS in the {\sf Search} box $\to$ Choose {\sf Eable DNS over HTTPS using}: change from {\sf Default Protection} $\to$ {\sf Increased Protection} or {\sf Max Protection} then {\it Choose provider: Cloudfare (Default)}. Then Medium can be accessed normally.
\end{itemize}

%------------------------------------------------------------------------------%

\subsection{\href{https://medium.com/mind-cafe/einsteins-formula-for-a-happy-life-b29aff61a9c7}{Einstein's Formula for a Happy Life: There's a science to everything}}

Everything we do in life is for a reason.

And so, quite naturally, for ages humans have wondered: \textit{For what reason do I exist?}

%
Some men chase women, some women chase money, some of both chase vodka with the soft drink of choice, yet every such chase leads to the same pot at the end of the rainbow: happiness.

\begin{quotation}
	``\textit{Happiness},'' \href{https://historyofeconomicthought.mcmaster.ca/aristotle/Ethics.pdf}{said Aristotle}, ``\textit{is the meaning and the purpose of life, the whole aim and end of human existence}.''
\end{quotation}
Armed with the above insight into the very reason for our existence, does not common-sense suggest it would be wise to inquire with the very man widely considered the smartest person in history?

%
If by chance you were to Google \textit{genius definition}, notice the following incredible inclusion:

\textbf{genius} [n]
\begin{enumerate}
	\item exceptional intellectual or creative power or other natural ability.
	
	``she was a teacher of genius''
	
	\textit{synonyms}: \textbf{brilliance}, great intelligence, great intellect, great ability, \textbf{cleverness}, brains, \textbf{erudition, wisdom, sagacity}, find mind, \textbf{wit, artistry, flair}, creative power, precocity, precociousness;
	\item an exceptionally intelligent person or one with exceptional skill in a particular area of activity.
	
	``a mathematical genius''
	
	\textit{synonyms}: brilliant person, mental giant, \textbf{mastermind}, Einstein, \textbf{intellectual, intellect, brain, highbrow, expert, master, artist, polymath}. 
\end{enumerate}
Doesn't seeing an actual person's last name - Einstein - listed along with abstract nouns stand out like seeing Shaquille O'Neal in a room filled with gymnasts?

%
In short, because writing is nothing but thinking on paper, to read the thoughts of another is to essentially converse with them.

And so, who better to ask about the secret to happiness than genius personified?

\subsubsection{Einstein's Formula for a Happy Life}
A few days before Einstein twirled into the Reaper's grim arms, his assistant - Dukas - found him in the hospital bed, ``in agony, unable to lift his head.''

%
Yet on the very next day, a mere 24 hours or so away from his death-day, Einstein ``asked Dukas to get him his glasses, papers, and pencil, and he proceeded to jot down a few calculations.''

%
``He worked as long as he could,'' \href{https://workbooks.colombo.ca/issue-37-15-april-2020/}{noted biographer} Walter Issacson, ``and when the pain got too great he went to sleep,'' for the final time.

Indeed, Einstein died doing the 1 thing he loved most - working.

Ah, circumstances reveal character!

%
``Genius is one percent inspiration,'' said Einstein, ``and 99 percent perspiration.''

Indeed, it's not by accident that no one has ever become great by accident.

After all, as \href{https://www.newyorker.com/magazine/1947/11/22/the-great-foreigner}{Einstein once noted}: ``Only a monomaniac gets what we commonly refer to as \textit{results}.''

%
Show me someone great and I'll show you someone obsessed.

Besides, what more is ``greatness'' than the child of an obsession?

%
For the above reason, when Einstein was asked for the secret to a happy life, though the questioner expected an answer long and sour, Einstein kept it short and sweet:
\begin{quotation}
	``\textit{If you want to live a happy life, tie it to a goal, not to people or things}.''
\end{quotation}
Here lies Einstein's formula for a happy life.

\subsubsection{Work - the Ultimate Shelter from Life's Storms}
Given that great minds think alike for the same reason passengers boarded the same train of thought inevitably end up at the same destination, we should hardly be surprised that Newton and Einstein - arguably the 2 greatest scientists in history - when confronted with life's storms both used the same formula for a happy life.

%
In 1665, the \href{https://www.historic-uk.com/HistoryUK/HistoryofEngland/The-Great-Plague/}{bubonic plague} struck London.

Never before nor since has the world seen anything like the deadly pandemic. And just as a pandemic shut down today's stores and universities, the same held true back then.

History repeats itself indeed.

%
While the world's stage was seemingly crumbling underneath his feet, Isaac Newton applied the formula for a happy life.

How?

He merely tied his life to an abstract goal, not to physical people or concrete things.

Over the span of roughly 18 months, Newton would revolutionize science.

%
Newton was later asked how he discovered the law of gravity.

He replied:
\begin{quotation}
	``\textit{By thinking about it all the time}.''
\end{quotation}
In short, because Newton had hitched his star to an abstract goal, which remains forever fixed, and not to life - which is forever unpredictable - he in effect found shelter from life's storms.

%
When Einstein's wife Elsa died, needless to say - he was crushed.

After all, according to Isaacson, Elsa served as a somewhat maternal figure to Einstein.

%
``She told him when to eat and where to go,'' \href{https://onlinereadfreenovel.com/walter-isaacson/page,171,42091-walter_isaacson_great_innovators_e-book_boxed_set.html}{Isaacson notes}.

``She packed his suitcases and doled out his pocket money. In public, she was protective of the man she called `the Professor.'''

%
Fortunately for Einstein, his formula for a happy life served as shelter from the storm.

\begin{quotation}
	``\textit{As long as I am able to work, I must not and will not complain, because work is the only thing that gives substance to life}.''
\end{quotation}

\subsubsection{In Closing: Dream With Mind $+$ Chase with Body $=$ the Formula for a Happy Life}
When the holy man was asked does he feel good due to having given up all the worldly pleasures, he answered: ``It's not so much that I feel good, but rather I no longer feel bad.''

%
What Einstein fully grasped - regarding his concise formula for a happy life - appears to boil down to this:

\begin{quotation}
	\textit{``Happiness'' can be likened to a supermodel, all glammed up for the runway. Contentment, however, is that same model when she returns home at night and then removes all the makeup}.
\end{quotation}
Or to put it another way, ``He is richest,'' said Socrates, ``who is content with the least.''

%
Indeed, the best things in life are not only free but they're not even ``things.''

After all, never has a hand touched \textit{love}.

Never has an eye spotted \textit{peace}.

Never has a nose sniffed \textit{dreams}.

Here lies the DNA of Einstein's formula for a happy life.

%
From aardvark to zebra, we humans are the only creatures to have ever graced the world's stage that can ``dream.''

To picture a goal with our 3rd eye coupled with pouring our heart and soul into manifesting it before our 2 eyes is the very reason for this magical gift called a ``mind.''

%
The above insight may have best been summed up by a character in the classic play \textit{The Secret of Freedom}:

\begin{quotation}
	``\textit{The only thing about a man that is man is his mind. Everything else you can find in a pig or a horse}.''
\end{quotation}
Perhaps there's much truth in the saying that we humans only value things we pay for.

%
Like those other abstract pleasures which came freely packaged at birth, such as \textit{love} and \textit{faith}, perhaps we take for granted our gift for dreaming up worthwhile goals, not to mention the ability to wed our lives to such dreaming.

%
Because example is better than precept, perhaps the occasion calls for ending this piece by recounting a real-life instance of how using Einstein's formula for a happy life sheltered 1 of my good friends from a brutal storm.

%
1 of my closest friends in college endured the ultimate heartbreak.

Her high school sweetheart and fiancé left her for another woman.

To say my pal was devastated would be an understatement.

Yet somehow$\ldots$ someway she embodied Maya Angelou's grand vision of a \href{https://www.poetryfoundation.org/poems/48985/phenomenal-woman}{Phenomenal Woman}.

%
How?

%
She simply rose from the canvas of doubt and fear, dusted herself off, and then poured all her time and energies into her studies.

A year or so later she got accepted into Loyola Law School.

%
Today, she's not only happily married but also a successful attorney.

In short, my pal used Einstein's formula for a happy life.

\subsubsection{The Takeaway}
Because Einstein was a mathematician at heart, he couldn't help but reduce the odds of achieving happiness down to an equation.

\begin{quotation}
	``\textit{You have to learn the rules of the game},'' he concluded, ``\textit{and then you have to play better than anyone else}.''
\end{quotation}
Dear reader, so far as the Game of Life is concerned, it appears the rules are such as to demand that we - the only creatures armed with an imagination - use it specifically to envision an abstract goal, better known as a ``dream.''

%
Perhaps the above insight explains why Einstein went so far as to have called ``\textit{imagination more important than knowledge}.''

In short, though wedding your life to dream-chasing may not sound like the most glorious endeavor, as Einstein noted - such a course is nevertheless your safest bet.

%
Most of all, given that no mortal knows for certain what the next moment has in store - from triumph to tragedy - wisdom dictates treating your life as a boat and your goal as an anchor, which will, in turn, lend stability during life's storms.

%
In short, Einstein's formula for a happy life can best be summed as follows:

\begin{quotation}
	\textit{Find out what you do best; then, find out how you can pay most of your attention to doing it; then, get someone to pay you for having paid most of your attention to mastering that 1 thing}.
\end{quotation}

\subsubsection{Sources}

\begin{enumerate}
	\item Bartlett, Robert (2012). \textit{Aristotle's Nicomachean Ethics}.
	\item Isaacson, Walter (2007). \textit{Einstein: His Life and Universe}.
	\item Tucci, Niccolo (1947). \textit{The Great Foreigner}.\hfill$\square$
\end{enumerate}

%------------------------------------------------------------------------------%

\subsection{\href{https://psiloveyou.xyz/why-mozart-called-love-the-key-to-becoming-a-genius-f2141ac4c9ae}{Why Mozart Called `Love' the Key to Becoming a Genius}}

\begin{quotation}
	``\textit{Love, love, love, that is the soul of genius}.'' - Mozart
\end{quotation}

\subsubsection{The Michelangelo Hotel}
About a year or so ago, an old friend let me know she'd be in town for the weekend.

She wanted to go sight-seeing in what she dubbed ``the city so nice they had to name it twice.''

%
``Gotcha!'' I said, readying myself to play tour guide.

%
As Fate would have it, my old pal would be staying in the Michelangelo Hotel.

%
As for why the namesake of this particular hotel was of special interest to me, ahem, just know Michelangelo is not only my fellow Pisces - long said to be the sign of the ``genius/weirdo,'' ranging from Einstein to Jobs - but he also embraced the view:

\begin{quotation}
	``\textit{If you knew how much work went into it, you wouldn't call it genius}.'' - Michelangelo
\end{quotation}
My friend told me to ``come up'' while she got dressed.

I respectfully declined.

%
``No, ma'am,'' I texted back. ``I'll wait in the lobby.''

\subsubsection{Every Child is Partially a Genius}
As I reclined in the comfy lounge chair, I spotted a little girl just up ahead.

And she, apparently, spotted a piano.

%
I readied myself to take notes.

After all, I'd learned long ago from another fellow Pisces - the philosopher Schopenhauer - that

\begin{quotation}
	``\textit{Every child is in a way a genius, and every genius is in a way a child}.''
\end{quotation}
With greedy eye, I watched the little girl stroll over to meet this big ol' box of wood.

I'm sure Mr. Piano welcomed his latest friendly visitor.

%
For roughly a minute or so, the child just sat and sat$\ldots$

%
She said not a word.

She merely stared at the motionless piano, as if seated bedside with an ailing friend. And then$\ldots$ and then - it happened!

%
Her tiny hands slowly touched a piano key.

And oh! the beautiful voice with which the wood expressed delight in response to the slightest of touch.

%
So sensitive$\ldots$ so true!

%
With her trembling hands lingering just above the keyboard, seemingly unsure of how to give expression to the urge, I pictured the little girl overhearing in her 3rd ear what Mozart must've heard \href{https://www.classicfm.com/composers/mozart/guides/first-composition-minuet-trio/#:~:text=His%20first%20documented%20composition%2C%20a,was%20just%20five%20years%20old.}{during his 1st composition} at just 5 years old, a Minuet and Trio in G major.

%
The little girl longed to re-create before her 2 ears the magical creation playing in her third ear.

Or at least I fancied she did.

%
Though the child hadn't yet acquired the skill for the task, the longing persisted.

Longing of this sort is good for the child at heart.

To be exact, such instinctive urges are the very voice of genius.

%
If only this precious child would've laid the offering before its shrine, if only$\ldots$

%
Ahhh, if by chance that child were to pursue the inspiration in good faith for all the remaining days of her life, if she were to \textbf{love the piano} like a caring mother does her only child, in time that pulsating genius would surely breathe thru her perception while moving thru her tiny hands - as did it Mozart's, as did it Chopin's, as did it Monk's.

%
In short,

\begin{quotation}
	\textit{What play is to the child, work is to the genius: both forms of re-creation rely heavily on imagination, solitude and love}.
\end{quotation}

\subsubsection{Genius $=$ a Form of Love}

\begin{quotation}
	``\textit{Neither a lofty degree of intelligence nor imagination nor both together go to the making of genius},'' Mozart declared. ``\textit{Love, love, love - that is the soul of genius}.''
\end{quotation}
Of all the words known to the King's English, there may be no 2 greater misunderstood words than \textit{love} and \textit{genius}.

%
``\textit{I love me some ice cream},'' I heard someone say the other day.

%
\textit{Wow}, I recall thinking.

\textit{So what word would she use to describe the emotion felt upon 1st seeing the smiling face of her newborn?}

After all, I doubted she would've knowingly equated the purest emotion of all with the taste of flavored, frozen cow's milk.

%
As for the word ``genius,'' whenever someone refers to me as being one, I'm quick to remind them: the word \textit{genius} is a verb, not a noun.

And the same holds true for \textit{love}.

%
Bingo!

\begin{quotation}
	``\textit{Genius is 1 percent talent},'' said Einstein, ``\textit{and 99 percent hard work}.''
\end{quotation}
Indeed, it's not by accident that no one has ever tasted genius by accident.

After all, in the Game of Life, as Einstein once noted: ``\textit{Only a monomaniac gets results}.''

%
Show me a genius and I'll show you a fanatic!

Besides, is not all love a form of obsession?

%
For the above reason, when a friend once told me to be careful because genius is only 1 step from insanity, without blinking I shot back:

\begin{quotation}
	``\textit{Ahem, but you still haven't told me if it was the step before or the step after?}''
\end{quotation}

\begin{quotation}
	``\textit{Got me looking so crazy in love},'' Beyoncé once sang.
\end{quotation}
Just as the girl in love with the boy can't turn her deep feelings of attachment on and off, genius doesn't come with a light switch.

%
Perhaps genius comes packaged as a gift and a curse.

After all, for ages it has been said,
\begin{quotation}
	``\textit{Genius is an infinite capacity for taking pains}.''
\end{quotation}
I recall my younger cousin, Don, once remarking he wanted to follow in my footsteps.

%
``\textit{Cool},'' I said.

``\textit{Just remember - no women, no booze, no weed}.''

I flashed a smile before continuing.

``\textit{Also, eat only the healthiest of foods, take brain supplements daily, run several miles daily, meditate daily. And, most of all, never$\ldots$ ever skip a day of writing!}''

%
He cringed.

``Nah, forget about it, then. \textbf{It's not worth it}, bro.''

%
Indeed, the matter of ``worth'' lies at the root of why Mozart once griped that

\begin{quotation}
	``\textit{People err who think my art comes easily to me. I assure you, dear friend, nobody has devoted so much time and thought to compositions as I}.''
\end{quotation}
Is not the word ``devotion'' synonymous with \textit{love}?

%
In short, this strange love affair with one's work, or ``genius'' for shorthand, may best be summed as follows:

\begin{quotation}
	\textit{You can have ANYTHING you want but not EVERYTHING}.
\end{quotation}

\subsubsection{In Closing}
\textsf{Like an old boxer, my brittle hands can no longer endure the pounding and must be bandaged before working.}

%
``\textit{I take you, my blessed genius, to be my wife, to have and to hold from this day forward, for better, for worse, for richer, for poorer, in sickness and in health, to love and to cherish, till death do us part},'' I mumble each morning before starting the sacred writing session.

%
For well over a decade now, for poorer days at the start, to somewhat richer days now, I've never once strayed from my marriage vows \textit{to have} the pen and \textit{to hold} the sheet of paper$\ldots$ till death do us part.

%
In sickness a few years ago, bedridden with the flu, I crawled out of bed.

With trembling hands, I typed and typed while mumbling, ``\textit{Till death do us part}.''

%
Perhaps the greatest sacrifice of all was my last ``serious'' relationship, which ended amicably.

%
On that fateful evening when my then-girlfriend and I said goodbye for the last time, she tearfully read the following letter to me, some of which is paraphrased from a passage by Greer, which we both loved:

\begin{quotation}
	\it
	Living with a genius sometimes feels like living alone. Everything had to be sacrificed for the work. Everything!
	
	Plans had to be canceled, meals had to be delayed. The sleep schedule was his to make, and it was as often late nights as it was early mornings. The routine was the unborn child of the house - the routine$\ldots$ the routine$\ldots$ the routine. The morning coffee and old philosophy books and brain supplements, and ghostly silence throughout the apartment until noon.
	
	Could he be tempted by a morning stroll in the park? Maybe. But that morning stroll meant work undone, and suffering, suffering, suffering. The work is to him what the ``miracle baby'' is to the barren woman. To him - the routine is religious.
	
	I learned to never take it personally. I learned he could never love a woman the way he loved his work.
\end{quotation}
In short, the word ``genius'' is but a fancy word for sacrifice.

And because the very \textbf{definition of love is sacrifice}, Mozart concluded:

\begin{quotation}
	``\textit{Love, love, love - that is the soul of genius}.''\hfill$\square$
\end{quotation}

%------------------------------------------------------------------------------%

\subsection{\href{https://medium.com/personal-growth/the-most-important-skill-nobody-taught-you-9b162377ab77}{The Most Important Skill Nobody Taught You}}

Before dying at the age of 39, Blaise Pascal made huge contributions to both physics and mathematics, notably in fluids, geometry, and probability.

%
This work, however, would influence more than just the realm of the natural sciences.

Many fields that we now classify under the heading of social science did, in fact, also grow out of the foundation he helped lay.

%
Interestingly enough, much of this was done in his teen years, with some of it coming in his twenties.

As an adult, inspired by a religious experience, he actually started to move towards philosophy and theology.

%
Right before his death, he was hashing out fragments of private thoughts that would later be released as a collection by the name of \href{https://designluck.com/recommends/pascals-pensees/}{Pensées}.

%
While the book is mostly a mathematician's case for choosing a life of faith and belief, the more curious thing about it is its clear and lucid ruminations on what it means to be human.

It's a blueprint of our psychology long before psychology was deemed a formal discipline.

%
There is enough thought-provoking material in it to quote, and it attacks human nature from a variety of different angles, but one of its most famous thoughts aptly sums up the core of his argument:

\begin{quotation}
	``\textit{All of humanity's problems stem from man's inability to sit quietly in a room alone}.''
\end{quotation}
According to Pascal, we fear the silence of existence, we dread boredom and instead choose aimless distraction, and we can't help but run from the problems of our emotions into the false comforts of the mind.

%
The issue at the root, essentially, is that we never learn the art of solitude.

\subsubsection{The Perils of Being Connected}
Today, more than ever, Pascal's message rings true.

If there is one word to describe the progress made in the last 100 years, it's connectedness.

%
Information technologies have dominated our cultural direction.

From the telephone to the radio to the TV to the internet, we have found ways to bring us all closer together, enabling constant worldly access.

%
I can sit in my office in Canada and transport myself to practically anywhere I want through Skype.

I can be on the other side of the world and still know what is going on at home with a quick browse.

%
I don't think I need to highlight the benefits of all this.

But the downsides are also beginning to show.

Beyond the current talk about privacy and data collection, there is perhaps an even more detrimental side-effect here.

%
\textit{We now live in a world where we're connected to everything except ourselves}.

%
If Pascal's observation about our inability to sit quietly in a room by ourselves is true of the human condition in general, then the issue has certainly been augmented by an order of magnitude due to the options available today.

%
The logic is, of course, seductive.

Why be alone when you never have to?

%
Well, the answer is that never being alone is not the same thing as never feeling alone.

Worse yet, the less comfortable you are with solitude, the more likely it is that you won't know yourself.

And then, you'll spend even more time avoiding it to focus elsewhere.

In the process, you'll become addicted to the same technologies that were meant to set you free.

%
Just because we can use the noise of the world to block out the discomfort of dealing with ourselves doesn't mean that this discomfort goes away.

%
Almost everybody thinks of themselves as self-aware.

They think they know how they feel and what they want and what their problems are. But the truth is that very few people really do.

And those that do will be the 1st to tell how fickle self-awareness is and how much alone time it takes to get there.

%
In today's world, people can go their whole lives without truly digging beyond the surface-level masks they wear; in fact, many do.

%
We are increasingly out of touch with who we are, and that's a problem.

\subsubsection{Boredom as a Mode of Stimulation}
If we take it back to the fundamentals - and this is something Pascal touches on, too - our aversion to solitude is really an aversion to boredom.

%
At its core, it's not necessarily that we are addicted to a TV set because there is something uniquely satisfying about it, just like we are not addicted to most stimulants because the benefits outweigh the downsides. Rather, what we are really addicted to is a state of not-being-bored.

%
Almost anything else that controls our life in an unhealthy way finds its root in our realization that we dread the nothingness of nothing. We can't imagine just \textit{being} rather than \textit{doing}.

And therefore, we look for entertainment, we seek company, and if those fail, we chase even higher highs.

%
We ignore the fact that never facing this nothingness is the same as never facing ourselves.

And never facing ourselves is why we feel lonely and anxious in spite of being so intimately connected to everything else around us.

%
Fortunately, there is a solution.

The only way to avoid being ruined by this fear - like any fear - is to face it.

It's to let the boredom take you where it wants so you can deal with whatever it is that is really going on with your sense of self.

That's when you'll hear yourself think, and that's when you'll learn to engage the parts of you that are masked by distraction.

%
The beauty of this is that, once you cross that initial barrier, you realize that being alone isn't so bad.

Boredom can provide its own stimulation.

%
When you surround yourself with moments of solitude and stillness, you become intimately familiar with your environment in a way that forced stimulation doesn't allow.

The world becomes richer, the layers start to peel back, and you see things for what they really are, in all their wholeness, in all their contradictions, and in all their unfamiliarity.

%
You learn that there are other things you are capable of paying attention to than just what makes the most noise on the surface.

Just because a quiet room doesn't scream with excitement like the idea of immersing yourself in a movie or a TV show doesn't mean that there isn't depth to explore there.

%
Sometimes, the direction that this solitude leads you in can be unpleasant, especially when it comes to introspection - your thoughts and your feelings, your doubts and your hopes - but in the long-term, it's far more pleasant than running away from it all without even realizing that you are.

%
Embracing boredom allows you to discover novelty in things you didn't know were novel; it's like being an unconditioned child seeing the world for the first time.

It also resolves the majority of internal conflicts.

\subsubsection{The Takeaway}
The more the world advances, the more stimulation it will provide as an incentive for us to get outside of our own mind to engage with it.

%
While Pascal's generalization that a lack of comfort with solitude is the root of all our problems may be an exaggeration, it isn't an entirely unmerited one.

%
Everything that has done so much to connect us has simultaneously isolated us.

We are so busy being distracted that we are forgetting to tend to ourselves, which is consequently making us feel more and more alone.

%
Interestingly, the main culprit isn't our obsession with any particular worldly stimulation.

It's the fear of nothingness - our addiction to a state of not-being-bored.

We have an instinctive aversion to simply \textit{being}.

%
Without realizing the value of solitude, we are overlooking the fact that, once the fear of boredom is faced, it can actually provide its own stimulation.

And the only way to face it is to make time, whether every day or every week, to just sit - with our thoughts, our feelings, with a moment of stillness.

%
The oldest philosophical wisdom in the world has one piece of advice for us: \textit{know yourself}.

And there is a good reason why that is.

%
Without knowing ourselves, it's almost impossible to find a healthy way to interact with the world around us.

Without taking time to figure it out, we don't have a foundation to built the rest of our lives on.

%
Being alone and connecting inwardly is a skill nobody ever teaches us.

That's ironic because it's more important than most of the ones they do.

%
\textit{Solitude may not be the solution to everything, but it certainly is a start}.\hfill$\square$

%------------------------------------------------------------------------------%



%------------------------------------------------------------------------------%



%------------------------------------------------------------------------------%

\section{Read Station Vietnam -- Trạm Đọc VN}

\begin{itemize}
	\item Official website: \url{https://tramdoc.vn/}.
	\item \href{https://www.facebook.com/tramdoc.vn}{Facebook{\tt/}Trạm Đọc - Read Station}.
\end{itemize}

\subsection{Lời bênh vực cho sự buồn chán: 200 năm tư tưởng về giá trị của sự nhàn rỗi từ những bộ óc vĩ đại nhất nhân loại}
``{\it Cứ giữ bình tĩnh \& chết trong buồn chán}. ``Tôi có thể bỏ qua cho mọi thứ trừ sự buồn chán.'' {\sc Hedy Lamarr} nói một cách châm biếm. Thật hợp lý khi người phụ nữ đã phát minh ra công nghệ đặt nền tảng cho sự ra đời của wifi này lại là tác giả của câu khẩu hiệu của Thời đại Thông tin. Ngày nay, trong sự tôn sùng dành cho năng suất làm việc, chúng ta coi sự buồn chán là điều không thể bào chữa được -- điều có thể coi như tương đương với 1 tội lỗi giết người. Chúng ta tránh xa nó như thể việc bị bắt gặp trong trạng thái làm việc thiếu năng suất chính là 1 thất bại cá nhân sâu sắc. Chúng ta không thể làm gì cả, nói gì tới sẵn sàng tự làm tất cả mọi thứ 1 mình.

Tuy vật, buồn chán không chỉ là 1 thứ cảm xúc thiết yếu -- với những khả năng liên quan tới sự suy tư, cô độc, \& tĩnh lặng -- buồn chán có vai trò quan trọng đối với cả cuộc sống tâm trí cũng như linh hồn con người, cân bằng trong cả khoa học \& nghệ thuật.

Khi {\sc Jane Goodall} bắt đầu biến giấc mơ thời thơ ấu thành hiện thực, bà dành ra 3 năm lăn lộn trên đất để kiên nhẫn lặp lại 1 công việc đòi hỏi khả năng chịu đựng rất cao đối với sự buồn chán -- điều có thể coi là gốc rễ của nghệ thuật quan sát \& đặt nền tảng cho mọi nghiên cứu khoa học. Năng lực chịu đựng buồn chán cũng quan trọng như chính nghệ thuật vậy. Không có buồn chán, có lẽ sẽ không có những ``giấc mơ ban ngày'' hay những không gian hồi tưởng. Không có những ``giấc mơ ban ngày tích cực'' sẽ không có sự sáng tạo; không có hồi tưởng, chúng ta sẽ không thể ứng phó mà chỉ biết phản ứng tức thời.

{\it Buồn chán chính là không e sợ cuộc sống bên trong} -- 1 kiểu dũng cảm tinh thần căn bản của con người. 1 vài suy tư trường tồn \& sâu sắc nhất về sự buồn chán \& những phúc lành nghịch lý nó mang lại đã đọc được trong vài năm qua.
\begin{enumerate}
	\item {\sc Bertrand Russell.} Trong cuốn The Conquest of Happiness (tạm dịch: Chinh phục Hạnh phúc), con mắt nhìn xa trông rộng của nhà triết học người Anh {\sc Bertrand Russell} đã để ý tới vấn đề của sự buồn chán, tại sao sự kinh sợ của chúng ta với nó không khác nào 1 vết thương tự đâm, \& hành trình loại bỏ nó khỏi cuộc sống đồng thời lấy đi 1 vài năng lực tối quan trọng với chúng ta như thế nào. Trong chương ``Buồn chán \& hưng phấn'', {\sc Russell} viết:
	\begin{quote}
		``Chúng ta ít buồn chán hơn tổ tiên của mình, nhưng lại sợ điều đó hơn họ. Chúng ta dần nhận ra, hay đúng hơn là tin rằng, sự buồn chán không phải 1 phần bản chất của con người, \& điều đó có thể được tránh được nhờ những nỗ lực không ngừng nghỉ theo đuổi sự hưng phấn.''
	\end{quote}
	Ông có 1 ghi chép đặc biệt mang tính thời đại giải thích cách thức ``vòng xoay khoái lạc'' ({\it hedonic treadmill}) đã bám sâu thâm căn cố đế, \& ngày càng vô nghĩa như thế nào trong nỗ lực chạy trốn khỏi sự buồn chán của chúng ta:
	\begin{quote}
		``Càng phát triển về mặt xã hội, sự theo đuổi của chúng ta đối với sự hưng phấn càng trở nên mạnh mẽ. Người ta liên tục chuyển từ chỗ này sang chỗ khác nếu có thể, nhảy múa \& tiệc tùng trong hoan lạc, nhưng vì 1 lý do nào đó, lại luôn mong đợi có thể tận hưởng điều này nhiều hơn nữa ở 1 nơi chốn mới. Những người phải làm việc để kiếm sống nhận phần buồn chán của mình như 1 điều tất yếu trong khi làm việc, nhưng với những người dư giả đủ để giải phóng bản thân khỏi công việc, lý tưởng của họ chính là 1 cuộc sống hoàn toàn thoát khỏi sự buồn chán. Đó là 1 lý tưởng cao quý, \& dù không muốn chỉ trích, nhưng tôi e là nó cũng chỉ như những lý tưởng khác, khó đạt được hơn nhiều những gì người mơ mộng về nó đã tưởng tượng. Rốt cục, mọi buổi sáng đều nhạt nhẽo so với những buổi đêm hoan lạc hôm trước. Rồi cũng sẽ đến tuổi trung niên, có lẽ thậm chí là cao niên. Khi 20 tuổi, người ta nghĩ rằng đến 30 tuổi là hết tất cả $\ldots$ Thật không khôn ngoan chút nào khi người ta tiêu xài vốn liếng quan trọng cả đời như tiền bạc. Có lẽ 1 vài nhân tố buồn chán là nguyên liệu cần thiết cho cuộc sống. Mong ước trốn tránh khỏi sự buồn chán là điều tự nhiên; sự thật là mọi chủng tộc người đều thể hiện ra như vậy khi có cơ hội $\ldots$ Những cuộc chiến tranh, tàn sát, \& hành quyết đều là 1 phần của ``chuyến bay'' thoát khỏi sự buồn chán; thậm chí cãi nhau với hàng xóm còn tốt hơn là không làm gì cả. Vì vậy, sự buồn chán trở thành 1 vấn đề cốt lõi đối với những người giảng đạo đức, vì chí ít phải 1 nửa số tội lỗi của loài người là do sợ buồn chán gây ra.''
	\end{quote}
	Tuy tình trạng này rất đáng phê bình, {\sc Russell} vẫn ghi nhận những giá trị sống nó mang lại, \& vạch ra 2 nhóm buồn chán chính:
	\begin{quote}
		``Sự buồn chán, tuy vậy, không nên bị coi là hoàn toàn xấu. Có 2 loại buồn chán: ``Fructifying'' \& ``Stultifying''. Loại thứ nhất sinh ra do thiếu thuốc, còn loại thứ 2 sinh ra do không có những hoạt động cần thiết.''
	\end{quote}
	Việc chúng ta chạy trốn điên cuồng khỏi sự buồn chán, như ông cảnh báo, đã tạo ra 1 mối quan hệ nghịch lý với sự hưng phấn, khiến chúng ta nhanh chóng nghiện \& bị hiệu ứng của nó làm tê liệt: [Reason: Dopamine]
	\begin{quote}
		``Những gì áp dụng với thuốc, trong chừng mực nào đó, cũng áp dụng với mọi dạng hưng phấn. 1 cuộc sống quá nhiều hưng phấn sẽ khiến người ta kiệt sức, theo đó chúng ta sẽ liên tục cần tới những kích thích tố mạnh hơn để thấy hồi hộp, 1 cảm giác được cho rằng là 1 phần thiết yếu của khoái lạc. 1 người quá quen với sự hưng phấn sẽ giống như 1 con bệnh thèm [hạt] tiêu, cuối cùng biết lượng tiêu thế nào có thể khiến người khác phải nghẹn. Có 1 nhân tố của sự buồn chán không thể tách rời khỏi việc tránh hưng phấn quá mức, \& sự hưng phấn quá mức không chỉ làm hao mòn sức khỏe, mà còn làm mất dần cảm giác với mọi khoái lạc, thay thế những kích thích đối với khoái cảm tuyệt vời trong cơ thể, sự lanh lợi đối với trí tuệ, \& những ngạc nhiên sâu sắc với cái đẹp $\ldots$ 1 sức mạnh tất yếu của việc chịu đựng sự buồn chán, do đó, là thiết yếu đối với 1 cuộc sống hạnh phúc, \& đó là 1 trong những điều cần thiết phải dạy cho giới trẻ.''
	\end{quote}
	Thật vậy, việc nuôi dưỡng năng lực cốt lõi này từ sớm sẽ củng cố hệ thống miễn dịch tâm lý cho người trưởng thành. Gần 1 thế kỷ trước khi có iPad, vật giờ đã nhanh chóng được nhét vào đôi tay chỉ thèm lướt màn hình của những đứa trẻ mới chập chững biết đi luôn thấy buồn chán tới mức cáu kỉnh, {\sc Russell} đã viết:
	\begin{quote}
		``Khả năng chịu đựng 1 cuộc sống ít nhiều đơn điệu nên được hình thành từ thời thơ bé. Những ông bố bà mẹ hiện đại thật đáng trách trong khoản này; họ đưa cho con trẻ quá nhiều những thú vui bị động $\ldots$ \& không nhận ra tầm quan trọng của việc đứa trẻ cần có 1 ngày nào đó tương tự như những ngày khác, tất nhiên là ngoại trừ những dịp hiếm có nào đó.''
	\end{quote}
	Thay vào đó, {\sc Russell} hô hào các bậc làm cha làm mẹ cho trẻ tự do trải nghiệm ``sự đơn điệu có ích'', để khơi nguồn sáng tạo \& những trò chơi tưởng tượng -- nói cách khác, niềm vui thơ ấu lớn lao \& thành tự học tập có tính chất phát triển chính là việc ``không làm gì cả với bất cứ ai mà ngoài chính mình''. {\sc Russell} viết:
	\begin{quote}
		``Niềm vui thích thời thơ ấu nên là do đứa trẻ tự tạo ra từ môi trường quanh nó với 1 chút nỗ lực \& sáng tạo. Những vui thú tràn đầy hứng khởi nhưng đồng thời không đòi hỏi chút nỗ lực nào về thể chất, e.g., đi xem kịch, thì nên rất hạn chế. Sự hưng phấn, về bản chất, giống như thuốc, càng ngày sẽ càng cần nhiều hơn, \& sự thụ động về thể chất khi hưng phấn này hoàn toàn đi ngược lại tự nhiên. 1 đứa trẻ, giống như những cây non, sẽ phát triển tốt nhất khi được để tự nhiên. Quá nhiều chuyến đi \& quá nhiều ấn tượng mới sẽ không tốt cho đứa trẻ, khiến chúng lớn lên mà không có khả năng chịu đựng nổi 1 sự đơn điệu cần thiết để tạo ra thành công.
		
		Tôi không có ý nói rằng sự đơn điệu cũng có giá trị của chính nó; ý tôi chỉ là những điều tưởng như đương nhiên là tốt đó là không thể đạt được nếu không có 1 mức độ đơn điệu nhất định trong đó $\ldots$ 1 thế hệ không thể chịu đựng nổi sự buồn chán sẽ là 1 thế hệ của những kẻ tiểu nhân, tách biệt 1 cách không chính đáng với những quy trình chậm rãi của tự nhiên, với họ mỗi động lực quan trọng sẽ dần dần khô héo, như thể những bông hoa bị cắt đưa vào lọ.''
	\end{quote}
	\item {\sc Søren Kierkegaard.}Nhà triết học người Đan Mạch Søren Kierkegaard (5.5.1813 -- 11.11.1855) là người có hiểu biết sâu sắc \& tầm nhìn vượt thời gian. Gần 200 năm trước, ông đã có thể giải thích về những vấn đề phù hợp với cả ngày nay như tâm lý học về những trò đùa quá đà \& bắt nạt trên mạng ({\it online trolling \& bullying}), lý do chúng ta phục tùng [theo đám đông], \& nguyên nhân lớn nhất dẫn tới sự không hạnh phúc.
	
	Ông cũng hướng con mắt tinh thông của mình tới các vấn đề của sự buồn chán, trong 1 chương của tác phẩm của ông viết năm 1843 {\it Eithe{\tt/}Or: A Fragment of Life} (tạm dịch: {\it Thế này hay thế kia: Một mảnh của cuộc sống}), ông cho rằng sự buồn chán chính là 1 sự trống rỗng có tính tồn tại, được định nghĩa không phải bởi sự thiếu vắng các kích thích mà bởi sự thiếu vắng ý nghĩa -- điều có lẽ giải thích tại sao ngày nay, thay vì bất kỳ thời điểm nào khác trong lịch sử, chúng ta chỉ cảm thấy hưng phấn quá mức chứ không buồn chán. {\sc Kierkegaard} của tuổi 30 tiếc nuối cuộc sống ``hoàn toàn vô nghĩa'' của mình \& viết:
	\begin{quote}
		``Sự buồn chán có thể kinh khủng thế nào chứ -- buồn chán đến phát ngấy; tôi không thấy biểu hiện nào mạnh mẽ hơn, chân thật hơn, như thể chỉ được ghi nhận bởi thứ như $\ldots$ tôi nằm trơ bất động \& mệt mỏi; điều duy nhất nhìn thấy được là sự trống rỗng, điều duy nhất để tôi tiếp tục sống là sự trống rỗng \& điều duy nhất giúp tôi hòa nhập cuộc sống mới cũng vẫn là sự trống rỗng. Tôi thậm chí không biết tới đớn đau $\ldots$ Sự đau đớn không thể giúp tôi tỉnh táo lại. Nếu tôi được ban tặng tất cả những vinh quang hay tất cả sự dày vò trên thế giới, chẳng có lựa chọn nào có thể biến chuyển tôi nhiều hơn điều còn lại; tôi sẽ không quay sang phía còn lại để đón nhận hay trốn tránh. Tôi đang chết dần. \& liệu điều gì có thể làm tôi khuây khỏa? Có lẽ nếu tôi cố gắng nhìn ra sự chân thực trường tồn qua mọi thử thách, sự nhiệt huyết vượt lên tất cả, 1 niềm tin có thể dời núi lấp biển; nếu tôi có thể biết được 1 ý tưởng kết hợp cả những cái vô hạn \& hữu hạn.''
	\end{quote}
	Ông khai sáng chúng ta về sự tôn sùng với năng suất \& sự bận rộn không cưỡng lại được của chúng ta như 1 rào ngăn trước sự buồn chán kinh khủng kia:
	\begin{quote}
		``Buồn chán là gốc rễ của mọi điều xấu xa. Thật hiếu kỳ khi biết rằng sự buồn chán, với bản chất yên bình \& an nhiên của nó, lại có thể có năng lực khởi động. Hiệu ứng nó tạo nên thật kỳ diệu, tuy nhiên điều đó lại thật khó ưa chứ chẳng hấp dẫn gì.''
	\end{quote}
	Điều này giải thích tại sao mọi danh mục những điều dễ thương sinh ra từ việc thiết lập những trò tiêu khiển vật chất của BuzzWorthy thật vô nghĩa khi trấn an những tâm hồn đang kêu gào vì sự buồn chán khủng khiếp sinh ra từ sự thiếu vắng những việc có ý nghĩa, điều đúng ra vốn là nhiệm vụ của triết học. {\sc Alan Watts}, 1 nhà hiền triết khác của thời đại, gọi những chiến lược vô vọng để tiêu khiển này là những ``khoái cảm không cần giải phóng''. Lưu ý: những ``trò tiêu khiển sai lầm'' này chính là nguồn gốc tồn tại của sự buồn chán -- 1 kiểu ``mỳ ăn liền'' -- {\sc Kierkegaard} bổ sung:
	\begin{quote}
		``Thật khó tin khi chính liều thuốc cho sự buồn chán lại có thể sinh ra buồn chán, nhưng điều này chỉ đúng chừng nào liều thuốc đó bị dùng sai cách. 1 trò tiêu khiển kỳ quặc, sai lầm cũng ẩn chứa sự buồn chán trong chính nó, vì thế nó cứ vậy tiến triển \& mang trong mình tính thời đại.''
	\end{quote}
	Tuy vậy, sự buồn chán chính là bản chất của chúng ta, ông lập luận:
	\begin{quote}
		``Mọi cá nhân đều nhạt nhẽo. Chính từ đó nói lên 1 khả năng phân loại. ``Nhạt nhẽo'' có thể khiến người khác cũng như làm cho chính mình thấy buồn chán. Kiểu người nhạt nhẽo đầu tiên là nhóm người bình dân, là đám đông, là con tàu vô tận của loài người nói chung; còn kiểu người thứ 2 là nhóm người được chọn, giới quý tộc. Thật buồn cười làm sao khi những người không khiến mình thấy buồn chán lại thường làm người khác thấy phát ngấy; ngược lại, những người làm mình thấy buồn chán lại giúp người khác giải khuây.''
	\end{quote}
	Gợi lại chính lời phê bình của mình về sự bận rộn của con người thời nay như 1 đánh lạc hướng chú ý khỏi cuộc sống, ông bổ sung:
	\begin{quote}
		``Thường thì những người không khiến mình thấy buồn chán lại hay bận rộn trong việc này hay việc khác, nhưng chính họ chính là những người nhạt nhẽo nhất, khó có thể chịu đựng nổi nhất $\ldots$ Nhóm còn lại, cao cấp hơn, là những người tự khiến mình buồn chán $\ldots$ Thường thì họ khiến người khác vui vẻ, với đám đông \& đồng nghiệp của họ. Họ càng khiến mình chán bao nhiêu, mức độ tiêu khiển họ mang lại cho người khác lại càng mạnh mẽ \& tưowng tự, khi sự buồn chán đạt cực độ, họ hoặc sẽ chết vì nó (nhóm thụ động) hoặc bật ra khỏi sự tò mò (nhóm chủ động).''
	\end{quote}
	Nếu vậy, chúng ta phải làm gì để bảo vệ chính mình khỏi những điều xấu xa vô cùng do buồn chán? Đổi lại, {\sc Kierkegaard} đưa ra giá trị của sự ``lười nhác'' -- 1 khái niệm thường được ông sử dụng như từ ``tĩnh'' chúng ta sử dụng ngày nay, 1 tính chất cần thiết đối với sự tồn tại có nhận thức trong cuộc sống của chính chúng ta. Ông viết:
	\begin{quote}
		``Lười nhác không bao giờ là nguồn gốc của sự xấu xa; trái lại, cuộc sống không buồn chán mới thật sự là ``hoàn hảo'' $\ldots$ Sự lười nhác, vì vậy, thay vì là nguồn gốc của sự xấu xa, lại chính là sự tốt đẹp chân chính. Sự buồn chán là nguồn gốc của xấu xa: chính điều này mới phải gạt đi. Sự lười nhác không xấu; thật vậy, vì có thể nói rằng những người không biết thế nào là lười nhác thì chẳng phải người bình thường.''
	\end{quote}
	\item {\sc Arthur Schopenhauer.} Rất lâu trước khi thuật ngữ ``vòng xoay khoái lạc'' ({\it hedonic treadmill}) được đặt ra bởi các nhà tâm lý học đương đại để mô tả kiểu tiêu dùng không cưỡng lại được \& sau khi đến những ngưỡng thỏa mãn thì chúng lại mất hào quang thế nào, nhà triết học vĩ đại người Đức {\sc Arthur Schopenhauer} (22.2.1788--21.9.1860) đã tư duy về vai trò của sự buồn chán -- điều ông định nghĩa là ``cảm giác trống rỗng trong cuộc sống'' -- trong vũ điệu bất tận không biết đến điểm thỏa mãn của con người. Trong cuốn {\it The Essays of Schopenhauer} (tạm dịch: Các bài viết của Schopenhauer), tác phẩm cho thấy phong cách của Schopenhauer cũng như lời phê bình có tính tiên đoán của ông về tính đạo đức của việc xuất bản online, ông viết:
	\begin{quote}
		``Cuộc sống chính nó là 1 nhiệm vụ, đó là kiếm kế sinh nhai. Nếu điều này được giải quyết, thì những gì chúng ta đạt được đều là gánh nặng, \& chúng bao gồm cả nhiệm vụ thứ 2, chính là loại bỏ gánh nặng này để không bị buồn chán, giống như loài chim săn mồi luôn chực chờ nhắm xuống bất cứ vật sống nào bảo đảm cho ham muốn của nó. Như vậy nhiệm vụ đầu tiên chính là đạt được 1 điều gì đó, \& nhiệm vụ thứ 2 sau đó chính là quên đi thành tích đó để nó không trở thành gánh nặng.''
	\end{quote}
	Với tính bi quan tiêu biểu của mình, ông lập luận rằng sự thỏa mãn các nhu cầu lúc nào cũng đưa tới buồn chán \& ``sự buồn chán sẽ ngay lập tức kéo theo những nhu cầu mới'', khiến chúng ta dần rơi vào vô nghĩa:
	\begin{quote}
		``Con người là 1 bản thể tập hợp của các nhu cầu, những điều rất khó để thỏa mãn $\ldots$ Nếu chúng ta thỏa mãn, thì tất cả những gì chúng ta nhận được là 1 trạng thái không đau đớn, khi anh ta chỉ có thể quy phục trước sự buồn chán. Đây là 1 minh chứng rõ ràng bản thân sự tồn tại là vô giá trị, vì sự buồn chán chỉ đơn thuần là cảm giác trống rỗng trong cuộc sống. E.g., nếu cuộc sống này là ham muốn tạo nên sự tồn tại của chúng ta, thực có giá trị tích cực, thì đã không có sự buồn chán; vì chỉ riêng sự tồn tại đã mang đến cho chúng ta mọi thứ, \& do đó, thỏa mãn chúng ta. Nhưng chúng ta sẽ tồn tại mà chẳng cảm thấy vui vẻ gì nếu như không theo đuổi điều gì đó; vì khoảng cách \& khó khăng cần vượt qua đại diện cho mục đích, 1 điều có thể khiến chúng ta thỏa mãn -- 1 ảo giác sẽ biến mất ngay khi chúng ta đạt được mục tiêu $\ldots$ Mọi khoái cảm nhục dục chẳng là gì khác ngoài 1 nỗ lực liên tục đạt được mục tiêu. Chừng nào chúng ta không làm 1 trong 2 cách này, mà bị ném trả lại vào sự tồn tại, chúng ta sẽ bị thuyết phục về sự trống rỗng \& vô nghĩa của nó, \& đó là thứ chúng ta gọi là buồn chán.''
	\end{quote}
	Tất nhiên {\sc Schopenhauer} là 1 nghệ nhân bậc thầy sử dụng sự bi quan làm nguyên liệu chính cho sáng tạo. Chúng ta không nhất thiết phải đi theo quan điểm tiêu cực như vậy để tìm thấy phần minh triết cốt lõi trong những ý tưởng xám xịt ảm đạm của ông ấy. Bởi lẽ, như {\sc Annie Dillard} đã viết trong tư duy tích cực của bà về việc ưu tiên hiện tại thay vì năng suất, thỏa mãn về cảm giác \& thỏa mãn về tinh thần là 2 điều hoàn toàn khác nhau -- chỉ có điều thứ nhất là giới hạn \& do đó, được ấn định thử thách với của sự buồn chán; khi theo đuổi điều thứ 2, sự buồn chán sẽ là bằng hữu thay vì kẻ thù của chúng ta, nền tảng tĩnh tại cần thiết cho sự suy ngẫm đưa chúng ta ra khỏi guồng bận rộn không thể cưỡng lại được để đi vào 1 trạn thái tồn tại với nhận thức sâu sắc trong hiện tại.
	\item {\sc Walter Benjamin.} Trong cuốn {\it Illumination: Essays \& Reflections} (tạm dịch: Khai sáng: Các bài viết \& hồi tưởng), nhà triết học người Đức, nhà lý luận văn hóa \& nhà phê bình văn học {\sc Walter Benjamin} (15.7.1892 - 26.09.1940) đã khám phá vai trò của sự buồn chán trong những suy ngẫm rộng hơn về vai trò của truyện kể trong việc tách biệt trí tuệ khỏi thông tin.
	
	Cho rằng sự phát triển của thông tin đã làm giảm đi vai trò của kể chuyện, ông coi sự dị ứng với sự buồn chán của chúng ta như 1 nỗi ưu phiền nguy hiểm trong Thời đại Thông tin. 1 nửa thế kỷ trước khi có những di căn như bây giờ, {\sc Benjamin} đã phê phán chứng bệnh tinh thần này:
	\begin{quote}
		``Không gì đưa 1 câu chuyện vào tâm trí dễ dàng hơn sự cô đọng mộc mạc có thể loại bỏ những phân tích tâm lý. Quá trình người kể bỏ dần từng lớp tâm lý càng tự nhiên bao nhiêu, câu chuyện càng chiếm tâm trí người nghe nhiều bấy nhiêu, càng hòa cùng với trải nghiệm của chính người nghe, anh ta càng có xu hướng kể lại chuyện đó cho người khác trong tương lai, dù sớm hay muộn. Quá trình hấp thụ dần dần này diễn ra theo chiều sâu trong trạn thái thư thái cần thiết -- 1 điều ngày càng trở nên hiếm có. Nếu giấc ngủ là cực hạn khi cơ thể nghỉ ngơi, sự buồn chán chính là cực hạn khi tâm trí nghỉ ngơi. Sự buồn chán là cánh chim mơ ước sinh ra từ trải nghiệm. Chỉ 1 tiếng xào xạc trong lá đã đủ khiến chú chim bay mất. Những nơi nó làm tổ, nơi mọi hoạt động đều gắn liền sâu sắc với sự buồn chán -- đều đã ``tuyệt chủng'' tại các thành phố \& thưa thớt dần tại các vùng quê. Cùng với đó là sự mất đi của khiếu lắng nghe \& cộng đồng những người biết lắng nghe cũng biến mất. Vì kể chuyện vẫn luôn là nghệ thuật lặp lại các câu chuyện, \& nghệ thuật này mất đi khi những câu chuyện không còn nữa. Nó mất đi vì không còn những tiếng khung dệt vang lên khi còn có người lắng nghe chúng. Người nghe càng dễ quên, họ càng dễ ấn tượng sâu sắc với những gì họ nghe \& ghi vào tâm trí. Khi nhịp công việc theo đuổi họ, họ lắng nghe như thể khiếu kể chuyện tự nó đi vào tâm trí họ. Điều này chính là bản chất của tấm mạng nuôi dưỡng khiếu kể chuyện. Đây là cách nó đang dần được tháo gỡ sau hàng nghìn năm được thêu dệt từ những hình thức thủ công cổ nhất.''
	\end{quote}
	\item {\sc Susan Sontag.} (16.1.1933--28.12.2004) dùng nhật ký để ghi lại ``chế độ'' ngốn sách rộng \& bao quát của bà khi tự thừa nhận đọc sách 8--10 tiếng{\tt/}ngày. Với trí tuệ mẫn tiệp, bà lượm từng mảnh ý tưởng thu được từ những trang sách -- bao gồm tư tưởng của {\sc Kierkegaard, Schopenhauer, \& Benjamin}, tất cả đều ghi lại trong nhật ký -- \& thêu dệt nên tư tưởng của riêng bà, vốn luôn là sự tổng hợp của 1 trí óc sáng tạo.
	
	Trong 1 chương mục của nhật ký trích từ cuốn {\it As Consciousness is Harness to Flesh: Journals and Notebooks} (tạm dịch: Khi nhận thức trang bị sức mạnh cho thân thể: Những bài viết và ký), 1 kho tàng những tư tưởng thông thái của {\sc Sontag} viết về tình yêu, nghệ thuật, viết lách, sự kiểm duyệt, \& cách ngôn, cũng như những minh họa sâu sắc về tình yêu, chính là 1 suy tư về vai trò của sự buồn chán đối với sự sáng tạo dưới hình thức của 1 kiểu chú ý: {\it Buồn chán thì mang cả nghĩa tốt lẫn nghĩa xấu.}
	\begin{quote}
		``{\sc Arthur Schopenhauer}, người đầu tiên viết về sự buồn chán (trong những bài luận của ông) xếp nó cùng hạng với ``sự đau đớn'', như 2 anh em sinh đôi xấu xa của sự sống (đau đớn cho người nghèo, buồn chán cho người giàu -- phân loại thế nào chỉ phụ thuộc vào mức độ giàu có). Mọi người nói rằng ``chán thật'' như thể đó là tiêu chuẩn cuối cùng của sự hấp dẫn, \& không có tác phẩm nghệ thuật nào có quyền khiến chúng ta buồn chán. Nhưng phần lớn những nghệ thuật thú vị trong thời đại chúng ta lại thật tẻ nhạt. {\sc Jassper John} nhạt nhẽo. {\sc Beckett} nhạt nhẽo, {\sc Robb-Grillet} cũng vậy. etc. Có lẽ chính nghệ thuật cũng nhạt nhẽo. (Đương nhiên điều này không có nghĩa là nghệ thuật nhạt nhẽo thì cần phải tốt đẹp). Chúng ta không nên kỳ vọng có được sự khuây khỏa nhờ nghệ thuật. Ít nhất là với nghệ thuật cao cấp.
		
		Buồn chán là 1 cách để học sự chú ý. Chúng ta đang học những trạng thái chú ý khác nhau -- e.g., thích nghe hơn nhìn -- nhưng chừng nào chúng ta còn hoạt động trong khung chú ý cũ chúng ta sẽ còn thấy X là nhạt nhẽo $\ldots$ e.g.: nghe để cảm nhận hơn là âm thanh (quá chú ý vào thông điệp được truyền tải). Có lẽ sau những lặp lại của những cụm từ, ngôn ngữ hay hình ảnh trong 1 thời gian dài trong những văn bản viết hay bản nhạc, hay phim ảnh, nếu chúng ta thấy chán, thì nên tự hỏi liệu mình có đang sử dụng đúng khung chú ý hay không. Hoặc là -- có lẽ chúng ta đang vận hành theo 1 khung, mà đáng ra phải là 2 khung chú ý đồng thời, để giúp giảm tải gánh nặng ở mỗi bên (như cảm nhận \& âm thanh).
	\end{quote}
	\item {\sc Renata Adler.} Vì lẽ buồn chán là 1 nhân tố cơ bản của cuộc sống, những khám phá về nó không nên bị giới hạn trong những tác phẩm nhận thức luận hay truyện phi giả tưởng. Trong tiểu thuyết {\it Speedboat} (tạm dịch: Tàu biển cao tốc) năm 1976, tác gia \& nhà phê bình {\sc Renata Adler} (sinh ngày 19.10.1938) đã mô tả tất cả những tác động nghịch lý qua lại giữa sự buồn chán \& sự chú ý:
	\begin{quote}
		 ``Bản chất của sự buồn chán không phải hiển nhiên hay thứ dễ nhìn ra. Ví như nó ngụ ý về khoảng thời gian kéo dài bao lâu. Thật dở hơi khi nói rằng ``trong 3 s đó, tôi thấy buồn chán''. Nó ám chỉ về sự thờ ơ nhưng đồng thời cũng đòi hỏi 1 mức độ chú ý nhất định. Chúng ta không thể nói 1 người thất buồn chán bởi 1 thứ anh ta không chú ý tới, hay khi anh ta đang trong trạng thái hôn mê, hoặc buồn ngủ. Nhưng tôi biết rằng, hay đúng hơn tôi nghĩ rằng mình biết những người lười nhác thường là những người bị làm cho buồn chán, \& những người như vậy (trừ khi họ ngủ rất nhiều) thường là kẻ xấu xa. Không phải tình cờ mà sự buồn chán \& độc ác lại là những mối bận tâm lớn nhất trong thời đại của chúng ta. Chúng sinh sôi trong 1 góc nào đó của tâm hồn con người.''
	\end{quote}
	\item {\sc Andrei Tarkovsky.} Nhà làm phim, nhà văn người Nga {\sc Andrei Tarkovsky} (4.4.1932--29.12.1986) là 1 trong những nhân vật có ảnh hưởng nhất trong lịch sử điện ảnh. {\sc Ingmar Bergman} xem ông là đạo diễn vĩ đại nhất, ``người đã phát minh ra hẳn 1 ngôn ngữ mới''. Với ngôn ngữ đó, các phim của ông nói về những khía cạnh đơn thuần nhất \& thường là khó khăng nhất trong cuộc sống theo 1 cách tinh tế \& khéo léo. Trong 1 trích đoạn từ 1 tài liệu cổ, ông đã khảo sát về sự cần thiết của sự cô đơn một mình: Do phụ đề video chỉ truyền đạt 1 phần rất chọn lọc những điều {\sc Tarkovsky} thực sự bàn tới -- 1 cách khá đánh lạc hướng -- nên tôi đã đề nghị bạn mình, {\sc Julia}, giúp sao chép lại, \& cô đã bằng lòng hỗ trợ:
	\begin{quote}
		``Ông muốn chia sẻ gì với mọi người?''
		
		``Tôi không rõ $\ldots$ Tôi nghĩ mình chỉ muốn nói rằng mọi người nên học cách ở 1 mình \& cố gắng dành càng nhiều thời gian có thể với chính mình càng tốt. Tôi cho rằng 1 trong những thiếu sót của người trẻ ngày nay chính là họ luôn cố gắng tụ tập quanh những sự kiện ồn ã, thậm chí thi thoảng còn khá hung hăng. Tôi cho rằng việc họ muốn xích lại gần nhau để khỏi phải thấy cô đơn có lẽ lại là 1 biểu hiện không tốt. Điều đó không có nghĩa là chúng ta cần phải cô đơn, nhưng chúng ta không nên thấy buồn chán với chính mình, vì nếu thật như vậy, theo tôi, chính là đang gặp nguy hiểm, xét từ góc độ về lòng tự tôn của bản thân''.		
	\end{quote}
	\item {\sc Adam Phillips.} \fbox{Trẻ con có cách đặt vấn đề tưởng đơn giản nhưng lại chính là những câu hỏi lớn lao có tính tồn tại.} Theo nhà phân tích tâm lý học nổi tiếng người Anh {\sc Adam Phillips} (sinh ngày 19.8.1954), câu hỏi ``Chúng ta sẽ làm gì bây giờ?'' chính là 1 trong số đó.
	
	Trong 1 bài luận rất tâm đắc với nhan đề {\it``On Being Bored''} (tạm dịch: Bàn về sự buồn chán), được lấy từ tuyển tập {\it On Kissing, Tickling, \& Being Bored: Psychoanalytic Essays on the Unexamined Life} năm 1993, Phillips viết:
	\begin{quote}
		``Tất cả người lớn đều ghi nhớ, trong rất nhiều điều khác, sự tẻ nhạt kinh khủng của tuổi thơ, \& cuộc sống buồn chán của mỗi đứa trẻ có thể được đánh vần bởi từng chữ trong từ đó: trạng thái lơ lửng đoán trước, khi mọi thứ đã khởi động nhưng chẳng có gì bắt đầu cả, tâm trạng thao thức kéo dài chứa đựng điều ước đầy nghịch lý \& ngốc nghếch nhất, mong ước có 1 ham muốn.''
	\end{quote}
	Tất nhiên {\sc Phillips} đã viết điều này $> 2$ thập kỷ trước khi Internet hiện đại đưa ra cho chúng ta 1 thuật ngữ phổ biến ``mạng xã hội'' đã tạo nên văn hóa ngày nay. Điều này cho thấy hiểu biết của ông có 1 tầng ý nghĩa mới, sâu sắc \& rất đáng để lưu tâm khi xem xét về năng lực của sự buồn chán -- không chỉ với trẻ con, mặc dù đặc biệt trong trường hợp của chúng, mà còn có người lớn -- trong bối cảnh thời đại tiếp cận liên tục với những dòng chảy khó hòa hợp từ những kích thích bên ngoài. Đây đặc biệt là sự dừng lại suy xét về chức năng phát triển của sự buồn chán đối với việc hình thành cơ chế tâm lý học \& cách chúng ta chú ý tới thế giới -- hoặc không phải vậy. {\sc Phillips} viết:
	\begin{quote}
		``Buồn chán thật ra là 1 quá trình nhất thời mà tại đó đứa trẻ vừa chờ đợi vừa tìm kiếm điều gì đó, mong muốn chúng được thỏa thuận 1 cách bí mật, \& theo nghĩa này, sự buồn chán khá giống với sự chú ý nửa vời. Trong sự tắc nghẹn mơ hồ, nỗi buồn chán đến phát điên đôi khi khiến đứa trẻ dần chạm tới 1 cảm giác trống rỗng đều đặn mà bên ngoài điều đó ham muốn thực sự của nó có thể kết tinh $\ldots$ Khả năng thấy buồn chán có thể là 1 thành tựu phát triển của đứa trẻ.''
	\end{quote}
	Tuy vật, sự buồn chán là trẻ con lại gợi ra sự khiển trách, cảm giác thất vọng \& buộc tội về sự thất bại từ người lớn. Điều này nghĩa là nếu ngay từ đầu người ta đã thừa nhận sự buồn chán thì để giảm nhẹ điều đó, 20 năm sau, họ sẽ đặt vào tay đứa trẻ 1 thiết bị kỹ thuật số. Theo nghĩa nào đó, chúng ta đối xử với sự buồn chán như cách chúng ta đối xử với chính tính trẻ con -- như 1 thứ phải vượt qua \& từ bỏ, hơn là đơn giản đó là 1 trạng thái khác, rất cần thiết là khác. {\sc Phillips} viết:
	\begin{quote}
		``Thực tế, sự buồn chán của con trẻ thường gặp phải sự phản đối 1 cách khó hiểu, \& người lớn thì luôn mong muốn làm thế nào để chúng không còn chú ý đến sự buồn chán ấy nữa, như thể họ đã quyết định rằng cuộc sống của con trẻ phải luôn luôn, hay phải được nhìn nhận như là luôn luôn thú vị. Đây là 1 trong những đòi hỏi áp đặt nhất của người lớn lên trẻ con, rằng chúng phải thích thú với điều gì đó, thay vì dành thời gian tìm hiểu điều gì khiến chúng yêu thích. Sự buồn chán là rất cần thiết trong quá trình con người sử dụng dành thời gian của chính mình để tìm kiếm điều mình yêu thích.''
	\end{quote}
\end{enumerate}
Translated from \href{https://www.brainpickings.org/2015/03/16/boredom/}{Brainpickings{\tt/}boredom}.'' -- \href{https://tramdoc.vn/tin-tuc/loi-benh-vuc-cho-su-buon-chan-200-nam-tu-tuong-ve-gia-tri-cua-su-nhan-roi-tu-nhung-bo-oc-vi-dai-nhat-nhan-loai-nnlAW.html}{Trạm Đọc{\tt/}lời bênh vực cho sự buồn chán: 200 năm tư tưởng về giá trị của sự nhàn rỗi từ những bộ óc vĩ đại nhất nhân loại}

%------------------------------------------------------------------------------%

\subsection{\href{http://tramdoc.vn/tin-tuc/phuong-trinh-hanh-phuc-cua-einstein-nzGgvW.html}{Phương Trình Hạnh Phúc của Einstein}}

\textbf{Chuyện gì cũng có công thức khoa học.}

%
Sống trên đời làm gì cũng cần lý do.

Từ lâu loài người đã đặt ra câu hỏi: chúng ta tồn tại để làm gì?

1 vài người đàn ông thích theo đuổi phụ nữ, 1 vài người phụ nữ thích theo đuổi tiền tài, 1 số khác thì theo đuổi những cuộc vui chuếnh choáng, nhưng tựu chung lại tất cả mọi người đều mưu cầu cùng 1 thứ: hạnh phúc đích thực.
\begin{quotation}
	``\textit{Hạnh phúc là ý nghĩa \& mục đích của cuộc đời, là đích đến \& kết cuộc cho sự tồn tại của nhân loại}.'' - Aristotle
\end{quotation}
Vì vậy để đạt được hạnh phúc ta thử tham khảo xem những nhà thông thái đã làm gì.

Einstein, người được xem là nhà khoa học vĩ đại nhất lịch sử loài người, đã viết ra 1 phương trình hạnh phúc:

\subsubsection{Phương trình hạnh phúc của Einstein}
Trong những ngày cuối đời, Dukas - trợ lý của Einstein - miêu tả rằng ông nằm trên giường bệnh \textit{trong đau đớn \& không thể ngẩng đầu lên}.

%
Tuy nhiên trong ngày tiếp theo, chỉ cách thời khắc sinh tử không đầy 24h, Einstein nhờ Dukas \textit{lấy giúp ông mắt kính, giấy \& bút chì để ông ghi vội 1 vài công thức}.

%
\textit{Ông ấy viết rất lâu}, Sử gia Walter Isaacson cho biết, \textit{cho đến khi cơn đau trở nặng không thể chịu đựng được nữa, ông ấy ngủ thiếp đi} - 1 giấc ngủ vĩnh hằng.

Einstein qua đời trong khi vẫn làm điều mà ông yêu thích nhất - làm việc.

Thật không ngoa khi nói hoàn cảnh làm bật lên nhân cách của 1 con người.

%
Einstein từng nói: \textit{Thiên tài chỉ có 1\% là bẩm sinh \& 99\% là nỗ lực}.

Trong suốt chiều dài lịch sử không có ai trở nên vĩ đại hoàn toàn dựa vào vận may.

Ông cho rằng: \textit{Chỉ có những người chăm chú vào duy nhất 1 vấn đề mới có thể gặt hái được thứ mà ta thường gọi là `thành quả'}.

%
Chỉ cho tôi 1 người vĩ đại \& tôi sẽ chỉ cho bạn 1 kẻ điên cuồng.

Có gì gần với ``sự vĩ đại'' hơn sự ám ảnh điên cuồng kia chứ?

%
Vì những lý do trên, có người đến hỏi Einstein bí quyết để có 1 cuộc sống hạnh phúc là gì.

Người phỏng vấn nghĩ rằng ông sẽ đáp lại rất dài dòng, nhưng ông chỉ trả lời ngắn gọn:
\begin{quotation}
	``\textit{Nếu bạn muốn sống 1 cuộc đời hạnh phúc, hãy gắn nó với 1 mục đích, đừng gắn nó với bất kỳ ai hay bất cứ điều gì.}''
\end{quotation}
Và đó là phương trình hạnh phúc của Einstein.

\subsubsection{Công việc - Ngôi nhà an toàn trước mọi khó khăn của cuộc đời}
Những bộ óc vĩ đại thường suy nghĩ giống nhau.

Chẳng lạ gì khi cả Newton \& Einstein - 2 bộ óc vĩ đại của loài người - khi đối mặt với khó khăn trong cuộc sống đều dùng chung 1 cách thức để vượt qua.

%
Năm 1665, dịch hạch hoành hành khắp London.

Lịch sử loài người chưa từng chứng kiến cơn bùng phát dịch nào khủng khiếp đến thế.

Sức tàn phá của nó không thua gì Covid-19 hiện nay, khiến các trường học \& cửa hàng phải đóng cửa.

%
Khi thế giới đang chìm trong hoảng loạn, Isaac Newton đã dùng đến công thức hạnh phúc của mình.

Dùng như thế nào?

Đơn giản là ông gắn cuộc đời mình vào 1 mục tiêu trừu tượng chứ không phải 1 con người hay bất kỳ vật thể hữu hình nào khác.

Trong vòng 18 tháng sống cách ly tại 1 trang trại, ông đã khởi đầu 1 số công trình nghiên cứu mà theo đánh giá của các nhà chuyên môn là ``làm thay đổi thế giới''.

%
Sau này, khi được hỏi làm cách nào mà ông tìm ra luật vạn vật hấp dẫn, Newton đã trả lời:
\begin{quotation}
	\textit{Tôi nghĩ về nó mọi lúc mọi nơi}.
\end{quotation}
Tóm lại, vì Newton gắn liền cuộc đời mình vào 1 mục tiêu lý tưởng - 1 điều không bao giờ thay đổi, chứ không phải cuộc đời, vốn luôn rất khó đoán, nên đã bảo vệ bản thân trước mọi biến động của cuộc sống.

%
Khi vợ của Einstein, bà Elsa, qua đời - ông rơi vào khủng hoảng.

Theo sử gia Isaacson, bà không chỉ là vợ mà còn giống như 1 người mẹ đối với Einstein.

%
\textit{Bà nói cho ông biết khi nào thì nên ăn \& nơi nào thì nên đi}, Isaacson miêu tả, \textit{Bà soạn hành lý cho ông rồi chuẩn bị sẵn tiền tiêu vặt}.

\textit{Ở những nơi công cộng, bà luôn che chở cho ông \& gọi ông là `Giáo sư'}.

%
May thay cho Einstein, phương trình hạnh phúc đã giúp ông vượt qua sự mất mát ấy.

\textit{Khi nào tôi còn có thể tiếp tục làm việc, tôi không nên \& sẽ không bao giờ than phiền, bởi vì công việc là nguồn sống duy nhất của đời tôi}.

\subsubsection{Phương trình Hạnh phúc $=$ Dùng tâm trí để mơ $+$ Dùng tất cả sức lực để theo đuổi giấc mơ đó}
Khi 1 nhà sư được hỏi ông có cảm thấy ổn không khi từ bỏ những thú vui trần thế, ông trả lời rằng: \textit{Tôi không cảm thấy tốt hơn, nhưng cũng không còn cảm thấy tệ nữa}.

%
Công thức để có 1 cuộc sống hạnh phúc mà Einstein đúc kết lại là:
\begin{quotation}
	``\textit{Hạnh phúc giống như 1 siêu mẫu chuẩn bị kỹ càng để  bước lên thảm đỏ. Tuy nhiên sự thỏa mãn lại là lúc khi cô người mẫu ấy trở về nhà \& gỡ bỏ lớp hóa trang}.''
\end{quotation}
Nói 1 cách khác, giống như Socrates từng nhận xét, \textit{Người giàu có nhất là người hài lòng với ít thứ nhất}.

%
Những điều tốt đẹp nhất trong cuộc sống không thể tốn tiền để đạt được cũng  không phải là vật chất hữu hình.

Liệu có ai từng chạm vào tình yêu?

Liệu có ai từng nhìn thấy hòa bình?

Liệu có ai từng ngửi thấy mơ ước?

Dưới đây là yếu tố cốt lõi trong phương trình Hạnh phúc của Einstein.

%
Trong tất cả các loài động vật, con người là loài duy nhất có ước mơ.

Đó là 1 món quà diệu kỳ được gọi là ``tâm thức''.

Chúng ta có thể tưởng tượng ra 1 viễn cảnh hoàn hảo rồi dồn hết toàn bộ năng lượng từ trái tim \& linh hồn để biến điều đó thành sự thật.

%
Thông điệp ở trên đã được 1 nhân vật trong vở kịch kinh điển \textit{The Secret of Freedom} (tạm dịch: Bí mật của Tự do) tóm tắt như sau:

%
\textit{Điều làm nên 1 con người chính là tâm thức của anh ta}.

\textit{Những thứ khác thì ngựa hay heo cũng có}.

%
Tuy nhiên, người ta thường nói những thứ cho không ít được ai quý trọng.

%
Từ lúc ra đời, mỗi con người đều được ban tặng nhiều phẩm hạnh tốt đẹp, ví dụ như tình yêu \& lòng tin.

Có lẽ vì thế mà chúng ta đã đánh giá thấp khả năng mơ đến 1 viễn cảnh xứng đáng chứ đừng nói đến khả năng biến ước mơ đó thành mục đích của cuộc đời.

Ước mơ, giống như tình yêu \& lòng tin, là những phước lành ta được ban tặng miễn phí từ khi mới lọt lòng.

Có lẽ vì vậy mà ta đã xem nhẹ khả năng mơ ước đến những mục tiêu đáng giá, chứ đừng nói đến khả năng kết nối cuộc đời với những mục tiêu như vậy.

%
Tôi xin được lấy ví dụ 1 người bạn của mình \& cách cô ấy dùng phương trình hạnh phúc của Einstein để vượt qua giông bão.

%
1 người bạn rất thân thời đại học của tôi đã trải qua 1 nỗi đau khôn tả.

Hôn phu của cô ấy, người cô quen từ thời trung học, phản bội cô \& đi theo 1 người phụ nữ khác.

Vậy mà bằng cách nào đó cô ấy đã vượt qua được cú sốc đó, giống hệt người phụ nữ trong bài thơ \textit{Phenomenal Woman} của Maya Angelou.

%
Cô đã làm điều đó bằng cách nào?

%
Cô tự vực bản thân dậy khỏi nghi ngờ \& sợ hãi, làm sạch trái tim  mình, rồi dồn hết thời gian \& năng lượng vào việc học.

Khoảng 1 năm sau cô được nhận vào Trường Luật Loyola.

%
Ngày hôm nay, không những có 1 cuộc hôn nhân viên mãn mà cô còn là 1 công tố viên thành đạt.

Bạn của tôi đã áp dụng phương trình Hạnh phúc của Einstein vào cuộc sống của cô ấy.

\textsf{Hạnh phúc cũng cần có phương trình của mình.}

\subsubsection{Kết luận}
Vì Einstein là 1 nhà toán học, ông xem những bí quyết dẫn đến thành công là thành tố trong 1 phương trình.

\textit{Bạn phải nắm quy tắc của trò chơi}, ông kết luận, \textit{và sau đó bạn phải chơi giỏi hơn những người khác}.

%
Dường như quy tắc của Trò Đời là chúng ta - sinh vật duy nhất có khả năng tưởng tượng - phải sử dụng nó để hướng tới 1 mục đích trừu tượng, hay còn được gọi là ``ước mơ.''

%
Có thể những lý do trên lý giải được vì sao Einstein cho rằng \textit{trí tưởng tượng quan trọng hơn kiến thức}.

Cho dù gắn liền cuộc sống với giấc mơ không hẳn là lựa chọn huy hoàng nhất, nhưng ông cho rằng đó là con đường dễ đi nhất trong cuộc đời này.

%
Tất cả chúng ta đều không biết vào giây phút kế tiếp chuyện gì sẽ xảy ra - đó có thể là vinh quang, cũng có thể là bi kịch.

1 cách sống khôn ngoan là xem cuộc đời như 1 chiếc thuyền \& biến ước mơ thành mỏ neo có thể giúp ta đứng vững qua giông bão.

%
Tóm lại, phương trình Hạnh phúc của cuộc đời chính là:
\begin{quotation}
	``\textit{Tìm ra thứ bạn làm tốt nhất; sau đó tìm cách để tập trung tối đa vào nó; sau đó tìm người trả tiền cho bạn vì bạn đã dành toàn bộ công sức để hoàn thiện kỹ năng đó}.''\hfill$\square$
\end{quotation}

\begin{flushright}
	Vy Vũ/Theo \href{https://medium.com/mind-cafe/einsteins-formula-for-a-happy-life-b29aff61a9c7}{Medium}
\end{flushright}

%------------------------------------------------------------------------------%

\subsection{\href{http://tramdoc.vn/tin-tuc/9-thoi-quen-ky-la-tao-nen-phong-cach-cua-cac-nha-van-vi-dai-nmwejW.html}{9 Thói Quen Kỳ Lạ Tạo Nên Phong Cách của Các Nhà Văn Vĩ Đại}}

\textbf{Bên cạnh tài năng thiên bẩm, những con người ấy còn sở hữu lòng nhiệt thành \& đam mê với nghề. Các thói quen có phần kỳ quặc của họ sẽ chắp cánh cho ngôn từ bay bổng trên trang giấy. Có thể, chỉ 1 cái nhìn thoáng qua về cuộc sống thường ngày cũng có thể giúp họ tìm thấy cảm hứng sáng tạo.}

%
Trong cuộc chiến với trang giấy trắng, tác giả cần phải có 1 chiến lược vững chắc.

Ý niệm này không chỉ áp dụng cho những nhà văn trẻ mà còn với cả \href{http://tramdoc.vn/tin-tuc/10-tac-pham-kinh-dien-tu-giai-nobel-van-hoc-da-duoc-dich-sang-tieng-viet-nyalW.html}{những tượng đài của văn học}.

Trên con đường tạo ra 1 kiệt tác, những bậc thầy ngôn từ cũng phải chật vật với sự kỳ vọng vào việc tạo ra tác phẩm tốt nhất \& khó khăn trong việc tìm kiếm cảm hứng cho bản thân mình.

%
Bên cạnh tài năng thiên bẩm, những con người ấy còn sở hữu lòng nhiệt thành \& đam mê với nghề. Các thói quen có phần kỳ quặc của họ sẽ chắp cánh cho ngôn từ bay bổng trên trang giấy.

Có thể, chỉ 1 cái nhìn thoáng qua về cuộc sống thường ngày cũng có thể giúp họ tìm thấy cảm hứng sáng tạo.

\subsubsection{Nằm để viết: Tư thế lý tưởng mang đến nguồn cảm hứng bất tận}
Với 1 số nhà văn, tư thế này giúp họ giải phóng năng lượng \& có thể tập trung hơn.

Sự êm ái của chiếc giường sẽ đem lại cảm hứng \& những từ ngữ hoàn hảo.

Có thể kể đến 1 số nhà văn sở hữu thói quen này như Mark Twain, George Orwell, Edith Wharton, Woody Allen \& Marcel Proust.

Giường \& ghế sofa là nơi đã giúp họ tạo ra vô số trang tuyệt phẩm.

Nhà văn, nhà viết kịch người Mỹ Truman Capote tự gọi mình là ``1 nhà văn nằm'', bởi vì ông không thể suy nghĩ \& viết ở những tư thế khác.

%
\textsf{Mark Twain trên chiếc giường của mình.}

\subsubsection{Đứng để sáng tác}
Ngược lại với thói quen trên, đứng viết dường như không còn quá xa lạ với những nhà văn theo đuổi dòng sách có yếu tố kịch tính, giật gân.

Danh sách này bao gồm \href{http://tramdoc.vn/tin-tuc/25-cuon-sach-ma-nguoi-dan-ong-nao-cung-nen-doc-it-nhat-mot-lan-trong-doi-nZN4aW.html}{Ernest Hemingway}, Charles Dickens, Virginia Woolf, Lewis Carroll \& Philip Roth.

Nguồn cảm hứng cho những tác phẩm lạ, có phần đặc biệt này đến từ chiếc bàn đứng.

Phương pháp này cũng đặc biệt có lợi cho những người quan tâm tới sức khỏe, bởi vì khoa học đã chứng minh đứng viết sẽ có lợi cho cột sống của chúng ta hơn.

\subsubsection{Những tờ giấy nhớ}
Vladimir Nabokov, tác giả của cuốn Lolita, Pale Fire \& Ada, đã có 1 cách tiếp cận hết sức đặc biệt với việc sáng tác.

Ông đã tạo ra những điều tuyệt vời chỉ với vài tờ giấy nhớ nhỏ nằm trong chiếc hộp mỏng.

Phong cách khác thường này giúp tác giả có thể tạo dựng nên hàng loạt bối cảnh khác nhau của cuốn tiểu thuyết, \& sau đó có thể thay đổi vị trí theo ý muốn.

Và  1 điều đặc biệt khác là, Nabokov luôn có những tờ giấy nhớ dưới gối để phòng khi có 1 ý tưởng bất chợt nảy ra trong đầu, ông sẽ lưu lại ngay lập tức. 

Biết đâu phương pháp của Nabokov cũng có ích với bạn thì sao?

\subsubsection{Sử dụng thủ thuật màu sắc}
Tác giả của những cuốn tiểu thuyết như  ``\textit{The Three Musketeers}'' - Ba chàng lính ngự lâm \& ``\textit{Count of Monte Cristo}'' - Bá tước Monte Cristo - Alexandre Dumas, đã sử dụng 1 hệ thống mã hóa màu sắc.

Có thể hiểu 1 cách đơn giản là trong mắt Alexandre mọi sự việc đều được ông quy về 1 màu sắc nào đó.

Các chuyên gia IT thường quy các điểm ảnh vi tính về hình ảnh thực tế: con chó, chiếc ô tô, cái cây,$\ldots$ thì ngược lại, nhà văn của chúng ta lại suy luận những sự vật thực tế về 1 màu sắc cụ thể. 

Ông cũng là người có nguyên tắc trong việc lựa chọn màu sắc cho tác phẩm của mình.

Quá thú vị đúng không?

Trong nhiều thập kỷ, thiên tài ấy đã sử dụng những màu sắc khác nhau cho những thể loại khác nhau của mình.

Màu xanh là màu của tiểu thuyết viễn tưởng.

Màu hồng dành cho những bài luận văn hay tác phẩm khoa học, còn màu vàng là để cho thơ.

Vậy tại sao lại không thử cách mã hóa này cho tác phẩm của bạn?

Hãy mạnh dạn thử qua mọi thứ trong cuộc sống này nhé!

\subsubsection{Tư thế dốc ngược}
Ấy là 1 phương pháp giúp chữa chứng ``đơ'' khi sáng tác -  ít nhất là tác giả nổi tiếng Dan Brown tin vào điều đó.

Chính ông đã thừa nhận rằng nó giúp ông thư giãn \& tập trung hơn.

Treo ngược càng lâu, tác dụng càng lớn.

\subsubsection{Lặng nhìn 1 bức tường trống không}
Francine Prose, tác giả của cuốn tiểu thuyết Blue Angel tin rằng 1 bức tường trống là 1 phép ẩn dụ phù hợp cho tất cả những tác phẩm nghệ thuật.

Trong thời gian làm việc, Francine Prose đã đẩy hết những chiếc bàn ra phía cửa sổ, để trước mặt là 1 bức tường gạch.

Sự ``đơn điệu''  này giúp chúng ta khắc phục tình trạng mất tập trung \& nhờ đó có thể làm việc hiệu quả trong nhiều giờ liền.

\subsubsection{Tự đối thoại}
Đã từng giành được nhiều giải thưởng, nhà biên kịch ``\textit{Mạng xã hội}'' Aaron Sorkin thừa nhận từng bị gãy cả mũi trong quá trình làm việc.

Thật bất ngờ, chuyện đó thực sự đã xảy ra như thế nào?

Ông ấy thích các cuộc đối thoại của mình trước gương.

Có 1 lần vì quá nhập tâm mà ông đã tự đập đầu mình vào gương.

Kịch tính hóa là tốt, nhưng đừng quên sự an toàn.

\subsubsection{Lột bỏ bớt quần áo hoặc không mặc gì cả khi viết}
Bạn đang phải đối mặt với những deadline dày đặc, gặp nhiều bế tắc trong sáng tác.

Những khó khăn ấy chắc chắn sẽ bị đánh bại nếu bạn chú ý đến 1 phương pháp gây tò mò của nhà văn Victor Hugo.

Ông chỉ viết khi không mặc quần áo.

Có 1 khoảng thời gian Victor Hugo ngập ngụa trong lịch trình làm việc dày đặc để hoàn thiện cuốn tiểu thuyết kinh điển ``\textit{Nhà thờ Đức Bà Paris}'', ông đã yêu cầu người hầu của mình giấu hết quần áo đi.

Điều này khiến ông không thể ra khỏi nhà.

Thậm chí là vào những ngày lạnh lẽo, Hugo cũng chỉ cho phép mình quấn 1 chiếc khăn trong lúc sáng tác.

1 hành động có chút điên rồ, quái dị.

\textsf{Nhà văn Victor Hugo.}

\subsubsection{Sự kích thích bằng cà-phê}
Tiểu thuyết gia người Pháp Honoré de Balzac đã nuôi dưỡng nguồn cảm hứng của mình bằng khoảng 50 tách cà-phê mỗi ngày.

Đây là lượng cà-phê đã truyền cảm hứng cho ông ở mức phù hợp.

Theo 1 số nghiên cứu, Balzac hầu như không ngủ trong khi viết cuốn The Human Comedy. Ngoài ra còn có 1 người cũng yêu thích cà-phê, đó là Voltaire.

Ông ấy cần tới 40 cốc mỗi ngày.\hfill$\square$

\begin{flushright}
	Nghiêm Anh | \url{Animedia.ru}
\end{flushright}

%------------------------------------------------------------------------------%

\subsection{\href{http://tramdoc.vn/tin-tuc/6-loi-ich-cua-thoi-quen-viet-moi-ngay-nzG1AW.html}{6 Lợi Ích của Thói Quen Viết Mỗi Ngày}}

\textbf{Tất cả chúng ta đều có thể giãi bày những suy nghĩ của mình lên mặt giấy. Điều đó có nghĩa rằng chúng ta đều là nhà văn, ngay cả khi không biết cách xoay chuyển cốt truyện 1 cách thuần thục như Tolstoy.}

%
Thói quen viết là 1 công cụ hữu ích để thể hiện bản thân, giúp phát triển khả năng sáng tạo \& tư duy.

Và để có được điều đó, bạn hoàn toàn không cần phải ép buộc mình trở thành 1 tiểu thuyết gia bị ám ảnh với việc nhốt mình trong 4 bức tường.

Đơn giản chỉ cần biến việc viết trở thành những thói quen thân thuộc \& từ từ, viết sẽ thay đổi cuộc sống của bạn.

\subsubsection{Viết là liệu trình tuyệt vời cho sức khỏe tinh thần}
Đa phần các nghiên cứu cho rằng, Expressive Writing (tạm dịch Văn biểu cảm - việc viết ra những gì bạn nghĩ \& cảm nhận) sẽ mang lại cảm giác hạnh phúc cho người viết.

1 ví dụ rõ ràng nhất về Expressive Writing là viết nhật ký.

Điều này cũng được nhắc tới trong việc viết blog, nó giúp cải thiện sức khỏe tượng tự như viết tay vậy.

%
Ngoài ra, việc không thể diễn đạt suy nghĩ của mình bằng văn bản sẽ cản trở việc giao tiếp với người khác, nhất là trong việc trao đổi kinh nghiệm \& cảm xúc.

Không dễ để cụ thể hóa những suy nghĩ của bạn \& đưa chúng vào  1 cuộc hội thoại.

Thói quen viết thường xuyên giúp đối phó với điều này.

%
Nghiên cứu của Laura King thuộc Đại học Southern Methodist cho thấy những người viết về mục tiêu, ước mơ \& thành tích của họ thường hạnh phúc \& khỏe mạnh hơn.

Tôi cũng nhận thấy tác dụng tương tự: những người bị căng thẳng trong công việc bắt đầu thói quen ghi chép lại trong vài ngày, sau đó cảm thấy tốt hơn \& năng suất của họ tăng 29\%.

(Adam Grant, nhà văn, nhà báo, giáo sư tại Trường Kinh doanh Wharton).

%
Bên cạnh đó, thói quen viết lách giúp bạn dễ dàng truyền đạt ngay cả những ý tưởng phức tạp cho người khác.

Nó giúp loại bỏ lời biện hộ ``\textit{Ở trong đầu có vẻ tốt hơn}''.

Dĩ nhiên là, khi viết bạn phải trình bày rõ ràng suy nghĩ của mình.

\subsubsection{Thói quen viết giúp bạn vượt qua những giai đoạn khó khăn}
Trong 1 nghiên cứu được thực hiện giữa các kỹ sư bị sa thải gần đây, người ta thấy rằng những người thường xuyên viết \& truyền đạt tâm tư của mình trên giấy sẽ nhanh chóng tìm được việc làm mới.

%
Các kỹ sư ghi lại suy nghĩ \& cảm xúc của họ về việc bị sa thải ít cảm thấy tức giận \& thù địch hơn đối với người chủ cũ.

Họ cũng uống ít hơn.

Kết quả là, 8 tháng sau, 52\% kỹ sư, những người thường xuyên viết, đã tìm được công việc toàn thời gian mới, so với chỉ 19\% ở nhóm được yêu cầu ``không viết''.

(Theo Adam Grant, nhà văn, nhà báo, giáo sư tại Trường Kinh doanh Wharton).

%
Những người tham gia thử nghiệm nói rằng: khi mô tả những tâm tư mà mình  không thể chia sẻ với ai, họ không phủ nhận những khó khăn, mà cố gắng chấp nhận \& vượt qua chúng.

Do đó, theo thời gian, những lo lắng dần phai nhạt.

%
Việc viết về những sự kiện đau buồn giúp bạn giải phóng cảm giác chán nản.

Tuy nhiên, bạn sẽ mất ít nhất 6 tháng để trải nghiệm toàn bộ lợi ích của hoạt động này.

Đồng thời, để thoát khỏi những lo lắng tiêu cực, bạn không nên ép buộc bản thân.

Việc viết lách, ghi chép phải tự nhiên, phải mang đến sự hài lòng cho người thực hiện công việc đó.

\subsubsection{Thói quen viết giúp bạn lấy lại động lực}
Ở 1 nghiên cứu khác cho rằng, đối với những đối tượng viết ra ít nhất 1 lần 1 tuần, họ nhận ra có nhiều thứ tốt đẹp đang diễn ra trong cuộc sống \& họ nhìn vào tương lai với phong thái lạc quan, có động lực nhiều hơn.

%
Nhưng có 1 ``ngoại lệ'': nếu bạn viết mỗi ngày, thì sẽ không có sự khác biệt đáng kể.

Điều này có nghĩa rằng: bất kỳ việc gì, nếu bạn làm việc đó quá thường xuyên mà chẳng có 1 mục đích thực sự, mong muốn thực sự, thì sự chán nản sẽ nhanh chóng kéo đến thôi.

\subsubsection{Khi viết, hãy cố gắng sắp xếp mọi thứ trong đầu}
Bạn đã bao giờ mở nhiều tab (cửa sổ trình duyệt web) trong trình duyệt của mình cùng 1 lúc chưa?

Thật khó để không bị nhầm lẫn \& không bị chúng làm phân tâm.

Ở trong đầu cũng vậy, cố gắng suy nghĩ nhiều ý tưởng, hành động, kế hoạch cùng 1 lúc cũng chẳng khác nào mở những tab giống nhau.

Thói quen viết sẽ hình thành suy nghĩ của bạn, bạn chuyển chúng từ não ra giấy \& giải phóng tâm trí mình.

\textsf{Thói quen viết sẽ hình thành suy nghĩ của bạn.}

\subsubsection{Viết giúp ích cho trí nhớ}
Thông tin sẽ dễ nhớ hơn khi bạn hiểu về tầm quan trọng của chúng \& diễn đạt chúng bằng ngôn từ của mình.

Có được 1 bài viết thú vị đòi hỏi bạn phải có nguyên tắc \& sự tự chủ trong tổ chức: cần phải thường xuyên tập trung, tìm kiếm các nguồn thông tin mới, cảm hứng \& kiến thức mới.

%
Khi tìm kiếm những ý tưởng mới, bạn sẽ phát triển được tư duy, khả năng phân tích \& nghiên cứu của mình, học cách đi sâu \& tìm ra những chủ đề mà bạn quan tâm.

Bằng cách dành thời gian để viết, bạn học được cách giải quyết vấn đề hiệu quả hơn.

%
Trong 1 khoảng thời gian, nếu viết cùng 1 chủ đề, bạn sẽ nhanh chóng chuyển từ những ý tưởng đại trà sang những ý tưởng mới mẻ, \& sau đó bạn có thể tạo ra gì đó độc đáo.

Vì vậy, nhiều nhà văn bắt đầu với 1 đoạn văn, sau đó chuyển thành 1 bài tiểu luận, sau đó là 1 bài luận rồi đến 1 loạt các bài báo, \& các bài báo trở thành 1 cuốn sách.

\subsubsection{Thói quen viết giúp bạn chấp nhận những lời chỉ trích}
Trong thế giới hiện đại, mọi người đều muốn làm cho mình được biết đến bằng cách này hay cách khác.

Mỗi người có thể xuất bản tác phẩm của mình \& chia sẻ nó với những người khác.

%
Hãy nghĩ mà xem, thật tuyệt vời khi bạn có thể ảnh hưởng đến người khác bằng ngôn từ của mình.

Khi ai đó viết cho bạn 1 lá thư cảm ơn về điều mà bạn đã chia sẻ, bạn có thể sẽ rất ngạc nhiên đấy.

%
Và khi đối mặt với những lời chỉ trích, các nhà văn sẽ trở nên kiên cường hơn.

Những lời nhận xét, ngay cả khi chúng phi lý thì đó cũng là 1 cách tôi luyện tuyệt vời.\hfill$\square$

\begin{flushright}
	Nghiêm Anh | Theo \url{https://animedia-company.cz/writing-regularly-6-benefits/}
\end{flushright}

%------------------------------------------------------------------------------%

\subsection{\href{http://tramdoc.vn/tin-tuc/haruki-murakami-khi-viet-toi-cham-toi-chon-bi-mat-va-ki-la-trong-ban-than-minh-nZNYLW.html}{Haruki Murakami: ``\textit{Khi viết, tôi chạm tới chốn bí mật \& kỳ lạ trong bản thân mình}''}}

\textbf{Trong bài phỏng vấn ngắn, Haruki Murakami sẽ nói về cuốn sách mới - \textit{Killing Commendatore} \& chia sẻ về quá trình sáng tác của ông: sự nghiêm khắc tuân thủ lịch trình hàng ngày giúp ông khai phóng trí tưởng tượng của mình thế nào \& tình yêu với việc giặt ủi có liên quan đến việc viết sách của ông ra sao.}

%
Cuốn tiểu thuyết mới của nhà văn Nhật Bản Haruki Murakami - \textit{Killing Commendatore} (ban đầu) kể về 1 chiếc chuông tự rung; 1 ý niệm trừu tượng đánh cắp thân xác của chàng trai trong 1 bức vẽ; chuyến du hành kì lạ \& thường xuyên tới địa ngục của 1 vài Ẩn Dụ Kép đáng sợ.

Đến 1 độ, chính tác giả cũng phải nói rằng, ``\textit{có nhiều thứ không có nghĩa gì cả}''.

%
Nhưng đây là Murakami, người viết nên những tác phẩm chơi đùa trên lằn ranh giữa thực \& hư, giữa cõi trần \& cõi mộng, giữa cuộc sống đời thường \& những sự bất thường đặc biệt.

Rất khó để diễn tả \textit{Killing Commendatore}.

Cuốn sách quá bao quát \& phức tạp, tuy thế nó cũng chứa đựng những yếu tố ta đã quen qua bao tác phẩm của Murakami: sự kì bí của tình yêu, sức nặng của lịch sử, tính siêu việt của nghệ thuật, sự đào sâu tìm tòi về những điều khó hiểu nằm ngoài khả năng lý giải của con người.

\textsf{Cuốn sách mới của Haruki Murakami: Killing Commendatore.}

Các tác phẩm của Murakami đã được dịch ra hơn 50 thứ tiếng.

Không chỉ tiểu thuyết, ông còn viết nhiều truyệt ngắn, nonfiction \& dịch sách tiếng Anh ra tiếng Nhật.

Tuần trước, tôi đã có 1 cuộc phỏng vấn với Murakami sau khi ông dành 1 tiếng đồng hồ chạy bộ quanh Công viên Trung tâm.

Nhấp 1 ngụm Starbucks, Murakami bắt đầu nói về những bí ẩn của quá trình sáng tạo, tình yêu của ông với việc giặt ủi \& sự nghiêm khắc tuân thủ lịch trình sáng tác hàng ngày giúp ông giải phóng trí tưởng tượng kì lạ của mình ra sao.

Dưới đây là tóm tắt cuộc trò chuyện của tôi \& vị tác giả nổi tiếng này.

\begin{itemize}
	\item \textbf{Ông lấy ý tưởng cho cuốn \textit{Killing Commendatore} từ đâu?}
	
	\begin{quotation}
		``\textit{Tôi cũng không biết}.
		
		\textit{Có thể tôi lấy ý tưởng từ đâu đó thẳm sâu trong tâm trí}.
		
		\textit{1 ngày nọ đột nhiên tôi muốn viết 1 hoặc 2 đoạn đầu tiên}.
		
		\textit{Khi ấy tôi cũng không biết câu chuyện sẽ dẫn đến đâu, thế là tôi cất nó vào trong ngăn kéo bàn làm việc \& tất cả những gì tôi làm tiếp theo là chờ đợi}.''
	\end{quotation}
	
	\item \textbf{Vậy phần còn lại của cuốn sách thì sao?}
	
	\begin{quotation}
		``\textit{Đến 1 ngày tôi nghĩ rằng mình có thể tiếp tục viết, thế là tôi bắt đầu viết \& cứ thế tiếp tục thôi}.
		
		\textit{Ta cứ đợi đúng lúc, không có gì phải vội vã cả}.
		
		\textit{Phải tự tin rằng ta sẽ nảy ra các ý tưởng, \& tôi có sự tự tin đó vì tôi đã cầm bút 40 năm nay rồi, tôi biết mình đang làm gì}.''
	\end{quotation}
	
	\item \textbf{Đối với ông quá trình sáng tác có khó khăn không?}
	\begin{quotation}
		``\textit{Khi tôi không sáng tác thì tôi dịch sách}.
		
		\textit{Dịch sách là 1 công việc ưa thích trong khi tôi chờ đợi ý tưởng tìm đến với mình: nó cho tôi cảm giác rằng tôi đang viết, dù không phải là đang viết cuốn tiểu thuyết của mình}.
		
		\textit{Cùng với đó, tôi chạy bộ, nghe nhạc \& làm việc nhà như ủi quần áo}.
		
		\textit{Tôi yêu việc ủi đồ}.
		
		\textit{Khi ủi đồ, tâm trí tôi thư thái \& thoải mái}.''
	\end{quotation}
	
	\item \textbf{Ông có đọc các đánh giá về sách của mình không?}
	
	\begin{quotation}
		``\textit{Tôi không đọc các đánh giá}.
		
		\textit{Nhiều nhà văn nói điều này \& họ đang nói dối, nhưng tôi thì không}.
		
		\textit{Vợ tôi thì lại đọc tất cả đánh giá, cô ấy đọc to tất cả những đánh giá tệ cho tôi nghe}.
		
		\textit{Cô ấy nói tôi phải chấp nhận những đánh giá không tốt, còn đánh giá tốt thì nên quên đi}.
		
		%
		\textit{Tôi là 1 người thực tế, nhưng mỗi khi viết, tôi lại đi tới chốn bí mật \& kì lạ trong bản thân mình}.
		
		\textit{Khi viết, tôi đang khám phá chính bản thân, khám phá thế giới bên trong mình}.
		
		\textit{Nếu bạn nhắm mắt lại \& chìm vào bản thân, bạn sẽ thấy 1 thế giới khác}.
		
		\textit{Nó giống như là khám phá 1 vũ trụ - vũ trụ bên trong bản thân}.
		
		\textit{Bạn sẽ đi tới những chốn khác nhau}.
		
		\textit{Những chốn ấy có thể nguy hiểm, có thể đáng sợ, quan trọng là bạn phải biết đường quay về}.''
	\end{quotation}    
	
	\item \textbf{Có vẻ như ông thấy khó khăn khi nói về những ý nghĩa ẩn sâu trong tác phẩm của mình?}
	
	\begin{quotation}
		``\textit{Mọi người luôn hỏi tôi về những chi tiết trong sách: ``\emph{Cái này nghĩa là gì? Cái kia nghĩa là sao?}}''
		
		\textit{Tôi không thể giải thích điều gì cả}.
		
		\textit{Tôi nói về bản thân, tôi nói về thế giới 1 cách ẩn dụ}.
		
		\textit{Bạn không thể giải thích hay phân tích các ẩn dụ được - bạn chỉ cần chấp nhận nó thôi}.
		
		\textit{1 cuốn sách cũng thế - tự nó cũng là 1 ẩn dụ}.''
	\end{quotation}
	
	\item \textbf{Ông nói rằng \textit{Killing Commendatore} là cuốn sách ông viết để tỏ lòng kính trọng tới \textit{The Great Gatsby} - cuốn sách mà ông đã dịch ra tiếng Nhật 10 năm trước. Có thể hiểu \textit{The Great Gatsby} là 1 bi kịch về giới hạn của giấc mơ Mỹ. Điều này có liên quan gì đến cuốn sách mới của ông không?}
	
	\begin{quotation}
		``\textit{The Great Gatsby là cuốn sách yêu thích của tôi}.
		
		\textit{Tôi đọc nó khi tôi 17 hay 18 tuổi, vừa mới ra trường}.
		
		\textit{Khi ấy tôi bị ấn tượng vì đó là 1 cuốn sách về giấc mơ \& cách người ta hành xử khi giấc mơ tan vỡ}.
		
		\textit{Đó là 1 chủ đề quan trọng đối với tôi}.
		
		\textit{Tôi không nghĩ nó nhất thiết phải là giấc mơ Mỹ, thay vào đó, nó là giấc mơ của 1 chàng trai, 1 giấc mơ nói chung}.''
	\end{quotation}
	
	\item \textbf{Ông thường mơ về điều gì?}
	
	\begin{quotation}
		``\textit{Tôi không mơ thường xuyên lắm, 1 tháng tôi chỉ mơ 1 hoặc 2 lần}.
		
		\textit{Hoặc cũng có thể tôi mơ nhiều hơn thế nhưng tôi không còn nhớ nữa}.
		
		\textit{Dù gì thì tôi cũng không cần mơ, vì tôi có thể viết}.''\hfill$\square$
	\end{quotation}    
\end{itemize}

\begin{flushright}
	Theo \href{https://www.independent.co.uk/arts-entertainment/books/features/haruki-murakami-interview-killing-commendatore-novel-new-a8578696.html}{Independent.co.uk}
	
	Lan Anh (biên dịch)
\end{flushright}

%------------------------------------------------------------------------------%

\subsection{\href{http://tramdoc.vn/tin-tuc/20-loi-khuyen-danh-cho-nhung-nha-van-tre-cua-william-faulkner-nZDe5W.html}{20 Lời Khuyên Dành cho Những Nhà Văn Trẻ của William Faulkner}}

\textbf{William Faulkner là 1 trong những nhà văn xuất sắc nhất của nước Mỹ. Ông là nhà văn sở hữu những cuốn sách như: tiểu thuyết ``\textit{Khu bảo tồn}'' (1931), ``\textit{Âm thanh \& cuồng nộ}'' (1929) được nhiều bạn đọc yêu mến.}

%
William Faulkner là 1 trong những nhà văn xuất sắc nhất của nước Mỹ.

Tư duy độc đáo \& sự nhạy bén tuyệt vời đã mang về cho ông giải thưởng Nobel ở tuổi 52, chưa kể tới 2 giải thưởng Pulitzer, 2 giải Sách quốc gia (National Book Awards) \& giúp ông trở thành tác giả được độc giả yêu mến.

Ông là nhà văn sở hữu những cuốn sách như: tiểu thuyết ``\textit{Khu bảo tồn}'' (1931), ``\textit{Âm thanh \& cuồng nộ}'' (1929) - cuốn tiểu thuyết mang về cho Faulkner giải Nobel Văn học,$\ldots$

Chúng có thể được nhiều thế hệ đọc đi đọc lại, \& người đọc chẳng thể hiểu được là tác giả đã xoay vần thế nào để tạo ra 1 tác phẩm như thế, cứ như thể có phép màu ở đây vậy.

Ở William có rất nhiều điều đáng để học hỏi.

Mặc dù không đặc biệt thích phỏng vấn, nhưng trong 1 vài dịp, ông vẫn có thể dành thời gian chia sẻ những kiến thức của mình.

Bên cạnh đó, William cũng tham gia giảng dạy văn học tại trường Đại học bang Virginia vào những năm 1957-1958, \& 1 số cuộc trò chuyện của ông với sinh viên sau đó đã được ghi lại.

Dưới đây là 20 lời khuyên hữu ích từ ông dành cho những nhà văn trẻ.

\subsubsection{``Trở thành nhà văn'' nghĩa là gì?}
Đừng cố trở thành 1 ``nhà văn'', hãy cứ viết thôi.

Lối suy nghĩ ``\textit{mình phải là 1 nhà văn}'' chính là con đường dẫn tới sự trì trệ.

Quá trình viết lách sẽ tạo nên sự hồi sinh, thăng tiến \& có khi là cả sự sống.

1 khi dừng lại, bạn sẽ bị bỏ xa.

Chẳng bao giờ là quá sớm để bắt đầu viết cả - tất cả những gì bạn cần làm là hãy đọc.

(Từ 1 cuộc phỏng vấn cho tạp chí Princeton, 1958)

\subsubsection{Tâm thế của 1 nhà văn khi viết}
Hãy cứ là kẻ nghiệp dư thôi.

Đừng bao giờ viết vì tiền mà chỉ nên viết vì đam mê \& niềm vui.

Viết dĩ nhiên sẽ mang đến niềm vui \& cũng cần được khích lệ.

Có thể không phải ở quá trình, nhưng khi sản phẩm đã hoàn thành, bạn sẽ bùng lên nhiệt huyết, cùng 1 cơn khát được sáng tạo thêm nữa.

Không nhất thiết phải cảm thấy tự hào, nhưng không có nghĩa là bạn phải lặng lẽ 1 mình, phải giữ bí mật rằng bạn đã tạo ra 1 tác phẩm như thế.

Bạn cần phải biết rằng mình đã làm tất cả những gì có thể.

Còn lần sau, hãy làm nó tốt hơn nữa.

(Từ 1 cuộc phỏng vấn cho tạp chí Princeton, 1958).

\subsubsection{Có thứ nào gọi là kỹ thuật viết hay không?}
Nếu muốn học kỹ thuật viết, hãy đi vào phòng phẫu thuật hoặc làm thợ xây xếp gạch chẳng hạn.

Chẳng có 1 khuôn mẫu nào có thể giúp bạn viết gì đó, \& cũng chẳng có con đường nào là dễ dàng cả.

1 nhà văn trẻ sẽ là 1 kẻ ngốc nếu cứ cố gắng chạy theo 1 khuôn mẫu.

Hãy học hỏi từ những sai lầm của mình, mọi người đều nhờ cách này mà thành công cả.

1 nhà văn giỏi phải biết rằng, chẳng ai có thể cho anh ta 1 lời khuyên xác đáng.

Anh ta yêu bản thân mình hơn hết.

\textbf{Bản thân thích nhà văn nào không quan trọng, mà quan trọng là anh ta luôn muốn vượt qua chính bản thân mình}.

(Từ 1 cuộc phỏng vấn trên tạp chí Paris Review, 1956)

\textsf{Chẳng có 1 khuôn mẫu nào có thể giúp bạn viết gì đó, \& cũng chẳng có con đường nào là dễ dàng cả.}

\subsubsection{Cách tốt nhất để bắt đầu 1 cuốn tiểu thuyết}
Trước tiên, hãy nghĩ ra 1 nhân vật.

Ngay sau khi bạn nghĩ ra nhân vật đó, mọi thứ sẽ trở nên chân thực, có hồn, \& chính nhân vật sẽ tự bắt tay vào công việc.

Tất cả những gì bạn phải làm là theo kịp nhân vật ấy \& tường thuật lại những gì anh ta làm, anh ta nói.

Nhưng bạn phải thật hiểu nhân vật của mình.

Bạn phải tin vào anh ta.

Quan trọng hơn là bạn phải cảm thấy rằng, nhân vật mà mình nghĩ ra thực sự tồn tại, sau đó, tất nhiên là bạn cần phải chọn lọc những sự kiện diễn ra sao cho chúng phù hợp với nhân vật ấy.

Từ đây, quá trình viết tiểu thuyết sẽ trở nên trôi chảy hơn.

Hầu hết, các cốt truyện được nghĩ ra trước cả khi bạn đặt bút lên giấy.

%
Nhưng nhân vật của bạn phải chân thật \& phải phù hợp với trải nghiệm của chính bạn.

Trải nghiệm là những gì bạn đã từng nói, từng đọc, đã từng tưởng tượng \& từng nghe, tất cả những gì được cho là phù hợp để đánh giá nhân vật của mình.

Và ngay khi anh ta xuất hiện, trở nên chân thực, ngay khi anh ta trở nên quan trọng với bạn thì bạn sẽ chẳng còn gặp phải 1 vấn đề nào trong việc viết tiểu thuyết nữa.

(Trích từ tiêu điểm ``Hỏi \& đáp'' với sinh viên đại học Virginia, 1958)

\subsubsection{Điều gì làm nên 1 nhà văn bậc thầy}
99\% tài năng $\ldots$ 99\% kỷ luật $\ldots$ 99\% lao động.

1 tiểu thuyết gia giỏi chẳng bao giờ thấy hài lòng với tác phẩm của mình.

``Nàng ấy'' có vẻ như chẳng bao giờ hoàn hảo trong mắt anh ta cả.

Hãy luôn mơ ước \& phấn đấu nhiều hơn những gì bạn nghĩ mình có thể.

Đừng cố gắng vượt qua những người cùng thời hoặc những người đi trước.

Hãy cố gắng vượt lên chính mình.

\textbf{Người sáng tạo là người bị con quỷ trong mình điều khiển.}

\textbf{Anh ta chẳng biết vì sao chúng chọn mình \& cũng chẳng có thời gian nghĩ về điều ấy.}

Anh ta chẳng biết xấu hổ, có thể ăn cắp, vay mượn, xin xỏ của mọi người chỉ để hoàn thành tác phẩm của mình.

Anh ta chỉ quan tâm đến nghệ thuật của riêng mình.

1 nhà văn tốt là 1 người vô lương tâm.

Ở anh ta có 1 giấc mơ.

Giấc mơ đó đàn áp anh ta \& anh ta cố gắng để thoát khỏi ``nàng ấy''.

Nhà văn ấy sẽ chẳng biết yên bình là gì cho tới khi anh ta giải quyết được điều đó.

\textbf{Tất cả đều bị gác qua 1 bên: danh dự, niềm tự hào, sự chính trực, lòng tin, hạnh phúc, mọi thứ, mục đích duy nhất chỉ để viết 1 cuốn sách.}

Và nếu 1 nhà văn phải cướp bóc từ mẹ của mình, anh ta chắc cũng chẳng ngần ngại.

Bài thơ ``\textit{Ode on a Grecian Urn}'' của John Keats còn tốn kém hơn cả vài người phụ nữ.

(Từ 1 cuộc phỏng vấn cho tạp chí Paris Review, năm 1956)

\subsubsection{Đừng viết đến mức kiệt sức}
Có 1 quy tắc duy nhất mà tôi luôn tuân theo, là ngắt quãng ở đỉnh điểm của các sự kiện.

Đừng viết đến mức kiệt sức.

Hãy luôn dừng lại khi quá trình đang diễn ra trôi chảy, vậy thì khi bắt đầu lại sẽ dễ dàng hơn.

Nếu tự vắt kiệt sức mình, bạn sẽ thấy mình đang ở trong 1 vùng chết, \& sẽ rất khó để thoát ra khỏi nó.

Giống như 1 câu nói nổi tiếng - ``\textit{Hãy rời đi khi đang ở trên đỉnh vinh quang}.'' (Trích mục ``Hỏi \& đáp'' với sinh viên Đại học Virginia, 1958)

\textsf{Đừng viết đến mức kiệt sức.}

\subsubsection{Đừng lạm dụng phương ngữ}
Tôi nghĩ rằng nên tránh dùng biệt ngữ nhất có thể vì nó chỉ gây hiểu lầm cho độc giả, những người chưa quen với điều đó mà thôi.

Không cần bắt nhân vật phải nói tiếng địa phương của họ mọi lúc.

Chỉ cần nhắc đến nó 1 vài lần là đủ, đồng thời đơn giản \& đừng quá phô trương.

(Từ 1 cuộc phỏng vấn cho tạp chí ``\textit{Good Word}'', 1958)

\subsubsection{Nhà văn nên xây dựng hình tượng nhân vật như thế nào?}
Sự thật thuần khiết đơn giản là xuất phát từ trái tim.

Đừng cố gắng để lộ tất cả ý tưởng của bạn cho người đọc.

Hãy cố gắng mô tả nhân vật theo cách mà bạn thấy.

Vay mượn 1 số đặc điểm từ người này, 1 vài đặc điểm từ người khác, \& bạn sẽ có 1 nhân vật khác - 1 nhân vật mà người đọc có thể nhìn \& hiểu nó.

(Từ 1 cuộc phỏng vấn cho tạp chí Princeton, 1958)

\subsubsection{Độ tuổi lý tưởng để bắt đầu viết}
Đối với văn xuôi, độ tuổi tốt nhất là từ 35 đến 45.

Ngọn lửa trong bạn vẫn cháy \& hơn hết là bạn đã có đủ nhận thức về cuộc sống.

Văn xuôi không cần vội vã.

Còn với thơ, độ tuổi đẹp nhất là từ 17 đến 26 tuổi.

Sự nhiệt huyết trong bạn dường như góp dồn lại vào 1 chiếc tên lửa \& chực chờ phóng lên cao.

(Từ 1 cuộc phỏng vấn cho tạp chí Western Review, năm 1947)

\subsubsection{Phong cách viết của 1 nhà văn}
Tôi không tạo ra phong cách của riêng mình.

Tôi nghĩ phong cách chỉ là công cụ để sáng tạo.

Còn 1 nhà văn nếu dành quá nhiều thời gian để sáng tạo ra phong cách hoặc đi theo phong cách của riêng mình, thì đơn giản là vì anh ta chẳng có gì để nói.

Anh ta biết điều đó \& sợ hãi sự tương đồng.

Vì vậy, anh ta chẳng sáng tạo ra gì ngoài phong cách - 1 thứ bảo vật quá quý giá.

Thế nên anh ta biến thành Walter Pater, đẹp đẽ mà rỗng tuếch.

Do đó, phong cách chỉ là 1 công cụ mà thôi.

Tôi tin rằng chính câu chuyện bạn tạo ra sẽ làm nên phong cách, 1 phong cách thật hay vào hôm nay \& 1 phong cách thật xuất sắc vào ngày hôm sau.

Giống như 1 người thợ mộc giỏi, 1 nhà văn nên lựa chọn \& sử dụng 1 phong cách viết, \& tôi nghĩ phong cách cũng chỉ là thứ yếu mà thôi.

(Trích mục "Hỏi \& đáp" với sinh viên Đại học Virginia, 1958)

\subsubsection{Viết dựa trên góc nhìn thực tế}

Bi kịch thời hiện đại mà ta gặp phải là nỗi sợ hãi tràn lan, ăn sâu tới mức thậm chí chúng ta học cách chịu đựng chúng.

Những vấn đề về tâm hồn chẳng còn được quan tâm đến nữa.

Hơn hết, câu hỏi: ``\textit{Khi nào tôi mới trở nên nổi tiếng?}'' dày vò chúng ta.

Vì thế, các nhà văn trẻ quên đi những vấn đề liên quan tới cảm xúc, những đối nghịch trong chính con người mình - thứ tạo nên 1 tác phẩm hay.

Bởi vì điều gì cần phải chịu đựng \& đòi hỏi sự chăm chỉ mới đáng để viết.

%
Họ phải hiểu điều đó.

Họ phải học được rằng sợ hãi là khởi nguồn của mọi thứ.

Và, khi nhận ra điều đó, hãy mãi mãi quên gã ta đi, đừng để lại gì trong ``phân xưởng'' của mình ngoại trừ sự thật, chân lý, \& những thứ cốt lõi như tình yêu, danh dự, lòng thương hại, sự tự hào, lòng thương xót \& sự hy sinh.

Chừng nào chưa ngộ ra điều này, họ sẽ còn khổ sở.

Họ không viết về tình yêu, mà về dục vọng, về những thất bại không có tổn thất, về những chiến thắng không hy vọng, \& tệ nhất, họ viết mà không có lòng trắc ẩn \& không chút thương xót.

Sự đau khổ của họ chẳng làm ai buồn rầu, cũng không để lại sẹo.

(Từ bài phát biểu của ông tại giải Nobel, 1949)

\subsubsection{Bạn nên đặt tên tác phẩm như thế nào?}
Tôi không nghĩ rằng, 1 cái tên hay là 1 cái tên phải dài.

Tôi nghĩ mọi thứ nên được cô đọng \& súc tích.

Tôi tin rằng 1 cuốn sách hay sẽ tự gọi tên mình.

Và ở trường hợp này, thì tên càng ngắn càng tốt. (Từ 1 cuộc phỏng vấn cho tạp chí ``\textit{Good Word}'', 1958)

%
Tất cả chúng ta đều đã thất bại trên con đường đến với giấc mơ về sự hoàn hảo.

Vì vậy, tôi trân trọng những nỗ lực tuyệt vời để làm điều không thể này của các nhà văn.

Có vẻ như, nếu có cơ hội viết lại các tác phẩm của mình, chắc chắn tôi sẽ viết chúng tốt hơn, \& điều này là hoàn toàn bình thường đối với 1 nhà văn.

Đây là lý do tại sao 1 nhà văn tiếp tục làm việc, cố gắng 1 lần nữa \& tin rằng lần này anh ta sẽ làm tốt hơn, sẽ chạm được tới sự lý tưởng.

Và đương nhiên là anh ta chẳng đạt được điều đó.

Tình trạng này là điều dĩ nhiên đối với 1 nhà văn.

Còn nếu khi 1 nhà văn nào đó thành công, khi tác phẩm của anh ta đạt tới mức độ lý tưởng thì đó cũng là thời điểm anh ta sẽ nhảy xuống từ đỉnh núi mang tên ``sự hoàn hảo'' để kết thúc sự tồn tại của bản thân.

Tôi đã thất bại trong việc viết thơ.

Dường như ai cũng bắt đầu bằng việc viết thơ nhưng sau đó thì thất bại, chuyển qua viết văn, ai ngờ văn còn khó hơn cả thơ.

Cuối cùng, 1 cây bút thất bại trong việc viết văn quyết định viết tiểu thuyết.

(Từ 1 cuộc phỏng vấn cho tạp chí Paris-Review, năm 1956)

\subsubsection{Khi cảm hứng đến, đừng trì hoãn việc viết}
Bạn sẽ luôn tìm thấy thời gian để viết cuốn sách của mình.

Bất cứ ai phủ định điều này thì chắc chắn họ đang lừa dối chính bản thân.

Trong trường hợp này thì việc viết hoàn toàn phụ thuộc vào cảm hứng.

Đừng chờ đợi.

Khi cảm hứng đến với bạn, hãy viết.

Đừng để đến sau này, đừng đợi tới khi bạn có thêm nhiều thời gian rồi sau đó cố gắng nhớ lại dòng suy nghĩ của mình \& thêm vào đó những câu chữ tinh tế.

Bạn sẽ chẳng bao giờ bắt gặp cùng 1 tâm trạng, bạn cũng sẽ không bao giờ nhìn thấy mọi thứ đẹp đẽ như khi chúng lần đầu xuất hiện.

(Từ 1 cuộc phỏng vấn cho tạp chí Western Review, năm 1947)

\subsubsection{1 nhà văn cần gì?}
Tất cả những gì 1 nhà văn cần là sự thanh thản, yên tĩnh \& niềm vui.

Nếu không gian không phù hợp, tất cả những gì anh ta phải đối mặt là áp lực, tức giận \& thất vọng.

Từ kinh nghiệm của bản thân, những gì tôi cần để lấy cảm hứng là giấy, thuốc lá, thức ăn \& 1 chút rượu whisky.

1 nhà văn chẳng cần đến tự do kinh tế.

Anh ta chỉ cần 1 cây bút chì \& giấy mà thôi.

Tôi chưa từng nghe tới việc 1 kiệt tác được viết bởi 1 nguồn tài trợ lớn.

1 nhà văn không yêu cầu tài trợ cho quá trình sáng tác của mình.

Anh ta còn bận sáng tác.

Trừ khi anh ta là 1 nhà văn hạng nhất, không thì anh cũng chỉ đơn giản đang tự lừa dối mình bằng ý niệm rằng: anh ta không có thời gian \& tiền bạc mà thôi [$\ldots$]

Không gì có thể phá hủy 1 nhà văn giỏi.

Điều duy nhất có thể khiến anh ta từ bỏ là cái chết của chính mình.

(Từ 1 cuộc phỏng vấn cho tạp chí Paris-Review, năm 1956)

\textsf{Tất cả những gì 1 nhà văn cần là sự thanh thản, yên tĩnh \& niềm vui.}

\subsubsection{Khi viết về những thứ bạn chưa từng trải nghiệm}
Không có giới hạn cho những gì bạn có thể viết.

1 nhà văn có thể nói về mọi thứ theo quan điểm mà anh ta có.

Nói cách khác, anh ta có thể viết về những thứ nằm ngoài trải nghiệm \& sự quan sát của mình.

Đừng cố giới hạn những khát vọng của anh ta.

Mục tiêu càng khó càng tốt.

Và nếu anh ta có thất bại, thì hãy coi đó là 1 thất bại lớn, không phải tầm thường.

(Trích mục "Hỏi \& đáp" với sinh viên Đại học Virginia, 1958)

\subsubsection{Sửa chữa có cần thiết sau khi hoàn thành 1 tác phẩm?}
Khi nguồn cảm hứng xuất hiện, hãy viết ra tất cả những gì bạn nghĩ đến.

Khi đọc lại những gì đã viết, hãy sử dụng những thứ bạn cảm thấy hay ho.

(Từ 1 cuộc phỏng vấn cho tạp chí Western Review, năm 1947)

%
Chẳng có câu chuyện nào được kể trong 1 câu, hoặc 1 đoạn văn, chúng đều không có giá trị.

Việc sửa chữa, thay thế với tôi là không khả thi.

Tôi là 1 kẻ lười biếng.

Tôi không thích làm việc, \& tôi cố gắng chỉnh sửa mọi thứ trong đầu ngay lập tức, suy nghĩ về nó trước khi thực hiện công việc khó khăn \& đáng ghét này trên giấy.

Tôi nghĩ sửa đổi, bổ sung thường là giai đoạn thứ 2 của việc viết 1 cuốn sách,\textbf{ bởi vì cuối cùng, khi cuốn tiểu thuyết được viết ra, nó vẫn không phải là thứ bạn mong muốn}.

Bởi thế bạn cố sửa chữa, rà soát, biên tập lại 1 cái gì đó, cố gắng đưa nó tới sự hoàn hảo mà đằng nào cũng chẳng đạt được.

Bạn làm sao vậy?

Tất nhiên là chẳng có cách nào để bạn đạt được điều đó đâu.

Nói cách khác, ý của tôi là: ``\textit{Những sửa chữa của người viết là điều cần thiết đối với chính anh ta, còn những sửa đổi của người biên tập là dành cho độc giả}.''

(Trích phần ``Hỏi \& đáp'' với sinh viên Đại học Virginia, 1958)

\subsubsection{3 thứ nhà văn cần có để bắt đầu 1 tác phẩm}
1 nhà văn cần 3 thứ: kinh nghiệm, óc quan sát \& trí tưởng tượng

Và 2 thứ trong số đó có thể bù đắp cho việc thiếu thứ còn lại.

Trong trường hợp của tôi, 1 câu chuyện thường bắt đầu bằng 1 suy nghĩ đơn giản, 1 ký ức, hoặc 1 bức tranh tưởng tượng.

Viết 1 tác phẩm là quá trình giải thích vì sao điều này xảy ra \& điều gì tiếp theo đó.

Nhà văn cố gắng tạo ra những nhân vật đáng tin nhất \& những tình huống thú vị nhất theo cách cảm xúc nhất mà anh ta biết.

Âm nhạc là thứ chính biểu hiện cảm xúc, nhưng tài năng của tôi lại là ngôn từ nên những thứ mà âm nhạc thuần túy có thể dễ dàng làm được thì tôi lại lúng túng khi thể hiện chúng trên giấy.

(Từ 1 cuộc phỏng vấn cho tạp chí Paris-Review, năm 1956)

\subsubsection{Cách tôi luyện tốt nhất đối với 1 nhà văn}
Hãy đọc, đọc \& đọc.

Đọc mọi thứ - từ những thứ rác rưởi, đến kinh điển, sách hay \& sách dở.

Hãy xem người khác làm điều đó như thế nào.

Người thợ mộc học nghề của mình thông qua quan sát.

Hãy đọc \& bạn sẽ lĩnh hội được điều này.

Và hãy viết.

Nếu tác phẩm của bạn đáng giá - bạn đã hiểu ra vấn đề, nếu không, hãy ném nó ra ngoài cửa sổ.

(Từ 1 cuộc phỏng vấn cho tạp chí Western Review, năm 1947)

\subsubsection{Hãy có thêm 1 công việc khác}
Đừng coi viết lách là công việc chính của bạn.

Hãy làm 1 công việc khác để bạn có tiền sống.

Việc bạn làm gì không quan trọng, miễn là tiền không phải là mục tiêu của bạn \& miễn là cuốn sách của bạn không có hạn chót.

Bạn có thể tìm đủ thời gian để viết, bất kể công việc chính của bạn mất bao lâu.

Tôi chưa gặp 1 nhà văn giỏi nào không tìm được thời gian để viết những gì anh ta muốn.

(Từ 1 cuộc phỏng vấn cho tạp chí Princeton, 1958)\hfill$\square$

\begin{flushright}
	Nghiêm Anh | Theo \url{Animedia.ru}
\end{flushright}

%------------------------------------------------------------------------------%

\subsection{\href{http://tramdoc.vn/tin-tuc/tu-sach-cua-gabriel-garcia-marquez-24-cuon-sach-tao-nen-mot-trong-nhung-nha-van-vi-dai-nhat-cua-nhan-loai-nnOn1W.html}{Tủ Sách Của Gabriel García Márquez: 24 Cuốn Sách Tạo nên 1 trong Những Nhà Văn Vĩ Đại Nhất của Nhân Loại}}

\textbf{Cuộc sống không đơn thuần chỉ là những trải nghiệm, mà còn là những kí ức \& cách ta kể lại những kí ức ấy.}

%
Cách hiệu quả nhất để hiểu được tâm trí của người khác chính là thông qua tủ sách của họ, những cuốn sách yêu thích của người đó - nền tảng để ta xây dựng cuộc sống nội tâm của ta, giá trị cốt lõi của bản thân ta.

Và ai lại có thể bỏ qua cơ hội khám phá những bộ óc lỗi lạc nhất của nhân loại chứ?

Tiếp nối tủ sách của những nhân vật như Leo Tolstoy, Susan Sontag, Alan Turing, Brian Eno, David Bowie, Stewart Brand, Carl Sagan, \& Neil deGrasse Tyson, là Grabiel García Márquez (6/3/1927 - 17/4/2014).

\textsf{Grabiel García Márquez.}

%
Được lồng ghép vào cuốn \textit{Sống để Kể lại} - cuốn tự truyện kể về khởi đầu đầy táo bạo của Garcia Marquez trên con đường viết lách - là những tác phẩm đã định hình bộ óc \& con đường sáng tạo của ông.

\begin{quotation}
	``\textit{Cuộc sống không đơn thuần chỉ là những trải nghiệm, mà còn là những kí ức \& cách ta kể lại những kí ức ấy}.''
\end{quotation}
García Márquez viết, \& những độc giả đồng điệu biết ngay tức khắc những cuốn sách đáng nhớ là những dấu mốc của cuộc đời \& những kỉ niệm quý giá.

%
Sau đây là những tác phẩm có ảnh hưởng sâu sắc nhất đối với García Márquez - từ những ngày niên thiếu của ông ở trường nội trú, khoảng thời gian mà ông kể lại: ``\textit{Ở đó tuyệt nhất là những lúc chúng tôi đọc sách cùng nghe trước giờ ngủ}.'' - \& những kỉ niệm ấm áp của ông về những tác phẩm ấy.

\subsubsection{\textit{Núi Thần} của Thomas Mann}
Thành công vang dội của Núi Thần $\ldots$ cần đến giáo viên can thiệp chúng tôi mới không thức suốt đêm chờ đợi nụ hôn của Hans Castorp \& Clavdia Chauchat.

Hoặc là căng thẳng tột độ khi tất cả chúng tôi ngồi dựng trên giường để không bỏ lỡ 1 từ nào trong cuộc đấu triết học giữ Naptha \& bạn của anh ta Settembrini.

Buổi đọc sách hôm ấy kéo dài đến hơn 1 tiếng \& kết thúc bằng 1 tràng pháo tay vang dội.

\subsubsection{\textit{Người Đàn Ông mang Mặt Nạ Sắt} của Alexandre Dumas}

\subsubsection{Ulysses của James Joyce}
1 ngày nọ Jorge Álvaro Espinosa, sinh viên luật, người đã chỉ bảo tôi đọc Kinh thánh \& bắt tôi học thuộc họ tên những bạn đồng hành của Job, đặt 1 cuốn sách dày cộp đến đáng sợ trước mặt tôi \& tuyên bố:

\begin{quotation}
	``\textit{Đây là cuốn Kinh thánh còn lại}.''
\end{quotation}
Dĩ nhiên, cuốn sách ấy là Ulysses của James Joyce, nhưng tôi chỉ đọc nó bữa đực bữa cái đến khi tôi mất hết kiên nhẫn.

Quả là vội vàng.

Hàng năm sau, khi tôi đã là 1 người trưởng thành, tôi đặt ra cho chính tôi phải đọc lại nó 1 cách nghiêm túc, \& bên cạnh giúp tôi khám phá cả 1 thế giới nội tâm sâu sắc trong chính mình, Ulysses cũng dạy tôi cách giải phóng ngôn từ \& sắp xếp thời gian \& cấu trúc trong những tác phẩm của mình.

\subsubsection{\textit{Âm Thanh \& Cuồng Nộ} của William Faulkner}
Tôi nhận thức được rằng việc đọc \textit{Ulysses} vào 20 tuổi, \& sau đó là \textit{Âm Thanh \& Cuồng Nộ}, là những mục tiêu non nớt \& quá xa vời, \& tôi quyết định đọc lại chúng với con mắt bớt phần thiên vị.

Qua đó, những gì mà trước đây đối với tôi là cứng nhắc \& khó hiểu trở nên đẹp đẽ \& giản đơn 1 cách kì diệu.

\subsubsection{\textit{Khi Tôi Nằm Chết} của William Faulkner}

\subsubsection{\textit{Cọ Hoang} của William Faulker}

\subsubsection{\textit{Oedipus Rex} của Sophocles}
[Nhà văn] Gustavo [Ibarra Merlano] đã dạy tôi hệ thống lại những ý tưởng rời rạc, \& sự phù phiếm của tâm hồn, điều mà tôi đang rất cần.

Với sự dịu dàng \& 1 nhân cách thép.

[$\ldots$]

%
Ông đọc sâu biết rộng \& tường tận về những học giả Công giáo đương thời, những người mà tôi chưa từng nghe tên.

Ông biết tất cả mọi thứ nên biết về thơ ca, đặc biệt là những tác phẩm kinh điển tiếng Hy Lạp \& Latinh mà ông đọc bản gốc$\ldots$

Tôi thấy thật đáng ngưỡng mộ rằng bên cạnh bộ óc \& nhân cách, ông có thể bơi \& có 1 thân hình không kém gì 1 vận động viên Olympic.

Điều làm ông lo ngại nhất về tôi là sự chối bỏ những tác phẩm kinh điển Hy Lạp \& Latinh, chúng đều có vẻ nhạt nhẽo \& vô dụng đối với tôi, ngoại trừ \textit{Ô-đi-xê}, cuốn mà tôi đã đọc nhiều lần, từng đoạn 1, ở trường trung học.

Và thế là trước khi chúng tôi nói lời tạm biệt, ông chọn lấy 1 cuốn sách bọc da từ thư viện \& đưa nó cho tôi 1 cách trịnh trọng.

``\textit{Anh có thể trở thành 1 nhà văn giỏi},'' ông nói, ``\textit{nhưng anh sẽ chẳng thể nào trở nên kiệt xuất nếu anh không có đủ kiến thức về văn học Hy Lạp cổ đại}.''

Cuốn sách ấy là tuyển tập toàn bộ các tác phẩm của Sophocles.

Từ đó trở đi Gustavo trở thành 1 trong những người quan trọng nhất trong cuộc đời tôi, vì \textit{Oedipus Rex} là 1 tác phẩm hoàn mỹ đối với tôi ngay từ lần đọc đầu tiên.

\subsubsection{\textit{Ngôi Nhà Bảy Chái} của Nathaniel Hawthorne}
[Gustavo Ibarra] đưa tôi mượn cuốn Ngôi Nhà Bảy Chái của Nathaniel Hawthorne, \& nó đã để lại dấu ấn sâu đậm đối với tôi suốt đời.

Chúng tôi cùng nhau đưa ra giả thiết về sự nghiệt ngã của những hoài niệm về hành trình lang thang của Ô-đi-xê, điều làm ta lạc lối \& không bao giờ tìm được lối thoát.

Nửa thế kỉ sau tôi tìm được câu trả lời cho giả thiết ấy trong 1 bài văn tuyệt hảo của Milan Kundera.

\subsubsection{\textit{Túp lều Bác Tôm} của Harriet Beecher Stower}

\subsubsection{\textit{Moby-Dick} của Herman Melville}

\subsubsection{\textit{Những Người con \& Những Người tình} của D.H. Lawrence}

\subsubsection{\textit{Nghìn Lẻ 1 Đêm}}
Tôi thậm chí còn dám nghĩ rằng vào thời đó những chuyện ly kỳ mà Scheherazade kể lại thật sự xảy ra trong cuộc sống hàng ngày, \& chỉ biến mất do thái độ ngạc nhiên \& thực tế đến hèn nhát của những thế hệ sau.

Nói cách khác, khó có thể nói rằng ai bây giờ lại có thể tin rằng người ta có thể bay trên những tấm thảm qua những thành phố \& rặng núi, hoặc rằng 1 nô lệ đến từ Cartagena de Indias có thể bị trừng phạt sống trong 1 cái chai trong vòng 200 năm, trừ khi nhà văn có thể thuyết phục độc giả tin vào câu chuyện ấy.

\subsubsection{\textit{Hóa Thân} của Franz Kaffka}
Tôi chẳng thể ngủ 1 cách yên bình như trước nữa.

[Cuốn sách] định hướng lại cuộc đời tôi ngay từ dòng đầu tiên, 1 trong những công cụ văn chương hiệu quả nhất trong nền văn học nhân loại: ``\textit{Khi Gregor Samsa tỉnh dậy vào 1 buổi sáng sau những cơn ác mộng anh ta thấy mình đã biến thành 1 con côn trùng khổng lồ}.''

[Tôi nhận ra rằng] không cần thiết phải chứng minh sự thật: điều mà nhà văn viết ra chính nó đã là sự thật, không cần bằng chứng nào ngoại trừ tài năng \& sức thuyết phục của anh ta.

Như Scheherazade, nhưng không phải trong thế giới diệu kì của cô, nơi mà mọi điều đều có thể mà là 1 thế giới hoang tàn mà mọi điều đều đã mất.

Khi tôi đọc xong \textit{Lột Xác} tôi thấy 1 niềm khao khát khó cưỡng được sống trong chốn thiên đường lạ lẫm ấy.

\subsubsection{\textit{Aleph \& những Truyện ngắn Khác} của Jorge Luis Borges}

\subsubsection{\textit{Tuyển tập Truyện ngắn} của Ernest Hemingway}

\subsubsection{\textit{Điểm Chọi Điểm} của Aldous Huxley}

\subsubsection{\textit{Của Chuột \& Người} của John Steinbeck}

\subsubsection{\textit{Chùm Nho Uất Hận} của John Steinbeck}

\subsubsection{\textit{Con đường Thuốc lá} của Erskine Caldwell}

\subsubsection{\textit{Tuyển tập Truyện ngắn} của Katherine Mansfield}

\subsubsection{\textit{Cuộc Chuyển đổi Manhattan} của John Dos Passos}

\subsubsection{\textit{Chân Dung Jennie} của Robert Nathan}

\subsubsection{\textit{Orlando} của Virginia Woolf}

\subsubsection{\textit{Bà Dalloway} của Virginia Woolf}
Đó là lần đầu tiên tôi nghe thấy cái tên Virginia Woolf, người mà ông [Gustavo Ibarra] gọi là Bà Woolf, cũng như Ông Faulkner.

Sự ngạc nhiên của tôi làm ông phát điên.

Ông cầm lấy chồng sách yêu thích của ông \& đặt chúng vào tay tôi.

%
``\textit{Đừng khách sáo},'' ông nói, ``\textit{cầm lấy hết đi, \& khi anh đã đọc xong hết ta sẽ đến lấy lại chúng bất kể anh ở đâu đi chăng nữa}.''

%
Đối với tôi chúng là 1 tài sản vô giá mà tôi không dám mạo hiểm khi tôi chẳng có lấy 1 cái hố để cất chúng vào.

Cuối cùng ông phó thác cho tôi cuốn Bà Dalloway bản tiếng Tây Ban Nha, với 1 lời tiên đoán khó chịu rằng tôi sẽ thuộc cuốn đó nằm lòng.

[$\ldots$]

%
Tôi về [nhà] cảm thấy như người đã khám phá ra cả thế giới.

%
\textit{Sống để Kể lại} là 1 cuốn sách kì diệu - câu chuyện đời khiêm tốn mà ấm áp của 1 trong những nhà văn vĩ đại nhất của nhân loại.

Hãy cùng đọc nó với Bà Woolf về ta nên đọc sách như thế nào.\hfill$\square$

\begin{flushright}
	Theo \href{https://www.brainpickings.org/2015/04/06/marquez-favorite-books/?fbclid=IwAR0JImPeIOfnzQxVa-czES1VWLGVG-kyYEdAQm6O2yl0DAJvyrA6V-kO08s}{Brainpickings}
	
	Quang Anh biên dịch
\end{flushright}

%------------------------------------------------------------------------------%

\subsection{\href{http://tramdoc.vn/tin-tuc/mot-ki-nang-qua-quan-trong-nhung-khong-ai-day-ban-o-mot-minh-lam-quen-voi-su-co-don-nyV68W.html}{1 Kỹ Năng Quá Quan Trọng Nhưng Không Ai Dạy Bạn: Ở 1 Mình, Làm Quen với Sự Cô Đơn}}

\textbf{Vấn đề duy nhất ở đây là: loài người chưa bao giờ học cách ở 1 mình.}

%
Trước khi qua đời ở tuổi 39, nhà khoa học lỗi lạc Blaise Pascal đã để lại 1 khối lượng lớn những đóng góp đồ sộ trong nhiều lĩnh vực như vật lý, toán học, cơ học chất lưu, đo lường \& xác suất.

%
Tuy nhiên, có 1 di sản khác mà ông để lại, không thuộc khoa học tự nhiên, còn truyền cảm hứng nhiều hơn cho hậu thế.

Pascal có 1 kho báu khác thuộc về khoa học xã hội, thậm chí, di sản này của ông còn to lớn \& vĩ đại hơn tất cả những thành tựu mà ông để lại.

%
1 điều thú vị là những triết lí sâu sắc chủ yếu được ông đúc kết khi tuổi đời còn rất trẻ.

Mãi đến khi trưởng thành \& tiếp xúc nhiều với tôn giáo, Pascal mới dần chuyển mình sang các lĩnh vực đậm tính triết học \& thần học. 

%
Ngay trước khi qua đời, Pascal đang tập hợp những triết lí của mình thành 1 tuyển tập thần học mà sau này được gọi với cái tên ``Cuốn sách Pensées''.

Tác phẩm chủ yếu nói về các giả thiết toán học khi được áp dụng vào cuộc sống để lựa chọn cho mình 1 đức tin.

Ngoài ra, cuốn sách còn thực sự kì bí ở những suy ngẫm về ý nghĩa của sự sống, của việc sinh ra là-1-con-người.

Nó là hình thái của triết học được ra đời trước cả khi triết học thực sự trở thành 1 lĩnh vực để nghiên cứu.

%
Có quá nhiều những suy nghĩ tâm đắc trong cuốn sách đáng được ``quote'' lại, chúng khiến người ta phải thực sự giật mình ở nhiều góc độ.

Tuy vậy, 1 trong những trích dẫn nổi tiếng nhất của Pascal đã tóm lược được những trăn trở cả đời của ông, cũng như mọi vấn đề của nhân loại.

%
Theo Pascal, \textit{chúng ta sợ sống \& tồn tại trong im lặng, sợ việc không là-1-cái-gì-đó-trên-đời.}

\textit{Chúng ta ghét sự nhàm chán, lặp lại \& tình nguyện để cho sự xao lãng xâm chiếm.}

\textit{Chúng ta không nghĩ ra cách nào khác ngoài chạy trốn khỏi các vấn đề cảm xúc bằng cách tự an ủi thậm chí là huyễn hoặc bản thân}.

%
\textbf{Vấn đề duy nhất ở đây là: loài người chưa bao giờ học cách ở 1 mình.}

%
Xã hội càng hiện đại, lời cảnh báo của Pascal càng chính xác.

Nếu có 1 từ nào đó diễn đạt chính xác những vấn đề của thế giới trong suốt 100 năm qua thì đó ắt hẳn là ``sự kết nối''.

%
Công nghệ thông tin đã \& đang xâm lấn thái quá vào việc định hướng văn hoá.

Từ điện thoại, đến radio rồi TV, mạng internet, chúng ta đã tìm ra cả ngàn cách để khiến loài người gần nhau hơn.

Tôi có thể ngồi tại văn phòng của mình ở Canada để tham dự 1 cuộc họp ở bất kì nơi nào trên thế giới chỉ qua Skype.

Tôi có thể bay đến bất kì nơi nào trên thế giới mà vẫn biết tình hình ở nhà chỉ bằng cách lướt web.

Thôi chẳng cần bàn đến lợi ích của sự kết nối, tuy nhiên, thứ nào nhiều lợi thì cũng đầy hại.

Người ta nói nhiều lắm rồi, về quyền riêng tư, về việc internet lén lút thu thập dữ liệu, tuy nhiên, còn có 1 ``thiệt hại'' to lớn khác mà không phải ai cũng biết.

\begin{quotation}
	\textit{Chúng ta đang sống trong 1 thế giới mà mọi thứ đều được kết nối, trừ bản thân mình}.
\end{quotation}
Nếu quan điểm của Pascal về việc ``con người không chịu nổi sự cô đơn'' là chính xác, vậy thì vấn đề sẽ ngày càng nghiêm trọng bởi con người thời nay có quá nhiều sự lựa chọn, họ sẽ nghĩ: ``Việc gì mình phải chịu đựng sự cô đơn?'' - khi đời người có quá nhiều cám dỗ?

%
Câu trả lời là: \textit{ở 1 mình khác với cô độc}.

Nếu bạn không chịu nổi việc ở 1 mình, bạn sẽ không bao giờ nhận thức được bản thân.

Càng như vậy, bạn càng đắm chìm vào sự xao lãng \& cứ thế, bạn lâm vào cảnh nghiện ngập, phụ thuộc vào công nghệ, những thứ vốn được chế tạo để giải phóng con người.

%
Đừng nghĩ rằng mình có thể dùng những náo nhiệt của thế giới để che đậy đi những rắc rối của bản thân, đồng nghĩa với việc những rắc rối ấy tự biến mất.

%
Hầu hết con người đều nghĩ mình đã quá hiểu rõ bản thân mình, họ tưởng rằng mình hiểu rõ bản thân, biết rõ cảm xúc của mình, hiểu rõ vấn đề của mình.

Nhưng thật ra, rất ít người có khả năng làm được điều đó.

Những người thật sự làm được sẽ ngay lập tức nói với bạn rằng ta không phải lúc nào cũng hiểu được chính mình, thậm chí, mất rất nhiều thời gian mới có thể làm được.

\begin{quotation}
	\textit{Ngày nay, con người có thể sống cả đời mà không nhận thức được gì về bản thân, ngoài cái vỏ bọc mà chúng ta tự dựng nên cho mình, chúng ta mất kết nối với chính bản thân ta, đó mới thực sự là vấn đề}.
\end{quotation}
\textbf{Nếu quay lại những nguyên lý của Pascal, ta sẽ thấy: căm ghét sự cô độc, rất gần với căm ghét sự nhàm chán.}

%
Vấn đề cốt lõi là ở đây.

Chúng ta nghiện xem TV bởi có cái gì đó rất hấp dẫn trên TV.

Ta nghiện chất kích thích vì lợi ích của nó (cho cá nhân ta) vượt trội hẳn những tác hại.

\begin{quotation}
	\textit{Có lẽ vậy, chúng ta ghét sự cô đơn bởi chúng ta đã nghiện 1 trạng thái mang tên ``không chán là được''}.
\end{quotation}
Tất cả những thứ điều khiển cuộc sống của ta 1 cách tiêu cực đều bắt nguồn từ việc: ta ghét phải đối mặt với ``hư không''.

Vì thế, ta lao đầu đi tìm trò tiêu khiển, tìm việc \& sau mỗi lần thất bại, tiêu chuẩn của ta lại càng ngày càng cao.

Ta lảng tránh 1 sự thật rằng nếu không đối mặt với sự chán nản, ta sẽ không bao giờ nhận thức được bản thân mình.

Và không nhận thức được bản thân mình chính là lí do ta thấy cô đơn, lo lắng, thay vì cảm thấy được kết nối với vạn vật xung quanh.

%
May thay, có 1 giải pháp cho vấn đề này.

Các duy nhất để chiến thắng nỗi sợ cô đơn đó chính là đối mặt với nó.

Hãy để sự chán nản đưa bạn đến nơi nào mà bạn vẫn kiểm soát được nó.

Lúc đó, bạn sẽ nghe thấy tiếng lòng của chính mình \& từ đó học được cách kết nối những phần của bản thân hiện vẫn đang còn xao lãng.

%
Thật tuyệt vời làm sao, khi bạn vượt qua được ranh giới đó, bạn sẽ thấy rằng, cô đơn chẳng phải là vấn đề gì đó quá to tát.

Sự chán nản \& nỗi cô đơn cũng có những tác động tích cực của chúng.

Khi bạn sẵn sàng đầm mình trong thanh tịnh, thế giới trở nên trù phú hơn, rõ rệt hơn.

%
Bạn sẽ học được rằng còn nhiều việc khác, thứ khác, đáng để bạn bận tâm hơn là những xô bồ của bề nổi cuộc sống.

1 căn phòng im lặng không có nghĩa là nó không có gì cho bạn khám phá.

%
Thi thoảng, việc ở-1-mình sẽ giúp bạn trải nghiệm những cảm giác không mấy dễ chịu, nhất là lúc bạn phải ``soi'' kĩ vào nội tâm của mình, những suy nghĩ, những cảm xúc, nghi ngại, hy vọng, nhưng về dài hạn, điều đó còn dễ chịu hơn nhiều so với việc lảng tránh tất cả mọi vấn đề trong cuộc sống.

%
Chịu đựng sự chán nản sẽ giúp bạn tìm thấy sự mới mẻ trong những thứ tưởng chừng chẳng có gì mới lạ; giống như 1 đứa trẻ vô tình nhìn thấy thế giới.

Điều này cũng giúp bạn giải quyết phần lớn những xung đột nội tâm của mình.

%
Thế giới càng tiến bộ, nó càng thúc đẩy chúng ta vượt qua những giới hạn của suy nghĩ.

Việc cho rằng: ``Không chịu được sự cô đơn là cốt rễ của mọi vấn đề'' có thể hơi ``nâng cao quan điểm'', nhưng chúng ta vẫn cần phải xem xét kĩ về nó.

%
Cái gì gắn kết chúng ta được thì cũng cô lập chúng ta được.

Tại sao ta cứ mải mê xao lãng với những việc không đâu để rồi càng ngày càng cảm thấy cô đơn hơn?

%
Thú vị thay, thủ phạm chính của việc ``\textbf{ghét cô đơn}'' không phải là cám dỗ cụ thể nào về vật chất.

Đó chỉ là nỗi sợ ``\textbf{hư không}'', dẫn đến việc nghiện trạng thái ``\textbf{miễn không chán là được}''.

Có thể, bản năng của chúng ta là ghét tồn tại.

%
Chừng nào còn không nhận ra ``\textbf{giá trị của sự thanh tịnh}'', chúng ta sẽ còn bỏ qua 1 sự thật rằng, chỉ khi nào ta dám đối mặt với sự chán nản, nó mới thực sự sản sinh ra những tác động tích cực.

Và để đối mặt với nó, ta cần thời gian, có thể mất vài ngày, vài tuần, chỉ đề ngồi, ngẫm, cảm nhận trong tĩnh lặng.

%
1 triết lí cổ xưa nhất trên thế giới khuyên ta duy nhất 1 điều: hãy tự nhận thức bản thân mình.

Lí do vì:

\begin{quotation}
	\textit{Không nhận thức được mình thì ta sẽ không bao giờ tìm ra cách để tương tác với thế giới}.
	
	\textit{Phải biết mình là ai đã, rồi ta mới có nền tảng để dựng nên từ đó 1 cuộc sống}.
\end{quotation}
Trớ trêu thay, 1 mình \& kết nối nội tâm là kĩ năng chẳng ai dạy ta.

Nhưng đó là kĩ năng quan trọng hơn hầu hết những thứ ta được dạy.

%
``\textbf{Ở 1 mình}'' có thể không giải quyết được mọi vấn đề, nhưng nó là bước khởi đầu, để từ đó ta tìm ra cách giải quyết mọi vấn đề.\hfill$\square$

\begin{flushright}
	Bản Dịch: CÀO CÀO (Kenh14)
	
	Bài Gốc: \href{https://medium.com/personal-growth/the-most-important-skill-nobody-taught-you-9b162377ab77}{The Most Important Skill Nobody Taught You}
\end{flushright}

%------------------------------------------------------------------------------%

\subsection{\href{http://tramdoc.vn/tin-tuc/sau-bi-kip-chon-sach-ma-mot-nao-cung-phai-biet-nVoMW.html}{6 Bí Kíp Chọn Sách mà Mọt Nào Cũng Phải Biết}}

\textbf{ Sách thì nhiều mà đời thì ngắn, làm sao ta mới tìm được 1 cuốn sách đáng đọc?}

%
Trung bình mỗi người Việt Nam chỉ đọc 4 cuốn sách mỗi năm, trong đó 2, 3 cuốn là sách giáo khoa, sách ngoài chỉ chiếm 1, 2.

%
Cũng theo 1 vài thống kê, những người thành đạt có 1 điểm chung là rất coi trọng việc đọc sách hàng ngày.

Điển hình trong số đó có Bill Gates (đọc khoảng 50 cuốn/năm) \& Mark Zuckerberg (2 tuần/cuốn).

%
Dưới đây là 1 vài tiêu chí giúp bạn đánh giá sách \& tìm được những cuốn ``đáng đồng tiền bát gạo'':
\begin{enumerate}
	\item \textbf{Giọng kể cuốn hút}: Nếu bạn cảm thấy bị giọng kể của người viết hấp dẫn ngay từ những trang đầu, cho dù đó là phần dẫn 1 cuốn tiểu thuyết hay lời mở đầu của 1 cuốn phi tiểu thuyết, nó có thể khiến bạn cười, thốt lên 1 lời cảm thán, phải suy nghĩ, thấy tò mò,$\ldots$ thì khả năng cao đó sẽ làm 1 cuốn đáng đọc với bạn.
	\item \textbf{Có định hướng rõ ràng}: 1 cuốn sách phi tiểu thuyết có thể nói về 1 mảng kiến thức, 1 ý tưởng xuyên suốt hay những hướng dẫn giúp vượt qua khó khăn, nhưng tất cả các chương phải xoay quanh 1 định hướng rõ ràng \& bạn có thể chắt lọc được gì ra từ đó.
	\item \textbf{Nhân vật chính hấp dẫn}: Cho dù đó có là 1 cuốn tiểu thuyết hay phi tiểu thuyết thì ít nhất nhân vật chính cũng nên là 1 người khiến bạn cảm thấy thú vị \& muốn khám phá thêm.
	
	Những thứ làm nên sức hấp dẫn của nhân vật chính thường bao gồm suy nghĩ, cách thể hiện suy nghĩ, cách tương tác với người khác cũng như độ độc đáo của họ.
	\item \textbf{Nhiều thông tin hữu ích}: Những thứ bạn có thể tiếp thu \& áp dụng (như tip, kỹ năng làm gì đó hay 1 cái nhìn cận cảnh về nhân vật hay nhóm người nào đó khiến bạn tò mò.
	
	Những chi tiết bạn có thể chắt lọc ra khi đọc: Từ cuốn sách này, bạn có thể tìm đọc, nghiên cứu thêm về những ý tưởng, lĩnh vực được truyền đạt trong đó.
	\item \textbf{Ham muốn đọc tiếp}: Những cuốn sách có thể khiến bạn không thể ngừng đọc cuốn sách \& cảm nhận được sự hối thúc muốn lật những trang sau để biết điều gì sẽ xảy ra.
	\item \textbf{Khoảnh khắc khi đọc xong}: Cuốn sách có thể khiến người đọc thở dài, muốn viết, muốn nghĩ về gì đó hay nói lại với ai đó về nó,$\ldots$ ngay sau khi đọc xong chắc chắn không phải 1 cuốn sách tầm thường, thậm chí trong nhiều trường hợp có thể là cuốn sách thay đổi cuộc đời bạn.
\end{enumerate}
\textbf{Vậy bạn nên chọn sách từ đâu?}
\begin{enumerate}
	\item Có \textbf{1 số nguồn gợi ý uy tín như Goodreads} (bạn có thể xem bạn bè đang đọc gì, đặt mục tiêu đọc bao nhiều cuốn trong năm nay hay tham khảo gợi ý sách từ những cá nhân nổi bật), danh sách Best-seller của New York Times (cập nhật hàng tuần) hay các bảng xếp hạng sách hay nhất năm có thể tìm được qua 1 lượt search Google.
	
	Tham khảo gợi ý từ bạn bè:
	\item \textbf{Bạn bè, người quen hay những người thành công, có sức ảnh hưởng lớn} đều có thể là những nguồn gợi ý sách giá trị.
	
	Bạn có thể tham khảo blog của Bill Gates, các blog sách nổi tiếng hay những cuốn sách gợi ý từ các cá nhân nổi bật khác.
	\item \textbf{Tìm sách theo tác giả}: Chắc hẳn những ai yêu sách luôn dễ dàng đọc ra vanh vách những tác giả họ yêu thích.
	
	Thường khi đã yêu thích lối viết của ai đó ở 1, 2 cuốn sách của họ, bạn sẽ thấy thích hầu hết các tác phẩm cùng người viết khác, điển hình như Dan Brown hay Malcolm Gladwell với những cuốn sách luôn khiến người đọc phải háo hức mỗi dịp tung ra.
	
	Nếu chưa tìm được những tác giả yêu thích thì bạn cũng có thể tham khảo luôn các tác giả từng đạt giải Nobel hay Pulitzer để đọc trước.
	\item \textbf{Chọn các cuốn phi tiểu thuyết theo lĩnh vực quan tâm}: Nếu bạn làm kinh doanh nhưng vẫn thích tìm hiểu tâm lý học hay khoa học đời sống thì hãy tham khảo những list sách nổi tiếng nhất về các lĩnh vực này trước tiên.
	\item \textbf{Chọn các tác phẩm kinh điển}: Lý do những cuốn này được xếp vào hàng kinh điển cũng bởi chúng chứa đựng những tư tưởng lớn lao có thể áp dụng vào cuộc sống cho đến tận ngày nay.
	
	Bạn rất có thể sẽ cảm thấy sửng sốt khi nhận ra cho dù mọi thứ có thay đổi chóng mặt qua nhiều thập kỷ, thế kỷ thì những vấn đề con người gặp phải, cách họ ứng phó với nghịch cảnh hay cách họ đối xử với nhau vẫn không hề đổi thay quá nhiều.\hfill$\square$
\end{enumerate}

\begin{flushright}
	Trạm Đọc (Read Station)
	
	Theo \href{http://cafebiz.vn/doc-sach-rat-tot-nhung-lam-sao-de-biet-do-la-mot-cuon-sach-dang-doc-20160820212051793.chn}{CafeBiz}
\end{flushright}

%------------------------------------------------------------------------------%

\subsection{\href{http://tramdoc.vn/tin-tuc/tro-lai-tu-bo-vuc-tram-cam-va-cuoc-chien-de-cuu-vot-chinh-minh-nYDm9W.html}{Trở Lại từ Bờ Vực: Trầm Cảm \& Cuộc Chiến để Cứu Vớt Chính Mình}}

\textbf{``$\ldots$ Khi bạn đến bệnh viện \& được cho là có thể gây tổn thương cho bản thân, bạn sẽ được đưa vào 1 căn phòng, nơi mọi người liên tục ra vào hỏi những câu giống nhau: ``Chị có kế hoạch tự sát không?'', ``Chị bị như thế này bao lâu rồi?'', ``Hiện giờ chị có sử dụng thuốc gì không?''}

%
Vấn đề của việc kết hôn với chàng trai bạn yêu năm 18 tuổi là gần như bạn sẽ luôn nhìn nhận anh ấy như 1 cậu trai gày gò, rụt rè bạn hôn ở bên cạnh tủ đồ sau khi tan học khi bạn nghĩ không có ai nhìn.

Nhiều năm sau, khi bạn nhìn lại người chồng tự tin, sơ mi cravat là lượt chuẩn bị đi làm, bạn vẫn thấy 1 nét gì đó của chàng trai đang yêu đi tàu 2 tiếng để thăm bạn tại ký túc xá trường đại học, \& nằm cạnh bạn cả đêm trên chiếc giường chật chội.

%
Trong năm đầu của cuộc hôn nhân, tôi thường ước rằng giá như mình \& Matt không gặp nhau khi còn quá trẻ vậy, giá như tôi gặp được Matt khi anh ấy đã là người đàn ông hiện giờ, thay vì là cậu trai trẻ trước kia.

Ăn mặc đẹp \& tốt bụng, vui tính \& tự tin, anh ấy là mẫu người đàn ông mà tôi sẽ tìm kiếm nhưng lại luôn cho rằng mình không thể nào tìm thấy.

%
Thay vào đó, chúng tôi gặp khi anh ấy vẫn đổ mồ hôi tay khi chúng tôi ở gần nhau \& anh ấy vẫn cần xin phép bố mẹ để sử dụng ô tô buổi tối trong tuần.

Khi chúng tôi kết hôn, tôi biết rằng anh ấy là tất cả những gì tôi mong muốn, 1 người tôi có thể dựa vào, yêu tôi \& hiểu được mong muốn được độc lập của tôi.

Nhưng đồng thời tôi cũng sợ sẽ mất anh nên tôi luôn kiềm chế bản thân để không làm gì tổn thương đến anh ấy.

%
Tôi có mối quan hệ phức tạp với chứng trầm cảm từ khi bắt đầu cuộc hôn nhân của mình.

Việc học của tôi không thuận lợi, dù luôn là vì lý do chính đáng (người thân trong gia đình qua đời, cảm thấy bị cô lập bởi những người tôi yêu thương), nên đương nhiên tôi nghĩ rằng mọi việc sẽ khá hơn khi tôi tốt nghiệp \& trở về nhà.

Mùa đông đầu tiên sau khi kết hôn, tôi lại thấy mình mất cân bằng, ngủ quá nhiều \& khóc quá thường xuyên, cảm thấy choáng ngợp \& kiệt sức dù chỉ làm những công việc đơn giản.

%
Ngày Valentine đầu tiên sau khi cưới của 2 đứa cũng là ngày mà khi trở về nhà, anh thấy tôi vẫn mặc nguyên bộ pijama, tê liệt bởi suy nghĩ của chính mình, quá sợ hãi để có thể di chuyển.

Cố hết sức để giúp, anh đưa tôi vào nhà tắm, gội đầu cho tôi khi tôi đứng yên đó, trần truồng \& thổn thức, không thực sự hiểu việc gì đang diễn ra.

Chỉ tới khi tôi quyết định sẽ tự tử thì tôi mới cảm thấy - sau 1 khoảng thời gian dài - rằng tôi có gì đấy để mà trông đợi.

Matt đã bắt đầu 1 công việc mới vài tháng trước, làm việc lâu \& kiếm được nhiều tiền.

Chúng tôi quyết định cho tôi đi học lại, để tôi có thể thay đổi công việc của mình, để tôi thấy hạnh phúc hơn.

Nhưng thay vào đó tôi chỉ đi lại quanh trường, cảm thấy lo lắng \& ngập ngụa trong bài tập \& kì vọng, trong lúc phân tích ưu khuyết điểm giữa treo cổ \& cắt cổ tay.

%
Buổi sáng tôi thấy mình cần phải đến phòng cấp cứu, tôi không gọi Matt, anh ấy đang tham dự 1 buổi hội thảo, mà gọi Sarah, 1 người bạn ít gặp nhưng dường như luôn có mặt khi tôi cảm thấy cuộc đời mình bắt đầu sụp đổ.

%
``\textit{Tớ sợ}'' tôi nói, đánh thức cô ấy vào lúc 5 giờ sáng.

%
``\textit{Đừng đi đâu},'' cô ấy nói.

``\textit{Tớ đến đón cậu ngay}.''

%
Khi bạn đến bệnh viện \& được cho là có thể gây tổn thương cho bản thân, bạn sẽ được đưa vào 1 căn phòng, nơi mọi người liên tục ra vào hỏi những câu giống nhau: “Chị có kế hoạch tự sát không?”, “Chị bị như thế này bao lâu rồi?”, “Hiện giờ chị có sử dụng thuốc gì không?”.

%
Sau khoảng 1 hay hai tiếng, Matt xuất hiện ở cửa phòng tôi.

Sarah đã gọi anh ngay sau cuộc gọi với tôi - đương nhiên rồi - nhưng tôi vẫn ngạc nhiên khi thấy anh.

Ngạc nhiên \& xấu hổ.

%
Lúc đầu anh ấy không nói gì, nhưng nhìn thấy khuôn mặt anh ấy khiến tôi nhận ra, lần đầu tiên, là tôi đã quá đà thế nào.

Tôi chỉ ước rằng anh ấy đừng nhìn tôi nữa.

%
``\textit{Đừng phán xét em},'' tôi nói. ``\textit{Anh đang phán xét em}.''

%
``\textit{Anh ấy không phán xét cậu đâu},'' Sarah trả lời.

``\textit{Anh ấy đang lo lắng cho cậu}.''

%
Nhưng lo lắng lại làm tôi khó chịu hơn là phán xét.

Chẳng thà anh cứ phán xét tôi, điều đấy hợp lý hơn.

Tôi đã làm 1 việc tồi tệ, lạm dụng sự tận tâm của anh dành cho tôi.

Tôi muốn anh ấy nổi giận, muốn anh ấy bảo tôi rằng anh ấy chịu đủ rồi, rằng anh ấy không kết hôn với 1 con vợ điên \& rằng anh ấy đã phạm sai lầm.

Tôi muốn anh ấy ghét tôi nhiều như tôi tự ghét chính mình.

%
Khi 1 nhân viên bảo vệ hộ tống tôi sang 1 căn phòng nhỏ tí (không khác lắm với căn phòng ký túc có cái giường chật chội Matt \& tôi từng nằm chung nhiều năm trước), tôi thấy bực bội.

Tôi chẳng biết tôi mong đợi cái quái gì khi đến bệnh viện \& thú nhận sự ``điên'' của mình, nhưng tôi thấy ngạc nhiên về việc quy trình có vẻ gượng ép \& cảm giác thiếu hợp tác.

Tôi đã khóc \& từ chối trả lời rất nhiều câu hỏi thăm dò (intake questions), van xin Matt đưa tôi ra khỏi nơi này.

%
``\textit{Anh không thể bỏ em lại đây},'' tôi rên rỉ.

%
Anh nhìn tôi, khuôn mặt vẫn còn nguyên cảm xúc đau đớn chưa giảm đi chút nào từ khi anh ấy đến.

%
Tôi khóc đến lúc thấy tim mình thắt lại, từ chối thuốc an thần lorazepam mà lẽ ra tôi phải uống từ lâu.

Matt ngồi yên lặng, nắm tay tôi \& không hỏi gì cả.

%
Tôi đã không thể hiểu được tại sao anh không nổi giận, tại sao anh không cảm thấy bị phản bội khi mà tôi đã sẵn sàng để anh ấy góa vợ khi mới chỉ 26 tuổi, vứt bỏ tất cả hi vọng \& ước mơ của cả 2.

Tôi tự thấy tức giận với mình thay cho anh ấy \& tự xỉ vả bản thân vì tất cả mọi việc tôi đã thất bại, trong vai trò 1 người trưởng thành, 1 cô vợ mới cưới, 1 người có tất cả \& cũng không có gì.

Tôi biểu tình bằng cách tuyệt thực tối hôm đấy.

Sau đó, tôi nhìn thấy 1 nhóm làm đồ thủ công trong căn phòng bên kia hành lang, đầy những người mà dường như cuộc đời họ cũng đã sụp đổ.

%
Khoa bệnh tâm thần, tôi đột nhiên cảm thấy, giống hệt như 1 cái nhà trẻ người lớn, nơi mà tất cả mọi thứ xung quanh đều được hướng dẫn \& cung cấp đầy đủ, nơi mà mọi người không kì vọng bạn sẽ làm được gì, nơi mà bạn không được nhận điện thoại hay email hay vật sắc nhọn hay trách nhiệm.

Bạn chỉ cần thức dậy, ăn sáng, nói về cảm xúc của mình \& trang trí bookmark khi tham gia trị liệu bằng nghệ thuật.

Cảm giác rất nhẹ nhõm, như là được hít thở khi đang đuối nước.

Tôi nhìn mọi người trang trí hoa giả lên mấy cục xốp.

Tôi thấy kiệt sức, nhưng đồng thời biết rõ rằng tôi muốn thử làm cho mọi thứ tốt đẹp hơn (ngay cả khi tôi không thể thừa nhận như vậy), nên tôi quay sang nói với Matt, ``\textbf{\textit{Có lẽ em nên thử cái này}.}''

Anh ấy trông ngạc nhiên, \& nhẹ nhõm: ``\textbf{\textit{Ồ, đương nhiên, chắc chắn rồi. Ý kiến hay đấy}.}''

%
Chúng tôi cùng nhau đi qua đấy, chần chừ.

``\textit{Anh sẽ chờ bên ngoài nhé, được không?}'' anh nói.

``\textit{Em vào đi}.''

Tôi ngồi xuống ghế, cảm thấy xấu hổ vì tôi đang ở đây, nhưng lại thấy vui sướng vì cuộc sống mới của tôi đơn giản biết nhường nào.

Tôi cần ở 1 nơi mà tất cả những gì tôi cần làm chỉ là tồn tại.

Tôi cần phải sắp xếp hoa giả \& được khen ngợi bởi 1 người phụ nữ được trả tiền để đối xử tốt với tôi.

Khi kiệt tác bằng hoa của tôi hoàn thành được 1 nửa (thành thật mà nói, nó trông như cái vòng cổ bằng mì macaroni), tôi liếc nhìn Matt đang đứng ở hành lang.

Tôi nhìn anh ấy 1 lúc, thực sự nhìn anh, người đàn ông tôi cưới, người mà đối với tôi giống gia đình hơn bất cứ người nào tôi từng gặp.

%
Tôi nhìn mái tóc dày mà tôi thích lùa ngón tay vào \& phần giữa cổ \& vai của anh, nơi mà tôi úp mặt vào hằng tối để ngửi mùi nước hoa đang nhạt dần \& cọ má vào đám râu lún phún hai-ngày-chưa-cạo khi anh đang ngủ say.

Tôi nghĩ đến việc anh ấy từng trông khác thế nào, anh ấy từng mạnh mẽ \& thông minh \& dũng cảm hơn bây giờ như thế nào, \& việc tôi ít khi còn được thấy những phẩm chất đấy nơi anh trong cuộc sống hàng ngày.

%
Anh bắt gặp tôi đang nhìn anh \& mỉm cười, ánh mắt anh vẫn ấm áp \& yêu thương như ngày nào, nhưng tôi chưa bao giờ chú ý.

Thay vào đó, tôi đã luôn chờ đợi anh ấy chứng tỏ tình yêu của anh bằng lời nói, luôn nghĩ rằng: ``\textit{Anh nói nữa đi. Hãy nói 3 lần. Và tiếp tục nói yêu em vì em không tin anh đâu}.''

%
Tôi đã luôn chờ đợi cái ngày mà hiện thực thức tỉnh anh \& khiến anh nhận ra anh đã sai lầm khi nghĩ tôi sẽ là người vợ tốt, khi trong thực tế tôi trầm cảm \& tìm cách tự hủy hoại mình.

Nhưng ngày đó không bao giờ đến.

%
Thay vào đó, anh đứng trong hành lang, theo dõi tôi đang trong cơn suy nhược tinh thần, \& tôi nhận thấy điều anh vẫn cố nói với tôi bấy lâu nay.

\begin{quotation}
	\it
	``Anh yêu em.
	
	Anh cảm thấy lo lắng cho em.
	
	Anh muốn em được hạnh phúc.
	
	Anh vẫn ở đây.''    
\end{quotation}
Đó là điều mà tôi cần hơn bất cứ thứ gì nhưng lại nghĩ mình không bao giờ có được.

Không phải cho những cô gái như tôi, với trí óc đen kịt \& nội tâm tan vỡ.

Đó là điều mà tôi có thể phải dành cả đời van xin để có được mà không nhận ra mình đã có nó.

Vào cái ngày mà tôi suýt nữa đã kết thúc cuộc đời, tôi mới nhận ra sự khởi đầu thực sự với cậu trai tôi cưới \& người đàn ông mà anh trở thành.\hfill$\square$

\begin{flushright}
	Trạm Đọc
	
	Bản dịch: \href{https://beautifulmindvn.com/2016/05/11/tro-lai-tu-bo-vuc/}{AnHD | BeautifulmindVN}
	
	Bài gốc: \href{https://www.nytimes.com/2015/06/07/style/crawling-back-from-the-ledge.html?_r=2}{Nytimes}
\end{flushright}

%------------------------------------------------------------------------------%

\subsection{\href{http://tramdoc.vn/tin-tuc/hay-hoc-nhung-tu-ky-quai-ngoai-ngu-giup-ban-mo-rong-the-gioi-nay-nDdaOW.html}{Hãy Học Những Từ ``Kỳ Quái'': Ngoại Ngữ Giúp Bạn Mở Rộng Thế Giới Này}}
\textbf{ Ngôn ngữ có trước hay cảm giác có trước?}

%
Bạn đang đứng ở 1 góc thị trấn nơi bạn từng sống ngày trước.

Bạn tản bộ qua ngôi nhà của người bạn gái từng quen hồi bạn 16 tuổi.

Bạn ngước nhìn lên thứ đã từng là cửa sổ phòng ngủ của nàng.

Từ phía ngoài, mọi thứ trông có vẻ vẫn như cũ, cho dù bây giờ đã có 1 ai khác sống ở đó.

Cha mẹ nàng đã nghỉ hưu, nàng đã lấy chồng \& có 2 đứa trẻ ở 1 thành phố khác.

Bạn bỗng cảm thấy hoài cổ sâu sắc về những sự kiện trong quá khứ mà bây giờ chắc chắn bạn không thể có những trải nghiệm như thế 1 lần nữa.

Không phải là bạn nhớ nàng nhiều bao nhiêu, 2 bạn đã từng là gì của nhau \& có nhiều kỉ niệm như thế nào.

Thật khó để giải thích cảm xúc đó là gì, cho dù nó rõ ràng đang làm lòng bạn cuộn sóng.

Ngay tại khoảnh khắc đó, điện thoại của bạn vang lên cuộc gọi đến.

Đó là 1 người bạn - hỏi bạn có khỏe không.

Bạn không biết nói gì cả, bạn không biết cách biểu đạt cảm xúc của mình, \& bạn nhanh chóng đổi chủ đề khác.

%
Đó là do bạn không nói tiếng Bồ Đào Nha!

Nếu bạn biết tiếng của đất nước này, bạn sẽ không phải chật vật tìm câu từ diễn tả cảm xúc của mình đến thế.

Bạn sẽ có 1 từ lý tưởng trong tay ngay lập tức.

Bạn có thể đơn giản nói với bạn mình rằng bạn đang ở trạng thái `\textbf{saudade}' \& họ sẽ hiểu.

Những từ mà không có nghĩa trực tiếp khi được dịch sang bất kì ngôn ngữ nào khác, từ đó hẳn là mang 1 khao khát u sầu, ngọt ngào xen lẫn đắng cay cho những điều tươi đẹp nay đã không còn: 1 câu chuyện tình, 1 ngôi nhà thơ ấu, 1 tình bạn.

\textbf{Nó là sự hòa trộn của niềm đau \& hạnh phúc rằng mình đã từng có 1 thời được sống với sự đáng yêu như thế}.

%
Vấn đề cơ bản - mối liên hệ giữa ngôn ngữ \& cảm giác - từ lâu đã gây tranh cãi trong triết học.

1 số học giả cho rằng cảm giác tách biệt với ngôn ngữ: ví dụ như những đứa trẻ có thể cảm nhận mọi thứ rất lâu trước khi chúng biết cách \textbf{ghim từ ngữ vào cảm giác của chúng}.

Tuy nhiên, các nhà triết học khác lại khẳng định rằng chúng ta sẽ không thể nhận biết được những cảm giác nhất định nếu không có những ngôn từ giúp ta nhận ra chúng.

%
\textit{Sự thật - như thường lệ - nằm đâu đó ở ranh giới mơ hồ ở giữa}.

Ngôn ngữ có thể không hoàn toàn tạo ra cảm giác, nhưng phần lớn nó chắc chắn làm sâu sắc \& rõ ràng cảm giác hơn.

Hiểu \& dùng đúng từ giúp chúng ta hiểu rõ bản thân, \& có thể ``\textit{bắc cầu}'' vào đời sống nội tâm 1 cách chính xác hơn \& an toàn hơn.

%
Hiện tượng này trở nên đặc biệt rõ ràng bất cứ khi ta bắt gặp những từ trong ngôn ngữ khác tập trung vào 1 cảm giác mà ngôn ngữ của chính chúng ta không có 1 thuật ngữ súc tích nào biểu đạt.

\textbf{Từ đó ta nhận thấy dùng đúng từ có tác dụng đến nhường nào trong việc khiến 1 cảm xúc trở nên rõ ràng \& dễ hiểu.}

%
Lấy ví dụ, từ \textbf{HÜZÜN} trong tiếng Thổ Nhĩ Kì, mang nghĩa 1 cảm giác u ám khi mọi thứ xuống dốc \& tình trạng đó - thường là thuộc về chính trị - có thể sẽ chỉ tồi tệ đi do những nhà lãnh đạo chính trị gian dối hành động dại dột \& không khôn ngoan.

Thật hữu dụng khi có 1 từ như thế này để biểu đạt \& đàm luận về nỗi đau thương được bóc tách trong tâm hồn người dân Thổ Nhĩ Kì.

%
Hoặc là, vui vẻ hơn, ở Na Uy có từ \textbf{FORELSKET}, 1 từ mang lại cảm giác phấn khích tại thời kì đầu của tình yêu, khi chúng ta không thể tin rằng 1 người hoàn hảo như thế có thể ``lạc lối'' vào cuộc đời ta \& lúc nào cũng nghĩ về tương lai tuyệt đẹp về 2 người.

Ta có thể biểu đạt rằng: ``\textit{Tôi đã bị chế ngự bởi forelsket khi những ngón tay của chúng tôi đan vào nhau}.''

%
Tuy nhiên không phải chỉ 1 vài từ vi diệu trong ngôn ngữ nước ngoài giúp làm sáng tỏ hơn tâm hồn chúng ta.

Nó là điều mà văn chương vĩ đại có thể làm.

1 bài thơ, 1 tiểu thuyết ``hay'' chính là thứ có thể gỡ bỏ bức tường xa cách tự tạo, giảm mức độ hiểu nhầm \& đưa ta trở lại với chính mình.

Sau 1 cuộc đời với việc đọc, chúng ta biết cách gắn những từ ngữ \& câu cú tới hầu hết mọi thứ chúng ta cảm nhận, cho dù nó có mong manh \& dễ tan biến đến mức nào.

%
Đó chính là món quà mà Shakespeare ca tụng trong ``\textit{Giấc mộng đêm hè}'' khi ông viết:
\begin{center}
	\it
	The poet's eye, in fine frenzy rolling,
	
	Doth glance from heaven to earth, from earth to heaven;
	
	And as imagination bodies forth
	
	The forms of things unknown, the poet's pen
	
	Turns them to shapes and gives to airy nothing
	
	A local habitation and a name.
\end{center}
Tạm dịch:
\begin{center}
	\it
	
	Đôi mắt của nhà thơ, lăn điên cuồng
	
	Từ thiên đường tới trái đất, từ trái đất tới thiên đường
	
	Và khi trí tưởng tượng tiến lên phía trước 
	
	Vật thể không tên, ngòi bút của nhà thơ
	
	Biến chúng thành hình dạng \& thổi hồn vào hư không
	
	1 nơi cư ngụ \& 1 cái tên.
\end{center}
Nhờ văn chương, chúng ta có thể được cứu thoát khỏi những ``bình chuông'' khóa kín cảm xúc.

Dùng đúng từ phá vỡ sự cô lập \& dẫn lối ta tới \textbf{TÌNH YÊU}.\hfill$\square$

\begin{flushright}
	Trạm Đọc
	
	Theo \href{http://www.thebookoflife.org/how-words-help-us-to-feel-things/}{The Book of Life}
\end{flushright}

%------------------------------------------------------------------------------%

\subsection{\href{http://tramdoc.vn/tin-tuc/su-tap-trung-quyet-dinh-trai-nghiem-trai-nghiem-quyet-dinh-cuoc-doi-nMOYnW.html}{Sự Tập Trung Quyết Định Trải Nghiệm, Trải Nghiệm Quyết Định Cuộc Đời}}

\textbf{William James đã đưa ra 1 trong những nhận định hay nhất về năng suất làm việc của thế kỷ 21 từ năm 1890. Trong cuốn Những nguyên tắc Tâm lý học (The Principles of Psychology), ông viết 1 câu giản dị nhưng vô cùng sâu sắc. ``Những điều mà tôi để ý tới làm nên trải nghiệm của tôi.''}

%
\textit{Sự tập trung của bạn quyết định những trải nghiệm mà bạn có, \& những trải nghiệm đó quyết định cuộc đời bạn}.

Hay nói cách khác: \textit{bạn phải kiểm soát sự tập trung của mình để kiểm soát cuộc đời}.

Ngày nay, thế giới có quá nhiều trải nghiệm trộn lẫn \& đan xen.

Chẳng hạn như làm việc tại nhà (hoặc trên tàu hỏa, máy bay hay bãi biển), vừa làm việc vừa trông con qua camera.

Sự phân tâm luôn chực ập tới chỉ sau 1 cú vuốt tay nhẹ trên màn hình điện thoại.

\subsubsection{Quản lý sự tập trung}
Để duy trì năng suất \& quản lý căng thẳng tốt hơn, chúng ta cần trau dồi kỹ năng quản lý sự tập trung.

%
Quản lý sự tập trung là việc thực hành kiểm soát sự phân tâm, sống với hiện tại, tìm thấy dòng chảy cảm xúc \& sự tập trung nguồn năng lượng, \& tối đa hóa khả năng tập trung.

Nhờ đó, bạn có thể \textbf{\textit{giải phóng thiên tài trong chính mình}}.

Để làm được như vậy, bạn cần phản ứng có chủ ý thay vì phản ứng 1 cách thụ động.

Đó chính là khả năng nhận ra sự tập trung của bạn đang bị phân tán (hoặc có khả năng bị) \& điều hướng nó.

%
Khả năng quản lý sự tập trung tốt hơn sẽ cải thiện năng suất làm việc, nhưng nó không chỉ dừng lại ở việc đánh dấu hoàn thành cho ``to-do list''.

Sau tất cả, bạn cần hướng đến năng lực kiến tạo cuộc đời mà bạn lựa chọn.

Không chỉ đơn thuần là rèn luyện sự tập trung mà còn là việc lấy lại sự kiểm soát thời gian \& những ưu tiên của bạn.

\subsubsection{Khát vọng \& Trải nghiệm}
Những người lãnh đạo mà tôi từng làm việc cùng chia sẻ.
\begin{quotation}
	``Tôi tin vào sức mạnh của việc đào tạo \& làm cố vấn cho team của mình.
	
	Điều quan trọng nhất tôi có thể làm với tư cách 1 nhà lãnh đạo là ủng hộ \& khuyến khích sự phát triển.
	
	Đây là cách tôi tạo nên sự khác biệt, \& là điều khiến tôi hài lòng trong công việc.''
\end{quotation}
Nhưng càng trò chuyện, tôi càng hiểu 1 ngày của họ thật sự diễn ra như thế nào.
\begin{quotation}
	``Tôi dành phần lớn thời gian vật lộn với email \& dập tắt các khủng hoảng.
	
	Tôi bắt đầu năm mới với 1 kế hoạch đào tạo cho đội của mình, nhưng nó đã đổ bể.
	
	Trao đổi 1-1 giữa tôi \& các thành viên không diễn ra thường xuyên như mong muốn.
	
	Chúng tôi đi quá sâu vào chi tiết mà thiếu đi 1 bức tranh toàn cảnh.''
\end{quotation}
\textit{Bởi vậy, tại sao chúng ta không nắm lấy trải nghiệm ta muốn có, \& kiến tạo cuộc đời mà ta khao khát?}

\textit{Tại sao lại có sự khác biệt đau đớn này giữa phiên bản chúng ta muốn trở thành \& cách ta phân bổ thời gian của mình?}

\subsubsection{Tập trung có chủ ý}
Đây chính là lúc khả năng quản lý sự tập trung đem đến 1 giải pháp.

Rèn luyện sự tập trung có nghĩa là chống lại sự phân tâm \& tạo cơ hội để theo đuổi những ưu tiên.

Đầu tiên, hãy kiểm soát:

\subsubsection{Yếu tố ngoại cảnh}

\paragraph{Kiểm soát công nghệ của bạn.} Nhớ rằng chúng ở đây để phục vụ bạn, không phải ngược lại!

Hãy giành quyền kiểm soát bằng cách tắt email \& các thông báo - được thiết kế đặc biệt để ``đánh cắp'' sự tập trung của bạn.

Điều này cho phép bạn tập trung hơn vào những công việc \& hoạt động mà mình lựa chọn.

Hãy đặt điện thoại của bạn ở chế độ im lặng \& khuất mắt thường xuyên hết mức có thể.

Đặc biệt khi đang làm việc.

\paragraph{Kiểm soát môi trường của bạn.} Thiết lập ranh giới với những người khác, đặc biệt là trong không gian văn phòng mở.

Ví dụ, sử dụng tai nghe hoặc treo biển ``không làm phiền'' khi bạn cần tập trung.

Nếu điều đó không hiệu quả, hãy cố gắng di chuyển sang không gian khác, hoặc tầng khác trong cùng tòa nhà.

Nếu vẫn không cải thiện được tình hình, bạn có thể cùng đồng nghiệp để thiết lập 1 khoảng thời gian nhất định.

Vài giờ trong ngày, hoặc ngày trong tuần.

1 ngày để tất cả mọi người thực hiện những công việc cần tập trung cao độ.

\subsubsection{Cơ chế não bộ}
Nhưng 1 sự thật bị lãng quên là.

Năng suất của chúng ta sụt giảm không chỉ bởi những tác động bên ngoài.

Bởi bộ não của chúng ta, do bị xáo trộn bởi môi trường làm việc hối hả ngày nay, đã trở thành cội nguồn của sự phân tâm.

\textbf{Ví dụ, vấn đề không chỉ là 1 email làm gián đoạn công việc của bạn.}

Thực tế là việc bị ``trói buộc'' với hộp thư điện tử khiến bạn cứ vài phút lại trông chờ 1 sự gián đoạn.

Bạn dần trở nên lo lắng không biết mình có quên làm những việc lặt vặt - như gửi email hoặc chuyển tiếp tài liệu.

Vì vậy, vừa chợt nghĩ tới việc nào bạn sẽ bắt tay vào làm ngay việc đó; nhưng cuối cùng bạn bị kẹt trong dòng chảy bất tận của hộp thư đến trước khi kịp nhận ra.

%
Hơn nữa, việc có mọi kiến thức trên thế giới ngay trong tầm tay - thông qua kết nối Internet trên điện thoại thông minh - khiến việc ở trong tình thế ``tôi không biết'' trở nên kém thoải mái.

Thật khó để cưỡng lại sự cám dỗ ``phải tìm hiểu ngay''.

\subsubsection{Yếu tố nội tại}
Vì vậy, bạn cũng phải học cách kiểm soát những yếu tố nội tại:

\paragraph{Kiểm soát hành vi của bạn.} Sử dụng những khoảng thời gian khi công nghệ đã bị ``thuần hóa'' \& tấm biển không-làm-phiền được treo lên để làm quen với việc đơn nhiệm (single-tasking).

Chỉ mở 1 cửa số trên màn hình máy tính.

Toàn tâm toàn ý vào 1 việc cho đến khi hoàn thành hoặc đến điểm dừng đã định sẵn.

Nghỉ ngơi bằng cách tránh xa máy tính của bạn.

Cố gắng ``ngắt kết nối'' hoàn toàn (không công nghệ) trong ít nhất môt tiếng hoặc hơn.

Càng thường xuyên càng tốt.

Đầu tiên, hãy thử trong 15-20 phút; sau đó nâng lên 1 tiếng, hay thậm chí 90 phút.

\paragraph{Kiểm soát tâm trí của bạn.} Đối với nhiều người trong chúng ta, đây là việc khó khăn nhất.

Tâm trí vốn dễ xao lãng.

Tập để ý khi nào tâm trí của bạn đang ``lang thang'', \& nhẹ nhàng hướng sự tập trung về nơi bạn muốn.

Nếu bạn nhớ ra 1 số việc nhỏ nhưng quan trọng khi đang tập trung cho việc khác, hãy ghi nó vào sổ tay \& quay lại sau.

Tương tự với những thông tin bạn muốn tra cứu trên mạng.

\subsubsection{Kết lại}
Rèn luyện khả năng quản lý sự tập trung sẽ không hoàn toàn loại bỏ sự xao lãng.

Kể từ khi bạn nhận ra mình phân tâm \& rèn luyện ``cơ bắp tập trung'', bạn sẽ dần kiểm soát cuộc sống của mình.

Từ đó, bạn có thể cống hiến nhiều hơn cho những điều thật sự quan trọng.

Đừng để sự phân tâm làm chệch hướng những khát khao \& dự định của bạn.

Thay vào đó, hãy kiểm soát sự tập trung để kiểm soát cuộc đời.

%
Maura Thomas là 1 diễn giả \& huấn luyện viên được giải thưởng quốc tế về năng suất cá nhân \& công ty, quản lý sự tập trung \& cân bằng giữa công việc \& cuộc sống.

Cô từng diễn thuyết tại diễn đàn TEDx, là nhà sáng lập của \textit{RegainYourTime} \& tác giả cuốn sách \textit{Personal Productivity Secrets} \& \textit{Work Without Walls}.

Cô được vinh danh là 1 trong Nhà Diễn giả Hàng đầu năm 2018 bởi tạp chí \textit{Inc. Magazine}.\hfill$\square$
\begin{flushright}
	Trạm Đọc - Readstation
	
	Nguồn: Maura Thomas (Harvard Business Review)
\end{flushright}

%------------------------------------------------------------------------------%

\section{\cite{Shapiro2014}. {\sc Dani Shapiro}. Still Writing: The Pleasures \& Perils of a Creative Life}
{\sf[591 Amazon ratings][4610 Goodreads ratings]}
\begin{itemize}
	\item {\sf Amazon review.} ``{\it Still Writing} offers up a cornucopia of wisdom, insights, \& practical lessons gleaned from {\sc Dani Shapiro}'s long experience as a celebrated writer \& teacher of writing. The beneficiaries are beginning writers, veteran writers \& everyone in between.'' -- {\sc Jennifer Egan}
	
	-- ``{\it Still Writing} cung cấp một kho tàng trí tuệ, hiểu biết sâu sắc, \& bài học thực tế được đúc kết từ kinh nghiệm lâu năm của {\sc Dani Shapiro} với tư cách là một nhà văn nổi tiếng \& giáo viên dạy viết. Những người hưởng lợi là những nhà văn mới vào nghề, những nhà văn kỳ cựu \& mọi người ở giữa.'' -- {\sc Jennifer Egan}
	
	From {\sc Dani Shapiro}, bestselling author of {\it Devotion} \& {\it Slow Motion}, comes a witty, heartfelt, \& practical look at exhilarating \& challenging process of storytelling. At once a memoir, a meditation on artistic process, \& advice on craft, {\it Still Writing} is an intimate companion to living a creative life. Writers -- \& anyone with an artistic temperament -- will find inspiration \& comfort in these pages. Offering lessons learned over 20 years of teaching \& writing, {\sc Shapiro} shares her own revealing insights to weave an indispensable almanac for modern writers.
	
	-- Từ {\sc Dani Shapiro}, tác giả bán chạy nhất của {\it Devotion} \& {\it Slow Motion}, một cái nhìn dí dỏm, chân thành, \& thực tế về quá trình kể chuyện đầy phấn khích \& đầy thử thách. Vừa là hồi ký, vừa là sự chiêm nghiệm về quá trình nghệ thuật, \& lời khuyên về nghề thủ công, {\it Still Writing} là người bạn đồng hành thân thiết với cuộc sống sáng tạo. Các nhà văn -- \& bất kỳ ai có tính khí nghệ thuật -- sẽ tìm thấy cảm hứng \& sự thoải mái trong những trang sách này. Với những bài học rút ra trong hơn 20 năm giảng dạy \& viết lách, {\sc Shapiro} chia sẻ những hiểu biết sâu sắc của riêng mình để tạo nên một cuốn niên giám không thể thiếu cho các nhà văn hiện đại.
	\item {\sf From Publisher.}
	\begin{itemize}
		\item From best-selling author of signal fires \& inheritance, an intimate \& eloquent guide to living a creative life
		\item Word after word, sentence after sentence, we build our writing lives
		\item Write a few sentences. Scratch them out. Write a few more.
		\item ``Practical, wise, \& inviting.'' -- Elle
		\item ``Instructive \& inspiring.'' -- Vanity Fair
		\item ``Both beautiful \& useful.'' -- {\sc Susan Orlean}
		\item ``Clear-eyed, honest, \& grounded.'' -- Kirkus reviews
		\item ``I loved it.'' -- {\sc Terry Tempest Williams}
	\end{itemize}
	\item {\sf About Author.} {\sc Dani Shapiro} is author of 11 books, \& host \& creator of hit podcast {\it Family Secrets}. Her most recent novel, {\it Signal Fires}, was named a best book of 2022 by {\it Time Magazine, Washington Post, Amazon}, \& others, \& is a national bestseller. Her most recent memoir, {\it Inheritance}, was an instant New York Times Bestseller, \& named a best book of 2019 by {\it Elle, Vanity Fair, Wired, \& Real Simple}. {\sc Dani}'s work has been published in 14 languages \& she's currently developing {\it Signal Fires} for its television adaptation. {\sc Dani}'s book on process \& craft of writing, {\it Still Writing}, is being reissued on occasion of its 10th anniversary in 2023. She occasionally teaches workshops \& retreats, \& is co-founder of Sirenland Writers Conference in Positano, Italy.
	\item ``I have to get lost so I can invent some way out.'' -- {\sc David Salle}
	\item {\sf Introduction.} I've heard it said that everything you need to know about life can be learned from watching baseball. I'm not what you'd call a sports fan, so I don't know if this is true, but I do believe in a similar philosophy, which is that everything you need to know about life can be learned from a genuine \& ongoing attempt to write.
	
	-- Tôi đã nghe nói rằng mọi thứ bạn cần biết về cuộc sống đều có thể học được từ việc xem bóng chày. Tôi không phải là người hâm mộ thể thao, vì vậy tôi không biết điều này có đúng không, nhưng tôi tin vào một triết lý tương tự, đó là mọi thứ bạn cần biết về cuộc sống đều có thể học được từ một nỗ lực viết thực sự \& liên tục.
	
	At least this has been the case for me.
	
	I have been writing all my life. Growing up, I wrote in soft-covered journals, in spiral-bound notebooks, in diaries with locks \& keys. I wrote love letters \& lies, stories \& missives. When I wasn't writing, I was reading. \& when I wasn't writing or reading, I was staring out window, lost in thought. Life was elsewhere -- I was sure of it -- \& writing was what took me there. In my notebooks, I escaped an unhappy \& lonely childhood. I tried to make sense of myself. I had no intention of becoming a writer. I didn't know that becoming a writer was possible. Still, writing was what saved me. It presented me with a window into infinite. \fbox{It allowed me to create order out of chaos.}
	
	-- Tôi đã viết suốt cuộc đời mình. Khi lớn lên, tôi viết trong những cuốn nhật ký bìa mềm, trong những cuốn sổ tay gáy lò xo, trong những cuốn nhật ký có ổ khóa \& chìa khóa. Tôi viết những bức thư tình \& những lời nói dối, những câu chuyện \& những lá thư. Khi không viết, tôi đọc. \& khi không viết hoặc đọc, tôi nhìn chằm chằm ra ngoài cửa sổ, chìm đắm trong suy nghĩ. Cuộc sống ở nơi khác -- tôi chắc chắn về điều đó -- \& viết chính là thứ đưa tôi đến đó. Trong những cuốn sổ tay của mình, tôi đã trốn thoát khỏi một tuổi thơ bất hạnh \& cô đơn. Tôi đã cố gắng hiểu rõ bản thân mình. Tôi không có ý định trở thành một nhà văn. Tôi không biết rằng trở thành một nhà văn là điều có thể. Tuy nhiên, viết chính là thứ đã cứu tôi. Nó mở ra cho tôi một cánh cửa sổ vào vô cực. \fbox{Nó cho phép tôi tạo ra trật tự từ sự hỗn loạn.}
	
	Of course, there's a huge difference between scribblings of a young girl in her journals -- I would never get out from under my bed if anyone were ever to read them -- \& sustained, grown-up work of crafting sth resonant \& lasting, a story that might shed light on our human condition. ``The good writer,'' {\sc Ralph Waldo Emerson} noted in his journal, ``seems to be writing about himself, but has his eye always on that thread of universe which runs through himself \& all things.''
	
	-- Tất nhiên, có một sự khác biệt lớn giữa những dòng chữ nguệch ngoạc của một cô gái trẻ trong nhật ký của cô ấy -- Tôi sẽ không bao giờ ra khỏi gầm giường nếu có ai đó đọc chúng -- \& công việc bền bỉ, trưởng thành của việc tạo ra thứ gì đó cộng hưởng \& lâu dài, một câu chuyện có thể làm sáng tỏ tình trạng con người của chúng ta. ``Nhà văn giỏi,'' {\sc Ralph Waldo Emerson} đã lưu ý trong nhật ký của mình, ``có vẻ như đang viết về chính mình, nhưng luôn để mắt đến sợi chỉ vũ trụ chạy qua chính mình \& mọi thứ.''
	
	Sitting down to write isn't easy. A few years ago, a local high school asked me if a student who is interested in becoming a writer might come \& observe me. Observe me! I had to decline. I couldn't imagine what poor student would think, watching me sit, then stand, sit again, decide that I needed more coffee, go downstairs \& make coffee, come back up, sit again, get up, comb my hair, sit again, stare at screen, check e-mail, stand up, pet dog, sit again $\ldots$
	
	-- Ngồi xuống để viết không phải là điều dễ dàng. Vài năm trước, một trường trung học địa phương đã hỏi tôi rằng liệu một học sinh nào đó muốn trở thành nhà văn có thể đến \& quan sát tôi không. Quan sát tôi! Tôi đã phải từ chối. Tôi không thể tưởng tượng được học sinh tội nghiệp đó sẽ nghĩ gì khi nhìn tôi ngồi, rồi đứng dậy, ngồi lại, quyết định rằng tôi cần thêm cà phê, xuống cầu thang \& pha cà phê, quay lại, ngồi lại, đứng dậy, chải tóc, ngồi lại, nhìn chằm chằm vào màn hình, kiểm tra email, đứng dậy, vuốt ve con chó, lại ngồi xuống $\ldots$
	
	You get the picture.
	
	Writing life requires courage, patience, persistence, empathy, openness, \& ability to deal with rejection. It requires willingness to be alone with oneself. To be gentle with oneself. To look at world without blinders on. To observe \& withstand what one sees. To be disciplined, \& at same time, take risks. To be willing to fail -- not just once, but again \& again, over course of a lifetime. ``Ever tried, ever failed,'' {\sc Samuel Beckett} once wrote. ``No matter. Try again. Fail again. Fail better.'' It requires what great editor {\sc Ted Solotoroff} once called {\it endurability}. It is this quality, most of all, that I think of when I look around a classroom at a group of aspiring writers. Some of them will be more gifted than others. Some of them will be driven, ambitious for success or fame, rather than by determination to do their best possible work. But of students I have taught, it is not necessarily most gifted, or the ones most focused on imminent literary fame (I think of these as short sprinters), but the ones who endure, who are still writing, decades later.
	
	-- Cuộc sống viết lách đòi hỏi lòng dũng cảm, sự kiên nhẫn, bền bỉ, sự đồng cảm, sự cởi mở, \& khả năng đối mặt với sự từ chối. Nó đòi hỏi sự sẵn lòng ở một mình với chính mình. Để nhẹ nhàng với chính mình. Để nhìn thế giới mà không bị che mắt. Để quan sát \& chịu đựng những gì mình nhìn thấy. Để có kỷ luật, \& đồng thời, chấp nhận rủi ro. Để sẵn sàng thất bại -- không chỉ một lần, mà là nhiều lần \& nhiều lần, trong suốt cuộc đời. ``Đã từng thử, đã từng thất bại,'' {\sc Samuel Beckett} đã từng viết. ``Không sao cả. Hãy thử lại. Thất bại lại. Thất bại tốt hơn.'' Nó đòi hỏi thứ mà biên tập viên vĩ đại {\sc Ted Solotoroff} từng gọi là {\it sức bền}. Đây chính là phẩm chất mà tôi nghĩ đến nhiều nhất khi nhìn quanh lớp học và thấy một nhóm các nhà văn đầy tham vọng. Một số người trong số họ sẽ có năng khiếu hơn những người khác. Một số người trong số họ sẽ có động lực, tham vọng thành công hoặc nổi tiếng, thay vì quyết tâm làm tốt nhất có thể công việc của mình. Nhưng trong số những học sinh tôi đã dạy, không nhất thiết họ là những người có năng khiếu nhất, hoặc những người tập trung nhất vào sự nổi tiếng trong văn chương (tôi coi họ là những vận động viên chạy nước rút), mà là những người bền bỉ, những người vẫn viết lách, sau nhiều thập kỷ.
	
	It is my hope that -- whether you're a writr or not -- this book will help you to discover or rediscover qualities necessary for a creative life. We are all unsure of ourselves. Everyone 1 of us walking the planet wonders, secretly, if we are getting it wrong. We stumble along. We love \& we lose. At times, we find unexpected strength, \& at other times, we succumb to our fears. We are impatient. We want to know what's around corner, \& writing life won't offer us this. It forces us into the here \& now. There is only this moment, when we put pen to page.
	
	-- Tôi hy vọng rằng -- dù bạn có phải là nhà văn hay không -- cuốn sách này sẽ giúp bạn khám phá hoặc tái khám phá những phẩm chất cần thiết cho một cuộc sống sáng tạo. Tất cả chúng ta đều không chắc chắn về bản thân mình. Mỗi người trong chúng ta bước đi trên hành tinh này đều tự hỏi, trong thâm tâm, liệu chúng ta có đang làm sai không. Chúng ta vấp ngã. Chúng ta yêu \& chúng ta mất mát. Đôi khi, chúng ta tìm thấy sức mạnh bất ngờ, \& đôi khi, chúng ta khuất phục trước nỗi sợ hãi của mình. Chúng ta thiếu kiên nhẫn. Chúng ta muốn biết điều gì đang chờ đợi mình, \& cuộc sống viết lách sẽ không mang lại cho chúng ta điều đó. Nó buộc chúng ta phải ở đây \& ngay bây giờ. Chỉ có khoảnh khắc này, khi chúng ta đặt bút lên trang giấy.
	
	Had I not, as a young woman, discovered that I was a writer, had I not met some extraordinarily generous role models \& teachers \& mentors who helped me along way, had I not begun to forge a path out of my own personal wilderness with words, I might not be here to tell this story. I was spinning, whirling, without any sense of who I was, or what I was made of. I was slowly, quietly killing myself. But after writing saved my life, the practice of it also became my teacher. It is impossible to spend your days writing \& not begin to know your own mind.
	
	-- Nếu như tôi không, khi còn là một phụ nữ trẻ, khám phá ra rằng mình là một nhà văn, nếu tôi không gặp một số hình mẫu vô cùng hào phóng \& giáo viên \& cố vấn đã giúp đỡ tôi trên chặng đường, nếu tôi không bắt đầu mở đường thoát khỏi sự hoang dã của riêng mình bằng những từ ngữ, thì có lẽ tôi đã không ở đây để kể câu chuyện này. Tôi đã quay cuồng, quay cuồng, không có bất kỳ cảm giác nào về việc mình là ai, hay mình được tạo nên từ điều gì. Tôi đã từ từ, lặng lẽ tự giết mình. Nhưng sau khi viết đã cứu mạng tôi, thì việc thực hành nó cũng đã trở thành người thầy của tôi. Không thể dành cả ngày để viết \& mà không bắt đầu hiểu được tâm trí của chính mình.
	
	The page is your mirror. What happens inside you is reflected back. You come face-to-face with your own resistance, lack of balance, self-loathing, \& insatiable ego -- \& also with your singular vision, guts, \& fortitude. No matter what you've achieved the day before, you begin each day at bottom of mountain. Isn't this true for most of us? A surgeon about to perform a difficult operation is at bottom of mountain. A lawyer delivering a closing argument. An actor waiting in the wings. A teacher on 1st day of school. Sometimes we may think that we're in charge, or that we have things figured out. Life is usually right there, though, ready to knock us over when we get too sure of ourselves. Fortunately, if we have learned lessons that years of practice have taught us, when this happens, we endure. We fail better. We sit up, dust ourselves off, \& begin again.
	
	-- Trang giấy là tấm gương của bạn. Những gì xảy ra bên trong bạn được phản chiếu lại. Bạn đối mặt với sự kháng cự, mất cân bằng, tự ghét, \& cái tôi không thể thỏa mãn -- \& cũng như tầm nhìn, lòng can đảm, \& sự kiên cường của riêng bạn. Bất kể bạn đã đạt được điều gì vào ngày hôm trước, bạn đều bắt đầu mỗi ngày ở chân núi. Điều này không đúng với hầu hết chúng ta sao? Một bác sĩ phẫu thuật sắp thực hiện một ca phẫu thuật khó đang ở chân núi. Một luật sư đưa ra lời biện hộ kết thúc. Một diễn viên đang chờ đợi trong cánh gà. Một giáo viên vào ngày đầu tiên đi học. Đôi khi chúng ta có thể nghĩ rằng mình đang chịu trách nhiệm hoặc chúng ta đã hiểu rõ mọi thứ. Tuy nhiên, cuộc sống thường ở ngay đó, sẵn sàng đánh gục chúng ta khi chúng ta quá tự tin vào bản thân. May mắn thay, nếu chúng ta đã học được những bài học mà nhiều năm thực hành đã dạy cho chúng ta, khi điều này xảy ra, chúng ta sẽ chịu đựng. Chúng ta thất bại tốt hơn. Chúng ta ngồi dậy, phủi bụi trên người, \& bắt đầu lại.
	
	``Endings are elusive, middles are nowhere to be found, but worst of all is to begin, to begin, to begin!'' -- {\sc Donald Barthelme}
	
	-- Kết thúc thì khó nắm bắt, phần giữa thì chẳng thấy đâu, nhưng tệ nhất là bắt đầu, bắt đầu, bắt đầu!
	\item {\sf Scars.} I grew up the only child of older parents. If I were to give you a list of all facts of my early life that made me a writer, this one would be near the top. {\it Only child. Older parents.} It now almost seems like a job requirement -- though back then, I wished it to be otherwise.  A lonely, isolated childhood isn't a prerequisite for a writing life, of course, but it certainly helped. My parents were observant Jews. We kept a kosher home. On Sabbath, from sundown on Friday evening until sundown on Saturday, we didn't drive, we didn't turn on lights, or radio, or television, \& I wasn't allowed to ride my bike, or play piano, or do homework. This left me with a lot of time to do nothing. Most Saturday mornings, I walked a half-mile to synagogue with my father while my mother stayed home with a sinus headache.
	
	-- {\sf Scars.} Tôi lớn lên là đứa con một của cha mẹ già. Nếu tôi đưa cho bạn danh sách tất cả các sự kiện trong cuộc sống thời thơ ấu khiến tôi trở thành một nhà văn, thì điều này sẽ ở gần đầu danh sách. {\it Con một. Cha mẹ già.} Bây giờ nó gần như là một yêu cầu công việc -- mặc dù trước đây, tôi ước nó không phải như vậy. Tất nhiên, một tuổi thơ cô đơn, biệt lập không phải là điều kiện tiên quyết cho cuộc sống viết lách, nhưng nó chắc chắn đã giúp ích. Cha mẹ tôi là người Do Thái sùng đạo. Chúng tôi giữ một ngôi nhà theo luật kosher. Vào ngày Sa-bát, từ lúc mặt trời lặn vào tối thứ Sáu cho đến khi mặt trời lặn vào thứ Bảy, chúng tôi không lái xe, không bật đèn, không bật radio, không bật tivi, \& Tôi không được phép đi xe đạp, không chơi piano hoặc làm bài tập về nhà. Điều này khiến tôi có nhiều thời gian để không làm gì cả. Hầu hết các sáng thứ Bảy, tôi đi bộ nửa dặm đến giáo đường Do Thái với cha tôi trong khi mẹ tôi ở nhà vì đau đầu do viêm xoang.
	
	Our house was silent \& spotless. Dirt, smudges, noise -- any kind of disarray would have been unthinkable. Housekeepers were always quitting. No one could keep house to my mother's standards. Every surface gleamed. Picture frames were dusted daily. Sheets \& pillowcases were ironed 3 times a week. My drawers were color-coordinated: blue Danskin tops perfectly folded next to blue Danskin bottoms. The exterminator came monthly. The toxic mold guy made binannual visits. Summers, the lawn man came every few days with his mower \& hedge trimmer, clipping our suburban New Jersey acre into shape.
	
	-- Ngôi nhà của chúng tôi im lặng \& sạch sẽ. Bụi bẩn, vết ố, tiếng ồn -- bất kỳ sự lộn xộn nào cũng không thể tưởng tượng được. Người giúp việc luôn nghỉ việc. Không ai có thể giữ nhà cửa theo tiêu chuẩn của mẹ tôi. Mọi bề mặt đều sáng bóng. Khung ảnh được phủi bụi hàng ngày. Ga trải giường \& vỏ gối được ủi 3 lần một tuần. Ngăn kéo của tôi được phối hợp màu sắc: phần trên Danskin màu xanh được gấp hoàn hảo bên cạnh phần dưới Danskin màu xanh. Người diệt côn trùng đến hàng tháng. Người chuyên xử lý nấm mốc độc hại đến thăm hai lần một năm. Summers, người làm vườn đến vài ngày một lần với máy cắt cỏ \& máy cắt tỉa hàng rào, cắt tỉa mẫu Anh ngoại ô New Jersey của chúng tôi cho gọn gàng.
\end{itemize}


%------------------------------------------------------------------------------%

%------------------------------------------------------------------------------%

%------------------------------------------------------------------------------%

%------------------------------------------------------------------------------%

\section{Wikipedia}

\subsection{Wikipedia{\tt/}style guide}
``A {\it style guide} is a set of standards for the writing, \href{https://en.wikipedia.org/wiki/Typesetting}{formatting}, \& design of \href{https://en.wikipedia.org/wiki/Document}{documents}. A book-length style guide is often called a {\it style manual} or a {\it manual of style} (MoS or MOS). A short style guide, typically ranging from several to several dozen pages, is often called a {\it style sheet}. The standards documented in a style guide are applicable either for general use, or prescribed use for an individual publication, particular organization, or specific field.

A style guide establishes standard style requirements to improve \href{https://en.wikipedia.org/wiki/Communication}{communication} by ensuring consistency within \& across documents. They may require certain \href{https://en.wikipedia.org/wiki/Best_practice}{best practices} in \href{https://en.wikipedia.org/wiki/Writing_style}{writing style}, \href{https://en.wikipedia.org/wiki/Usage_(language)}{usage}, \href{https://en.wikipedia.org/wiki/Composition_(language)}{language composition}, \href{https://en.wikipedia.org/wiki/Composition_(visual_arts)}{visual composition}, \href{https://en.wikipedia.org/wiki/Orthography}{orthography}, \& \href{https://en.wikipedia.org/wiki/Typography}{typography} by setting standards of usage in areas such as \href{https://en.wikipedia.org/wiki/Punctuation}{punctuation}, \href{https://en.wikipedia.org/wiki/Capitalization}{capitalization}, \href{https://en.wikipedia.org/wiki/Citation#Styles}{citing sources}, formatting of numbers \& dates, \href{https://en.wikipedia.org/wiki/Table_(information)}{table} appearance \& other areas. For \href{https://en.wikipedia.org/wiki/Academic_publishing}{academic} \& \href{https://en.wikipedia.org/wiki/Technical_communication}{technical} documents, a guide may also enforce the best practice in \href{https://en.wikipedia.org/wiki/Ethics}{ethics} (e.g. \href{https://en.wikipedia.org/wiki/Author}{authorship}, \href{https://en.wikipedia.org/wiki/Research_ethics}{research ethics}, \& disclosure) \& compliance (\href{https://en.wikipedia.org/wiki/Technical_standard}{technical} \& \href{https://en.wikipedia.org/wiki/Regulatory_compliance}{regulatory}). For translation, a style guide may even be used to enforce consistent grammar, tones, \& localization decisions such as \href{https://en.wikipedia.org/wiki/Unit_of_measurement}{units of measure}.

Style guides are specialized in a variety of ways, from the general use of a broad public audience, to a wide variety of specialized uses (such as for students \& scholars of various \href{https://en.wikipedia.org/wiki/Academic_discipline}{acadmeic disciplines}, \href{https://en.wikipedia.org/wiki/Medical_publishing}{medicine}, \href{https://en.wikipedia.org/wiki/Journalism}{journalism}, the \href{https://en.wikipedia.org/wiki/Legal_publication}{law}, \href{https://en.wikipedia.org/wiki/Government_Publishing_Office}{government}, business, \& specific \href{https://en.wikipedia.org/wiki/Industry_(economics)}{industries}). The term {\it house style} refers to the conventions defined by the style guide of a particular \href{https://en.wikipedia.org/wiki/Publisher}{publisher} or other organization.

\subsubsection{Varieties}
Style guides vary widely in scope \& size. Writers working in many large industries or professional sectors reference a specific style guide, written for their usage in specialized documents within their fields. For the most part, these guides are relevant \& useful for peer-to-peer specialist documentation or to help writers working in specific industries or sectors communicate highly technical information in scholarly articles or industry \href{https://en.wikipedia.org/wiki/White_paper}{white papers}.

Professional style guides of different countries can be referenced for authoritative advice on their respective language(s), e.g., the \href{https://en.wikipedia.org/wiki/New_Oxford_Style_Manual}{New Oxford Style Manual} from \href{https://en.wikipedia.org/wiki/Oxford_University_Press}{Oxford University Press}, UK; \& \href{https://en.wikipedia.org/wiki/The_Chicago_Manual_of_Style}{The Chicago Manual of Style} from the \href{https://en.wikipedia.org/wiki/University_of_Chicago_Press}{University of Chicago Press}, US; both Australia \& Canada have style guides -- available online -- created by their governments.

\paragraph{Sizes.} See also \href{https://en.wikipedia.org/wiki/List_of_style_guides}{Wikipedia{\tt/}list of style guides}. The variety in scope \& length is enabled by the cascading of 1 style over another, analogous to how styles cascade \href{https://en.wikipedia.org/wiki/Style_sheet_(web_development)}{in web development} \& \href{https://en.wikipedia.org/wiki/Style_sheet_(desktop_publishing)}{in desktop} cascade over \href{https://en.wikipedia.org/wiki/CSS}{CSS} styles.

In many cases, a project such as a \href{https://en.wikipedia.org/wiki/Book}{book}, \href{https://en.wikipedia.org/wiki/Academic_journal}{journal}, or \href{https://en.wikipedia.org/wiki/Monograph}{monograph} series typically has a short style sheet that cascades over the somewhat larger style guide of an organization such as a \href{https://en.wikipedia.org/wiki/Publishing}{publishing} company, whose specific content is usually called {\it house style}. Most house styles, in turn, cascade over an industry-wide style manual that is even more comprehensive. Examples of industry style guides include:
\begin{enumerate}
	\item \href{https://en.wikipedia.org/wiki/The_Associated_Press_Stylebook}{\it The Associated Press Stylebook} (AP Stylebook) \& {\it The Canadian Press Stylebook} for journalism
	\item \href{https://en.wikipedia.org/wiki/The_Chicago_Manual_of_Style}{\it The Chicago Manual of Style} (CMoS) \& \href{https://en.wikipedia.org/wiki/Hart%27s_Rules}{Oxford style} for general academic writing \& publishing
	\item \href{https://en.wikipedia.org/wiki/MHRA_Style_Guide}{Modern Humanities Research Association (MHRA) style} \& \href{https://en.wikipedia.org/wiki/ASA_style}{American Sociological Association (ASA) style} for the arts \& humanities
	\item \href{https://en.wikipedia.org/wiki/Oxford_Standard_for_Citation_of_Legal_Authorities}{Oxford Standard for Citation of Legal Authorities} (OSCOLA) \& \href{https://en.wikipedia.org/wiki/Bluebook}{Bluebook style} for law
	\item \href{https://en.wikipedia.org/wiki/GPO_style_manual}{US Government Publishing Office (USGPO) style} \& \href{https://en.wikipedia.org/wiki/Australian_Government_Publishing_Service}{Australian Government Publishing Service (AGPS) style} for government publication
\end{enumerate}
Finally, these reference works cascade over the \href{https://en.wikipedia.org/wiki/Orthography}{orthographic} norms of the language in use (e.g., \href{https://en.wikipedia.org/wiki/English_orthography}{English orthography} for English-language publications). This, of course, may be subject to national variety, e.g., \href{https://en.wikipedia.org/wiki/Comparison_of_American_and_British_English}{British, American, Canadian, \& Australian English}.

\paragraph{Topics.} Some style guides focus on specific topic areas e.g. \href{https://en.wikipedia.org/wiki/Graphic_design}{graphic design}, including \href{https://en.wikipedia.org/wiki/Typography}{typography}. Website style guides cover a publication's visual \& technical aspects as well as text.

Guides in specific scientific \& technical fields may cover \href{https://en.wikipedia.org/wiki/Nomenclature}{nomenclature} to specify names of classifying labels that are clear, standardized, \& \href{https://en.wikipedia.org/wiki/Ontology}{ontologically} sound (e.g., \href{https://en.wikipedia.org/wiki/Taxonomy_(biology)}{taxonomy}, \href{https://en.wikipedia.org/wiki/Chemical_nomenclature}{chemical nomenclature}, \& \href{https://en.wikipedia.org/wiki/Gene_nomenclature}{gene nomenclature}).

Style guides that cover \href{https://en.wikipedia.org/wiki/Usage_(language)}{usage} may suggest descriptive terms for people which avoid \href{https://en.wikipedia.org/wiki/Racism}{racism}, \href{https://en.wikipedia.org/wiki/Sexism}{sexism}, \href{https://en.wikipedia.org/wiki/Homophobia}{homophobia}, etc. Style guides increasingly incorporate \href{https://en.wikipedia.org/wiki/Accessibility}{accessibility} conventions for audience members with visual, mobility, or other disabilities.

\paragraph{Web style guides.} Since the rise of the digital age, websites have allowed for an expansion of style guide conventions that account for digital behavior such as screen reading (reading from a digitalized screen rather than a physical document). Screen reading requires web style guides to focus more intently on a user experience subjected to multichannel surfing. Though web style guides can also vary widely, they tend to prioritize similar values concerning brevity, terminology, syntax, tone, structure, typography, graphics, \& errors.

\subsubsection{Updating}
Most style guides are revised periodically to accommodate changes in conventions \& usage. The frequency of updating \& the \href{https://en.wikipedia.org/wiki/Revision_control}{revision control} are determined by the subject. For style manuals in \href{https://en.wikipedia.org/wiki/Reference_work}{reference-work} format, new \href{https://en.wikipedia.org/wiki/Edition_(book)}{editions} typically appear every 1--20 years. E.g., the {\it Ap Stylebook} is revised annually, \& the Chicago, \& ASA manuals are in their 17th, 7th, \& 6th editions, respectively, as of 2023. Many house styles \& individual project styles change more frequently, especially for new projects.'' -- \href{https://en.wikipedia.org/wiki/Style_guide}{Wikipedia{\tt/}style guide}

%------------------------------------------------------------------------------%

\subsection{Wikipedia{\tt/}English writing style}
``An {\it English writing style} is a combination of features in an \href{https://en.wikipedia.org/wiki/English_language}{English language} \href{https://en.wikipedia.org/wiki/Writing}{composition} that has become characteristics of a particular writer, a genre, a particular organization, or a profession more broadly (e.g., \href{https://en.wikipedia.org/wiki/Legal_writing}{legal writing}).

An individual's writing style may be distinctive for particular themes, personal idiosyncrasies of phrasing \&{\tt/}or \href{https://en.wikipedia.org/wiki/Idiolect}{idiolet}; recognizable combinations of these patterns may be defined metaphorically as a writer's ``voice.''

Organizations that employ writers or commission written work from individuals may require that writers conform to a ``\href{https://en.wikipedia.org/wiki/Style_guide}{house style}'' defined by the organization. This conformity enables a more consistent readability of composite works produced by many authors \& promotes usability of, e.g., references to other cited works.

In many kinds of professional writing aiming for effective transfer of information, adherence to a standardized style can facilitate the comprehension of readers who are already accustomed to it. Many of these standardized styles are documented in \href{https://en.wikipedia.org/wiki/Style_guide}{style guides}.

\subsubsection{Personal styles}
All writing has some style, even if the author is not thinking about a personal style. It is important to understand that style reflects meaning. E.g., if a writer wants to express a sense of euphoria\footnote{an extremely strong feeling of happiness \& excitement that usually lasts only a short time.}, he or she might write in a style overflowing with expressive modifiers. Some writers use styles that are very specific, e.g. in pursuit of an artistic effect. Stylistic rule-breaking is exemplified by the poet. An example is \href{https://en.wikipedia.org/wiki/E._E._Cummings}{E. E. Cummings}, whose writing consists mainly of only \href{https://en.wikipedia.org/wiki/Lower_case}{lower case} letters, \& often uses unconventional \href{https://en.wikipedia.org/wiki/Typography}{typography}, \href{https://en.wikipedia.org/wiki/White_space_(visual_arts)}{spacing}, \& \href{https://en.wikipedia.org/wiki/Punctuation}{punctuation}. Even in non-artistic writing, every person who writes has his or her own personal style.

\subsubsection{Proprietary styles}
Many large publications define a house style to be used throughout the publication, a practice almost universal among newspapers \& well-known magazines. These styles can cover the means of expression \& sentence structures, such as those adopted by \href{https://en.wikipedia.org/wiki/Time_(magazine)#Style}{\it Time}. They may also include features peculiar to a publication; the practice at \href{https://en.wikipedia.org/wiki/The_Economist#Tone_and_voice}{\it The Economist}, e.g., is that articles are rarely attributed to an individual author. General characteristics have also been prescribed for different categories of writing, e.g. in \href{https://en.wikipedia.org/wiki/News_style}{journalism}, the use of \href{https://en.wikipedia.org/wiki/SI#SI_writing_style}{SI units}, or \href{https://en.wikipedia.org/wiki/Questionnaire_construction}{questionnaire construction}.

\subsubsection{Academic styles}
University students, especially graduate students, are encouraged to write papers in an approved style. This practice promotes readability \& ensures that references to cited works are noted in a uniform way. Typically, students are encouraged to use a style commonly adopted by journals publishing articles in the field of study. The list of {\it Style Manuals \& Guides}, from the \href{https://en.wikipedia.org/wiki/University_of_Memphis}{University of Memphis} Libraries, includes 30 academic style manuals that are currently in print, \& 12 that are available online. Citation of referenced works is a key element in academic style.

The requirements for writing \& citing articles accessed online may sometimes differ from those for writing \& citing printed works. Some of the details are covered in {\it The Columbia Guide to Online Style}.'' -- \href{https://en.wikipedia.org/wiki/English_writing_style}{Wikipedia{\tt/}English writing style}

%------------------------------------------------------------------------------%

\subsection{Wikipedia{\tt/}rat race}
``A {\it rat race} is an endless, self-defeating, or pointless pursuit. The phrase equates humans to rats attempting to earn a reward such as cheese, in vain. It may also refer to a competitive struggle to get ahead financially or routinely.

The term is commonly associated with an exhausting, repetitive lifestyle that leaves no time for relaxation or enjoyment.

\subsubsection{Etymology}
In the late 1800s, the term ``rat-run'' was used meaning ``maze-like passages by which rats move about their territory'', commonly used in a derogatory sense.

By the 1930s actual rat races of some sort are frequently mentioned among carnival \& gambling attractions.

By 1934, ``rat-race'' was also used in reference to aviation training, referring to a ``\href{https://en.wikipedia.org/wiki/Follow-the-leader}{follow-the-leader}'' game in which a trainee fighter pilot had to copy all the actions (loops, rolls, spins, \href{https://en.wikipedia.org/wiki/Immelmann_turn}{Immelmann turns} etc.) performed by an experienced pilot.

From 1939, the phrase took on the meaning of ``competitive struggle'' referring to a person's work \& life.

\subsubsection{Historical usage}
\href{https://en.wikipedia.org/wiki/The_Rat_Race_(novel)}{\it The Rat Race} was used as a title for a novel written by \href{https://en.wikipedia.org/wiki/Jay_Franklin}{\sc Jay Franklin} in 1947 for \href{https://en.wikipedia.org/wiki/Colliers_Magazine}{Colliers Magazine} \& 1st published in book form in 1950. It is dedicated {\it To those few rats in Washington who do not carry brief-cases}.

The term ``rat race'' was used in an article about \href{https://en.wikipedia.org/wiki/Samuel_Goudsmit}{\sc Samuel Goudsmit} published in 1953 entitled: {\it A Farewell to String \& Sealing Wax\~I} in which \href{https://en.wikipedia.org/wiki/Daniel_Lang_(writer)}{\sc Daniel Lang} wrote:
\begin{quote}
	Sometimes when his sardonic mood is on him, he wonders whether the \href{https://en.wikipedia.org/wiki/Synchrotron}{synchrotrons}, the \href{https://en.wikipedia.org/wiki/Betatron}{betatrons}, the \href{https://en.wikipedia.org/wiki/Cosmotron}{cosmotrons}, \& all the other contrivances physicists have lately rigged up to create energy by accelerating particles of matter aren't playing a wry joke on their inventors. ``They are accelerating us too,'' he says, in a voice that still betrays a trace of the accent of his native Holland. In protesting against the speedup, {\sc Goudsmit} can speak with authority, for in the course of only a few years, he, like many other contemporary physicists, has seen his way of life change from a tranquil one of contemplation of a rat race.
\end{quote}
\href{https://en.wikipedia.org/wiki/Philip_K._Dick}{\sc Philip K. Dick} used the term in ``\href{https://en.wikipedia.org/wiki/The_Last_of_the_Masters}{The Last of the Masters}'' published in 1954:
\begin{quote}
	``Maybe,'' McLean said softly, ``you \& I can then get off this rat race. You \& I \& all the rest of us. \& live like human beings.'' ``Rat race,'' Fowler murmured. ``Rats in a maze. Doing tricks. Performing chores thought up by somebody else.'' McClean caught Fowler's eye. ``By somebody of another species.''
\end{quote}
\href{https://en.wikipedia.org/wiki/Jim_Bishop}{\sc Jim Bishop} used the term rat race in his book {\it The Golden Ham: A Candid Biography of Jackie Gleason}. The term occurs in a letter \href{https://en.wikipedia.org/wiki/Jackie_Gleason}{\sc Jackie Gleason} wrote to his wife in which he says: ``Television is a rat race, \& remember this, even if you win you are still a rat.''

\href{https://en.wikipedia.org/wiki/William_H._Whyte}{\sc William H. Whyte} used the term rat race in \href{https://en.wikipedia.org/wiki/The_Organization_Man}{The Organization Man} published in 1956:
\begin{quote}
	The word {\it collective} most of them can't bring themselves to use -- except to describe foreign countries or organizations they don't work for -- but they are keenly aware of how much more deeply beholden they are to organization than were their elders. They are wry about it, to be sure; they talk of the ``treadmill,'' the ``rat race,'' of the inability to control one's direction.
\end{quote}
\href{https://en.wikipedia.org/wiki/Merle_A._Tuve}{\sc Merle A. Tuve} used the term rat race in a 1959 article entitled ``Is Science Too Big for the Scientist?'', writing:
\begin{quote}
	There is a growing conviction among many of my friends in academic circles that the university today is no place for a scholar in science. A professor's life nowadays is a rat-race of busyness \& activity, managing contracts \& projects, guiding teams of assistants, bossing crews of technicians, making numerous trips, sitting on committees for government agencies, \& engaging in other distractions necessary to keep the whole frenetic business from collapse.
\end{quote}

\subsubsection{Solutions}
``Escaping the rat race'' can have a number of different meanings:
\begin{itemize}
	\item Movement from work or geographical location into (typically) a more rural area
	\item \href{https://en.wikipedia.org/wiki/Retirement}{Retirement}, quitting or ceasing work
	\item Moving from a job of high strenuosity to 1 of lesser strenuosity, like the \href{https://en.wikipedia.org/wiki/Tang_ping}{tang ping} lifestyle of young Chinese
	\item Adopting a \href{https://en.wikipedia.org/wiki/Buddha-like_mindset}{Buddha-like mindset}
	\item Changing to a different job altogether
	\item \href{https://en.wikipedia.org/wiki/Remote_work}{Remote work}
	\item Becoming \href{https://en.wikipedia.org/wiki/Financial_independence}{financially independent} from an employer
	\item Living in harmony with nature
	\item Developing an inner attitude of detachment from materialistic pursuits
	\item Alienation from the norms of society
\end{itemize}

\subsubsection{Music}
[$\ldots$]'' -- \href{https://en.wikipedia.org/wiki/Rat_race}{Wikipedia{\tt/}rat race}

%------------------------------------------------------------------------------%

\subsection{Wikipedia{\tt/}Town Mouse \& Country Mouse}
``{\it The Town Mouse \& the Country Mouse}'' is ` of \href{https://en.wikipedia.org/wiki/Aesop%27s_Fables}{Aesop's Fables}. It is number 352 in the \href{https://en.wikipedia.org/wiki/Perry_Index}{Perry Index} \& type 112 in \href{https://en.wikipedia.org/wiki/Aarne%E2%80%93Thompson}{Aarne--Thompson}'s folk tale index. Like several other elements in Aesop's fables, ``town mouse \& country mouse'' has become an English idiom.

\subsubsection{Story}
In the original tale, a proud town mouse visits his cousin in the country. The country mouse offers the city mouse a meal of simple country cuisine, at which the visitor scoffs \& invites the country mouse back to the city for a taste of the ``fine life'' \& the 2 cousins dine on \href{https://en.wikipedia.org/wiki/White_bread}{white bread} \& other fine foods. But their rich feast is interrupted by a cat which fores the rodent cousins to abandon their meal \& retreat back into their mouse hole for safety. The town mouse tells the country mouse that that cat killed his mother \& father \& that he is frequently the target of attacks. After hearing this, the country mouse decides to return home, preferring security to opulence or, as the 13th-century preacher \href{https://en.wikipedia.org/wiki/Odo_of_Cheriton}{Odo of Cheriton} phrased it, ``I'd rather gnaw a bean than be gnawed by continual fear''.

\subsubsection{Spread}
The story was widespread in Classical times \& there is an early Greek version by \href{https://en.wikipedia.org/wiki/Babrius}{\sc Babrius} (Fable 108). \href{https://en.wikipedia.org/wiki/Horace}{\sc Horace} included it as part of 1 of his satires (II.6), ending on this story in a poem comparing town living unfavorably to life in the country. \href{https://en.wikipedia.org/wiki/Marcus_Aurelius}{\sc Marcus Aurelius} alludes to it in his \href{https://en.wikipedia.org/wiki/Meditations}{Meditations}, Book 11.22; ``Think of the country mouse \& of the town mouse, \& of the alarm \& trepidation\footnote{great worry or fear about something unpleasant that may happen.} of the town mouse''.

However, it seems to have been the 12th century Anglo-Norman write \href{https://en.wikipedia.org/wiki/Walter_of_England}{Walter of England} who contributed most to the spread of the fable throughout medieval Europe. His Latin version (or that of Odo of Cheriton) has been credited as the source of the fable that appeared in the Spanish {\it Libro de Buen Amor} of \href{https://en.wikipedia.org/wiki/Juan_Ruiz}{\sc Juan Ruiz} in the 1t half of the 14th century. Walter has also the source for several manuscript collections of Aesop's fables in Italian \& equally of the popular {\it Esopi fabulas} by Accio Zucco da Sommacampagna, the 1st printed collection of Aesop's fables in that language (Verona, 1479), in which the story of the town mouse \& the country mouse appears as fable 12. This consists of 2 sonnets, the 1st of which tells the story \& the 2nd contains a moral reflection.

\subsubsection{British variations}

\subsubsection{Eastern analogies}

\subsubsection{Later adaptations}
[$\ldots$]'' -- \href{https://en.wikipedia.org/wiki/The_Town_Mouse_and_the_Country_Mouse}{Wikipedia{\tt/}Town Mouse \& Country Mouse}

Read also \href{https://www.reddit.com/r/ChainsawMan/comments/ghj22b/just_a_little_appreciation_to_the_brilliance_of/}{Reddit{\tt/}Just a Little Appreciation to The Brilliance of The Town Mouse and Country Mouse Fable}.

%------------------------------------------------------------------------------%

\section{Miscellaneous}

%------------------------------------------------------------------------------%

\printbibliography[heading=bibintoc]
	
\end{document}