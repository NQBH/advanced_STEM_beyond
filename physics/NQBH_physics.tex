\documentclass{article}
\usepackage[backend=biber,natbib=true,style=alphabetic,maxbibnames=50]{biblatex}
\addbibresource{/home/nqbh/reference/bib.bib}
\usepackage[utf8]{vietnam}
\usepackage{tocloft}
\renewcommand{\cftsecleader}{\cftdotfill{\cftdotsep}}
\usepackage[colorlinks=true,linkcolor=blue,urlcolor=red,citecolor=magenta]{hyperref}
\usepackage{amsmath,amssymb,amsthm,enumitem,float,graphicx,mathtools,tikz}
\usetikzlibrary{angles,calc,intersections,matrix,patterns,quotes,shadings}
\allowdisplaybreaks
\newtheorem{assumption}{Assumption}
\newtheorem{baitoan}{}
\newtheorem{cauhoi}{Câu hỏi}
\newtheorem{conjecture}{Conjecture}
\newtheorem{corollary}{Corollary}
\newtheorem{dangtoan}{Dạng toán}
\newtheorem{definition}{Definition}
\newtheorem{dinhly}{Định lý}
\newtheorem{dinhnghia}{Định nghĩa}
\newtheorem{example}{Example}
\newtheorem{ghichu}{Ghi chú}
\newtheorem{hequa}{Hệ quả}
\newtheorem{hypothesis}{Hypothesis}
\newtheorem{lemma}{Lemma}
\newtheorem{luuy}{Lưu ý}
\newtheorem{nhanxet}{Nhận xét}
\newtheorem{notation}{Notation}
\newtheorem{note}{Note}
\newtheorem{principle}{Principle}
\newtheorem{problem}{Problem}
\newtheorem{proposition}{Proposition}
\newtheorem{question}{Question}
\newtheorem{remark}{Remark}
\newtheorem{theorem}{Theorem}
\newtheorem{vidu}{Ví dụ}
\usepackage[left=1cm,right=1cm,top=5mm,bottom=5mm,footskip=4mm]{geometry}
\def\labelitemii{$\circ$}
\DeclareRobustCommand{\divby}{%
	\mathrel{\vbox{\baselineskip.65ex\lineskiplimit0pt\hbox{.}\hbox{.}\hbox{.}}}%
}
\setlist[itemize]{leftmargin=*}
\setlist[enumerate]{leftmargin=*}

\title{Physics -- Vật Lý}
\author{Nguyễn Quản Bá Hồng\footnote{A Scientist {\it\&} Creative Artist Wannabe. E-mail: {\tt nguyenquanbahong@gmail.com}. Bến Tre City, Việt Nam.}}
\date{\today}

\begin{document}
\maketitle
\begin{abstract}
	This text is a part of the series {\it Some Topics in Advanced STEM \& Beyond}:
	
	{\sc url}: \url{https://nqbh.github.io/advanced_STEM/}.
	
	Latest version:
	\begin{itemize}
		\item {\it Physics -- Vật Lý}.
		
		PDF: {\sc url}: \url{https://github.com/NQBH/advanced_STEM_beyond/blob/main/physics/NQBH_physics.pdf}.
		
		\TeX: {\sc url}: \url{https://github.com/NQBH/advanced_STEM_beyond/blob/main/physics/NQBH_physics.tex}.
	\end{itemize}
\end{abstract}
\tableofcontents

%------------------------------------------------------------------------------%

\section{Wikipedia}

\subsection{Wikipedia{\tt/}advection}
``In the field of \href{https://en.wikipedia.org/wiki/Physics}{physics}, \href{https://en.wikipedia.org/wiki/Engineering}{engineering}, \& \href{https://en.wikipedia.org/wiki/Earth_science}{earth sciences}, {\it advection} is the \href{https://en.wikipedia.org/wiki/Transport}{transport} of a substance or quantity by bulk motion.

The properties of that substance are carried with it.

Generally the majority of the advected substance is a fluid.

The properties that are carried with the advected substance are \href{https://en.wikipedia.org/wiki/Conservation_of_energy}{conserved} properties e.g. \href{https://en.wikipedia.org/wiki/Energy}{energy}.

An example of advection is the transport of \href{https://en.wikipedia.org/wiki/Pollutant}{pollutants} or \href{https://en.wikipedia.org/wiki/Silt}{silt} in a \href{https://en.wikipedia.org/wiki/River}{river} by bulk water flow downstream.

Another commonly advected quantity is energy or \href{https://en.wikipedia.org/wiki/Enthalpy}{enthalpy}.

Here the fluid may be any material that contains thermal energy, e.g. \href{https://en.wikipedia.org/wiki/Water}{water} or \href{https://en.wikipedia.org/wiki/Air}{air}.

In general, any substance or conserved, \href{https://en.wikipedia.org/wiki/Intensive_and_extensive_properties}{extensive} quantity can be advected by a fluid that can hold or contain the quantity or substance.

%
During advection, a fluid transports some conserved quantity or material via bulk motion.

The fluid's motion is described mathematically as a \href{https://en.wikipedia.org/wiki/Vector_field}{vector field}, \& the transported material is described by a \href{https://en.wikipedia.org/wiki/Scalar_field}{scalar field} showing its distribution over space.

Advection requires currents in the fluid, \& so cannot happen in rigid solids.

It does not include transport of substances by \href{https://en.wikipedia.org/wiki/Molecular_diffusion}{molecular diffusion}.

%
Advection is sometimes confused with the more encompassing process of \href{https://en.wikipedia.org/wiki/Convection}{convection} which is the combination of advective transport \& diffusive transport.

%
In \href{https://en.wikipedia.org/wiki/Meteorology}{meteorology} \& \href{https://en.wikipedia.org/wiki/Physical_oceanography}{physical oceanography}, advection often refers to the transport of some property of the atmosphere or \href{https://en.wikipedia.org/wiki/Ocean}{ocean}, e.g., \href{https://en.wikipedia.org/wiki/Heat}{heat}, humidity (see \href{https://en.wikipedia.org/wiki/Water_vapor}{moisture}) or \href{https://en.wikipedia.org/wiki/Salinity}{salinity}.

Advection is important for the formation of \href{https://en.wikipedia.org/wiki/Orographic}{orographic} clouds \& the precipitation of water from clouds, as part of the \href{https://en.wikipedia.org/wiki/Hydrological_cycle}{hydrological cycle}.

\subsubsection{Distinction between advection \& convection}
The term {\it advection} often serves as a synonym for \href{https://en.wikipedia.org/wiki/Convection}{{\it convection}}, \& this correspondence of terms is used in the literature.

More technically, convection applies to the movement of a fluid (often due to density gradients created by thermal gradients), whereas advection is the movement of some material by the velocity of the fluid.

Thus, somewhat confusingly, it is technically correct to think of momentum being advected by the velocity field in the Navier-Stokes equations, although the resulting motion would be considered to be convection.

Because of the specific use of the term convection to indicate transport in association with thermal gradients, it is probably safer to use the term advection if one is uncertain about which terminology best describes their particular system.

\subsubsection{Meteorology}
In \href{https://en.wikipedia.org/wiki/Meteorology}{meteorology} \& \href{https://en.wikipedia.org/wiki/Physical_oceanography}{physical oceanography}, advection often refers to the horizontal transport of some property of the atmosphere or \href{https://en.wikipedia.org/wiki/Ocean}{ocean}, e.g., \href{https://en.wikipedia.org/wiki/Heat}{heat}, humidity or salinity, \& convection generally refers to vertical transport (vertical advection).

Advection is important for the formation of \href{https://en.wikipedia.org/wiki/Orographic_cloud}{orographic clouds} (terrain-forced convection) \& the precipitation of water from clouds, as part of the \href{https://en.wikipedia.org/wiki/Hydrological_cycle}{hydrological cycle}.

\subsubsection{Other quantities}
The advection equation also applies if the quantity being advected is represented by a \href{https://en.wikipedia.org/wiki/Probability_density_function}{probability density function} at each point, although accounting for diffusion is more difficult.

\subsubsection{Mathematics of advection}
The {\it advection equation} is the \href{https://en.wikipedia.org/wiki/Partial_differential_equation}{PDE} that governs the motion of a conserved \href{https://en.wikipedia.org/wiki/Scalar_field}{scalar field} as it is advected by a known \href{https://en.wikipedia.org/wiki/Velocity_field}{velocity vector field}.

It is derived using the scalar field's \href{https://en.wikipedia.org/wiki/Conservation_law}{conservation law}, together with \href{https://en.wikipedia.org/wiki/Gauss's_theorem}{Gauss's theorem}, \& taking the \href{https://en.wikipedia.org/wiki/Infinitesimal}{infinitesimal} limit.

%
One easily visualized example of advection is the transport of ink dumped into a river.

As the river flows, ink will move downstream in a ``pulse'' via advection, as the water's movement itself transports the ink.

If added to a lake without significant bulk water flow, the ink would simply disperse outwards from its source in a \href{https://en.wikipedia.org/wiki/Diffusion}{diffusive} manner, which is not advection.

Note that as it moves downstream, the ``pulse'' of ink will also spread via diffusion.

The sum of these processes is called \href{https://en.wikipedia.org/wiki/Convection}{convection}.

\paragraph{The advection equation.} In Cartesian coordinates the advection \href{https://en.wikipedia.org/wiki/Operator_(mathematics)}{operator} is
\begin{align*}
	{\bf u}\cdot\nabla = u_x\partial_x + u_y\partial_y + u_z\partial_z,
\end{align*}
where ${\bf u} = (u_x,u_y,u_z)$ is the \href{https://en.wikipedia.org/wiki/Velocity_field}{velocity field}, \& $\nabla$ is the \href{https://en.wikipedia.org/wiki/Del}{del} operator (note that \href{https://en.wikipedia.org/wiki/Cartesian_coordinate_system}{Cartesian coordinates} are used here).

%
The advection equation for a conserved quantity described by a \href{https://en.wikipedia.org/wiki/Scalar_field}{scalar field} $\psi$ is expressed mathematically by a \href{https://en.wikipedia.org/wiki/Continuity_equation}{continuity equation}:
\begin{align*}
	\partial_t\psi + \nabla\cdot\left(\psi{\bf u}\right) = 0,
\end{align*}
where $\nabla\cdot$  is the divergence operator \& again ${\bf u}$ is the \href{https://en.wikipedia.org/wiki/Velocity_field}{velocity vector field}.

Frequently, it is assumed that the flow is \href{https://en.wikipedia.org/wiki/Incompressible_flow}{incompressible}, i.e., the \href{https://en.wikipedia.org/wiki/Velocity_field}{velocity field} satisfies 
\begin{align*}
	\nabla\cdot{\bf u} = 0.
\end{align*}
In this case, ${\bf u}$ is said to be \href{https://en.wikipedia.org/wiki/Solenoidal}{solenoidal}.

If this is so, the above equation can be rewritten as
\begin{align*}
	\partial_t\psi + {\bf u}\cdot\nabla\psi = 0.
\end{align*}
In particular, if the flow is steady, then
\begin{align*}
	{\bf u}\cdot\nabla\psi = 0,
\end{align*}
which shows that $\psi$ is constant along a streamline.

Hence, $\partial_t\psi = 0$, so $\psi$ does not vary in time.

%
If a vector quantity ${\bf a}$ (e.g., a \href{https://en.wikipedia.org/wiki/Magnetic_field}{magnetic field}) is being advected by the \href{https://en.wikipedia.org/wiki/Solenoidal}{solenoidal} \href{https://en.wikipedia.org/wiki/Velocity_field}{velocity field} ${\bf u}$, the advection equation above becomes: 
\begin{align*}
	\partial_t{\bf a} + ({\bf u}\cdot\nabla){\bf a} = 0.
\end{align*}
Here, ${\bf a}$ is a vector field instead of the scalar field $\psi$.

\paragraph{Solving the equation.} {\sf A simulation of the advection equation where ${\bf u} = (\sin t,\cos t)$ is solenoidal.}

%
The advection equation is not simple to solve \href{https://en.wikipedia.org/wiki/Numerical_analysis}{numerically}: the system is a \href{https://en.wikipedia.org/wiki/Hyperbolic_partial_differential_equation}{hyperbolic partial differential equation}, \& interest typically centers on \href{https://en.wikipedia.org/wiki/Continuous_function}{discontinuous} ``shock'' solutions (which are notoriously difficult for numerical schemes to handle).

%
Even with 1 space dimension \& a constant \href{https://en.wikipedia.org/wiki/Velocity_field}{velocity field}, the system remains difficult to simulate.

The equation becomes
\begin{align*}
	\partial_t\psi + u_x\partial_x\psi = 0,
\end{align*}
where $\psi = \psi(t,x)$ is the scalar field being advected \& $u_x$ is the $x$ component of the vector ${\bf u} = (u_x,0,0)$.

\paragraph{Treatment of the advection operator in the compressible NSEs.} According to Zang,[Zang, Thomas (1991). ``On the rotation \& skew-symmetric forms for incompressible flow simulations''. Applied Numerical Mathematics. 7: 27--40. Bibcode:1991ApNM....7...27Z. doi:10.1016/0168-9274(91)90102-6.] numerical simulation can be aided by considering the \href{https://en.wikipedia.org/wiki/Skew-symmetric_matrix}{skew symmetric} form for the advection operator.
\begin{align*}
	\frac{1}{2}{\bf u}\cdot\nabla{\bf u} + \frac{1}{2}\nabla({\bf u}{\bf u}) \mbox{ where } \nabla({\bf u}{\bf u}) = \left[\nabla({\bf u}u_x),\nabla({\bf u}u_y),\nabla({\bf u}u_z)\right]
\end{align*}
and ${\bf u}$ is the same as above.

%
Since skew symmetry implies only imaginary eigenvalues, this form reduces the ``blow up'' \& ``spectral blocking'' often experienced in numerical solutions with sharp discontinuities (see Boyd[Boyd, John P. (2000). {\it Chebyshev \& Fourier Spectral Methods} 2nd edition. Dover. p. 213.]).

%
Using \href{https://en.wikipedia.org/wiki/Vector_calculus_identities#Vector_dot_product}{vector calculus identities}, these operators can also be expressed in other ways, available in more software packages for more coordinate systems.
\begin{align*}
	{\bf u}\cdot\nabla{\bf u} &= \nabla\left(\frac{\|{\bf u}\|^2}{2}\right) + (\nabla\times{\bf u})\times{\bf u},\\
	\frac{1}{2}{\bf u}\cdot\nabla{\bf u} + \frac{1}{2}\nabla({\bf u}{\bf u}) &= \nabla\left(\frac{\|{\bf u}\|^2}{2}\right) + (\nabla\times{\bf u})\times{\bf u} + \frac{1}{2}{\bf u}(\nabla\cdot{\bf u}).
\end{align*}
This form also makes visible that the \href{https://en.wikipedia.org/wiki/Skew-symmetric_matrix}{skew symmetric} operator introduces error when the velocity field diverges.

Solving the advection equation by numerical methods is very challenging \& there is a large scientific literature about this.'' -- \href{https://en.wikipedia.org/wiki/Advection}{Wikipedia{\tt/}advection}

%------------------------------------------------------------------------------%

\subsection{Wikipedia{\tt/}diffusion}
{\sf Some particles are \href{https://en.wikipedia.org/wiki/Dissolution_(chemistry)}{dissolved} in a glass of water. At 1st, the particles are all near 1 top corner of the glass. If the particles randomly move around (``diffuse'') in the water, they eventually become distributed randomly \& uniformly from an area of high concentration to an area of low, \& organized (diffusion continues, but with no net \href{https://en.wikipedia.org/wiki/Flux}{flux}).}

``{\it Diffusion} is the net movement of anything (e.g., atoms, ions, molecules, energy) generally from a region of higher \href{https://en.wikipedia.org/wiki/Concentration}{concentration} to a region of lower concentration. Diffusion is driven by a gradient in \href{https://en.wikipedia.org/wiki/Gibbs_free_energy}{Gibbs free energy} or \href{https://en.wikipedia.org/wiki/Chemical_potential}{chemical potential}. It is possible to diffuse ``uphill'' from a region of lower concentration to a region of higher concentration, as in \href{https://en.wikipedia.org/wiki/Spinodal_decomposition}{spinodal decomposition}. Diffusion is a stochastic process due to the inherent randomness of the diffusing entity \& can be used to model many real-life stochastic scenarios. Therefore, diffusion \& the corresponding mathematical models are used in several fields beyond physics, e.g. \href{https://en.wikipedia.org/wiki/Statistics}{statistics}, \href{https://en.wikipedia.org/wiki/Probability_theory}{probability theory}, \href{https://en.wikipedia.org/wiki/Information_theory}{information theory}, \href{https://en.wikipedia.org/wiki/Neural_networks}{neural networks}, \href{https://en.wikipedia.org/wiki/Finance}{finance}, \& \href{https://en.wikipedia.org/wiki/Marketing}{marketing}.

The concept of diffusion is widely used in many fields, including physics (\href{https://en.wikipedia.org/wiki/Molecular_diffusion}{particle diffusion}), chemistry, biology, sociology, economics, statistics, data science, \& finance (diffusion of people, ideas, data, \& price values). The central idea of diffusion, however, is common to all of these: a substance or collection undergoing diffusion spreads out from a point or location at which there is a higher concentration of that substance or collection.

A \href{https://en.wikipedia.org/wiki/Gradient}{gradient} is the change in the value of a quantity; e.g., concentration, \href{https://en.wikipedia.org/wiki/Pressure}{pressure}, or \href{https://en.wikipedia.org/wiki/Temperature}{temperature} with the change in another variable, usually \href{https://en.wikipedia.org/wiki/Distance}{distance}. A change in concentration over a distance is called a \href{https://en.wikipedia.org/wiki/Molecular_diffusion}{concentration gradient}, a change in pressure over a distance is called a \href{https://en.wikipedia.org/wiki/Pressure_gradient}{pressure gradient}, \& a change in temperature over a distance is called a \href{https://en.wikipedia.org/wiki/Temperature_gradient}{temperature gradient}.

The word {\it diffusion} derives from the Latin word, {\it diffundere}, which means ``to spread out''.

A distinguishing feature of diffusion is that it depends on particle \href{https://en.wikipedia.org/wiki/Random_walk}{random walk}, \& results in mixing or mass transport without requiring directed bulk motion. Bulk motion, or bulk flow, is the characteristic of \href{https://en.wikipedia.org/wiki/Advection}{advection}. The term \href{https://en.wikipedia.org/wiki/Convection}{convection} is used to describe the combination of both \href{https://en.wikipedia.org/wiki/Transport_phenomena}{transport phenomena}.

-- 1 đặc điểm nổi bật của sự khuếch tán là nó phụ thuộc vào bước đi ngẫu nhiên của hạt và dẫn đến sự trộn lẫn hoặc vận chuyển khối lượng mà không cần chuyển động khối có hướng. Chuyển động khối, hay dòng chảy khối, là đặc điểm của sự tiến bộ. Thuật ngữ đối lưu được sử dụng để mô tả sự kết hợp của cả hai hiện tượng vận chuyển.

In a diffusion process can be described by \href{https://en.wikipedia.org/wiki/Fick%27s_laws_of_diffusion}{Fick's laws}, it is called a normal diffusion (or Fickian diffusion), otherwise, it is called an \href{https://en.wikipedia.org/wiki/Anomalous_diffusion}{anomalous diffusion} (or non-Fickian diffusion).

When talking about the extent of diffusion, 2 length scales are used in 2 different scenarios:
\begin{enumerate}
	\item \href{https://en.wikipedia.org/wiki/Brownian_motion}{Brownian motion} of an \href{https://en.wikipedia.org/wiki/Impulse_response}{impulsive} point source (e.g., 1 single spray of perfume) -- the square root of the \href{https://en.wikipedia.org/wiki/Mean_squared_displacement}{mean squared displacement} from this point. In Fickian diffusion, this is $\sqrt{2nDt}$, where $n$ is the dimension of this Brownian motion;
	\item \href{https://en.wikipedia.org/wiki/Fick%27s_laws_of_diffusion#Example_solutions_and_generalization}{Constant concentration source} in 1D -- the diffusion length. In Fickian diffusion, this is $2\sqrt{Dt}$.
\end{enumerate}
{\sf Diffusion from a microscopic \& a macroscopic point of view. Initially, there are \href{https://en.wikipedia.org/wiki/Solute}{solute} molecules on the left side of a barrier \& none on the right. The barrier is removed, \& the solute diffuses to fill the whole container. Top: A single molecule moves around randomly. Middle: With more molecules, there is a statistical trend that the solute fills the container more \& more uniformly. Bottom: With an enormous number of solute molecules, all randomness is gone: The solute appears to move smoothly \& deterministically from high-concentration areas to low-concentration areas. There is no microscopic \href{https://en.wikipedia.org/wiki/Force}{force} pushing molecules rightward, but there {\it appears} to be one in the bottom panel. This apparent force is called an \href{https://en.wikipedia.org/wiki/Entropic_force}{entropic force}.}

\subsubsection{Diffusion vs. bulk flow}
``Bulk flow'' is the movement{\tt/}flow of an entire body due to a pressure gradient (e.g., water coming out of a tap). ``Diffusion'' is the gradual movement{\tt/}dispersion of concentration within a body with no net movement of matter. An example of a process where both \href{https://en.wikipedia.org/wiki/Mass_flow_(life_sciences)}{bulk motion} \& diffusion occur is human breathing.

1st, there is a ``bulk flow'' process. The \href{https://en.wikipedia.org/wiki/Lungs}{lungs} are located in the \href{https://en.wikipedia.org/wiki/Thoracic_cavity}{thoracic cavity}, which expands as the 1st step in external respiration. This expansion leads to an increase in volume of the \href{https://en.wikipedia.org/wiki/Pulmonary_alveolus}{alveoli} in the lungs, which causes a decrease in pressure in the alveoli. This creates a pressure gradient between the \href{https://en.wikipedia.org/wiki/Air}{air} outside the body at relatively high pressure \& the alveoli\footnote{phế nang.} at relatively low pressure. The air moves down the pressure gradient through the airways of the lungs \& into the alveoli until the pressure of the air \& that in the alveoli are equal, i.e., the movement of air by bulk flow stops once there is no longer a pressure gradient.

2nd, there is a ``diffusion'' process. The air arriving in the alveoli has a higher concentration of oxygen than the ``stale'' air in the alveoli. The increase in oxygen concentration creates a concentration gradient for oxygen between the air in the alveoli \& the blood in the \href{https://en.wikipedia.org/wiki/Capillaries}{capillaries} that surround the alveoli. Oxygen then moves by diffusion, down the concentration gradient, into the blood. The other consequence of the air arriving in alveoli is that the concentration of \href{https://en.wikipedia.org/wiki/Carbon_dioxide}{carbon dioxide} $\rm CO_2$ in the alveoli decreases. This creates a concentration gradient for carbon dioxide to diffuse from the blood into the alveoli, as fresh air has a very low concentration of carbon dioxide compared to the \href{https://en.wikipedia.org/wiki/Blood}{blood} in the body.

3rd, there is another ``bulk flow'' process. The pumping action of the \href{https://en.wikipedia.org/wiki/Heart}{heart} then transports the blood around the body. As the left ventricle of the heart contracts, the volume decreases, which increases the pressure in the ventricle. This creates a pressure gradient between the heart \& the capillaries, \& blood moves through \href{https://en.wikipedia.org/wiki/Blood_vessel}{blood vessels} by bulk flow down the pressure gradient.

\subsubsection{Diffusion in the context of different disciplines}
{\sf Diffusion furnaces used for \href{https://en.wikipedia.org/wiki/Thermal_oxidation}{thermal oxidation}}. ``There are 2 ways to introduce the notion of {\it diffusion}: either a \href{https://en.wiktionary.org/wiki/phenomenon}{phenomenological approach} starting with \href{https://en.wikipedia.org/wiki/Fick%27s_laws_of_diffusion}{Fick's laws of diffusion} \& their mathematical consequences, or a physical \& atomistic one, by considering the {\it\href{https://en.wikipedia.org/wiki/Random_walk}{random walk} of the diffusion particles}.

In the phenomenological approach, {\it diffusion is the movement of a substance from a region of high concentration to a region of low concentration without bulk motion}. According to Fick's laws, the diffusion \href{https://en.wikipedia.org/wiki/Flux#Flux_as_flow_rate_per_unit_area}{flux} is proportional to the negative gradient $-\nabla u$ of concentrations. It goes from regions of higher concentration to regions of lower concentration. Sometime later, various generalizations of Fick's laws were developed in the frame of \href{https://en.wikipedia.org/wiki/Thermodynamics}{thermodynamics} \& \href{https://en.wikipedia.org/wiki/Non-equilibrium_thermodynamics}{non-equilibrium thermodynamics}.

From the {\it atomistic point of view}, diffusion is considered as a result of the random walk of the diffusion particles. In \href{https://en.wikipedia.org/wiki/Molecular_diffusion}{molecular diffusion}, the moving molecules in a gas, liquid, or solid are self-propelled by kinetic energy. Random walk of small particles in suspension in a fluid was discovered in 1827 by \href{https://en.wikipedia.org/wiki/Robert_Brown_(botanist,_born_1773)}{\sc Robert Brown}, who found that minute particle suspended in a liquid medium \& just large enough to be visible under an optical microscope exhibit a rapid \& continually irregular motion of particles known as Brownian movement. The theory of the \href{https://en.wikipedia.org/wiki/Brownian_motion}{Brownian motion} \& the atomistic backgrounds of diffusion were developed by \href{https://en.wikipedia.org/wiki/Albert_Einstein}{\sc Albert Einstein}. The concept of diffusion is typically applied to any subject matter involving random walks in \href{https://en.wikipedia.org/wiki/Statistical_ensemble_(mathematical_physics)}{ensembles} of individuals.

In chemistry \& \href{https://en.wikipedia.org/wiki/Materials_science}{materials science}, diffusion also refers to the movement of fluid molecules in porous solids. Different types of diffusion are distinguished in porous solids. \href{https://en.wikipedia.org/wiki/Molecular_diffusion}{Molecular diffusion} occurs when the collision with another molecule is more likely than the collision with the pore walls. Under such conditions, the diffusivity is similar to that in a non-confined space \& is proportional to the mean free path. \href{https://en.wikipedia.org/wiki/Knudsen_diffusion}{Knudsen diffusion} occurs when the pore diameter is comparable to or smaller than the mean free path of the molecule diffusing through the pore. Under this condition, the collision with the pore walls becomes gradually more likely \& the diffusivity is lower. Finally there is configurational diffusion, which happens if the molecules have comparable size to that of the pore. Under this condition, the diffusivity is much lower compared to molecular diffusion \& small differences in the kinetic diameter of the molecule cause large differences in \href{mass diffusivity}.

\href{https://en.wikipedia.org/wiki/Biologist}{Biologists} often use the terms ``net movement'' or ``net diffusion'' to describe the movement of ions or molecules by diffusion. E.g., oxygen can diffuse through cell membranes so long as there is a higher concentration of oxygen outside the cell. However, because the movement of molecules is random, occasionally oxygen molecules move out of the cell (against the concentration gradient). Because there are more oxygen molecules outside the cell, the probability that oxygen molecules will enter the cell is higher than the probability that oxygen molecules will leave the cell. Therefore, the ``net'' movement of oxygen molecules (the difference between the number of molecules either entering or leaving the cell) is into the cell. I.e., there is a {\it net movement} of oxygen molecules down the concentration gradient.

\subsubsection{History of diffusion in physics}
In the scope of time, diffusion in solids was used long before the theory of diffusion was created. E.g., \href{https://en.wikipedia.org/wiki/Pliny_the_Elder}{\sc Pliny the Elder} had previously described the \href{https://en.wikipedia.org/wiki/Cementation_process}{cementation process}, which produces steel from the element \href{https://en.wikipedia.org/wiki/Iron}{iron} (Fe) through carbon diffusion. Another example is well known for many centuries, the diffusion of colors of \href{https://en.wikipedia.org/wiki/Stained_glass}{stained glass} or \href{https://en.wikipedia.org/wiki/Earthenware}{earthenwave} \& \href{https://en.wikipedia.org/wiki/Chinese_ceramics}{Chinese ceramics}.

In modern science, the 1st systematic experimental study of diffusion was performed by \href{https://en.wikipedia.org/wiki/Thomas_Graham_(chemist)}{\sc Thomas Graham}. He studied diffusion in gases, \& the main phenomenon was described by him in 1831--1833:
\begin{quote}
	``$\ldots$ gases of different nature, when brought into contact, do not arrange themselves according to their density, the heaviest undermost, \& the lighter uppermost, but they spontaneously diffuse, mutually \& equally, through each other, \& so remain in the intimate state of mixture for any length of time.''
\end{quote}
The measurements of Graham contributed to \href{https://en.wikipedia.org/wiki/James_Clerk_Maxwell}{\sc James Clerk Maxwell} deriving, in 1867, the coefficient of diffusion for $\rm CO_2$ in the air. The error rate is $< 5\%$.

In 1855, \href{https://en.wikipedia.org/wiki/Adolf_Fick}{\sc Adolf Fick}, the 26-year-old anatomy demonstrator from Z\"urich, proposed \href{https://en.wikipedia.org/wiki/Fick%27s_laws_of_diffusion}{Fick's law of diffusion}. He used Graham's research, stating his goal as ``the development of a fundamental law, for the operation of diffusion in a single element of space''. He asserted a deep analogy between diffusion \& conduction of heat or electricity, creating a formalism similar to \href{https://en.wikipedia.org/wiki/Thermal_conduction}{Fourier's law for heat conduction} (1822) \& \href{https://en.wikipedia.org/wiki/Ohm%27s_law}{Ohm's law} for electric current (1827).

\href{https://en.wikipedia.org/wiki/Robert_Boyle}{\sc Robert Boyle} demonstrated diffusion in solids in the 17th century by penetration of zinc Zn into a copper Cu coin. Nevertheless, diffusion in solids was not systematically studied until the 2nd part of the 19th century. \href{https://en.wikipedia.org/wiki/William_Chandler_Roberts-Austen}{\sc William Chandler Roberts-Austen}, the well-known British metallurgist \& former assistant of {\sc Thomas Graham} studied systematically solid state diffusion on the example of gold in lead in 1896:
\begin{quote}
	``$\ldots$ My long connection with Graham's researches made it almost a duty to attempt to extend his work on liquid diffusion to metals.''
\end{quote}
In 1858, \href{https://en.wikipedia.org/wiki/Rudolf_Clausius}{Rudolf Clausius} introduced the concept of the \href{https://en.wikipedia.org/wiki/Mean_free_path}{mean free path}. In the same year, \href{https://en.wikipedia.org/wiki/James_Clerk_Maxwell}{\sc James Clerk Maxwell} developed the 1st atomistic theory of transport processes in gases. The modern atomistic theory of diffusion \& Brownian motion was developed by \href{https://en.wikipedia.org/wiki/Albert_Einstein}{\sc Albert Einstein}, \href{https://en.wikipedia.org/wiki/Marian_Smoluchowski}{Marian Smoluchowski}, \& \href{https://en.wikipedia.org/wiki/Jean-Baptiste_Perrin}{Jean-Baptise Perrin}. \href{https://en.wikipedia.org/wiki/Ludwig_Boltzmann}{\sc Ludwig Boltzmann}, in the development of the atomistic backgrounds of the macroscopic \href{https://en.wikipedia.org/wiki/Transport_phenomena}{transport processes}, introduced the \href{https://en.wikipedia.org/wiki/Boltzmann_equation}{Boltzmann equation}, which has served mathematics \& physics with a source of transport process ideas \& concerns for $> 140$ years.

In 1920--1921, \href{https://en.wikipedia.org/wiki/George_de_Hevesy}{\sc George de Hevesy} measured \href{https://en.wikipedia.org/wiki/Self-diffusion}{self-diffusion} using \href{https://en.wikipedia.org/wiki/Radioisotope}{raioisotopes}. He studied self-diffusion of radioactive isotopes of lead in the liquid \& solid lead.

\href{https://en.wikipedia.org/wiki/Yakov_Frenkel}{\sc Yakov Frenkel} (sometimes, Jakov{\tt/}Jacob Frenkel) proposed, \& elaborated in 1926, the idea of diffusion in crystals through local defects (vacancies \& \href{https://en.wikipedia.org/wiki/Interstitial_defect}{interstitial} atoms). He concluded, the diffusion process in condensed matter is an ensemble of elementary jumps \& quasichemical interactions of particles \& defects. He introduced several mechanisms of diffusion \& found rate constants from experimental data.

Sometime later, \href{https://en.wikipedia.org/wiki/Carl_Wagner}{\sc Carl Wagner} \& \href{https://en.wikipedia.org/wiki/Walter_H._Schottky}{\sc Walter H. Schottky} developed Frenkel's ideas about mechanisms of diffusion further. Presently, it is universally recognized that atomic defects are necessary to mediate diffusion in crystals.

\href{https://en.wikipedia.org/wiki/Henry_Eyring_(chemist)}{\sc Henry Eyring}, with co-authors, applied his theory of \href{https://en.wikipedia.org/wiki/Transition_state_theory}{absolute reaction rates} to Frenkel's quasichemical model of diffusion. The analogy between \href{https://en.wikipedia.org/wiki/Chemical_kinetics}{reaction kinetics} \& diffusion leads to various nonlinear versions of Fick's law.

\subsubsection{Basic models of diffusion}

\subsubsection{Diffusion in physics}

\subsubsection{Random walk (random motion)}

'' -- \href{https://en.wikipedia.org/wiki/Diffusion}{Wikipedia{\tt/}diffusion}

%------------------------------------------------------------------------------%

\subsection{Wikipedia{\tt/}physics}
``

'' -- \href{https://en.wikipedia.org/wiki/Physics}{Wikipedia{\tt/}physics}

%------------------------------------------------------------------------------%

\subsection{Wikipedia{\tt/}hydrostatics}
{\sf Table of Hydraulics \& Hydrostatics, from the 1728 Cyclop\ae dia.}

%
{\it Fluid statics} or {\it hydrostatics} is the branch of \href{https://en.wikipedia.org/wiki/Fluid_mechanics}{fluid mechanics} that studies ``\href{https://en.wikipedia.org/wiki/Fluid}{fluids} at rest \& the pressure in a fluid or exerted by a fluid on an immersed body''.[``Hydrostatics''. Merriam-Webster. Retrieved Sep 11, 2018.]

%
It encompasses the study of the conditions under which fluids are at rest in \href{https://en.wikipedia.org/wiki/Mechanical_equilibrium}{stable equilibrium} as opposed to \href{https://en.wikipedia.org/wiki/Fluid_dynamics}{fluid dynamics}, the study of fluids in motion.

Hydrostatics are categorized as a part of the fluid statics, which is the study of all fluids, incompressible or not, at rest.

%
Hydrostatics is fundamental to \href{https://en.wikipedia.org/wiki/Hydraulics}{hydraulics}, the engineering of equipment for storing, transporting \& using fluids.

It is also relevant to \href{https://en.wikipedia.org/wiki/Geophysics}{geophysics} \& \href{https://en.wikipedia.org/wiki/Astrophysics}{astrophysics} (e.g., in understanding \href{https://en.wikipedia.org/wiki/Plate_tectonics}{plate tectonics} \& the anomalies of the \href{https://en.wikipedia.org/wiki/Gravity_of_Earth}{Earth's gravitational field}), to \href{https://en.wikipedia.org/wiki/Meteorology}{meteorology}, to \href{https://en.wikipedia.org/wiki/Medicine}{medicine} (in the context of \href{https://en.wikipedia.org/wiki/Blood_pressure}{blood pressure}), \& many other fields.

%
Hydrostatics offers physical explanations for many phenomena of everyday life, such as why \href{https://en.wikipedia.org/wiki/Atmospheric_pressure}{atmospheric pressure} changes with \href{https://en.wikipedia.org/wiki/Altitude}{altitude}, why wood \& oil float on water, \& why the surface of still water is always level.

\subsubsection{History}
Some principles of hydrostatics have been known in an empirical \& intuitive sense since antiquity, by the builders of boats, \href{https://en.wikipedia.org/wiki/Cistern}{cisterns}, \href{https://en.wikipedia.org/wiki/Aqueduct_(water_supply)}{aqueducts} \& \href{https://en.wikipedia.org/wiki/Fountain}{fountains}.

\href{https://en.wikipedia.org/wiki/Archimedes}{Archimedes} is credited with the discovery of \href{https://en.wikipedia.org/wiki/Archimedes'_Principle}{Archimedes' Principle}, which relates the \href{https://en.wikipedia.org/wiki/Buoyancy}{buoyancy} force on an object that is submerged in a fluid to the weight of fluid displaced by the object.

The \href{https://en.wikipedia.org/wiki/Roman_Empire}{Roman} engineer \href{https://en.wikipedia.org/wiki/Vitruvius}{Vitruvius} warned readers about \href{https://en.wikipedia.org/wiki/Lead}{lead} pipes bursting under hydrostatic pressure.[Marcus Vitruvius Pollio (ca. 15 BCE), ``The Ten Books of Architecture'', Book VIII, Chapter 6. At the University of Chicago's Penelope site. Accessed on 2013-02-25.]

%
The concept of pressure \& the way it is transmitted by fluids was formulated by the \href{https://en.wikipedia.org/wiki/France}{French} \href{https://en.wikipedia.org/wiki/Mathematician}{mathematician} \& \href{https://en.wikipedia.org/wiki/Philosopher}{philosopher} \href{https://en.wikipedia.org/wiki/Blaise_Pascal}{Blaise Pascal} in 1647.

\paragraph{Hydrostatics in ancient Greece \& Rome.}

\subparagraph{Pythagorean Cup.} Main article: \href{https://en.wikipedia.org/wiki/Pythagorean_cup}{Pythagorean cup}.

%
The ``fair cup'' or \href{https://en.wikipedia.org/wiki/Pythagorean_cup}{Pythagorean cup}, which dates from about the 6th century BC, is a hydraulic technology whose invention is credited to the Greek mathematician \& geometer Pythagoras.

It was used as a learning tool.

%
The cup consists of a line carved into the interior of the cup, \& a small vertical pipe in the center of the cup that leads to the bottom.

The height of this pipe is the same as the line carved into the interior of the cup.

The cup may be filled to the line without any fluid passing into the pipe in the center of the cup.

However, when the amount of fluid exceeds this fill line, fluid will overflow into the pipe in the center of the cup.

Due to the drag that molecules exert on one another, the cup will be emptied.

\subparagraph{Heron's fountain.} Main article: \href{https://en.wikipedia.org/wiki/Heron's_fountain}{Heron's fountain}.

%
\href{https://en.wikipedia.org/wiki/Heron's_fountain}{Heron's fountain} is a device invented by \href{https://en.wikipedia.org/wiki/Heron_of_Alexandria}{Heron of Alexandria} that consists of a jet of fluid being fed by a reservoir of fluid.

The fountain is constructed in such a way that the height of the jet exceeds the height of the fluid in the reservoir, apparently in violation of principles of hydrostatic pressure.

The device consisted of an opening \& 2 containers arranged one above the other.

The intermediate pot, which was sealed, was filled with fluid, \& several \href{https://en.wikipedia.org/wiki/Cannula}{cannula} (a small tube for transferring fluid between vessels) connecting the various vessels.

Trapped air inside the vessels induces a jet of water out of a nozzle, emptying all water from the intermediate reservoir. 

\paragraph{Pascal's contribution in hydrostatics.} Main article: \href{https://en.wikipedia.org/wiki/Pascal's_Law}{Pascal's Law}.

%
Pascal made contributions to developments in both hydrostatics \& hydrodynamics.

\href{https://en.wikipedia.org/wiki/Pascal's_Law}{Pascal's Law} is a fundamental principle of fluid mechanics that states that any pressure applied to the surface of a fluid is transmitted uniformly throughout the fluid in all directions, in such a way that initial variations in pressure are not changed.

\subsubsection{Pressure in fluids at rest}
Due to the fundamental nature of fluids, a fluid cannot remain at rest under the presence of a \href{https://en.wikipedia.org/wiki/Shear_stress}{shear stress}.

However, fluids can exert \href{https://en.wikipedia.org/wiki/Pressure}{pressure} \href{https://en.wikipedia.org/wiki/Surface_normal}{normal} to any contacting surface.

If a point in the fluid is thought of as an infinitesimally small cube, then it follows from the principles of equilibrium that the pressure on every side of this unit of fluid must be equal.

If this were not the case, the fluid would move in the direction of the resulting force.

Thus, the pressure on a fluid at rest is \href{https://en.wikipedia.org/wiki/Isotropic}{isotropic}; i.e., it acts with equal magnitude in all directions.

This characteristic allows fluids to transmit force through the length of pipes or tubes; i.e., a force applied to a fluid in a pipe is transmitted, via the fluid, to the other end of the pipe.

This principle was 1st formulated, in a slightly extended form, by Blaise Pascal, \& is now called \href{https://en.wikipedia.org/wiki/Pascal's_law}{Pascal's law}.

\paragraph{Hydro pressure.} See also: \href{https://en.wikipedia.org/wiki/Vertical_pressure_variation}{Vertical pressure variation}.

%
In a fluid at rest, all frictional \& inertial stresses vanish \& the state of stress of the system is called {\it hydrostatic}.

When this condition of $V = 0$ is applied to the \href{https://en.wikipedia.org/wiki/Navier-Stokes_equations}{Navier-Stokes equations}, the gradient of pressure becomes a function of body forces only.

For a \href{https://en.wikipedia.org/wiki/Barotropic_fluid}{barotropic fluid} in a conservative force field like a gravitational force field, the pressure exerted by a fluid at equilibrium becomes a function of force exerted by gravity.

%
The hydrostatic pressure can be determined from a control volume analysis of an infinitesimally small cube of fluid.

Since pressure is defined as the force exerted on a test area ($p = \frac{F}{A}$, with $p$: pressure, $F$: force normal to area $A$, $A$: area), \& the only force acting on any such small cube of fluid is the weight of the fluid column above it, hydrostatic pressure can be calculated according to the following formula:
\begin{align*}
	p(z) - p(z_0) = \frac{1}{A}\int_{z_0}^z {\rm d}z'\iint_A {\rm d}x'{\rm d}y'\rho(z')g(z') = \int_{z_0}^z {\rm d}z'\rho(z')g(z'),
\end{align*}
where:
\begin{itemize}
	\item $p$ is the hydrostatic pressure (Pa),
	\item $\rho$ is the fluid \href{https://en.wikipedia.org/wiki/Density}{density} ($\rm kg/m^3$),
	\item $g$ is \href{https://en.wikipedia.org/wiki/Gravity}{gravitational} acceleration ($\rm m/s^2$),
	\item $A$ is the test area ($m^2$),
	\item $z$ is the height (parallel to the direction of gravity) of the test area (m),
	\item $z_0$ is the height of the \href{https://en.wikipedia.org/wiki/Pressure_measurement#Absolute,_gauge_and_differential_pressures_-_zero_reference}{zero reference point of the pressure} (m).
\end{itemize}
For water \& other liquids, this integral can be simplified significantly for many practical applications, based on the following 2 assumptions: Since many liquids can be considered \href{https://en.wikipedia.org/wiki/Incompressible}{incompressible}, a reasonable good estimation can be made from assuming a constant density throughout the liquid.

(The same assumption cannot be made within a gaseous environment.)

Also, since the height $h$ of the fluid column between $z$ \& $z_0$ is often reasonably small compared to the radius of the Earth, one can neglect the variation of $g$.

Under these circumstances, the integral is simplified into the formula:
\begin{align*}
	p - p_0 = \rho gh,
\end{align*}
where $h$ is the height $z - z_0$ of the liquid column between the test volume \& the zero reference point of the pressure.

This formula is often called \href{https://en.wikipedia.org/wiki/Simon_Stevin}{Stevin's} law.[3][4]

Note that this reference point should lie at or below the surface of the liquid.

Otherwise, one has to split the integral into 2 (or more) terms with the constant $\rho_{\rm liquid}$ \& $\rho(z')$ above.

E.g., the \href{https://en.wikipedia.org/wiki/Pressure_measurement#Absolute,_gauge_and_differential_pressures_-_zero_reference}{absolute pressure} compared to vacuum is:
\begin{align*}
	p = \rho gH + p_{\rm atm},
\end{align*}
where $H$ is the total height of the liquid column above the test area to the surface, \& $p_{\rm atm}$ is the \href{https://en.wikipedia.org/wiki/Atmospheric_pressure}{atmospheric pressure}, i.e., the pressure calculated from the remaining integral over the air column from the liquid surface to infinity.

This can easily be visualized using a \href{https://en.wikipedia.org/wiki/Pressure_prism}{pressure prism}.

%
Hydrostatic pressure has been used in the preservation of foods in a process called \href{https://en.wikipedia.org/wiki/Pascalization}{pascalization}.[5]

\paragraph{Medicine.} In medicine, hydrostatic pressure in \href{https://en.wikipedia.org/wiki/Blood_vessel}{blood vessels} is the pressure of the blood against the wall.

It is the opposing force to \href{https://en.wikipedia.org/wiki/Oncotic_pressure}{oncotic pressure}.

\paragraph{Atmospheric pressure.} \href{https://en.wikipedia.org/wiki/Statistical_mechanics}{Statistical mechanics} shows that, for a pure \href{https://en.wikipedia.org/wiki/Ideal_gas}{ideal gas} of constant temperature, $T$, its pressure, $p$ will vary with height, $h$, as: 
\begin{align*}
	p(h) = p(0)e^{-\frac{Mgh}{kT}},
\end{align*}
where:
\begin{itemize}
	\item $g$ is the \href{https://en.wikipedia.org/wiki/Standard_gravity}{acceleration due to gravity}
	\item $T$ is the \href{https://en.wikipedia.org/wiki/Absolute_temperature}{absolute temperature}
	\item $k$ is \href{https://en.wikipedia.org/wiki/Boltzmann_constant}{Boltzmann constant}
	\item $M$ is the mass of a single \href{https://en.wikipedia.org/wiki/Molecule}{molecule} of gas
	\item $p$ is the pressure
	\item $h$ is the height
\end{itemize}
This is known as the \href{https://en.wikipedia.org/wiki/Barometric_formula}{barometric formula}, \& maybe derived from assuming the pressure is \href{https://en.wikipedia.org/wiki/Hydrostatic_pressure}{hydrostatic}.

%
If there are multiple types of molecules in the gas, the \href{https://en.wikipedia.org/wiki/Partial_pressure}{partial pressure} of each type will be given by this equation.

Under most conditions, the distribution of each species of gas is independent of the other species.

\paragraph{Buoyancy.} Main article: \href{https://en.wikipedia.org/wiki/Buoyancy}{Buoyancy}.

%
Any body of arbitrary shape which is immersed, partly or fully, in a fluid will experience the action of a net force in the opposite direction of the local pressure gradient.

If this pressure gradient arises from gravity, the net force is in the vertical direction opposite that of the gravitational force.

This vertical force is termed buoyancy or buoyant force \& is equal in magnitude, but opposite in direction, to the weight of the displaced fluid.

Mathematically,
\begin{align*}
	F = \rho gV,
\end{align*}
where $\rho$ is the density of the fluid, $g$ is the acceleration due to gravity, \& $V$ is the volume of fluid directly above the curved surface.[Fox, Robert; McDonald, Alan; Pritchard, Philip (2012). {\it Fluid Mechanics} (8 ed.). John Wiley \& Sons. pp. 76--83. ISBN 978-1-118-02641-0.]

In the case of a \href{https://en.wikipedia.org/wiki/Ship}{ship}, e.g., its weight is balanced by pressure forces from the surrounding water, allowing it to float.

If more cargo is loaded onto the ship, it would sink more into the water - displacing more water \& thus receive a higher buoyant force to balance the increased weight.

%
Discovery of the principle of buoyancy is attributed to \href{https://en.wikipedia.org/wiki/Archimedes}{Archimedes}. 

\paragraph{Hydrostatic force on submerged surfaces.} The horizontal \& vertical components of the hydrostatic force acting on a submerged surface are given by the following:[6]
\begin{align*}
	F_{\rm h} &= p_cA,\\
	F_{\rm v} &= \rho gV,
\end{align*}
where:
\begin{itemize}
	\item $p_c$ is the pressure at the centroid of the vertical projection of the submerged surface
	\item $A$ is the area of the same vertical projection of the surface
	\item $\rho$ is the density of the fluid
	\item $g$ is the acceleration due to gravity
	\item $V$ is the volume of fluid directly above the curved surface
\end{itemize}

\subsubsection{Liquids (fluids with free surfaces)}
Liquids can have free surfaces at which they interface with gases, or with a \href{https://en.wikipedia.org/wiki/Vacuum}{vacuum}.

In general, the lack of the ability to sustain a \href{https://en.wikipedia.org/wiki/Shear_stress}{shear stress} entails that free surfaces rapidly adjust towards an equilibrium.

However, on small length scales, there is an important balancing force from \href{https://en.wikipedia.org/wiki/Surface_tension}{surface tension}.

\paragraph{Capillary action.} When liquids are constrained in vessels whose dimensions are small, compared to the relevant length scales, \href{https://en.wikipedia.org/wiki/Surface_tension}{surface tension} effects become important leading to the formation of a \href{https://en.wikipedia.org/wiki/Meniscus_(liquid)}{meniscus} through \href{https://en.wikipedia.org/wiki/Capillary_action}{capillary action}.

This capillary action has profound consequences for biological systems as it is part of 1 of the 2 driving mechanisms of the flow of water in \href{https://en.wikipedia.org/wiki/Plant}{plant} \href{https://en.wikipedia.org/wiki/Xylem}{xylem}, the \href{https://en.wikipedia.org/wiki/Transpirational_pull}{transpirational pull}.

\paragraph{Hanging drops.} Without surface tension, \href{https://en.wikipedia.org/wiki/Drop_(liquid)}{drops} would not be able to form.

The dimensions \& stability of drops are determined by surface tension.

The drop's surface tension is directly proportional to the cohesion property of the fluid.'' -- \href{https://en.wikipedia.org/wiki/Hydrostatics}{Wikipedia{\tt/}Hydrostatics}

%------------------------------------------------------------------------------%



%------------------------------------------------------------------------------%

\section{Miscellaneous}

%------------------------------------------------------------------------------%

\printbibliography[heading=bibintoc]
	
\end{document}