\documentclass{beamer}
\usetheme{Madrid}
\usecolortheme{default}
\usepackage[utf8]{vietnam}
\usepackage[backend=biber,natbib=true,style=alphabetic,maxbibnames=50]{biblatex}
\addbibresource{/home/nqbh/reference/bib.bib}
\usepackage{amsmath,amssymb,amsthm,enumitem,float,graphicx,mathtools,tikz}
\title{Discrete Mathematics for Computer Science\\Toán Rời Rạc Cho Khoa Học Máy Tính}
\author{Nguyễn Quản Bá Hồng}
\institute{Presented at University of Management \& Technology of HCMC (UMT)}
\logo{\includegraphics[height=5mm]{UMT_logo}}
\date{\today}

\begin{document}
\frame{\titlepage}
\begin{frame}
	\frametitle{Table of Contents}
	\tableofcontents
\end{frame}

\section{Introduction to Discrete Mathematics}

\begin{frame}
	\frametitle{What is Discrete Mathematics?}
	
	\begin{definition}[Discrete mathematics]
		\emph{Discrete mathematics}: study of countable, distinct, or separate mathematical structures.
	\end{definition}
	Old nickname: Finite Mathematics, to distinguished with ``Continuous Mathematics'', including e.g., Calculus, Mathematical Analysis.
	
	\begin{example}[Pixel]
		Phones, computer monitors, televisions, modern screens, \& Disney cartoons, animated films for kids \& for adults, e.g., Rick \& Morty.
	\end{example}
	
\end{frame}

\begin{frame}
	\frametitle{Some Critical Thinking Questions}
	{\bf Targets.} Typical super-lazy unmotivated undergraduate{\tt/}graduate students majored in Natural Science.
	\begin{block}{Some purpose-driven questions}
		\begin{itemize}\it
			\item[$\bullet$] Why do undergraduate or graduate students need to learn mathematics?
			\item[$\bullet$] Which type of mathematics do undergraduate or graduate students need to learn?
			\item[$\bullet$] Why do CS-major students need to study Discrete Mathematics?
		\end{itemize}
	\end{block}
\end{frame}

\begin{frame}
	\frametitle{Motivations}
	\begin{itemize}
		\item[$\bullet$] Learn Discrete Mathematics just for fun, for entertaining yourself.
		
		\begin{example}[\href{https://www.imdb.com/title/tt0119217}{\it Good Will Hunting} (1997)]
			{\sc Will Hunting} learned History, Sociology, Psychology $\Psi$, Advanced Mathematics, Combinatorial Discrete Mathematics to flirt hot girls in bars, \& even Advanced Organic Chemistry for fun \& to help her girlfriend.
		\end{example}
		
		\item[$\bullet$] Learn enough Discrete Mathematics to understand different branches of Computer Science.
		
		{\it Main Goal}: Focus strongly on writing programs, developing software, \& building useful applications.
		\item[$\bullet$] If looking for research-oriented jobs, especially Theoretical Computer Science, then need to learn Discrete Mathematics much harder.
		
		{\it Main Goal}: Build some new useful theories, then find their theoretical- or practical real-world applications.
	\end{itemize}
\end{frame}

\section{Applications of Discrete Mathematics}

\begin{frame}
	\frametitle{Combinatorics using SciPy}
	\begin{problem}[Permutation, arrangement, combination]
		Given $n,k\in\mathbb{N}^\star$, $k\le n$. Use {\sf Pascal{\tt/}Python{\tt/}C{\tt/}C++} to compute the numbers of permutations $P_n$, of arrangements $A_n^k$, of combinations $C_n^k$.
	\end{problem}
	
	\begin{proof}[Solution]
		$P_n = n!,A_n^k = \frac{n!}{(n - k)!},C_n^k = \frac{n!}{k!(n - k)!}$. Run {\tt combinatorics.py}.		
	\end{proof}
	
	\begin{problem}[Pascal triangle]
		Given $n\in\mathbb{N}^\star$. Use {\sf Pascal{\tt/}Python{\tt/}C{\tt/}C++} to print the 1st $n + 1$ lines of the Pascal triangle.
	\end{problem}
\end{frame}

\begin{frame}
	\frametitle{References}
	\begin{itemize}
		\item[][GKP89] {\sc Ronald L. Graham, Donald Erwin Knuth, Oren Patashmik}. {\it Concrete Mathematics: A Foundation for Computer Science}. 2e.
		
		\item[][WR21] {\sc Ryan T. White, Archana Tikayat Ray}. {\it Practical Discrete Mathematics: Discover math principles that fuel algorithms for computer science \& machine learning with Python}.		
	\end{itemize}
	
\end{frame}

\end{document}