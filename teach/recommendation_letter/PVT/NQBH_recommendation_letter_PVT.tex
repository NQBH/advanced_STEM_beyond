\documentclass[11pt]{article}
\usepackage[backend=biber,natbib=true,style=alphabetic,maxbibnames=50]{biblatex}
\addbibresource{/home/nqbh/reference/bib.bib}
\usepackage[utf8]{vietnam}
\usepackage{tocloft}
\renewcommand{\cftsecleader}{\cftdotfill{\cftdotsep}}
\usepackage[colorlinks=true,linkcolor=blue,urlcolor=red,citecolor=magenta]{hyperref}
\usepackage{amsmath,amssymb,amsthm,enumitem,float,graphicx,mathtools,tikz}
\usetikzlibrary{angles,calc,intersections,matrix,patterns,quotes,shadings}
\allowdisplaybreaks
\newtheorem{assumption}{Assumption}
\newtheorem{baitoan}{}
\newtheorem{cauhoi}{Câu hỏi}
\newtheorem{conjecture}{Conjecture}
\newtheorem{corollary}{Corollary}
\newtheorem{dangtoan}{Dạng toán}
\newtheorem{definition}{Definition}
\newtheorem{dinhly}{Định lý}
\newtheorem{dinhnghia}{Định nghĩa}
\newtheorem{example}{Example}
\newtheorem{ghichu}{Ghi chú}
\newtheorem{hequa}{Hệ quả}
\newtheorem{hypothesis}{Hypothesis}
\newtheorem{lemma}{Lemma}
\newtheorem{luuy}{Lưu ý}
\newtheorem{nhanxet}{Nhận xét}
\newtheorem{notation}{Notation}
\newtheorem{note}{Note}
\newtheorem{principle}{Principle}
\newtheorem{problem}{Problem}
\newtheorem{proposition}{Proposition}
\newtheorem{question}{Question}
\newtheorem{remark}{Remark}
\newtheorem{theorem}{Theorem}
\newtheorem{vidu}{Ví dụ}
\usepackage[left=3cm,right=2cm,top = 2.5cm,bottom=3cm]{geometry}
\def\labelitemii{$\circ$}
\DeclareRobustCommand{\divby}{%
    \mathrel{\vbox{\baselineskip.65ex\lineskiplimit0pt\hbox{.}\hbox{.}\hbox{.}}}%
}
\setlist[itemize]{leftmargin=*}
\setlist[enumerate]{leftmargin=*}

\begin{document}

\begin{figure}[H]
    \includegraphics[width=6cm]{UMT_logo}
\end{figure}

\begin{flushleft}
     MSc. {\sc Nguyen Quan Ba Hong}
     
     Department of Artificial Intelligence {\it\&} Data Science (AI-DS),
     
     School of Technology (SOT),
     
     UMT University of Management {\it\&} Technology Ho Chi Minh City.
     
     Address: Street no. 60 -- Cat Lai, Thu Duc, Ho Chi Minh city, Viet Nam.
     
     E-mail: {\sf nguyenquanbahong@gmail.com} {\it\&} {\sf hong.nguyenquanba@umt.edu.vn}.
     
     Website: \url{https://nqbh.github.io/}. GitHub: \url{https://github.com/NQBH}.
\end{flushleft}

\begin{flushright}
    ThS. {\sc Nguyễn Quản Bá Hồng}
    
    Bộ Môn Trí Tuệ Nhân Tạo {\it\&} Khoa Học Dữ Liệu (AI-DS),
    
    Khoa Công Nghệ (SOT),
    
    Trường Đại học Quản lý {\it\&} Công nghệ Thành phố Hồ Chí Minh.
    
    Địa chỉ: Đường 60 -- Phường Cát Lái, Khu đô thị, Thủ Đức, Hồ Chí Minh, Việt Nam.
    
    E-mail: {\sf nguyenquanbahong@gmail.com} {\it\&} {\sf hong.nguyenquanba@umt.edu.vn}.
    
    Website: \url{https://nqbh.github.io/}. GitHub: \url{https://github.com/NQBH}.
\end{flushright}
\vspace{5mm}
\begin{center}
    \LARGE
    \textbf{\textsf{Recommendation Letter for Undergraduate Student}}
\end{center}
As a Mathematics \& Computer Science lecturer at University of Management {\it\&} Technology Ho Chi Minh City, I, {\sc Nguyen Quan Ba Hong}, hereby certify the outstanding quality of {\sc Phan Vinh Tien}, a 3rd-year undergraduate student in Computer Science \& Information Technology, majored in Mathematics for Artificial Intelligence, especially for Machine Learning \& Deep Learning. {\sc Phan Vinh Tien} has some outstanding academic achievements as follows:
\begin{enumerate}
    \item Special Prize due to achieving both Gold medals in Algebra B \& Analysis B, Vietnamese Mathematical Olympiad for College Students in Sai Gon University, Apr 2025.
    \item Research in Computer Vision in Emotion Recognition.
    \item Conducting Bachelor Thesis in Machine Learning, Deep Learning, \& Optimizations for Combinatorics \& Graph Theory. Advisors: Associate Professor {\sc Tran Dan Thu}, MSc. {\sc Nguyen Quan Ba Hong}, \& ME. {\sc Huynh Le Phu Trung}.
\end{enumerate}

\begin{center}
    \LARGE
    \textbf{\textsf{Thư Giới Thiệu Cho Sinh Viên Đại Học}}
\end{center}
Với vai trò là 1 giảng viên về Toán học \& Khoa học Máy tính tại Trường Đại Học Quản lý {\it\&} Công nghệ Thành phố Hồ Chí Minh, tôi, {\sc Nguyễn Quản Bá Hồng} xác nhận các phẩm chất học tập \& nghiên cứu của sinh viên {\sc Phan Vĩnh Tiến}, 1 sinh viên năm 3 Đại học về Khoa Học Máy Tính \& Công Nghệ Thông Tin, đi sâu về Học Máy \& Học Sâu. {\sc Phan Vĩnh Tiến} đã gặt hái 1 số thành tích cũng như đang có cái tiến triển về học tập \& nghiên cứu như sau:
\begin{enumerate}
    \item Giải Đặc biệt do được cả Giải Nhất (Huy chương Vàng) cả 2 môn Đại số \& Giải tích bảng B, kỳ thi Olympic Toán Sinh viên Toàn Quốc tại Đại Học Sài Gòn, tháng 4 2025.
    \item Thực hiện đề tài Nghiên cứu Khoa học cấp trường UMT về lĩnh vực Thị Giác Máy Tính với đề tài Nhận diện Cảm xúc của Gương mặt.
    \item Đang thực hiện Khóa luận Tốt nghiệp với đề tài về Học Máy, Học Sâu, \& Toán Tối Ưu cho Tổ Hợp \& Lý Thuyết Đồ Thị, dưới sự hướng dẫn của PGS.TS. {\sc Trần Đan Thư}, ThS. {\sc Nguyễn Quản Bá Hồng}, \& ThS. {\sc Huỳnh Lê Phú Trung}.
\end{enumerate}
Qua các thành tích xuất xác \& quá trình học tập nghiêm túc của {\sc Phan Vĩnh Tiến}. Tôi viết thư giới thiệu này để đảm bảo em có năng lực, trình độ chuyên môn, \& sự vượt khó để tham gia Trường Hè do Viện Nghiên Cứu về Toán Cao Cấp tổ chức.

%------------------------------------------------------------------------------%

\begin{flushright}    
    Mathematics \& Computer Science lecturer\\in Department of Artificial Intelligence {\it\&} Data Science (AI-DS),\\School of Technology (SOT),\\UMT University of Management {\it\&} Technology Ho Chi Minh City
    
    ThS. {\sc Nguyễn Quản Bá Hồng}
\end{flushright}

%------------------------------------------------------------------------------%

\printbibliography[heading=bibintoc]

\end{document}