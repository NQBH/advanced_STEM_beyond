\documentclass{article}
\usepackage[backend=biber,natbib=true,style=alphabetic,maxbibnames=50]{biblatex}
\addbibresource{/home/nqbh/reference/bib.bib}
\usepackage[utf8]{vietnam}
\usepackage{tocloft}
\renewcommand{\cftsecleader}{\cftdotfill{\cftdotsep}}
\usepackage[colorlinks=true,linkcolor=blue,urlcolor=red,citecolor=magenta]{hyperref}
\usepackage{amsmath,amssymb,amsthm,enumitem,float,graphicx,mathtools,tikz}
\usetikzlibrary{angles,calc,intersections,matrix,patterns,quotes,shadings}
\allowdisplaybreaks
\newtheorem{assumption}{Assumption}
\newtheorem{baitoan}{Bài toán}
\newtheorem{cauhoi}{Câu hỏi}
\newtheorem{conjecture}{Conjecture}
\newtheorem{corollary}{Corollary}
\newtheorem{dangtoan}{Dạng toán}
\newtheorem{definition}{Definition}
\newtheorem{dinhly}{Định lý}
\newtheorem{dinhnghia}{Định nghĩa}
\newtheorem{example}{Example}
\newtheorem{ghichu}{Ghi chú}
\newtheorem{goal}{Goal}
\newtheorem{hequa}{Hệ quả}
\newtheorem{hypothesis}{Hypothesis}
\newtheorem{lemma}{Lemma}
\newtheorem{luuy}{Lưu ý}
\newtheorem{nhanxet}{Nhận xét}
\newtheorem{notation}{Notation}
\newtheorem{note}{Note}
\newtheorem{principle}{Principle}
\newtheorem{problem}{Problem}
\newtheorem{proposition}{Proposition}
\newtheorem{question}{Question}
\newtheorem{remark}{Remark}
\newtheorem{theorem}{Theorem}
\newtheorem{vidu}{Ví dụ}
\usepackage[left=1cm,right=1cm,top=5mm,bottom=5mm,footskip=4mm]{geometry}
\def\labelitemii{$\circ$}
\DeclareRobustCommand{\divby}{%
	\mathrel{\vbox{\baselineskip.65ex\lineskiplimit0pt\hbox{.}\hbox{.}\hbox{.}}}%
}
\setlist[itemize]{leftmargin=*}
\setlist[enumerate]{leftmargin=*}

\title{Lecture Note: Combinatorics {\it\&} Graph Theory\\Bài Giảng: Tổ Hợp {\it\&} Lý Thuyết Đồ Thị}
\author{Nguyễn Quản Bá Hồng\footnote{A Scientist {\it\&} Creative Artist Wannabe. E-mail: {\tt nguyenquanbahong@gmail.com, hong.nguyenquanba@umt.edu.vn}. Bến Tre City, Việt Nam.}}
\date{\today}

\begin{document}
\maketitle
\begin{abstract}
	This text is a part of the series {\it Some Topics in Advanced STEM \& Beyond}:
	
	{\sc url}: \url{https://nqbh.github.io/advanced_STEM/}.
	
	Latest version:
	\begin{itemize}
		\item {\it Lecture Note: Combinatorics \& Graph Theory -- Bài Giảng: Tổ Hợp \& Lý Thuyết Đồ Thị}.
		
		PDF: {\sc url}: \url{https://github.com/NQBH/advanced_STEM_beyond/blob/main/combinatorics/lecture/NQBH_combinatorics_graph_theory_lecture.pdf}.
		
		\TeX: {\sc url}: \url{https://github.com/NQBH/advanced_STEM_beyond/blob/main/combinatorics/lecture/NQBH_combinatorics_graph_theory_lecture.tex}.
		\item {\it Slide: Combinatorics \& Graph Theory -- Slide Bài Giảng: Tổ Hợp \& Lý Thuyết Đồ Thị}.
		
		PDF: {\sc url}: \url{https://github.com/NQBH/advanced_STEM_beyond/blob/main/combinatorics/slide/NQBH_combinatorics_graph_theory_slide.pdf}.
		
		\TeX: {\sc url}: \url{https://github.com/NQBH/advanced_STEM_beyond/blob/main/combinatorics/slide/NQBH_combinatorics_graph_theory_slide.tex}.
		\item {\it Survey: Combinatorics \& Graph Theory -- Khảo Sát: Tổ Hợp \& Lý Thuyết Đồ Thị}.
		
		PDF: {\sc url}: \url{https://github.com/NQBH/advanced_STEM_beyond/blob/main/combinatorics/NQBH_combinatorics.pdf}.
		
		\TeX: {\sc url}: \url{https://github.com/NQBH/advanced_STEM_beyond/blob/main/combinatorics/NQBH_combinatorics.tex}.
		\item Codes:
		\begin{itemize}
			\item C{\tt/}C++: \url{https://github.com/NQBH/advanced_STEM_beyond/blob/main/combinatorics/C++}.
			\item Python: \url{https://github.com/NQBH/advanced_STEM_beyond/blob/main/combinatorics/Python}.
		\end{itemize}
	\end{itemize}
\end{abstract}
\tableofcontents

%------------------------------------------------------------------------------%

\section{Basic Combinatorics -- Tổ Hợp Cơ Bản}

%------------------------------------------------------------------------------%

\subsection{Nguyên lý bù trừ}

\begin{dinhly}[Nguyên lý bù trừ{\tt/}nguyên lý bao hàm--loại trừ]
	\item(i) Với 2 tập hợp hữu hạn $A,B$ bất kỳ, $|A\cup B| = |A| + |B| - |A\cap B|$, $|A\backslash B| = |A| - |A\cap B|$.
	\item(ii) Với 3 tập hợp hữu hạn $A,B,C$ bất kỳ, $|A\cup B\cup C| = |A| + |B| + |C| - |A\cap B| - |B\cap C| - |C\cap A| + |A\cap B\cap C|$.
	\item(iii) Với $n\in\mathbb{N}^\star$, $A_i$, $i = 1,\ldots,n$, là $n$ tập hợp hữu hạn bất kỳ:
	\begin{equation*}
		\left|\bigcup_{i=1}^n A_i\right| = \sum_{T\subseteq\{1,\ldots,n\},\,T\ne\emptyset} (-1)^{|T| + 1}\left|\bigcap_{i\in T} A_i\right|.
	\end{equation*}
	Từ đó suy ra,
	\begin{equation*}
		\left|\bigcup_{i=1}^n A_i\right|\ge\sum_{i=1}^n |A_i| - \sum_{1\le i < j\le n} |A_i\cap A_j|.
	\end{equation*}
\end{dinhly}

%------------------------------------------------------------------------------%

\subsection{Mathematical induction \& recurrence -- Quy nạp \& truy hồi}

%------------------------------------------------------------------------------%

\subsection{Pigeonhole principle \& Ramsey theory -- Nguyên lý chuồng bồ câu \& lý thuyết Ramsey}

%------------------------------------------------------------------------------%

\subsection{Counting rules \& Stirling number of type 1 \& type 2}

%------------------------------------------------------------------------------%

\subsection{Hoán vị \& tổ hợp}

\begin{baitoan}[Consecutive coin toss -- Gieo các đồng xu liên tiếp]
	Cho $n,k\in\mathbb{N}^\star$, $k\le n$. Tung 1 đồng xu đồng chất ngẫu nhiên $n$ lần. Tính xác suất lý thuyết của sự kiện: (a) Toàn bộ đều là mặt sấp (ngửa). (b) Có đúng $k$ lần xuất hiện mặt sấp (ngửa). (c) Có ít nhất $k$ lần xuất hiện mặt sấp (ngửa). (d) Có đúng $k$ lần xuất hiện mặt sấp (ngửa) liên tiếp nhau. (e) Có ít nhất $k$ lần xuất hiện mặt sấp (ngửa) liên tiếp nhau.
\end{baitoan}

\begin{proof}[Giải]
	Gọi $X_i\in\{S,N\}$ là biến cố ngẫu nhiên biểu diễn mặt đồng xu trong lần tung thứ $i$, $\forall i = 1,\ldots,n$. Không gian mẫu: $|\Omega| = \prod_{i=1}^n 2 = 2^n$. (a) Vì chỉ có 1 trường hợp thuận lợi là $(S,S,\ldots,S)$ nên $\mathbb{P}(X_i = S,\ \forall i = 1,\ldots,n) = \mathbb{P}(|\{i;X_i = S\}| = n) = \dfrac{1}{2^n}$. Tương tự, vì chỉ có 1 trường hợp thuận lợi là $(N,N,\ldots,N)$ nên $\mathbb{P}(X_i = N,\ \forall i = 1,\ldots,n) = \mathbb{P}(|\{i;X_i = N\}| = n) = \dfrac{1}{2^n}$. (b) $\mathbb{P}(|\{i;X_i = S\}| = k) = \mathbb{P}(|\{i;X_i = N\}| = k) = \dfrac{C_n^k}{2^n}$, $\forall k = 0,\ldots,n$. (c) $\mathbb{P}(|\{i;X_i = S\}|\ge k) = \mathbb{P}(|\{i;X_i = N\}|\ge k) = \dfrac{C_n^k + C_n^{k+1} + \cdots + C_n^n}{2^n} = \dfrac{\sum_{i=k}^n C_n^i}{2^n}$, $\forall k = 0,\ldots,n$. (d) $\mathbb{P} = \dfrac{n - k + 1}{2^n}$. (e) $\mathbb{P} = \dfrac{\sum_{i=k}^n (n - i + 1)}{2^n} = \dfrac{(n + 1)(n - k + 1) - \dfrac{(n + k)(n - k + 1)}{2}}{2^n}$.
	
\end{proof}

\begin{baitoan}[Simultaneous coin toss -- Gieo các đồng xu đồng thời]
	Cho $n,k\in\mathbb{N}^\star$, $k\le n$. Tung đồng thời $n$ đồng xu đồng chất ngẫu nhiên. Tính xác suất lý thuyết của sự kiện: (a) Toàn bộ đều là mặt sấp (ngửa). (b) Có đúng $k$ lần xuất hiện mặt sấp (ngửa). (c) Có ít nhất $k$ lần xuất hiện mặt sấp (ngửa).
\end{baitoan}

\begin{proof}[Giải]
	Gọi $X$ là biến cố ngẫu nhiên chỉ số mặt S xuất hiện khi tung đồng thời $n$ đồng xu. (a) $\mathbb{P}(X = n) = \mathbb{P}(X = 0) = \frac{1}{n + 1}$. (b) $\mathbb{P}(X = k) = \frac{1}{n + 1}$
\end{proof}

\begin{baitoan}[Consecutive 2 dice rolls -- Gieo 2 xúc xắc lần lượt]
	Gieo lần lượt 2 con xúc xắc. Tính xác suất lý thuyết của sự kiện: (a) 2 mặt có cùng số chấm, khác số chấm. (b) Số chấm 2 mặt có cùng tính chẵn lẻ, khác tính chẵn lẻ. (c) Số chấm 2 mặt đều là số nguyên tố, đều là hợp số, có ít nhất 1 số nguyên tố, có ít nhất 1 hợp số. (d) Số chấm 1 mặt là ước (bội) của số chấm trên mặt còn lại. (e) Tổng số chấm 2 mặt bằng $n\in\mathbb{N}$.
\end{baitoan}
{\sf Ans.} (e) $f(n) = (\min\{n - 1, 6\} - \max\{n - 6,1\} + 1){\bf1}_{n\in\{2,3,\ldots,12\}}$.

\begin{baitoan}[Simultaneous 2 dice rolls -- Gieo 2 xúc xắc đồng thời]
	Gieo đồng thời 2 con xúc xắc. Tính xác suất lý thuyết của sự kiện: (a) 2 mặt có cùng số chấm, khác số chấm. (b) Số chấm 2 mặt có cùng tính chẵn lẻ, khác tính chẵn lẻ. (c) Số chấm 2 mặt đều là số nguyên tố, đều là hợp số, có ít nhất 1 số nguyên tố, có ít nhất 1 hợp số. (d) Số chấm 1 mặt là ước (bội) của số chấm trên mặt còn lại. (e) Tổng số chấm 2 mặt bằng $n\in\mathbb{N}$.
\end{baitoan}

\begin{baitoan}[Consecutive $n$ dice rolls -- Gieo $n$ xúc xắc lần lượt]
	Gieo lần lượt $n\in\mathbb{N}^\star$ con xúc xắc. Tính xác suất lý thuyết của sự kiện: (a) $n$ mặt có cùng số chấm. (b) $n$ mặt có khác số chấm. (c) Số chấm $n$ mặt có cùng tính chẵn lẻ. (d) Số chấm 1 mặt là ước (bội) của số chấm trên các mặt còn lại. (e) Tổng số chấm $n$ mặt bằng $a\in\mathbb{N}$.
\end{baitoan}

\begin{baitoan}[Simultaneous $n$ dice rolls -- Gieo $n$ xúc xắc đồng thời]
	Gieo đồng thời $n\in\mathbb{N}^\star$ con xúc xắc. Tính xác suất lý thuyết của sự kiện: (a) $n$ mặt có cùng số chấm. (b) $n$ mặt có khác số chấm. (c) Số chấm $n$ mặt có cùng tính chẵn lẻ. (d) Số chấm 1 mặt là ước (bội) của số chấm trên các mặt còn lại. (e) Tổng số chấm $n$ mặt bằng $a\in\mathbb{N}$.
\end{baitoan}

\begin{baitoan}[Squares \& rectangles with same perimeter -- Hình vuông \& hình chữ nhật cùng chu vi]
	Cho $n\in\mathbb{N}^\star$. Viết n thành tổng 2 số: $n = a + b$. Tính xác suất để $a,b$ cùng là độ dài cạnh của 1 hình vuông, xác suất để $a,b$ là độ dài 2 cạnh của 1 hình chữ nhật nếu: (a) $a,b\in\mathbb{N}^\star$. (b) $a,b\in\mathbb{N}$.
\end{baitoan}

\begin{baitoan}[Squares \& rectangles with same area -- hình vuông \& hình chữ nhật cùng diện tích]
	Cho $a\in\mathbb{N}^\star,a\ge2$ có phân tích thừa số nguyên tố $a = \prod_{i=1}^{n} p_i^{a_i} = p_1^{a_1}p_2^{a_2}\cdots p_n^{a_n}$ với $p_i$ là số nguyên tố, $a_i\in\mathbb{N}^\star$, $\forall i = 1,2,\ldots,n$. (a) Viết ngẫu nhiên a thành tích của 2 số: $a = bc$. Tính xác suất để $b,c$ là độ dài 2 cạnh của 1 hình chữ nhật, xác suất để $b,c$ cùng là độ dài cạnh của 1 hình vuông nếu: (i) $b,c\in\mathbb{N}$. (ii) $b,c\in\mathbb{Z}$. (b) Lấy ngẫu nhiên 2 số $b,c\in\mbox{\rm Ư}(a)$. Tính xác suất để phân số $\dfrac{b}{c}$: (i) tối giản. (ii) không tối giản.
\end{baitoan}

\begin{definition}[Prime-counting function]
	The {\rm prime-counting function} is the function counting the number of prime numbers less than or equal to some real number x, denoted by $\pi(x)\coloneqq|\{p\in\mathbb{N}^\star|p \mbox{ is a prime},\ p\le x\}|$.
\end{definition}

\begin{dinhnghia}[Hàm đếm số số nguyên tố]
	{\rm Hàm đếm số số nguyên tố} là hàm đếm số số nguyên tố nhỏ hơn hoặc bằng $x\in\mathbb{R}$, ký hiệu là $\pi(x)\coloneqq|\{p\in\mathbb{N}^\star|p \mbox{ là số nguyên tố},\ p\le x\}|$.
\end{dinhnghia}

\begin{baitoan}[Prime, composite -- số nguyên tố, hợp số]
	Cho $m,n,k\in\mathbb{N}^\star$. Đặt $A_n = \{1,2,\ldots,n\}$ là tập hợp $n$ số nguyên dương đầu tiên, $\forall n\in\mathbb{N}^\star$. (a) Lấy m số từ $A_n$. Tính xác suất để m số này cùng chẵn, cùng lẻ, có ít nhất 1 số chẵn, có ít nhất 1 số lẻ, có đúng k số chẵn, có đúng k số lẻ, có ít nhất k số chẵn, có ít nhất k số lẻ. (b) Lấy m số phân biệt từ $A_n$. Tính xác suất để m số này đều là số nguyên tố, đều là hợp số, có đúng k số nguyên tố, có đúng k hợp số, có ít nhất 1 số nguyên tố, có ít nhất 1 hợp số, có ít nhất k số nguyên tố, có ít nhất k hợp số. (c) Viết chương trình {\sf Pascal, Python C{\tt/}C++} để mô phỏng việc tính các xác suất đó.
\end{baitoan}

\begin{baitoan}[Odd, even -- chẵn, lẻ]
	Cho $a,b\in\mathbb{Z},a < b$, $n,k\in\mathbb{N}^\star,n\ge2,k\le n$. Đặt $A = [a,b]\cap\mathbb{Z} = \{a,a + 1,a + 2,\ldots,b - 1,b\}$. (a) Lấy 2 số từ tập A. Xét 2 trường hợp phân biệt, không nhất thiết phân biệt. Tính xác suất để 2 số này cùng tính chẵn lẻ, khác tính chẵn lẻ. (b) Lấy n số từ tập A. Tính xác suất để n số này đều chẵn, đều lẻ, cùng tính chẵn lẻ, có đúng k số chẵn, k số lẻ, có ít nhất k số chẵn, k số lẻ. (c) Viết chương trình {\sf Pascal, Python C{\tt/}C++} để mô phỏng việc tính các xác suất đó.
\end{baitoan}

\begin{baitoan}[VMC2024B4]
	(a) Đếm số cách chọn ra 3 viên gạch, mỗi viên từ 1 hàng trong $3\times5$ viên gạch xếp xen kẽ, sao cho không có 2 viên gạch nào được lấy ra nằm kề nhau (2 viên gạch được gọi là kề nhau nếu có chung 1 phần của 1 cạnh).
	\begin{figure}[H]
		\centering
		\includegraphics[width=8cm]{brick3x5}
	\end{figure}
	(b) Đếm số cách chọn ra 4 viên gạch, mỗi viên từ 1 hàng trong $4\times5$ viên gạch xếp xen kẽ, sao cho không có 2 viên gạch nào được lấy ra nằm kề nhau.
	\begin{figure}[H]
		\centering
		\includegraphics[width=8cm]{brick4x5}
	\end{figure}
	(c) Cho $m,n\in\mathbb{N}^\star$. Đếm số cách chọn ra $m$ viên gạch, mỗi viên từ 1 hàng trong $m\times n$ viên gạch xếp xen kẽ, sao cho không có 2 viên gạch nào được lấy ra nằm kề nhau. (d) Cho $m,n,k\in\mathbb{N}^\star$. Đếm số cách chọn ra $k$ viên gạch, không nhất thiết mỗi viên từ 1 hàng trong $m\times n$ viên gạch xếp xen kẽ, sao cho không có 2 viên gạch nào được lấy ra nằm kề nhau. (e${}^\star$) Mở rộng cho trường hợp $m\times n$ với số gạch mỗi hàng có thể khác nhau, cụ thể là hàng $i$ chứa $a_i\in\mathbb{N}^\star$ viên gạch, $\forall i = 1,\ldots,m$ với 2 trường hợp: (i) Mỗi viên từ 1 hàng. (ii) Lấy $k\in\mathbb{N}^\star$ viên gạch, mỗi hàng có thể lấy nhiều viên.
\end{baitoan}

\begin{nhanxet}[Left-right symmetry -- Đối xứng trái phải]
	Nếu số viên gạch của mỗi hàng bằng nhau \& được sắp xen kẽ như (a) \& (b), thì thứ tự viên gạch đầu tiên từ bên trái của mỗi hàng lồi ra hay thụt vào không quan trọng, vì có thể lấy đối xứng gương trái--phải để chuyển đổi 2 trường hợp đó. Cũng chú ý đến tính đối xứng trên--dưới (top-bottom symmetry).
\end{nhanxet}

\begin{proof}
	Số cách chọn gạch từ 3 hàng, mỗi hàng $n$ viên gạch: $(n - 1)(n - 2)^2 + (n - 1)^2 = (n - 1)(n^2 - 3n + 3)$, $\forall n\in\mathbb{N}^\star$, $n\ge2$. Số cách chọn gạch từ 4 hàng, mỗi hàng $n\in\mathbb{N}^\star$ viên gạch: $(n^2 - 3n + 3)^2$, $\forall n\in\mathbb{N}^\star$, $n\ge2$.
\end{proof}

\begin{itemize}
	\item C++ codes:
	\begin{itemize}
		\item (DKAK): \url{https://github.com/NQBH/advanced_STEM_beyond/blob/main/VMC/C++/brick_DPAK.cpp}.
		\item (NLDK): \url{https://github.com/NQBH/advanced_STEM_beyond/blob/main/VMC/C++/brick_NLDK.cpp}.
	\end{itemize}
\end{itemize}

%------------------------------------------------------------------------------%

\subsection{Hệ số nhị thức \& đa thức}

%------------------------------------------------------------------------------%

\subsection{Phân vùng số nguyên \& nguyên tắc loại suy}

%------------------------------------------------------------------------------%

\section{Graph Theory -- Lý Thuyết Đồ Thị}
\textbf{\textsf{Resources -- Tài nguyên.}}
\begin{enumerate}
	\item \cite{Andreescu_Dospinescu2010}. {\sc Titu Andreescu, Gabriel Dospinescu}. {\it Problems From the Book}. Chap. 6: {\it Some Classical Problems in Extremal Graph Theory -- Vài Bài Toán Cổ Điển trong Lý Thuyết Đồ Thị Cực Trị}, pp. 119--136.
	
	\item \cite{Valiente2002, Valiente2021}. {\sc Gabriel Valiente}. {\it Algorithms on Trees \& Graphs With Python Code}. 2e.
\end{enumerate}

\subsection{Trees \& graphs: Some basic concepts -- Cây \& đồ thị: Vài khái niệm cơ bản}
The notion of graph which is most useful in computer science is that of a directed graph or just a graph. A graph is a combinatorial structure consisting of a finite nonempty set of objects, called vertices, together with a finite (possibly empty) set of ordered pairs of vertices, called directed edges or arcs.

-- Khái niệm đồ thị hữu ích nhất trong khoa học máy tính là đồ thị có hướng hoặc chỉ là đồ thị. Đồ thị là 1 cấu trúc tổ hợp bao gồm 1 tập hợp hữu hạn không rỗng các đối tượng, được gọi là các đỉnh, cùng với 1 tập hợp hữu hạn (có thể rỗng) các cặp đỉnh có thứ tự, được gọi là các cạnh có hướng hoặc cung.

\begin{definition}[Directed graph, \cite{Valiente2021}, Def. 1.1, p. 3]
	A {\rm graph} $G = (V,E)$ consists of a finite nonempty set $V$ of vertices \& a finite set $E\subseteq V\times V$ of edges. The {\rm order} of a graph $G = (V,E)$, denoted by $n$, is the number of vertices, $n = |V|$ \& the {\rm size}, denoted by $m$, is the number of edges, $m = |E|$. An edge $e = (v,w)$ is said to be {\rm incident} with vertices $v$ \& $w$, where $v$ is the {\rm source} \& $w$ the {\rm target} of edge $e$, \& vertices $v,w$ are said to be {\rm adjacent}. Edges $(u,v),(v,w)$ are said to be {\rm adjacent}, as are edges $(u,v),(w,v)$, \& also edges $(v,u),(v,w)$.
\end{definition}

\begin{dinhnghia}[Đồ thị có hướng]
	1 {\rm đồ thị} $G = (V,E)$ bao gồm 1 tập hữu hạn không rỗng $V$ các đỉnh \& 1 tập hữu hạn $E\subseteq V\times V$ các cạnh. {\rm Bậc} của 1 đồ thị $G = (V,E)$, ký hiệu là $n$, là số đỉnh, $n = |V|$ \& {\rm size}, ký hiệu là $m$, là số cạnh, $m = |E|$. Một cạnh $e = (v,w)$ được gọi là {\rm incident} với các đỉnh $v$ \& $w$, trong đó $v$ là {\rm source} \& $w$ {\rm target} của cạnh $e$, \& các đỉnh $v,w$ được gọi là {\rm kề}. Các cạnh $(u,v),(v,w)$ được gọi là {\rm kề}, cũng như các cạnh $(u,v),(w,v)$, \& cũng vậy các cạnh $(v,u),(v,w)$.
\end{dinhnghia}
Graphs are often drawn as a set of points in the plane \& a set of arrows, each of which joins 2 (not necessarily different) points. In a drawing of a graph $G = (V,E)$, each vertex $v\in V$ is drawn as a point or a small circle \& each edge $(v,w)\in E$ is drawn as an arrow from point or circle of vertex $v$ to the point or circle corresponding to vertex $w$.

-- Đồ thị thường được vẽ như một tập hợp các điểm trên mặt phẳng \& một tập hợp các mũi tên, mỗi mũi tên nối 2 điểm (không nhất thiết phải khác nhau). Trong bản vẽ đồ thị $G = (V,E)$, mỗi đỉnh $v\in V$ được vẽ như một điểm hoặc một đường tròn nhỏ \& mỗi cạnh $(v,w)\in E$ được vẽ như một mũi tên từ điểm hoặc đường tròn của đỉnh $v$ đến điểm hoặc đường tròn tương ứng với đỉnh $w$.

A vertex has 2 degrees in a graph, one given by the number of edges coming into the vertex \& the other given by the number of edges in the graph going out of the vertex.

-- Mỗi đỉnh có 2 bậc trong đồ thị, một bậc được xác định bởi số cạnh đi vào đỉnh \& bậc còn lại được xác định bởi số cạnh trong đồ thị đi ra khỏi đỉnh.

\begin{definition}[\cite{Valiente2021}, Def. 1.2, p. 4]
	The {\rm indegree} of a vertex $v$ in a graph $G = (V,E)$ is the number of edges in $G$ whose target is $v$, i.e., ${\rm indeg}(v) = |\{(u,v)|(u,v)\in E\}|$. The {\rm outdegree} of a vertex $v$ in a graph $G = (V,E)$ is the number of edges in $G$ whose source is $v$, i.e., ${\rm outdeg}(v) = |\{(v,w)|(v,w)\in E\}|$. The {\rm degree} of a vertex $v$ in a graph $G = (V,E)$ is the sum of the indegree \& the outdegree of the vertext, i.e., ${\rm deg}(v) = {\rm indeg}(v) + {\rm outdeg}(v)$.
\end{definition}
A basic relationship between the size of a graph \& the degree of its vertices, which will prove to be very useful in analyzing the computational complexity of algorithms on graphs:

-- Mối quan hệ cơ bản giữa kích thước của đồ thị \& bậc của các đỉnh, sẽ rất hữu ích trong việc phân tích độ phức tạp tính toán của các thuật toán trên đồ thị:

\begin{theorem}
	Let $G = (V,E)$ be a graph with $n$ vertices \& $m$ edges, \& let $V = \{v_1,\ldots,v_n\}$. Then
	\begin{equation*}
		\sum_{i=1}^n {\rm indeg}(v_i) = \sum_{i=1}^n {\rm outdeg}(v_i) = m.
	\end{equation*}
\end{theorem}
Walks, trails, \& paths in a graph are alternating sequences of vertices \& edges in the graph s.t. each edge in the sequence is preceded by its source vertex \& followed by its target vertex. Trails are walks having no repeated edges, \& paths are trails having no repeated vertices.

-- Đường đi, đường mòn, \& đường đi trong đồ thị là chuỗi xen kẽ các đỉnh \& cạnh trong đồ thị, tức là mỗi cạnh trong chuỗi được đi trước bởi đỉnh nguồn \& theo sau bởi đỉnh đích. Đường mòn là đường đi không có cạnh lặp lại, \& đường đi là đường mòn không có đỉnh lặp lại.

Denote by $d(V),C(V)$ the number, \& the set of vertices adjacent to a vertex $V$, respectively. A graph is said to have a {\it complete $k$-subgraph} if there are $k$ vertices any 2 of which are connected. A graph is said to be {\it $k$-free} if it does not contain a complete $k$-subgraph.

\begin{lemma}[\cite{Andreescu_Dospinescu2010}, Example 1, p. 121, Zarankiewicz's lemma]
	If $G$ is a $k$-free graph, then there exists a vertex having degree at most $\left\lfloor\dfrac{k - 2}{k - 1}n\right\rfloor$.
\end{lemma}
Zarankiewicz's lemma is the main step in the proof of Turan's theorem -- a famous classical result about $k$-free graphs.

\begin{theorem}[\cite{Andreescu_Dospinescu2010}, Example 2, p. 123, Turan's theorem]
	The greatest number of edges of a $k$-free graph with $n$ vertices is
	\begin{equation*}
		\frac{k - 2}{k - 1}\cdot\frac{n^2 - r^2}{2} + \binom{r}{2},
	\end{equation*}
	where $r$ is the remainder left by $n$ when divided to $k - 1$.
\end{theorem}

\begin{dinhnghia}[\cite{Ha_Thanh_to_hop}, Def. 7.2, p. 249, Đỉnh cô lập, lá]
	Cho $G$ là 1 đồ thị. Đỉnh có bậc $0$ được gọi là {\rm đỉnh cô lập}, đỉnh có bậc $1$ được gọi là {\rm lá}.
\end{dinhnghia}

\begin{dinhnghia}[\cite{Ha_Thanh_to_hop}, Def. 7.3, p. 249, Đồ thị chính quy]
	1 đồ thị được gọi là {\rm chính quy bậc $d$} hoặc {\rm$d$-chính quy} nếu mỗi đỉnh có bậc bằng $d\in\mathbb{N}$.
\end{dinhnghia}

\begin{dinhnghia}[\cite{Ha_Thanh_to_hop}, Def. 7.3, p. 249, Đỉnh thị khối]
	1 đồ thị được gọi là {\rm đồ thị bậc $3$} nếu nó chính quy bậc $3$, i.e., mỗi đỉnh đồ thị có bậc bằng $3$.
\end{dinhnghia}

\begin{goal}[Tính khả dĩ của dãy bậc của đồ thị]
	Tìm vài dấu hiệu hoặc vài điều kiện cần \& đủ để có thể quyết định liệu 1 dãy số nguyên dương $(a_i)_{i=1}^n\subset\mathbb{N}$ cho trươc có  thể thể là dãy bậc của đồ thị mà không phải vẽ biểu đồ.
\end{goal}

\begin{dinhnghia}[\cite{Ha_Thanh_to_hop}, Def. 7.6, p. 249, Dãy bậc của đồ thị, chuỗi đồ thị]
	{\rm Chuỗi bậc} của đồ thị là dãy bậc của các đỉnh của nó theo thứ tự không tăng. 1 dãy số nguyên không âm không tăng được gọi là {\rm đồ thị} nếu tồn tại 1 đồ thị có chuỗi bậc chính xác là dãy số nguyên không âm đó.
\end{dinhnghia}

\begin{vidu}[Sequence $1,1,\ldots,1$]
	$1,1,1$ không phải là 1 dãy đồ thị vì không thể xây dựng 1 đồ thị có 3 đỉnh sao cho tất cả 3 bậc là $1$. Nhưng $1,1$ \& $1,1,1,1$, hay nói chung các dãy chỉ toàn số $1$ với độ dài là 1 số chẵn, i.e., $\{1\}_{i=1}^{2n}$, $\forall n\in\mathbb{N}^\star$, là các dãy đồ thị, nhưng bất kỳ dãy chỉ toàn số $1$ với độ dài là 1 số lẻ, i.e., $\{1\}_{i=1}^{2n+1}$, $\forall n\in\mathbb{N}^\star$, thì không phải là 1 dãy đồ thị (why?)
\end{vidu}

\begin{dinhly}[Euler's, \cite{Ha_Thanh_to_hop}, Thm. 7.9, p. 250]
	\label{thm: Euler}
	Cho $G = (V,E)$ là đồ thị tổng quát với $d_1,\ldots,d_{|V|}\in\mathbb{N}$ là bậc của các đỉnh. Khi đó $\sum_{i=1}^{|V|} d_i = d_1 + d_2 + \cdots + d_{|V|} = 2|E|$. Nói riêng, số đỉnh của $G$ có bậc lẻ là số chẵn.
\end{dinhly}
Briefly:
\begin{equation*}
	d_1,\ldots,d_{|V|}\mbox{ are degrees of vertices of a graph } G = (V,E)\Rightarrow\sum_{i=1}^{|V|} d_i = 2|E|\Rightarrow|\{i;d_i\not{\divby}\ 2\}|\divby2.
\end{equation*}
Chú ý chiều ngược lại chưa chắc đúng:
\begin{vidu}
	Dãy số $7,5,5,4,3,2,2,0$ không mâu thuẫn với Định lý \ref{thm: Euler} nhưng nó không phải là đồ thị (why?).
\end{vidu}

\begin{question}
	Có thể suy ra được những hệ quả nào từ đẳng thức $\sum_{i=1}^{|V|} d_i = 2|E|$?
\end{question}

\begin{baitoan}
	Cho $G = (V,E)$ là đồ thị tổng quát với $d_1,\ldots,d_p\in\mathbb{N}$ là bậc của các đỉnh. Chứng minh: (i) Bậc cao nhất $d_{\max}\coloneqq\max_{1\le i\le p} d_i$ thỏa $d_{\max}\ge\dfrac{2|E|}{|V|}$. (ii)
\end{baitoan}




%------------------------------------------------------------------------------%

\section{Posets, Kết Nối, Lưới Boolean}

%------------------------------------------------------------------------------%

\section{Miscellaneous}

%------------------------------------------------------------------------------%

\printbibliography[heading=bibintoc]
	
\end{document}