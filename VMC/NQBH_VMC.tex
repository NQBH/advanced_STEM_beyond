\documentclass{article}
\usepackage[backend=biber,natbib=true,style=alphabetic,maxbibnames=50]{biblatex}
\addbibresource{/home/nqbh/reference/bib.bib}
\usepackage[utf8]{vietnam}
\usepackage{tocloft}
\renewcommand{\cftsecleader}{\cftdotfill{\cftdotsep}}
\usepackage[colorlinks=true,linkcolor=blue,urlcolor=red,citecolor=magenta]{hyperref}
\usepackage{amsmath,amssymb,amsthm,enumitem,float,graphicx,mathtools,tikz}
\usetikzlibrary{angles,calc,intersections,matrix,patterns,quotes,shadings}
\allowdisplaybreaks
\newtheorem{assumption}{Assumption}
\newtheorem{baitoan}{}
\newtheorem{cauhoi}{Câu hỏi}
\newtheorem{conjecture}{Conjecture}
\newtheorem{corollary}{Corollary}
\newtheorem{dangtoan}{Dạng toán}
\newtheorem{definition}{Definition}
\newtheorem{dinhly}{Định lý}
\newtheorem{dinhnghia}{Định nghĩa}
\newtheorem{example}{Example}
\newtheorem{ghichu}{Ghi chú}
\newtheorem{hequa}{Hệ quả}
\newtheorem{hypothesis}{Hypothesis}
\newtheorem{lemma}{Lemma}
\newtheorem{luuy}{Lưu ý}
\newtheorem{nhanxet}{Nhận xét}
\newtheorem{notation}{Notation}
\newtheorem{note}{Note}
\newtheorem{principle}{Principle}
\newtheorem{problem}{Problem}
\newtheorem{proposition}{Proposition}
\newtheorem{question}{Question}
\newtheorem{remark}{Remark}
\newtheorem{theorem}{Theorem}
\newtheorem{vidu}{Ví dụ}
\usepackage[left=1cm,right=1cm,top=5mm,bottom=5mm,footskip=4mm]{geometry}
\def\labelitemii{$\circ$}
\DeclareRobustCommand{\divby}{%
	\mathrel{\vbox{\baselineskip.65ex\lineskiplimit0pt\hbox{.}\hbox{.}\hbox{.}}}%
}
\setlist[itemize]{leftmargin=*}
\setlist[enumerate]{leftmargin=*}

\title{Vietnamese Mathematical Olympiad for College Students [VMC]\\Olympic Toán Sinh Viên Toàn Quốc}
\author{Nguyễn Quản Bá Hồng\footnote{A Scientist {\it\&} Creative Artist Wannabe. E-mail: {\tt nguyenquanbahong@gmail.com}. Bến Tre City, Việt Nam.}}
\date{\today}

\begin{document}
\maketitle
\begin{abstract}
	This text is a part of the series {\it Some Topics in Advanced STEM \& Beyond}:
	
	{\sc url}: \url{https://nqbh.github.io/advanced_STEM/}.
	
	Latest version:
	\begin{itemize}
		\item {\it Vietnamese Mathematical for College Students [VMC] -- Olympic Toán Sinh Viên Toàn Quốc}.
		
		PDF: {\sc url}: \url{https://github.com/NQBH/advanced_STEM_beyond/blob/main/VMC/NQBH_VMC.pdf}.
		
		\TeX: {\sc url}: \url{https://github.com/NQBH/advanced_STEM_beyond/blob/main/VMC/NQBH_VMC.tex}.
	\end{itemize}
\end{abstract}
\tableofcontents

%------------------------------------------------------------------------------%

\section{Algebra -- Đại Số}

\subsection{Matrix -- Ma trận}

%------------------------------------------------------------------------------%

\section{Analysis -- Giải Tích}

\subsection{Sequence -- Dãy số}
\textbf{\textsf{Resources -- Tài nguyên.}}
\begin{enumerate}
	\item \cite{Khai_so_hoc_day_so}. {\sc Phan Huy Khải}. {\it Các Chuyên Đề Số Học Bồi Dưỡng Học Sinh Giỏi Toán Trung Học. Chuyên Đề 2: Số Học \& Dãy Số}.
	{\sc VMS -- Hội Toán Học Việt Nam}. {\it Kỷ Yếu Kỳ Thi Olympic Toán Học Sinh Viên--Học Sinh Lần 28}.
	\item {\sc VMS -- Hội Toán Học Việt Nam}. {\it Kỷ Yếu Kỳ Thi Olympic Toán Học Sinh Viên--Học Sinh Lần 29}. Huế, 2--8.4.2023.
\end{enumerate}

\begin{baitoan}[General recursive sequences -- Dãy truy hồi tổng quát]
	Cho dãy số $(u_n)_{n=1}^\infty$ được xác định bởi công thức truy hồi
	\begin{equation}
		\boxed{u_n = f(u_{n-1},u_{n-2},\ldots,u_{n-m}),\ \forall m,n\in\mathbb{N}^\star,\ m < n.}
	\end{equation}
	Tìm các tính chất tổng quát của dãy theo 1 số dạng đặc biệt của hàm $f$ để lập thành các mệnh đề \& định lý, rồi chứng minh chúng.
\end{baitoan}
{\sf Vài phương pháp phổ biến để giải bài toán dãy số.}
\begin{itemize}
	\item Tìm cách xác định công thức số hạng tổng quát của dãy số: Thử vài trường hợp đầu để dự đoán công thức chính xác rồi chứng minh bằng quy nạp toán học.
	\item Sử dụng phương trình đặc trưng của lý thuyết dãy số.
\end{itemize}

\begin{baitoan}[VMC2023B]
	Cho $(u_n)_{n=1}^\infty$ là dãy số được xác định bởi $u_n = \prod_{k=1}^n \left(1 + \frac{1}{4^k}\right)$, $\forall n\in\mathbb{N}^\star$. (a) Tìm tất cả $n\in\mathbb{N}^\star$ thỏa $u_n > \frac{5}{4}$. (b) Chứng minh $u_n\le2023$, $\forall n\in\mathbb{N}^\star$. (c) Chứng minh dãy số $(u_n)_{n=1}^\infty$ hội tụ.
\end{baitoan}

\begin{proof}
	(a) $u_{n+1} = \left(1 + \frac{1}{4^{n+1}}\right)u_n > u_n$, $\forall n\in\mathbb{N}^\star$, suy ra $(u_n)$ đơn điệu tăng, mà $u_1 = \frac{5}{4}$ nên $u_n > \frac{5}{4}\Leftrightarrow n\ge2$. (b)
\end{proof}

\begin{remark}
	Gặp phải dãy số $(u_n)_{n=1}^\infty$ có công thức mỗi số hạng là 1 tích thì thử tính $\frac{u_{n+1}}{u_n}$ xem có đơn giản hóa được không. Gặp phải dãy số $(u_n)_{n=1}^\infty$ có công thức mỗi số hạng là 1 tổng thì thử tính $u_{n+1} - u_n$ xem có đơn giản hóa được không.
\end{remark}

\begin{baitoan}[Recursive sequence vs. ANN]
	Tìm mối liên hệ giữa các dãy số cho bởi công thức truy hồi (recursive sequences) \& mạng lưới nơ-ron nhân tạo (artificial neural networks, abbr., ANNs).
\end{baitoan}

%------------------------------------------------------------------------------%

\subsection{Integral -- Tích phân}

%------------------------------------------------------------------------------%

\section{Miscellaneous}

%------------------------------------------------------------------------------%

\printbibliography[heading=bibintoc]
	
\end{document}