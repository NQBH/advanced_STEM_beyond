\documentclass{article}
\usepackage[backend=biber,natbib=true,style=alphabetic,maxbibnames=50]{biblatex}
\addbibresource{/home/nqbh/reference/bib.bib}
\usepackage[utf8]{vietnam}
\usepackage{tocloft}
\renewcommand{\cftsecleader}{\cftdotfill{\cftdotsep}}
\usepackage[colorlinks=true,linkcolor=blue,urlcolor=red,citecolor=magenta]{hyperref}
\usepackage{amsmath,amssymb,amsthm,enumitem,float,graphicx,mathtools,tikz}
\usetikzlibrary{angles,calc,intersections,matrix,patterns,quotes,shadings}
\allowdisplaybreaks
\newtheorem{assumption}{Assumption}
\newtheorem{baitoan}{}
\newtheorem{cauhoi}{Câu hỏi}
\newtheorem{conjecture}{Conjecture}
\newtheorem{corollary}{Corollary}
\newtheorem{dangtoan}{Dạng toán}
\newtheorem{definition}{Definition}
\newtheorem{dinhly}{Định lý}
\newtheorem{dinhnghia}{Định nghĩa}
\newtheorem{example}{Example}
\newtheorem{ghichu}{Ghi chú}
\newtheorem{hequa}{Hệ quả}
\newtheorem{hypothesis}{Hypothesis}
\newtheorem{lemma}{Lemma}
\newtheorem{luuy}{Lưu ý}
\newtheorem{nhanxet}{Nhận xét}
\newtheorem{notation}{Notation}
\newtheorem{note}{Note}
\newtheorem{principle}{Principle}
\newtheorem{problem}{Problem}
\newtheorem{proposition}{Proposition}
\newtheorem{question}{Question}
\newtheorem{remark}{Remark}
\newtheorem{theorem}{Theorem}
\newtheorem{vidu}{Ví dụ}
\usepackage[left=1cm,right=1cm,top=5mm,bottom=5mm,footskip=4mm]{geometry}
\def\labelitemii{$\circ$}
\DeclareRobustCommand{\divby}{%
	\mathrel{\vbox{\baselineskip.65ex\lineskiplimit0pt\hbox{.}\hbox{.}\hbox{.}}}%
}
\setlist[itemize]{leftmargin=*}
\setlist[enumerate]{leftmargin=*}

\title{Literary -- Văn Chương}
\author{Nguyễn Quản Bá Hồng\footnote{A Scientist {\it\&} Creative Artist Wannabe. E-mail: {\tt nguyenquanbahong@gmail.com}. Bến Tre City, Việt Nam.}}
\date{\today}

\begin{document}
\maketitle
\begin{abstract}
	This text is a part of the series {\it Some Topics in Advanced STEM \& Beyond}:
	
	{\sc url}: \url{https://nqbh.github.io/advanced_STEM/}.
	
	Latest version:
	\begin{itemize}
		\item {\it Literary -- Văn Chương}.
		
		PDF: {\sc url}: \url{https://github.com/NQBH/advanced_STEM_beyond/blob/main/literary/NQBH_literary.pdf}.
		
		\TeX: {\sc url}: \url{https://github.com/NQBH/advanced_STEM_beyond/blob/main/literary/NQBH_literary.tex}.
	\end{itemize}
\end{abstract}
\tableofcontents

%------------------------------------------------------------------------------%

\section{{\sc William Strunk Jr., E. B. White}. The Elements of Style}
{\bf \textsf{Resources -- Tài nguyên.}}
\begin{enumerate}
	\item \cite{Strunk_element_style}. {\sc William Strunk Jr.} {\it The Elements of Style}.
	\item \cite{Strunk_White_element_style}. {\sc William Strunk Jr., E. B. White}. {\it The Elements of Style}.
\end{enumerate}

\subsection{Wikipedia{\tt/}The Elements of Style}
``{\it The Elements of Style} (also called {\it Strunk \& White}) is a \href{https://en.wikipedia.org/wiki/Style_guide}{style guide} for formal grammar used in \href{https://en.wikipedia.org/wiki/American_English}{American English} writing. The 1st publishing was written by \href{https://en.wikipedia.org/wiki/William_Strunk_Jr.} n 1918, \& published by \href{https://en.wikipedia.org/wiki/Harcourt_(publisher)}{Harcourt} in 1920, comprising 8 ``elementary rules of usage,'' 10 ``elementary principles of composition,'' ``a few matters of form,'' a list of 49 ``words \& expressions commonly misused,'' \& a list of 57 ``words often misspelled.'' Writer \& editor \href{https://en.wikipedia.org/wiki/E._B._White}{E. B. White} greatly enlarged \& revised the book for publication by \href{https://en.wikipedia.org/wiki/Macmillan_Publishers}{Macmillan} in 1959. That was the 1st edition of the book, which \href{https://en.wikipedia.org/wiki/Time_(magazine)}{Time} recognized in 2011 as 1 of the 100 best \& most influenced non-fiction books written in English since 1923. American wit Dorothy Parker said, regarding the book:
\begin{quote}
	``If you have any young friends who aspire to become writers, the 2nd-greatest favor you can do them is to present them with copies of {\it The Elements of Style}. The 1st greatest, of course, is to shoot them now, while they're happy.'' -- \href{https://en.wikipedia.org/wiki/The_Elements_of_Style}{Wikipedia{\tt/}Elements of Style}
\end{quote}

\subsubsection{Content}
See \href{https://en.wikipedia.org/wiki/The_Elements_of_Style}{Wikipedia{\tt/}The Elements of Style}. ``Strunk concentrated on the cultivation of good writing \& composition; the original 1918 edition exhorted writers to ``omit needless words'', use the \href{https://en.wikipedia.org/wiki/Active_voice}{active voice}, \& employ \href{https://en.wikipedia.org/wiki/Parallelism_(grammar)}{parallelism} appropriately.'' [$\ldots$] ``The 3rd edition of {\it The Elements of Style} (1979) features 54 points: a list of common word-usage errors; 11 rules of punctuation \& grammar; 11 principles of writing; 11 matters of form; \&, in Chap. V, 21 reminders for better style. The final reminder, the 21st, ``Prefer the standard to the offbeat\footnote{{\bf offbeat} [a] [usually before noun] ({\it informal}) different from what most people expect, \textsc{synonym}: {\bf unconventional}.}'', is thematically integral\footnote{{\bf integral} [a] {\bf 1.} being an essential part of something; {\bf 2.} [usually before noun] included as part of something, rather than supplied separately; {\bf 3.} [usually before noun] having all the parts that are necessary for something to be complete.} to the subject of {\it The Elements of Style}, yet does stand as a discrete\footnote{{\bf discrete} [a] ({\it formal or specialist}) independent of other things of the same type, \textsc{synonym}: {\bf separate}.} essay about writing lucid\footnote{{\bf lucid} [a] {\bf 1.} clearly expressed; easy to understand, \textsc{synonym}: clear; {\bf 2.} able to think clearly, especially when somebody cannot usually do this.} prose\footnote{{\bf prose} [n] [uncountable] writing that is not poetry.}. To write well, White advises writers to have the proper\footnote{{\bf proper} [a] {\bf 1.} [only before noun] ({\it especially British English}) right, appropriate or correct; according to the rules, \textsc{opposite}: {\bf improper}; {\bf 2.} [only before noun] {\it British English}) considered to be real \& of a good enough standard; {\bf 3.} socially \& morally acceptable, \textsc{opposite}: {\bf improper}; {\bf 4.} [after noun] according to the most exact meaning of the word; {\bf 5. proper to somebody{\tt/}something} belonging to a particular type of person or thing; natural in a particular situation or place.} mind-set, that they write to please themselves, \& that they aim for ``1 moment of felicity\footnote{{\bf felicity} [n] {\bf 1.} [uncountable] great happiness; {\bf 2.} [uncountable] the quality of being well chosen or suitable; {\bf 3. felicities} [plural] well-chosen or successful features, especially in a speech or piece of writing.}'', a phrase by \href{https://en.wikipedia.org/wiki/Robert_Louis_Stevenson}{Robert Louis Stevenson}. Thus Strunk's 1918 recommendation:
\begin{quotation}
	``Vigorous\footnote{{\bf vigorous} [a] {\bf 1.} involving physical strength, effort or energy; {\bf 2.} done with determination, energy or enthusiasm; {\bf 3.} strong \& healthy.} writing is concise\footnote{{\bf concise} [a] giving only the information that is necessary \& important, using few words.}. A sentence should contain no unnecessary words, a paragraph no unnecessary sentences, for the same reason that a drawing should have no unnecessary lines \& a machine no unnecessary parts. This requires not that the writer make all his sentences short, or that he avoid all detail \& treat his subjects only in outline, but that he make every word tell.'' -- ``Elementary Principles of Composition'', {\it The Element of Style} \cite{Strunk1918}''
\end{quotation}
[$\ldots$] ``The 4th edition of {\it The Elements of Style} (2000), published 54 years after Strunk's death, omits his stylistic\footnote{{\bf stylistic} [a] [only before noun] connected with the style that a writer, artist or musician uses.} advice about masculine\footnote{{\bf masculine} [a] {\bf 1.} having the qualities or appearance considered to be typical of men; connected with or like men; {\bf 2.} (in some languages) belonging to a class of nouns, pronouns or adjectives that have masculine gender, not feminine or neuter.} pronouns: ``unless the antecedent\footnote{{\bf antecedent} [n] a thing or an event that exists or comes before something else \& has an influence on it; [a] existing or coming before something else, \& having an influence on it.} is or must be feminine''. In its place, the following sentence has been added: ``many writers find the use of the generic {\it he} or {\it his} to rename indefinite antecedents limiting or offensive.'' Further, the retitled entry ``They. He or she'', in Chap. IV: {\it Misused Words \& Expressions}, advises the writer to avoid an ``unintentional emphasis on the masculine''.'' -- \href{https://en.wikipedia.org/wiki/The_Elements_of_Style#Content}{Wikipedia{\tt/}The Element of Style{\tt/}content}

\subsubsection{Reception}
``{\it The Elements of Style} was listed as 1 of the 100 best \& most influential\footnote{{\bf influential} [a] having a lot of influence on the way that somebody{\tt/}something behaves or develops, or on the way that somebody thinks.} books written in English since 1923 by {\it Time} in its 2011 list. Upon its release, Charles Poor, writing for \href{https://en.wikipedia.org/wiki/The_New_York_Times}{{\it The New York Times}}, called it ``a splendid\footnote{{\bf splendid} [a] ({\it especially British English}) {\bf 1.} very impressive; very beautiful; {\bf 2.} ({\it old-fashioned}) excellent; very good, \textsc{synonym}: great.} trophy for all who are interested in reading \& writing.'' American poet \href{https://en.wikipedia.org/wiki/Dorothy_Parker}{Dorothy Parker} has, regarding the book, said:
\begin{quotation}
	``If you have any young friends who aspire to become writers, the 2nd-greatest favor you can do them is to present them with copies of {\it The Elements of Style}. The 1st-greatest, of course, is to shoot them now, while they're happy.''
\end{quotation}
Criticism\footnote{{\bf criticism} [n] {\bf 1.} [uncountable, countable] the act of expressing disapproval of somebody{\tt/}something \& opinions about their faults or bad qualities; a statement showing disapproval; {\bf 2.} [uncountable] the work or activity of analyzing \& making fair, careful judgments about somebody{\tt/}something, especially books, music, etc.} of {\it Strunk \& White} has largely focused on claims that it has a \href{https://en.wikipedia.org/wiki/Linguistic_prescriptivism}{prescriptivist}\footnote{{\bf prescriptive} [a] {\bf 1.} telling people what should be done or how something should be done; {\bf 2.} ({\it linguistics}) telling people how a language should be used, rather than describing how it is used, \textsc{opposite}: {\bf descriptive}.} nature, or that it has become a general \href{https://en.wikipedia.org/wiki/Anachronism}{anachronism}\footnote{{\bf anachronism} [n] {\bf 1.} [countable] a person, a custom or an idea that seems old-fashioned \& does not belong to the present; {\bf 2.} [countable, uncountable] something that is placed, e.g., in a book or play, in the wrong period of history; the fact of placing something in the wrong period of history.} in the face of modern English usage.

In criticizing {\it The Elements of Style}, \href{https://en.wikipedia.org/wiki/Geoffrey_Pullum}{Geoffrey Pullum}, professor of \href{https://en.wikipedia.org/wiki/Linguistics}{linguistics} at the \href{https://en.wikipedia.org/wiki/University_of_Edinburgh}{University of Edinburgh}, \& co-author of \href{https://en.wikipedia.org/wiki/The_Cambridge_Grammar_of_the_English_Language}{{\it The Cambridge Grammar of the English Language}} (2002), said that:
\begin{quotation}
	``The book's toxic mix of \href{https://en.wikipedia.org/wiki/Linguistic_purism}{purism}\footnote{{\bf purism} [n] [uncountable] the belief that things should be done in the traditional way \& that there are correct forms in languages, art, etc. that should be followed.}, \href{https://en.wikipedia.org/wiki/Atavism}{atavism}, \& personal \href{https://en.wikipedia.org/wiki/Eccentricity_(behavior)}{eccentricity}\footnote{{\bf eccentricity} [n] {\bf 1.} [uncountable] behavior that people think is strange or unusual; the quality of being unusual \& different from other people; {\bf 2.} [countable, usually plural] an unusual act or habit.} is not underpinned\footnote{{\bf underpin} [v] to support or form the basis of something.} by a proper grounding\footnote{{\bf grounding} [n] [singular, uncountable] knowledge \& understanding of the basic parts of a subject; a basis for something.} in English grammar. It is often so misguided that the authors appear not to notice their own egregious\footnote{{\bf egregious} [a] ({\it formal}) extremely bad.} flouting\footnote{{\bf flout} [v] {\bf flout something} to show that you have no respect for a law, etc. by openly not obeying it, \textsc{synonym}: {\bf defy}.} of its own rules $\ldots$ It's sad. Several generations of college students learned their grammar from the uninformed\footnote{{\bf uninformed} [a] having or showing a lack of knowledge or information about something, \textsc{opposite}: informed.} bossiness\footnote{{\bf bossiness} [n] [uncountable] ({\it disapproving}) bossy behavior.} of {\it Strunk \& White}, \& the result is a nation of educated people who know they feel vaguely\footnote{{\bf vaguely} [adv] {\bf 1.} in a way that is not detailed or exact; {\bf 2.} slightly.} anxious\footnote{{\bf anxious} [a] {\bf 1. anxious (about something)} feeling worried or nervous; {\bf 2.} wanting something very much.} \& insecure\footnote{{\bf insecure} [a] {\bf 1.} not confident, especially about yourself or your abilities, \textsc{opposite}: {\bf secure}; {\bf 2.} not safe or protected, \textsc{opposite}: {\bf secure}.} whenever they write {\it however} or {\it than me} or {\it was} or {\it which}, but can't tell you why.''
\end{quotation}
Pullum has argued, e.g., that the authors misunderstood what constitutes the \href{https://en.wikipedia.org/wiki/English_passive_voice}{passive voice}\footnote{NQBH: Personally, I prefer the passive voice to the active one.}, \& he criticized their proscription\footnote{{\bf proscription} [n] [countable, uncountable] ({\it formal}) {\bf proscription (against{\tt/}on something)} the act of saying officially that something is banned; the stat of being banned.} of established \& unproblematic\footnote{{\bf unproblematic} [a] not having or causing problems, \textsc{opposite}: {\bf problematic}.} English usages, e.g. the \href{https://en.wikipedia.org/wiki/Split_infinitive}{split infinitive} \& the use of {\it which} in a restrictive \href{https://en.wikipedia.org/wiki/English_relative_clause#That_or_which}{relative clause}. On \href{https://en.wikipedia.org/wiki/Language_Log}{Language Log}, a blog about language written by \href{https://en.wikipedia.org/wiki/Linguists}{linguists}, he further criticized {\it The Elements of Style} for promoting \href{https://en.wikipedia.org/wiki/Linguistic_prescriptivism}{linguistic precriptivism} \& \href{https://en.wikipedia.org/wiki/Hypercorrection}{hypercorrection} among \href{https://en.wikipedia.org/wiki/Anglophones}{Anglophones}, \& called it ``the book that ate American's brain''.

\href{https://en.wikipedia.org/wiki/The_Boston_Globe}{{\it The Boston Globe}}'s review described {\it The Elements of Style Illustrated} (2005), with illustrations by Maira Kalman, as an ``aging zombie of a book $\ldots$ a hodgepodge\footnote{{\bf hodgepodge} [n] ({\it North American English}) (also {\bf hotchpotch}, {\it especially in British English}) [singular] ({\it informal}) a number of things mixed together without any particular order or reason.}, its now-antiquated\footnote{{\bf antiquated} [a] ({\it usually disapproving}) (of things or ideas) old-fashioned \& no longer suitable for modern conditions, \textsc{synonym}: {\bf outdated}.} \href{https://en.wikipedia.org/wiki/Pet_peeve}{pet peeves} jostling for\footnote{{\bf jostle for} [phrasal verb] {\bf jostle for something} to compete strongly \& with force for something.} space with 1970s taboos\footnote{{\bf taboo} [n] {\bf 1. taboo (against{\tt/}on something)} a cultural or religious custom that does not allow people to do, use or talk about a particular thing; {\bf 2. taboo (against{\tt/}on something)} a general agreement not to do something or talk about something.} \& 1990s computer advice''.

Nevertheless, many contemporary\footnote{{\bf contemporary} [a] {\bf 1.} belonging to the present time, \textsc{synonym} {\bf modern}; {\bf 2.} (especially of people \& society) belonging to the same time as somebody{\tt/}something else.} authors still recommend it highly. Their praise\footnote{{\bf praise} [v] {\bf 1.} to express your approval or admiration for somebody{\tt/}something; {\bf 2. praise God} to express your thanks to or your respect for God.} tends to focus on its characterization\footnote{{\bf characterization} [n] [uncountable, countable] {\bf 1. characterization (of something)} the process of discovering or describing the qualities or features of something; the result of this process; {\bf 2.} the way in which the characters in a story, play or film are made to seem real.} of \fbox{good writing \& how to achieve it}, grammar being just 1 element of that purpose. In \href{https://en.wikipedia.org/wiki/On_Writing:_A_Memoir_of_the_Craft}{On writing} (2000, p. 11), \href{https://en.wikipedia.org/wiki/Stephen_King}{Stephen King} writes:
\begin{quotation}
	``There is little or no detectable \href{https://en.wikipedia.org/wiki/Bullshit}{bullshit} in that book. (Of course, it's short; at 85 pages it's much shorter than this one.) I'll tell you right now that every aspiring writer should read {\it The Elements of Style}. Rule 17 in the chapter titled {\it Principles of Composition} is `Omit needless words.' I will try to do that here.''
\end{quotation}
In 2011, Tim Skern remarked that {\it The Elements of Style} ``remains the best book available on writing good English.''

In 2013, \href{https://en.wikipedia.org/wiki/Nevile_Gwynne}{Nevile Gwynne} reproduced {\it The Elements of Style} in his work \href{https://en.wikipedia.org/wiki/Gwynne%27s_Grammar}{{\it Gwynne's Grammar}}. Britt Peterson of the \href{https://en.wikipedia.org/wiki/Boston_Globe}{{\it Boston Globe}} wrote that his inclusion of the book was a ``curious\footnote{{\bf curious} [a] {\bf 1.} having a strong desire to know about something; {\bf 2.} strange \& unusual.} addition''.

In 2016, the Open Syllabus Project lists {\it The Elements of Style} as the most frequently assigned text in US academic \href{https://en.wikipedia.org/wiki/Syllabus}{syllabuses}, based on an analysis of 933,635 texts appearing in over 1 million syllabuses.'' -- \href{https://en.wikipedia.org/wiki/The_Elements_of_Style#Reception}{Wikipedia{\tt/}The Elements of Style{\tt/}reception}

``The 1st writer I watched at work was my stepfather, E. B. White.\footnote{Sự ảnh hưởng, đặc biệt đến nhân cách \& việc lựa chọn nghề nghiệp, của những hình mẫu đầu tiên mà ta, 1 cách tình cờ hay được số phận sắp đặt, gặp gỡ trong cuộc đời.} Each Tuesday morning, he would close his study door \& sit down to write the ``Notes \& Comment'' page for {\it The New Yorker}. The task was familiar to him -- he was required to file a few hundred words of editorial\footnote{{\bf editorial} [a] [usually before noun] connected with the task of preparing something e.g. a newspaper, a book, or a television or radio programme, to be published or broadcast; [n] an important article in a journal or a newspaper, that expresses the editor's opinion about an issue.} of personal commentary on some topic in or out of the news that week -- but the sounds of his typewriter\footnote{{\bf typewriter} [n] a machine that produces writing similar to print. It has keys that you press to make metal letters or signs hit a piece of paper through a long, narrow piece of cloth covered with ink ($=$ colored liquid).} \footnote{NQBH: I like the term ``typewriter'' in any literary scene., which sounds traditional \& sexy, opposite to personal notebooks{\tt/}laptop now: modern \& robust.} from his room came in hesitant\footnote{{\bf hesitant} [a] slow to speak or act because you feel uncertain, embarrassed or unwilling.} bursts\footnote{{\bf burst} [v] {\bf 1.} [intransitive, transitive] to break open or apart, especially because of pressure from inside; to make something break in this way; {\bf 2.} [intransitive] {\bf $+$ adv.{\tt/}prep.} to go or come from somewhere suddenly; {\bf burst into something} [phrasal verb] to start producing something suddenly \& with great force; [n] a short period of a particular activity or strong emotion that often starts suddenly.}, with long silences in between. Hours went by. Summoned at last for lunch, he was silent \& preoccupied\footnote{{\bf preoccupied} [a] thinking \&{\tt/}or worrying continuously about something so that you do not pay attention to other things.}, \& soon excused himself to get back to the job. When the copy went off at last, in the afternoon RFD pouch\footnote{{\bf pouch} [n] {\bf 1.} a small bag, usually made of leather, \& often carried in a pocket or attached to a belt; {\bf 2.} a large bag for carrying letters, especially official ones; {\bf 3.} a pocket of skin on the stomach of some female marsupial animals, e.g. kangaroos, in which they carry their young; {\bf 4.} a pocket of skin in the cheeks of some animals, e.g. hamsters, in which they store food.} -- we were in Maine, a day's mail away from New York -- he rarely seemed satisfied. \fbox{``It isn't good enough.''}\footnote{``The quest for perfection can never end.''} he said sometimes, \fbox{``I wish it were better.''}

\fbox{Writing is hard}, even for authors who do it all the time. Less frequent practitioners -- the job applicant; the business executive with an annual report to get out; the high school senior with a Faulkner assignment; the graduate-school student with her thesis proposal; the writer of a letter of condolence\footnote{{\bf condolence} [n] [countable, usually plural, uncountable] sympathy that you feel for somebody when a person in their family or that they know well has died; an expression of this sympathy.} -- often get stuck in an awkward\footnote{{\bf awkward} [a] {\bf 1.} embarrassed; making you feel embarrassed; {\bf 2.} difficult to deal with, \textsc{synonym}: {\bf difficult}; {\bf 3.} not convenient; {\bf 4.} difficult because of its shape or design; {\bf 5.} not moving in an easy way; not comfortable or elegant.} passage or find a muddle\footnote{{\bf muddle} [v] ({\it especially British English}) {\bf 1.} to put things in the wrong order or mix them up; {\bf 2.} muddle somebody (up) to confuse somebody; {\bf 3.} muddle somebody{\tt/}something (up)$|$ {\bf muddle A (up) with B} to confuse 1 person or thing with another, \textsc{synonym}: {\bf mix up}.} on their screens, \& then blame themselves. What should be easy \& flowing looks tangled\footnote{{\bf tangled} [a] {\bf 1.} twisted together in an untidy way; {\bf 2.} complicated, \& not easy to understand.} or feeble\footnote{{\bf feeble} [a] {\bf 1.} very weak; {\bf 2.} not effective; not showing energy or effort.} or overblown\footnote{{\bf overblown} [a] {\bf 1.} that is made to seem larger, more impressive or more important than it really is, \textsc{synonym}: {\bf exaggerated}; {\bf 2.} (of flowers) past the best, most beautiful stage.} -- not what was meant at all. \fbox{What's wrong with me}, each one thinks. \fbox{Why can't I get this right?}''

[$\ldots$] White knew that a compendium\footnote{{\bf compendium} [n] (plural {\bf compendia, compendiums}) a collection of facts, drawings \& photographs on a particular subject, especially in a book.} of specific tips -- about singular \& plural verbs, parentheses, the ``that'' -- ``which'' scuffle\footnote{{\bf scuffle} [n] {\bf scuffle (with somebody) $|$ scuffle (between A \& B)} a short \& not very violent fight or struggle; [v] {\bf 1.} [intransitive] {\bf scuffle (with somebody)} (of 2 or more people) to fight or struggle with each other for a short time, in a way that is not very serious; {\bf 2.} [intransitive] {\bf $+$ adv.{\tt/}prep.} to move quickly making a quiet rubbing noise.}, \& many others -- could clear up a recalcitrant\footnote{{\bf recalcitrant} [a] ({\it formal}) unwilling to obey rules or follow instructions; difficult to control.} sentence or subclause when quickly reconsulted\footnote{{\bf consult} [v] {\bf 1.} [transitive, intransitive] to discuss something with somebody to get their permission for something, or to help you make a decision; {\bf 2.} [transitive, intransitive] to go to somebody for information or advice, especially an expert e.g. a doctor or lawyer; {\bf 3.} [transitive] {\bf consult something} to look in or at something to get information, \textsc{synonym}: {\bf refer to something}.}, \& that the larger principles needed to be kept in plain sight, like a wall sampler.

How simple they look, set down here in White's last chapter: ``\fbox{Write in a way that comes naturally},'' ``\fbox{Revise \& rewrite},'' ``\fbox{Do not explain too much},'' \& the rest; above all, the cleansing\footnote{{\bf cleanse} [v] {\bf 1.} [transitive, intransitive] {\bf cleanse (something)} to clean your skin or a wound; {\bf 2.} [transitive] {\bf cleanse somebody (of{\tt/}from something}) ({\it literary}) to take away somebody's guilty feelings or sin.}, clarion\footnote{{\bf clarion} [n] {\bf 1.} a medieval trumpet with clear shrill tones; {\bf 2.} the sound of or as if of a clarion' [a] brilliantly clear; loud \& clear.} ``Be clear.'' How often I have turned to them, in the book or in my mind, while trying to start or unblock or revise some piece of my own writing! They help -- they really do. They work. They are the way.

E. B. White's prose is celebrated for its ease\footnote{{\bf ease} [n] [uncountable] {\bf 1.} lack of difficulty or effort, \textsc{opposite}: {\bf difficulty}; {\bf 2.} the state of feeling relaxed or comfortable, without anxiety, problems or pain.} \& clarity\footnote{{\bf clarity} [n] [uncountable] {\bf 1.} the quality of being expressed clearly; {\bf 2.} the ability to think about or understand something clearly; {\bf 3.} if a picture, substance or sound has clarity, you can see or hear it very clearly, or see through it easily.} -- just think of {\it Charlotte's Web} -- but maintaining this standard required endless attention. When the new issue of {\it The New Yorker} turned up in Maine, I sometimes saw him reading his ``Comment'' piece over to himself, with only a slightly different expression than the one he'd worn on the day it went off. Well, O.K., he seemed to be saying. \fbox{At least I got the elements right.}

This edition has been modestly\footnote{{\bf modest} [a] {\bf 1.} fairly limited or small in amout; {\bf 2.} not expensive, rich or impressive; {\bf 3.} (of people, especially women, or their clothes) not showing too much of the body; not intended to attract attention, especially in a sexual way; {\bf 4.} ({\it approving}) not talking much about your own abilities or possessions.} updated, with word processors \& air conditioners making their 1st appearance among White's references, \& with a light redistribution of genders to permit a feminine pronoun or female farmer to take their places among the males who once innocently\footnote{{\bf innocent} [a] {\bf 1.} not guilty of a crime, etc.; not having done something wrong, \textsc{opposite}: {\bf guilty}; {\bf 2.} [only before noun] suffering harm or being killed because of a crime, war, etc. although not directly involved in it; {\bf 3.} having little experience of evil or unpleasant things, or of sexual matters; {\bf 4.} not intended to cause harm or upset somebody, \textsc{synonym}: {\bf harmless}.} served him.'' [$\ldots$] ``What is not here is anything about E-mail -- the rules-free, lower-case flow that cheerfully keeps us in touch these days. E-mail is conversation, \& it may be replacing the sweet \& endless talking we once sustained\footnote{{\bf sustain} [v] {\bf 1. sustain somebody{\tt/}something} to provide enough of what somebody{\tt/}something needs in order to live or exist; {\bf 2.} to make something continue for some time without becoming less, \textsc{synonym}: {\bf maintain}; {\bf 3. sustain something} ({\it formal}) to experience something bad, \textsc{synonym}: {\bf suffer}; {\bf 4. sustain something} to provide evidence to support an opinion, a theory, etc., \textsc{synonym}: {\bf uphold}; {\bf 5. sustain something} ({\it law}) to decide that a claim, etc. is valid, \textsc{synonym}: {\bf uphold}.} (\& tucked away\footnote{{\bf tuck away} [phrasal verb] {\bf tuck something $\leftrightarrow$ away} {\bf 1. be tucked away} to be located in a quiet place, where not many people go; {\bf 2.} to hide something somewhere or keep it in a safe place; {\bf 3.} ({\it British English, informal}) to eat a lot of food.}) within the informal letter. But we are all writers \& readers as well as communicators, with \fbox{the need at times to please \& satisfy ourselves} (as White put it) with the \fbox{clear \& almost perfect thought}.'' -- \cite[{\it Foreword} by Roger Angell]{Strunk_White2019}

``I [E. B. White] passed the course, graduated from the university, \& \fbox{forgot the book but not the professor}.'' [$\ldots$]

``{\it The Elements of Style}, when I [E. B. White] reexamined it in 1957, seemed to me to contain \fbox{rich deposits\footnote{{\bf deposit} [n] {\bf 1.} a layer of a substance that has been left somewhere, especially by a river or flood, or is found at the bottom of a liquid; {\bf 2.} a layer of a substance that has formed naturally underground; {\bf 3.} [usually singular] {\bf a deposit (on something)} a sum of money that is given as the 1st part of a larger payment; {\bf 4.} (in the British political system) the amount of money that a candidate in an election to Parliament has to pay, \& that is returned if they get enough votes.} of gold}. It was Will Strunk's {\it parvum opus}\footnote{{\bf parvum opus} [from Latin] [n] a little work, a small but meaningful work of an artist or writer.}, his attempt to cut the vast tangle\footnote{{\bf tangle} [n] {\bf 1.} a twisted mass of threads, hair, etc. that cannot be easily separated; {\bf 2.} a lack of order; a confused state; {\bf 3.} ({\it informal}) a disagreement or fight; [v] [transitive, intransitive] {\bf tangle (something) up} to twist something into an untidy mass; to become twisted in this way.} of English rhetoric\footnote{{\bf rhetoric} [n] [uncountable] {\bf 1.} ({\it often disapproving} speech or writing that is intended to influence people, but that is not completely honest or sincere; {\bf 2.} the skill of using language in speech or writing in a special way that influences or entertains people.)} down to size \& write its rules \& principles on the head of a pin\footnote{{\bf pin} [n] {\bf 1.} a short thin piece of stiff wire with a sharp point at 1 end \& a round head at the other, used to hold or attach things; {\bf 2.} a short piece of metal or other material, used to hold things together; {\bf 3.} a piece of metal with a sharp point, worn for decoration; {\bf 4.} 1 of the metal parts that stick out of an electric plug \& fit into a socket; [v] {\bf pin something ($+$ adv.{\tt/}prep.)} to attach something onto another thing or join things together with a pin, etc.; {\bf pin something down} [phrasal verb] to explain or understand something exactly.}. Will himself had hung the tag ``little'' on the book; he referred to it sardonically\footnote{{\bf sardonically} [adv] ({\it disapproving}) in a way that shows that you think that you are better than other people \& do not take them seriously, \textsc{synonym}: {\bf mockingly}.} \& with secret pride as ``the {\it little book},'' always giving the word ``little'' a special twist, as though he were putting a spin on a ball. In its original form, it was a 43 page summation of the case for cleanliness, accuracy\footnote{{\bf accuracy} [n] {\bf 1.} [uncountable] the state of being exact or correct, \textsc{opposite}: {\bf inaccuracy}; {\bf 2.} [uncountable, countable] ({\it specialist}) the degree to which the result of a measurement or calculation matches the correct value or a standard, \textsc{opposite}: {\bf inaccuracy}.}, \& brevity\footnote{{\bf brevity} [n] [uncountable] {\bf 1.} the quality of using few words when speaking or writing; {\bf 2. brevity (of something)} the fact of lasting a short time.} in the use of English. Today, 52 years later, its vigor\footnote{{\bf vigor} [n] [uncountable] {\bf 1.} effort, energy, \& enthusiasm; {\bf 2. vigor (of something)} physical strength; good health.} is unimpaired\footnote{{\bf unimpaired} [a] ({\it formal}) not damaged or made less good, \textsc{opposite}: {\bf impaired}.}, \& for sheer\footnote{{\bf sheer} [a] {\bf 1.} [only before noun] used to emphasize the size, degree or amount of something; nothing but; {\bf 2.} very steep.} pith\footnote{{\bf pith} [n] [uncountable] {\bf 1.} a soft dry white substance inside the skin of oranges \& some other fruits; {\bf 2.} the essential or most important part of something.} I think it probably sets a record that is not likely to be broken. Even after I got through tampering with\footnote{{\bf tamper with} [phrasal verb] {\bf tamper with something} to make changes to something without permission, especially in order to damage it, \textsc{synonym}: interfere with.} it, it was still a tiny thing, \fbox{a barely tarnished\footnote{{\bf tarnished} [v] {\bf 1.} [intransitive, transitive] if mental tarnishes or something tarnishes it, it no longer looks bright \& shiny; {\bf 2.} [transitive, often passive] to damage the good opinion people have of somebody{\tt/}something, \textsc{synonym}: {\bf taint}; [n] [singular, uncountable] a thin layer on the surface of a metal that makes it look darker \& less bright.} gem\footnote{{\bf gem} [n] {\bf 1.} (also less frequent {\bf gemstone}) a precious stone that has been cut \& polished \& is used in jewellery, \textsc{synonym}: {\bf jewel, precious stone}; {\bf 2.} a person, place or thing that is especially good.}}. 7 rules of usage, 11 principles of composition\footnote{{\bf composition} [n] {\bf 1.} [uncountable] the different parts that something is made of; the way in which the different parts are organized; {\bf 2.} [countable] a piece of music or a poem; {\bf 3.} [uncountable] the act of writing a piece of music or a poem; {\bf 4.} [uncountable] ({\it art}) the arrangement of people of objects in a painting, photograph or scene of a film.}, a few matters of form, \& a list of words \& expressions commonly misused -- that was the sum \& substance\footnote{{\bf substance} [n] {\bf 1.} a type of solid, liquid or gas that has particular qualities; {\bf 2.} [countable] a drug or chemical, especially an illegal one, that has a particular effect on the mind or body; {\bf 3.} [uncountable] the most important or main part of something; {\bf 4.} [uncountable] ({\it formal}) importance; {\bf 5.} [uncountable] the quality of being based on facts or the truth.} of Prof. Strunk's work. Somewhat audaciously\footnote{{\bf audaciously} [adv] ({\it formal}) in a way that shows you are willing to take risks or to do something that shocks people.}, \& in an attempt to give my publisher his money's worth, I [E. B. White] added a chapter called ``An Approach to Style,'' setting forth my own prejudices\footnote{{\bf prejudice} [n] [uncountable, countable] an unreasonable dislike of a person, group, etc., especially when it is based on their race, religion, sex, etc.}, my notions of error, my articles of faith. This chapter (Chap. V) is addressed particularly to those who feel that English prose composition is not only a necessary skill but a sensible pursuit as well -- a way to spend one's days. I think Prof. Strunk would not object to that.''

[$\ldots$] ``I have now completed a 3rd revision. Chap. IV has been refurbished\footnote{{\bf refurbish} [v] {\bf refurbish something} to clean \& decorate a room, building, etc. in order to make it more attractive, more useful, etc.} with words \& expressions of a recent vintage\footnote{{\bf vintage} [n] {\bf 1.} the wine that was produced in a particular year or place; the year in which it was produced; {\bf 2.} [usually singular] the period or season of gathering grapes for making wine; [a] [only before noun] {\bf 1. vintage} wine is of very good quality \& has been stored for several years; {\bf 2.} (British English) (of a vehicle) made between 1919 \& 1930 \& admired for its style \& interest; {\bf 3.} typical of a period in the past \& of high quality; the best work of the particular person; {\bf 4. vintage year} a particular good \& successful year.}; 4 rules of usage have been added to Chap. I. Fresh examples have been added to some of the rules \& principles, amplification\footnote{{\bf amplification} [n] [uncountable] {\bf 1. amplification (of something)} the process of increasing the amplitude of an electrical signal; {\bf 2.} (biochemistry) {\bf amplification (of something)} the process by which many copies of something, e.g. a gene, are made; {\bf 3. amplification (of something)} the action of making something greater or easier to notice; {\bf 4.} the action of adding details to a story, statement, etc.; details added to a story, statement, etc.} has reared\footnote{{\bf rear} [v] {\bf 1. rear somebody{\tt/}something} [often passive] to care for young children or animals until they are fully grown, \textsc{synonym}: {\bf raise}; {\bf 2. rear something} to breed or keep animals or birds, e.g. on a farm; {\bf something rears its head} [idiom] (of something unpleasant) to appear or happen; [n] (usually {\bf the rear}) [singular] the back part of something; [a] [only before noun] at or near the back of something.} its head in a few places in the text where I felt an assault\footnote{{\bf assault} [n] {\bf 1.} [uncountable, countable] the crime of attacking somebody physically; in law, {\bf assault} is an act that threatens physical harm to somebody, whether or not actual harm is done: {\it to commit}{\tt/}{\it be charged with assault}; {\bf 2.} [countable] (by an army, etc.) the act of attacking somebody{\tt/}something, \textsc{synonym}: {\bf attack}; {\bf 3.} [countable, usually singular, uncountable] an act of criticizing or attacking somebody{\tt/}something severely; [v] {\bf assault somebody} to attack somebody physically.} could successfully be made on the bastions\footnote{{\bf bastion} [n] {\bf 1.} ({\it formal}) a group of people or a system that protects a way of life or a belief when it seems that it may disappear; {\bf 2.} a place that military forces are defending.} of its brevity, \& in general the book has received a thorough overhaul\footnote{{\bf overhaul} [n] an examination of a machine or system, including doing repairs on it or making changes to it; [v] {\bf 1. overhaul something} to examine every part of a machine, system, etc. \& make any necessary changes or repairs; {\bf 2. overhaul somebody} to come from behind a person you are competing against in a race \& go past them, \textsc{synonym}: {\bf overtake}.} -- to correct errors, delete bewhiskered\footnote{{\bf bewhiskered} [a] {\bf 1.} having whiskers; bearded; {\bf 2.} ancient, as a witticism, expression, etc.; pass\'e; hoary.} entries, \& enliven\footnote{{\bf enliven} [v] ({\it formal}) {\bf enliven something} to make something more interesting or more fun.} the argument.

Prof. Strunk was a positive man. His book contains rules of grammar phrased as direct orders. In the main I [E. B. White] have not tried to soften his commands, or modify his pronouncements\footnote{{\bf pronouncement} [n] a formal public statement.}, or remove the special objects of his scorn\footnote{{\bf scorn} [n] [uncountable] a strong feeling that somebody{\tt/}something is stupid or not good enough, usually shown by the way you speak, \textsc{synonym}: {\bf contempt}; [v] {\bf 1. scorn somebody{\tt/}something} to feel or show that you think somebody{\tt/}something is stupid \& you do not respect them or it, \textsc{synonym}: {\bf dismiss}; {\bf 2.} ({\it formal}) to refuse to have or do something because you are too proud.}. I have tried, instead, to preserve\footnote{{\bf preserve} [v] {\bf 1. preserve something} to keep a particular quality or feature; {\bf 2.} to keep something safe from harm, in good condition or in its original state; {\bf 3.} to prevent something from decaying, by treating it in a particular way; [n] [singular] an activity, job or interest that is thought to be suitable for 1 particular person or group of people.} the flavor\footnote{{\bf flavor} [n] {\bf 1.} [uncountable] {\bf flavor (of something)} how food or drink tastes, \textsc{synonym}: {\bf taste}; {\bf 2.} [countable] a particular type of taste; {\bf 3.} [singular] a particular quality or atmosphere; {\bf 4.} [singular] {\bf a{\tt/}the flavor of something} an idea of what something is like.} of his discontent\footnote{{\bf discontent} [n] (also {\bf discontentment}) {\bf 1.} [uncountable] a feeling of being unhappy because you are not satisfied with a particular situation, \textsc{synonym}: {\bf dissatisfaction}; {\bf 2.} [countable] {\bf discontent (of somebody)} a thing that makes you feel unhappy \& not satisfied with a particular situation, \textsc{synonym}: {\bf dissatisfaction}.} while slightly enlarging the scope of the discussion. {\it The Elements of Style} does not pretend\footnote{{\bf pretend} [v] {\bf 1.} to behave in a particular way, in order to make other people believe something that is not true; {\bf 2.} (usually used in negative sentences \& questions) to claim to be, do or have something, especially when this is not true.} to survey\footnote{{\bf survey} [n] {\bf 1. survey} (of somebody{\tt/}something) an investigation of the opinions, behavior, etc. of a particular group of people, which is usually done by asking them questions; {\bf 2.} an act of examining \& recording the measurements, features, etc. of an area of land in order to make a map or plan of it; {\bf 3. survey (of something)} a general study, view or description of something; [v] {\bf 1. survey somebody{\tt/}something} to investigate the opinions or behavior of a group of people by asking them a series of questions; {\bf 2. survey something} to study \& give a general description of something; {\bf 3. survey something} to measure \& record the features of an area of land, e.g. in order to make a map or in preparation for building; {\bf 4. survey something} to look carefully at the whole of something, especially in order to get a general impression of it, \textsc{synonym}: {\bf inspect}.} the whole field. Rather it proposes\footnote{{\bf propose} [v] {\bf 1.} to suggest a plan or an idea for people to consider \& decide on; {\bf 2.} to suggest an explanation of something for people to consider.} to give in brief space the principal\footnote{{\bf principal} [a] [only before noun] main; most important.} requirements of plain\footnote{{\bf plain} [a] {\bf 1.} easy to see or understand, \textsc{synonym}: {\bf clear}; {\bf 2.} [only before noun] expressed in a clear \& simple way, without using technical language; {\bf 3.} not trying to deceive anyone; honest \& direct; {\bf 4.} not decorated or complicated; simple; in computing, {\bf plain text} is data representing text that is not written in code or using special formatting \& can be read, displayed or printed without much processing: {\it Mathematical formulae are an example of content that cannot be represented satisfactorily via plain text.}; {\bf 5.} without marks or a pattern on it; {\bf 6.} [only before noun] (used for emphasis) simple; nothing but. \textsc{synonym}: {\bf sheer}.} English style. It concentrates\footnote{{\bf concentrate} [v] {\bf 1.} [transitive, often passive] {\bf concentrate something $+$ adv.{\tt/}prep.} to bring something together in 1 place; {\bf 2.} [intransitive, transitive] to give all your attention to something \& not think about anything else; {\bf 3.} [transitive] {\bf concentrate something} to increase the strength of a substance by reducing its volume, e.g. by boiling it; {\bf concentrate on something} [phrasal verb] to spend more time doing 1 particular thing than others; [n] [countable, uncountable] {\bf concentrate (of something)} a substance that is made stronger because water or other substances have been removed.} on fundamentals\footnote{{\bf fundamentals} [n] [plural] {\bf fundamentals (of something)} the basic \& most important parts of something.}: the rules of usage \& principles of composition most commonly violated\footnote{{\bf violet} [v] {\bf 1. violate something} to go against or refuse to obey a law, an agreement, etc.; {\bf 2. violate something} to not treat something with respect.}.

The reader will soon discover that these rules \& principles are in the form of sharp commands, Sergeant\footnote{{\bf sergeant} [n] (abbr., {\bf Sergt, Sgt}) {\bf 1.} a member of 1 of the middle ranks in the army \& the air force, below an officer; {\bf 2.} (in a UK) a police officer just below the rank of an inspector; {\bf 3.} (in the US) a police officer just below the rank of a lieutenant or caption.} Strunk snapping\footnote{{\bf snap} [v] {\it break} {\bf 1.} [transitive, intransitive] to break something suddenly with a sharp noise; to be broken in this way; {\it take photograph} {\bf 2.} [transitive, intransitive] ({\it informal}) to take a photograph; {\it open}{\tt/}{\it close}{\tt/}{\it move into position} {\bf 3.} [intransitive, transitive] to move, or to move something, into a particular position quickly, especially with a sudden sharp noise; {\it speak impatiently} {\bf 4.} [transitive, intransitive] to speak or say something in an impatient, usually angry, voice; {\it of animal} {\bf 5.} [intransitive] {\bf snap (at somebody{\tt/}something)} to try to bite somebody{\tt/}something, \textsc{synonym}: {\bf nip}; {\it lose control} {\bf 6.} [intransitive] to suddenly be unable to control your feelings any longer because the situation has become too difficult; {\it fasten clothing} {\bf 7.} [intransitive, transitive] {\bf snap (something)} ({\it North American English}) to fasten a piece of clothing with a snap; {\it in American football} {\bf 8.} [transitive] {\bf snap something} ({\it sport}) (in American football) to start play by passing the ball back between your legs.} orders to his platoon\footnote{{\bf platoon} [n] a small group of soldiers that is part of a company \& commanded by a lieutenant.}. ``Do not join independent clauses with a comma.'' (Rule 5.) ``Do not break sentences in 2.'' (Rule 6.) ``Use the active voice.'' (Rule 14.) ``Omit\footnote{{\bf omit} [v] {\bf 1.} to not include something{\tt/}somebody, either deliberately or because you have forgotten it{\tt/}them, \textsc{synonym}: {\bf leave somebody{\tt/}something out (of something)}; {\bf 2. omit to do something} to not do or fail to do something.} needless\footnote{{\bf needless} [a] (of something bad) not necessary; that could be avoided, \textsc{synonym}: unnecessary.} words.'' (Rule 17.) ``Avoid a succession\footnote{{\bf succession} [n] {\bf 1.} [countable, usually singular] a number of things or people that follow each other in time or order, \textsc{synonym}: {\bf series}; {\bf 2.} [uncountable] the act of taking over an official position or title; {\bf 3.} [uncountable] the right to take over an official position or title, especially to become the king or queen of a country.} of loose\footnote{{\bf loose} [a] {\bf 1.} not firmly fixed where it should be; that can become separated from something; {\bf 2.} not tightly packed together; not solid or hard; {\bf 3.} not strictly organized or controlled; {\bf 4.} not exact; not very careful; {\bf 5.} (of clothes) not fitting closely, \textsc{opposite}: {\bf tight}; {\bf 6.} not tied together; not held in position by anything or contained in anything; {\bf 7.} ({\it medical}) (of body waste) having too much liquid in it.} sentences.'' (Rule 18.) ``In summaries, keep to 1 tense.'' (Rule 21.) Each rule or principle is followed by a short hortatory\footnote{{\bf hortatory} [a] trying to strongly encourage or persuade someone to do something.} essay, \& usually the exhortation\footnote{{\bf exhortation} [n] [countable, uncountable] ({\it formal}) {\bf exhortation (to do something)} an act of trying very hard to persuade somebody to do something.} is followed by, or interlarded\footnote{{\bf interlard} [v] (used with object) {\bf 1.} to diversify by adding or interjecting something unique, striking, or contrasting (usually followed by {\it with}); {\bf 2.} (of things) to be intermixed in.} with, examples in parallel columns -- the true vs. the false, the right vs. the wrong, the timid\footnote{{\bf timid} [a] shy \& nervous; not brave.} vs. the bold, the ragged\footnote{{\bf ragged} [a] {\bf 1.} (of clothes) old \& torn, \textsc{synonym}: {\bf shabby}; {\bf 2.} (of people) wearing old or torn clothes; {\bf 3.} having an outline, an edge or a surface that is not straight or even; {\bf 4.} not smooth or regular; not showing control or careful preparation; {\bf 5.} ({\it informal}) very tired, especially after physical effort.} vs. the trim\footnote{{\bf trim} [v] {\bf 1. trim something} to make something neater, smaller, better, etc., by cutting parts from it; {\bf 2.} to cut away unnecessary parts from something; {\bf 3.} [usually passive] {\bf trim something (with something)} to decorate something, especially around its edges.}. From every line there peers out at me the puckish\footnote{{\bf puckish} [a] [usually before noun] ({\it literary}) enjoying playing tricks on other people, \textsc{synonym}: {\bf mischievous}.} face of my professor, his short hair parted neatly\footnote{{\bf neat} [a] {\bf 1.} in good order; carefully done or arranged; {\bf 2.} simple but clever; {\bf 3.} containing or made out of just 1 substance; not mixed with anything else.} in the middle \& combed down over his forehead, his eyes blinking incessantly\footnote{{\bf incessantly} [adv] ({\it usually disapproving}) without stopping, \textsc{synonym}: {\bf constantly}.} behind steel-rimmed spectacles\footnote{{\bf spectacle} [n] {\bf 1.} [countable, uncountable] {\bf spectacle (of something)} a performance or an event that is very impressive \& exciting to look at; {\bf 2.} [singular] {\bf spectacle (of something)} an unusual, embarrassing or sad sight or situation that attracts a lot of attention; {\bf 3.} ({\bf spectacles}) [plural] [{\it formal}] $=$ {\bf glass}.} as though he had just emerged into strong light, his lips nibbling each other like nervous horses, his smile shuttling to \& fro under a carefully edged mustache.

``Omit needless words!'' cries the author on p. 23, \& into that imperative\footnote{{\bf imperative} [n] a thing that is very important \& needs immediate attention or action; [a] [not usually before noun] very important \& needing immediate attention or action, \textsc{synonym}: {\bf vital}.} Will Strunk \fbox{really put his heart \& soul}. In the days when I was sitting in his class, he omitted so many needless words, \& omitted them so forcibly\footnote{{\bf forcibly} [adv] {\bf 1.} in a way that involves the use of physical force; {\bf 2.} in a way that makes something very clear.} \& with such eagerness\footnote{{\bf eager} [a] very interested \& excited by something that is going to happen or about something that you want to do, \textsc{synonym}: {\bf keen}.} \& obvious relish\footnote{{\bf relish} [v] to get great pleasure from something; to want very much to do or have something, \textsc{synonym}: {\bf enjoy}; [n] {\bf 1.} [uncountable] great pleasure; {\bf 2.} [uncountable, countable] a cold, thick, spicy sauce made from fruit \& vegetables that have been boiled, that is served with meat, cheese, etc.}, that he often seemed in the position of having shortchanged\footnote{{\bf short-change} [v] [often passive] {\bf 1. short-change somebody} to give back less than the correct amount of money to somebody who has paid for something with more than the exact price; {\bf 2. short-change somebody} to treat somebody unfairly by not giving them what they have earned or deserve.} himself -- a man left with nothing more to say yet with time to fill, a radio prophet who had outdistanced\footnote{{\bf outdistance} [v] {\bf outdistance somebody{\tt/}something} to leave somebody{\tt/}something behind by going faster, further, etc.; to be better than somebody{\tt/}something, \textsc{synonym}: {\bf outstrip}.} the clock. Will Strunk got out of this predicament\footnote{{\bf predicament} [n] a difficult or an unpleasant situation, especially one where it is difficult to know what to do, \textsc{synonym}: {\bf quandary}.} by a simple trick: he uttered\footnote{{\bf utter} [v] {\bf utter something} to make a sound with your voice; to say something.} every sentence 3 times. When he delivered his oration\footnote{{\bf oration} [n] ({\it formal}) a formal speech made on a public occasion, especially as part of a ceremony.} on brevity to the class, he leaned forward over his desk, grasped his coat lapels\footnote{{\bf lapel} [n] 1 of the 2 front parts of the top of a coat or jacket that are joined to the collar \& are folded back.} in his hands, \&, in a husky\footnote{{\bf husky} [a] {\bf 1.} (of a person of their voice) sounding deep, quiet \& rough, sometimes in an attractive way; {\bf 2.} ({\it North American English}) with a large, strong body; [n] (North American English also {\bf huskie}) a large strong dog with thick hair, used for pulling sledges across snow.}, conspiratorial\footnote{{\bf conspiratorial} [a] {\bf 1.} connected with, or making you think of, a conspiracy ($=$ a secret plan to do something illegal); {\bf 2.} (of a person's behavior) suggesting that a secret is being shared.} voice, said, ``Rule 17. Omit needless words! Omit needless words! Omit needless word!''

He was a memorable\footnote{{\bf memorable} [a] special, good or unusual \& therefore worth remembering; easy to remember.} man, friendly \& funny. Under the remembered sting of his kindly lash\footnote{{\bf lash} [v] {\bf 1.} [intransitive, transitive] to hit somebody{\tt/}something with great force, \textsc{synonym}: {\bf pound}; {\bf 2.} [transitive] {\bf lash somebody{\tt/}something} to hit a person or an animal with a whip, rope, stick, etc., \textsc{synonym}: {\bf beat}.}, I have been trying to omit needless words since 1919, \& although there are still many words that cry for omission \& the huge task will never be accomplished, it is exciting to me to reread to masterly Strunkian elaboration\footnote{{\bf elaboration} [n] [uncountable, countable] {\bf 1.} the act of explaining or describing something in a more detailed way; {\bf 2.} the process of developing a plan, an idea, etc. \& making it complicated or detailed; {\bf 3. elaboration (of something)} ({\it biology}) the production of a substance or structure from elements or simpler constituents in a natural process.} of this noble\footnote{{\bf noble} [a] {\bf 1.} belonging to a family of high social rank, \textsc{synonym}: {\bf aristocratic}; {\bf 2.} having or showing fine personal qualities that people admire, e.g. courage, honesty \& care for others; [n] a person who comes from a family of high social rank; a member of the nobility, \textsc{synonym}: {\bf aristocratic}.} theme\footnote{{\bf theme} [n] the subject of a talk, piece of writing, exhibition, etc.; an idea that keeps returning in a piece of research or a work of art or literature.}. It goes:
\begin{quotation}
	{\it Vigorous writing is concise. A sentence should contain no unnecessary words, a paragraph no unnecessary sentences, for the same reason that a drawing should have no unnecessary lines \& a machine no unnecessary parts. This requires not that the writer make all sentences short or avoid all detail \& treat subjects only in outline, but that every word tell.}
\end{quotation}
There you have a short, valuable essay on the nature \& beauty of brevity -- 59 words that could change the world. Having recovered from his adventure in prolixity\footnote{{\bf prolixity} [n] [uncountable] ({\it formal}) the fact of using too many words \& therefore creating a piece of writing, a speech, etc., that is boring.} (59 words were a lot of words in the tight world of William Strunk Jr.), the professor proceeds to give a few quick lessons in pruning\footnote{{\bf pruning} [n] [uncountable] {\bf 1.} the activity of cutting off some of the branches from a tree, bush, etc. so that it will grow better \& stronger; {\bf 2.} the act of making something smaller by removing parts; the act of cutting out parts of something.}. Students learn to cut the dead-wood from ``this is a subject that,'' reducing it to ``this subject,'' a saving of 3 words. They learn to trim\footnote{{\bf trim} [v] {\bf 1. trim something} to make something neater, smaller, better, etc., by cutting parts from it; {\bf 2.} to cut away unnecessary parts from something; {\bf 3.} [usually passive] {\bf trim something (with something)} to decorate something, especially around its edges.} ``used for fuel purposes'' down to ``used for fuel.'' They learn that they are being chatterboxes\footnote{{\bf chatterbox} [n] ({\it informal}) a person who talks a lot, especially a child.} when they say ``the question as to whether'' \& that they should just say ``whether'' -- a saving of 4 words out of a possible 5.

The professor devotes\footnote{{\bf devote} [v] {\bf devote yourself to somebody{\tt/}something} to give most of your time, energy or attention to somebody{\tt/}something, \textsc{synonym}: {\bf dedicate}; {\bf devote something to something}: to give an amount of time, attention or resources to something.} a special paragraph to the vile\footnote{{\bf vile} [a] {\bf 1.} ({\it informal}) extremely unpleasant or bad, \textsc{synonym}: {\bf disgusting}; {\bf 2.} ({\it formal}) morally bad; completely unacceptable, \textsc{synonym}: {\bf wicked}.} expression {\it the fact that}, a phrase that causes him to quiver\footnote{{\bf quiver} [v] to shake slightly; to make a slight movement, \textsc{synonym}: {\bf tremble}; [n] {\bf 1.} an emotion that has an effect on your body; a slight movement in part of your body; {\bf 2.} a case for carrying arrows.} with revulsion\footnote{{\bf revulsion} [n] [uncountable, singular] ({\it formal}) a strong feeling of horror, \textsc{synonym}: {\bf disgust, repugnance}.}. The expression, he says, should be ``revised out of every sentence in which it occurs.'' But a shadow\footnote{{\bf shadow} [n] {\bf 1.} [countable] the dark area or shape produced by somebody{\tt/}something coming between light \& a surface; {\bf 2.} [uncountable] ({\bf shadows} [plural]) darkness, especially that produced by somebody{\tt/}something coming between light \& a surface; {\bf 3.} [singular] the strong (usually bad) influence of somebody{\tt/}something.} of gloom\footnote{{\bf gloom} [n] {\bf 1.} [uncountable, singular] a feeling of being sad \& without hope, \textsc{synonym}: {\bf depression}; {\bf 2.} [uncountable] ({\it literary}) almost total darkness.} seems to hang over the page, \& you feel that he knows how hopeless his cause is. I suppose I have written {\it the fact that} a thousand times in the heat of composition, revised it out maybe 500 times in the cool aftermath\footnote{{\bf aftermath} [n] [usually singular] the situation that exists as a result of an important (\& usually unpleasant) event, especially a war, an accident, etc.}. To be batting only .500 this late in the season, to fail half the time to connect with this fat pitch, saddens me, for it seems a betrayal of the man who showed me how to swing\footnote{{\bf swing} [v] {\bf 1.} [intransitive, transitive] to change to make somebody{\tt/}something change from 1 opinion or mood to another; {\bf 2.} [intransitive, transitive] to turn or change direction suddenly; to make something do this; {\bf 3.} [intransitive, transitive] to move backwards or forwards or from side to side while hanging from a fixed point; to make something do this; {\bf 4.} [intransitive, transitive] to move or make something move with a wide curved movement; [n] a change from 1 opinion or situation to another; the amount by which something changes.} at it \& made the swinging seem worthwhile.

I treasure\footnote{{\bf treasure} [n] {\bf 1.} [uncountable] a collection of valuable things e.g. gold, silver \& jewelery; {\bf 2.} [countable, usually plural] a highly valued object; {\bf 3.} [singular] a person who is much loved or valued; [v] {\bf treasure something} to have or keep something that you love \& that is extremely valuable to you, \textsc{synonym}: {\bf cherish}.} {\it The Elements of Style} for its sharp\footnote{{\bf sharp} [a] {\bf 1.} [usually before noun] (especially of a change in something) sudden \& fast; {\bf 2.} [usually before noun] (especially of a difference in something) clear \& definite; {\bf 3.} (especially of something that can cut or make a hole in something) having a fine edge or point, \textsc{opposite}: {\bf blunt}; {\bf 4.} (of a person or what they say) critical or severe; {\bf 5.} (of a physical feeling or an emotion) very strong \& sudden, often like being cut or wounded, \textsc{synonym}: {\bf intense}; {\bf 6.} changing direction suddenly; {\bf 7.} (of people or their minds or eyes) quick to notice or understand things or to react.} advice, but I treasure it even more for the \fbox{audacity}\footnote{{\bf audacity} [n] [uncountable] behavior that is brave but likely to shock or offend people, \textsc{synonym}: {\bf nerve}.} \& self-confidence\footnote{{\bf self-confidence} [n] [uncountable] confidence in yourself \& your abilities, \textsc{synonym}: {\bf self-assurance, confidence}.} of its author. \fbox{Will knew where he stood.} He was so sure of where he stood, \& made his position so clear \& so plausible, that his peculiar\footnote{{\bf peculiar} [a] belonging to or connected with 1 particular place, situation, person, etc., \& not others.} stance\footnote{{\bf stance} [n] the opinions that somebody has about something \& expresses publicly, \textsc{synonym}: {\bf position}.} has continued to invigorate\footnote{{\bf invigorate} [v] {\bf 1. invigorate somebody} to make somebody feel healthy \& full of energy; {\bf 2. invigorate something} to make a situation, an organization, etc. efficient \& successful.} me -- \&, I am sure, thousands of other ex-students -- during the years that have intervened\footnote{{\bf intervene} [v] {\bf 1.} [intransitive] to become involved in a situation in order to improve it or stop it from getting worse; {\bf 2.} [intransitive] to happen in the time between events; {\bf 3.} [intransitive] to exist or be found in the space between things; {\bf 4.} [intransitive] to happen in a way  that delays something or prevents it from happening.} since our 1st encounter\footnote{{\bf encounter} [v] {\bf 1. encounter something} to experience something, especially something unpleasant or difficult, while you are trying to do something else, \textsc{synonym}: {\bf run into something}; {\bf 2. encounter something{\tt/}somebody} to discover or experience something, or meet somebody, especially something{\tt/}somebody new, unusual or unexpected, \textsc{synonym}: {\bf come across somebody{\tt/}something}; [n] a meeting, especially one that is sudden or unexpected.}. He had a number of likes \& dislikes that were almost as whimsical\footnote{{\bf whimsical} [a] unusual \& not serious in a way that is either funny or annoying.} as the choice of a necktie, yet he made them seem utterly\footnote{{\bf utter} [a] [only before noun] used to emphasize how complete something is, \textsc{synonym}: {\bf total}; [v] {\bf utter something} to make a sound with your voice; to say something.} convincing. He disliked the word {\it forceful}\footnote{{\bf forceful} [a] {\bf 1.} (of people) expressing opinion firmly \& clearly in a way that persuades other people to believe them, \textsc{synonym}: {\bf assertive}; {\bf 2.} (of opinions, etc.) expressed firmly \& clearly so that other people believe them; {\bf 3.} using force; {\bf 4.} (of action) strong \& effective.} \& advised us to use {\it forcible}\footnote{{\bf forcible} [a] [only before noun] involving the use of physical force.} instead. He felt that the word {\it clever}\footnote{{\bf clever} [a] {\bf 1.} (especially British English) quick at learning \& understanding things, \textsc{synonym}: {\bf intelligent}; {\bf 2. clever (at something{\tt/}doing somethign)} (especially British English) skillful; {\bf 3.} showing intelligence or skill, e.g. in the design of an object, in an idea or somebody's actions.} was greatly overused: ``It is best restricted to ingenuity\footnote{{\bf ingenuity} [n] [uncountable] the ability to invent things or solve problems in clever new ways, \textsc{synonym}: {\bf inventiveness}.} displayed in small matters.'' He despised\footnote{{\bf despise} [v] (not used in the progressive tenses) to dislike \& have no respect for somebody{\tt/}something.} the expression {\it student body}, which he termed gruesome\footnote{{\bf gruesome} [a] very unpleasant \& filling you with horror, usually because it is connected with death or injury.}, \& made a special trip downtown to the {\it Alumni News} office 1 day to protest\footnote{{\bf protest} [n] [uncountable, countable] the expression of strong disagreement with or opposition to something; a statement or an action that shows this.} the expression \& suggest that {\it studentry} be substituted\footnote{{\bf substitute} [v] [intransitive, transitive] to take the place of somebody{\tt/}something else; to use somebody{\tt/}something instead of somebody{\tt/}something else; [n] a person or thing that you use or have instead of the usual one.} -- a coinage\footnote{{\bf coinage} [n] {\bf 1.} [uncountable] the coins used in a particular place or at a particular time; coins of a particular type; {\bf 2.} [countable, uncountable] a word or phrase that has been invented recently; the process of inventing a word or phrase.} of his own, which he felt was similar to {\it citizenry}\footnote{{\bf citizenry} [n] [singular $+$ singular or plural verb] ({\it formal}) all the citizens of a particular town, country, etc.}. I am told that the {\it News} editor was so charmed by the visit, if not by the word, that he ordered the student body buried, never to rise again. {\it Studentry} has taken its place. It's not much of an improvement, but it does sound less cadaverous\footnote{{\bf cadaverous} [a] ({\it literary}) (of a person) extremely pale, thin \& looking ill.}, \& it made Will Strunk quite happy.

Some years ago, when the heir\footnote{{\bf heir} [n] {\bf 1.} a person who has the legal right to receive somebody's property, money or title when that person dies; {\bf 2.} a person who is thought to continue the work or a tradition started by somebody else.} to the throne of England was a child, I noticed a headline in the {\it Times} about Bonnie Prince Charlie: ``CHARLES' TONSILS OOUT.'' Immediately Rule 1 leapt to mind.
\begin{quotation}
	{\bf 1.} Form the possessive singular of nouns by adding {\it 's}. Follow this rule whatever the final consonant\footnote{{\bf consonant} [n] {\bf 1.} (phonetics) a speech sound made by completely or partly stopping the flow of air being breathed out through the mouth; {\bf 2.} a letter of the alphabet that represents a consonant sound.}. Thus write, {\it Charles's friend, Burns's poems, the witch's malice\footnote{{\bf malice} [n] [uncountable] a desire to harm somebody caused by a feeling of hate.}}.
\end{quotation}
Clearly, Will Strunk had foreseen\footnote{{\bf foreseen} [v] to know about something before it happens.}, as far back as 1918, the dangerous tonsillectomy\footnote{{\bf tonsillectomy} [n] ({\it medical}) a medical operation to remove the tonsils.} of a prince, in which the surgeon removes the tonsils \& the {\it Times} copy desk removes the final {\it s}. He started his book with it. I commend Rule 1 to the {\it Times}, \& I trust that Charles's throat, not Charles' throat, is in fine shape today.

Style rules of this sort are, of course, somewhat a matter of individual preference\footnote{{\bf preference} [n] {\bf 1.} [countable, usually singular, uncountable] a greater interest in or desire for somebody{\tt/}something than somebody{\tt/}something else; {\bf 2.} [countable] a thing that is liked better or best.}, \& even the established rules of grammar are open to challenge. Prof. Strunk, although 1 of the most inflexible\footnote{{\bf inflexible} [a] {\bf 1.} ({\it disapproving}) that cannot be changed or made more suitable for a particular situation, \textsc{synonym}: {\bf rigid}; {\bf 2.} ({\it disapproving}) (of people or organizations) unwilling to change their opinions, decision or behavior.} \& choosy\footnote{{\bf choosy} [a] ({\it informal}) careful in choosing; difficult to please, \textsc{synonym}: {\bf fussy, picky}.} of men, was quick to acknowledge\footnote{{\bf acknowledge} [v]  {\bf 1.} to accept that something is true or exists; {\bf 2.} to accept that somebody{\tt/}something has a particular quality, importance or status, \textsc{synonym}: {\bf recognize}; {\bf 3. acknowledge somebody{\tt/}something} to publicly express thanks fo help or inspiration; {\bf 4. acknowledge something} to tell somebody that you have received something that they sent to you.} the fallacy\footnote{{\bf fallacy} [n] {\bf 1.} [countable] a false idea that many people believe is true; {\bf 2.} [uncountable, countable] a false way of thinking about something.} of inflexibility \& the danger of doctrine\footnote{{\bf doctrine} [n] {\bf 1.} [countable, uncountable] {\bf doctrine (of something)} a belief or principle, or set of beliefs or principles, held by a religion, a political party or a legal system; {\bf 2.} ({\bf Doctrine}) [countable] (US) a statement of government policy, especially foreign policy.}. ``It is an old observation,'' he wrote, ``that the best writers sometimes disregard\footnote{{\bf disregard} [v] {\bf disregard something} to not consider something; to treat something as unimportant, \textsc{synonym}: {\bf ignore}.} the rules of rhetoric\footnote{{\bf rhetoric} [n] [uncountable] {\bf 1.} ({\it often disapproving}) speech or writing that is intended to influence people, but that is not completely honest or sincere; {\bf 2.} the skill of using language in speech or writing in a special way that influences or entertains people.}. \texttt{[stop translating here]} When they do so, however, the reader will usually find in the sentence some compensating merit, attained at the cost of the violation. Unless he is certain of doing as well, he will probably do best to follow the rules.''

It is encouraging to see how perfectly a book, even a dusty rule book, perpetuates \& extends the spirit of a man. Will Strunk loved the clear, the brief, the bold, \& his book is clear, brief, bold. Boldness is perhaps its chief distinguishing mark. On p. 26, explaining 1 of his parallels, he says, ``The lefthand version gives the impression that the writer is undecided or timid, apparently unable or afraid to choose 1 form of expression \& hold to it.'' \& his original Rule 11 was ``Make definite assertions.'' That was Will all over. He scorned the vague, the tame, the colorless, the irresolute. He felt it was worse to be irresolute than to be wrong. I remember a day in class when he leaned far forward, in his characteristic pose -- the pose of a man about to impart a secret -- \& croaked, ``If you don't know how to pronounce a word, say it loud! If you don't know how to pronounce a word, say it loud!'' This comical piece of advice struck me as sound at the time, \& I still respect it.\fbox{ Why compound ignorance with inaudibility?} \fbox{Why run \& hide?}

All through {\it The Elements of Style} one finds evidence of the author's deep sympathy for the reader. Will felt that the reader was in serious trouble most of the time, floundering in a swamp, \& that it was the duty of anyone attempting to write English to drain this swamp quickly \& get the reader up on dry ground, or at least to throw a rope. In revising the text, I have tried to hold steadily in mind this belief of his, this concern for the bewildered reader.

In the English classes of today, ``the little book'' is surrounded by longer, lower textbooks -- books with permissive steering \& automatic transitions. Perhaps the book has become something of a curiosity. To me, it still seems to maintain its original poise, standing, in a drafty time, erect, resolute, \& assured. I still find the Strunkian wisdom a comfort, the Strunkian humor a delight, \& the Strunkian attitude forward right-\&-wrong a blessing undisguised.'' -- \cite[Introduction (by E. B. White)]{Strunk_White2019}

%------------------------------------------------------------------------------%

\subsection*{Foreword}

%------------------------------------------------------------------------------%

\subsection*{Introduction}

%------------------------------------------------------------------------------%

\subsection{Elementary Rules of Usage}

\subsubsection{Form the possessive singular of nouns by adding 's.}

%------------------------------------------------------------------------------%

\subsubsection{In a series of 3 or more terms with a single conjunction, use a comma after each term except the last.}

%------------------------------------------------------------------------------%

\subsubsection{Enclose parenthetic expressions between commas.}

%------------------------------------------------------------------------------%

\subsubsection{Place a comma before a conjunction introducing an independent clause.}

%------------------------------------------------------------------------------%

\subsubsection{Do not join independent clauses with a comma.}

%------------------------------------------------------------------------------%

\subsubsection{Do not break sentences in 2.}

%------------------------------------------------------------------------------%

\subsubsection{Use a colon after an independent clause to introduce a list of particulars, an appositive, an amplification, or an illustrative quotation.}

%------------------------------------------------------------------------------%

\subsubsection{Use a dash to set off an abrupt break or interruption \& to announce a long appositive or summary.}

%------------------------------------------------------------------------------%

\subsubsection{The number of the subject determines the number of the verb.}

%------------------------------------------------------------------------------%

\subsubsection{Use the proper case of pronoun.}

%------------------------------------------------------------------------------%

\subsubsection{A participial phrase at the beginning of a sentence must refer to the grammatical subject.}

%------------------------------------------------------------------------------%

\subsection{Elementary Principles of Composition}

\subsubsection{Choose a suitable design \& hold to it.}
``A basic structural design underlies every kind of writing. Writers will in part follow this design, in part deviate from it, according to their skills, their needs, \& the unexpected events that accompany the act of composition. Writing, to be effective, must follow closely the thoughts of the writer, but not necessarily in the order in which those thoughts occur. This calls for a scheme of procedure. In some cases, the best design is no design, as with a love letter, which is simply an outpouring, or with a casual essay, which is a ramble. But in most cases, planning must be a deliberate prelude to writing. The 1st principle of composition, therefore, is to foresee or determine the shape of what is to come \& pursue that shape.

A sonnet is built on a 14-line frame, each line containing 5 feet. Hence, sonneteers know exactly where they are headed, although they may not know how to get there. Most forms of composition are less clearly defined, more flexible, but all have skeletons to which the writer will bring the flesh \& the blood. The more clearly the writer perceives the shape, the better are the chances of success.'' -- \cite[p. 29]{Strunk_White_element_style}

%------------------------------------------------------------------------------%

\subsubsection{Make the paragraph the unit of composition: 1 paragraph to each topic.}
``The paragraph is a convenient unit; it serves all forms of literary work. As long as it holds together, a paragraph may be of any length -- a single, short sentence or a passage of great duration.

If the subject on which you are writing is of slight extent, or if you intend to treat it briefly, there may be no need to divide it into topics. Thus, a brief description, a brief book review, a brief account of a single incident, a narrative merely outlining an action, the setting forth of a single idea -- any 1 of these is best written in a single paragraph. After the paragraph has been written, examine it to see whether division will improve it.

Ordinarily, however, a subject requires division into topics, each of which should be dealt with in a paragraph. The object of treating each topic in a paragraph by itself is, of course, to aid the reader. The beginning of each paragraph is a signal that a new step in the development of the subject has been reached.

As a rule, single sentences should not be written or printed as paragraphs. An exception may be made of sentences of transition, indicating the relation between the parts of an exposition or argument.

In dialogue, each speech, even if only a single word, is usually a paragraph by itself; i.e., a new paragraph begins with each change of speaker. The application of this rule when dialogue \& narrative are combined is best learned from examples in well-edited works of fiction. Sometimes a writer, seeking to create an effect of rapid talk or for some other reason, will elect not to set off each speech in a separate paragraph \& instead will run speeches together. The common practice, however, \& the one that serves best in most instances, is to give each speech a paragraph of its own.

As a rule, begin each paragraph either with a sentence that suggests the topic or with a sentence that helps the transition. If a paragraph forms part of a larger composition, its relation to what precedes, or its function as a part of the whole, may need to be expressed. This can sometimes be done by a mere word or phrase ({\it again, therefore, for the same reason}) in the 1st sentence. Sometimes, however, it is expedient to get into the topic slowly, by way of a sentence or 2 of introduction or transition.

In narration \& description, the paragraph sometimes begins with a concise, comprehensive statement serving to hold together the details that follows.
\begin{quotation}\it
	The breeze served us admirably.
	
	The campaign opened with a series of reverses.
	
	The next 10 or 12 pages were filled with a curious set of entries.
\end{quotation}
But when this device, or any device, is too often used, it becomes a mannerism. More commonly, the opening sentence simply indicates by its subject the direction the paragraph is to take.
\begin{quotation}\it
	At length I thought I might return toward the stockade.
	
	He picked up the heavy lamp from the table \& began to explore.
	
	Another flight of steps, \& they emerged on the roof.
\end{quotation}
In animated narrative, the paragraphs are likely to be short \& without any semblance of a topic sentence, the writer rushing headlong, event following event in rapid succession. The break between such paragraphs merely serves the purpose of a rhetorical pause, throwing into prominence some detail of the action.

In general, remember that paragraphing calls for a good eye as well as a logical mind. Enormous blocks of print look formidable to readers, who are often reluctant to tackle them. Therefore, breaking long paragraphs in 2, even if it is not necessary to do so for sense, meaning, or logical development, is often a visual help. But remember, too, that firing off many short paragraphs in quick succession can be distracting. Paragraph breaks used only for show read like the writing of commerce or of display advertising. Moderation \& a sense of order should be the main considerations in paragraphing.'' -- \cite[pp. 30--31]{Strunk_White_element_style}

%------------------------------------------------------------------------------%

\subsubsection{Use the active voice.}
``The active voice is usually more direct \& vigorous than the passive:
\begin{example}
	I shall always remember my 1st visit to Boston.
\end{example}
This is much better than
\begin{example}
	My 1st visit to Boston will always be remembered by me.
\end{example}
The latter sentence is less direct, less bold, \& less concise. If the writer tries to make it more concise by omitting ``by me,''
\begin{example}
	My 1st visit to Boston will always be remembered,
\end{example}
it becomes indefinite: is it the writer or some undisclosed person or the world at large that will always remember this visit?

This rule does not, of course, mean that the writer should entirely discard the passive voice, which is frequently convenient \& sometimes necessary.
\begin{example}
	The dramatists of the Restoration are little esteemed today.
	
	Modern readers have little esteem for the dramatists of the Restoration.
\end{example}
The 1st would be the preferred form in a paragraph on the dramatists of the Restoration, the 2nd in a paragraph on the tastes of modern readers. The need to make a particular word the subject of the sentence will often, as in these examples, determine which voice is to be used.

The habitual use of the active voice, however, makes for forcible writing. This is true not only in narrative concerned principally with action but in writing of any kind. Many a tame sentence of description or exposition can be made lively \& emphatic by substituting a transitive in the active voice for some such perfunctory expression as {\it there is} or {\it could be heard}.
\begin{example}
	There were a great number of dead leaves lying on the ground. $\to$ Dead leaves covered the ground.
	
	At dawn the crowing of a rooster could be heard. $\to$ The cock's crow came with dawn.
	
	The reason he left college was that his health became impaired. $\to$ Failing health compelled him to leave college.
	
	It was not long before she was very sorry that she had said what she had. $\to$ She soon repented her words.
\end{example}
Note, in the examples above, that when a sentence is made stronger, it usually becomes shorter. Thus, brevity is a by-product of vigor.'' -- \cite[p. 32]{Strunk_White_element_style}

%------------------------------------------------------------------------------%

\subsubsection{Put statements in positive form.}
``Make definite assertions. Avoid tame, colorless, hesitating, noncommittal language. Use the word {\it not} as a means of denial or in antithesis, never as a means of evasion.
\begin{example}
	He was not very often on time. $\to$ He usually came late.
	
	She did not think that studying Latin was a sensible way to use one's time. $\to$ She thought the study of Latin a waste of time.
	
	\emph{The Taming of the Shrew} is rather weak in spots. Shakespeare does not portray Katharine as a very admirable character, nor does Bianca remain long in memory as an important character in Shakespeare's works. $\to$ The women in \emph{The Taming of the Shrew} are unattractive. Katharine is disagreeable, Bianca insignificant.
\end{example}
The last example, before correction, is indefinite as well as negative. The corrected version, consequently, is simply a guess at the writer's intention.

All 3 examples show the weakness inherent in the word {\it not}. Consciously or unconsciously, the reader is dissatisfied with being told only what is not; the reader wishes to be told what is. Hence, as a rule, it is better to express even a negative in positive form.
\begin{example}
	not honest $\to$ dishonest, not important $\to$ trifling, did not remember $\to$ forgot, did not pay any attention to $\to$ ignored, did not have much confidence in $\to$ distrusted.
\end{example}
Placing negative \& positive in opposition makes for a stronger structure.
\begin{example}
	Not charity, but simple justice.
	
	Not that I loved Caesar less, but that I loved Rome more.
	
	Ask not what your country can do for you -- ask what you can do for your country.
\end{example}
Negative words other than {\it not} are usually strong.
\begin{example}
	Her loveliness I never knew
	
	Until she smiled on me.
\end{example}
Statements qualified with unnecessary auxiliaries or conditionals sound irresolute.
\begin{example}
	If you would let us know the time of your arrival, we would be happy to arrange your transportation from the airport. $\to$ If you will let us know the time of your arrival, we shall be happy to arrange your transportation from the airport.
	
	Applicants can make a good impression by being neat \& punctual. $\to$  Applicants will make a good impression if they are neat \& punctual.
	
	Plath may be ranked among those modem poets who died young. $\to$ Plath was one of those modern poets who died young.
\end{example}
If your every sentence admits a doubt, your writing will lack authority. Save the auxiliaries {\it would, should, could, may, might}, \& {\it can} for situations involving real uncertainty.'' -- \cite[pp. 33--34]{Strunk_White_element_style}

%------------------------------------------------------------------------------%

\subsubsection{Use definite, specific, concrete  language.}
``Prefer the specific to the general, the definite to the vague, the concrete to the abstract.
\begin{example}
	A period of unfavorable weather set in. $\to$ It rained every day for a week.
	
	He showed satisfaction as he took possession of his well-earned reward. $\to$ He grinned as he pocketed the coin.
\end{example}
If those who have studied the art of writing are in accord on any 1 point, it is this: the surest way to arouse \& hold the readers attention is by being specific, definite, \& concrete. The greatest writers -- Homer, Dante, Shakespeare -- are effective largely because they deal in particulars \& report the details that matter. Their words call up pictures.

Jean Stafford, to cite a more modern author, demonstrates in her short story ``In the Zoo'' how prose is made vivid by the use of words that evoke images \& sensations:
\begin{example}
	$\ldots$ Daisy \& I in time found asylum in a small menagerie down by the railroad tracks. It belonged to a gentle alcoholic ne'er-do- well, who did nothing all day long but drink bathtub gin in rickeys \&  play solitaire \&  smile to himself \&  talk to his animals. He had a little, stunted red vixen \&  a deodorized skunk, a parrot from Tahiti that spoke Parisian French, a woebegone coyote, \&  two capuchin monkeys, so serious \&  humanized, so small \&  sad \&  sweet, \&  so religious-looking with their tonsured heads that it was impossible not to think their gibberish was really an ordered language with a grammar that someday some philologist would understand.
	
	Gran knew about our visits to Mr. Murphy \&  she did not object, for it gave her keen pleasure to excoriate him when we came home. His vice was not a matter of guesswork; it was an established fact that he was half-seas over from dawn till midnight. ``With the black Irish,'' said Gran, ``the taste for drink is taken in with the mother's milk \&  is never mastered. Oh, I know all about those promises to join the temperance movement \&  not to touch another drop. The way to Hell is paved with good intentions.'' -- Excerpt from ``In the Zoo'' from Bad Characters by Jean Stafford.
\end{example}
If the experiences of Walter Mitty, of Molly Bloom, of Rabbit Angstrom have seemed for the moment real to countless readers, if in reading Faulkner we have almost the sense of inhabiting Yoknapatawpha County during the decline of the South, it is because the details used are definite, the terms concrete. It is not that every detail is given -- that would be impossible, as well as to no purpose -- but that all the significant details are given, \&  with such accuracy \&  vigor that readers, in imagination, can project themselves into the scene.

In exposition \&  in argument, the writer must likewise never lose hold of the concrete; \&  even when dealing with general principles, the writer must furnish particular instances of their application.

In his {\it Philosophy of Style}, Herbert Spencer gives 2 sentences to illustrate how the vague \&  general can be turned into the vivid \&  particular:
\begin{example}
	In proportion as the manners, customs, \&  amusements of a nation are cruel \&  barbarous, the regulations of their penal code will be severe. $\to$ In proportion as men delight in battles, bullfights, \&  combats of gladiators, will they punish by hanging, burning, \&  the rack.
\end{example}
To show what happens when strong writing is deprived of its vigor, George Orwell once took a passage from the Bible \&  drained it of its blood. On the left, below, is Orwell’s translation; on the right, the verse from Ecclesiastes (King James Version).
\begin{example}
	Objective consideration of contemporary phenomena compels the conclusion that success or failure in competitive activities exhibits no tendency to be commensurate with innate capacity, but that a considerable element of the unpredictable must inevitably be taken into account. $\to$ I returned, \&  saw under the sun, that the race is not to the swift, nor the battle to the strong, neither yet bread to the wise, nor yet riches to men of understanding, nor yet favor to men of skill; but time \&  chance happeneth to them all.'' -- \cite[pp. 35--36]{Strunk_White_element_style}
\end{example}

%------------------------------------------------------------------------------%

\subsubsection{Omit needless words.}
``Vigorous writing is concise. A sentence should contain no unnecessary words, a paragraph no unnecessary sentences, for the same reason that a drawing should have no unnecessary lines \&  a machine no unnecessary parts. This requires not that the writer make all sentences short, or avoid all detail \&  treat subjects only in outline, but that every word tell.

Many expressions in common use violate this principle.
\begin{example}
	the question as to whether $\to$ whether (the question whether), there is no doubt but that $\to$ no doubt (doubtless), used for fuel purposes $\to$ used for fuel, he is a man who $\to$ he, in a hasty manner $\to$ hastily, this is a subject that $\to$ this subject, Her story is a strange one. $\to$ Her story is strange. the reason why is that $\to$ because.
\end{example}
{\it The fact that} is an especially debilitating expression. It should be revised out of every sentence in which it occurs.
\begin{example}
	owing to the fact that $\to$ since (because), in spite of the fact that $\to$ though (although), call your attention to the fact that $\to$ remind you (notify you), I was unaware of the fact that $\to$ I was unaware that (did not know), the fact that he had not succeeded $\to$ his failure, the fact that I had arrived $\to$ my arrival.
\end{example}
See also the words {\it case, character, nature} in Chap. IV. {\it Who is, which was}, \&  the like are often superfluous.
\begin{example}
	His cousin, who is a member of the same firm $\to$ His cousin, a member of the same firm
	
	Trafalgar, which was Nelson's last battle $\to$ Trafalgar, Nelson’s last battle.
\end{example}
As the active voice is more concise than the passive, \&  a positive statement more concise than a negative one, many of the examples given under Rules 14 \&  15 illustrate this rule as well.

A common way to fall into wordiness is to present a single complex idea, step by step, in a series of sentences that might to advantage be combined into one.
\begin{example}
	Macbeth was very ambitious. This led him to wish to become king of Scotland. The witches told him that this wish of his would come true. The king of Scotland at this time was Duncan. Encouraged by his wife, Macbeth murdered Duncan. He was thus enabled to succeed Duncan as king. (51 words)
	
	$to$ Encouraged by his wife, Macbeth achieved his ambition \&  realized the prediction of the witches by murdering Duncan \&  becoming king of Scotland in his place. (26 words)'' -- \cite[pp. 37--38]{Strunk_White_element_style}
\end{example}

%------------------------------------------------------------------------------%

\subsubsection{Avoid a succession of loose sentences.}
``This rule refers especially to loose sentences of a particular type: those consisting of 2 clauses, the 2nd introduced by a conjunction or relative. A writer may err by making sentences too compact \& periodic. An occasional loose sentence prevents the style from becoming too formal \& gives the reader a certain relief. Consequently, loose sentences are common in easy, unstudied writing. The danger is that there may be too many of them.

An unskilled writer will sometimes construct a whole paragraph of sentences of this kind, using as connectives {\it and, but}, and, less frequently, {\it who, which, when, where}, \& {\it while}, these last in nonrestrictive senses. (See Rule 3.)
\begin{example}
	The 3rd concert of the subscription series was given last evening, \& a large audience was in attendance. Mr. Edward Appleton was the soloist, \& the Boston Symphony Orchestra furnished the instrumental music. The former showed himself to be an artist of the 1st rank, while the latter proved itself fully deserving of its high reputation. The interest aroused by the series has been very gratifying to the Committee, \& it is planned to give a similar series annually hereafter. The 4th concert will be given on Tuesday, May 10, when an equally attractive program will be presented.
\end{example}
Apart from its triteness \& emptiness, the paragraph above is bad because of the structure of its sentences, with their mechanical symmetry \& singsong. Compare these sentences from the chapter ``What I Believe'' in E. M. Forster's {\it 2 Cheers for Democracy}:
\begin{example}
	I believe in aristocracy, though -- if that is the right word, \& if a democrat may use it. Not an aristocracy of power, based upon rank \& influence, but an aristocracy of the sensitive, the considerate \& the plucky. Its members are to be found in all nations \& classes, \& all through the ages, \& there is a secret understanding between them when they meet. They represent the true human tradition, the 1 permanent victory of our queer race over cruelty \& chaos. Thousands of them perish in obscurity, a few are great names. They are sensitive for others as well as for themselves, they are considerate without being fussy, their pluck is not swankiness but the power to endure, \& they can take a joke.
\end{example}
A writer who has written a series of loose sentences should recast enough of them to remove the monotony, replacing them with simple sentences, sentences of 2 clauses joined by a semicolon, periodic sentences of 2 clauses, or sentences (loose or periodic) of 3 clauses -- whichever best represent the real relations of the thought.'' -- \cite[pp. 39--40]{Strunk_White_element_style}

%------------------------------------------------------------------------------%

\subsubsection{Express coordinate ideas in similar form.}
``This principle, that of parallel construction, requires that expressions similar in content \& function be outwardly similar. The likeness of form enables the reader to recognize more readily the likeness of content \& function. The familiar Beatitudes exemplify the virtue of parallel construction.
\begin{quotation}
	Blessed are the poor in spirit: for theirs is the kingdom of heaven.
	
	Blessed are they that mourn: for they shall be comforted.
	
	Blessed are the meek: for they shall inherit the earth.
	
	Blessed are they which do hunger \& thirst after righteousness: for they shall be filled.
\end{quotation}
The unskilled writer often violates this principle, mistakenly believing in the value of constantly varying the form of expression. When repeating a statement to emphasize it, the writer may need to vary its form. Otherwise, the writer should follow the principle of parallel construction.
\begin{example}
	Formerly, science was taught by the textbook method, while now the laboratory method is employed. $\to$ Formerly, science was taught by the textbook method; now it is taught by the laboratory method.
\end{example}
The lefthand version gives the impression that the writer is undecided or timid, apparently unable or afraid to choose one form of expression \& hold to it. The righthand version shows that the writer has at least made a choice \& abided by it.

By this principle, an article or a preposition applying to all the members of a series must either be used only before the first term or else be repeated before each term.
\begin{example}
	The French, the Italians, Spanish, \& Portuguese $\to$ The French, the Italians, the Spanish, \& the Portuguese
	
	In spring, summer, or in winter $\to$ In spring, summer, or winter (In spring, in summer, or in winter).
\end{example}
Some words require a particular preposition in certain idiomatic uses. When such words are joined in a compound construction, all the appropriate prepositions must be included, unless they are the same.
\begin{example}
	His speech was marked by disagreement and scorn for his opponent's position. $\to$ His speech was marked by disagreement with and scorn for his opponent's position.
\end{example}
Correlative expressions ({\it both, \&; not, but; not only, but also; either, or; 1st, 2nd, 3rd}; \& the like) should be followed by the same grammatical construction. Many violations of this rule can be corrected by rearranging the sentence.
\begin{example}
	It was both a long ceremony and very tedious. $\to$ The ceremony was both long and tedious.
	
	A time for not words but action. $\to$ A time not for words but for action.
	
	Either you must grant his request or incur his ill will. $\to$ You must either grant his request or incur his ill will.
	
	My objections are, 1st, the injustice of the measure; 2nd, that it is unconstitutional. $\to$ My objections are, 1st, that the measure is unjust; 2nd, that it is unconstitutional.
\end{example}
It may be asked, what if you need to express a rather large number of similar ideas -- say, 20? Must you write 20 consecutive sentences of the same pattern? On closer examination, you will probably find that the difficulty is imaginary -- that these 20 ideas can be classified in groups, \& that you need apply the principle only within each group. Otherwise, it is best to avoid the difficulty by putting statements in the form of a table.'' -- \cite[pp. 41--42]{Strunk_White_element_style}

%------------------------------------------------------------------------------%

\subsubsection{Keep related words together.}

%------------------------------------------------------------------------------%

\subsubsection{In summaries, keep to 1 tense.}

%------------------------------------------------------------------------------%

\subsubsection{Place the emphatic words of a sentence at the end.}

%------------------------------------------------------------------------------%

\subsection{A Few Matters of Form}

%------------------------------------------------------------------------------%

\subsection{Words \& Expressions Commonly Misused}

%------------------------------------------------------------------------------%

\subsection{An Approach to Style (With a List of Reminders)}

\subsubsection{Place yourself in the background.}
``Write in a way that draws the reader's attention to the sense \& substance of the writing, rather than to the mood \& temper of the author. If the writing is solid \& good, the mood \& temper of the writer will eventually be revealed \& not at the expense of the work. Therefore, the 1st piece of advice is this: to achieve style, begin by affecting none -- i.e., place yourself in the background. A careful \& honest writer does not need to worry about style. As you become proficient in the use of language, your style will emerge, because you yourself will emerge, \& when this happens you will find it increasingly easy to break through the barriers that separate you from other minds, other hearts -- which is, of course, the purpose of writing, as well as its principal reward. Fortunately, the act of composition, or creation, disciplines the mind; writing is 1 way to go about thinking, \& the practice \& habit of writing not only drain the mind but supply it, too.'' -- \cite[p. 78]{Strunk_White_element_style}

%------------------------------------------------------------------------------%

\subsubsection{Write in a way that comes naturally.}
``Write in a way that comes easily \& naturally to you, using words \& phrases that come readily to hand. But do not assume that becaues you have acted naturally your product is without flaw.

The use of language begins with imitation. The infant imitates the sounds made by its parents; the child imitates 1st the spoken language, then the stuff of books. The imitative life continues long after the writer is secure in the language, for it is almost impossible to avoid imitating what one admires. Never imitate consciously, but do not worry about being an imitator; take pains instead to admire what is good. Then when you write in a way that comes naturally, you will echo the halloos that bear repeating.'' -- \cite[p. 79]{Strunk_White_element_style}

%------------------------------------------------------------------------------%

\subsubsection{Work from a suitable design.}
``Before beginning to compose something, gauge the nature \& extent of the enterprise \& work from a suitable design. (See Chap. II, Rule 12.) Design informs even the simplest structure, whether of brick \& steel or of prose. You raise a pup tent from 1 sort of vision, a cathedral from another. This does not mean that you must sit with a blueprint always in front of you, merely that you had best anticipate what you are getting into. To compose a laundry list, you can work directly from the pile of soiled garments, ticking them off 1 by 1. By to write a biography, you will need at least a rough scheme; you cannot plunge in blindly \& start ticking off fact after fact about your subject, lest you miss the forest for the trees \& there be no end to your labors.

Sometimes, of course, impulse \& emotion are more compelling than design. If you are deeply troubled \& are composing a letter appealing for mercy or for love, you had best not attempt to organize your emotions; the prose will have a better chance if the emotions are left in disarray -- which you'll probably have to do anyway, since feelings do not usually lend themselves to rearrangement. But even the kind of writing that is essentially adventurous \& impetuous will on examination be found to have a secret plan: Columbus didn't just sail, he sailed west, \& the New World took shape from this simple \&, we now think, sensible design.'' -- \cite[p. 80]{Strunk_White_element_style}

%------------------------------------------------------------------------------%

\subsubsection{Write with nouns \& verbs.}
``Write with nouns \& verbs, not with adjectives \& adverbs. The adjective hasn't been built that can pull a weak or inaccurate noun out of a tight place. This is not to disparage adjectives \& adverbs; they are indispensable parts of speech. Occasionally they surprise us with their power, as in
\begin{quotation}\it
	Up the airy mountain,
	
	Down the rushy glen,
	
	We daren't go a-hunting
	
	For fear of little men $\ldots$
\end{quotation}
The nouns {\it mountain} \& {\it glen} are accurate enough, but had the mountain not become airy, the glen rushy, William Ailing-ham might never have got off the ground with this poem. In general, however, it is nouns \& verbs, not their assistants, that give good writing its toughness \& color.'' -- \cite[p. 81]{Strunk_White_element_style}

%------------------------------------------------------------------------------%

\subsubsection{Revise \& rewrite.}
``Revising is part of writing. Few writers are so expert that they can produce what they are after on the 1st try. Quite often you will discover, on examining the completed work, that there are serious flaws in the arrangement of the material, calling for transpositions. When this is the case, a word processor can save you time \& labor as you rearrange the manuscript. You can select material on your screen \& move it to a more appropriate spot, or, if you cannot find the right spot, you can move the material to the end of the manuscript until you decide whether to delete it. Some writers find that working with a printed copy of the manuscript helps them to visualize the process of change; others prefer to revise entirely on screen. Above all, do not be afraid to experiment with what you have written. Save both the original \& the revised versions; you can always use the computer to restore the manuscript to its original condition, should that course seem best. Remember, it is no sign of weakness or defeat that your manuscript ends up in need of major surgery. This is a common occurrence in all writing, \& among the best writers.'' -- \cite[p. 82]{Strunk_White_element_style}

%------------------------------------------------------------------------------%

\subsubsection{Do not overwrite.}
``Rich, ornate prose is hard to digest, generally unwholesome, \& sometimes nauseating. If the sickly-sweet word, the overblown phrase are your natural form of expression, as is sometimes the case, you will have to compensate for it by a show of vigor, \& by writing something as meritorious as the Songs of Songs, which is Solomon's.

When writing with a computer, you must guard against wordiness. The click \& flow of a word processor can be seductive, \& you may find yourself adding a few unnecessary words or even a whole passage just to experience the pleasure of running your fingers over the keyboard \& watching your words appear on the screen. It is always a good idea to reread your writing later \& ruthlessly delete the excess.'' -- \cite[p. 83]{Strunk_White_element_style}

%------------------------------------------------------------------------------%

\subsubsection{Do not overstate.}
``When you overstate, readers will be instantly on guard, \& everything that has preceded your overstatement as well as everything that follows it will be suspect in their minds because they have lost confidence in your judgment or your poise. Overstatement is 1 of the common faults. A single overstatement, wherever or however it occurs, diminishes the whole, \& a single carefree superlative has the power to destroy, for readers, the object of your enthusiasm.'' -- \cite[p. 84]{Strunk_White_element_style}

%------------------------------------------------------------------------------%

\subsubsection{Avoid the use of qualifiers.}
``{\it Rather, very, little, pretty} -- these are the leeches that infest the pond of prose, sucking the blood of words. The constant use of the adjective {\it little} (except to indicate size) is particularly debilitating; we should all try to do a little better, we should all be very watchful of this rule, for it is a rather important one, \& we are pretty sure to violate it now \& then.'' -- \cite[p. 85]{Strunk_White_element_style}

%------------------------------------------------------------------------------%

\subsubsection{Do not affect a breezy manner.}
``The volume of writing is enormous, these days, \& much of it has a sort of windiness about it, almost as though the author were in a state of euphoria. ``Spontaneous me,'' say Whitman, \&, in his innocence, let loose the hordes of uninspired scribblers who would 1 day confuse spontaneity with genius.

The breezy style is often the work of an egocentric, the person who imagines that everything that comes to mind is of general interest \& that uninhibited prose creates high spirits \& carries the day. Open any alumni magazine, turn to the class notes, \& you are quite likely to encounter old Spontaneous Me at work -- an aging collegian who writes something like this:
\begin{quotation}\it
	Well, guys, here I am again dishing the dirt about your disorderly classmates, after passing a week in the Big Apple trying to catch the Columbia hoops tilt \& then a cab-ride from hell through the West Side casbah. \& speaking of news, howzabout tossing a few primo items this way?
\end{quotation}
This is an extreme example, but the same wind blows, at lesser velocities, across vast expanses of journalistic prose. The author in this case has managed in 2 sentences to commit most of the unpardonable sins: he obviously has nothing to say, he is showing off \& directing the attention of the reader to himself, he is using slang with neither provocation nor ingenuity, he adopts a patronizing air by throwing in the word {\it primo}, he is humorless (though full of fun), dull, \& empty. He has not done his work. Compare his opening remarks with the following -- a plunge directly into the news:
\begin{quotation}\it
	Clyde Crawford, who stroked the varsity shell in 1958, is swinging an oar again after a lapse of 40 years. Clyde resigned last spring as executive sales manager of the Indiana Flotex Company \& is now a gondolier in Venice.
\end{quotation}
This, although conventional, is compact, informative, unpretentious. The writer has dug up an item of news \& presented it in a straightforward manner. What the 1st writer tried to accomplish by cutting rhetorical capers \& by breeziness, the 2nd writer managed to achieve by good reporting, by keeping a tight rein on his material, \& by staying out of the act.'' -- \cite[p. 87]{Strunk_White_element_style}

%------------------------------------------------------------------------------%

\subsubsection{Use orthodox spelling.}
``In ordinary composition, use orthodox spelling. Do not write {\it nite} for {\it night, thru} for {\it through, pleez} for {\it please}, unless you plan to introduce a complete system of simplified spelling \& are prepared to take the consequences.

In the original edition of {\it The Elements of Style}, there was a chapter on spelling. In it, the author had this to say:
\begin{quotation}\it
	The spelling of English words is not fixed \& invariable, nor does it depend on any other authority than general agreement. At the present day there is practically unanimous agreement as to the spelling of most words $\ldots$ At any given moment, however, a relatively small number of words may be spelled in more than 1 way. Gradually, as a rule, 1 of these forms comes to be generally preferred, \& the less customary form comes to look obsolete \& is discarded. From time to time new forms, mostly simplifications, are introduced by innovators, \& either win their place or die of neglect.
	
	The practical objection to unaccepted \& oversimplified spellings is the disfavor with which they are received by the reader. They distract his attention \& exhaust his patience. He reads the form though automatically, without thought of its needless complexity; he reads the abbreviation tho \& mentally supplies the missing letters, at the cost of a fraction of his attention. The writer has defeated his own purposed.
\end{quotation}
The language manages somehow to keep pace with events. A word that has taken hold in our century is {\it thru-way}; it was born of necessity \& is apparently here to stay. In combination with {\it way, thru} is more serviceable than {\it through}; it is a high-speed word for readers who are going 65. {\it Throughway} would be too long to fit on a road sign, too slow to serve the speeding eye. It is conceivable that because of our thruways, {\it through} will eventually become {\it thru} -- after many more thousands of miles of travel.'' -- \cite[p. 88]{Strunk_White_element_style}

%------------------------------------------------------------------------------%

\subsubsection{Do not explain too much.}
``It is seldom advisable to tell all. Be sparing, e.g., in the use of adverbs after ``he said,'' ``she replied,'' \& the like: ``he said consolingly''; ``she replied grumblingly.'' Let the conversation itself disclose the speaker's manner of condition. Dialogue heavily weighted with adverbs after the attributive verb is cluttery \& annoying. Inexperienced writers not only overwork their adverbs but load their attributives with explanatory verbs: ``he consoled,'' ``she congratulated.'' They do this, apparently, in the belief that the word {\it said} is always in need of support, or because they have been told to do it by experts in the art of bad writing.'' -- \cite[p. 89]{Strunk_White_element_style}

%------------------------------------------------------------------------------%

\subsubsection{Do not construct awkward adverbs.}
``Adverbs are easy to build. Take an adjective or a participle, add {\it -ly}, \& behold! you have an adverb. But you'd probably be better off without it. Do not write {\it tangledly}. The word itself is a tangle. Do not even write {\it tiredly}. Nobody says {\it tangledly} \& not many people say {\it tiredly}. Words that are not used orally are seldom the ones to put on paper.
\begin{example}
	He climbed tiredly to bed. $\to$ He climbed wearily to bed.
	
	The lamp cord lay tangledly beneath her chair. $\to$ The lamp cord lay in tangles beneath her chair.
\end{example}
Do not dress words up by adding {\it -ly} to them, as though putting a hat on a horse.
\begin{example}
	overly $\to$ over, muchly $\to$ much, thusly $\to$ thus.'' -- \cite[p. 90]{Strunk_White_element_style}
\end{example}


%------------------------------------------------------------------------------%

\subsubsection{Make sure the reader knows who is speaking.}
``Dialogue is a total loss unless you indicate who the speaker is. In long dialogue passages containing no attributives, the reader may become lost \& be compelled to go back \& reread in order to puzzle the thing out. Obscurity is an imposition on the reader, to say nothing of its damage to the work.

In dialogue, make sure that your attributives do not awkwardly interrupt a spoken sentence. Place them where the break would come naturally in speech -- i.e., where the speaker would pause for emphasis, or take a breath. The best test for locating an attributive is to speak the sentence aloud.
\begin{example}
	``Now, my boy, we shall see,'' he said, ``how well you have learned your lesson.'' $\to$ ``Now, my boy,'' he said, ``we shall see how well you have learned your lesson.''
	
	``What's more, they would never,'' she added, ``consent to the plan.'' $\to$  ``What's more,'' she added, ``they would never consent to the plan.'''' -- \cite[p. 91]{Strunk_White_element_style}
\end{example}

%------------------------------------------------------------------------------%

\subsubsection{Avoid fancy words.}
``Avoid the elaborate, the pretentious, the coy, \& the cute. Do not be tempted by a 20-dollar word when there is a 10-center handy, ready \& able. Anglo-Saxon is a livelier tongue than Latin, so use Anglo-Saxon words. In this, as in so many matters pertaining to style, one's ear must be one's guide: {\it gut} is a lustier noun than {\it intestine}, but the 2 words are not interchangeable, because {\it gut} is often inappropriate, being too coarse for the context. Never call a stomach a tummy without good reason.

If you admire fancy words, if every sky is {\it beauteous}, every blonde {\it curvaceous}, every intelligent child prodigious, if you are tickled by {\it discombobulate}, you will have a bad time with Reminder 14. What is wrong, you ask, with {\it beauteous?} No one knows, for sure. There is nothing wrong, really, with any word -- all are good, but some are better than others. A matter of ear, a matter of reading the books that sharpen the ear.

The line between the fancy \& the plain, between the atrocious \& the felicitous, is sometimes alarmingly fine. The opening phrase of the Gettysburg address is close to the line, at least by our standards today, \& Mr. Lincoln, knowingly or unknowingly, was flirting with disaster when he wrote ``4 score \& 7 years ago.'' The President could have got into his sentence with plain ``87'' -- a saving of 2 words \& less of a strain on the listeners' powers of multiplication. But Lincoln's ear must have told him to go ahead with 4 score \& 7. By doing so, he achieved cadence while skirting the edge of fanciness. Suppose he had blundered over the line \& written, ``In the year of our Lord seventeen hundred \& seventy-six.'' His speech would have sustained a heavy blow. Or suppose he had settle for ``87.'' In that case he would have got into his introductory sentence too quickly; the timing would have been bad.

The question of ear is vital. Only the writer whose ear is reliable is in a position to use bad grammar deliberately; this writer knows for sure when a colloquialism is better than formal phrasing \& is able to sustain the work at a level of good taste. So cock your ear. Years ago, students were warned not to end a sentence with a preposition; time, of course, has softened that rigid decree. Not only is the preposition acceptable at the end, sometimes it is more effective in that spot than anywhere else. ``A claw hammer, not an ax, was the tool he murdered her with.'' This is preferable to ``A claw hammer, not an ax, was the tool with which he murdered her.'' Why? Because it sounds more violent, more like murder. A matter of ear.

\& would you write ``The worst tennis player around here is I'' or ``The The worst tennis player around here is me''? The 1st is good grammar, the 2nd is good judgment -- although the {\it me} might not do in all contexts.

The split infinitive is another trick of rhetoric in which the ear must be quicker than the handbook. Some infinitives seem to improve on being split, just as a stick of round stovewood does. ``I cannot bring myself to really like the fellow.'' The sentence is relaxed, the meaning is clear, the violation is harmless \& scarcely perceptible. Put the other way, the sentence becomes stiff, needlessly formal. A matter of ear.

There are times when the ear not only guides us through difficult situations but also saves us from minor or major embarrassments of prose. The ear, e.g., must decide when to omit {\it that} from a sentence, when to retain it. ``She knew she could do it'' is preferable to ``She knew that she could do it'' -- simpler \& just as clear. Bu tin many cases the {\it that} is needed. ``He felt that his big nose, which was sunburned, made him look ridiculous.'' Omit the {\it that} \& you have ``He felt his big nose $\ldots$'''' -- \cite[p. 93]{Strunk_White_element_style}

%------------------------------------------------------------------------------%

\subsubsection{Do not use dialect unless your ear is good.}
``Do not attempt to use dialect unless you are a devoted student of the tongue you hope to reproduce. If you use dialect, be consistent. The reader will become impatient or confused upon finding 2 or more versions of the same word or expression. In dialect it is necessary to spell phonetically, or at least ingeniously, to capture unusual inflections. Take, e.g., the word {\it once}. It often appears in dialect writing as {\it oncet}, but {\it oncet} looks as though it should be pronounced ``onset.'' A better spelling would be {\it wunst}. But if you write it {\it oncet} once, write it that way throughout. The best dialect writers, by \& large, are economical of their talents; they use the minimum, not the maximum, of deviation from the norm, thus sparing their readers as well as convincing them.'' -- \cite[p. 94]{Strunk_White_element_style}

%------------------------------------------------------------------------------%

\subsubsection{Be clear.}
``Clarity is not the prize in writing, nor it is always the principal mark of a good style. There are occasions when obscurity serves a literary yearning, if not a literary purpose, \& there are writers whose mien is more overcast than clear. But since writing is communication, clarity can only be a virtue. \& although there is no substitute for merit in writing, clarity comes closest to being one. Even to a writer who is being intentionally obscure or wild of tongue we can say, ``be obscure clearly! Be wild of tongue in a way we can understand!'' Even to writers of market letters, telling us (but not telling us) which securities are promising, we can say, ``Be cagey plainly! Be elliptical in a straightforward fashion!''

Clarity, clarity, clarity. When you become hopelessly mired in a sentence, it is best to start fresh; do not try to fight your way through against the terrible odds of syntax. Usually what is wrong is that the construction has become too involved at some point; the sentence needs to be broken apart \& replaced by 2 or more shorter sentences.

Muddiness is not merely a disturber of prose, it is also destroyer of life, of hope: death on the highway caused by a badly worded road sign, heartbreak among lovers caused by a misplaced phrase in a well-intentioned letter, anguish of a traveler expecting to be met at a railroad station \& not being met because of a slipshod telegram. Think of the tragedies that are rooted in ambiguity, \& be clear! When you say something, make sure you have said it. The chances of your having said it are only fair.'' -- \cite[p. 95]{Strunk_White_element_style}

%------------------------------------------------------------------------------%

\subsubsection{Do not inject opinion.}
``Unless there is a good reason for its being there, do not inject opinion into a piece of writing. We all have opinions about almost everything, \& the temptation to toss them in is great. To air one's views gratuitously, however, is to imply that the demand for them is brisk, which may not be the case, \& which, in any event, may not be relevant to the discussion. Opinions scattered indiscriminately about leave the mark of egotism on a work. Similarly, to air one's views at an improper time may be in bad taste. If you have received a letter inviting you to speak at the dedication of a new cat hospital, \& you have cats, your reply, declining the invitation, does not necessarily have to cover the full range of your emotions. You must make it clear that you will not attend, but you do not have to let fly at cats. The writer of the letter asked a civil question; attack cats, then, only if you can do so with good humor, good taste, \& in such a way that your answer will be courteous as well as responsive. Since you are out of sympathy with cats, you may quite properly give this as a reason for not appearing at the dedicatory ceremonies of a cat hospital. But bear in mind that your opinion of cats was not sought, only your services as a speaker. Try to keep things straight.'' -- \cite[p. 96]{Strunk_White_element_style}

%------------------------------------------------------------------------------%

\subsubsection{Use figures of speech sparingly.}
``The simile is a common device \& a useful one, but similes coming in rapid fire, one right on top of another, are more distracting than illuminating. Readers need time to catch their breath; they can't be expected to compare everything with something else, \& no relief in sight.

When you use metaphor, do not mix it up. I.e., don't start by calling something a swordfish \& end by calling it an hourglass.'' -- \cite[p. 97]{Strunk_White_element_style}

%------------------------------------------------------------------------------%

\subsubsection{Do not take shortcuts at the cost of clarity.}
``Do not use initials for the names of organizations or movements unless you are certain the initials will be readily understood. Write things out. Not everyone knows that MADD means Mothers Against Drunk Driving, \& even if everyone did, there are babies being born every minute who will someday encounter the name for the 1st time. They deserve to see the words, not simply the initials. A good rule is to start your article by writing out names in full, \& then, later, when your readers have got their bearings, to shorten them.

Many shortcuts are self-defeating; they waste the reader's time instead of conserving it. There are all sorts of rhetorical stratagems \& devices that attract writers who hope to be pithy, but most of them are simply bothersome. The longest way round is usually the shortest home, \& the one truly reliable shortcut in writing is to choose words that are strong \& surefooted to carry readers on their way.'' -- \cite[p. 98]{Strunk_White_element_style}

%------------------------------------------------------------------------------%

\subsubsection{Avoid foreign languages.}
``The writer will occasionally find it convenient or necessary to borrow from other languages. Some writers, however, from sheer exuberance or a desire to show off, sprinkle their work liberally with foreign expressions, with no regard for the reader's comfort. It is a bad habit. Write in English.'' -- \cite[p. 99]{Strunk_White_element_style}

%------------------------------------------------------------------------------%

\subsubsection{Prefer the standard to the offbeat.}
``Young writers will be drawn at every turn toward eccentricities in language. They will hear the beat of new vocabularies, the exciting rhythms of special segments of their society, each speaking a language of its own. All of us come under the spell of these unsettling drums; the problem for beginners is to listen to them, learn the words, feel the vibrations, \& not be carried away.

Youths invariably speak to other youths in a tongue of their own devising: they renovate the language with a wild vigor, as they would a basement apartment. By the time this paragraph sees print, {\it psyched, nerd, ripoff, dude, geek}, \& {\it funky} will be the words of yesteryear, \& we will be fielding more recent ones that have come bouncing into our speech -- some of them into our dictionary as well. A new word is always up for survival. Many do survive. Others grow stale \& disappear. Most are, at least in their infancy, more approximate to conversation than to composition.

Today, the language of advertising enjoys an enormous circulation. With its deliberate infractions of grammatical rules \& its crossbreeding of the parts of speech, it profoundly influences the tongues \& pens of children \& adults. Your new kitchen range is so revolutionary it {\it obsoletes} all other ranges. Your counter top is beautiful because it is {\it accessorized} with gold-plated faucets. Your cigarette tastes good {\it like} a cigarette should. \&, {\it like the man says}, you will want to try one. You will also, in all probability, want to try writing that way, using that language. You do so at your peril, for it is the language of mutilation.

Advertisers are quite understandably interested in what they call ``attention getting.'' The man photographed must have lost an eye or grown a pink beard, or he must have 3 arms or be sitting wrong-end-to on a horse. This technique is proper in its place, which is the world of selling, but the young writer had best not adopt the device of mutilation in ordinary composition, whose purpose is to engage, not paralyze, the readers senses. Buy the gold-plated faucets if you will, but do not accessorize your prose. To use the language well, do not begin by hacking it to bits; accept the whole body of it, cherish its classic form, its variety, \& its richness.

Another segment of society that has constructed a language of its own is business. People in business say that toner cartridges are {\it in short supply}, that they have {\it updated} the next shipment of these cartridges, \& that they will {\it finalize} their recommendations at the next meeting of the board. They are speaking a language familiar \& dear to them. Its portentous nouns \& verbs invest ordinary events with high adventure; executives walk among toner cartridges, caparisoned like knights. We should tolerate them -- every person of spirit wants to ride a white horse. The only question is whether business vocabulary is helpful to ordinary prose. Usually, the same ideas can be expressed less formidably, if one makes the effort. A good many of the special words of business seem designed more to express the user's dreams than to express a precise meaning. Not all such words, of course, can be dismissed summarily; indeed, no word in the language can be dismissed offhand by anyone who has a healthy curiosity. {\it Update} isn't a bad word; in the right setting it is useful. In the wrong setting, though, it is destructive, \& the trouble with adopting coinages too quickly is that they will bedevil one by insinuating themselves where they do not belong. This may sound like rhetorical snobbery, or plain stuffiness; but you will discover, in the course of your work, that the setting of a word is just as restrictive as the setting of a jewel. The general rule here is to prefer the standard. {\it Finalize}, for instance, is not standard; it is special, \& it is a peculiarly fuzzy \& silly word. Does it mean ``terminate,'' or does it mean ``put into final form''? One can't be sure, really, what it means, \& one gets the impression that the person using it doesn't know, either, \& doesn't want to know.

The special vocabularies of the law, of the military, of government are familiar to most of us. Even the world of criticism has a modest pouch of private words ({\it luminous, taut}), whose only virtue is that they are exceptionally nimble \& can escape from the garden of meaning over the wall. Of these critical words, Wilcott Gibbs once wrote, ``$\ldots$ they are detached from the language \& inflated like little balloons.'' The young writer should learn to spot them -- words that at 1st glance seem freighted with delicious meaning but that soon burst in air, leaving nothing but a memory of bright sound.

The language is perpetually in flux: it is a living stream, shifting, changing, receiving new strength from a thousand tributaries, losing old forms in the backwaters of time. To suggest that a young writer not swim in the main stream of this turbulence would be foolish indeed, \& such is not the intent of these cautionary remarks. The intent is to suggest that in choosing between the formal \& the informal, the regular \& the offbeat, the general \& the special, the orthodox \& the heretical, the beginner err on the side of conservatism, on the side of established usage. No idiom is taboo, no accent forbidden; there is simply a better chance of doing well if the writer holds a steady course, enters the stream of English quietly, \& does not thrash about.

``But,'' you may ask, ``what if it comes natural to me to experiment rather than conform? What if I am a pioneer, or even a genius?'' Answer: then be one. But do not forget that what may seem like pioneering may be merely evasion, or laziness -- the disinclination to submit to discipline. Writing good standard English is no cinch, \& before you have managed it you will have encountered enough rough country to satisfy even the most adventurous spirit.

Style takes its final shape more from attitudes of mind than from principles of composition, for, as an elderly practitioner once remarked, ``Writing is an act of faith, not a trick of grammar.'' This moral observation would have no place in a rule book were it not that style {\it is} the writer, \& therefore what you are, rather than what you know, will at last determine your style. If you write, you must believe -- in the truth \& worth of the scrawl, in the ability of the reader to receive \& decode the message. No one can write decently who is distrustful of the reader's intelligence, or whose attitude is patronizing.

Many references have been made in this book to ``the reader,'' who has been much in the news. It is now necessary to warn you that you concern for the reader must be pure: you must sympathize with the reader's plight (most readers are in trouble about half the time) but never seek to know the reader's wants. Your whole duty as a writer is to please \& satisfy yourself, \& the true writer always plays to an audience of one. Start sniffing the air, or glancing at the Trend Machine, \& you are as good as dead, although you may make a nice living.

Full of belief, sustained \& elevated by the power of purpose, armed with the rules of grammar, you are ready to exposure. At this point, you may well pattern yourself on the fully exposed cow of Robert Louis Stevenson's rhyme. This friendly \& commendable animal, you may recall, was ``blown by all the winds that pass{\tt/}\& wet with all the showers.'' \& so must you as a young writer be. In our modern idiom, we should say that you must get wet all over. Mr. Stevenson, working in a plainer style, said it with felicity, \& suddenly 1 cow, out of so many, received the gift of immortality. Like the steadfast writer, she is at home in the wind \& the rain; \&, thanks to 1 moment of felicity, she will live on \& on \& one.'' -- \cite[pp. 101--103]{Strunk_White_element_style}

%------------------------------------------------------------------------------%

\subsection{Afterword}
``Will Strunk \& E. B. White were unique collaborators. Unlike Gilbert \& Sullivan, or Woodward \& Bernstein, they worked separately \& decades apart.

We have no way of knowing whether Prof. Strunk took particular notice of Elwyn Brooks White, a student of his at Cornell University in 1919. Neither teacher nor pupil could have realized that their names would be linked as they now are. Nor could they have imagined that 38 years after they met, White would take this little gem of a textbook that Strunk had written for his students, polish it, expand it, \& transform it into a classic.

E. B. White shared Strunk's sympathy for the reader. To Strunk's do's \& don'ts he added passages about the power of words \& the clear expression of thoughts \& feelings. To the nuts \& bolts of grammar he added a rhetorical dimension.

The editors of this edition have followed in White's footsteps, once again providing fresh examples \& modernizing usage where appropriate. {\it The Elements of Style} is still a little book, small enough \& important enough to carry in your pocket, as I carry mine. It has helped me to write better. I believe it can do the same for you.

\textsc{Charles Osgood}'' -- \cite[p. 104]{Strunk_White_element_style}

%------------------------------------------------------------------------------%

\subsection{Glossary}

\begin{itemize}
	\item {\bf adjectival modifier:} A word, phrase, or clause that acts as an adjective in qualifying the meaning of a noun or pronoun, {\it Your} country; a {\it turn-of-the-century} style; people {\it who are always late}.
	\item {\bf adjective:} A word that modifies, quantifies, or otherwise describes a noun or pronoun. {\it Drizzly} November; midnight {\it dreary}; {\it only} requirement.
	\item {\bf adverb:} A word that modifies or otherwise qualifies a verb, an adjective, or another adverb. Gestures {\it gracefully}; {\it exceptionally} quiet engine.
	\item $\ldots$
\end{itemize}

%------------------------------------------------------------------------------%

\section{{\sc William Zinsser}. On Well Writing}

%------------------------------------------------------------------------------%

%------------------------------------------------------------------------------%

%------------------------------------------------------------------------------%

%------------------------------------------------------------------------------%

%------------------------------------------------------------------------------%

\section{Miscellaneous}

%------------------------------------------------------------------------------%

\printbibliography[heading=bibintoc]
	
\end{document}