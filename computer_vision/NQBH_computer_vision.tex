\documentclass{article}
\usepackage[backend=biber,natbib=true,style=alphabetic,maxbibnames=50]{biblatex}
\addbibresource{/home/nqbh/reference/bib.bib}
\usepackage[utf8]{vietnam}
\usepackage{tocloft}
\renewcommand{\cftsecleader}{\cftdotfill{\cftdotsep}}
\usepackage[colorlinks=true,linkcolor=blue,urlcolor=red,citecolor=magenta]{hyperref}
\usepackage{amsmath,amssymb,amsthm,enumitem,float,graphicx,mathtools,tikz}
\usetikzlibrary{angles,calc,intersections,matrix,patterns,quotes,shadings}
\allowdisplaybreaks
\newtheorem{assumption}{Assumption}
\newtheorem{baitoan}{}
\newtheorem{cauhoi}{Câu hỏi}
\newtheorem{conjecture}{Conjecture}
\newtheorem{corollary}{Corollary}
\newtheorem{dangtoan}{Dạng toán}
\newtheorem{definition}{Definition}
\newtheorem{dinhly}{Định lý}
\newtheorem{dinhnghia}{Định nghĩa}
\newtheorem{example}{Example}
\newtheorem{ghichu}{Ghi chú}
\newtheorem{hequa}{Hệ quả}
\newtheorem{hypothesis}{Hypothesis}
\newtheorem{lemma}{Lemma}
\newtheorem{luuy}{Lưu ý}
\newtheorem{nhanxet}{Nhận xét}
\newtheorem{notation}{Notation}
\newtheorem{note}{Note}
\newtheorem{principle}{Principle}
\newtheorem{problem}{Problem}
\newtheorem{proposition}{Proposition}
\newtheorem{question}{Question}
\newtheorem{remark}{Remark}
\newtheorem{theorem}{Theorem}
\newtheorem{vidu}{Ví dụ}
\usepackage[left=1cm,right=1cm,top=5mm,bottom=5mm,footskip=4mm]{geometry}
\def\labelitemii{$\circ$}
\DeclareRobustCommand{\divby}{%
	\mathrel{\vbox{\baselineskip.65ex\lineskiplimit0pt\hbox{.}\hbox{.}\hbox{.}}}%
}
\setlist[itemize]{leftmargin=*}
\setlist[enumerate]{leftmargin=*}

\title{Computer Vision -- Thị Giác Máy Tính}
\author{Nguyễn Quản Bá Hồng\footnote{A Scientist {\it\&} Creative Artist Wannabe. E-mail: {\tt nguyenquanbahong@gmail.com}. Bến Tre City, Việt Nam.}}
\date{\today}

\begin{document}
\maketitle
\begin{abstract}
	This text is a part of the series {\it Some Topics in Advanced STEM \& Beyond}:
	
	{\sc url}: \url{https://nqbh.github.io/advanced_STEM/}.
	
	Latest version:
	\begin{itemize}
		\item {\it Computer Vision -- Thị Giác Máy Tính}.
		
		PDF: {\sc url}: \url{https://github.com/NQBH/advanced_STEM_beyond/blob/main/computer_vision/NQBH_computer_vision.pdf}.
		
		\TeX: {\sc url}: \url{https://github.com/NQBH/advanced_STEM_beyond/blob/main/computer_vision/NQBH_computer_vision.tex}.
	\end{itemize}
\end{abstract}
\tableofcontents

%------------------------------------------------------------------------------%

\section{Basic Computer Vision}

%------------------------------------------------------------------------------%

\section{Wikipedia}

\subsection{Wikipedia{\tt/}computer vision}
``{\it Computer vision} tasks include methods for \href{https://en.wikipedia.org/wiki/Image_sensor}{acquiring}, \href{https://en.wikipedia.org/wiki/Image_processing}{processing}, \href{https://en.wikipedia.org/wiki/Image_analysis}{analyzing}, \& understanding \href{https://en.wikipedia.org/wiki/Digital_image}{digital images}, \& extraction of \href{https://en.wikipedia.org/wiki/High-dimensional}{high-dimensional} data from real world in order to produce numerical or symbolic information, e.g., in form of decisions. ``Understanding'' in this context signifies transformation of visual images (input to \href{https://en.wikipedia.org/wiki/Retina}{retina}) into descriptions of world that make sense to thought processes \& can elicit appropriate action. This image understanding can be seen as disentangling of symbolic information from image data using models constructed with aid of geometry, physics, statistics, \& learning theory.

\href{https://en.wikipedia.org/wiki/Scientific_discipline}{Scientific discipline} of computer vision is concerned with theory behind artificial systems that extract information from images. Image data can take many forms, e.g. video sequences, views from multiple cameras, multi-dimensional data from a 3D scanner, 3D point clouds from LiDaR sensors, or medical scanning devices. Technological discipline of computer vision seeks to apply its theories \& models to construction of computer vision systems.

Subdisciplines of computer vision include \href{https://en.wikipedia.org/wiki/3D_reconstruction}{scene recontruction}, \href{https://en.wikipedia.org/wiki/Object_detection}{object detection}, \href{https://en.wikipedia.org/wiki/Event_detection}{event detection}, \href{https://en.wikipedia.org/wiki/Activity_recognition}{activity recognition}, \href{https://en.wikipedia.org/wiki/Video_tracking}{video tracking}, \href{https://en.wikipedia.org/wiki/Object_recognition}{object recognition}, \href{https://en.wikipedia.org/wiki/3D_pose_estimation}{3D pose estimation}, learning, indexing, \href{https://en.wikipedia.org/wiki/Motion_estimation}{motion estimation}, \href{https://en.wikipedia.org/wiki/Visual_servoing}{visual servoing}, 3D scene modeling, \& \href{https://en.wikipedia.org/wiki/Digital_photograph_restoration}{image restoration}.

\subsubsection{Definition}

\subsubsection{History}

\subsubsection{Related fields}

\subsubsection{Applications}

\subsubsection{Typical tasks}

\subsubsection{System methods}

\subsubsection{Hardware}

'' -- \href{https://en.wikipedia.org/wiki/Computer_vision}{Wikipedia{\tt/}computer vision}

%------------------------------------------------------------------------------%

\section{Miscellaneous}

%------------------------------------------------------------------------------%

\printbibliography[heading=bibintoc]
	
\end{document}