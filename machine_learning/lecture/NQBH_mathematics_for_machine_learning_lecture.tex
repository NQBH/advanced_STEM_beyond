\documentclass{article}
\usepackage[backend=biber,natbib=true,style=alphabetic,maxbibnames=50]{biblatex}
\addbibresource{/home/nqbh/reference/bib.bib}
\usepackage[utf8]{vietnam}
\usepackage{tocloft}
\renewcommand{\cftsecleader}{\cftdotfill{\cftdotsep}}
\usepackage[colorlinks=true,linkcolor=blue,urlcolor=red,citecolor=magenta]{hyperref}
\usepackage{amsmath,amssymb,amsthm,enumitem,float,graphicx,mathtools,tikz}
\usetikzlibrary{angles,calc,intersections,matrix,patterns,quotes,shadings}
\allowdisplaybreaks
\newtheorem{assumption}{Assumption}
\newtheorem{baitoan}{Bài toán}
\newtheorem{cauhoi}{Câu hỏi}
\newtheorem{conjecture}{Conjecture}
\newtheorem{corollary}{Corollary}
\newtheorem{dangtoan}{Dạng toán}
\newtheorem{definition}{Definition}
\newtheorem{dinhly}{Định lý}
\newtheorem{dinhnghia}{Định nghĩa}
\newtheorem{example}{Example}
\newtheorem{ghichu}{Ghi chú}
\newtheorem{hequa}{Hệ quả}
\newtheorem{hypothesis}{Hypothesis}
\newtheorem{lemma}{Lemma}
\newtheorem{luuy}{Lưu ý}
\newtheorem{nhanxet}{Nhận xét}
\newtheorem{notation}{Notation}
\newtheorem{note}{Note}
\newtheorem{principle}{Principle}
\newtheorem{problem}{Problem}
\newtheorem{proposition}{Proposition}
\newtheorem{question}{Question}
\newtheorem{remark}{Remark}
\newtheorem{theorem}{Theorem}
\newtheorem{vidu}{Ví dụ}
\usepackage[left=1cm,right=1cm,top=5mm,bottom=5mm,footskip=4mm]{geometry}
\def\labelitemii{$\circ$}
\DeclareRobustCommand{\divby}{%
	\mathrel{\vbox{\baselineskip.65ex\lineskiplimit0pt\hbox{.}\hbox{.}\hbox{.}}}%
}
\setlist[itemize]{leftmargin=*}
\setlist[enumerate]{leftmargin=*}

\title{Lecture: Mathematics for Machine Learning}
\author{Nguyễn Quản Bá Hồng\footnote{A Scientist {\it\&} Creative Artist Wannabe. E-mail: {\tt nguyenquanbahong@gmail.com}. Bến Tre City, Việt Nam.}}
\date{\today}

\begin{document}
\maketitle
\begin{abstract}
	This text is a part of the series {\it Some Topics in Advanced STEM \& Beyond}:
	
	{\sc url}: \url{https://nqbh.github.io/advanced_STEM/}.
	
	Latest version:
	\begin{itemize}
		\item {\it }.
		
		PDF: {\sc url}: \url{.pdf}.
		
		\TeX: {\sc url}: \url{.tex}.
		\item {\it }.
		
		PDF: {\sc url}: \url{.pdf}.
		
		\TeX: {\sc url}: \url{.tex}.
	\end{itemize}
\end{abstract}
\tableofcontents

%------------------------------------------------------------------------------%

\section{Linear Regression -- Hồi Quy Tuyến Tính}
\textbf{\textsf{References -- Tài nguyên.}}
\begin{enumerate}
	\item \cite{Bach2024}. {\sc Francis Bach}. {\it Learning Theory from First Principles}. Chap. 3: Linear Least-Squares Regression.
	\item \cite{Deisenroth_Faisal_Ong2023}. {\sc Marc Peter Deisenroth, A. Aldo Faisal, Cheng Soon Ong}. {\it Mathematics for Machine Learning}. Chap. 9: Linear Regression.
	\item {\sc Andrew Ng}'s Machine Learning Specialization slides: \url{}.
	\item \cite{Tiep_ML_co_ban}. {\sc Vũ Hữu Tiệp}. {\it Machine Learning Cơ Bản}. Chap. 7: Hồi Quy Tuyến Tính.
	\item Wikipedia: \href{https://en.wikipedia.org/wiki/Linear_function}{Wikipedia{\tt/}linear function}. \href{https://en.wikipedia.org/wiki/Linear_regression}{Wikipedia{\tt/}linear regression}.
\end{enumerate}

\subsection{Introduction to linear regression}
{\it1st impression.} Hồi quy tuyến tính (linear regression) là:
\begin{itemize}
	\item 1 thuật toán hồi quy mà đầu ra là 1 hàm số tuyến tính (linear function, i.e., $y = ax + b$, $a,b\in\mathbb{R}$, $a\ne0$) của đầu vào. A brief notation:
	\begin{equation*}
		{\rm output} = {\tt linear\_function}({\rm input}) = a{\rm input} + b.
	\end{equation*}
	\item Thuật toán đơn giản nhất trong nhóm các thuật toán học có giám sát (supervised learning algorithms).
\end{itemize}

\begin{problem}[Housing prices -- giá cả nhà đất{\tt/}bất động sản]
	Let $m\in\mathbb{N}^\star$. Suppose we have a 2-column table whose the 1st column consists of sizes of $m$ houses: ${\bf x} = (x_1,x_2,\ldots,x_m)$ \& the 2nd one consists of their corresponding housing prices ${\bf y} = (y_1,y_2,\ldots,y_m)$. Here $m$ is the number of training examples, ${\bf x}$ is the vector of input variable or features (size), \& ${\bf y}$ is the vector of output variables or target variables (price).
\end{problem}

\begin{baitoan}[Ước lượng giá nhà, \cite{Tiep_ML_co_ban}, Sect. 7.1, p. 100]
	Xét bài toán ước lượng giá của 1 căn phòng rộng $x_1\ {\rm m}^2$, có $x_2$ phòng ngủ, \& cách trung tâm thành phố $x_3$ km. Giả sử có 1 tập dữ diệu của $N = 10000$ căn nhà trong thành phố đó. Liệu có thể dự đoán được giá $y$ của 1 căn nhà mới (i.e., khác $N$ căn nhà đã có dữ liệu) thông qua 3 thông số về diện tích $x_1\in(0,\infty)$, số phòng ngủ $x_2\in\mathbb{N}$, \& khoảng cách tới trung tâm thành phố $x_3\in[0,\infty)$?
\end{baitoan}
{\sf Mathematical notations -- Ký hiệu toán học.} Đặt ${\bf x} = [x_1,x_2,x_3]\in(0,\infty)\times\mathbb{N}\times[0,\infty)$ là 1 vector cột chứa dữ liệu đầu vào (inputs) gồm 3 thông số diện tích $x_1$, số phòng ngủ $x_2$, \& khoảng cách đến trung tâm thành phố $x_3$; ${\bf x}$ được gọi là {\it vector đặc trưng}. Đặt giá nhà (đầu ra{\tt/}output) $y\in(0,\infty)$.

{\sf Reasoning -- Biện luận.} Nếu diện tích nhà càng lớn thì giá nhà càng cao: $x_1\uparrow\Rightarrow y\uparrow$, nếu nhà có càng nhiều phòng ngủ thì giá nhà càng cao: $x_2\uparrow\Rightarrow y\uparrow$, \& nếu nhà càng ở gần trung tâm thành phố thì giá nhà càng cao: $x_3\downarrow0\Rightarrow y\uparrow$. Có thể sử dụng mô hình đầu ra là 1 hàm tuyến tính đơn giản của đầu vào:
\begin{equation}
	\label{linear relation}
	\tag{lnr}
	y\approx\hat{y}\coloneqq f({\bf x}) = w_1x_1 + w_2x_2 + w_3x_3 = \sum_{i=1}^3 w_ix_i = {\bf x}^\top{\bf w} = {\bf x}\cdot{\bf w} = \langle{\bf x},{\bf w}\rangle,
\end{equation}
với ${\bf w} = [w_1,w_2,w_2]^\top$ là {\it vector trọng số} (weight vector) với $w_1,w_2\in(0,\infty),w_3\in(-\infty,0)$ (why?) cần tìm. Mối quan hệ cho bởi \eqref{linear relation} là 1 mối quan hệ tuyến tính.

Về hàm tuyến tính, see, e.g., \href{https://en.wikipedia.org/wiki/Linear_function}{Wikipedia{\tt/}linear function}.

%------------------------------------------------------------------------------%

\section{Overfitting -- Quá Khớp}

%------------------------------------------------------------------------------%

\section{$K$ Neighbors -- $K$ Lân Cận}

%------------------------------------------------------------------------------%

\section{Phân Cụm $K$-Means}

%------------------------------------------------------------------------------%

\section{Artificial Neural Networks (ANNs) -- Mạng Neuron Nhân Tạo}

%------------------------------------------------------------------------------%

\section{Perception Learning Algorithm -- Thuật Toán Học Perceptron}

%------------------------------------------------------------------------------%

\section{Logistic Regression -- Hồi Quy Logistic}

%------------------------------------------------------------------------------%

\section{Softmax Regression -- Hồi Quy Softmax}

%------------------------------------------------------------------------------%

\section{Deep Neural Networks \& Backpropagation -- Mạng Neuron Đa Tầng \& Lan Truyền Ngược}

%------------------------------------------------------------------------------%

\section{Miscellaneous}

%------------------------------------------------------------------------------%

\printbibliography[heading=bibintoc]
	
\end{document}