\documentclass{article}
\usepackage[backend=biber,natbib=true,style=alphabetic,maxbibnames=50]{biblatex}
\addbibresource{/home/nqbh/reference/bib.bib}
\usepackage[utf8]{vietnam}
\usepackage{tocloft}
\renewcommand{\cftsecleader}{\cftdotfill{\cftdotsep}}
\usepackage[colorlinks=true,linkcolor=blue,urlcolor=red,citecolor=magenta]{hyperref}
\usepackage{amsmath,amssymb,amsthm,enumitem,float,graphicx,mathtools,tikz}
\usetikzlibrary{angles,calc,intersections,matrix,patterns,quotes,shadings}
\allowdisplaybreaks
\newtheorem{assumption}{Assumption}
\newtheorem{baitoan}{}
\newtheorem{cauhoi}{Câu hỏi}
\newtheorem{conjecture}{Conjecture}
\newtheorem{corollary}{Corollary}
\newtheorem{dangtoan}{Dạng toán}
\newtheorem{definition}{Definition}
\newtheorem{dinhly}{Định lý}
\newtheorem{dinhnghia}{Định nghĩa}
\newtheorem{example}{Example}
\newtheorem{ghichu}{Ghi chú}
\newtheorem{hequa}{Hệ quả}
\newtheorem{hypothesis}{Hypothesis}
\newtheorem{lemma}{Lemma}
\newtheorem{luuy}{Lưu ý}
\newtheorem{nhanxet}{Nhận xét}
\newtheorem{notation}{Notation}
\newtheorem{note}{Note}
\newtheorem{principle}{Principle}
\newtheorem{problem}{Problem}
\newtheorem{proposition}{Proposition}
\newtheorem{question}{Question}
\newtheorem{remark}{Remark}
\newtheorem{theorem}{Theorem}
\newtheorem{vidu}{Ví dụ}
\usepackage[left=1cm,right=1cm,top=5mm,bottom=5mm,footskip=4mm]{geometry}
\def\labelitemii{$\circ$}
\DeclareRobustCommand{\divby}{%
	\mathrel{\vbox{\baselineskip.65ex\lineskiplimit0pt\hbox{.}\hbox{.}\hbox{.}}}%
}
\setlist[itemize]{leftmargin=*}
\setlist[enumerate]{leftmargin=*}

\title{Neuroscience -- Khoa Học Thần Kinh}
\author{Nguyễn Quản Bá Hồng\footnote{A Scientist {\it\&} Creative Artist Wannabe. E-mail: {\tt nguyenquanbahong@gmail.com}. Bến Tre City, Việt Nam.}}
\date{\today}

\begin{document}
\maketitle
\begin{abstract}
	This text is a part of the series {\it Some Topics in Advanced STEM \& Beyond}:
	
	{\sc url}: \url{https://nqbh.github.io/advanced_STEM/}.
	
	Latest version:
	\begin{itemize}
		\item {\it Neuroscience -- Khoa Học Thần Kinh}.
		
		PDF: {\sc url}: \url{https://github.com/NQBH/advanced_STEM_beyond/blob/main/neuroscience/NQBH_neuroscience.pdf}.
		
		\TeX: {\sc url}: \url{https://github.com/NQBH/advanced_STEM_beyond/blob/main/neuroscience/NQBH_neuroscience.tex}.
	\end{itemize}
\end{abstract}
\tableofcontents

%------------------------------------------------------------------------------%

\section{Wikipedia{\tt/}Neuroscience}
``{\it Neuroscience} is the \href{https://en.wikipedia.org/wiki/Science}{scientific study} of the \href{https://en.wikipedia.org/wiki/Nervous_system}{nervous system} (the \href{https://en.wikipedia.org/wiki/Brain}{brain}, \href{https://en.wikipedia.org/wiki/Spinal_cord}{spinal cord}, \& \href{https://en.wikipedia.org/wiki/Peripheral_nervous_system}{peripheral nervous system}), its functions, \& its disorders. It is a \href{https://en.wikipedia.org/wiki/Multidisciplinary_approach}{multidisciplinary} science that combines \href{https://en.wikipedia.org/wiki/Physiology}{physiology}, \href{https://en.wikipedia.org/wiki/Anatomy}{anatomy}, \href{https://en.wikipedia.org/wiki/Molecular_biology}{molecular biology}, \href{https://en.wikipedia.org/wiki/Developmental_biology}{developmental biology}, \href{https://en.wikipedia.org/wiki/Cytology}{cytology}, psychology, physics, computer science, chemistry, medicine, statistics, \& \href{https://en.wikipedia.org/wiki/Mathematical_Modeling}{mathematical modeling} to understand the fundamental \& emergent properties of \href{https://en.wikipedia.org/wiki/Neuron}{neurons}, \href{https://en.wikipedia.org/wiki/Glia}{glia}, \& \href{https://en.wikipedia.org/wiki/Neural_circuit}{neural circuits}. The understanding of the biological basis of \href{https://en.wikipedia.org/wiki/Learning}{learning}, \href{https://en.wikipedia.org/wiki/Memory}{memory}, \href{https://en.wikipedia.org/wiki/Behavior}{behavior}, \href{https://en.wikipedia.org/wiki/Perception}{perception}, \& \href{https://en.wikipedia.org/wiki/Consciousness}{consciousness} has been described by \href{https://en.wikipedia.org/wiki/Eric_Kandel}{Eric Kandel} as the ``epic challenge'' of the \href{https://en.wikipedia.org/wiki/Biology}{biological sciences}.

The scope of neuroscience has broadened over time to include different approaches used to study the nervous system at different scales. The techniques used by \href{https://en.wikipedia.org/wiki/Neuroscientist}{neuroscientists} have expanded enormously, from molecular \& \href{https://en.wikipedia.org/wiki/Cell_biology}{cellular} studies of individual neurons to \href{https://en.wikipedia.org/wiki/Neuroimaging}{imaging} of \href{https://en.wikipedia.org/wiki/Sensory_neuron}{sensory}, \href{https://en.wikipedia.org/wiki/Motor_neuron}{motor}, \& \href{https://en.wikipedia.org/wiki/Cognition}{cognitive} tasks in the brain.

\subsection{History}
See \href{https://en.wikipedia.org/wiki/History_of_neuroscience}{Wikipedia{\tt/}history of neuroscience}.

\subsection{Modern neuroscience}
See \href{https://en.wikipedia.org/wiki/Outline_of_neuroscience}{Wikipedia{\tt/}outline of neuroscience}.

\subsection{Major branches}

\subsection{Careers in neuroscience}

\subsection{Neuroscience organizations}

\subsection{Engineering applications of neuroscience}

\subsection{Nobel prizes related to neuroscience}

%------------------------------------------------------------------------------%

\section{Miscellaneous}

%------------------------------------------------------------------------------%

\printbibliography[heading=bibintoc]
	
\end{document}