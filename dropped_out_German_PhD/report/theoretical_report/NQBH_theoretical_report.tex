\documentclass{book}
\usepackage[backend=biber,natbib=true,style=authoryear]{biblatex}
\addbibresource{/home/nguyen/1_NQBH/math/bib.bib}
\usepackage[utf8]{inputenc}
\usepackage{graphicx}
\usepackage[colorlinks=true,linkcolor=blue,urlcolor=red,citecolor=magenta]{hyperref}
\usepackage{amsmath,amssymb,amsthm}
\allowdisplaybreaks
\usepackage{tcolorbox}
\numberwithin{equation}{section}
\newtheorem{assumption}{Assumption}[section]
\newtheorem{lemma}{Lemma}[section]
\newtheorem{corollary}{Corollary}[section]
\newtheorem{definition}{Definition}[section]
\newtheorem{proposition}{Proposition}[section]
\newtheorem{theorem}{Theorem}[section]
\newtheorem{notation}{Notation}[section]
\newtheorem{remark}{Remark}[section]
\newtheorem{example}{Example}[section]
\newtheorem{ques}{Question}[section]
\newtheorem{problem}{Problem}[section]
\newtheorem{conjecture}{Conjecture}[section]
\usepackage[left=0.5in,right=0.5in,top=1.5cm,bottom=1.5cm]{geometry}
\usepackage{fancyhdr}
\pagestyle{fancy}
\fancyhf{}
\addtolength{\headheight}{0pt}% obsolete
\lhead{\scriptsize \chaptername~\thechapter}
\rhead{\scriptsize \nouppercase{\leftmark}} %\nouppercase !
\renewcommand{\chaptermark}[1]{\markboth{#1}{}}
\cfoot{\thepage}
\def\labelitemii{$\circ$}

\title{Theoretical Report}
\author{Nguyen Quan Ba Hong}
\date{\today}

\begin{document}
\maketitle
\setcounter{secnumdepth}{6}
\setcounter{tocdepth}{6}
\tableofcontents

%------------------------------------------------------------------------------%

\chapter{Ingredients}

\begin{remark}[Purpose]
    In this theoretical report, I have been trying to gather all the necessary/relevant ingredients for the big picture because the nature of the project
    \begin{tcolorbox}
        \href{https://www.romsoc.eu/optimal-shape-design-of-air-ducts-in-combustion-engines/}{\textsc{ROMSOC $\triangleright$ Project 11: Optimal Shape Design of Air Ducts in Combustion Engines}}
    \end{tcolorbox}
    is to couple 2 big fields: shape/topology optimization with PDEs/turbulence models in fluid dynamics/Computational Fluid Dynamics (CFD), and also because:
    \begin{itemize}
        \item ``\emph{A unique characteristic of shape optimization is that it makes bonds from different areas of mathematics, such as differential geometry, Riemmanian geometry, real and complex analysis, partial differential equations, topology and set theory}.'' - Kevin Sturm, \cite{Sturm2015}
        \item ``\emph{The scope of research is frighteningly broad because it touches on areas that include classical geometry, modern partial differential equations, geometric measure theory, topological groups, and constrained optimization, with applications to classical mechanics of continuous media such as fluid mechanics, elasticity theory, fracture theory, modern theories of optimal design, optimal location and shape of geometric objects, free and moving boundary problems, and image processing.}'' - \cite{Delfour_Zolesio2011}
    \end{itemize}
    Thus, I am aware enough to know that I should  not, and cannot, attempt to attack all the possible combinations/branches of this ROMSOC project in my PhD.
    
    Instead, what I have been trying to achieve in this theoretical report is to design a general\footnote{For me, personally, 1 of the most beautiful perfections I always want to pursue is generality.} enough framework or to draw a big and clear enough picture for a long run instead of 3-year (at least) duration of a PhD. This framework can/will be kept developing/extending/upgrading if I still work on this field after my PhD (provided I can graduate) or can be transferred easily to another person who was/is/will be interested in this project.
\end{remark}

\begin{remark}[Principle(s)]
    To obtain the most generality, I think I need/have to separate them 1st, then push each of them as far/general as I can. After, or simultaneously with that, I will try to combine them together into the most general combination which I can, also obtaining partial results along the road up there.
\end{remark}

\section*{Minimap}
Now I will list all relevant fields/areas which I have collected for this project.

Here is a \textit{mini-map} of this chapter:

\begin{tcolorbox}
    \textsc{PDEs.} Consider both classical and modern ones:
    \begin{enumerate}
        \item \textbf{Classic PDEs.}
        
        $\to$ \cite{Sokolowski_Zolesio1992} dealt rigorously with classical linear PDEs and \textit{variational inequalities}. It seems to me that shape optimization for semilinear/quasilinear is promising.
        \item \textbf{PDEs in fluid mechanics/fluid mechanics $\triangleright$ fluid dynamics/fluid mechanics $\triangleright$ Computational Fluid Dynamics (CFD).}\footnote{The symbol $\triangleright$ will be used to indicate sub-field/sub-area relation in this text.}
        
        $\to$ a \textit{vast} and \textit{overwhelming} world in mathematical analysis/numerical analysis/physical/engineering perspectives.
        \item \textbf{Turbulence $\triangleright$ Turbulence modeling.}
        
        $\to$ a \textit{wicked sick} world, with full of open problems in both mathematics and physics, in mathematical analysis/numerical analysis/physical/engineering perspectives. \href{https://en.wikipedia.org/wiki/Richard_Feynman}{Richard Feynman} has described turbulence as the most important unsolved problem in classical physics.
        
        $\star$ \textbf{Unsolved problem in physics:}
        \begin{quotation}
            ``\textit{Is it possible to make a theoretical model to describe the behavior of a turbulent flow - in particular, its internal structures?}''
        \end{quotation}        
    \end{enumerate}
\end{tcolorbox}

\begin{tcolorbox}
    \textsc{Numerical analysis.}
    \begin{enumerate}
        \item \textbf{FDM.} $\to$ perhaps for completeness only, but engineers always use FDM as a numerical reference result to test their methods.
        \item \textbf{FEM.} $\to$ important, since a lot of cutting-edge closed- and open-source for shape optimization are FEM-based. By \href{https://en.wikipedia.org/wiki/Computational_fluid_dynamics#Finite_element_method}{Wikipedia/CFD/FEM},
        
        \begin{quotation}
            ``\textit{The finite element method (FEM) is used in structural analysis of solids, but is also applicable to fluids. However, the FEM formulation requires special care to ensure a conservative solution. The FEM formulation has been adapted for use with fluid dynamics governing equations. Although FEM must be carefully formulated to be conservative, it is much more stable than the finite volume approach. However, FEM can require more memory and has slower solution times than the FVM.}''
        \end{quotation}
        \item \textbf{FVM.} $\to$ most important, since we are dealing with PDEs/conservation laws in fluid mechanics/dynamics, not elliptic/parabolic PDEs.
        \item \textbf{Miscellaneous.} $\to$ see, e.g., \href{https://en.wikipedia.org/wiki/Computational_fluid_dynamics}{Wikipedia/CFD}.
    \end{enumerate}
\end{tcolorbox}

\begin{tcolorbox}
    \textsc{Mathematical optimization.}
    
    In the context of this project, we will consider mostly PDEs-constrained optimization.
    \begin{enumerate}
        \item \textbf{Optimal control.} $\to$ needed for the Lagrangian framework and the perspective of calculus of variations.
        \item \textbf{Size optimization.} $\to$ for the sake of completeness only, see e.g. \cite{Haslinger_Makinen2003}.
        \item \textbf{Topology optimization.} $\to$ necessary to obtain a good enough initial geometry before conducting shape optimization, see \cite{Hintermueller_Laurain2007} for a general optimization process of structural/topological optimization.
        \item \textbf{Shape optimization.} $\to$ the \textit{heart} of this project.
    \end{enumerate}
\end{tcolorbox}

\begin{tcolorbox}
    \textsc{Miscellaneous (add-ons).}
    \begin{enumerate}
        \item \textbf{Classic geometry.} $\to$ obvious because of the term ``shape''.
        \item \textbf{Set theory.} $\to$ important to categorize different types of shapes/domains, (see, e.g., \cite{Delfour_Zolesio2011}), e.g.:
        \begin{itemize}
            \item \textit{$C^k$/$C^\infty$/$C^{k,l}$-domains}
            
            $\to$ See e.g., \cite{Delfour_Zolesio2011,Sokolowski_Zolesio1992}
            \item \textit{Nonsmooth domain} (e.g., domains with corners)
            
            $\to$ See \cite{Grisvard1985}, but this text is mostly used for 2D, not 3D.
            \item \textit{Polyhedral domains} (also, \textit{domains of polyhedral type}) 
            
            $\to$ This perfectly matches the discretization/triangulation of the domain in FVM. See e.g., \cite{Mazya_Rossmann2007,Mazya_Rossmann2009,Mazya_Rossmann2010}.
            
            \begin{remark}
                Shape optimization for polyhedral domains seems advanced (e.g., weighted Sobolev spaces and their associated embedding results) to me in PhD. Hence, this should be treated later.
            \end{remark}
        \end{itemize}
        \item \textbf{Differential geometry.}
        
        $\to$ definitely an advanced area, but critical to push, towards the perfection, the treatment of the boundary of the domains needing optimizing, i.e., for \textit{tangential calculus} in shape optimization when dealing with boundary-type cost functionals or using boundary formulas for both boundary-/domain-type cost functionals.
        \item \textbf{Riemannian geometry.} $\to$ only for completeness, sound hard to me.
        \item \textbf{Real \& complex analysis.} $\to$ I do not fully understand what this means, only knowing about the notions of Eulerian/Gateaux/Fr\'echet (semi-)differentiability of cost functionals.
        \item \textbf{Topological groups.} $\to$ I am not sure how important this is, but \cite{Delfour_Zolesio2011} mentioned these in some 1st chapters for a general framework.
        \item \textbf{Geometric measure theory.} $\to$ advanced, but critical to treat nonsmooth domains, e.g., domains with corners, singularities, or even just Lebesgue-measurable domains. $\to$ for post-PhD perhaps.
    \end{enumerate}
\end{tcolorbox}

\begin{tcolorbox}
    \textsc{Applications.} (many in the literature and several in the promising future)
    
    In general, for classical mechanics of continuous media, see, e.g., \cite{Delfour_Zolesio2011,Sturm2015}.
    \begin{enumerate}
        \item \textbf{Fluid mechanics.}
        \item \textbf{Elasticity theory.} $\to$ classical, see, e.g., \cite{Sokolowski_Zolesio1992,Sturm2015,Delfour_Zolesio2011}.
        \item \textbf{Fracture theory.} $\to$ seem advanced, but sound critical for solid mechanics.
        \item \textbf{Modern theories of optimal design.}
        \item \textbf{Optimal locations \& shape of geometric objects.}
        \item \textbf{Free \& moving boundary problems (FBP/MBP).}
        \item \textbf{Image processing.}
    \end{enumerate}
\end{tcolorbox}

\begin{remark}[Core references]
    \begin{itemize}
        Here are the list of core references which should be stuck for fields interested:
        \item For shape optimization, stick with \cite{Delfour_Zolesio2011,Sokolowski_Zolesio1992,Younes2019}.
        \item For NSEs, stick with \cite{Tsai2018}.
        \item For rigorous turbulence, stick with \cite{Rebollo_Lewandowski2014}.
        \item For practical turbulence, stick with ?
    \end{itemize}
\end{remark}
The purpose of this chapter is to list/gather all ingredients in order to able to combine/couple some of these to form delicious, but challenging enough to be possible, recipes in Chap. 2.

\section{PDEs}
With the help of \href{https://en.wikipedia.org/wiki/Partial_differential_equation#Classification}{Wikipedia/PDE/classification}, I am mainly interested in the following groups/types of PDEs:

\begin{enumerate}
    \item \fbox{\textbf{Classic PDEs}}, e.g.:
    \begin{enumerate}
        \item \fbox{\href{https://en.wikipedia.org/wiki/First-order_partial_differential_equation}{\textbf{1st-order PDEs}}}
        
        \textsc{References.} $\to$ any classical undergraduate textbooks for PDEs, and
        \begin{itemize}
            \item \textbf{Classic \& heuristic.} \cite[Chap. 3: Nonlinear 1st-Order PDE]{Evans2010}.
            \item \textbf{Functional analysis, especially operator, perspective.} \cite{Brezis2011}.
        \end{itemize}
        \item \fbox{\textbf{Elliptic PDEs}}
        
        \textsc{References.}
        \begin{itemize}
            \item \textbf{Heuristic \& rigorous.} \cite[Chap. 6: 2nd-Order Elliptic Equations]{Evans2010}.
            \item \textbf{Functional analysis perspective.} \cite[Chap. 9: Sobolev Spaces \& the Variational Formulation of Elliptic BVPs in $N$ Dimensions]{Brezis2011}.
            \item \textbf{Courant's lecture note.} \cite{Han_Lin2011}.
            \item \textbf{Classic \& most rigorous.} \cite{Gilbarg_Trudinger2001}.
            \item \textbf{Old \& classic.} \cite{Ladyzhenskaya_Uraltseva1968}.
            \item \textbf{In nonsmooth domains.} \cite{Grisvard1985}.
            \item \textbf{In polyhedral domains.} \cite{Mazya_Rossmann2010}.
            \item \textbf{Numerics.} \cite{Knabner_Angermann2003}.
        \end{itemize}
        \item \fbox{\textbf{Parabolic PDEs}}
        
        \textsc{References.}
        \begin{itemize}
            \item \textbf{Heuristic \& rigorous.} \cite[Sect. 7.1: 2nd-order parabolic equations]{Evans2010}.
            \item \textbf{Functional analysis perspective.} \cite[Chap. 10: Evolution Problems: The Heat Equation \& the Wave Equation]{Brezis2011}.
            \item \textbf{Old \& classic.} \cite{Friedman1964}.
            \item \textbf{Old, classic, \& advanced.} \cite{Ladyzhenskaja_Solonnikov_Uralceva1968}.
            \item \textbf{Semigroup.} \cite{Lunardi1995}.
            \item \textbf{Semigroup.} \cite{Ladyzhenskaya1991}.
            \item \textbf{Numerics.} \cite{Knabner_Angermann2003}.
        \end{itemize}
        \item \fbox{\textbf{Hyperbolic PDEs}}
        
        $\to$ careful with shock waves and discontinuities.
        
        \textsc{References.}
        \begin{itemize}
            \item \textbf{Heuristic \& rigorous.} \cite[Sect. 7.2: 2nd-order hyperbolic equations]{Evans2010}. \cite[Sect. 7.3: Hyperbolic equations of 1st-order equations]{Evans2010}.
            \item \textbf{Functional analysis perspective.} \cite[Chap. 10: Evolution Problems: The Heat Equation \& the Wave Equation]{Brezis2011}.
            \item \cite{Taylor2011}.
            \item \textbf{Courant's lecture note.} \cite{Lax2006}.
            \item \textbf{Universitext.} \cite{Alinhac2009}.
            \item \textbf{Advanced.} \cite{Benzoni-Gavage_Serre2007}.
            \item \textbf{Control theory.} \cite{Lasiecka_Triggiani2000}
            \item \textbf{Geometric optics.} \cite{Rauch2012}.
            \item \textbf{Wave phenomenon.} \cite{Ikawa2000}.
        \end{itemize}
        \item \fbox{\textbf{Hamilton-Jacobi equations}}
        
        \textsc{References.}
        \begin{itemize}
            \item \textbf{Heuristic \& rigorous.} \cite[Chap. 10: Hamilton-Jacobi Equations]{Evans2010}.
        \end{itemize}
        \item \fbox{\textbf{Nonlinear wave equations}}
        
        \textsc{References.}
        \begin{itemize}
            \item \textbf{Heuristic \& rigorous.} \cite[Chap. 12: Nonlinear Wave Equations]{Evans2010}.
            \item \textbf{Advanced.} \cite{Lannes2013}.
        \end{itemize}
    \end{enumerate}
    \item \fbox{\textbf{PDEs in fluid mechanics/fluid mechanics $\triangleright$ fluid dynamics/fluid mechanics $\triangleright$ CFD}}
    
    (for a good list, see \href{https://en.wikipedia.org/wiki/Fluid_dynamics}{Wikipedia/fluid dynamics} or \textsc{Star-CCM+}/OpenFOAM Documentations):
    \begin{enumerate}
        \item \fbox{\textbf{Navier-Stokes equations (NSEs)}} (in general).
        
        \textsc{References.}
        \begin{itemize}
            \item \textbf{Classic.} \cite{Temam1977}, \cite{Temam2000} $\to$ there is no online version, I have to borrow this book from the WIAS library\footnote{This is the only book that I cannot download so far.}.
            \item \textbf{Brief.} \cite{Tartar2006}.
            \item \textbf{Lecture notes (state-of-the-art).}  \cite{Tsai2018}.
            \item \textbf{Advanced.} \cite{Ladyzhenskaya1969}.
            \item \textbf{Brief.} \cite{Tartar2006}.
            \item \textbf{In polyhedral domains.} \cite{Mazya_Rossmann2007}, \cite{Mazya_Rossmann2009}.
            \item \textbf{Boundary control $+$ FSI.} \cite{Lasiecka_Szulc_Zochowski2018}.
            \item \cite{Hou_Pei2019}.
            \item \cite{Sene_Ngom_Ngom2019}.
            \item \cite{Kim_Huang2018}.
            \item \cite{Camano_Oyarzua_Ruiz-Baier_Tierra2018}.
            \item \textbf{Instationary $+$ mixed boundary conditions.} \cite{Kim_Cao2017}.
            \item \textbf{Weak solvability $+$ steady variational inequality $+$ mixed boundary conditions.} \cite{Kracmar_Neustupa2001}.
            \item \textbf{Soret \& Dufour effects $+$ Navier slip \& convective boundary conditions.} \cite{Kaladhar_Komuraiah_Madhusudhan-Reddy2019}.
            \item \textbf{Weak solution $+$ steady $+$ mixed boundary conditions.} \cite{Hou_Pei2019}.
            \item \textbf{Global stabilization.} \cite{Sene_Ngom_Ngom2019}.
        \end{itemize}
        \item \fbox{\textbf{Conservation laws}}\footnote{Conservation laws are best treated by FVMs by the nature of their derivations.}: Conservation of mass $+$ momentum $+$ energy.
        
        \textsc{References.}
        \begin{itemize}
            \item \textbf{Heuristic \& rigorous.} \cite[Chap. 11: Systems of Conservation Laws]{Evans2010}.
            \item \cite{Lax1987}.
        \end{itemize}
        \item \fbox{\textbf{Compressible vs. incompressible flow}}
        \item \fbox{\textbf{Newtonian vs. non-Newtonian flow}}
        \item \fbox{\textbf{Stationary vs. instationary flow}}\footnote{Also, steady-state/time-independent vs. unsteady-state/time-dependent/evolution.}
        \item \fbox{\textbf{Laminar vs. turbulent flow}}
        \item \fbox{\textbf{Subsonic vs. transonic, supersonic \& hypersonic folows}}
        \item \fbox{\textbf{Reactive vs. non-reactive flows}} $\to$ PDEs in chemistry.
    \end{enumerate}
    \item \fbox{\textbf{Turbulence models}}
    
    \textsc{References.}
    \begin{itemize}
        \item \textbf{Standard/Classical \& physics-oriented.} \cite{Pope2000}.
        \item \textbf{Modern, rigorous, \& mathematics-oriented.} \cite{Rebollo_Lewandowski2014}.
        \item See [inserted links] current available turbulence models in
        \begin{itemize}
            \item OpenFOAM: 
            \item \textsc{Star-CCM+}: [internal link]
        \end{itemize}
    \end{itemize}
    \begin{itemize}
        \item RANS
        \item LES
        \item DES ($=$ RANS $+$ LES).
    \end{itemize}
\end{enumerate}

\subsection{Preliminaries}

\subsubsection{Function spaces}
Let $\Omega$ be an open subset of $\mathbb{R}^N$.
\begin{enumerate}
    \item \textbf{Continuous $C^0$ functions.} Denote by $C(\Omega)$ or $C^0(\Omega)$ the space of continuous functions from $\Omega$ to $\mathbb{R}$.
    \item \textbf{$C^k$ functions.} For an integer $k\ge 1$,
    \begin{align*}
        C^k\left(\Omega\right) := \left\{f\in C^{k-1}\left(\Omega\right);\partial^\alpha f\in C\left(\Omega\right),\ \forall\alpha,\ \left|\alpha\right| = k\right\}.
    \end{align*}
    \item \textbf{Continuously differentiable functions with compact support.} $\mathcal{D}^k(\Omega)$ or $C_c^k(\Omega)$ (resp., $\mathcal{D}(\Omega)$ or $C_c^\infty(\Omega)$) will denote the space of all $k$-times (resp., infinitely) continuously differentiable functions with compact support contained in the open set $\Omega$.
    \item \textbf{Bounded continuous functions.} Denote by $\mathcal{B}^0(\Omega)$ the space of bounded continuous functions from $\Omega$ to $\mathbb{R}$, and, for an integer $k\ge 1$, the space
    \begin{align*}
        \mathcal{B}^k\left(\Omega\right) := \left\{f\in\mathcal{B}^{k-1}\left(\Omega\right);\partial^\alpha f\in\mathcal{B}^0\left(\Omega\right),\ \forall\alpha,\ \left|\alpha\right| = k\right\},
    \end{align*}
    i.e., the space of all functions in $\mathcal{B}^0(\Omega)$ whose derivatives of order $\le k$ are continuous and bounded in $\Omega$.
    
    Endowed with the norm \textbf{(2.16)}
    \begin{align*}
        \|f\|_{C^k(\Omega)} := \max_{0\le\left|\alpha\right|\le k}\sup_{x\in\Omega} \left|\partial^\alpha f(x)\right|,
    \end{align*}
    $\mathcal{B}^k(\Omega)$ is a Banach space.
    \item If a function $f$ is bounded and uniformly continuous\footnote{A function $f:\Omega\to\mathbb{R}$ is uniformly continuous if for each $\varepsilon > 0$ there exists $\delta > 0$ s.t. for all $x$ and $y$ in $\Omega$ s.t. $|x - y| < \delta$ we have $|f(x) - f(y)| < \varepsilon$.} on $\Omega$, it possesses a unique, continuous extension to the closure $\overline{\Omega}$ of $\Omega$.
    
    Denote by $C^k\left(\overline{\Omega}\right)$ the space of functions $f$ in $C^k(\Omega)$ for which $\partial^\alpha f$ is bounded and uniformly continuous on $\Omega$ for all $\alpha$, $0\le|\alpha|\le k$.
    \item \textbf{Functions vanishing at the boundary.} A function $f$ in $C^k(\Omega)$ is said to \textit{vanish at the boundary} of $\Omega$ if for every $\alpha$, $0\le|\alpha|\le k$, and $\varepsilon > 0$ there exists a compact subset $K$ of $\Omega$ s.t., for all $x\in\Omega\cap K^c$, $|\partial^\alpha f(x)|\le\varepsilon$.
    
    Denote by $C_0^k(\Omega)$ the space of all such functions.
    
    Clearly
    \begin{align*}
        \boxed{C_0^k(\Omega)\subset C^k(\overline{\Omega})\subset\mathcal{B}^k(\Omega)\subset C^k(\Omega).}
    \end{align*}
    Endowed with the norm (2.16), $C_0^k(\Omega)$, $C^k(\overline{\Omega})$, and $\mathcal{B}^k(\Omega)$ are Banach spaces.
    \item Finally
    \begin{align*}
        C^\infty(\Omega) := \bigcap_{k\ge 0} C^k(\Omega),\ C_0^\infty(\Omega) := \bigcap_{k\ge 0} C_0^k(\Omega),\ \mbox{ and } \mathcal{B}(\Omega) := \bigcap_{k\ge 0} \mathcal{B}^k(\Omega).
    \end{align*}
    When $f$ is a vector function from $\Omega$ to $\mathbb{R}^m$, the corresponding spaces will be denoted by $C_0^k(\Omega)^m$ or $C_0^k(\Omega,\mathbb{R}^m)$, $C^k(\overline{\Omega})^m$ or $C^k(\Omega,\mathbb{R}^m)$, $\mathcal{B}^k(\Omega)^m$ or $\mathcal{B}^k(\Omega,\mathbb{R}^m)$, $C^k(\Omega)^m$ or $C^k(\Omega,\mathbb{R}^m)$, etc.
    \item \textbf{H\"older $C^{0,l}$ and Lipschitz $C^{0,1}$ continuous functions.} Given $\lambda$, $0 < \lambda\le 1$, a function $f$ is \textit{$(0,\lambda)$-Hölder continuous} in $\Omega$ if
    \begin{align*}
        \exists c > 0,\ \forall x,y\in\Omega,\ \left|f(y) - f(x)\right|\le c\left|x - y\right|^\lambda.
    \end{align*}
    When $\lambda = 1$, we also say that $f$ is \textit{Lipschitz} or \textit{Lipschitz continuous}.
    \item \textbf{$(k,\lambda)$-H\"older continuous functions.} For $k\ge 1$, $f$ is \textit{$(k,\lambda)$-Hölder continuous} in $\Omega$ if
    \begin{align*}
        \forall\alpha,\ 0\le|\alpha|\le k,\ \exists c > 0,\ \forall x,y\in\Omega,\ \left|\partial^\alpha f(y) - \partial^\alpha f(x)\right|\le c\left|x - y\right|^\lambda.
    \end{align*}
    Denote by $C^{k,\lambda}(\Omega)$ the space of all $(k,\lambda)$-Hölder continuous functions on $\Omega$.
    \item Define for $k\ge 0$ the subspaces\footnote{The notation $C^{k,\lambda}(\overline{\Omega})$ should not be confused with the notation $C^{k,\lambda}(\Omega)$ for $(k,\lambda)$-Hölder continuous functions in $\Omega$ without the uniform boundedness assumption in $\Omega$.
        
        In particular, $C^{k,\lambda}(\overline{\mathbb{R}^N})$ is contained in but not equal to $C^{k,\lambda}(\mathbb{R}^N)$.}
    \begin{align*}
        C^{k,\lambda}(\overline{\Omega}) := \left\{f\in C^k(\overline{\Omega});\forall\alpha,\ 0\le|\alpha|\le k,\ \exists c > 0,\ \forall x,y\in\Omega,\ \left|\partial^\alpha f(y) - \partial^\alpha f(x)\right|\le c\left|x - y\right|^\lambda\right\}
    \end{align*}
    of $C^k(\overline{\Omega})$.
    
    By definition for each $\alpha$, $0\le|\alpha|\le k$, $\partial^\alpha f$ has a unique, bounded, continuous extension to $\overline{\Omega}$.
    
    Endowed with the norm \textbf{(2.21)}
    \begin{align*}
        \|f\|_{C^{k,\lambda}(\Omega)} := \max\left\{\|f\|_{C^k(\Omega)},\ \max_{0\le\left|\alpha\right|\le k}\sup_{x,y\in\Omega,\ x\ne y} \frac{\left|\partial^\alpha f(y) - \partial^\alpha f(x)\right|}{\left|x - y\right|^\lambda}\right\}
    \end{align*}
    $C^{k,\lambda}(\overline{\Omega})$ is a Banach space.
    \item Denote by $C_0^{k,\lambda}(\Omega) := C^{k,\lambda}(\overline{\Omega})\cap C_0^k(\Omega)$.
    \item Assumption 2.3:
    
    \begin{example}
        \begin{enumerate}
            \item $\Theta = \mathcal{B}^0(\mathbb{R}^N,\mathbb{R}^N)$, space of bounded continuous functions.
            
            For $g\in\mathcal{B}^0(\mathbb{R}^N,\mathbb{R}^N)$and $I + f\in\mathcal{F}(\mathcal{B}^0(\mathbb{R}^N,\mathbb{R}^N))$, the composition $g\circ(I + f)$ is continuous and $\|g\circ(I + f)\|_{C^0} = \|g\|_{C^0} < \infty$.
            \item $\Theta = C^0(\overline{\mathbb{R}^N},\mathbb{R}^N)$, space of bounded uniformly continuous functions.
            
            For $g\in C^0(\overline{\mathbb{R}^N},\mathbb{R}^N)$ and $I + f\in\mathcal{F}(C^0(\overline{\mathbb{R}^N},\mathbb{R}^N))$, $\|g\circ(I + f)\|_{C^0} = \|g\|_{C^0} < \infty$, and since $g$ and $I + f$ are uniformly continuous, so is the composition $g\circ(I + f)$.
            \item $\Theta = C_0^0(\mathbb{R}^N,\mathbb{R}^N)$, a subspace of $C^0(\overline{\mathbb{R}^N},\mathbb{R}^N)$.
            
            From (2) $\|g\circ(I + f)\|_{C^0} = \|g\|_{C^0} < \infty$ and $g\circ(I + f)\in C^0(\overline{\mathbb{R}^N},\mathbb{R}^N)$.
            
            It remains to show that $g\circ(I + f)\in C_0^0(\mathbb{R}^N,\mathbb{R}^N)$.
            
            Since $g\in C_0^0(\mathbb{R}^N,\mathbb{R}^N)$, for all $\varepsilon > 0$, there exists $\rho_0 > 0$ s.t. for all $x$ s.t. $|x| > \rho_0$, $|g(x)| < \varepsilon$.
            
            But $f$ also belongs to $C_0^0(\mathbb{R}^N,\mathbb{R}^N)$ and there exists $\rho\ge 2\rho_0$ s.t. for all $x$ s.t. $|x| > \rho$, $|f(x)| < \rho_0$.
            
            In particular, $|x + f(x)|\ge|x| - |f(x)| > \rho - \rho_0\ge\rho_0$ and hence $|g(x + f(x))| < \varepsilon$.
            
            In conclusion for all $\varepsilon > 0$, there exists $\rho > 0$ s.t. for all $x$ s.t. $|x| > \rho$, $|g(x + f(x))| < \varepsilon$ and $g\circ(I + f)\in C_0^0(\mathbb{R}^N,\mathbb{R}^N)$.
            \item $\Theta = C^{0,1}(\overline{\mathbb{R}^N},\mathbb{R}^N)$.
            
            This is the space of bounded Lipschitz continuous functions on $\mathbb{R}^N$ that is contained in $C^0(\overline{\mathbb{R}^N},\mathbb{R}^N)$.
            
            For $f,g\in C^{0,1}(\overline{\mathbb{R}^N},\mathbb{R}^N)$, $\|f\circ(I + g)\|_{C^0} = \|f\|_{C^0} < \infty$.
            
            Since $g$ is Lipschitz with Lipschitz constant $c(g)$, $I + g$ is also Lipschitz with Lipschitz constant $1 + c(g)$ and the composition is also Lipschitz with constant $c(f)(1 + c(g))$.
        \end{enumerate}
    \end{example}
    \item Assumption 2.4 is a little trickier since it involves not only norms but also a ``uniform continuity'' of the function $g$ that is not verified for $g\in\mathcal{B}^0(\mathbb{R}^N,\mathbb{R}^N)$ and for the Lipschitz part $c(g)$ of the norm of $C^{0,1}(\overline{\mathbb{R}^N},\mathbb{R}^N)$.
    
    \begin{example}[Checking Assumption 2.4]
        (1) $\Theta = \mathcal{B}^0(\mathbb{R}^N,\mathbb{R}^N)$.
        
        In that space there exists some $g$ that is not uniformly continuous.
        
        For that $g$ there exists $\varepsilon > 0$ s.t. for all $n > 0$, there exists $x_n,y_n$, $|y_n - x_n| < \frac{1}{n}$ s.t. $|g(y_n) - g(x_n)|\ge\varepsilon$.
        
        Define the function $\gamma_n(x) = y_n - x_n$.
        
        Therefore, there exists $\varepsilon > 0$ s.t. for all $n$ with $\|\gamma_n\|_{C^0} < \frac{1}{n}$,
        \begin{align*}
            \|g\circ(I + \gamma_n) - g\|_{C^0}\ge|g(x_n + \gamma_n(x_n)) - g(x_n)| = |g(x_n + y_n - x_n) - g(x_n)|\ge\varepsilon
        \end{align*}
        and Assumption 2.4 is not verified.
        
        However, $\mathcal{B}^k(\mathbb{R}^N,\mathbb{R}^N)\subset C^0(\mathbb{R}^N,\mathbb{R}^N)$ and for all $x\in\mathbb{R}^N$, the mapping $f\mapsto f(x):\mathcal{B}^0(\mathbb{R}^N,\mathbb{R}^N)\to\mathbb{R}$ is continuous.
        
        (2) and (3) $\Theta = C^0(\overline{\mathbb{R}^N},\mathbb{R}^N)$.
        
        By the uniform continuity of $g\in C^0(\overline{\mathbb{R}^N},\mathbb{R}^N)$, for all $\varepsilon > 0$ there exists $\delta > 0$ s.t.
        \begin{align*}
            \forall x,y\in\mathbb{R}^N,\ |y - x| < \delta\Rightarrow|g(y) - g(x)| < \frac{\varepsilon}{2}.
        \end{align*}
        For $\|\gamma\|_{C^0} < \delta$
        \begin{align*}
            \forall x\in\mathbb{R}^N,\ |(I + \gamma)(x) - x| &= |\gamma(x)|\le\|\gamma\|_{C^0} < \delta,\\
            \forall x\in\mathbb{R}^N,\ |g(x + \gamma(x)) - g(x)| < \frac{\varepsilon}{2}&\Rightarrow\|g\circ(I + \gamma) - g\|_{C^0}\le\frac{\varepsilon}{2} < \varepsilon,
        \end{align*}
        and Assumption 2.4 is verified.
        
        In addition, $C^0(\overline{\mathbb{R}^N},\mathbb{R}^N)\subset C^0(\mathbb{R}^N,\mathbb{R}^N)$ and for all $x\in\mathbb{R}^N$, the mapping $f\mapsto f(x): C^0(\overline{\mathbb{R}^N},\mathbb{R}^N)\to\mathbb{R}$ is continuous.
        
        (4) $\Theta = C^{0,1}(\overline{\mathbb{R}^N},\mathbb{R}^N)$.
        
        Since $g\in C^{0,1}(\overline{\mathbb{R}^N},\mathbb{R}^N)$
        \begin{align*}
            \forall x,y\in\mathbb{R}^N,\ |g(y) - g(x)|\le c(g)|y - x|.
        \end{align*}
        Therefore for
        \begin{align*}
            \|\gamma\|_{C^{0,1}} := \max\{\|\gamma\|_{C^0},c(\gamma)\} < \delta,
        \end{align*}
        $\|\gamma\|_{C^0} < \delta$ and we get
        \begin{align*}
            \|g\circ(I + \gamma) - g\|_{C^0}\le c(g)\|\gamma\|_{C^0} < c(g)\delta.
        \end{align*}
        Unfortunately, for the part $c(g\circ(I + \gamma))$ of the $C^{0,1}$-norm, for all $x\ne y\in\mathbb{R}^N$
        \begin{align*}
            \frac{|g(y + \gamma(y)) - g(y) - \left(g(x + \gamma(x)) - g(x)\right)|}{|y - x|}&\le c(g) + c(g)c(I + \gamma),\\
            c\left(g\circ(I + \gamma) - g\right)\le c(g)[1 + c(\gamma)] &< c(g)[1 + \delta],
        \end{align*}
        and we cannot satisfy Assumption 2.4.
        
        However, $C^{0,1}(\overline{\mathbb{R}^N},\mathbb{R}^N)\subset C^0(\mathbb{R}^N,\mathbb{R}^N)$ and for all $x\in\mathbb{R}^N$, the mapping $f\mapsto f(x): C^{0,1}(\overline{\mathbb{R}^N},\mathbb{R}^N)\to\mathbb{R}$ is continuous.
    \end{example}
    \item \textbf{Sobolev space $W^{m,p}$.}
\end{enumerate}

\subsubsection{Compactness theorems}

\begin{theorem}[Ascoli-Arzel\`a theorem]
    Let $\Omega$ be a bounded open subset of $\mathbb{R}^N$. A subset $\mathcal{K}$ of $C(\overline{\Omega})$ is precompact in $C(\overline{\Omega})$ provided the following 2 conditions hold:
    \begin{itemize}
        \item[(i)] there exists a constant $M$ s.t. for all $f\in\mathcal{K}$ and $x\in\Omega$, $\left|f(x)\right|\le M$;
        \item[(ii)] for every $\varepsilon > 0$ there exists $\delta > 0$ s.t. if $f\in\mathcal{K}$, $x,y$ in $\Omega$, and $|x - y| < \delta$, then $|f(x) - f(y)| < \varepsilon$.
    \end{itemize}
\end{theorem}

\subsubsection{Embedding theorems}
\begin{enumerate}
    \item Classical:
    
    \begin{theorem}[R. A. Adams (1)]
        Let $k\ge 0$ be an integer and $0 < \nu < \lambda\le 1$ be real numbers. Then the following embedding exist: \textbf{(2.22)-(2.24)}
        \begin{align*}
            C^{k+1}(\overline{\Omega})&\to C^k(\overline{\Omega}),\\
            C^{k,\lambda}(\overline{\Omega})&\to C^k(\overline{\Omega}),\\
            C^{k,\lambda}(\overline{\Omega})&\to C^{k,\nu}(\overline{\Omega}).
        \end{align*}
        If $\Omega$ is bounded, then the embeddings (2.23) and (2.24) are compact. If $\Omega$ is convex, we have the further embeddings \textbf{(2.25)-(2.26)}
        \begin{align*}
            C^{k+1}(\overline{\Omega})&\to C^{k,1}(\overline{\Omega}),\\
            C^{k+1}(\overline{\Omega})&\to C^{k,\nu}(\overline{\Omega}).
        \end{align*}
        If $\Omega$ is convex and bounded, then embeddings (2.22) and (2.26) are compact.
    \end{theorem}
    As a consequence of the 2nd part of the theorem, the definition of $C^{k,\lambda}(\overline{\Omega})$ simplifies when $\Omega$ is convex: \textbf{(2.27)}
    \begin{align*}
        C^{k,\lambda}(\overline{\Omega}) := \left\{f\in\mathcal{B}^k(\Omega);\forall\alpha,\ |\alpha| = k,\ \exists c > 0,\ \forall x,y\in\Omega,\ \left|\partial^\alpha f(y) - \partial^\alpha f(x)\right|\le c\left|x - y\right|^\lambda\right\},
    \end{align*}
    and its norm is equivalent to the norm \textbf{(2.28)}
    \begin{align*}
        \|f\|_{C^{k,\lambda}(\Omega)} := \|f\|_{C^k(\Omega)} + \max_{|\alpha| = k}\sup_{x,y\in\Omega,\, x\ne y} \frac{\left|\partial^\alpha f(y) - \partial^\alpha f(x)\right|}{\left|x - y\right|^\lambda}.
    \end{align*}
    When $f$ is a vector function from $\Omega$ to $\mathbb{R}^m$, the corresponding spaces will be denoted by $C^{k,\lambda}(\overline{\Omega})^m$ or $C^{k,\lambda}(\overline{\Omega},\mathbb{R}^m)$.
    \item \textbf{Identity $C^{k,1}(\overline{\Omega}) = W^{k+1,\infty}(\Omega)$: From Convex to Path-Connected Domains via the Geodesic Distance.} By Rademacher's theorem (cf., e.g., L. C. Evans and R. F. Gariepy [1])
    \begin{align*}
        C^{k,1}(\overline{\Omega})\subset W_{\rm loc}^{k+1,\infty}(\Omega),
    \end{align*}
    and, when $\Omega$ is convex,
    \begin{align*}
        C^{k,1}(\overline{\Omega}) = W^{k+1,\infty}(\Omega).
    \end{align*}
    This last convexity assumption can be relaxed as follows. From Rademacher's theorem and the inequality (cf. H. Brézis [1, p. 154])
    \begin{align*}
        \forall u\in W^{1,\infty}(\Omega) \mbox{ and } \forall x,y\in\Omega,\ \left|u(x) - u(y)\right|\le\left\|\nabla u\right\|_{L^\infty(\Omega)}{\rm dist}_\Omega(x,y),
    \end{align*}
    have:
    
    \begin{theorem}
        Let $\Omega$ be a bounded, path-connected, open subset of $\mathbb{R}^N$ s.t. \textbf{(2.29)}
        \begin{align*}
            \exists c,\ \forall x,y\in\overline{\Omega},\ {\rm dist}_\Omega(x,y)\le c|x - y|.
        \end{align*}
        Then we have $W^{1,\infty}(\Omega) = C^{0,1}(\overline{\Omega})$ algebraically and topologically, and there exists a constant $c'$ s.t.
        \begin{align*}
            \forall u\in W^{1,\infty}(\Omega) \mbox{ and } \forall x,y\in\overline{\Omega},\ \left|u(x) - u(y)\right|\le c'\left\|\nabla u\right\|_{L^\infty(\Omega)}\left|x - y\right|.
        \end{align*}
        Here for all integers $k\ge 0$, $W^{k+1,\infty}(\Omega) = C^{k,1}(\overline{\Omega})$ algebraically and topologically.
    \end{theorem} 
    This is true for Lipschitzian domains (D. Gilbarg and N. S. Trudinger [1]).
    
    \begin{corollary}
        Let $\Omega$ be a bounded, open, path-connected, and locally Lipschitzian\footnote{Cf. Definition 5.2 of Sect. 5 and Theorem 5.3, which says that a locally Lipschitzian domain is equi-Lipschitzian when $\partial\Omega$ is bounded.} subset of $\mathbb{R}^N$. Then property (2.29) and Theorem 2.6 are verified.
    \end{corollary}
\end{enumerate}




\subsection{Elliptic PDEs}

\subsection{Parabolic PDEs}

\subsection{Hyperbolic PDEs}

\subsection{Navier Stokes equations}


\subsection{Turbulence models}

\subsubsection{LES}
\begin{itemize}
    \item Spalart-Allmaras
    \item Smagorinsky
\end{itemize}

\subsubsection{RANS}
\begin{itemize}
    \item NSE-KT
    \item $k$-$\epsilon$ (levels 1,2,$\ldots$ statistics) $\to$ careful with boundary layer, wall laws.
    
    \textsc{References.}
    \begin{itemize}
        \item \textbf{Classic but heuristic/informal only.} \cite{Mohammadi_Pironneau1994}.
        \item \textbf{Modern and rigorous!} \cite{Rebollo_Lewandowski2014}.
    \end{itemize}
    \item $k$-$\omega$
\end{itemize}

\subsubsection{VMS}

\subsubsection{RSM}

\subsection{Boundary conditions}

\section{Numerical analysis}
\begin{enumerate}
    \item \fbox{\textbf{Finite Difference Methods (FDMs)}}
    
    \textsc{References.}
    \begin{itemize}
        \item \textbf{Classic.} \cite{LeVeque2007}.
    \end{itemize}
    \item \fbox{\textbf{Finite Element Methods (FEMs)}}
    
    \textsc{References.}
    \begin{itemize}
        \item \textbf{Classic.} \cite{Brenner_Scott2008}.
        \item \textbf{FEM for LES book.} \cite{John2004}.
        \item \textbf{FEM for LES article.} \cite{John_Layton2002}.
    \end{itemize}
    \item \fbox{\textbf{Finite Volume Methods (FVMs)}}
    
    \textsc{References.}
    \begin{itemize}
        \item \textbf{Classic.} \cite{LeVeque2002}.
        \item \cite{Moukalled_Mangani_Darwish2016}.
    \end{itemize}
\end{enumerate}

\subsection{FDMs}
Perhaps for the sake of completeness only.

\subsection{FEMs}

\subsection{FVMs}

\section{Mathematical optimization}
\begin{enumerate}
    \item \fbox{\textbf{Optimal control}}
    
    \textsc{References.}
    \begin{itemize}
        \item \cite{Troltzsch2010}.
        \item \cite{Kroener2011}.
        \item \cite{Kroener2019}.
    \end{itemize}
    \item \fbox{\textbf{Topology optimization}}
    
    \textsc{References.}
    \begin{itemize}
        \item \cite{Othmer_Kaminski_Giering2006}.
        \item \cite{Sigmund_Maute2013}.
    \end{itemize}
    \item \fbox{\textbf{Shape optimization}}
    
    \textsc{References.}
    \begin{itemize}
        \item \textbf{Original paper.} \cite{Micheletti1972}.
        \item \textbf{Lecture (formal computation).} \cite{Schmidt2020}.
        \item \textbf{Most rigorous book.} \cite{Delfour_Zolesio2011} $\to$ take this book as the core \& stick with its structure.
        \item \textbf{Most rigorous book's companion.} \cite{Sokolowski_Zolesio1992}.
        \item \textbf{PhD thesis.} \cite{Sturm2015}.
        \item \cite{Fischer_Lindemann_Ulbrich_Ulbrich2017}.
        \item \cite{Haslinger_Makinen2003}.
        \item \cite{Kasumba_Kunisch2010}.
        \item \cite{Kasumba_Kunisch2012}.
        \item \cite{Hintermuller2005}.
        \item \cite{Fuhr_Schulz_Welker2018}.
        \item \cite{Boisgerault_Zolesio2000}.
        \item \cite{Laurain_Sturm2016}.
        \item \cite{Heners_Radtle_Hinze_Duster2018}.
        \item \cite{Feppon_Allaire_Bordeu_Cortial_Dapogny2019}.
        \item \cite{Haslinger_Stebel2011}.
        \item \cite{Brandenburg_Lindemann_Ulbrich_Ulbrich2009}.
        \item \cite{Gunzburger_Kim_Manservisi2000}.
        \item \cite{Gunzburger_Kim1998}.
        \item \cite{Bulicek_Haslinger_Malek_Stebel2009}.
        \item \cite{Fursikov1981}.
        \item \cite{Allaire_Henrot2001}.
        \item \cite{Schmidt_Schulz2010}.
    \end{itemize}
    \item \fbox{\textbf{Adjoint}}
    \begin{enumerate}
        \item \fbox{\textbf{Discrete adjoint}}
        
        \textsc{References.}
        \begin{itemize}
            \item 
        \end{itemize}
        \item \fbox{\textbf{Continuous adjoint}}
        
        \textsc{References.}
        \begin{itemize}
            \item \cite{Othmer2008}.
            \item \cite{Othmer_Villiers_Weller2007}.
            \item \cite{Jameson_Martinelli_Pierce1998}.
            \item \cite{Papadimitriou_Giannakoglou2007}.
            \item \cite{Anderson_Venkatakrishnan1999}.
            \item \cite{Turgeon_Pelletier_Borggaard_Etienne2007}.
            \item \cite{Zymaris_Papadimitriou_Giannakoglou_Othmer2009}.
            \item \cite{Zymaris_Papadimitriou_Giannakoglou_Othmer2010}.
            \item \cite{Papoutsis-Kiachagias_Giannakoglou2016}.
            \item \cite{Bueno-Orovio_Castro_Palacios_Zuazua2012}.
            \item \cite{Baeza_Castro_Palacios_Zuazua2009}.
        \end{itemize}
    \end{enumerate}
\end{enumerate}

\subsection{Optimal control}

\subsection{Topological optimization}

\subsection{Shape optimization}
\begin{enumerate}
    \item \textbf{Basic dilemma.} \textit{Parametrize geometries by functions or functions by geometries?}
\end{enumerate}

\subsubsection{Geometries}
\begin{enumerate}
    \item \textbf{Target.} Classical descriptions of nonempty subsets of the finite-dimensional Euclidean space that are characterized by the smoothness or properties of their boundary.
    \item \textbf{Approaches.}
    \begin{enumerate}
        \item \textbf{Differential geometry's.} Assume that we can associate with each point of the boundary a diffeomorphism from a neighborhood of that point that \textit{locally flattens} the boundary.\footnote{used by Stephan Schmidt.}
        \item \textbf{Level set.} Assume that the set is the union of the positive \textit{level sets} of a continuous function and that the zero level is its boundary.
        \item \textbf{Functional analysis's.} Assume that in each point of the boundary the set is locally the \textit{epigraph} of a function.
    \end{enumerate}
    \item \textbf{Functionality/practicality.} \cite{Delfour_Zolesio2011}: ``The smoothness of the set is then characterized by the smoothness of the corresponding diffeomorphism, level function, or graph function.
    
    Those definitions are equivalent for sufficiently smooth sets.
    
    Domains that are locally the epigraph of a continuous function in the 3rd category also belong to the 1st category, but the converse is generally not true.''
\end{enumerate}

\paragraph{Connected Space, Path-Connected Space, \& Geodesic Distance.}

\begin{definition}
    \begin{itemize}
        \item[(i)] A \emph{connected space} is a topological space which cannot be represented as the disjoint union of 2 or more nonempty open subsets.
        \item[(ii)] A topological space $X$ is said to be \emph{path-connected} (or \emph{pathwise connected} or \textit{$0$-connected}) if for any 2 points $x$ and $y$ in $X$ there exists a continuous function $f$ from the unit interval $[0,1]$ to $X$ with $f(0) = x$ and $f(1) = y$. (This function is called a \emph{path} in $X$ from $x$ to $y$.)
        \item[(iii)] Given a path-connected subset $S$ of $\mathbb{R}^N$, the \emph{geodesic distance} ${\rm dist}_S(x,y)$ between 2 points $x$ and $y$ of $S$ is the infimum of the lengths of the paths in $S$ joining $x$ and $y$.
    \end{itemize}
\end{definition}

\begin{definition}
    Given a subset $A$ of $\mathbb{R}^N$,
    \begin{equation*}
        \#(A) := \left\{\begin{split}
            &\mbox{number of connected components of } A, &&\mbox{ if } A\ne\emptyset,\\
            &0, &&\mbox{ if } A = \emptyset.
        \end{split}\right.
    \end{equation*}
    We say that the set $A$ is \emph{hole-free} if $A^c$ is connected or $A = \mathbb{R}^N$.
\end{definition}
If we adopt the \textit{convention} that the empty set $\emptyset$ is both connected and path-connected, then a set $A$ is connected if and only if its complement $A^c$ is hole-free.

\begin{theorem}
    Let $A$ and $\Omega$ be subsets of $\mathbb{R}^N$.
    \begin{itemize}
        \item[(i)] A set $A\ne\emptyset$ is connected iff $\#(A) = 1$; a set $A$ is hole-free iff $\#\left(A^c\right)\le 1$.
        \item[(ii)] An open set $\Omega\ne\emptyset$ is path-connected iff $\#(\Omega) = 1$; an open set $\Omega$ is hole-free iff $\#\left(\Omega^c\right)\le 1$.
    \end{itemize}
\end{theorem}

\paragraph{Bouligand's Contingent Cone, Dual Cone, \& Normal Cone}
\begin{definition}
    Let $\emptyset\ne\Omega\subset\mathbb{R}^N$ and let $x\in\overline{\Omega}$. The \emph{dual cone} associated with $\Omega$ is defined as \textbf{(2.2)}
    \begin{align*}
        \Omega^* := \left\{x^*\in\mathbb{R}^N;\forall x\in\Omega,\ x^*\cdot x\ge 0\right\}.
    \end{align*}
    $\Omega^*$ is a closed convex cone in 0.
\end{definition}

\begin{definition}
    Let $\emptyset\ne\Omega\subset\mathbb{R}^N$ and let $x\in\overline{\Omega}$.
    \begin{itemize}
        \item[(i)] The vector $h\in\mathbb{R}^N$ is an \emph{admissible direction} for $\Omega$ in $x$ if there exists a sequence $\{\varepsilon_n > 0\}$, $\varepsilon_n\downarrow 0$ as $n\to\infty$, s.t. \textbf{(2.3)}
        \begin{align*}
            \forall n,\ \exists x_n\in\Omega \mbox{ and } \lim_{n\to\infty} \frac{x_n - x}{\varepsilon_n} = h.
        \end{align*}
        $T_x\Omega$ will denote the set of all admissible directions for $\Omega$ in $x$ and it will be referred to as \emph{Bouligand's contingent cone}\footnote{The \textit{contingent cone} $T_\Omega(x)$ was introduced in the 1930s by G. Bouligand [1].
            
            It naturally occurs in the \textit{viability theory} of differential equations (cf. M. Nagumo [1] or J.-P. Aubin and A. Cellina [1, p. 174 and p. 180]).} to $\Omega$ in $x$.
        \item[(ii)] The dual cone $\left(-T_x\Omega\right)^*$ associated with $\Omega$ and a point $x\in\overline{\Omega}$ will be referred to as the \emph{normal cone} to $\Omega$ in $x\in\overline{\Omega}$.
    \end{itemize}
\end{definition}

\paragraph{Sets Locally Described by a Homeomorphism or a Diffeomorphism.}
\begin{enumerate}
    \item \textbf{Sets of Classes $C^k$ and $C^{k,l}$.} Let $\{e_1,\ldots,e_n\}$ be the standard unit orthonormal basis in $\mathbb{R}^N$.
    
    Use the notation $\zeta = \left(\zeta',\zeta_N\right)$ for a point $\zeta = \left(\zeta_1,\ldots,\zeta_N\right)$ in $\mathbb{R}^N$, where $\zeta' = \left(\zeta_1,\ldots,\zeta_{N-1}\right)$.
    
    Denote by $B$ the open unit ball in $\mathbb{R}^N$ and define the sets \textbf{(3.1)-(3.2)}
    \begin{align*}
        B_0 &:= \left\{\zeta\in B;\zeta_N = 0\right\},\\
        B_+ &:= \left\{\zeta\in B;\zeta_N > 0\right\},\\
        B_- &:= \left\{\zeta\in B;\zeta_N < 0\right\}.
    \end{align*}

    \begin{definition}
        Let $\Omega$ be a subset of $\mathbb{R}^N$ s.t. $\partial\Omega\ne\emptyset$, $0\le k$ be an integer or $+\infty$, and $0\le l\le 1$ be a real number.
        \begin{itemize}
            \item[(i)] - $\Omega$ is said to be \emph{locally} of class $C^k$ at $x\in\partial\Omega$ if there exist
            \begin{itemize}
                \item[(a)] a neighborhood $U(x)$ of $x$, and
                \item[(b)] a bijective map $g_x:U(x)\to B$ with the following properties: \textbf{(3.3)-(3.5)}
                \begin{align*}
                    g_x&\in C^k\left(U(x);B\right),\ h_x := g_x^{-1}\in C^k\left(B;U(x)\right),\\
                    {\rm int}\,&\Omega\cap U(x) = h_x\left(B_+\right),\\
                    \Gamma_x &:= \partial\Omega\cap U(x) = h_x\left(B_0\right),\ B_0 = g_x\left(\Gamma_x\right).
                \end{align*}
            \end{itemize}
            - Given $0 < l < 1$, $\Omega$ is said to be \emph{locally $(k,l)$-Hölderian at} $x\in\partial\Omega$ if conditions (a) and (b) are satisfied with a map $g_x\in C^{k,l}\left(U(x),B\right)$ with inverse $h_x = g_x^{-1}\in C^{k,l}\left(B,U(x)\right)$.
            
            - $\Omega$ is said to be \emph{locally $k$-Lipschitzian at} $x\in\partial\Omega$ if conditions (a) and (b) are satisfied with a map $g_x\in C^{k,1}\left(U(x),B\right)$ with inverse $h_x = g_x^{-1}\in C^{k,1}\left(B,U(x)\right)$.
        \end{itemize}
        \item[(ii)] - $\Omega$ is said to be \emph{locally of class} $C^k$ if, for each $x\in\partial\Omega$, $\Omega$ is locally of \emph{class} $C^k$ at $x$.
        
        - Given $0 < l < 1$, $\Omega$ is said to be \emph{locally $(k,l)$-Hölderian} if, for each $x\in\partial\Omega$, $\Omega$ is locally $(k,l)$-Hölderian at $x$.
        
        - $\Omega$ is said to be \emph{locally $k$-Lipschitzian} if, for each $x\in\partial\Omega$, $\Omega$ is locally $k$-Lipschitzian at $x$.
    \end{definition}

    \begin{remark}
        By definition ${\rm int}\,\Omega\ne\emptyset$ and ${\rm int}\,\Omega^c\ne\emptyset$.
        
        The above definitions are usually given for an open set $\Omega$ called a \emph{domain}.
        
        This terminology naturally arises in PDEs, where this open set is indeed the \emph{domain} in which the solution of the PDE is defined.\footnote{Classically for $k\ge 1$ a bounded open domain $\Omega$ in $\mathbb{R}^N$ is said to be of class $C^k$ if its boundary is a $C^k$-submanifold of $\mathbb{R}^N$ of codimension 1 and $\Omega$ is located on 1 side of its boundary $\Gamma = \partial\Omega$ (cf. S. Agmon [1, Def. 9.2, p. 178]).}
        
        At this point it is not necessary to assume that the set $\Omega$ is open.
        
        The family of diffeomorphisms $\{g_x:x\in\partial\Omega\}$ characterizes the equivalence class of sets
        \begin{align*}
            \left[\Omega\right] = \left\{A\subset\mathbb{R}^N;{\rm int}\,A = {\rm int}\,\Omega \mbox{ and } \partial A = \partial\Omega\right\}
        \end{align*}
        with the same interior and boundary.
        
        The sets ${\rm int}\,\Omega$ and $\partial\Omega$ are invariants for the equivalence class $[\Omega]$ of sets of class $C^{k,l}$.
        
        The notation $\Gamma$ for $\partial\Omega$ and the standard terminology \emph{domain} for the unique (open) set ${\rm int}\,\Omega$ associated with the class $[\Omega]$ will be used.
        
        In what follows, a $C^{k,l}$-mapping $g_x$ with $C^{k,l}$-inverse will be called a \emph{$C^{k,l}$-diffeomorphism}.
    \end{remark}

    \begin{remark}
        See in Sect. 5 that domains that are locally the epigraph of a $C^{k,l}$, $k\ge 1$ (resp., Lipschitzian), function are of class $C^{k,l}$ (resp., $C^{0,1}$), but domains which are of class $C^{0,1}$ are not necessarily locally the epigraph of a Lipschitzian function (cf. Examples 5.1 and 5.2).
        
        Nevertheless globally $C^{0,1}$-mappings with a $C^{0,1}$ inverse are important since they transport $L^p$-functions onto $L^p$-functions and $W^{1,p}$-functions onto $W^{1,p}$-functions (cf. J. Nečas [1, Lems. 3.1 and 3.2, pp. 65-66]).
    \end{remark}
    \item \textbf{Unit exterior normal.} For sets of class $C^1$, the unit exterior normal to the boundary $\Gamma = \partial\Omega$ can be characterized through the Jacobian matrices of $g_x$ and $h_x$.
    
    By definition of $B_0$, $\{e_1,\ldots,e_{N-1}\}\subset B_0$ and the tangent space $T_y\Gamma$, $\Gamma = \partial\Omega$, at $y$ to $\Gamma_x$ is the vector space spanned by the $N - 1$ vectors \textbf{(3.7)}
    \begin{align*}
        \left\{Dh_x\left(\zeta',0\right){\bf e}_i;1\le i\le N - 1\right\},\ \left(\zeta',0\right) = g_x(y)\in B_0,
    \end{align*}
    where $Dh_x(\zeta)$ is the Jacobian matrix of $h_x$ at the point $\zeta$:
    \begin{align*}
        \left(Dh_x\right)_{lm} := \partial_m\left(h_x\right)_l.
    \end{align*}
    So from (3.7) a normal vector field to $\Gamma_x$ at $y\in\Gamma_x$ is given by \textbf{(3.8)}
    \begin{align*}
        m_x(y) = -\left(Dh_x\right)^{-\top}\left(\zeta',0\right){\bf e}_N = - \left(Dg_x\right)^\top {\bf e}_N,\ h_x\left(\zeta',0\right) = y,
    \end{align*}
    since
    \begin{align*}
        -m_x(y)\cdot Dh_x\left(\zeta',0\right){\bf e}_i = {\bf e}_N\cdot {\bf e}_i = \delta_{iN},\ 1\le i\le N.
    \end{align*}
    Thus the \textit{outward unit normal field} ${\bf n}(y)$ at $y\in\Gamma_x$ is given by \textbf{(3.9)}
    \begin{align*}
        \forall h_x\left(\zeta',0\right) = y\in\Gamma_x,\ {\bf n}(y) &= -\frac{\left(Dh_x\right)^{-\top}\left(\zeta',0\right){\bf e}_N)}{\left|\left(Dh_x\right)^{-\top}\left(\zeta',0\right){\bf e}_N)\right|},\\
        \forall y\in\Gamma_x,\ {\bf n}(y) &= -\frac{\left(Dh_x\right)^{-\top}\left(h_x^{-1}(y)\right){\bf e}_N}{\left|\left(Dh_x\right)^{-\top}\left(h_x^{-1}(y)\right){\bf e}_N\right|}.
    \end{align*}
    It can be verified that ${\bf n}$ is uniquely defined on $\Gamma$ by checking that for $y\in\Gamma_x\cap\Gamma_{x'}$, ${\bf n}$ is uniquely defined by (3.9).
    \item \textbf{Boundary Integral for Sets of Class $C^1$.} The family of neighborhoods $U(x)$ associated with all the points $x$ of $\Gamma$ is an open cover of $\Gamma$.
    
    If $\Gamma = \partial\Omega$ is assumed to be compact, then there exists a finite open subcover: i.e., there exists a finite sequence of points $\{x_j;1\le j\le m\}$ of $\Gamma$ s.t. $\Gamma\subset U_1\cup\cdots\cup U_m$, where $U_j = U(x_j)$.
    
    For simplicity, index all the previous symbols by $j$ instead of $x_j$.
    
    %
    The boundary integration on $\Gamma$ is obtained by using a partition of unity $\{r_j;1\le j\le m\}$ for the family of open neighborhoods $\{U_j;1\le j\le m\}$ of $\Gamma$: \textbf{(3.10)}
    \begin{align*}
        r_j\in\mathcal{D}\left(U_j\right),\ 0\le r_j(x)\le 1,\ \sum_{j=1}^m r_j(x) = 1 \mbox{ in a neighborhood } U \mbox{ of } \Gamma,
    \end{align*}
    s.t. $\overline{U}\subset\bigcup_{j=1}^m U_j$, where $\mathcal{D}(U_j)$ is the set of all infinitely continuously differentiable functions with compact support in $U_j$.
    
    If $f\in C(\Gamma)$, then \textbf{(3.11)}
    \begin{align*}
        \left(fr_j\right)\circ h_j\in C(B_0),\ 1\le j\le m.
    \end{align*}
    Define the boundary integral of $fr_j$ on $\Gamma_j$ as \textbf{(3.12)}
    \begin{align*}
        \int_{\Gamma_j} fr_j{\rm d}\Gamma := \int_{B_0} \left(fr_j\right)\circ h_j\left(\zeta',0\right)\omega_j\left(\zeta'\right){\rm d}\zeta',\ \Gamma_j = U(x_j)\cap\Gamma,
    \end{align*}
    where $\omega_j = \omega_{x_j}$ and $\omega_x$ is the \textit{density term} \textbf{(3.13)}
    \begin{align*}
        \omega_x\left(\zeta'\right) &= \left|m_x\left(h_x\left(\zeta',0\right)\right)\right|\left|\det Dh_x\left(\zeta',0\right)\right|,\\
        m_x(y) &= -\left(Dh_x\right)^{-\top}\left(h_x^{-1}(y)\right){\bf e}_N = -\left(Dg_x\right)^\top(y){\bf e}_N.
    \end{align*}
    From this define the boundary integral of $f$ on $\Gamma$ as \textbf{(3.14)}
    \begin{align*}
        \int_\Gamma f{\rm d}\Gamma := \sum_{j=1}^m \int_{\Gamma_j} fr_j{\rm d}\Gamma.
    \end{align*}
    The expression on the RHS of (3.12) results from the parametrization of the boundary $\Gamma_j$ by $B_0$ through the diffeomorphism $h_j$.
    \item \textbf{Integral on Submanifolds.} In differential geometry there is a general procedure to define the canonical density for a $d$-dimensional submanifold $V$ in $\mathbb{R}^N$ parametrized by a $C^k$-mapping, $k\ge 1$ (cf., e.g., M. Berger and B. Gostiaux [1, Def. 2.1.1, p. 48 (56) and Prop. 6.62, p. 214 (239)]).
    
    \begin{definition}
        Let $\emptyset\ne S\subset\mathbb{R}^N$, let $k\ge 1$ and $1\le d\le N$ be integers, and let $0\le l\le 1$ be a real number.
        \begin{itemize}
            \item[(i)] Given $x\in S$, $S$ is said to be locally a $C^k$- (resp., $C^{k,l}$-) \emph{submanifold} of dimension $d$ at $x$ in $\mathbb{R}^N$ if there exists an open subset $U(x)$ of $\mathbb{R}^N$ containing $x$ and a $C^k$- (resp., $C^{k,l}$-) diffeomorphism $g_x$ from $U(x)$ onto its open image $g_x\left(U(x)\right)$, s.t.
            \begin{align*}
                g_x\left(U(x)\cap S\right) = g_x\left(U(x)\right)\cap R^d,
            \end{align*}
            where $R^d = \left\{(x_1,\ldots,x_d,0,\ldots,0)\in\mathbb{R}^N;\forall(x_1,\ldots,x_d)\in\mathbb{R}^d\right\}$.
            \item[(ii)] $S$ is said to be a $C^k$- (resp., $C^{k,l}$-) \emph{submanifold} of dimension $d$ in $\mathbb{R}^N$ if, for each $x\in S$, it is locally a $C^k$- (resp., $C^{k,l}$-) \emph{submanifold} of dimension $d$ at $x$ in $\mathbb{R}^N$.
        \end{itemize}
    \end{definition}
    For $k\ge 1$ the \textit{canonical density} $\omega_x$ on the submanifold $S$ at a point $y\in U(x)\cap S$ is given by
    \begin{align*}
        \omega_x = \sqrt{\left|\det B\right|},
    \end{align*}
    where $d\times d$ matrix $B$ is given by
    \begin{align*}
        (B)_{ij} = \left(Dh_x{\bf e}_i\right)\cdot\left(Dh_x{\bf e}_j\right),\ 1\le i,j\le d,\ h_x = g_x^{-1} \mbox{ on } g_x\left(U(x)\right).
    \end{align*}
    In the special case $d = N - 1$, it is easy to verify that
    \begin{align*}
        \left(Dh_x\right)^\top Dh_x = \begin{bmatrix}
            C^\top C & C^\top c\\ c^\top C & c^\top c
        \end{bmatrix},\ B = C^\top C,
    \end{align*}
    where $C$ is the $\left(N\times(N - 1)\right)$-matrix and $c$ is the $N$-vector defined by
    \begin{align*}
        C_{ij} = \left\{Dh_x\right\}_{ij},\ 1\le i\le N,\ 1\le j\le N - 1,\ c = Dh_x{\bf e}_N.
    \end{align*}
    Denote by $M(A)$ the matrix of cofactors associated with a matrix $A$: $M(A)_{ij}$ is equal to the determinant of the matrix obtained after deleting the $i$th row and the $j$th column times $(-1)^{i+j}$.
    
    Then $M(A) = (\det A)A^{-\top}$, $M(A^\top) = M(A)^\top$, and for 2 invertible matrices $A_1$ and $A_2$, $M(A_1A_2) = M(A_1)M(A_2)$.
    
    As a result
    \begin{align*}
        \det B = M\left(Dh_x^\top Dh_x\right)_{NN} = {\bf e}_N\cdot M\left(Dh_x^\top Dh_x\right){\bf e}_N,
    \end{align*}
    where $M\left(Dh_x^\top Dh_x\right)_{NN}$ is the $NN$-cofactor of the matrix $Dh_x^\top Dh_x$.
    
    Then
    \begin{align*}
        M\left(Dh_x^\top Dh_x\right)_{NN} = {\bf e}_N\cdot M\left(Dh_x^\top Dh_x\right){\bf e}_N = {\bf e}_N\cdot M\left(Dh_x^\top\right)M\left(Dh_x\right){\bf e}_N = \left|M\left(Dh_x\right){\bf e}_N\right|^2.
    \end{align*}
    In view of the previous considerations and (3.13)
    \begin{align*}
        \sqrt{\det B} = \left|M\left(Dh_x\right){\bf e}_N\right| = \left|\det Dh_x\right|\left|\left(Dh_x\right)^{-\top}{\bf e}_N\right| = \omega_x.
    \end{align*}
    \item \textbf{Hausdorff Measures.} Definition 3.2 gives the classical construction of a $d$-dimensional \textit{surface measure} on the boundary of a $C^1$-domain.
    
    In 1918, F. Hausdorff [1] introduced a $d$-dimensional measure in $\mathbb{R}^N$ which gives the same surface measure for smooth sub-manifolds but is defined on all measurable subsets of $\mathbb{R}^N$.
    
    When $d = N$, it is equal to the Lebesgue measure.
    
    To complete the discussion we quote the definition from F. Morgan [1, p. 8] or L. C. Evans and R. F. Gariepy [1, p. 60].
    
    \begin{definition}
        For any subset $S$ of $\mathbb{R}^N$, define the \emph{diameter} of $S$ as
        \begin{align*}
            {\rm diam}(S) := \sup\left\{\left|x - y\right|;x,y\in S\right\}.
        \end{align*}
        Let $\alpha(d)$ denote the Lebesgue measure of the unit ball in $\mathbb{R}^d$. The $d$-dimensional Hausdorff measure $H_d(A)$ of a subset $A$ of $\mathbb{R}^N$ is defined by the following process. For $\delta$ small, cover $A$ efficiently by countably many sets $S_j$ with ${\rm diam}(S_j)\le\delta$, add up all the terms
        \begin{align*}
            \alpha(d)\left(\frac{{\rm diam}\left(S_j\right)}{2}\right)^d,
        \end{align*}
        and take the limit as $\delta\to 0$:
        \begin{align*}
            H_d(A) := \lim_{\delta\downarrow 0} \inf_{A\subset\bigcup S_j,\,{\rm diam}\left(S_j\right)\le\delta} \sum_j \alpha(d)\left(\frac{{\rm diam}\left(S_j\right)}{2}\right)^d,
        \end{align*}
        where the infimum is taken over all countable covers $\{S_j\}$ of $A$ whose members have diameter at most $\delta$.
    \end{definition}
    For $0\le d < \infty$, $H_d$ is a \textit{Borel regular} measure, but not a \textit{Radon} measure for $d < N$ since it is not necessarily finite on each compact subset of $\mathbb{R}^N$.
    
    The \textit{Hausdorff dimension} of a set $A\subset\mathbb{R}^N$ is defined as \textbf{(3.15)}
    \begin{align*}
        H_{\dim}(A) := \inf\left\{0\le s < \infty;H_s(A) = 0\right\}.
    \end{align*}
    By definition $H_{\rm dim}(A)\le N$ and
    \begin{align*}
        \forall k > H_{\dim}(A),\ H_k(A) = 0.
    \end{align*}
    If a submanifold $S$ of dimension $d$, $1\le d < N$, of Definitions 3.2 and 3.3 is characterized by a single $C^1$-diffeomorphism $g$, i.e.,
    \begin{align*}
        g(S) = \mathbb{R}^d \mbox{ and }  S = h\left(\mathbb{R}^d\right),\ h = g^{-1},
    \end{align*}
    then for any Lebesgue-measurable set $E\subset\mathbb{R}^d$
    \begin{align*}
        \int_E \omega{\rm d}x = H_d\left(h(E)\right).
    \end{align*}
    This is a generalization to submanifolds of codimension greater than 1 of formula (3.12).
    
    %
    The definition of the Hausdorff measure and the Hausdorff dimension extend from integers $d$ to reals $s$, $0\le s\le\infty$, by modifying Definition 3.3 as follows.
    
    \begin{definition}
        For any real $s$, $0\le s\le\infty$, the $s$-dimensional Hausdorff measure $H_s(A)$ of a subset $A$ of $\mathbb{R}^N$ is defined by the following process. For $\delta$ small, cover $A$ efficiently by countably many sets $S_j$ with ${\rm diam}(S_j)\le\delta$, and add all the terms
        \begin{align*}
            \alpha(s)\left(\frac{{\rm diam}\left(S_j\right)}{2}\right)^d,
        \end{align*}
        where \textbf{(3.16)}
        \begin{align*}
            \alpha(s) := \frac{\pi^{s/2}}{\Gamma\left(\frac{s}{2} + 1\right)},\ \Gamma(t) := \int_0^\infty e^{-x}x^{t - 1}.
        \end{align*}
        Take the limit as $\delta\to 0$:
        \begin{align*}
            H_s(A) := \lim_{\delta\downarrow 0}\inf_{A\subset\bigcup S_j,\,{\rm diam}\left(S_j\right)\le\delta} \sum_j \alpha(s)\left(\frac{{\rm diam}\left(S_j\right)}{2}\right)^d,
        \end{align*}
        where the infimum is taken over all countable covers $\{S_j\}$ of $A$ whose members have diameter at most $\delta$. The Hausdorff dimension is defined by the same formula (3.15).
    \end{definition}
    \item \textbf{Fundamental Forms and Principal Curvatures.} Consider a set $\Omega$ locally of class $C^2$ in $\mathbb{R}^N$.
    
    Its boundary $\Gamma = \partial\Omega$ is an $(N - 1)$-dimensional submanifold of $\mathbb{R}^N$ of class $C^2$.
    
    At each point $x\in\Gamma$ there is a $C^2$-diffeomorphism $g_x$ from a neighborhood $U(x)$ of $x$ onto $B$.
    
    Denoting its inverse by $h_x = g_x^{-1}$, the \textit{covariant basis} at a point $y\in U(x)\cap\Gamma$ is defined as
    \begin{align*}
        a_\alpha(y) := \frac{\partial h_x}{\partial\zeta_\alpha}\left(\zeta',0\right),\ \alpha = 1,\ldots,N - 1,\ h_x\left(\zeta',0\right) = y,
    \end{align*}
    and $a_N$ is chosen as the \textit{inward unit normal}:
    \begin{align*}
        a_N(y) := \frac{\left(Dh_x\left(h_x^{-1}(y)\right)\right)^{-\top}{\bf e}_N}{\left|\left(Dh_x\left(h_x^{-1}(y)\right)\right)^{-\top}{\bf e}_N\right|}.
    \end{align*}
    The standard convention that the Greek indices range from 1 to $N - 1$ and the Roman indices from 1 to $N$ will be followed together with Einstein's rule of summation over repeated indices.
    
    The associated \textit{contravariant basis} $\{a^i\} = \{a^i(y)\}$ is defined from the covariant one $\{a_i\} = \{a_i(y)\}$ as
    \begin{align*}
        a^i\cdot a_k = \delta_{ij},
    \end{align*}
    where $\delta_{ij}$ is the Kronecker index function.
    
    The 1st, 2nd, and 3rd fundamental forms $a$, $b$, and $c$ are defined as
    \begin{align*}
        a_{\alpha\beta} := a_\alpha\cdot a_\beta,\ b_{\alpha\beta} := -a_\alpha\cdot a_{N,\beta},\ c_{\alpha\beta} := b_{\alpha}^\lambda b_{\lambda\beta},
    \end{align*}
    where
    \begin{align*}
        a_{N,\beta} = \frac{\partial a_N}{\partial\zeta_\beta},\ b_\alpha^\lambda = a^\lambda\cdot a^\mu b_{\mu\alpha}.
    \end{align*}
    The above definitions extend to sets of class $C^{1,1}$ for which $h_x\in C^{1,1}(B)$ and hence $h_x\in C^{1,1}(B_0)$.
    
    So by Rademacher's theorem in dimension $N - 1$ (cf., e.g., L. C. Evans and R. F. Gariepy [1]), $h_x\in W^{2,\infty}(B_0)$ and the definitions of the 2nd and 3rd fundamental form still make sense $H_{N-1}$ a.e. on $\Gamma$.
    
    The eigenvalues of $b_{\alpha\beta}$ are the $(N - 1)$ \textit{principal curvatures} $\kappa_i$, $1\le i\le N - 1$, of the submanifold $\Gamma$.
    
    The \textit{mean curvature} $H$ and the \textit{Gauss curvature} $K$ are defined as
    \begin{align*}
        H := \frac{1}{N - 1}\sum_{\alpha = 1}^{N - 1} \kappa_\alpha \mbox{ and } K := \prod_{\alpha = 1}^{N - 1} \kappa_\alpha.
    \end{align*} 
    The choice of the \textit{inner normal} for $a_N$ is necessary to make the principal curvatures of the sphere (boundary of the ball) positive.
    
    The factor $\frac{1}{N - 1}$ is used to make the mean curvature of the unit sphere equal to 1 in all dimensions.
    
    The reader should keep in mind the long-standing differences in usage between geometry and PDEs, where the outer unit normal is used in the integration-by-parts formulae for the Euclidean space $\mathbb{R}^N$.
    
    For integration by parts on submanifolds of $\mathbb{R}^N$ the sum of the principal curvatures will naturally occur rather than the mean curvature.
    
    It will be convenient to \textit{redefine} $H$ as the sum of the principal curvatures and introduce the notation $\overline{H}$ for the classical mean curvature: \textbf{(3.17)}
    \begin{align*}
        H := \sum_{\alpha = 1}^{N - 1} \kappa_\alpha,\ \overline{H} := \frac{1}{N - 1}\sum_{\alpha = 1}^{N - 1} \kappa_\alpha.
    \end{align*}
\end{enumerate}

\paragraph{Sets Globally Described by the Level Sets of a Function.}
\begin{enumerate}
    \item From the definition of a set of class $C^{k,l}$, $k\ge 1$ and $0\le l\le 1$, the set $\Omega$ can also be locally described by the level sets of the $C^{k,l}$-function \textbf{(4.1)}
    \begin{align*}
        f_x(y) := g_x(y)\cdot {\bf e}_N,
    \end{align*}
    since by definition
    \begin{align*}
        {\rm int}\,\Omega\cap U(x) &= \left\{y\in U(x);f_x(y) > 0\right\},\\
        \partial\Omega\cap U(x) &= \left\{y\in U(x);f_x(y) = 0\right\}.
    \end{align*}
    The boundary $\partial\Omega$ is the zero level set of $f_x$ and the gradient
    \begin{align*}
        \nabla f_x(y) = \left(Dg_x\right)^\top(y){\bf e}_N\ne 0
    \end{align*}
    is normal to that level set.
    \item Thus the \textit{exterior normal} to $\Omega$ is given by
    \begin{align*}
        {\bf n}(y) = -\frac{\nabla f_x(y)}{\left|\nabla f_x(y)\right|} = -\frac{(Dg_x)^\top(y){\bf e}_N}{\left|(Dg_x)^\top(y){\bf e}_N\right|} = -\frac{\left(Dh_x\left(g_x(y)\right)\right)^{-\top}{\bf e}_N}{\left|\left(Dh_x\left(g_x(y)\right)\right)^{-\top}{\bf e}_N\right|}.
    \end{align*}
    \item From this local construction of the functions $\{f_x;x\in\partial\Omega\}$, a global function on $\mathbb{R}^N$ can be constructed to characterize $\Omega$.
    
    \begin{theorem}
        Given $k\ge 1$, $0\le l\le 1$, and a set $\Omega$ in $\mathbb{R}^N$ of class $C^{k,l}$ with compact boundary, there exists a Lipschitz continuous $f:\mathbb{R}^N\to\mathbb{R}$ s.t. \textbf{(4.2)}
        \begin{align*}
            {\rm int}\,\Omega = \left\{y\in\mathbb{R}^N;f(y) > 0\right\} \mbox{ and } \partial\Omega = \left\{y\in\mathbb{R}^N;f(y) = 0\right\},
        \end{align*}
        and a neighborhood $W$ of $\partial\Omega$ s.t. \textbf{(4.3)}
        \begin{align*}
            f\in C^{k,l}(W) \mbox{ and } \nabla f\ne 0 \mbox{ on } W \mbox{ and } {\bf n} = -\frac{\nabla f}{\left|\nabla f\right|},
        \end{align*}
        where ${\bf n}$ is the outward unit normal to $\Omega$ on $\partial\Omega$.
    \end{theorem}
    \item This theorem has a converse.
    
    \begin{theorem}
        Associate with a continuous function $f:\mathbb{R}^N\to\mathbb{R}$ the set \textbf{(4.6)}
        \begin{align*}
            \Omega := \left\{y\in\mathbb{R}^N;f(y) > 0\right\}.
        \end{align*}
        Assume that \textbf{(4.7)}
        \begin{align*}
            f^{-1}(0) := \left\{y\in\mathbb{R}^N;f(y) = 0\right\}\ne\emptyset,
        \end{align*}
        and that there exists a neighborhood $V$ of $f^{-1}(0)$ s.t. $f\in C^{k,l}(V)$ for some $k\ge 1$ and $0\le l\le 1$ and that $\nabla f\ne 0$ in $f^{-1}(0)$. Then $\Omega$ is a set of class $C^{k,l}$, \textbf{(4.8)}
        \begin{align*}
            {\rm int}\,\Omega = \Omega \mbox{ and } \partial\Omega = f^{-1}(0).
        \end{align*}
    \end{theorem}
    \item The important theorem of Sard.
    
    \begin{theorem}[J. Dieudonné (3, Vol. III, sect. 16.23, p. 167)]
        Let $X$ and $Y$ be 2 differential manifolds, $f:X\to Y$ be a $C^\infty$-mapping, and $E$ be the set of critical points of $f$. Then $f(E)$ is negligible in $Y$, and $Y - f(E)$ is dense in $Y$.
    \end{theorem}
    Combining this theorem with Theorem 4.1, this means that, for almost all $t$ in the range of a $C^\infty$-function $f:\mathbb{R}^N\to\mathbb{R}$, the set
    \begin{align*}
        \left\{y\in\mathbb{R}^N;f(x) > t\right\}
    \end{align*}
    is of class $C^\infty$ in the sense of Definition 3.1.
    \item The following theorem extends and completes Sard's theorem.
    
    \begin{theorem}[H. Federer (5, Thm. 3.4.3, p. 316)]
        If $m > \nu\ge 0$ and $k\ge 1$ are integers, $A$ is an open subset of $\mathbb{R}^m$, $B\subset A$, $Y$ is a normed vector space, and
        \begin{align*}
            f:A\to Y \mbox{ is a map of class } k,\ \dim\operatorname{Im}Df(x)\le\nu \mbox{ for } x\in B,
        \end{align*}
        then
        \begin{align*}
            H_{\nu + \frac{m - \nu}{k}}\left(f(B)\right) = 0.
        \end{align*}
    \end{theorem}
\end{enumerate}

\paragraph{Sets Locally Described by the Epigraph of a Function.}
\begin{enumerate}
    \item \textbf{Local $C^0$ Epigraphs, $C^0$ Epigraphs, \& Equi-$C^0$ Epigraphs \& the Space $\mathcal{H}$ of Dominating Functions.} Let ${\bf e}_N$ in $\mathbb{R}^N$ be a \textit{unit reference vector} and introduce the following notation: \textbf{(5.1)}
    \begin{align*}
        H := \left\{{\bf e}_N\right\}^\bot,\ \forall\zeta\in\mathbb{R}^N,\ \zeta' := P_H\left(\zeta\right),\ \zeta_N := {\bf e}_N\cdot\zeta
    \end{align*}
    for the associated \textit{reference hyperplane} $H$ orthogonal to ${\bf e}_N$, the orthogonal projection $P_H$ onto $H$, and the \textit{normal component} $\zeta_N$ of $\zeta$.
    
    The vector $\zeta$ is equal to $\zeta' + \zeta_N{\bf e}_N$ and will often be denoted by $\left(\zeta',\zeta_N\right)$.
    
    In practice the vector ${\bf e}_N$ is chosen as ${\bf e}_N = (0,\ldots,0,1)$, but the actual form of the unit vector ${\bf e}_N$ is not important.
    
    %
    A \textit{direction} in $\mathbb{R}^N$ is specified by a unit vector $d\in\mathbb{R}^N$, $|d| = 1$.
    
    Alternatively, it can be specified by $A{\bf e}_N$ for some matrix $A$ in the \textit{orthogonal subgroup} of $N\times N$ matrices \textbf{(5.2)}
    \begin{align*}
        \operatorname{O}(N) := \left\{A;A^\top A = AA^\top = I\right\}.
    \end{align*}
    Conversely, for any unit vector $d\in\mathbb{R}^N$, $|d| = 1$, there exists\footnote{Complete the orthonormal basis w.r.t. $d$ by adding $N - 1$ unit vectors $d_1,\ldots,d_{N-1}$ and construct the matrix whose columns are the vectors $A = \left[d_1\ d_2\ \ldots d_{N-1}\ d\right]$ for which $A{\bf e}_N = d$.} $A\in\operatorname{O}(N)$ s.t. $d = A{\bf e}_N$.
    
    For all $x\in\mathbb{R}^N$ and $A\in\operatorname{O}(N)$, it is easy to check that $|Ax| = |x| = |A^{-1}x|$ for all $x\in\mathbb{R}^N$.
    
    \begin{definition}
        Let ${\bf e}_N$ be a unit vector in $\mathbb{R}^N$, $H$ be the hyperplane $\{{\bf e}_N\}^\bot$, and $\Omega$ be a subset of $\mathbb{R}^N$ with nonempty boundary $\partial\Omega$.
        \begin{itemize}
            \item[(i)] $\Omega$ is said to be \emph{locally a $C^0$ epigraph} if for each $x\in\partial\Omega$ there exist
            \begin{itemize}
                \item[(a)] an open neighborhood $\mathcal{U}(x)$ of $x$;
                \item[(b)] a matrix $A_x\in\operatorname{O}(N)$;
                \item[(c)] a bounded open neighborhood $V_x$ of 0 in $H$ s.t. \textbf{(5.3)}
                \begin{align*}
                    \mathcal{U}(x)\subset\left\{y\in\mathbb{R}^N;P_H\left(A_x^{-1}\left(y - x\right)\right)\in V_x\right\};
                \end{align*}
                \item[(d)] and a function $a_x\in C^0(V_x)$ s.t. $a_x(0) = 0$ and \textbf{(5.4)-(5.5)}
                \begin{align*}
                    \mathcal{U}(x)\cap\partial\Omega &= \mathcal{U}(x)\cap\left\{x + A_x\left(\zeta' + \zeta_N{\bf e}_N\right);\zeta'\in V_x,\ \zeta_N = a_x\left(\zeta'\right)\right\},\\
                    \mathcal{U}(x)\cap{\rm int}\,\Omega &= \mathcal{U}(x)\cap\left\{x + A_x\left(\zeta' + \zeta_N{\bf e}_N\right);\zeta'\in V_x,\ \zeta_N > a_x\left(\zeta'\right)\right\}.
                \end{align*}
            \end{itemize}
            \item[(ii)] $\Omega$ is said to be a \emph{$C^0$ epigraph} if it is locally a $C^0$ epigraph and the neighborhoods $\mathcal{U}(x)$ and $V_x$ can be chosen in such a way that $V_x$ and $A_x^{-1}\left(\mathcal{U}(x) - x\right)$ are independent of $x$: there exist bounded open neighborhoods $V$ of 0 in $H$ and $U$ of 0 in $\mathbb{R}^N$ s.t. $P_H(U)\subset V$ and
            \begin{align*}
                \forall x\in\partial\Omega,\ V_x = V \mbox{ and } \exists A_x\in\operatorname{O}(N) \mbox{ s.t. } \mathcal{U}(x) = x + A_xU.
            \end{align*}
            \item[(iii)] $\Omega$ is said to be \emph{an equi-$C^0$ epigraph} if it is a $C^0$ epigraph and the family of functions $\{a_x;x\in\partial\Omega\}$ is uniformly bounded and equicontinuous: \textbf{(5.6)}
            \begin{align*}
                &\exists c > 0 \mbox{ s.t. } \forall x\in\partial\Omega,\ \forall\xi'\in V,\ \left|a_x\left(\xi'\right)\right|\le c,\\
                &\forall\varepsilon > 0,\ \exists\delta > 0 \mbox{ s.t. } \forall x\in\partial\Omega,\ \forall y,\forall z\in V \mbox{ s.t. } \left|z - y\right| < \delta,\ \left|a_x(z) - a_x(y)\right| < \varepsilon.
            \end{align*}
        \end{itemize}
    \end{definition}
    
    \begin{remark}
        Conditions (5.3)-(5.5) are equivalent to the following three conditions: \textbf{(5.7)-(5.9)}
        \begin{align*}
            \mathcal{U}(x)\cap\partial\Omega&\subset\left\{x + A_x\left(\zeta' + \zeta_N{\bf e}_N\right);\zeta'\in V_x,\ \zeta_N = a_x\left(\zeta'\right)\right\},\\
            \mathcal{U}(x)\cap\operatorname{int}\Omega&\subset\left\{x + A_x\left(\zeta' + \zeta_N{\bf e}_N\right);\zeta'\in V_x,\ \zeta_N > a_x\left(\zeta'\right)\right\},\\
            \mathcal{U}(x)\cap\operatorname{int}\Omega^c&\subset\left\{x + A_x\left(\zeta' + \zeta_N{\bf e}_N\right);\zeta'\in V_x,\ \zeta_N < a_x\left(\zeta'\right)\right\}.
        \end{align*}
        In particular, $\Omega$ is locally a $C^0$ epigraph if and only if $\Omega^c$ is locally a $C^0$ epigraph.
        
        As a result, conditions (5.3), (5.4), and (5.5) are also equivalent to conditions (5.3),
        (5.4), and the following condition:
        \begin{align*}
            \mathcal{U}(x)\cap\operatorname{int}\Omega^c\subset\left\{x + A_x\left(\zeta' + \zeta_N{\bf e}_N\right);\zeta'\in V_x,\ \zeta_N < a_x\left(\zeta'\right)\right\}
        \end{align*}
        in place of condition (5.5).
    \end{remark}

    \begin{remark}
        Note that from (5.3) $P_H(A_x^{-1}(\mathcal{U}(x) - x))\subset V_x$ and that this yields the condition $P_H(U)\subset V$ in part (iii) of the definition.
    \end{remark}
    
    \begin{remark}
        It is always possible to redefine $\mathcal{U}(x) - x$ or $V_x$ to be open balls or open hypercubes centered in 0.
        
        E.g., in the case of balls, for $V_x' = B_H(0,\rho)\subset V_x$, choose the associated neighborhood $\mathcal{U}'(x) = \mathcal{U}(x)\cap\{y\in\mathbb{R}^N;P_H(A_x^{-1}(y - x))\in B_H(0,\rho)\}$.
        
        Similarly, for $\mathcal{U}'(x) = B(x,r)\subset\mathcal{U}(x)$ choose the associated neighborhood $V_x' = P_H(A_x^{-1}(B(x,r) - x)) = B_H(0,r)\subset V_x$, since, from (5.3), $V_x\supset P_H(A_x^{-1}(\mathcal{U}(x) - x))\supset P_H(A_x^{-1}B(x,r) - x) = B_H(0,r)$.
        
        In both cases properties (5.3) to (5.5) are verified for the new neighborhoods.
    \end{remark}
    
    \begin{remark}
        Sets $\Omega\subset\mathbb{R}^N$ that are locally $C^0$ epigraphs have a boundary $\partial\Omega$ with zero $N$-dimensional Lebesgue measure.
    \end{remark}
    The 3 cases considered in Definition 5.1 differ only when the boundary $\partial\Omega$ is unbounded.
    
    \begin{notation}
        \begin{itemize}
            \item[(i)] Under condition (5.3), a point $y\in\mathcal{U}(x)\subset\mathbb{R}^N$ is represented by $(\zeta',\zeta_N)\in V_x\times\mathbb{R}$, where
            \begin{align*}
                \zeta' := P_H\left(A_x^{-1}(y - x)\right)\in V_x\subset H,\ \zeta_N := A_x^{-1}(y - x)\cdot{\bf e}_N(x).
            \end{align*}
            Can also identify $H$ with $R^{N - 1} := \{(z',0)\in\mathbb{R}^N;z'\in\mathbb{R}^{N - 1}\}$.
            \item[(ii)] The graph, epigraph, and hypograph of $a_x:V_x\to\mathbb{R}$ will be denoted as follows:
            \begin{align*}
                A_x^0 &:= \left\{x + A_x(\zeta' + \zeta_N{\bf e}_N);\forall\zeta'\in V_x,\ \zeta_N = a_x(\zeta')\right\},\\
                A_x^+ &:= \left\{x + A_x(\zeta' + \zeta_N{\bf e}_N);\forall\zeta'\in V_x,\ \zeta_N > a_x(\zeta')\right\},\\
                A_x^- &:= \left\{x + A_x(\zeta' + \zeta_N{\bf e}_N);\forall\zeta'\in V_x,\ \zeta_N < a_x(\zeta')\right\}.
            \end{align*}
        \end{itemize}
    \end{notation}
    The equi-$C^0$ epigraph case is important since we shall see in Theorem 5.2 that the 3 cases of Definition 5.1 are equivalent when the boundary $\partial\Omega$ is compact.
    
    For an arbitrary boundary, the equicontinuity of the graph functions $\{a_x;x\in\partial\Omega\}$ can be expressed in terms of the continuity in 0 of a \textit{dominating function} $h$.
    
    For this purpose, define the \textit{space of dominating functions} as follows: \textbf{(5.11)}
    \begin{align*}
        \mathcal{H} := \left\{h:[0,\infty)\to[0,\infty);h(0) = 0 \mbox{ and } h \mbox{ is continuous in } 0\right\}.
    \end{align*}
    By continuity in 0, $h$ is locally bounded in 0:
    \begin{align*}
        \forall c > 0,\ \exists\rho > 0 \mbox{ s.t. } \forall\theta\in[0,\rho],\ |h(\theta)|\le c.
    \end{align*}
    The only part of the function $h$ that will really matter is in a bounded neighborhood of zero.
    
    Its extension outside to $[0,\infty)$ can be relatively arbitrary.
    
    \begin{theorem}
        Let $\Omega$ be a $C^0$-epigraph. The following conditions are equivalent:
        \begin{itemize}
            \item[(i)] $\Omega$ is an equi-$C^0$-epigraph.
            \item[(ii)] There exist $\rho > 0$ and $h\in\mathcal{H}$ s.t. $B_H(0,\rho)\subset V$ and for all $x\in\partial\Omega$ \textbf{(5.12)}
            \begin{align*}
                \forall\zeta',\xi'\in B_H(0,\rho) \mbox{ s.t. } |\xi' - \zeta'| < \rho,\ |a_x(\xi') - a_x(\zeta')|\le h(|\xi' - \zeta'|).
            \end{align*}
            \item[(iii)] There exist $\rho > 0$ and $h\in\mathcal{H}$ s.t. $B_H(0,\rho)\subset V$ and for all $x\in\partial\Omega$ \textbf{(5.13)}
            \begin{align*}
                \forall\xi'\in B_H(0,\rho),\ a_x(\xi')\le h(|\xi'|).
            \end{align*}
            Inequalities (5.12) and (5.13) for $h\in\mathcal{H}$ are also verified with $\limsup h$ that also belongs to $\mathcal{H}$.
        \end{itemize}
    \end{theorem}

    \begin{theorem}
        When the boundary $\partial\Omega$ is compact, the 3 properties of Definition 5.1 are equivalent:
        \begin{align*}
            \mbox{locally } C^0 \mbox{ epigraph}\Leftrightarrow C^0 \mbox{ epigraph}\Leftrightarrow\mbox{ equi-}C^0 \mbox{ epigraph}. 
        \end{align*}
        Under such conditions, there exists neighborhoods $V$ of 0 in $H$ and $U$ of 0 in $\mathbb{R}^N$ s.t. $P_HU\subset V$ and the neighborhoods $V_x$ of 0 in $H$ and the neighborhoods $\mathcal{U}(x)$ of $x$ in $\mathbb{R}^N$ can be chosen of the form \textbf{(5.16)}
        \begin{align*}
            V_x = V \mbox{ and } \exists A_x\in\operatorname{O}(N) \mbox{ s.t. } \mathcal{U}(x) = x + A_xU.
        \end{align*}
        Moreover, the family of functions $\{a_x:V\to\mathbb{R};x\in\partial\Omega\}$ can be chosen uniformly bounded and equicontinuous w.r.t. $x\in\partial\Omega$. In addition, there exist $\rho > 0$ s.t. $B_H(0,\rho)\subset V$ and a function $h\in\mathcal{H}$ s.t. \textbf{(5.17)}
        \begin{align*}
            \forall x\in\partial\Omega,\ \forall\xi',\zeta'\in B_H(0,\rho),\ |\xi' - \zeta'| < \rho,\ |a_x(\xi') - a_x(\zeta')|\le h(|\xi' - \zeta'|).
        \end{align*}
    \end{theorem}
    \item \textbf{Local $C^{k,l}$-Epigraphs \& Hölderian/Lipschitzian Sets.} Extend the 3 definitions of $C^0$ epigraphs to $C^{k,l}$ epigraphs.
    
    $C^{0,1}$ is the important family of \textit{Lipschitzian sets} and $C^{0,l}$, $0 < l < 1$, of \textit{Hölderian sets}.
    
    \begin{definition}
        Let $\Omega$ be a subset of $\mathbb{R}^N$ s.t. $\partial\Omega\ne\emptyset$. Let $k\ge 0$, $0\le l\le 1$.
        \begin{itemize}
            \item[(i)] $\Omega$ is said to be \emph{locally a $C^{k,l}$ epigraph} if for each $x\in\partial\Omega$ there exist
            \begin{itemize}
                \item[(a)] an open neighborhood $\mathcal{U}(x)$ of $x$;
                \item[(b)] a matrix $A_x\in\operatorname{O}(N)$;
                \item[(c)] a bounded open neighborhood $V_x$ of 0 in $H$ s.t. \textbf{(5.20)}
                \begin{align*}
                    \mathcal{U}(x)\subset\left\{y\in\mathbb{R}^N;P_H\left(A_x^{-1}\left(y - x\right)\right)\in V_x\right\};
                \end{align*}
                \item[(d)] and a function $a_x\in C^{k,l}(V_x)$ s.t. $a_x(0) = 0$ and \textbf{(5.21)}
                \begin{align*}
                    \mathcal{U}(x)\cap\partial\Omega &= \mathcal{U}(x)\cap\left\{x + A_x\left(\zeta' + \zeta_N{\bf e}_N\right);\zeta'\in V_x,\ \zeta_N = a_x\left(\zeta'\right)\right\},\\
                    \mathcal{U}(x)\cap\operatorname{int}\Omega &= \mathcal{U}(x)\cap\left\{x + A_x\left(\zeta' + \zeta_N{\bf e}_N\right);\zeta'\in V_x,\ \zeta_N > a_x\left(\zeta'\right)\right\},
                \end{align*}
                where $\zeta' = \left(\zeta_1,\ldots,\zeta_{N - 1}\right)\in\mathbb{R}^{N - 1}$.
            \end{itemize}
            We shall say that $\Omega$ is \emph{locally Lipschitzian} in the $C^{0,1}$ case and \emph{locally Hölderian of index $l$} in the $C^{0,l}$, $0 < l < 1$, case.
        \end{itemize}
        \item[(ii)] $\Omega$ is said to be a \emph{$C^{k,l}$ epigraph} if it is locally a $C^{k,l}$ epigraph and the neighborhoods $\mathcal{U}(x)$ and $V_x$ can be chosen in such a way that $V_x$ and $A_x^{-1}\left(\mathcal{U}(x) - x\right)$ are independent of $x$: there exist bounded open neighborhoods $V$ of 0 in $H$ and $U$ of 0 in $\mathbb{R}^N$ s.t. $P_H(U)\subset V$ and
        \begin{align*}
            \forall x\in\partial\Omega,\ V_x := V \mbox{ and } \exists A_x\in\operatorname{O}(N) \mbox{ s.t. } \mathcal{U}(x) := x + A_xU.
        \end{align*}
        We shall say that $\Omega$ is Lipschitzian in the $C^{0,1}$ case and Hölderian of index $l$ in the $C^{0,l}$, $0 < l < 1$, case.
        \item[(iii)] Given $(k,l)\ne(0,0)$, $\Omega$ is said to be \emph{an equi-$C^{k,l}$ epigraph} if it is a $C^{k,l}$ epigraph and \textbf{(5.23)}
        \begin{align*}
            \exists c > 0,\ \forall x\in\partial\Omega,\ \left\|a_x\right\|_{C^{k,l}(V)}\le c,
        \end{align*}
        where $\|f\|_{C^{k,l}(V)}$ is the norm on the space $C^{k,l}(V)$ as defined in (2.21):
        \begin{align*}
            \|f\|_{C^{k,l}(V)} &:= \|f\|_{C^k(V)} + \max_{0\le\left|\alpha\right|\le k} \sup_{x,y\in V,\,x\ne y} \frac{\left|\partial^\alpha f(y) - \partial^\alpha f(x)\right|}{\left|y - x\right|^l}, \mbox{ if } 0 < l\le 1,\\
            C^{k,0}(V) &:= C^k(V) \mbox{ and } \|f\|_{C^{k,0}(V)} := \|f\|_{C^k(V)}.
        \end{align*}
        $\Omega$ is said to be \emph{an equi-$C^{0,0}$ epigraph} if it is an equi-$C^0$ epigraph in the sense of Definition 5.1 (iii).
        
        We shall say that $\Omega$ is \emph{equi-Lipschitzian} in the $C^{0,1}$ case and \emph{equi-Hölderian of index $l$} in the $C^{0,l}$, $0 < l < 1$, case.
    \end{definition}

    \begin{remark}
        In Definition 5.1 (iii) for $(k,l)\ne(0,0)$ it follows from condition (5.23) that the functions $\{a_x;x\in\partial\Omega\}$ are uniformly equicontinuous on $V$.
        
        In the case $(k,l) = 0,0)$ this is no longer true.
        
        So we have to include it by using Definition 5.1 (iii).
    \end{remark}

    \begin{theorem}
        When the boundary $\partial\Omega$ is compact, the three types of sets of Definition 5.2 are equi-$C^{k,l}$ epigraphs in the sense of Definitions 5.2 (iii) and 5.1 (iii).
    \end{theorem}

    Sets $\Omega$ that are \textit{locally a $C^{k,l}$ epigraph} are Lebesgue measurable in $\mathbb{R}^N$, their ``volume'' is locally finite, and their boundary has zero ``volume''.
    
    \begin{theorem}
        Let $k\ge 0$, let $0\le l\le 1$, and let $\Omega$ be \emph{locally a $C^{k,l}$ epigraph}.
        \begin{itemize}
            \item[(i)] The complement $\Omega^c$ is also locally a $C^{k,l}$ epigraph, and \textbf{(5.24)}
            \begin{align*}
                \operatorname{int}\Omega&\ne\emptyset,\ \overline{\operatorname{int}\Omega} = \overline{\Omega}, \mbox{ and } \partial\left(\operatorname{int}\Omega\right) = \partial\Omega,\\
                \operatorname{int}\Omega^c&\ne\emptyset,\ \overline{\operatorname{int}\Omega^c} = \overline{\Omega^c}, \mbox{ and } \partial\left(\operatorname{int}\Omega^c\right) = \partial\Omega.
            \end{align*}
            Moreover, for all $x\in\partial\Omega$ \textbf{(5.25)}
            \begin{align*}
                \operatorname{m}\left(\mathcal{U}(x)\cap\partial\Omega\right) = 0\Rightarrow\operatorname{m}\left(\partial\Omega\right) = 0,
            \end{align*}
            where $\operatorname{m}$ is the $N$-dimensional Lebesgue measure.
            \item[(ii)] If $\Omega\ne\emptyset$ is open, then $\Omega = \operatorname{int}\overline{\Omega}$.
        \end{itemize}
    \end{theorem}

    \begin{lemma}
        \begin{itemize}
            \item[(i)] $\partial\left(\operatorname{int}\Omega\right) = \partial\Omega\Leftrightarrow\overline{\operatorname{int}\Omega} = \overline{\Omega}$.
            \item[(ii)] $\partial\left(\operatorname{int}\Omega^c\right) = \partial\Omega\Leftrightarrow\overline{\operatorname{int}\Omega^c} = \overline{\Omega^c}$.
        \end{itemize}
    \end{lemma}
    \item \textbf{Local $C^{k,l}$-Epigraphs \& Sets of Class $C^{k,l}$.} Sets or domains which are locally the epigraph of a $C_{k,l}$-function, $k\ge 0$, $0\le l\le 1$ (resp., locally Lipschitzian), are sets of class $C^{k,l}$ (resp., $C^{0,1}$).
    
    However, we shall see from Examples 5.1 and 5.2 that a domain of class $C^{0,1}$ is generally not locally the epigraph of a Lipschitzian function.
    
    \begin{theorem}
        Let $l$, $0\le l\le 1$, be a real number.
        \begin{itemize}
            \item[(i)] If $\Omega$ is locally a $C^{k,l}$ epigraph for $k\ge 0$, then it is locally of class $C^{k,l}$.
            \item[(ii)] If $\Omega$ is locally of class $C^{k,l}$ for $k\ge 1$, then it is locally a $C^{k,l}$ epigraph.
        \end{itemize}
    \end{theorem}
    
    \begin{example}[R. Adams, N. Aronszajn, and K. T. Smith (1)]
        Consider the open convex (Lipschitzian) set
        \begin{align*}
            \Omega_0 = \left\{\rho e^{i\theta};0 < \rho < 1,\ 0 < \theta < \frac{\pi}{2}\right\}
        \end{align*}
        and its image $\Omega = T(\Omega_0)$ by the $C^{0,1}$-diffeomorphism (see Fig. 2.3)
        \begin{align*}
            T\left(\rho e^{i\theta}\right) = \rho e^{i\left(\theta - \log\rho\right)},\ T^{-1}\left(\rho e^{i\theta}\right) = \rho e^{i\left(\theta + \log\rho\right)}.
        \end{align*}
        It is readily seen that, as $\rho$ goes to zero, the image of the 2 pieces of the boundary of $\Omega_0$ corresponding to $\theta = 0$ and $\theta = \frac{\pi}{2}$ begin to spiral around the origin.
        
        As a result $\Omega$ is not locally the epigraph of a function at the origin.
    \end{example}
    
    \begin{example}
        This 2nd example can be found in F. Murat and J. Simon [1], where it is attributed to M. Zerner.
        
        Consider the Lipschitzian function $\lambda$ defined on $[0,1]$ as follows: $\lambda(0) = 0$, and on each interval $\left[\frac{1}{3^{n+1}},\frac{1}{3^n}\right]$
        \begin{equation*}
            \lambda(s) := \left\{\begin{split}
                &2\left(s - \frac{1}{3^{n+1}}\right), &&\frac{1}{3^{n+1}}\le s\le\frac{2}{3^{n+1}},\\
                -&2\left(s - \frac{1}{3^n}\right), &&\frac{2}{3^{n+1}}\le s\le\frac{1}{3^n},
            \end{split}\right.
        \end{equation*}
        where $n$ ranges over all integers $n\ge 0$.
        
        Associate with $\lambda$ and a real $\delta > 0$ the set
        \begin{align*}
            \Omega := \left\{\left(x_1,x_2\right)\in\mathbb{R}^2;0 < x_1 < 1,\ \left|x_2 - \lambda\left(x_1\right)\right| < \delta x_1\right\}
        \end{align*}
        The set $\Omega$ is the image of the triangle
        \begin{align*}
            \Omega_0 := \left\{\left(x_1,x_2\right)\in\mathbb{R}^2;0 < x_1 < 1,\ \left|x_2\right| < \delta x_1\right\}
        \end{align*}
        through the $C^{0,1}$-homeomorphism
        \begin{align*}
            T\left(x_1,x_2\right) := \left(x_1,x_2 + \lambda\left(x_1\right)\right),\ T^{-1}\left(y_1,y_2\right) = \left(y_1,y_2 - \lambda\left(y_1\right)\right).
        \end{align*}
        Since the triangle $\Omega_0$ is Lipschitzian, its image is a set of class $C^{0,1}$.
        
        But $\Omega$ is not locally Lipschitzian in $(0,0)$ since $\Omega$ zigzags like a lightning bolt as it gets closer to the origin.
        
        Thus, however small the neighborhood around $(0,0)$, a direction ${\bf e}_N(0,0)$ cannot be found to make the domain locally the epigraph of a function.
    \end{example}
    \item \textbf{Locally Lipschitzian Sets: Some Examples \& Properties.} The important subfamily of sets that are locally a Lipschitzian epigraph (i.e., locally a $C^{0,1}$ epigraph) enjoys most of the properties of smooth domains.
    
    Convex sets and domains that are locally the epigraph of a $C^1$- or smoother function are locally Lipschitzian.
    
    In a bounded path-connected Lipschitzian set, the geodesic distance between any 2 points is uniformly bounded.
    
    Their boundary has zero volume and locally finite boundary measure (cf. Theorem 5.7 below).
    
    Sobolev spaces defined on $\Omega$ have linear extension to $\mathbb{R}^N$.
    \begin{enumerate}
        \item \textbf{Examples \& Continuous Linear Extensions.} According to this definition the whole space $\mathbb{R}^N$ and the closed unit ball with its center or a small crack removed are not locally Lipschitzian since in the 1st case the boundary is empty and in the 2nd case the conditions cannot be satisfied at the center or along the crack, which is a part of the boundary.
        
        Similarly the set
        \begin{align*}
            \Omega := \bigcup_{n = 1}^\infty \Omega_n,\ \Omega_n := \left\{y\in\mathbb{R}^N;\left|y - \frac{1}{2^n}\right| < \frac{1}{2^{n+2}}\right\}
        \end{align*}
        is not locally Lipschitzian since the conditions of Definition 5.2 (i) are not satisfied in $0\in\partial\Omega$.
        
        However, the set
        \begin{align*}
            \Omega := \bigcup_{n = 1}^\infty \Omega_n,\ \Omega_n := \left\{y\in\mathbb{R}^N;\left|y - n\right| < \frac{1}{2^{n+2}}\right\}
        \end{align*}
        is locally Lipschitzian, but not Lipschitzian or equi-Lipschitzian in the sense of Definition 5.2.
        
        %
        1 of the important properties of a bounded (open) Lipschitzian domain $\Omega$ is the existence of a \textit{continuous linear extension} of functions of the Sobolev space $H^k(\Omega)$ from $\Omega$ to functions in $H^k(\mathbb{R}^N)$ defined on $\mathbb{R}^N$, i.e. \textbf{(5.28)}
        \begin{align*}
            E:H^k(\Omega)\to H^k\left(\mathbb{R}^N\right) \mbox{ and } \forall\varphi\in H^k(\Omega),\ \left.\left(E\varphi\right)\right|_\Omega = \varphi,
        \end{align*}
        (cf. the Calderón extension theorem in R. A. Adams [1, p. 83, Thm. 4.32, p. 91] and J. Nečas [1, Thm. 3.10, p. 80]).
        
        This property is also important in the existence of optimal domains when it is uniform for a given family.
        
        E.g., this will occur for the family of domains satisfying the uniform cone property of section 6.4.1.
        \item \textbf{Convex Sets.}
        \begin{theorem}
            Any convex subset $\Omega$ of $\mathbb{R}^N$ s.t. $\Omega\ne\mathbb{R}^N$ and $\operatorname{int}\Omega\ne\emptyset$ is locally Lipschitzian, and for each $x\in\partial\Omega$ the neighborhood $V_x$ and the function $a_x$ can be chosen convex.
        \end{theorem}
        \item \textbf{Boundary Measure \& Integral for Lipschitzian Sets.} The \textit{boundary measure} on $\partial\Omega$, which was defined in Sect. 3.2 from the local $C^1$-diffeomorphism, can also be defined from the local graph representation of the boundary and extended to the Lipschitzian case.
        
        %
        Assuming that $\partial\Omega$ is compact, there is a finite subcover of open neighborhoods $\mathcal{U}(x)$ for $\partial\Omega$ that can be represented by a finite family of Lipschitzian graphs.
        
        Let $\{\mathcal{U}_j\}_{j=1}^m$, $\mathcal{U}_j := \mathcal{U}(x_j)$, be a finite open cover of $\partial\Omega$ corresponding to some finite sequence $\{x_j\}_{j=1}^m$ of points of $\partial\Omega$.
        
        Denote by ${\bf e}_N$, $H$, $\{V_j\}_{j=1}^m$, $V_j = V_{x_j}$, and $\{a_j\}$, $a_j = a_{x_j}$, the associated elements of Definition 5.2.
        
        Introduce the notation \textbf{(5.29)}
        \begin{align*}
            \Gamma = \partial\Omega,\ \Gamma_j = \Gamma\cap\mathcal{U}_j,\ 1\le j\le m,
        \end{align*}
        and the map $h_j = h_{x_j}$ and its inverse $g_j = g_{x_j}$ as defined in (5.26)-(5.27):
        \begin{align*}
            h_x(\xi) &:= x + A_x\left(\xi' + \left[\xi_N + a_x\left(\xi'\right)\right]{\bf e}_N\right),\\
            g_x(y) &:= \left(P_HA_x^{-1}\left(y - x\right),{\bf e}_N\cdot A_x^{-1}\left(y - x\right) - a_x\left(P_HA_x^{-1}\left(y - x\right)\right)\right).
        \end{align*}
        By Rademacher's theorem the Lipschitzian function $a_x$ is differentiable almost everywhere in $V_x$ and belongs to $W^{1,\infty}(V_x)$ (cf., e.g., L. C. Evans and R. F. Gariepy [1]).
        
        Since $h_j$ is defined from the Lipschitzian function $a_j$, $Dh_j$ is defined a.e. in $V_x\times\mathbb{R}$, but also $H_{N-1}$ a.e. on $V_x\times\{0\}$.
        
        Thus the canonical density for $C^1$-domains still makes sense and is given by the same formula (3.13): \textbf{(5.30)}
        \begin{align*}
            \omega_j(\zeta') := \omega_{x_j}(\zeta') = \left|\det Dh_j\left(\zeta',0\right)\right|\left|\left(Dh_j\right)^{-\top}\left(\zeta',0\right){\bf e}_N\right|.
        \end{align*}
        It is easy to verify that for almost all $\xi'\in V_j$
        \begin{align*}
            Dh_j\left(\xi',\xi_N\right) = A_j\begin{bmatrix}
                1 & 0 & \cdots & \cdots & 0\\ 0 & \ddots & & & \vdots\\ \vdots & & \ddots & & \vdots\\ 0 & & & \ddots & 0\\ \partial_1a_j(\xi') & \partial_2a_j(\xi') & \cdots & \partial_{N - 1}a_j(\xi') & 1
            \end{bmatrix},\ \det Dh_j\left(\xi',\xi_N\right) = 1,
        \end{align*}
        where $\partial_ia_j$ is the partial derivative of $a_j$ w.r.t. the $i$th component of $\zeta' = \left(\zeta_1,\ldots,\zeta_{N - 1}\right)\in V_j$.
        
        The matrix $Dh_j$ is invertible and its coefficients belong to $L^\infty$ and $(A_j)^{-\top} = A_j$.
        
        By direct computation \textbf{(5.31)}
        \begin{align*}
            \left(Dh_j\right)^{-\top}\left(\xi',\xi_N\right) = A_j\begin{bmatrix}
                1 & 0 & \cdots & 0 & -\partial_1a_j(\xi')\\ 0 & \ddots & & \vdots & -\partial_2a_j(\xi')\\ \vdots & & \ddots & 0 & \vdots\\ \vdots & & & \ddots & -\partial_{N - 1}a_j(\xi')\\
                0 & 0 & \cdots & 0 & 1
            \end{bmatrix},
        \end{align*}
        and finally \textbf{(5.32)}
        \begin{align*}
            \omega_j(\zeta') = \left|\left(Dh_j\right)^{-\top}\left(\zeta',0\right){\bf e}_N\right| = \left|(A_j)^{-1}\left(Dh_j\right)^{-\top}\left(\zeta',0\right){\bf e}_N\right| = \sqrt{1 + \left|\nabla a_j\left(\zeta'\right)\right|^2}.
        \end{align*}
        Let $\{r_1,\ldots,r_m\}$ be a partition of unity for the $\mathcal{U}_j$'s, i.e., \textbf{(5.33)}
        \begin{align*}
            r_j\in\mathcal{D}\left(\mathcal{U}_j\right),\ 0\le r_j(x)\le 1,\ \sum_{j=1}^m r_j(x) = 1 \mbox{ in a neighborhood } \mathcal{U} \mbox{ of } \Gamma,
        \end{align*}
        s.t. $\overline{\mathcal{U}}\subset\bigcup_{j = 1}^m \mathcal{U}_j$.
        
        For any function $f$ in $C^0(\Gamma)$ the functions $f_j$ and $f_j\circ h_j$ defined as \textbf{(5.34)}
        \begin{equation*}
            \left\{\begin{split}
                f_j(y) &= f(y)r_j(y), &&y\in\Gamma_j\\
                f_j\circ h_j\left(\zeta',0\right) &= f_j\left(x + A_j\left(\xi' + a_j\left(\xi'\right){\bf e}_N\right)\right), &&\zeta'\in V_j,
            \end{split}\right.
        \end{equation*}
        respectively, belong to $C^0(\Gamma_j)$ and $C^0(V_j)$.
        
        The integral of $f$ on $\Gamma$ is then defined as \textbf{(5.35)}
        \begin{align*}
            \int_\Gamma f{\rm d}\Gamma := \sum_{j=1}^m \int_{\Gamma_j} f_j{\rm d}\Gamma_j,\ \int_{\Gamma_j} f_j{\rm d}\Gamma := \int_{V_j} f_j\left(h_j\left(\zeta',0\right)\right)\omega_j\left(\zeta'\right){\rm d}\zeta'.
        \end{align*}
        Since $\omega_j\in L^\infty(V_j)$, $1\le j\le m$, this integral is also well-defined for all $f$ in $L^1(\Gamma)$; i.e., the function $f_j\circ h_j\omega_j$ belongs to $L^1(V_j)$ for all $j$, $1\le j\le m$.
        
        %
        As in Sect. 2, the tangent plane to $\Gamma_j$ at $y = h_j\left(\zeta',0\right)$ is defined by
        \begin{align*}
            Dh_j\left(\zeta',0\right)A_jH,
        \end{align*}
        where $A_jH$ is the hyperplane orthogonal to $A_j{\bf e}_N$.
        
        An outward normal field to $\Gamma_j$ is given by
        \begin{align*}
            m_x(\zeta) = -\left(Dh_j\right)^{-\top}\left(\zeta',\zeta_N\right){\bf e}_N = A_j\begin{bmatrix}
                \partial_1a_j\left(\zeta'\right)\\ \vdots\\ \partial_{N - 1}a_j\left(\zeta'\right)\\ -1
            \end{bmatrix},
        \end{align*}
        and the unit outward normal to $\Gamma_j$ at $y = h_j\left(\zeta',0\right)\in\Gamma_j$ is given by \textbf{(5.36)}
        \begin{align*}
            {\bf n}(y) = {\bf n}\left(h_j\left(\zeta',0\right)\right) = \frac{1}{\sqrt{1 + \left|\nabla a_j\left(\zeta'\right)\right|^2}}A_j\begin{bmatrix}
                \partial_1a_j\left(\zeta'\right)\\ \vdots\\ \partial_{N - 1}a_j\left(\zeta'\right)\\ -1
            \end{bmatrix}.
        \end{align*}
        The matrix $A_j$ can be removed by redefining the functions $a_x$ on the space $A_jH$ orthogonal to $A_j{\bf e}_N$.
        
        The formulae also hold for $C^{k,l}$-domains, $k\ge 1$.
        
        %
        We have seen that sets that are locally a \textit{$C^{k,l}$ epigraph}, $k\ge 0$, have nice properties.
        
        E.g., their boundary has zero ``volume''.
        
        However, they generally have no locally finite ``surface measure'' (cf. M. C. Delfour, N. Doyon, and J.-P. Zolésio [1, 2] for specific examples) as we shall see in Sect. 6.5.
        
        Fortunately, Lipschitzian sets have locally finite ``surface measure''.
        
        \begin{theorem}
            Let $\Omega\subset\mathbb{R}^N$ be locally Lipschitzian. Then \textbf{(5.37)}
            \begin{align*}
                H_{N - 1}\left(\mathcal{U}(x)\cap\partial\Omega\right)\le\sqrt{1 + c_x^2}m_{N - 1}\left(V_x\right),
            \end{align*}
            where $m_{N-1}$ and $H_{N-1}$ are the respective $(N - 1)$-dimensional Lebesgue measure and $(N - 1)$-dimensional Hausdorff measure, and $c_x > 0$ is the Lipschitz constant of $a_x$ in $V_x$.
            
            If, in addition, $\partial\Omega$ is compact, then \textbf{(5.38)}
            \begin{align*}
                H_{N - 1}\left(\partial\Omega\right) < \infty.
            \end{align*}
        \end{theorem}
        \item \textbf{Geodesic Distance in a Domain and in Its Boundary.}
        
        \begin{theorem}
            Let $\Omega\subset\mathbb{R}^N$ be bounded, open, and locally Lipschitzian.
            \begin{itemize}
                \item[(i)] If $\Omega$ is path-connected, then \textbf{(5.39)}
                \begin{align*}
                    \exists c_\Omega,\ \forall x,y\in\overline{\Omega},\ \operatorname{dist}_{\overline{\Omega}}(x,y)\le c_\Omega\left|x - y\right|.
                \end{align*}
                \item[(ii)] If $\partial\Omega$ is path-connected, then \textbf{(5.40)}
                \begin{align*}
                    \exists c_{\partial\Omega},\ \forall x,y\in\partial\Omega,\ \operatorname{dist}_{\partial\Omega}(x,y)\le c_{\partial\Omega}\left|x - y\right|.
                \end{align*}
            \end{itemize}
        \end{theorem}
        
        \begin{remark}
            The boundary of a path-connected, bounded, open, and locally Lipschitzian set is generally not path-connected.
            
            As an example consider the set $\{x\in\mathbb{R}^2;1 < |x| < 2\}$ whose boundary is made up of the 2 disjoint circles of radii 1 and 2.
        \end{remark}
    
        \begin{remark}
            An open subset of $\mathbb{R}^N$ with compact path-connected boundary is generally not path-connected as can be seen from a domain in $\mathbb{R}^2$ made up of 2 tangent squares.
        \end{remark}
        \item \textbf{Nonhomogeneous Neumann \& Dirichlet Problems.} With the help of the previous definition of boundary measure in Sect. 5.4.3, we can now make sense of the nonhomogeneous Neumann and Dirichlet problems (for the Laplace equation) in Lipschitzian domains.
        
        Assuming that the functions $a_j$ belong to $W^{1,\infty}(V_j)$, it can be shown that the classical Stokes divergence theorem holds for such domains.
        
        Given a bounded smooth domain $D$ in $\mathbb{R}^N$ and a Lipschitzian domain $\Omega$ in $D$, then \textbf{(5.41)}
        \begin{align*}
            \forall\vec{\varphi}\in C^1\left(\overline{D},\mathbb{R}^N\right),\ \int_\Omega \nabla\cdot\vec{\varphi}{\rm d}x = \int_\Gamma \vec{\varphi}\cdot{\bf n}{\rm d}\Gamma,
        \end{align*}
        where the outward unit normal field ${\bf n}$ is defined by (5.36) for almost all $y\in\Gamma_j$ (with $y = h_x\left(\zeta',0\right)$, $\zeta'\in V_j$) as
        \begin{align*}
            {\bf n}(y) = A_j\frac{\left(\partial_1a_j\left(\zeta'\right),\ldots,\partial_{N - 1}a_j\left(\zeta'\right),-1\right)^\top}{\sqrt{1 + \left|\nabla a_j\left(\zeta'\right)\right|^2}}.
        \end{align*}
        The trace of a function $g$ in $W^{1,1}(D)$ on $\Gamma$ is defined through a $W^{1,\infty}(D)^N$-extension $N$ of the normal ${\bf n}$ as \textbf{(5.42)}
        \begin{align*}
            \forall\varphi\in\mathcal{D}\left(\mathbb{R}^N\right),\ \int_\Gamma g\varphi{\rm d}\Gamma := \int_\Omega \nabla\cdot\left(g\varphi N\right){\rm d}x,
        \end{align*}
        (cf. S. Agmon, A. Douglis, and L. Nirenberg [1, 2]).
        
        The RHS is well-defined in the usual sense so that by the Stokes divergence theorem we obtain the trace $g|_\Gamma$ defined on $\Gamma$.
        
        This trace is uniquely defined, denoted by $\gamma_\Gamma g$, and \textbf{(5.43)}
        \begin{align*}
            \gamma_\Gamma\in\mathcal{L}\left(W^{1,1}(\Omega),L^1(\Gamma)\right).
        \end{align*}
        Let $\Omega$ be a Lipschitzian domain in $D$.
    \end{enumerate}
\end{enumerate}

\paragraph{Sets Locally Described by a Geometric Property.}
\begin{enumerate}
    \item A large class of sets $\Omega$ can be characterized by geometric \textit{segment properties}.
    
    The basic \textit{segment property} is equivalent to the property that the set is locally a $C^0$ epigraph (cf. Sect. 6.2).
    
    E.g., it is sufficient to get the density of $C^k(\overline{\Omega})$ in the Sobolev space $W^{m,p}(\Omega)$ for any $m\ge 1$ and $k\ge m$.
    
    In this section, we establish the equivalence of the \textit{segment}, the \textit{uniform segment}, and the more recent\footnote{This was introduced starting with the uniform cusp property in M. C. Delfour and J.-P. Zolésio [37] in 2001 and further refined and generalized in the sequence of papers by M. C. Delfour, N. Doyon, and J.-P. Zolésio [1, 2] in 2005 and M. C. Delfour and J.-P. Zolésio [43] in 2007.} \textit{uniform fat segment} properties with the respective \textit{locally} $C^0$, the $C^0$, and the \textit{equi-$C^0$ epigraph} properties (cf. Sects. 6.2-6.4).
    
    Under the uniform fat segment property the local functions all have the same modulus of continuity specified by the continuity of the dominating function at the origin (cf. Sect. 6.3).
    
    For sets with a compact boundary the 3 segment properties are equivalent (cf. Sect. 6.1).
    
    However, for sets with an unbounded boundary the uniform segment property is generally too \textit{meager} to make the local epigraphs uniformly bounded and equicontinuous.
    
    %
    Sect. 6.4 specializes the uniform fat segment property to the uniform \textit{cusp} and \textit{cone properties}, which imply that $\partial\Omega$ is, respectively, an equi-Hölderian and equi-Lipschitzian epigraph.
    
    %
    Sect. 6.5 discusses the existence of a locally finite boundary measure.
    
    We have already seen in Sect. 5.4.3 that Lipschitzian sets have a locally finite boundary or surface (that is Hausdorff $H_{N - 1}$) measure.
    
    So we can make sense of boundary conditions associated with a PDE on the domain.
    
    In general, this is not true for Hölderian domains.
    
    To illustrate that point, we construct examples of Hölderian sets for which the Hausdorff dimension of the boundary is strictly greater than $N - 1$ and hence $H_{N - 1}(\partial\Omega) = +\infty$.
    \item ${\bf e}_N$ is a unit vector in $\mathbb{R}^N$ and $H = \{{\bf e}_N\}^\bot$.
    
    The \textit{open segment} between 2 distinct points $x$ and $y$ of $\mathbb{R}^N$ will be denoted by
    \begin{align*}
        (x,y) := \left\{x + t\left(y - x\right);\forall t,\ 0 < t < 1\right\}.
    \end{align*}
    
    \begin{definition}
        Let $\Omega$ be a subset of $\mathbb{R}^N$ s.t. $\partial\Omega\ne\emptyset$.
        \begin{itemize}
            \item[(i)] $\Omega$ is said to satisfy the \emph{segment property} if
            \begin{align*}
                \forall x\in\partial\Omega,\ \exists r > 0,\ \exists\lambda > 0,\ \exists A_x\in\operatorname{O}(N), \mbox{ s.t. } \forall y\in B(x,r)\cap\overline{\Omega},\ \left(y,y + \lambda A_x{\bf e}_N\right)\subset\operatorname{int}\Omega.
            \end{align*}
            \item[(ii)] $\Omega$ is said to satisfy the \emph{uniform segment property} if
            \begin{align*}
                \exists r > 0,\ \exists\lambda > 0 \mbox{ s.t. } \forall x\in\partial\Omega,\ \exists A_x\in\operatorname{O}(N),
            \end{align*}
            s.t.
            \begin{align*}
                \forall y\in B(x,r)\cap\overline{\Omega},\ \left(y,y + \lambda A_x{\bf e}_N\right)\subset\operatorname{int}\Omega.
            \end{align*}
            \item[(iii)] $\Omega$ is said to satisfy the \emph{uniform fat segment property} if there exist $r > 0$, $\lambda > 0$, and an open region $\mathcal{O}$ of $\mathbb{R}^N$ containing the segment $(0,\lambda {\bf e}_N)$ and not 0 s.t. for all $x\in\partial\Omega$, \textbf{(6.1)}
            \begin{align*}
                \exists A_x\in\operatorname{O}(N) \mbox{ s.t. }\forall y\in B(x,r)\cap\overline{\Omega},\ y + A_x\mathcal{O}\subset\operatorname{int}\Omega.
            \end{align*}
        \end{itemize}
    \end{definition}
    Of course, when $\Omega$ is an \textit{open domain}, $\Omega = \operatorname{int}\Omega$, and we are back to the standard definition.
    
    Yet, as in Definition 3.1, those definitions involve only $\partial\Omega$, $\overline{\Omega}$, and $\operatorname{int}\Omega$ and the properties remain true for all sets in the class
    \begin{align*}
        \left[\Omega\right]_b = \left\{A\subset\mathbb{R}^N;\overline{A} = \overline{\Omega} \mbox{ and } \partial A = \partial\Omega\right\}.
    \end{align*}
    A set having the segment property must have an $(N - 1)$-dimensional boundary and cannot simultaneously lie on both sides of any given part of its boundary.
    
    In fact, a domain satisfying the segment property is locally a $C^0$ epigraph in the sense of Definition 5.1, as we shall see in Sect. 6.2.
    \item 1st give a few general properties.
    
    \begin{theorem}
        Let $\Omega$ be a subset of $\mathbb{R}^N$ s.t. $\partial\Omega\ne\emptyset$.
        \begin{itemize}
            \item[(i)] If \textbf{(6.2)}
            \begin{align*}
                \forall x\in\partial\Omega,\ \exists\lambda > 0,\ \exists A_x\in\operatorname{O}(N),\ \left(x,x + \lambda A_x{\bf e}_N\right)\in\operatorname{int}\Omega,
            \end{align*}
            then $\operatorname{int}\Omega\ne\emptyset$ and
            \begin{align*}
                \overline{\operatorname{int}\Omega} = \overline{\Omega} \mbox{ and } \partial\left(\operatorname{int}\Omega\right) = \partial\Omega.
            \end{align*}
            \item[(ii)] $\Omega$ satisfies the segment property iff $\Omega^c$ satisfies the segment property. In particular
            \begin{align*}
                \operatorname{int}\Omega&\ne\emptyset,\ \overline{\operatorname{int}\Omega} = \overline{\Omega}, \mbox{ and } \partial\left(\operatorname{int}\Omega\right) = \partial\Omega,\\
                \operatorname{int}\Omega^c&\ne\emptyset,\ \overline{\operatorname{int}\Omega^c} = \overline{\Omega^c}, \mbox{ and } \partial\left(\operatorname{int}\Omega^c\right) = \partial\Omega
            \end{align*}
        \end{itemize}
    \end{theorem}
    \item Summarize the equivalences between the epigraph and segment properties that will be detailed in Theorems 6.5-6.7 with the associated constructions and additional motivation.
    
    \begin{theorem}
        Let $\Omega$ be a subset of $\mathbb{R}^N$ s.t. $\partial\Omega\ne\emptyset$.
        \begin{itemize}
            \item[(i)]
            \begin{itemize}
                \item[(a)] $\Omega$ is locally a $C^0$ epigraph iff $\Omega$ has the segment property;
                \item[(b)] $\Omega$ is a $C^0$ epigraph iff $\Omega$ has the uniform segment property;
                \item[(c)] $\Omega$ is an equi-$C^0$ epigraph iff $\Omega$ has the uniform fat segment property.
            \end{itemize}
            \item[(ii)] If, in addition, $\partial\Omega$ is compact, the 6 properties are equivalent and there exists a dominating function $h\in\mathcal{H}$ that satisfies the properties of Theorems 5.1 and 5.2.
        \end{itemize}
    \end{theorem}
    \item We quote the following density results from R. A. Adams [1, Thm. 3.18, p. 54].
    
    \begin{theorem}
        If $\Omega$ has the segment property, then the set
        \begin{align*}
            \left\{f|_{\operatorname{int}\Omega};\forall f\in C_0^\infty\left(\mathbb{R}^N\right)\right\}
        \end{align*}
        of restrictions of functions of $C_0^\infty\left(\mathbb{R}^N\right)$ to $\operatorname{int}\Omega$ is dense in $W^{m,p}(\operatorname{int}\Omega)$ for $1\le p < \infty$ and $m\ge 1$. In particular, $C^k\left(\overline{\operatorname{int}\Omega}\right)$ is dense in $W^{m,p}(\operatorname{int}\Omega)$ for any $m\ge 1$ and $k\ge m$.
    \end{theorem}
    \item \textbf{Equivalence of Geometric Segment \& $C^0$ Epigraph Properties.} As for sets that are locally a $C^0$ epigraph in Definition 5.1, the 3 cases of Definition 6.1 differ only when $\partial\Omega$ is unbounded.
    
    1st deal with the 1st 2 and will come back to the 3rd one in Sect. 6.3.
    
    \begin{theorem}
        Let $\partial\Omega\ne\emptyset$ be compact. Then the segment property and the uniform segment property of Definition 6.1 are equivalent.
    \end{theorem}

    \begin{theorem}
        Let $\Omega$ be a subset of $\mathbb{R}^N$ s.t. $\partial\Omega\ne\emptyset$.
        \begin{itemize}
            \item[(i)] If $\Omega$ satisfies the segment property, then $\Omega$ is locally a $C^0$ epigraph. For $x\in\partial\Omega$, let $B(x,r_x)$, $\lambda_x$, and $A_x$ be the associated open ball, the height, and the rotation matrix. Then there exists $\rho_x$, \textbf{(6.3)}
            \begin{align*}
                0 < \rho_x\le r_{x\lambda} := \min\left\{r_x,\frac{\lambda_x}{2}\right\},
            \end{align*}
            which is the largest radius s.t.
            \begin{align*}
                B_H\left(0,\rho_x\right)\subset\left\{P_H\left(A_x^{-1}(y - x)\right);\forall y\in B\left(x,r_{x\lambda}\right)\cap\partial\Omega\right\}.
            \end{align*}
            The neighborhoods of Definition 5.1 can be chosen as \textbf{(6.4)}
            \begin{align*}
                V_x &:= B_H\left(0,\rho_x\right) \mbox{ and }\\
                \mathcal{U}(x) &:= B\left(x,r_{x\lambda}\right)\cap\left\{y\in\mathbb{R}^N;P_H\left(A_x^{-1}(y - x)\right)\in V_x\right\},
            \end{align*}
            where\footnote{Note that $\mathcal{U}(x) - \{x\} = A_x\left[B\left(0,r_{x\lambda}\right)\cap\left\{\zeta;P_H\zeta\in V_x\right\}\right]$.} $B_H(0,\rho_x)$ denotes the open ball of radius $\rho_x$ in the hyperplane $H$. For each $\zeta'\in V_x$, there exists a unique $y_{\zeta'}\in\partial\Omega\cap\mathcal{U}(x)$ s.t. $P_HA_x^{-1}\left(y_{\zeta'} - x\right) = \zeta'$ and the function
            \begin{align*}
                \zeta'\mapsto a_x\left(\zeta'\right) := \left(y_{\zeta'} - x\right)\cdot\left(A_x{\bf e}_N\right):V_x\to\mathbb{R}
            \end{align*}
            is well-define, bounded \textbf{(6.5)}
            \begin{align*}
                \forall\zeta'\in V_x,\ \left|a_x\left(\zeta'\right)\right| < r_{x\lambda},
            \end{align*}
            and uniformly continuous in $V_x$, i.e., $a_x\in C^0\left(\overline{V_x}\right)$.
            \item[(ii)] Conversely, if $\Omega$ is locally a $C^0$ epigraph, both $\Omega$ and $\Omega^c$ satisfy the segment property.
        \end{itemize}
    \end{theorem}

    \begin{theorem}
        Let $\Omega$ be a subset of $\mathbb{R}^N$ s.t. $\partial\Omega\ne\emptyset$. $\Omega$ has the uniform segment property iff $\Omega$ is a $C^0$ epigraph. In particular, the neighborhoods $\mathcal{U}(x)$ at $x\in\partial\Omega$ in $\mathbb{R}^N$ and $V_x$ at 0 in $H$ can be chosen in the following way: given $r_\lambda = \min\left\{r,\frac{\lambda}{2}\right\}$ and the largest radius $\rho > 0$ s.t. $B_H(0,\rho)\subset\left\{P_Hz;\forall z\in B(x,r_\lambda)\cap\partial\Omega\right\}$, let
        \begin{align*}
            V := B_H(0,\rho) \mbox{ and } U := B(0,r_\lambda)\cap\left\{y;P_Hz\in B_H(0,\rho)\right\}
        \end{align*}
        and, for each $x\in\partial\Omega$, define
        \begin{align*}
            V_x := V \mbox{ and } \mathcal{U}(x) := x + A_xU.
        \end{align*}
        Moreover, the functions $a_x:V\to\mathbb{R}$ can be chosen uniformly bounded,
        \begin{align*}
            \forall x\in\partial\Omega,\ \forall\zeta'\in V,\ \left|a_x(\zeta')\right|\le r_\lambda,
        \end{align*}
        and uniformly continuous in $V$, i.e., $a_x\in C^0(\overline{V})$.
    \end{theorem}

    \begin{remark}
        It is important to notice that, when $\partial\Omega$ is unbounded, the \emph{uniform segment property} does not generally imply that the family of functions $\{a_x;x\in\partial\Omega\}$ can be chosen uniformly bounded and equicontinuous.
        
        To restore that property the segment will have to be \emph{fattened}, as we shall see in the next section.
    \end{remark}
    
    \begin{remark}
        The modulus of continuity in 0 of the function $h$ of part (ii) determines the modulus of continuity of the family of functions $\{a_x;x\in\partial\Omega\}$.
    \end{remark}
    \item \textbf{Equivalence of the Uniform Fat Segment \& the Equi-$C^0$ Epigraph Properties.} The \textit{uniform fat segment property} strengthens the \textit{uniform segment property} by fattening the segment of Definition 6.1 (ii).
    
    Before proving the equivalence with an equi-$C^0$ epigraph, 1st show that the open set $\mathcal{O}$ of Definition 6.1 (iii) is equivalent to a parametrized axisymmetrical open region of the form \textbf{(6.6)}
    \begin{align*}
        \mathcal{O}\left(h,\rho,\lambda\right) := \left\{\zeta' + \zeta_N{\bf e}_N\in\mathbb{R}^N;\zeta'\in B_H\left(0,\rho\right),\ \limsup h\left(\left|\zeta'\right|\right) < \zeta_N < \lambda\right\}
    \end{align*}
    for some $\rho > 0$, $\lambda > 0$ and a \textit{dominating function} $h$ that belongs to the \textit{space of dominating functions} introduced in (5.11) (see Fig. 2.5): \textbf{(6.7)}
    \begin{align*}
        \mathcal{H} = \left\{h;[0,\infty)\to[0,\infty);h(0) = 0 \mbox{ and } h \mbox{ is continuous in } 0\right\}.
    \end{align*}
    The use of the $\limsup h$ rather than $h$ is necessary to make $\mathcal{O}(h,\rho,\lambda)$ open since we only assume that $h$ is continuous in 0.
    
    \begin{lemma}
        \begin{itemize}
            \item[(i)] Given $\rho > 0$, $\lambda > 0$, and $h\in\mathcal{H}$, the region $\mathcal{O}(h,\rho,\lambda)$ contains the segment $(0,\lambda {\bf e}_N)$, does not contain 0, and is open.
            \item[(ii)] Let $\lambda > 0$ be a real number and $\mathcal{O}$ be an open subset of $\mathbb{R}^N$ containing $(0,\lambda {\bf e}_N)$ and not 0. Then there exist $\rho > 0$ and a continuous function $h:[0,\rho]\to[0,\infty)$ which is monotone strictly increasing s.t. the set
            \begin{align*}
                \mathcal{O}\left(h,\rho,\frac{\lambda}{2}\right) := \left\{\xi;\xi'\in B_H(0,\rho) \mbox{ and } h\left(|\zeta'|\right) < \xi_N < \frac{\lambda}{2}\right\}
            \end{align*}
            is open and contains $\left(0,\frac{\lambda {\bf e}_N}{2}\right)$ and not 0, and $\mathcal{O}\left(h,\rho,\frac{\lambda}{2}\right)\subset\mathcal{O}$.
        \end{itemize}
    \end{lemma}

    \begin{theorem}
        Let $\Omega$ be a subset of $\mathbb{R}^N$ s.t. $\partial\Omega\ne\emptyset$. $\Omega$ is an equi-$C^0$ epigraph iff $\Omega$ has the uniform fat segment property.
    \end{theorem}
    \item \textbf{Uniform Cone/Cusp Properties and Hölderian/Lipschitzian Sets.} Make the connection between the notions of Hölderian and Lipschitzian sets of Definition 5.2 and the dominating function $h$ generating the set $\mathcal{O}(h,\rho,\lambda)$ of the uniform fat segment property.
    
    \begin{definition}
        Let $\Omega$ be a subset of $\mathbb{R}^N$ s.t. $\partial\Omega\ne\emptyset$.
        \begin{itemize}
            \item[(i)] $\Omega$ is \emph{locally Lipschitzian} if $\Omega$ is \emph{locally a $C^{0,1}$ epigraph}.
            
            $\Omega$ is \emph{locally Hölderian} if $\Omega$ is \emph{locally a $C^{0,l}$ epigraph} for some $l$, $0 < l < 1$.
            \item[(ii)] $\Omega$ is \emph{Lipschitzian} if $\Omega$ is a \emph{$C^{0,1}$ epigraph}.
            
            $\Omega$ is \emph{Hölderian} if $\Omega$ is \emph{a $C^{0,l}$ epigraph} for some $l$, $0 < l < 1$.
            \item[(iii)] $\Omega$ is \emph{equi-Lipschitzian} if $\Omega$ is \emph{an equi-$C^{0,1}$ epigraph}.
            
            $\Omega$ is \emph{equi-Hölderian} if $\Omega$ is \emph{an equi-$C^{0,l}$ epigraph} for some $l$, $0 < l < 1$.
        \end{itemize}
    \end{definition}
    It is natural to associate with Definition 6.2 the following family of dominating functions:
    \begin{align*}
        h_k(\theta) = \lambda\left(\frac{\theta}{\rho}\right)^l,\ 0 < l\le 1, \mbox{ in } \mathcal{H}.
    \end{align*}
    For $l = 1$, the region $\mathcal{O}(h_1,\rho,\lambda)$ is the open \textit{cone} of height $\lambda$ and aperture $\omega$ given by $\tan\omega = \frac{\rho}{\lambda}$, $0 < \omega < \frac{\pi}{2}$,
    \begin{align*}
        h_1(\theta) = \lambda\frac{\theta}{\rho} = \frac{1}{\tan\omega}\theta.
    \end{align*}
    For $0 < l < 1$, $\mathcal{O}(h_l,\rho,\lambda)$ defines a \textit{cuspidal region} around $\lambda {\bf e}_N$ of height $\lambda$.
    \begin{enumerate}
        \item \textbf{Uniform Cone Property \& Lipschitzian Sets.} Lipschitzian sets have also been equivalently characterized by a purely geometric \textit{uniform cone property} which seems to have been originally introduced by S. Agmon [1].
        
        In this section we establish the equivalence of the 2 sets of definitions and express the previous properties and formulae of Definition 5.2 in terms of the parameters of the cone.
        
        1 of the important properties of a family of open domains satisfying the uniform cone property is that the extension operator from the domains to $\mathbb{R}^N$ is uniformly continuous.
        
        \textbf{Notation 6.1.} Given $\lambda > 0$ and $0 < \omega < \frac{\pi}{2}$, denote by $C(\lambda,\omega)$ the open cone
        \begin{align*}
            C(\lambda,\omega) := \left\{y\in\mathbb{R}^N;\frac{1}{\tan\omega}\left|P_H(y)\right| < y\cdot {\bf e}_N < \lambda\right\},
        \end{align*} 
        where $P_H$ is the orthogonal projection onto the hyperplane $H = \{{\bf e}_N\}^\bot$ orthogonal to the direction ${\bf e}_N$.
        
        Further, associate with an arbitrary point $x\in\mathbb{R}^N$, an $A_x\in\operatorname{O}(N)$, a direction $A_x{\bf e}_N$, and the translated and rotated cone $x + A_xC(\lambda,\omega)$.
        
        It is readily seen that
        \begin{align*}
            C(\lambda,\omega) = \mathcal{O}\left(h,\rho,\lambda\right) \mbox{ with } h(\theta) = \frac{\theta}{\tan\omega} \mbox{ and } \rho = \lambda\tan\omega.
        \end{align*}
        
        \begin{definition}
            Let $\Omega$ be a subset of $\mathbb{R}^N$ s.t. $\partial\Omega\ne\emptyset$.
            \begin{itemize}
                \item[(i)] $\Omega$ is said to satisfy the \emph{local uniform cone property} if
                \begin{align*}
                    \forall x\in\partial\Omega,\ \exists\lambda > 0,\ \exists\omega > 0,\ \exists r > 0,\ \exists A_x\in\operatorname{O}(N),
                \end{align*}
                s.t.
                \begin{align*}
                    \forall y\in B(x,r)\cap\overline{\Omega},\ y + A_xC\left(\lambda,\omega\right)\subset\operatorname{int}\Omega.
                \end{align*}
                \item[(ii)] $\Omega$ is said to satisfy the \emph{uniform cone property} if
                \begin{align*}
                    \exists\lambda > 0,\ \exists\omega > 0,\ \exists r >0,\ \forall x\in\partial\Omega,\ \exists A_x\in\operatorname{O}(N),
                \end{align*}
                s.t.
                \begin{align*}
                    \forall y\in B(x,r)\cap\overline{\Omega},\ y + A_xC\left(\lambda,\omega\right)\subset\operatorname{int}\Omega.
                \end{align*}
            \end{itemize}
        \end{definition}
    
        \begin{theorem}
            Let $\Omega$ be a subset of $\mathbb{R}^N$ s.t. $\partial\Omega\ne\emptyset$. $\Omega$ is equi-Lipschitzian iff $\Omega$ satisfies the uniform cone property.
        \end{theorem}
        \item \textbf{Uniform Cusp Property \& Hölderian Sets.}
        \begin{definition}
            Let $\Omega$ be a subset of $\mathbb{R}^N$ s.t. $\partial\Omega\ne\emptyset$, $0 < l < 1$, and $h_l(\theta) = \lambda\left(\frac{\theta}{\rho}\right)^l$.
            \begin{itemize}
                \item[(i)] $\Omega$ is said to satisfy the \emph{cusp property} of index $l$, $0 < l < 1$, if
                \begin{align*}
                    \forall x\in\partial\Omega,\ \exists h\in\mathcal{H},\ \exists\lambda > 0,\ \exists r > 0,\ \exists A_x\in\operatorname{O}(N),
                \end{align*}
                s.t.
                \begin{align*}
                    \forall y\in B(x,r)\cap\overline{\Omega},\ y + A_x\mathcal{O}\left(h_l,\rho,\lambda\right)\subset\operatorname{int}\Omega.
                \end{align*}
                \item[(ii)] $\Omega$ is said to satisfy the \emph{uniform cusp property} of index $l$, $0 < l < 1$, if
                \begin{align*}
                    \exists\lambda > 0,\ \exists h\in\mathcal{H},\ \exists r > 0,\ \forall x\in\partial\Omega,\ \exists A_x\in\operatorname{O}(N),
                \end{align*}
                s.t.
                \begin{align*}
                    \forall y\in B(x,r)\cap\overline{\Omega},\ y + A_x\mathcal{O}\left(h_l,\rho,\lambda\right)\subset\operatorname{int}\Omega.
                \end{align*}
            \end{itemize}
        \end{definition}
    
        The Lipschitzian case of Definitions 6.2 and 6.3 corresponds to $l = 1$ and $\rho = \lambda\tan\omega$ in the above definition, but we wanted to keep the 2 terminologies distinct.
        
        \begin{remark}
            In some applications, it might be interesting to relax the uniform cusp property by permitting the axis of the cuspidal region to bend: this makes the region look like a horn, and the corresponding property becomes a \emph{horn condition} or \emph{property}.
            
            Horn-shaped domains have been studied in several contexts in the literature.
            
            In particular conditions on domains have been introduced in the context of extension operators and embedding theorems: the domains of F. John [1]; the $(\varepsilon,\delta)$-domains of P. W. John [1]; and the domains satisfying a \emph{flexible horn condition} (which is a broader notion than the previous two) of O. V. Besov [1, 2].
        \end{remark}
        
        \begin{theorem}
            Let $\Omega$ be a subset of $\mathbb{R}^N$ s.t.  $\partial\Omega\ne\emptyset$ and let $0 < l < 1$. $\Omega$ is equi-Hölderian of index $l$ if and only if $\Omega$ has the uniform cusp property of index $l$.
        \end{theorem}
    \end{enumerate}
    \item \textbf{Hausdorff Measure \& Dimension of the Boundary.} Lipschitzian sets have a locally finite boundary measure $H_{N - 1}(\partial\Omega)$.
    
    In particular, when $\partial\Omega$ is compact, $H_{N - 1}(\partial\Omega)$ is finite.
    
    Construct examples of Hölderian sets of index $\alpha$, $0 < \alpha < 1$, with compact boundary (verifying the uniform cusp property) for which the Hausdorff dimension of the boundary is strictly greater than $N - 1$ and hence their boundary measure $H_{N - 1}(\partial\Omega) = +\infty$.
    
    1st give an upper bound.
    
    \begin{theorem}
        Let $\Omega$ be a subset of $\mathbb{R}^N$ with compact boundary. If $\Omega$ satisfies the uniform cusp property of Definition 6.4 (ii) associated with the function $h(\theta) = \theta^\alpha$, $0 < \alpha < 1$, then the Hausdorff dimension of $\partial\Omega$ is less than or equal to $N - \alpha$.
    \end{theorem}
    It is possible to construct examples of sets verifying the uniform cusp property for which the Hausdorff dimension of the boundary is strictly greater than $N - 1$ and hence $H_{N - 1}(\partial\Omega) = +\infty$.
    
    \begin{example}
        This following 2D example of an open domain with compact boundary satisfying the uniform cusp condition for the function $h(\theta) = \theta^\alpha$, $0 < \alpha < 1$, can easily be generalized to an $N$-dimensional example.
        
        Consider the open domain
        \begin{align*}
            \Omega := \left\{(x,y);-1 < x\le 0 \mbox{ and } 0 < y < 2\right\}\cap\left\{(x,y);0 < x < 1 \mbox{ and } f(x) < y < 2\right\}\cap\left\{(x,y);1\le x\le 2 \mbox{ and } 0 < y < 2\right\}
        \end{align*}
        in $\mathbb{R}^2$ (see Fig. 2.7), where $f:[0,1]\to\mathbb{R}$ is defined as follows:
        \begin{align*}
            f(x) := d_C(x)^\alpha,\ 0\le x\le 1,
        \end{align*}
        and $C$ is the Cantor set on the interval $[0,1]$ and $d_C(x)$ is the distance function from the point $x$ to the set $C$ ($d_C$ is uniformly Lipschitzian of constant 1).
        
        This function is equal to 0 on $C$.
        
        Any point in $[0,1]\backslash C$ belongs to 1 of the intervals of length $3^{-k}$, $k\ge 1$, which has been deleted from $[0,1]$ in the sequential construction of the Cantor set.
        
        Therefore the distance function $d_C(x)$ is equal to the distance function to the 2 end points of that interval.
        
        In view of this special structure it can be shown that
        \begin{align*}
            \forall x,y\in[0,1],\ \left|d_C(y)^\alpha - d_C(x)^\alpha\right|\le\left|y - x\right|^\alpha.
        \end{align*}
        Denote by $\Gamma$ the piece of the boundary $\partial\Omega$ specified by the function $f = d_C$.
        
        On $\Gamma$ the uniform cusp condition is verified with $\rho = \frac{1}{6}$, $\lambda = \left(\frac{1}{6}\right)^\alpha$, and $h(\theta) = \theta^\alpha$ (see Fig. 2.8).
        
        Clearly the number $N_\Omega(\varepsilon)$ of hypercubes of dimension $N$ and side $\varepsilon$ required to cover $\partial\Omega$ in greater than the number $N_\Gamma(\varepsilon)$ of hypercubes of dimension $N$ and side $\varepsilon$ required to cover $\Gamma$.
        
        The construction of the Cantor set is done by sequentially deleting intervals.
        
        At step $k = 0$ the interval $\left(\frac{1}{3},\frac{2}{3}\right)$ of width $\frac{1}{3}$ is removed.
        
        At step $k$ a total of $2^k$ intervals of width $\frac{1}{3^{k+1}}$ are removed.
        
        Thus if we pick $\varepsilon = 3^{-(k+1)}$, the interval $[0,1]$ can be covered with exactly $3^{k+1}$ intervals.
        
        Here we are interested in finding a lower bound to the total number of squares of side $\varepsilon$ necessary to cover $\Gamma$.
        
        For this purpose we keep only the $2^k$ intervals removed at step $k$.
        
        Vertically it takes
        \begin{align*}
            \left\lfloor\frac{\left(2^{-1}3^{-(k+1)}\right)^\alpha}{3^{-(k+1)}}\right\rfloor\ge\frac{\left(2^{-1}3^{-(k+1)}\right)^\alpha}{3^{-(k+1)}}
            - 1.
        \end{align*}
        Then we have for $\beta\ge 0$ \cite[p. 119]{Delfour_Zolesio2011} $\ldots$
        
        Therefore given $0 < \alpha < \frac{\log 2}{\log 3}$
        \begin{align*}
            \forall\beta,\ 0\le\beta + \alpha < \frac{\log 2}{\log 3},\ H_{1 + \beta}(\partial\Omega) = +\infty,
        \end{align*}
        and the Hausdorff dimension of $\partial\Omega$ is strictly greater than 1.
    \end{example}

    Given $0 < \alpha < 1$, it is possible to construct an \textit{optimal example} of a set verifying the uniform cusp property for which the Hausdorff dimension of the boundary is exactly $N - \alpha$ and hence $H_{N - 1}(\partial\Omega) = +\infty$.
    
    \begin{example}[Optimal example of a set that verifies the uniform cusp property with $h(\theta) = |\theta|^\alpha$, $0 < \alpha < 1$, and whose boundary has Hausdorff dimension exactly equal to $N - \alpha$] \cite[pp. 120--122]{Delfour_Zolesio2011}.        
    \end{example}
\end{enumerate}

\subsubsection{Courant Metrics on Images of a Set}
\begin{enumerate}
    \item A natural way to construct a family of variable domains is to consider the images of a fixed subset of $\mathbb{R}^N$ by some family of transformations of $\mathbb{R}^N$.
    
    The structure and the topology of the images can be specified via the natural algebraic and topological structures of the space of transformations or equivalence classes of transformations for which the full power of function analytic methods is available.
    
    There are many ways to do that, and specific constructions and choices are again very much problem dependent.
    \item In 1972 A. M. Micheletti [1] introduced what may be 1 of the 1st complete metric topologies on a family of domains of class $C^k$, $k\ge 1$, that are the images of a fixed open domain through a family of $C^k$-diffeomorphisms of $\mathbb{R}^N$.
    
    There the natural underlying algebraic structure is the \textit{group structure} of the composition of transformations with the identity transformation as the neutral element.
    
    Her analysis culminates with the construction of a complete metric on the quotient of the group by an appropriate closed subgroup of transformations leaving the fixed subset unaltered.
    
    She called it the \textit{Courant metric}\footnote{``Nello studiare la continuità dell'n-esimo autovalore dell'operatore di Laplace $-\Delta_\Omega$ relativo ad un aperto limitato $\Omega$ con dati di Dirichlet nulli, considerato comme funzione dell'aperto $\Omega$, Courant introduce una nozione di vicinanza tra due domini basata su un diffeomorphismo del tipo $I + \psi$ con $\psi\in C^1\left(\mathbb{R}^m\right)$, che transforma l'uno nell'altro.'' (cf. A. M. Micheletti [1]).} because it is proved in the book of R. Courant and D. Hilbert [1, p. 420] that the $n$th eigenvalue of the Laplace operator depends continuously on the domain $\Omega$, where $\Omega = (I + f)\Omega_0$ is the image of a fixed domain $\Omega_0$ by $I + f$ and $f$ is a smooth mapping.
    
    Her constructions naturally extend to other families of transformations of $\mathbb{R}^N$ or of fixed holdalls $D$.
    \item In Sect. 2 we 1st extend her generic constructions associated with the space $C_0^k(\mathbb{R}^N,\mathbb{R}^N)$ of mappings from $\mathbb{R}^N$ into $\mathbb{R}^N$ to a larger family of Banach spaces of mappings e.g. $C^k(\overline{\mathbb{R}^N},\mathbb{R}^N)$, $C^{k,1}(\overline{\mathbb{R}^N},\mathbb{R}^N)$, or $\mathcal{B}^k(\mathbb{R}^N,\mathbb{R}^N)$, and beyond to Fr\'echet spaces e.g. $\mathcal{B}^k(\mathbb{R}^N,\mathbb{R}^N)$ or $C_0^\infty(\mathbb{R}^N,\mathbb{R}^N)$ of infinitely continuously differentiables mappings.
    
    We emphasize the \textit{geodesic character} of the construction of the metric and its interpretation as trajectories of bounded variation on the group.
    \item The next step in the construction is the choice of the closed subgroup of transformations of $\mathbb{R}^N$ that is very much problem dependent.
    
    Originally, it was chosen as the set of transformations that leave the underlying set or pattern unaltered.
    
    However, in some applications, it could be unaltered up to a translation, a rotation, or a flip.
    
    The underlying set or pattern can be a closed set or an open crack-free set.\footnote{Cf. Definition 7.1 in Chap. 8.}
    
    This includes closed submanifolds of $\mathbb{R}^N$.
    
    It is shown that, as long as the subgroup is closed, we get a complete Courant metric on the quotient group.
    
    In this section we also characterize the tangent space to the group of transformations of $\mathbb{R}^N$ that leads to the Courant metric.
    
    It is an example of an infinite-dimensional manifold (cf. Theorems 2.14 and 2.17 in Sects. 2.5-2.6).
    \item In Sect. 3, we free the constructions from the framework of bounded continuously differentiable transformations to reach the spaces of all homeomorphisms or $C^k$-diffeomorphisms of $\mathbb{R}^N$ or an open subset $D$ of $\mathbb{R}^N$.
    
    Again it is shown that they are complete metric spaces.
    
    Hence, from Sect. 2, their quotient by a closed subgroup yields a Courant metric and a complete metric topology.
    
    With such larger spaces, it now becomes possible to consider subgroups involving not only translations but also isometries, symmetries, or flips in $\mathbb{R}^N$ or $D$.
\end{enumerate}

\paragraph{Generic Constructions of Micheletti.}
\begin{enumerate}
    \item The original constructions of A. M. Micheletti [1] were carried out for the Banach space $C_0^k(\mathbb{R}^N,\mathbb{R}^N)$, $k\ge 0$.\footnote{$C_0^k(\mathbb{R}^N,\mathbb{R}^N)$ denotes the space of $k$ times continuously differentiable functions from $\mathbb{R}^N$ to $\mathbb{R}^N$, whose derivatives vanish at infinity.
        
        It is also denoted by $\mathcal{B}_0^k(\mathbb{R}^N,\mathbb{R}^N)$ in the literature.}
    
    Generalize them to Banach spaces $\Theta$ of mappings from $\mathbb{R}^N$ into $\mathbb{R}^N$ under fairly general assumptions.
    \item \textbf{Space $\mathcal{F}(\Theta)$ of Transformations of $\mathbb{R}^N$.} Associate with a real vector space $\Theta$ of mappings from $\mathbb{R}^N$ to $\mathbb{R}^N$ the following space of transformations of $\mathbb{R}^N$: \textbf{(2.1)}
    \begin{align*}
        \mathcal{F}(\Theta) := \left\{I + f;f\in\Theta,\ \left(I + f\right) \mbox{ bijective and } (I + f)^{-1} - I\in\Theta\right\},
    \end{align*}
    where $x\mapsto I(x) := x:\mathbb{R}^N\to\mathbb{R}^N$ is the identity mapping.\footnote{For $\Theta = C_0^k(\mathbb{R}^N,\mathbb{R}^N)$, this definition is equivalent to the one of A. M. Micheletti [1]:
        \begin{align*}
            \mathcal{F}\left(C_0^k(\mathbb{R}^N,\mathbb{R}^N)\right) := \left\{F:\mathbb{R}^N\to\mathbb{R}^N;F - I\in C_0^k(\mathbb{R}^N,\mathbb{R}^N) \mbox{ and } F^{-1}\in C^k(\mathbb{R}^N,\mathbb{R}^N)\right\}.
    \end{align*}}
    It will be shown that $\mathcal{F}(\Theta)$ is a group for the composition $(F\circ G)(x) := F(G(x))$ of transformations of $\mathbb{R}^N$.
    
    Equivalent right-invariant metrics on $\mathcal{F}(\Theta)$ will be introduced under Assumptions 2.1-2.2 and related to some notion of geodesics in the group $\mathcal{F}(\Theta)$.
    
    The completeness of $\mathcal{F}(\Theta)$ will require Assumption 2.3 and either
    \begin{itemize}
        \item[(i)] the assumption that $\Theta\subset C^0(\mathbb{R}^N,\mathbb{R}^N)$ and, for all $x\in\mathbb{R}^N$, the mapping $f\mapsto f(x):\Theta\to\mathbb{R}^N$ is continuous that makes the group a complete metric space, or
        \item[(ii)] Assumption 2.4 that makes the group a complete (metric) topological group.
    \end{itemize}
    Finally, the right-invariant \textit{Courant metric} will be defined as the quotient metric \textbf{(2.2)}
    \begin{align*}
        \forall F,G\in\mathcal{F}(\Theta),\ d_{\mathcal{G}}\left([F],[H]\right) := \inf_{G,\widetilde{G}\in\mathcal{G}} d\left(F\circ G,H\circ\widetilde{G}\right),\ [F] := F\circ\mathcal{G}
    \end{align*}
    associated with the quotient group $\mathcal{F}(\Theta)/\mathcal{G}$ of $\mathcal{F}(\Theta)$ by a closed subgroup $\mathcal{G}$ of $\mathcal{F}(\Theta)$.
    
    The subgroup considered by A. M. Micheletti [1] was
    \begin{align*}
        \mathcal{G}(\Omega_0) := \left\{F\in\mathcal{F}(\Theta);F(\Omega_0) = \Omega_0\right\}
    \end{align*}
    to quotient out all $F$ whose image of the fixed subset $\Omega_0$ of $\mathbb{R}^N$ is $\Omega_0$ and make the quotient group $\mathcal{F}(\Theta)/\mathcal{G}(\Omega_0)$ isomorphic to the set of images \textbf{(2.3)}
    \begin{align*}
        \mathcal{X}(\Omega_0) := \left\{F(\Omega_0);\forall F\in\mathcal{F}(\Theta)\right\}
    \end{align*}
    of $\Omega_0$ by the elements of $\mathcal{F}(\Theta)$.
    
    The abstract results will be applied with $\Theta$ equal to Banach spaces e.g. $C_0^k(\mathbb{R}^N,\mathbb{R}^N)\subset C^k(\overline{\mathbb{R}^N},\mathbb{R}^N)\subset\mathcal{B}^k(\mathbb{R}^N,\mathbb{R}^N)$ and $C^{k,1}(\overline{\mathbb{R}^N},\mathbb{R}^N)$, $k\ge 0$, and, through special constructions, to the Fréchet spaces $C_0^\infty(\mathbb{R}^N,\mathbb{R}^N)\subset\mathcal{B}(\mathbb{R}^N,\mathbb{R}^N) = \bigcap_{k\ge 0} \mathcal{B}^k(\mathbb{R}^N,\mathbb{R}^N)$.
    
    \textbf{Assumption 2.1.} \textit{$\Theta$ is a real vector space of mappings from $\mathbb{R}^N$ into $\mathbb{R}^N$ and}
    \begin{align*}
        \forall g\in\Theta,\ \forall I + f\in\mathcal{F}(\Theta),\ g\circ(I + 
        f)\in\Theta.
    \end{align*}
    
    \begin{theorem}
        Let $\Theta$ be a real vector space of mappings from $\mathbb{R}^N$ into $\mathbb{R}^N$. $\mathcal{F}(\Theta)$ is a group for the composition $\circ$ iff Assumption 2.1 is verified. In particular $(I + f)^{-1} - I = -f\circ(I + f)^{-1}\in\Theta$.
    \end{theorem}

    \begin{example}
        \begin{enumerate}
            \item $\Theta = \mathcal{B}^0(\mathbb{R}^N,\mathbb{R}^N)$, space of bounded continuous functions.
            
            For $g\in\mathcal{B}^0(\mathbb{R}^N,\mathbb{R}^N)$and $I + f\in\mathcal{F}(\mathcal{B}^0(\mathbb{R}^N,\mathbb{R}^N))$, the composition $g\circ(I + f)$ is continuous and $\|g\circ(I + f)\|_{C^0} = \|g\|_{C^0} < \infty$.
            \item $\Theta = C^0(\overline{\mathbb{R}^N},\mathbb{R}^N)$, space of bounded uniformly continuous functions.
            
            For $g\in C^0(\overline{\mathbb{R}^N},\mathbb{R}^N)$ and $I + f\in\mathcal{F}(C^0(\overline{\mathbb{R}^N},\mathbb{R}^N))$, $\|g\circ(I + f)\|_{C^0} = \|g\|_{C^0} < \infty$, and since $g$ and $I + f$ are uniformly continuous, so is the composition $g\circ(I + f)$.
            \item $\Theta = C_0^0(\mathbb{R}^N,\mathbb{R}^N)$, a subspace of $C^0(\overline{\mathbb{R}^N},\mathbb{R}^N)$.
            
            From (2) $\|g\circ(I + f)\|_{C^0} = \|g\|_{C^0} < \infty$ and $g\circ(I + f)\in C^0(\overline{\mathbb{R}^N},\mathbb{R}^N)$.
            
            It remains to show that $g\circ(I + f)\in C_0^0(\mathbb{R}^N,\mathbb{R}^N)$.
            
            Since $g\in C_0^0(\mathbb{R}^N,\mathbb{R}^N)$, for all $\varepsilon > 0$, there exists $\rho_0 > 0$ s.t. for all $x$ s.t. $|x| > \rho_0$, $|g(x)| < \varepsilon$.
            
            But $f$ also belongs to $C_0^0(\mathbb{R}^N,\mathbb{R}^N)$ and there exists $\rho\ge 2\rho_0$ s.t. for all $x$ s.t. $|x| > \rho$, $|f(x)| < \rho_0$.
            
            In particular, $|x + f(x)|\ge|x| - |f(x)| > \rho - \rho_0\ge\rho_0$ and hence $|g(x + f(x))| < \varepsilon$.
            
            In conclusion for all $\varepsilon > 0$, there exists $\rho > 0$ s.t. for all $x$ s.t. $|x| > \rho$, $|g(x + f(x))| < \varepsilon$ and $g\circ(I + f)\in C_0^0(\mathbb{R}^N,\mathbb{R}^N)$.
            \item $\Theta = C^{0,1}(\overline{\mathbb{R}^N},\mathbb{R}^N)$.
            
            This is the space of bounded Lipschitz continuous functions on $\mathbb{R}^N$ that is contained in $C^0(\overline{\mathbb{R}^N},\mathbb{R}^N)$.
            
            For $f,g\in C^{0,1}(\overline{\mathbb{R}^N},\mathbb{R}^N)$, $\|f\circ(I + g)\|_{C^0} = \|f\|_{C^0} < \infty$.
            
            Since $g$ is Lipschitz with Lipschitz constant $c(g)$, $I + g$ is also Lipschitz with Lipschitz constant $1 + c(g)$ and the composition is also Lipschitz with constant $c(f)(1 + c(g))$.
        \end{enumerate}
    \end{example}
    Associate with $F\in\mathcal{F}(\Theta)$ the following function of $I$ and $F$: \textbf{(2.5)}
    \begin{align*}
        d(I,F) := \inf_{F = F_1\circ\cdots\circ F_n,\, F_i\in\mathcal{F}(\Theta)} \sum_{i=1}^n \|F_i - I\|_\Theta + \|F_i^{-1} - I\|_\Theta,
    \end{align*}
    where the infimum is taken over all \textit{finite factorizations} of $F$ in $\mathcal{F}(\Theta)$ of the form
    \begin{align*}
        F = F_1\circ\cdots\circ F_n,\ F_i\in\mathcal{F}(\Theta).
    \end{align*}
    In particular $d(I,F) = d(I,F^{-1})$.
    
    Extend this function to all $F$ and $G$ in $\mathcal{F}(\Theta)$: \textbf{(2.6)}
    \begin{align*}
        d(F,G) := d\left(I,G\circ F^{-1}\right).
    \end{align*}
    By definition, $d$ is right-invariant since for all $F$, $G$, and $H$ in $\mathcal{F}(\Theta)$
    \begin{align*}
        d(F,G) = d\left(F\circ H,G\circ H\right).
    \end{align*}
    To show that $d$ is a metric\footnote{A function $d:X\times X\to\mathbb{R}$ is said to be a \textit{metric} on $X$ if (cf. J. Dugundji [1])
        \begin{itemize}
            \item[(i)] $d(F,G)\ge 0$, for all $F$, $G$,
            \item[(ii)] $d(F,G) = 0\Leftrightarrow F = G$,
            \item[(iii)] $d(F,G) = d(G,F)$, for all $F$, $G$,
            \item[(iv)] $d(F,H)\le d(F,G) + d(G,H)$, for all $F$, $G$, $H$. 
        \end{itemize}
        A metric $d$ on a group $(\mathcal{F},\circ)$ is said to be \textit{right-invariant} if for all $H\in\mathcal{F}$, $d(F\circ H,G\circ H) = d(F,G)$.} on $\mathcal{F}(\Theta)$ necessitates a 2nd assumption.
    
    \textbf{Assumption 2.2.} $(\Theta,\|\cdot\|)$ is a normed vector space of mappings from $\mathbb{R}^N$ into $\mathbb{R}^N$ and for each $r > 0$ there exists a continuous function $c_0(r)$ s.t. \textbf{(2.7)}
    \begin{align*}
        \forall\{f_i\}_{i=1}^n\subset\Theta \mbox{ s.t. } \sum_{i=1}^n \|f_i\| < \alpha 
        < r;
    \end{align*}
    then \textbf{(2.8)}
    \begin{align*}
        \left\|(I + f_1)\circ\cdots\circ(I + f_n) - I\right\| < \alpha c_0(r).
    \end{align*}
    
    \begin{theorem}
        Under Assumptions 2.1 and 2.2, $d$ is a right-invariant metric on $\mathcal{F}(\Theta)$.
    \end{theorem}

    \begin{example}
        (1), (2), (3) Since the norm on the 3 spaces $\mathcal{B}^0(\mathbb{R}^N,\mathbb{R}^N)$, $C^0(\overline{\mathbb{R}^N},\mathbb{R}^N)$, and $C_0^0(\mathbb{R}^N,\mathbb{R}^N)$ is the same and Assumption 2.2 involves only the norm, it is sufficient to check it for $\mathcal{B}^0(\mathbb{R}^N,\mathbb{R}^N)$.
        
        Consider $\{f_i\}_{i=1}^n\subset\mathcal{B}^0(\mathbb{R}^N,\mathbb{R}^N)$ s.t. \textbf{(2.9)}
        \begin{align*}
            \sum_{i=1}^n \|f_i\|_{C^0} < \alpha < r.
        \end{align*}
        Using the convention\footnote{NQBH corrected: instead of $(I + f_n)\circ(I + f_n) = I + f_n$.} $(I + f_n)\circ(I + f_n) = I + f_n + f_n\circ(I + f_n)$ in the summation\footnote{NQBH corrected: $c_0(r) = 1$ instead of $c_0(r) = r$.}
        \begin{align*}
            (I + f_1)\circ\cdots\circ(I + f_n) - I &= I + f_n - I + \sum_{i=1}^{n-1} (I + f_i)\circ\cdots\circ(I + f_n) - (I + f_{i-1})\circ\cdots\circ(I + f_n)\\
            &= f_n + \sum_{i=1}^{n-1} f_i\circ(I + f_{i+1})\circ\cdots\circ(I + f_n)\\
            \Rightarrow\left\|(I + f_1)\circ\cdots\circ(I + f_n) - 
            I\right\|_{C^0}&\le\sum_{i=1}^n \|f_i\|_{C^0} < \alpha < r\Rightarrow c_0(r) = 1.
        \end{align*}
        (4) $\Theta = C^{0,1}(\overline{\mathbb{R}^N},\mathbb{R}^N)$.
        
        Consider a sequence $\{f_i\}_{i=1}^n\subset C^{0,1}(\overline{\mathbb{R}^N},\mathbb{R}^N)$ s.t. \textbf{(2.10)}
        \begin{align*}
            \sum_{i=1}^n \max\left\{\|f_i\|_{C^0},c(f_i)\right\} < \alpha < r.
        \end{align*}
        From the previous example
        \begin{align*}
            \left\|(I + f_1)\circ\cdots\circ(I + f_n) - I\right\|_{C^0}\le\sum_{i=1}^n \|f_i\|_{C^0} < \alpha.
        \end{align*}
        In addition,
        \begin{align*}
            (I + f_1)\circ\cdots\circ(I + f_n) - I &= f_n + \sum_{i=1}^{n-1} f_i\circ(I + f_{i+1})\circ\cdots\circ(I + f_n)\\
            \Rightarrow c\left((I + f_1)\circ\cdots\circ(I + f_n) - I\right)&\le c(f_n) + \sum_{i=1}^{n-1} c(f_i)\left(1 + c(f_{i+1})\right)\cdots\left(1 + c(f_1)\right)\\
            &\le c(f_n) + \sum_{i=1}^{n-1} c(f_i)e^{\sum_{k=1}^{i+1} c(f_k)}\le\left(\sum_{i=1}^n c(f_i)\right)e^{\sum_{i=1}^n c(f_i)} < \alpha e^\alpha\\
            \Rightarrow\left\|(I + f_1)\circ\cdots\circ(I + f_n) _ I\right\|_{C^{0,1}} &< \alpha\max\{e^\alpha,1\} < \alpha e^r,
        \end{align*}
        and we can choose $c_0(r) = e^r$.
    \end{example}
    An important aspect of the previous construction was to \textit{build} in the triangle inequality to make $d$ a (right-invariant) metric.
    
    However, there are other ways to introduce a topology on $\mathcal{F}(\Theta)$.
    
    The interest of this specific choice will become apparent later on.
    
    E.g., the original \textit{Courant metric} of A. M. Micheletti [1] in 1972 was constructed from the following slightly different right-invariant metric on $\mathcal{F}(\Theta)$: given $H\in\mathcal{F}(\Theta)$, consider finite factorizations $H_1\circ\cdots\circ H_m$, $H_i\in\mathcal{F}(\Theta)$, $1\le i\le m$, of $H$ and finite factorizations $G_1\circ\cdots\circ G_n$, $G_j\in\mathcal{F}(\Theta)$, $1\le j\le n$, of $H^{-1}$, and define \textbf{(2.11)}
    \begin{align*}
        d_1(I,H) := \inf_{H = H_1\circ\cdots\circ H_m,\ m\ge 1} \sum_{i=1}^m \left\|H_i - I\right\| + \inf_{H^{-1} = G_1\circ\cdots\circ G_n,\, n\ge 1} \sum_{i=1}^n \|G_i - I\|,
    \end{align*}
    where the infima are taken over all \textit{finite factorizations} of $H$ and $H^{-1}$ in $\mathcal{F}(\Theta)$ and \textbf{(2.12)}
    \begin{align*}
        \forall F,G\in\mathcal{F}(\Theta),\ d_1(G,F) := d_1\left(I,F\circ G^{-1}\right).
    \end{align*}
    Another example is the topology generated by the right-invariant \textit{semimetric}\footnote{Given a space $X$, a function $d:X\times X\to\mathbb{R}$ is said to be a \emph{semimetric} if
        \begin{itemize}
            \item[(i)] $d(F,G)\ge 0$, for all $F$, $G$,
            \item[(ii)] $d(F,G) = 0\Leftrightarrow F = G$,
            \item[(iii)] $d(F,G) = d(G,F)$, for all $F$, $G$.
        \end{itemize}
        This notion goes back to Fréchet and Menger.} \textbf{(2.13)}
    \begin{align*}
        \forall H\in\mathcal{F}(\Theta),\ d_0(I,H) &:= \|H - I\| + \|H^{-1} - I\|,\\
        \forall F,G\in\mathcal{F}(\Theta),\ d_0(G,F) &:= d_0\left(I,F\circ G^{-1}\right).
    \end{align*}
    Given $r > 0$ and $F\in\mathcal{F}(\Theta)$, let $B_r(F) = \{G\in\mathcal{F}(\Theta);d_0(F,G) < r\}$.
    
    Let $\tau_0$ be the weakest topology on $\mathcal{F}(\Theta)$ s.t. the family $\{B_r(F);0\le r < \infty\}$ is the base for $\tau_0$.
    
    By definition
    \begin{align*}
        d_1(F,G)\le d(F,G)\le d_0(F,G)
    \end{align*}
    and the injections $(\mathcal{F},d_1)\to(\mathcal{F},d)\to(\mathcal{F},d_0)$ are continuous.
    
    \begin{theorem}
        Under Assumptions 2.1 and 2.2, the topologies generated by $d_1$, $d$, and $d_0$ are equivalent.
    \end{theorem}
    F. Murat and J. Simon [1] in 1976 considered as candidates for $\Theta$ the spaces
    \begin{align*}
        W^{k+1,\overline{c}}(\mathbb{R}^N,\mathbb{R}^N) := \left\{f\in W^{k,\infty}(\mathbb{R}^N,\mathbb{R}^N);\forall 0\le|\alpha|\le k + 1,\ \partial^{\alpha} f\in C(\overline{\mathbb{R}^N},\mathbb{R}^N)\right\}
    \end{align*}
    and $W^{k+1,\infty}(\mathbb{R}^N,\mathbb{R}^N)$, $k\ge 0$.
    
    The 1st space $W^{k+1,\overline{c}}(\mathbb{R}^N,\mathbb{R}^N)$ is equivalent to the space $C^{k+1}(\overline{\mathbb{R}^N},\mathbb{R}^N)$, $k\ge 0$, algebraically and topologically.
    
    The 2nd space $W^{k+1,\infty}(\mathbb{R}^N,\mathbb{R}^N)$ coincides with $C^{k,1}(\overline{\mathbb{R}^N},\mathbb{R}^N)$, $k\ge 0$.
    
    For the spaces $C^{k+1}(\overline{\mathbb{R}^N},\mathbb{R}^N)$ and $C^{k,1}(\overline{\mathbb{R}^N},\mathbb{R}^N)$, $k\ge 0$, they obtained the following \textit{pseudo triangle inequality} for the semimetric $d_0$: \textbf{(2.14)}
    \begin{align*}
        d_0(F_1,F_3)\le d_0(F_1,F_2) + d_0(F_2,F_3) + d_0(F_1,F_2)d_0(F_2,F_3)P\left[d_0(F_1,F_2) + d_0(F_2,F_3)\right],\ \forall F_1,F_2,F_3,
    \end{align*}
    where $P:\mathbb{R}^+\to\mathbb{R}^+$ is a continuous increasing function.
    
    With that additional property, they called $d_0$ a \textit{pseudodistance}\footnote{This terminology is not standard.
        
        In the literature a \textit{pseudometric} or \textit{pseudodistance} is usually reserved for a function $d:X\times X\to\mathbb{R}$ satisfying
        \begin{itemize}
            \item[(i)] $d(F,G)\ge 0$, for all $F$, $G$,
            \item[(ii')] $d(F,G) = 0\Leftarrow F = G$,
            \item[(iii)] $d(F,G) = d(G,F)$, for all $F$, $G$.
            \item[(iv)] $d(F,H)\le d(F,G) + d(G,H)$, for all $F$, $G$, $H$.
        \end{itemize}
        Condition (ii) for a metric is weakened to condition (ii') for a pseudodistance.} and showed that they could construct a metric from it.
    
    \begin{lemma}
        Assume that the semimetric $d_0$ satisfies the pseudo triangle inequality (2.14). Then, for all $\alpha$, $0 < \alpha < 1$, there exists a constant $\eta_\alpha > 0$ s.t. the function $d_0^{(\alpha)}:\mathcal{F}(\Theta)\times\mathcal{F}(\Theta)\to\mathbb{R}^+$ defined as
        \begin{align*}
            d_0^{(\alpha)}(F_1,F_2) := \inf\left\{d_0(F_1,F_2),\eta_\alpha\right\}^\alpha
        \end{align*}
        is a metric on $E$.
    \end{lemma}
    The Micheletti construction with the metric $d_1$ was given in 2001 for the spaces $C^k(\overline{\mathbb{R}^N},\mathbb{R}^N)$ and $C^{k,1}(\overline{\mathbb{R}^N},\mathbb{R}^N)$, $k\ge 0$, in M. C. Delfour and J.-P. Zolésio [37] so that pseudodistances can be completely bypassed.
    
    The metrics $d$ and $d_1$ constructed from the semimetric $d_0$ under the general Assumptions 2.1 and 2.2 are both more general and interesting since they are of \textit{geodesic type} by the use of infima over finite factorizations $(I + \theta_1)\circ\cdots\circ(I + \theta_n)$ of $I + \theta$ when the semimetric $d_0$ is interpreted as an \textit{energy} \textbf{(2.15)}
    \begin{align*}
        \sum_{i=1}^n d_0\left(I + \theta_i,I\right) = \sum_{i=1}^n \|\theta_i\| + \left\|-\theta_i\circ\left(I + \theta_i\right)^{-1}\right\|
    \end{align*} 
    over families of piecewise constant trajectories 
    $T:[0,1]\to\mathcal{F}(\Theta)$ starting from $I$ at time 0 to $I + \theta$ at time 1 in the group $\mathcal{F}(\Theta)$ by assigning times $t_i$, $0 = t_0 < t_1 < \cdots < t_n < t_{n+1} = 1$, at each jump: \textbf{(2.16)}
    \begin{equation*}
        T(t) := \left\{\begin{split}
            &I, &&0\le t < t_1,\\
            &(I + \theta_1), &&t_1 < t < t_2,\\
            &(I + \theta_2)\circ(I + \theta_1), &&t_2 < t < t_3,\\
            &\ldots\\
            &(I + \theta_i)\circ\cdots\circ(I + \theta_1), &&t_i < t < t_{i+1},\\
            &\ldots\\
            &(I + \theta_n)\circ\cdots\circ(I + \theta_1), &&t_n < t\le 1.
        \end{split}\right.
    \end{equation*}
    The \textit{jump in the group} $\mathcal{F}(\Theta)$ at time $t_i$ is given by
    \begin{align*}
        \left[T(t)\right]_{t_i} := T(t_i^+)\circ T(t_i^-)^{-1} - I = \left[(I + \theta_i)\circ\cdots\circ(I + \theta_1)\right]\circ\left[(I + \theta_{i-1})\circ\cdots\circ(I + \theta_1)\right]^{-1} - I = \theta_i
    \end{align*}
    as an element of $\Theta$ that could be viewed as the tangent space to $\mathcal{F}(\Theta)\subset I + \Theta$.
    
    %
    The choice of $d$ in a \textit{geodesic context} is more interesting than the choice of $d_1$ that would involve a double infima in the energy (2.15).
    
    At this stage, in view of the equivalence Theorem 2.3, working with the metric $d$ or $d_1$ is completely equivalent.
    
    \textbf{Assumption 2.3.} $(\Theta,\|\cdot\|)$ is a normed vector space of mappings from $\mathbb{R}^N$ into $\mathbb{R}^N$ and there exists a continuous function $c_1$ s.t. for all $I + f\in\mathcal{F}(\Theta)$ and $g\in\Theta$ \textbf{(2.17)}
    \begin{align*}
        \left\|g\circ(I + f)\right\|\le\|g\|c_1\left(\|f\|\right).
    \end{align*}
    The next assumption is a kind of \textit{uniform continuity}.
    
    \textbf{Assumption 2.4.} $(\Theta,\|\cdot\|)$ is a normed vector space of mappings from $\mathbb{R}^N$ into $\mathbb{R}^N$. For each $g\in\Theta$, for all $\varepsilon > 0$, there exists $\delta > 0$ s.t. \textbf{(2.18)}
    \begin{align*}
        \forall\gamma\in\Theta \mbox{ s.t. } \|\gamma\| < \delta,\ \left\|g\circ\left(I + \gamma\right) - g\right\| < \varepsilon.
    \end{align*}
    
    \begin{theorem}
        Under Assumptions 2.1 to 2.4, $\mathcal{F}(\Theta)$ is a topological (metric) group.
    \end{theorem}

    \begin{example}[Checking Assumption 2.3]
        (1), (2), (3) Since the norm on the 3 spaces $\mathcal{B}^0(\mathbb{R}^N,\mathbb{R}^N)$, $C^0(\overline{\mathbb{R}^N},\mathbb{R}^N)$, and $C_0^0(\mathbb{R}^N,\mathbb{R}^N)$ is the same and Assumption 2.2 involves only the norm, it is sufficient to check it for $\mathcal{B}^0(\mathbb{R}^N,\mathbb{R}^N)$.
        
        Consider $g\in\mathcal{B}^0(\mathbb{R}^N,\mathbb{R}^N)$ and $I + f\in\mathcal{F}(\mathcal{B}^0(\mathbb{R}^N,\mathbb{R}^N))$.
        
        Since $I + f$ is a bijection \textbf{(2.19)}
        \begin{align*}
            \|g\circ(I + f)\|_{C^0}\le\|g\|_{C^0}
        \end{align*}
        and the constant function $c_1(r) = 1$ satisfies Assumption 2.3.
        
        (4) $\Theta = C^{0,1}(\overline{\mathbb{R}^N},\mathbb{R}^N)$.
        
        Consider $g\in C^{0,1}(\overline{\mathbb{R}^N},\mathbb{R}^N)$ and $I + f\in\mathcal{F}(C^{0,1}(\overline{\mathbb{R}^N},\mathbb{R}^N))$.
        
        From (1)
        \begin{align*}
            \|g\circ(I + f)\|_{C^0}\le\|g\|_{C^0}
        \end{align*}
        and since all the functions are Lipschitz
        \begin{align*}
            c\left(g\circ(I + f)\right)&\le c(g)(1 + c(f))\\
            \Rightarrow\|g\circ(I + f)\|_{C^{0,1}}&\le\|g\|_{C^{0,1}}(1 + c(f))\le\|g\|_{C^{0,1}}\left(1 + \|f\|_{C^{0,1}}\right)
        \end{align*}
        and the continuous function $c_1(r) = 1 + r$ satisfies Assumption 2.3.
    \end{example}
    Assumption 2.4 is a little trickier.
    
    It involves not only norms but also a ``uniform continuity'' of the function $g$ that is not verified for $g\in\mathcal{B}^0(\mathbb{R}^N,\mathbb{R}^N)$ and for the Lipschitz part $c(g)$ of the norm of $C^{0,1}(\overline{\mathbb{R}^N},\mathbb{R}^N)$.
    
    \begin{example}[Checking Assumption 2.4]
        (1) $\Theta = \mathcal{B}^0(\mathbb{R}^N,\mathbb{R}^N)$.
        
        In that space there exists some $g$ that is not uniformly continuous.
        
        For that $g$ there exists $\varepsilon > 0$ s.t. for all $n > 0$, there exists $x_n,y_n$, $|y_n - x_n| < \frac{1}{n}$ s.t. $|g(y_n) - g(x_n)|\ge\varepsilon$.
        
        Define the function $\gamma_n(x) = y_n - x_n$.
        
        Therefore, there exists $\varepsilon > 0$ s.t. for all $n$ with $\|\gamma_n\|_{C^0} < \frac{1}{n}$,
        \begin{align*}
            \|g\circ(I + \gamma_n) - g\|_{C^0}\ge|g(x_n + \gamma_n(x_n)) - g(x_n)| = |g(x_n + y_n - x_n) - g(x_n)|\ge\varepsilon
        \end{align*}
        and Assumption 2.4 is not verified.
        
        However, $\mathcal{B}^k(\mathbb{R}^N,\mathbb{R}^N)\subset C^0(\mathbb{R}^N,\mathbb{R}^N)$ and for all $x\in\mathbb{R}^N$, the mapping $f\mapsto f(x):\mathcal{B}^0(\mathbb{R}^N,\mathbb{R}^N)\to\mathbb{R}$ is continuous.
        
        (2) and (3) $\Theta = C^0(\overline{\mathbb{R}^N},\mathbb{R}^N)$.
        
        By the uniform continuity of $g\in C^0(\overline{\mathbb{R}^N},\mathbb{R}^N)$, for all $\varepsilon > 0$ there exists $\delta > 0$ s.t.
        \begin{align*}
            \forall x,y\in\mathbb{R}^N,\ |y - x| < \delta\Rightarrow|g(y) - g(x)| < \frac{\varepsilon}{2}.
        \end{align*}
        For $\|\gamma\|_{C^0} < \delta$
        \begin{align*}
            \forall x\in\mathbb{R}^N,\ |(I + \gamma)(x) - x| &= |\gamma(x)|\le\|\gamma\|_{C^0} < \delta,\\
            \forall x\in\mathbb{R}^N,\ |g(x + \gamma(x)) - g(x)| < \frac{\varepsilon}{2}&\Rightarrow\|g\circ(I + \gamma) - g\|_{C^0}\le\frac{\varepsilon}{2} < \varepsilon,
        \end{align*}
        and Assumption 2.4 is verified.
        
        In addition, $C^0(\overline{\mathbb{R}^N},\mathbb{R}^N)\subset C^0(\mathbb{R}^N,\mathbb{R}^N)$ and for all $x\in\mathbb{R}^N$, the mapping $f\mapsto f(x): C^0(\overline{\mathbb{R}^N},\mathbb{R}^N)\to\mathbb{R}$ is continuous.
        
        (4) $\Theta = C^{0,1}(\overline{\mathbb{R}^N},\mathbb{R}^N)$.
        
        Since $g\in C^{0,1}(\overline{\mathbb{R}^N},\mathbb{R}^N)$
        \begin{align*}
            \forall x,y\in\mathbb{R}^N,\ |g(y) - g(x)|\le c(g)|y - x|.
        \end{align*}
        Therefore for
        \begin{align*}
            \|\gamma\|_{C^{0,1}} := \max\{\|\gamma\|_{C^0},c(\gamma)\} < \delta,
        \end{align*}
        $\|\gamma\|_{C^0} < \delta$ and we get
        \begin{align*}
            \|g\circ(I + \gamma) - g\|_{C^0}\le c(g)\|\gamma\|_{C^0} < c(g)\delta.
        \end{align*}
        Unfortunately, for the part $c(g\circ(I + \gamma))$ of the $C^{0,1}$-norm, for all $x\ne y\in\mathbb{R}^N$
        \begin{align*}
            \frac{|g(y + \gamma(y)) - g(y) - \left(g(x + \gamma(x)) - g(x)\right)|}{|y - x|}&\le c(g) + c(g)c(I + \gamma),\\
            c\left(g\circ(I + \gamma) - g\right)\le c(g)[1 + c(\gamma)] &< c(g)[1 + \delta],
        \end{align*}
        and we cannot satisfy Assumption 2.4.
        
        However, $C^{0,1}(\overline{\mathbb{R}^N},\mathbb{R}^N)\subset C^0(\mathbb{R}^N,\mathbb{R}^N)$ and for all $x\in\mathbb{R}^N$, the mapping $f\mapsto f(x): C^{0,1}(\overline{\mathbb{R}^N},\mathbb{R}^N)\to\mathbb{R}$ is continuous.
    \end{example}

    \begin{remark}
        Since the spaces $C_0^k(\mathbb{R}^N,\mathbb{R}^N)\subset C^k(\overline{\mathbb{R}^N},\mathbb{R}^N)\subset\mathcal{B}^k(\mathbb{R}^N,\mathbb{R}^N)$ have the same norm and Assumptions 2.2 and 2.3 involve only the norms, they will always be verified for $C_0^k(\mathbb{R}^N,\mathbb{R}^N)$ and $C^k(\mathbb{R}^N,\mathbb{R}^N)$ if they are verified for $\mathcal{B}^k(\mathbb{R}^N,\mathbb{R}^N)$, $k\ge 0$.
        
        Solely Assumption 2.1 will require special attention.
    \end{remark}
    The fact that Assumption 2.4 is not always verified will not prevent us from proving that the space $\left(\mathcal{F}(\Theta),d\right)$ is complete.
    
    The completeness will be proved in Theorem 2.6 under Assumptions 2.1 to 2.3, where Assumption 2.4 will be replaced by the condition $\Theta\subset C^0(\mathbb{R}^N,\mathbb{R}^N)$ and, for all $x\in\mathbb{R}^N$, the mapping $f\mapsto f(x):\Theta\to\mathbb{R}^N$ is continuous.
    
    The following theorem will be proved in Theorems 2.11-2.13 of section 2.5 and Theorems 2.15 and 2.16 of Sect. 2.6.
    
    \begin{theorem}
        Let $k\ge 0$ be an integer.
        \begin{itemize}
            \item[(i)] $C_0^k(\mathbb{R}^N,\mathbb{R}^N)$ and $C^k(\overline{\mathbb{R}^N},\mathbb{R}^N)$ satisfy Assumptions 2.1 to 2.4,
            \begin{align*}
                C_0^k(\mathbb{R}^N,\mathbb{R}^N)\subset C^k(\overline{\mathbb{R}^N},\mathbb{R}^N)\subset C^0(\mathbb{R}^N,\mathbb{R}^N),
            \end{align*}
            and, for all $x\in\mathbb{R}^N$, the mapping $f\mapsto f(x):C^k(\overline{\mathbb{R}^N},\mathbb{R}^N)\to\mathbb{R}^N$ is continuous.
            \item[(ii)] $\mathcal{B}^k(\mathbb{R}^N,\mathbb{R}^N)$ and $C^{k,1}(\overline{\mathbb{R}^N},\mathbb{R}^N)$ verify Assumptions 2.1 to 2.3,
            \begin{align*}
                C^{k,1}(\overline{\mathbb{R}^N},\mathbb{R}^N)\subset\mathcal{B}^k(\mathbb{R}^N,\mathbb{R}^N)\subset C^0(\mathbb{R}^N,\mathbb{R}^N),
            \end{align*}
            and, for all $x\in\mathbb{R}^N$, the mapping $f\mapsto f(x):\mathcal{B}^k(\mathbb{R}^N,\mathbb{R}^N)\to\mathbb{R}^N$ is continuous.
        \end{itemize}
    \end{theorem}
    
    \begin{theorem}
        \begin{itemize}
            \item[(i)] Let Assumptions 2.1 to 2.3 be verified for a Banach space $\Theta$ contained in $C^0(\mathbb{R}^N,\mathbb{R}^N)$ s.t., for all $x\in\mathbb{R}^N$, the mapping $f\mapsto f(x):\Theta\to\mathbb{R}^N$ is continuous. Then $(\mathcal{F}(\Theta),d)$ is a complete metric space.
            \item[(ii)] Let Assumptions 2.1 to 2.4 be verified for a Banach space $\Theta$. Then $\mathcal{F}(\Theta)$ is a complete topological (metric) group.
        \end{itemize}
    \end{theorem}

    \begin{remark}
        $\mathcal{F}(C^{0,1}(\overline{\mathbb{R}^N},\mathbb{R}^N))$ is a complete metric space since the assumptions of Theorem 2.6 are verified, but it is not a topological group since Assumption 2.4 is not verified.
    \end{remark}
    \item \textbf{Diffeomorphisms for $\mathcal{B}(\mathbb{R}^N,\mathbb{R}^N)$ \& $C_0^\infty(\mathbb{R}^N,\mathbb{R}^N)$.} Consider as other candidates for $\Theta$ the spaces \textbf{(2.20)-(2.21)}
    \begin{align*}
        \mathcal{B}(\mathbb{R}^N,\mathbb{R}^N) &:= \bigcap_{k=0}^\infty \mathcal{B}^k(\mathbb{R}^N,\mathbb{R}^N) =  \bigcap_{k=0}^\infty C^k(\overline{\mathbb{R}^N},\mathbb{R}^N),\\
        C_0^\infty(\mathbb{R}^N,\mathbb{R}^N) &:= \bigcap_{k=0}^\infty C_0^k(\mathbb{R}^N,\mathbb{R}^N),
    \end{align*}
    and the resulting groups $\mathcal{F}(\mathcal{B}(\mathbb{R}^N,\mathbb{R}^N))$
    \begin{align*}
        \left\{I + f;f\in\mathcal{B}(\mathbb{R}^N,\mathbb{R}^N),\ (I + f) \mbox{ bijective and } (I + f)^{-1} - I\in\mathcal{B}(\mathbb{R}^N,\mathbb{R}^N)\right\},
    \end{align*}
    and $\mathcal{F}\left(C_0^\infty(\mathbb{R}^N,\mathbb{R}^N)\right)$
    \begin{align*}
        \left\{I + f;f\in C_0^\infty(\mathbb{R}^N,\mathbb{R}^N),\ (I + f) \mbox{ bijective and } (I + f)^{-1} - I\in C_0^\infty(\mathbb{R}^N,\mathbb{R}^N)\right\},
    \end{align*}
    that satisfy Assumption 2.1 but not the other assumptions since $\mathcal{B}(\mathbb{R}^N,\mathbb{R}^N)$ and $C_0^\infty(\mathbb{R}^N,\mathbb{R}^N)$ are Fréchet but not Banach spaces.
    
    To get around this, observe that
    \begin{align*}
        \mathcal{F}\left(\mathcal{B}(\mathbb{R}^N,\mathbb{R}^N)\right) &= \bigcap_{k=0}^\infty \mathcal{F}\left(\mathcal{B}^k(\mathbb{R}^N,\mathbb{R}^N)\right) = \bigcap_{k=0}^\infty \mathcal{F}\left(C^k(\overline{\mathbb{R}^N},\mathbb{R}^N)\right),\\
        \mathcal{F}\left(C_0^\infty(\mathbb{R}^N,\mathbb{R}^N)\right) &= \bigcap_{k=0}^\infty \mathcal{F}\left(C_0^k(\mathbb{R}^N,\mathbb{R}^N)\right),
    \end{align*}
    where $\left(\mathcal{F}(C^k(\overline{\mathbb{R}^N},\mathbb{R}^N)),d_k\right)$ and $\left(\mathcal{F}(C_0^k(\mathbb{R}^N,\mathbb{R}^N)),d_k\right)$ are topological (metric) groups with the same metrics $\{d_k\}$ and the following monotony property:
    \begin{align*}
        d_k(F,G)\le d_{k+1}(F,G)
    \end{align*}
    resulting from the monotony of the norms
    \begin{align*}
        \|f\|_{C^k} = \max_{0\le i\le k} \|f\|_{C^i}\le\max_{0\le i\le k+1} \|f\|_{C^i} = \|f\|_{C^{k+1}}.
    \end{align*}
    
    \begin{theorem}
        The spaces $\mathcal{F}(\mathcal{B}(\mathbb{R}^N,\mathbb{R}^N))$ and $\mathcal{F}(C_0^\infty(\mathbb{R}^N,\mathbb{R}^N))$ are complete topological (metric) groups for the distance \textbf{(2.22)}
        \begin{align*}
            d_\infty(F,G) := \sum_{k=0}^\infty \frac{1}{2^k}\frac{d_k(F,G)}{1 + d_k(F,G)},
        \end{align*}
        where $d_k$ is the metric associated with $C^k(\overline{\mathbb{R}^N},\mathbb{R}^N)$ and $C_0^k(\mathbb{R}^N,\mathbb{R}^N)$.\footnote{In view of (2.20), it is important to note that $\mathcal{F}(\mathcal{B}(\mathbb{R}^n,\mathbb{R}^N))$ is a topological group by using the family $(\mathcal{F}(C^k(\overline{\mathbb{R}^N},\mathbb{R}^N)),d_k)$ in Lemma 2.2 and not the family $(\mathcal{F}(\mathcal{B}^k(\mathbb{R}^N,\mathbb{R}^N)),d_k)$.}
    \end{theorem}
    This theorem is a consequence of the following general lemma.
    
    \begin{lemma}
        Assume that $(\mathcal{F}_k,d_k)$, $k\ge 0$, is a family of groups each complete w.r.t. their metric $d_k$ and that \textbf{(2.23)}
        \begin{align*}
            \forall k\ge 0,\ \mathcal{F}_{k+1}\subset\mathcal{F}_k \mbox{ and } d_k(F,G)\le d_{k+1}(F,G),\ \forall F,G\in\bigcap_{k=0}^\infty \mathcal{F}_k.
        \end{align*}
        Then the intersection $\mathcal{F}_\infty = \bigcap_{k=0}^\infty \mathcal{F}_k$ is a group that is complete w.r.t. the metric $d_\infty$ defined in (2.22). If, in addition, each $(\mathcal{F}_k,d_k)$ is a topological (metric) group, then $\mathcal{F}_\infty$ is a topological (metric) group.
    \end{lemma}
    \item \textbf{Closed Subgroups $\mathcal{G}$.} Given an arbitrary nonempty subset $\Omega_0$ of $\mathbb{R}^N$, consider the family \textbf{(2.24)}
    \begin{align*}
        \mathcal{X}\left(\Omega_0\right) := \left\{F\left(\Omega_0\right)\subset\mathbb{R}^N;\forall F\in\mathcal{F}(\Theta)\right\}
    \end{align*}
    of images of $\Omega_0$ by the elements of $\mathcal{F}(\Theta)$.
    
    By introducing the subgroup \textbf{(2.25)}
    \begin{align*}
        \mathcal{G}\left(\Omega_0\right) := \left\{F\in\mathcal{F}(\Theta);F\left(\Omega_0\right) = \Omega_0\right\},
    \end{align*}
    we obtain a bijection \textbf{(2.26)}
    \begin{align*}
        [F]\mapsto F\left(\Omega_0\right):\mathcal{F}(\Theta)/\mathcal{G}\left(\Omega_0\right)\to\mathcal{X}\left(\Omega_0\right)
    \end{align*}
    between the set of images $\mathcal{X}(\Omega_0)$ and the quotient space $\mathcal{F}(\Theta)/\mathcal{G}(\Omega_0)$.
    
    Using this bijection, the topological structure of $\mathcal{X}(\Omega_0)$ will be identified with the topological structure of the quotient space.
    
    \begin{lemma}
        Let Assumptions 2.1 to 2.3 hold with $\Theta\subset C^0(\mathbb{R}^N,\mathbb{R}^N)$ and, for all $x\in\mathbb{R}^N$, let the mapping $f\mapsto f(x):\Theta\to\mathbb{R}^N$ be continuous.\footnote{This means that $\mathcal{F}(\Theta)\subset I + \Theta\subset\operatorname{Hom}(\mathbb{R}^N,\mathbb{R}^N)$.} If the nonempty subset $\Omega_0$ of $\mathbb{R}^N$ is closed or if $\Omega_0 = \operatorname{int}\overline{\Omega_0}$,\footnote{Such a set will be referred to as a \textit{crack-free} set in Definition 7.1 (ii) of Chap. 8.
            
            Indeed, by definition $\Omega$ is crack-free if $\overline{\Omega^c} = \overline{\overline{\Omega}^c}$.
            
            If,in addition, $\Omega$ is open, then $\Omega^c = \overline{\Omega^c} = \overline{\overline{\Omega}^c}$ and $\Omega = \overline{\overline{\Omega}^c}^c = \operatorname{int}\overline{\Omega}$.} the family \textbf{(2.27)}
        \begin{align*}
            \mathcal{G}\left(\Omega_0\right) := \left\{F\in\mathcal{F}(\Theta);F\left(\Omega_0\right) = \Omega_0\right\}
        \end{align*}
        is a closed subgroup of $\mathcal{F}(\Theta)$.
    \end{lemma}

    \begin{remark}
        A. M. Micheletti [1] assumes that $\Omega_0$ is a bounded connected open domain of class $C^3$ in order to make all the images $F(\Omega_0)$ bounded connected open domains of class $C^3$ with $\Theta = C_0^3(\mathbb{R}^N,\mathbb{R}^N)$.
        
        Open domains of class $C^k$, $k\ge 1$, are locally $C^k$ epigraphs and domains that are locally $C^0$ epigraphs are crack-free\footnote{Cf. Definition 7.1 (ii) of Chapter 8.} by Theorem 5.4 (ii) of Chapter 2.
    \end{remark}
    
    \begin{remark}
        F. Murat and J. Simon [1] were mainly interested in families of images of locally Lipschitzian (epigraph) domains.
        
        Recall that $\mathcal{F}(C^1(\overline{\mathbb{R}^N},\mathbb{R}^N))$ transports locally Lipschitzian (epigraph) domains onto locally Lipschitzian (epigraph) domains (cf. A. Bendali [1] and A. Djadane [1]), but that $\mathcal{F}(C^{0,1}(\mathbb{R}^N,\mathbb{R}^N))$ does not, as shown in Examples 5.1-5.2 of Sect. 5.3 in Chap. 2.
    \end{remark}
    Of special interest are the closed submanifolds of $\mathbb{R}^N$ of codimension $\ge 1$.
    
    E.g., we can choose $\Omega_0 = S^n$, $2\le n < N$, the $n$-sphere of radius 1 centered in 0.
    
    We get a family of closed curves in $\mathbb{R}^3$ for $n = 2$ and of closed surfaces in $\mathbb{R}^3$ for $n = 3$.
    
    Similarly, by choosing $\Omega_0 = \overline{B_1^n(0)}$, the closed unit ball of dimension $n$, we get a family of curves for $n = 1$ and a family of surfaces for $n = 2$.
    
    The choice of $[0,1]^n$ yields a family of surfaces with \textit{corners} for $n = 2$.
    
    %
    Explicit expressions can be obtained for normals, curvatures, and density measures as in Chap. 2.
    
    \begin{example}[Closed surfaces in $\mathbb{R}^3$]
        For $\Omega_0 = S^2$ in $\mathbb{R}^3$, $f\in C^1(\overline{\mathbb{R}^3},\mathbb{R}^3)$, and $F = I + f$, the tangent field at the point $\xi\in S^2$ is $T_\xi S^2 = \{\xi\}^\top$ and the unit normal is $\xi$.
        
        At a point $x = F(\xi)$ of the image $F(S^2)$, the tangent space and the unit normal are given by
        \begin{align*}
            T_{F(\xi)}F\left(S^2\right) = DF(\xi)\{\xi\}^\bot \mbox{ and } {\bf n}(F(x)) = \frac{DF(\xi)^{-\top}\xi}{\left|DF(\xi)^{-\top}\xi\right|}.
        \end{align*}
    \end{example}
    
    \begin{example}[Curves in $\mathbb{R}^N$]
        Let ${\bf e}_N$ be a unit vector in $\mathbb{R}^N$, let $H = \{{\bf e}_N\}^\bot$, and let
        \begin{align*}
            \Omega_0 := \left\{\zeta_N{\bf e}_N;\zeta_N\in[0,1]\right\}\subset\mathbb{R}^N.
        \end{align*}
        For $f\in C^1(\overline{\mathbb{R}^N},\mathbb{R}^N)$, and $F = I + f$, the tangent field at the point $\xi\in\Omega_0$ is $T_\xi\Omega_0 = \mathbb{R}{\bf e}_N$ and the normal space $(T_\xi\Omega_0)^\bot$ is $H$.
        
        At a point $x = F(\xi)$ of the curve $F(\Omega_0)$, the tangent and normal spaces are given by
        \begin{align*}
            T_{F(\xi)}F\left(\Omega_0\right) = \mathbb{R}DF(\xi){\bf e}_N \mbox{ and } T_{F(\xi)}F\left(\Omega_0\right)^\bot = \left\{DF(\xi){\bf e}_N\right\}^\bot.
        \end{align*}
    \end{example}
    In practice, one might want to use other subgroups, such as all the translations by some vector $a\in\mathbb{R}^N$: \textbf{(2.28)-(2.29)}
    \begin{align*}
        &F_a(x) := a + x,\ x\in\mathbb{R}^N,\ \Rightarrow F_a - I = a\in\Theta,\\
        \Rightarrow &F_a^{-1}(x) - a + x = F_{-a}(x) \Rightarrow F_a^{-1} - I = -a\in\Theta
    \end{align*}
    for all $\Theta$ that contain the constant functions e.g. $\mathcal{B}^k(\mathbb{R}^N,\mathbb{R}^N)$, $C^k(\overline{\mathbb{R}^N},\mathbb{R}^N)$, and $C^{k,1}(\overline{\mathbb{R}^N},\mathbb{R}^N)$, but not $C_0^k(\overline{\mathbb{R}^N},\mathbb{R}^N)$.
    
    Therefore \textbf{(2.30)}
    \begin{align*}
        \mathcal{G}_\tau := \left\{F_a;\forall a\in\mathbb{R}^N\right\}
    \end{align*}
    is a subgroup of $\mathcal{F}(\Theta)$.
    
    In that case, it is also a closed subgroup of $\mathcal{F}(\Theta)$:
    \begin{align*}
        F_a\circ F_b(x) = a + b + x = F_{a+b}(x),\ F_0 = I
    \end{align*}
    and $(\mathcal{G}_\tau,\circ)$ is isomorphic to the 1D closed abelian group $(\mathbb{R}^N,+)$.
    
    This means that all the topologies on $\mathcal{G}_\tau$ are equivalent and $(\mathcal{G}_\tau,\circ)$ is closed in $\mathcal{F}(\Theta)$.
    \item \textbf{Courant Metric on the Quotient Group $\mathcal{F}(\Theta)/\mathcal{G}$.} When $\mathcal{F}(\Theta)$ is a topological right-invariant metric group and $\mathcal{G}$ is a closed subgroup of $\mathcal{F}(\Theta)$, the \textit{quotient metric} \textbf{(2.31)}
    \begin{align*}
        d_{\mathcal{G}}\left(F\circ\mathcal{G},H\circ\mathcal{G}\right) := \inf_{G,\widetilde{G}\in\mathcal{G}} d\left(F\circ G,H\circ\widetilde{G}\right)
    \end{align*}
    induces a complete metric topology on the quotient group $\mathcal{F}(\Theta)/\mathcal{G}$.\footnote{Cf., e.g., D. Montgomery and L. Zippin [1 sect. 1.22, p. 34, sect. 1.23, p. 36])
        \begin{itemize}
            \item[(i)] If $\mathcal{G}$ is a topological group whose open sets at $e$ have a countable basis, then $\mathcal{G}$ is metrizable and, moreover, there exists a metric which is right-invariant.
            \item[(ii)] If $\mathcal{G}$ is a closed subgroup of a metric group $\mathcal{F}$, then $\mathcal{F}/\mathcal{G}$ is metrizable.
    \end{itemize}}
    The \textit{quotient metric} $d_{\mathcal{G}\left(\Omega_0\right)}$ with $\Theta = C_0^k(\overline{\mathbb{R}^N},\mathbb{R}^N)$ was referred to as the \textit{Courant metric} by A. M. Micheletti [1].
    
    In that case $\mathcal{F}(\Theta)$ is a topological (metric) group.
    
    However, we have seen that for $\Theta = C^{0,1}(\overline{\mathbb{R}^N},\mathbb{R}^N)$ and $\mathcal{B}^0(\mathbb{R}^N,\mathbb{R}^N)$, $\mathcal{F}(\Theta)$ is not a topological group.
    
    Yet, the completeness of $(\mathcal{F}(\Theta),d)$ and the right-invariance of $d$ are sufficient to recover the result for an arbitrary closed subgroup $\mathcal{G}$ of $\mathcal{F}$.
    
    It is important to have the weakest possible assumptions on $\mathcal{G}$ to secure the widest range of applications.
    
    \begin{theorem}
        Assume that $(\mathcal{F},d)$ is a group with right-invariant metric $d$ that is complete for the topology generated by $d$. For any closed subgroup $\mathcal{G}$ of $\mathcal{F}$, the function $d_{\mathcal{G}}:\mathcal{F}\times\mathcal{F}\to\mathbb{R}$, \textbf{(2.32)}
        \begin{align*}
            d_{\mathcal{G}}\left(F\circ\mathcal{G},H\circ\mathcal{G}\right) := \inf_{G\in\mathcal{G}} d\left(F,H\circ G\right),
        \end{align*}
        is a right-invariant metric on $\mathcal{F}(\Theta)/\mathcal{G}$ and the space $(\mathcal{F}/\mathcal{G},d_{\mathcal{G}})$ is complete. The topology induced by $d_{\mathcal{G}}$ coincides with the quotient topology of $\mathcal{F}/\mathcal{G}$.
    \end{theorem}
    We now have all the ingredients to conclude.
    
    \begin{theorem}
        Let $\Theta$ be any of the spaces $\mathcal{B}^k(\mathbb{R}^N,\mathbb{R}^N)$, $C^k(\overline{\mathbb{R}^N},\mathbb{R}^N)$, $C_0^k(\mathbb{R}^N,\mathbb{R}^N)$, $C^{k,1}(\overline{\mathbb{R}^N},\mathbb{R}^N)$, $k\ge 0$, or $C^\infty(\overline{\mathbb{R}^N},\mathbb{R}^N)$, $C_0^\infty(\mathbb{R}^N,\mathbb{R}^N)$.
        \begin{itemize}
            \item[(i)] The group $(\mathcal{F}(\Theta),d)$ is a complete right-invariant metric space. For $\Theta$ equal to $C^k(\overline{\mathbb{R}^N},\mathbb{R}^N)$ or $C_0^k(\mathbb{R}^N,\mathbb{R}^N)$, $0\le k\le\infty$, $(\mathcal{F}(\Theta),d)$ is also a topological group.
            \item[(ii)] For any closed subgroup $\mathcal{G}$ of $\mathcal{F}(\Theta)$, the function $d_{\mathcal{G}}:\mathcal{F}(\Theta)\times\mathcal{F}(\Theta)\to\mathbb{R}$, \textbf{(2.34)}
            \begin{align*}
                d_{\mathcal{G}}\left(F\circ\mathcal{G},H\circ\mathcal{G}\right) := \inf_{G\in\mathcal{G}} d\left(F,H\circ G\right),
            \end{align*}
            is a right-invariant metric on $\mathcal{F}(\Theta)/\mathcal{G}$ and the space $(\mathcal{F}(\Theta)/\mathcal{G},d_{\mathcal{G}})$ is complete. The topology induced by $d_{\mathcal{G}}$ coincides with the quotient topology of $\mathcal{F}(\Theta)/\mathcal{G}$.
            \item[(iii)] Let $\Omega_0$, $\emptyset\ne\Omega_0\subset\mathbb{R}^N$, be closed or let $\Omega_0 = \operatorname{int}\overline{\Omega_0}$. 
            
            Then \textbf{(2.35)}
            \begin{align*}
                \mathcal{G}\left(\Omega_0\right) := \left\{F\in\mathcal{F}(\Theta);F\left(\Omega_0\right) = \Omega_0\right\}
            \end{align*}
            is a closed subgroup of $\mathcal{F}(\Theta)$.
        \end{itemize}
    \end{theorem}
    Finally, it is interesting to recall that A. M. Micheletti [1] used the quotient metric to prove the following theorem.
    
    \begin{theorem}
        Fix $k = 3$, the Courant metric of $\mathcal{F}(C_0^3(\mathbb{R}^N,\mathbb{R}^N))/\mathcal{G}(\Omega_0)$, and a bounded open connected domain $\Omega_0$ of class $C^3$ in $\mathbb{R}^N$. Let $\mathcal{X}(\Omega_0)$ be the family of images of $\Omega_0$ by elements of $\mathcal{F}(C_0^3(\mathbb{R}^N,\mathbb{R}^N))/\mathcal{G}(\Omega_0)$.
        
        The subset of all bounded open domains $\Omega$ in $\mathcal{X}(\Omega_0)$ s.t. the spectrum of the Laplace operator $-\Delta$ on $\Omega$ with homogeneous Dirichlet conditions on the boundary $\partial\Omega$ does not have all its eigenvalues simple is of the 1st category.\footnote{A set $B$ is said to be \textit{nowhere dense} if its closure has no interior or, alternatively, if $B^c$ is dense.
            
            A set is said to be of the \textit{1st category} if it is the countable union of nowhere dense sets (cf. J. Dugundji [1, Def. 10.4, p. 250]).}
    \end{theorem}
    This theorem says that, up to an arbitrarily small perturbation of the domain, the eigenvalues of the Laplace operator of a $C^3$-domain can be made simple.
    \item \textbf{Assumptions for $\mathcal{B}^k(\mathbb{R}^N,\mathbb{R}^N)$, $C^k(\overline{\mathbb{R}^N},\mathbb{R}^N)$, and $C_0^k(\mathbb{R}^N,\mathbb{R}^N)$.}
    \begin{enumerate}
        \item \textbf{Checking the Assumptions.} By definition, for all integers $k\ge 0$
        \begin{align*}
            C_0^k(\mathbb{R}^N,\mathbb{R}^N)&\subset C^k(\overline{\mathbb{R}^N},\mathbb{R}^N)\subset\mathcal{B}^k(\mathbb{R}^N,\mathbb{R}^N)\subset\mathcal{B}^0(\mathbb{R}^N,\mathbb{R}^N)\subset C^0(\mathbb{R}^N,\mathbb{R}^N),\\
            \forall x\in\mathbb{R}^N,\ f&\mapsto f(x):\mathcal{B}^0(\mathbb{R}^N,\mathbb{R}^N)\to\mathbb{R}^N,
        \end{align*}
        with continuous injections between the spaces and continuity of the evaluation map.
        
        These spaces are all Banach spaces endowed with the same norm:
        \begin{align*}
            \|f\|_{\mathcal{C}^k} := \max_{0\le\left|\alpha\right|\le k} \left\|\partial^\alpha f\right\|_C.
        \end{align*}
        As a consequence, the evaluation map $f\mapsto f(x)$ is continuous for all the spaces.
        
        %
        The following results are quoted directly from A. M. Micheletti [1] and apply to all spaces $\Theta\subset C^k(\mathbb{R}^N,\mathbb{R}^N)$.
        
        Denote by
        \begin{align*}
            S(x)[y_1][y_2],\ldots,[y_k]
        \end{align*}
        at the point $x\in\mathbb{R}^N$ a $k$-linear form with arguments $y_1,\ldots,y_k$.
        
        \begin{lemma}
            Assume that $F$ and $G$ are 2 mappings from $\mathbb{R}^N$ to $\mathbb{R}^N$ s.t. $F$ is $k$-times differentiable in an open neighborhood of $x$ and $G$ is $k$-times differentiable in an open neighborhood of $F(x)$. Then the $k$th derivative of $G\circ F$ in $x$ is the sum of a finite number of $k$-linear applications on $\mathbb{R}^N$ of the form \textbf{(2.36)}
            \begin{align*}
                \left(h_1,\ldots,h_k\right)\mapsto G^{(l)}\left(F(x)\right)\left[F^{(\lambda_1)}(x)[h_1][h_1]\ldots[h_{\lambda_1}]\right]\ldots\left[F^{(\lambda_l)}(x)\left[h_{k - \lambda_l + 1}\right]\ldots[h_k]\right],
            \end{align*}
            where $l = 1,\ldots,k$ and $\lambda_1 + \cdots + \lambda_l = k$.
        \end{lemma}
    
        \begin{lemma}
            Given $g$ and $f$ in $C^k(\mathbb{R}^N,\mathbb{R}^N)$, let $\psi = g\circ(I + f)$. Then for each $x\in\mathbb{R}^N$
            \begin{align*}
                |\psi(x)| &= |g(x + f(x))|,\\
                |\psi^{(1)}(x)|&\le|g^{(1)}(x + f(x))|\left[1 + |f^{(1)}(x)|\right],\\
                |\psi^{(i)}(x)|&\le|g^{(1)}(x + f(x))||f^{(i)}(x)| + \sum_{j=2}^i |g^{(j)}(x + f(x))|a_j\left(|f^{(1)}(x)|,\ldots,|f^{(i-1)}(x)|\right)
            \end{align*}
            for $i = 2,\ldots,k$, where $a_j$ is a polynomial with positive coefficients.
        \end{lemma}
    
        \begin{theorem}
            Assumptions 2.1 and 2.3 are verified for the spaces $C_0^k(\mathbb{R}^N,\mathbb{R}^N)$, $C^k(\overline{\mathbb{R}^N},\mathbb{R}^N)$, and $\mathcal{B}(\mathbb{R}^N,\mathbb{R}^N)$, $k\ge 0$.
        \end{theorem}
    
        \begin{theorem}
            Assumption 2.2 is verified for the spaces $C_0^k(\mathbb{R}^N,\mathbb{R}^N)$, $C^k(\overline{\mathbb{R}^N},\mathbb{R}^N)$, and $\mathcal{B}^k(\mathbb{R}^N,\mathbb{R}^N)$, $k\ge 0$.
        \end{theorem}
        It is a consequence of the following lemma of A. M. Micheletti [1].
        
        \begin{lemma}
            Given integers $r\ge 0$ and $s > 0$ there exists a constant $c(r,s) > 0$ with the following property: if the sequence $f_1,\ldots,f_n$ in $C^r(\mathbb{R}^N,\mathbb{R}^N)$ is s.t. \textbf{(2.38)}
            \begin{align*}
                \sum_{i=1}^n \|f_i\|_{C^r} < \alpha,\ 0 < \alpha < s,
            \end{align*}
            then for the map $F = (I + f_n)\circ\cdots\circ(I + f_1)$, \textbf{(2.39)}
            \begin{align*}
                \|F - I\|_{C^r}\le\alpha c(r,s).
            \end{align*}
        \end{lemma}
    
        \begin{theorem}
            Assumption 2.4 is verified for $C^k(\overline{\mathbb{R}^N},\mathbb{R}^N)$ and $C_0^k(\mathbb{R}^N,\mathbb{R}^N)$.
        \end{theorem}
        It is a consequence of the following slightly modified versions of lemmas of A. M. Micheletti [1].
        
        \begin{lemma}
            Given $f\in C^k(\overline{\mathbb{R}^N},\mathbb{R}^N)$ and $\gamma\in C^0(\mathbb{R}^N,\mathbb{R}^N)$, for all $\varepsilon > 0$, there exists $\delta > 0$ s.t. \textbf{(2.44)}
            \begin{align*}
                \forall\gamma,\ \|\gamma\|_{C^0} < \delta,\ \sup_{0\le i\le k} \|f^{(i)}\circ(I + \gamma) - f^{(i)}\|_{C^0} < \varepsilon.
            \end{align*}
        \end{lemma}
    
        \begin{lemma}
            For each $g\in C^k(\overline{\mathbb{R}^N},\mathbb{R}^N)$, for all $\varepsilon > 0$, there exists $\delta > 0$ s.t. \textbf{(2.45)}
            \begin{align*}
                \forall\gamma\in C^k(\overline{\mathbb{R}^N},\mathbb{R}^N) \mbox{ s.t. } \|\gamma\|_{C^k} < \delta,\ \|g\circ(I + \gamma) - g\|_{C^k} < \varepsilon.
            \end{align*}
        \end{lemma}
        \item \textbf{Perturbations of the Identity and Tangent Space.} By construction
        \begin{align*}
            \mathcal{F}(\Theta) \subset I + \Theta
        \end{align*}
        and the tangent space in each point of the affine space $I + \Theta$ is $\Theta$.
        
        In general, $I + \Theta$ is not a subset of $\mathcal{F}(\Theta)$, but, for some spaces of transformations $\Theta$, there is a sufficiently small ball $B$ around 0 in $\Theta$ s.t. the so-called \textit{perturbations of the identity}, $I + B$, are contained in $\mathcal{F}(\Theta)$.
        
        In particular, for $\theta\in B$, the family of transformations $T_t = I + t\theta\in\mathcal{F}(\Theta)$, $0\le t\le 1$, is a $C^1$-path in $\mathcal{F}(\Theta)$ and the tangent space to each point of $\mathcal{F}(\Theta)$ can be shown to be exactly $\Theta$.
        
        \begin{theorem}
            Let $\Theta$ be equal to $C_0^k(\mathbb{R}^N,\mathbb{R}^N)$, $C^k(\overline{\mathbb{R}^N},\mathbb{R}^N)$, or $\mathcal{B}^k(\mathbb{R}^N,\mathbb{R}^N)$, $k\ge 1$.
            \begin{itemize}
                \item[(i)] The map $f\mapsto I + f:B(0,1)\subset\Theta\to\mathcal{F}(\Theta)$ is continuous.
                \item[(ii)] For all $F\in\mathcal{F}(\Theta)$, the tangent space $T_F\mathcal{F}(\Theta)$ to $\mathcal{F}(\Theta)$ is $\Theta$.
            \end{itemize}
        \end{theorem}
    
        \begin{remark}
            The generic framework of Micheletti is similar to an infinite-dimensional Riemannian manifold.
        \end{remark}
    \end{enumerate}
    \item \textbf{Assumptions for $C^{k,1}(\overline{\mathbb{R}^N},\mathbb{R}^N)$ \& $C_0^{k,1}(\mathbb{R}^N,\mathbb{R}^N)$.}
    \begin{enumerate}
        \item \textbf{Checking the Assumptions.} By definition, for all integers $k\ge 0$\footnote{NQBH corrected: $\forall x,y\in\mathbb{R}^N$ instead of $\forall x,y\in\Omega$.}
        \begin{align*}
            C^{k,1}(\overline{\mathbb{R}^N},\mathbb{R}^N) := \left\{f\in C^k(\overline{\mathbb{R}^N},\mathbb{R}^N);\forall\alpha,\ 0\le|\alpha|\le k,\ \exists c > 0,\ \forall x,y\in\mathbb{R}^N,\ |\partial^\alpha f(y) - \partial^\alpha f(x)|\le c|x - y|\right\}
        \end{align*}
        endowed with the same norm: \textbf{(2.49)-(2.50)}
        \begin{align*}
            \|f\|_{C^{k,1}} := \max\left\{\max_{0\le\left|\alpha\right|\le k} \|\partial^\alpha f\|_C,\ c_k(f)\right\} = \max\left\{\|f\|_{C^k},c_k(f)\right\},\\
            c(f) := \sup_{y\ne x} \frac{|f(y) - f(x)|}{|y - x|} \mbox{ and } \forall k\ge 1,\ c_k(f) := \sum_{|\alpha| = k} c\left(\partial^\alpha f\right),
        \end{align*}
        and the convention $c_0(f) = c(f)$.
        
        $C_0^{k,1}(\mathbb{R}^N,\mathbb{R}^N)$ is the space of functions $f\in C^{k,1}(\mathbb{R}^N,\mathbb{R}^N)$ s.t. $f$ and all its derivatives go to zero at infinity.
        
        They are all Banach spaces.
        
        By definition
        \begin{align*}
            C_0^{k,1}(\mathbb{R}^N,\mathbb{R}^N)&\subset C^{k,1}(\overline{\mathbb{R}^N},\mathbb{R}^N)\subset\mathcal{B}^0(\mathbb{R}^N,\mathbb{R}^N)\subset
            C^0(\mathbb{R}^N,\mathbb{R}^N),\\
            \forall x\in\mathbb{R}^N,\ f&\mapsto f(x):\mathcal{B}^0(\mathbb{R}^N,\mathbb{R}^N)\to\mathbb{R}^N
        \end{align*}
        with continuous injections between the spaces and continuity of the evaluation map.
        
        As a consequence, the evaluation map $f\mapsto f(x)$ is continuous for all the spaces.
        
        We deal only with the case $C^{k,1}(\overline{\mathbb{R}^N},\mathbb{R}^N)$.
        
        The case $C_0^{k,1}(\mathbb{R}^N,\mathbb{R}^N)$ is obtained from the case $C^{k,1}(\overline{\mathbb{R}^N},\mathbb{R}^N)$ as in the previous section.
        
        \begin{theorem}
            Assumptions 2.1 and 2.3 are verified for the spaces $C_0^{k,1}(\mathbb{R}^N,\mathbb{R}^N)$ and $C^{k,1}(\overline{\mathbb{R}^N},\mathbb{R}^N)$, $k\ge 0$.
        \end{theorem}
        
        \begin{theorem}
            Assumption 2.2 is verified for $C_0^{k,1}(\mathbb{R}^N,\mathbb{R}^N)$ and $C^{k,1}(\overline{\mathbb{R}^N},\mathbb{R}^N)$, $k\ge 0$.
        \end{theorem}
        \item \textbf{Perturbations of the Identity \& Tangent Space.} By construction
        \begin{align*}
            \mathcal{F}(\Theta)\subset I + \Theta,
        \end{align*}
        and the tangent space in each point of the affine space $I + \Theta$ is $\Theta$.
        
        In general, $I + \Theta$ is not a subset of $\mathcal{F}(\Theta)$, but, for some spaces of transformations $\Theta$, there is a sufficiently small ball $B$ around 0 in $\Theta$ s.t. the so-called \textit{perturbations of the identity}, $I + B$, are contained in $\mathcal{F}(\Theta)$.
        
        In particular, for $\theta\in B$, the family of transformations $T_t = I + t\theta\in\mathcal{F}(\Theta)$, $0\le t\le 1$, is a $C^1$-path in $\mathcal{F}(\Theta)$ and the tangent space to each point of $\mathcal{F}(\Theta)$ can be shown to be exactly $\Theta$.
        
        \begin{theorem}
            Let $\Theta$ be equal to $C_0^{k,1}(\mathbb{R}^N,\mathbb{R}^N)$ or $C^{k,1}(\overline{\mathbb{R}^N},\mathbb{R}^N)$, $k\ge 0$.
            \begin{itemize}
                \item[(i)] The map $f\mapsto I + f:B(0,1)\subset\Theta\to\mathcal{F}(\Theta)$ is continuous.
                \item[(ii)] For all $F\in\mathcal{F}(\Theta)$, the tangent space $T_F\mathcal{F}(\Theta)$ to $\mathcal{F}(\Theta)$ is $\Theta$.
            \end{itemize}
        \end{theorem}
        
        \begin{remark}
            The generic framework of Micheletti is similar to an infinite-dimensional Riemannian manifold.
        \end{remark}
    \end{enumerate}
\end{enumerate}

\paragraph{Generalization to All Homeomorphisms \& $C^k$-Diffeomorphisms.} 
\begin{enumerate}
    \item With a variation of the generic constructions associated with the Banach space $\Theta$ of Sect. 2, it is possible to construct a metric on the whole space of homeomorphisms $\operatorname{Hom}(\mathbb{R}^N,\mathbb{R}^N)$ or on the whole space $\operatorname{Diff}_k(\mathbb{R}^N,\mathbb{R}^N)$ of $C^k$-diffeomorphisms of $\mathbb{R}^N$, $k\ge 1$, or of an open subset $D$ of $\mathbb{R}^N$.
    \item 1st, recall that, for an open subset $D$ of $\mathbb{R}^N$, the topology of the vector space $C^k(D,\mathbb{R}^N)$ can be specified by the family of seminorms \textbf{(3.1)}
    \begin{align*}
        \forall K \mbox{ compact } \subset D,\ n_K(F) := \|F\|_{C^k(K)}.
    \end{align*}
    This topology is equivalent to the one generated by the monotone increasing subfamily $\{K_i\}_{i\ge 1}$ of compact sets \textbf{(3.2)}
    \begin{align*}
        K_i := \left\{x\in D;d_{D^c}(x)\ge\frac{1}{i} \mbox{ and } |x|\le i\right\},\ i\ge 1,
    \end{align*}
    where $d_{D^c}(x) = \inf_{y\in D^c} |y - x|$.
    
    This leads to the construction of the metric \textbf{(3.3)}
    \begin{align*}
        \delta_k(F,G) := \sum_{i=1}^\infty \frac{1}{2^i}\frac{n_{K_i}(F - G)}{1 + n_{K_i}(F - G)}
    \end{align*}
    that generates the same topology as the initial family of seminorms and makes $C^k(D,\mathbb{R}^N)$ a  Fréchet space with metric $\delta_k$.
    \item Given an open subset $D$ of $\mathbb{R}^N$, consider the group \textbf{(3.4)}
    \begin{align*}
        \operatorname{Hom}_k(D) := \left\{F\in C^k(D,\mathbb{R}^N);F:D\to D \mbox{ is bijective and } F^{-1}\in C^k(D,\mathbb{R}^N)\right\}
    \end{align*}
    of transformations of $D$.
    
    For $k = 0$, $\operatorname{Hom}_0(D) = \operatorname{Hom}(D)$ and for $k\ge 1$, $\operatorname{Hom}_k(D) = \operatorname{Diff}_k(D)$, where the condition $F^{-1}\in C^k(D,\mathbb{R}^N)$ is redundant.
    
    1 possible choice of a topology on $\operatorname{Hom}_k(D)$ is the topology induced by $C^k(D,\mathbb{R}^N)$.
    
    This is the so-called \textit{weak topology} on $\operatorname{Hom}_k(D)$ (cf., e.g., M. Hirsch [1, Chap. 2]).
    
    However, $\operatorname{Hom}_k(D)$ is not necessarily complete for that topology.
    \item In order to get completeness, we adapt the constructions of A. M. Micheletti [1].
    
    Associate with $F\in\operatorname{Hom}_k(D)$ and each compact subset $K\subset D$ the following function of $I$ and $F$: \textbf{(3.5)}
    \begin{align*}
        q_k(I,F) := \inf_{F = F_1\circ\cdots\circ F_n,\,F_l\in\operatorname{Hom}_k(D)} \sum_{l=1}^n n_K(I,F_l) + n_K\left(I,F_l^{-1}\right),
    \end{align*} 
    where the infimum is taken over all \textit{finite factorizations} $F = F_1\circ\cdots\circ F_n$, $F_i\in\operatorname{Hom}_k(D)$, of $F$ in $\operatorname{Hom}_k(D)$.
    
    By construction, $q_K(I,F^{-1}) = q_K(I,F)$.
    
    Extend this definition to all pairs $F$ and $G$ in $\operatorname{Hom}_k(D)$: \textbf{(3.6)}
    \begin{align*}
        q_K(F,G) := q_k\left(I,G\circ F^{-1}\right).
    \end{align*}
    The function $q_K$ is a \textit{pseudometric}\footnote{A map $d:X\times X\to\mathbb{R}$ on a set $X$ is called a \textit{pseudometric} (or \textit{écart}, or \textit{gauge}) whenever
        \begin{itemize}
            \item[(i)] $d(x,y)\ge 0$, for all $x$ and $y$,
            \item[(ii)] $x = y\Rightarrow d(x,y) = 0$,
            \item[(iii)] $d(x,y) = d(y,x)$,
            \item[(iv)] $d(x,z)\le d(x,y) + d(y,z)$
        \end{itemize}
        (J. Dugundji [1, p. 198]).
        
        This notion is analogous to the one of seminorm for topological vector spaces, but for a group the terminology semimetric (cf. footnote 7) is used for a metric without the triangle inequality.} and the family $\{q_K:K \mbox{ compact } \subset D\}$ defines a topology on $\operatorname{Hom}_k(D)$.
    
    This topology is metrizable for the metric \textbf{(3.7)}
    \begin{align*}
        d_k(F,G) := \sum_{i=1}^\infty \frac{1}{2^i}\frac{q_{K_i}(F,G)}{1 + q_{K_i}(F,G)},
    \end{align*}
    constructed from the monotone increasing subfamily $\{K_i\}_{i\ge 1}$ of compact sets defined in (3.2).
    
    By definition, $d_k(F,G) = d_k(G,F)$ and $d_k$ is symmetrical.
    
    It is right-invariant since for all $F$, $G$, and $H$ in $\operatorname{Hom}_k(D)$
    \begin{align*}
        \forall F,G,H\in\operatorname{Hom}_k(D),\ d_k(F,G) &= d_k\left(F\circ H,G\circ H\right),\\
        \forall F\in\operatorname{Hom}_k(D),\ d_k(I,F) &= d_k\left(I,F^{-1}\right).
    \end{align*}
    
    \begin{theorem}
        $\operatorname{Hom}_k(D)$, $k\ge 0$, is a group under composition, $d_k$ is a right-invariant metric on $\operatorname{Hom}_k(D)$, and $(\operatorname{Hom}_k(D),d_k)$ is a complete metric space.
    \end{theorem}
    We need Lemmas 2.4 and 2.5 and the analogues of Lemma 2.6 and Theorem 2.11 for the spaces $C^k(K,\mathbb{R}^N)$, $K$ a compact subset of $D$, whose elements are bounded and uniformly continuous.
    
    \begin{lemma}
        Given integers $r\ge 0$ and $s > 0$ there exists a constant $c(r,s) > 0$ with the following property: for any compact $K\subset D$ and a sequence $F_1,\ldots,F_n$ in $C^r(K,\mathbb{R}^N)$ is s.t. \textbf{(3.8)}
        \begin{align*}
            \sum_{l=1}^n \|F_l - I\|_{C^r(K)} < \alpha,\ 0 < \alpha < s;
        \end{align*}
        then for the map $F = F_n\circ\cdots\circ F_1$, \textbf{(3.9)}
        \begin{align*}
            \|F - I\|_{C^r(K)}\le\alpha c(r,s).
        \end{align*}
    \end{lemma}
    
    \begin{theorem}
        Given $k\ge 0$, there exists a continuous function $c_1$ s.t. for all $F,G\in\operatorname{Hom}_k(D)$ and any compact $K\subset D$ \textbf{(3.10)}
        \begin{align*}
            \|F\circ G\|_{C^k(K)}\le\|F\|_{C^k(K)}c_1\left(\|G\|_{C^k(K)}\right).
        \end{align*}
    \end{theorem}

    \begin{remark}
        In section 2 it was possible to consider in $\mathcal{F}(\Theta)$ a subgroup $\mathcal{G}$ of translations since
        \begin{align*}
            F_a(x) = a + x\Rightarrow (F_a - I)(x) = a\in\Theta
        \end{align*}
        for spaces $\Theta$ that contain constants.
        
        Yet, since all of them were spaces of bounded functions in $\mathbb{R}^N$, it was not possible to include rotations or flips.
        
        But this can be done now in $\operatorname{Hom}_k(\mathbb{R}^N)$, and we can quotient out not only by the subgroup of translations but also by the subgroups of isometries or rotation that are important in image processing.
    \end{remark}
\end{enumerate}

\subsubsection{Transformations Generated by Velocities}
\begin{enumerate}
    \item In Chap. 3 we have constructed quotient groups of transformations $\mathcal{F}(\Theta)/\mathcal{G}$ and their associated complete Courant metrics.
    
    Such spaces are neither linear nor convex.
    
    In this chapter, we specialize the results of Chap. 3 to spaces of transformations that are generated at time $t = 1$ by the flow of a \textit{velocity field} over a generic time interval $[0,1]$ with values in the tangent space $\Theta$.
    
    The main motivation is to introduce a notion of semiderivatives in the direction $\theta\in\Theta$ on such groups as well as a tractable criterion for 
    continuity via $C^1$ or continuous paths in the quotient group endowed with the Courant metric.
    
    This point of view was adopted by J.-P. Zolésio [2, 7] as early as 1973 and considerably expanded in his \textit{thèse d'état} in 1979.
    
    1 of his motivations was to solve a \textit{shape differential equation} of the type $\mathcal{A}V(t) + G(\Omega_t(V)) = 0$, $t > 0$, where $G$ is the \textit{shape gradient} of a functional and $\mathcal{A}$ a duality operator.\footnote{Cf. J.-P. Zolésio [7] and the recent book by M. Moubachir and J.-P. Zolésio [1].}
    
    At that time most people were using a simple perturbation of the identity to compute shape derivatives.
    
    The 1st comprehensive book systematically promoting the \textit{velocity method} was published in 1992 by J. Soko\l owski and J.-P. Zolésio [9].
    
    Structural theorems for the Eulerian shape derivative of smooth domains were 1st given in 1979 by J.-P. Zolésio [7] and generalized to nonsmooth domains in 1992 by M. C. Delfour and J.-P. Zolésio [14].
    
    The velocity point of view was also adopted in 1994 by R. Azencott [1], in 1995 by A. Trouvé [2], and in 1998 by A. Trouvé [3] and L. Younes [2] to construct complete metrics and \textit{geodesic paths} in spaces of diffeomorphisms generated by a velocity field with a broad spectrum of applications to imaging.
    
    The reader is referred to the forth-coming book of L. Younes [6] for a comprehensive exposition of this work and to related papers e.g. the ones of P. W. Michor and D. Mumford [1, 2, 3] and L. Younes, P. W. Michor, I. Shah, and D. Mumford [1].
    \item In view of the above motivations, this chapter begins with Sect. 2, which specializes the results of Chap. 3 to transformations generated by velocity fields.
    
    It also explores the connections between the constructions of Azencott and Micheletti that implicitly use a notion of \textit{geodesic path with discontinuities}.
    \item Sect. 3 motivates and adapts the definitions of \textit{Gateaux} and \textit{Hadamard} semiderivatives in topological vector spaces (cf. Chap. 9) to shape functionals defined on shape spaces.
    
    The analogue of the Gateaux semiderivative for sets is obtained by the \textit{method of perturbation of the identity operator}, while the analogue of the Hadamard semiderivative comes from the \textit{velocity (speed) method}.
    
    The 1st notion does not extend to submanifolds and does not incorporate the \textit{chain rule} for the semiderivative of the composition of functions (e.g., to get the semi-derivative w.r.t. the parameters of an a priori parametrized geometry), while Hadamard does.
    
    In Chap. 9, flows of velocity fields will be adopted as the natural framework for defining shape semiderivatives.
    
    In Sect. 3.3 the velocity and transformation viewpoints will be emphasized through a series of examples of commonly used families of transformations of sets.
    
    They include \textit{$C^k$-domains, Cartesian graphs, polar coordinates}, and \textit{level sets}.
    \item The following 2 sections give technical results that will be used to characterize continuity and semidifferentiability along local paths without restricting the analysis to the subgroup $G_\Theta$ of $F(\Theta)$ of Sect. 2.
    
    In Sect. 4 we establish the equivalence between deformations obtained from a family of $C^1$-paths and deformations obtained from the flow of a velocity field.
    
    Sect. 4.1 gives the equivalence under relatively general conditions.
    
    Sect. 4.2 shows that Lipschitzian perturbations of the identity operator can be generated by the flow of a nonautonomous velocity field.
    
    In Sect. 4.3 the conditions of section 4.2 are sharpened for the special families of velocity fields in $C_0^k(\mathbb{R}^N,\mathbb{R}^N)$, $C^k(\overline{\mathbb{R}^N},\mathbb{R}^N)$, and $C^{k,1}(\overline{\mathbb{R}^N},\mathbb{R}^N)$.
    
    The \textit{constrained case} where the family of domains are subsets of a fixed \textit{holdall} is studied in Sect. 5.
    
    In both Sects. 4-5 we show that, under appropriate conditions, starting from a family of transformations is locally equivalent to starting from a family of velocity fields.
    
    This key result bridges the 2 points of view.
    \item Sect. 6 builds on the results of Sect. 4 to establish that the continuity of a shape function w.r.t. the Courant metric on $\mathcal{F}(\Theta)/\mathcal{G}$ is equivalent to its continuity along the flows of all velocity fields $V$ in $C^0([0,\tau];\Theta)$ for $\Theta$ equal to $C_0^{k+1}(\mathbb{R}^N,\mathbb{R}^N)$, $C^{k+1}(\overline{\mathbb{R}^N},\mathbb{R}^N)$, and $C^{k,1}(\overline{\mathbb{R}^N},\mathbb{R}^N)$, $k\ge 0$.
    
    This result is of both intrinsic and practical interest since it is generally easier to check the continuity along paths than directly w.r.t. the Courant metric.
    
    Finally, from the discussion in Sect. 3, the technical results of Sect. 4 will again be used in Chap. 9 to construct local $C^1$-paths generated by velocity fields $V$ in $C^0([0,\tau];\Theta)$ to define the shape semiderivatives.
\end{enumerate}

\paragraph{Metrics on Transformations Generated by Velocities.}
\begin{enumerate}
    \item \textbf{Subgroup $G_\Theta$ of Transformations Generated by Velocities.} From Chap. 3 recall the definition of the group
    \begin{align*}
        \mathcal{F}(\Theta) := \left\{I + \theta;\theta\in\Theta,\ (I + \theta)^{-1}\exists, \mbox{ and } (I + \theta)^{-1} - I\in\Theta\right\}
    \end{align*}
    for the Banach space $\Theta = C^{0,1}(\overline{\mathbb{R}^N},\mathbb{R}^N)$ with norm \textbf{(2.1)}
    \begin{align*}
        \|\theta\|_\Theta := \max\left\{\sup_{x\in\mathbb{R}^N} |\theta(x)|,\ c(\theta)\right\},\ c(\theta) := \sup_{y\ne x} \frac{|\theta(y) - \theta(x)|}{|y - x|},
    \end{align*}
    and the associated complete metric $d$ on $\mathcal{F}(\Theta)$.
    
    This section specializes to a subgroup of transformations $I + \theta$ of $\mathcal{F}(\Theta)$ that are specified via a vector field $V$ on a \textit{generic interval} $[0,1]$ with the following properties:
    \begin{itemize}
        \item[(i)] for all $x\in\mathbb{R}^N$, the function $t\mapsto V(t,x):[0,1]\to\mathbb{R}^N$ belongs to $L^1(0,1;\mathbb{R}^N)$ and \textbf{(2.2)}
        \begin{align*}
            \int_0^1 \|V(t)\|_C{\rm d}t = \int_0^1 \sup_{x\in\mathbb{R}^N} |V(t,x)|{\rm d}t < \infty;
        \end{align*}
        \item[(ii)] for almost all $t\in[0,1]$, the function $x\mapsto V(t,x):\mathbb{R}^N\to\mathbb{R}^N$ is Lipschitzian and \textbf{(2.3)}
        \begin{align*}
            \int_0^1 c\left(V(t)\right){\rm d}t = \int_0^1 \sup_{x\ne y} \frac{\left|V(t,y) - V(t,x)\right|}{|y - x|}{\rm d}t < \infty.
        \end{align*}
    \end{itemize}
    In view of the assumptions on $V$, it is natural to introduce the norm \textbf{(2.4)}
    \begin{align*}
        \|V\|_{L^1(0,1;\Theta)} := \int_0^1 \|V(t)\|_\Theta{\rm d}t.
    \end{align*}
    Associate with $V\in L^1(0,1;\Theta)$ the flow \textbf{(2.5)}
    \begin{align*}
        \frac{dT_t(V)}{dt} = V(t)\circ T_t(V),\ T_0(V) = I.
    \end{align*}
    In view of the assumptions on $V$, for each $X\in\mathbb{R}^N$, the differential equation \textbf{(2.6)}
    \begin{align*}
        x'(t;X) = V\left(t,x(t;X)\right),\ x(0;X) = X,
    \end{align*}
    has a unique solution in $W^{1,1}(0,1;\mathbb{R}^N)\subset C([0,1],\mathbb{R}^N)$, the mapping $X\mapsto x(\cdot;X):\mathbb{R}^N\to C([0,1],\mathbb{R}^N)$ is continuous, and for all $t\in[0,1]$ the mapping $X\mapsto T_t(V)(X) := x(t;X):\mathbb{R}^N\to\mathbb{R}^N$ is bijective.
    
    Define \textbf{(2.7)}
    \begin{align*}
        G_\Theta := \left\{T_1(V);\forall V\in L^1([0,1];\Theta)\right\}.
    \end{align*}
    
    \begin{theorem}
        \begin{itemize}
            \item[(i)] The set $G_\Theta$ with the composition $\circ$, \textbf{(2.8)}
            \begin{align*}
                \forall V_1,V_2\in L^1(0,1;\Theta),\ T_1(V_1)\circ T_1(V_2),
            \end{align*}
            is a subgroup of $\mathcal{F}(\Theta)$ and for all $V\in L^1(0,1;\Theta)$, \textbf{(2.9)}
            \begin{align*}
                \sup_{0\le t\le 1} \|T_t(V) - I\|_\Theta = \sup_{0\le t\le 1} \|\theta_V(t)\|_\Theta\le\|V\|_{L^1}e^{2\left(1 + \|V\|_{L^1}\right)},
            \end{align*}
            where $\theta_V(t) := T_t(V) - I$ and $L^1$ stands for $L^1(0,1;\Theta)$. In particular, for all $t$, $T_t(V)\in\mathcal{F}(\Theta)$ and $t\mapsto T_t(V):[0,1]\to\mathcal{F}(\Theta)$ is a continuous path.
            \item[(ii)] For all $V\in L^1(0,1;\Theta)$, $(T_1(V))^{-1} = T_1(V^-)$ for $V^-(t,x) = -V(1 - t,x)$, \textbf{(2.10)-(2.11)}
            \begin{align*}
                -\theta_V\circ(I + \theta_V)^{-1} &= (T_1(V))^{-1} - I = T_1(V^-) - I = \theta_{V^-},\\
                \|V^-\|_{L^1(0,1;\Theta)} &= \|V\|_{L^1(0,1;\Theta)},
            \end{align*}
            and \textbf{(2.12)}
            \begin{align*}
                \sup_{0\le t\le 1} \|T_t(V)^{-1} - I\|_\Theta = \sup_{0\le t\le 1} \|\theta_V^-(t)\|_\Theta\le\|V\|_{L^1}e^{2\left(1 + \|V\|_{L^1}\right)}.
            \end{align*}
            \item[(iii)] For all $V\in L^1(0,1;\Theta)$, \textbf{(2.13)-(2.14)}
            \begin{align*}
                \|T_1(V) - I\|_\Theta + \|(T_1(V))^{-1} - I\|_\Theta&\le 2\|V\|_{L^1(0,1;\Theta)}e^{2\left(1 + \|V\|_{L^1(0,1;\Theta)}\right)}\\
                \Rightarrow d(T_1(V),I)&\le 2\|V\|_{L^1(0,1;\Theta)}e^{2\left(1 + \|V\|_{L^1(0,1;\Theta)}\right)}.
            \end{align*}
            \item[(iv)] For all $V,W\in L^1(0,1;\Theta)$, \textbf{(2.15)}
            \begin{align*}
                \max\left\{\sup_{0\le t\le 1} \|\theta_W(t) - \theta_V(t)\|_\Theta,\ \sup_{0\le t\le 1} \|\theta_{W^-}(t) - \theta_{V-}(t)\|_\Theta\right\}\le\|V\|_{L^1}e^{1 + \|V\|_{L^1}}\left(1 + \|W\|_{L^1}e^{2\left(1 + \|W\|_{L^1}\right)}\right)\|W(s) - V(s)\|_{L^1},
            \end{align*}
            where $L^1$ stands for $L^1(0,1;\Theta)$.
        \end{itemize}
    \end{theorem}

    \begin{remark}
        Note that the condition $(I - \theta)^{-1}$ exists and $(I - \theta)^{-1}\in\Theta$ in the definition of $\mathcal{F}(\Theta)$ is automatically verified for the transformations $I + \theta_V$ generated by a velocity field $V\in L^1(0,1;\Theta)$.
    \end{remark}
    \item \textbf{Complete Metrics on $G_\Theta$ and Geodesics.} Since $G_\Theta$ is a subgroup of the complete group $\mathcal{F}(\Theta)$, its closure $\overline{G}_\Theta$ w.r.t. the metric $d$ of $\mathcal{F}(\Theta)$ is a closed subgroup whose elements can be ``approximated'' by elements constructed from the flow of a velocity field.
    
    Is $(G_\Theta,d)$ complete?
    
    Can a velocity field be associated with a limit element?
    
    It is very unlikely that $(G_\Theta,d)$ can be complete unless we strengthen the metric $d$ to make the associated Cauchy sequence of velocities $\{V_n\}$ converge in $L^1(0,1;\Theta)$.
    
    %
    We have seen that for each $V\in L^1(0,1;\Theta)$, we have a continuous path $t\mapsto T_t(V)$ in $G_\Theta$.
    
    Since the part of the metric on $I + \theta_V$ is defined as an infimum over all finite factorizations $(I + \theta_{V_1})\circ\cdots \circ(I + \theta_{V_n})$ of $I + \theta_V$ and $(I + \theta_{V_1^-})\circ\cdots\circ(I + \theta_{V_n^-})$ of $(I + \theta_V)^{-1}$,
    \begin{align*}
        T_1(V) &= T_1(V_1)\circ T_1(V_2)\circ\cdots\circ T_1(V_n),\\
        d\left(I,T_1(V)\right) &= \inf_{T_1(V) = T_1(V_1)\circ\cdots\circ T_1(V_n)} \sum_{i=1}^n \|\theta_{V_i}\|_\Theta + \|\theta_{V_i^-}\|_\Theta,
    \end{align*}
    is there a velocity $V^*\in L^1(0,1;\Theta)$ that achieves that infimum?
    
    Do we have a \textit{geodesic path} between $I$ and $T_1(V)$ that would be achieved by $V^*\in L^1(0,1;\Theta)$?
    
    %
    Given a velocity field we have constructed a continuous path $t\mapsto T_t(V):[0,1]\to\mathcal{F}(\Theta)$.
    
    At each $t$, there exists $\delta > 0$ sufficiently small s.t. the path $t\mapsto T_{t+s}(V)\circ T_t^{-1}(V):[0,\delta]\to\mathcal{F}(\Theta)$ is differentiable in $t$ in the group $\mathcal{F}(\Theta)$: \textbf{(2.17)-(2.18)}
    \begin{align*}
        \frac{T_{t+s}(V)\circ T_t^{-1}(V) - I}{s} &= \frac{1}{s}\int_t^{t+s} V(r)\circ T_r(V){\rm d}r\circ T_t^{-1}(V)\to V(t)\circ T_t(V)\circ T_t^{-1}(V) = V(t),\\
        \Rightarrow\frac{d_{\mathcal{F}}T_t}{dt} &= V(t) \mbox{ a.e. in } \Theta.
    \end{align*}
    With this definition a jump from $I + \theta_1$ to $[I + \theta_2]\circ[I + \theta_1]$ at time $\frac{1}{2}$ becomes
    \begin{align*}
        T_{1/2 + s}(V)\circ T_{1/2}^{-1}(V) - I = [I + \theta_2]\circ[I + \theta_1]\circ[I + \theta_1]^{-1} - I = I + \theta_2 - I = \theta_2,
    \end{align*}
    i.e., a \textit{Dirac delta function} $\theta_2$ in the tangent space $\Theta$ to $\mathcal{F}(\Theta)$ at $\frac{1}{2}$.
    
    Now the norm of the velocity $V = \theta_2\delta_{1/2}$ in the space of measures becomes
    \begin{align*}
        \|V\|_{M_1\left((0,1);\Theta\right)} = \int_0^1 \|\theta_2\|_\Theta\delta_{1/2}{\rm d}t = \|\theta_2\|_\Theta.
    \end{align*}
    If we have $n$ such jumps,
    \begin{align*}
        \|V\|_{M_1\left((0,1);\Theta\right)} = \sum_{i=1}^n \|\theta_i\|_\Theta.
    \end{align*}
    The metric of Micheletti is the infimum of this norm over all finite factorizations.
    
    %
    A consequence of this analysis is that a factorization of an element $I + \theta$ of the form $(I + \theta_1)\circ(I + \theta_2)\circ\cdots\circ(I + \theta_n)$ is in fact a path $t\mapsto T_t:[0,1]\to\mathcal{F}(\Theta)$ of bounded variations in $\mathcal{F}(\Theta)$ between $I$ and $I + \theta$ by assigning times $0 < t_1 < \cdots < t_n < 1$ to each jump in the group $\mathcal{F}(\Theta)$:
    \begin{equation*}
        T_t(V) := \left\{\begin{split}
            &I, &&0\le t < t_1,\\
            &I + \theta_1, &&t_1\le t < t_2,\\
            &(I + \theta_1)\circ(I + \theta_2), && t_2\le t < t_3,\\
            &\ldots\\
            &(I + \theta_1)\circ\cdots\circ(I + \theta_n), &&t_n\le t\le 1,
        \end{split}\right.
    \end{equation*}
    \begin{align*}
        V(t) := \sum_{i=1}^n 
        \theta_i\delta(t_i)\Rightarrow\frac{d_{\mathcal{F}}T_t}{dt} = V(t),\ \|V\|_{M_1\left((0,1);\Theta\right)} = \sum_{i=1}^n \|\theta_i\|_\Theta,
    \end{align*}
    where the time-derivative is taken in the group sense (2.17) and $V$ is a bounded measure.
    
    With this interpretation in mind, for $V\in L^1(0,1;\Theta)$ the metric of Micheletti \textit{should} reduce to
    \begin{align*}
        d\left(I,T_1(V)\right) = \inf_{v\in L^1(0,1;\Theta),\,T_1(v) = T_1(V)} \int_0^1 \|v(t)\|_\Theta{\rm d}t = \inf_{v\in L^1(0,1;\Theta),\,T_1(v) = T_1(V)} \int_0^1 \left\|\frac{d_{\mathcal{F}}T_t(v)}{dt}\right\|_\Theta{\rm d}t,
    \end{align*}
    where the derivative is in the group $\mathcal{F}(\Theta)$.
    
    As a result we could talk of a geodesic path in  $\mathcal{F}(\Theta)$ between $I$ and $T_1(V)$.
    
    If this could be justified, the next question is whether $G_\Theta$ is complete in $(\mathcal{F}(\Theta),d)$ or complete w.r.t. the metric constructed from the function \textbf{(2.19)-(2.20)}
    \begin{align*}
        d_G\left(I,T_1(V)\right) &:= \inf_{v\in L^1(0,1;\Theta),\,T_1(v) = T_1(V)} \int_0^1 \|v(t)\|_\Theta{\rm d}t,\\
        d_G\left(T_1(V),T_1(W)\right) &:= d\left(T_1(V)\circ T_1(W)^{-1},I\right).
    \end{align*}
    By definition
    \begin{align*}
        d_G\left(T_1(V)^{-1},I\right) = d_G\left(T_1(V),I\right),\ d_G\left(T_1(V),T_1(W)\right) = d_G\left(T_1(W),T_1(V)\right),
    \end{align*}
    and $d_G$ is right-invariant.
    
    The infimum is necessary since the velocity taking $I$ to $T_1(V)$ is not unique.
    
    \begin{theorem}
        Let $\Theta$ be equal to $C^{k,1}(\overline{\mathbb{R}^N},\mathbb{R}^N)$, $C^{k+1}(\overline{\mathbb{R}^N},\mathbb{R}^N)$, $C_0^{k+1}(\mathbb{R}^N,\mathbb{R}^N)$, or $\mathcal{B}^{k+1}(\mathbb{R}^N,\mathbb{R}^N)$, $k\ge 0$. Then $d_G$ is a right-invariant metric on $G_\Theta$.
    \end{theorem}
    
    \begin{remark}
        The theorem is also true for the right-invariant metric \textbf{(2.21)-(2.22)}\footnote{NQBH corrected: the RHS should be $\inf_{W\in L^1(0,1;\Theta),\,T_1(W) = T_1(V)} d\left(T_1(W),T_1(V)\right) + \|W\|_{L^1(0,1;\Theta)}$ instead of $d\left(T_1(W),T_1(V)\right) + \inf_{W\in L^1(0,1;\Theta),\,T_1(W) = T_1(V)} \|W\|_{L^1(0,1;\Theta)}$.}
        \begin{align*}
            d_G'\left(T_1(V),I\right) &:= \inf_{W\in L^1(0,1;\Theta),\,T_1(W) = T_1(V)} d\left(T_1(W),T_1(V)\right) + \|W\|_{L^1(0,1;\Theta)},\\
            d_G'\left(T_1(V),T_1(W)\right) &:= d_G'\left(T_1(V)\circ T_1(W)^{-1},I\right).
        \end{align*}
    \end{remark}

    Getting the completeness with the metric $d_G$ and even $d_G'$ is not obvious. 
    
    For a Cauchy sequence $\{T_1(V_n)\}$, the  $L^1$-convergence of the velocities $\{V_n\}$ to some $V$ would yield the convergence of $\{T_1(V_n)\}$ to $T_1(V)$ and even the convergence of the geodesics, but the metrics do not seem sufficiently strong to do that.
    
    To appreciate this point, 1st establish an inequality that really follows from the triangle inequality for $d_G$.
    
    Associate with $W$ a $w$ s.t. $T_1(w) = T_1(W)$, with $V$ a $v$ s.t. $T_1(v) = T_1(V)$, and let $u$ be s.t. $T_1(u) = T_1(W)\circ T_1(V)^{-1}$.
    
    Then the concatenation $w*v^-$ is s.t. $T_1(v*u) = T_1(W)$.
    
    One could think of many ways to strengthen the metric.
    
    E.g., one could introduce the minimization of the whole $W^{1,1}(0,1;\Theta)$-norm of the path $t\mapsto T_t(v)$ w.r.t. $v$ from which the minimizing velocity $v(t) =\partial_tT_t\circ T_t^{-1}$ could be recovered, but the triangle inequality would be lost.
    \item \textbf{Constructions of Azencott \& Trouvé.} In 1994 R. Azencott [1] defined the following program.
    
    Given a smooth manifold $M$, consider the time continuous curve $T = (T_t)_{0\le t\le 1}$ in $\operatorname{Aut}(M)$ solutions of \textbf{(2.23)}
    \begin{align*}
        \frac{\partial T_t}{\partial t} = V(t)\circ T_t,\ T_0 = I,
    \end{align*}
    for a continuous time family $V(t)$ of vector fields in a Hilbert space $H\subset\operatorname{Aut}(M)$ and define \textbf{(2.24)}
    \begin{align*}
        d(I,\phi) = \inf_{V\in C([0,1];H),\,T_1(V) = \phi} l(\phi) \mbox{ with } l(\phi) := \int_0^1 |V(t)|_I{\rm d}t.
    \end{align*}
    A fully rigorous construction of this distance, in the context of infinite-dimensional Lie groups, was performed by A. Trouvé [2] in 1995 and [3] in 1998.
    
    He shows that the subset of $\phi\in\operatorname{Aut}(M)$ for which $d(I,\phi)$ can be defined is a subgroup $\mathcal{A}$ of invertible mappings.
    
    He extends this distance between 2 arbitrary mappings $\varphi$ and $\psi$ by right-invariance:
    \begin{align*}
        d(\phi,\psi) := d\left(I,\psi\circ\phi^{-1})\right).
    \end{align*}
    Hence, in this extended framework, the new problem should be to find in $\mathcal{A}$
    \begin{align*}
        \hat{\phi} = \operatorname{argmin}\frac{1}{2}\int_M \left|\widetilde{f} - f\circ\phi\right|^2{\rm d}x + \frac{1}{2}d\left({\rm Id},\phi\right)^2,
    \end{align*}
    where $M$ is a smooth manifold, the function $f:M\to\mathbb{R}$ is a \textit{high dimensional representation template}, and $\widetilde{f}$ is a new \textit{observed image} belonging to this family (cf. A. Trouvé [3]).
    
    This approach was further developed by the group of Azencott (cf., e.g., L. Younes [1]).
    
    %
    Quoting from the introduction of A. Trouvé [3] in 1998:
    \begin{quotation}
        $\ldots$Hence, for a large deformation $\phi$, one can consider $\phi$ as the concatenation of small deformations $\phi_{u_i}$.
        
        More precisely, if $\phi = \Phi_n$, where the $\Phi_k$ are recursively defined by $\Phi_0 = {\rm Id}$ and $\phi_{k+1} = \phi_{u_{k+1}}\circ\Phi_k$, the family $\Phi = \left(\Phi_k\right)_{0\le k\le n}$ defines a polygonal line in $\operatorname{Aut}(M)$ whose length $l(\Phi)$ can be approximated by $l(\phi) = \sum_{k=1}^n |u_k|_e$.
        
        At a naive level, one could define the distance $d({\rm Id},\phi)$ as the infimum of $l(\Phi_n)$ for all family $\Phi$ s.t. $\Phi_n = \phi$.
        
        For a more rigorous setting, one should consider the time continuous curve $\Phi = (\Phi_t)_{0\le t\le 1}$ in $\operatorname{Aut}(M)$. $\ldots$
    \end{quotation}
    In fact what he describes as a naive approach is the very precise construction of A. M. Micheletti [1] in 1972.
    
    He is certainly not the 1st author who has overlooked that paper in Italian, where the central completeness result that is of interest to us was just a lemma in her analysis of the continuity of the first eigenvalue of the Laplacian.
    
    We brought it up only in the 2001 edition of this book.
    
    %
    By introducing a Hilbert space $H\subset\Theta = C^{0,1}(\overline{\mathbb{R}^N},\mathbb{R}^N)$ and minimizing over $L^2(0,1;H)$ rather than $L^1(0,1;\Theta)$, the bounded sequence $\{v_n\}$ lives in $L^2(0,1;H)$, where there exist a $v\in L^2(0,1;H)$ and a weakly converging subsequence to $v$.
    
    By using the complete metric of Micheletti, we had the existence of a $F\in\mathcal{F}(\Theta)$ s.t. $T_1(v_n)\to F$, but now we have a candidate $v$ for which $F = T_1(v)$.
    
    This is very much analogous to the \textit{Hilbert uniqueness method} of J.-L. Lions [1] in the context of the controllability of PDEs (cf. also R. Glowinski and J.-L. Lions [1, 2]).
    
    \begin{remark}
        It is important to understand that finding a complete metric of \emph{Courant type} for the various spaces of diffeomorphisms is a much less restrictive problem than constructing a metric from geodesics.
        
        The existence of a geodesic is not required for the Courant metrics.
    \end{remark}
\end{enumerate}

\paragraph{Semiderivatives via Transformations Generated by Velocities.}
\begin{enumerate}
    \item \textbf{Shape Function.} All spaces $\mathcal{F}(\Theta)$ and $\mathcal{F}(\Theta)/\mathcal{G}(\Omega_0)$ in Chap. 3 associated with the family of sets \textbf{(3.1)}
    \begin{align*}
        \mathcal{X}\left(\Omega_0\right) = \left\{F\left(\Omega_0\right);\forall F\in\mathcal{F}(\Theta)\right\}
    \end{align*}
    are nonlinear and nonconvex and the elements of $\mathcal{F}(\Theta)/\mathcal{G}(\Omega_0)$ are equivalence classes of transformations.
    
    Defining a differential w.r.t. such spaces is similar to defining a differential on an infinite-dimensional manifold.
    
    Fortunately, the tangent space is invariant and equal to $\Theta$ in every point of $F(\Theta)$.
    
    This considerably simplifies the analysis.
    
    \begin{definition}
        Given a nonempty subset $D$ of $\mathbb{R}^N$, consider the set $\mathcal{P}(D) = \{\Omega;\Omega\subset D\}$ of subsets of $D$. The set $D$ will be referred to as the underlying \emph{holdall} or \emph{universe}. A \emph{shape functional} is a map \textbf{(3.2)}
        \begin{align*}
            J:\mathcal{A}\to E
        \end{align*}
        from some \emph{admissible family} $\mathcal{A}$ of sets in $\mathcal{P}(D)$ into a topological space $E$.
    \end{definition}
    E.g., $\mathcal{A}$ could be the set $\mathcal{X}(\Omega)$.
    
    $D$ can represent some physical or mechanical constraint, a submanifold of $\mathbb{R}^N$, or some mathematical constraint.
    
    In most cases, it can be chosen large and as smooth as necessary for the analysis.
    
    In the unconstrained case, $D$ is equal to $\mathbb{R}^N$.
    \item \textbf{Gateaux \& Hadamard Semiderivatives.} Consider the solution (flow) of the differential equation \textbf{(3.3)}
    \begin{align*}
        \frac{dx}{dt}(t,X) = V\left(t,x(t,X)\right),\ t\ge 0,\ x(0,X) = X,
    \end{align*}
    for velocity fields $V(t)(x) := V(t,x)$ (cf. Fig. 4.1).
    
    \textsf{Fig. 4.1. Transport of $\Omega$ by the velocity field $V$.}
    
    It generates the family of transformations $\{T_t:\mathbb{R}^N\to\mathbb{R}^N;t\ge 0\}$ defined as follows: \textbf{(3.4)}
    \begin{align*}
        X\mapsto T_t(X) := x(t;X):\mathbb{R}^N\to\mathbb{R}^N.
    \end{align*}
    Given an initial set $\Omega\subset\mathbb{R}^N$, associate with each $t > 0$ the new set \textbf{(3.5)}
    \begin{align*}
        \Omega_t := T_t(\Omega) = \left\{T_t(X);\forall X\in\Omega\right\}.
    \end{align*}
    This perturbation of the initial set is the basis of the \textit{velocity (speed ) method}.
    
    %
    The choice of the terminology ``velocity'' to describe this method is accurate but may become ambiguous in problems where the variables involved are themselves ``physical velocities'': this situation is commonly encountered in continuum mechanics.
    
    In such cases it may be useful to distinguish between the ``artificial velocity'' and the ``physical velocity''.
    
    This is at the origin of the terminology \textit{speed method}, which has often been used in the literature.
    
    The latter terminology is convenient, but is not as accurate as the \textit{velocity method}.
    
    We shall keep both terminologies and use the one that is most suitable in the context of the problem at hand.
    
    %
    It is important to work with the weakest notion of \textit{semiderivative} that preserves the basic elements of the \textit{classical differential calculus} e.g. the \textit{chain rule} for the differentiation of the composition of functions.
    
    It should be able to handle the norm of a function or functions defined as an \textit{upper} or \textit{lower envelope} of a family of differentiable functions.
    
    Since the spaces of geometries are generally nonlinear and nonconvex, we also need a notion of semiderivative that yields differentials in the tangent space as for manifolds.
    
    We are looking for a very general notion of semidifferential \textit{but not more}!
    
    In that context the most suitable candidate is the Hadamard semiderivative\footnote{In his 1937 paper, M. Fréchet [1] extends to function spaces the Hadamard derivative and promotes it as more general than his own Fréchet derivative since it does not require the space to have a metric in infinite-dimensional spaces.
        
        In the same paper, he also introduces a relaxation of the Hadamard derivative that is almost a semiderivative.
        
        A precise definition of the Hadamard semiderivative explicitly appears in J.-P. Penot [1] in 1978 and in A. Bastiani [1] in 1964 as a well-known notion.} that will be introduced later.
    
    Important nondifferentiable functions are Hadamard semidifferentiable and the chain rule is verified.
    
    %
    We proceed step by step.
    
    A 1st candidate for the \textit{shape semiderivative} of $J$ at $\Omega$ in the direction $V$ would be \textbf{(3.6)}
    \begin{align*}
        dJ(\Omega;V) := \lim_{t\downarrow 0} \frac{J\left(\Omega_t(V)\right) - J(\Omega)}{t} \mbox{ (when the limit exists).}
    \end{align*}
    This very weak notion depends on the history of $V$ for $t > 0$ and will be too weak for most of our purposes.
    
    E.g., its composition with another function would not verify the chain rule.
    
    To make it compatible with the usual notion of semiderivative in a direction $\theta$ of the tangent space $\Theta$ to $\mathcal{F}(\Theta)$ in $F(\Omega)$, it must be strengthened as follows.
    
    \begin{definition}
        Given $\theta\in\Theta$ \textbf{(3.7)}
        \begin{align*}
            dJ(\Omega;\theta) := \lim_{V\in C^0([0,\tau],\Theta),\,V(0) = \theta,\,t\downarrow 0} \frac{J\left(\Omega_t(V)\right) - J(\Omega)}{t},
        \end{align*}
        where the limit depends only on $\theta$ and is independent of the choice of $V$ for $t > 0$.
    \end{definition}
    Equivalently, the limit $dJ(\Omega;\theta)$ is independent of the way we approach $\Omega$ for a fixed direction $\theta$.
    
    This is precisely the adaptation of the Hadamard semiderivative.\footnote{Cf. footnote 2 in Sect. 2 of Chap. 9 for the definitions and properties and the comparison of the various notions of derivatives and semiderivatives in Banach spaces.}
    
    Definitions based on a \textit{perturbation of the identity} \textbf{(3.8)}
    \begin{align*}
        T_t(X) := x(t) = X + t\theta(X),\ t\ge 0,
    \end{align*}
    can be recast in the velocity framework by choosing the velocity field $V(t) = \theta\circ T_t^{-1}$, i.e. \textbf{(3.9)}
    \begin{align*}
        \forall x\in\mathbb{R}^N,\ \forall t\ge 0,\ V(t,x) := \theta\left(T_t^{-1}(x)\right) = \theta\circ\left(I + t\theta\right)^{-1}(x),
    \end{align*}
    which requires the existence of the inverse for sufficiently small $t$.
    
    With that choice for each $X$, $x(t) = T_t(X)$ is the solution of the differential equation \textbf{(3.10)}
    \begin{align*}
        \frac{dx}{dt}(t) = V\left(t,x(t)\right),\ x(0) = X.
    \end{align*}
    Under appropriate continuity and differentiability assumptions, \textbf{(3.11)-(3.12)}
    \begin{align*}
        \forall x\in\mathbb{R}^N,\ V(0)(x) &:= V(0,x) = \theta(x),\\
        \forall x\in\mathbb{R}^N,\ \dot{V}(0)(x) &:= \left.\frac{\partial V}{\partial t}(t,x)\right|_{t=0} = - \left[D\theta(x)\right]\theta(x),
    \end{align*}
    where $D\theta(x)$ is the Jacobian matrix of $\theta$ at the point $x$.
    
    In compact notation, \textbf{(3.13)}
    \begin{align*}
        V(0) = \theta \mbox{ and } \dot{V}(0) = -[D\theta]\theta.
    \end{align*}
    This last computation shows that at time 0, the points of the domain $\Omega$ are simultaneously affected by the velocity field $V(0) = \theta$ and the acceleration field $\dot{V}(0) = -[D\theta]\theta$.
    
    The same result will be obtained without \textit{acceleration term} by replacing $I + t\theta$ by the transformation $T_t$ generated by the solutions of the differential equation
    \begin{align*}
        \frac{dx}{dt}(t) = \theta(x(t)),\ x(0) = X,\ T_t(X) := x(t),
    \end{align*}
    for which $V(0) = \theta$ and $\dot{V} = 0$.
    
    Under suitable assumptions the 2 methods will produce the same 1st-order semiderivative.
    
    %
    The definition of a shape semiderivative \textbf{(3.14)}
    \begin{align*}
        dJ(\Omega;\theta) := \lim_{t\downarrow 0} \frac{J\left([I + t\theta]\Omega\right) - J(\Omega)}{t}
    \end{align*}
    by perturbation of the identity is similar to a Gateaux semiderivative that will not verify the chain rule.
    
    Moreover, 2nd-order semiderivatives will differ by an acceleration term that will appear in the expression obtained by the method of perturbation of the identity.
    \item \textbf{Examples of Families of Transformations of Domains.} In Sect. 3.2 we have formally introduced a notion of semiderivative of a shape functional via transformations generated by flows of velocity fields and perturbations of the identity operator as special cases of the velocity method.
    
    %
    Before proceeding with a more abstract treatment, the authors present several examples of definitions of shape semiderivatives that can be found in the literature.
    
    The authors consider special classes of domains ($C^\infty$, $C^k$, Lipschitzian), Cartesian graphs, polar coordinates, and level sets that provide classical examples of parametrized and/or constrained deformations.
    
    In each case we construct the associated underlying family (not necessarily unique) of transformations $\{T_t;0\le t\le\tau\}$.
    \begin{enumerate}
        \item \textbf{$C^\infty$-Domains.} Let $\Omega$ be an open domain of class $C^\infty$ in $\mathbb{R}^N$.
        
        Recall from Sect. 5 of Chap. 2 that in any point $x\in\Gamma$ the unit normal normal is given by \textbf{(3.15)}
        \begin{align*}
            \forall y\in\Gamma_x := \mathcal{U}(x)\cap\Gamma,\ {\bf n}(y) = \frac{{\bf m}_x(y)}{\left|{\bf m}_x(y)\right|},
        \end{align*}
        where \textbf{(3.16)-(3.17)}
        \begin{align*}
            {\bf m}_x(y) &= -\left(Dh_x\right)^{-\top}\left(h_x^{-1}(y)\right){\bf e}_N \mbox{ in } \mathcal{U}(x)\\
            \Rightarrow{\bf n} &= -\frac{\left(Dh_x\right)^{-\top}{\bf e}_N}{\left|\left(Dh_x\right)^{-\top}{\bf e}_N\right|}\circ h_x^{-1}.
        \end{align*}
        When $\Gamma$ is compact, it is possible to find a finite sequence of points $\{x_j;1\le j\le J\}$ in $\Gamma$ s.t.
        \begin{align*}
            \Gamma\subset\mathcal{U} := \bigcup_{j=1}^J \mathcal{U}_j,\ \mathcal{U}_j := \mathcal{U}(x_j).
        \end{align*}
        As in the definition of the boundary integral, associate with $\{U_j\}$ a partition of unity $\{r_j\}$:
        \begin{align*}
            r_j\in\mathcal{D}\left(\mathcal{U}_j\right),\ 0\le r_j\le 1,\ \sum_{j=1}^J r_j = 1 \mbox{ in } \mathcal{U}_0,
        \end{align*}
        for some neighborhood $\mathcal{U}_0$ of $\Gamma$ s.t.
        \begin{align*}
            \Gamma\subset\mathcal{U}_0\subset\overline{\mathcal{U}_0}\subset\mathcal{U}.
        \end{align*}
        For the $C^\infty$-domain $\Omega$ the normal satisfies
        \begin{align*}
            {\bf n} = \sum_{j=1}^J r_j{\bf n} = \sum_{j=1}^J r_j\frac{{\bf m}_j}{\left|{\bf m}_j\right|}\in C^\infty\left(\Gamma;\mathbb{R}^N\right)
        \end{align*}
        since
        \begin{align*}
            \forall j,\ \frac{{\bf m}_j}{\left|{\bf m}_j\right|}\circ h_j\in C^\infty(B;\mathbb{R}^N).
        \end{align*}
        Given any $\rho\in C^\infty(\Gamma)$ and $t\ge 0$, consider the following perturbation $\Gamma_t$ of $\Gamma$ along the normal field ${\bf n}$: \textbf{(3.18)}
        \begin{align*}
            \Gamma_t := \left\{x\in\mathbb{R}^N;x = X + t\rho(X){\bf n}(X),\ \forall X\in\Gamma\right\}.
        \end{align*}
        Claim for $\tau$ sufficiently small and all $t$, $0\le t\le\tau$, the set $\Gamma_t$ is the boundary of a $C^\infty$-domain $\Omega_t$ by constructing a transformation $T_t$ of $\mathbb{R}^N$ which maps $\Omega$ onto $\Omega_t$ and $\Gamma$ onto $\Gamma_t$.
        
        1st construct an extension ${\bf N}\in\mathcal{D}(\mathbb{R}^N,\mathbb{R}^N)$ of the normal field ${\bf n}$ on $\Gamma$.
        
        Define \textbf{(3.19)}
        \begin{align*}
            {\bf m} := \sum_{j=1}^J r_j{\bf m}_j\in\mathcal{D}\left(\mathcal{U},\mathbb{R}^N\right).
        \end{align*}
        By construction \textbf{(3.20)}
        \begin{align*}
            {\bf m}\in C^\infty\left(\Gamma;\mathbb{R}^N\right),
        \end{align*}
        ${\bf m}\ne{\bf 0}$ on $\Gamma$, and there exists a neighborhood $\mathcal{U}_1$ of $\Gamma$ contained in $\mathcal{U}_0$ where ${\bf m}\ne{\bf 0}$ since ${\bf m}$ is at least $C^1$.
        
        %
        Now construct a function $r_0$ in $\mathcal{D}(\mathcal{U}_1)$, $0\le r_0(x)\le 1$, and a neighborhood $\mathcal{V}$ of $\Gamma$ s.t.
        \begin{align*}
            r_0 = 1 \mbox{ in } \mathcal{V} \mbox{ and } \Gamma\subset\mathcal{V}\subset\overline{\mathcal{V}}\subset\mathcal{U}_1.
        \end{align*}
        Define the vector field \textbf{(3.21)}
        \begin{align*}
            \forall{\bf x}\in\mathbb{R}^N,\ {\bf N}({\bf x}) := r_0({\bf x})\frac{{\bf m}({\bf x})}{\left|{\bf m}({\bf x})\right|}.
        \end{align*}
        Hence ${\bf N}$ belongs to $\mathcal{D}(\mathbb{R}^N,\mathbb{R}^N)$ since $\operatorname{supp}{\bf N}\subset\overline{\mathcal{V}}$ is compact.
        
        Moreover
        \begin{align*}
            {\bf N} = \frac{{\bf m}}{\left|{\bf m}\right|} \mbox{ in } \mathcal{V}\Rightarrow{\bf N}(x) = {\bf n}(x) \mbox{ on } \Gamma = \Gamma\cap\mathcal{V}.
        \end{align*}
        For each $j$, $\rho\circ h_j\in C^\infty(\overline{B_0})$ and the extensions
        \begin{align*}
            \rho_j^h(\zeta) = \rho_j^h(\zeta',\zeta_N) := \rho\left(h_j\left(\zeta',0\right)\right),\ \forall\zeta\in B,\ \tilde{\rho}_j := \rho_j^h\circ g_j,
        \end{align*}
        belong, respectively, to $C^\infty(\overline{B})$ and $C^\infty(\overline{\mathcal{U}_j})$. Then
        \begin{align*}
            \tilde{\rho} := \sum_{j=1}^J r_j\tilde{\rho}_j\in\mathcal{D}(\mathbb{R}^N)
        \end{align*}
        is an extension of $\rho$ from $\Gamma$ to $\mathbb{R}^N$ with compact support since $\operatorname{supp}r_j\subset\mathcal{U}_j\subset\mathcal{U}$.
        
        %
        Define the following transformation of $\mathbb{R}^N$ \textbf{(3.22)}
        \begin{align*}
            T_t(X) := X + t\tilde{\rho}(X){\bf N}(X),\ t\ge 0.
        \end{align*}
        By construction, $\tilde{\rho}{\bf N}$ is uniformly Lipschitzian in $\mathbb{R}^N$ and, by Theorem 4.2 in Sect. 4.2, there exists $0 < \tau$ s.t. $T_t$ is bijective and bicontinuous from $\mathbb{R}^N$ onto itself.
        
        As a result from J. Dugundji [1] for $0\le t\le\tau$
        \begin{align*}
            \Omega_t &= T_t(\Omega) = \left[I + t\tilde{\rho}{\bf N}\right](\Omega),\\
            \partial\Omega_t &= T_t(\partial\Omega) = \left[I + t\tilde{\rho}{\bf N}\right](\partial\Omega) = \left[I + t\rho{\bf n}\right](\partial\Omega) = \Gamma_t.
        \end{align*}
        Since the domain $\Omega_t$ is specified by its boundary $\Gamma_t$, it depends only on $\rho$ and not on its extension $\tilde{\rho}$.
        
        The special transformation $T_t$ introduced here is of class $C^\infty$, i.e., $T_t\in C^\infty(\mathbb{R}^N,\mathbb{R}^N)$, and $\frac{1}{t}[T_t - I]$ is proportional to the normal field ${\bf n}$ on $\Gamma$, but it is not proportional to the normal ${\bf n}_t$ on $\Gamma_t$ for $t > 0$.
        
        In other words, at $t = 0$ the deformation is along ${\bf n}$, but at $t > 0$ the deformation is generally not along ${\bf n}_t$.
        
        %
        If $J(\Omega)$ is a real-valued shape functional defined on $C^\infty$-domains in $D$, the semiderivative (if it exists) is defined as follows: for all $\rho\in C^\infty(\Gamma)$ \textbf{(3.23)}
        \begin{align*}
            d_{\bf n}J(\Omega;\tilde{\rho}) := \lim_{t\downarrow 0} \frac{J\left(\left(I + t\tilde{\rho}{\bf N}\right)\Omega\right) - J(\Omega)}{t}.
        \end{align*}
        It turns out that this limit depends only on $\rho$ and not on its extension $\tilde{\rho}$.
        \item \textbf{$C^k$-Domains.} When $\Omega$ is a domain of class $C^k$ with boundary $\Gamma$, the normal field ${\bf n}$ belongs to $C^{k-1}(\Gamma,\mathbb{R}^N)$.
        
        Therefore, choosing deformations along the normal would yield transformations $\{T_t\}$ mapping $C^k$-domains $\Omega$ onto $C^{k-1}$-domains $\Omega_t = T_t(\Omega)$.
        
        The obvious way to deal with $C^k$-domains is to relax the constraint that the perturbation $\tilde{\rho}{\bf N}$ be carried by the normal.
        
        Choose vector fields $\theta$ in $\mathcal{D}^k(\mathbb{R}^N,\mathbb{R}^N)$ and consider the family of transformations \textbf{(3.24)}
        \begin{align*}
            T_t = I + t\theta,\ \Omega_t = T_t(\Omega),\ t\ge 0.
        \end{align*}
        This is a generalization of the family of transformations (3.22) in Sect. 3.3.1 from $\tilde{\rho}{\bf N}$ to $\theta$.
        
        For $k\ge 1$, $\theta$ is again uniformly Lipschitzian in $\mathbb{R}^N$ and, by Theorem 4.2 in Sect. 4.2, there exists $\tau > 0$ s.t. $T_t$ is bijective and bicontinuous from $\mathbb{R}^N$ onto itself.
        
        Thus for $0\le t\le\tau$,
        \begin{align*}
            \Omega_t = T_t(\Omega) = [I + t\theta](\Omega) \mbox{ and } \partial\Omega_t = T_t(\partial\Omega) = [I + t\theta](\partial\Omega).
        \end{align*}
        A more restrictive approach to get around the lack of sufficient smoothness of the normal ${\bf n}$ to $\Gamma$ would be to introduce a transverse field ${\bf p}$ on $\Gamma$ s.t. \textbf{(3.25)}
        \begin{align*}
            {\bf p}\in C^k\left(\Gamma;\mathbb{R}^N\right),\ \forall x\in\Gamma,\ {\bf p}(x)\cdot{\bf n}(x) > 0.
        \end{align*}
        Given ${\bf p}$ and $\rho\in C^k(\Gamma)$ define for $t\ge 0$ \textbf{(3.26)}
        \begin{align*}
            \Gamma_t := \left\{x\in\mathbb{R}^N;x = X + t\rho(X){\bf p}(X),\ \forall X\in\Gamma\right\}.
        \end{align*}
        Choosing $C^k$-extensions $\tilde{\rho}$ and $\tilde{\bf p}$ of $\rho$ and ${\bf p}$ we can go back to the case where \textbf{(3.27)}
        \begin{align*}
            \theta = \tilde{\rho}\tilde{\bf p}\in\mathcal{D}^k(\mathbb{R}^N,\mathbb{R}^N).
        \end{align*}
        For any $\theta\in\mathcal{D}^k(\mathbb{R}^N,\mathbb{R}^N)$  the semiderivative is defined as \textbf{(3.28)}
        \begin{align*}
            d_kJ(\Omega;\theta) := \lim_{t\downarrow 0} \frac{J\left((I + t\theta)(\Omega)\right) - J(\Omega)}{t}.
        \end{align*}
        \item \textbf{$C^k$-Domains.} When $\Omega$ is a domain of class $C^k$ with boundary $\Gamma$, the normal field ${\bf n}$ belongs to $C^{k-1}(\Gamma,\mathbb{R}^N)$.
        
        Therefore, choosing deformations along the normal would yield transformations $\{T_t\}$ mapping $C^k$-domains $\Omega$ onto $C^{k-1}$-domains $\Omega_t = T_t(\Omega)$.
        
        The obvious way to deal with $C^k$-domains is to relax the constraint that the perturbation $\tilde{\rho}{\bf N}$ be carried by the normal.
        
        Choose vector fields $\theta$ in $\mathcal{D}^k(\mathbb{R}^N,\mathbb{R}^N)$ and consider the family of transformations \textbf{(3.24)}
        \begin{align*}
            T_t = I + t\theta,\ \Omega_t = T_t(\Omega),\ t\ge 0.
        \end{align*}
        This is a generalization of the family of transformations (3.22) in Sect. 3.3.1 from $\tilde{\rho}{\bf N}$ to $\theta$.
        
        For $k\ge 1$, $\theta$ is again uniformly Lipschitzian in $\mathbb{R}^N$ and, by Theorem 4.2 in Sect. 4.2, there exists $\tau > 0$ s.t. $T_t$ is bijective and bicontinuous from $\mathbb{R}^N$ onto itself.
        
        Thus for $0\le t\le\tau$,
        \begin{align*}
            \Omega_t = T_t(\Omega) = [I + t\theta](\Omega) \mbox{ and } \partial\Omega_t = T_t(\partial\Omega) = [I + t\theta](\partial\Omega).
        \end{align*}
        A more restrictive approach to get around the lack of sufficient smoothness of the normal ${\bf n}$ to $\Gamma$ would be to introduce a transverse field ${\bf p}$ on $\Gamma$ s.t. \textbf{(3.25)}
        \begin{align*}
            {\bf p}\in C^k\left(\Gamma;\mathbb{R}^N\right),\ \forall x\in\Gamma,\ {\bf p}(x)\cdot{\bf n}(x) > 0.
        \end{align*}
        Given ${\bf p}$ and $\rho\in C^k(\Gamma)$ define for $t\ge 0$ \textbf{(3.26)}
        \begin{align*}
            \Gamma_t := \left\{x\in\mathbb{R}^N;x = X + t\rho(X){\bf p}(X),\ \forall X\in\Gamma\right\}.
        \end{align*}
        Choosing $C^k$-extensions $\tilde{\rho}$ and $\tilde{\bf p}$ of $\rho$ and ${\bf p}$ we can go back to the case where \textbf{(3.27)}
        \begin{align*}
            \theta = \tilde{\rho}\tilde{\bf p}\in\mathcal{D}^k(\mathbb{R}^N,\mathbb{R}^N).
        \end{align*}
        For any $\theta\in\mathcal{D}^k(\mathbb{R}^N,\mathbb{R}^N)$  the semiderivative is defined as \textbf{(3.28)}
        \begin{align*}
            d_kJ(\Omega;\theta) := \lim_{t\downarrow 0} \frac{J\left((I + t\theta)(\Omega)\right) - J(\Omega)}{t}.
        \end{align*}
        \item \textbf{Polar Coordinates \& Star-Shaped Domains.} In some examples domains are star-shaped w.r.t. a point.
        
        Since a domain can always be translated, there is no loss of generality in assuming that this point is the origin.
        
        Then such domains $\Omega$ can be parametrized as follows: \textbf{(3.35)}
        \begin{align*}
            \Omega := \left\{x\in\mathbb{R}^N;x = \rho\zeta,\ \zeta\in S_{N-1},\ 0\le\rho < f(\zeta)\right\},
        \end{align*}
        where $S_{N-1}$ is the unit sphere in $\mathbb{R}^N$, \textbf{(3.36)}
        \begin{align*}
            S_{N-1} := \left\{x\in\mathbb{R}^N;|x| = 1\right\},
        \end{align*}
        and $f:S_{N-1}\to\mathbb{R}_+$ is a positive continuous mapping from $S_{N-1}$ s.t. \textbf{(3.37)}
        \begin{align*}
            m := \min\left\{f(\zeta);\zeta\in S_{N-1}\right\} > 0.
        \end{align*}
        Given any $g\in C^0(S_{N-1})$ and a sufficiently small $t\ge 0$ the perturbed domains are defined as \textbf{(3.38)}
        \begin{align*}
            \Omega_t := \left\{x\in\mathbb{R}^N;x = \rho\zeta,\ \zeta\in S_{N-1},\ 0\le\rho < f(\zeta) + tg(\zeta)\right\}.
        \end{align*}
        E.g., choose $t$, $0\le t\le t_1$, for some \textbf{(3.39)}
        \begin{align*}
            t_1 = \frac{m}{\|g\|_{C^0\left(S_{N-1}\right)}} > 0
        \end{align*}
        and define the transformation $T_t$ as \textbf{(3.40)}
        \begin{equation*}
            \left\{\begin{split}
                T_t(X) &= 0, &&\mbox{ if } X = 0,\\
                T_t(X) &= \left[\rho + t\frac{\rho}{f(\zeta)}g(\zeta)\right]\zeta, &&\mbox{ if } X = \rho\zeta\ne 0.
            \end{split}\right.
        \end{equation*}
        As in the previous example, $T_t$ is not unique and for any continuous increasing function $\lambda:\mathbb{R}_+\to\mathbb{R}_+$ s.t. $\lambda(0) = 0$ and $\lambda(t) = 1$ the transformation \textbf{(3.41)}
        \begin{equation*}
            \left\{\begin{split}
                T_t(X) &= 0, &&\mbox{ if } X = 0,\\
                T_t(X) &= \left[\rho + t\lambda\left(\frac{\rho}{f(\zeta)}\right)g(\zeta)\right]\zeta, &&\mbox{ if } X = \rho\zeta\ne 0.
            \end{split}\right.
        \end{equation*}
        yields the same domain $\Omega_t$.
        \item \textbf{Level Sets.} In Sects. 3.3.1 to 3.3.4, the perturbed domain $\Omega_t$ always appears in the form $\Omega_t = T_t(\Omega)$, where $T_t$ is a bijective transformation of $\mathbb{R}^N$ and $T_t$ is of the form $I + t\theta$.
        
        In some free boundary problems (e.g., plasma physics, propagation of fronts) the free boundary $\Gamma$ is a level curve of a smooth function $u$ defined over an open domain $D$.
        
        Assume that $D$ is bounded open with smooth boundary $\partial D$.
        
        Let $u\in C^2(\overline{D})$ be a positive function on $\overline{D}$ s.t. \textbf{(3.42)}
        \begin{align*}
            u\ge 0 \mbox{ in } \overline{D},\ u = 0 \mbox{ on } \partial D,\ \exists \mbox{ a unique } x_u\in D \mbox{ s.t. } \forall x\in\overline{D} - \{x_u\},\ \left|\nabla u(x)\right| > 0.
        \end{align*}
        If $m = \max\{u(x);x\in\overline{D}\}$, then for each $t$ in $[0,m)$ the level set \textbf{(3.43)}
        \begin{align*}
            \Gamma_t = u^{-1}(t)
        \end{align*}
        is a $C^2$-submanifold of $\mathbb{R}^N$ in $D$, which is the boundary of the open set \textbf{(3.44)}
        \begin{align*}
            \Omega_t = \left\{x\in D;u(x) > t\right\}.
        \end{align*}
        By definition $\Omega_0 = D$, for all $t_1 > t_2$, $\Omega_{t_1}\subset\Omega_{t_2}$, and the domains $\Omega_t$ converge in the Hausdorff complementary topology\footnote{Cf. Sect. 2 of Chap. 6.} to the point $x_u$.
        
        The outward unit normal field on $\Gamma_t$ is given by \textbf{(3.45)}
        \begin{align*}
            x\in\Gamma_t,\ {\bf n}_t(x) = -\frac{\nabla u(x)}{\left|\nabla u(x)\right|}.
        \end{align*}
        This suggests introducing the velocity field \textbf{(3.46)}
        \begin{align*}
            \forall x\in D - \{x_u\},\ V(x) := \frac{\nabla u(x)}{\left|\nabla u(x)\right|^2},
        \end{align*}
        which is continuous everywhere but at $x = x_u$.
        
        If $V$ were continuous everywhere, then for each $X$, the path $x(t;X)$ generated by the differential equation \textbf{(3.47)}
        \begin{align*}
            \frac{dx}{dt}(t) = V(x(t)),\ x(0) = X
        \end{align*}
        would have the property that \textbf{(3.48)}
        \begin{align*}
            u(x(t)) = u(X) + t
        \end{align*}
        since formally \textbf{(3.49)}
        \begin{align*}
            \frac{d}{dt}u(x(t)) = \nabla u(x(t))\cdot\frac{dx}{dt}(t) = 1. 
        \end{align*}
        This means that the map $X\mapsto T_t(X) = x(t;X)$ constructed from (3.47) would map the level set \textbf{(3.50)}
        \begin{align*}
            \Gamma_0 = \left\{X\in\overline{D};u(X) = 0\right\}
        \end{align*}
        onto the level set \textbf{(3.51)}
        \begin{align*}
            \Gamma_t = \left\{x\in\overline{D};u(x) = t\right\}
        \end{align*}
        and eventually $\Omega_0$ onto $\Omega_t$.
        
        Unfortunately, it is easy to see that this last property fails on the function $u(x) = 1 - x^2$ defined on the unit disk.
        
        %
        To get around this difficulty, introduce for some arbitrarily small $\varepsilon$, $0 < \varepsilon < \frac{m}{2}$, an infinitely differentiable function $\rho_\varepsilon:\mathbb{R}^N\to[0,1]$ s.t. \textbf{(3.52)}
        \begin{equation*}
            \rho_\varepsilon(x) = \left\{\begin{split}
                &0, &&\mbox{ if } \left|\nabla u(x)\right| < \varepsilon,\\
                &1, &&\mbox{ if } \left|\nabla u(x)\right| > 2\varepsilon,
            \end{split}\right.
        \end{equation*}
        and the velocity \textbf{(3.53)}
        \begin{align*}
            V_\varepsilon(x) = \rho_\varepsilon(x)V(x),\ x\in\overline{D}.
        \end{align*}
        As above, define the transformation \textbf{(3.54)}
        \begin{align*}
            X\mapsto T_t^\varepsilon(X) = x(t;X),
        \end{align*}
        where$x(t;X)$ is the solution of the differential equation \textbf{(3.55)}
        \begin{align*}
            \frac{dx}{dt}(t) = V_\varepsilon(x(t)),\ t\ge 0,\ x(0) = X\in\overline{D}.
        \end{align*}
        For $0\le t < m - 2\varepsilon$, $T_t$ maps $\Gamma_0$ onto $\Gamma_t$; for $0\le s < m - \varepsilon$ s.t. $s + t < m - \varepsilon$, $T_t$ maps $\Gamma_s$ onto $\Gamma_{t+s}$.
        
        However, for $s > m - \varepsilon$, $T_t$ is the identity operator.
        
        As a result, for $0\le t < m - 2\varepsilon$, \textbf{(3.56)}
        \begin{align*}
            T_t\left(\Omega_0\right) = \Omega_t \mbox{ and } T_t\left(\Gamma_0\right) = \Gamma_t.
        \end{align*}
        Of course $\varepsilon > 0$ is arbitrary and we can make the construction for $t$'s arbitrary close to $m$.
        
        This is an example that can be handled by the velocity (speed) method and not by a perturbation of the identity.
        
        Here the domains $\Omega_t$ are implicitly constrained to stay within the larger domain $D$.
        
        We shall see in Sect. 5 how to introduce and characterize such a constraint.
        
        %
        Another example of description by level sets is provided by the \textit{oriented distance function} $b_\Omega$ for some open domain $\Omega$ of class $C^2$ with compact boundary $\Gamma$ (cf. Chap. 7).
        
        We shall see that there exists $h > 0$ and a neighborhood
        \begin{align*}
            U_h(\Gamma) = \left\{x\in\mathbb{R}^N;\left|b_\Omega(x)\right| < h\right\}
        \end{align*}
        s.t. $b_\Omega\in C^2(U_h(\Gamma))$.
        
        Then for $0\le t < h$ the flow corresponding to the velocity field $V = \nabla b_\Omega$ maps $\Omega$ its boundary $\Gamma$ onto
        \begin{align*}
            T_t(\Omega) &= \Omega_t = \left\{x\in\mathbb{R}^N;\left|b_\Omega(x)\right| < t\right\},\\
            T_t(\Gamma) &= \Gamma_t = \left\{x\in\mathbb{R}^N;\left|b_\Omega(x)\right| = t\right\}.
        \end{align*}
    \end{enumerate}
\end{enumerate}

\paragraph{Unconstrained Families of Domains.}
\begin{enumerate}
    \item Here study equivalences between the \textit{velocity method} (cf. J.-P. Zolésio [12, 8]) and methods using a family of transformations.
    
    In Sect. 4.1: give some general conditions to construct a family of transformations of $\mathbb{R}^N$ from a velocity field.
    
    Conversely, show how to construct a velocity field from a family of transformations of $\mathbb{R}^N$.
    
    In Sect. 4.2, this construction is applied to Lipschitzian perturbations of the identity.
    
    In Sect. 4.3, the various equivalences of Sect. 4.1 are specialized to velocities in $C_0^{k+1}(\mathbb{R}^N,\mathbb{R}^N)$, $C^{k+1}(\overline{\mathbb{R}^N},\mathbb{R}^N)$, and $C^{k,1}(\overline{\mathbb{R}^N},\mathbb{R}^N)$, $k\ge 0$.
    \item \textbf{Equivalence between Velocities \& Transformations.} Let the real number $\tau > 0$ and the map $V:[0,\tau]\times\mathbb{R}^N\to\mathbb{R}^N$ be given.
    
    If $t$ is interpreted as an artificial time, the map $V$ can be viewed as the (time-dependent) velocity field $\{V(t);0\le t\le\tau\}$ defined on $\mathbb{R}^N$: \textbf{(4.1)}
    \begin{align*}
        x\mapsto V(t)(x) := V(t,x):\mathbb{R}^N\to\mathbb{R}^N.
    \end{align*}
    Assume that there exists $\tau = \tau(V) > 0$ s.t. \textbf{(4.2)}
    \begin{align*}
        (V)\ \forall x\in\mathbb{R}^N,\ V(\cdot,x)\in C\left([0,\tau];\mathbb{R}^N\right),\ \exists c > 0,\ \forall x,y\in\mathbb{R}^N,\ \left\|V(\cdot,y) - V(\cdot,x)\right\|_{C\left([0,\tau];\mathbb{R}^N\right)}\le c\left|y - x\right|,
    \end{align*}
    where $V(\cdot,x)$ is the function $t\mapsto V(t,x)$.
    
    Note that $V$ is continuous on $[0,\tau]\times\mathbb{R}^N$.
    
    Hence it is uniformly continuous on $[0,\tau]\times D$ for any bounded open subset $D$ of $\mathbb{R}^N$ and \textbf{(4.3)}
    \begin{align*}
        V(\cdot)\in C\left([0,\tau];C\left(\overline{D};\mathbb{R}^N\right)\right)
    \end{align*}
    Associate with $V$ the solution $x(t;V)$ of the vector ordinary differential equation \textbf{(4.4)}
    \begin{align*}
        \frac{dx}{dt}(t) = V\left(t,x(t)\right),\ t\in[0,\tau],\ x(0) = X\in\mathbb{R}^N
    \end{align*}
    and define the transformations \textbf{(4.5)}
    \begin{align*}
        X\mapsto T_t(V)(X) := x_V(t;X):\mathbb{R}^N\to\mathbb{R}^N
    \end{align*}
    and the maps (whenever the inverse of $T_t$ exists) \textbf{(4.6)-(4.7)}
    \begin{align*}
        (t,X)&\mapsto T_V(t,X) := T_t(V)(X):[0,\tau]\times\mathbb{R}^N\to\mathbb{R}^N,\\
        (t,x)&\mapsto T_V^{-1}(t,x) := T_t^{-1}(V)(x):[0,\tau]\times\mathbb{R}^N\to\mathbb{R}^N.
    \end{align*}
    \textbf{Notation 4.1.} In what follows we shall drop the $V$ in $T_V(t,X)$, $T_V^{-1}(t,x)$, and $T_t(V)$ whenever no confusion arises.
    
    \begin{theorem}
        \begin{itemize}
            \item[(i)] Under assumptions (V) the map $T$ specified by (4.4)-(4.6) has the following properties: \textbf{(4.8)}
            \begin{align*}
                (T1)\ \forall X\in\mathbb{R}^N,\ &T(\cdot,X)\in C^1\left([0,\tau];\mathbb{R}^N\right) \mbox{ and } \exists c > 0,\\
                \forall X,Y\in\mathbb{R}^N,\ &\left\|T(\cdot,Y) - T(\cdot,X)\right\|_{C^1\left([0,\tau];\mathbb{R}^N\right)}\le c\left|Y - X\right|,
            \end{align*}
            \begin{align*}
                (T2)\ \forall t\in[0,\tau],\ X\mapsto T_t(X) = T(t,X):\mathbb{R}^N\to\mathbb{R}^N \mbox{ is bijective},
            \end{align*}
            \begin{align*}
                (T3)\ \forall x\in\mathbb{R}^N,\ &T^{-1}(\cdot,x)\in C\left([0,\tau];\mathbb{R}^N\right) \mbox{ and } \exists c > 0,\\
                \forall x,y\in\mathbb{R}^N,\ &\left\|T^{-1}(\cdot,y) - T^{-1}(\cdot,x)\right\|_{C\left([0,\tau];\mathbb{R}^N\right)}\le c\left|y - x\right|.
            \end{align*}
            \item[(ii)] Given a real number $\tau > 0$ and a map $T:[0,\tau]\times\mathbb{R}^N\to\mathbb{R}^N$ satisfying assumptions (T1) to (T3), the map \textbf{(4.9)}
            \begin{align*}
                (t,x)\mapsto V(t,x) := \frac{\partial T}{\partial t}\left(t,T_t^{-1}(x)\right):[0,\tau]\times\mathbb{R}^N\to\mathbb{R}^N
            \end{align*}
            satisfies conditions (V), where $T_t^{-1}$ is the inverse of $X\mapsto T_t(X) = T(t,X)$. If, in addition, $T(0,\cdot) = I$, then $T(\cdot,X)$ is the solution of (4.4) for that $V$.
            \item[(iii)] Given a real number $\tau > 0$ and a map $T:[0,\tau]\times\mathbb{R}^N\to\mathbb{R}^N$ satisfying assumptions (T1) and (T2) and $T(0,\cdot) = I$, then there exists $\tau' > 0$ s.t. the conclusion of part (ii) hold on $[0,\tau']$.
        \end{itemize}
    \end{theorem}
    A more general version of this theorem for constrained domains (Theorem 5.1) will be given and proved in Sect. 5.1.
    
    This equivalence theorem says that we can start either from a family of velocity fields  $\{V(t)\}$ on $\mathbb{R}^N$ or a family of transformations $\{T_t\}$ of $\mathbb{R}^N$ provided that the map $V$, $V(t,x) = V(t)(x)$, satisfies (V) or the map $T$, $T(t,X) = T_t(X)$, satisfies (T1) to (T3).
    
    %
    Starting from $V$, the family of homeomorphisms $\{T_t(V)\}$ generates the family \textbf{(4.13)}
    \begin{align*}
        \Omega_t := T_t(V)(\Omega) = \left\{T_t(V)(X);X\in\Omega\right\}
    \end{align*}
    of perturbations of the initial domain $\Omega$.
    
    Interior (resp., boundary) points of $\Omega$ are mapped onto interior (resp., boundary) points of $\Omega_t$.
    
    This is the basis of the \textit{velocity method} which will be used to define shape derivatives.
    \item \textbf{Perturbations of the Identity.} In examples it is usually possible to show that the transformation $T$ satisfies assumptions (T1) to (T3) and construct the corresponding velocity field $V$ defined in (4.9).
    
    E.g., consider perturbations of the identity to the 1st ($A = 0$) or 2nd order: for $t\ge 0$ and $X\in\mathbb{R}^N$, \textbf{(4.14)}
    \begin{align*}
        T_t(X) := X + tU(X) + \frac{t^2}{2}A(X),
    \end{align*}
    where $U$ and $A$ are transformations of $\mathbb{R}^N$.
    
    It turns out that for Lipschitzian transformations $U$ and $A$, assumptions (T1) to (T3) are satisfied in some interval $[0,\tau]$.
    
    \begin{theorem}
        Let $U$ and $A$ be 2 uniform Lipschitzian transformations of $\mathbb{R}^N$: there exists $c > 0$ s.t. for all $X,Y\in\mathbb{R}^N$,
        \begin{align*}
            \left|U(Y) - U(X)\right|\le c\left|Y - X\right| \mbox{ and } \left|A(Y) - A(X)\right|\le c\left|Y - X\right|.
        \end{align*}
        There exists $\tau > 0$ s.t. the map $T$ given by (4.14) satisfies conditions (T1) to (T3) on $[0,\tau]$. The associated velocity $V$ given by \textbf{(4.15)}
        \begin{align*}
            (t,x)\mapsto V(t,x) = U\left(T_t^{-1}(x)\right) + tA\left(T_t^{-1}(x)\right):[0,\tau]\times\mathbb{R}^N\to\mathbb{R}^N
        \end{align*}
        satisfies conditions (V) on $[0,\tau]$.
    \end{theorem}

    \begin{remark}
        Observe that from (4.14) and (4.15) \textbf{(4.16)}
        \begin{align*}
            V(0) = U,\ \dot{V}(0)(x) = \frac{\partial V}{\partial t}(t,x)|_{t = 0} = A - [DU]U,
        \end{align*}
        where $DU$ is the Jacobian matrix of $U$.
        
        The term $\dot{V}(0)$ is an \emph{acceleration} at $t = 0$ which will always be present even when $A = 0$, but it can be eliminated by choosing $A = [DU]U$.
    \end{remark}
    \item \textbf{Equivalence for Special Families of Velocities.} In this section we specialize Theorem 4.1 to velocities in $C^{k,1}(\overline{\mathbb{R}^N},\mathbb{R}^N)$, $C_0^{k+1}(\mathbb{R}^N,\mathbb{R}^N)$, and $C^{k+1}(\overline{\mathbb{R}^N},\mathbb{R}^N)$, $k\ge 0$.
    
    The following notation will be convenient:
    \begin{align*}
        f(t) := T_t - I,\ f'(t) = \frac{dT_t}{dt},\ g(t) := T_t^{-1} - I,
    \end{align*}
    whenever $T_t^{-1}$ exists and the identities
    \begin{align*}
        g(t) &= -f(t)\circ T_t^{-1} = -f(t)\circ[I + g(t)],\\
        V(t) &= \frac{dT_t}{dt}\circ T_t^{-1} = f'(t)\circ T_t^{-1} = f'(t)\circ[I + g(t)].
    \end{align*}
    Recall also for a function $F:\mathbb{R}^N\to\mathbb{R}^N$ the notation
    \begin{align*}
        c(F) := \sup_{y\ne x} \frac{\left|F(y) - F(x)\right|}{|y - x|} \mbox{ and } \forall k\ge 1,\ c_k(F) := \sum_{|\alpha| = k} c\left(\partial^\alpha F\right).
    \end{align*}
    
    \begin{theorem}
        Let $k\ge 0$ be an integer.
        \begin{itemize}
            \item[(i)] Given $\tau > 0$ and a velocity field $V$ s.t. \textbf{(4.19)}
            \begin{align*}
                V\in C\left([0,\tau];C^k(\overline{\mathbb{R}^N},\mathbb{R}^N)\right) \mbox{ and } c_k\left(V(t)\right)\le c
            \end{align*}
            for some constant $c > 0$ independent of $t$, the map $T$ given by (4.4)-(4.6) satisfies conditions (T1), (T2), and \textbf{(4.20)}
            \begin{align*}
                f\in C^1\left([0,\tau];C^k(\overline{\mathbb{R}^N},\mathbb{R}^N)\right)\cap C\left([0,\tau];C^{k,1}(\overline{\mathbb{R}^N},\mathbb{R}^N)\right),\ c_k\left(f'(t)\right)\le c,
            \end{align*}
            for some constant $c > 0$ independent of $t$. Moreover, conditions (T3) are satisfied and there exists $\tau > 0$ s.t. \textbf{(4.21)}
            \begin{align*}
                g\in C\left([0,\tau'];C^k(\overline{\mathbb{R}^N},\mathbb{R}^N)\right), \ c_k\left(g(t)\right)\le ct,
            \end{align*}
            for some constant $c$ independent of $t$.
            \item[(ii)] Given $\tau > 0$ and $T:[0,\tau]\times\mathbb{R}^N\to\mathbb{R}^N$ satisfying conditions (4.20) and $T(0,\cdot) = I$, there exists $\tau' > 0$ s.t. the velocity field $V(t) = f'(t)\circ T_t^{-1}$ satisfies conditions (V) and (4.19) in $[0,\tau']$.
        \end{itemize}
    \end{theorem}
    We now turn to the case of velocities in $C_0^k(\mathbb{R}^N,\mathbb{R}^N)$.
    
    As in Chap. 2, it will be convenient to use the notation $\mathcal{C}_0^k$ for the space $C_0^k(\mathbb{R}^N,\mathbb{R}^N)$, $\mathcal{C}^k(\overline{\mathbb{R}^N})$ for the space $C^k(\overline{\mathbb{R}^N},\mathbb{R}^N)$, and $\mathcal{C}^{k,1}(\overline{\mathbb{R}^N})$ for the space $C^{k,1}(\overline{\mathbb{R}^N},\mathbb{R}^N)$.
    
    \begin{theorem}
        Let $k\ge 1$ be an integer.
        \begin{itemize}
            \item[(i)] Given $\tau > 0$ and a velocity field $V$ s.t. \textbf{(4.23)}
            \begin{align*}
                V\in C([0,\tau];C_0^k(\mathbb{R}^N,\mathbb{R}^N)),
            \end{align*}
            the map $T$ given by (4.4)–(4.6) satisfies conditions (T1), (T2), and \textbf{(4.24)}
            \begin{align*}
                f\in C^1([0,\tau];C_0^k(\mathbb{R}^N,\mathbb{R}^N)).
            \end{align*}
            Moreover, conditions (T3) are satisfied and there exists $\tau' > 0$ s.t. \textbf{(4.25)}
            \begin{align*}
                g\in C([0,\tau'];C_0^k(\mathbb{R}^N,\mathbb{R}^N)).
            \end{align*}
            \item[(ii)] Given $\tau > 0$ and $T:[0,\tau]\times\mathbb{R}^N\to\mathbb{R}^N$ satisfying conditions (4.24) and $T(0,\cdot) = I$, there exists $\tau' > 0$ s.t. the velocity field $V(t) = f'(t)\circ T_t^{-1}$ satisfies conditions (V) and (4.23) on $[0,\tau']$.
        \end{itemize}
    \end{theorem}
    The proof of the last theorem is based on the fact that the vector functions involved are uniformly continuous.
    
    The fact that they vanish at infinity is not an essential element of the proof.
    
    Therefore, the theorem is valid with $C^k(\overline{\mathbb{R}^N},\mathbb{R}^N)$ in place of $C_0^k(\mathbb{R}^N,\mathbb{R}^N)$.
    
    \begin{theorem}
        Let $k\ge 1$ be an integer.
        \begin{itemize}
            \item[(i)] Given $\tau > 0$ and a velocity field $V$ s.t. \textbf{(4.28)}
            \begin{align*}
                V\in C([0,\tau];C^k(\overline{\mathbb{R}^N},\mathbb{R}^N)),
            \end{align*}
            the map $T$ given by (4.4)-(4.6) satisfies conditions (T1), (T2), and \textbf{(4.29)}
            \begin{align*}
                f\in C^1([0,\tau];C^k(\overline{\mathbb{R}^N},\mathbb{R}^N)).
            \end{align*}
            Moreover, conditions (T3) are satisfied and there exists $\tau' > 0$ s.t. \textbf{(4.30)}
            \begin{align*}
                g\in C([0,\tau'];C^k(\overline{\mathbb{R}^N},\mathbb{R}^N)).
            \end{align*}
            \item[(ii)] Given $\tau > 0$ and $T:[0,\tau]\times\mathbb{R}^N\to\mathbb{R}^N$ satisfying conditions (4.29) and $T(0,\cdot) = I$, there exists $\tau' > 0$ s.t. the velocity field $V(t) = f'(t)\circ T_t^{-1}$ satisfies conditions (V) and (4.28) on $[0,\tau']$.
        \end{itemize}
    \end{theorem}
\end{enumerate}

\paragraph{Constrained Families of Domains.}
\begin{enumerate}
    \item Turn to the case where the family of admissible domains $\Omega$ is constrained to lie in a fixed larger subset $D$ of $\mathbb{R}^N$ or its closure.
    
    E.g., $D$ can be an open set or a closed submanifold of $\mathbb{R}^N$.
    \item \textbf{Equivalence between Velocities \& Transformations.} Given a nonempty subset $D$ of $\mathbb{R}^N$, consider a family of transformations \textbf{(5.1)}
    \begin{align*}
        T:[0,\tau]\times\overline{D}\to\mathbb{R}^N
    \end{align*}
    for some $\tau = \tau(T) > 0$ with the following properties: \textbf{(5.2)}
    \begin{itemize}
        \item $({\rm T1}_D)$ $\forall X\in\overline{D}$, $T(\cdot,X)\in C^1([0,\tau];\mathbb{R}^N)$ and $\exists c > 0$, $\forall X,Y\in\overline{D}$, $\|T(\cdot,Y) - T(\cdot,X)\|_{C^1\left([0,\tau];\mathbb{R}^N\right)}\le c|Y - X|$,
        \item $({\rm T2}_D)$ $\forall t\in[0,\tau]$, $X\mapsto T_t(X) = T(t,X):\overline{D}\to\overline{D}$ is bijective,
        \item $({\rm T3}_D)$ $\forall x\in\overline{D}$, $T^{-1}(\cdot,x)\in C([0,\tau];\mathbb{R}^N)$ and $\exists c > 0$, $\forall x,y\in\overline{D}$, $\|T^{-1}(\cdot,y) - T^{-1}(\cdot,x)\|_{C\left([0,\tau];\mathbb{R}^N\right)}\le c|y - x|$,
    \end{itemize}
    where under assumption $({\rm T2}_D)$ $T^{-1}$ is defined from the inverse of $T_t$ as \textbf{(5.3)}
    \begin{align*}
        (t,x)\mapsto T^{-1}(t,x) := T_t^{-1}(x):[0,\tau]\times\overline{D}\to\mathbb{R}^N.
    \end{align*}
    These 3 properties are the analogue for $\overline{D}$ of the same 3 properties obtained for $\mathbb{R}^N$.
    
    In fact, Theorem 4.1 extends from $\mathbb{R}^N$ to $\overline{D}$ by adding 1 assumption to (V).
    
    Specifically, consider a velocity field \textbf{(5.4)}
    \begin{align*}
        V:[0,\tau]\times\overline{D}\to\mathbb{R}^N
    \end{align*}
    for which there exists $\tau = \tau(V) > 0$ s.t. \textbf{(5.5)}
    \begin{itemize}
        \item $({\rm V1}_D)$ $\forall x\in\overline{D}$, $V(\cdot,x)\in C([0,\tau];\mathbb{R}^N)$, $\exists c > 0$, $\forall x,y\in\overline{D}$, $\|V(\cdot,y) - V(\cdot,x)\|_{C([0,\tau];\mathbb{R}^N)}\le c|y - x|$,
        \item $({\rm V2}_D)$ $\forall x\in\overline{D}$, $\forall t\in[0,\tau]$, $\pm V(t,x)\in T_{\overline{D}}(x)$,
    \end{itemize}
    where $T_{\overline{D}}(x)$ is Bouligand's contingent cone\footnote{This is an equivalent characterization of Bouligand's contingent cone of Definition 2.4 in Sect. 2.4 of Chap. 2.} to $\overline{D}$ at the point $x$ in $\overline{D}$ \textbf{(5.6)}
    \begin{align*}
        T_{\overline{D}}(X) := \bigcap_{\varepsilon > 0} \bigcap_{\alpha > 0} \bigcup_{0 < h < \alpha} \left[\frac{1}{h}(\overline{D} - X) + \varepsilon B\right]
    \end{align*}
    and $B$ is the unit disk in $\mathbb{R}^N$ (cf. J.-P. Aubin and A. Cellina [1, p. 176]).
    
    This definition is equivalent to \textbf{(5.7)}
    \begin{align*}
        T_{\overline{D}}(X) = \limsup_{t\downarrow 0} \left\{\frac{\overline{D} - X}{t}\right\} = \left\{v;\liminf_{t\downarrow 0} \frac{1}{t}d_D(X + tv) = 0\right\}
    \end{align*}
    (cf. J.-P. Aubin and H. Frankowska [1, pp. 121–122, 17, 21]).
    
    Note that when $D$ is bounded in $\mathbb{R}^N$,
    \begin{align*}
        V(\cdot)\in C([0,\tau];C(\overline{D};\mathbb{R}^N)\cap\operatorname{Lip}(\overline{D};\mathbb{R}^N)) = C([0,\tau];C^{0,1}(\overline{D};\mathbb{R}^N)).
    \end{align*}
    When $D$ is equal to $\mathbb{R}^N$, $T_{\overline{D}}(x) = \mathbb{R}^N$ for all $x$ and condition $({\rm V2}_D)$ can be dropped.
    
    When $D$ is equal to the boundary $\partial A$ of a set $A$ of class $C^{1,1}$ in $\mathbb{R}^N$, $\partial A$ is a $C^{1,1}$-submanifold of $\mathbb{R}^N$ and
    \begin{align*}
        \forall x\in\partial A,\ \pm V(t,x)\in T_{\partial A}(x)\Leftrightarrow\forall x\in\partial 
        A,\ V(t,x)\cdot\nabla b_A(x) = 0;
    \end{align*}
    i.e., at each point of $\partial A$, the velocity field is tangent to $\partial A$: it belongs to the tangent linear space of $\partial A$.
    
    %
    The next theorem is a generalization of Theorem 4.1 from $\mathbb{R}^N$ to an arbitrary set $D$ which shows the equivalence between velocity and transformation viewpoints.
    
    \begin{theorem}
        \begin{itemize}
            \item[(i)] Let $\tau > 0$ and let $V$ be a family of velocity fields satisfying conditions $({\rm V1}_D)$ and $({\rm V2}_D)$ an consider the family of transformations \textbf{(5.8)}
            \begin{align*}
                (t,X)\mapsto T(t,X) = x(t;X):[0,\tau]\times\overline{D}\to\mathbb{R}^N,
            \end{align*}
            where $x(\cdot,X)$ is the solution of \textbf{(5.9)}
            \begin{align*}
                \frac{dx}{dt}(t) = V\left(t,x(t)\right),\ 0\le t\le\tau,\ x(0) = X.
            \end{align*}
            Then the family of transformations $T$ satisfies conditions $({\rm T1}_D)$ to $({\rm T3}_D)$.
            \item[(ii)] Conversely, given a family of transformations $T$ satisfying conditions $({\rm T1}_D)$ to $({\rm T3}_D)$, the family of velocity fields \textbf{(5.10)}
            \begin{align*}
                (t,x)\mapsto V(t,x) = \frac{\partial T}{\partial t}\left(t,T_t^{-1}(x)\right):[0,\tau]\times\overline{D}\to\mathbb{R}^N
            \end{align*}
            satisfies conditions $({\rm V1}_D)$ and $({\rm V2}_D)$. If, in addition, $T(0,\cdot) = I$, then $T(\cdot,X)$ is the solution of (5.9) for that $V$.
            \item[(iii)] Given a real number $\tau > 0$ and a map $T:[0,\tau]\times\overline{D}\to\overline{D}$ satisfying assumptions $({\rm T1}_D)$ and $({\rm T2}_D)$ and $T(0,\cdot) = I$, then there exists $\tau' > 0$ s.t. the conclusions of part (ii) hold on $[0,\tau']$.
        \end{itemize}
    \end{theorem}

    \begin{remark}
        Condition $({\rm V2}_D)$ is a generalization to an arbitrary set $D$ of the following condition used by J.-P. Zolésio [12] in 1979: for all $x$ in $\partial D$,
        \begin{equation*}
            \left\{\begin{split}
                &V(t,x)\cdot{\bf n}(x) = 0, &&\mbox{ if the outward normal } {\bf n}(x) \mbox{ exists},\\
                &0, &&\mbox{ otherwise}.
            \end{split}\right.
        \end{equation*}
    \end{remark}
    \item \textbf{Transformation of Condition $({\rm V2}_D)$ into a Linear Constraint.} Condition $({\rm V2}_D)$ is equivalent to \textbf{(5.24)}
    \begin{align*}
        \forall t\in[0,\tau],\ \forall x\in\overline{D},\ V(t,x)\in\left\{-T_D(x)\right\}\cap T_D(x)
    \end{align*}
    since $T_{\overline{D}}(x) = T_D(x)$.
    
    If $T_D(x)$ were convex, then the above intersection would be a closed linear subspace of $\mathbb{R}^N$.
    
    This is true when $D$ is convex.
    
    In that case $T_D(x) = C_D(x)$, where $C_D(x)$ is the Clarke tangent cone and \textbf{(5.25)}
    \begin{align*}
        L_D(x) = \{-C_D(x)\}\cap C_D(x)
    \end{align*}
    is a closed linear subspace of $\mathbb{R}^N$.
    
    This means that $({\rm V2}_D)$ reduces to \textbf{(5.26)}
    \begin{align*}
        \forall t\in[0,\tau],\ \forall x\in\overline{D},\ V(t,x)\in L_D(x).
    \end{align*}
    It turns out that for continuous vector fields $V(t,\cdot)$, the equivalence of $({\rm V2}_D)$ and (5.26) extends to arbitrary domains $D$.
    
    This equivalence generally fails for discontinuous vector fields.
    
    Other equivalences might be possible between $T_D$ and some intermediary convex cone between $C_D$ and $T_D$, but there is no evidence so far of that fact.
    
    For smooth bounded open domains $\Omega$, the 2 cones coincide and the condition reduces to $V\cdot{\bf n} = 0$, ${\bf n}$ the normal to $\partial\Omega$, and $V(t,x)$ belongs to the tangent space to $\partial\Omega$ in each point of $\partial\Omega$.
    
    \begin{theorem}
        \begin{itemize}
            \item[(i)] Given a velocity field $V$ satisfying $({\rm V1}_D)$, condition $({\rm V2}_D)$ is equivalent to
            \begin{align*}
                ({\rm V2}_C)\ \forall t\in[0,\tau],\ \forall x\in\overline{D},\ V(t,x)\in L_D(x) = \{-C_D(x)\}\cap C_D(x),
            \end{align*}
            where $C_D(x)$ is the (closed convex ) Clarke tangent cone to $\overline{D}$ at $x$ \textbf{(5.27)}
            \begin{align*}
                C_D(x) = \left\{v\in\mathbb{R}^N;\lim_{t\downarrow 0,y\to_{\overline{D}} x} \frac{d_D(y + tv)}{t} = 0\right\},
            \end{align*}
            and $\to_{\overline{D}}$ denotes the convergence in $\overline{D}$.
            \item[(ii)] $L_D(x)$ is a closed linear subspace of $\mathbb{R}^N$.
        \end{itemize}
    \end{theorem}
    
    \begin{proof}        
        The equivalence of $({\rm V2}_D)$ and $({\rm V2}_C)$ is a direct consequence of the following lemma.
        
        \begin{lemma}
            Given a vector field $W\in C(\overline{D};\mathbb{R}^N)$ the following 2 conditions are equivalent: \textbf{(5.28)-(5.29)}
            \begin{align*}
                \forall x\in\overline{D},\ W(x)&\in T_D(x),\\
                \forall x\in\overline{D},\ W(x)&\in C_D(x)
            \end{align*}
        \end{lemma}
    \end{proof}
    
    \begin{remark}
        Lemma 5.1 essentially says that for continuous vector fields we can relax the condition of M. Nagumo [1]'s theorem from $({\rm V2}_D)$ involving the Bouligand contingent cone to $({\rm V2}_C)$ involving the smaller Clarke convex tangent cone.
        
        In dimension $N = 3$, $L_D(x)$ is $\{0\}$, a line, a plane, or the whole space.
    \end{remark}
    
    \textbf{Notation 5.1.} In what follows, it will be convenient to introduce the spaces and subspaces \textbf{(5.30)}
    \begin{align*}
        \mathcal{L} = \left\{V:[0,\tau]\times\mathbb{R}^N\to\mathbb{R}^N; V \mbox{ satisfies (V) on } \mathbb{R}^N\right\},
    \end{align*}
    and for an arbitrary domain $D$ in $\mathbb{R}^N$ \textbf{(5.31)}
    \begin{align*}
        \mathcal{L}_D = \left\{V:[0,\tau]\times\overline{D}\to\mathbb{R}^N;V \mbox{ satisfies } ({\rm V1}_D) \mbox{ and } ({\rm V2}_C) \mbox{ on } \overline{D}\right\}.
    \end{align*}
    For any integers $k\ge 0$ and $m\ge 0$ and any compact subset $K$ of $\mathbb{R}^N$ define the following subspaces of $\mathcal{L}$: \textbf{(5.32)}
    \begin{align*}
        \mathcal{V}_K^{m,k} = C^m\left([0,\tau],\mathcal{D}^k(K,\mathbb{R}^N)\right)\cap\mathcal{L},
    \end{align*}
    where $\mathcal{D}^k(K,\mathbb{R}^N)$ is the space of $k$-times continuously differentiable transformations of $\mathbb{R}^N$ with compact support in $K$.
    
    In all cases $\mathcal{V}_K^{m,k}\subset\mathcal{L}_K$.
    
    As usual $\mathcal{D}^\infty(K,\mathbb{R}^N)$ will be written $\mathcal{D}(K,\mathbb{R}^N)$.
\end{enumerate}

\paragraph{Continuity of Shape Functions along Velocity Flows.}
\begin{enumerate}
    \item Throughout this section, assume that $\Omega$ is a closed subset or an open crack-free\footnote{Cf. Definition 7.1 (ii) of Chap. 8.} subset of $\mathbb{R}^N$ and that $B$ is a Banach space.
    
    %
    Since the construction of the Courant metric is relatively abstract, can the continuity of a shape functional be characterized in a more direct way?
    
    The answer is yes thanks to the sharp theorems giving the equivalence between transformations and velocities in the previous section without restricting the analysis to the subgroup $G_\Theta$ defined by (2.7) in Sect. 2.
    
    This is due to the fact that continuity and semiderivatives are local properties and that the velocity character naturally pops out as a sequence of transformations goes to identity.
    \item In this section, we give a characterization via velocities of the continuity of a shape functional of the form \textbf{(6.1)}
    \begin{align*}
        \Omega'\mapsto J(\Omega'):\mathcal{X}(\Omega)\to B,\ \mathcal{X}(\Omega) = \left\{F(\Omega);\forall F\in\mathcal{F}(\Theta)\right\}
    \end{align*}
    w.r.t. the Courant metric topology of the quotient $\mathcal{F}(\Theta)/\mathcal{G}(\Omega)$ using the equivalence Theorems 4.3-4.5.
    
    Checking the continuity along flows of a velocity is easier and more natural.
    
    As in Definition 3.1, assume that $J$ verifies the
    compatibility condition of Definition 3.1:
    \begin{align*}
        \forall H,F\in\mathcal{F}(\Theta) \mbox{ s.t. } H\left(F(\Omega)\right) = F(\Omega),\ J(H(F(\Omega))) = J(F(\Omega)).
    \end{align*}
    We specifically consider the continuity of shape functionals w.r.t. the Courant metric associated with $\Theta$ equal to $C_0^{k+1}(\mathbb{R}^N,\mathbb{R}^N)$, $C^{k+1}(\overline{\mathbb{R}^N},\mathbb{R}^N)$, and $C^{k,1}(\overline{\mathbb{R}^N},\mathbb{R}^N)$, $k\ge 0$, but similar equivalences are true for the other spaces of Chap. 3.
    
    \begin{remark}
        An obvious consequence of Theorems 6.1-6.3 is that when a shape functional is semidifferentiable in the sense of Definition 3.2 (Hadamard), it is continuous at that point for the Courant metric topology in complete analogy with the classical Euclidean calculus.
    \end{remark}
    \item We begin with the space $\mathcal{C}_0^k(\mathbb{R}^N) = C_0^k(\mathbb{R}^N,\mathbb{R}^N)$ of A. M. Micheletti [1].
    
    \begin{theorem}
        Let $k\ge 1$ be an integer, $B$ be a Banach space, and $\Omega$ be a nonempty open subset of $\mathbb{R}^N$. Consider a shape functional $J:N_\Omega([I])\to B$ defined in a neighborhood $N_\Omega([I])$ of $[I]$ in $\mathcal{F}(C_0^k(\mathbb{R}^N,\mathbb{R}^N))/\mathcal{G}(\Omega)$. Then $J$ is continuous at $\Omega$ for the Courant metric iff \textbf{(6.2)}
        \begin{align*}
            \lim_{t\downarrow 0} J\left(T_t(\Omega)\right) = J(\Omega)
        \end{align*}
        for all families of velocity fields $\{V(t);0\le t\le\tau\}$ satisfying the condition \textbf{(6.3)}
        \begin{align*}
            V\in C([0,\tau];C_0^k(\mathbb{R}^N,\mathbb{R}^N)).
        \end{align*}
    \end{theorem}
    \item The case of the Courant metric associated with the space $\mathcal{C}^k(\overline{\mathbb{R}^N}) = C^k(\overline{\mathbb{R}^N},\mathbb{R}^N)$ is a corollary to Theorem 6.1.
    
    \begin{theorem}
        Let $k\ge 1$ be an integer, $B$ be a Banach space, and $\Omega$ be a nonempty open subset of $\mathbb{R}^N$. Consider a shape functional $J:N_\Omega([I])\to B$ defined in a neighborhood $N_\Omega([I])$ of $[I]$ in $\mathcal{F}(C^k(\overline{\mathbb{R}^N},\mathbb{R}^N))/\mathcal{G}(\Omega)$.
        
        Then $J$ is continuous at $\Omega$ for the Courant metric iff \textbf{(6.4)}
        \begin{align*}
            \lim_{t\downarrow 0} J\left(T_t(\Omega)\right) = J(\Omega)
        \end{align*}
        for all families of velocity fields $\{V(t);0\le t\le\tau\}$ satisfying the condition \textbf{(6.5)}
        \begin{align*}
            V\in C([0,\tau];C^k(\overline{\mathbb{R}^N},\mathbb{R}^N)).
        \end{align*}
    \end{theorem}
    \item The proof of the theorem for the Courant metric topology associated with the space $\mathcal{C}^{k,1}(\mathbb{R}^N) = C^{k,1}(\mathbb{R}^N,\mathbb{R}^N)$ is similar to the proof of Theorem 6.1 with obvious changes.
    
    \begin{theorem}
        Let $k\ge 0$ be an integer, $\Omega$ be a nonempty open subset of $\mathbb{R}^N$, and $B$ be a Banach space. Consider a shape functional $J:N_\Omega([I])\to B$ defined in a neighborhood $N_\Omega([I])$ of $[I]$ in $\mathcal{F}(C^{k,1}(\overline{\mathbb{R}^N},\mathbb{R}^N))/\mathcal{G}(\Omega)$. Then $J$ is continuous at $\Omega$ for the Courant metric iff \textbf{(6.6)}
        \begin{align*}
            \lim_{t\downarrow 0} J\left(T_t(\Omega)\right) = J(\Omega)
        \end{align*}
        for all families $\{V(t);0\le t\le\tau\}$ of velocity fields in $C^{k,1}(\overline{\mathbb{R}^N},\mathbb{R}^N)$ satisfying the conditions \textbf{(6.7)}
        \begin{align*}
            V\in C([0,\tau];C^k(\overline{\mathbb{R}^N},\mathbb{R}^N)) \mbox{ and } c_k\left(V(t)\right)\le c
        \end{align*}
        for some constant $c$ independent of $t$.
    \end{theorem}
    
    \begin{remark}
        The conclusions of Theorems 6.1--6.3 are generic.
        
        They also have their counterpart in the constrained case.
        
        The difficulty lies in the 2nd part of the theorem, which requires a special construction to make sure that the family of transformations $\{T_t;0\le t\le\tau\}$ constructed from the sequence $\{T_n\}$ are homeomorphisms of $\overline{D}$. 
    \end{remark}
\end{enumerate}

\subsubsection{Metrics via Characteristic Functions}
\begin{enumerate}
    \item The constructions of the metric topologies of Chap. 3 are limited to families of sets which are the image of a fixed set by a family of homeomorphisms or diffeomorphisms.
    
    If it is connected, bounded, or $C^k$, $k\ge 1$, then the images will have the same properties under $C^k$-transformations of $\mathbb{R}^N$.
    
    In this chapter we considerably enlarge the family of available sets by relaxing the smoothness assumption to the mere Lebesgue measurability and even just measurability to include \textit{Hausdorff measures}.
    
    This is done by identifying $\Omega\subset\mathbb{R}^N$ with its \textit{characteristic function} \textbf{(1.1)}
    \begin{equation*}
        \chi_\Omega(x) := \left\{\begin{split}
            &1, &&\mbox{ if } x\in\Omega,\\
            &0, &&\mbox{ if } x\notin\Omega,
        \end{split}\right. \mbox{ if } \Omega\ne\emptyset \mbox{ and } \chi_\emptyset := 0.
    \end{equation*}
    1st introduce an Abelian group structure\footnote{Cf. the group structure on subsets of a holdall $D$ in Sect. 2.2 of Chap. 2.} on characteristic functions of \textit{measurable}\footnote{Measurable in the sense of L. C. Evans and R. F. Gariepy [1].
        
        It simultaneously covers Hausdorff and Lebesgue measures.} subsets of a fixed holdall $D$ in Sect. 2 and then construct complete metric spaces of equivalence classes of measurable characteristic functions via the $L^p$-norms, $1\le p < \infty$, that turn out to be algebraically and topologically equivalent.
    \item Starting with Sect. 3, we specialize to the Lebesgue measure in $\mathbb{R}^N$.
    
    The complete topology generated by the metric defined as the $L^p$-norm of the difference of 2 characteristic functions is called the \textit{strong topology} in Sect. 3.1.
    
    In Sect. 3.2 we consider the \textit{weak $L^p$-topology} on the family of characteristic functions.
    
    Weak limits of sequences of characteristic functions are functions with values in $[0,1]$ which belong to the closed convex hull of the equivalence classes of measurable characteristic functions.
    
    This occurs in optimization problems where the function to be optimized depends on the solution of a PDE on the variable domain, as was shown, e.g., by F. Murat [1] in 1971.
    
    It usually corresponds to the appearance of a \textit{microstructure} or a \textit{composite material} in mechanics.
    
    1 important example in that category was the analysis of the optimal thickness of a circular plate by K.-T. Cheng and N. Olhoff [2, 1] in 1981.
    
    The reader is referred to the work of F. Murat and L. Tartar [3, 1], initially published in 1985, for a comprehensive treatment of the calculus of variations and homogenization.
    
    Sect. 3.3 deals with the question of finding a \textit{nice representative} in the equivalence class of sets.
    
    Can it be chosen open?
    
    We introduce the \textit{measure theoretic representative} and characterize its interior, exterior, and boundary.
    
    It will be used later in Sect. 6.4 to prove the compactness of the family of Lipschitzian domains verifying a uniform cone property.
    
    Sect. 3.4 shows that the family of convex subsets of a fixed bounded holdall is closed in the strong topology.
    \item The use of the $L^p$-topologies is illustrated in Sect. 4 by revisiting the optimal design problem studied by J. Céa and K. Malanowski [1] in 1970.
    
    By relaxing the family of characteristic functions to functions with values in the interval $[0,1]$ a saddle point formulation is obtained.
    
    Functions with values in the interval $[0,1]$ can also be found in the modeling of a dam by H. W. Alt [1] and H. W. Alt and G. Gilardi [1], where the ``liquid saturation'' in the soil is a function with values in $[0,1]$.
    
    Another problem amenable to that formulation is the buckling of columns as will be illustrated in Sect. 5 using the work of S. J. Cox and M. L. Overton [1].
    
    It is 1 of the very early optimal design problems, formulated by Lagrange in 1770.
    \item The \textit{Caccioppoli} or \textit{finite perimeter sets} of the celebrated Plateau problem are revisited in Sect. 6.
    
    Their characteristic function is a function of bounded variation.
    
    They provide the 1st example of compact families of characteristic functions in $L^p(D)$-strong, $1\le p < \infty$.
    
    This was developed mainly by R. Caccioppoli [1] and E. De Giorgi [1] in the context of J. A. F. Plateau [1]'s problem of minimal surfaces.
    
    Sect. 6.3 exploits the embedding\footnote{$\operatorname{BV}(D)$ is the space of functions of bounded variations in $D$.} of $\operatorname{BV}(D)\cap L^\infty(D)$ into the Sobolev spaces $W^{\varepsilon,p}(D)$, $0 < \varepsilon < \frac{1}{p}$, $p\ge 1$, to introduce a cascade of complete metrics between $L^p(D)$ and $\operatorname{BV}(D)$ on characteristic functions.
    
    In Sect. 6.4 we show that the family of Lipschitzian domains in a fixed bounded holdall verifying the uniform cone property of Sect. 6.4.1 in Chap. 2 is also compact.
    
    This condition naturally yields a uniform bound on the perimeter of the sets in the family and hence can be viewed as a special case of the 1st compactness theorem.
    
    Sect. 7 gives an example of the use of the perimeter in the Bernoulli free boundary problem and in particular for the water wave.
    
    There the energy associated with the surface tension of the water is proportional to the perimeter, i.e., the surface area of the free boundary.
\end{enumerate}

\paragraph{Abelian Group Structure on Measurable Characteristic Functions.}
\begin{enumerate}
    \item \textbf{Group Structure on $X_\mu(\mathbb{R}^N)$.} From Sect. 2.2 in Chap. 2, the Abelian group structure on the subsets of $\mathbb{R}^N$ extends to the corresponding family of their characteristic functions.
    
    %
    \begin{theorem}
        Let $\emptyset\ne D\subset\mathbb{R}^N$. The space of characteristic functions $\operatorname{X}(D)$ endowed with the multiplication $\Delta$ and the neutral multiplicative element $\chi_\emptyset = 0$ \textbf{(2.1)}
        \begin{align*}
            \chi_A\Delta\chi_B := \left|\chi_A - \chi_B\right| = \chi_{(A\cap B^c)\cup(B\cap A^c)} = \chi_{A\Delta B}
        \end{align*}
        is an Abelian group, i.e. \textbf{(2.2)}
        \begin{align*}
            \chi_A\Delta\chi_B = \chi_B\Delta\chi_A,\ \chi_A\Delta\chi_\emptyset = \chi_A,\ \chi_A\Delta\chi_A = \chi_\emptyset,\ \chi_A^{-1} = \chi_A.
        \end{align*}
    \end{theorem}

    \begin{remark}
        We chose the product terminology $\Delta$, but we could have chosen to call that algebraic operation an addition $\oplus$.
    \end{remark}
    \item \textbf{Measure Spaces.}
    \begin{definition}
        Let $X$ be a set and $\mathcal{P}(X)$ be the collection of subsets of $X$.
        \begin{itemize}
            \item[(i)] A mapping $\mu:\mathcal{P}(X)\to[0,\infty]$ is called a \emph{measure} on $X$ if
            \begin{itemize}
                \item[(a)] $\mu(\emptyset) = 0$, and
                \item[(b)] $\mu(A)\le\sum_{k=1}^\infty \mu(A_k)$, whenever $A\subset\bigcup_{k=1}^\infty A_k$.
            \end{itemize}
            \item[(ii)] A subset $A\subset X$ is \emph{$\sigma$-finite w.r.t. $\mu$} if it can be written $A = \bigcup_{k=1}^\infty B_k$, where $B_k$ is $\mu$-measurable and $\mu(B_k) < \infty$ for $k = 1,2,\ldots$.
            \item[(iii)] A measure $\mu$ on $\mathbb{R}^N$ is a \emph{Radon} measure if $\mu$ is Borel regular and $\mu(K) < \infty$ for each compact set $K\subset\mathbb{R}^N$.
        \end{itemize}
    \end{definition}
    The \textit{Lebesgue measure} $\operatorname{m}_N$ on $\mathbb{R}^N$ is a Radon measure since $\mathbb{R}^N$ is $\sigma$-finite w.r.t. $\operatorname{m}_N$.
    
    The \textit{$s$-dimensional Hausdorff measure} $H_s$, $0\le s < N$, on $\mathbb{R}^N$ (cf. Sect. 3.2.3 in Chap. 2) is \textit{Borel regular} but not necessarily a Radon measure since bounded $H_s$-measurable subsets of $\mathbb{R}^N$ are not necessarily $\sigma$-finite w.r.t. $H_s$.
    
    Yet, a Borel regular measure $\mu$ can be made $\sigma$-finite by restricting it to a $\mu$-measurable set $A\subset\mathbb{R}^N$ s.t. $\mu(A) < \infty$ or, more generally, to a $\sigma$-finite $\mu$-measurable set $A\subset\mathbb{R}^N$.
    
    For smooth submanifolds of dimension $s$, $s$ an integer, $H_s$ gives the same area as the integral of the canonical density in Sect. 3.2.2 of Chap. 2.
    
    %
    Given a $\mu$-measurable subset $D\subset\mathbb{R}^N$, let $L^p(D,\mu)$ denote the Banach space of equivalence classes of $\mu$-measurable functions s.t. $\int |f|^p{\rm d}\mu < \infty$, $1\le p < \infty$, and let $L^\infty(D,\mu)$ be the space of equivalence classes of $\mu$-measurable functions s.t. $\operatorname{ess}\sup_D |f| < \infty$.
    
    Denote by $[A]_\mu$ the equivalence class of $\mu$-measurable subsets of $D$ that are equal almost everywhere and
    \begin{align*}
        \operatorname{X}_\mu(D) := \left\{\chi_A;A\subset D \mbox{ is } \mu\mbox{-measurable}\right\}.
    \end{align*}
    Since $A\Delta B$ is $\mu$-measurable for $A$ and $B$ $\mu$-measurable, $\{[A]_\mu;A\subset D \mbox{ is } \mu\mbox{-measurable}\}$ is a subgroup of $\mathcal{P}(D)$ and \textbf{(2.3)}
    \begin{align*}
        \operatorname{X}_\mu(D)\cap L^1(D,\mu) = \left\{\chi_A;A\subset D \mbox{ is } \mu\mbox{-measurable and } \mu(A) < \infty\right\}
    \end{align*}
    is a group in $L^p(D,\mu)$ w.r.t. $\Delta$.
    
    When $\mu = \operatorname{m}_N$, drop the subscript $\mu$.
    \item \textbf{Complete Metric for Characteristic Functions in $L^p$-Topologies.} Now that we have revealed the underlying group structure of $\operatorname{X}_\mu(D)\cap L^1(D,\mu)$, we construct a metric on that group.
    
    We have a structure similar to the Courant metric of A. M. Micheletti [1] in Chap. 3 on groups of transformations of $\mathbb{R}^N$.
    
    We identify equivalence classes of $\mu$-measurable subsets of $D$ and characteristic functions via the bijection $[A]_\mu\leftrightarrow\chi_A$.
    
    \begin{theorem}
        Let $1\le p < \infty$, let $\mu$ be a measure on $\mathbb{R}^N$, and let $\emptyset\ne D\subset\mathbb{R}^N$ be $\mu$-measurable.
        \begin{itemize}
            \item[(i)] $\operatorname{X}_\mu(D)\cap L^1(D,\mu)$ is closed in $L^p(D,\mu)$. The function
            \begin{align*}
                \rho_D([A_1]_\mu,[A_2]_\mu) := \|\chi_{A_2} - \chi_{A_1}\|_{L^p(D,\mu)}
            \end{align*}
            defines a complete metric structure on the Abelian group $\operatorname{X}_\mu(D)\cap L^1(D,\mu)$ that makes it a topological group. If $\mu(D) < \infty$, then $\operatorname{X}_\mu(D)\cap L^1(D,\mu) = \operatorname{X}_\mu(D)$.
            \item[(ii)] If, in addition, $D$ is $\sigma$-finite w.r.t. $\mu$ for a family $\{D_k\}$ of $\mu$-measurable subsets of $D$ s.t. $\mu(D_k) < \infty$, for all $k\ge 1$, then $\operatorname{X}_\mu(D)$ is closed in $L_{\rm loc}^p(D,\mu)$ and
            \begin{align*}
                \rho([A_1]_\mu,[A_2]_\mu) := \sum_{k=1}^\infty \frac{1}{2^n}\frac{\|\chi_{A_2} - \chi_{A_1}\|_{L^p(D_k,\mu)}}{1 + \|\chi_{A_2} - \chi_{A_1}\|_{L^p(D_k,\mu)}}
            \end{align*}
            defines a complete metric structure on the Abelian group $\operatorname{X}_\mu(D)$ that makes it a topological group. When $\mu$ is a Radon measure on $\mathbb{R}^N$, the assumption that $D$ is $\sigma$-finite w.r.t. $\mu$ can be dropped.        
        \end{itemize}
    \end{theorem}
    \begin{remark}
        For the Lebesgue measure $\operatorname{m}_N$ and $D$ measurable (resp., measurable and bounded), $\operatorname{X}(D)$ is a topological Abelian group in $L_{\rm loc}^p(D,\operatorname{m}_N)$ (resp. $L^p(D,\operatorname{m}_N)$).
        
        For Hausdorff measures $H_s$, $0\le s < N$, the theorem does not say more.
    \end{remark}
    For $p = 1$ and $\mu$-measurable sets $A_1$ and $A_2$ in $D$, the metric $\rho_D([A_1]_\mu,[A_2]_\mu)$ is the $\mu$-measure of the symmetric difference
    \begin{align*}
        A_1\Delta A_2 = (A_1^c\cap A_2)\cup(A_2^c\cap A_1)
    \end{align*}
    and for $1\le p < \infty$ all the topologies are equivalent on $\operatorname{X}_\mu(D)\cap L^1(D,\mu)$.
    
    \begin{theorem}
        Let $1\le p < \infty$, let $\mu$ be a measure on $\mathbb{R}^N$, and let $\emptyset\ne D\subset\mathbb{R}^N$ be $\mu$-measurable.
        \begin{itemize}
            \item[(i)] The topologies induced by $L^p(D,\mu)$ on $\operatorname{X}_\mu(D)\cap L^1(D,\mu)$ are all equivalent for $1\le p < \infty$.
            \item[(ii)] If, in addition, $D$ is $\sigma$-finite w.r.t. $\mu$ for a family $\{D_k\}$ of $\mu$-measurable subsets of $D$ s.t. $\mu(D_k) < \infty$, for all $k\ge 1$, then the topologies induced by $L_{\rm loc}^p(D,\mu)$ on $\operatorname{X}_\mu(D)$ are all equivalent for $1\le p < \infty$.
        \end{itemize}
    \end{theorem}

    \begin{remark}
        At this stage, it is not clear what the tangent space to this topological Abelian group is.
        
        If $\mu$ is a Radon measure, then for all $\varphi\in\mathcal{D}(\mathbb{R}^N)$
        \begin{align*}
            \frac{1}{\mu\left(B_\varepsilon(x)\right)}\int_{\mathbb{R}^N} \left[\chi_A\Delta\chi_{B_\varepsilon(x)}\right]\Delta\chi_A^{-1}\varphi{\rm d}\mu = \frac{1}{\mu\left(B_\varepsilon(x)\right)}\int_{\mathbb{R}^N} \chi_{B_\varepsilon(x)}\varphi{\rm d}\mu,
        \end{align*}
        and by the Lebesgue-Besicovitch differentiation theorem
        \begin{align*}
            \frac{1}{\mu\left(B_\varepsilon(x)\right)}\int_{\mathbb{R}^N} \left[\chi_A\Delta\chi_{B_\varepsilon(x)}\right]\Delta\chi_A^{-1}\varphi{\rm d}\mu\to\varphi(x),\ \mu \mbox{ a.e. in } \mathbb{R}^N.
        \end{align*}
    \end{remark}
\end{enumerate}

\paragraph{Lebesgue Measurable Characteristic Functions.}
\begin{enumerate}
    \item In this section we specialize to the Lebesgue measure and the corresponding family of characteristic functions \textbf{(3.1)}
    \begin{align*}
        \operatorname{X}(\mathbb{R}^N) := \left\{\chi_\Omega;\forall\Omega \mbox{ Lebesuge measurable in } \mathbb{R}^N\right\}.
    \end{align*}
    Clearly $\operatorname{X}(\mathbb{R}^N)\subset L^\infty(\mathbb{R}^N)$, and for all $p\ge 1$, $\operatorname{X}(\mathbb{R}^N)\subset L_{\rm loc}^p(\mathbb{R}^N)$.
    
    Also associate with $\emptyset\ne D\subset\mathbb{R}^N$ the set \textbf{(3.2)}
    \begin{align*}
        \operatorname{X}(D) := \left\{\chi_\Omega;\forall\Omega \mbox{ Lebesgue measurable in } D\right\}.
    \end{align*}
    The above definitions are special cases of the definitions of Sect. 2.3, and Theorems 2.2 and 2.3 apply to $\operatorname{m}_N$ as a Radon measure.
    \item \textbf{Strong Topologies \& $C^\infty$-Approximations.} In view of the equivalence Theorem 2.3, we introduce the notion of strong convergence.
    
    \begin{definition}
        \begin{itemize}
            \item[(i)] If $D$ is a bounded measurable subset of $\mathbb{R}^N$, a sequence $\{\chi_n\}$ in $\operatorname{X}(D)$ is said to be \emph{strongly convergent} in $D$ if it converges in $L^p(D)$-strong for some $p$, $1\le p < \infty$.
            \item[(ii)] A sequence $\{\chi_n\}$ in $\operatorname{X}(\mathbb{R}^N)$ is said to be \emph{locally strongly convergent} if it converges in $L_{\rm loc}^p(\mathbb{R}^N)$-strong for some $p$, $1\le p < \infty$.
        \end{itemize}
    \end{definition}
    The following approximation theorem will also be useful.
    
    \begin{theorem}
        Let $\Omega$ be an arbitrary Lebesgue measurable subset of $\mathbb{R}^N$. There exists a sequence $\{\Omega_n\}$ of open $C^\infty$-domains in $\mathbb{R}^N$ s.t.
        \begin{align*}
            \chi_{\Omega_n}\to\chi_\Omega \mbox{ in } L_{\rm loc}^1(\mathbb{R}^N).
        \end{align*}
    \end{theorem}
    \item \textbf{Weak Topologies \& Microstructures.} Some \textit{shape optimization} problems lead to apparent paradoxes.
    
    Their solution is no longer a geometric domain associated with a characteristic function, but a \textit{fuzzy domain} associated with the relaxation of a characteristic function to a function with values ranging in $[0,1]$.
    
    The intuitive notion of a geometric domain is relaxed to the notion of a \textit{probability distribution} of the presence of points of the set.
    
    When the underlying problem involves 2 different materials characterized by 2 constants $k_1\ne k_2$, the occurrence of such a solution can be interpreted as the \textit{mixing} or \textit{homogenization} of the 2 materials at the microscale.
    
    This is also referred to as a \textit{composite material} or a \textit{microstructure}.
    
    Somehow this is related to the fact that the strong convergence needs to be relaxed to the weak $L^p$-convergence and the space $\operatorname{X}(D)$ needs to be suitably enlarged.
    
    %
    Even if $X(D)$ is strongly closed and bounded in $L^p(D)$, it is not strongly compact.
    
    However, for $1 < p < \infty$ its closed convex hull $\overline{\rm co}\operatorname{X}(D)$ is weakly compact in the reflexive Banach space $L^p(D)$.
    
    In fact, \textbf{(3.3)}
    \begin{align*}
        \overline{\rm co}\operatorname{X}(D) = \left\{\chi\in L^p(D);\chi(x)\in[0,1] \mbox{ a.e. in } D\right\}.
    \end{align*}
    Indeed, by definition $\overline{\rm co}\operatorname{X}(D)\subset\{\chi\in L^p(D);\chi(x)\in[0,1]\mbox{ a.e. in } D\}$.
    
    Conversely any $\chi$ that belongs to the RHS of (3.3) can be approximated by a sequence of convex combinations of elements of $\operatorname{X}(D)$.
    
    Choose
    \begin{align*}
        \chi_n = \sum_{m=1}^n \frac{1}{n}\chi_{B_{nm}},\ B_{nm} = \left\{x;\chi(x)\ge\frac{m}{n}\right\},
    \end{align*}
    for which $|\chi_n(x) - \chi(x)| < \frac{1}{n}$.
    
    The elements of $\overline{\rm co}\operatorname{X}(D)$ are not necessarily characteristic functions of a domain; i.e., the identity
    \begin{align*}
        \chi(x)\left(1 - \chi(x)\right) = 0 \mbox{ a.e. in } D
    \end{align*}
    is not necessarily satisfied.
    
    %
    We 1st give a few basic results and then consider a classical example from the \textit{theory of homogenization} of differential equations.
    
    \begin{lemma}
        Let $D$ be a bounded open subset of $\mathbb{R}^N$, let $K$ be a bounded subset of $\mathbb{R}$, and let
        \begin{align*}
            \mathcal{K} := \left\{k:D\to\mathbb{R};k \mbox{ is measurable and } k(x)\in K \mbox{ a.e. in } D\right\}.
        \end{align*}
        \begin{itemize}
            \item[(i)] For any $p$, $1\le p < \infty$, and any sequence $\{k_n\}\subset\mathcal{K}$ the following statements are equivalent:
            \begin{itemize}
                \item[(a)] $\{k_n\}$ converges in $L^\infty(D)$-weak$\star$;
                \item[(b)] $\{k_n\}$ converges in $L^p(D)$-weak;
                \item[(c)] $\{k_n\}$ converges in $\mathcal{D}(D)'$,
            \end{itemize}
            where $\mathcal{D}(D)'$ is the space of scalar distributions on $D$.
            \item[(ii)] If $K$ is bounded, closed, and convex, then $\mathcal{K}$ is convex and compact in $L^\infty(D)$-weak$\star$, $L^p(D)$-weak, and $\mathcal{D}(D)'$.
        \end{itemize}
        The above results remain true in the vectorial case when $\mathcal{K}$ is the set of mappings $k:D\to K$ for some bounded subset $K\subset\mathbb{R}^p$ and a finite integer $p\ge 1$.
    \end{lemma}
    
    \begin{proof}
        See \cite[p. 216]{Delfour_Zolesio2011}.\footnote{In a metric space the compactness is equivalent to the sequential compactness.
            
            For the weak topology we use the fact that if $E$ is a separable normed space, then, in its topological dual $E'$, any closed ball is a compact metrizable space for the weak topology.
            
            Since $\mathcal{K}$ is a bounded subset of the normed reflexive separable Banach space $L^p(D)$, $1\le p < \infty$, the weak compactness of $\mathcal{K}$ coincides with its weak sequential compactness (cf. J. Dieudonné [1, Vol. 2, Chap. XII, sect. 12.15.9, p. 75]).}
    \end{proof}
    In view of this equivalence we adopt the following terminology.
    
    \begin{definition}
        Let $\emptyset\ne D\subset\mathbb{R}^N$ be bounded open. A sequence $\{\chi_n\}$ in $\operatorname{X}(D)$ is said to be \emph{weakly convergent} if it converges for some topology between $L^\infty(D)$-weak$\star$ and $\mathcal{D}(D)'$.
    \end{definition}
    It is interesting to observe that working with the weak convergence makes sense only when the limit element is not a characteristic function.
    
    \begin{theorem}
        Let the assumptions of the theorem be satisfied. Let $\{\chi_n\}$ and $\chi$ be elements of $\operatorname{X}(\mathbb{R}^N)$ (resp. $\operatorname{X}(D)$) s.t. $\chi_n\rightharpoonup\chi$ weakly in $L^2(D)$ (resp., $L_{\rm loc}^2(\mathbb{R}^N)$). Then for all $p$, $1\le p < \infty$,
        \begin{align*}
            \chi_n\to\chi \mbox{ strongly in } L^p(D) \mbox{ (resp., } L_{\rm loc}^p(\mathbb{R}^N)).
        \end{align*}
    \end{theorem}
    Working with the weak convergence creates new phenomena and difficulties.
    
    E.g. when a characteristic function is present in the coefficient of the higher-order term of a differential equation the weak convergence of a sequence of characteristic functions $\{\chi_n\}$ to some element $\chi$ of $\overline{\rm co}\operatorname{X}(D)$,
    \begin{align*}
        \chi_n\rightharpoonup\chi \mbox{ in } L^2(D)\mbox{-weak }\Rightarrow\chi_n\rightharpoonup\chi \mbox{ in } L^\infty(D)\mbox{-weak}\star,
    \end{align*}
    does not imply the weak convergence in $H^1(D)$ of the sequence $\{y(\chi_n)\}$ of solutions to the solution of the differential equation corresponding to $y(\chi)$,
    \begin{align*}
        y(\chi_n)\not\rightharpoonup y(\chi) \mbox{ in } H^1(D)\mbox{-weak}.
    \end{align*}
    By compactness\footnote{This is true since $D$ is a bounded Lipschitzian domain.
        
        Examples of domains between 2 spirals can be constructed where the injection is not compact.} of the injection of $H^1(D)$ into $L^2(D)$ this would have implied convergence in $L^2(D)$-strong:
    \begin{align*}
        y(\chi_n)\to y(\chi) \mbox{ in } L^2(D)\mbox{-strong}.
    \end{align*}
    This fact was pointed out in 1971 by F. Murat [1] in the following example, which will be rewritten to emphasize the role of the characteristic function.
    
    \begin{example}
        Consider the following BVP \textbf{(3.4)}
        \begin{equation*}
            \left\{\begin{split}
                -\frac{d}{dx}\left(k\frac{dy}{dx}\right) + ky &= 0 \mbox{ in } D= (0,1),\\
                y(0) &= 1 \mbox{ and } y(1) = 2,
            \end{split}\right.
        \end{equation*}
        where \textbf{(3.5)}
        \begin{align*}
            \mathcal{K} = \left\{k = k_1(x)\chi + k_2(x)(1 - \chi);\chi\in\operatorname{X}(D)\right\}
        \end{align*}
        with \textbf{(3.6)}
        \begin{align*}
            k_1(x) = 1 - \sqrt{\frac{1}{6} - \frac{x^2}{6}},\ k_2(x) = 1 + \sqrt{\frac{1}{6} - \frac{x^2}{6}}.
        \end{align*}
        This is equivalent to \textbf{(3.7)}
        \begin{align*}
            \mathcal{K} = \left\{k\in L^\infty(0,1);k(x)\in\left\{k_1(x),k_2(x)\right\} \mbox{ a.e. in } [0,1]\right\}.
        \end{align*}
        Associate with the integer $p\ge 1$ the function \textbf{(3.8)}
        \begin{equation*}
            k^p(x) = \left\{\begin{split}
                &1 - \sqrt{\frac{1}{2} - \frac{x^2}{6}}, &&\frac{m}{p} < x\le\frac{2m + 1}{p},\\
                &1 + \sqrt{\frac{1}{2} - \frac{x^2}{6}}, &&\frac{2m + 1}{2p} < x\le\frac{m + 1}{p},
            \end{split}\right.\ 0\le m\le p - 1,
        \end{equation*}
        and the characteristic function \textbf{(3.9)}
        \begin{equation*}
            \chi_p(x) = \left\{\begin{split}
                &1, &&\frac{m}{p} < x\le\frac{2m + 1}{p},\\
                &0, &&\frac{2m + 1}{2p} < x\le\frac{m + 1}{p},
            \end{split}\right.\ 0\le m\le p - 1.
        \end{equation*}
        It is readily seen that $k^p = k_1\chi_p + k_2\left(1 - \chi_p\right)$.
        
        He shows that for each $p$, $k^p\in\mathcal{K}$, (resp., $\chi_p\in\operatorname{X}(D)$) and that \textbf{(3.10)-(3.11)}
        \begin{align*}
            k^p&\rightharpoonup k_\infty = 1\ \left(\mbox{resp., } \chi_p\rightharpoonup\frac{1}{2}\right) \mbox{ in } L^\infty(0,1)\mbox{-weak}\star,\\
            \frac{1}{k^p}&\rightharpoonup\frac{1}{2}\left(\frac{1}{k_1} + \frac{1}{k_2}\right) = \frac{1}{\frac{1}{2} + \frac{x^2}{6}} \mbox{ in } L^\infty(0,1)\mbox{-weak}\star.
        \end{align*}
        Moreover, \textbf{(3.12)}
        \begin{align*}
            y_p\rightharpoonup y \mbox{ in } H^1(0,1)\mbox{-weak},
        \end{align*}
        where $y_p$ denotes the solution of (3.4) corresponding to $k = k^p$ and $y$ the solution of the BVP \textbf{(3.13)}
        \begin{equation*}
            \left\{\begin{split}
                -\frac{d}{dx}\left[\left(\frac{1}{2} + \frac{x^2}{6}\right)\frac{dy}{dx}\right] + y &= 0 \mbox{ in } (0,1),\\
                y(0) = 1,\ y(1) &= 2.
            \end{split}\right.
        \end{equation*}
        Define the function \textbf{(3.14)}
        \begin{align*}
            k_H(x) = \frac{1}{2} + \frac{x^2}{6},
        \end{align*}
        which corresponds to \textbf{(3.15)}
        \begin{align*}
            \chi_H(x) = \frac{1}{2}\left[1 + \sqrt{\frac{1}{2} - \frac{x^2}{6}}\right]\in\overline{\rm co}\operatorname{X}(D).
        \end{align*}
        Notice that $k_H$ appears in the 2nd-order term and $k_\infty$ in the 0th-order term in (3.13): \textbf{(3.16)}
        \begin{align*}
            -\frac{d}{dx}\left(k_H\frac{dy}{dx}\right) + k_\infty y = 0 \mbox{ in } (0,1).
        \end{align*}
        It is easy to check that \textbf{(3.17)}
        \begin{align*}
            y(x) = 1 + x^2 \mbox{ in } [0,1],
        \end{align*}
        which is not equal to the solution $y_\infty$ of (3.4) for the weak limit $k_\infty = 1$: \textbf{(3.18)}
        \begin{align*}
            y_\infty(x) = \frac{2\left(e^x - e^{-x}\right) + e^{1 - x} - e^{-(1 - x)}}{e - e^{-1}}.
        \end{align*}
    \end{example}
    To the authors' knowledge this was the beginning of the theory of homogenization in France.
    
    This is only 1 part of F. Murat [1]'s example.
    
    He also constructs an objective function for which the lower bound is not achieved by an element of $\mathcal{K}$.
    
    This is a nonexistence result.
    
    %
    The above example uses space varying coefficients $k_1(x)$ and $k_2(x)$.
    
    However, it is still valid for 2 positive constants $k_1 > 0$ and $k_2 > 0$.
    
    It is easy to show that
    \begin{align*}
        k_\infty = \frac{k_1 + k_2}{2},\ \chi_\infty = \frac{1}{2}, \mbox{ and } \frac{1}{k_H} = \frac{1}{2}\left(\frac{1}{k_1} + \frac{1}{k_2}\right),\ \chi_H = \frac{k_2}{k_1 + k_2},
    \end{align*}
    and the solution $y$ of the BVP (3.16) is given by
    \begin{align*}
        y(x) = \frac{2\sinh cx + \sinh c(1 - x)}{\sinh c}, \mbox{ where } c = \frac{k_1 + k_2}{2\sqrt{k_1k_2}}\ge 1,
    \end{align*}
    and the solution $y_\infty$ by
    \begin{align*}
        y_\infty(x) = \frac{2\sinh x + \sinh (1 - x)}{\sinh 1}.
    \end{align*}
    Thus for $k_1\ne k_2$, or equivalently $c > 1$, $y\ne y_\infty$.
    
    %
    In fact, using the same sequence of $\chi_p$'s, the sequence $y_p = y(\chi_p)$ of solutions of (4.5) weakly converges to $y_H = y(\chi_H)$, which is different from the solution $y_\infty = y(\chi_\infty)$ for $k_1\ne k_2$.
    
    This is readily seen by noticing that since $\chi_\infty$ and $\chi_H$ are constant
    \begin{align*}
        -\Delta(k_Hy_H) &= \chi_\infty f = -\Delta(k_\infty y_\infty)\\
        \Rightarrow k_Hy_H &= k_\infty y_\infty\Rightarrow y_H = \frac{(k_1 + k_2)^2}{4k_1k_2}y_\infty\ne y_\infty, \mbox{ for } k_1\ne k_2.
    \end{align*}
    Despite this example, we shall see in the next section that in some cases we obtain the existence (and uniqueness) of a maximizer in $\overline{\rm co}\operatorname{X}(D)$ which belongs to $\operatorname{X}(D)$.
    \item \textbf{Nice or Measure Theoretic Representative.} Since the strong and weak $L^p$-topologies are defined on equivalence classes $[\Omega]$ of (Lebesgue) measurable subsets $\Omega$ of $\mathbb{R}^N$, it is natural to ask if there is a \textit{nice representative} that is generic of the class $[\Omega]$.
    
    E.g. we have seen in Chap. 2 that within the equivalence class of a set of class $C^k$ there is a unique open and a unique closed representative and that all elements of the class have the same interior, boundary, and exterior.
    
    The same question will again arise for finite perimeter sets.
    
    As an illustration of what is meant by a nice representative, consider the smiling and the expressionless suns in Fig. 5.1.
    
    \textsf{Fig. 5.1 Smiling sun $\Omega$ and expressionless sun $\tilde{\Omega}$.}
    
    The expressionless sun is obtained by adding missing points and lines ``inside'' $\Omega$ and removing the rays ``outside'' $\Omega$.
    
    This ``restoring/cleaning'' operation can be formalized as follows.
    
    \begin{definition}
        Associate with a Lebesgue measurable set $\Omega$ in $\mathbb{R}^N$ the sets
        \begin{align*}
            \Omega_0 &:= \left\{x\in\mathbb{R}^N;\exists\rho > 0 \mbox{ s.t. } \operatorname{m}(\Omega\cap B(x,\rho)) = 0\right\},\\
            \Omega_1 &:= \left\{x\in\mathbb{R}^N;\exists\rho > 0 \mbox{ s.t. } \operatorname{m}(\Omega\cap B(x,\rho)) = \operatorname{m}(B(x,\rho))\right\},\\
            \Omega_\bullet &:= \left\{x\in\mathbb{R}^N;\forall\rho > 0 \mbox{ s.t. } 0 < \operatorname{m}(\Omega\cap B(x,\rho)) < \operatorname{m}(B(x,\rho))\right\}
        \end{align*}
        and the \emph{measure theoretic exterior} ${\rm O}$, \emph{interior} ${\rm I}$, and \emph{boundary} $\partial_*\Omega$
        \begin{align*}
            {\rm O} &:= \left\{x\in\mathbb{R}^N;\lim_{r\downarrow 0} \frac{\operatorname{m}(B(x,r)\cap\Omega)}{\operatorname{m}(B(x,r))} = 0\right\},\\
            {\rm I} &:= \left\{x\in\mathbb{R}^N;\lim_{r\downarrow 0} \frac{\operatorname{m}(B(x,r)\cap\Omega)}{\operatorname{m}(B(x,r))} = 1\right\},\\
            \partial_*\Omega &:= \left\{x\in\mathbb{R}^N;0\le\liminf_{r\downarrow 0} \frac{\operatorname{m}(B(x,r)\cap\Omega)}{\operatorname{m}(B(x,r))} < \limsup_{r\downarrow 0} \frac{\operatorname{m}(B(x,r)\cap\Omega)}{\operatorname{m}(B(x,r))} \le 1\right\}.
        \end{align*}
        We shall say that ${\rm I}$ is the \emph{nice} or \emph{measure theoretic representative} of $\Omega$.
    \end{definition}
    The 6 sets $\Omega_0$, $\Omega_1$, $\Omega_\bullet$, ${\rm O}$, ${\rm I}$, and $\partial_*\Omega$ are \textit{invariant} for all sets in the equivalence class $[\Omega]$ of $\Omega$.
    
    They are 2 different partitions of $\mathbb{R}^N$ \textbf{(3.19)}
    \begin{align*}
        \Omega_0\cup\Omega_1\cup\Omega_\bullet = \mathbb{R}^N \mbox{ and } {\rm O}\cup{\rm I}\cup\partial_*\Omega = \mathbb{R}^N
    \end{align*}
    like $\operatorname{int}\Omega^c\cup\operatorname{int}\Omega\cup\partial\Omega = \mathbb{R}^N$.
    
    The next theorem links the 6 invariant sets and describes some of the interesting properties of the measure theoretic representative.
    
    \begin{theorem}
        Let $\Omega$ be a Lebesgue measurable set in $\mathbb{R}^N$ and let $\Omega_0$, $\Omega_1$, $\Omega_\bullet$, ${\rm O}$, ${\rm I}$, and $\partial_*\Omega$ be the sets constructed from $\Omega$ in Definition 3.3.
        \begin{itemize}
            \item[(i)] The sets ${\rm O}$, ${\rm I}$, and $\partial_*\Omega$ are invariants of the equivalence class $[\Omega]$. Moreover, they are Borel measurable and
            \begin{align*}
                \chi_I = \chi_\Omega,\ \chi_{\rm O} = \chi_{\Omega^c}, \mbox{ and } \chi_{\partial_*\Omega} = 0 \mbox{ a.e. in } \mathbb{R}^N.
            \end{align*}
            \item[(ii)] The sets $\Omega_0$, $\Omega_1$ and $\Omega_\bullet$ are invariants of the equivalence class $[\Omega]$. The sets $\Omega_0$ and $\Omega_1$ are open, $\Omega_\bullet$ is closed, and \textbf{(3.20)-(3.21)}
            \begin{align*}
                \operatorname{int}\Omega\subset\Omega_1 = \operatorname{int}{\rm I},\ \operatorname{int}\Omega^c\subset\Omega_0 = \operatorname{int}{\rm O},\ \partial\Omega&\supset\overline{I^c}\cap\overline{{\rm O}^c} = \Omega_\bullet\supset\partial_*\Omega,\\
                \partial_*\Omega\cup\partial{\rm O}\cup\partial{\rm I}&\subset\Omega_\bullet\subset\partial\Omega.
            \end{align*}
            \item[(iii)] The condition $\operatorname{m}(\partial\Omega) = 0$ implies $\operatorname{m}(\partial{\rm I}) = 0$. When $\operatorname{m}(\partial{\rm I}) = 0$, then $\operatorname{int}{\rm I}$ and $\overline{\rm I}$ can be chosen as the respective open and closed representatives of $\Omega$. In particular, this is true for Lipschitzian sets and hence for convex sets.
        \end{itemize}
    \end{theorem}
    The inclusions (3.20) indicate that the measure theoretic interior and exterior are enlarged and that the measure theoretic boundary is reduced.
    
    In general those operations do not commute with the set theoretic operations.
    
    E.g., they do not commute with the closure, as can be seen from the following simple example.
    
    \begin{example}
        Consider the set $\Omega$ of all rational numbers in $[0,1]$. Then
        \begin{equation*}
            \left.\begin{split}
                {\rm O} &= \Omega_0 = \mathbb{R}\\
                {\rm I} &= \Omega_1 = \emptyset\\
                \partial_*\Omega &= \Omega_\bullet = \emptyset
            \end{split}\right|\Rightarrow\operatorname{I}(\Omega) = \emptyset,
        \end{equation*}
        \begin{equation*}
            \left.\begin{split}
                {\rm O}(\overline{\Omega}) &= (\overline{\Omega})_0 = \mathbb{R}\backslash[0,1]\\
                {\rm I}(\overline{\Omega}) &= (\overline{\Omega})_1 = (0,1)\\
                \partial_*\Omega(\overline{\Omega}) &= (\overline{\Omega})_\bullet = \{0,1\}
            \end{split}\right|\Rightarrow\operatorname{I}(\overline{\Omega}) = (0,1).
        \end{equation*}
    \end{example}
    However, the complement operation commutes with the other set of operations and determines the same sets \textbf{(3.22)-(3.23)}
    \begin{align*}
        (\Omega^c)_0 &= \Omega_1,\ (\Omega^c)_1 = \Omega_0,\ (\Omega^c)_\bullet = \Omega_\bullet,\\
        \operatorname{I}(\Omega^c) &= \operatorname{O}(\Omega),\ \operatorname{I}(\Omega) = \operatorname{O}(\Omega^c),\ \partial_*(\Omega) = \partial_*(\Omega^c).
    \end{align*}
    The next theorem is a companion to Lemma 5.1 in Chap. 2.
    
    \begin{theorem}
        The following conditions are equivalent:
        \begin{itemize}
            \item[(i)] $\overline{\operatorname{int}\Omega} = \overline{\Omega}$ and $\overline{\operatorname{int}\Omega^c} = \overline{\Omega^c}$.
            \item[(ii)] $\operatorname{int}\Omega = \Omega_1$, $\operatorname{int}\Omega^c = \Omega_0$, $\partial\Omega = \partial\Omega_0 = \partial\Omega_1$ ($= \Omega_\bullet$).
        \end{itemize}
        If either of the above 2 conditions is satisfied, then
        \begin{align*}
            \operatorname{int}{\rm I} = \operatorname{int}\Omega,\ \operatorname{int}{\rm O} = \operatorname{int}\Omega^c,\ \overline{{\rm I}^c}\cap\overline{{\rm O}^c} = \partial\Omega = \partial\overline{\Omega} = \partial\overline{\Omega^c}.
        \end{align*}
    \end{theorem}
    \item \textbf{The Family of Convex Sets.} Convex sets will play a special role here and in the subsequent chapters.
    
    We shall say that an equivalence class $[\Omega]$ of Lebesgue measurable subsets of $D$ is \textit{convex} if there exists a convex Lebesgue measurable subset $\Omega^*$ of $D$ s.t. $[\Omega] = [\Omega^*]$.
    
    We also introduce the notation \textbf{(3.24)}
    \begin{align*}
        \mathcal{C}(D) := \left\{\chi_\Omega;\Omega \mbox{ convex subset of } D\right\}.
    \end{align*}
    
    \begin{theorem}
        Let $\Omega\ne\emptyset$ be a convex subset of $\mathbb{R}^N$.
        \begin{itemize}
            \item[(i)] $\overline{\Omega}$ and $\operatorname{int}\Omega$ are convex.\footnote{By convention $\emptyset$ is convex.}
            \item[(ii)] $(\operatorname{int}\Omega)^c = \overline{\Omega^c} = \overline{\overline{\Omega}^c} = \overline{\operatorname{int}\Omega^c}$.
            
            If $\operatorname{int}\Omega\ne\emptyset$, then $\overline{\operatorname{int}\Omega} = \overline{\Omega}$.
            
            If $\operatorname{int}\Omega\ne\emptyset$ and $\Omega\ne\mathbb{R}^N$, then $\partial\Omega\ne\emptyset$ and $\operatorname{int}\Omega^c\ne\emptyset$.
            \item[(iii)] Given a measurable (resp., bounded measurable) subset $D$ in $\mathbb{R}^N$
            \begin{align*}
                \forall 1\le p < \infty,\ \mathcal{C}(D) \mbox{ is closed in } L_{\rm loc}^p(D) \mbox{ (resp., } L^p(D)).
            \end{align*}
        \end{itemize}
    \end{theorem}
    Show in Sect. 6.1 (Corollary 1) that for bounded domains $D$, $\mathcal{C}(D)$ is also compact for the $L^p(D)$ topology, $1\le p < \infty$.
    \item \textbf{Sobolev Spaces for Measurable Domains.} The lack of a priori smoothness on $\Omega$ may introduce technical difficulties in the formulation of some BVPs.
    
    However, it is possible to relax such BVPs from smooth bounded open connected domains $\Omega$ to measurable domains (cf. J.-P. Zolésio [7]).
    
    E.g. consider the homogeneous Dirichlet BVP \textbf{(3.25)}
    \begin{align*}
        -\Delta y = f \mbox{ in } \Omega,\ y = 0 \mbox{ on } \Gamma
    \end{align*}
    over a bounded open connected domain $\Omega$ with a boundary $\Gamma$ of class $C^1$ and associate with its solution $y = y(\Omega)$ the objective function and volume constraint \textbf{(3.26)}
    \begin{align*}
        J(\Omega) = \frac{1}{2}\int_\Omega |y - g|^2{\rm d}x,\ \int_\Omega {\rm d}x = \pi.
    \end{align*}
    There is a priori no reason to assume that an optimal (minimizing) domain $\Omega^*$ is of class $C^1$ or is connected.
    
    So the problem must be suitably relaxed to a large enough class of domains, which preserves the meaning of the underlying function spaces, the well-posedness of the original problem, and the volume.
    
    To extend problem (3.25)-(3.26) to Lebesgue measurable sets, we 1st have to make sense of the Sobolev space for measurable subsets $\Omega$ of $D$.
    
    \begin{theorem}
        Let $D$ be an open domain in $\mathbb{R}^N$. For any Lebesgue measurable subset $\Omega$ of $\mathbb{R}^N$, the spaces \textbf{(3.27)-(3.28)}
        \begin{align*}
            H_\bullet^1(\Omega;D) &:= \left\{\varphi\in H_0^1(D);(1 - \chi_\Omega)\nabla\varphi = 0 \mbox{ a.e. in } D\right\},\\
            H_\diamond^1(\Omega;D) &:= \left\{\varphi\in H_0^1(D);(1 - \chi_\Omega)\varphi = 0 \mbox{ a.e. in } D\right\}
        \end{align*}
        are closed subspaces of $H_0^1(D)$ and hence Hilbert spaces.\footnote{Observe that for 2 open domains in the same equivalence class, the spaces defined by (3.27)-(3.28) coincide.
            
            Therefore their functions do not see cracks in the underlying domain.
            
            The case of cracks will be handled in Chap. 8 by capacity methods.} Similarly, for any $\chi\in\overline{\rm co}\operatorname{X}(D)$, \textbf{(3.29)-(3.30)}
        \begin{align*}
            H_\bullet^1(\chi;D) &:= \left\{\varphi\in H_0^1(D);(1 - \chi)\nabla\varphi = 0 \mbox{ a.e. in } D\right\},\\
            H_\diamond^1(\chi;D) &:= \left\{\varphi\in H_0^1(D);(1 - \chi)\varphi = 0 \mbox{ a.e. in } D\right\}
        \end{align*}
        are also closed subspaces of $H_0^1(D)$ and hence Hilbert spaces. Furthermore\footnote{From L. C. Evans and R. F. Gariepy [1, Thm. 4, p. 130], $\nabla\varphi = 0$ a.e. on $\{\varphi = 0\}$.}    
        \begin{align*}
            H_\diamond^1(\Omega;D)\subset H_\bullet^1(\Omega;D),\ H_\diamond^1(\chi;D)\subset H_\bullet^1(\chi;D).
        \end{align*}
    \end{theorem}
    Assuming that $D$ is bounded, the variational problems \textbf{(3.31)-(3.32)}
    \begin{align*}
        \mbox{find } y &= y(\Omega)\in H_\bullet^1(\Omega;D) \mbox{ s.t. } \forall\varphi\in H_\bullet^1(\Omega;D),\ \int_D \nabla y\cdot\nabla\varphi{\rm d}x = \int_D \chi_\Omega f\varphi{\rm d}x,\\
        \mbox{find } y &= y(\Omega)\in H_\diamond^1(\Omega;D) \mbox{ s.t. } \forall\varphi\in H_\diamond^1(\Omega;D),\ \int_D \nabla y\cdot\nabla\varphi{\rm d}x = \int_D \chi_\Omega f\varphi{\rm d}x
    \end{align*}
    now make sense and have unique solutions for measurable subsets $\Omega$ of $D$ (or even $\chi\in\overline{\rm co}\operatorname{X}(D)$), and the associated objective function \textbf{(3.33)}
    \begin{align*}
        J(\Omega) = h\left(\chi_\Omega,y(\Omega)\right),\ h(\chi,\varphi) := \frac{1}{2}\int_D \chi|\varphi - g|^2{\rm d}x
    \end{align*}
    can be minimized over all measurable subsets $\Omega$ of $D$ (or all $\chi\in\overline{\rm co}\operatorname{X}(D)$) with fixed measure $\operatorname{m}(\Omega) = \pi$.
    
    %
    The above problems are now well-posed, and their restriction to smooth bounded open connected domains coincides with the initial problem (3.25)-(3.26).
    
    Indeed, if $\Omega$ is a connected open domain with a boundary $\Gamma$ of class $C^1$, then $\Gamma$ has a zero Lebesgue measure and the definition (3.28) of $H_\bullet^1(\Omega;D)$ coincides with $H_\diamond^1(\Omega;D)$.
    
    Therefore, the problem specified by (3.31)-(3.33) is a well-defined extension of problem (3.25)-(3.26).
    
    In general, when $\Omega$ contains holes, the elements of $H_\bullet^1(\Omega;D)$ are not necessarily equal to zero on the boundary of each hole and can be equal to different constants from hole to hole, as in physical problems involving a potential.
    
    \begin{example}
        Let $D = (-2,2)$, $\Omega = (-2,-1)\cup(1,2)$, and $f = 1$ on $D$.
        
        The solution of (3.31) is given by
        \begin{equation*}
            \left\{\begin{split}
                -\frac{d^2y}{dx^2} &= 1 \mbox{ on } (-2,-1),\\
                y(-2) &= 0,\ \frac{dy}{dx}(-1) = 0,
            \end{split}\right.\
            \left\{\begin{split}
                -\frac{d^2y}{dx^2} &= 1 \mbox{ on } (1,2),\\
                y(2) &= 0,\ \frac{dy}{dx}(1) = 0,
            \end{split}\right.\ y = \frac{1}{2} \mbox{ on } [-1,1],
        \end{equation*}
        \begin{equation*}
            \Rightarrow y(x) = \left\{\begin{split}
                &-(x + 2)\frac{x}{2} &&\mbox{ in } (-2,-1),\\
                &\frac{1}{2} &&\mbox{ in } [-1,1].\\
                &-(x - 2)\frac{x}{2} &&\mbox{ in } (1,2).
            \end{split}\right.
        \end{equation*}
    \end{example}
    
    \begin{example}
        Let $D = B(0,2)$, the open ball of radius 2 centered in 0 in $\mathbb{R}^2$, $H = \overline{B(0,1)}$, $\Omega = {\rm C}_DH$, and $f = 1$.
        
        In polar coordinates the solution of (3.31) is given by
        \begin{equation*}
            \left\{\begin{split}
                &-\frac{1}{r}\frac{d}{dr}\left(r\frac{dy}{dr}\right) = 1 \mbox{ in } (1,2),\\
                &y(2) = 0,\ \frac{dy}{dr}(1) = 0,
            \end{split}\right. \mbox{ and } y(r) = \frac{1}{2}\log\frac{1}{2} + \frac{3}{4} \mbox{ in } [0,1],
        \end{equation*}
        or explicitly
        \begin{equation*}
            y(r) = \left\{\begin{split}
                &\frac{1}{2}\log\frac{r}{2} + 1 - \frac{r^2}{4} &&\mbox{ in } (1,2),\\
                &\frac{1}{2}\log\frac{1}{2} + \frac{3}{4} &&\mbox{ in } [0,1].
            \end{split}\right.
        \end{equation*}
    \end{example}
    The last example is a special case that retains the characteristics of the 1D example.
    
    In higher dimensions the normal derivative is not necessarily zero on the ``internal boundary'' of $\Omega$.
    
    \begin{example}
        Let $D = B(0,1)$ in $\mathbb{R}^2$, let $H$ be a bounded open connected hole in $D$ s.t. $\overline{H}\subset D$, and let $\Omega = {\rm C}_D\overline{H}$.
        
        Then it can be checked that the solution of (3.31) is of the form
        \begin{align*}
            -\Delta y = f \mbox{ in } \Omega,\ y = 0 \mbox{ on } \partial D,
        \end{align*}
        where the constant $c$ on $H$ is determined by the condition
        \begin{align*}
            \forall\varphi\in H_\bullet^1(\Omega;D),\ \int_{\partial\Omega} \partial_{\bf n}y\varphi{\rm d}\gamma = 0,
        \end{align*}
        or equivalently
        \begin{align*}
            \int_{\partial H} \partial_{\bf n}y{\rm d}\gamma = 0.
        \end{align*}
    \end{example}
    
    \begin{example}
        Let $D = (-2,2)\times(-2,2)$ and let $\Omega_1$ and $\Omega_2$ be 2 open squares in $G = (-1,1)\times(-1,1)$ s.t. $\overline{\Omega_1}\subset G$, $\overline{\Omega_2}\subset G$, and $\overline{\Omega_1}\cap\overline{\Omega_2} = \emptyset$.
        
        Define the domain as
        \begin{align*}
            \Omega = \Omega_0\cup\Omega_1\cup\Omega_2,\ \Omega_0 = D\backslash\overline{G}
        \end{align*}
        Then the solution of (3.31) is characterized by
        \begin{align*}
            -\Delta y = f \mbox{ in } \Omega,\ y = 0 \mbox{ on } \partial D,\ y = c \mbox{ on } G\backslash[\overline{\Omega}_1\cup\overline{\Omega}_2],
        \end{align*}
        where the constant $c$ is determined by the condition
        \begin{align*}
            \int_{\partial\Omega_0^{\rm int}} \partial_{\bf n}y{\rm d}\gamma + \int_{\partial\Omega_1} \partial_{\bf n}y{\rm d}\gamma + \int_{\partial\Omega_2} \partial_{\bf n}y{\rm d}\gamma = 0
        \end{align*}
        and $\partial\Omega_0^{\rm int} = \partial G$, the interior boundary of $\Omega_0$.
    \end{example}
\end{enumerate}

\paragraph{Some Compliance Problems with 2 Materials.}
\begin{enumerate}
    \item As a 1st example consider the optimal compliance problem where the optimization variable is the distribution of 2 materials with different physical characteristics within a fixed domain $D$.
    
    It cannot a priori be assumed that the 2 regions are separated by a smooth boundary and that each region is connected.
    
    The optimal solution may lead to a nonsmooth interface and even to the mixing of the 2 materials.
    
    This type of solution occurs in control or optimization problems over a bounded nonconvex subset of a function space.
    
    In general, their relaxed solution lies in the closed convex hull of that subset.
    
    In control theory, the phenomenon is known as \textit{chattering control}.
    
    To illustrate this approach it is best to consider a generic example.
    
    In Sect. 4.1, we consider a variation of the optimal design problem studied by J. Céa and K. Malanowski [1] in 1970 and then discuss its original version in Sect. 4.2.
    
    The variation of the problem is constructed in such a way that the form of the associated variational equation is similar to the one of Example 3.1, which provided a counterexample to the weak continuity of the solution.
    
    Yet the maximization of the minimum energy yields as solution a characteristic function in both cases.
    
    However, this is no longer true for the minimization of the same minimum energy as was shown by F. Murat and L. Tartar [1, 3].
    \item \textbf{Transmission Problem \& Compliance.} Let $D\subset\mathbb{R}^N$ be a bounded open domain with Lipschitzian boundary $\partial D$.
    
    Assume that the domain $D$ is partitioned into 2 subdomains $\Omega_1$ and $\Omega_2$ separated by a smooth boundary $\partial\Omega_1\cap\partial\Omega_2$ as illustrated in Fig. 5.3.
    
    \textsf{Fig. 5.3. Fixed domain $D$ and its partition into $\Omega_1$ and $\Omega_2$.}
    
    Domain $\Omega_1$ (resp., $\Omega_2$) is made up of a material characterized by a constant $k_1 > 0$ (resp., $k_2 > 0$).
    
    Let $y$ be the solution of the \textit{transmission problem} \textbf{(4.1)}
    \begin{equation*}
        \left\{\begin{split}
            -k_1\Delta y &= f \mbox{ in } \Omega_1,\ -k_2\Delta y = 0 \mbox{ in } \Omega_2,\\
            y &= 0 \mbox{ on } \partial D,\ k_1\partial_{{\bf n}_1}y + k_2\partial_{{\bf n}_2}y = 0 \mbox{ on } \partial\Omega_1\cap\partial\Omega_2,
        \end{split}\right.
    \end{equation*}
    where $n_1$ (resp., $n_2$) is the unit outward normal to $\Omega_1$ (resp., $\Omega_2$) and $f$ is a given function in $L^2(D)$.
    
    Our objective is to maximize the equivalent of the \textit{compliance} \textbf{(4.2)}
    \begin{align*}
        J(\Omega_1) = -\int_{\Omega_1} fy{\rm d}x
    \end{align*}
    over all domains $\Omega_1$ in $D$.
    
    In mechanics the compliance is associated with the total work of the body forces $f$.
    
    %
    Denote by $\Omega$ the domain $\Omega_1$.
    
    By complementarity $\Omega_2 = {\rm C}_D\overline{\Omega}$ and let $\chi = \chi_\Omega$.
    
    Problem (4.1) can be rewritten in the following variational form: \textbf{(4.3)}
    \begin{align*}
        \mbox{find } y = y(\chi)\in H_0^1(D) \mbox{ s.t. } \forall\varphi\in H_0^1(D),\ \int_D \left[k_1\chi + k_2(1 - \chi)\right]\nabla y\cdot\nabla\varphi{\rm d}x = \int_D \chi f\varphi{\rm d}x.
    \end{align*}
    For $k_1 > 0$ and $k_2 > 0$ and all $\chi$ in $\operatorname{X}(D)$ \textbf{(4.4)}
    \begin{align*}
        k(x) &:= k_1\chi(x) + k_2\left(1 - \chi(x)\right),\\
        0 < \min\{k_1,k_2\}&\le k(x)\le\max\{k_1,k_2\} \mbox{ a.e. in } D.
    \end{align*}
    By the Lax-Milgram theorem, the variational equation (4.3) still makes sense and has a unique solution $y = y(\chi)$ in $H_0^1(D)$ that coincides with the solution of the BVP \textbf{(4.5)}
    \begin{align*}
        -\nabla\cdot\left(k\nabla y\right) = \chi f \mbox{ in } D,\ y = 0 \mbox{ on } \partial D.
    \end{align*}
    Notice that the high-order term has the same form as the term in Example 3.1.
    
    As for the objective function, it can be rewritten as \textbf{(4.6)}
    \begin{align*}
        J(\chi) = -\int_D \chi fy(\chi){\rm d}x.
    \end{align*}
    Thus the IBVP (4.1) has been transformed into the variational problem (4.3), and the initial objective function (4.2) into (4.6).
    
    Both make sense for $\chi$ in $\operatorname{X}(D)$ and even in $\overline{\rm co}\operatorname{X}(D)$.
    
    The family of characteristic functions $\operatorname{X}(D)$ and its closed convex hull $\overline{\rm co}\operatorname{X}(D)$ have been defined and characterized in (3.2) and (3.3).
    
    %
    The objective function (4.6) can be further rewritten as a minimum \textbf{(4.7)}
    \begin{align*}
        J(\chi) = \min_{\varphi\in H_0^1(D)} E(\chi,\varphi)
    \end{align*}
    for the \textit{energy function} \textbf{(4.8)}
    \begin{align*}
        E\left(\chi,\varphi\right) := \int_D (k_1\chi + k_2(1 - \chi))|\nabla\varphi|^2 - 2\chi f\varphi{\rm d}x
    \end{align*}
    associated with the variational problem (4.3).
    
    The initial optimal design problem becomes a max-min problem \textbf{(4.9)}
    \begin{align*}
        \max_{\chi\in\operatorname{X}(D)} J(\chi) = \max_{\chi\in\operatorname{X}(D)} \min_{\varphi\in H_0^1(D)} E\left(\chi,\varphi\right),
    \end{align*}
    where $\operatorname{X}(D)$ can be considered as a subset of $L^p(D)$ for some $p$, $1\le p < \infty$.
    
    %
    This problem can also be relaxed to functions $\chi$ with value in $[0,1]$ \textbf{(4.10)}
    \begin{align*}
        \max_{\chi\in\overline{\rm co}\operatorname{X}(D)} J(\chi) = \max_{\chi\in\overline{\rm co}\operatorname{X}(D)} \min_{\varphi\in H_0^1(D)} E\left(\chi,\varphi\right).
    \end{align*}
    We shall refer to this problem as the \textit{relaxed problem}.
    
    %
    These formulations have been introduced by J. Céa and K. Malanowski [1], who used the variable $k(x)$ and the following \textit{equality constraint} on its integral:
    \begin{align*}
        \int_D k(x){\rm d}x = \gamma \mbox{ or } \int_D \chi(x){\rm d}x = \frac{\gamma - k_2\operatorname{m}(D)}{k_1 - k_2}
    \end{align*}
    for some appropriate $\gamma > 0$.
    
    Moreover the force $f$ in (4.1)-(4.3) was exerted everywhere in $D$ and not only in $\Omega$.
    
    So the objective function (4.2)-(4.6) was the integral of $fy$ over all of $D$.
    
    We shall see in Sect. 4.2 that the fact that the support of $f$ is $\Omega$ or all of $D$ does not affect the nature of the results.
    
    %
    In both maximization problems (4.1)-(4.2) and (4.3)-(4.6) it is necessary to introduce a \textit{volume constraint} in order to avoid a trivial solution.
    
    Notice that by using (4.3) with $\varphi = y(\chi)$ the objective function (4.6) becomes
    \begin{align*}
        J(\chi) = -\int_D \left(k_1\chi + k_2(1 - \chi)\right)|\nabla y(\chi)|^2{\rm d}x\le 0.
    \end{align*}
    So maximizing $J(\chi)$ is equivalent to minimizing the integral of $k(x)|\nabla y|^2$.
    
    Therefore, $\chi = 0$, (only material $k_2$) is a maximizer since the corresponding solution of (4.3) is $y = 0$.
    
    In order to eliminate this situation we introduce the following constraint on the volume of material $k_1$: \textbf{(4.11)}
    \begin{align*}
        \int_D \chi{\rm d}x\ge\alpha > 0
    \end{align*}
    for some $\alpha$, $0 < \alpha\le\operatorname{m}(D)$.
    
    The case $\alpha = \operatorname{m}(D)$ yields the unique solution $\chi = 1$ (only material $k_1$).
    
    So we can further assume that $\alpha < \operatorname{m}(D)$.
    
    %
    For $0 < \alpha < \operatorname{m}(D)$, the optimal design problem becomes \textbf{(4.12)}
    \begin{align*}
        \max_{\chi\in\operatorname{X}(D),\,\int_D \chi{\rm d}x\ge\alpha} J(\chi) = \max_{\chi\in\operatorname{X}(D),\,\int_D \chi{\rm d}x\ge\alpha} \min_{\varphi\in H_0^1(D)} E\left(\chi,\varphi\right)
    \end{align*}
    and its relaxed version \textbf{(4.13)}:
    \begin{align*}
        \max_{\chi\in\overline{\rm co}\operatorname{X}(D),\,\int_D \chi{\rm d}x\ge\alpha} J(\chi) = \max_{\chi\in\overline{\rm co}\operatorname{X}(D),\,\int_D \chi{\rm d}x\ge\alpha} \min_{\varphi\in H_0^1(D)} E\left(\chi,\varphi\right).
    \end{align*}
    We shall now show that problem (4.13) has a unique solution $\chi^*$ in $\overline{\rm co}\operatorname{X}(D)$ and that $\chi^*$ is a characteristic function, $\chi^*\in\operatorname{X}(D)$, for which the inequality constraint is saturated: \textbf{(4.14)}
    \begin{align*}
        \chi^*\in\operatorname{X}(D) \mbox{ and } \int_D \chi^*{\rm d}x = \alpha.
    \end{align*}
    
    At this juncture it is advantageous to incorporate the volume inequality constraint into the problem formulation by introducing a Lagrange multiplier $\lambda\ge 0$.
    
    The formulation of the relaxed problem becomes \textbf{(4.15)}
    \begin{align*}
        \max_{\chi\in\overline{\rm co}\operatorname{X}(D)} \min_{\varphi\in H_0^1(D),\,\lambda\ge 0} G\left(\chi,\varphi,\lambda\right),\ G\left(\chi,\varphi,\lambda\right) := E\left(\chi,\varphi\right) + \lambda\left(\int_D \chi{\rm d}x - \alpha\right)
    \end{align*}
    and $E(\chi,\varphi)$ is the energy function given in (4.7).
    
    %
    We 1st establish the existence of saddle point solutions to problem (4.15).
    
    We use a general result in I. Ekeland and R. Temam [1, Prop. 2.4, p. 164].
    
    The set $\overline{\rm co}\operatorname{X}(D)$ is a nonempty bounded closed convex subset of $L^2(D)$ and the set $H_0^1(D)\times[\mathbb{R}^+\cup\{0\}]$ is trivially closed and convex.
    
    The function $G$ is concave-convex with the following properties: \textbf{(4.16)-(4.17)}
    \begin{align*}
        &\forall\chi\in\overline{\rm co}\operatorname{X}(D),\ (\varphi,\lambda)\mapsto G(\chi,\varphi,\lambda) \mbox{ is convex, continuous, and } \exists\chi_0\in\overline{\rm co}\operatorname{X}(D) \mbox{ s.t. } \lim_{\|\varphi\|_{H_0^1} + |\lambda|\to\infty} G\left(\chi_0,\varphi,\lambda\right) = +\infty;\\
        &\forall\varphi\in H_0^1(D),\ \forall\lambda\ge 0, \mbox{ the map } \chi\mapsto G(\chi,\varphi,\lambda) \mbox{ is affine and continuous for } L^2(D)\mbox{-strong}.
    \end{align*}
    For the 1st condition recall that $0 < \alpha < \operatorname{m}(D)$ and pick $\chi_0 = 1$ on $D$.
    
    To check the 2nd condition pick any sequence $\{\chi_n\}$ in $\overline{\rm co}\operatorname{X}(D)$ which converges to some $\chi$ in $L^2(D)$-strong.
    
    Then the sequence also converges in $L^2(D)$-weak and by Lemma 3.1 in $L^\infty(D)$-weak${}^\star$.
    
    Hence $G\left(\chi_n,\varphi,\lambda\right)$ converges to $G(\chi,\varphi,\lambda)$.
    
    %
    The set of saddle points $\left(\hat{\chi},y,\hat{\lambda}\right)$ is of the form $X\times Y\subset\overline{\rm co}\operatorname{X}(D)\times\{H_0^1(D)\times[\mathbb{R}^+\cup\{0\}]\}$ and is completely characterized by the following variational equation and inequalities (cf. I. Ekeland and R. Temam [1, Prop. 1.6, p. 157]): \textbf{(4.18)-(4.19)}
    \begin{align*}
        \forall\varphi\in H_0^1(D),\ \int_D \left[k_2 + \left(k_1 - k_2\right)\hat{\chi}\right]\nabla y\cdot\nabla\varphi - \hat{\chi}f\varphi{\rm d}x &= 0,\\
        \forall\chi\in\overline{\rm co}\operatorname{X}(D),\ \int_D \left[(k_1 - k_2)|\nabla y|^2 - 2fy + \hat{\lambda}\right]\left(\chi - \hat{\chi}\right){\rm d}x&\le 0,\\
        \left(\int_D \hat{\chi}{\rm d}x - \alpha\right)\hat{\lambda} = 0,\ \int_D \hat{\chi}{\rm d}x - \alpha\ge 0,\ \hat{\lambda}&\ge 0.
    \end{align*}
    But for each $\chi\in\overline{\rm co}\operatorname{X}(D)$ there exists a unique $y(\chi)$ solution of (4.18) and then
    \begin{align*}
        \forall\hat{\chi}\in X,\ \{\hat{\chi}\}\times Y\subset\{\hat{\chi}\}\times\{\{y(\hat{\chi})\}\times[\mathbb{R}^+\cup\{0\}]\}.
    \end{align*} 
    Therefore, $y(\hat{\chi})$ is independent of $\hat{\chi}\in X$, i.e., $Y = \{y\}\times\Lambda$ and
    \begin{align*}
        \forall\hat{\chi}\in X,\ \hat{\lambda}\in\Lambda,\ y(\hat{\chi}) = y.
    \end{align*}
    %
    For each $\hat{\lambda}$, inequality (4.19) is completely equivalent to the following characterization of the maximizer $\hat{\chi}\in X$: \textbf{(4.21)}
    \begin{equation*}
        \hat{\chi}(x) = \left\{\begin{split}
            &1,&&\mbox{ if } (k_1 - k_2)|\nabla y|^2 - 2fy + \hat{\lambda} > 0,\\
            &\in[0,1],&&\mbox{ if } (k_1 - k_2)|\nabla y|^2 - 2fy + \hat{\lambda} = 0,\\
            &0,&&\mbox{ if } (k_1 - k_2)|\nabla y|^2 - 2fy + \hat{\lambda} < 0.\\
        \end{split}\right.
    \end{equation*}
    Associate with an arbitrary $\lambda\ge 0$ the sets \textbf{(4.22)}
    \begin{align*}
        D_+(\lambda) &= \left\{x\in D;(k_1 - k_2)|\nabla y|^2 - 2fy + \lambda > 0\right\},\\
        D_0(\lambda) &= \left\{x\in D;(k_1 - k_2)|\nabla y|^2 - 2fy + \lambda = 0\right\}.
    \end{align*}
    In order to complete the characterization of the optimal triplets, we need the following general result.
    
    \begin{lemma}
        Consider a function \textbf{(4.23)}
        \begin{align*}
            G:A\times B\to\mathbb{R}
        \end{align*}
        for some sets $A$ and $B$. Define \textbf{(4.24)-(4.25)}
        \begin{align*}
            g &:= \inf_{x\in A}\sup_{y\in B} G(x,y),\ A_0 := \left\{x\in A;\sup_{y\in B} G(x,y) = g\right\},\\
            h &:= \sup_{y\in B}\inf_{x\in A} G(x,y),\ B_0 := \left\{y\in B;\inf_{x\in A} G(x,y) = h\right\}.
        \end{align*}
        When $g = h$ the set of saddle points (possibly empty) will be denoted by \textbf{(4.26)}
        \begin{align*}
            S := \left\{(x,y)\in A\times B;g = G(x,y) = h\right\}.
        \end{align*}
        Then the following hold.
        \begin{itemize}
            \item[(i)] In general $h\le g$ and \textbf{(4.27)}
            \begin{align*}
                \forall(x_0,y_0)\in A_0\times B_0,\ h\le G(x_0,y_0)\le g.
            \end{align*}
            \item[(ii)] If $h = g$, then $S = A_0\times B_0$.
        \end{itemize}
    \end{lemma}
    Associate with an arbitrary solution $(\hat{\chi},y,\hat{\lambda})$, the characteristic function \textbf{(4.29)}
    \begin{equation*}
        \chi_{\hat{\lambda}} = \left\{\begin{split}
            &\chi_{D_+(\hat{\lambda})},&&\mbox{ if } D_+(\hat{\lambda})\ne\emptyset,\\
            &0,&&\mbox{ if } D_+(\hat{\lambda}) = \emptyset.
        \end{split}\right.
    \end{equation*}
    Then from (4.21)
    \begin{align*}
        \hat{\chi}\left\{(k_1 - k_2)|\nabla y|^2 - 2fy + \hat{\lambda}\right\} = \chi_{\hat{\lambda}}\left\{(k_1 - k_2)|\nabla y|^2 - 2fy + \hat{\lambda}\right\} \mbox{ a.e. in } D,
    \end{align*}
    and
    \begin{align*}
        G\left(\hat{\chi},y,\hat{\lambda}\right) &= \int_D \left[\hat{\chi}k_1 + \left(1 - \hat{\chi}\right)k_2\right]|\nabla y|^2 - 2\hat{\chi}fy{\rm d}x + \hat{\lambda}\left(\int_D \hat{\chi}{\rm d}x - \alpha\right)\\
        &= \int_D k_2|\nabla y|^2 + \hat{\chi}\left[(k_1 - k_2)|\nabla y|^2 - 2fy + \hat{\lambda}\right]{\rm d}x - \hat{\lambda}\alpha\\
        &= \int_D k_2|\nabla y|^2 + \chi_{\hat{\lambda}}\left[(k_1 - k_2)|\nabla y|^2 - 2fy + \hat{\lambda}\right]{\rm d}x - \hat{\lambda}\alpha\\
        &= G\left(\chi_{\hat{\lambda}},y,\hat{\lambda}\right).
    \end{align*}
    From Lemma 4.1, $\left(\chi_{\hat{\lambda}},y,\hat{\lambda}\right)$ is also a saddle point.
    
    So there exists a maximizer $\chi_{\hat{\lambda}}\in\operatorname{X}(D)$ that is a characteristic function and necessarily
    \begin{align*}
        \max_{\chi\in\operatorname{X}(D)}\min_{\varphi\in H_0^1(D),\,\lambda\ge 0} G(\chi,\varphi,\lambda) = \max_{\chi\in\overline{\rm co}\operatorname{X}(D)}\min_{\varphi\in H_0^1(D),\,\lambda\ge 0} G(\chi,\varphi,\lambda).
    \end{align*}
    But we can show more than that.
    
    If there exists $\hat{\lambda}\in\Lambda$ s.t. $\hat{\lambda} > 0$, then by construction $\hat{\chi}\ge\chi_{\hat{\lambda}}$ and by (4.20)
    \begin{align*}
        \alpha = \int_D \hat{\chi}{\rm d}x\ge\int_D \chi_{\hat{\lambda}}{\rm d}x = \alpha.
    \end{align*}
    Since $\hat{\chi} = \chi_{\hat{\lambda}} = 1$ a.e. in $D_+(\hat{\lambda})$, then $\hat{\chi} = \chi_{\hat{\lambda}}$.
    
    But, always by construction, $\chi_{\hat{\lambda}}$ is independent of $\hat{\chi}$.
    
    Thus the maximizer is unique, it is a characteristic function, and its integral is equal to $\alpha$.
    
    %
    The case $\Lambda = \{0\}$ is a degenerate one.
    
    Set $\varphi = y$ in (4.18) and regroup the terms as follows:
    \begin{align*}
        0 = \int_D \left(k_2 + (k_1 - k_2)\hat{\chi}\right)|\nabla y|^2 - fy\hat{\chi}{\rm d}x = \int_D \left(k_2 + \frac{k_1 - k_2}{2}\hat{\chi}\right)|\nabla y|^2{\rm d}x + \int_D \left(\frac{k_ 1 - k_2}{2}\hat{\chi}|\nabla y|^2 - fy\right)\hat{\chi}{\rm d}x.
    \end{align*}
    The integrand of the first integral is positive.
    
    From the characterization of $\hat{\chi}$, the integrand of the 2nd one is also positive.
    
    Hence they are both zero a.e. in $D$.
    
    As a result
    \begin{align*}
        \operatorname{m}\left(D_+(0)\right) = 0 \mbox{ and } \nabla y = 0.
    \end{align*}
    Therefore $y = 0$ in $D$.
    
    Hence the saddle points are of the general form $(\hat{\chi},y,\hat{\lambda}) = (\hat{\chi},0,0)$, $\hat{\chi}\in X$.
    
    In particular, $D_+(0) = \emptyset$ and from our previous considerations $\chi_\emptyset$; i.e., $\hat{\chi} = 0$ is a solution.
    
    But this is impossible since $\int_D \hat{\chi}{\rm d}x\ge\alpha > 0$.
    
    Therefore $\hat{\lambda} = 0$ cannot occur.
    
    %
    In conclusion, there exists a (unique for $\alpha > 0$) maximizer $\chi^*$ in $\overline{\rm co}\operatorname{X}(D)$ that is in fact a characteristic function, and necessarily \textbf{(4.30)}
    \begin{align*}
        \max_{\chi\in\operatorname{X}(D),\,\int_D \chi{\rm d}x\ge\alpha}\min_{\varphi\in H_0^1(D)} E(\chi,\varphi) = \max_{\chi\in\overline{\rm co}\operatorname{X}(D),\,\int_D \chi{\rm d}x\ge\alpha}\min_{\varphi\in H_0^1(D)} E(\chi,\varphi)
    \end{align*}
    Moreover, for $0 < \alpha\le\operatorname{m}(D)$, \textbf{(4.31)}
    \begin{align*}
        \int_D \chi^*{\rm d}x = \alpha,
    \end{align*}
    and $\chi^*$ is the unique solution of the problem with an equality constraint: \textbf{(4.32)}
    \begin{align*}
        \max_{\chi\in\operatorname{X}(D),\,\int_D \chi{\rm d}x = \alpha}\min_{\varphi\in H_0^1(D)} E(\chi,\varphi) = \max_{\chi\in\overline{\rm co}\operatorname{X}(D),\,\int_D \chi{\rm d}x\ge\alpha}\min_{\varphi\in H_0^1(D)} E(\chi,\varphi).
    \end{align*}
    As a numerical illustration of the theory consider (4.1) over the diamond-shaped domain \textbf{(4.33)}
    \begin{align*}
        D = \left\{(x,y);|x| + |y| < 1\right\}
    \end{align*}
    with the function $f$ of Fig. 5.4.
    
    \textsf{Fig. 5.4. The function $f(x,y) = 56\left(1 - |x| - |y|\right)^6$.}
    
    This function has a sharp peak in $(0,0)$ which has been scaled down in the picture.
    
    The variational form (4.3) of the BVP was approximated by continuous piecewise linear finite elements on each triangle, and the function $\chi$ by a piecewise constant function on each triangle.
    
    The constant on each triangle was constrained to lie between 0 and 1 together with the global constraint on its integral over the whole domain $D$.
    
    Fig. 5.5 shows the optimal partition (the domain has been rotated by 45 degrees to save space).
    
    \textsf{Fig. 5.5. Optimal distribution and isotherms with $k_1 = 2$ (black) and $k_2 = 1$ (white) for the problem of Sect. 4.1.}
    
    The grey triangles correspond to the region $D_0(\hat{\lambda}_m)$, where $\hat{\chi}\in[0,1]$.
    
    The presence of this grey zone in the approximated problem is due to the fact that equality for the total area where $\hat{\chi} = 1$ could not be exactly achieved with the chosen triangulation of the domain.
    
    Thus the problem had to adjust the value of $\hat{\chi}$ between 0 and 1 in a few triangles in order to achieve equality for the integral of $\hat{\chi}$.
    
    For this example the Lagrange multiplier associated with the problem is strictly positive.
    \item \textbf{The Original Problem of Céa and Malanowski.} Now put the force $f$ everywhere in the fixed domain $D$.
    
    The same technique and the same conclusion can be drawn: there exists at least 1 maximizer of the compliance, which is a characteristic function.
    
    For completeness we give the main elements below.
    
    %
    Fix the bounded open Lipschitzian domain $D$ in $\mathbb{R}^N$ and let $\Omega$ be a smooth subset of $D$.
    
    Let $y$ be the solution of the \textit{transmission problem} \textbf{(4.34)}
    \begin{equation*}
        \left\{\begin{split}
            -k_1\Delta y &= f \mbox{ in } \Omega,\ -k_2\Delta y = f \mbox{ in } D\backslash\overline{\Omega},\\
            y &= 0 \mbox{ on } \partial D,\\
            k_1\partial_{{\bf n}_1}y + k_2\partial_{{\bf n}_2}y &= 0 \mbox{ on } \partial\Omega\cap D,
        \end{split}\right.
    \end{equation*}
    where ${\bf n}_1$ (resp., ${\bf n}_2$) is the unit outward normal to $\Omega$ (resp., ${\rm C}_D\overline{\Omega}$) and $f$ is a given function in $L^2(D)$.
    
    Again problem (4.34) can be reformulated in terms of the characteristic function $\chi = \chi_\Omega$: \textbf{(4.35)}
    \begin{align*}
        \mbox{find } y = y(\chi)\in H_0^1(D) \mbox{ s.t. } \forall\varphi\in H_0^1(D),\ \int_D \left[k_1\chi + k_2(1 - \chi)\right] \nabla y\cdot\nabla\varphi{\rm d}x = \int_D f\varphi{\rm d}x,
    \end{align*}
    with the objective function \textbf{(4.36)}
    \begin{align*}
        J(\chi) = -\int_D fy(\chi){\rm d}x
    \end{align*}
    to be maximized over all $\chi\in\operatorname{X}(D)$.
    
    As in the previous case, the function $J(\chi)$ can be rewritten as the minimum of the energy function \textbf{(4.37)}
    \begin{align*}
        E\left(\chi,\varphi\right) := \int_D \left[k_1\chi + k_2(1 - \chi)\right]|\nabla\varphi|^2 - 2f\varphi{\rm d}x
    \end{align*}
    over $H_0^1(D)$, \textbf{(4.38)}
    \begin{align*}
        J(\chi) = \min_{\varphi\in H_0^1(D)} E(\chi,\varphi),
    \end{align*}
    and we have the relaxed max-min problem \textbf{(4.39)}
    \begin{align*}
        \max_{\chi\in\overline{\rm co}\operatorname{X}(D)} \min_{\varphi\in H_0^1(D)} E(\chi,\varphi).
    \end{align*}
    
    Without constraint on the integral of $\chi$, the problem is trivial and $\chi = 1$ (resp., $\chi = 0$) if $k_1 > k_2$ (resp., $k_2 > k_1$).
    
    In other words it is optimal to use only the strong material.
    
    In order to make the problem nontrivial assume that the strong material is $k_1$, i.e., $k_1 > k_2$, and put an upper bound on the volume of material $k_1$ which occupies the part $\Omega$ of $D$: \textbf{(4.40)}
    \begin{align*}
        \int_D \chi{\rm d}x\le\alpha,\ 0 < \alpha < \operatorname{m}(D).
    \end{align*}
    The case $\alpha = 0$ trivially yields $\chi = 0$.
    
    Under assumption (4.40) the case $\chi = 1$ with only the strong material $k_1$ is no longer admissible.
    
    Thus we consider for $0 < \alpha < \operatorname{m}(D)$ the problem \textbf{(4.41)}
    \begin{align*}
        \max_{\chi\in\overline{\rm co}\operatorname{X}(D),\,\int_D \chi{\rm d}x\le\alpha} \min_{\varphi\in H_0^1(D)} E(\chi,\varphi).
    \end{align*}
    We shall now show that problem (4.41) has a unique solution $\chi^*$ in $\overline{\rm co}\operatorname{X}(D)$ and that in fact $\chi^*$ is a characteristic function for which the inequality constraint is saturated: \textbf{(4.42)}
    \begin{align*}
        \int_D \chi^*{\rm d}x = \alpha.
    \end{align*}
    This solves the original problem of Céa and Malanowski with the equality constraint: \textbf{(4.43)}
    \begin{align*}
        \max_{\chi\in\operatorname{X}(D),\,\int_D \chi{\rm d}x = \alpha} \min_{\varphi\in H_0^1(D)} E(\chi,\varphi).
    \end{align*}
    
    As in Sect. 4.1 it is convenient to reformulate the problem with a Lagrange multiplier $\lambda\ge 0$ for the constraint inequality (4.40): \textbf{(4.44)}
    \begin{align*}
        \max_{\chi\in\operatorname{X}(D)} \min_{\varphi\in H_0^1(D),\,\lambda\ge 0} G(\chi,\varphi,\lambda),\ G(\chi,\varphi,\lambda) = E(\chi,\varphi) - \lambda\left(\int_D \chi{\rm d}x - \alpha\right).
    \end{align*}
    We then relax the problem to $\overline{\rm co}\operatorname{X}(D)$: \textbf{(4.45)}
    \begin{align*}
        \max_{\chi\in\overline{\rm co}\operatorname{X}(D)} \min_{\varphi\in H_0^1(D),\,\lambda\ge 0} G(\chi,\varphi,\lambda).
    \end{align*}
    By the same arguments as the ones used in Sect. 4.1 we have the existence of saddle points $(\hat{\chi},y,\hat{\lambda})$ which are completely characterized by \textbf{(4.46)-(4.48)}
    \begin{align*}
        \forall\varphi\in H_0^1(D),\ \int_D \left[k_2 + (k_1 - k_2)\hat{\chi}\right]\nabla y\cdot\nabla\varphi - f\varphi{\rm d}x &= 0,\\
        \forall\chi\in\overline{\rm co}\operatorname{X}(D),\ \int_D \left[(k_1 - k_2)|\nabla y|^2 - \hat{\lambda}\right]\left(\chi - \hat{\chi}\right){\rm d}x&\le 0,\\
        \left(\int_D \hat{\chi}{\rm d}x - \alpha\right)\hat{\lambda} = 0,\ \int_D \hat{\chi}{\rm d}x - \alpha\le 0,\ \hat{\lambda}&\ge 0.
    \end{align*}
    As before $y$ is unique and the set of saddle points o $G$ is the closed convex set
    \begin{align*}
        X\times\left\{\{y\}\times\Lambda\right\}\subset\operatorname{X}(D)\times\left\{\{y\}\times\{\lambda;\lambda\ge 0\}\right\}.
    \end{align*}
    The closed convex set $\Lambda$ has a minimal element $\hat{\lambda}_m\ge 0$.
    
    %
    For each $\lambda$, each maximizer $\hat{\chi}\in X$ is necessarily of the form \textbf{(4.49)}
    \begin{equation*}
        \hat{\chi}(x) = \left\{\begin{split}
            &1,&&\mbox{ if } (k_1 - k_2)|\nabla y|^2 - \hat{\lambda} > 0,\\
            &\in[0,1],&&\mbox{ if } (k_1 - k_2)|\nabla y|^2 - \hat{\lambda} = 0,\\
            &0,&&\mbox{ if } (k_1 - k_2)|\nabla y|^2 - \hat{\lambda} < 0.
        \end{split}\right.
    \end{equation*}
    Associate with each $\lambda > 0$ the sets \textbf{(4.50)-(4.52)}
    \begin{align*}
        D_+(\lambda) &= \left\{x\in D;(k_1 - k_2)|\nabla y|^2 - \lambda > 0\right\},\\
        D_0(\lambda) &= \left\{x\in D;(k_1 - k_2)|\nabla y|^2 - \lambda = 0\right\},\\
        D_-(\lambda) &= \left\{x\in D;(k_1 - k_2)|\nabla y|^2 - \lambda < 0\right\}.
    \end{align*}
    Define the characteristic function $\chi_m = \chi_{D_+(\lambda_m)}$.
    
    By construction all $\hat{\chi}\in X$ are of the form \textbf{(4.53)}
    \begin{align*}
        \chi_m\le\hat{\chi},\ \int_D \hat{\chi}{\rm d}x\le\alpha.
    \end{align*}
    Again it is easy to show that $(\chi_m,y,\hat{\lambda}_m)$ is also a saddle point of $G$.
    
    Therefore, we have a maximizer $\chi_m\in\operatorname{X}(D)$ over $\overline{\rm co}\operatorname{X}(D)$ which is a characteristic function and
    \begin{align*}
        \int_D \chi_m{\rm d}x\le\int_D \hat{\chi}{\rm d}x\le\alpha.
    \end{align*}
    If there exists $\hat{\lambda} > 0$ in $\Lambda$, then for all $\hat{\chi}\in X$
    \begin{align*}
        \int_D \chi_m{\rm d}x = \alpha = \int_D \hat{\chi}{\rm d}x.
    \end{align*}
    As a result the maximizer $\chi_m$ is unique, it is a characteristic function, and its integral is equal to $\alpha$.
    
    %
    The case $\Lambda = \{0\}$ cannot occur since the triplet $(1,y_1,0)$ would be a saddle point, where $y_1$ is the solution of the variational equation (4.34) for $\chi = 1$.
    
    To see this, 1st observe that for $k_1 > k_2$
    \begin{align*}
        \forall\chi,\varphi,\ G(\chi,\varphi,0) = E(\chi,\varphi)\le E(1,\varphi) = G(1,\varphi,0).
    \end{align*}
    As a result
    \begin{align*}
        &\inf_{\varphi,\lambda} G(\chi,\varphi,\lambda)\le\inf_\varphi G(\chi,\varphi,0)\le\inf_\varphi G(1,\varphi,0),\\
        &\inf_\varphi G(1,\varphi,0) = G(1,y_1,0)\le\sup_\chi\inf_\varphi G(\chi,\varphi,0)\\
        \Rightarrow&\sup_\chi\inf_{\varphi,\lambda} G(\chi,\varphi,\lambda)\le\sup_\chi\inf_\varphi G(\chi,\varphi,0) = G(1,y_1,0).
    \end{align*}
    But we know that there exists a unique $y\in H_0^1(D)$ s.t.
    \begin{align*}
        \sup_\chi\inf_\varphi G(\chi,\varphi,0)\le\sup_\chi G(\chi,y,0) = \inf_{\varphi,\lambda}\sup_\chi G(\chi,\varphi,\lambda).
    \end{align*}
    So from the above 2 inequalities,
    \begin{align*}
        \sup_\chi\inf_{\varphi,\lambda} G(\chi,\varphi,\lambda) = G(1,y_1,0) = \inf_{\varphi,\lambda}\sup_\chi G(\chi,\varphi,\lambda),
    \end{align*}
    $(1,y_1,0)$ is a saddle point of $G$, and
    \begin{align*}
        \alpha\ge\int_D \chi{\rm d}x = \operatorname{m}(D).
    \end{align*}
    This contradicts the fact that $\alpha < \operatorname{m}(D)$.
    
    %
    In conclusion in all cases there exists a (unique when $0\le\alpha < \operatorname{m}(D)$) maximizer $\chi^*$ in $\overline{\rm co}\operatorname{X}(D)$, which is in fact a characteristic function, and
    \begin{align*}
        \max_{\chi\in\operatorname{X}(D),\,\int_D \chi{\rm d}x\ge\alpha}\min_{\varphi\in H_0^1(D)} E(\chi,\varphi) = \max_{\chi\in\overline{\rm co}\operatorname{X}(D),\,\int_D \chi{\rm d}x\ge\alpha}\min_{\varphi\in H_0^1(D)} E(\chi,\varphi).
    \end{align*}
    Moreover, for $0\le\alpha < \operatorname{m}(D)$
    \begin{align*}
        \int_D \chi^*{\rm d}x = \alpha.
    \end{align*}
    This is precisely \textit{the} solution of the original problem of Céa and Malanowski and
    \begin{align*}
        \max_{\chi\in\operatorname{X}(D),\,\int_D \chi{\rm d}x = \alpha}\min_{\varphi\in H_0^1(D)} E(\chi,\varphi) = \max_{\chi\in\overline{\rm co}\operatorname{X}(D),\,\int_D \chi{\rm d}x\ge\alpha}\min_{\varphi\in H_0^1(D)} E(\chi,\varphi).
    \end{align*}
    Again, as an illustration of the theoretical results, consider (4.34) over the diamond-shaped domain $D$ defined in (4.33) with the function $f$ of Fig. 5.4 in Sect. 4.1.
    
    %
    The variational form (4.35) of the BVP was approximated in the same way as the variational form (4.3) of (4.1) in Sect. 4.1.
    
    Fig. 5.6 shows the optimal partition of the domain (rotated 45 degrees).
    
    The grey triangles correspond to the region $D_0(\hat{\lambda}_m)$, where $\hat{\chi}\in[0,1]$.
    
    The black region corresponds to the points where $|\nabla y|^2 > \frac{\hat{\lambda}_m}{k_1 - k_2}$ and the white region to the ones where $|\nabla y|^2 < \frac{\hat{\lambda}_m}{k_1 - k_2}$, as can be readily seen in Fig. 5.6.
    
    \textsf{Fig. 5.6. Optimal distribution and isotherms with $k_1 = 2$ (black) and $k_2 = 1$ (white) for the problem of Céa and Malanowski.}
    
    For this example the Lagrange multiplier associated with the problem is strictly positive.
    
    It is interesting to compare this computation with the one of Fig. 5.5 in Sect. 4.1, where the support of the force was restricted to $\Omega$.
    
    \begin{remark}
        The formulation and the results remain true if the generic variational equation is replaced by a variational inequality (unilateral problem)
        \begin{align*}
            \max_{\chi\in\operatorname{X}(D),\,\int_D \chi{\rm d}x = \alpha} \min_{\varphi\in\left\{\psi\in H_0^1(D);\psi\ge 0 \mbox{ in } D\right\}} E(\chi,\varphi).
        \end{align*}
    \end{remark}
    \item \textbf{Relaxation and Homogenization}
    In Sects. 4.1 and 4.2 the possible homogenization phenomenon predicted in Example 3.1 of Sect. 3.2 did not take place, and in both examples the solution was a characteristic function.
    
    Yet, it occurs when the maximization is changed to a minimization in the problem of J. Céa and K. Malanowski [1] of Sect. 4.2:
    \begin{align*}
        \inf_{\chi\in\operatorname{X}(D),\,\int_D \chi{\rm d}x = \alpha} \min_{\varphi\in H_0^1(D)} E(\chi,\varphi).
    \end{align*}
    In 1985 F. Murat and L. Tartar [1] gave a general framework to study this class of problems by relaxation.
    
    It is based on the use of L. C. Young [2]'s generalized functions (measures).
    
    They present a fairly complete analysis of the homogenization theory of 2nd-order elliptic problems of the form
    \begin{align*}
        -\sum_{ij} \partial_{x_i}\left(a_{ij}(x)\partial_{x_j}u\right) = f.
    \end{align*}
    They give as examples the maximization and minimization versions of the problem of Sect. 4.2.
    
    This material, which would have deserved a whole chapter in this book, is fortunately available in English (cf. F. Murat and L. Tartar [3]).
    
    More results on composite materials can be found in the book edited by A. Cherkaev and R. Kohn [1], which gathers a selection of translations of key papers originally written in French and Russian.
\end{enumerate}

\paragraph{Buckling of Columns.}
\begin{enumerate}
    \item \textbf{Buckling of Columns.} 1 of the very early optimal design problem was formulated by Lagrange in 1770 (cf. I. Todhunter and K. Pearson [1]) and later studied by T. Clausen in 1849.
    
    It consists in finding the best profile of a vertical column to prevent buckling.
    
    This problem and other problems related to columns have been revisited in a series of papers by S. J. Cox [1], S. J. Cox and M. L. Overton [1], S. J. Cox [2], and S. J. Cox and C. M. McCarthy [1].
    
    Since Lagrange many authors have proposed solutions, but a complete theoretical and numerical solution for the buckling of a column was given only in 1992 by S. J. Cox and M. L. Overton [1].
    
    %
    Consider a normalized column of unit height and unit volume.
    
    Denote by $l$ the \textit{magnitude of the normalized axial load} and by $u$ the resulting transverse displacement.
    
    Assume that the potential energy is the sum of the bending and elongation energies
    \begin{align*}
        \int_0^1 EI|u''|^2{\rm d}x - l\int_0^1 |u'|^2{\rm d}x,
    \end{align*}
    where $I$ is the 2nd moment of area of the column's cross section and $E$ is its Young's modulus.
    
    For sufficiently small load $l$ the minimum of this potential energy w.r.t. all admissible $u$ is zero.
    
    \textit{Euler's buckling load} $\lambda$ of the column is the largest $l$ for which this minimum is zero.
    
    This is equivalent to finding the following minimum: \textbf{(5.1)}
    \begin{align*}
        \lambda := \inf_{0\ne u\in V} \frac{\int_0^1 EI|u''|^2{\rm d}x}{\int_0^1 |u'|^2{\rm d}x},
    \end{align*}
    where $V = H_0^2(0,1)$ corresponds to the clamped case, but other types of boundary conditions can be contemplated.
    
    This is an eigenvalue problem with a special Rayleigh quotient.
    
    Assume that $E$ is constant and that the 2nd moment of area $I(x)$ of the column's cross section at the height $x$, $0\le x\le 1$, is equal to a constant $c$ times its cross-sectional area $A(x)$,
    \begin{align*}
        I(x) = cA(x)\Rightarrow\int_0^1 A(x){\rm d}x = 1.
    \end{align*}
    Normalizing $\lambda$ by $cE$ and taking into account the engineering constraints
    \begin{align*}
        \exists 0 < A_0 < A_1,\ \forall x\in[0,1],\ 0 < A_0\le A(x)\le A_1,
    \end{align*}
    we finally get \textbf{(5.2)-(5.3)}
    \begin{align*}
        &\sup_{A\in\mathcal{A}} \lambda(A),\ \lambda(A) := \inf_{0\ne u\in V} \frac{\int_0^1 A|u''|^2{\rm d}x}{\int_0^1 |u'|^2{\rm d}x},\\
        \mathcal{A} &:= \left\{A\in L^2(0,1);A_0\le A\le A_1 \mbox{ and } \int_0^1 A(x){\rm d}x = 1\right\}.
    \end{align*}
    This problem can also be reformulated by rewriting
    \begin{align*}
        A(x) = A_0 + \chi(x)(A_1 - A_0),\ \int_0^1 \chi(x){\rm d}x = \alpha := \frac{1 - A_0}{A_1 - A_0}
    \end{align*}
    for some $\chi\in\overline{\rm co}\operatorname{X}([0,1])$.
    
    Clearly the problem makes sense only for $0 < A_0\le 1$.
    
    Then \textbf{(5.4)}
    \begin{align*}
        \sup_{\chi\in\overline{\rm co}\operatorname{X}([0,1]),\,\int_0^1 \chi(x){\rm d}x = \alpha} \tilde{\lambda}(\chi),\ \tilde{\lambda}(\chi) := \inf_{0\ne u\in V} \frac{\int_0^1 \left[A_0 + (A_1 - A_0)\chi\right]|u''|^2{\rm d}x}{\int_0^1 |u'|^2{\rm d}x}.
    \end{align*}
    Rayleigh's quotient is not a nice convex concave function w.r.t. $(v,A)$, and its analysis necessitates tools different from the ones of Sect. 4.
    
    1 of the original elements of the paper of S. J. Cox and M. L. Overton [1] was to replace Rayleigh's quotient by G. Auchmuty [1]'s dual variational principle for the eigenvalue problem (5.1).
    
    We 1st recall the existence of solution to the minimization of Rayleigh's quotient.
    
    In what follows we shall use the norm $\|u'\|_{L^2}$ for the space $H_0^1(0,1)$ and $\|u''\|_{L^2}$ for the space $H_0^2(0,1)$.
    
    \begin{theorem}
        There exists at least one nonzero solution $u\in V$ to the minimization problem \textbf{(5.5)}
        \begin{align*}
            \lambda(A) := \inf_{0\ne u\in V} \frac{\int_0^1 A|u''|^2{\rm d}x}{\int_0^1 |u'|^2{\rm d}x}.
        \end{align*}
        Then $\lambda(A_0) > 0$ and for all $A\in\mathcal{A}$, $\lambda(A)\ge\lambda(A_0)$, and \textbf{(5.6)}
        \begin{align*}
            \forall v\in V,\ \int_0^1 |v'|^2{\rm d}x\le\lambda(A_0)^{-1}\int_0^1 A|v''|^2{\rm d}x.
        \end{align*}
        The solutions are completely characterized by the variational equation: \textbf{(5.7)}
        \begin{align*}
            \exists u\in V,\ \forall v\in V,\ \int_0^1 Au''v''{\rm d}x = \lambda(A)\int_0^1 u'v'{\rm d}x.
        \end{align*}
    \end{theorem}
    The dual variational principle of G. Auchmuty [1] for the eigenvalue problem (5.5) can be chosen as \textbf{(5.8)-(5.9)}
    \begin{align*}
        \mu(A) &:= \inf\left\{L(A,v);v\in H_0^2(0,1)\right\},\\
        L(A,v) &:= \frac{1}{2}\int_0^1 A|v''|^2{\rm d}x - \left(\int_0^1 |v'|^2{\rm d}x\right)^{1/2}.
    \end{align*}
    
    \begin{theorem}
        For each $A\in\mathcal{A}$, there exists at least one minimizer of $L(A,v)$, \textbf{(5.10)}
        \begin{align*}
            \mu(A) = -\frac{1}{2\lambda(A)},
        \end{align*}
        and the set of minimizers of (5.8) is given by \textbf{(5.11)}
        \begin{align*}
            E(A) := \left\{u\in H_0^2(0,1);u \mbox{ is solution of (5.7) and } \left(\int_0^1 |u'|^2{\rm d}x\right)^{1/2} = \frac{1}{\lambda(A)}\right\}.
        \end{align*}
    \end{theorem}

    \begin{theorem}
        \begin{itemize}
            \item[(i)] The set $\mathcal{A}$ is compact in the $L^2(0,1)$-weak topology.
            \item[(ii)] The function $A\mapsto\mu(A)$ is concave and upper semicontinuous w.r.t. the $L^2(0,1)$-weak topology.
            \item[(iii)] There exists $A$ in $\mathcal{A}$ which maximizes $\mu(A)$ over $A$ and, a fortiori, which maximizes $\lambda(A)$ over $\mathcal{A}$.
        \end{itemize}
    \end{theorem}
\end{enumerate}

\paragraph{Caccioppoli or Finite Perimeter Sets.}
\begin{enumerate}
    \item The notion of a \textit{finite perimeter set} has been introduced and developed mainly by R. Caccioppoli [1] and E. De Giorgi [1] in the context of J. A. F. Plateau [1]'s problem, named after the Belgian physicist and professor (1801-1883), who did experimental observations on the geometry of soap films.
    
    A modern treatment of this subject can be found in the book of E. Giusti [1].
    
    1 of the difficulties in studying the \textit{minimal surface problem} is the description of such surfaces in the usual language of differential geometry.
    
    E.g., the set of possible singularities is
    not known.
    
    Finite perimeter sets provide a geometrically significant solution to Plateau's problem without having to know ahead of time what all the possible singularities of the solution can be.
    
    The characterization of all the singularities of the solution is a difficult problem which can be considered separately.
    
    %
    This was a very fundamental contribution to the theory of variational problems, where the optimization variable is the geometry of a domain.
    
    This point of view has been expanded, and a variational calculus was developed by F. J. Almgren, Jr. [1].
    
    This is the \textit{theory of varifolds}.
    
    \begin{quotation}
        Perhaps the one most important virtue of varifolds is that it is possible to obtain a geometrically significant solution to a number of variational problems, including Plateau's problem, without having to know ahead of time what all the possible singularities of the solution can be.
    \end{quotation}
    \item \textbf{Finite Perimeter Sets.} Given an open subset $D$ of $\mathbb{R}^N$, consider $L^1$-functions $f$ on $D$ with distributional gradient $\nabla f$ in the space $M^1(D)^N$ of (vectorial) bounded measures; i.e.,
    \begin{align*}
        \vec{\varphi}\mapsto\langle\nabla f,\vec{\varphi}\rangle_{\mathcal{D}} := -\int_D f\nabla\cdot\vec{\varphi}{\rm d}x:\ \mathcal{D}^1(D;\mathbb{R}^N)\to\mathbb{R}
    \end{align*}
    is continuous w.r.t. the topology of uniform convergence in $D$: \textbf{(6.1)}
    \begin{align*}
        \|\nabla f\|_{M^1(D)^N} := \sup_{\vec{\varphi}\in\mathcal{D}^1(D;\mathbb{R}^N),\,\|\vec{\varphi}\|_C\le 1} \langle\nabla f,\vec{\varphi}\rangle_{\mathcal{D}} < \infty,\ \|\vec{\varphi}\|_C := \sup_{x\in D} |\vec{\varphi}(x)|_{\mathbb{R}^N},
    \end{align*}
    where $M^1(D) := \mathcal{D}^0(D)'$ is the topological dual of $\mathcal{D}^0(D)$, and
    \begin{align*}
        \nabla f\in\mathcal{L}(\mathcal{D}^0(D;\mathbb{R}^N),\mathbb{R})\equiv\mathcal{L}(\mathcal{D}^0(D),\mathbb{R})^N = M^1(D)^N.
    \end{align*}
    Such functions are known as \textit{functions of bounded variation}.
    
    The space \textbf{(6.2)}
    \begin{align*}
        \operatorname{BV}(D) := \left\{f\in L^1(D);\nabla f\in M^1(D)^N\right\}
    \end{align*}
    endowed with the norm \textbf{(6.3)}
    \begin{align*}
        \|f\|_{\operatorname{BV}(D)} = \|f\|_{L^1(D)} + \|\nabla f\|_{M^1(D)^N}
    \end{align*}
    is a Banach space.
    
    \begin{definition}[L. C. Evans and R. F. Gariepy (1)]
        A function $f\in L_{\rm loc}^1(U)$ in an open subset $U$ of $\mathbb{R}^N$ has \emph{locally bounded variation} if for each bounded open subset $V$ of $U$ s.t. $\overline{V}\subset U$, $f\in\operatorname{BV}(V)$. The set of all such functions will be denoted by $\operatorname{BV}_{\rm loc}(U)$.
    \end{definition}
    
    \begin{theorem}
        Given an open subset $U$ of $\mathbb{R}^N$, $f$ belongs to $\operatorname{BV}_{\rm loc}(U)$ iff for each $x\in U$ there exists $\rho > 0$ s.t. $\overline{B(x,\rho)}\subset U$ and $f\in\operatorname{BV}(B(x,\rho))$, $B(x,\rho)$ the open ball of radius $\rho$ in $x$.
    \end{theorem}
    For more details and properties see also F. Morgan [1, p. 117], H. Federer [5, sect. 4.5.9], L. C. Evans and R. F. Gariepy [1], W. P. Ziemer [1], and R. Temam [1].
    
    %
    As in the previous section, consider measurable subsets $\Omega$ of a fixed bounded open subset $D$ of $\mathbb{R}^N$.
    
    Their characteristic functions $\chi_\Omega\in\operatorname{X}(D)$ are $L^1(D)$-functions \textbf{(6.4)}
    \begin{align*}
        \|\chi_\Omega\|_{L^1(D)} = \int_D \chi_\Omega{\rm d}x = \operatorname{m}(\Omega)\le\operatorname{m}(D) < \infty
    \end{align*}
    with distributional gradient \textbf{(6.5)}
    \begin{align*}
        \forall\vec{\varphi}\in(\mathcal{D}(D))^N,\ \langle\nabla\chi_\Omega,\vec{\varphi}\rangle_D := -\int_D \chi_\Omega\nabla\cdot\vec{\varphi}{\rm d}x.
    \end{align*}
    When $\Omega$ is an open domain with boundary $\Gamma$ of class $C^1$, then by the Stokes divergence theorem
    \begin{align*}
        -\int_{\Omega\cap D} \nabla\cdot\vec{\varphi}{\rm d}x = -\int_{\partial(\Omega\cap D)} \vec{\varphi}\cdot{\bf n}{\rm d}\Gamma = -\int_{\Gamma\cap D} \vec{\varphi}\cdot{\bf n}{\rm d}\Gamma,
    \end{align*}
    where ${\bf n}$ is the outward normal field along $\partial(\Omega\cap D)$.
    
    Since $\Gamma$ is of class $C^1$ the normal field ${\bf n}$ along $\Gamma$ belongs to $C^0(\Gamma)$, and from the last identity the maximum is
    \begin{align*}
        P_D(\Omega) := \|\nabla\chi_\Omega\|_{M^1(D)} = \int_{\Gamma\cap D} |{\bf n}|^2{\rm d}\Gamma = \int_{\Gamma\cap D} {\rm d}\Gamma = H_{N-1}(\Gamma\cap D),
    \end{align*}
    the $(N - 1)$-dimensional Hausdorff measure of $\Gamma\cap D$.
    
    As a result \textbf{(6.6)}
    \begin{align*}
        H_{N-1}(\Gamma) = P_D(\Omega) + H_{N-1}(\Gamma\cap\partial D).
    \end{align*}
    Thus the norm of the gradient provides a natural relaxation of the notion of perimeter to the following larger class of domains.
    
    \begin{definition}
        Let $\Omega$ be a Lebesgue measurable subset of $\mathbb{R}^N$.
        \begin{itemize}
            \item[(i)] The \emph{perimeter of $\Omega$ w.r.t. an open subset $D$ of $\mathbb{R}^N$} is defined as \textbf{(6.7)}
            \begin{align*}
                P_D(\Omega) := \|\nabla\chi_\Omega\|_{M^1(D)^N}.
            \end{align*}
            $\Omega$ is said to have \emph{finite perimeter w.r.t. $D$} if $P_D(\Omega)$ is finite. The family of measurable characteristic functions with finite measure and finite relative perimeter in $D$ will be denoted as
            \begin{align*}
                \operatorname{BX}(D) := \left\{\chi_\Omega\in\operatorname{X}(D);\chi_\Omega\in\operatorname{BV}(D)\right\}.
            \end{align*}
            \item[(ii)] $\Omega$ is said to have \emph{locally finite perimeter} if for all bounded open subsets $D$ of $\mathbb{R}^N$, $\chi_\Omega\in\operatorname{BV}(D)$, i.e., $\chi_\Omega\in\operatorname{BV}_{\rm loc}(\mathbb{R}^N)$.
            \item[(iii)] $\Omega$ is said to have \emph{finite perimeter} if $\chi_\Omega\in\operatorname{BV}(\mathbb{R}^N)$.
        \end{itemize}
    \end{definition}
    
    \begin{theorem}
        Let $\Omega$ be a Lebesgue measurable subset of $\mathbb{R}^N$. $\Omega$ has \emph{locally finite perimeter}, i.e., $\chi_\Omega\in\operatorname{BV}_{\rm loc}(\mathbb{R}^N)$, iff for each $x\in\partial\Omega$ there exists $\rho > 0$ s.t. $\chi_\Omega\in\operatorname{BV}(B(x,\rho))$, $B(x,\rho)$ the open ball of radius $\rho$ in $x$. 
    \end{theorem}
    The interest behind this construction is twofold.
    
    1st, the notion of perimeter of a set is extended to measurable sets; 2nd, this framework provides a 1st compactness theorem which will be useful in obtaining existence of optimal domains.
    
    \begin{theorem}
        Assume that $D$ is a bounded open domain in $\mathbb{R}^N$ with a Lipschitzian boundary $\partial D$. Let $\{\Omega_n\}$ be a sequence of measurable domains in D for which there exists a constant $c > 0$ s.t. \textbf{(6.8)}
        \begin{align*}
            \forall n,\ P_D(\Omega_n)\le c.
        \end{align*}
        Then there exist a measurable set $\Omega$ in $D$ and a subsequence $\{\Omega_{n_k}\}$ s.t. \textbf{(6.9)}
        \begin{align*}
            \chi_{\Omega_{n_k}}\to\chi_\Omega \mbox{ in } L^1(D) \mbox{ as } k\to\infty \mbox{ and } P_D(\Omega)\le\liminf_{k\to\infty} P_D(\Omega_{n_k})\le c.
        \end{align*}
        Moreover $\nabla\chi_{\Omega_{n_k}}$ ``converges in measure'' to $\nabla\chi_\Omega$ in $M^1(D)^N$; i.e., for all $\vec{\varphi}$ in $\mathcal{D}^0(D,\mathbb{R}^N)$, \textbf{(6.10)}
        \begin{align*}
            \lim_{k\to\infty} \langle\nabla\chi_{\Omega_{n_k}},\vec{\varphi}\rangle_{M^1(D)^N}\to\langle\nabla\chi_\Omega,\vec{\varphi}\rangle_{M^1(D)^N}.
        \end{align*}
    \end{theorem}

    \begin{example}[The staircase]
        In (6.9) the inequality can be strict, as can be seen from the following example (cf. Fig. 5.7).
        
        For each $n\ge 1$, define the set
        \begin{align*}
            \Omega_n = \bigcup_{j=1}^n \left[\frac{j}{n},\frac{j+1}{n}\right]\times\left[0,1 - \frac{j}{n}\right].
        \end{align*}
        Its limit is the set
        \begin{align*}
            \Omega = \left\{(x,y);0\le x\le 1,\ 0\le y\le x\right\}.
        \end{align*}
        It is easy to check that the sets $\Omega_n$ are contained in the holdall $D = (-,1,2)\times(-1,2)$ and that
        \begin{align*}
            \forall n\ge 1,\ P_D(\Omega_n) &= 4,\ P_D(\Omega) = 2 + \sqrt{2},\\
            \chi_{\Omega_n}&\to\chi_\Omega \mbox{ in } L^p(D),\ 1\le p < \infty.
        \end{align*}
        Each $\Omega_n$ is uniformly Lipschitzian, but the Lipschitz constant and the 2 neighborhoods of Definition 3.2 in Chap. 2 cannot be chosen independently of $n$.
    \end{example}
    
    \begin{corollary}
        Let $D$ be open bounded and Lipschitzian. There exists a constant $c > 0$ s.t. for all convex domains $\Omega$ in $D$
        \begin{align*}
            P_D(\Omega)\le c,\ \operatorname{m}(\Omega)\le c
        \end{align*}
        and the set $\mathcal{C}(D)$ of convex subsets of $D$ is compact in $L^p(D)$ for all $p$, $1\le p < \infty$.
    \end{corollary}
    This theorem and the lower (resp., upper) semicontinuity of the shape function $\Omega\mapsto J(\Omega)$ will provide existence results for domains in the class of finite perimeter sets in $D$.
    
    1 example is the transmission problem (4.1) to (4.3) of Sect. 4.1 with the objective function \textbf{(6.11)}
    \begin{align*}
        J(\Omega) = \frac{1}{2}\int_\Omega |y(\Omega) - g|^2{\rm d}x + \alpha P_D(\Omega),\ \alpha > 0,
    \end{align*}
    for some $g\in L^2(D)$.
    
    We shall come back later to the homogeneous Dirichlet BVP (3.25)-(3.26).
    
    %
    It is important to recall that even if a set $\Omega$ in $D$ has a finite perimeter $P_D(\Omega)$, its \textit{relative boundary} $\Gamma\cap D$ can have a nonzero $N$-dimensional Lebesgue measure.
    
    To illustrate this point consider the following adaptation of Example 1.10 in E. Giusti [1, p. 7].
    
    \begin{example}
        Let $D = B(0,1)$ in $\mathbb{R}^2$ be the open ball in 0 of radius 1. For $i\ge 1$, let $\{x_i\}$ be an ordered sequence of all points in $D$ with rational coordinates.
        
        Associate with each $i$ the open ball
        \begin{align*}
            B_i = \left\{x\in D;|x - x_i| < \rho_i\right\},\ 0 < \rho_i\le\min\left\{\frac{1}{2^i},1 - |x_i|\right\}.
        \end{align*}
        Define the new sequence of open subsets of $D$,
        \begin{align*}
            \Omega_n = \bigcup_{i=1}^n B_i,
        \end{align*}
        and notice that for all $n\ge 1$,
        \begin{align*}
            \operatorname{m}(\partial\Omega_n) = 0,\ P_D(\Omega_n)\le 2\pi,
        \end{align*}
        where $\operatorname{m} = \operatorname{m}_2$ is the Lebesgue measure in $\mathbb{R}^2$ and $\partial\Omega_n$ is the boundary of $\Omega_n$.
        
        Moreover, since the sequence of sets $\{\Omega_n\}$ is increasing,
        \begin{align*}
            \chi_{\Omega_n}\to\chi_\Omega \mbox{ in } L^1(D),\ \Omega = \bigcup_{i=1}^\infty B_i,\ P_D(\Omega)\le\liminf_{n\to\infty} P_D(\Omega_n)\le 2\pi.
        \end{align*}
        Observe that $\overline{\Omega} = \overline{D}$ and that $\partial\Omega = \overline{\Omega}\cap\overline{\Omega^c}\supset\overline{D}\cap\Omega^c$. Thus
        \begin{align*}
            \operatorname{m}(\partial\Omega) = \operatorname{m}(\overline{D}\cap\Omega^c)\ge\operatorname{m}(\overline{D}) - \operatorname{m}(\Omega)\ge\frac{2\pi}{3}
        \end{align*}
        since
        \begin{align*}
            \operatorname{m}(D) = \pi \mbox{ and } \operatorname{m}(\Omega)\le\sum_{i=1}^\infty \pi\frac{1}{2^{2i}} = \frac{\pi}{3}.
        \end{align*}
        Recall that
        \begin{align*}
            \operatorname{m}(\Omega_n)\le\sum_{i=1}^n \operatorname{m}(B_i)\le\sum_{i=1}^n \pi\frac{1}{2^{2i}}\Rightarrow\operatorname{m}(\overline{D})\le\sum_{i=1}^\infty \operatorname{m}(B_i)\le\sum_{i=1}^\infty \pi\frac{1}{2^{2i}}. 
        \end{align*}
        For $p$, $1\le p < \infty$, the sequence of characteristic functions $\{\chi_{\Omega_n}\}$ converges to $\chi_\Omega$ in $L^p(D)$-strong.
        
        However, for all $n$, $\operatorname{m}(\partial\Omega_n) = 0$, but $\operatorname{m}(\partial\Omega) > 0$.
    \end{example}
    In fact we can associate with the perimeter $P_D(\Omega)$ the \textit{reduced boundary}, $\partial^*\Omega$, which is the set of all $x\in\partial\Omega$ for which the normal ${\bf n}(x)$ exists.
    
    We quote the following interesting theorems from W. H. Fleming [1, p. 455] (cf. also E. De Giorgi [2], L. C. Evans and R. F. Gariepy [1, Thm. 1, sect. 5.1, p. 167, Notation, pp. 168-169, Lem. 1, p. 208], and H. Federer [2]).
    
    \begin{theorem}
        Let $\chi_\Omega\in\operatorname{BV}_{\rm loc}(\mathbb{R}^N)$. There exists a Radon measure $\|\partial\Omega\|$ on $\mathbb{R}^N$ and a $\|\partial\Omega\|$-measurable function $\nu_\Omega:\mathbb{R}^N\to\mathbb{R}^N$ s.t.
        \begin{itemize}
            \item[(i)] $|\nu_\Omega(x)| = 1$, $\|\partial\Omega\|$ a.e. and
            \item[(ii)] $\int_{\mathbb{R}^N} \chi_\Omega\nabla\cdot\varphi{\rm d}x = -\int_{\mathbb{R}^N} \varphi\cdot\nu_\Omega{\rm d}\|\partial\Omega\|$, for all $\varphi\in C_c^1(\mathbb{R}^N,\mathbb{R}^N)$.
        \end{itemize}
    \end{theorem}
    
    \begin{definition}
        Let $\chi_\Omega\in\operatorname{BV}_{\rm loc}(\mathbb{R}^N)$. The point $x\in\mathbb{R}^N$ belongs to the \emph{reduced boundary} $\partial^*\Omega$ if
        \begin{itemize}
            \item[(i)] $\|\partial\Omega\|(B_r(x)) > 0$, for all $r > 0$,
            \item[(ii)] $\lim_{r\downarrow 0} \frac{1}{m(B_r(x))}\int_{B_r(x)} \nu_\Omega{\rm d}\|\partial\Omega\| = \nu_\Omega(x)$, and
            \item[(iii)] $|\nu_\Omega(x)| = 1$.
        \end{itemize}
    \end{definition}
    
    \begin{theorem}
        Let $\Omega$ have finite perimeter $P(\Omega)$. Let $\partial^*\Omega$ denote the reduced boundary of $\Omega$. Then
        \begin{itemize}
            \item[(i)] $\partial^*\Omega\subset\partial_*\Omega\subset\Omega_\bullet\subset\partial\Omega$ and $\overline{\partial^*\Omega} = \partial\Omega$,
            \item[(ii)] $P(\Omega) = H_{N-1}(\partial^*\Omega)$ (cf. E. Giusti [1, Chap. 4]),
            \item[(iii)] the Gauss-Green theorem holds with $\partial^*\Omega$, and
            \item[(iv)] $\operatorname{m}(\partial^*\Omega) = \operatorname{m}(\partial_*\Omega) = 0$, $H_{N-1}(\partial_*\Omega\backslash\partial^*\Omega) = 0$.
        \end{itemize}
    \end{theorem}
    We quote the following density theorem (cf. E. Giusti [1, Thm. 1.24 and Lem. 1.25, p. 23]), which complements Theorem 3.1.
    
    \begin{theorem}
        Let $\Omega$ be a bounded measurable domain in $\mathbb{R}^N$ with finite perimeter. Then there exists a sequence $\{\Omega_j\}$ of $C^\infty$-domains s.t. as $j$ goes to $\infty$
        \begin{align*}
            \int_{\mathbb{R}^N} |\chi_{\Omega_j} - \chi_\Omega|{\rm d}x\to 0 \mbox{ and } P(\Omega_j)\to P(\Omega).
        \end{align*}
    \end{theorem}
    \item \textbf{Decomposition of the Integral along Level Sets.} Some useful theorems on the decomposition of the integral along the \textit{level sets} of a function.
    
    The 1st one was used by J.-P. Zolésio [6] (Grad's model in plasma physics) in 1979 and [9, Sect. 4.4, p. 95] in 1981, by R. Temam [1] (monotone rearrangements) in 1979, by J. M. Rakotoson and R. Temam [1] in 1987, and more recently in 1995 by M. C. Delfour and J.-P. Zolésio [19, 20, 21, 25] (intrinsic formulation of models of shells).
    
    Quote the version given in L. C. Evans and R. F. Gariepy [1, Prop. 3, p. 118].
    
    The original theorem can be found in H. Federer [3] and L. C. Young [1].
    
    \begin{theorem}
        Let $f:\mathbb{R}^N\to\mathbb{R}$ be Lipschitz continuous with
        \begin{align*}
            |\nabla f| > 0 \mbox{ a.e.}
        \end{align*}
        For any Lebesgue summable function $g:\mathbb{R}^N\to\mathbb{R}$, we have the following decomposition of the integral along the level sets of $f$:
        \begin{align*}
            \int_{\{f > t\}} g{\rm d}x = \int_t^\infty\left(\int_{\{f = s\}} \frac{g}{|\nabla f|}{\rm d}H_{N-1}\right){\rm d}s.
        \end{align*}
    \end{theorem}
    The 2nd theorem (W. H. Fleming and R. Rishel [1]) uses a BV function instead of a Lipschitz function.
    
    \begin{theorem}
        Let $D$ be an open subset of $\mathbb{R}^N$. For any $f\in\operatorname{BV}(D)$ and real $t$, let $E_t := \{x;f(x) < t\}$. Then
        \begin{align*}
            \|\nabla f\|_{M^1(D)^N} = \int_{-\infty}^\infty P(E_t){\rm d}t.
        \end{align*}
    \end{theorem}
    (cf., e.g., H. Whitney [1, Chap. 11]).
    
    %
    This is known as the \textit{co-area formula} (see also E. Giusti [1, Thm. 1.23, p. 20] and E. De Giorgi [4]).
    \item \textbf{Domains of Class $W^{\varepsilon,p}(D)$, $0\le\varepsilon < 1/p$, $p\ge 1$, and a Cascade of Complete Metric Spaces.} There is a general property enjoyed by functions in $\operatorname{BV}(D)$.
    
    \begin{theorem}
        Let $D$ be a bounded open Lipschitzian domain in $\mathbb{R}^N$.
        \begin{itemize}
            \item[(i)] $\operatorname{BV}(D)\subset W^{\varepsilon,1}(D)$, $0\le\varepsilon < 1$.
            \item[(ii)] $\operatorname{BV}(D)\cap L^\infty(D)\subset W^{\varepsilon,p}(D)$, $0\le\varepsilon < \frac{1}{p}$, $1\le p < \infty$.
        \end{itemize}
    \end{theorem}
    This theorem says that for a Caccioppoli set $\Omega$ in $D$,
    \begin{align*}
        \forall p\ge 1,\ 0\le\varepsilon < \frac{1}{p},\ \chi_\Omega\in W^{\varepsilon,p}(D).
    \end{align*}
    The special case $p = 2$ was proved by C. Baiocchi, V. Comincioli, E. Magenes, and G. A. Pozzi [1] in the context of the celebrated problem of the dam.
    
    They showed that the domain $\Omega$ is the hypograph of a continuous monotonically decreasing function on a closed interval is $H^\varepsilon(D) = W^{\varepsilon,2}(D)$, $0\le\varepsilon < \frac{1}{2}$, in $\mathbb{R}^2$.
    
    %
    For a characteristic function $\chi_\Omega$ of a measurable set $\Omega$ and $\varepsilon > 0$, we have $0\le\varepsilon < \frac{1}{p}$ for all $p\ge 1$ and \textbf{(6.12)}
    \begin{align*}
        \|\chi_\Omega\|_{W^{\varepsilon,p}(D)}^p = \int_D\int_D \frac{|\chi_\Omega(x) - \chi_\Omega(y)|}{|x - y|^{N + p\varepsilon}}{\rm d}x{\rm d}y + \|\chi_\Omega\|_{L^p(D)}^p.
    \end{align*}
    The $W^{\varepsilon,p}(D)$-norm is equivalent to the $W^{\varepsilon',p'}(D)$ for all pairs $(p',\varepsilon')$, $p'\ge 1$, $0\le\varepsilon' < \frac{1}{p'}$, s.t. $p'\varepsilon' = p\varepsilon$.
    
    At this stage, it is not clear whether that norm is related to some Hausdorff measure of the relative boundary $\Gamma\cap D$ of the set.
    
    %
    For $p = 2$ and a characteristic function, $\chi_\Omega$, of a measurable set $\Omega$, we get \textbf{(6.13)}
    \begin{align*}
        \|\chi_\Omega\|_{H^\varepsilon(D)}^2 = 2\int_\Omega\int_{{\rm C}_D\Omega} |x - y|^{-(N + 2\varepsilon)}{\rm d}x{\rm d}y + \operatorname{m}(\Omega\cap D).
    \end{align*}
    The space $\operatorname{X}(D)\cap W^{\varepsilon,2}(D)$ is a closed subspace of the Hilbert space $W^{\varepsilon,2}(D)$ and the square of its norm is differentiable.
    
    A direct consequence for optimization problems is that a penalization term of the form $\|\chi_\Omega\|_{W^{\varepsilon,2}(D)}^2$ is now differentiable and can be used in various minimization problems or to regularize the objective function to obtain existence of approximate solutions.
    
    E.g., to obtain existence results when minimizing a function $J(\Omega)$ (defined for all measurable sets $\Omega$ in $D$), we can consider a regularized problem in the following form: \textbf{(6.14)}
    \begin{align*}
        J_\alpha(\Omega) = J(\Omega) + \alpha\|\chi_\Omega\|_{H^\varepsilon(D)}^2,\ \alpha > 0.
    \end{align*}
    Sect. 6.1 provided the important compactness Theorem 6.3 for Caccioppoli or finite perimeter sets.
    
    For a bounded open holdall $D$, this also provides a new complete metric topology on $\operatorname{X}(D)\cap\operatorname{BV}(D)$ when endowed with the metric
    \begin{align*}
        \rho_{\operatorname{BV}}(\chi_1,\chi_2) := \|\chi_2 - \chi_1\|_{\operatorname{BV}(D)} = \|\chi_2 - \chi_1\|_{L^1(D)} + \|\nabla\chi_2 - \nabla\chi_1\|_{M^1(D)}.
    \end{align*}
    Similarly, the intermediate spaces of Theorem 6.9 between $\operatorname{BV}(D)$ and $L^p(D)$ provide little rougher metrics for $p\ge 1$ and $\varepsilon < \frac{1}{p}$ on $\operatorname{X}(D)\cap W^{\varepsilon,p}(D)$ when endowed with the metric
    \begin{align*}
        \rho_{W^{\varepsilon,p}}(\chi_1,\chi_2) := \|\chi_2 - \chi_1\|_{W^{\varepsilon,p}(D)}.
    \end{align*}
    
    \begin{theorem}
        Given a bounded open holdall $D$ in $\mathbb{R}^N$, $\operatorname{X}(D)\cap\operatorname{BV}(D)$ and $\operatorname{X}(D)\cap W^{\varepsilon,p}(D)$, $p\ge 1$, $\varepsilon < \frac{1}{p}$, are complete metric spaces and
        \begin{align*}
            \operatorname{X}(D)\cap\operatorname{BV}(D)\subset\operatorname{X}(D)\cap W^{\varepsilon,p}(D)\subset\operatorname{X}(D).
        \end{align*}
    \end{theorem}
    \item \textbf{Compactness \& Uniform Cone Property.} In Theorem 6.3 of Sect. 6.1 we have seen a 1st compactness theorem for a family of sets with a uniformly bounded perimeter.
    
    In this section we present a 2nd compactness theorem for measurable domains satisfying a \textit{uniform cone property} due to D. Chenais [1, 4, 6].
    
    We provide a proof of this result that emphasizes the fact that the perimeter of sets in that family is uniformly bounded.
    
    It then becomes a special case of Theorem 6.3.
    
    This result will also be obtained as Corollary 2 to Theorem 13.1 in section 13 of Chap. 7.
    
    It will use completely different arguments and apply to families of sets that do not even have a finite boundary measure.
    
    \begin{theorem}
        Let $D$ be a bounded open holdall in $\mathbb{R}^N$ with uniformly Lipschitzian
        boundary $\partial D$. For $r > 0$, $\omega > 0$, and $\lambda > 0$ consider the family \textbf{(6.15)}
        \begin{align*}
            L(D,r,\omega,\lambda) := \left\{\Omega\subset D;\Omega \mbox{ is Lebesgue measurable and satisfies the uniform cone property for } (r,\omega,\lambda)\right\}.
        \end{align*}
        For $p$, $1\le p < \infty$, the set \textbf{(6.16)}
        \begin{align*}
            X(D,r,\omega,\lambda) := \left\{\chi_\Omega;\forall\Omega\in L(D,r,\omega,\lambda)\right\}
        \end{align*}
        is compact in $L^p(D)$ and there exists a constant $p(D,r,\omega,\lambda) > 0$ s.t. \textbf{(6.17)}
        \begin{align*}
            \forall\omega\in L(D,r,\omega,\lambda),\ H_{N-1}(\partial\Omega)\le p(D,r,\omega,\lambda).
        \end{align*}
    \end{theorem}
    We 1st show that the perimeter of the sets of the family $L(D,r,\omega,\lambda)$ is uniformly bounded and use Theorem 6.8 to show that any sequence of $X(D,r,\omega,\lambda)$ has a subsequence converging to the characteristic function of some finite perimeter set $\Omega$.
    
    We use the full strength of the compactness of the injection of $\operatorname{BV}(D)$ into $L^1(D)$ rather than checking directly the conditions under which a subset of $L^p(D)$ is relatively compact.
    
    The proof will be completed by showing that the nice representative (Definition 3.3) ${\rm I}$ of $\Omega$ satisfies the same uniform cone property.
    
    The proof uses some elements of D. Chenais [4, 6]'s original proof.
    
    \begin{lemma}
        Let $\chi_{\Omega_n}\to\chi_\Omega$ in $L^p(D)$, $1\le p < \infty$, for $\Omega\subset D$, $\Omega_n\subset D$, and let ${\rm I}$ be the measure theoretic representative of $\Omega$. Then
        \begin{align*}
            \forall x\in\overline{\rm I},\ \forall R > 0,\ \exists N(x,R) > 0,\ \forall n\ge N(x,R),\ \operatorname{m}\left(B(x,R)\cap\Omega_n\right) > 0.
        \end{align*}
        Moreover,
        \begin{align*}
            &\forall x\in\partial{\rm I},\ \forall R > 0,\ \exists N(x,R) > 0,\ \forall n\ge N(x,R),\\
            &\operatorname{m}\left(B(x,R)\cap\Omega_n\right) > 0 \mbox{ and } \operatorname{m}\left(B(x,R)\cap\Omega_n^c\right) > 0
        \end{align*}
        and $B(x,R)\cap\partial\Omega_n\ne\emptyset$.
    \end{lemma}
\end{enumerate}

\paragraph{Existence for the Bernoulli Free Boundary Problem.}
\begin{enumerate}
    \item \textbf{An Example: Elementary Modeling of the Water Wave.} Consider a fluid in a domain $\Omega$ in $\mathbb{R}^3$ and assume that the velocity of the flow ${\bf u}$ satisfies the \textit{Navier-Stokes equations} \textbf{(7.1)}
    \begin{align*}
        {\bf u}_t - D{\bf u}{\bf u} - \nu\Delta{\bf u} + \nabla p = -\rho g \mbox{ in } \Omega,\ \nabla\cdot{\bf u} = 0 \mbox{ in } \Omega,
    \end{align*}
    where $\nu > 0$ and $\rho > 0$ are the respective \textit{viscosity} and \textit{density} of the fluid, $p$ is the pressure, and $g$ is the gravity constant.
    
    The 2nd equation characterizes the incompressibility of the fluid.
    
    A standard example considered by physicists is the water wave in a channel.
    
    The boundary conditions on $\partial\Omega$ are the \textit{sliding conditions} at the bottom $S$ and on the free boundary $\Gamma$, i.e. \textbf{(7.2)}
    \begin{align*}
        {\bf u}\cdot{\bf n} = 0 \mbox{ on } S\cup\Gamma.
    \end{align*}
    Assume that a \textit{stationary regime} has been reached so that the velocity of the fluid is no longer a function of the time.
    
    Furthermore assume that the motion of the fluid is \textit{irrotational}.
    
    By the classical \textbf{Hodge decomposition}, the velocity can be written in the form \textbf{(7.3)}
    \begin{align*}
        {\bf u} = \nabla\varphi + \operatorname{curl}\xi.
    \end{align*}
    As $\operatorname{curl}{\bf u} = \operatorname{curl}\operatorname{curl}\xi = 0$ we conclude that $\xi = 0$, since $\operatorname{curl}\operatorname{curl}$ is a \textit{good} isomorphism.
    
    Then ${\bf u} = \nabla\varphi$ and the incompressibility condition becomes \textbf{(7.4)}
    \begin{align*}
        \Delta\varphi = 0 \mbox{ in } \Omega.
    \end{align*}
    Therefore the NSE reduces to \textbf{(7.5)}
    \begin{align*}
        D(\nabla\varphi)\nabla\varphi + \nu\nabla(\Delta\varphi) + \nabla p = -\nabla(\rho gz),
    \end{align*}
    but \textbf{(7.6)}
    \begin{align*}
        D(\nabla\varphi)\nabla\varphi = \frac{1}{2}\nabla(|\nabla\varphi|^2),
    \end{align*}
    so that \textbf{(7.7)}
    \begin{align*}
        \nabla\left(\frac{1}{2}|\nabla\varphi|^2 + p + \rho gz\right) = 0 \mbox{ in } \Omega.
    \end{align*}
    Then if $\Omega$ is connected, there exists a constant $c$ s.t. \textbf{(7.8)}
    \begin{align*}
        \frac{1}{2}|\nabla\varphi|^2 + p + \rho gz = c \mbox{ in } \Omega.
    \end{align*}
    In the domain $\Omega$, that \textit{Bernoulli condition} yields explicitly the pressure $p$ as a function of the velocity $|\nabla\varphi|$ and the height $z$ in the fluid.
    
    The flow is now assumed to be independent of the transverse variable $y$ so that the initial 3D problem in a perfect channel reduces to a 2D one.
    
    Since the problem has been reduced to a 2D one, introduce the harmonic conjugate $\psi$, the so-called \textit{stream function}, so that the boundary condition $\partial_{\bf n}\varphi = 0$ takes the form $\psi =$ constant on each connected component of the boundary $\partial\Omega$.
    
    %
    On the \textit{free boundary} at the top of the wave, the pressure is related to the existing \textit{atmospheric pressure} $p_a$ through the surface tension $\sigma > 0$ and the mean curvature $H$ of the free boundary.
    
    Actually \textbf{(7.9)}
    \begin{align*}
        p - p_a = -\sigma H
    \end{align*}
    on the free boundary of the stationary wave, where $H$ is the mean curvature associated with the fluid domain.
    
    Also, from the Cauchy conditions, we have $|\nabla\psi| = |\nabla\varphi|$ so that the boundary condition (7.8) on the free boundary of the wave becomes \textbf{(7.10)}
    \begin{align*}
        \frac{1}{2}|\nabla\psi|^2 + \rho gz + \sigma H = p_a.
    \end{align*}
    In order to simplify the presentation we replace the equation $\Delta\psi = 0$ by $\Delta\psi = f$ in $\Omega$ to avoid a \textit{forcing term} on a part of the boundary, and we show that the resulting free boundary problem has the following \textit{shape variational formulation}.
    
    Let $D$ be a fixed, sufficiently large, smooth, bounded, and open domain in $\mathbb{R}^2$.
    
    Let $a$ be a real number s.t. $0 < a < \operatorname{m}(D)$.
    
    To find $\Omega\subset D$, $\operatorname{m}(\Omega) = a$ and $\psi\in H_0^1(D)$ s.t. \textbf{(7.11)}
    \begin{align*}
        -\Delta\psi = f \mbox{ in } \Omega,
    \end{align*}
    and $\psi =$ constant and satisfies the boundary condition (7.10) on the free part $\partial\Omega\cap D$ of the boundary.
    
    %
    For a fixed $\Omega$, the solution of this problem is a minimizing element of the following variational problem: \textbf{(7.12)}
    \begin{align*}
        J(\Omega) := \inf_{\varphi\in H_\diamond^1(\Omega;D)} \int_\Omega \left(\frac{1}{2}|\nabla\varphi|^2 - f\varphi + \rho gz\right){\rm d}x + \sigma P_D(\Omega),
    \end{align*}
    where \textbf{(7.13)}
    \begin{align*}
        H_\diamond^1(\Omega;D) := \left\{u\in H_0^1(D);u(x) = 0 \mbox{ a.e. } x \mbox{ in } D\backslash\Omega\right\}
    \end{align*}
    is the relaxation of the definition of the Sobolev space $H_0^1(\Omega)$ \footnote{In Chap. 8 the space $H_0^1(\Omega;D)$ of extensions by zero to $D$ of elements of $H_0^1(\Omega)$ is defined for $\Omega$ open.
        
        It is then characterized in Lemma 6.1 of Chap. 8 by a capacity condition on the complement $D\backslash\Omega$.
        
        This characterization extends to quasi-open sets $\Omega$ as defined in Sect. 6 of Chap. 8.} for any measurable subset $\Omega$ of $D$.
    
    Its properties were studied in Theorem 3.6.
    
    %
    With that formulation there exists a measurable $\Omega^*\subset D$, $|\Omega^*| = a$, s.t. \textbf{(7.14)}
    \begin{align*}
        \forall\Omega\subset D,\ |\Omega| = a,\ J(\Omega^*)\le J(\Omega).
    \end{align*}
    By using the methods of Chap. 10 it follows that if $\partial\Omega^*$ is sufficiently smooth, the \textit{shape Euler condition} $dJ(\Omega^*;V) = 0$ yields the original free boundary problem and the free boundary condition (7.10).
    
    The existence of a solution will now follow from Theorem 6.3 in Sect. 6.1.
    
    The case without surface tension is physically important.
    
    It occurs in phenomena with ``evaporation'' (cf. M. Souli and J.-P. Zolésio [2]).
    
    In that case the previous Bernoulli condition becomes
    \begin{align*}
        |\partial_{\bf n}\xi|^2 = g^2\ (g\ge 0)
    \end{align*}
    on the free boundary, so that if we assume (in the channel setting) that there is no cavitation or recirculation in the fluid, then $\partial_{\bf n}\xi > 0$ on the free boundary and we get the Neumann-like condition $\partial_{\bf n}\xi = g$ together with the Dirichlet condition.
    
    In Sect. 7.4: consider the \textit{case with surface tension}.
    \item \textbf{Existence for a Class of Free Boundary Problems.} Consider the following free boundary problem, studied in J.-P. Zolésio [25, 26, 29]: find $\Omega$ in a fixed holdall $D$ and a function $y$ on $\Omega$ s.t. \textbf{(7.15)}
    \begin{align*}
        -\Delta y = f \mbox{ in } \Omega,\ y = 0 \mbox{ and } \partial_{\bf n}y = Q^2 \mbox{ on } \partial\Omega,
    \end{align*}
    where $f$ and $Q$ are appropriate functions defined in $D$.
    
    To study this type of problem H. W. Alt and L. A. Caffarelli [1] have introduced the following function: \textbf{(7.16)}
    \begin{align*}
        J(\varphi) := \int_D \frac{1}{2}|\nabla\varphi|^2 - f\varphi{\rm d}x + \int_D Q^2\chi_{\{\varphi > 0\}}{\rm d}x,
    \end{align*}
    which is minimized over \textbf{(7.17)}
    \begin{align*}
        K := \left\{u\in H_0^1(D);u(x)\ge 0 \mbox{ a.e. in } D\right\},
    \end{align*}
    where $\chi_{\{\varphi > 0\}}$ (resp., $\chi_{\{\varphi\ne 0\}}$) is the characteristic function of the set $\{x\in D;\varphi(x) > 0\}$\footnote{This set defined up to a set of zero measure is a \textit{quasi-open} set in the sense of section 6 in Chap. 8.} (resp., $\{x\in D;\varphi(x)\ne 0\}$).
    
    The existence of a solution is based on the following lemma.
    
    \begin{lemma}
        Let $\{u_n\}$ and $\{\chi_n\}$ be 2 converging sequences s.t. $u_n\to u$ in $L^2(D)$-strong, and let the $\chi_n$'s be characteristic functions, $\chi_n(1 - \chi_n) = 0$, which converge to some function $\lambda$ in $L^2(D)$-weak. Then \textbf{(7.18)}
        \begin{align*}
            \forall n,\ \left(1 - \chi_n\right)u_n = 0\Rightarrow\lambda\ge\chi_{\{u\ne 0\}}.
        \end{align*}
    \end{lemma}

    \begin{proposition}
        Let $f$ and $Q$ be 2 elements of $L^2(D)$ s.t. $f\ge 0$ almost everywhere. There exists $u$ in $K$ which minimizes the function $J$ over the positive cone $K$ of $H_0^1(D)$.
    \end{proposition}

    Obviously in the upper bound in (7.20), $Q^2$ is a positive function.
    
    Moreover
    \begin{itemize}
        \item[(i)] the set $\Omega$ is given by $\{x\in D;u(x) > 0\}$, and
        \item[(ii)] the restriction $u|_\Omega$ of $u$ to $\Omega$ is a weak solution of the free boundary problem \textbf{(7.21)}
        \begin{align*}
            -\Delta u = f \mbox{ in } \Omega,\ u = 0 \mbox{ and } \partial_{\bf n}u = Q^2 \mbox{ on } \partial\Omega.
        \end{align*}
    \end{itemize}
    Effectively, the minimization problem (7.16)-(7.17) can be written as a shape optimization problem.
    
    1st introduce for any measurable $\Omega\subset D$ the positive cone
    \begin{align*}
        H_0^1(\Omega)_+ := \left\{u\in H_\diamond^1(\Omega;D);u(x)\ge 0 \mbox{ a.e. } x\in D\right\}
    \end{align*}
    in the Hilbert space $H_\diamond^1(\Omega;D)$.
    
    Then consider the following shape optimization problem: \textbf{(7.22)}
    \begin{align*}
        \inf\left\{E(\Omega);\Omega \mbox{ is measurable subset in } D\right\},
    \end{align*}
    where the energy function $E$ is given by \textbf{(7.23)}
    \begin{align*}
        E(\Omega) := \min_{\varphi\in H_0^1(\Omega)_+} \left(\int_\Omega \frac{1}{2}|\nabla\varphi|^2 - f\varphi{\rm d}x\right) + \int_\Omega Q^2{\rm d}x.
    \end{align*}
    The necessary condition associated with the minimum could be obtained by the techniques introduced in Sect. 2.1 of Chap. 10.
    
    The important difference with the previous formulation (7.16)-(7.17) is that the independent variable in the objective function is no longer a function but a domain.
    
    The shape formulation (7.22)-(7.23) now makes it possible to handle constraints on the volume or the perimeter of the domain, which were difficult to incorporate in the 1st formulation.
    
    %
    As we have seen in Theorem 3.6, $H_\diamond^1(\Omega;D)$ endowed with the norm of $H_0^1(D)$ is a Hilbert space, and $H_0^1(\Omega)_+$ a closed convex cone in $H_\diamond^1(\Omega;D)$, so that for any measurable subset $\Omega$ in $D$, problem (7.23) has a unique solution $y$ in $H_0^1(\Omega)_+$.
    
    Thus we have the following equivalence between problems (7.22)-(7.23) and the minimization (7.16)-(7.17) of $J$ over $K$.
    
    \begin{proposition}
        Let $u$ be a minimizing element of $J$ over $K$. Then
        \begin{align*}
            \Omega := \left\{x\in D;u(x) > 0\right\}
        \end{align*}
        is a solution of problem (7.22) and $y = u|_\Omega$ is a solution of (7.23). Conversely, if $\Omega$ is a measurable subset of $D$ and $y$ is a solution of (7.22)-(7.23) in $H_0^1(\Omega)_+$, then the element $u$ defined by
        \begin{equation*}
            u(x) := \left\{\begin{split}
                &y(x), &&\mbox{ if } x\in\Omega,\\
                &0, &&\mbox{ if } x\in D\backslash\Omega,
            \end{split}\right.
        \end{equation*}
        belongs to $K$ and minimizes $J$ over $K$.
    \end{proposition}
    \item \textbf{Weak Solutions of Some Generic Free Boundary Problems.}
    \begin{enumerate}
        \item \textbf{Problem without Constraint.} Problem (7.22)-(7.23) can be relaxed as follows: given any $f$ in $L^2(D)$, $G$ in $L^1(D)$, \textbf{(7.24)}
        \begin{align*}
            (\mathcal{P}_0)\ \inf\left\{E(\Omega);\Omega\subset D \mbox{ measurable}\right\},
        \end{align*}
        where for any measurable subset $\Omega$ of $D$ the energy function is now defined by \textbf{(7.25)}
        \begin{align*}
            E(\Omega) := \min_{\varphi\in H_\diamond^1(\Omega;D)} E_D(\varphi) + \int_\Omega G{\rm d}x,\ E_D(\varphi) := \int_D \frac{1}{2}|\nabla\varphi|^2 - f\varphi{\rm d}x,
        \end{align*}
        where $H_\diamond^1(\Omega;D)$ is defined in (7.13).
        
        By Theorem 3.6 $H_\diamond^1(\Omega;D)$ is contained in $H_\bullet^1(\Omega;D)$, and for any element $u$ in $H_\diamond^1(\Omega;D)$ we have $\nabla u(x) = 0$ for almost all $x$ in $D\backslash\Omega$ so that \textbf{(7.26)-(7.27)}
        \begin{align*}
            &\forall\varphi\in H_\diamond^1(\Omega;D),\ E_D(\varphi) = \int_\Omega \frac{1}{2}|\nabla\varphi|^2 - f\varphi{\rm d}x\\
            &\Rightarrow E(\Omega) = \min_{\varphi\in H_\diamond^1(\Omega;D)} \left(\int_\Omega \frac{1}{2}|\nabla\varphi|^2 - f\varphi{\rm d}x + \int_\Omega G{\rm d}x\right).
        \end{align*}
        We have the following existence result for problem $(\mathcal{P}_0)$.
        
        \begin{theorem}
            For any $f$ in $L^2(D)$, $G = Q^2$ in $L^1(D)$, there exists at least one solution to problem $(\mathcal{P}_0)$.
        \end{theorem}
        \item \textbf{Constraint on the Measure of the Domain $\Omega$.} Consider an important variation of problem (7.22)-(7.23).
        
        Given any $f$ in $L^2(D)$, $G$ in $L^1(D)$, and a real number $\alpha$, $0 < \alpha < \operatorname{m}(D)$, \textbf{(7.30)}
        \begin{align*}
            (\mathcal{P}_0^\alpha)\ \inf\left\{E(\Omega);\Omega\subset D \mbox{ measurable and } \operatorname{m}(\Omega) = \alpha\right\}.
        \end{align*}
        We have the following existence result for problem $(\mathcal{P}_0^\alpha)$.
        
        \begin{theorem}
            For any $f$ in $L^2(D)$, $G = 0$, and any real number $\alpha$, $0 < \alpha < \operatorname{m}(D)$, there exists at least one solution to problem $(\mathcal{P}_0^\alpha)$.
        \end{theorem}
    
        \begin{corollary}
            Assume that $G = 0$ and $f$ in $L^2(D)$ and $f = Q^2$ in $L^1(D)$. Then the following problem has an optimal solution: \textbf{(7.33)}\begin{align*}
                (\mathcal{P}_0^{\alpha-})\ \inf_{\Omega\subset D,\,\operatorname{m}(\Omega)\le\alpha} E(\Omega).
            \end{align*}
        \end{corollary}
    \end{enumerate}
    \item \textbf{Weak Existence with Surface Tension.} Problems $(\mathcal{P}_0)$, $(\mathcal{P}_0^\alpha)$, and $(\mathcal{P}_0^{\alpha-})$ have optimal solutions, but as they are associated with the homogeneous Dirichlet boundary condition, $u$ in $H_\diamond^1(\Omega;D)$, the optimal domain $\Omega$ is in general not allowed to have holes, that is to say, roughly speaking, that the topology of $\Omega$ is a priori specified.
    
    In many examples it turns out that the solution $u$ physically corresponds to a potential and the classical homogeneous Dirichlet boundary condition is not the appropriate one.
    
    The physical condition is that the potential $u$ should be constant on each connected component of the boundary $\partial\Omega$ in $D$.
    
    When $\Omega$ is a simply connected domain in $\mathbb{R}^2$, then $\partial\Omega$ has a single connected component so the constant can be taken as zero.
    
    In general this constant can be fixed only in 1 connected component; in the others the constant is an unknown of the problem.
    
    %
    The minimization problems $(\mathcal{P}_0)$, $(\mathcal{P}_0^\alpha)$, and $(\mathcal{P}_0^{\alpha-})$ associated with the Hilbert space $H_\bullet^1(\Omega;D)$ fail (in the sense that the previous techniques for existence of an optimal $\Omega$ fail).
    
    The main reason is that Lemma 7.1 is no longer true when $u_n$ is replaced by $\nabla u_n$ weakly converging in $L^2(D)^N$.
    
    The key idea is to recover the equivalent of Lemma 7.1 by imposing the strong $L^2(D)$-convergence of the sequence $\{u_n\}$.
    
    In practice $\{u_n\}$ corresponds to the sequence of characteristic functions $\chi_{\Omega_n}$ of a minimizing sequence $\{\Omega_n\}$.
    
    To obtain the strong $L^2(D)$-convergence of a subsequence we add a constraint on the perimeters.
    
    Consider the family of finite perimeter sets in $D$ of Definition 6.2 (ii) in section 6.1,
    \begin{align*}
        \operatorname{BPS}(D) := \left\{\Omega\subset D;\chi_\Omega\in\operatorname{BV}(D)\right\},
    \end{align*}
    where $\operatorname{BV}(D)$ is defined in (6.2).
    
    Introduce the following problem indexed by $\sigma > 0$: \textbf{(7.34)-(7.35)}
    \begin{align*}
        (\mathcal{P}_\sigma^\alpha)\ &\inf_{\Omega\subset D,\,\operatorname{m}(\Omega) = \alpha} E_\sigma(\Omega),\ E_\sigma(\Omega) := E(\Omega) + \sigma P_D(\Omega),\\
        &E(\Omega) := \min_{\varphi\in H_\bullet^1(\Omega;D)} \left(\int_\Omega \frac{1}{2}|\nabla\varphi|^2 - f\varphi{\rm d}x + \int_\Omega G{\rm d}x\right).
    \end{align*}
    
    \begin{theorem}
        Let $f$ in $L^2(D)$, $G$ in $L^1(D)$, $\sigma > 0$, and $0\le\alpha < \operatorname{m}(D)$. Then problem $(P_\sigma^\alpha)$ has at least one optimal solution $\Omega$ in $\operatorname{BPS}(D)$.
    \end{theorem}    
\end{enumerate}

\subsubsection{Metrics via Distance Functions}
\begin{enumerate}
    \item In Chap. 5 the characteristic function was used to embed the equivalence classes of measurable subsets of $D$ into $L^p(D)$ or $L_{\rm loc}^p(D)$, $1\le p < \infty$, and induce a complete metric on the equivalence classes of measurable sets.
    
    This construction is generic and extends to other set-dependent functions embedded in an appropriate function space.
    
    The Hausdorff metric is the result of such a construction, where the distance function plays the role of the characteristic function.
    
    The distance function embeds equivalence classes of subsets $A$ of a closed holdall $D$ with the same closure $\overline{A}$ into the space $C(D)$ of continuous functions.
    
    When $D$ is bounded the $C^0$-norm of the difference of the distance functions of 2 subsets of $D$ is the Hausdorff metric.
    
    The Hausdorff topology has many much desired properties.
    
    In particular, for $D$ bounded, the set of equivalence classes of nonempty subsets $A$ of $D$ is compact.
    \item Yet, the volume and perimeter are not continuous w.r.t. the Hausdorff topology.
    
    This can be fixed by changing the space $C(D)$ to the space $W^{1,p}(D)$ since distance functions also belong to that space.
    
    With that metric the volume is now continuous.
    
    The price to pay is the loss of compactness when $D$ is bounded.
    
    But other sequentially compact families can easily be constructed.
    
    By analogy with Caccioppoli sets we introduce the sets for which the elements of the Hessian matrix of 2nd-order derivatives of the distance function are bounded measures.
    
    They are called sets of \textit{bounded curvature}.
    
    Their closure is a Caccioppoli set, and they enjoy other interesting properties.
    
    Convex sets belong to that family.
    
    General compactness theorems are obtained for such families under global or local conditions.
    
    This chapter also deals with the family of open sets characterized by the distance function to their complement.
    
    They are discussed in parallel along with the distance functions to a set.
    \item The properties of distance functions and Hausdorff and Hausdorff complementary metric topologies are studied in Sect. 2.
    
    The projections, skeletons, cracks, and differentiability properties of distance functions are discussed in Sect. 3.
    
    $W^{1,p}$-topologies are introduced in Sect. 4 and related to characteristic functions.
    
    The compact families of \textit{sets of bounded and locally bounded curvature} are characterized in Sect. 5.
    \item The notion of \textit{reach} and Federer's sets of \textit{positive reach} are studied in Sect. 6.
    
    The smoothness of smooth closed submanifolds is related to the smoothness of the square of the distance function in a neighborhood.
    
    Approximation of distance functions by dilated sets/tubular neighborhoods and their critical points are presented in Sect. 7.
    
    Convex sets are characterized in Sect. 8 along with the special family of convex sets.
    
    Both sets of positive reach and convex sets will be further investigated in Chap. 7.
    
    Finally Sect. 9 gives several compactness theorems under global and local conditions on the Hessian matrix of the distance function.
\end{enumerate}

\paragraph{Uniform Metric Topologies.}
\begin{enumerate}
    \item \textbf{Family of Distance Functions $C_d(D)$.} In this section we review some properties of distance functions and present the general approach that will be followed in this chapter.
    
    \begin{definition}
        Given $A\subset\mathbb{R}^N$ the \emph{distance function} from a point $x$ to $A$ is defined as \textbf{(2.1)}
        \begin{equation*}
            d_A(x) := \left\{\begin{split}
                &\inf_{y\in A} |y - x|, &&A\ne\emptyset,\\
                &+\infty, &&A = \emptyset,
            \end{split}\right.
        \end{equation*}
        and the family of all distance functions of nonempty subsets of $D$ as \textbf{(2.2)}
        \begin{align*}
            C_d(D) := \left\{d_A;\forall A,\ \emptyset\ne A\subset\overline{D}\right\}.
        \end{align*}
        When $D = \mathbb{R}^N$ the family $C_d(\mathbb{R}^N)$ is denoted by $C_d$.
    \end{definition}
    By definition $d_A$ is finite in $\mathbb{R}^N$ iff $A\ne\emptyset$.
    
    Recall the following properties of distance functions.
    
    \begin{theorem}
        Assume that $A$ and $B$ are nonempty subsets of $\mathbb{R}^N$.
        \begin{itemize}
            \item[(i)] The map $x\mapsto d_A(x)$ is uniformly Lipschitz continuous in $\mathbb{R}^N$: \textbf{(2.3)}
            \begin{align*}
                \forall x,y\in\mathbb{R}^N,\ |d_A(y) - d_A(x)|\le|y - x|
            \end{align*}
            and $d_A\in C_{\rm loc}^{0,1}(\overline{\mathbb{R}^N})$.\footnote{A function $f$ belongs to $C_{\rm loc}^{0,1}(\overline{\mathbb{R}^N})$ if for all bounded open subsets $D$ of $\mathbb{R}^N$ its restriction to $D$ belongs to $C^{0,1}(\overline{D})$.}
            \item[(ii)] There exists $p\in\overline{A}$ s.t. $d_A(x) = |p - x|$ and $d_A = d_{\overline{A}}$ in $\mathbb{R}^N$.
            \item[(iii)] $\overline{A} = \{x\in\mathbb{R}^N;d_A(x) = 0\}$.
            \item[(iv)] $d_A = 0$ in $\mathbb{R}^N$ $\Leftrightarrow\overline{A} = \mathbb{R}^N$.
            \item[(v)] $\overline{A}\subset\overline{B}\Leftrightarrow d_A\ge d_B$.
            \item[(vi)] $d_{A\cup B} = \min\{d_A,d_B\}$.
            \item[(vii)] $d_A$ is (Fr\'echet) differentiable almost everywhere and \textbf{(2.4)}
            \begin{align*}
                |\nabla d_A(x)|\le 1 \mbox{ a.e. in } \mathbb{R}^N.
            \end{align*}
        \end{itemize}
    \end{theorem}
    \item \textbf{Pompéiu-Hausdorff Metric on $C_d(D)$.} Let $D$ be a nonempty subset of $\mathbb{R}^N$ and associate with each nonempty subset $A$ of $D$ the equivalence class
    \begin{align*}
        [A]_d := \left\{B;\forall B,\ B\subset\overline{D} \mbox{ and } \overline{B} = \overline{A}\right\},
    \end{align*}
    since from Theorem 2.1 (v) $d_A = d_B$ iff $\overline{A} = \overline{B}$.
    
    So $\overline{A}$ is the invariant closed representative of the class $[A]_d$.
    
    Consider the set
    \begin{align*}
        \mathcal{F}_d(D) := \left\{[A]_d;\forall A,\ \emptyset\ne A\subset\overline{D}\right\}
    \end{align*}
    that can be identified with the set
    \begin{align*}
        \left\{A;\forall A,\ \emptyset\ne A \mbox{ closed } \subset\overline{D}\right\}.
    \end{align*}
    In general, there is no \textit{open representative} in the class $[A]_d$ since
    \begin{align*}
        d_{\overline{A}} = d_A\le d_{\operatorname{int}A},
    \end{align*}
    where $\operatorname{int}A = \overline{A^c}^c$ denotes the interior of $A$.
    
    By the definition of $[A]_d$ the map
    \begin{align*}
        [A]_d\mapsto d_A:\mathcal{F}_d(D)\to C_d(D)\subset C(\overline{D})
    \end{align*}
    is injective.
    
    So $\mathcal{F}_d(D)$ can be identified with the subset of distance functions $C_d(D)$ in $C(\overline{D})$.
    
    The distance function plays the same role as the set $\operatorname{X}(D)$ in $L^p(D)$ of equivalence classes of characteristic functions of measurable sets.
    
    %
    When $D$ is bounded, $C(D)$ is a Banach space when endowed with the norm \textbf{(2.5)}
    \begin{align*}
        \|f\|_{C(D)} = \sup_{x\in D} |f(x)|.
    \end{align*}
    As for the characteristic functions of Chap. 5, this induces a complete metric \textbf{(2.6)}
    \begin{align*}
        \rho([A]_d,[B]_d) := \|d_A - d_B\|_{C(D)} = \sup_{x\in D} |d_A(x) - d_B(x)|
    \end{align*}
    on $\mathcal{F}_d(D)$, which turns out to be equal to the classical Pompéiu-Hausdorff metric\footnote{The ``écart mutuel'' between 2 sets was introduced by D. Pompéiu [1] in his thesis presented in Paris in March 1905.
        
        This is the 1st example of a metric between 2 sets in the literature.
        
        It was studied in more detail by F. Hausdorff [2, ``Quellenangaben'', p. 280, and Chap. VIII, Sect. 6] in 1914.}
    \begin{align*}
        \rho_H([A]_d,[B]_d) := \max\left\{\sup_{x\in B} d_A(x),\sup_{y\in A} d_B(y)\right\}
    \end{align*}
    (cf. J. Dugundji [1, Chap. IX, Prob. 4.8, p. 205] for the definition of $\rho_H$).
    
    Indeed for $x\in D$, $x_A\in A$, and $x_B\in B$
    \begin{align*}
        |x - x_A|\le|x - x_B| + |x_B - x_A|\Rightarrow\ldots\Rightarrow d_A(x) - d_B(x)\le\sup_{y\in B} d_A(y).
    \end{align*}
    By interchanging the roles of $A$ and $B$
    \begin{align*}
        \forall x\in D,\ |d_A(x) - d_B(x)|\le\max\left\{\sup_{z\in B} d_A(z),\sup_{y\in A} d_B(y)\right\}.
    \end{align*}
    
\end{enumerate}

\subsubsection{Cost functionals}

\subsection{Adjoint}

\subsubsection{Discrete adjoint}

\subsubsection{Continuous adjoint}

\section{Miscellaneous}

\begin{remark}
    Shape optimization is the intersection of set theory, differential geometry, optimization, etc. see \cite{Sturm2015}. So, I will list here some developing branches.
\end{remark}

\subsection{Set theory}
\textsc{References.}
\begin{itemize}
    \item \textbf{Naive.} \cite{Halmos1974}\footnote{My 1st book in reading Advanced Mathematics.}.
    \item \textbf{Classic.} \cite{Kaplansky1977}.
    \item \textbf{Ready for shape optimization.} \cite{Delfour_Zolesio2011} $\to$ the important role of Set Theory in Shape Optimization.
\end{itemize}

\subsection{Measure theory}
\begin{itemize}
    \item \textbf{Classics.} \cite{Evans_Gariepy2015}.
\end{itemize}

\subsection{Differential geometry}
\textsc{References.}
\begin{itemize}
    \item \textbf{Classic of an expert.} \cite{Carmo2016}.
    \item \textbf{Modern.} \cite{Kuhnel2015}.
\end{itemize}

\subsection{Calculus of variations}
\textsc{References.}
\begin{itemize}
    \item \textbf{Classic.} \cite[Chap. 8: The Calculus of Variations]{Evans2010}.
\end{itemize}

%------------------------------------------------------------------------------%

\chapter{Recipes}

Coupling process$\ldots$

\begin{remark}
    This is exactly the point I need your advice to know which recipes should/will work.
\end{remark}

\section{Optimal control of PDEs}

\section{PDE-constrained topology optimization}

\section{PDE-constrained shape optimization}

\section{Fluid-structure interaction (FSI)}

\section{PDE-constrained topology/shape optimization with neural network/deep learning/machine learning/Artificial Intelligence?}
There are some references on coupling shape optimization with neural network/deep learning/machine learning/AI from the last 2-3 years, e.g.:
\begin{itemize}
    \item $\ldots$[insert refs]
\end{itemize}



- Add famous names of the corresponding communities



%------------------------------------------------------------------------------%


%------------------------------------------------------------------------------%

\printbibliography[heading=bibintoc]
\end{document}