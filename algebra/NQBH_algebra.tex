\documentclass{article}
\usepackage[backend=biber,natbib=true,style=alphabetic,maxbibnames=50]{biblatex}
\addbibresource{/home/nqbh/reference/bib.bib}
\usepackage[utf8]{vietnam}
\usepackage{tocloft}
\renewcommand{\cftsecleader}{\cftdotfill{\cftdotsep}}
\usepackage[colorlinks=true,linkcolor=blue,urlcolor=red,citecolor=magenta]{hyperref}
\usepackage{amsmath,amssymb,amsthm,enumitem,float,graphicx,mathtools,tikz}
\usetikzlibrary{angles,calc,intersections,matrix,patterns,quotes,shadings}
\allowdisplaybreaks
\newtheorem{assumption}{Assumption}
\newtheorem{baitoan}{}
\newtheorem{cauhoi}{Câu hỏi}
\newtheorem{conjecture}{Conjecture}
\newtheorem{corollary}{Corollary}
\newtheorem{dangtoan}{Dạng toán}
\newtheorem{definition}{Definition}
\newtheorem{dinhly}{Định lý}
\newtheorem{dinhnghia}{Định nghĩa}
\newtheorem{example}{Example}
\newtheorem{ghichu}{Ghi chú}
\newtheorem{hequa}{Hệ quả}
\newtheorem{hypothesis}{Hypothesis}
\newtheorem{lemma}{Lemma}
\newtheorem{luuy}{Lưu ý}
\newtheorem{nhanxet}{Nhận xét}
\newtheorem{notation}{Notation}
\newtheorem{note}{Note}
\newtheorem{principle}{Principle}
\newtheorem{problem}{Problem}
\newtheorem{proposition}{Proposition}
\newtheorem{question}{Question}
\newtheorem{remark}{Remark}
\newtheorem{theorem}{Theorem}
\newtheorem{vidu}{Ví dụ}
\usepackage[left=1cm,right=1cm,top=5mm,bottom=5mm,footskip=4mm]{geometry}
\DeclareRobustCommand{\divby}{%
	\mathrel{\vbox{\baselineskip.65ex\lineskiplimit0pt\hbox{.}\hbox{.}\hbox{.}}}%
}
\def\labelitemii{$\circ$}
\setlist[itemize]{leftmargin=*}
\setlist[enumerate]{leftmargin=*}

\title{Linear Algebra -- Đại Số Tuyến Tính}
\author{Nguyễn Quản Bá Hồng\footnote{A Scientist {\it\&} Creative Artist Wannabe. E-mail: {\tt nguyenquanbahong@gmail.com}. Bến Tre City, Việt Nam.}}
\date{\today}

\begin{document}
\maketitle
\begin{abstract}
	This text is a part of the series {\it Some Topics in Advanced STEM \& Beyond}:
	
	{\sc url}: \url{https://nqbh.github.io/advanced_STEM/}.
	
	Latest version:
	\begin{itemize}
		\item {\it Linear Algebra -- Đại Số Tuyến Tính}.
		
		PDF: {\sc url}: \url{https://github.com/NQBH/advanced_STEM_beyond/blob/main/linear_algebra/NQBH_linear_algebra.pdf}.
		
		\TeX: {\sc url}: \url{https://github.com/NQBH/advanced_STEM_beyond/blob/main/linear_algebra/NQBH_linear_algebra.tex}.
	\end{itemize}
\end{abstract}
\tableofcontents

%------------------------------------------------------------------------------%

\section{Basic}
Tôi được giải Nhì Đại số Olympic Toán Sinh viên 2014 (VMC2014) khi còn học năm nhất Đại học \& được giải Nhất Đại số Olympic Toán Sinh viên 2015 (VMC2015) khi học năm 2 Đại học. Nhưng điều đó không có nghĩa là tôi giỏi Đại số. Bằng chứng là 10 năm sau khi nhận các giải đó, tôi đang tự học lại Đại số tuyến tính với hy vọng có 1 hay nhiều cách nhìn mới mẻ hơn \& mang tính ứng dụng hơn cho các đề tài cá nhân của tôi.

\noindent\textbf{\textsf{Resources -- Tài nguyên.}}
\begin{itemize}
	\item \cite{Hung_linear_algebra}. {\sc Nguyễn Hữu Việt Hưng}. {\it Đại Số Tuyến Tính}.
	\item \cite{Tiep_ML_co_ban}. {\sc Vũ Hữu Tiệp}. {\it Machine Learning Cơ Bản}.
	
	Mã nguồn cuốn ebook ``Machine Learning Cơ Bản'': \url{https://github.com/tiepvupsu/ebookMLCB}.
	
	Phép nhân từng phần{\tt/}tích Hadamard (Hadamard product) thường xuyên được sử dụng trong ML. Tích Hadamard của 2 ma trận cùng kích thước $A,B\in\mathbb{R}^{m\times n}$, được ký hiệu là $A\odot B = (a_{ij}b_{ij})_{i,j=1}^{m,n}\in\mathbb{R}^{m\times n}$.
	
	Việc chuyển đổi hệ cơ sở sử dụng ma trận trực giao có thể được coi như 1 phép xoay trục tọa độ. Nhìn theo 1 cách khác, đây cũng chính là 1 phép xoay vector dữ liệu theo chiều ngược lại, nếu ta coi các trục tọa độ là cố định.
	
	Việc phân tích 1 đại lượng toán học ra thành các đại lượng nhỏ hơn mang lại nhiều hiệu quả. Phân tích 1 số thành tích các thừa số nguyên tố giúp kiểm tra 1 số có bao nhiêu ước số. Phân tích đa thức thành nhân tử giúp tìm nghiệm của đa thức. Việc phân tích 1 ma trận thành tích của các ma trận đặc biệt cũng mang lại nhiều lợi ích trong việc giải hệ phương trình tuyến tính, tính lũy thừa của ma trận, xấp xỉ ma trận, $\ldots$
	
	{\bf Phép phân tích trị riêng.} Cách biểu diễn 1 ma trận vuông $A$ với ${\bf x}_i\ne{\bf 0}$ là các vector riêng của 1 ma trận vuông $A$ ứng với các giá trị riêng lặp hoặc phức $\lambda_i$: $A{\bf x}_i = \lambda_i{\bf x}_i$, $\forall i = 1,\ldots,n$: $A = X\Lambda X^{-1}$ với $\Lambda = {\rm diag}(\lambda_1,\ldots,\lambda_n)$, $X = [{\bf x}_1,\ldots,{\bf x}_n]$.
	
	{\bf Norm -- Chuẩn.} Khoảng cách Euclid chính là độ dài đoạn thẳng nối 2 điểm trong mặt phẳng. Đôi khi, để đi từ 1 điểm này tới 1 điểm kia, không thể đi bằng đường thẳng vì còn phụ thuộc vào hình dạng đường đi nối giữa 2 điểm. Cf. đường trắc địa trong Hình học Vi phân -- geodesics in Differential Geometry. Việc đo khoảng cách giữa 2 điểm dữ liệu nhiều chiều rất cần thiết trong ML -- chính là lý do khái niệm {\it chuẩn} (norm) ra đời.
	
	{\bf Trace -- Vết.} {\it Vết} (trace) của 1 ma trận vuông $A$ được ký hiệu là $\operatorname{trace}A$ là tổng tất cả các phần tử trên đường chéo chính của nó. Hàm vết xác định trên tập các ma trận vuông được sử dụng nhiều trong tối ưu vì nó có các tính chất đẹp.
	
	{\bf Kiểm tra gradient.} Việc tính gradient của hàm nhiều biến thông thường khá phức tạp \& rất dễ mắc lỗi. Trong thực nghiệm, có 1 cách để kiểm tra liệu gradient tính được có chính xác không. Cách này dựa trên định nghĩa của đạo hàm cho hàm 1 biến.
	\item \cite{Trefethen_Bau1997,Trefethen_Bau2022}. {\sc Lloyd N. Trefethen, David Bau III}. {\it Numerical Linear Algebra}.
\end{itemize}

%------------------------------------------------------------------------------%

\section{Wikipedia}

\subsection{Wikipedia{\tt/}direct sum}
``The {\it direct sum} is an \href{https://en.wikipedia.org/wiki/Operation_(mathematics)}{operation} between \href{https://en.wikipedia.org/wiki/Mathematical_structure}{structures} in \href{https://en.wikipedia.org/wiki/Abstract_algebra}{abstract algebra}, a branch of mathematics. It is defined differently, but analogously, for different kinds of structures. E.g., the direct sum of 2 abelian groups $A,B$ is another abelian group $A\oplus B$ consisting of the ordered pairs $(a,b)$ where $a\in A,b\in B$. To add ordered pairs, we define the sum $(a,b) + (c,d)\coloneqq(a + c,b + d)$, i.e., addition is defined coordinate-wise. E.g., the direct sum $\mathbb{R}\oplus\mathbb{R}$ where $\mathbb{R}$ is \href{https://en.wikipedia.org/wiki/Real_coordinate_space}{real coordinate space}, is the \href{https://en.wikipedia.org/wiki/Cartesian_plane}{Cartesian plane} $\mathbb{R}^2$. A similar process can be used to form the direct sum of 2 \href{https://en.wikipedia.org/wiki/Vector_space}{vector spaces} or 2 \href{https://en.wikipedia.org/wiki/Module_(mathematics)}{modules}.

We can also form direct sums with any finite number of summands, e.g., $A\oplus B\oplus C$, provided $A,B,C$ are the same kinds of algebraic structures (e.g., all abelian groups, or all vector spaces). This relies on the fact that the direct sum is \href{https://en.wikipedia.org/wiki/Associative}{associative} \href{https://en.wikipedia.org/wiki/Up_to}{up to} \href{https://en.wikipedia.org/wiki/Isomorphism}{isomorphism}. I.e., $(A\oplus B)\oplus C\cong A\oplus(B\oplus C)$ for any algebraic structures $A,B,C$ of the same kind. The direct sum is also \href{https://en.wikipedia.org/wiki/Commutative}{commutative} up to isomorphism, i.e., $A\oplus B\cong B\oplus A$ for any algebraic structures $A,B$ of the same kind.

The direct sum of finitely many abelian groups, vector spaces, or modules is \href{https://en.wikipedia.org/wiki/Isomorphism}{canonically isomorphic} to the corresponding \href{https://en.wikipedia.org/wiki/Direct_product}{direct product}. This is false, however, for some algebraic object, like nonabelian groups.

In the case where infinitely many objects are combined, the direct sum \& direct product are not isomorphic, even for abelian groups, vector spaces, or modules. E.g., consider the direct sum \& direct product of (countably) infinitely many copies of the integers. All element in the direct product is an infinite sequence, e.g., $(1,2,3,\ldots)$ but in the direct sum, there is a requirement that all but finitely many coordinates be zero, so the sequence $(1,2,3,\ldots)$ would be an element of the direct product but not of the direct sum, while $(1,2,0,0,\ldots)$ would be an element of both. Often, if a $+$ sign is used, all but finitely many coordinates must be zero, while if some form of multiplication is used, all but finitely many coordinates must be 1. In more technical language, if the summands are $(A_i) _{i\in I}$, the direct sum $\bigoplus_{i\in I} A_i$ is defined to be the set of tuples $(a_i)_{i\in I}$ with $a_i\in A_i$ s.t. $a_i = 0$ for all but finitely many $i$. The direct sum $\bigoplus_{i\in I} A_i$ is contained in the \href{https://en.wikipedia.org/wiki/Direct_product}{direct product} $\prod_{i\in I} A_i$, but is strictly smaller when the \href{https://en.wikipedia.org/wiki/Index_set}{index set} $I$ is infinite, because an element of the direct product can have infinitely many nonzero coordinates.

\subsubsection{Examples}
The $xy$-plane, a 2D \href{https://en.wikipedia.org/wiki/Vector_space}{vector space}, can be thought of as the direct sum of 2 1D vector spaces, namely the $x$ \& $y$ axes. In this direct sum, the $x,y$ axes intersect only at the origin (the zero vector). Addition is defined coordinate-wise, i.e., $(x_1,y_1) + (x_2,y_2)\coloneqq(x_1 + x_2,y_1 + y_2)$, which is the same as vector addition.

Given 2 structures $A,B$, their direct sum is written as $A\oplus B$. Given an \href{https://en.wikipedia.org/wiki/Indexed_family}{indexed family} of structures $A_i$, indexed with $i\in I$, the direct sum may be written $A = \bigoplus_{i\in I} A_i$. Each $A_i$ is called a {\it direct summand} of $A$. If the index set is finite, the direct sum is the same as the direct product. In the case of groups, if the group operation is written as $+$ the phrase ``direct sum'' is used, while if the group operation is written $*$ the phrase ``direct product'' is used. When the index set is infinite, the direct sum is not the same as the direct product since the direct sum has the extra requirement that all but finitely many coordinates must be 0.

\paragraph{Internal \& external direct sums.} A distinction is made between internal \& external direct sums, though the 2 are isomorphic. If the summands are defined 1st, \& then the direct sum is defined in terms of the summands, we have an external direct sum. E.g., if we define the real numbers $\mathbb{R}$ \& then define $\mathbb{R}\oplus\mathbb{R}$ the direct sum is said to be {\it external}.

If, on the other hand, 1st define some algebraic structure $S$ \& then write $S$ as a direct sum of 2 substructures $V,W$, then the direct sum is said to be internal. In this case, each element of $S$ is expressible uniquely as an algebraic combination of an element of $V$ \& an element of $W$. For an example of an internal direct sum, consider $\mathbb{Z}_6$ (the integers modulo 6), whose elements are $\{0,1,2,3,4,5\}$. This is expressible as an internal direct sum $\mathbb{Z}_6 = \{0,2,4\}\oplus\{0,3\}$.

\subsubsection{Types of direct sum}
[$\ldots$]

\subsubsection{Homomorphisms}
The direct sum $\bigoplus_{i\in I} A_i$ comes equipped with a \href{https://en.wikipedia.org/wiki/Projection_(mathematics)}{\it projection} \href{https://en.wikipedia.org/wiki/Homomorphism}{homomorphism} $\pi_j:\bigoplus_{i\in I} A_i\to A_j$ for each $j\in I$ \& a {\it coprojection} $\alpha_j:A_j\to\bigoplus_{i\in I} A_i$ for each $j\in I$. Given another algebraic structure $B$ (with the same additional structure) \& homomorphisms $g_j:A_j\to B$, $\forall j\in I$, there is a unique homomorphism $g:\bigoplus_{i\in I} A_i\to B$, called the sum of the $g_j$, s.t. $g\alpha_j = g_j$, $\forall j$. Thus the direct sum is the \href{https://en.wikipedia.org/wiki/Coproduct}{coproduct} in the appropriate \href{https://en.wikipedia.org/wiki/Category_(mathematics)}{category}.'' -- \href{https://en.wikipedia.org/wiki/Direct_sum}{Wikipedia{\tt/}direct sum}

%------------------------------------------------------------------------------%

\section{Miscellaneous}

%------------------------------------------------------------------------------%

\printbibliography[heading=bibintoc]
	
\end{document}