\documentclass[12pt,landscape,a4paper]{article}
\usepackage[utf8]{vietnam}
\usepackage[colorlinks=true,linkcolor=blue,urlcolor=red,citecolor=magenta]{hyperref}
\usepackage{float,schedule}
\usepackage[left=0.5in,right=0.5in,top=2.5cm,bottom=1.5cm]{geometry}

\title{Phân Tích Ma Trận Đề Thi Olympic Toán Sinh Viên 2023--2025}
\begin{document}
\maketitle

%------------------------------------------------------------------------------%

\section*{Ký hiệu, viết tắt}

\begin{enumerate}
	\item VMC: Vietnamese Mathematical Olympiad for College Students -- Kỳ Thi Olympic Toán Sinh Viên \& Học Sinh Toàn Quốc.
	\item A$i$, $i = 1,2,\ldots,6$: Bài số $i$ của bảng A. B$i$, $i = 1,2,\ldots,6$: Bài số $i$ của bảng B.
	\item  $\equiv$ : các bài toán trùng nhau của bảng A \& bảng B.
	\item $>$: 2 bài toán có nội dung gần giống nhưng khó hơn ở vài số hạng được thêm.
\end{enumerate}

%------------------------------------------------------------------------------%

\begin{table}[H]
	\centering
	\begin{tabular}{|l|r|r|r|}
		\hline
		Chủ đề & VMC2023 & VMC2024 & VMC2025 \\
		\hline
		Dãy số hội tụ, giới hạn dãy số & A1 $\equiv$ B1 & A1 $>$ B1 & A1 $\equiv$ B1 \\
		\hline
		Tính đơn điệu của hàm số &  &  & A2 $>$ B2 \\
		\hline
		Tìm công thức dãy truy hồi &  &  & A3 $\equiv$ B3 \\
		\hline
		Hàm liên tục & A2 $\equiv$ B2 & A2 $\equiv$ B2, B5 & A4 $\equiv$ B4 \\
		\hline
		Đạo hàm, khả vi & A2 $\equiv$ B2, A5, B5 & A2 $\equiv$ B2, A4 $>$ B4 &  \\
		\hline
		Tích phân & A4, B4 & A5 & A5, B5 \\
		\hline
		Cực trị & A2 & A3 $\equiv$ B3 &  \\
		\hline
		Toán thực tế & A3 $\equiv$ B3 &  &  \\
		\hline
	\end{tabular}
\end{table}
{\sf Nhận xét.} Thường các đề bảng A \& bảng B trong 3 năm gần đây 2023--2025 sẽ có 3--4 câu trùng nhau, có vài câu ở bảng A thêm vài số hạng hoặc vài ý chứng minh phức tạp hơn bảng B nhưng giống nhau các hàm số \& các giả thiết.

\end{document}