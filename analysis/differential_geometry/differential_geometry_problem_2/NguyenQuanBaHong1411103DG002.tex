\documentclass[a4paper]{article}
\usepackage{longtable,float,hyperref,color,amsmath,amsxtra,amssymb,latexsym,amscd,amsthm,amsfonts,graphicx}
\numberwithin{equation}{section}
\allowdisplaybreaks
\usepackage{fancyhdr}
\pagestyle{fancy}
\fancyhf{}
\fancyhead[RE,LO]{\footnotesize \textsc \leftmark}
\cfoot{\thepage}
\renewcommand{\headrulewidth}{0.5pt}
\setcounter{tocdepth}{3}
\setcounter{secnumdepth}{3}
\usepackage{imakeidx}
\makeindex[columns=2, title=Alphabetical Index, 
           options= -s index.ist]
\title{\huge Differential Geometry Assignment 002}
\author{\textsc{Nguyen Quan Ba Hong}\footnote{Student ID: 1411103.}\\
{\small Students at Faculty of Math and Computer Science,}\\ 
{\small Ho Chi Minh University of Science, Vietnam} \\
{\small \texttt{email. nguyenquanbahong@gmail.com}}\\
{\small \texttt{blog. \url{www.nguyenquanbahong.com}} 
\footnote{Copyright \copyright\ 2016-2017 by Nguyen Quan Ba Hong, Student at Ho Chi Minh University of Science, Vietnam. This document may be copied freely for the purposes of education and non-commercial research. Visit my site \texttt{\url{www.nguyenquanbahong.com}} to get more.}}}
\begin{document}
\maketitle
\begin{abstract}
This context contains my solutions to \textbf{Problems 4, 8, 17}, Chapter 2, \cite{1}.
\end{abstract}
\newpage
\tableofcontents
\newpage
\section{Problems}
\textbf{Problem 1.1 (Exercise 4, p.49, \cite{1}).} \textit{A regular curve between two points $p,q$ in $\mathbb{R}^n$ with minimal length is necessarily the line segment from $p$ to $q$.}\\
\\
\textit{Hint.} Consider the Schwarz inequality $\left\langle {X,Y} \right\rangle  \le \left\| X \right\| \cdot \left\| Y \right\|$ for the tangent vector and the difference vector $p-q$.\\
\\
\textsc{Proof.} Let $c:\left[ {0,1} \right] \to {\mathbb{R}^n}$ be a regular curve between two given points $p,q$ in $\mathbb{R}^n$, i.e., $c\left(0\right)=p$, $c\left(1\right)=q$. The length of this curve is
\begin{align}
\label{1.1}
L\left( c \right) = \int_0^1 {\left\| {\dot c} \right\|dt} .
\end{align}
Applying the Schwarz inequality $\left\langle {X,Y} \right\rangle  \le \left\| X \right\| \cdot \left\| Y \right\|$ for the tangent vector ${\dot c}$ and the difference vector $q-p$ gives
\begin{align}
\label{1.2}
\left\langle {\dot c,q - p} \right\rangle  \le \left\| {\dot c} \right\| \cdot \left\| {q - p} \right\|.
\end{align}
Combining \eqref{1.1} and \eqref{1.2} yields
\begin{align}
L\left( c \right) &= \int_0^1 {\left\| {\dot c} \right\|dt} \\
&\ge \int_0^1 {\frac{{\left\langle {\dot c,q - p} \right\rangle }}{{\left\| {q - p} \right\|}}dt}\label{1.4} \\
& = \frac{{\left\langle {\int_0^1 {\dot cdt} ,q - p} \right\rangle }}{{\left\| {q - p} \right\|}}\label{1.5}\\
& = \frac{{\left\langle {c\left( b \right) - c\left( a \right),q - p} \right\rangle }}{{\left\| {q - p} \right\|}}\\
& = \frac{{\left\langle {q - p,q - p} \right\rangle }}{{\left\| {q - p} \right\|}}\\
& = \left\| {q - p} \right\|,
\end{align}
where we have used the following lemma to deduce the equality between \eqref{1.4} and \eqref{1.5}.\\
\\
\textbf{Lemma 1.2.} \textit{Let $f:\left[ {0,1} \right] \to {R^n}$ and $\alpha \in \mathbb{R}^n$ be an integrable vector-valued function and a fixed vector, respectively. Then the following equality holds}
\begin{align}
\label{1.9}
\int_0^1 {\left\langle {f\left( t \right),\alpha } \right\rangle dt}  = \left\langle {\int_0^1 {f\left( t \right)dt} ,\alpha } \right\rangle .
\end{align}
\textit{Proof of Lemma 1.2.} Let $\alpha  = \left( {{\alpha _1}, \ldots ,{\alpha _n}} \right)$ and $f\left( t \right) = \left( {{f_1}\left( t \right), \ldots ,{f_n}\left( t \right)} \right)$, we have
\begin{align}
\int_0^1 {\left\langle {f\left( t \right),\alpha } \right\rangle dt}  &= \int_0^1 {\sum\limits_{i = 1}^n {{\alpha _i}{f_i}\left( t \right)} dt} \\
 &= \sum\limits_{i = 1}^n {{\alpha _i}\int_0^1 {{f_i}\left( t \right)dt} } \\
 &= \left\langle {\int_0^1 {f\left( t \right)dt} ,\alpha } \right\rangle .
\end{align}
Hence, \eqref{1.9} holds. \hfill $\square$\\

Return to our problem, the equality holds if and only if there exists $\lambda \in \mathbb{R}$ such that $\dot c = \lambda \left( {q - p} \right)$. This is equivalent to $c\left(t\right) = \lambda \left( {q - p} \right)t + C$ where $C$ is a constant. Using $c\left(0\right)=p$, $c\left(1\right)=q$ for this parametrization of $c$ yields
\begin{align}
c\left( t \right) = \left( {q - p} \right)t + p = \left( {1 - t} \right)p + tq,
\end{align}
which is the line segment from $p$ to $q$. This completes our proof. \hfill $\square$\\
\\
\textbf{Problem 1.3 (Exercise 8, p.50, \cite{1}).} \textit{The Frenet two-frame of a plane curve with given curvature function $\kappa \left(s\right)$ can be described by the exponential series for the matrix}
\begin{align}
\left( {\begin{array}{*{20}{c}}
0&{\int_0^s {\kappa \left( t \right)dt} }\\
{ - \int_0^s {\kappa \left( t \right)dt} }&0
\end{array}} \right).
\end{align}
\textit{It follow that}
\begin{align}
\left( {\begin{array}{*{20}{c}}
{{e_1}\left( s \right)}\\
{{e_2}\left( s \right)}
\end{array}} \right) = \sum\limits_{i = 0}^\infty  {\frac{1}{{i!}}{{\left( {\begin{array}{*{20}{c}}
0&{\int_0^s \kappa  }\\
{ - \int_0^s \kappa  }&0
\end{array}} \right)}^i}} .
\end{align}
\textsc{Proof.} ``\textit{Not only does every plane curve uniquely determine its curvature function $\kappa \left(s\right)$, but also conversely, the curvature function $\kappa$ also determines the curve, up to Euclidean motions, i.e., up to the prescription of a point on the curve and the tangent of the curve at that point.}'', see p.15, \cite{1}. 

Let the curvature function $\kappa \left(s\right)$ be given. Then one can set 
\begin{align}
\label{1.16}
{e_1}\left( s \right) = \left( {\cos \left( {\alpha \left( s \right)} \right),\sin \left( {\alpha \left( s \right)} \right)} \right),
\end{align}
with a function $\alpha \left(s\right)$ which is to be found. Necessarily one has
\begin{align}
\label{1.17}
{e_2}\left( s \right) = \left( { - \sin \left( {\alpha \left( s \right)} \right),\cos \left( {\alpha \left( s \right)} \right)} \right).
\end{align}
The Frenet equation says that $\kappa {e_2} = {e_1}' = \alpha '{e_2}$, hence $\kappa =\alpha '$. By a judicious choice of adapted coordinate system we can assume that for $s=0$, the curve passes through the origin with $e_1\left(0\right)=\left(1,0\right)$; then $\alpha \left(0\right) =0$, and hence
\begin{align}
\alpha \left( s \right) = \int_0^s {\kappa \left( t \right)dt} .
\end{align}
Then \eqref{1.16} and \eqref{1.17} becomes
\begin{align}
{e_1}\left( s \right) &= \left( {\cos \left( {\int_0^s {\kappa \left( t \right)dt} } \right),\sin \left( {\int_0^s {\kappa \left( t \right)dt} } \right)} \right),\\
{e_2}\left( s \right) &= \left( { - \sin \left( {\int_0^s {\kappa \left( t \right)dt} } \right),\cos \left( {\int_0^s {\kappa \left( t \right)dt} } \right)} \right).
\end{align}

We now need the following lemma, which is the equality appeared in p.31, \cite{1}. 
\\
\textbf{Lemma 1.4.} \textit{Given a realnumber $K$, the following equality holds}
\begin{align}
\label{1.21}
\left( {\begin{array}{*{20}{c}}
{\cos K}&{\sin K}\\
{ - \sin K}&{\cos K}
\end{array}} \right) = \sum\limits_{i = 0}^\infty  {\frac{1}{{i!}}{{\left( {\begin{array}{*{20}{c}}
0&K\\
{ - K}&0
\end{array}} \right)}^i}} .
\end{align}
\textit{Proof of Lemma 1.4.} It is easy to prove 
\begin{align}
\label{1.22}
{\left( {\begin{array}{*{20}{c}}
0&K\\
{ - K}&0
\end{array}} \right)^{2i}} = {\left( { - 1} \right)^i}\left( {\begin{array}{*{20}{c}}
{{K^{2i}}}&0\\
0&{{K^{2i}}}
\end{array}} \right)\mbox{ for } i \in {\mathbb{N}*}
\end{align}
and
\begin{align}
\label{1.23}
{\left( {\begin{array}{*{20}{c}}
0&K\\
{ - K}&0
\end{array}} \right)^{2i + 1}} = {\left( { - 1} \right)^i}\left( {\begin{array}{*{20}{c}}
0&{{K^{2i + 1}}}\\
{ - {K^{2i + 1}}}&0
\end{array}} \right)\mbox{ for } i \in \mathbb{N},
\end{align}
by induction.

We recall that the Maclaurin series expansions of $\sin K$ and $\cos K$ are given by
\begin{align}
\sin K &= \sum\limits_{i = 0}^\infty  {{{\left( { - 1} \right)}^i}\frac{{{K^{2i + 1}}}}{{\left( {2i + 1} \right)!}}} ,\\
\cos K &= \sum\limits_{i = 0}^\infty  {{{\left( { - 1} \right)}^i}\frac{{{K^{2i}}}}{{\left( {2i} \right)!}}}.\label{1.25}
\end{align}
Using \eqref{1.22}-\eqref{1.25}, we can transform the right-hand side of \eqref{1.21} as follows.
\begin{align}
&\ \sum\limits_{i = 0}^\infty  {\frac{1}{{i!}}{{\left( {\begin{array}{*{20}{c}}
0&K\\
{ - K}&0
\end{array}} \right)}^i}}  \\
=&\ \sum\limits_{i = 0}^\infty  {\frac{1}{{\left( {2i} \right)!}}{{\left( {\begin{array}{*{20}{c}}
0&K\\
{ - K}&0
\end{array}} \right)}^{2i}}}  + \sum\limits_{i = 0}^\infty  {\frac{1}{{\left( {2i + 1} \right)!}}{{\left( {\begin{array}{*{20}{c}}
0&K\\
{ - K}&0
\end{array}} \right)}^{2i + 1}}} \\
 =&\ \sum\limits_{i = 0}^\infty  {\frac{{{{\left( { - 1} \right)}^i}}}{{\left( {2i} \right)!}}\left( {\begin{array}{*{20}{c}}
{{K^{2i}}}&0\\
0&{{K^{2i}}}
\end{array}} \right)}  + \sum\limits_{i = 0}^\infty  {\frac{{{{\left( { - 1} \right)}^i}}}{{\left( {2i + 1} \right)!}}\left( {\begin{array}{*{20}{c}}
0&{{K^{2i + 1}}}\\
{ - {K^{2i + 1}}}&0
\end{array}} \right)} \\
 =&\ \left( {\begin{array}{*{20}{c}}
{\sum\limits_{i = 0}^\infty  {\frac{{{{\left( { - 1} \right)}^i}}}{{\left( {2i} \right)!}}{K^{2i}}} }&0\\
0&{\sum\limits_{i = 0}^\infty  {\frac{{{{\left( { - 1} \right)}^i}}}{{\left( {2i} \right)!}}{K^{2i}}} }
\end{array}} \right) \\
&+ \left( {\begin{array}{*{20}{c}}
0&{\sum\limits_{i = 0}^\infty  {\frac{{{{\left( { - 1} \right)}^i}}}{{\left( {2i + 1} \right)!}}{K^{2i + 1}}} }\\
{ - \sum\limits_{i = 0}^\infty  {\frac{{{{\left( { - 1} \right)}^i}}}{{\left( {2i + 1} \right)!}}{K^{2i + 1}}} }&0
\end{array}} \right)\\
=&\ \left( {\begin{array}{*{20}{c}}
{\sum\limits_{i = 0}^\infty  {{{\left( { - 1} \right)}^i}\frac{{{K^{2i}}}}{{\left( {2i} \right)!}}} }&{\sum\limits_{i = 0}^\infty  {{{\left( { - 1} \right)}^i}\frac{{{K^{2i + 1}}}}{{\left( {2i + 1} \right)!}}} }\\
{ - \sum\limits_{i = 0}^\infty  {{{\left( { - 1} \right)}^i}\frac{{{K^{2i + 1}}}}{{\left( {2i + 1} \right)!}}} }&{\sum\limits_{i = 0}^\infty  {{{\left( { - 1} \right)}^i}\frac{{{K^{2i}}}}{{\left( {2i} \right)!}}} }
\end{array}} \right)\\
 =&\ \left( {\begin{array}{*{20}{c}}
{\cos K}&{\sin K}\\
{ - \sin K}&{\cos K}
\end{array}} \right).
\end{align}
Hence, \eqref{1.21} holds. \hfill $\square$\\

Return to our problem, applying Lemma 1.4 for $K = \int_0^s {\kappa \left( t \right)dt} $ yields
\begin{align}
\left( {\begin{array}{*{20}{c}}
{{e_1}\left( s \right)}\\
{{e_2}\left( s \right)}
\end{array}} \right) &= \left( {\begin{array}{*{20}{c}}
{\cos \left( {\int_0^s {\kappa \left( t \right)dt} } \right)}&{\sin \left( {\int_0^s {\kappa \left( t \right)dt} } \right)}\\
{ - \sin \left( {\int_0^s {\kappa \left( t \right)dt} } \right)}&{\cos \left( {\int_0^s {\kappa \left( t \right)dt} } \right)}
\end{array}} \right)\\
& = \sum\limits_{i = 0}^\infty  {\frac{1}{{i!}}{{\left( {\begin{array}{*{20}{c}}
0&{\int_0^s {\kappa \left( t \right)dt} }\\
{ - \int_0^s {\kappa \left( t \right)dt} }&0
\end{array}} \right)}^i}} .
\end{align}
This completes our proof. \hfill $\square$\\
\\
\textbf{Problem 1.5 (Exercise 17, p.52, \cite{1}).} \textit{In the orthogonal (but not normal) three-frame $c',c'',c'\times c''$ the Frenet equations of a space curve take the equivalent form}
\begin{align}
\label{1.35}
\left( {\begin{array}{*{20}{c}}
{c'}\\
{c''}\\
{c' \times c''}
\end{array}} \right)' = \left( {\begin{array}{*{20}{c}}
0&1&0\\
{ - {\kappa ^2}}&{\frac{{\kappa '}}{\kappa }}&\tau \\
0&{ - \tau }&{\frac{{\kappa '}}{\kappa }}
\end{array}} \right)\left( {\begin{array}{*{20}{c}}
{c'}\\
{c''}\\
{c' \times c''}
\end{array}} \right).
\end{align}
\textit{Here the entries of the matrix depend in some sense rationally (i.e., without roots) on ${\kappa ^2} = \left\langle {c'',c''} \right\rangle $ and $\tau$, because of the relation}
\begin{align}
\label{1.36}
\frac{{\kappa '}}{\kappa } = \frac{1}{2}\left( {\log \left( {{\kappa ^2}} \right)} \right)'.
\end{align}
\textsc{Proof.} We recall that the accompanying three-frame of a space curve is given by
\begin{align}
\label{1.37}
{e_1} &= c',\\
{e_2} &= \frac{{c''}}{{\left\| {c''} \right\|}} = \frac{{c''}}{\kappa },\\
{e_3} &= {e_1} \times {e_2} = \frac{{c' \times c''}}{\kappa }, \label{1.39}
\end{align}
The Frenet equation is given by
\begin{align}
\label{1.40}
{e_1}' &= \kappa {e_2},\\
{e_2}' &=  - \kappa {e_1} + \tau {e_3},\label{1.41}\\
{e_3}' &=  - \tau {e_2}, \label{1.42}
\end{align} 
Combining \eqref{1.37}-\eqref{1.39} with \eqref{1.41} yields
\begin{align}
 - \kappa c' + \frac{\tau }{\kappa }c' \times c'' &= \left( {\frac{{c''}}{\kappa }} \right)'\\
& = \frac{{c'''\kappa  - c''\kappa '}}{{{\kappa ^2}}},
\end{align}
i.e.,
\begin{align}
\label{1.45}
c''' =  - {\kappa ^2}c' + \frac{{\kappa '}}{\kappa }c'' + \tau c' \times c''
\end{align}
Combining \eqref{1.37}-\eqref{1.39} with \eqref{1.42} yields
\begin{align}
 - \frac{\tau }{\kappa }c'' &= \left( {\frac{{c' \times c''}}{\kappa }} \right)'\\
 &= \frac{{\left( {c' \times c''} \right)'\kappa  - c' \times c''\kappa '}}{{{\kappa ^2}}},
\end{align}
i.e.,
\begin{align}
\label{1.48}
\left( {c' \times c''} \right)' =  - \tau c'' + \frac{{\kappa '}}{\kappa }c' \times c''.
\end{align}
Then \eqref{1.45} and \eqref{1.48} give \eqref{1.35}. And \eqref{1.36} is obvious by calculating
\begin{align}
\frac{1}{2}\left( {\log \left( {{\kappa ^2}} \right)} \right)' = \frac{1}{2} \cdot \frac{{2\kappa \kappa '}}{{{\kappa ^2}}} = \frac{{\kappa '}}{\kappa }.
\end{align}
This completes our proof. \hfill $\square$\\
\\
\\
\\
\begin{center}
\textsc{The End}
\end{center}
\newpage
\begin{thebibliography}{999}
\bibitem {1} Wolfgang K\"{u}hnel, \textit{Differential Geometry, Curves - Surfaces - Manifolds}, Second Edition, Student Mathematical Library, Volume 16, AMS.
\end{thebibliography}
\end{document}