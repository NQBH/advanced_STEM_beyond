\documentclass{article}
\usepackage[backend=biber,natbib=true,style=alphabetic,maxbibnames=50]{biblatex}
\addbibresource{/home/nqbh/reference/bib.bib}
\usepackage[utf8]{vietnam}
\usepackage{tocloft}
\renewcommand{\cftsecleader}{\cftdotfill{\cftdotsep}}
\usepackage[colorlinks=true,linkcolor=blue,urlcolor=red,citecolor=magenta]{hyperref}
\usepackage{amsmath,amssymb,amsthm,enumitem,float,graphicx,mathtools,tikz,tipa}
\usetikzlibrary{angles,calc,intersections,matrix,patterns,quotes,shadings}
\allowdisplaybreaks
\newtheorem{assumption}{Assumption}
\newtheorem{baitoan}{Bài toán}
\newtheorem{cauhoi}{Câu hỏi}
\newtheorem{conjecture}{Conjecture}
\newtheorem{corollary}{Corollary}
\newtheorem{dangtoan}{Dạng toán}
\newtheorem{definition}{Definition}
\newtheorem{dinhly}{Định lý}
\newtheorem{dinhnghia}{Định nghĩa}
\newtheorem{example}{Example}
\newtheorem{ghichu}{Ghi chú}
\newtheorem{hequa}{Hệ quả}
\newtheorem{hypothesis}{Hypothesis}
\newtheorem{lemma}{Lemma}
\newtheorem{luuy}{Lưu ý}
\newtheorem{nhanxet}{Nhận xét}
\newtheorem{notation}{Notation}
\newtheorem{note}{Note}
\newtheorem{principle}{Principle}
\newtheorem{problem}{Problem}
\newtheorem{proposition}{Proposition}
\newtheorem{question}{Question}
\newtheorem{remark}{Remark}
\newtheorem{theorem}{Theorem}
\newtheorem{vidu}{Ví dụ}
\usepackage[left=1cm,right=1cm,top=5mm,bottom=5mm,footskip=4mm]{geometry}
\def\labelitemii{$\circ$}
\DeclareRobustCommand{\divby}{%
	\mathrel{\vbox{\baselineskip.65ex\lineskiplimit0pt\hbox{.}\hbox{.}\hbox{.}}}%
}
\setlist[itemize]{leftmargin=*}
\setlist[enumerate]{leftmargin=*}

\title{Lecture Note: Mathematical Analysis -- Bài Giảng: Giải Tích Toán Học}
\author{Nguyễn Quản Bá Hồng\footnote{A Scientist {\it\&} Creative Artist Wannabe. E-mail: {\tt nguyenquanbahong@gmail.com, hong.nguyenquanba@umt.edu.vn}. Bến Tre City, Việt Nam.}}
\date{\today}

\begin{document}
\maketitle
\begin{abstract}
	This text is a part of the series {\it Some Topics in Advanced STEM \& Beyond}:
	
	{\sc url}: \url{https://nqbh.github.io/advanced_STEM/}.
	
	Latest version:
	\begin{itemize}
		\item {\it Lecture Note: Mathematical Analysis -- Bài Giảng: Giải Tích Toán Học}.
		
		PDF: {\sc url}: \url{https://github.com/NQBH/advanced_STEM_beyond/blob/main/analysis/lecture/NQBH_mathematical_analysis_lecture.pdf}.
		
		\TeX: {\sc url}: \url{https://github.com/NQBH/advanced_STEM_beyond/blob/main/analysis/lecture/NQBH_mathematical_analysis_lecture.tex}.
		\item {\it Slide: Mathematical Analysis -- Slide: Giải Tích Toán Học}.
		
		PDF: {\sc url}: \url{https://github.com/NQBH/advanced_STEM_beyond/blob/main/analysis/slide/NQBH_mathematical_analysis_slide.pdf}.
		
		\TeX: {\sc url}: \url{https://github.com/NQBH/advanced_STEM_beyond/blob/main/analysis/slide/NQBH_mathematical_analysis_slide.tex}.
		\item Codes:
		\begin{itemize}
			\item C++: \url{https://github.com/NQBH/advanced_STEM_beyond/tree/main/analysis/C++}.
			\item Python: \url{https://github.com/NQBH/advanced_STEM_beyond/tree/main/analysis/Python}.
		\end{itemize}
	\end{itemize}
\end{abstract}
\tableofcontents

%------------------------------------------------------------------------------%

\section{Basic Mathematical Analysis -- Giải Tích Toán Học Cơ Bản}
\textbf{\textsf{Resources -- Tài nguyên.}}
\begin{enumerate}
	\item {\sc Đặng Đình Áng}. {\it Nhập Môn Giải Tích}.
	\item \cite{Rudin1976}. {\sc Walter Rudin}. {\it Principles of Mathematical Analysis}.
	
	\item \cite{Tao_analysis_1}. {\sc Terence Tao}. {\it Analysis I}.
	
	\item \cite{Tao_analysis_2}. {\sc Terence Tao}. {\it Analysis II}.
\end{enumerate}

\begin{question}[Definition of mathematical analysis]
	What is mathematical analysis? Cf. mathematical analysis with other types of analysis.
\end{question}
For answers, see, e.g., \cite[Chap. 1, Sect. 1.1: {\it What Is Analysis?}, pp. 1--2]{Tao_analysis_1}, \href{https://en.wikipedia.org/wiki/Mathematical_analysis}{Wikipedia{\tt/}mathematical analysis}. For other types of analysis, see, e.g., \href{https://en.wikipedia.org/wiki/Analysis}{Wikipedia{\tt/}analysis}.

\begin{question}[Motivation of mathematical analysis]
	Why do mathematical analysis?
\end{question}
For answers, see, e.g., \cite[Chap. 1, Sect. 1.2: {\it Why Do Analysis?}, pp. 2--10]{Tao_analysis_1}

\begin{example}[Division by zero \& infinity]
	The cancellation law for multiplication $ac = bc\Rightarrow a = b$ does not work when $c = 0$ \& $c = \pm\infty$. The cancellation law for addition $a + c = b + c\Rightarrow a = b$.
\end{example}

\begin{example}[Cancellation properties]
	
\end{example}
See, e.g., \href{https://en.wikipedia.org/wiki/Cancellation_property}{Wikipedia{\tt/}cancellation property}.

\begin{example}[Geometric series -- Chuỗi hình học]
	When does the geometric series $G(a)\coloneqq\sum_{i=0}^\infty \frac{1}{a^i}$ converge? When does $G(a)$ diverge? 
\end{example}

%------------------------------------------------------------------------------%

\subsection{Numbers -- Các loại số}
Trong chương trình Toán phổ thông, học sinh đã được học: số tự nhiên ở chương trình Toán 6 \cite{SGK_Toan_6_CD_tap_1, SGK_Toan_6_CD_tap_2}, \& số hữu tỷ \& số thực ở chương trình Toán 7,

%------------------------------------------------------------------------------%

\subsection{Notations \& conventions -- Ký hiệu \& quy ước}
Đặt tập hợp các đa thức (polynomial) 1 biến với hệ số nguyên, hệ số hữu tỷ, hệ số thực, hệ số phức lần lượt cho bởi:
\begin{align*}
	\mathbb{Z}[x]&\coloneqq\left\{\sum_{i=0}^n a_ix^i;n\in\mathbb{N},\ a_i\in\mathbb{Z},\ \forall i = 0,\ldots,n,\ a_n\ne0\right\},\\
	\mathbb{Q}[x]&\coloneqq\left\{\sum_{i=0}^n a_ix^i;n\in\mathbb{N},\ a_i\in\mathbb{Q},\ \forall i = 0,\ldots,n,\ a_n\ne0\right\},\\
	\mathbb{R}[x]&\coloneqq\left\{\sum_{i=0}^n a_ix^i;n\in\mathbb{N},\ a_i\in\mathbb{R},\ \forall i = 0,\ldots,n,\ a_n\ne0\right\},\\
	\mathbb{C}[x]&\coloneqq\left\{\sum_{i=0}^n a_ix^i;n\in\mathbb{N},\ a_i\in\mathbb{C},\ \forall i = 0,\ldots,n,\ a_n\ne0\right\}.
\end{align*}
Ta có quan hệ hiển nhiên $\mathbb{N}[x]\subset\mathbb{Z}[x]\subset\mathbb{Q}[x]\subset\mathbb{R}[x]\subset\mathbb{C}[x]$. Tổng quát, với $\mathbb{F}$ là 1 trường bất kỳ, tập hợp các đa thức 1 biến với hệ số thuộc trường $\mathbb{F}$ (e.g., $\mathbb{Z},\mathbb{Z}_p,\mathbb{Q},\mathbb{R},\mathbb{C}$) cho bởi:
\begin{equation*}
	\mathbb{F}[x]\coloneqq\left\{\sum_{i=0}^n a_ix^i;n\in\mathbb{N},\ a_i\in\mathbb{F},\ \forall i = 0,\ldots,n,\ a_n\ne0\right\}.
\end{equation*}
Tập xác định của đa thức có thể là toàn bộ trường số thực $\mathbb{R}$ hoặc trường số phức $\mathbb{C}$, i.e., $D_P = {\rm dom}(P) = \mathbb{R}$ or $D_P = {\rm dom}(P) = \mathbb{C}$, tùy vào trường $\mathbb{F}$ của các hệ số \& mục đích sử dụng đa thức.

\begin{problem}[Cf: Calculus vs. Mathematical Analysis]
	Distinguish \& compare Calculus vs. Mathematical Analysis.
\end{problem}
Analysis is more pure mathematics. Calculus is more applied mathematics.

\begin{problem}[Examples \& counterexamples in mathematical analysis -- Ví dụ \& phản ví dụ trong phân tích toán học]
	Find, from simple to advanced, examples \& counterexamples to each mathematical concepts \& mathematical results, including lemmas, propositions, theorems, \& consequences.
	
	-- Tìm các ví dụ \& phản ví dụ từ đơn giản đến nâng cao cho mỗi khái niệm toán học \& kết quả toán học, bao gồm các bổ đề, mệnh đề, định lý, \& hệ quả.
\end{problem}

\begin{problem}[Python {\tt SymPy}]
	Study {\tt SymPy} to support calculus \& mathematical analysis.
\end{problem}

\begin{definition}[Neighborhood, \cite{Wrede_Spiegel2010}, p. 6]
	The set of all points $x$ s.t. $|x - a| < \delta$, where $\delta > 0$, is called a $\delta$ {\rm neighborhood} of the point $a$. The set of all points $x$ s.t. $0 < |x - a| < \delta$, in which $x = a$ is excluded, is called a {\rm deleted $\delta$ neighborhood} of $a$ or an open ball of radius $\delta$ about $a$.
\end{definition}

\begin{theorem}[Bolzano--Weierstrass theorem]
	Every bounded infinite set has at least 1 limit point.
\end{theorem}

\begin{definition}[Algebraic- \& transcendental numbers -- số đại số \& số siêu việt]
	A number $x\in\mathbb{R}$ which is a solution to the {\rm polynomial equation}
	\begin{equation}
		\label{polynomial eqn}
		\sum_{i=0}^n a_ix^i = a_nx^n + a_{n-1}x^{n-1} + \cdots + a_1x + a_0 = 0,
	\end{equation}
	where $n\in\mathbb{N}^\star$, called the {\rm degree} of the equation, $a_i\in\mathbb{Z}$, $\forall i = 0,1,\ldots,n$, $a_n\ne0$, is called an {\rm algebraic number}. A number which cannot be expressed as a solution of any polynomial equation with integer coefficients is called a {\rm transcendental number}.
\end{definition}

\begin{theorem}[Common transcendental numbers]
	$\pi,e$ are transcendental.
\end{theorem}

\begin{theorem}[Countability of sets of algebraic- \& transcendental numbers]
	(i) The set of algebraic numbers is a countably infinite set. (ii) The set of transcendental numbers is noncountably infinite.
\end{theorem}

%------------------------------------------------------------------------------%

\section{Sequence -- Dãy Số}
\begin{itemize}\sf\small
	\item \textbf{sequence} [n] {\tt/}\textipa{'si:kw@ns}{\tt/} 1. [countable] \textit{sequence (of sth)} a set of events, actions, numbers, etc. which have a particular order \& which lead to a particular result; 2. [countable, uncountable] the order that events, actions, etc. happen in or should happen in; 3. [countable] a part of a film that deals with 1 subject or topic or consists of 1 scene. [v] 1. \textit{sequence sth} (specialist) to arrange things into a sequence; 2. \textit{sequence sth} (biology) to identify the order in which a set of genes or parts of molecules are arranged.
\end{itemize}
\textbf{\textsf{Resources -- Tài nguyên.}}
\begin{enumerate}
	\item \cite{Rudin1976}. {\sc Walter Rudin}. {\it Principles of Mathematical Analysis}. Chap. 3: Numerical Sequences \& Series.
	
	\item \cite{Tao_analysis_1}. {\sc Terence Tao}. {\it Analysis I}.
	
	\item \cite{Tao_analysis_2}. {\sc Terence Tao}. {\it Analysis II}.
	
	\item \cite{Wrede_Spiegel2010}. {\sc Robert Wrede, Murray R. Spiegel}. {\it Advanced Calculus}. 3e. Schaum's Outline Series. Chap. 2: Sequences.
\end{enumerate}
This section deals primarily with sequences of real- \& complex numbers, sequences in Euclidean spaces, or even in metric spaces.

-- Phần này chủ yếu đề cập đến các dãy số thực \& phức, các dãy trong không gian Euclid hoặc thậm chí trong không gian metric.

%------------------------------------------------------------------------------%

\subsection{Definition of a sequence -- Định nghĩa của dãy số}

\begin{definition}[Numerical sequence -- dãy số, \cite{Wrede_Spiegel2010}, p. 25]
	A {\rm sequence} is a set of numbers $u_1,u_2,\ldots$ in a definite order of arrangement (i.e., a {\rm correspondence} with the natural numbers or a subset thereof) \& formed according to a definite rule. Each number in the sequence is called a {\rm term}; $u_n$ is called the {\rm$n$th term}. The sequence is called {\rm finite} or {\rm infinite} according as there are or are not a finite number of terms. The sequence $u_1,u_2,\ldots$ is also designated briefly by $\{u_n\}$.
\end{definition}
Có thể hiểu khái niệm dãy (sequence) ở đây 1 cách tổng quát hơn là 1 dãy các đối tượng Toán học hoặc Tin học, e.g., dãy số phức $\{a_n\}_{n=1}^\infty$ là 1 dãy gồm các số $a_n\in\mathbb{C}$, $\forall n = 1,2,\ldots$, dãy các hàm số thực $\{f_n\}_{n=1}^\infty$ là 1 dãy gồm các hàm số $f_n:\mathbb{R}\to\mathbb{R}$, $\forall n = 1,2,\ldots$, hay dãy các dãy $\{\{a_{m,n}\}_{n=1}^\infty\}_{m=1}^\infty$ tức 1 dãy gồm các phần tử của dãy lại là các dãy số $\{a_{m,n}\}_{n=1}^\infty$, $\forall m = 1,2,\ldots$ Trước hết, ta tập trung là khái niệm dãy đơn giản nhất: dãy số -- numerical sequence, trước khi đến với khái niệm {\it hội tụ đều} của dãy hàm (uniform convergence of sequences of functions).

%------------------------------------------------------------------------------%

\subsection{Convergent- \& divergent sequences -- Dãy số hội tụ \& dãy số phân kỳ}

\begin{definition}[Limit of a sequence, \cite{Wrede_Spiegel2010}, p. 25]
	A number $l\in\mathbb{R}$ is called the {\rm limit} of an infinite sequence $u_1,u_2,\ldots$ if for any positive number $\epsilon$ we can find a positive number $N$ depending on $\epsilon$ s.t. $|u_n - l| < \epsilon$, $\forall n\in\mathbb{N}$, $n > N$. In such case we write $\lim_{n\to+\infty} u_n = l$.
\end{definition}

\begin{definition}[Convergent sequences, \cite{Rudin1976}, Def. 3.1, p. 47]
	\label{def: convergent sequence in metric space}
	A sequence $\{p_n\}$ in a metric space $X$ is said to {\rm converge} if there is a point $p\in X$ with the following property: For every $\varepsilon > 0$ there is an integer $N$ such that $n\ge N$ implies that $d(p_n,p) < \varepsilon$. (Here $d$ denotes the {\rm distance} in $X$.) In this case we also say that $\{p_n\}$ converges to $p$, or that $p$ is the {\rm limit} of $\{p_n\}$, \& we write $p_n\to p$, or $p_n\to p$ as $n\to\infty$, or $\lim_{n\to+\infty} p_n = p$. If $\{p_n\}$ does not converge, it is said to {\rm diverge}.
\end{definition}

\begin{remark}
	Định nghĩa \ref{def: convergent sequence in metric space} về dãy hội tụ trong các không gian metric không chỉ phụ thuộc vào bản thân dãy $\{p_n\}$ mà còn vào chính không gian metric $X$. Nhân tiện, vì ở đây đang xét không gian metric mà mỗi phần tử của nó được coi là 1 điểm (point), nên thành phần của dãy số được ký hiệu là $p_n$ để ám chỉ bản chất của mỗi phần tử của dãy là 1 điểm trong không gian metric tổng quát $X$. Nếu $X = \mathbb{R}$ hoặc $X = \mathbb{C}$ thì mỗi điểm trên trục số thực hoặc 1 số phức $z = a + bi$ tương ứng với điểm $(a,b)$ trên mặt phẳng phức $\mathbb{R}^2$, khi đó ký hiệu $p_n$ có thể được thay bởi các ký hiệu quen thuộc hơn cho số (numerals), e.g., $a_n,x_n,\ldots$
\end{remark}
In cases of possible ambiguity, we can be more precise \& specify ``convergent in $X$'' rather than ``convergent''.

-- Trong trường hợp có thể có sự mơ hồ, chúng ta có thể chính xác hơn \& cụ thể hơn ``hội tụ trong $X$'' thay vì ``hội tụ''.

\begin{theorem}[Some important properties of convergent sequences in metric spaces, \cite{Rudin1976}, Thm. 3.2, p. 48]
	Let $\{p_n\}$ be a sequence in a metric space $X$.
	\item(a) $\{p_n\}$ converges to $p\in X$ iff every neighborhood of $p$ contains all but finitely many of the terms of $\{p_n\}$.
	\item(b) {\rm(Uniqueness of limit)} If $p\in X,p'\in X$, \& if $\{p_n\}$ converges to $p$ \& to $p'$, then $p' = p$.
	\item(c) If $\{p_n\}$ converges, then $\{p_n\}$ is bounded.
	\item(d) If $E\subset X$ \& if $p$ is a limit point of $E$, then there is a sequence $\{p_n\}$ in $E$ such that $p = \lim_{n\to+\infty} p_n$.
\end{theorem}

\begin{baitoan}[\cite{Rudin1976}, p. 48, +1]
	(a) Prove that the sequence $\{\frac{1}{n}\}$ converges in $\mathbb{R} = \mathbb{R}^1$ (to $0$), but fails to converge in the set of all positive real numbers, with $d(x,y)\coloneqq|x - y|$, $\forall x,y\in X$. (b) Find similar or more advanced examples.
\end{baitoan}

\begin{baitoan}[\cite{VMS_VMC2023}, 1.1, p. 30, HCMUT]
	Cho $f\in C^1(\mathbb{R},\mathbb{R})$ thỏa $f'(x) < 0$, $\forall x\in\mathbb{R}$. Xét dãy số $\{a_n\}$:
	\begin{equation*}
		\left\{\begin{split}
			a_1 &= 1,\\
			a_{n+1} &= a_n - \frac{f(a_n)}{f'(a_n)},\ \forall n\in\mathbb{N}^\star.
		\end{split}\right.
	\end{equation*}
	(a) Nếu $f(x) > 0$, $\forall x\in\mathbb{R}$, tính $\lim_{n\to+\infty} a_n$. (b) Nếu $f(2023) = 0$ \& $f\in C^2(\mathbb{R})$ thỏa $f''(x) > 0$, $\forall x\in\mathbb{R}$, tính $\lim_{n\to+\infty} a_n$.
\end{baitoan}

\begin{baitoan}[\cite{VMS_VMC2023}, 1.2, p. 30, VNUHCM UIT]
	Cho dãy số $\{u_n\}_{n=1}^\infty$ thỏa
	\begin{equation*}
		\left\{\begin{split}
			u_0&\ge-2,\\
			u_n &= \sqrt{2 + u_{n-1}},\ \forall n\in\mathbb{N}^\star.
		\end{split}\right.
	\end{equation*}
	(a) Chứng minh $\{u_n\}$ có giới hạn hữu hạn. Tính $\lim_{n\to+\infty} u_n$. (b) Cho 2 dãy $\{v_n\}_{n=1}^\infty,\{w_n\}_{n=1}^\infty$ đặt bởi
	\begin{equation*}
		\left\{\begin{split}
			v_n &= 4^n|u_n - 2|,\\
			w_n &= \frac{u_1u_2\cdots u_n}{2^n},\ \forall n\in\mathbb{N}^\star.
		\end{split}\right.
	\end{equation*}
	Tính $\lim_{n\to+\infty} v_n,\lim_{n\to+\infty} w_n$.
\end{baitoan}

\begin{baitoan}[\cite{VMS_VMC2023}, 1.3, p. 30, ĐH Đồng Tháp]
	Xét dãy số $\{u_n\}_{n=1}^\infty$ đặt bởi
	\begin{equation*}
		u_1 = \frac{3}{2},\ u_n = 1 + \frac{1}{2}\arctan u_{n-1},\ \forall n\in\mathbb{N}^\star.
	\end{equation*}
	Chứng minh $\{u_n\}_{n=1}^\infty$ hội tụ.
\end{baitoan}

\begin{baitoan}[\cite{VMS_VMC2023}, 1.4, p. 31, ĐH Đồng Tháp]
	Cho dãy số $\{a_n\}_{n=1}^\infty$ đặt bởi
	\begin{equation*}
		a_1 = 1,\ a_{n+1} = \frac{n^2 - 1}{a_n} + 2,\ \forall n\in\mathbb{N}^\star.
	\end{equation*}
	(a) Chứng minh $n\le a_n\le n + 1$, $\forall n\in\mathbb{N}^\star$. (b) Đặt $S_n^{(3)}\coloneqq\sum_{i=1}^n a_i^3$. Tính $\lim_{n\to+\infty} \dfrac{S_n^{(3)}}{n^4}$.
\end{baitoan}

\begin{baitoan}[\cite{VMS_VMC2023}, 1.5, p. 31, ĐHGTVT]
	Cho dãy số $\{a_n\}_{n=1}^\infty$ đặt bởi
	\begin{equation*}
		a_1 > 0,\ a_{n+1} = \frac{a_n^2}{a_n^2 - a_n + 1},\ \forall n\in\mathbb{N}^\star.
	\end{equation*}
	Chứng minh $\{a_n\}_{n=1}^\infty$ giảm \& tính $\lim_{n\to+\infty} a_n$.
\end{baitoan}

\begin{baitoan}[\cite{VMS_VMC2023}, 1.6, p. 31, ĐH Hùng Vương, Phú Thọ]
	Cho dãy số $\{u_n\}_{n=1}^\infty$ đặt bởi
	\begin{equation*}
		\left\{\begin{split}
			u_0 &= 0,\ u_1 = \beta,\\
			u_{n+1} &= \frac{u_n + u_{n-1}}{2},\ \forall n\in\mathbb{N}^\star.
		\end{split}\right.
	\end{equation*}
	(a) Tìm công thức số hạng tổng quát của $\{u_n\}_{n=1}^\infty$. (b) Tính $\lim_{n\to+\infty} u_n$.
\end{baitoan}

\begin{baitoan}[\cite{VMS_VMC2023}, 1.7, p. 31, ĐHKH, Thái Nguyên]
	Cho dãy số $\{u_n\}_{n=1}^\infty$ đặt bởi
	\begin{equation*}
		x_n = \sum_{i=1}^n \frac{i}{(i + 1)!} = \frac{1}{2!} + \frac{2}{3!} + \cdots + \frac{n}{(n + 1)!},\ \forall n\in\mathbb{N}^\star.
	\end{equation*}
	Tính $\lim_{n\to+\infty} \sqrt[n]{\sum_{i=1}^{2023} x_i^n} = \lim_{n\to+\infty} \sqrt[n]{x_1^n + x_2^n + \cdots + x_{2023}^n}$.
\end{baitoan}

\begin{baitoan}[\cite{VMS_VMC2023}, 1.8, p. 31, ĐH Mỏ--Địa chất]
	Tính
	\begin{equation*}
		\lim_{n\to+\infty} \frac{\left(\prod_{i=1}^n i^{i^{2021}}\right)^{\frac{1}{n^{2022}}}}{n^{\frac{1}{2022}}} = \lim_{n\to+\infty} \frac{\left(1^{1^{2021}}\cdot2^{2^{2021}}\cdots n^{n^{2021}}\right)^{\frac{1}{n^{2022}}}}{n^{\frac{1}{2022}}}.
	\end{equation*}
\end{baitoan}

\begin{baitoan}[\cite{VMS_VMC2023}, 1.9, pp. 31--32, ĐHSPHN2]
	Cho dãy số $\{x_n\}_{n=1}^\infty$ đặt bởi
	\begin{equation*}
		x_1\in(0,1),\ x_{n+1} = \frac{1}{n}\sum_{i=1}^n \ln(1 + x_i),\ \forall n\in\mathbb{N}^\star.
	\end{equation*}
	(a) Chứng minh dãy $\{x_n\}_{n=1}^\infty$ có giới hạn hữu hạn. (b) Chứng minh $\lim_{n\to+\infty} \dfrac{n(x_n - x_{n+1})}{x_n^2} = \dfrac{1}{2}$.
\end{baitoan}

\begin{baitoan}[\cite{VMS_VMC2023}, 1.10, p. 32, ĐH Trà Vinh]
	Cho dãy số $\{x_n\}_{n=1}^\infty$ đặt bởi
	\begin{equation*}
		a_1 = a_2 = 1,\ a_{n+2} = \frac{1}{a_{n+1}} + a_n,\ \forall n\in\mathbb{N}^\star.
	\end{equation*}
	Tính $x_{2022}$.
\end{baitoan}

\begin{baitoan}[\cite{VMS_VMC2023}, 1.11, p. 32, ĐH Trà Vinh]
	Cho 2 dãy số $\{x_n\}_{n=1}^\infty,\{y_n\}_{n=1}^\infty$ đặt bởi
	\begin{equation*}
		x_1 = y_1 = \sqrt{3},\ x_{n+1} = x_n + \sqrt{1 + x_n^2},\ y_{n+1} = \frac{1}{1 + \sqrt{1 + y_n^2}},\ \forall n\in\mathbb{N}^\star.
	\end{equation*}
	Chứng minh $x_ny_n\in(2,3)$, $\forall n\ge2$ \& $\lim_{n\to+\infty} y_n = 0$.
\end{baitoan}

\begin{baitoan}[\cite{VMS_VMC2023}, 1.11, p. 32, ĐH Vinh]
	Cho dãy số $\{x_n\}_{n=1}^\infty$ đặt bởi
	\begin{equation*}
		x_n = \prod_{i=1}^n \left(1 + \frac{1}{2^i}\right) = \left(1 + \frac{1}{2}\right)\left(1 + \frac{1}{2^2}\right)\cdots\left(1 + \frac{1}{2^n}\right),\ \forall n\in\mathbb{N}^\star.
	\end{equation*}
	(a) Tìm tất cả $n\in\mathbb{N}^\star$ thỏa $x_n > \frac{15}{8}$. (b) Chứng minh $\{x_n\}_{n=1}^\infty$ hội tụ.
\end{baitoan}

\begin{baitoan}[\cite{VMS_VMC2024}, p. 32, 1.1, VNUHCM UIT]
	Cho $a,b\in\mathbb{R}$, $a < b$. Xét dãy số
	\begin{equation*}
		\left\{\begin{split}
			x_0 &= a,\ x_1 = b,\\
			x_{n+1} &= x_n + \frac{1}{2}x_{n-1}\left(1 - \cos\frac{\pi}{n}\right).
		\end{split}\right.
	\end{equation*}
	Chứng minh $\{x_n\}$ hội tụ.
\end{baitoan}

\begin{baitoan}[\cite{VMS_VMC2024}, p. 32, 1.2, ĐH Đồng Tháp]
	Cho dãy số $\{u_n\}_{n=1}^\infty$ đặt bởi
	\begin{equation*}
		u_n = \sum_{i=1}^n \frac{i}{(i + 1)!} = \frac{1}{2!} + \frac{2}{3!} + \frac{3}{4!} + \cdots + \frac{n}{(n + 1)!},\ \forall n\in\mathbb{N}^\star.
	\end{equation*}
	(a) Tìm $n\in\mathbb{N}$ lớn nhất để $u_n < \dfrac{2023}{2024}$. (b) Tính giới hạn $\lim_{n\to+\infty} \sqrt[n]{\sum_{i=1}^{2024} u_i^n} = \sqrt[n]{u_1^n + u_2^n + \cdots + u_{2024}^n}$.
\end{baitoan}

\begin{baitoan}[\cite{VMS_VMC2024}, p. 32, 1.3, ĐHGTVT]
	Cho dãy số $\{a_n\}_{n=1}^\infty$ thỏa $\frac{1}{2} < a_n < 1$, $\forall n\in\mathbb{N}^\star$. Dãy số $\{x_n\}$ đặt bởi
	\begin{equation*}
		x_1 = a_1,\ x_{n+1} = \frac{2(a_{n+1} + x_n) - 1}{1 + 2a_{n+1}x_n},\ \forall n\in\mathbb{N}^\star.
	\end{equation*}
	(a) Chứng minh dãy số $\{x_n\}_{n=1}^\infty$ tăng \& bị chặn trên. (b) Tìm $\lim_{n\to+\infty} x_n$.
\end{baitoan}

\begin{baitoan}[\cite{VMS_VMC2024}, p. 33, 1.4, ĐH Vinh]
	Cho dãy số $\{x_n\}_{n=1}^\infty$ đặt bởi
	\begin{equation*}
		\left\{\begin{split}
			x_1 &= 2024,\\
			x_{n+1} &= \frac{x_n^2}{3\lfloor x_n\rfloor + 4},\ \forall n\in\mathbb{N}^\star.
		\end{split}\right.
	\end{equation*}
	(a) Chứng minh $x_8 < 1$. (b) Chứng minh $\{x_n\}_{n=1}^\infty$ hội tụ \& tìm giới hạn.
\end{baitoan}

%------------------------------------------------------------------------------%

\section{Function -- Hàm Số}

\begin{baitoan}[\cite{VMS_VMC2023}, 3.1, p. 33, HCMUT]
	(a) Chứng minh tồn tại hàm số $f\in C^2(\mathbb{R},\mathbb{R})$ thỏa $xf''(x) + 2f'(x) = x^{2023}$, $\forall x\in\mathbb{R}$. (b) Giả sử $g\in C^2(\mathbb{R},\mathbb{R})$ thỏa $xg''(x) + 2g'(x)\ge x^{2023}$, $\forall x\in\mathbb{R}$. Chứng minh $\int_{-1}^1 x(g(x) + x^{2023})\,{\rm d}x\ge\dfrac{2}{2025}$.
\end{baitoan}

\begin{baitoan}[\cite{VMS_VMC2023}, 3.2, p. 33, ĐH Đồng Tháp]
	Cho hàm $f(x)x = 2(x - 1) - \arctan x$, $\forall x\in\mathbb{R}$. Chứng minh phương trình $f(x) = 0$ có nghiệm duy nhất là $a\in(1,\sqrt{3})$.
\end{baitoan}

\begin{proposition}[Luật bình phương nghịch đảo]
	Mỗi sự gia tăng khoảng cách từ nguồn cho ra kết quả giảm mức độ âm thanh theo tỷ lệ nghịch với bình phương của sự gia tăng khoảng cách.
\end{proposition}

\begin{baitoan}[\cite{VMS_VMC2023}, 3.3, pp. 33--34, ĐH Đồng Tháp]
	Sử dụng luật bình phương nghịch đảo, giải quyết bài toán: 1 người có 1 mảnh đất lớn có chiều dài mặt tiền là $l$ {\rm m} ở giữa 2 quán karaoke thường phát ra âm thanh có cường độ lần lượt là $I_1,I_2$. Người này định xây 1 ngôi nhà nhỏ trên mảnh đất đó nhưng muốn tìm vị trí sao cho chịu ảnh hưởng của âm thanh từ 2 quán karaoke là ít nhất. Giúp người này nếu biết: (a) Cường độ âm thanh $I_1 = I_2$. (b) Cường độ âm thanh $I_1 = 8I_2$. (c) $I_1 = aI_2$ với $a\in(0,\infty)$ cho trước.
\end{baitoan}

\begin{baitoan}[\cite{VMS_VMC2023}, 3.5, p. 34, ĐH Hùng Vương, Phú Thọ]
	Cho hàm
	\begin{equation*}
		f(x) = \left\{\begin{split}
			&x^2\sin\frac{1}{x} + \alpha x&&\mbox{if } x\ne0,\\
			&0&&\mbox{if } x = 0.
		\end{split}\right.
	\end{equation*}
	(a) Tính $f'(x)$ khi $x\ne0$. (b) Tính $f'(0)$. (c) Chứng minh hàm $f(x)$ không đơn điệu trên mỗi khoảng mở chứa điểm $0$.
\end{baitoan}

\begin{baitoan}[\cite{VMS_VMC2023}, 3.6, p. 34, ĐH Hùng Vương, Phú Thọ]
	(a) Gia đình bác Nam muốn xây 1 cái bể hình hộp với đáy là hình vuông có thể tích $V = 10\ {\rm m}^3$. Biết giá thành để xây mỗi $\rm m^2$ mặt đấy là $a = 700000$ đồng \& 1 mặt bên là $b = 500000$ đồng. Để tổng chi phí xây dựng là nhỏ nhất thì bác Nam nên xây bể với kích thước như thế nào? (b) Giải bài toán với $a,b,V\in(0,\infty)$ bất kỳ.
\end{baitoan}

\begin{baitoan}[\cite{VMS_VMC2023}, 3.7, pp. 34--35, ĐHKH Thái Nguyên]
	Tìm các hàm liên tục $f:\mathbb{R}\to\mathbb{R}$, $f\not\equiv0$, thỏa
	\begin{equation*}
		f(x + y) = 2023^yf(x) + 2023^xf(y),\ \forall x,y\in\mathbb{R}.
	\end{equation*}
	Từ đó tính
	\begin{equation*}
		\lim_{x\to0} \frac{e^{f(x)} - 1}{\sin f(x)},\ \lim_{n\to+\infty} \frac{n}{f^{(n)}(0)}.
	\end{equation*}
\end{baitoan}

\begin{baitoan}[\cite{VMS_VMC2023}, 3.8, p. 35, ĐH Mỏ--Địa chất]
	Tính
	\begin{equation*}
		\lim_{(x,y,z)\to(0,0,0)} \frac{\sin x^2 + \sin y^2 + \sin z^2}{x^2 + y^2 + z^2}.
	\end{equation*}
\end{baitoan}

\begin{baitoan}[\cite{VMS_VMC2023}, 3.9, p. 35, ĐH Mỏ--Địa chất]
	Gọi $y_1(x),y_2(x),y_3(x)$ là 3 nghiệm của phương trình vi phân $y''' + a(x)y'' + b(x)y' c(x)y = 0$ thỏa $y_1^2(x) + y_2^2(x) + y_3^2(x) = 1$, $\forall x\in\mathbb{R}$. Tìm các hằng số $\alpha,\beta$ để hàm $z = (y_1'(x))^2 + (y_2'(x))^2 + (y_3'(x))^2$ là nghiệm của phương trình vi phân $z' + \alpha a(x)z + \beta c(x) = 0$.
\end{baitoan}

\begin{baitoan}[\cite{VMS_VMC2023}, 3.10, p. 35, ĐH Mỏ--Địa chất]
	Trên hình ellipse $\dfrac{x^2}{a^2} + \dfrac{y^2}{b^2} = 1$, tìm tất cả các điểm $T = (x_0,y_0)$ thỏa: tam giác bị giới hạn bởi các đường thẳng $x = 0,y = 0$ \& tiếp tuyến với ellipse tại điểm $T$ có diện tích nhỏ nhất.
\end{baitoan}

\begin{baitoan}[\cite{VMS_VMC2023}, 3.11, p. 35, FTU Hà Nội]
	Chứng minh đa thức $f(x) = \sum_{i=0}^{2022} (-1)^i\frac{x^i}{i!} = 1 - x + \dfrac{x^2}{2!} - \dfrac{x^3}{3!} + \cdots + \dfrac{x^{2022}}{2022!}$ không có nghiệm thực.
\end{baitoan}

\begin{baitoan}[\cite{VMS_VMC2023}, 3.12, p. 35, ĐHSPHN2]
	Cho $f\in C(\mathbb{R},\mathbb{R})$, $a,b\in\mathbb{R}$, $a < b$. 1 điểm $x$ được gọi là 1 {\rm điểm mù} nếu tồn tại 1 điểm $y\in\mathbb{R}$ với $y > x$ sao cho $f(y) > f(x)$. Giả sử tất cả các điểm thuộc khoảng mở $I = (a,b)$ là các điểm mù \& $a,b$ không phải là 2 điểm mù. Chứng minh $f(a) = f(b)$.
\end{baitoan}

\begin{baitoan}[\cite{VMS_VMC2023}, 3.13, p. 36, ĐH Trà Vinh]
	Chứng minh hàm số $f(x) = x^{x^x}$ đồng biến trên $(0,\infty)$ \& $\lim_{x\to0^+} f(x) = 0$.
\end{baitoan}

\begin{baitoan}[\cite{VMS_VMC2023}, 3.14, p. 36, ĐH Vinh]
	Cho hàm
	\begin{equation*}
		f(x) = \left\{\begin{split}
			&\sqrt[3]{x^2}\sin\frac{1}{x^{2023}}&&\mbox{if } x\ne0,\\
			&0&&\mbox{if } x = 0.
		\end{split}\right.
	\end{equation*}
	(a) Chứng minh hàm số $f$ liên tục tại $x = 0$. (b) Hàm số $f$ có khả vi tại $x = 0$ hay không?
\end{baitoan}

\begin{baitoan}[\cite{VMS_VMC2023}, 3.15, p. 36, ĐH Vinh]
	Cho hàm $f\in C([0,1],\mathbb{R})$, khả vi trên khoảng $(0,1)$, thỏa $f(0) = 0$, \& $|f'(x)|\le2023|f(x)|$, $\forall x\in(0,1)$. Chứng minh $f(x) = 0$, $\forall x\in[0,1]$.
\end{baitoan}

\begin{baitoan}[\cite{VMS_VMC2023}, 3.16, p. 36, ĐH Vinh]
	Giả sử hàm $f:(0,\infty)\to\mathbb{R}$ khả vi trên khoảng $(0,\infty)$ \& thỏa 2 điều kiện: (i) $|f(x)|\le2023$, $\forall x\in(0,\infty)$; (ii) $f(x)f'(x)\ge2022\cos x$, $\forall x\in(0,\infty)$. Có tồn tại $\lim_{x\to+\infty} f(x)$ không?
\end{baitoan}

%------------------------------------------------------------------------------%

\section{Continuity -- Sự Liên Tục}

\begin{definition}[\cite{Tao_analysis_1}, Def. 6.1.1, p. 109: distance between 2 reals]
	Given $x,y\in\mathbb{R}$, their distance $d(x,y)$ is defined to be $d(x,y)\coloneqq|x - y|\in[0,\infty)$.
\end{definition}

\begin{definition}[\cite{Tao_analysis_1}, Def. 6.1.2, p. 109: $\varepsilon$-close reals]
	Let $\varepsilon > 0$ be a real number. $x,y\in\mathbb{R}$ is said to be {\rm$\varepsilon$-close} iff $d(x,y)\le\varepsilon$.
\end{definition}

%------------------------------------------------------------------------------%

\section{Series -- Chuỗi Số}

\begin{baitoan}[\cite{VMS_VMC2023}, 2.1, p. 32, VNUHCM UIT]
	Cho dãy số $\{x_n\}_{n=1}^\infty\subset(0,\infty)$ thỏa $\sum_{n=1}^\infty \dfrac{x_n}{(2n - 1)^2} < 1$. Chứng minh $\sum_{k=1}\sum_{n=1}^k \dfrac{x_n}{k^3} < 2$.
\end{baitoan}

\begin{baitoan}[\cite{VMS_VMC2023}, 2.2, p. 32, ĐHGTVT]
	Cho dãy số $\{a_n\}_{n=1}^\infty\subset(0,\infty)$ đặt bởi
	\begin{equation*}
		a_1 > 0,\ a_{n+1} = \frac{a_n^2}{a_n^2 - a_n + 1},\ \forall n\in\mathbb{N}^\star.
	\end{equation*}
	Tính $\sum_{n=1}^\infty a_n$.
\end{baitoan}

\begin{baitoan}[\cite{VMS_VMC2023}, 2.2, p. 32, ĐH Mỏ--Địa chất]
	Gọi $S$ là dãy con của dãy điều hòa $\left\{\dfrac{1}{n}\right\}_{n=1}^\infty = 1,\frac{1}{2},\frac{1}{3}\ldots,\frac{1}{n},\ldots$ \& có tổng hữu hạn. Gọi $c(n)$ là số lượng các phần tử của $S$ có số thứ tự trong dãy mẹ (điều hòa) ban đầu không vượt quá $n$. Chứng minh $\lim_{n\to+\infty} \dfrac{c(n)}{n} = 0$.
\end{baitoan}

\begin{baitoan}[\cite{VMS_VMC2024}, p. 33, 2.1, ĐHCNTT TpHCM]
	Khảo sát sự hội tụ của chuỗi số
	\begin{equation*}
		\sum_{i=1}^{+\infty} \frac{\beta\sin^2l\alpha}{1 + \beta\sin^2k\alpha},\ \alpha\notin\{k\pi:k\in\mathbb{Z}\},\,\beta > 0.
	\end{equation*}
\end{baitoan}

%------------------------------------------------------------------------------%

\section{Derivative \& Differentiability -- Đạo Hàm \& Tính Khả Vi}

\begin{baitoan}[\cite{VMS_VMC2023}, p. 36, 4.1, VNUHCM UIT]
	Cho hàm $f\in C^2(\mathbb{R})$ thỏa $f(0) = 2$, $f'(0) = -2$, $f(1) = 1$. Chứng minh tồn tại $c\in(0,1)$ thỏa $f(c)f'(c) + f''(c) = 0$.
\end{baitoan}

\begin{baitoan}[\cite{VMS_VMC2023}, p. 37, 4.2, ĐH Đồng Tháp]
	Cho $f$ khả vi trên $(a,\infty)$, $\forall a\in(0,\infty)$ \& $\lim_{x\to+\infty} f'(x) = 0$. Chứng minh $\lim_{x\to+\infty} \dfrac{f(x)}{x} = 0$.
\end{baitoan}

\begin{baitoan}[\cite{VMS_VMC2023}, p. 37, 4.3, ĐH Đồng Tháp]
	Cho $f$ là hàm số có đạo hàm $f'$ đồng biến trên $[0,2]$ \& $f(0) = -1,f(2) = 1$. Chứng minh tồn tại $a,b,c\in[0,2]$ thỏa $f'(a)f'(b)f'(c) = 1$.
\end{baitoan}

\begin{baitoan}[\cite{VMS_VMC2023}, p. 37, 4.4, ĐHGTVT]
	Cho $f\in C^\infty(\mathbb{R})$ thỏa $f^{(n)}(0) = 0$, $\forall n\in\mathbb{N}$ \& $f^{(n)}(x)x\ge0$, $\forall k\in\mathbb{N}^\star$, $\forall x\in(0,\infty)$. Chứng minh $f(x) = 0$, $\forall x\in(0,\infty)$.	 
\end{baitoan}

\begin{baitoan}[\cite{VMS_VMC2023}, p. 37, 4.5, ĐH Hùng Vương, Phú Thọ]
	Giả sử hàm $f\in C([1,2023])$, khả vi trong khoảng $(1,2023)$, \& $f(2023) = 0$. Chứng minh tồn tại $c\in(1,2023)$ thỏa
	\begin{equation*}
		f'(c) = \frac{2024 - 2023c}{1 - c}f(c).
	\end{equation*}
\end{baitoan}

\begin{baitoan}[\cite{VMS_VMC2023}, p. 37, 4.6, ĐHKH Thái Nguyên]
	Giả sử $f(x)\in C^\infty([-1,1])$, $f^{(n)}(0) = 0$, $\forall n\in\mathbb{N}$, \& tồn tại $\alpha\in(0,1)$ thỏa $\sup_{x\in[-1,1]} |f^{(n)}(x)|\le\alpha^nn!$, $\forall n\in\mathbb{N}$. Chứng minh $f(x)\equiv0$ trên đoạn $[-1,1]$.
\end{baitoan}

\begin{baitoan}[\cite{VMS_VMC2023}, p. 37, 4.7, ĐHSPHN2]
	Cho $f\in C([a,b])$ khả vi trên $(a,b)$. Giả sử $f'(x) > 0$, $\forall x\in(a,b)$. Chứng minh $\forall x_1,x_2\in\mathbb{R}$ thỏa $a\le x_1 < x_2\le b$ \& $f(x_1)f(x_2) > 0$ thì luôn tồn tại $c\in(x_1,x_2)$ thỏa
	\begin{equation*}
		\frac{x_1f(x_2) - x_2f(x_1)}{f(x_2) - f(x_1)} = c - \frac{f(c)}{f'(c)}.
	\end{equation*}
\end{baitoan}

\begin{baitoan}[\cite{VMS_VMC2024}, p. 33, 3.1, VNUHCM UIT]
	Cho $f$ là hàm số thực trên $(0,\infty)$. Giả sử
	\begin{equation*}
		f(x^\alpha) = f(x)\sin^2\alpha + f(1)\cos^2\alpha,\ \forall x\in(0,\infty),\ \forall\alpha\in\mathbb{R}.
	\end{equation*}
	Chứng minh $f$ khả vi tại $1$.
\end{baitoan}

\begin{baitoan}[\cite{VMS_VMC2024}, p. 34, 3.2, ĐH Đồng Tháp]
	(a) Chứng minh với mỗi $n\in\mathbb{N}^\star$, phương trình $2x = \sqrt{x + n} + \sqrt{x + n + 1}$ có nghiệm dương duy nhất, ký hiệu là $x_n$. (b) Tính $a\coloneqq\lim_{n\to+\infty} \dfrac{x_n}{\sqrt{n}},b\coloneqq \lim_{n\to+\infty} x_n - a\sqrt{n}$.
\end{baitoan}

\begin{baitoan}[\cite{VMS_VMC2024}, p. 34, 3.3, ĐH Đồng Tháp]
	Cho
	\begin{equation*}
		f(x) = \left\{\begin{split}
			&x^2\left|\cos\frac{\pi}{x}\right|&&\mbox{if } x\ne0,\\
			&0&&\mbox{if } x = 0.
		\end{split}\right.
	\end{equation*}
	Chứng minh $f$ khả vi tại $0$ nhưng $f$ không khả vi tại các điểm $x_n\coloneqq\dfrac{2}{2n + 1}$ với $n\in\mathbb{Z}$.
\end{baitoan}

\begin{baitoan}[\cite{VMS_VMC2024}, p. 34, 3.4, ĐH Đồng Tháp]
	Giả sử $f$ khả vi liên tục trên $(0,\infty)$, $f(0) = 1$. Chứng minh nếu $|f(x)|\le e^{-x}$, $\forall x\ge0$ thì tồn tại $x_0 > 0$ để $f'(x_0) = -e^{-x_0}$.
\end{baitoan}

\begin{baitoan}[\cite{VMS_VMC2024}, p. 34, 3.5, ĐHGTVT]
	Cho $a\in\mathbb{R}$, $b\in(0,\infty)$. Hàm $f$ xác định trên $[-1,1]$, được cho bởi
	\begin{equation*}
		f(x) = \left\{\begin{split}
			&x^a\sin x^{-b}&&\mbox{if } x\ne0,\\
			&0&&\mbox{if } x = 0.
		\end{split}\right.
	\end{equation*}
	(a) Tìm tất cả các giá trị của $a$ để hàm $f$ liên tục trên $[-1,1]$. (b) Tìm tất cả các giá trị của $a$ để tồn tại $f'(0)$. (c) Tìm điều kiện của $a,b$ để tồn tại $f''(0)$.
\end{baitoan}

\begin{baitoan}[\cite{VMS_VMC2024}, p. 35, 3.7, HUS]
	Cho $f:\mathbb{R}\to\mathbb{R}$ là hàm số được xác định bởi công thức
	\begin{equation*}
		f(x) = \left\{\begin{split}
			&x^2 + a&&\mbox{if } x\le0,\\
			&be^x + x&&\mbox{if } x > 0,
		\end{split}\right.
	\end{equation*}
	với $a,b\in\mathbb{R}$: tham số. Xác định $a,b$ để $f$ có nguyên hàm trên $\mathbb{R}$.
\end{baitoan}

\begin{baitoan}[\cite{VMS_VMC2024}, p. 35, 3.8, ĐH Vinh]
	Cho hàm $f\in C(\mathbb{R},\mathbb{R})$ thỏa $f_{2024}(x) = x$, $\forall x\in\mathbb{R}$ với
	\begin{equation*}
		\left\{\begin{split}
			f_{n+1}(x) &= f(f_n(x)),\ \forall x\in\mathbb{R},\,\forall n\in\mathbb{N}^\star,\\
			f_1(x) &= f(x),\ \forall x\in\mathbb{R}
		\end{split}\right.
	\end{equation*}
	Chứng minh $f_2(x) = x$, $\forall x\in\mathbb{R}$.
\end{baitoan}

\begin{baitoan}[\cite{VMS_VMC2024}, p. 35, 3.9, ĐH Vinh]
	Cho hàm
	\begin{equation*}
		f(x) = \left(\frac{2023^x + 2024^x}{2}\right)^{\frac{1}{x}},\ x > 0.
	\end{equation*}
	(a) Tìm $\lim_{x\to0^+} f(x)$. (b) Chứng minh $f$ là hàm số đơn điệu tăng trên $(0,+\infty)$.
\end{baitoan}

\begin{baitoan}[\cite{VMS_VMC2024}, p. 36, 4.1, HCMUT]
	(a) Cho $f\in C^3(\mathbb{R},[0,+\infty))$ thỏa $\max_{x\in\mathbb{R}} |f'''(x)|\le1$. Chứng minh
	\begin{equation*}
		f''(x)\ge-\sqrt[3]{\frac{3}{2}f(x)},\ \forall x\in\mathbb{R}.
	\end{equation*}
	(b) Tìm tất cả các hàm số $f$ thỏa mãn điều kiện của (a) thỏa
	\begin{equation*}
		f''(x) = -\sqrt[3]{\frac{3}{2}f(x)},\ \forall x\in\mathbb{R}.
	\end{equation*}
\end{baitoan}

\begin{baitoan}[\cite{VMS_VMC2024}, p. 36, 4.2, VNUHCM UIT]
	Cho hàm số $f:[0,1]\to\mathbb{R})$ liên tục trên $[0,1]$, khả vi trên $(0,1)$ sao cho $\exists M > 0$, $\exists c\in[0,1]$ thỏa $f(c) = 0$ \&
	\begin{equation*}
		|f'(x)|\le M|f(x)|,\ \forall x\in(0,1).
	\end{equation*}
	Chứng minh $f(x) = 0$, $\forall x\in[0,1]$.
\end{baitoan}

\begin{baitoan}[\cite{VMS_VMC2024}, p. 36, 4.3, ĐH Đồng Tháp]
	Cho $f$ khả vi trên $\mathbb{R}$ \& $f'$ giảm ngặt trên $\mathbb{R}$. (a) Chứng minh
	\begin{equation*}
		f(x + 1) - f(x) < f'(x) < f(x) - f(x - 1),\ \forall x\in\mathbb{R}.
	\end{equation*}
	(b) Chứng minh nếu tồn tại $\lim_{x\to+\infty} f(x) = L$ thì $\lim_{x\to+\infty} f'(x) = 0$. (c) Tìm hàm số $g$ khả vi trên $\mathbb{R}$ \& tồn tại $\lim_{x\to+\infty} g(x) = L$ nhưng $\lim_{x\to+\infty} g'(x)\ne0$.
\end{baitoan}

\begin{baitoan}[\cite{VMS_VMC2024}, p. 37, 4.4, ĐHGTVT]
	Giả sử $V$ là tập hợp các hàm liên tục $f:[0,1]\to\mathbb{R}$ \& khả vi trên $(0,1)$ thỏa $f(0) = 0,f(1) = 1$. Xác định các giá trị $\alpha\in\mathbb{R}$ để với mỗi $f\in V$, luôn tồn tại $\xi\in(0,1)$ thỏa $f(\xi) + \alpha = f'(\alpha)$.
\end{baitoan}

\begin{baitoan}[\cite{VMS_VMC2024}, p. 37, 4.5, HUS]
	Cho $f:[0,3]\to\mathbb{R}$ là hàm liên tục trên $[0,3]$ \& khả vi trong $(0,3)$. Chứng minh tồn tại $c\in(0,3)$ thỏa $2f'(c) = f(3) - f(2) + f(1) - f(0)$.
\end{baitoan}

\begin{baitoan}[\cite{VMS_VMC2024}, p. 37, 4.6, ĐH Mỏ--Địa chất]
	Giả sử có chuỗi có 2 đầu hướng ra vô cực
	\begin{equation*}
		\cdots + f''(x) + f'(x) + f(x) + \int_0^x f(t)\,{\rm d}t + \int_0^x\int_0^t f(s)\,{\rm d}s\,{\rm d}t + \cdots
	\end{equation*}
	\& hội tụ đều trên khoảng $(-1,1)$. Chuỗi là biểu diễn của số nào?
\end{baitoan}

\begin{baitoan}[\cite{VMS_VMC2024}, p. 37, 4.7, ĐH Vinh]
	Cho hàm $f\in C^2(\mathbb{R},\mathbb{R})$ \& thỏa $f(x)\le2024$, $\forall x\in\mathbb{R}$. Chứng minh tồn tại $x\in\mathbb{R}$ thỏa $f''(x) = 0$.
\end{baitoan}

%------------------------------------------------------------------------------%

\section{Integral -- Tích Phân}

\begin{baitoan}[\cite{VMS_VMC2023}, p. 38, 5.1, VNUHCM UIT]
	Cho hàm $f:(-1,1)\to\mathbb{R}$ khả vi đến cấp 2 thỏa $f(0) = 1$ \& $f''(x) + 2f'(x) + f(x)\ge1$, $\forall x\in(-1,1)$. Tìm {\rm GTNN} của $\int_{-1}^1 e^xf(x)\,{\rm d}x$.
\end{baitoan}

\begin{baitoan}[\cite{VMS_VMC2023}, p. 38, 5.2, ĐH Đồng Tháp]
	Cho hàm $f:[0,2023]\to(0,\infty)$ khả tích \& $f(x)f(2023 - x) = 1$, $\forall x\in[0,2023]$. Chứng minh $\int_0^{2023} f(x)\,{\rm d}x\ge2023$.
\end{baitoan}

\begin{baitoan}[\cite{VMS_VMC2023}, p. 38, 5.3, ĐHGTVT]
	Cho hàm $f\in C([0,1])$ thỏa $\int_0^1 f(x)\,{\rm d}x = \int_0^1 xf(x)\,{\rm d}x$. Chứng minh tồn tại $c\in(0,1)$ thỏa $cf(c) + 2023\int_0^c f(x)\,{\rm d}x = 0$.
\end{baitoan}

\begin{baitoan}[\cite{VMS_VMC2023}, p. 38, 5.4, ĐHGTVT]
	Tính
	\begin{equation*}
		I\coloneqq\int_{-\pi}^\pi \frac{\sin nx}{(1 + 2023^x)\sin x}\,{\rm d}x.
	\end{equation*}
\end{baitoan}

\begin{baitoan}[\cite{VMS_VMC2023}, p. 38, 5.5, ĐHGTVT]
	Cho hàm $f$ dương, khả tích trên $[a,b]$, $0 < m\le f(x)\le M$, $\forall x\in[a,b]$. Chứng minh
	\begin{equation*}
		(b - a)^2\le\int_a^b f(x)\,{\rm d}x\int_a^b \frac{{\rm dx}}{f(x)}\le\frac{(m + M)^2}{4mM}(b - a)^2.
	\end{equation*}
\end{baitoan}

\begin{baitoan}[\cite{VMS_VMC2023}, p. 39, 5.6, ĐHKH Thái Nguyên]
	Cho hàm $h\in C([0,1])$ thỏa $\int_0^1 xh(x)\,{\rm d}x = \int_0^1 h(x)\,{\rm d}x$. Chứng minh tồn tại $\beta\in(0,1)$ thỏa $\beta h(\beta^2) = \frac{2023}{2}\int_0^{\beta^2} h(x)\,{\rm d}x$.
\end{baitoan}

\begin{baitoan}[\cite{VMS_VMC2023}, p. 39, 5.7, ĐHKH Thái Nguyên]
	Cho $f\in C([0,\pi])$ thỏa $f(0) > 0$ \& $\int_0^\pi f(x)\,{\rm d}x < 2$. Chứng minh phương trình $f(x) = \sin x$ có ít nhất 1 nghiệm trong khoảng $(0,\pi)$.
\end{baitoan}

\begin{baitoan}[\cite{VMS_VMC2023}, p. 39, 5.8, ĐH Mỏ--Địa chất]
	Cho $f\in C([0,1]),g\in C([0,1],(0,\infty))$ với $f$ không giảm. Chứng minh
	\begin{equation*}
		\left(\int_0^t f(x)g(x)\,{\rm d}x\right)\left(\int_0^1 g(x)\,{\rm d}x\right)\le\left(\int_0^t g(x)\,{\rm d}x\right)\left(\int_0^1 f(x)g(x)\,{\rm d}x\right),\ \forall t\in[0,1].
	\end{equation*}
\end{baitoan}

\begin{baitoan}[\cite{VMS_VMC2023}, p. 39, 5.9, ĐH Mỏ--Địa chất]
	Cho $f\in C([0,1])$ thỏa $\int_0^1 f(x)\,{\rm d}x = 0$. Chứng minh tồn tại điểm $c\in(0,1)$ thỏa $\int_0^c xf(x)\,{\rm d}x = 0$.
\end{baitoan}

\begin{baitoan}[\cite{VMS_VMC2023}, p. 39, 5.10, ĐHSPHN2]
	Gọi ${\cal F}$ là lớp tất cả các hàm khả vi $f:\mathbb{R}\to(0,\infty)$ thỏa
	\begin{equation*}
		|f'(x) - f'(y)|\le2023|x - y|,\ \forall x,y\in\mathbb{R}.
	\end{equation*}
	Chứng minh
	\begin{equation*}
		(f'(x))^2 < 4046f(x),\ \forall x\in\mathbb{R}.
	\end{equation*}
\end{baitoan}

\begin{baitoan}[\cite{VMS_VMC2023}, p. 40, 5.11, ĐHSPHN2]
	Giả sử $f\in C^2([a,b])$ thỏa $f(a)\ne-f(b)$ \& $\int_a^b f(x)\,{\rm d}x = 0$. Tìm {\rm GTNN} của
	\begin{equation*}
		A\coloneqq\frac{(b - a)^3}{(f(a) + f(b))^2}\int_a^b (f''(x))^2\,{\rm d}x.
	\end{equation*}
\end{baitoan}

\begin{baitoan}[\cite{VMS_VMC2023}, p. 40, 5.12, ĐH Trà Vinh]
	Tính
	\begin{equation*}
		I\coloneqq\int_0^{2\pi} \ln(\sin x + \sqrt{1 + \sin^2x})\,{\rm d}x.
	\end{equation*}
\end{baitoan}

\begin{baitoan}[\cite{VMS_VMC2023}, p. 40, 5.12, ĐH Vinh]
	Cho $f\in C([0,1])$ thỏa $xf(y) + yf(x)\le1$, $\forall x,y\in[0,1]$. Chứng minh: (a) $f(x)\le\dfrac{1}{2x}$, $\forall x\in(0,1]$. (b) $\int_0^1 f(x)\,{\rm d}x\le\dfrac{\pi}{4}$.
\end{baitoan}

\begin{baitoan}[\cite{VMS_VMC2024}, p. 37, 5.1, VNUHCM UIT]
	Cho $\alpha\in(0,\infty)$ \& $f\in C([0,1])$ nghịch biến, $a\in(0,1)$ thỏa
	\begin{equation*}
		\int_0^a f(t)\,{\rm d}t < \frac{a}{2025},\ f(0) = \beta > 0.
	\end{equation*}
	Chứng minh phương trình $f(x) = x^{2024}$ có nghiệm trong $[0,1]$.
\end{baitoan}

\begin{baitoan}[\cite{VMS_VMC2024}, p. 38, 5.2, ĐH Đồng Tháp]
	Giả sử $f\in C^1([0,1])$ thỏa $f(0) = 0$, $0\le f'(x)\le1$, $\forall x\in[0,1]$. Xét hàm số
	\begin{equation*}
		F(t) = \left(\int_0^t f(x)\,{\rm d}x\right)^2 - \int_0^t (f(x))^3\,{\rm d}x,\ \forall t\in[0,1].
	\end{equation*}
	(a) Chứng minh $F$ đồng biến trên $[0,1]$. (b) Chứng minh
	\begin{equation*}
		\left(\int_0^1 f(x)\,{\rm d}x\right)^2\ge\int_0^1 (f(x))^3\,{\rm d}x.
	\end{equation*}
	Cho vài ví dụ về hàm $f$ để đẳng thức xảy ra.
\end{baitoan}

\begin{baitoan}[\cite{VMS_VMC2024}, p. 38, 5.3, ĐHGTVT]
	Cho $f:[0,1]\to(0,+\infty)$ là 1 hàm khả tích thỏa $f(x)f(1 - x) = 1$, $\forall x\in[0,1]$. Chứng minh $\int_0^1 f(x)\,{\rm d}x\ge1$.
\end{baitoan}

\begin{baitoan}[\cite{VMS_VMC2024}, p. 38, 5.4, HUS]
	Cho $f:[0,1]\to\mathbb{R}$ là hàm khả tích trên $[0,1]$ \& liên tục trên $(0,1)$. Chứng minh tồn tại $a,b\in(0,1)$ phân biệt sao cho
	\begin{equation*}
		\int_0^1 f(x)\,{\rm d}x = \frac{f(a) + f(b)}{2}.
	\end{equation*}
\end{baitoan}

\begin{baitoan}[\cite{VMS_VMC2024}, p. 38, 5.5, ĐH Mỏ--Địa chất]
	Tính tích phân
	\begin{equation*}
		\iiiint_{x^2 + y^2 + z^2 + t^2\le1} e^{x^2 + y^2 - z^2 - t^2}\,{\rm d}x\,{\rm d}y\,{\rm d}z\,{\rm d}t.
	\end{equation*}
\end{baitoan}

\begin{baitoan}[\cite{VMS_VMC2024}, p. 38, 5.6, ĐH Vinh]
	Chứng minh
	\begin{equation*}
		\frac{9}{8\pi} < \int_{\frac{\pi}{6}}^{\frac{\pi}{3}} \left(\frac{\sin x}{x}\right)^2\,{\rm d}x < \frac{3}{2\pi}.
	\end{equation*}
\end{baitoan}

%------------------------------------------------------------------------------%

\subsection{SymPy{\tt/integrals} module}
See \url{https://docs.sympy.org/latest/modules/integrals/integrals.html}. The {\tt integrals} module in {\tt SymPy} implements methods to calculate definite \& indefinite integrals of expressions. Principal method in this module is {\tt integrate()}:
\begin{itemize}
	\item {\tt integrate(f, x)} returns the indefinite integral $\int f\,{\rm d}x$
	\item {\tt integrate(f, (x, a, v))} returns the definite integral $\int_a^b f\,{\rm d}x$.
\end{itemize}

\begin{problem}[Integration of elementary functions]
	Use {\tt SymPy} to compute definite- \& indefinite integrals of elementary functions as many as possible.
\end{problem}

\begin{problem}[Integration of nonelementary functions]
	Use {\tt SymPy} to compute definite- \& indefinite integrals of nonelementary functions as many as possible.
\end{problem}

\begin{example}[Integral of error function]
	The indefinite integral of the nonelementary function $e^{-x^2}{\rm erf}(x)$, where ${\rm erf}(x)$ is the {\rm error function}, is given by
	\begin{equation*}
		\int e^{-x^2}{\rm erf}(x)\,{\rm d}x = \frac{\sqrt{\pi}}{4}{\rm erf}(x).
	\end{equation*}
\end{example}
Run the following Python code:
\begin{verbatim}
	from sympy import *
	x = Symbol('x')
	print(integrate(exp(-x**2)*erf(x), x))
\end{verbatim}
to obtain the following output:
\begin{verbatim}
	sqrt(pi)*erf(x)**2/4
\end{verbatim}
For more information about the error function, see, e.g., \href{https://en.wikipedia.org/wiki/Error_function}{Wikipedia{\tt/}error function}.

%------------------------------------------------------------------------------%

\subsection{Leibniz integral rule -- Quy tắc tích phân Leibniz}
In \href{https://en.wikipedia.org/wiki/Calculus}{calculus}, the {\it Leibniz integral rule} for differentiation under the integral sign, named after \href{https://en.wikipedia.org/wiki/Gottfried_Wilhelm_Leibniz}{\sc Gottfried Wilhelm Leibniz}.

\begin{theorem}[Leibniz integral rule -- Quy tắc tích phân Leibniz]
	For an integral of the form $\int_{a(x)}^{b(x)} f(t,x)\,{\rm d}t$ where $a(x),b(x)\in\mathbb{R}$ \& the integrands are functions dependent on $x$, the derivative of this integral is expressible as
	\begin{equation}
		\label{Leibniz integral rule}
		\tag{Lintr}
		\frac{d}{dx}\left(\int_{a(x)}^{b(x)} f(t,x)\,{\rm d}t\right) = f(b(x),x)\frac{d}{dx}b(x) - f(a(x),x)\frac{d}{dx}a(x) + \int_{a(x)}^{b(x)} \partial_xf(t,x)\,{\rm d}t,
	\end{equation}
	where the \href{https://en.wikipedia.org/wiki/Partial_derivative}{partial derivative} $\partial_x = \frac{\partial}{\partial x}$ indicates that inside the integral, only the variation of $f(t,x)$ with $x$ is considered in taking the derivative.
\end{theorem}

%------------------------------------------------------------------------------%

\section{Functional Equation -- Phương Trình Hàm}

\begin{baitoan}[\cite{VMS_VMC2023}, 6.1, p. 40, VNUHCM UIT]
	Tìm tất cả các hàm số $f\in C^2(\mathbb{R},(0,\infty))$ thỏa
	\begin{equation*}
		f''(x)f(x)\ge2(f'(x))^2,\ \forall x\in\mathbb{R}.
	\end{equation*}
\end{baitoan}

\begin{baitoan}[\cite{VMS_VMC2023}, 6.2, p. 40, ĐH Hùng Vương, Phú Thọ]
	Tìm tất cả các hàm số $f\in C(\mathbb{R})$ thỏa $f(1) = 2023$ \& $f(x + y) = 2023^xf(y) + 2023^yf(x)$, $\forall x,y\in\mathbb{R}$.
\end{baitoan}

\begin{baitoan}[\cite{VMS_VMC2023}, 6.3, p. 40, ĐH Hùng Vương, Phú Thọ]
	Tìm tất cả các hàm số $f(x)\in C^1([0,1])$ có $f(1) = f(0$ \& thỏa
	\begin{equation*}
		\int_0^1 \left(\dfrac{f'(x)}{f(x)}\right)^2\,{\rm d}x\le1.
	\end{equation*}
\end{baitoan}

\begin{baitoan}[\cite{VMS_VMC2023}, 6.4, p. 41, ĐH Mỏ--Địa chất]
	Cho $r,s\in\mathbb{Q}$. Tìm tất cả các hàm số $f:\mathbb{Q}\to\mathbb{Q}$ thỏa
	\begin{equation*}
		f(x + f(y)) = f(x + r) + y + s,\ \forall x,y\in\mathbb{Q}.
	\end{equation*}
\end{baitoan}

\begin{baitoan}[\cite{VMS_VMC2023}, 6.5, p. 41, FTU Hà Nội]
	Tìm tất cả các hàm số thực $f:(0,\infty)\to(0,\infty)$ thỏa
	\begin{equation*}
		f(x + f(y)) = xf\left(1 + f\left(\frac{y}{x}\right)\right),\ \forall x,y\in(0,\infty).
	\end{equation*}
\end{baitoan}

\begin{baitoan}[\cite{VMS_VMC2023}, 6.6, p. 41, ĐH Trà Vinh]
	Tìm tất cả các hàm số $f(x)$ thỏa
	\begin{equation*}
		f\left(\frac{x + 1}{x - 1}\right) = 2f(x) + \frac{3}{x - 1},\ \forall x\ne1.
	\end{equation*}
\end{baitoan}

\begin{baitoan}[\cite{VMS_VMC2023}, 6.7, p. 41, ĐH Trà Vinh]
	Tìm tất cả các hàm số $f(x)\in C^1([0,1])$ thỏa $f(1) = ef(0)$ \&
	\begin{equation*}
		\int_0^1 \left(\frac{f'(x)}{f(x)}\right)^2\,{\rm d}x\le1.
	\end{equation*}
\end{baitoan}

\begin{baitoan}[\cite{VMS_VMC2024}, p. 38, 6.1, HUS]
	Cho $f:(0,1)\to\mathbb{R}$ là 1 hàm khả vi thỏa $(f'(x))^2 - 3f'(x) + 2 = 0$, $\forall x\in(0,1)$. Tìm $f$. (b) Mở rộng bài toán cho dạng phương trình hàm phức tạp hơn.
\end{baitoan}

%------------------------------------------------------------------------------%

\section{Fourier transform -- Biến đổi Fourier}
\textbf{\textsf{Resources -- Tài nguyên.}}
\begin{enumerate}
	\item \cite{Tao_Fourier_analysis}. {\sc Terence Tao}. {\it Higher Order Fourier Analysis}.
\end{enumerate}

%------------------------------------------------------------------------------%

\subsection{Discrete Fourier transform -- Biến đổi Fourier rời rạc}
See, e.g., \href{https://en.wikipedia.org/wiki/Discrete_Fourier_transform}{Wikipedia{\tt/}discrete Fourier transform}. In mathematics, the {\it discrete Fourier transform (DFT)} converts a finite sequence of equally-spaced \href{https://en.wikipedia.org/wiki/Sampling_(signal_processing)}{samples} of a function into a same-length sequence of equally-spaced samples of the \href{https://en.wikipedia.org/wiki/Discrete-time_Fourier_transform}{discrete-time Fourier transform} (DTFT), which is a complex-valued function of frequency. The interval at which the DTFT is sampled is the reciprocal of the duration of the input sequence.

\begin{definition}[Discrete Fourier transform]
	The {\rm discrete Fourier transform} transforms a \href{https://en.wikipedia.org/wiki/Sequence}{sequence} of $N$ complex numbers ${\bf x} = \{x_n\}_{n=0}^{N-1}\coloneqq x_0,x_1,\ldots,x_{N-1}$ into another sequence of complex numbers, ${\bf X} = \{X_n\}_{n=0}^{N-1}\coloneqq X_0,X_1,\ldots,X_{N-1}$ defined by
	\begin{equation}
		\label{discrete Fourier transform}
		\tag{dFt}
		X_k\coloneqq\sum_{n=0}^{N-1} x_ne^{-i2\pi\frac{k}{N}n}.
	\end{equation}
	The transform is sometimes denoted by the symbol ${\cal F}$, as in ${\bf X} = {\cal F}\{{\bf x}\}$ or ${\cal F}({\bf x})$ or ${\cal F}{\bf x}$.
\end{definition}

%------------------------------------------------------------------------------%

\section{Miscellaneous}

\subsection{See also}

\begin{enumerate}
	\item \cite{Strogatz_infinite_power}. {\sc Steven Strogatz}. {\it Infinite Powers: How Calculus Reveals the Secrets of the Universe}.
	
	\item \cite{Strogatz_infinite_power_VN}. {\sc Steven Strogatz}. {\it Infinite Powers: How Calculus Reveals the Secrets of the Universe -- Sức Mạnh Vô Hạn: Giải Tích Toán Khám Phá Bí Mật Của Vũ Trụ Như Thế Nào?}.
	
	{\sf Nhận xét.} 1 quyển sách hay về thường thức về lịch sử phát triển của Giải tích Toán học \& các ý tưởng cơ bản nhất của Giải tích. Khuyến khích đọc thử, cũng như các tác phẩm thường thức Khoa học Tự nhiên nói chung \& Toán học nói riêng khác của tác giả {\sc Steven Strogatz}.
	\item TS. {\sc Huỳnh Quang Vũ}. {\it Các Bài Giảng Giải Tích}. \url{https://sites.google.com/view/hqvu/teaching}.
	\begin{itemize}
		\item Bộ Môn Giải Tích, Khoa Toán - Tin học, Faculty of Mathematics \& Computer Science, HCMUS. \href{https://drive.google.com/file/d/1NA1G0NSIVjnu_zG7e0JTnOvGfFqmuuVg/view}{\it Giáo Trình Vi Tích Phân 1}.
		\item Bộ Môn Giải Tích, Khoa Toán - Tin học, Faculty of Mathematics \& Computer Science, HCMUS. \href{https://drive.google.com/file/d/1Td7-zDZYFdop6f1IXvsPO0S4Cxc7ccd3/view}{\it Giáo Trình Vi Tích Phân 2}.
	\end{itemize}
	\item {\it Vietnamese Mathematical Olympiad for High School- \& College Students (VMC) -- Olympic Toán Học Học Sinh \& Sinh Viên Toàn Quốc}.
	
	PDF: {\sc url}: \url{https://github.com/NQBH/advanced_STEM_beyond/blob/main/VMC/NQBH_VMC.pdf}.
	
	\TeX: {\sc url}: \url{https://github.com/NQBH/advanced_STEM_beyond/blob/main/VMC/NQBH_VMC.tex}.
	\begin{itemize}
		\item Codes:
		\begin{itemize}
			\item C++ code: \url{https://github.com/NQBH/advanced_STEM_beyond/tree/main/VMC/C++}.
			\item Python code: \url{https://github.com/NQBH/advanced_STEM_beyond/tree/main/VMC/Python}.
		\end{itemize}
		\item Resource: \url{https://github.com/NQBH/advanced_STEM_beyond/tree/main/VMC/resource}.
		\item Figures: \url{https://github.com/NQBH/advanced_STEM_beyond/tree/main/VMC/figure}.
	\end{itemize}
	\item {\it Olympic Tin Học Sinh Viên OLP \& ICPC}.
	
	PDF: {\sc url}: \url{https://github.com/NQBH/advanced_STEM_beyond/blob/main/OLP_ICPC/NQBH_OLP_ICPC.pdf}.
	
	\TeX: {\sc url}: \url{https://github.com/NQBH/advanced_STEM_beyond/blob/main/OLP_ICPC/NQBH_OLP_ICPC.tex}.
	\begin{itemize}
		\item Codes:
		\begin{itemize}
			\item C: \url{https://github.com/NQBH/advanced_STEM_beyond/tree/main/OLP_ICPC/C}.
			\item C++: \url{https://github.com/NQBH/advanced_STEM_beyond/tree/main/OLP_ICPC/C++}.
			\item Python: \url{https://github.com/NQBH/advanced_STEM_beyond/tree/main/OLP_ICPC/Python}.
		\end{itemize}
	\end{itemize}
\end{enumerate}

%------------------------------------------------------------------------------%

\printbibliography[heading=bibintoc]
	
\end{document}