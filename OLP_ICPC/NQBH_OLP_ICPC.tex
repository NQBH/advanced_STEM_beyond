\documentclass{article}
\usepackage[backend=biber,natbib=true,style=alphabetic,maxbibnames=50]{biblatex}
\addbibresource{/home/nqbh/reference/bib.bib}
\usepackage[utf8]{vietnam}
\usepackage{tocloft}
\renewcommand{\cftsecleader}{\cftdotfill{\cftdotsep}}
\usepackage[colorlinks=true,linkcolor=blue,urlcolor=red,citecolor=magenta]{hyperref}
\usepackage{amsmath,amssymb,amsthm,enumitem,float,graphicx,mathtools,tikz}
\usetikzlibrary{angles,calc,intersections,matrix,patterns,quotes,shadings}
\allowdisplaybreaks
\newtheorem{assumption}{Assumption}
\newtheorem{baitoan}{}
\newtheorem{cauhoi}{Câu hỏi}
\newtheorem{conjecture}{Conjecture}
\newtheorem{corollary}{Corollary}
\newtheorem{dangtoan}{Dạng toán}
\newtheorem{definition}{Definition}
\newtheorem{dinhly}{Định lý}
\newtheorem{dinhnghia}{Định nghĩa}
\newtheorem{example}{Example}
\newtheorem{ghichu}{Ghi chú}
\newtheorem{hequa}{Hệ quả}
\newtheorem{hypothesis}{Hypothesis}
\newtheorem{lemma}{Lemma}
\newtheorem{luuy}{Lưu ý}
\newtheorem{nhanxet}{Nhận xét}
\newtheorem{notation}{Notation}
\newtheorem{note}{Note}
\newtheorem{principle}{Principle}
\newtheorem{problem}{Problem}
\newtheorem{proposition}{Proposition}
\newtheorem{question}{Question}
\newtheorem{remark}{Remark}
\newtheorem{theorem}{Theorem}
\newtheorem{vidu}{Ví dụ}
\usepackage[left=1cm,right=1cm,top=5mm,bottom=5mm,footskip=4mm]{geometry}
\def\labelitemii{$\circ$}
\DeclareRobustCommand{\divby}{%
	\mathrel{\vbox{\baselineskip.65ex\lineskiplimit0pt\hbox{.}\hbox{.}\hbox{.}}}%
}
\setlist[itemize]{leftmargin=*}
\setlist[enumerate]{leftmargin=*}

\title{Olympic Tin Học Sinh Viên OLP \& ACM-ICPC}
\author{Nguyễn Quản Bá Hồng\footnote{A Scientist {\it\&} Creative Artist Wannabe. E-mail: {\tt nguyenquanbahong@gmail.com}. Bến Tre City, Việt Nam.}}
\date{\today}

\begin{document}
\maketitle
\begin{abstract}
	This text is a part of the series {\it Some Topics in Advanced STEM \& Beyond}:
	
	{\sc url}: \url{https://nqbh.github.io/advanced_STEM/}.
	
	Latest version:
	\begin{itemize}
		\item {\it Olympic Tin Học Sinh Viên OLP \& ICPC}.
		
		PDF: {\sc url}: \url{https://github.com/NQBH/advanced_STEM_beyond/blob/main/OLP_ICPC/NQBH_OLP_ICPC.pdf}.
		
		\TeX: {\sc url}: \url{https://github.com/NQBH/advanced_STEM_beyond/blob/main/OLP_ICPC/NQBH_OLP_ICPC.tex}.
		\item Codes:
		\begin{itemize}
			\item C: \url{https://github.com/NQBH/advanced_STEM_beyond/tree/main/OLP_ICPC/C}.
			\item C++: \url{https://github.com/NQBH/advanced_STEM_beyond/tree/main/OLP_ICPC/C++}.
			\item Python: \url{https://github.com/NQBH/advanced_STEM_beyond/tree/main/OLP_ICPC/Python}.
		\end{itemize}
	\end{itemize}
\end{abstract}
\tableofcontents

%------------------------------------------------------------------------------%

\section{Basic Competitive Programming -- Lập Trình Thi Đấu Cơ Bản}
\textbf{\textsf{Resources -- Tài nguyên.}}
\begin{enumerate}
	\item \cite{Laaksonen2020}. {\sc Antti Laaksonen}. {\it Guide to Competitive Programming: Learning \& Improving Algorithms Through Contests}.
	\item \cite{Thu_Phuong_Tien_Triet_NMLT}. {\sc Trần Đan Thư, Nguyễn Thanh Phương, Đinh Bá Tiến, Trần Minh Triết}. {\it Nhập Môn Lập Trình}.
	\item \cite{Thu_Phuong_Tien_Triet_Phuong_KTLT}. {\sc Trần Đan Thư, Nguyễn Thanh Phương, Đinh Bá Tiến, Trần Minh Triết, Đặng Bình Phương}. {\it Kỹ Thuật Lập Trình}.
	\item \cite{Thu_Tien_Khang_PPLTHDT}. {\sc Trần Đan Thư, Đinh Bá Tiến, Nguyễn Tấn Trần Minh Khang}. {\it Phương Pháp Lập Trình Hướng Đối Tượng}.
\end{enumerate}

\subsection{Various types of inputs \& outputs -- Các dạng dữ liệu đầu vào \& đầu ra}
To compile a C++ program in Linux, run in Terminal:
\begin{verbatim}
$ g++ -O2 -Wall program_name.cpp -o program_name
$ ./program_name
\end{verbatim}
or if you want to transfer input file into it \& print output into Terminal screen:
\begin{verbatim}
$ ./program_name < program_name.inp
\end{verbatim}
or if you want to transfer input file into it \& print output into a file:
\begin{verbatim}
$ ./program_name < program_name.inp > program_name.out
\end{verbatim}
See, e.g., \cite{Laaksonen2020}.

\begin{itemize}
	\item Geeks4Geeks{\tt/std::endl} vs. \verb|\n| in C++: \url{https://www.geeksforgeeks.org/endl-vs-n-in-cpp/}. 
	\item i++ vs. ++i: \href{https://stackoverflow.com/questions/24886/is-there-a-performance-difference-between-i-and-i-in-c}{StackOverflow{\tt/}Is there a performance difference between i++ \& ++i in C?}
\end{itemize}

\subsection{Repeat{\tt/}Loop -- Lặp}

\subsection{String data -- Kiểu dữ liệu chuỗi}

%------------------------------------------------------------------------------%

\subsection{Array data -- Kiểu dữ liệu mảng}
Về mặt toán học, kiểu dữ liệu mảng là dãy số hữu hạn $(a_i)_{i=1}^n = (a_1,a_2,\ldots,a_n)$. Về mặt Tin học, kiểu dữ liệu mảng được ký hiệu bởi {\tt a[1..n]}.

\begin{baitoan}[\cite{Duc_200_BT_Python}, 141., pp. 140--141: Count digit -- Đếm chữ số]
	Cho dãy số $n$ số nguyên dương $A[1..n]$ \& 1 chữ số $k$. Đếm số lần xuất hiện chữ số $k$ trong dãy $A$ đã cho. E.g., với dãy $A[] = (11,12,13,14,15)$, thì chữ số $k = 1$ xuất hiện $6$ lần trong dãy $A$.
	\item {\sf Input.} Dòng đầu tiên của đầu vào chứa số nguyên $T\in\mathbb{N}^\star$ cho biết số bộ dữ liệu cần kiểm tra. Mỗi bộ dữ liệu gồm: (i) Dòng đầu chứa lần lượt $n,k\in\mathbb{N}$ là số phần tử trong dãy $A[]$ \& chữ số $k$. (ii) Dòng thứ 2 chứa $n$ số nguyên cách nhau 1 dấu cách, mô tả các phần tử của dãy $A$.
	\item {\sf Output.} Ứng với mỗi bộ dữ liệu, in ra 1 dòng chứa kết quả của bài toán tương ứng với bộ dữ liệu đầu vào đó.
	\item {\sf Constraint.} $1\le T\le100,1\le n\le100,0\le k\le9,1\le A[i]\le1000$, $\forall i = 1,\ldots,n$.
\end{baitoan}

\begin{itemize}
	\item Input: \url{https://github.com/NQBH/advanced_STEM_beyond/blob/main/OLP_ICPC/input/count_digit.inp}.
	\item Output: \url{https://github.com/NQBH/advanced_STEM_beyond/blob/main/OLP_ICPC/output/count_digit.out}.
	\item Python: \url{https://github.com/NQBH/advanced_STEM_beyond/blob/main/OLP_ICPC/Python/count_digit.py}.
	\item C++: ?
\end{itemize}

\begin{baitoan}[\cite{Duc_200_BT_Python}, 141., pp. 140--141: Count digit -- Đếm chữ số]
	Cho dãy số nguyên $a[1],a[2],\ldots,a[n]$. Thực hiện nhiệm vụ: Chia dãy thành 2 phần trái \& phải, trong đó phần trái gồm $\frac{n}{2}$ phần tử đầu tiên \& phần phải gồm các phần tử còn lại. Tính tổng các phần tử của mỗi phần, cuối cùng tính \& in ra tích 2 tổng tìm được.
	\item {\sf Input.} Dòng đầu tiên của đầu vào chứa $t\in\mathbb{N}^\star$ cho biết số bộ dữ liệu cần kiểm tra. Mỗi bộ dữ liệu gồm: (i) Dòng đầu chứa $n\in\mathbb{N}^\star$ cho biết số phần tử của dãy. (ii) Dòng 2 chứa $n$ số nguyên cách nhau bởi dấu cách, là các phần tử của dãy.
	\item {\sf Output.} Ứng với mỗi bộ dữ liệu, in ra 1 dong chứa kết quả của bài toán tương ứng với bộ dữ liệu đầu vào đó.
	\item {\sf Constraint.} $1\le t\le100,1\le n\le100,1\le A[i]\le100$, $\forall i = 1,\ldots,n$.
\end{baitoan}

\begin{itemize}
	\item Input: \url{https://github.com/NQBH/advanced_STEM_beyond/blob/main/OLP_ICPC/input/prod_left_right_sums.inp}.
	\item Output: \url{https://github.com/NQBH/advanced_STEM_beyond/blob/main/OLP_ICPC/output/prod_left_right_sums.out}.
	\item Python: \url{https://github.com/NQBH/advanced_STEM_beyond/blob/main/OLP_ICPC/Python/prod_left_right_sums.py}.
	\item C++: ?
\end{itemize}

%------------------------------------------------------------------------------%

\section{Olympic Tin THCS \& THPT}

\begin{baitoan}[\cite{Trung_HSG_THPT_Tin}, HSG12 Tp. Hà Nội 2020--2021, Prob. 1, p. 80: Find mid -- Tìm giữa]
	(a) Cho $l,r\in\mathbb{N}^\star$. Tìm $m\in[l,r)\cap\mathbb{N}^\star$ để chênh lệch giữa tổng các số nguyên liên tiếp từ $l$ đến $m$ \& tổng các số nguyên liên tiếp từ $m + 1$ đến $r$ là nhỏ nhất. (b) Mở rộng cho $l,r\in\mathbb{Z}$. (c$\star$) Thay tổng bởi tổng bình phương, tổng lập phương, tổng lũy thừa bậc $a\in\mathbb{R}$.
	\item {\sf Input.} 2 số $l,r\in\mathbb{N}^\star$, $l < r\le10^9$.
	\item {\sf Output.} Gồm 1 số nguyên duy nhất là $m$ thỏa mãn.
	\item {\sf Limits.} Subtask 1: $60\%$ các test có $l < r\le10^3$. Subtask 2: $40\%$ các test còn lại có $l < r\le10^9$.
\end{baitoan}

\begin{itemize}
	\item Input: \url{https://github.com/NQBH/advanced_STEM_beyond/blob/main/OLP_ICPC/input/find_mid.inp}.
	\item Output: 
	\item C++: \url{https://github.com/NQBH/advanced_STEM_beyond/blob/main/OLP_ICPC/C++/find_mid.cpp}.
	\item Python: \url{https://github.com/NQBH/advanced_STEM_beyond/blob/main/OLP_ICPC/Python/find_mid.py}.
\end{itemize}

%------------------------------------------------------------------------------%

\section{VNOI}

\begin{baitoan}[gcd in Pascal triangle -- ƯCLN trong tam giác Pascal, \url{https://oj.vnoi.info/problem/gpt}]
	Tam giác Pascal là 1 cách sắp xếp hình học của các hệ số nhị thức vào 1 tam giác. Hàng thứ $n\in\mathbb{N}$ của tam giác bao gồm các hệ số trong khai triển của đa thức $f(x,y) = (x + y)^n$. I.e., phần tử tại cột thứ $k$, hàng thứ $n$ của tam giác Pascal là $C_n^k = \binom{n}{k}$, i.e., tổ hợp chập $k$ của $n$ phần tử $0\le k\le n$. Cho $n\in\mathbb{N}$. Tính ${\rm GPT}(n)$ là ƯCLN của các số nằm giữa 2 số 1 trên hàng thứ $n$ của tam giác Pascal.
	\item {\sf Input.} Dòng đầu ghi $T$ là số lượng test. $T$ dòng tiếp theo, mỗi dòng ghi 1 số nguyên $n$.
	\item {\sf Output.} Gồm $T$ dòng, mỗi dòng ghi ${\rm GPT}(n)$ tương ứng.
	\item {\sf Constraint.} $1\le T\le20$, $2\le n\le10^9$.
\end{baitoan}
{\it Phân tích.} Công thức khai triển nhị thức Newton: $(a + b)^n = \sum_{i=0}^n C_n^ia^{n-i}b^i$, $\forall n\in\mathbb{N}$, see, e.g., \href{https://en.wikipedia.org/wiki/Binomial_theorem}{Wikipedia{\tt/}binomial theorem}. Cần tính $\gcd(\{C_n^i;1\le i\le n - 1\}) = \gcd(C_n^1,C_n^2,\ldots,C_n^{n-1})$. Chú ý mỗi hàng của tam giác Pascal có tính chất đối xứng nên chỉ cần xét ``1 nửa'' là đủ. Cụ thể hơn: $C_n^k = C_n^{n-k}$, $\forall k\in\mathbb{N}$, $k\le n$, nên
\begin{equation*}
	\{C_n^1,\ldots,C_n^{n-1}\} = \{C_n^1,\ldots,C_n^{\lfloor\frac{n}{2}\rfloor}\} = \left\{\begin{split}
		&\{C_n^1,\ldots,C_n^{\frac{n-1}{2}}\}&&\mbox{if } n\not{\divby}\ 2,\\
		&\{C_n^1,\ldots,C_n^{\frac{n}{2}}\}&&\mbox{if } n\divby2,
	\end{split}\right.
\end{equation*}
nên thay vì xét $i = 1,\ldots,n-1$, chỉ cần xét $i = 1,\ldots,\lfloor\frac{n}{2}\rfloor$ là đủ.

\begin{theorem}
	\begin{equation*}
		\gcd\{C_n^i\}_{i=1}^{n-1} = \left\{\begin{split}
			&p&&\mbox{if } n = p^k\mbox{ for some prime } p\mbox{ \& some } n\in\mathbb{N}^\star,\\
			&1&&\mbox{if } n\ne p^k\mbox{ for all prime } p\mbox{ \& any } n\in\mathbb{N}^\star.
		\end{split}\right.
	\end{equation*}
\end{theorem}
See also, e.g.:
\begin{itemize}
	\item \href{https://math.stackexchange.com/questions/2067235/gcd-of-binomial-coefficients}{Mathematics StackExchange{\tt/}gcd of binomial coefficients}.
\end{itemize}

%------------------------------------------------------------------------------%

\section{CSES Problem Set}
Link: \url{https://cses.fi/problemset/}.

\subsection{Introductory Problems}

\begin{problem}[CSES]
	There are $n$ concert tickets available, each with a certain price. Then, $m$ customers arrive, one after another. Each customer announces the maximum price they are willing to pay for a ticket, \& after this, they will get a ticket with nearest possible price such that it does not exceed the maximum price.
	\item {\sf Input.} 1st input line contains $n,m\in\mathbb{N}$: number of tickets \& number of customers. The next line contains $n$ integers $h_1,h_2,\ldots,h_n$: the price of each ticket. The last line contains $m$ integers $t_1,t_2,\ldots,t_m$: the maximum price for each customer in the order they arrive.
	\item {\sf Output.} Print, for each customer, the price that they will pay for their ticket. After this, ticket cannot be purchased again. If a customer cannot get any ticket, print $-1$.
	\item {\sf Constraints.} $1\le m,n\le 2\cdot10^5$, $1\le h_i,t_i\le10^9$.
\end{problem}

\subsection{Dynamic Programming}
\textbf{\textsf{Resources -- Tài nguyên.}}
\begin{enumerate}
	\item \cite{Bertsekas2005,Bertsekas2017}. {\sc Dimitri P. Bertsekas}. {\it Dynamic Programming \& Optimal Control. Vol. I}. 3e. 4e (can't download yet).
	\item \cite{Bertsekas2007,Bertsekas2012} {\sc Dimitri P. Bertsekas}. {\it Dynamic Programming \& Optimal Control. Vol. II}. 3e. 4e (can't download yet).
\end{enumerate}

\subsection{Graph Algorithms}

\subsection{Range Queries}

\subsection{Mathematics}

\begin{problem}[CSES{\tt/}Josephus Queries, \url{https://cses.fi/problemset/task/2164}]
	Consider a game where there are $n\in\mathbb{N}^\star$ children, numbered $1,2,\ldots,n$, in a circle. During the game, every 2nd child is removed from circle, until there are no children left. Task: process $q$ queries of the form: ``when there are $n$ children, who is the $k$th child that will be removed?''
	\begin{itemize}
		\item {\sf Input.} The 1st input line has an integer $q$: the number of queries. After this, there are $q$ lines that describe the queries. Each line has 2 integers $n,k$: the number of children \& the position of the child.
		\item {\sf Output.} Print $q$ integers: the answer for each query.
	\end{itemize}
\end{problem}
It seems to me that {\sf Jack97} (nickname: \verb|abortion_grandmaster|) proposed this problem.

Codes:
\begin{itemize}
	\item C++: \url{https://github.com/NQBH/advanced_STEM_beyond/blob/main/OLP_ICPC/C%2B%2B/gcd_Pascal_triangle.cpp}.
\end{itemize}

\begin{problem}[CSES{\tt/}Dice Probability, \url{https://cses.fi/problemset/task/1725}]
	Throw a dice $n\in\mathbb{N}^\star$ times, \& every throw produces an outcome between $1$ \& $6$. What is the probability that the sum of outcomes is between $a,b\in\mathbb{Z}$?
	\begin{itemize}
		\item {\sf Input.} The only input line contains 3 integers $n,a,b\in\mathbb{N}^\star$.
		\item {\sf Output.} Print probability rounded to 6 decimal places (rounding half to even).
		\item {\sf Constraints.} $1\le n\le100,1\le a\le b\le6n$.
		\item {\sf Example.} Input: {\tt2 9 10}. Output: {\tt0.194444}.
	\end{itemize}
\end{problem}
{\it Phân tích.} Gọi $n$ outcomes là $a_1,\ldots,a_n\in\{1,\ldots,6\}$. Sum of outcomes: $S\coloneqq\sum_{i=1}^n a_i\in\{n,\ldots,6n\}$.

\subsection{String Algorithms}

\subsection{Geometry}

\subsection{Advanced Techniques}

\subsection{Additional Problems}

%------------------------------------------------------------------------------%

\section{OLP}

%------------------------------------------------------------------------------%

\section{ICPC}

%------------------------------------------------------------------------------%

\section{Miscellaneous}

\subsection{Contributors}

\begin{enumerate}
	\item {\sc Võ Ngọc Trâm Anh.} C++ codes.
	\item {\sc Đặng Phúc An Khang.} C++ codes.
	\begin{itemize}
		\item {\it Combinatorics \& Number Theory in Competitive Programming -- Tổ Hợp \& Lý Thuyết Số trong Lập Trình Thi Đấu}.
	\end{itemize}
	\item {\sc Nguyễn Lê Anh Khoa}: C++ codes.
	\item {\sc Phan Vĩnh Tiến.} C++ codes.
\end{enumerate}

\subsection{Donate or Buy Me Coffee}
Donate (but do not donut) or buy me some coffee via NQBH's bank account information at \url{https://github.com/NQBH/publication/blob/master/bank/NQBH_bank_account_information}.

\subsection{See also}

\begin{enumerate}
	\item {\it Vietnamese Mathematical Olympiad for High School- \& College Students (VMC) -- Olympic Toán Học Học Sinh \& Sinh Viên Toàn Quốc}.
	
	PDF: {\sc url}: \url{https://github.com/NQBH/advanced_STEM_beyond/blob/main/VMC/NQBH_VMC.pdf}.
	
	\TeX: {\sc url}: \url{https://github.com/NQBH/advanced_STEM_beyond/blob/main/VMC/NQBH_VMC.tex}.
	\begin{itemize}
		\item Codes:
		\begin{itemize}
			\item C++ code: \url{https://github.com/NQBH/advanced_STEM_beyond/tree/main/VMC/C++}.
			\item Python code: \url{https://github.com/NQBH/advanced_STEM_beyond/tree/main/VMC/Python}.
		\end{itemize}
		\item Resource: \url{https://github.com/NQBH/advanced_STEM_beyond/tree/main/VMC/resource}.
		\item Figures: \url{https://github.com/NQBH/advanced_STEM_beyond/tree/main/VMC/figure}.
	\end{itemize}
\end{enumerate}

%------------------------------------------------------------------------------%

\printbibliography[heading=bibintoc]
	
\end{document}