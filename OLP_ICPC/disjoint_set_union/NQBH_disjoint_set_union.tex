\documentclass{article}
\usepackage[backend=biber,natbib=true,style=alphabetic,maxbibnames=50]{biblatex}
\addbibresource{/home/nqbh/reference/bib.bib}
\usepackage[utf8]{vietnam}
\usepackage{tocloft}
\renewcommand{\cftsecleader}{\cftdotfill{\cftdotsep}}
\usepackage[colorlinks=true,linkcolor=blue,urlcolor=red,citecolor=magenta]{hyperref}
\usepackage{amsmath,amssymb,amsthm,enumitem,fancyvrb,float,graphicx,mathtools,tikz}
\usetikzlibrary{angles,calc,intersections,matrix,patterns,quotes,shadings}
\allowdisplaybreaks
\newtheorem{assumption}{Assumption}
\newtheorem{baitoan}{Bài toán}
\newtheorem{cauhoi}{Câu hỏi}
\newtheorem{conjecture}{Conjecture}
\newtheorem{corollary}{Corollary}
\newtheorem{dangtoan}{Dạng toán}
\newtheorem{definition}{Definition}
\newtheorem{dinhly}{Định lý}
\newtheorem{dinhnghia}{Định nghĩa}
\newtheorem{example}{Example}
\newtheorem{ghichu}{Ghi chú}
\newtheorem{hequa}{Hệ quả}
\newtheorem{hypothesis}{Hypothesis}
\newtheorem{lemma}{Lemma}
\newtheorem{luuy}{Lưu ý}
\newtheorem{nhanxet}{Nhận xét}
\newtheorem{notation}{Notation}
\newtheorem{note}{Note}
\newtheorem{principle}{Principle}
\newtheorem{problem}{Problem}
\newtheorem{proposition}{Proposition}
\newtheorem{question}{Question}
\newtheorem{remark}{Remark}
\newtheorem{theorem}{Theorem}
\newtheorem{vidu}{Ví dụ}
\usepackage[left=1cm,right=1cm,top=5mm,bottom=5mm,footskip=4mm]{geometry}
\def\labelitemii{$\circ$}
\DeclareRobustCommand{\divby}{%
    \mathrel{\vbox{\baselineskip.65ex\lineskiplimit0pt\hbox{.}\hbox{.}\hbox{.}}}%
}
\setlist[itemize]{leftmargin=*}
\setlist[enumerate]{leftmargin=*}

\title{Disjoint Set Union (DSU) -- Hợp Tập Hợp Rời Rạc}
\author{Nguyễn Quản Bá Hồng\footnote{A scientist- {\it\&} creative artist wannabe, a mathematics {\it\&} computer science lecturer of Department of Artificial Intelligence {\it\&} Data Science (AIDS), School of Technology (SOT), UMT Trường Đại học Quản lý {\it\&} Công nghệ TP.HCM, Hồ Chí Minh City, Việt Nam.\\E-mail: {\sf nguyenquanbahong@gmail.com} {\it\&} {\sf hong.nguyenquanba@umt.edu.vn}. Website: \url{https://nqbh.github.io/}. GitHub: \url{https://github.com/NQBH}.}}
\date{\today}

\begin{document}
\maketitle
\begin{abstract}
    This text is a part of the series {\it Some Topics in Advanced STEM \& Beyond}:

    {\sc url}: \url{https://nqbh.github.io/advanced_STEM/}.

    Latest version:
    \begin{itemize}
        \item {\it Disjoint Set Union (DSU) -- Hợp Tập Hợp Rời Rạc}.

        PDF: {\sc url}: \url{.pdf}.

        \TeX: {\sc url}: \url{.tex}.
        \item {\it }.

        PDF: {\sc url}: \url{.pdf}.

        \TeX: {\sc url}: \url{.tex}.
    \end{itemize}
\end{abstract}
\tableofcontents

%------------------------------------------------------------------------------%

\section{Introduction to Disjoint Set Union -- Nhập Môn Hợp Tập Hợp Rời Rạc}
\textbf{\textsf{Resources -- Tài nguyên.}}
\begin{enumerate}
    \item \href{https://cp-algorithms.com/data_structures/disjoint_set_union.html}{Algorithms for Competitive Programming{\tt/}disjoint set union}.

    \item {\sc Benjamin Qi, Andrew Wang, Nathan Gong, Michael Cao}. \href{https://usaco.guide/gold/dsu?lang=cpp}{USACO Guide{\tt/}Disjoint Set Union}.

    {\sf Abstract.} The Disjoint Set Union (DSU) data structure, which allows you to add edges to a graph \& test whether 2 vertices of a graph are connected.

    -- Cấu trúc dữ liệu Disjoint Set Union (DSU), cho phép bạn thêm các cạnh vào đồ thị \& kiểm tra xem 2 đỉnh của đồ thị có được kết nối hay không.

    \item {\sc Ngô Quang Nhật}. \href{https://wiki.vnoi.info/algo/data-structures/disjoint-set-union}{VNOI Wiki{\tt/}Disjoint Set Union}.

    \item \href{https://en.wikipedia.org/wiki/Disjoint-set_data_structure}{Wikipedia{\tt/}disjoint-set data structure}.
\end{enumerate}
Disjoint Set Union (abbr., DSU) là 1 cấu trúc dữ liệu hữu dụng, thường xuất hiện trong các kỳ thi Lập trình Thi Đấu, \& có thể được dùng để quản lý 1 cách hiệu 1 tập hợp của các tập hợp.

\begin{baitoan}
    Cho 1 đồ thị $G = (V,E)$ có $|V| = n\in\mathbb{N}^\star$ đỉnh, ban đầu không có cạnh nào, $E = \emptyset$. Ta cần xử lý 2 loại truy vấn:
    \begin{enumerate}
        \item Thêm 1 cạnh giữa 2 đỉnh $x,y\in V$ trong đồ thị, i.e., $E = E\cup\{\{x,y\}\}$ nếu $G$ là đồ thị vô hướng \& $E = E\cup\{(x,y)\}$ nếu $G$ là đồ thị có hướng.
        \item In ra {\tt yes} nếu như 2 đỉnh $x,y$ nằm trong cùng 1 thành phần liên thông. In ra {\tt no} nếu ngược lại.
    \end{enumerate}
\end{baitoan}

%------------------------------------------------------------------------------%

\subsection{Data Structure Disjoint Set Union -- Cấu trúc dữ liệu Disjoint Set Union}
Cho tiện, ta đánh số $n$ đỉnh của đồ thị $G$ bởi $1,2,\ldots,n$ (trường hợp $n$ đỉnh được dán nhãn bởi $v_1,v_2,\ldots,v_n$ hoàn toàn tương tự vì ta có thể làm việc trên chỉ số $i$ của $v_i$), khi đó $V = [n]$. Giả sử $G$ có $c\coloneqq$ \verb|num_connected_component| $\in\mathbb{N}^\star$ (số thành phần liên thông) $C_1,C_2,\ldots,C_c$ với $\{C_i\}_{i=1}^c$ là 1 phân hoạch của $V = [n]$, i.e.:
\begin{equation*}
    \bigcup_{i=1}^c C_i = [n],\ C_i\cap C_j = \emptyset,\ \forall i,j\in[c],\ i\ne j.
\end{equation*}
Nếu ta coi mỗi đỉnh của đồ thị $G = (V,E)$ là 1 phần tử \& mỗi thành phần liên thông (connected component) trong đồ thị là 1 tập hợp, truy vấn 1 sẽ trở thành gộp 2 tập hợp lần lượt chứa phần tử $x,y$ thành 1 tập hợp mới \& truy vấn 2 trở thành việc hỏi 2 phần tử $x,y$ có nằm trong cùng 1 tập hợp không.

Để tiện tính toán \& lý luận về mặt toán học cho riêng cấu trúc dữ liệu DSU, sau đây là 1 định nghĩa lai Toán--Tin mang tính cá nhân của tác giả [NQBH], hoàn toàn không chính thống trong Lý thuyết Đồ thị:

\begin{dinhnghia}[Chỉ số thành phần liên thông]
    \label{def: index component}
    Cho đồ thị vô hướng $G = (V,E)$ với $V = [n]$. Gọi $C(i)\subset[n]$ là thành phần liên thông của $G = (V,E)$ chứa đỉnh $i\in[n]$ \& gọi chỉ số của thành phần liên thông chứa đỉnh $i$ là ${\rm cid}(i)$, i.e., $i\in C_{{\rm cid}(i)}\equiv C(i)$, với hàm ${\rm cid}:[n]\to[c]$ được gọi là {\rm hàm chỉ số liên thông}.
\end{dinhnghia}
Với định nghĩa \ref{def: index component}, ta có ngay
\begin{equation*}
    \left\{\begin{split}
        &i\in C(i) = C_{{\rm cid}(i)}\subset[n],\\
        &i,j\mbox{ are connected},\ \forall j\in C(i).
    \end{split}\right.\ \forall i\in[n].
\end{equation*}
Ở đây, ta coi mỗi đỉnh của đồ thị tự liên thông với chính nó theo nghĩa đỉnh đó đến được (reachability) chính đỉnh đó thông qua 1 đường đi có độ dài 0, được gọi là 1 {\it đường đi tầm thường}, chứ không phải theo nghĩa khuyên (loop).

\begin{lemma}[A characterization of connectedness -- 1 đặc trưng hóa của tính liên thông]
    \label{lem: characterization connectedness}
    Cho đồ thị vô hướng $G = (V,E)$.
    (i) 2 đỉnh trên 1 đồ thị $G$ không liên thông với nhau, i.e., không có đường đi nào trên $G$ nối chúng khi \& chỉ khi chúng thuộc 2 thành phần liên thông khác nhau, i.e.,
    \begin{equation*}
        i,j\mbox{ are not connected}\Leftrightarrow C(i)\ne C(j)\Leftrightarrow C(i)\cap C(j) = \emptyset\Leftrightarrow{\rm cid}(i)\ne{\rm cid}(j),\ \forall i,j\in[n].
    \end{equation*}
    \item(ii) 2 đỉnh trên đồ thị $G$ liên thông với nhau, i.e., có đường đi trên trên $G$ nối chúng khi \& chỉ khi chúng cùng thuộc 1 thành phần liên thông, i.e.:
    \begin{equation*}
        i,j\mbox{ are connected}\Leftrightarrow C(i) = C(j)\Leftrightarrow C(i)\cap C(j)\ne\emptyset\Leftrightarrow{\rm cid}(i) = {\rm cid}(j),\ \forall i,j\in[n].
    \end{equation*}
\end{lemma}
Với truy vấn 1, giả sử 2 đỉnh $i,j\in[n]$ chưa có cạnh nối chúng trực tiếp, i.e., $\{i,j\}\notin E$. Có 2 trường hợp xảy ra:
\begin{itemize}
    \item Trường hợp 1: Giả sử $i,j$ cùng thuộc 1 thành phần liên thông, theo Lem. \ref{lem: characterization connectedness}, có $C(i) = C(j)$, ${\rm cid}(i) = {\rm cid}(j)$ nên ta chỉ cần thêm cạnh $\{i,j\}$ vào tập cạnh $E$: $E\leftarrow E\cup\{\{i,j\}\}$ hay \verb|edge_list.append({i, j})| mà không cần cập nhật $c$ tập liên thông $\{C_i\}_{i=1}^c$ hay hàm chỉ số liên thông ${\rm cid}(\cdot)$.
    \item Trường hợp 2: Giả sử $i,j$ thuộc 2 thành phần liên thông khác nhau, theo \ref{lem: characterization connectedness}, có $C(i)\ne C(j)$, ${\rm cid}(i)\ne {\rm cid}(j)$
\end{itemize}



%------------------------------------------------------------------------------%

\section{Miscellaneous}

%------------------------------------------------------------------------------%

\printbibliography[heading=bibintoc]

\end{document}