\documentclass[oneside]{article}
\usepackage[utf8]{inputenc}
\title{Optimal Shape Design of Air Ducts in Combustion Engines: Design a General Shape Optimization Framework}
\author{Michael Hinterm\"uller\thanks{Weierstrass Institute for Applied Analysis and Stochastics (WIAS), Mohrenstrasse 39, 10117 Berlin, Germany.} \and Hong Quan Ba Nguyen${}^*$}
\date{\today}

\begin{document}
\maketitle
\thispagestyle{empty}
\begin{abstract}
    In order to optimize the shape design of air ducts in combustion engines, we consider a shape optimization problem subject to an instationary incompressible viscous Navier-Stokes equations in 3D with appropriate physical boundary conditions in the duct geometry. An inflow profile is given at the inlet, a no-slip boundary condition is imposed on the wall, and a do-nothing boundary condition on the outlet. To find optimal shapes, we choose a mixed cost functional to achieve the flow uniformity at the outlet and minimize the dissipated power of our fluid dynamics device in a well balanced way.
    
    In this talk, we try to design a general shape optimization framework via the continuous adjoint approach for the problem proposed, especially replacing Navier-Stokes equations by some widely used turbulence models, for high Reynolds number flows, to be able to capture turbulence phenomena. Their adjoint PDEs and the first-order shape derivatives of the chosen mixed cost functional and a general one have been computed and will be presented in this talk as the first step to design a gradient descent algorithm and develop CFD shape-optimization software packages.
\end{abstract}
\textit{Keywords}: Shape optimization, Navier-Stokes equations, turbulence models, adjoint-based method.
\end{document}