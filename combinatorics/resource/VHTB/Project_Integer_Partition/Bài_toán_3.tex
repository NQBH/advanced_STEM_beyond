\documentclass[12pt]{article}
\usepackage[utf8]{inputenc}
\usepackage[vietnamese]{babel}
\usepackage{amsmath, amssymb}
\usepackage{listings}
\usepackage{enumitem}

\title{Bài Toán 3: Số phân hoạch tự liên hợp}
\author{Đồ Án Phân Hoạch Số Nguyên}
\date{}

\begin{document}

\maketitle

\section*{Phát biểu bài toán}
Cho $n, k \in \mathbb{N}$. Hãy thực hiện các yêu cầu sau:
\begin{enumerate}[label=(\alph*)]
    \item Đếm số phân hoạch tự liên hợp của $n$ có đúng $k$ phần tử, ký hiệu $p_k^{\text{self}}(n)$
    \item Với $k$ bất kỳ, in ra tất cả các phân hoạch tự liên hợp của $n$
    \item Thiết lập công thức đệ quy truy hồi tính $p_k^{\text{self}}(n)$
\end{enumerate}

\section*{Định nghĩa: Phân hoạch tự liên hợp}
Phân hoạch $\lambda = (\lambda_1, \lambda_2, \dots, \lambda_r)$ là tự liên hợp nếu biểu đồ Ferrers của nó bằng chính chuyển vị của nó. Ví dụ: $(3,1,1)$ và $(5,3,1)$ là phân hoạch tự liên hợp.

\section*{Công thức đếm (derivation)}
Số phân hoạch tự liên hợp của $n$ bằng số tập các số nguyên dương lẻ phân biệt sao cho tổng của chúng là $n$.

Công thức tổng quát:
\[
p^{\text{self}}(n) = \text{số phân hoạch của } n \text{ thành tổng của các số lẻ phân biệt}
\]

\section*{Công thức đệ quy}
Gọi $dp[i][j]$ là số phân hoạch của $i$ dùng các số lẻ phân biệt $\leq j$. Ta có:

\[
dp[i][j] = dp[i][j-2] + dp[i-j][j] \quad \text{nếu } j \leq i
\]
Ngược lại:
\[
dp[i][j] = dp[i][j-2]
\]
Với khởi tạo: $dp[0][j] = 1$, $dp[i][0] = 0$ với $i > 0$

\section*{Chú thích các biến}
\begin{itemize}
    \item \texttt{n} – tổng cần phân hoạch
    \item \texttt{dp[i][j]} – số phân hoạch của $i$ dùng số lẻ phân biệt $\leq j$
    \item \texttt{j} tăng theo bước 2 (chỉ số lẻ)
\end{itemize}

\end{document}
