\documentclass{article}
\usepackage[backend=biber,natbib=true,style=alphabetic,maxbibnames=50]{biblatex}
\addbibresource{/home/nqbh/reference/bib.bib}
\usepackage[utf8]{vietnam}
\usepackage{tocloft}
\renewcommand{\cftsecleader}{\cftdotfill{\cftdotsep}}
\usepackage[colorlinks=true,linkcolor=blue,urlcolor=red,citecolor=magenta]{hyperref}
\usepackage{amsmath,amssymb,amsthm,enumitem,float,graphicx,mathtools,tikz}
\usetikzlibrary{angles,calc,intersections,matrix,patterns,quotes,shadings}
\allowdisplaybreaks
\newtheorem{assumption}{Assumption}
\newtheorem{baitoan}{}
\newtheorem{cauhoi}{Câu hỏi}
\newtheorem{conjecture}{Conjecture}
\newtheorem{corollary}{Corollary}
\newtheorem{dangtoan}{Dạng toán}
\newtheorem{definition}{Definition}
\newtheorem{dinhly}{Định lý}
\newtheorem{dinhnghia}{Định nghĩa}
\newtheorem{example}{Example}
\newtheorem{ghichu}{Ghi chú}
\newtheorem{hequa}{Hệ quả}
\newtheorem{hypothesis}{Hypothesis}
\newtheorem{lemma}{Lemma}
\newtheorem{luuy}{Lưu ý}
\newtheorem{nhanxet}{Nhận xét}
\newtheorem{notation}{Notation}
\newtheorem{note}{Note}
\newtheorem{principle}{Principle}
\newtheorem{problem}{Problem}
\newtheorem{proposition}{Proposition}
\newtheorem{question}{Question}
\newtheorem{remark}{Remark}
\newtheorem{theorem}{Theorem}
\newtheorem{vidu}{Ví dụ}
\usepackage[left=1cm,right=1cm,top=5mm,bottom=5mm,footskip=4mm]{geometry}
\DeclareRobustCommand{\divby}{%
	\mathrel{\vbox{\baselineskip.65ex\lineskiplimit0pt\hbox{.}\hbox{.}\hbox{.}}}%
}
\def\labelitemii{$\circ$}
\setlist[itemize]{leftmargin=*}
\setlist[enumerate]{leftmargin=*}

\title{Linear Algebra -- Đại Số Tuyến Tính}
\author{Nguyễn Quản Bá Hồng\footnote{A Scientist {\it\&} Creative Artist Wannabe. E-mail: {\tt nguyenquanbahong@gmail.com}. Bến Tre City, Việt Nam.}}
\date{\today}

\begin{document}
\maketitle
\begin{abstract}
	This text is a part of the series {\it Some Topics in Advanced STEM \& Beyond}:
	
	{\sc url}: \url{https://nqbh.github.io/advanced_STEM/}.
	
	Latest version:
	\begin{itemize}
		\item {\it Linear Algebra -- Đại Số Tuyến Tính}.
		
		PDF: {\sc url}: \url{https://github.com/NQBH/advanced_STEM_beyond/blob/main/linear_algebra/NQBH_linear_algebra.pdf}.
		
		\TeX: {\sc url}: \url{https://github.com/NQBH/advanced_STEM_beyond/blob/main/linear_algebra/NQBH_linear_algebra.tex}.
	\end{itemize}
\end{abstract}
\tableofcontents

%------------------------------------------------------------------------------%

\section{Basic}
Tôi được giải Nhì Đại số Olympic Toán Sinh viên 2014 (VMC2014) khi còn học năm nhất Đại học \& được giải Nhất Đại số Olympic Toán Sinh viên 2015 (VMC2015) khi học năm 2 Đại học. Nhưng điều đó không có nghĩa là tôi giỏi Đại số. Bằng chứng là 10 năm sau khi nhận các giải đó, tôi đang tự học lại Đại số tuyến tính với hy vọng có 1 hay nhiều cách nhìn mới mẻ hơn \& mang tính ứng dụng hơn cho các đề tài cá nhân của tôi.

\noindent\textbf{\textsf{Resources -- Tài nguyên.}}
\begin{itemize}
	\item \cite{Hung_linear_algebra}. {\sc Nguyễn Hữu Việt Hưng}. {\it Đại Số Tuyến Tính}.
	\item \cite{Tiep_ML_co_ban}. {\sc Vũ Hữu Tiệp}. {\it Machine Learning Cơ Bản}.
	
	Mã nguồn cuốn ebook ``Machine Learning Cơ Bản'': \url{https://github.com/tiepvupsu/ebookMLCB}.
	
	Phép nhân từng phần{\tt/}tích Hadamard (Hadamard product) thường xuyên được sử dụng trong ML. Tích Hadamard của 2 ma trận cùng kích thước $A,B\in\mathbb{R}^{m\times n}$, được ký hiệu là $A\odot B = (a_{ij}b_{ij})_{i,j=1}^{m,n}\in\mathbb{R}^{m\times n}$.
	
	Việc chuyển đổi hệ cơ sở sử dụng ma trận trực giao có thể được coi như 1 phép xoay trục tọa độ. Nhìn theo 1 cách khác, đây cũng chính là 1 phép xoay vector dữ liệu theo chiều ngược lại, nếu ta coi các trục tọa độ là cố định.
	
	Việc phân tích 1 đại lượng toán học ra thành các đại lượng nhỏ hơn mang lại nhiều hiệu quả. Phân tích 1 số thành tích các thừa số nguyên tố giúp kiểm tra 1 số có bao nhiêu ước số. Phân tích đa thức thành nhân tử giúp tìm nghiệm của đa thức. Việc phân tích 1 ma trận thành tích của các ma trận đặc biệt cũng mang lại nhiều lợi ích trong việc giải hệ phương trình tuyến tính, tính lũy thừa của ma trận, xấp xỉ ma trận, $\ldots$
	
	{\bf Phép phân tích trị riêng.} Cách biểu diễn 1 ma trận vuông $A$ với ${\bf x}_i\ne{\bf 0}$ là các vector riêng của 1 ma trận vuông $A$ ứng với các giá trị riêng lặp hoặc phức $\lambda_i$: $A{\bf x}_i = \lambda_i{\bf x}_i$, $\forall i = 1,\ldots,n$: $A = X\Lambda X^{-1}$ với $\Lambda = {\rm diag}(\lambda_1,\ldots,\lambda_n)$, $X = [{\bf x}_1,\ldots,{\bf x}_n]$.
	
	{\bf Norm -- Chuẩn.} Khoảng cách Euclid chính là độ dài đoạn thẳng nối 2 điểm trong mặt phẳng. Đôi khi, để đi từ 1 điểm này tới 1 điểm kia, không thể đi bằng đường thẳng vì còn phụ thuộc vào hình dạng đường đi nối giữa 2 điểm. Cf. đường trắc địa trong Hình học Vi phân -- geodesics in Differential Geometry. Việc đo khoảng cách giữa 2 điểm dữ liệu nhiều chiều rất cần thiết trong ML -- chính là lý do khái niệm {\it chuẩn} (norm) ra đời.
	
	{\bf Trace -- Vết.} {\it Vết} (trace) của 1 ma trận vuông $A$ được ký hiệu là $\operatorname{trace}A$ là tổng tất cả các phần tử trên đường chéo chính của nó. Hàm vết xác định trên tập các ma trận vuông được sử dụng nhiều trong tối ưu vì nó có các tính chất đẹp.
	\item \cite{Trefethen_Bau1997,Trefethen_Bau2022}. {\sc Lloyd N. Trefethen, David Bau III}. {\it Numerical Linear Algebra}.
\end{itemize}


%------------------------------------------------------------------------------%

\section{Miscellaneous}

%------------------------------------------------------------------------------%

\printbibliography[heading=bibintoc]
	
\end{document}