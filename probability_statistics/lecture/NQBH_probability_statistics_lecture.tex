\documentclass{article}
\usepackage[backend=biber,natbib=true,style=alphabetic,maxbibnames=50]{biblatex}
\addbibresource{/home/nqbh/reference/bib.bib}
\usepackage[utf8]{vietnam}
\usepackage{tocloft}
\renewcommand{\cftsecleader}{\cftdotfill{\cftdotsep}}
\usepackage[colorlinks=true,linkcolor=blue,urlcolor=red,citecolor=magenta]{hyperref}
\usepackage{amsmath,amssymb,amsthm,enumitem,float,graphicx,mathtools,tikz}
\usetikzlibrary{angles,calc,intersections,matrix,patterns,quotes,shadings}
\allowdisplaybreaks
\newtheorem{assumption}{Assumption}
\newtheorem{baitoan}{}
\newtheorem{cauhoi}{Câu hỏi}
\newtheorem{conjecture}{Conjecture}
\newtheorem{corollary}{Corollary}
\newtheorem{dangtoan}{Dạng toán}
\newtheorem{definition}{Definition}
\newtheorem{dinhly}{Định lý}
\newtheorem{dinhnghia}{Định nghĩa}
\newtheorem{example}{Example}
\newtheorem{ghichu}{Ghi chú}
\newtheorem{hequa}{Hệ quả}
\newtheorem{hypothesis}{Hypothesis}
\newtheorem{lemma}{Lemma}
\newtheorem{luuy}{Lưu ý}
\newtheorem{nhanxet}{Nhận xét}
\newtheorem{notation}{Notation}
\newtheorem{note}{Note}
\newtheorem{principle}{Principle}
\newtheorem{problem}{Problem}
\newtheorem{proposition}{Proposition}
\newtheorem{question}{Question}
\newtheorem{remark}{Remark}
\newtheorem{theorem}{Theorem}
\newtheorem{vidu}{Ví dụ}
\usepackage[left=1cm,right=1cm,top=5mm,bottom=5mm,footskip=4mm]{geometry}
\def\labelitemii{$\circ$}
\DeclareRobustCommand{\divby}{%
	\mathrel{\vbox{\baselineskip.65ex\lineskiplimit0pt\hbox{.}\hbox{.}\hbox{.}}}%
}
\setlist[itemize]{leftmargin=*}
\setlist[enumerate]{leftmargin=*}

\title{Lecture: Probability Theory {\it\&} Statistics Theory\\Bài Giảng: Lý Thuyết Xác Suất {\it\&} Lý Thuyết Thống Kê}
\author{Nguyễn Quản Bá Hồng\footnote{A scientist- {\it\&} creative artist wannabe, a mathematics {\it\&} computer science lecturer of Department of Artificial Intelligence {\it\&} Data Science (AIDS), School of Technology (SOT), UMT Trường Đại Học Quản Lý {\it\&} Công Nghệ Thành Phố Hồ Chí Minh, \url{https://www.umt.edu.vn}, Việt Nam.\\E-mail: {\sf nguyenquanbahong@gmail.com} {\it\&} {\sf hong.nguyenquanba@umt.edu.vn}. Website: \url{https://nqbh.github.io/}. GitHub: \url{https://github.com/NQBH}.}}
\date{\today}

\begin{document}
\maketitle
\begin{abstract}
	This text is a part of the series {\it Some Topics in Advanced STEM \& Beyond}:
	
	{\sc url}: \url{https://nqbh.github.io/advanced_STEM/}.
	
	Latest version:
	\begin{itemize}
		\item {\it Lecture: Probability Theory \& Statistics Theory -- Bài Giảng: Lý Thuyết Xác Suất \& Lý Thuyết Thống Kê}.
		
		PDF: {\sc url}: \url{https://github.com/NQBH/advanced_STEM_beyond/blob/main/probability_statistics/lecture/NQBH_probability_statistics_lecture.pdf}.
		
		\TeX: {\sc url}: \url{https://github.com/NQBH/advanced_STEM_beyond/blob/main/probability_statistics/lecture/NQBH_probability_statistics_lecture.tex}.
		\item {\it }.
		
		PDF: {\sc url}: \url{.pdf}.
		
		\TeX: {\sc url}: \url{.tex}.
	\end{itemize}
\end{abstract}
\tableofcontents

%------------------------------------------------------------------------------%

\section{Basic Probability Theory -- Lý Thuyết Xác Suất Cơ Bản}
\textbf{\textsf{Resources -- Tài nguyên.}}
\begin{enumerate}
	\item \cite{Deisenroth_Faisal_Ong2024}. {\sc Marc Peter Deisenroth, A. Aldo Faisal, Cheng Soon Ong}. {\it Mathematics for Machine Learning}. Chap. 6: Probability \& Distributions.
	\item \cite{Jaynes2003}. {\sc E. T. Jaynes}. {\it Probability Theory: The Logic of Science}.
	\item \cite{Kallenberg2021}. {\sc Olav Kallenberg}. {\it Foundations of Modern Probability}. Series Probability Theory \& Stochastic Modelling.
	\item \cite{Klenke2020}. {\sc Achim Klenke}. {\it Probability Theory -- A Comprehensive Course}.
	\item \cite{Trong_Thanh_Minh_xac_suat}. {\sc Đặng Đức Trọng, Đinh Ngọc Thanh, Nguyễn Đăng Minh}. {\it Lý Thuyết Xác Suất}.
	\item \cite{Varadhan2001}. {\sc S. R. S. Varadhan}. {\it Probability Theory}. Series Courant Lecture Notes in Mathematics.
\end{enumerate}

%------------------------------------------------------------------------------%

\subsection{A big picture in probability theory -- 1 bức tranh lớn trong lý thuyết xác suất}
{\it Probability theory} or {\it probability calculus} is the branch of mathematics concerned with \href{https://en.wikipedia.org/wiki/Probability}{probability}. Although there are several different \href{https://en.wikipedia.org/wiki/Probability_interpretations}{probability interpretations}, probability theory treats the concept in a rigorous mathematical manner by expressing it through a set of \href{https://en.wikipedia.org/wiki/Axioms_of_probability}{axioms of probability} (see also, e.g., \cite{Popper_logic_science,Popper_logic_khoa_hoc}). Typically these axioms formalize probability in terms of a \href{https://en.wikipedia.org/wiki/Probability_space}{probability space}, which assigns a \href{https://en.wikipedia.org/wiki/Measure_(mathematics)}{measure} taking values between 0 \& 1, termed the \href{https://en.wikipedia.org/wiki/Probability_measure}{probability measure}, to a set of outcomes called the \href{https://en.wikipedia.org/wiki/Sample_space}{sample space}. Any specified subset of the sample space is called an \href{https://en.wikipedia.org/wiki/Event_(probability_theory)}{event}.

-- {\it Lý thuyết xác suất} hoặc {\it phép tính xác suất} là nhánh toán học liên quan đến xác suất. Mặc dù có 1 số cách diễn giải xác suất khác nhau, lý thuyết xác suất xử lý khái niệm này theo cách toán học nghiêm ngặt bằng cách thể hiện nó thông qua 1 tập hợp các tiên đề. Thông thường, các tiên đề này chính thức hóa xác suất theo không gian xác suất, gán 1 phép đo có giá trị từ 0 đến 1, được gọi là phép đo xác suất, cho 1 tập hợp các kết quả được gọi là không gian mẫu. Bất kỳ tập hợp con nào được chỉ định của không gian mẫu được gọi là 1 sự kiện.

Central subsets in probability theory include discrete \& continuous \href{https://en.wikipedia.org/wiki/Random_variable}{random variables}, \href{https://en.wikipedia.org/wiki/Probability_distributions}{probability distributions}, \& \href{https://en.wikipedia.org/wiki/Stochastic_process}{stochastic processes} (which provide mathematical abstractions of \href{https://en.wikipedia.org/wiki/Determinism}{non-deterministic} or uncertain processes or measured quantities that may either be single occurrences or evolve over time in a random fashion). Although it is not possible to perfectly predict random events, much can be said about their behavior. 2 major results in probability theory describing such behavior are the \href{https://en.wikipedia.org/wiki/Law_of_large_numbers}{law of large numbers} \& the \href{https://en.wikipedia.org/wiki/Central_limit_theorem}{central limit theorem}.

-- Các chủ đề trung tâm trong lý thuyết xác suất bao gồm các biến ngẫu nhiên rời rạc và liên tục, phân phối xác suất và các quá trình ngẫu nhiên (cung cấp các khái niệm trừu tượng về mặt toán học của các quá trình không xác định hoặc không chắc chắn hoặc các đại lượng đo lường có thể là các sự kiện đơn lẻ hoặc phát triển theo thời gian 1 cách ngẫu nhiên). Mặc dù không thể dự đoán hoàn hảo các sự kiện ngẫu nhiên, nhưng có thể nói nhiều về hành vi của chúng. 2 kết quả chính trong lý thuyết xác suất mô tả hành vi như vậy là quy luật số lớn và định lý giới hạn trung tâm.

As a mathematical foundation for \href{https://en.wikipedia.org/wiki/Statistics}{statistics}, probability theory is essential to many human activities that involve quantitative analysis of data. Methods of probability theory also apply to descriptions of complex systems given only partial knowledge of their state, as in \href{https://en.wikipedia.org/wiki/Statistical_mechanics}{statistical mechanics} or \href{https://en.wikipedia.org/wiki/Sequential_estimation}{sequential estimation}. A great discovery of 20th-century physics was the probabilistic nature of physical phenomena at atomic scales, described in \href{https://en.wikipedia.org/wiki/Quantum_mechanics}{quantum mechanics}.

-- Là nền tảng toán học cho thống kê, lý thuyết xác suất là điều cần thiết cho nhiều hoạt động của con người liên quan đến phân tích định lượng dữ liệu. Các phương pháp của lý thuyết xác suất cũng áp dụng cho các mô tả về các hệ thống phức tạp chỉ được cung cấp 1 phần kiến thức về trạng thái của chúng, như trong cơ học thống kê hoặc ước tính tuần tự. Một khám phá lớn của vật lý thế kỷ XX là bản chất xác suất của các hiện tượng vật lý ở quy mô nguyên tử, được mô tả trong cơ học lượng tử.

For history of probability theory, see, e.g., \href{https://en.wikipedia.org/wiki/History_of_probability}{Wikipedia{\tt/}probability theory}.

\subsection{Treatment of probability theory -- Xử lý lý thuyết xác suất}
Most introductions to probability theory treat discrete probability distributions \& continuous probability distributions separately. The measure theory-based treatment of probability covers the discrete, continuous, a mix of the 2, \& more.

-- Hầu hết các phần giới thiệu về lý thuyết xác suất đều xử lý riêng biệt các phân phối xác suất rời rạc \& phân phối xác suất liên tục. Phần xử lý xác suất dựa trên lý thuyết đo lường bao gồm phân phối xác suất rời rạc, liên tục, hỗn hợp của cả 2, \& nhiều hơn nữa.

%------------------------------------------------------------------------------%

\subsection{Measure space \& probability space -- Không gian đo \& không gian xác suất}
2 giả định về không gian mẫu $\Omega$:
\begin{enumerate}
	\item Không gian mẫu $\Omega$ là đồng xác suất.
	\item Không gian mẫu $\Omega$ là không đồng xác suất.
\end{enumerate}

\subsubsection{Probability space -- Không gian xác suất}
See, e.g., \href{https://en.wikipedia.org/wiki/Probability_space}{probability space}.

%------------------------------------------------------------------------------%

\subsection{Random variable -- Biến ngẫu nhiên}
Đặt $(X = a)$ chỉ biến cố ``$X$ lấy giá trị $a$''.

%------------------------------------------------------------------------------%

\subsection{Discrete random variable -- Biến ngẫu nhiên rời rạc}
Các biến ngẫu nhiên rời rạc trong lý thuyết xác suất thường được biểu diễn bằng các bảng phân phối xác suất:
\begin{table}[H]
	\centering
	\begin{tabular}{|c|c|c|c|c|}
		\hline
		$X$ & $x_1$ & $x_2$ & $\ldots$ & $x_n$ \\
		\hline
		$P(X = x_i)$ & $p_1$ & $p_2$ & $\ldots$ & $p_n$ \\
		\hline
	\end{tabular}
\end{table}
Xét không gian xác suất $(\Omega,\mathfrak{M},P)$ \& ánh xạ $X:\Omega\to\mathbb{R}$.

\begin{dinhnghia}
	$X$ được gọi là 1 {\rm biến ngẫu nhiên} trên $\Omega$ khi
	\begin{equation*}
		(X\le x)\coloneqq\{\omega\in\Omega;X(\omega)\le x\} = X^{-1}((-\infty,x])\in\mathfrak{M},\ \forall x\in\mathbb{R}.
	\end{equation*}
	Khi đó, hàm $F_X:\mathbb{R}\to\mathbb{R}$ xác định bởi
	\begin{equation}
		\label{cdf}
		\tag{cdf}
		F_X(x)\coloneqq P(X\le x),\ \forall x\in\mathbb{R},
	\end{equation}
	được gọi là {\rm hàm phân phối tích lũy (cumulative distribution function, abbr., cdf)} của biến ngẫu nhiên $X$.
\end{dinhnghia}

%------------------------------------------------------------------------------%

\subsection{Continuous random variable -- Biến ngẫu nhiên liên tục}

%------------------------------------------------------------------------------%

\subsection{Limit theorems -- Các định lý giới hạn}

%------------------------------------------------------------------------------%

\section{Basic Statistics Theory -- Lý Thuyết Thống Kê Cơ Bản}
\textbf{\textsf{Resources -- Tài nguyên.}}
\begin{enumerate}
	\item \cite{Trong_Thanh_thong_ke}. {\sc Đặng Đức Trọng, Đinh Ngọc Thanh}. {\it Lý Thuyết Thống Kê}.
\end{enumerate}

\subsection{Data description -- Mô tả dữ liệu}

%------------------------------------------------------------------------------%

\subsection{Sample theory -- Lý thuyết mẫu}

%------------------------------------------------------------------------------%

\subsection{Estimation theory -- Lý thuyết ước lượng}

%------------------------------------------------------------------------------%

\subsection{Testing theory -- Lý thuyết kiểm định}

%------------------------------------------------------------------------------%

\subsection{Linear regression model -- Mô hình hồi quy tuyến tính}

%------------------------------------------------------------------------------%

\section{Miscellaneous}

\subsection{See also}
For the derivation of the definitions of probability, see, e.g.:
\begin{itemize}
	\item \cite{Popper_logic_science}. {\sc Karl Popper}. {\it The Logic of Scientific Discovery}.
	
	\item \cite{Popper_logic_khoa_hoc}. {\sc Karl Popper}. {\it The Logic of Scientific Discovery -- Logic Của Sự Khám Phá Khoa Học}.
\end{itemize}

%------------------------------------------------------------------------------%

\printbibliography[heading=bibintoc]
	
\end{document}