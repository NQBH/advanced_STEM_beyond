\documentclass{article}
\usepackage[backend=biber,natbib=true,style=alphabetic,maxbibnames=50]{biblatex}
\addbibresource{/home/nqbh/reference/bib.bib}
\usepackage[utf8]{vietnam}
\usepackage{tocloft}
\renewcommand{\cftsecleader}{\cftdotfill{\cftdotsep}}
\usepackage[colorlinks=true,linkcolor=blue,urlcolor=red,citecolor=magenta]{hyperref}
\usepackage{amsmath,amssymb,amsthm,enumitem,float,graphicx,mathtools,tikz}
\usetikzlibrary{angles,calc,intersections,matrix,patterns,quotes,shadings}
\allowdisplaybreaks
\newtheorem{assumption}{Assumption}
\newtheorem{baitoan}{}
\newtheorem{cauhoi}{Câu hỏi}
\newtheorem{conjecture}{Conjecture}
\newtheorem{corollary}{Corollary}
\newtheorem{dangtoan}{Dạng toán}
\newtheorem{definition}{Definition}
\newtheorem{dinhly}{Định lý}
\newtheorem{dinhnghia}{Định nghĩa}
\newtheorem{example}{Example}
\newtheorem{ghichu}{Ghi chú}
\newtheorem{hequa}{Hệ quả}
\newtheorem{hypothesis}{Hypothesis}
\newtheorem{lemma}{Lemma}
\newtheorem{luuy}{Lưu ý}
\newtheorem{nhanxet}{Nhận xét}
\newtheorem{notation}{Notation}
\newtheorem{note}{Note}
\newtheorem{principle}{Principle}
\newtheorem{problem}{Problem}
\newtheorem{proposition}{Proposition}
\newtheorem{question}{Question}
\newtheorem{remark}{Remark}
\newtheorem{theorem}{Theorem}
\newtheorem{vidu}{Ví dụ}
\usepackage[left=1cm,right=1cm,top=5mm,bottom=5mm,footskip=4mm]{geometry}
\def\labelitemii{$\circ$}
\DeclareRobustCommand{\divby}{%
	\mathrel{\vbox{\baselineskip.65ex\lineskiplimit0pt\hbox{.}\hbox{.}\hbox{.}}}%
}
\setlist[itemize]{leftmargin=*}
\setlist[enumerate]{leftmargin=*}

\title{Philosophy $\Phi$ -- Triết Học $\Phi$}
\author{Nguyễn Quản Bá Hồng\footnote{A Scientist {\it\&} Creative Artist Wannabe. E-mail: {\tt nguyenquanbahong@gmail.com}. Bến Tre City, Việt Nam.}}
\date{\today}

\begin{document}
\maketitle
\begin{abstract}
	This text is a part of the series {\it Some Topics in Advanced STEM \& Beyond}:
	
	{\sc url}: \url{https://nqbh.github.io/advanced_STEM/}.
	
	Latest version:
	\begin{itemize}
		\item {\it Philosophy $\Phi$ -- Triết Học $\Phi$}.
		
		PDF: {\sc url}: \url{https://github.com/NQBH/advanced_STEM_beyond/blob/main/philosophy/NQBH_philosophy.pdf}.
		
		\TeX: {\sc url}: \url{https://github.com/NQBH/advanced_STEM_beyond/blob/main/philosophy/NQBH_philosophy.tex}.
	\end{itemize}
\end{abstract}
\tableofcontents

%------------------------------------------------------------------------------%

\section{Wikipedia}

\subsection{Wikipedia{\tt/}logical intuition}
``{\it Logical Intuition}, or {\it mathematical intuition} or {\it rational intuition}, is a series of instinctive foresight, know-how, \& savviness\footnote{{\it savvy} [a] having practical knowledge \& understanding of something; having common sense.} often associated with the ability to perceive \href{https://en.wikipedia.org/wiki/Logic}{logical} or mathematical truth -- \& the ability to solve mathematical challenges efficiency. Humans apply logical intuition in proving mathematical \href{https://en.wikipedia.org/wiki/Theorem}{theorems}, validating logical arguments, developing algorithms \& \href{https://en.wikipedia.org/wiki/Heuristic}{heuristics}, \& in related contexts where mathematical challenges are involved. The ability to recognize logical or mathematical truth \& identify viable methods may vary from person to person, \& may even be a result of knowledge \& experience, which are subject to cultivation. The ability may not be realizable in a computer program by means other than \href{https://en.wikipedia.org/wiki/Genetic_programming}{genetic programming} or \href{https://en.wikipedia.org/wiki/Evolutionary_programming}{evolutionary programming}.

\subsubsection{History}
\href{https://en.wikipedia.org/wiki/Plato}{Plato} \& \href{https://en.wikipedia.org/wiki/Aristotle}{Aristotle} considered intuition a means for perceiving ideas, significant enough that for Aristotle, intuition comprised the only means of knowing principles that are \href{https://en.wikipedia.org/wiki/A_priori_and_a_posteriori}{not subject to argument}.

\href{https://en.wikipedia.org/wiki/Henri_Poincar%C3%A9}{\sc Henri Poincar\'e} distinguished logical intuition from \href{https://en.wikipedia.org/wiki/Intuition}{other forms of intuition}. In his book \href{https://en.wikipedia.org/wiki/The_Value_of_Science}{The Value of Science}, he points out that:
\begin{quote}
	{\it``$\ldots$ There are many kinds of intuition. I have said how much the intuition of pure number, when comes rigorous mathematical induction, differs from sensible intuition to which the imagination, properly so called, is the {\it principal contributor}.''}
	
	-- $\ldots$ Có nhiều loại trực giác. Tôi đã nói rằng trực quan về số thuần túy, khi có quy nạp toán học nghiêm ngặt, khác biệt đến mức nào với trực giác nhạy cảm mà trí tưởng tượng, được gọi 1 cách chính xác, là {\it đóng góp chính}.	
\end{quote}
The passage goes on to assign 2 roles to logical intuition: to permit one to choose which \href{https://en.wikipedia.org/wiki/Axiom}{route} (axiom) to follow in search of scientific \href{https://en.wikipedia.org/wiki/Truth}{truth}, \& to allow one to comprehend logical developments.

\href{https://en.wikipedia.org/wiki/Bertrand_Russell}{\sc Bertrand Russell}, though critical of intuitive \href{https://en.wikipedia.org/wiki/Mysticism#Intuitive_insight_and_enlightenment}{mysticism}, pointed out that the degree to which a truth is \href{https://en.wikipedia.org/wiki/Self-evidence}{self-evident} according to logical intuition can vary, from 1 situation to another, \& stated that some self-evident truths are practically \href{https://en.wikipedia.org/wiki/Infallibility#In_philosophy}{infallible}\footnote{{\it infallible} [a] 1. never wrong; never making mistakes, $\ne$ {\it fallible}; 2. that never fails; always doing what it is supposed to do.}:
\begin{quote}
	{\it``When a certain number of logical principles have been admitted, the rest can be deduced from them; but the propositions deduced are often just as self-evident as those that were assumed without proof. All arithmetic, moreover, can be deduced from the general principles of logic, yet the simple propositions of arithmetic, such as `$2 + 2 = 4$', are just as self-evident as the principles of logic.''}
	
	-- Khi 1 số nguyên tắc logic nhất định đã được thừa nhận, phần còn lại có thể được suy ra từ chúng; nhưng các mệnh đề được suy ra thường hiển nhiên như những mệnh đề được giả định mà không có bằng chứng. Hơn nữa, tất cả số học đều có thể được suy ra từ các nguyên tắc chung của logic, tuy nhiên các mệnh đề đơn giản của số học, chẳng hạn như `$2 + 2 = 4$', cũng hiển nhiên như các nguyên tắc logic.
\end{quote}
\href{https://en.wikipedia.org/wiki/Kurt_G%C3%B6del}{\sc Kurt G\"odel} demonstrated based on his \href{https://en.wikipedia.org/wiki/G%C3%B6del%27s_incompleteness_theorems}{incompleteness theorems} that intuition-based \href{https://en.wikipedia.org/wiki/Propositional_calculus}{propositional calculus} cannot be \href{https://en.wikipedia.org/wiki/Many-valued_logic}{finitely valued}. G\"odel also likened logical intuition to sense perception, \& considered the mathematical constructs that humans perceive to have an independent \href{https://en.wikipedia.org/wiki/Philosophical_realism}{existence} of their own. Under this line of reasoning, the human mind's ability to sense such abstract constructs may not be finitely implementable.

\subsubsection{Discussion}
Dissent (bất đồng chính kiến) regarding the value of intuition in a logical or mathematical context may often hinge on the breadth of the definition of intuition \& the psychological underpinning of the word. Dissent regarding the implications of logical intuition in the fields of AI \& \href{https://en.wikipedia.org/wiki/Cognitive_computing}{cognitive computing} may similarly hinge on definitions. However, similarity between the potentially infinite nature of logical intuition posited by G\"odel \& the \href{https://en.wikipedia.org/wiki/Hard_problem_of_consciousness}{hard problem of consciousness} posited by \href{https://en.wikipedia.org/wiki/David_Chalmers}{\sc David Chalmers} suggest that the realms of intuitive knowledge \& experiential consciousness may both have aspects that are not reducible to classical physics concepts.'' -- \href{https://en.wikipedia.org/wiki/Logical_intuition}{Wikipedia{\tt/}logical intuition}

%------------------------------------------------------------------------------%

\subsection{Wikipedia{\tt/}abstract \& concrete}
``In philosophy \& the \href{https://en.wikipedia.org/wiki/The_arts}{arts},  a fundamental distinction is between things that are {\it abstract} \& things that are {\it concrete}. While there is no general \href{https://en.wikipedia.org/wiki/Consensus_decision-making}{consensus} as to how to precisely defined the 2, examples include that things like \href{https://en.wikipedia.org/wiki/Number}{numbers}, \href{https://en.wikipedia.org/wiki/Set_(mathematics)}{sets}, \& \href{https://en.wikipedia.org/wiki/Idea}{ideas} are abstract objects, while plants, dogs, \& planets are concrete objects. Popular suggestions for a definition include that the distinction between concreteness vs. abstractness is, respectively: between
\begin{enumerate}
	\item existence inside vs. outside \href{https://en.wikipedia.org/wiki/Spacetime}{space-time};
	\item having causes \& effects vs. not;
	\item being related, in \href{https://en.wikipedia.org/wiki/Metaphysics}{metaphysics}, to \href{https://en.wikipedia.org/wiki/Particulars}{particulars} vs. \href{https://en.wikipedia.org/wiki/Universal_(metaphysics)}{universals};
	\item belonging to either the physical vs. the mental realm (or the mental-\&-physical realm vs. neither).
\end{enumerate}
Another view is that it is the distinction between contingent existence vs. necessary existence; however, philosophers differ on which type of existence here defines abstractness, as opposed to concreteness. Despite this diversity of views, there is broad agreement concerning most objects as to whether they are abstract or concrete, s.t. most interpretations agree, e.g., that rocks are concrete objects while numbers are abstract objects.

Abstract objects are most commonly used in philosophy, particularly metaphysics, \& \href{https://en.wikipedia.org/wiki/Semantics}{semantics}. They are sometimes called {\it abstracta} in contrast to {\it concreta}. The term {\it abstract object} is said to have been coined by \href{https://en.wikipedia.org/wiki/Willard_Van_Orman_Quine}{\sc Willard Van Orman Quine}. \href{https://en.wikipedia.org/wiki/Abstract_object_theory}{Abstract object theory} is a discipline that studies the nature \& role of abstract objects. It holds that properties can be related to objects in 2 ways: through exemplification \& through encoding. Concrete objects exemplify their properties while abstract objects merely encode them. This approach is also known as the \href{https://en.wikipedia.org/wiki/Dual_copula_strategy}{dual copula strategy}.

\subsubsection{In philosophy}
The \href{https://en.wikipedia.org/wiki/Type%E2%80%93token_distinction}{type-token distinction} identifies physical objects that are tokens of a particular type of thing. The ``type'' of which it is a part is in itself an abstract object. The abstract-concrete distinction is often introduced \& initially understood in terms of \href{https://en.wikipedia.org/wiki/Paradigm}{paradigmatic} examples of objects of each kind. Examples of abstract vs. concrete objects: tennis vs. a tennis match, redness vs. red light reflected off of an apple \& hitting one's eyes, 5 vs. 5 cars, justice vs. a just action, humanity (the property of being human) vs. human population (the set of all humans).

Abstract objects have often garnered the interest of philosophers because they raise problems for popular theories. In \href{https://en.wikipedia.org/wiki/Ontology}{ontology}, abstract objects are considered problematic for \href{https://en.wikipedia.org/wiki/Physicalism}{physicalism} \& some forms of \href{https://en.wikipedia.org/wiki/Metaphysical_naturalism}{naturalism}. Historically, the most important ontological dispute about abstract objects has been the \href{https://en.wikipedia.org/wiki/Problem_of_universals}{problem of universals}. In \href{https://en.wikipedia.org/wiki/Epistemology}{epistemology}, abstract objects are considered problematic for \href{https://en.wikipedia.org/wiki/Empiricism}{empiricism}. If abstracta lack causal powers \& spatial location, how do we know about them? It is hard to say how they can affect our sensory experiences, \& yet we seem to agree on a wide range of claims about them.

Some, e.g. \href{https://en.wikipedia.org/wiki/Ernst_Mally}{\sc Ernst Mally}, \href{https://en.wikipedia.org/wiki/Edward_Zalta}{\sc Edward Zalta}, \& arguably, \href{https://en.wikipedia.org/wiki/Plato}{\sc Plato} in his \href{https://en.wikipedia.org/wiki/Theory_of_Forms}{\it Theory of Forms}, have held that abstract objects constitute the defining subject matter of \href{https://en.wikipedia.org/wiki/Metaphysics}{metaphysics} or philosophical inquiry more broadly. To the extent that philosophy is independent of empirical research, \& to the extent that empirical questions do not inform questions about abstracta, philosophy would seem especially suited to answering these latter questions.

In \href{https://en.wikipedia.org/wiki/Modern_philosophy}{modern philosophy}, the distinction between abstract \& concrete was explored by \href{https://en.wikipedia.org/wiki/Immanuel_Kant}{\sc Immanuel Kant} \& \href{https://en.wikipedia.org/wiki/G._W._F._Hegel}{G. W. F. Hegel}.

\href{https://en.wikipedia.org/wiki/Gottlob_Frege}{\sc Gottlob Frege} said that abstract objects, e.g., propositions, were members of $\frac{1}{3}$ realm, different from the external world or from internal \href{https://en.wikipedia.org/wiki/Consciousness}{consciousness}. See \href{https://en.wikipedia.org/wiki/Popper%27s_three_worlds}{{\sc Popper}'s 3 worlds}.

\paragraph{Abstract objects \& causality.} Another popular proposal for drawing the abstract-concrete distinction contends that an object is abstract if it lacks \href{https://en.wikipedia.org/wiki/Causality}{causal} power. A causal power has the ability to affect something causally. Thus, the empty set is abstract because it cannot act on other objects. 1 problem with this view is that it is not clear exactly what it is to have causal power. For a more detailed exploration of the abstract-concrete distinction, see the relevant \href{https://en.wikipedia.org/wiki/Stanford_Encyclopedia_of_Philosophy}{\it Stanford Encyclopedia of Philosophy} article.

\paragraph{Quasi-abstract entities.} Recently, there has been some philosophical interest in the development of a 3rd category of objects known as the quasi-abstract. Quasi-abstract objects have drawn particular attention in the area of \href{https://en.wikipedia.org/wiki/Social_ontology}{social ontology} \& \href{https://en.wikipedia.org/wiki/Documentality}{documentality}. Some argue that the over-adherence to the \href{https://en.wikipedia.org/wiki/Platonism}{platonist} duality of the concrete \& the abstract has led to a large category of social objects having been overlooked or rejected as \href{https://en.wikipedia.org/wiki/Nonexistent_object}{nonexistent} because they exhibit characteristics that the traditional duality between concrete \& abstract regards as incompatible. Specifically, the ability to have temporal location, but not spatial location, \& have causal agency (if only by acting through representatives). These characteristics are exhibited by a number of social objects, including states of the international legal system.

\subsubsection{Concrete \& abstract thought in psychology}
\href{https://en.wikipedia.org/wiki/Jean_Piaget}{\sc Jean Piaget} uses the terms ``concrete'' \& ``formal'' to describe 2 different types of learning. Concrete thinking involves facts \& descriptions about everyday, tangible objects, while abstract (\href{https://en.wikipedia.org/wiki/Formal_Operational#Formal_operational_stage}{formal operational}) thinking involves a mental process. Abstract idea vs. Concrete idea: dense things sink vs. It will sink if its density is greater than the density of the fluid, you breathe in oxygen \& breathe out carbon dioxide vs. gas exchange takes place between the air in the alveoli \& the blood, plants get water through their roots vs. water diffuses through the cell membrane of the root hair cells.'' -- \href{https://en.wikipedia.org/wiki/Abstract_and_concrete}{Wikipedia{\tt/}abstract \& concrete}

%------------------------------------------------------------------------------%

\subsection{Wikipedia{\tt/}natural law}
``{\it Natural law} (Latin: {\it ius naturale, lex naturalis}) is a system of law based on a close observation of \href{https://en.wikipedia.org/wiki/Natural_order_(philosophy)}{natural order} \& \href{https://en.wikipedia.org/wiki/Human_nature}{human nature}, from which values, thought by natural law's proponents to be intrinsic to human nature, can be deduced \& applied independently of \href{https://en.wikipedia.org/wiki/Positive_law}{positive law} (the express enacted laws of a \href{https://en.wikipedia.org/wiki/Sovereign_state}{state} or \href{https://en.wikipedia.org/wiki/Society}{society}). According to the theory of law called \href{https://en.wikipedia.org/wiki/Jusnaturalism}{jusnaturalism}, all people have inherent rights, conferred not by act of legislation but by ``\href{https://en.wikipedia.org/wiki/God}{God}, \href{https://en.wikipedia.org/wiki/Nature}{nature}, or \href{https://en.wikipedia.org/wiki/Reason}{reason}''. Natural law theory can also refer to ``theories of ethics, theories of politics, theories of civil law, \& theories of religious morality.''
	
\subsubsection{History}

\subsubsection{Contemporary jurisprudence}

\subsubsection{Methodology}

'' -- \href{https://en.wikipedia.org/wiki/Natural_law}{Wikipedia{\tt/}natural law}

%------------------------------------------------------------------------------%

\subsection{Wikipedia{\tt/}universal law}
``In \href{https://en.wikipedia.org/wiki/Law}{law} \& \href{https://en.wikipedia.org/wiki/Ethics}{ethics}, {\it universal law} or {\it universal principle} refers to concepts of legal \href{https://en.wikipedia.org/wiki/Legitimacy_(political)}{legitimacy} actions, whereby those \href{https://en.wikipedia.org/wiki/Value_(personal_and_cultural)}{principles} \& rules for governing human beings' conduct which are most universal in their acceptability, their applicability, translation, \& \href{https://en.wikipedia.org/wiki/Philosophical}{philosophical} basis, are therefore considered to be most legitimate.

\subsubsection{Debate}
Cognition, experiences \& intuition are the starting points of legal thought, which has to be seen through the glasses of universality \& abstractness. Notwithstanding this assumption, ``legal principles
\begin{enumerate}
	\item do not contain only logic \& reason \&
	\item they can be different in different situations despite their equal naming.
\end{enumerate}
The legal rules can be identical in different legal orders while they carry different wants''.

On 1 side ``universality, abstraction, \& theory itself are defined in a way that undermines the perspectives of some while privileging the perspectives of others''; on the other side, ``the aspiration to universality itself may stand in the way of its realization if it seals off from view the bias built into legal norms, public practices, \& established institutions''.

\subsubsection{Examples}
There are 12 universal laws.
\begin{enumerate}
	\item Law of Divine Oneness
	\item Law of Vibration
	\item Law of Action
	\item Law of Correspondence
	\item Law of Cause \& Effect
	\item Law of Compensation
	\item Law of Attraction
	\item Law of Perpetual Transmutation of Energy
	\item Law of Relativity
	\item Law of Polarity
	\item Law of Rhythm
	\item Law of Gender
\end{enumerate}
'' -- \href{https://en.wikipedia.org/wiki/Universal_law}{Wikipedia{\tt/}universal law}

%------------------------------------------------------------------------------%

\subsection{Wikipedia{\tt/}universality (philosophy)}
``In philosophy, {\it universality} or {\it absolutism} is the idea that universal facts exist \& can be progressively discovered, as opposed to \href{https://en.wikipedia.org/wiki/Relativism}{relativism}, which asserts that all facts are relative to one's perspective. Absolutism \& relativism have been explored at length in contemporary \href{https://en.wikipedia.org/wiki/Analytic_philosophy}{analytic philosophy}.

\subsubsection{Universality in ethics}
Main article: \href{https://en.wikipedia.org/wiki/Moral_universalism}{Wikipedia{\tt/}moral universalism}. When used in the context of ethics, the meaning of {\it universal} refers to that which is true for ``all similarly situated individuals''. \href{https://en.wikipedia.org/wiki/Rights}{Rights}, e.g. in \href{https://en.wikipedia.org/wiki/Natural_rights}{natural rights}, or in the 1789 \href{https://en.wikipedia.org/wiki/Declaration_of_the_Rights_of_Man_and_of_the_Citizen}{Declaration of the Rights of Man \& of the Citizen}, for those heavily influenced by the philosophy of the \href{https://en.wikipedia.org/wiki/Age_of_Enlightenment}{Enlightenment} \& its conception of a \href{https://en.wikipedia.org/wiki/Human_nature}{human nature}, could be considered universal. The 1948 \href{https://en.wikipedia.org/wiki/Universal_Declaration_of_Human_Rights}{Universal Declaration of Human Rights} is inspired by such principles.

Universal moralities contrast with \href{https://en.wikipedia.org/wiki/Moral_relativism}{moral relativisms}, which seek to account for differing ethical positions between people \& \href{https://en.wikipedia.org/wiki/Social_norm}{cultural norms}.

\subsubsection{Universality about truth}
In logic, or the consideration of valid arguments, a \href{https://en.wikipedia.org/wiki/Proposition}{proposition} is said to have universality if it can be conceived as being true in all possible contexts without creating a \href{https://en.wikipedia.org/wiki/Contradiction}{contradiction}. A \href{https://en.wikipedia.org/wiki/Universalism}{universalist conception of truth} accepts 1 or more universals, whereas a \href{https://en.wikipedia.org/wiki/Relativism}{relativist conception of truth} denies the existence of some or all \href{https://en.wikipedia.org/wiki/Problem_of_universals}{universals}.

\subsubsection{Universal in metaphysics}
Main article: \href{https://en.wikipedia.org/wiki/Universal_(metaphysics)}{Wikipedia{\tt/}universal (metaphysics)}. In \href{https://en.wikipedia.org/wiki/Metaphysics}{metaphysics}, a \href{https://en.wikipedia.org/wiki/Universal_(metaphysics)}{universal} is a proposed \href{https://en.wikipedia.org/wiki/Type_(metaphysics)}{type}, \href{https://en.wikipedia.org/wiki/Property_(metaphysics)}{property}, or \href{https://en.wikipedia.org/wiki/Relation_(metaphysics)}{relation} which can be instantiated by many different \href{https://en.wikipedia.org/wiki/Particular}{particulars}. While universals are related to the concept o universality, the concept is importantly distinct, see the main page on universals for a full treatment of the topic.'' -- \href{https://en.wikipedia.org/wiki/Universality_(philosophy)}{Wikipedia{\tt/}universality (philosophy)}

%------------------------------------------------------------------------------%

\section{{\sc Blaise Pascal}. {\it Pens\'ees} \& Other Writings}

\begin{enumerate}
	\item \cite{Pascal_pensees}.{\sc Blaise Pascal}. {\it Pens\'ees}.
	
	{\sf Note.} {\it Passages} erased by {\sc Pascal} are enclosed in square brackets, thus []. {\it Words}, added or corrected by the editor of the text, are similarly denoted, but are in italics. It has been seen fit to transfer Fragment 514 of the French edition to the Notes. All subsequent Fragments have accordingly been renumbered.
	\begin{itemize}
		\item {\sf Introduction by {\sc T. S. Eliot}.}
		\item {\sf Thoughts on Mind \& on Style.}
		
		\fbox{1} {\it The difference between the mathematical \& the intuitive mind.} -- In the one the principles are palpable, but removed from ordinary use; so that for want of habit it is difficult to turn one's mind in that direction: but if one turns it thither ever so little, one sees the principles fully, \& one must have a quite inaccurate mind who reasons wrongly from principles so plain that it is almost impossible they should escape notice.
		
		-- {\it Sự khác biệt giữa toán học \& tâm trí trực quan.} -- Trong trường hợp thứ nhất, các nguyên tắc là rõ ràng, nhưng tách biệt khỏi cách sử dụng thông thường; do đó, nếu không có thói quen, rất khó để hướng tâm trí theo hướng đó: nhưng nếu hướng tâm trí theo hướng đó 1 chút, người ta sẽ thấy các nguyên tắc 1 cách đầy đủ, \& người ta hẳn phải có 1 tâm trí hoàn toàn không chính xác, người lý luận sai lầm từ các nguyên tắc quá rõ ràng đến mức gần như không thể thoát khỏi sự chú ý.
		
		But in the intuitive mind the principles are found in common use, \& are before the eyes of everybody. One has only to look, \& no effort is necessary; it is only a question of good eyesight, but it must be good, for the principles are so subtle \& so numerous, that it is almost impossible but that some escape notice. Now the omission of 1 principle leads to error; thus one must have very clear sight to see all the principles, \& in the next place an accurate mind not to draw false deductions from known principles.
		
		-- Nhưng trong tâm trí trực giác, các nguyên tắc được tìm thấy trong cách sử dụng phổ biến, \& nằm trước mắt mọi người. Người ta chỉ cần nhìn, \& không cần nỗ lực; chỉ là vấn đề về thị lực tốt, nhưng phải tốt, vì các nguyên tắc rất tinh tế \& rất nhiều, đến nỗi gần như không thể không có 1 số người thoát khỏi sự chú ý. Bây giờ việc bỏ sót 1 nguyên tắc dẫn đến sai lầm; do đó, người ta phải có thị lực rất rõ ràng để thấy tất cả các nguyên tắc, \& tiếp theo là 1 tâm trí chính xác để không rút ra những suy luận sai lầm từ các nguyên tắc đã biết.
				
		All mathematicians would then be intuitive if they had clear sight, for they do not reason incorrectly from principles known to them; \& intuitive minds would be mathematical if they could turn their eyes to the principles of mathematics to which they are unused.
		
		-- Tất cả các nhà toán học sẽ có trực giác nếu họ có tầm nhìn rõ ràng, vì họ không lý luận sai từ các nguyên tắc mà họ biết; \& những tâm trí trực giác sẽ có tư duy toán học nếu họ có thể hướng mắt đến các nguyên tắc toán học mà họ chưa quen.
		
		The reason, therefore, that some intuitive minds are not mathematical is that they cannot at all turn their attention to the principles of mathematics. But the reason that mathematicians are not intuitive is that they do not see what is before them, \& that, accustomed to the exact \& plain principles of mathematics, \& not reasoning till they have well inspected \& arranged their principles, they are lost in matters of intuition where the principles do not allow of such arrangement. They are scarcely seen; they are felt rather than seen; there is the greatest difficulty in making them felt by those who do not of themselves perceive them. These principles are so fine \& so numerous that a very delicate \& very clear sense is needed to perceive them, \& to judge rightly \& justly when they are perceived, without for the most part being able to demonstrate them in order as in mathematics; because the principles are not known to us in the same way, \& because it would be an endless matter to undertake it. We must see the matter at once, at 1 glance, \& not by a process of reasoning, at least to a certain degree. \& thus it is rare that mathematicians are intuitive, \& that men of intuition are mathematicians, because mathematicians wish to treat matters of intuition mathematically, \& make themselves ridiculous, wishing to begin with definitions \& then with axioms, which is not the way to proceed in this kind of reasoning. Not that the mind does not do so, but it does it tacitly, naturally, \& without technical rules; for the expression of it is beyond all men, \& only a few can feel it.
		
		-- Do đó, lý do tại sao 1 số tâm trí trực quan không phải là toán học là vì họ không thể hướng sự chú ý của mình vào các nguyên tắc toán học. Nhưng lý do tại sao các nhà toán học không trực quan là vì họ không nhìn thấy những gì trước mắt họ, \& rằng, quen với các nguyên tắc toán học chính xác \& rõ ràng, \& không lý luận cho đến khi họ kiểm tra kỹ \& sắp xếp các nguyên tắc của mình, họ bị lạc vào các vấn đề trực quan mà các nguyên tắc không cho phép sắp xếp như vậy. Chúng hầu như không được nhìn thấy; chúng được cảm nhận hơn là được nhìn thấy; có khó khăn lớn nhất trong việc khiến chúng cảm nhận được bởi những người không tự mình nhận thức được chúng. Những nguyên tắc này rất tinh tế \& rất nhiều đến nỗi cần có 1 giác quan rất tinh tế \& rất rõ ràng để nhận thức chúng, \& để phán đoán đúng \& công bằng khi chúng được nhận thức, mà phần lớn không thể chứng minh chúng theo thứ tự như trong toán học; bởi vì chúng ta không biết các nguyên tắc theo cùng 1 cách, \& bởi vì sẽ là 1 vấn đề vô tận để thực hiện nó. Chúng ta phải nhìn thấy vấn đề ngay lập tức, chỉ trong nháy mắt, \& không phải bằng 1 quá trình lý luận, ít nhất là ở 1 mức độ nhất định. \& do đó hiếm khi các nhà toán học có trực giác, \& những người có trực giác là nhà toán học, bởi vì các nhà toán học muốn xử lý các vấn đề trực giác theo phương pháp toán học, \& làm cho mình trở nên lố bịch, muốn bắt đầu bằng các định nghĩa \& sau đó bằng các tiên đề, đó không phải là cách tiến hành trong loại lý luận này. Không phải là tâm trí không làm như vậy, nhưng nó làm điều đó 1 cách ngầm hiểu, tự nhiên, \& không có các quy tắc kỹ thuật; vì cách diễn đạt của nó vượt quá tất cả mọi người, \& chỉ 1 số ít có thể cảm nhận được.
		
		Intuitive minds, on the contrary, being thus accustomed to judge at a single glance, are so astonished when they are presented with propositions of which they understand nothing, \& the way to which is through definitions \& axioms so sterile, \& which they are not accustomed to see thus in detail, that they are repelled \& disheartened.
		
		-- Ngược lại, những tâm trí trực giác, vốn quen với việc phán đoán chỉ qua 1 cái nhìn, sẽ rất ngạc nhiên khi họ được trình bày những mệnh đề mà họ không hiểu gì cả, \& cách tiếp cận thông qua các định nghĩa \& tiên đề vô bổ, \& mà họ không quen xem xét chi tiết như vậy, đến nỗi họ cảm thấy khó chịu \& nản lòng.
		
		But dull minds are never either intuitive or mathematical.
		
		-- Nhưng những tâm trí chậm chạp không bao giờ có trực giác hoặc khả năng toán học.
		
		Mathematicians who are only mathematicians have exact minds, provided all things are explained to them by means of definitions \& axioms; otherwise they are inaccurate \& insufferable, for they are only right when the principles are quite clear.
		
		-- Các nhà toán học thực chất là các nhà toán học có tư duy chính xác, miễn là mọi thứ được giải thích cho họ thông qua các định nghĩa \& tiên đề; nếu không, chúng sẽ không chính xác \& khó chịu, vì chúng chỉ đúng khi các nguyên tắc thực sự rõ ràng.
		
		\& men of intuition who are only intuitive cannot have the patience to reach to 1st principles of things speculative \& conceptual, which they have never seen in the world, \& which are altogether out of the common.
		
		-- \& những người trực giác chỉ có trực giác không thể đủ kiên nhẫn để đạt tới những nguyên lý đầu tiên của những thứ mang tính suy đoán \& khái niệm, những thứ mà họ chưa từng thấy trên thế giới, \& hoàn toàn khác thường.
		
		\fbox{2} There are different kinds of right understanding; some have right understanding in a certain order of things, \& not in others, where they go astray. Some draw conclusions well from a few premises, \& this displays an acute judgment.
		
		-- Có nhiều loại hiểu biết đúng đắn khác nhau; 1 số hiểu biết đúng đắn theo 1 trật tự nhất định của sự vật, \& không hiểu đúng đắn theo những trật tự khác, nơi họ đi lạc. Một số rút ra kết luận tốt từ 1 vài tiền đề, \& điều này thể hiện sự phán đoán sắc sảo.
		
		Others draw conclusions well where there are many premises.
		
		-- Những người khác đưa ra kết luận tốt khi có nhiều tiền đề.
		
		E.g., the former easily learn hydrostatics, where the premises are few, but the conclusions are so fine that only the greatest acuteness can reach them.
		
		-- Ví dụ, người đầu tiên dễ dàng học được thủy tĩnh học, trong đó có ít tiền đề nhưng kết luận lại rất tinh tế đến mức chỉ có sự nhạy bén nhất mới có thể đạt tới.
		
		\& in spite of that these persons would perhaps not be great mathematicians, because mathematics contain a great number of premises, \& there is perhaps a kind of intellect that can search with ease a few premises to the bottom, \& cannot in the least penetrate those matters in which they are many premises.
		
		-- \& mặc dù những người này có lẽ không phải là những nhà toán học vĩ đại, bởi vì toán học chứa đựng rất nhiều tiền đề, \& có lẽ có 1 loại trí tuệ có thể dễ dàng tìm kiếm 1 vài tiền đề đến tận cùng, \& không thể thâm nhập vào những vấn đề có nhiều tiền đề.
		
		There are then 2 kinds of intellect: the one able to penetrate acutely \& deeply into the conclusions of given premises, \& this is the precise intellect; the other able to comprehend a great number of premises without confusing them, \& this is the mathematical intellect. The one has force \& exactness, the other comprehension. Now the one quality can exist without the other; the intellect can be strong \& narrow, \& can also be comprehensive \& weak.
		
		-- Sau đó có 2 loại trí tuệ: loại có khả năng thâm nhập sâu sắc \& vào kết luận của các tiền đề đã cho, \& đây là trí tuệ chính xác; loại còn lại có khả năng hiểu được 1 số lượng lớn các tiền đề mà không làm chúng nhầm lẫn, \& đây là trí tuệ toán học. Loại này có sức mạnh \& chính xác, loại kia có khả năng hiểu biết. Bây giờ, 1 phẩm chất có thể tồn tại mà không có phẩm chất kia; trí tuệ có thể mạnh \& hẹp, \& cũng có thể toàn diện \& yếu.
		
		\fbox{3} Those who are accustomed to judge by feeling do not understand the process of reasoning, for they would understand at 1st sight, \& are not used to seek for principles. \& others, on the contrary, who are accustomed to reason from principles, do not at all understand matters of feeling, seeking principles, \& being unable to see at a glance.
		
		-- Những người quen phán đoán bằng cảm tính không hiểu được quá trình lý luận, vì họ có thể hiểu ngay từ cái nhìn đầu tiên, \& không quen tìm kiếm nguyên tắc. \& ngược lại, những người khác quen lý luận theo nguyên tắc, hoàn toàn không hiểu vấn đề về cảm tính, tìm kiếm nguyên tắc, \& không thể nhìn thấy ngay trong cái nhìn thoáng qua.
		
		\fbox{4} {\it Mathematics, intuition.} -- True eloquence makes light of eloquence, true morality makes light of morality; i.e., the morality of the judgment, which has no rules, makes light of the morality of the intellect.
		
		-- {\it Toán học, trực giác.} -- Sự hùng biện thực sự coi thường sự hùng biện, đạo đức thực sự coi thường đạo đức; tức là đạo đức của sự phán đoán, không có quy tắc, coi thường đạo đức của trí tuệ.
		
		For it is to judgment that perception belongs, as science belongs to intellect. Intuition is the part of judgment, mathematics of intellect.
		
		-- Vì nhận thức thuộc về phán đoán, cũng như khoa học thuộc về trí tuệ. Trực giác là 1 phần của phán đoán, toán học của trí tuệ.
		
		To make light of philosophy is to be a true philosopher.
		
		-- Coi thường triết học mới là 1 triết gia thực thụ.
		
		\fbox{5} Those who judge of a work by rule are in regard to others as those who have a watch are in regard to others. One says, ``It is 2 hours ago''; the other says, ``It is only 3-quarters of an hour.'' I look at my watch, \& say to the one, ``You are weary,'' \& to the other, ``Time gallops with you''; for it is only $1.5$ hour ago, \& I laugh at those who tell me that time goes slowly with me, \& that I judge by imagination. They do not know that I judge by my watch.
		
		-- Những người đánh giá 1 tác phẩm theo quy tắc thì liên quan đến những người khác như những người có đồng hồ liên quan đến những người khác. Một người nói, ``Đã 2 giờ rồi''; người kia nói, ``Mới chỉ 3 phần tư giờ.'' Tôi nhìn vào đồng hồ của mình, \& nói với người này, ``Bạn mệt rồi,'' \& với người kia, ``Thời gian phi nước đại với bạn''; vì mới chỉ 1,5 giờ trước, \& Tôi cười những người nói với tôi rằng thời gian trôi chậm với tôi, \& rằng tôi đánh giá bằng trí tưởng tượng. Họ không biết rằng tôi đánh giá bằng đồng hồ của mình.
		
		\fbox{6} Just as we harm the understanding, we harm the feelings also.
		
		-- Khi chúng ta làm tổn hại đến sự hiểu biết, chúng ta cũng làm tổn hại đến cảm xúc.
		
		The understanding \& the feelings are moulded by intercourse; the understanding \& feelings are corrupted by intercourse. Thus good or bad society improves or corrupts them. It is, then, all-important to know how to choose in order to improve \& not to corrupt them; \& we cannot make this choice, if they be not already improved \& not corrupted. Thus a circle is formed, \& those are fortunate who escape it.
		
		-- Sự hiểu biết \& cảm xúc được định hình bởi giao hợp; sự hiểu biết \& cảm xúc bị làm hư hỏng bởi giao hợp. Do đó, xã hội tốt hay xấu cải thiện hoặc làm hư hỏng chúng. Do đó, điều quan trọng nhất là biết cách lựa chọn để cải thiện \& không làm hư hỏng chúng; \& chúng ta không thể đưa ra lựa chọn này, nếu chúng chưa được cải thiện \& không bị hư hỏng. Do đó, 1 vòng tròn được hình thành, \& những người may mắn thoát khỏi nó.
		
		\fbox{7} The greater intellect one has, the more originality one finds in men. Ordinary persons find no difference between men.
		
		-- Người ta càng có trí tuệ cao thì càng thấy tính độc đáo ở con người. Người bình thường không thấy có sự khác biệt giữa con người.
		
		\fbox{8} There are many people who listen to a sermon in the same way as they listen to vespers.
		
		-- Có nhiều người lắng nghe bài giảng theo cách tương tự như lắng nghe kinh chiều.
		
		\fbox{9} When we wish to correct with advantage, \& to show another that he errs, we must notice from what side he views the matter, for on that side it is usually true, \& admit that truth to him, but reveal to him the side on which it is false. He is satisfied with that, for he sees that he was not mistaken, \& that he only failed to see all sides. Now, no one is offended at not seeing everything; but one does not like to be mistaken, \& that perhaps arises from the fact that man naturally cannot see everything, \& that naturally he cannot err in the side he looks at, since the perceptions of our senses are always true.
		
		-- Khi chúng ta muốn sửa chữa 1 cách có lợi, \& để chỉ cho người khác thấy rằng anh ta sai, chúng ta phải để ý xem anh ta nhìn nhận vấn đề từ phía nào, vì ở phía đó thường là đúng, \& thừa nhận sự thật đó với anh ta, nhưng tiết lộ cho anh ta biết phía mà nó sai. Anh ta hài lòng với điều đó, vì anh ta thấy rằng anh ta không nhầm, \& rằng anh ta chỉ không nhìn thấy mọi phía. Bây giờ, không ai bị xúc phạm vì không nhìn thấy mọi thứ; nhưng người ta không thích bị nhầm, \& điều đó có lẽ phát sinh từ thực tế là con người tự nhiên không thể nhìn thấy mọi thứ, \& rằng tự nhiên anh ta không thể sai lầm ở phía mà anh ta nhìn vào, vì nhận thức của các giác quan của chúng ta luôn đúng.
		
		\fbox{10} People are generally better persuaded by the reasons which they have themselves discovered than by those which have come into the mind of others.
		
		-- Mọi người thường dễ bị thuyết phục hơn bởi những lý do mà họ tự khám phá ra hơn là những lý do xuất hiện trong tâm trí của người khác.
		
		\fbox{11} All great amusements are dangerous to the Christian life; but among all those which the world has invented there is none more to be feared than the theater. It is a representation of the passions so natural \& so delicate that it excites them \& gives birth to them in our hearts, \&, above all, to that of love, principally when it is represented as very chaste \& virtuous. For the more innocent it appears to innocent souls, the more they are likely to be touched by it. Its violence pleases our self-love, which immediately forms a desire to produce the same effects which are seen so well represented; \&, at the same time, we make ourselves a conscience founded on the propriety of the feelings which we see there, by which the fear of pure souls is removed, since they imagin that it cannot hurt their purity to love with a love which seems to them so reasonable.
		
		-- Mọi trò tiêu khiển lớn đều nguy hiểm cho đời sống Cơ đốc; nhưng trong số tất cả những trò mà thế gian đã phát minh ra, không có trò nào đáng sợ hơn trò kịch. Nó là sự thể hiện của những đam mê rất tự nhiên \& rất tinh tế đến nỗi nó kích thích chúng \& sinh ra chúng trong trái tim chúng ta, \& trên hết, là tình yêu, chủ yếu là khi nó được thể hiện là rất trong sáng \& đức hạnh. Vì nó càng có vẻ ngây thơ đối với những tâm hồn ngây thơ, thì chúng càng có khả năng bị nó chạm đến. Sự bạo lực của nó làm thỏa mãn lòng tự ái của chúng ta, điều này ngay lập tức hình thành mong muốn tạo ra những hiệu ứng giống như được thể hiện rất rõ ràng; \&, đồng thời, chúng ta tự tạo cho mình 1 lương tâm dựa trên sự phù hợp của những cảm xúc mà chúng ta thấy ở đó, nhờ đó nỗi sợ hãi của những tâm hồn trong sáng được loại bỏ, vì họ tưởng rằng không thể làm tổn thương sự trong sáng của họ khi yêu bằng 1 tình yêu mà đối với họ có vẻ hợp lý như vậy.
		
		So we depart from the theater with our heart so filled with all the beauty \& tenderness of love, the soul \& the mind so persuaded of its innocence, that we are quite ready to receive its 1st impressions, or rather to seek an opportunity of awakening them in the heart of another, in order that we may receive the same pleasures \& the same sacrifices which we have seen so well represented in the theater.
		
		-- Vì vậy, chúng ta rời khỏi nhà hát với trái tim tràn ngập vẻ đẹp \& sự dịu dàng của tình yêu, tâm hồn \& trí óc tin chắc vào sự ngây thơ của mình, đến nỗi chúng ta hoàn toàn sẵn sàng đón nhận những ấn tượng đầu tiên, hay đúng hơn là tìm kiếm cơ hội đánh thức chúng trong trái tim của người khác, để chúng ta có thể nhận được những thú vui \& những hy sinh giống như những gì chúng ta đã thấy được thể hiện rất tốt trong nhà hát.
		
		\fbox{12} Scaramouch, who only thinks of 1 thing.
		
		-- Scaramouch chỉ nghĩ đến 1 điều.
		
		The doctor, who speaks for 15 mins after he has said everything, so full is he of the desire of talking.
		
		-- Vị bác sĩ, người nói chuyện thêm 15 phút sau khi đã nói hết mọi thứ, vì ông ấy rất muốn được nói.
		
		\fbox{13} One likes to see the error, the passion of Cleobuline, because she is unconscious of it. She would be displeasing, if she were not deceived.
		
		-- Người ta thích nhìn thấy lỗi lầm, sự đam mê của Cleobuline, vì cô ấy không ý thức được điều đó. Cô ấy sẽ rất khó chịu nếu không bị lừa dối.
		
		\fbox{14} When a natural discourse paints a passion or an effect, one feels within oneself the truth of what one reads, which were there before, although one did not know it. Hence one is inclined to love him who makes us feel it, for he has not shown us his own riches, but ours. \& thus this benefit renders him pleasing to us, besides that such community of intellect as we have with him necessarily inclines the heart to love.
		
		-- Khi 1 bài diễn thuyết tự nhiên vẽ nên 1 niềm đam mê hay 1 hiệu ứng, người ta cảm thấy trong chính mình sự thật của những gì mình đọc, vốn đã có trước đó, mặc dù người ta không biết điều đó. Do đó, người ta có xu hướng yêu người khiến chúng ta cảm thấy điều đó, vì người đó không cho chúng ta thấy sự giàu có của riêng mình, mà là của chúng ta. \& do đó, lợi ích này khiến người đó làm chúng ta hài lòng, bên cạnh đó, cộng đồng trí tuệ như chúng ta có với người đó nhất thiết khiến trái tim hướng đến tình yêu.
		
		\fbox{15} Eloquence, which persuades by sweetness, not by authority; as a tyrant, not as a king.
		
		-- Sự hùng biện, thuyết phục bằng sự ngọt ngào, không phải bằng uy quyền; như 1 bạo chúa, không phải như 1 vị vua.
		
		\fbox{16} Eloquence is an art of saying things in such a way -- (1) that those to whom we speak may listen to them without pain \& with pleasure; (2) that they feel themselves interested, so that self-love leads them more willingly to reflection upon it.
		
		-- Sự hùng biện là nghệ thuật diễn đạt mọi việc theo cách -- (1) để những người mà chúng ta nói chuyện có thể lắng nghe mà không đau đớn \& với sự thích thú; (2) để họ cảm thấy mình quan tâm, để tình yêu bản thân khiến họ sẵn sàng suy ngẫm về điều đó hơn.
		
		It consists, then, in a correspondence which we seek to establish between the head \& the heart of those to whom we speak on the 1 hand, \& on the other, between the thoughts \& the expressions which we employ. This assumes that we have studied well the heart of man so as to know all its powers, \& then to find the just proportions of the discourse which we wish to adapt to them. We must put ourselves in the place of those who are to hear us, \& make trial on our own heart of the turn which we give to our discourse in order to see whether one is made for the other, \& whether we can assure ourselves that the hearer will be, as it were, forced to surrender. We ought to restrict ourselves, so far as possible, to the simple \& natural, \& not to magnify that which is little, or belittle that which is great. It is not enough that a thing be beautiful; it must be suitable to the subject, \& there must be in it nothing of excess or defect.
		
		-- Sau đó, nó bao gồm 1 sự tương ứng mà chúng ta tìm cách thiết lập giữa đầu \& trái tim của những người mà chúng ta nói chuyện với 1 mặt, \& mặt khác, giữa những suy nghĩ \& những biểu hiện mà chúng ta sử dụng. Điều này giả định rằng chúng ta đã nghiên cứu kỹ lưỡng trái tim của con người để biết tất cả các sức mạnh của nó, \& sau đó tìm ra tỷ lệ chính xác của bài phát biểu mà chúng ta muốn điều chỉnh cho họ. Chúng ta phải đặt mình vào vị trí của những người sẽ nghe chúng ta, \& thử nghiệm trên trái tim của chính mình về lượt mà chúng ta dành cho bài phát biểu của mình để xem liệu người này có dành cho người kia không, \& liệu chúng ta có thể tự đảm bảo rằng người nghe sẽ, như thể, bị buộc phải đầu hàng. Chúng ta nên hạn chế bản thân, trong khả năng có thể, vào những điều đơn giản \& tự nhiên, \& không phóng đại những gì nhỏ bé, hoặc hạ thấp những gì vĩ đại. Một điều gì đó đẹp thôi là chưa đủ; nó phải phù hợp với chủ đề, \& không được có bất cứ điều gì dư thừa hay khiếm khuyết trong đó.
		
		\fbox{17} Rivers are roads which move, \& which carry us whether we desire to go.
		
		-- Sông là những con đường chuyển động, \& đưa chúng ta đi dù chúng ta có muốn đi hay không.
		
		\fbox{18} When we do not know the truth of a thing, it is of advantage that there should exist a common error which determines the mind of man, as, e.g., the moon, to which is attributed the change of reasons, the progress of diseases, etc. For the chief malady of man is restless curiosity about things which he cannot understand; \& it is not so bad for him to be in error as to be curious to no purpose.
		
		-- Khi chúng ta không biết sự thật của 1 điều gì đó, thì sẽ có lợi nếu tồn tại 1 lỗi chung quyết định tâm trí con người, chẳng hạn như mặt trăng, được cho là nguyên nhân gây ra sự thay đổi lý trí, sự tiến triển của bệnh tật, v.v. Vì căn bệnh chính của con người là sự tò mò không ngừng về những thứ mà họ không thể hiểu; \& không tệ khi họ mắc lỗi bằng việc tò mò không có mục đích.
		
		The manner in which {\sc Epictetus, Montaigne, \& Salomon de Tultie} wrote, is the most usual, the most suggestive, the most remembered, \& the oftenest quoted; because it is entirely composed of thoughts born from the common talk of life. As when we speak of the common error which exists among men that the moon is the cause of everything, we never fail to say that {\sc Salomon de Tultie} says that when we do not know the truth of a thing, it is of advantage that there should exists a common error, etc.; which is the thought above.
		
		-- Cách mà {\sc Epictetus, Montaigne, \& Salomon de Tultie} viết, là cách thông thường nhất, gợi ý nhất, được nhớ đến nhiều nhất, \& được trích dẫn nhiều nhất; bởi vì nó hoàn toàn bao gồm những suy nghĩ nảy sinh từ cuộc nói chuyện chung của cuộc sống. Giống như khi chúng ta nói về lỗi chung tồn tại giữa con người rằng mặt trăng là nguyên nhân của mọi thứ, chúng ta không bao giờ không nói rằng {\sc Salomon de Tultie} nói rằng khi chúng ta không biết sự thật của 1 điều gì đó, thì việc tồn tại 1 lỗi chung là có lợi, v.v.; đó là suy nghĩ ở trên.
		
		\fbox{19} The last thing one settles in writing a book is what one should put in 1st.
		
		-- Điều cuối cùng mà người ta cần lưu ý khi viết 1 cuốn sách là nên đưa những gì vào đầu tiên.
		
		\fbox{20} {\it Order.} -- Why should I undertake to divide my virtues into 4 rather than into 6? Why should I rather establish virtue in 4, in 2, in 1? Why into {\it Abstine et sustine} rather than into ``Follow Nature,'' or ``Conduct your private affairs without injustice,'' as {\sc Plato}, or anything else? But there, you will say, everything is contained in 1 word. yes, but it is useless without explanation, \& when we come to explain it, as soon as we unfold this maxim which contains all the rest, they emerge in that 1st confusion which you desired to avoid. So, when they are all included in one, they are hidden \& useless, as in a chest, \& never appear save in their natural confusion. Nature has established them all without including one in the other.
		
		-- {\it Thứ tự.} -- Tại sao tôi phải chia đức hạnh của mình thành 4 thay vì thành 6? Tại sao tôi lại phải thiết lập đức hạnh thành 4, thành 2, thành 1? Tại sao thành {\it Abstine et sustine} thay vì thành ``Tuân theo Tự nhiên'', hay ``Tiến hành công việc riêng tư của bạn mà không có bất công'', như {\sc Plato}, hay bất cứ điều gì khác? Nhưng ở đó, bạn sẽ nói, mọi thứ đều nằm trong 1 từ. đúng, nhưng sẽ vô ích nếu không có lời giải thích, \& khi chúng ta bắt đầu giải thích, ngay khi chúng ta mở ra câu châm ngôn này chứa đựng tất cả phần còn lại, chúng xuất hiện trong sự nhầm lẫn đầu tiên mà bạn muốn tránh. Vì vậy, khi tất cả chúng được bao gồm trong một, chúng bị ẩn \& vô dụng, như trong 1 cái rương, \& không bao giờ xuất hiện ngoại trừ trong sự nhầm lẫn tự nhiên của chúng. Tự nhiên đã thiết lập tất cả chúng mà không bao gồm chúng trong nhau.
		
		\fbox{21} Nature has made all her truths independent of 1 another. Our art makes one dependent on the other. But this is not natural. Each keeps its own place.
		
		-- Thiên nhiên đã tạo ra tất cả chân lý của mình độc lập với nhau. Nghệ thuật của chúng ta làm cho cái này phụ thuộc vào cái kia. Nhưng điều này không tự nhiên. Mỗi cái giữ vị trí riêng của nó.
		
		\fbox{22} Let no one say that I have said nothing new; the arrangement of the subject is new. When we play tennis, we both play with the same ball, but 1 of us places it better.
		
		-- Đừng ai nói rằng tôi không nói điều gì mới; cách sắp xếp chủ đề là mới. Khi chúng ta chơi quần vợt, cả hai chúng ta đều chơi với cùng 1 quả bóng, nhưng 1 trong chúng ta đặt nó tốt hơn.
		
		I had as soon it said that I used words employed before. \& in the same way if the same thoughts in a different arrangement do not form a different discourse, no more do the same words in their different arrangement form different thoughts!
		
		-- Tôi đã từng nói rằng tôi sử dụng những từ đã dùng trước đó. \& tương tự như vậy, nếu cùng 1 ý nghĩ được sắp xếp khác nhau không tạo thành 1 diễn ngôn khác nhau, thì cùng 1 từ được sắp xếp khác nhau cũng không tạo thành những ý nghĩ khác nhau!		
		
		\fbox{23} Words differently arranged have a different meaning, \& meanings differently arranged have different effects.
		
		-- Các từ được sắp xếp khác nhau sẽ mang lại ý nghĩa khác nhau, \& các ý nghĩa được sắp xếp khác nhau sẽ có hiệu quả khác nhau.
		
		\fbox{24} {\it Language.} -- We should not turn the mind from 1 thing to another, except for relaxation, \& that when it is necessary \& the time suitable, \& not otherwise. For he that relaxes out of season wearies, \& he who wearies us out of season makes us languid, since we turn quite away. So much does our perverse lust like to do the contrary of what those wish to obtain from us without giving us pleasure, the coin for which we will do whatever is wanted.
		
		-- {\it Ngôn ngữ.} -- Chúng ta không nên chuyển tâm trí từ việc này sang việc khác, ngoại trừ việc thư giãn, \& khi cần thiết \& thời điểm thích hợp, \& không được làm khác. Vì người thư giãn ngoài mùa sẽ mệt mỏi, \& người làm chúng ta mệt mỏi ngoài mùa sẽ khiến chúng ta uể oải, vì chúng ta quay đi hoàn toàn. Dục vọng đồi trụy của chúng ta thích làm ngược lại những gì những người muốn có được từ chúng ta mà không làm chúng ta vui, đồng tiền mà chúng ta sẽ làm bất cứ điều gì được yêu cầu.
		
		\fbox{25} {\it Eloquence.} -- It requires the pleasant \& the real; but the pleasant must itself be drawn from the true.
		
		-- {\it Sự hùng biện.} -- Nó đòi hỏi sự dễ chịu \& sự thực; nhưng bản thân sự dễ chịu phải được rút ra từ sự thực.
		
		\fbox{26} Eloquence is a painting of thought; \& thus those who, after having painted it, add something more, make a picture instead of a portrait.
		
		-- Sự hùng biện là bức tranh của tư tưởng; \& do đó, những người sau khi đã vẽ xong, thêm thắt điều gì đó, tạo nên 1 bức tranh thay vì 1 bức chân dung.
		
		\fbox{27} {\it Miscellaneous. Language.} -- Those who make atitheses by forcing words are like those who make false windows for symmetry. Their rule is not to speak accurately, but to make apt figures of speech.
		
		-- {\it Lặt Vặt. Ngôn ngữ.} -- Những người tạo ra những sự kiện bất thường bằng cách ép buộc từ ngữ giống như những người tạo ra những cửa sổ giả để tạo sự cân xứng. Quy tắc của họ không phải là nói chính xác, mà là tạo ra những hình ảnh ẩn dụ thích hợp.
		
		\fbox{28} Symmetry is what we see at a glance; based on the fact that there is no reason for any difference, \& based also on the face of man; whence it happens that symmetry is only wanted in breadth, not in height or depth.
		
		-- Tính đối xứng là thứ chúng ta nhìn thấy ngay từ cái nhìn đầu tiên; dựa trên thực tế là không có lý do gì cho bất kỳ sự khác biệt nào, \& cũng dựa trên khuôn mặt của con người; do đó tính đối xứng chỉ được mong muốn ở chiều rộng, không phải chiều cao hay chiều sâu.
		
		\fbox{29} When we see a natural style, we are astonished \& delighted; for we expected to see an author, \& we find a man. Whereas those who have good taste, \& who seeing a book expect to find a man, are quite surprised to find an author. {\it Plus poetice quam humane locutus es}. Those honor Nature well, who teach that she can speak on everything, even on theology.
		
		-- Khi chúng ta thấy 1 phong cách tự nhiên, chúng ta ngạc nhiên \& vui mừng; vì chúng ta mong đợi nhìn thấy 1 tác giả, \& chúng ta tìm thấy 1 người đàn ông. Trong khi những người có gu thẩm mỹ tốt, \& những người nhìn thấy 1 cuốn sách mong đợi tìm thấy 1 người đàn ông, thì khá ngạc nhiên khi tìm thấy 1 tác giả. {\it Plus poetice quam humane locutus es}. Những người tôn vinh Thiên nhiên tốt, những người dạy rằng cô ấy có thể nói về mọi thứ, thậm chí về thần học.
				
		\fbox{30} We only consult the ear because the heart is wanting. The rule is uprightness.
		
		-- Chúng ta chỉ tham khảo tai vì trái tim thiếu sự chính trực. Quy tắc là sự ngay thẳng.
		
		Beauty of omission, of judgment.
		
		-- Vẻ đẹp của sự bỏ sót, của sự phán đoán.
		
		\fbox{31} All the false beauties which we blame into Cicero have their admirers, \& in great number.
		
		-- Tất cả những vẻ đẹp giả tạo mà chúng ta đổ lỗi cho Cicero đều có rất nhiều người ngưỡng mộ.
		
		\fbox{32} There is a certain standard of grace \& beauty which consists in a certain relation between our nature, such as it is, weak or strong, \& the thing which pleases us.
		
		-- Có 1 tiêu chuẩn nhất định về sự duyên dáng \& vẻ đẹp bao gồm mối quan hệ nhất định giữa bản chất của chúng ta, dù yếu hay mạnh, \& điều làm chúng ta hài lòng.
		
		Whatever is formed according to this standard pleases us, be it house, song, discourse, verse, prose, woman, birds, rivers, trees, rooms, dress, etc. Whatever is not made according to this standard displeases those who have good taste.
		
		-- Bất cứ thứ gì được hình thành theo tiêu chuẩn này đều làm chúng ta hài lòng, dù đó là ngôi nhà, bài hát, bài diễn văn, câu thơ, văn xuôi, phụ nữ, chim chóc, sông ngòi, cây cối, phòng ốc, trang phục, v.v. Bất cứ thứ gì không được tạo ra theo tiêu chuẩn này đều làm mất lòng những người có gu thẩm mỹ tốt.
		
		\& as there is a perfect relation between a song \& a house which are made after a good model, because they are like this good model, though each after its kind; even so there is a perfect relation between things made after a bad model. Not that the bad model is unique, for there are many; but each bad sonnet, e.g., on whatever false model it is formed, is just like a woman dressed after that model.
		
		-- \& vì có 1 mối quan hệ hoàn hảo giữa 1 bài hát \& 1 ngôi nhà được tạo ra theo 1 mô hình tốt, bởi vì chúng giống như mô hình tốt này, mặc dù mỗi mô hình theo loại của nó; cũng vậy có 1 mối quan hệ hoàn hảo giữa những thứ được tạo ra theo 1 mô hình xấu. Không phải là mô hình xấu là duy nhất, vì có rất nhiều; nhưng mỗi bài sonnet xấu, ví dụ, trên bất kỳ mô hình sai lầm nào mà nó được hình thành, thì cũng giống như 1 người phụ nữ ăn mặc theo mô hình đó.
		
		Nothing makes us understand better the ridiculousness of a false sonnet than to consider nature \& the standard, \& then to imagine a woman or a house made according to that standard.
		
		-- Không gì giúp chúng ta hiểu rõ hơn sự lố bịch của 1 bài thơ sonnet giả tạo hơn là xem xét thiên nhiên \& chuẩn mực, \& rồi tưởng tượng ra 1 người phụ nữ hoặc 1 ngôi nhà được xây dựng theo chuẩn mực đó.
		
		\fbox{33} {\it Poetical beauty.} -- As we speak of poetical beauty, so ought we to speak of mathematical beauty \& medical beauty. But we do not do so; \& the reason is that we know well what is the object of mathematics, \& that it consists in proofs, \& what is the object of medicine, \& that it consists in healing. But we do not know in what grace consists, which is the object of poetry. We do not know the natural model which we ought to imitate; \& through lack of this knowledge, we have coined fantastic terms, ``The golden age,'' ``The wonder of our times,'' ``Fatal,'' etc., \& call this jargon poetical beauty.
		
		-- {\it Vẻ đẹp thơ ca.} -- Khi chúng ta nói về vẻ đẹp thơ ca, chúng ta cũng nên nói về vẻ đẹp toán học \& vẻ đẹp y khoa. Nhưng chúng ta không làm như vậy; \& lý do là chúng ta biết rõ mục đích của toán học là gì, \& rằng nó bao gồm các bằng chứng, \& mục đích của y học là gì, \& rằng nó bao gồm việc chữa bệnh. Nhưng chúng ta không biết ân sủng bao gồm điều gì, đó là mục đích của thơ ca. Chúng ta không biết mô hình tự nhiên mà chúng ta nên bắt chước; \& do thiếu kiến thức này, chúng ta đã đặt ra những thuật ngữ kỳ lạ, ``Thời đại hoàng kim'', ``Kỳ quan của thời đại chúng ta'', ``Chết chóc'', v.v., \& gọi thuật ngữ này là vẻ đẹp thơ ca.
		
		But whoever imagines a woman after this model, which consists in saying little things in big words, will see a pretty girl adorned with mirrors \& chains, at whom he will smile; because we know better wherein consists the charm of woman than the charm of verse. But those who are ignorant would admire her in this dress, \& there are many villages in which she would be taken for the queen; hence we call sonnets made after this model ``Village Queens.''
		
		-- Nhưng bất kỳ ai tưởng tượng ra một người phụ nữ theo mô hình này, bao gồm việc nói những điều nhỏ nhặt bằng những từ ngữ lớn lao, sẽ thấy một cô gái xinh đẹp được trang điểm bằng gương \& dây xích, người mà anh ta sẽ mỉm cười; bởi vì chúng ta biết rõ hơn về sự quyến rũ của phụ nữ hơn là sự quyến rũ của thơ ca. Nhưng những người thiếu hiểu biết sẽ ngưỡng mộ cô ấy trong bộ váy này, \& có nhiều ngôi làng mà cô ấy sẽ được coi là nữ hoàng; do đó chúng ta gọi những bài thơ sonnet được tạo ra theo mô hình này là ``Nữ hoàng làng.''		
		 
		\item {\sf The Misery of Man without God.}
		\item {\sf Of the Necessity of the Wager.}
		\item {\sf Of the Means of Belief.}
		\item {\sf Justice \& the Reason of Effects.}
		\item {\sf The Philosophers.}
		\item {\sf Morality \& Doctrine.}
		\item {\sf The Fundamentals of the Christian Religion.}
		\item {\sf Perpetuity.}
		\item {\sf Typology.}
		\item {\sf The Prophecies.}
		\item {\sf Proofs of Jesus Christ.}
		\item {\sf The Miracles.}
		\item {\sf Appendix: Polemical Fragments.}
		\item {\sf Notes.}
	\end{itemize}
	
	
	\item {\sc Blaise Pascal}. {\it Pens\'ees \& Other Writings}.
	
	{\sf Oxford World's Classics.} {\sc Blaise Pascal} (1623--62), sickly as a youth \& in poor health throughout his life, was precociously gifted as a mathematician \& scientist, but is now chiefly remembered for his commitment to the religious group associated with the monastery of Port-Royal, generally if inappropriately referred to as `Jansenist'. He never wrote a book, but did compose a brilliant series of satirical \& polemical flysheets, subsequently known as the {\it Lettres provinciales}, against the relaxation of strict \& immutable moral norms, \& left a series of abandoned texts, of which the most important were post-humously published after heavy editing by his family \& admirers for purposes of religious edification. They have ever since been known after their modish 17th-century title as the {\it Pens\'ees}. Some of the texts were notes produced in the context of preparing a religious apologetic.
	
	{\sc Pascal} mixed in the upper bourgeois society of his time, was on friendly terms with important scientists, administrators, legal officials, \& some of the higher nobility. In the expression of his religious views, {\sc Pascal} brought French prose writing to new peaks of lyrical, satirical, \& polemical brilliance. Although he failed to resolve the religious problems which confronted him, he was the most brilliant religious philosopher of his century \& the author of the finest satirical work written in 17th-century France.
	
	{\sc Anthony Levi} took premature retirement as Buchanan Professor of French Language \& Literature in the University of Saint Andrews in order to undertake full-time research. His books include a 2-volume {\it Guide to French Literature}, 1992 \& 1994, \& an annotated edition of {\sc Erasmus}'s {\it Praise of Folly}.
	
	{\sc Honor Levi} read French at Warwick, discovered \& published for her Edinburgh Ph.D. the legal inventory of Richelieu's possessions at death, \& reconstituted the original layout of what is now the Palais-Royal in Paris. She taught French, published papers given in France at historical symposia, wrote on French art history \& literature, \& published a series fo encyclopedia articles on historic French towns.
	\begin{itemize}
		\item {\sf Introduction.}
		\item {\sf Note on the Text.}
		\item {\sf Select Bibliography.}
		\item {\sf A Chronology of {\sc Blaise Pascal}.}
		\item {\sf Pens\'ees.}
		\item {\sf Discussion with Monsieur de Sacy.}
		\item {\sf The Art of Persuasion.}
		\item {\sf Writings on Grace.}
		\begin{itemize}
			\item {\sf Letter on the Possibility of the Commandments.}
			\item {\sf Treatise concerning Predestination.}
		\end{itemize}
		\item {\sf Explanatory Notes.}
	\end{itemize}
\end{enumerate}

%------------------------------------------------------------------------------%

\section{Jordan Peterson. 12 Rules for Life}

\begin{quotation}
	\textit{``The most influential public intellectual in the Western world right now.''} -- New York Times
\end{quotation}

\section*{Introduction}
``\textit{12 Rules for Life: An Antidote\footnote{\textbf{antidote} [n] \textbf{1.} \textbf{antidote (to something)} a substance that controls the effects of a poison or disease; \textbf{2.} \textbf{antidote (to something)} anything that takes away the effects of something unpleasant.} to Chaos\footnote{\textbf{chaos} [n] [uncountable] a state of complete confusion \& lack of order; in physics, \textbf{chaos} is the property of a complex system whose behavior is so unpredictable that it appears random, especially because small changes in conditions can have very large effects; \textbf{chaos theory} is the branch of mathematics that deals with these complex systems.}} is a 2018 \href{https://en.wikipedia.org/wiki/Self-help_book}{self-help book} by the Canadian clinical\footnote{\textbf{clinical} [a] [only before noun] connected with the examination \& treatment of patients \& their illnesses.} psychologist\footnote{\textbf{psychologist} [n] a scientist who studies psychology.} \href{https://en.wikipedia.org/wiki/Jordan_Peterson}{Jordan Peterson}. It provides life advice through essays in abstract ethical\footnote{\textbf{ethical} [a] \textbf{1.} connected with beliefs \& principles about what is right \& wrong; \textbf{2.} morally correct or acceptable.} principles, psychology, mythology\footnote{\textbf{mythology} [n] [uncountable, countable] \textbf{1.} ancient myths in general; the ancient myths of a particular culture, society, etc.; \textbf{2.} \textbf{mythology (of something)} ideas that many people think are true but are in fact false.}, religion\footnote{\textbf{religion} [n] \textbf{1.} [uncountable] the belief in the existence of a god or gods, \& the activities that are connected with the worship of them; \textbf{2.} [countable] 1 of the systems of belief that are based on the belief in the existence of a particular god or gods.}, \& personal anecdotes\footnote{\textbf{anecdote} [n] [countable, uncountable] \textbf{1.} \textbf{anecdote (about somebody{\tt/}something)} a short, interesting or funny story about a real person or event; \textbf{2.} a personal account of an event, especially one that is considered as possibly not true or accurate.}.''[$\ldots$] ``The book is written in a more accessible style than his previous academic book, \href{https://en.wikipedia.org/wiki/Maps_of_Meaning:_The_Architecture_of_Belief}{Maps of Meaning: The Artchitecture of Belief} (1999). A sequel, \href{https://en.wikipedia.org/wiki/Beyond_Order}{Beyond Order: 12 More Rules for Life}, was published in Mar 2021.'' -- \href{https://en.wikipedia.org/wiki/12_Rules_for_Life}{Wikipedia{\tt/}12 Rules for Life}

\subsection*{Overview}

\paragraph*{Background.} ``Peterson's interest in writing the book grew out of a personal hobby of answering questions posted on \href{https://en.wikipedia.org/wiki/Quora}{Quora}; 1 such question being
\begin{question}
	\fbox{``What are the most valuable things everyone should know?'',}
\end{question}
to which his answer comprised 42 rules. The early vision \& promotion of the book aimed to include all rules, with the title ``42''. Peterson stated that it ``isn't only written for other people. It's warning to me.'''' -- \href{https://en.wikipedia.org/wiki/12_Rules_for_Life#Background}{Wikipedia{\tt/}12 Rules for Life{\tt/}overview{\tt/}background}

\paragraph*{12 Rules.} ``The book is divided into chapters with each title representing 1 of the following 12 specific rules for life as explained through an essay.
\begin{enumerate}
	\item ``Stand up straight with your shoulders back.''
	\item ``Treat yourself like you are someone you are responsible for helping.''
	\item ``Make friends with people who want the best for you.''
	\item ``Compare yourself to who you were yesterday, not to who someone else is today.''
	\item ``Do not let your children do anything that makes you dislike them.''
	\item ``Set your house in perfect order before you criticize the world.''
	\item ``Pursue what is meaningful (not what is expedient\footnote{\textbf{expedient} [n] an action that is useful or necessary for a particular purpose, but not always fair or right.}).''
	\item ``Tell the truth -- or, at least, don't lie.''
	\item ``Assume that the person you are listening to might know something you don't.''
	\item ``Be precise in your speech.''
	\item ``Do not bother children when they are skate-boarding.''
	\item ``Pet a cat when you encounter\footnote{\textbf{encounter} [v] \textbf{1.} \textbf{encounter something} to experience something, especially something unpleasant or difficult, while you are trying to do something else, \textsc{synonym}: \textbf{run into something}; \textbf{2.} \textbf{encounter something{\tt/}somebody} to discover or experience something, or meet somebody, especially something{\tt/}somebody new, unusual or unexpected, \textsc{synonym}: \textbf{come across somebody{\tt/}something}; [n] a meeting, especially one that is sudden or unexpected.} one on the street.'''' -- \href{https://en.wikipedia.org/wiki/12_Rules_for_Life#12_Rules}{Wikipedia{\tt/}12 Rules for Life{\tt/}overview{\tt/}content}
\end{enumerate} 

\paragraph*{Content.} ``The book's central idea is that ``\fbox{suffering is built into the structure of \href{https://en.wikipedia.org/wiki/Being}{being}}'' \& although it can be unbearable\footnote{\textbf{unbearable} [a] too painful, annoying or unpleasant to deal with or accept, \textsc{synonym}: \textbf{intolerable}, \textsc{opposite}: \textbf{bearable}.}, people have a choice either to withdraw\footnote{\textbf{withdraw} [v] \textbf{1.} [transitive, intransitive] (used especially about armed forces) to make people leave a place; to leave a place; \textbf{2.} [intransitive] \textbf{withdraw (to something)} to leave a room; to go away from other people; \textbf{3.} [transitive] to move something back, out or away from something; \textbf{4.} [transitive] to take money out of a bank account or financial institution; \textbf{5.} [intransitive] to stop taking part in something; \textbf{6.} [intransitive] to stop wanting to speak to, or be with, other people; \textbf{7.} [transitive] to no longer provide or offer something; to no longer make something available; \textbf{8.} [transitive] \textbf{withdraw something} to say that you no longer agree with what you said before.}, which is a ``suicidal\footnote{\textbf{suicidal} [a] (of people) very unhappy or depressed \& feeling that they want to kill themselves; (of behavior) showing this.} gesture\footnote{\textbf{gesture} [n] \textbf{1.} [countable, uncountable] \textbf{gesture (of something)} something that you do or say to show a particular feeling or intention; \textbf{2.} [countable, uncountable] a movement that you make with your hands, your head or your face to show a particular meaning.}'', or to face \& transcend\footnote{\textbf{transcend} [v] \textbf{transcend something} to be or go beyond the usual limits of something.} it. Living in a world of chaos \& order,\fbox{ everyone has ``darkness''} that can \fbox{``turn them into the monsters they're capable of being''} to satisfy their \fbox{dark impulses\footnote{\textbf{impulse} [n] \textbf{1.} [countable, usually singular, uncountable] a sudden strong wish or need to do something, without stopping to think about the results; \textbf{2.} [countable, usually singular] something that causes somebody{\tt/}something to do something or to develop \& make progress; \textbf{3.} [countable] a brief electric current, e.g. one that travels from a nerve to a muscle; \textbf{4.} [countable] (\textit{physics}) the change in momentum of an object due to a force.} in the right situations}. Scientific experiments like the \href{https://en.wikipedia.org/wiki/Inattentional_blindness#Invisible_Gorilla_Test}{Invisible Gorilla Test} show that perception\footnote{\textbf{perception} [n] \textbf{1.} [uncountable, countable] an idea, a belief or an image you have as a result of how you see or understand something; \textbf{2.} [uncountable] the way you notice things or the ability to notice things with the senses; in biology, \textbf{perception} refers to the processes in the nervous system by which a living thing becomes aware of events \& things outside itself; \textbf{3.} [uncountable] the ability to understand the true nature of something, \textsc{synonym}: \textbf{insight}.} is adjusted to aims, \& it is \fbox{better to seek \href{https://en.wikipedia.org/wiki/Meaning_(psychology)}{meaning} rather than happiness}. Peterson notes:
\begin{quotation}
	``It's all very well to think the meaning of life is happiness, but what happens when you're unhappy? Happiness is a great side effect. When it comes, accept it gratefully\footnote{\textbf{grateful} [a] \textbf{1.} feeling or showing thanks because somebody has done something kind for you or has done as you asked; \textbf{2.} used to make a request, especially in a letter or in a formal situation.}. But it's fleeting\footnote{\textbf{fleeting} [a] [usually before noun] lasting only a short time, \textsc{synonym}: \textbf{brief}.} \& unpredictable\footnote{\textbf{unpredictable} [a] that cannot be predicted because it changes a lot or depends on too many different things, \textsc{opposite}: \textbf{predictable}.}. It's not something to aim at -- because it's not an aim. \& if happiness is the purpose of life, what happens when you're unhappy? Then you're a failure.''
\end{quotation}
The book advances the idea that \fbox{people are born with an instinct\footnote{\textbf{instinct} [n] [uncountable, countable] a natural tendency for people \& animals to behave in a particular way, using the knowledge \& abilities that they were born with rather than thought or training.} for ethics \& meaning}, \& should take responsibility\footnote{\textbf{responsibility} [n] \textbf{1.} [uncountable, countable] a duty to deal with or take care of somebody{\tt/}something, so that you may be blamed if something goes wrong; \textbf{2.} [uncountable] \textbf{responsibility (for something)} blame for something bad that has happened; \textbf{3.} [countable, uncountable] a moral duty to behave well with regard to somebody{\tt/}something.} to search for meaning above their own interests (Rule 7, ``Pursue what is meaningful, not what is expedient''). Such thinking is reflected both in contemporary\footnote{\textbf{contemporary} [a] \textbf{1.} belonging to the present time, \textsc{synonym}: \textbf{modern}; \textbf{2.} (especially of people \& society) belonging to the same time as somebody{\tt/}something else; [n] a person or thing living or existing at the same time as somebody{\tt/}something else, especially somebody who is about the same age as somebody else.} stories e.g. \href{https://en.wikipedia.org/wiki/Pinocchio_(1940_film)}{Pinocchio}, \href{https://en.wikipedia.org/wiki/The_Lion_King}{The Lion King}, \& \href{https://en.wikipedia.org/wiki/Harry_Potter}{Harry Potter}, \& in ancient stories from the \href{https://en.wikipedia.org/wiki/Bible}{Bible}. To ``stand up straight with your shoulders back'' (Rule 1) is to ``accept the terrible responsibility of life'', to make self-sacrifice\footnote{\textbf{self-sacrifice} [n] [uncountable] (\textit{approving}) the act of not allowing yourself to have or do something in order to help other people.}, because the individual must rise above \href{https://en.wikipedia.org/wiki/Victimisation}{victimization}\footnote{\textbf{victimize} [v] [often passive] \textbf{victimize somebody} to make somebody suffer unfairly because you do not like them, their opinions or something that they have done.} \& ``conduct his or her life in a manner that requires the rejection\footnote{\textbf{rejection} [n] [uncountable, countable] \textbf{1.} the act of refusing to accept or consider something; \textbf{2.} the act of refusing to accept somebody for a job or position; \textbf{3.} the decision not to use, sell, publish, etc. something because its quality is not good enough; \textbf{4.} \textbf{rejection (of something)} an occasion when somebody's body does not accept a new organ after a transplant operation, by producing substances that attack the organ; \textbf{5.} the act of failing to give a person or an animal enough care or affection.} of immediate gratification\footnote{\textbf{gratification} [n] [uncountable, countable] (\textit{formal}) the state of feeling pleasure when something goes well for you or when your desires are satisfied; something that gives you pleasure, \textsc{synonym}: \textbf{satisfaction}.}, of natural \& perverse\footnote{\textbf{perverse} [a] showing a deliberate \& determined desire to behave in a way that most people think is wrong, unacceptable or unreasonable.} desires alike.'' The comparison to \href{https://en.wikipedia.org/wiki/Neurology}{neurological}\footnote{\textbf{neurological} [a] relating to nerves or to the science of neurology.} structures \& behavior of \href{https://en.wikipedia.org/wiki/Lobsters}{lobsters} is used as a natural example to the formation\footnote{\textbf{formation} [n] \textbf{1.} [uncountable] the action of forming something; the process of being formed; \textbf{2.} [countable] a thing that has been formed, especially in a particular place or in a particular way; \textbf{3.} [countable, uncountable] a particular arrangement or pattern of people or things.} of \href{https://en.wikipedia.org/wiki/Hierarchy}{social hierarchies}\footnote{\textbf{hierarchy} [n] \textbf{1.} [countable, uncountable] a system, especially in a society or an organization, in which people are organized into different levels of importance from highest to lowest; \textbf{2.} [countable] a system that ideas or beliefs can be arranged into.}.

The other parts of the work explore \& criticize the state of young men; the upbringing\footnote{\textbf{upbringing} [n] [singular, uncountable] the way in which a child is cared for \& taught how to behave while it is growing up.} that ignores \href{https://en.wikipedia.org/wiki/Sex_differences_in_humans}{sex differences} between boys \& girls (criticism of \href{https://en.wikipedia.org/wiki/Overprotective}{over-protection} \& \href{https://en.wikipedia.org/wiki/Tabula_rasa}{tabula rasa} model in \href{https://en.wikipedia.org/wiki/Social_science}{social sciences}); male-female \href{https://en.wikipedia.org/wiki/Interpersonal_relationship}{interpersonal relationships}; \href{https://en.wikipedia.org/wiki/School_shooting}{school shootings}; religion \& moral \href{https://en.wikipedia.org/wiki/Nihilism}{nihilism}\footnote{\textbf{nihilism} [n] [uncountable] (\textit{philosophy}) the belief that life has no meaning or purpose \& that religious \& moral principles have no value.}; \href{https://en.wikipedia.org/wiki/Relativism}{relativism}\footnote{\textbf{relativism} [n] [uncountable] the belief that truth is not always \& generally valid, but can be judged only in relation to other things, e.g. your personal situation.}; \& lack of respect for the values that built \href{https://en.wikipedia.org/wiki/Western_world}{Western society}.

In the last chapter, Peterson outlines the ways in which one can cope with the most tragic\footnote{\textbf{tragic} [a] \textbf{1.} making you feel very sad, usually because somebody has died or suffered a lot; \textbf{2.} [usually before noun] connected with tragedy ($=$ the style of literature).} events, events that are often \fbox{out of one's control}. In it, he describes his own personal struggle upon discovering that his daughter, Mikhaila, had a rare bone disease. The chapter is a meditation\footnote{\textbf{meditation} [n] \textbf{1.} [uncountable] the practice of thinking deeply, usually in silence, especially for religious reasons or in order to make your mind calm; \textbf{2.} [countable, usually plural] \textbf{meditation (on something)} serious thoughts on a particular subject that somebody writes down or speaks.} on how to maintain\footnote{\textbf{maintain} [v] \textbf{1.} \textbf{maintain something} to cause or enable a condition or situation to continue, \textsc{synonym}: \textbf{preserve}; \textbf{2.} \textbf{maintain something} to keep something at the same level or rate; \textbf{3.} to state strongly that something is true, even when some other people may not believe it; \textbf{4.} \textbf{maintain somebody{\tt/}something} to support somebody{\tt/}something over a long period of time by providing money, paying for food, etc.; \textbf{5.} \textbf{maintain something} to keep a building, machine, etc. in good condition by checking or repairing it regularly; \textbf{6.} \textbf{maintain a record} to write something down as a record \& keep adding the most recent information, \textsc{synonym}: \textbf{keep}.} a watchful\footnote{\textbf{watchful} [a] paying attention to what is happening in case of danger, accidents, etc.} eye on, \& cherish\footnote{\textbf{cherish} [v] (\textit{formal}) \textbf{1.} \textbf{cherish somebody{\tt/}something} to love somebody{\tt/}something very much \& want to protect them or it; \textbf{2.} \textbf{cherish something} to keep an idea, a hope or a pleasant feeling in your mind for a long time.}, life's small redeemable\footnote{\textbf{redeemable} [a] \textbf{redeemable (against something)} that can be exchanged for money or goods.} qualities (i.e., ``pet a cat when you encounter one''). It also outlines a practical way to deal with hardship\footnote{\textbf{hardship} [n] [uncountable, countable] a situation that is difficult \& unpleasant because you do not have enough money, food, clothes, etc.}: to shorten one's temporal\footnote{\textbf{temporal} [a] \textbf{1.} connected with or limited by time; \textbf{2.} connected with the real physical world, not spiritual matters; \textbf{3.} (\textit{anatomy}) near the temples at the side of the head.} scope of responsibility (e.g., focusing on the next minute rather than the next 3 months).

Canadian psychiatrist \& psychoanalyst \href{https://en.wikipedia.org/wiki/Norman_Doidge}{Norman Doidge} wrote \cite{Peterson2018}'s foreword.'' -- \href{https://en.wikipedia.org/wiki/12_Rules_for_Life#Content}{Wikipedia{\tt/}12 Rules for Life{\tt/}overview{\tt/}content}

\begin{quotation}
	``The most influential public intellectual\footnote{\textbf{intellectual} [a] [usually before noun] connected with or using a person's ability to think in a logical way \& understand things, \textsc{synonym}: \textbf{mental}; [n] a person who is well educated \& enjoys activities in which they have to think seriously about things.} in the Western world right now.'' -- New York Times
\end{quotation}

%------------------------------------------------------------------------------%

\subsection{Foreword by Noman Doidge}
``Rules? More rules? Really? Isn't life complicated enough, restricting enough, without abstract rules that don't take our unique, individual situations into account? \& given that our brains are plastic, \& all develop differently based on our life experiences, why even expect that a few rules might be helpful to us all?

People don't clamor for rules, even in the Bible $\ldots$ as when Moses comes down the mountain, after a long absence, bearing the tablets inscribed with 10 commandments, \& finds the Children of Israel in revelry. They'd been Pharaoh's slaves \& subject to his tyrannical regulations for 400 years, \& after that Moses subjected them to the harsh desert wilderness for another 40 years, to purify them of their slavishness. Now, free at last, they are unbridled, \& have lost all control as they dance wildly around an idol, a golden calf, displaying all manner of corporeal corruption.

``I've got some good news $\ldots$ \& I've got some bad news,'' the lawgiver yells to them. ``Which do you want 1st?''

``The good news!'' the hedonists reply.

``I got Him from 15 commandments down to 10!''

``Hallelujah!'' cries the unruly crowd. ``\& the bad?''

``Adultery is still in.''

So rules there will be -- but, please, not too many. We are ambivalent about rules, even when we know they are good for us. If we are spirited souls, if we have character, rules seem restrictive, an affront to our sense of agency \& our pride in working out our own lives. Why should we be judged according to another's rule?

\& judged we are. After all, God didn't give Moses ``The 10 Suggestions,'' he gave Commandments; \& if I'm a free agent, my 1st reaction to a command might just be that nobody, not even God, tells me what to do, even if it's good for me. But the story of the golden calf also reminds us that without rules we quickly become slaves to our passions -- \& there's nothing freeing about that.

\& the story suggests something more: unchaperoned, \& left to our own untutored judgment, we are quick to aim low \& worship qualities that are beneath us -- in this case, an artificial animal that brings out our own animal instincts in a completely unregulated way. The old Hebrew story makes it clear how the ancients felt about our prospects for civilized behavior in the absence of rules that seek to elevate our gaze \& raise our standards.

1 neat thing about the Bible story is that it doesn't simply list its rules, as lawyers or legislators or administrators might; it embeds them in a dramatic tale that illustrates why we need them, thereby making them easier to understand. Similarly, in this book Prof. Peterson doesn't just propose his 12 rules, he tells stories, too, bringing to bear his knowledge of many fields as he illustrates \& explains why the best rules do not ultimately restrict us but instead facilitate our goals \& make for fuller, freer lives. p. 6

'' -- \cite[pp. 5--]{Peterson2018}



%------------------------------------------------------------------------------%

\subsection{Overture}

%------------------------------------------------------------------------------%

\subsection{Rule 1: Stand Up Straight with Your Shoulders Back}

%------------------------------------------------------------------------------%

\subsection{Rule 2: Treat Yourself Like Someone You Are Responsible for Helping}

%------------------------------------------------------------------------------%

\subsection{Rule 3: Make Friends with People Who Want The Best for You}

%------------------------------------------------------------------------------%

\subsection{Rule 4: Compare Yourself to Who You Were Yesterday, Not to Who Someone Else Is Today}

%------------------------------------------------------------------------------%

\subsection{Rule 5: Do Not Let Your Children Do Anything That Makes You Dislike Them}

%------------------------------------------------------------------------------%

\subsection{Rule 6: Set Your House In Perfect Order Before You Criticize The World}

%------------------------------------------------------------------------------%

\subsection{Rule 7: Pursue What Is Meaningful (Not What Is Expedient)}

\subsubsection{Get While The Getting's Good}
``Life is suffering. That's clear. There is no more basic, irrefutable truth. It's basically what God tells Adam \& Eve, immediately before he kicks them out of Paradise.
\begin{quotation}
	Unto the woman he said, I will greatly multiply thy sorrow \& thy conception; in sorrow thou shalt bring 4th children; \& thy desire shall be to thy husband, \& he shall rule over thee.
	
	\& unto Adam he said, Because thou hast hearkened unto the voice of thy wife, \& hast eaten of the tree, of which I commanded thee, saying, Thou shalt not eat of it: cursed is the ground for thy sake; in sorrow shalt thou eat of it all the days of thy life;
	
	Thorns also \& thistles shall it bring 4th to thee; \& thou shalt eat the herb of the field;
	
	By the sweat of your brow you will eat your food until you return to the ground, since from it you were taken; for dust you are \& to dust you will return.'' (Genesis 3:16-19. KJV)
\end{quotation}
What in the world should be done about that?

The simplest, most obvious, \& most direct answer? Pursue pleasure. Follow your impulses. Live for the moment. Do what's expedient. Lie, cheat, steal, deceive, manipulate -- but don't get caught. In an ultimately meaningless universe, what possible difference could it make? \& this is by no means a new idea. The fact of life's tragedy \& the suffering that is part of it has been used to justify the pursuit of immediate selfish gratification for a very long time.
\begin{quotation}
	Short \& sorrowful is our life, \& there is no remedy when a man comes to his end, \& no one has been known to return from Hades.
	
	Because we were born by mere chance, \& hereafter we shall be as though we had never been; because the breath in our nostrils is smoke, \& reason is a spark kindled by the beating of our hearts.
	
	When it is extinguished, the body will turn to ashes, \& the spirit will dissolve like empty air. Our name will be forgotten in time \& no one will remember our works; our life will pass away like the traces of a cloud, \& be scattered like mist that is chased by the rays of the sun \& overcome by its heat.
	
	For our allotted time is the passing of a shadow, \& there is no return from our death, because it is sealed up \& no one turns back.
	
	Come, therefore, let us enjoy the good things that exist, \& make use of the creation to the full as in youth.
	
	Let us take our fill of costly wine \& perfumes, \& let no flower of spring pass by us.
	
	Let us crown ourselves with rosebuds before they wither.
	
	Let none of us fail to share in our revelry, everywhere let us leave signs of enjoyment, because this is our portion, \& this our lot.
	
	Let us oppress the righteous poor man; let us not spare the widow nor regard the gray hairs of the aged.
	
	But let our might be our law of right, for what is weak proves itself to be useless.
	
	(Wisdom 2:1-11, RSV).
\end{quotation}
The pleasure of expediency may be fleeing, but it's pleasure, nonetheless, \& that's something to stack up against the terror \& pain of existence. Every man for himself, \& the devil take the hindmost, as the old proverb has it. Why not simply take everything you can get, whenever the opportunity arises? Why not determine to live in that manner?

Or is there an alternative, more powerful \& more compelling?

Our ancestors worked out very sophisticated answers to such questions, but we still don't understand them very well. This is because they are in large part still implicit -- manifest primarily in ritual \& myth \&, as of yet, incompletely articulated. We act them out \& represent them in stories, but we're not yet wise enough to formulate them explicitly. We're still chimps in a troupe, or wolves in a pack. we know how to behave. We know who's who, \& why. We've learned that through experience. Our knowledge has been shaped by our interaction with others. We've established predictable routines \& patterns of behavior -- but we don't really understand them, or know where they originated. They've evolved over great expanses of time. No one was formulating them explicitly (at least not in the dimmest reaches of the past), even though we've been telling each other how to act forever. 1 day, however, not so long ago, we woke up. We were already doing, but we started \textit{noticing} what we were doing. We started using our bodies as devices to represent their own actions. We started imitating \& dramatizing. We invented ritual. We started acting out our own experiences. Then we started to tell stories. We coded our observations of our own drama in these stories. In this manner, the information that was 1st only embedded in our behavior became represented in our stories. But we didn't \& still don't understand what it all means.

The Biblical narrative of Paradise \& the Fall is 1 such story, fabricated by our collective imagination, working over the centuries. It provides a profound account of the nature of Being, \& points the way to a mode of conceptualization \& action well-matched to that nature. In the Garden of Eden, prior to the dawn of self-consciousness -- so goes the story -- human beings were sinless. Our primordial parents, Adam \& Eve, walked with God. Then, tempted by the snake, the 1st couple ate from the tree of the knowledge of good \& evil, discovered Death \& vulnerability, \& turned away from God. Mankind was exiled from Paradise, \& began its effortful mortal existence. The idea of sacrifice enters soon afterward, beginning with the account of Cain \& Abel, \& developing through the Abrahamic adventures \& the Exodus: After much contemplation, struggling humanity learns that God's favor could be gained, \& his wrath averted, through proper sacrifice -- \&, also, that bloody murder might be motivated among those unwilling or unable to succeed in this manner.'' -- \cite[pp. 183--185]{Peterson2018}

\subsubsection{The Delay of Gratification}

%------------------------------------------------------------------------------%

\subsection{Rule 8: Tell The Truth -- Or, At Least, Don't Lie}

%------------------------------------------------------------------------------%

\subsection{Rule 9: Assume That The Person You Are Listening to Might Know Something You Don't}

%------------------------------------------------------------------------------%

\subsection{Rule 10: Be Precise In Your Speech}

%------------------------------------------------------------------------------%

\subsection{Rule 11: Do Not Bother Children When They Are Skateboarding}

%------------------------------------------------------------------------------%

\subsection{Rule 12: Pet A Cat When You Encounter One On The Street}

%------------------------------------------------------------------------------%

\subsection{Coda}

%------------------------------------------------------------------------------%

\subsection{Endnotes}

%------------------------------------------------------------------------------%

\section{Jordan Peterson. Beyond Order}

%------------------------------------------------------------------------------%

\section{Meaning of Life -- Giá Trị Của Sự Sống}

%------------------------------------------------------------------------------%

\section{Meaning of Death -- Giá Trị Của Cái Chết}

%------------------------------------------------------------------------------%

\section{Miscellaneous}

%------------------------------------------------------------------------------%

\printbibliography[heading=bibintoc]
	
\end{document}