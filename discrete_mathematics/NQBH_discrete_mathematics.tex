\documentclass{article}
\usepackage[backend=biber,natbib=true,style=alphabetic,maxbibnames=50]{biblatex}
\addbibresource{/home/nqbh/reference/bib.bib}
\usepackage[utf8]{vietnam}
\usepackage{tocloft}
\renewcommand{\cftsecleader}{\cftdotfill{\cftdotsep}}
\usepackage[colorlinks=true,linkcolor=blue,urlcolor=red,citecolor=magenta]{hyperref}
\usepackage{amsmath,amssymb,amsthm,enumitem,float,graphicx,mathtools,tikz}
\usetikzlibrary{angles,calc,intersections,matrix,patterns,quotes,shadings}
\allowdisplaybreaks
\newtheorem{assumption}{Assumption}
\newtheorem{baitoan}{}
\newtheorem{cauhoi}{Câu hỏi}
\newtheorem{conjecture}{Conjecture}
\newtheorem{corollary}{Corollary}
\newtheorem{dangtoan}{Dạng toán}
\newtheorem{definition}{Definition}
\newtheorem{dinhly}{Định lý}
\newtheorem{dinhnghia}{Định nghĩa}
\newtheorem{example}{Example}
\newtheorem{ghichu}{Ghi chú}
\newtheorem{hequa}{Hệ quả}
\newtheorem{hypothesis}{Hypothesis}
\newtheorem{lemma}{Lemma}
\newtheorem{luuy}{Lưu ý}
\newtheorem{nhanxet}{Nhận xét}
\newtheorem{notation}{Notation}
\newtheorem{note}{Note}
\newtheorem{principle}{Principle}
\newtheorem{problem}{Problem}
\newtheorem{proposition}{Proposition}
\newtheorem{question}{Question}
\newtheorem{remark}{Remark}
\newtheorem{theorem}{Theorem}
\newtheorem{vidu}{Ví dụ}
\usepackage[left=1cm,right=1cm,top=5mm,bottom=5mm,footskip=4mm]{geometry}
\def\labelitemii{$\circ$}
\DeclareRobustCommand{\divby}{%
	\mathrel{\vbox{\baselineskip.65ex\lineskiplimit0pt\hbox{.}\hbox{.}\hbox{.}}}%
}
\setlist[itemize]{leftmargin=*}
\setlist[enumerate]{leftmargin=*}

\title{Discrete Mathematics -- Toán Rời Rạc}
\author{Nguyễn Quản Bá Hồng\footnote{A Scientist {\it\&} Creative Artist Wannabe. E-mail: {\tt nguyenquanbahong@gmail.com}. Bến Tre City, Việt Nam.}}
\date{\today}

\begin{document}
\maketitle
\begin{abstract}
	This text is a part of the series {\it Some Topics in Advanced STEM \& Beyond}:
	
	{\sc url}: \url{https://nqbh.github.io/advanced_STEM/}.
	
	Latest version:
	\begin{itemize}
		\item {\it }.
		
		PDF: {\sc url}: \url{.pdf}.
		
		\TeX: {\sc url}: \url{.tex}.
	\end{itemize}
\end{abstract}
\tableofcontents

%------------------------------------------------------------------------------%

\section{}

%------------------------------------------------------------------------------%

\section{Miscellaneous}

%------------------------------------------------------------------------------%

\section{Wikipedia's}

\subsection{Wikipedia{\tt/}discrete mathematics}
``{\it Discrete mathematics} is study of \href{https://en.wikipedia.org/wiki/Mathematical_structures}{mathematical structures} that can be considered ``discrete'' (in a way analogous to \href{https://en.wikipedia.org/wiki/Discrete_variable}{discrete variables}, having a \href{https://en.wikipedia.org/wiki/Bijection}{bijection} with $\mathbb{N}$) rather than ``continuous'' (analogously to \href{https://en.wikipedia.org/wiki/Continuous_function}{continuous functions}). Objects studied in discrete mathematics include integers, \href{https://en.wikipedia.org/wiki/Graph_(discrete_mathematics)}{graphs}, \& \href{https://en.wikipedia.org/wiki/Statement_(logic)}{statements} in \href{https://en.wikipedia.org/wiki/Mathematical_logic}{logic}. By contrast, discrete mathematics excludes topics in ``continuous mathematics'' e.g. real numbers, calculus or \href{https://en.wikipedia.org/wiki/Euclidean_geometry}{Euclidean geometry}. Discrete objects can often be \href{https://en.wikipedia.org/wiki/Enumeration}{enumerated} by integers; more formally, discrete mathematics has been characterized as branch of mathematics dealing with \href{https://en.wikipedia.org/wiki/Countable_set}{countable sets} (finite sets or sets with same \href{https://en.wikipedia.org/wiki/Cardinality}{cardinality} as $\mathbb{N}$). However, there is no exact definition of term ``discrete mathematics''.

Set of objects studied in discrete mathematics can be finite or infinite. Term {\it finite mathematics} is sometimes applied to parts of field of discrete mathematics that deals with finite sets, particularly those areas relevant to business.

{\sf Graphs e.g. these are among objects studied by discrete mathematics, for their interesting \href{https://en.wikipedia.org/wiki/Graph_property}{mathematical properties}, their usefulness as models of real-world problems, \& their importance in developing computer algorithms.}

Research in discrete mathematics increased in latter half of 20th century partly due to development of \href{https://en.wikipedia.org/wiki/Digital_computers}{digital computers} which operate in ``discrete'' steps \& store data in ``discrete'' bits. Concepts \& notations from discrete mathematics are useful in studying \& describing objects \& problems in branches of computer science, e.g. \href{https://en.wikipedia.org/wiki/Computer_algorithm}{computer algorithms}, \href{https://en.wikipedia.org/wiki/Programming_language}{programming languages}, \href{https://en.wikipedia.org/wiki/Cryptography}{cryptography}, \href{https://en.wikipedia.org/wiki/Automated_theorem_proving}{automated theorem proving}, \& \href{https://en.wikipedia.org/wiki/Software_development}{software development}. Conversely, computer implementations are significant in applying ideas from discrete mathematics to real-world problems.

Although main objects of study in discrete mathematics are discrete objects, \href{https://en.wikipedia.org/wiki/Analysis_(mathematics)}{analytic} methods from ``continuous'' mathematics are often employed as well.

In university curricula, discrete mathematics are discrete objects, \href{https://en.wikipedia.org/wiki/Analysis_(mathematics)}{analytic} methods from ``continuous'' mathematics are often employed as well.

In university curricula, discrete mathematics appeared in 1980s, initially as a computer science support course; its contents were somewhat haphazard at time. Curriculum has thereafter developed in conjunction with efforts by \href{https://en.wikipedia.org/wiki/Association_for_Computing_Machinery}{ACM} \& \href{https://en.wikipedia.org/wiki/Mathematical_Association_of_America}{MAA} into a course that is basically intended to develop \href{https://en.wikipedia.org/wiki/Mathematical_maturity}{mathematical maturity} in 1st-year students; therefore, it is nowadays a prerequisite for mathematics majors in some universities as well. Some high-school-level discrete mathematics textbooks have appeared as well. At this level, discrete mathematics is sometimes seen as a preparatory course, like \href{https://en.wikipedia.org/wiki/Precalculus}{precalculus} in this respect.

\href{https://en.wikipedia.org/wiki/Fulkerson_Prize}{Fulkerson Prize} is awarded for outstanding papers in discrete mathematics.

\subsubsection{Topics}

\subsubsection{Challenges}

'' -- \href{https://en.wikipedia.org/wiki/Discrete_mathematics}{Wikipedia{\tt/}discrete mathematics}

%------------------------------------------------------------------------------%

\printbibliography[heading=bibintoc]
	
\end{document}