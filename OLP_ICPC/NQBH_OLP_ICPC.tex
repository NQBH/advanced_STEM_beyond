\documentclass{article}
\usepackage[backend=biber,natbib=true,style=alphabetic,maxbibnames=50]{biblatex}
\addbibresource{/home/nqbh/reference/bib.bib}
\usepackage[utf8]{vietnam}
\usepackage{tocloft}
\renewcommand{\cftsecleader}{\cftdotfill{\cftdotsep}}
\usepackage[colorlinks=true,linkcolor=blue,urlcolor=red,citecolor=magenta]{hyperref}
\usepackage{amsmath,amssymb,amsthm,enumitem,fancyvrb,float,graphicx,mathtools,tikz}
\usetikzlibrary{angles,calc,intersections,matrix,patterns,quotes,shadings}
\allowdisplaybreaks
\newtheorem{assumption}{Assumption}
\newtheorem{baitoan}{Bài toán}
\newtheorem{cauhoi}{Câu hỏi}
\newtheorem{conjecture}{Conjecture}
\newtheorem{corollary}{Corollary}
\newtheorem{dangtoan}{Dạng toán}
\newtheorem{definition}{Definition}
\newtheorem{dinhly}{Định lý}
\newtheorem{dinhnghia}{Định nghĩa}
\newtheorem{example}{Example}
\newtheorem{ghichu}{Ghi chú}
\newtheorem{hequa}{Hệ quả}
\newtheorem{hypothesis}{Hypothesis}
\newtheorem{lemma}{Lemma}
\newtheorem{luuy}{Lưu ý}
\newtheorem{nhanxet}{Nhận xét}
\newtheorem{notation}{Notation}
\newtheorem{note}{Note}
\newtheorem{principle}{Principle}
\newtheorem{problem}{Problem}
\newtheorem{proposition}{Proposition}
\newtheorem{question}{Question}
\newtheorem{remark}{Remark}
\newtheorem{theorem}{Theorem}
\newtheorem{vidu}{Ví dụ}
\usepackage[left=1cm,right=1cm,top=5mm,bottom=5mm,footskip=4mm]{geometry}
\def\labelitemii{$\circ$}
\DeclareRobustCommand{\divby}{%
	\mathrel{\vbox{\baselineskip.65ex\lineskiplimit0pt\hbox{.}\hbox{.}\hbox{.}}}%
}
\setlist[itemize]{leftmargin=*}
\setlist[enumerate]{leftmargin=*}

\title{Olympic Tin Học Sinh Viên OLP {\it\&} ACM-ICPC}
\author{Nguyễn Quản Bá Hồng\footnote{A Scientist {\it\&} Creative Artist Wannabe. Website: \url{https://nqbh.github.io}. GitHub: \url{https://github.com/NQBH}.\\E-mail: {\sf nguyenquanbahong@gmail.com, hong.nguyenquanba@umt.edu.vn}. Bến Tre \& Hồ Chí Minh Cities, Việt Nam.}}
\date{\today}

\begin{document}
\maketitle
\begin{abstract}
	This text is a part of the series {\it Some Topics in Advanced STEM \& Beyond}:
	
	{\sc url}: \url{https://nqbh.github.io/advanced_STEM/}.
	
	Latest version:
	\begin{itemize}
		\item {\it Olympic Tin Học Sinh Viên OLP \& ICPC}.
		
		PDF: {\sc url}: \url{https://github.com/NQBH/advanced_STEM_beyond/blob/main/OLP_ICPC/NQBH_OLP_ICPC.pdf}.
		
		\TeX: {\sc url}: \url{https://github.com/NQBH/advanced_STEM_beyond/blob/main/OLP_ICPC/NQBH_OLP_ICPC.tex}.
		\item Codes:
		\begin{itemize}
			\item C: \url{https://github.com/NQBH/advanced_STEM_beyond/tree/main/OLP_ICPC/C}.
			\item C++: \url{https://github.com/NQBH/advanced_STEM_beyond/tree/main/OLP_ICPC/C++}.
			\item Python: \url{https://github.com/NQBH/advanced_STEM_beyond/tree/main/OLP_ICPC/Python}.
		\end{itemize}
	\end{itemize}
\end{abstract}
\tableofcontents

%------------------------------------------------------------------------------%

\section*{Preliminaries -- Kiến thức chuẩn bị}
\textbf{\textsf{Resources -- Tài nguyên.}}
\begin{enumerate}
	\item \cite{Wu_Wang2016}. {\sc Yonghui Wu, Jiande Wang}. {\it Data Structure Practice for Collegiate Programming Contests \& Education}.
	\item \cite{Wu_Wang2018}. {\sc Yonghui Wu, Jiande Wang}. {\it Algorithm Design Practice for Collegiate Programming Contests \& Education}.	
	\item Codeforces \url{https://codeforces.com/}.
	\item CSES Problem Sets. \url{https://cses.fi/problemset/}.
\end{enumerate}
Some critical-thinking questions:
\begin{question}[Generalization; main ideas of a solution{\tt/}proof]
	What are main ideas of a solution or a proof of a problem that can be used to generalize the original problem?
\end{question}

\begin{question}[Link\footnote{Watch, e.g., \href{https://www.imdb.com/title/tt14976292/}{IMDb{\tt/}Shi Guang Dai Li Ren $\star$ Link Click} (2021--).}]
	Can we draw some link(s) between different problems? Even they are in different categories: algebra, analysis, \& combinatorics.
\end{question}

\begin{remark}[Repeat \& mathematical induction -- Lặp \& quy nạp toán học]
	\label{rmk: repeat}
	Nếu bài toán có chứa $n\in\mathbb{N}^\star$ tổng quát hoặc chứa số tự nhiên của năm ra đề, e.g., 2025, thì đưa $2025$ về $n\in\mathbb{N}^\star$, rồi sử dụng các kỹ thuật toán học để đưa về phép lặp, hoặc sử dụng phương pháp quy nạp toán học (method mathematical induction).
\end{remark}

\subsection*{Notation -- Ký hiệu}

\begin{itemize}
	\item $\overline{m,n}\coloneqq\{m,m + 1,\ldots,n - 1, n\}$, $\forall m,n\in\mathbb{Z}$, $m\le n$. Hence the notation ``for $i\in\overline{m,n}$'' means ``for $i = m,m + 1,\ldots,n$'', i.e., chỉ số{\tt/}biến chạy $i$ chạy từ $m\in\mathbb{Z}$ đến $n\in\mathbb{Z}$. Trong trường hợp $a,b\in\mathbb{R}$, ký hiệu $\overline{a,b}\coloneqq\overline{\lceil a\rceil,\lfloor b\rfloor}$ có nghĩa như định nghĩa trước đó với $m\coloneqq\lceil a\rceil,n\coloneqq\lfloor b\rfloor\in\mathbb{Z}$; khi đó ký hiệu ``for $i\in\overline{a,b}$'' với $a,b\in\mathbb{R}$, $a\le b$ có nghĩa là ``for $i = \lceil a\rceil,\lceil a\rceil + 1,\ldots,\lfloor b\rfloor - 1,\lfloor b\rfloor$, i.e., chỉ số{\tt/}biến chạy $i$ chạy từ $\lceil a\rceil$ đến $\lfloor b\rfloor\in\mathbb{Z}$.
	\item $\lfloor x\rfloor,\{x\}$ lần lượt được gọi là {\it phần nguyên \& phần lẻ} (integer- \& fractional parts) của $x\in\mathbb{R}$, see, e.g., \href{https://en.wikipedia.org/wiki/Floor_and_ceiling_functions}{Wikipedia{\tt/}floor \& ceiling functions}, \href{https://en.wikipedia.org/wiki/Fractional_part}{Wikipedia{\tt/}fractional part}.
	\item $x_+\coloneqq\max\{x,0\}$, $x_-\coloneqq\max\{-x,0\} = -\min\{x,0\}$ lần lượt được gọi là {\it phần dương \& phần âm} (positive- \& negative parts) của $x\in\mathbb{R}$.
	\item s.t.: abbreviation of `such that'.
	\item w.l.o.g.: abbreviation of `without loss of generality'.
\end{itemize}

%------------------------------------------------------------------------------%

\section{Basic Competitive Programming -- Lập Trình Thi Đấu Cơ Bản}
\textbf{\textsf{Resources -- Tài nguyên.}}
\begin{enumerate}
	\item \cite{Laaksonen2020}. {\sc Antti Laaksonen}. {\it Guide to Competitive Programming: Learning \& Improving Algorithms Through Contests}.
	\item \cite{Thu_Phuong_Tien_Triet_NMLT}. {\sc Trần Đan Thư, Nguyễn Thanh Phương, Đinh Bá Tiến, Trần Minh Triết}. {\it Nhập Môn Lập Trình}.
	\item \cite{Thu_Phuong_Tien_Triet_Phuong_KTLT}. {\sc Trần Đan Thư, Nguyễn Thanh Phương, Đinh Bá Tiến, Trần Minh Triết, Đặng Bình Phương}. {\it Kỹ Thuật Lập Trình}.
	\item \cite{Thu_Tien_Khang_PPLTHDT}. {\sc Trần Đan Thư, Đinh Bá Tiến, Nguyễn Tấn Trần Minh Khang}. {\it Phương Pháp Lập Trình Hướng Đối Tượng}.
\end{enumerate}

\subsection{The art of handling inputs \& formatting outputs -- Nghệ thuật xử lý các dạng đầu vào \& định dạng các dạng đầu ra}
To handle various types of inputs \& format various types of outputs, see, e.g.:
\begin{itemize}
	\item \href{http://poj.org/faq.htm}{Peking University Judge Online for ACM{\tt/}ICPC (POJ){\tt/}FAQ}.
	See, e.g., \cite[Chap. 2, Subsect. 2.1.1, pp. 10--11]{Laaksonen2020, Laaksonen2024}.
\end{itemize}
To compile a C++ program in Linux, run in Terminal:
\begin{verbatim}
$ g++ -O2 -Wall program_name.cpp -o program_name
$ ./program_name
\end{verbatim}
or if you want to transfer input file into it \& print output into Terminal screen:
\begin{verbatim}
$ ./program_name < program_name.inp
\end{verbatim}
or if you want to transfer input file into it \& print output into a file:
\begin{verbatim}
$ ./program_name < program_name.inp > program_name.out
\end{verbatim}

\begin{itemize}
	\item Geeks4Geeks{\tt/std::endl} vs. \verb|\n| in C++: \url{https://www.geeksforgeeks.org/endl-vs-n-in-cpp/}. 
	\item {\tt i++} vs. {\tt++i}: \href{https://stackoverflow.com/questions/24886/is-there-a-performance-difference-between-i-and-i-in-c}{StackOverflow{\tt/}Is there a performance difference between i++ \& ++i in C?}
\end{itemize}

\subsection{Repeat{\tt/}Loop -- Lặp}

\subsection{String data -- Kiểu dữ liệu chuỗi}

%------------------------------------------------------------------------------%

\subsection{Array data -- Kiểu dữ liệu mảng}
Về mặt toán học, kiểu dữ liệu mảng là dãy số hữu hạn $(a_i)_{i=1}^n = (a_1,a_2,\ldots,a_n)$. Về mặt Tin học, kiểu dữ liệu mảng được ký hiệu bởi {\tt a[1..n]}.

\begin{baitoan}[\cite{Duc_200_BT_Python}, 141., pp. 140--141: Count digit -- Đếm chữ số]
	Cho dãy số $n$ số nguyên dương $A[1..n]$ \& 1 chữ số $k$. Đếm số lần xuất hiện chữ số $k$ trong dãy $A$ đã cho. E.g., với dãy $A[] = (11,12,13,14,15)$, thì chữ số $k = 1$ xuất hiện $6$ lần trong dãy $A$.
	\item {\sf Input.} Dòng đầu tiên của đầu vào chứa số nguyên $T\in\mathbb{N}^\star$ cho biết số bộ dữ liệu cần kiểm tra. Mỗi bộ dữ liệu gồm: (i) Dòng đầu chứa lần lượt $n,k\in\mathbb{N}$ là số phần tử trong dãy $A[]$ \& chữ số $k$. (ii) Dòng thứ 2 chứa $n$ số nguyên cách nhau 1 dấu cách, mô tả các phần tử của dãy $A$.
	\item {\sf Output.} Ứng với mỗi bộ dữ liệu, in ra 1 dòng chứa kết quả của bài toán tương ứng với bộ dữ liệu đầu vào đó.
	\item {\sf Constraint.} $1\le T\le100,1\le n\le100,0\le k\le9,1\le A[i]\le1000$, $\forall i = 1,\ldots,n$.
\end{baitoan}

\begin{itemize}
	\item Input: \url{https://github.com/NQBH/advanced_STEM_beyond/blob/main/OLP_ICPC/input/count_digit.inp}.
	\item Output: \url{https://github.com/NQBH/advanced_STEM_beyond/blob/main/OLP_ICPC/output/count_digit.out}.
	\item Python: \url{https://github.com/NQBH/advanced_STEM_beyond/blob/main/OLP_ICPC/Python/count_digit.py}.
	\item C++: ?
\end{itemize}

\begin{baitoan}[\cite{Duc_200_BT_Python}, 141., pp. 140--141: Count digit -- Đếm chữ số]
	Cho dãy số nguyên $a[1],a[2],\ldots,a[n]$. Thực hiện nhiệm vụ: Chia dãy thành 2 phần trái \& phải, trong đó phần trái gồm $\frac{n}{2}$ phần tử đầu tiên \& phần phải gồm các phần tử còn lại. Tính tổng các phần tử của mỗi phần, cuối cùng tính \& in ra tích 2 tổng tìm được.
	\item {\sf Input.} Dòng đầu tiên của đầu vào chứa $t\in\mathbb{N}^\star$ cho biết số bộ dữ liệu cần kiểm tra. Mỗi bộ dữ liệu gồm: (i) Dòng đầu chứa $n\in\mathbb{N}^\star$ cho biết số phần tử của dãy. (ii) Dòng 2 chứa $n$ số nguyên cách nhau bởi dấu cách, là các phần tử của dãy.
	\item {\sf Output.} Ứng với mỗi bộ dữ liệu, in ra 1 dong chứa kết quả của bài toán tương ứng với bộ dữ liệu đầu vào đó.
	\item {\sf Constraint.} $1\le t\le100,1\le n\le100,1\le A[i]\le100$, $\forall i = 1,\ldots,n$.
\end{baitoan}

\begin{itemize}
	\item Input: \url{https://github.com/NQBH/advanced_STEM_beyond/blob/main/OLP_ICPC/input/prod_left_right_sums.inp}.
	\item Output: \url{https://github.com/NQBH/advanced_STEM_beyond/blob/main/OLP_ICPC/output/prod_left_right_sums.out}.
	\item Python: \url{https://github.com/NQBH/advanced_STEM_beyond/blob/main/OLP_ICPC/Python/prod_left_right_sums.py}.
	\begin{verbatim}
t = int(input())
for _ in range(t):
    n = int(input())
    a = list(map(int, input().split()))
    lsum = rsum = 0
    for i in range(n//2):
        lsum += a[i]
    for i in range(n//2, n):
        rsum += a[i]
    print(lsum * rsum)
	\end{verbatim}
	\item C++: ?
\end{itemize}

\subsubsection{Kỹ thuật mảng chỉ số cho kiểu dữ liệu mảng}
{\bf A general idea.} Giả sử có dãy số $\{a_n\}_{n=1}^n$ được lưu với mảng {\tt a = a[0], a[1],...,a[n - 1]} với $a_i =$ {\tt a[i - 1]}, $\forall i\in[n]$. Giả sử có $m\in\mathbb{N}^\star$ mảng chỉ số $\{f_i\}_{i=1}^m$ mà mỗi mảng có số phần tử là 1 hàm của $n$, $f_i:[n]\to\mathbb{R}$, mà $m$ mảng chỉ số này lại liên quan hay ràng buộc với nhau theo những cách nào đấy, biểu diễn được bằng công thức toán, e.g., $f_i(n) = F_i(f_1,f_2,\ldots,f_{i-1},f_{i+1},\ldots,f_n)$. Tìm hiểu cấu trúc toán học, cấu trúc giải thuật, \& tạo ra các ví dụ để minh họa ý tưởng tổng quát này.

Kỹ thuật {\it sliding window} cũng là 1 trường hợp riêng của ý tưởng này với $m = 2$, $f_1(i) =$ left index (chỉ số trái), $f_2(i) =$ right index (chỉ số phải) \& ta thường lấy tổng $\sum_{\tt left\_index}^{\tt right\_index} a[i]$, tích $\prod_{\tt left\_index}^{\tt right\_index} a[i]$, hoặc 1 hàm nào đấy của các phần tử bị giới hạn bởi 2 chỉ số trái \& phải này, e.g., $\sum_{\tt left\_index}^{\tt right\_index} f(a[i])$ or $F(a[{\tt left\_index}],\ldots,a[{\tt right\_index}])$.

\begin{problem}[Techniques of additional arrays, R+4]
	Establish the general \& rigorous frameworks for the idea of using additional arrays to micro manage or to get insights of a given array in some higher levels.
\end{problem}

%------------------------------------------------------------------------------%

\section{Introductory Problems -- Các Bài Toán Mở Đầu}

\begin{problem}[\href{https://cses.fi/problemset/task/1068}{CSES Problem Set{\tt/}weird algorithm}]
    Consider an algorithm that takes as input a positive integer $n$. If $n$ is even, the algorithm divides it by $2$, \& if $n$ is odd, the algorithm multiplies it by $3$ \& adds $1$. The algorithm repeats this, until $n = 1$. E.g., the sequence for $n = 3$ is as follows: $3\to10\to5\to16\to8\to4\to2\to1$. Simulate the execution of the algorithm for a given value of $n$.
    \item {\sf Input.} The only input line contains an integer $n\in\mathbb{N}^\star$.
    \item {\sf Output.} rint a line that contains all values of $n$ during the algorithm.
    \item {\sf Constraints.} $n\in[10^6]$.
    \item {\sf Sample.}
    \begin{table}[H]
        \centering
        \begin{tabular}{|l|l|}
            \hline
            \verb|weird_algorithm.inp| & \verb|weird_algorithm.out| \\
            \hline
            3 & 3 10 5 16 8 4 2 1 \\
            \hline
        \end{tabular}
    \end{table}
\end{problem}

\begin{problem}[\href{https://cses.fi/problemset/task/1083}{CSES Problem Set{\tt/}missing number}]
    You are given all numbers in $[n]$ except one. Find the missing number.
    \item {\sf Input.} The 1st input line has an integer $n\in\mathbb{N}^\star$. The 2nd line contains $n  - 1$ numbers. Each number is distinct \& between $1$ \& $n$ (inclusive) .
    \item {\sf Output.} Print the missing number.
    \item {\sf Constraints.} $n\overline{2,2\cdot10^5}$.
    \item {\sf Sample.}
    \begin{table}[H]
        \centering
        \begin{tabular}{|l|l|}
            \hline
            \verb|missing_number.inp| & \verb|missing_number.out| \\
            \hline
            5 & 4 \\
            2 3 1 5 & \\
            \hline
        \end{tabular}
    \end{table}
\end{problem}

\begin{problem}[\href{https://cses.fi/problemset/task/1069}{CSES Problem Set{\tt/}repetitions}]
    You are given a DNA sequence: a string consisting of characters {\tt A, C, G, T}. Find the longest repetition in the sequence. This is a maximum-length substring containing only 1 type of character.
    \item {\sf Input.} The only input line contains a string of $n\in\mathbb{N}^\star$ characters.
    \item {\sf Output.} Print $1$ integer: the length of the longest repetition.
    \item {\sf Constraints.} $n\in[10^6]$.
    \item {\sf Sample.}
    \begin{table}[H]
        \centering
        \begin{tabular}{|l|l|}
            \hline
            \verb|repetition.inp| & \verb|repetition.out| \\
            \hline
            ATTCGGGA & 3 \\
            \hline
        \end{tabular}
    \end{table}
\end{problem}

\begin{problem}[\href{https://cses.fi/problemset/task/1094}{CSES Problem Set{\tt/}increasing array}]
    You are given an array of $n\in\mathbb{N}^\star$ integers. You want to modify the array so that it is increasing, i.e., every element is at least as large as the previous element. On each move, you may increase the value of any element by $1$. What is the minimum number of moves required?
    \item {\sf Input.} The 1st input line contains an integer $n\in\mathbb{N}^\star$: the size of the array. Then, the 2nd line contains $n$ integers $x_1,x_2,\ldots,x_n\in\mathbb{N}^\star$: the contents of the array.
    \item {\sf Output.} Print the minimum number of moves.
    \item {\sf Constraints.} $n\in[2\cdot10^5],x_i\in[10^9]$, $\forall i\in[n]$.
    \item {\sf Sample.}
    \begin{table}[H]
        \centering
        \begin{tabular}{|l|l|}
            \hline
            \verb|increasing_array.inp| & \verb|increasing_array.out| \\
            \hline
            5 & 5 \\
            3 2 5 1 7 & \\
            \hline
        \end{tabular}
    \end{table}
\end{problem}

\begin{problem}[\href{https://cses.fi/problemset/task/1070}{CSES Problem Set{\tt/}permutations}]
    A permutation of $[n]$ is called {\rm beautiful} if there are no adjacent elements whose difference is $1$. Given $n$, construct a beautiful permutation if such a permutation exists.
    \item {\sf Input.} The 1st input line contains an integers $n\in\mathbb{N}^\star$.
    \item {\sf Output.} Print a beautiful permutation of integers $1,2,\ldots,n$. If there are several solutions, you may print any of them. If there are no solutions, print {\tt NO SOLUTION}.
    \item {\sf Constraints.} $n\in[10^6]$.
    \item {\sf Sample.}
    \begin{table}[H]
        \centering
        \begin{tabular}{|l|l|}
            \hline
            \verb|permutation.inp| & \verb|permutation.out| \\
            \hline
            5 & 4 2 5 3 1 \\
            \hline
            3 & NO SOLUTION \\
            \hline
        \end{tabular}
    \end{table}
\end{problem}

\begin{problem}[\href{https://cses.fi/problemset/task/1071}{CSES Problem Set{\tt/}number spiral}]
    A {\rm number spiral} is an infinite grid whose upper-left square has number $1$. Here are the 1st $5$ layers of the spiral
    \begin{table}[H]
        \centering
        \begin{tabular}{|c|c|c|c|c|}
            \hline
            1 & 2 & 9 & 10 & 25 \\
            \hline
            4 & 3 & 8 & 11 & 24 \\
            \hline
            5 & 6 & 7 & 12 & 23 \\
            \hline
            16 & 15 & 14 & 13 & 22 \\
            \hline
            17 & 18 & 19 & 20 & 21 \\
            \hline
        \end{tabular}
    \end{table}
    Find out the number is row $y$ \& column $x$.
    \item {\sf Input.} The 1st input line contains an integer $t\in\mathbb{N}^\star$: the number of tests. After this, there are $t$ lines, each containing integers $y,z\in\mathbb{N}^\star$.
    \item {\sf Output.} For each test, print the number in row $y$ \& column $x$. 
    \item {\sf Constraints.} $t\in[10^5],x,y\in[10^9]$.
    \item {\sf Sample.}
    \begin{table}[H]
        \centering
        \begin{tabular}{|l|l|}
            \hline
            \verb|number_spiral.inp| & \verb|number_spiral.out| \\
            \hline
            3 & 8 \\
            2 3 & 1 \\
            1 1 & 15 \\
            4 2 & \\
            \hline
        \end{tabular}
    \end{table}
\end{problem}

\begin{problem}[\href{https://cses.fi/problemset/task/1072}{CSES Problem Set{\tt/}two knights}]
    Count for $k\in[n]$ the number of ways $2$ knights can be placed on a $k\times k$ chessboard so that they do not attack each other.
    \item {\sf Input.} The only input line contains an integer $n\in\mathbb{N}^\star$.
    \item {\sf Output.} Print $n$ integers: the results.
    \item {\sf Constraints.} $n\in[10^4]$.
    \item {\sf Sample.}
    \begin{table}[H]
        \centering
        \begin{tabular}{|l|l|}
            \hline
            \verb|two_knight.inp| & \verb|two_knight.out| \\
            \hline
            8 & 0 \\
            & 6 \\
            & 28 \\
            & 96 \\
            & 252 \\
            & 550 \\
            & 1056 \\
            & 1848 \\
            \hline
        \end{tabular}
    \end{table}
\end{problem}

\begin{problem}[\href{https://cses.fi/problemset/task/1092}{CSES Problem Set{\tt/}two sets}]
    Divide the numbers $[n]$ into 2 sets of equal sum.
    \item {\sf Input.} The only input line contains an integer $n\in\mathbb{N}^\star$.
    \item {\sf Output.} Print {\tt YES}, if the division is possible, \& {\tt NO} otherwise. After this, if the division is possible, print an example of how to create the sets. 1st, print the number of elements in the 1st set followed by the elements themselves in a separate line, \& then, print the 2nd set in a similar way.
    \item {\sf Constraints.} $n\in[10^6]$.
    \item {\sf Sample.}
    \begin{table}[H]
        \centering
        \begin{tabular}{|l|l|}
            \hline
            \verb|two_set.inp| & \verb|two_set.out| \\
            \hline
            7 & YES \\
            & 4 \\
            & 1 2 4 7\\
            & 3 \\
            & 3 5 6\\
            \hline
            6 & NO
        \end{tabular}
    \end{table}
\end{problem}

\begin{problem}[\href{https://cses.fi/problemset/task/1617}{CSES Problem Set{\tt/}bit strings}]
    Calculate the number of bit strings of length $n\in\mathbb{N}^\star$, e.g., if $n = 3$, the correct answer is $8$, because the possible bit strings are $000,001,010,011,100,101,110,111$.
    \item {\sf Input.} The only input line has an integer $n\in\mathbb{N}^\star$.
    \item {\sf Output.} Print the result modulo $10^9 + 7$.
    \item {\sf Constraints.} $n\in[10^6]$.
    \item {\sf Sample.}
    \begin{table}[H]
        \centering
        \begin{tabular}{|l|l|}
            \hline
            \verb|bit_string.inp| & \verb|bit_string.out| \\
            \hline
            3 & 8 \\
            \hline
        \end{tabular}
    \end{table}
\end{problem}

\begin{proof}[Solution]
    The number of bit strings of length $n\in\mathbb{N}^\star$ is $2^n$.
\end{proof}

\begin{problem}[\href{https://cses.fi/problemset/task/1618}{CSES Problem Set{\tt/}trailing zeros}]
    Calculate the number of trailing zeros in the factorial $n!$, e.g., $20! = 2432902008176640000$ \& it has $4$ trailing zeros.
    \item {\sf Input.} The only input line has an integer $n\in\mathbb{N}^\star$.
    \item {\sf Output.} Print the number of trailing zeros is $n!$.
    \item {\sf Constraints.} $n\in[10^6]$.
    \item {\sf Sample.}
    \begin{table}[H]
        \centering
        \begin{tabular}{|l|l|}
            \hline
            \verb|trailing_zero.inp| & \verb|trailing_zero.out| \\
            \hline
            20 & 4 \\
            \hline
        \end{tabular}
    \end{table}
\end{problem}

\begin{problem}[CSES Problem Set{\tt/}coin piles]
    You have $2$ coin piles containing $a,b\in\mathbb{N}$ coins. On each move, you can either remove $1$ coin from the left pile \& $2$ coins from the right pile, or $2$ coins from the left pile \& $1$ coin from the right pile. Efficiently find out if you can empty both the piles.
    \item {\sf Input.} The 1st input line has an integer $t$ integers $n,m\in\mathbb{N}^\star$: the number of tests. After this, there are $t$ lines, each of which has $2$ integers $a,b\in\mathbb{N}$: the numbers of coins in the piles.
    \item {\sf Output.} For each test, print {\tt YES} if you can empty the piles \& {\tt NO} otherwise.
    \item {\sf Constraints.} $t\in[10^5],a,b\in\overline{0,10^9}$.
    \item {\sf Sample.}
    \begin{table}[H]
        \centering
        \begin{tabular}{|l|l|}
            \hline
            \verb|coin_pile.inp| & \verb|coin_pile.out| \\
            \hline
            3 & YES \\
            2 1 & NO \\
            2 2 & YES \\
            3 3 & \\
            \hline
        \end{tabular}
    \end{table}
\end{problem}

\begin{problem}[\ref{https://cses.fi/problemset/task/1755}{CSES Problem Set{\tt/}palindrome reorder}]
    Given a string, recorder its letters in such a way that it becomes a palindrome (i.e., it reads the same forwards \& backwards).
    \item {\sf Input.} The only input line has a string of length $n\in\mathbb{N}^\star$ consisting of characters {\tt A--Z}.
    \item {\sf Output.} Print a palindrome consisting of the characters of the original string. You may print any valid solution. If there are no solutions, print {\tt NO SOLUTION}.
    \item {\sf Constraints.} $n\in[10^6]$.
    \item {\sf Sample.}
    \begin{table}[H]
        \centering
        \begin{tabular}{|l|l|}
            \hline
            \verb|palindrome_reorder.inp| & \verb|palindrome_reorder.out| \\
            \hline
            AAAACACBA & AACABACAA \\
            \hline
        \end{tabular}
    \end{table}
\end{problem}

\begin{problem}[\href{https://cses.fi/problemset/task/2205}{CSES Problem Set{\tt/}gray code}]
    A {\rm Gray code} (not Gay code) is a list of all $2^n$ bit strings of length $n\in\mathbb{N}^\star$, where any $2$ successive strings differ in exactly $1$ bit (i.e., their Hamming distance is $1$). Create a Gray code for a given length $n$.
    \item {\sf Input.} The only input line has an integer $n\in\mathbb{N}^\star$.
    \item {\sf Output.} Print $2^n$ lines that describe the Gray code. You can print any valid solution.
    \item {\sf Constraints.} $n\in[16]$.
    \item {\sf Sample.}
    \begin{table}[H]
        \centering
        \begin{tabular}{|l|l|}
            \hline
            \verb|gray_code.inp| & \verb|gray_code.out| \\
            \hline
            2 & 00 \\
            & 01 \\
            & 11 \\
            & 10 \\
            \hline
        \end{tabular}
    \end{table}
\end{problem}

\begin{problem}[\href{https://cses.fi/problemset/task/2165}{CSES Problem Set{\tt/}tower of Hanoi}]
    The Tower of Hanoi game consists of $3$ stacks (left, middle, \& right) \& $n\in\mathbb{N}^\star$ round disks of different sizes. Initially, the left stack has all the disks, in increasing order of size from top to bottom. The goal is to move all the disks to the right stack using the middle stack. On each move you can move the uppermost disk from a stack to another stack. In addition, it is not allowed to place a larger disk on a smaller disk. Find a solution that minimizes the number of moves.
    \item {\sf Input.} The only input line has an integer $n\in\mathbb{N}^\star$: the number of of disks.
    \item {\sf Output.} 1st print an integer $k\in\mathbb{N}^\star$: the minimum number of moves. After this, print $k$ lines that describe the moves. Each line has $2$ integers $a,b\in[3]$: you move a disk from stack $a$ to stack $b$.
    \item {\sf Constraints.} $n\in[16]$.
    \item {\sf Sample.}
    \begin{table}[H]
        \centering
        \begin{tabular}{|l|l|}
            \hline
            \verb|tower_Hanoi.inp| & \verb|tower_Hanoi.out| \\
            \hline
            2 & 3 \\
            & 1 2 \\
            & 1 3 \\
            & 2 3 \\
            \hline
        \end{tabular}
    \end{table}
\end{problem}

\begin{problem}[\href{https://cses.fi/problemset/task/1622}{CSES Problem Set{\tt/}creating strings}]
    Given a string, generate all different strings that can be created using its characters.
    \item {\sf Input.} The only input line has a string of length $n\in\mathbb{N}^\star$. Each character is between {\tt a--z}.
    \item {\sf Output.} 1st print an integer $k$: the number of strings. Then print $k$ lines: the strings in alphabetical order.
    \item {\sf Constraints.} $n\in[8]$.
    \item {\sf Sample.}
    \begin{table}[H]
        \centering
        \begin{tabular}{|l|l|}
            \hline
            \verb|creating_string.inp| & \verb|creating_string.out| \\
            \hline
            aabac & 20 \\
            & aaabc \\
            & aaacb \\
            & aabac \\
            & aabca \\
            & aacab \\
            & aacba \\
            & abaac \\
            & abaca \\
            & abcaa \\
            & acaab \\
            & acaba \\
            & acbaa \\
            & baaac \\
            & baaca \\
            & bacaa \\
            & bcaaa \\
            & caaab \\
            & caaba \\
            & cabaa \\
            & cbaaa \\
            \hline
        \end{tabular}
    \end{table}
\end{problem}

\begin{problem}[\href{https://cses.fi/problemset/task/1623}{CSES Problem Set{\tt/}apple division}]
    There are $n\in\mathbb{N}^\star$ apples with known weights. Divide the apples into $2$ groups so that the difference between the weights of the groups is minimal.
    \item {\sf Input.} The 1st input line has an integer $n\in\mathbb{N}^\star$: the number of apples. The next line has $n$ integers $p_1,p_2,\ldots,p_n$: the weight of each apple.
    \item {\sf Output.} Print $1$ integer: the minimum difference between the weights of the groups.
    \item {\sf Constraints.} $n\in[20],p_i\in[10^9]$.
    \item {\sf Sample.}
    \begin{table}[H]
        \centering
        \begin{tabular}{|l|l|}
            \hline
            \verb|apple_division.inp| & \verb|apple_division.out| \\
            \hline
            5 & 1 \\
            3 2 7 4 1 & \\
            \hline
        \end{tabular}
    \end{table}
    \item {\sf Explanation.} Group 1 has weights $2,3,4$ (total weight $9$), \& group 2 has weights $1,7$ (total weight $8$).
\end{problem}

\begin{problem}[\href{https://cses.fi/problemset/task/1624}{CSES Problem Set{\tt/}chessboard \& queens}]
    Place $8$ queens on a chessboard so that no $2$ queens are attacking each other. As an additional challenge, each square is either free or reserved, \& you can only place queens on the free squares. However, the reserved squares do not prevent queens from attacking each other. Count the number of possible ways are there to place the queens.
    \item {\sf Input.} The input line has $8$ lines, \& each of them has $8$ characters. Each square is either free {\tt.} or reversed {\tt*}.
    \item {\sf Output.} Print $1$ integer: the number of ways you can place the queens.
    \item {\sf Constraints.} $n\in[10^5],m\in[100],x_i\in\{0,1,\ldots,m\}$.
    \item {\sf Sample.}
    \begin{table}[H]
        \centering
        \begin{tabular}{|l|l|}
            \hline
            \verb|chessboard_queen.inp| & \verb|chessboard_queen.out| \\
            \hline
            ........ & 65 \\
            ........ & \\
            ..*..... & \\
            ........ & \\
            ........ & \\
            .....**. & \\
            ...*.... & \\
            ........ & \\
            \hline
        \end{tabular}
    \end{table}
\end{problem}

\begin{problem}[\href{https://cses.fi/problemset/task/3399}{CSES Problem Set{\tt/}Raab game I}]
    Consider a $2$ play game where each player has $n\in\mathbb{N}^\star$ cards numbered $1,2,\ldots,n$. On each turn both players place $1$ of their cards on the table. The player who placed the higher card gets $1$ point. If the cards are equal, neither player gets a point. The game continues until all cards have been played. You are given the number of cards $n$ \& the players's scores at the end of the game, $a,b\in\mathbb{N}$. Give an example of how the game could have played out.
    \item {\sf Input.} The 1st input line contains $1$ integer $t\in\mathbb{N}^\star$: the number of tests. Then there are $t$ lines, each with $3$ integers $n,a,b\in\mathbb{N}^\star$.
    \item {\sf Output.} For each test case print {\tt YES} if there is a game with the given outcome \& {\tt NO} otherwise. If the answer is {\tt YES}, print an example of $1$ possible game. Print $2$ line representing the order in which the players place their cards. You can give any valid example.
    \item {\sf Constraints.} $t\in[10^3],n\in[100],a,b,\in\overline{0,n}$.
    \item {\sf Sample.}
    \begin{table}[H]
        \centering
        \begin{tabular}{|l|l|}
            \hline
            \verb|Raab_game I.inp| & \verb|Raab_game I.out| \\
            \hline
            5 & YES \\
            4 1 2 & 1 4 3 2 \\
            2 0 1 & 2 1 3 4 \\
            3 0 0 & NO \\
            2 1 1 & YES \\
            4 4 1 & 1 2 3 \\
            & 1 2 3 \\
            & YES \\
            & 1 2 \\
            & 2 1 \\
            & NO \\
            \hline
        \end{tabular}
    \end{table}
\end{problem}

\begin{problem}[\href{https://cses.fi/problemset/task/3419}{CSES Problem Set{\tt/}mex grid construction}]
    Construct an $n\times n$ grid where each square has the smallest nonnegative integer that does not appear to the left on the same row or above on the same column.
    \item {\sf Input.} The only input line has an integer $n\in\mathbb{N}^\star$.
    \item {\sf Output.} Print the grid according to the example.
    \item {\sf Constraints.} $n\in[100]$.
    \item {\sf Sample.}
    \begin{table}[H]
        \centering
        \begin{tabular}{|l|l|}
            \hline
            \verb|mex_grid_construction.inp| & \verb|mex_grid_construction.out| \\
            \hline
            5 & \\
            & 0 1 2 3 4 \\
            & 1 0 3 2 5 \\
            & 2 3 0 1 6 \\
            & 3 2 1 0 7 \\
            & 4 5 6 7 0 \\            
            \hline
        \end{tabular}
    \end{table}
\end{problem}

\begin{problem}[\href{https://cses.fi/problemset/task/3217}{CSES Problem Set{\tt/}knight moves grid}]
    There is a knight on a $n\times n$ chessboard. For each square, print the minimum number of moves the knight needs to do to reach the top-left corner.
    \item {\sf Input.} The only input line has an integer $n\in\mathbb{N}^\star$.
    \item {\sf Output.} Print the minimum number of moves for each square.
    \item {\sf Constraints.} $n\in\overline{4,10^3}$.
    \item {\sf Sample.}
    \begin{table}[H]
        \centering
        \begin{tabular}{|l|l|}
            \hline
            \verb|knight_move_grid.inp| & \verb|knight_move_grid.out| \\
            \hline
            8 & \\
            & 0 3 2 3 2 3 4 5 \\
            & 3 4 1 2 3 4 3 4 \\
            & 2 1 4 3 2 3 4 5 \\
            & 3 2 3 2 3 4 3 4 \\
            & 2 3 2 3 4 3 4 5 \\
            & 3 4 3 4 3 4 5 4 \\
            & 4 3 4 3 4 5 4 5 \\
            & 5 4 5 4 5 4 5 6 \\
            \hline
        \end{tabular}
    \end{table}
\end{problem}

\begin{problem}[\href{https://cses.fi/problemset/task/3311}{CSES Problem Set{\tt/}grid coloring I}]
    You are given an $n\times m$ grid where each cell contains $1$ character {\tt A, B, C} or {\tt D}. For each cell, you must change the character to {\tt A, B, C} or {\tt D}. The new character must be different from the old one. Change the characters in every cell s.t. no $2$ adjacent cells have the same character.
    \item {\sf Input.} The 1st input line has $2$ integers $n,m\in\mathbb{N}^\star$: the number of rows \& columns. The next $n$ lines each have $m$ characters: the description of the grid.
    \item {\sf Output.} Print $n$ lines each with $m$ characters: the description of the final grid. You may print any valid solution. If no solution exists, just print {\tt IMPOSSIBLE}.
    \item {\sf Constraints.} $m,n\in[500]$.
    \item {\sf Sample.}
    \begin{table}[H]
        \centering
        \begin{tabular}{|l|l|}
            \hline
            \verb|grid_coloring_I.inp| & \verb|grid_coloring_I.out| \\
            \hline
            3 4 & CDCD \\
            AAAA & DCDC \\
            BBBB & ABAB \\
            CCDD & \\
            \hline
        \end{tabular}
    \end{table}
\end{problem}

\begin{problem}[\href{https://cses.fi/problemset/task/2431}{CSES Problem Set{\tt/}digit queries}]
    Consider an infinite string that consists of all positive integers in increasing order: $12345678910111213141516171819202122232425\ldots$. Process $q\in\mathbb{N}^\star$ queries of the form: what is the digit at position $k\in\mathbb{N}^\star$ in the string?
    \item {\sf Input.} The 1st input line has an integer $q\in\mathbb{N}^\star$: the number of queries. After this, there are $q$ lines that describe the queries. Each line has an integer $k$: a $1$-indexed position in the string.
    \item {\sf Output.} For each query, print the corresponding digit.
    \item {\sf Constraints.} $q\in[10^3],k\in[10^{18}]$.
    \item {\sf Sample.}
    \begin{table}[H]
        \centering
        \begin{tabular}{|l|l|}
            \hline
            \verb|digit_query.inp| & \verb|digit_query.out| \\
            \hline
            3 & 7 \\
            7 & 4 \\
            19 & 1 \\
            12 & \\
            \hline
        \end{tabular}
    \end{table}
\end{problem}

\begin{problem}[\href{https://cses.fi/problemset/task/1743}{CSES Problem Set{\tt/}string reorder}]
    Reorder the characters of a string so that no $2$ adjacent characters are the same. What is the lexicographically minimal such string?
    \item {\sf Input.} The only input line has a string of length $n\in\mathbb{N}^\star$ consisting of characters {\tt A--Z}.
    \item {\sf Output.} Print the lexicographically minimal reordered string where no $2$ adjacent characters are the same. If it is not possible to create such a string, print {\tt-1}.
    \item {\sf Constraints.} $n\in[10^6]$.
    \item {\sf Sample.}
    \begin{table}[H]
        \centering
        \begin{tabular}{|l|l|}
            \hline
            \verb|string_reorder.inp| & \verb|string_reorder.out| \\
            \hline
            HATTIVATTI & AHATITITVT \\
            \hline
        \end{tabular}
    \end{table}
\end{problem}

\begin{problem}[\href{https://cses.fi/problemset/task/1625}{CSES Problem Set{\tt/}grid path description}]
    There are $88418$ paths in a $7\times7$ grid from the upper-left square to the lower-left square. Each path corresponds to a $48$-character description consisting of characters {\tt D, U, L, R} (down, up, left, right, resp.). You are given a description of a path which may also contains characters {\tt?} (any direction). Calculate the number of paths that match the description.
    \item {\sf Input.} The only input line has a $48$-character string of characters {\tt?, D, U, L, R}.
    \item {\sf Output.} Print $1$ integer: the total number of paths.
    \item {\sf Sample.}
    \begin{table}[H]
        \centering
        \begin{tabular}{|l|l|}
            \hline
            \verb|grid_path_description.inp| & \verb|grid_path_description.out| \\
            \hline
            ??????R??????U??????????????????????????LD????D? & 201 \\
            & \\
            \hline
        \end{tabular}
    \end{table}
\end{problem}

%------------------------------------------------------------------------------%

\section{Practice for Simple Computing -- Thực Hành Tính Toán Đơn Giản}

\begin{enumerate}
	\item \cite{Wu_Wang2016}. {\sc Yonghui Wu, Jiande Wang}. {\it Data Structure Practice for Collegiate Programming Contests \& Education}.
	\item \cite{Wu_Wang2018}. {\sc Yonghui Wu, Jiande Wang}. {\it Algorithm Design Practice for Collegiate Programming Contests \& Education}.
\end{enumerate}

\begin{problem}[\cite{Wu_Wang2016}, p. 4: financial management]
	{\sc Larry} graduated this year \& finally has a job. He's making a lot of money, but somehow never seems to have enough. {\sc Larry} has decided that he needs to get a hold of his financial portfolio \& solve his financial problems. The 1st step is to figure out what's been going on with his money. {\sc Larry} has his bank account statements \& wants to see how much money he has. Help {\sc Larry} by writing a program to take his closing balance from each of the past 12 months \& calculate his average account balance.
	\item {\sf Input.} The input will be 12 lines. Each line will contain the closing balance of his bank account for a particular month. Each number will be positive \& displayed to the penny. No dollar sign will be included.
	\item {\sf Output.} The output will be a single number, the average (mean) of the closing balances for the 12 months. It will be rounded to the nearest penny, preceded immediately by a dollar sign, \& followed by the end of the line. There will be no other spaces or characters in the output.
	\item {\sf Source.} ACM Mid-Atlantic United States 2001.
	\item {\sf IDs for online judges.} POJ 1004, ZOJ 1048, UVA 2362.
\end{problem}
\textbf{\textsf{Math Analysis.}} Let $\{a_i\}_{i=1}^{12}\subset[0,\infty)$ be monthly incomes of 12 months. Compute their average by the formula $\overline{a} = \frac{1}{12}\sum_{i=1}^{12} a_i$. This can be generalized to $n\in\mathbb{N}^\star$ months with a sequence of monthly incomes $\{a_i\}_{i=1}^n\subset[0,\infty)$ with its average value given by the formula $\overline{a}\coloneqq\frac{1}{n}\sum_{i=1}^n a_i$.

\textbf{\textsf{CS Analysis.}} The income of 12 months {\tt a[0..11]} is input by a {\tt for} statement \& the total income ${\tt sum}\coloneqq\sum_{i=0}^{11} a[i]$ is calculated. Then the average monthly income {\tt avg = sum/12} is calculated. Finally, {\tt avg} is output in accordance with the problem's output format by utilizing {\tt printf}'s format functionalities via \verb|printf("$%.2f", avg)|.
\begin{itemize}
	\item Input: \url{https://github.com/NQBH/advanced_STEM_beyond/blob/main/OLP_ICPC/input/financial_management.inp}.
	\item Output: \url{https://github.com/NQBH/advanced_STEM_beyond/blob/main/OLP_ICPC/output/financial_management.out}.
	\item C++: \url{https://github.com/NQBH/advanced_STEM_beyond/blob/main/OLP_ICPC/C++/financial_management.cpp}.
\begin{verbatim}
#include <iostream>
using namespace std;
int main() {
    double avg, sum = 0.0, a[12] = {0};
    for (int i = 0; i < 12; ++i) { // input income of 12 months a[0..11] & summation
        cin >> a[i];
        sum += a[i];
    }
    avg = sum/12; // compute average monthly
    printf("$%.2f", avg); // output average monthly
    return 0;
}		
\end{verbatim}
	
	\begin{remark}[array of 0s]
		The technique \verb|a[12] = {0}| initializes an array $a$ with all zero elements.
	\end{remark}
	\item Python: \url{https://github.com/NQBH/advanced_STEM_beyond/blob/main/OLP_ICPC/Python/financial_management.py}.
	\begin{verbatim}
		sum = 0
		for __ in range(12):
		a = float(input())
		sum += a
		print("${:.2f}".format(sum / 12))
	\end{verbatim}
\end{itemize}

\begin{baitoan}[Basic statistical data sample -- mẫu dữ liệu thống kê cơ bản]
	Cho 1 mẫu dữ liệu $(a_i)_{i=1}^n$. Tính trung bình, độ lệch chuẩn, phương sai của mẫu.
\end{baitoan}



\begin{problem}[\cite{Wu_Wang2016}, pp. 5--6: doubles]
	As part of an arithmetic competency program, your students will be given randomly generated lists of 2--15 unique positive integers \& asked to determine how many items in each list are twice some other item in same list. You will need a program to help you with the grading. This program should be able to scan the lists \& output the correct answer for each one. E.g., given the list {\tt1 4 3 2 9 7 18 22} your program should answer $3$, as $2$ is twice $1$, $4$ is twice $2$, \& $18$ is twice $9$.
	\item {\sf Input.} The input file will consist of 1 or more lists of numbers. There will be 1 list of numbers per line. Each list will contain from 2--15 unique positive integers. No integer will be $> 99$. Each line will be terminated with the integer $0$, which is not considered part of the list. A line with the single number $-1$ will mark the end of the file. Some lists may not contain any doubles.
	\item {\sf Output.} The output will consist of 1 line per input list, containing a count of the items that are double some other item.
	\item {\sf Source.} ACM Mid-Central United States 2003.
	\item {\sf IDs for online judges.} POJ 1552, ZOJ 1760, UVA 2787.
\end{problem}

\begin{remark}[Multiple test cases -- đa bộ test]
	For any problem with multiple test cases, a loop is used to deal with multiple test cases. The loop enumerates every test case.
	
	-- Đối với bất kỳ vấn đề nào có nhiều trường hợp thử nghiệm, một vòng lặp được sử dụng để xử lý nhiều trường hợp thử nghiệm. Vòng lặp liệt kê mọi trường hợp thử nghiệm.
\end{remark}

\begin{itemize}
	\item Input: \url{https://github.com/NQBH/advanced_STEM_beyond/blob/main/OLP_ICPC/input/double.inp}.
	\item Output: \url{https://github.com/NQBH/advanced_STEM_beyond/blob/main/OLP_ICPC/output/double.out}.
	\item C++:
	\begin{itemize}
		\item \url{https://github.com/NQBH/advanced_STEM_beyond/blob/main/OLP_ICPC/C++/double.cpp}.
		\begin{verbatim}
			#include <iostream>
			using namespace std;
			int main() {
				int i, j, n, count, a[20];
				cin >> a[0]; // input 1st element
				while (a[0] != -1) { // if it is not end of input, input a new test case
					n = 1;
					for ( ; ; ++n) {
						cin >> a[n];
						if (a[n] == 0) break;
					}
					count = 0; // determine how many items in each list are twice some other item
					for (i = 0; i < n - 1; ++i) { // enumerate all pairs
						for (j = i + 1; j < n; ++j) {
							if (a[i]*2 == a[j] || a[j]*2 == a[i]) // accumulation
							++count;
						}
					}
					cout << count << endl; // output result
					cin >> a[0]; // input 1st element of next test case
				}
				return 0;
			}
		\end{verbatim}
		\item \url{https://github.com/NQBH/advanced_STEM_beyond/blob/main/OLP_ICPC/C++/double_DPAK.cpp}: use map \& vector.
	\end{itemize} 
\end{itemize}
Bài toán có thể mở rộng từ double thành triple, multiple, or just equal.

\begin{problem}[\cite{Wu_Wang2016}, pp. 7--8: sum of consecutive prime numbers]
	Some positive integers can be represented by a sum of 1 or more consecutive prime numbers. How many such representations does a give positive integer have? E.g., the integer $53$ has 2 representations $5 + 7 + 11 + 13 + 17$ \& $53$. The integer $41$ has 3 representations: $2+ 3 + 5 + 7 + 11 + 13, 11 + 13 + 17$, \& $41$. The integer $3$ has only 1 representation, which is $3$. The integer $20$ has no such representations. Note: summands must be consecutive prime numbers, so neither $7 + 13$ nor $3 + 5 + 5 + 7$ is a valid representation for the integer $20$. Your mission is to write a program that reports the number of representations for the given positive integer.
	\item {\sf Input.} The input is a sequence of positive integers, each in a separate line. The integers are between $2$ \& $10000$, inclusive. The end of the input is indicated by a zero.
	\item {\sf Output.} The output should be composed of lines each corresponding to an input line, except the last zero. An output line includes the number of representations for the input integer as the sum of 1 or more consecutive prime numbers. No other characters should be inserted in the output.
	\item {\sf Source.} ACM Japan 2005.
	\item {\sf IDs for online judges}. POJ 2739, UVA 3399.
\end{problem}

\begin{problem}[\cite{Wu_Wang2016}, pp. 9--10: I think I need a houseboat]
	{\sc Fred Mapper} is considering purchasing some land in Louisiana to build his house on. In the process of investigating the land, he learned that the state of Louisiana is actually shrinking by $50$ square miles each year, due to erosion caused by the Mississippi River. Since {\sc Fred} is hoping to live in this house for the rest of his life, he needs to know if his land is going to be lost to erosion.
	
	After doing more research, {\sc Fred} has learned that the land that is being lost forms a semicircle. This semicircle is part of a circle centered at $(0,0)$, with the line that bisects the circle being the $x$ axis. Locations below the $x$ axis are in the water. The semicircle has an area of $0$ at the beginning of year $1$.
	\item {\sf Input.} The 1st line of input will be a positive integer indicating how many data sets will be included $N$. Each of the next $N$ lines will contain the $x,y$ Cartesian coordinates of the land {\sc Fred} is considering. These will be floating-point numbers measured in miles. The $y$ coordinate will be nonnegative. $(0,0)$ will not be given.
	\item {\sf Output.} For each data set, a single line of output should appear. This line should take the form of
	\begin{center}
		Property $N$: This property will begin eroding in year $Z$.
	\end{center}
	where $N$ is the data set (counting from $1$) \& $Z$ is the 1st year (start from $1$) this property will be within the semicircle AT THE END OF YEAR $Z$. $Z$ must be an integer. After the last data set, this should print out ``END OF OUTPUT.''
	\item {\sf Source.} ACM Mid-Atlantic United States 2001.
	\item {\sf Note.} No property will appear exactly on the semicircle boundary: it will be either inside or outside. This problem will be judged automatically. Your answer must match exactly, including the capitalization, punctuation, \& white space. This includes the periods at the ends of the lines. All locations are given in miles.
	\item {\sf IDs for online judges.} POJ 1005, ZOJ 1049, UVA 2363.
\end{problem}

\begin{proof}[Mathematics analysis]
	Gọi $S_n$, $r_n$ lần lượt là tổng diện tích đất sạt lở \& bán kính của hình bán nguyệt của mảnh đất sạt lở đến hết năm $n$, $S_1 = 0,S_n = S_{n-1} + 50 = 50(n - 1) = \dfrac{\pi r_n^2}{2}\Rightarrow r_n = \sqrt{\dfrac{100(n - 1)}{\pi}}$, $\forall n\in\mathbb{N}^\star$. Tìm $n$ thỏa mãn $r_{n-1} < \sqrt{x^2 + y^2} < r_n$, tuwong 
\end{proof}

\begin{problem}[\cite{Wu_Wang2016}, p. 12: Hangover]
	How far can you make a stack of cards overhang a table? If you have 1 card, you can create a maximum overhang of half a card length. (We're assuming that the cards must be perpendicular to the table.) With 2 cards, you can make the top card overhang the bottom one by half a card length, \& the bottom one overhang the table by a third of a card length, for a total maximum overhang of $\frac{1}{2} + \frac{1}{3} = \frac{5}{6}$ card lengths. In general, you can make $n$ cards overhang by $\sum_{i=2}^{n+1} \frac{1}{i} = \frac{1}{2} + \frac{1}{3} + \cdots + \frac{1}{n + 1}$ card lengths, where the top card overhangs the 2nd by $\frac{1}{2}$, the 2nd overhangs the 3rd by $\frac{1}{3}$, the 3rd overhangs the 4th by $\frac{1}{4}$, \& so on, \& the bottom card overhangs the table by $\frac{1}{n + 1}$.
	\item {\sf Input.} The input consists of 1 or more test cases, followed by a line containing the number $0.00$ that signals the end of the input. Each test case is a single line containing a positive floating-point number $c$ whose value is at least $0.01$ \& at most $5.20$; $c$ will contain exactly 3 digits.
	\item {\sf Output.} For each test case, output the minimum number of cards necessary to achieve an overhang of at least $c$ card lengths. Use the exact output format shown in the examples.
	\item {\sf Source.} ACM Mid-Central United States 2001.
	item {\sf IDs for online judges.} POJ 1003, UVA 2294.
\end{problem}

\begin{problem}[\cite{Wu_Wang2016}, p. 17: Sum]
	Your task is to find the sum of all integer numbers lying between $1$ \& $N$ inclusive.
	\item {\sf Input.} The input consists of a single integer $N$ that is not greater than $10000$ by its absolute value.
	\item {\sf Output.} Write a single integer number that is the sum of all integer numbers lying between $1$ \& $N$ inclusive.
	\item {\sf Source.} Source: ACM 2000, Northeastern European Regional Programming Contest (test tour).
	\item {\sf ID for online judge.} Ural 1068.
\end{problem}


\begin{baitoan}[\cite{Trung_HSG_THPT_Tin}, HSG12 Tp. Hà Nội 2020--2021, Prob. 1, p. 80: Find mid -- Tìm giữa]
	(a) Cho $l,r\in\mathbb{N}^\star$. Tìm $m\in[l,r)\cap\mathbb{N}^\star$ để chênh lệch giữa tổng các số nguyên liên tiếp từ $l$ đến $m$ \& tổng các số nguyên liên tiếp từ $m + 1$ đến $r$ là nhỏ nhất. (b) Mở rộng cho $l,r\in\mathbb{Z}$. (c$\star$) Thay tổng bởi tổng bình phương, tổng lập phương, tổng lũy thừa bậc $a\in\mathbb{R}$.
	\item {\sf Input.} 2 số $l,r\in\mathbb{N}^\star$, $l < r\le10^9$.
	\item {\sf Output.} Gồm 1 số nguyên duy nhất là $m$ thỏa mãn.
	\item {\sf Limits.} Subtask 1: $60\%$ các test có $l < r\le10^3$. Subtask 2: $40\%$ các test còn lại có $l < r\le10^9$.
\end{baitoan}

\begin{itemize}
	\item Input: \url{https://github.com/NQBH/advanced_STEM_beyond/blob/main/OLP_ICPC/input/find_mid.inp}.
	\item Output: \url{https://github.com/NQBH/advanced_STEM_beyond/blob/main/OLP_ICPC/output/find_mid.out}.
	\item C++: \url{https://github.com/NQBH/advanced_STEM_beyond/blob/main/OLP_ICPC/C++/find_mid.cpp}.
	\item Python: \url{https://github.com/NQBH/advanced_STEM_beyond/blob/main/OLP_ICPC/Python/find_mid.py}.
\end{itemize}

%------------------------------------------------------------------------------%

\section{Ad Hoc Problems}

\begin{itemize}\sf
	\item \textbf{ad hoc} [a] (from Latin) arranged or happening when necessary \& not planned in advance; [adv] (from Latin) in a way that is arranged or happens when necessary \& is not planned in advance.
	
	-- (từ tiếng Latin) được sắp xếp hoặc xảy ra khi cần thiết \& không được lên kế hoạch trước; [adv] (từ tiếng Latin) theo cách được sắp xếp hoặc xảy ra khi cần thiết \& không được lên kế hoạch trước.
\end{itemize}
For the definitions of ``ad hoc'', see also, e.g., \href{https://vi.wikipedia.org/wiki/Ad_hoc}{viWikipedia{\tt/}ad hoc}, \href{https://en.wikipedia.org/wiki/Ad_hoc}{enWikipedia{\tt/}ad hoc}.

%------------------------------------------------------------------------------%

\subsection{Solving Ad Hoc Problems by Mechanism Analysis}

\begin{problem}[\cite{Wu_Wang2018}, 1.1.1, p. 1, Factstone Benchmark]
	{\sc Amtel} has announced that it will release a 128-bit computer chip by 2010, a 256-bit computer by 2020, \& so on, continuing its strategy of doubling the word size every 10 years. ({\sc Amtel} released a 64-bit computer in 2000, a 32-bit computer in 1990, a 16-bit computer in 1980, an 8-bit computer in 1970, \& a 4-bit computer, its 1st, in 1960.) {\sc Amtel} will use a new benchmark -- the {\rm Factstone} -- to advertise the vastly improved capacity of its new chips. The {\rm Factstone} rating is defined to be the largest integer $n$ such that $n!$ can be represented as an unsigned integer in a computer word. Given a year $1960\le y\le2160$, what will be the {\rm Factstone} rating of {\sc Amtel}'s most recently released chip?
	\item {\sf Input.} There are several test cases. For each test case, there is 1 line of input containing $y$. A line containing $0$ follows that last test case.
	\item {\sf Output.} For each test case, output a line giving the Factstone rating.
	\item {\sf Source.} Waterloo local 2005.09.24
	\item {\sf IDs for Online Judges.} POJ 2661, UVA 10916
\end{problem}

\begin{baitoan}[\cite{Wu_Wang2018}, 1.1.1, p. 1, Điểm chuẩn Factstone]
	{\sc Amtel} đã thông báo rằng họ sẽ phát hành một con chip máy tính 128-bit vào năm 2010, một máy tính 256-bit vào năm 2020, \& cứ thế, tiếp tục chiến lược tăng gấp đôi kích thước từ sau mỗi 10 năm. ({\sc Amtel} đã phát hành một máy tính 64-bit vào năm 2000, một máy tính 32-bit vào năm 1990, một máy tính 16-bit vào năm 1980, một máy tính 8-bit vào năm 1970, \& một máy tính 4-bit, máy tính đầu tiên của họ, vào năm 1960.) {\sc Amtel} sẽ sử dụng một chuẩn mực mới -- {\rm Factstone} -- để quảng cáo cho khả năng được cải thiện đáng kể của các con chip mới của mình. Xếp hạng {\rm Factstone} được định nghĩa là số nguyên $n$ lớn nhất sao cho $n!$ có thể được biểu diễn dưới dạng một số nguyên không dấu trong một từ máy tính. Với năm $1960\le y\le2160$, xếp hạng {\rm Factstone} của chip mới nhất được phát hành của {\sc Amtel} sẽ là bao nhiêu?
	\item {\sf Đầu vào.} Có một số trường hợp thử nghiệm. Đối với mỗi trường hợp thử nghiệm, có 1 dòng đầu vào chứa $y$. Một dòng chứa $0$ theo sau trường hợp thử nghiệm cuối cùng đó.
	\item {\sf Đầu ra.} Đối với mỗi trường hợp thử nghiệm, đầu ra là một dòng cho biết xếp hạng Factstone.
\end{baitoan}
{\sf Analysis.} For a given year $y\in[1960,2160]\cap\mathbb{N}$, 1st the number of bits for the computer in this year is calculated, \& then the largest integer $n$, i.e., the {\it Factstone rating}, that $n!$ can be represented as an unsigned integer in a computer word is calculated. Since the computer was a 4-bit computer in 1960 \& {\sc Amtel} doubles the word size every 10 years, the number of bits for the computer in year $y$ is $b = 2^{2 + \lfloor\frac{y - 1960}{10}\rfloor}\in\mathbb{N}^\star$. The largest unsigned integer for $b$-bit is $2^b - 1\in\mathbb{N}^\star$. If $n!$ is the largest unsigned integer $\le 2^b - 1$, then $n$ is the Factstone rating in year $y$. There are 2 calculation methods:
\begin{enumerate}
	\item Calculate $n!$ directly, which is slow \& easily leads to overflow.
	\item Logarithms are used to calculate $n!$, based on the following inequality
	\begin{equation*}
		n!\le 2^b - 1\Rightarrow\log_2 n! = \sum_{i=1}^n \log_2 i = \log_2 1 + \log_2 2 + \cdots + \log_2 n\le\log_2 (2^b - 1) < \log_2 2^b = b,
	\end{equation*}
	$n$ can be calculated by a loop: Initially $i\coloneqq1$, repeat {\tt++i}, \& $\log_2 i$ is accumulated until the sum is $> b$. Then $i - 1$ is the Factstone rating.
\end{enumerate}
Codes:
\begin{itemize}
	\item C++: \url{https://github.com/NQBH/advanced_STEM_beyond/blob/main/OLP_ICPC/C++/Factstone_benchmark.cpp}.
	\begin{verbatim}
#include <stdio.h>
#include <math.h>

int main() {
    int y;
    while (1 == scanf("%d", &y) && y) { // input test cases
        double w = log(4);
        for (int Y = 1960; Y <= y; Y += 10)
            w *= 2;
        int i = 1; // accumulation log2 i until > w
        double f = 0; // f: sum of accumulation for log2 i
        while (f < w)
            f += log((double)++i);
        printf("%d\n", i - 1); // output Factstone rating
    }
    if (y) printf("fishy ending %d\n", y);
}
	\end{verbatim}
	\item Python:
\end{itemize}

\begin{baitoan}[Mở rộng \cite{Wu_Wang2018}, 1.1.1, p. 1, Điểm chuẩn Factstone]
	{\sc Amtel} đã thông báo rằng kể từ năm $y_0\in\mathbb{N}^\star$ cho trước, họ sẽ phát hành 1 con chip $b^{b_0}$-``bit'' computer (``bit'' ở đây được hiểu theo nghĩa rộng là theo cơ số $b\in\mathbb{N}^\star$, $b\ge2$, chứ không phải hệ nhị phân 2-bit), \& cứ sau $\delta_y$ năm \& $\delta_m$ tháng ($\delta_m\in\overline{1,11}$), số ``bit'' sẽ gấp $b\in\mathbb{N}$, $b\ge2$ lên so với trước đó. {\sc Amtel} sẽ sử dụng một chuẩn mực mới -- {\rm Factstone} -- để quảng cáo cho khả năng được cải thiện đáng kể của các con chip mới của mình. Xếp hạng {\rm Factstone} được định nghĩa là số nguyên $n$ lớn nhất sao cho $n!$ có thể được biểu diễn dưới dạng một số nguyên không dấu trong một từ máy tính. Với năm $y_0\le y$, xếp hạng {\rm Factstone} của chip mới nhất được phát hành của {\sc Amtel} sẽ là bao nhiêu?
	\item {\sf Đầu vào.} Có một số trường hợp thử nghiệm. Đối với mỗi trường hợp thử nghiệm, có 1 dòng đầu vào chứa $y$. Một dòng chứa $0$ theo sau trường hợp thử nghiệm cuối cùng đó.
	\item {\sf Đầu ra.} Đối với mỗi trường hợp thử nghiệm, đầu ra là một dòng cho biết xếp hạng Factstone.
\end{baitoan}

\begin{remark}[$\log$: product $\mapsto$ sum]
	When you encounter a product function of $n$, i.e., $f(n)$, e.g. $n!$ above, use logarithm to transform products into sums.
\end{remark}

\begin{question}[Sum $\leftrightarrows$ Product]
	Làm sao để chuyển 1 tổng thành 1 tích? Làm sao chuyển 1 tích thành 1 tổng?
\end{question}

\begin{proof}[Answer]
	Chuyển tổng thành tích: $e^{a + b} = e^ae^b$, $\forall a,b\in\mathbb{R}$. Tổng quát:
	\begin{equation*}
		e^{\sum_{i=1}^n a_i} = \prod_{i=1}^n e^{a_i},\ \forall a_i\in\mathbb{R},\ \forall n\in\mathbb{N}^\star,\ \forall i = 1,\ldots,n.
	\end{equation*}
	Chuyển tích thành tổng: $\ln(ab) = \ln a + \ln b$, $\forall a,b\in(0,\infty)$. Tổng quát:
	\begin{equation*}
		\ln\prod_{i=1}^n a_i = \sum_{i=1}^n \ln a_i,\ \forall a_i\in(0,\infty),\ \forall n\in\mathbb{N}^\star,\ \forall i = 1,\ldots,n.
	\end{equation*}
	{\it Note}: Có thể thay $\ln x$ bởi $\log x,\log_a x$ với $a\in(0,\infty)$ bất kỳ.
\end{proof}

\begin{problem}[\cite{Wu_Wang2018}, 1.1.2, p. 3, Bridge]
	Consider that $n$ people wish to cross a bridge at night. A group of at most 2 people may cross at any time, \& each group must have a flashlight. Only 1 flashlight is available among the $n$ people, so some sort of shuttle arrangement must be arranged in order to return the flashlight so that more people may cross. 
	
	Each person has a different crossing speed; the speed of a group is determined by the speed of the slower member. Your job is to determine a strategy that gets all $n$ people across the bridge in the minimum time.
	\item {\sf Input.} The 1st line of input contains $n$, followed by $n$ lines giving the crossing times for each of the people. There are not more than $1000$ people, \& nobody takes more than $100$ seconds to cross the bridge.
	\item {\sf Output.} The 1st line of output must contain the total number of seconds required for all $n$ people to cross the bridge. The following lines give a strategy for achieving this time. Each line contains either 1 or 2 integers, indicating which person or people form the next group to cross. (Each person is indicated by the crossing time specified in the input. Although many people may have the same crossing time, the ambiguity is of no consequence.) Note that the crossings alternate directions, as it is necessary to return the flashlight so that more may cross. If more than 1 strategy yields the minimal time, any one will do.
	\item {\sf Source.} POJ 2573, ZOJ 1877, UVA 10037
	\item {\sf IDs for Online Judge.} Waterloo local 2000.09.30
\end{problem}

\begin{baitoan}[\cite{Wu_Wang2018}, 1.1.2, p. 3, ``Đi cầu'']
	Hãy xem xét $n$ người muốn đi qua cầu vào ban đêm. Một nhóm tối đa 2 người có thể đi qua bất kỳ lúc nào, \& mỗi nhóm phải có một chiếc đèn pin. Chỉ có 1 chiếc đèn pin trong số $n$ người, vì vậy phải sắp xếp một số loại hình sắp xếp đưa đón để trả lại đèn pin để nhiều người hơn có thể đi qua.
	
	Mỗi người có tốc độ đi qua khác nhau; tốc độ của một nhóm được xác định bởi tốc độ của thành viên chậm hơn. Nhiệm vụ của bạn là xác định một chiến lược giúp tất cả $n$ người đi qua cầu trong thời gian ngắn nhất.
	\item {\sf Đầu vào.} Dòng đầu tiên của đầu vào chứa $n$, theo sau là $n$ dòng cho biết thời gian đi qua của mỗi người. Không có quá $1000$ người, \& không ai mất hơn $100$ giây để đi qua cầu.
	\item {\sf Đầu ra.} Dòng đầu tiên của đầu ra phải chứa tổng số giây cần thiết để tất cả $n$ người đi qua cầu. Các dòng sau đưa ra một chiến lược để đạt được thời gian này. Mỗi dòng chứa 1 hoặc 2 số nguyên, cho biết người hoặc những người nào tạo thành nhóm tiếp theo để vượt sông. (Mỗi người được chỉ định theo thời gian vượt sông được chỉ định trong đầu vào. Mặc dù nhiều người có thể có cùng thời gian vượt sông, nhưng sự mơ hồ không quan trọng.) Lưu ý rằng các lần vượt sông thay đổi hướng, vì cần phải trả lại đèn pin để nhiều người có thể vượt sông. Nếu có nhiều hơn 1 chiến lược tạo ra thời gian tối thiểu, bất kỳ chiến lược nào cũng được.
\end{baitoan}
\textbf{\textsf{Computer Science Analysis.}} The strategy that gets all $n$ people, numbered $P_1,\ldots,P_n$, across the bridge in the minimum time is: fast people should return the flashlight to help slow people. Because a group of $\le2$ people may cross the bridge each time, we solve the problem by analyzing members of groups. 1st, $n$ people's crossing times, denoted by $t_1,\ldots,t_n$, are sorted in descending order: $t_{i_1}\ge t_{i_2}\ge\cdots t_{i_n}$ where $(i_1,\ldots,i_n)$ is some rearrangement of $(1,\ldots,n)$, i.e., $\{i_1,\ldots,i_n\} = \{1,\ldots,n\}$. Suppose that in the current sequence (i.e., after some people have crossed the bridge \& hence being not counted in the current sequence), $A,B$ are the current fastest person $P_A$ \& the current 2nd fastest person $P_B$'s crossing times, respectively, $a,b$ are the current slowest person $P_a$ \& the current 2nd slowest person $P_b$'s crossing time, respectively. Obviously, $A < B < b < a$. There are 2 methods for making the current slowest person \& the current 2nd slowest person to cross the bridge:
\begin{itemize}
	\item {\it Method 1}: The fastest person $P_A$ helps the slowest person $P_a$ \& the 2nd slowest person $P_b$ to cross the bridge. The steps:
	\begin{enumerate}
		\item The fastest person $P_A$ \& the slowest person $P_a$ cross the bridge, which takes time $\max\{A,a\} = a$.
		\item The fastest person $P_A$ is back, which takes time $A$.
		\item The fastest person $P_A$ \& the 2nd slowest person $P_b$ cross the bridge, which takes time $\max\{A,b\} = b$.
		\item The fastest person is back, which takes time $A$.
	\end{enumerate}
	It takes times $a + A + b + A = 2A + a + b$.
	\item {\it Method 2}: The fastest person $P_A$ \& the 2nd fastest person $P_B$ help the current slowest person $P_a$ \& the current 2nd slowest person $P_b$ to cross the bridge. The steps:
	\begin{enumerate}
		\item The fastest person $P_A$ \& the 2nd fastest person $P_B$ cross the bridge, which takes time $\max\{A,B\} = B$.
		\item The fastest person $P_A$ is back \& returns the flashlight to the slowest person $P_a$ \& the 2nd slowest person $P_b$, which takes time $A$.
		\item The slowest person $P_a$ \& the 2nd slowest person $P_b$ cross the bridge \& give the flashlight to the 2nd fastest person $P_B$, which takes time $\max\{a,b\} = a$.
		\item The 2nd faster person $P_B$ is back, which takes time $B$.
	\end{enumerate}
	It takes time $B + A + a + B = 2B + A + a$. Note: In Method 2, the roles of the fastest person $P_A$ \& the 2nd fastest person $P_B$ are the same \& hence they will take the same time, indeed: $B + B + a + A = 2B + a + A$.
\end{itemize}
Each time, we need to compare Method 1 \& Method 2. If $2A + a + b < 2B + A + a\Leftrightarrow A + b < 2B$, then we use Method 1, else we use Method 2 (in the case $2A + a + b = 2B + A + a\Leftrightarrow A + b = 2B$, either of them can be used). \& each time the current slowest person \& the current 2nd slowest person cross the bridge. Finally, there are 2 cases depending on $n$ being even or odd (since only 2 persons can cross the bridge in each turn):
\begin{itemize}
	\item Case 1: If there are only 2 persons who need to cross the bridge, then the 2 persons cross the bridge. It takes time $B$.
	\item Case 2: There are 3 persons who need to cross the bridge. 1st, the fastest person \& the slowest person cross the bridge. Then, the fastest person is back. Finally, the last 2 persons cross the bridge. It  takes time $\max\{A,a\} + A + \max\{A,b\} = a + A + b$.
\end{itemize}
\textbf{\textsf{Mathematical Analysis.}}

Codes:
\begin{itemize}
	\item C++: 
\end{itemize}

%------------------------------------------------------------------------------%

\subsection{Solving Ad Hoc Problems by Statistical Analysis}
Unlike mechanism analysis, statistical analysis begins with a partial solution to the problem, \& the overall global solution is found based on analyzing the partial solution. Solving problems by statistical analysis is a bottom-up method.

-- Không giống như phân tích cơ chế, phân tích thống kê bắt đầu bằng một giải pháp cục bộ cho vấn đề, \& giải pháp toàn cục tổng thể được tìm thấy dựa trên việc phân tích giải pháp cục bộ. Giải quyết vấn đề bằng phân tích thống kê là phương pháp từ dưới lên.

\begin{problem}[\cite{Wu_Wang2018}, 1.2.1., p. 6, Ants]
	An army of ants walk on a horizontal pole of length $l$ {\rm cm}, each with a constant speed of $1$ {\rm cm{\tt/}s}. When a walking ant reaches an end of the pole, it immediately falls off it. When 2 ants meet, they turn back \& start walking in opposite directions. We know the original positions of ants on the pole; unfortunately, we do not know the directions in which the ants are walking. Your task is to compute the earliest \& the latest possible times needed for all ants to fall off the pole.
	\item {\sf Input.} The 1st line of input contains 1 integer giving the number of cases that follow. The data for each case start with 2 integer numbers: the length of pole (in {\rm cm}) \& $n$, the number of ants residing on the pole. These 2 numbers are followed by $n$ integers giving the position of each ant on the pole as the distance measured from the left end of the pole, in no particular order. All input integers are $\le10^6$, \& they are separated by whitespace.
	\item {\sf Output.} For each case of input, output 2 numbers separated by a single space. The 1st number is the earliest possible time when all ants fall off the pole (if the directions of their walks are chosen appropriately), \& the 2nd number is the latest possible such time.
	\item {\sf Source.} Waterloo local 2004.09.19
	\item {\sf IDs for Online judges.} POJ 1852, ZOJ 2376, UVA 10714
\end{problem}

\begin{baitoan}[\cite{Wu_Wang2018}, 1.2.1., p. 6, Kiến]
	Một đội quân kiến đi trên một cột nằm ngang dài $l$ {\rm cm}, mỗi con có tốc độ không đổi là $1$ {\rm cm{\tt/}s}. Khi một con kiến đi đến một đầu của cột, nó ngay lập tức rơi khỏi cột. Khi 2 con kiến gặp nhau, chúng quay lại \& bắt đầu đi theo hướng ngược nhau. Chúng ta biết vị trí ban đầu của các con kiến trên cột; thật không may, chúng ta không biết hướng mà các con kiến đang đi. Nhiệm vụ của bạn là tính toán thời gian sớm nhất \& thời gian muộn nhất có thể cần thiết để tất cả các con kiến rơi khỏi cột.
	\item {\sf Đầu vào.} Dòng đầu tiên của đầu vào chứa 1 số nguyên cho biết số trường hợp theo sau. Dữ liệu cho mỗi trường hợp bắt đầu bằng 2 số nguyên: chiều dài của cột (tính bằng {\rm cm}) \& $n$, số kiến trú ngụ trên cột. 2 số này được theo sau bởi $n$ số nguyên cho biết vị trí của mỗi con kiến trên cột là khoảng cách được đo từ đầu bên trái của cột, không theo thứ tự cụ thể. Tất cả các số nguyên đầu vào là $\le10^6$, \& chúng được phân cách bằng khoảng trắng.
	\item {\sf Đầu ra.} Đối với mỗi trường hợp đầu vào, đầu ra 2 số được phân cách bằng một khoảng trắng. Số thứ nhất là thời gian sớm nhất có thể khi tất cả các con kiến rơi khỏi cột (nếu hướng đi của chúng được chọn một cách thích hợp), \& số thứ 2 là thời gian muộn nhất có thể như vậy.
\end{baitoan}
\textbf{\textsf{Analysis.}} 

\begin{problem}[\cite{Wu_Wang2018}, 1.3.1., pp. 12--13, Perfection]
	From the article Number Theory in the 1994 Microsoft Encarta: ``If $a,b,c\in\mathbb{Z}$ s.t. $a = bc$, $a$ is called a multiple of $b$ or of $c$, \& $b$ or $c$ is called a {\rm divisor} or {\rm factor} of $a$. If $c\ne\pm1$, $b$ is called a {\rm proper divisor} of $a$.
\end{problem}

%------------------------------------------------------------------------------%

\section{VNOI}

\begin{baitoan}[gcd in Pascal triangle -- ƯCLN trong tam giác Pascal, \url{https://oj.vnoi.info/problem/gpt}]
	Tam giác Pascal là 1 cách sắp xếp hình học của các hệ số nhị thức vào 1 tam giác. Hàng thứ $n\in\mathbb{N}$ của tam giác bao gồm các hệ số trong khai triển của đa thức $f(x,y) = (x + y)^n$. I.e., phần tử tại cột thứ $k$, hàng thứ $n$ của tam giác Pascal là $C_n^k = \binom{n}{k}$, i.e., tổ hợp chập $k$ của $n$ phần tử $0\le k\le n$. Cho $n\in\mathbb{N}$. Tính ${\rm GPT}(n)$ là ƯCLN của các số nằm giữa 2 số 1 trên hàng thứ $n$ của tam giác Pascal.
	\item {\sf Input.} Dòng đầu ghi $T$ là số lượng test. $T$ dòng tiếp theo, mỗi dòng ghi 1 số nguyên $n$.
	\item {\sf Output.} Gồm $T$ dòng, mỗi dòng ghi ${\rm GPT}(n)$ tương ứng.
	\item {\sf Constraint.} $1\le T\le20$, $2\le n\le10^9$.
\end{baitoan}
{\it Phân tích.} Công thức khai triển nhị thức Newton: $(a + b)^n = \sum_{i=0}^n C_n^ia^{n-i}b^i$, $\forall n\in\mathbb{N}$, see, e.g., \href{https://en.wikipedia.org/wiki/Binomial_theorem}{Wikipedia{\tt/}binomial theorem}. Cần tính $\gcd(\{C_n^i;1\le i\le n - 1\}) = \gcd(C_n^1,C_n^2,\ldots,C_n^{n-1})$. Chú ý mỗi hàng của tam giác Pascal có tính chất đối xứng nên chỉ cần xét ``1 nửa'' là đủ. Cụ thể hơn: $C_n^k = C_n^{n-k}$, $\forall k\in\mathbb{N}$, $k\le n$, nên
\begin{equation*}
	\{C_n^1,\ldots,C_n^{n-1}\} = \{C_n^1,\ldots,C_n^{\lfloor\frac{n}{2}\rfloor}\} = \left\{\begin{split}
		&\{C_n^1,\ldots,C_n^{\frac{n-1}{2}}\}&&\mbox{if } n\not{\divby}\ 2,\\
		&\{C_n^1,\ldots,C_n^{\frac{n}{2}}\}&&\mbox{if } n\divby2,
	\end{split}\right.
\end{equation*}
nên thay vì xét $i = 1,\ldots,n-1$, chỉ cần xét $i = 1,\ldots,\lfloor\frac{n}{2}\rfloor$ là đủ.

\begin{theorem}
	\begin{equation*}
		\gcd\{C_n^i\}_{i=1}^{n-1} = \left\{\begin{split}
			&p&&\mbox{if } n = p^k\mbox{ for some prime } p\mbox{ \& some } n\in\mathbb{N}^\star,\\
			&1&&\mbox{if } n\ne p^k\mbox{ for all prime } p\mbox{ \& any } n\in\mathbb{N}^\star.
		\end{split}\right.
	\end{equation*}
\end{theorem}
See also, e.g.:
\begin{itemize}
	\item \href{https://math.stackexchange.com/questions/2067235/gcd-of-binomial-coefficients}{Mathematics StackExchange{\tt/}gcd of binomial coefficients}.
\end{itemize}

%------------------------------------------------------------------------------%

\section{Recurrence Relation -- Quan Hệ Hồi Quy}
\textbf{\textsf{Resources -- Tài nguyên.}}
\begin{enumerate}
	\item \cite{Andrica_Bagdasar2020}. {\sc Dorin Andrica, Ovidiu Bagdasar}. {\it Recurrent Sequences: Key Results, Applications, \& Problems}.
\end{enumerate}
Let $X$ be an arbitrary set. A function $f:\mathbb{N}\to X$ defines a {\it sequence} $(x_n)_{n=0}^\infty$ of elements of $X$, where $x_n = f(n)$, $\forall n\in\mathbb{N}$. The set of all sequences with elements in $X$ is denoted by $X^\mathbb{N}$, while $X^n$ denotes Cartesian product of $n$ copies of $X$, where $X$ will be chosen as $\mathbb{C}$, the Euclidean space $\mathbb{R}^m$, the algebra $M_r(A)$ of the $r\times r$ matrices with entries in a ring $A$, etc. The set $X^\mathbb{N}$ has numerous important subsets. E.g., when $X = \mathbb{R}$, the set of real numbers $\mathbb{R}^\mathbb{N}$ includes sequences which are bounded, monotonous, convergent, positive, nonzero, periodic, etc.

-- Cho $X$ là 1 tập hợp tùy ý. Một hàm $f:\mathbb{N}\to X$ định nghĩa một {\it dãy} $(x_n)_{n=0}^\infty$ các phần tử của $X$, trong đó $x_n = f(n)$, $\forall n\in\mathbb{N}$. Tập hợp tất cả các dãy có các phần tử trong $X$ được ký hiệu là $X^\mathbb{N}$, trong khi $X^n$ biểu thị tích Descartes của $n$ bản sao của $X$, trong đó $X$ sẽ được chọn là $\mathbb{C}$, không gian Euclidean $\mathbb{R}^m$, đại số $M_r(A)$ của các ma trận $r\times r$ có các phần tử trong vành $A$, v.v. Tập hợp $X^\mathbb{N}$ có nhiều tập con quan trọng. Ví dụ, khi $X = \mathbb{R}$, tập hợp các số thực $\mathbb{R}^\mathbb{N}$ bao gồm các dãy số bị chặn, đơn điệu, hội tụ, dương, khác không, tuần hoàn, v.v.

When $a\in X$ is fixed, in {\it implicit form}, a recurrence relation is defined by
\begin{equation}
	\label{recurrence relation: implicit form}
	F_n(x_n,x_{n-1},\ldots,x_0) = a,\ \forall n\in\mathbb{N}^\star,
\end{equation}
where $F_n:X^{n+1}\to X$ is a function of $n + 1$ variables, $n\in\mathbb{N}^\star$. In general, the implicit form of a recurrence relation does not define uniquely the sequence $(x_n)_{n=0}^\infty$.

The {\it explicit form} of a recurrence relation is
\begin{equation}
	\label{recurrence relation: explicit form}
	x_n = f_n(x_{n-1},\ldots,x_0),\ \forall n\in\mathbb{N}^\star,
\end{equation}
where $f_n:X^n\to X$ is a function, $\forall n\in\mathbb{N}^\star$. The relations \eqref{recurrence relation: explicit form} give the rule to construct the term $x_n$ of the sequence $(x_n)_{n\ge0}$ from the 1st term $x_0$: $x_1 = f_1(x_0),x_2 = f_2(x_1,x_0),\ldots$, i.e., \eqref{recurrence relation: explicit form} is a functional type relation.

In mathematics, a {\it recurrence relation} is an equation according to which the $n$th term of a sequence of numbers is equal to some combination of the previous terms, i.e.:
\begin{equation}
	\label{explicit recurrent sequence 0}
	\left\{\begin{split}
		u_0&\in\mathbb{F},\\
		u_n &= f_n(n,u_0,u_1,\ldots,u_{n-1}),\ \forall n\in\mathbb{N}^\star,
	\end{split}\right.
\end{equation}
if $u_0$ is the initial element, which is an element of the given field $\mathbb{F}$, of the sequence $\{u_n\}_{n=0}^\infty$, where $f_n:\mathbb{N}^\star\times\mathbb{F}^n\to\mathbb{F}$ is a scalar-valued function of $(n + 1)$-dimensional-vector-valued argument, $\forall n\in\mathbb{N}^\star$; \&
\begin{equation}
	\label{explicit recurrent sequence 1}
	\left\{\begin{split}
		u_1&\in\mathbb{F},\\
		u_n &= f_n(n,u_1,\ldots,u_{n-1}),\ \forall n\in\mathbb{N},\ n\ge2,
	\end{split}\right.
\end{equation}
if $u_1$ is the initial element, which is an element of the given field $\mathbb{F}$, of the sequence $\{u_n\}_{n=1}^\infty$, where $f_n:\mathbb{N}_{\ge2}\times\mathbb{F}^{n-1}\to\mathbb{F}$ is a scalar-valued function of $n$-dimensional-vector-valued argument, $\forall n\in\mathbb{N}$, $n\ge2$.

\begin{question}
	What define uniquely \eqref{explicit recurrent sequence 0}?
\end{question}

\begin{proof}[Answer]
	The solutions $\{u_n\}_{n=0}^\infty$ defined by \eqref{explicit recurrent sequence 0}, are uniquely determined in terms of $u_0\in\mathbb{F},\{f_n\}_{n=1}^\infty$. Analogously, the solutions $\{u_n\}_{n=0}^\infty$ defined by \eqref{explicit recurrent sequence 1}, are uniquely determined in terms of $u_1\in\mathbb{F},\{f_n\}_{n=2}^\infty$.
\end{proof}

\begin{remark}[Starting index of a sequence]
	The starting index of a sequence $\{u_n\}_{n\in\{0,1\}}^\infty$ can be $0$, which is commonly used in Computer Science \& various programming languages, or $1$, which is commonly used in Mathematics.
\end{remark}
Often, only $k$ previous terms of the sequence appear in the equation, for a parameter $k$ that is independent of $n$; this number $k$ is called the {\it order} of the relation. If the values of the 1st $k$ numbers in the sequence have been given, the rest of the sequence can be calculated by repeatedly applying the equation.

In {\it linear recurrences}, the $n$th term is equated to a \href{https://en.wikipedia.org/wiki/Linear_function}{linear function} of the $k$ previous terms. A famous example is the recurrence for the \href{https://en.wikipedia.org/wiki/Fibonacci_number}{Fibonacci numbers}
\begin{equation}
	\label{Fibonacci sequence}
	\tag{Fib}
	\left\{\begin{split}
		F_0 & = F_1 = 1,\\
		F_n &= F_{n-1} + F_{n-2},\ \forall n\in\mathbb{N},\,n\ge2,
	\end{split}\right.
\end{equation}
where the order $k = 2$ \& the linear function merely adds the 2 previous terms. This example is a \href{https://en.wikipedia.org/wiki/Linear_recurrence_with_constant_coefficients}{linear recurrence with constant coefficients}, because the coefficients of the linear function (1 \& 1) are constants that do not depend on $n$. For these recurrences, one can express the general term of the sequence as a \href{https://en.wikipedia.org/wiki/Closed-form_expression}{closed-form expression} of $n$. \href{https://en.wikipedia.org/wiki/P-recursive_equation}{Linear recurrences with polynomial coefficients} depending on $n$ are also important, because many common \href{https://en.wikipedia.org/wiki/Elementary_functions}[elementary functions] \& \href{https://en.wikipedia.org/wiki/Special_functions}{special functions} have a \href{https://en.wikipedia.org/wiki/Taylor_series}{Taylor series} whose coefficients satisfy such a recurrence relation (see \href{https://en.wikipedia.org/wiki/Holonomic_function}{Wikipedia{\tt/}holonomic function}).

Def: Solving a recurrence relation means obtaining a \href{https://en.wikipedia.org/wiki/Closed-form_solution}{closed-form solution}: a non-recursive function of $n$.

The concept of a recurrence relation can be extended to \href{https://en.wikipedia.org/wiki/Multidimensional_array}{multidimensional arrays}, i.e., \href{https://en.wikipedia.org/wiki/Indexed_families}{indexed families} that are indexed by \href{https://en.wikipedia.org/wiki/Tuple}{tuples} of naturals.

\begin{definition}[Recurrence relation]
	A \emph{recurrence relation} is an equation that expresses each element of a sequence as a function of preceding ones. More precisely, in the case where only the immediately preceding element is involved, a {\rm1st order recurrence relation} has the form
	\begin{equation}
		\label{1st order recurrence relation}
		\boxed{\left\{\begin{split}
			u_0&\in X,\\
			u_n &= \varphi(n,u_{n-1}),\ \forall n\in\mathbb{N}^\star,
		\end{split}\right.}		
	\end{equation}
	where $\varphi:\mathbb{N}\times X\to X$ is a function, where $X$ is a set to which the elements of a sequence must belong. For any $u_0\in X$, this defines a unique sequence with $u_0$ as its 1st element, called the {\rm initial value}, which is easy to modify the definition for getting sequences starting from the term of index $1$ or higher.
	
	A {\rm recurrence relation of order $k\in\mathbb{N}^\star$} has the form
	\begin{equation}
		\label{kth order recurrence relation}
		\boxed{\left\{\begin{split}
			u_0,u_1,\ldots,u_{k-1}&\in X,\\
			u_n &= \varphi(n,k,u_{n-1},u_{n-2},\ldots,u_{n-k}),\ \forall n\in\mathbb{N},\,n\ge k,
		\end{split}\right.}
	\end{equation}
	where $\varphi:\mathbb{N}^2\times X^k\to X$ is a function that involves $k$ consecutive elements of the sequence. In this case, $k$ initial values are needed for defining a sequence.
\end{definition}

\begin{remark}[Explicit- vs. implicit recurrence relations]
	The explicit recurrence relations are the recurrence relations that can be given as \eqref{explicit recurrent sequence 0} or \eqref{explicit recurrent sequence 1}; meanwhile the implicit recurrence relations are the recurrence relations that can be given as
	\begin{equation}
		\label{implicit recurrent sequence 0}
		\left\{\begin{split}
			u_0&\in\mathbb{F},\\
			f_n(n,u_0,u_1,\ldots,u_{n-1},u_n) &= 0,\ \forall n\in\mathbb{N}^\star,
		\end{split}\right.
	\end{equation}
	if $u_0$ is the initial element, which is an element of the given field $\mathbb{F}$, of the sequence $\{u_n\}_{n=0}^\infty$, where $f_n:\mathbb{N}^\star\times\mathbb{F}^{n+1}\to\mathbb{F}$ is a scalar-valued function of $(n + 1)$-dimensional-vector-valued argument, $\forall n\in\mathbb{N}^\star$; \&
	\begin{equation}
		\label{implicit recurrent sequence 1}
		\left\{\begin{split}
			u_1&\in\mathbb{F},\\
			f_n(n,u_1,\ldots,u_{n-1},u_n) &= 0,\ \forall n\in\mathbb{N},\ n\ge2,
		\end{split}\right.
	\end{equation}
	if $u_1$ is the initial element, which is an element of the given field $\mathbb{F}$, of the sequence $\{u_n\}_{n=1}^\infty$, where $f_n:\mathbb{N}_{\ge2}\times\mathbb{F}^n\to\mathbb{F}$ is a scalar-valued function of $n$-dimensional-vector-valued argument, $\forall n\in\mathbb{N}$, $n\ge2$. The wellposednesses of \eqref{implicit recurrent sequence 0} \& \eqref{implicit recurrent sequence 1} require that the corresponding recurrent equation has a unique solution to be able to define $u_n$ uniquely.
\end{remark}

\begin{example}[Factorial]
	The \href{https://en.wikipedia.org/wiki/Factorial}{factorial} is defined by the recurrence relation $n! = n\cdot(n - 1)!$, which is \eqref{1st order recurrence relation} with $X = \mathbb{N}^\star$, $u_0 = 0! = 1$, $\varphi(x,y) = xy$, $\forall x,y\in X = \mathbb{N}^\star$ so that $u_n = \varphi(n,u_{n-1}) = nu_{n-1} = n(n - 1)! = n!$, $\forall n\in\mathbb{N}^\star$. This is an example of a {\rm linear recurrence with polynomial coefficients} of order $1$, with the simple polynomial (in $n$) $n$ as its only coefficient.
\end{example}

\begin{example}[Logistic map]
	An example of a recurrence relation is the \href{https://en.wikipedia.org/wiki/Logistic_map}{logistic map} defined by
	\begin{equation}
		\label{logistic map}
		\tag{lgt}
		\left\{\begin{split}
			x_0&\in\mathbb{R},\\
			x_{n+1} &= rx_n(1 - x_n),
		\end{split}\right.
	\end{equation}
	for a given constant $r$. The behavior of the sequence depends dramatically on $r$, but is stable when the initial condition $x_0$ varies (proofs?)
\end{example}

%------------------------------------------------------------------------------%

\subsection{Linear recurrence with constant coefficients -- Hồi quy tuyến tính với hệ số hằng}
See, e.g., \href{https://en.wikipedia.org/wiki/Linear_recurrence_with_constant_coefficients}{Wikipedia{\tt/}linear recurrence with constant coefficients}. In mathematics (including combinatorics, linear algebra, \& \href{https://en.wikipedia.org/wiki/Dynamical_systems}{dynamical system}), a {\it linear recurrence with constant coefficients} (also known as a {\it linear recurrence relation} or {\it linear difference equation}) sets equal to 0 a \href{https://en.wikipedia.org/wiki/Polynomial}{polynomial} that is linear in the various iterates of a variable -- i.e., in the values of the elements of a sequence. The polynomial's linearity means that each of its terms has degree 0 or 1. A linear recurrence denotes the evolution of some variable over time, with the current \href{https://en.wikipedia.org/wiki/Discrete_time}{time period} or discrete moment in time denoted as $t$, 1 period earlier denoted as $t - 1$, 1 period later as $t + 1$, etc.

The \href{https://en.wikipedia.org/wiki/Equation_solving}{solution} of such an equation is a function of $t$, \& not of any iterate values, giving the value of the iterate at any time. To find the solution, it is necessary to know the specific values (known as \href{https://en.wikipedia.org/wiki/Initial_condition}{\it initial conditions}) of $n$ of the iterates, \& normally these are the $n$ iterates that are oldest. The equation or its variable is said to be \href{https://en.wikipedia.org/wiki/Lyapunov_stability#Definition_for_discrete-time_systems}{\it stable} if from any set of initial conditions the variable's limit as time goes to $\infty$ exists; this limit is called the \href{https://en.wikipedia.org/wiki/Steady_state}{\it steady state}.

Difference equations are used in a variety of contexts, e.g. in \href{https://en.wikipedia.org/wiki/Economics}{economics} to model the evolution through time of variables e.g. \href{https://en.wikipedia.org/wiki/Gross_domestic_product}{gross domestic product}, the \href{https://en.wikipedia.org/wiki/Inflation_rate}{inflation rate}, the \href{https://en.wikipedia.org/wiki/Exchange_rate}{exchange rate}, etc. They are used in modeling such \href{https://en.wikipedia.org/wiki/Time_series}{time series} because values of these variables are only measured at discrete intervals. In \href{https://en.wikipedia.org/wiki/Econometrics}{econometric} applications, linear difference equations are modeled with \href{https://en.wikipedia.org/wiki/Stochastic_process}{stochastic terms} in the form of \href{https://en.wikipedia.org/wiki/Autoregressive_model}{autoregressive (AR) models} \& in models e.g. \href{https://en.wikipedia.org/wiki/Vector_autoregression}{vector autoregression} (VAR) \& \href{https://en.wikipedia.org/wiki/Autoregressive_moving_average}{autoregressive moving average} (ARMA) models that combine AR with other features.

\begin{definition}[Linear recurrence with constant coefficients]
	A {\rm linear recurrence with constant coefficients} is an equation of the following form, written in terms of parameters $a_1,\ldots,a_n,b$:
	\begin{equation}
		\label{linear recurrence with constant coefficients}
		y_n = \sum_{i=1}^k a_iy_{n-i} + b,
	\end{equation}
	or equivalently as
	\begin{equation}
		\label{linear recurrence with constant coefficients 1}
		y_{n + k} = \sum_{i=1}^n a_iy_{n + k - i} + b,
	\end{equation}
\end{definition}

%------------------------------------------------------------------------------%

\section{Dynamic Programming -- Quy Hoạch Động}
\textbf{\textsf{Resource -- Tài nguyên.}}
\begin{enumerate}
	\item \href{https://en.wikipedia.org/wiki/Dynamic_programming}{Wikipedia{\tt/}dynamic programming}.
	\item \cite{Thu_Phuong_Tien_Triet_Phuong_KTLT}. {\sc Trần Đan Thư, Nguyễn Thanh Phương, Đinh Bá Tiến, Trần Minh Triết, Đặng Bình Phương}. {\it Kỹ Thuật Lập Trình}. Chap. 9: Kỹ Thuật Quy Hoạch Động.
	\item \cite{Bertsekas2005,Bertsekas2017}. {\sc Dimitri P. Bertsekas}. {\it Dynamic Programming \& Optimal Control. Vol. I}. 3e. 4e (can't download yet).
	\item \cite{Bertsekas2007,Bertsekas2012} {\sc Dimitri P. Bertsekas}. {\it Dynamic Programming \& Optimal Control. Vol. II}. 3e. 4e (can't download yet).
\end{enumerate}

\begin{baitoan}[\cite{Thu_Phuong_Tien_Triet_Phuong_KTLT}, p. 441]
	(a) Tìm $(x,y)\in\mathbb{R}^2$ thỏa $x^2 + y^2\le1$ để $x + y$ đạt {\rm GTNN, GTLN}. (b) Tìm $(x,y)\in\mathbb{R}^2$ thỏa $x^2 + y^2\le r^2$ để $x + y$ đạt {\rm GTNN, GTLN} với $r\in(0,\infty)$. (c) Phát biểu ý nghĩa hình học của bài toán.
\end{baitoan}
Phát biểu bài toán tối ưu{\tt/}bài toán quy hoạch:
\begin{align*}
	&\max_{x^2 + y^2\le r^2} x + y = \sqrt{2},\ {\arg\max}_{x^2 + y^2\le r^2} x + y = \left(\frac{1}{\sqrt{2}},\frac{1}{\sqrt{2}}\right),\\
	&\min_{x^2 + y^2\le r^2} x + y = -\sqrt{2},\ {\arg\min}_{x^2 + y^2\le r^2} x + y = \left(-\frac{1}{\sqrt{2}},-\frac{1}{\sqrt{2}}\right),
\end{align*}

\begin{definition}[Fibonacci sequences]
	{\sf Fibonacci sequences} are defined by
	\begin{equation*}
		\left\{\begin{split}
			F_0 &= 0,\ F_1 = 1,\\
			F_n &= F_{n - 1} + F_{n - 2},\ \forall n\in\mathbb{N},\,n\ge2.
		\end{split}\right.
	\end{equation*}
\end{definition}

\begin{definition}[Lucas sequences]
	The sequence of {\sf Lucas numbers} are defined by
	\begin{equation*}
		\left\{\begin{split}
			L_0 &= 2,\ L_1 = 1,\\
			L_n &= L_{n - 1} + L_{n - 2},\ \forall n\in\mathbb{N},\,n\ge2.
		\end{split}\right.
	\end{equation*}
\end{definition}

\begin{baitoan}[Fibonacci numbers -- Số Fibonacci]
	(a) Tính dãy số Fibonacci \& dãy Lucas bằng: (i) Truy hồi $O(a^n)$ với $a\approx1.61803$. (ii) Quy hoạch động $O(n)$. (iii) Quy hoạch động cải tiến. (b) Trong mỗi thuật toán, tính cụ thể số lần gọi hàm tính $F_i,L_i$, với $i = 0,1,\ldots,n$, số phép cộng đã thực hiện. Tính time- \& space complexities. (c) Mở rộng bài toán cho dãy số truy hồi cấp 2 với hệ số hằng $\{u_n\}_{n=1}^\infty$ được xác định bởi:
	\begin{equation*}
		\left\{\begin{split}
			u_0 &= \alpha,\ u_1 = \beta,\\
			u_{n+2} &= au_{n+1} + bu_n,\ \forall n\in\mathbb{N},
		\end{split}\right.
	\end{equation*}
	với $a,b,\alpha,\beta\in\mathbb{C}$ cho trước. (d) Mở rộng bài toán cho dãy số truy hồi cấp 2 với hệ số thay đổi $\{u_n\}_{n=1}^\infty$ được xác định bởi:
	\begin{equation*}
		\left\{\begin{split}
			u_0 &= \alpha,\ u_1 = \beta,\\
			u_n &= a(n)u_{n-1} + b(n)u_{n-2},\ \forall n\in\mathbb{N},\,n\ge2.
		\end{split}\right.
	\end{equation*}
	với $\alpha,\beta\in\mathbb{C}$, $a,b:\mathbb{N}_{\ge2}\to\mathbb{C}$ là 2 hàm giá trị phức cho trước.
\end{baitoan}

\begin{proof}
	Cho $n\in\mathbb{N}$. Đặt $f(n,i)$ là số lần xuất hiện (frequency) của $F_i$ khi tính $F_n$ bằng công thức truy hồi. Dễ chứng minh bằng phương pháp quy nạp Toán học: $f(n,n - i) = F_{i + 1}$, $\forall i = 0,1,\ldots,n$.
	\begin{itemize}
		\item Số lần call hàm truy hồi $= \sum_{i=0}^n f_{n-i} = \sum_{i=0}^n F_{i+1} = \sum_{i=1}^{n+1} F_i = F_{n+3} - 1$.
		\item Số lần thực hiện phép cộng $= F_{n+1} - 1$.
		\item Tốn $n + 1$ ô nhớ để chứa $F_0,F_1,\ldots,F_n$.
	\end{itemize}
\end{proof}
C++: \url{https://github.com/NQBH/advanced_STEM_beyond/blob/main/OLP_ICPC/C++/Fibonacci.cpp}.
\begin{verbatim}
#include <iostream>
using namespace std;
const long nMAX = 10000;

long fib(long i) {
    if (i == 1 || i == 2)
        return 1;
    else
        return fib(i - 1) + fib(i - 2);
}

// \cite{Thu_Phuong_Tien_Triet_Phuong_KTLT}, p. 443
long fib_recurrence(long n) {
    long ans, Fn_1, Fn_2;
    if (n <= 2)
        ans = 1;
    else {
        Fn_1 = fib_recurrence(n - 1);
        Fn_2 = fib_recurrence(n - 2);
        ans = Fn_1 + Fn_2;
    }
    return ans;
}

// \cite{Thu_Phuong_Tien_Triet_Phuong_KTLT}, p. 443
long fib_dynamic(long n) {
    long F[nMAX + 1];
    F[0] = 0;
    F[1] = F[2] = 1;
    for (int i = 2; i <=n; ++i)
        F[i] = F[i - 1] + F[i - 2];
    return F[n];
}

// \cite{Thu_Phuong_Tien_Triet_Phuong_KTLT}, p. 443
long fib_dynamic_improved(long n) {
    long lastF = 1, F = 1;
    int i = 1;
    while (i < n) {
        F += lastF;
        lastF = F - lastF;
        ++i;
    }
    return F;
}

int main() {
    long n, i;
    cin >> n;
    cout << "Fibonacci sequence of length " << n << ":\n";
	
    for (i = 0; i <= n; ++i)
        cout << fib(i) << " ";
    cout << "\n";
	
    for (i = 0; i <= n; ++i)
        cout << fib_recurrence(i) << " ";
    cout << "\n";
	
    for (i = 0; i <= n; ++i)
        cout << fib_dynamic(i) << " ";
    cout << "\n";
	
    for (i = 0; i <= n; ++i)
        cout << fib_dynamic_improved(i) << " ";
    cout << "\n";
}
\end{verbatim}

\begin{baitoan}[\cite{Trung_THCS_Tin}, Đăk Nông THCS 2022--2023, 4: virus, p. 32]
	Flashback là 1 loại virus máy tính sinh sản rất nhanh khi gặp môi trường thuận lợi \& là 1 loại virus nguy hiểm, có tốc độ lây lan nhanh trong môi trường mạng. Flashback lần đầu được phát hiện vào năm 2011 bởi công ty diệt virus Intego dưới dạng 1 bản cài đặt flash giả \& chúng sinh sản theo quy luật:
	\begin{itemize}
		\item Ngày đầu tiên (ngày thứ $0$) có $n$ cá thể ở mức $1$.
		\item Ở mỗi ngày tiếp theo, mỗi cá thể mức $i$ sinh ra $i$ cá thể mức $1$, các cá thể mới sẽ sinh sôi, phát triển từ ngày hôm sau.
		\item Bản thân cá thể thứ $i$ sẽ phát triển thành mức $i + 1$ \& chu kỳ phát triển trong ngày chấm dứt.
	\end{itemize}
	\item {\sf Requirement.} Xác định sau $k$ ngày trong môi trường có bao nhiêu cá thể.
	\item {\sf Input.} Vào từ file {\tt flashback.inp} gồm: Dòng 1 chứa số bộ test $t\in\mathbb{N}^\star$. $t$ dòng tiếp theo, mỗi dòng chứa $n,k\in\mathbb{N}^\star$, ràng buộc $n\in[1000],k\le[10^5]$.
	\item {\sf Output.} Ghi ra file {\tt flashback.out} 1 số nguyên duy nhất là số dư của kết quả tìm được chia cho $10^9 + 7$.
	\item {\sf Limit.}
	\begin{itemize}
		\item Subtask 1: có 40\% số test ứng với $n\le100,k\le1000$.
		\item Subtask 2: có 60\% số test ứng với $n\le1000,k\le10^5$.
	\end{itemize}
	\item {\sf Sample.}
	\begin{table}[H]
		\centering
		\begin{tabular}{|l|l|}
			\hline
			{\tt flashback.inp} & {\tt flashback.out} \\
			\hline
			5 & 65 \\
			5 3 & 130 \\
			10 3 & 170 \\
			5 4 & 2563 \\
			11 6 & 232767 \\
			999 6 & \\
			\hline
		\end{tabular}
	\end{table}
	\noindent(a) Mô tả thành mô hình toán học. (b) Viết chương trình {\sf C{\tt/}C++, Pascal, Python} để giải.
\end{baitoan}

\begin{proof}
	Gọi $a(d,l)$ là số cá cá thể ở mức $l$ vào ngày $d$ ($d$: day, $l$: level). Thiết lập công thức truy hồi:
	
	Kết quả cuối cùng: $n(F_1 + \sum_{i=1}^k F_{2i}) = n\sum_{i=1}^k F_{2i} = nF_{2k+1}$.
	
	C++ codes:
	\begin{itemize}
		\item DAK's C++: \url{https://github.com/NQBH/advanced_STEM_beyond/blob/main/OLP_ICPC/C%2B%2B/DAK_virus.cpp}.
\begin{verbatim}

\end{verbatim}
		\item NHT's C++: \url{https://github.com/NQBH/advanced_STEM_beyond/blob/main/OLP_ICPC/C%2B%2B/NHT_virus.cpp}.
\begin{verbatim}
#include <bits/stdc++.h>
#define ll long long
using namespace std;

const int MOD = 1e9 + 7;
int fib[2000009];   

int main() {
    ios::sync_with_stdio(false);
    cin.tie(nullptr);
	
    fib[0] = 0;
    fib[1] = 1; 
    for(int i = 2; i <= 200005; i++) {
        fib[i] = (fib[i - 1] + fib[i - 2]) % MOD;
    }
	
    int T;
    cin >> T;
    while(T--) {
        int n, k;
        cin >> n >> k;
        cout << n*fib[2*k+1] << '\n';
    }
    return 0;
}
\end{verbatim}
	\end{itemize}
\end{proof}


\begin{baitoan}[\cite{Trung_THCS_Tin}, Hà Nội HSG9 2021--2022, 5: cổ phiếu VNI, p. 37]
	Bình mua bán cổ phiếu VNI trên thị trường chứng khoán. Giả sử giá của 1 cổ phiếu VNI trong $n\in\mathbb{N}^\star$ ngày lần lượt là $a_1,a_2,\ldots,a_n$. Biết mỗi ngày Bình chỉ thực hiện 1 trong 3 hoạt động:
	\begin{enumerate}
		\item Mua 1 cổ phiếu VNI
		\item Bán số lượng cổ phiếu bất kỳ mà Bình đang sở hữu
		\item Không thực hiện bất kỳ giao dịch nào.
	\end{enumerate}
	\item {\sf Requirement.} Bình thực hiện mua bán cổ phiếu VNI như thế nào để thu được lợi nhuận lớn nhất nếu Bình tham gia mua bán bắt đầu từ ngày thứ $t\in\mathbb{N}^\star$ cho trước?
	\item {\sf Input.} Đọc từ file {\tt vni.inp} gồm:
	\begin{itemize}
		\item Dòng 1: $n\in[10^5]$ là số ngày biết giá cổ phiếu.
		\item Dòng 2: gồm $n$ số nguyên dương $a_1,a_2,\ldots,a_n$ tương ứng là giá cổ phiếu VNI trong từng ngày $a_i\in[10^9]$, $1\le i\le n$.
		\item Dòng 3: $q\in[10^5]$ là số lượng truy vấn.
		\item $q$ dòng tiếp theo, mỗi dòng gồm $t\in[n]$ thể hiện cho ngày đầu tiên mà Bình tham gia mua bán cổ phiếu VNI.
	\end{itemize}
	\item {\sf Output.} Ghi ra file {\tt vni.out} gồm $q$ dòng, mỗi dòng 1 số nguyên duy nhất là lợi nhuận lớn nhất mà Bình thu được ở mỗi truy vấn tương ứng.
	\item {\sf Limit.}
	\begin{itemize}
		\item Subtask 1: có 50\% số test tương ứng $n\in[10^3],q = 1$.
		\item Subtask 2: có 30\% số test tương ứng $n\in[10^5],q = 1$.
		\item Subtask 3: có 20\% số test còn lại không có ràng buộc gì thêm.
	\end{itemize}
	\item {\sf Sample.}
	\begin{table}[H]
		\centering
		\begin{tabular}{|l|l|}
			\hline
			{\tt flashback.inp} & {\tt flashback.out} \\
			\hline
			4 & 7 \\
			1 2 5 4 & 0 \\
			2 & \\
			1 & \\
			3 & \\
			\hline
		\end{tabular}
	\end{table}
\end{baitoan}

\begin{baitoan}[\cite{Trung_THCS_Tin}, BRVT HSG9 2022--2023, 2: đố vui tin học, p. 57]
	Để tổng kết phát thưởng cho cuộc thi Đố vui Tin học. Ban tổ chức có $n\in\mathbb{N}^\star$ phần quà được đánh số thứ tự từ $1$ đến $n$, phần quà thứ $i$ có giá trị là $a_i$. Ban tổ chức yêu cầu học sinh chọn các phần quà theo quy tắc:
	\begin{itemize}
		\item Phần quà chọn sau phải có số thứ tự lớn hơn phần quà chọn trước đó.
		\item Phần quà chọn sau phải có giá trị lớn hơn phần quà chọn trước đó ít nhất $k$ giá trị.
	\end{itemize}
	\item {\sf Requirement.} Giúp các học sinh lựa chọn theo quy tắc ban tổ chức đặt ra sao cho số lượng phần quà được chọn là nhiều nhất.
	\item {\sf Input.} Vào từ file {\tt gift.inp}:
	\begin{itemize}
		\item Dòng 1 chứa số bộ test $t\in\mathbb{N}^\star$.
		\item Với mỗi bộ test:
		\begin{itemize}
			\item Dòng 1 chứa $n\in[10^4],k\in[10^3]$.
			\item Dòng 2 chứa $n$ số nguyên dương $a_i\in[10^6]$ là giá trị của phần quà thứ $i$, $\forall i\in[n]$.
		\end{itemize}
	\end{itemize}
	\item {\sf Output.} Ghi ra file {\tt gift.out} 1 dòng duy nhất chứa số lượng phần quà nhiều nhất thỏa mãn yêu cầu của Ban tổ chức.
	\item {\sf Sample.}
	\begin{table}[H]
		\centering
		\begin{tabular}{|l|l|}
			\hline
			{\tt gift.inp} & {\tt gift.out} \\
			\hline
			4 & 3 \\
			5 2 & 5 \\
			4 5 6 4 8 & 4 \\
			10 2 & 3 \\
			4 3 6 5 7 6 9 10 8 12 & \\
			10 3 & \\
			4 3 6 5 7 6 9 10 8 12 & \\
			10 4 & \\
			4 3 6 5 7 6 9 10 8 12 & \\
			\hline
		\end{tabular}
	\end{table}
\end{baitoan}

\begin{baitoan}[\cite{Trung_THCS_Tin}, BRVT HSG9 2022--2023, 3: trò chơi, p. 58]
	Nhân dịp kỷ niệm ngày thành lập Đoàn, cô Tổng phụ trách tổ chức 1 trò chơi có thưởng cho các bạn lớp 9: Có $n\in\mathbb{N}^\star$ ô vuông được vẽ thẳng hàng trên sân trường, các ô vuông được đánh số thứ tự từ $1$ đến $n$. Mỗi ô vuông $i$ có giá trị năng lượng là $h_i$. 1 học sinh đang ở ô thứ $i$, bạn ấy có thể nhảy tới ô vuông tiếp theo theo các cách:
	\begin{itemize}
		\item Nếu đang ở ô thứ $i$, có thể nhảy đến ô vuông thứ tự $i + 1,i + 2,\ldots,i + k$.
		\item Chi phí năng lượng tiêu hao cho 1 lần nhảy là $|h_i - h_j|$ với $h_j$ là ô đích mà bạn nhảy tới.
	\end{itemize}
	Học sinh nào di chuyển từ ô $1$ đến ô $n$ với chi phí năng lượng thấp nhất sẽ được cô thưởng 1 phần quà.
	\item {\sf Yêu cầu.} Tìm chi phí thấp nhất để giúp các học sinh nhảy từ ô vuông thứ $1$ đến ô vuông thứ $n$.
	\item {\sf Input.} Vào từ file {\tt game.inp}:
	\begin{itemize}
		\item Dòng 1 chứa số bộ test $t\in\mathbb{N}^\star$.
		\item Với mỗi bộ test:
		\begin{itemize}
			\item Dòng 1 chứa $2\le n\le10^5,k\in[100]$, lần lượt là số ô vuông \& số ô vuông tối đa mà học sinh có thể nhảy qua.
			\item Dòng 2 chứa $n$ giá trị $h_i\in[10^4]$, mỗi số cách nhau 1 ký tự trắng là chi phí năng lượng của ô thứ $i$, $\forall i\in[n]$.
		\end{itemize}
	\end{itemize}
	\item {\sf Output.} Ghi ra file {\tt game.out} 1 số nguyên là chi phí năng lượng ít nhất.
	\item {\sf Sample.}
	\begin{table}[H]
		\centering
		\begin{tabular}{|l|l|}
			\hline
			{\tt game.inp} & {\tt game.out} \\
			\hline
			1 & 20 \\
			5 3 &  \\
			10 25 35 40 20 & \\
			\hline
		\end{tabular}
	\end{table}
\end{baitoan}

\begin{baitoan}[\cite{Trung_THCS_Tin}, Lâm Đồng HSG9 2022--2023, 4: biểu diễn văn nghệ, p. 82]
	Trong 1 chương trình nghệ thuật diễn ra liên tục trong $n\in\mathbb{N}^\star$ giờ, công ty X có danh sách $m\in\mathbb{N}^\star$ nghệ sĩ khác nhau có thể thuê để biểu diễn. Thời điểm bắt đầu biểu diễn được tính bằng $0$. Để đơn giản trong quản lý \& sắp xếp, các nghệ sĩ được đánh số thứ tự từ $1$ đến $m$, nghệ sĩ thứ $i\in[n]$ biểu diễn trong thời điểm $s_i$ đến thời điểm $t_i$, $0\le s_i < t_i\le n$, với tiền công là $c_i$, $0\le c_i\le10^6$.
	\item {\sf Requirement.} Viết chương trình thuê các nghệ sĩ để bất cứ thời điểm nào cũng luôn có ít nhất 1 nghệ sĩ biểu diễn đồng thời tổng chi phí thuê là nhỏ nhất.
	\item {\sf Input.} Vào từ file \verb|art_performance.inp| gồm:
	\begin{itemize}
		\item Dòng 1 chứa số bộ test $t\in\mathbb{N}^\star$.
		\item Với mỗi bộ test:
		\begin{itemize}
			\item Dòng đầu tiên chứa $n,m\in[400]$.
			\item $m$ dòng tiếp theo, mỗi dòng chứa 3 số nguyên không âm $s_i,t_i,c_i$.
		\end{itemize}
	\end{itemize}
	\item {\sf Output.} Ghi ra file \verb|art_performance.out| 1 số nguyên là chi phí thuê nhỏ nhất.
	\item {\sf Sample.}
	\begin{table}[H]
		\centering
		\begin{tabular}{|l|l|}
			\hline
			\verb|art_performance.inp| & \verb|art_performance.out| \\
			\hline
			1 & 20 \\
			9 5 &  \\
			0 5 25 & \\
			1 3 18 & \\
			3 7 21 & \\
			4 6 38 & \\
			7 9 20 & \\
			\hline
		\end{tabular}
	\end{table}
\end{baitoan}

\begin{baitoan}[\cite{Trung_THCS_Tin}, TS10 chuyên Tin Bình Dương 2023--2024, 2: ghép tranh, p. 93]
	Trò chơi thứ 2 của lớp 9A là trò chơi ghép tranh. Cô chủ nhiệm cho 1 bức tranh có $n\in\mathbb{N}^\star$ mảnh ghép \& $k\in\mathbb{N}^\star$, $1\le k\le n\le50$. Lần lượt từng tổ sẽ thay phiên nhau lên ghép tranh với số mảnh ghép sử dụng không vượt quá $k$. An nhận thấy phải tìm được tất cả các cách ghép tranh mới có thể chiến thắng trò chơi. Có thể có nhiều cách ghép tranh, 2 cách ghép khác nhau nếu tồn tại 1 cách ghép giúp hoàn thành được bức tranh \& bị bỏ qua ở cách kia.
	\item {\sf Requirement.} Giúp An xác định số cách ghép tranh khác nhau để tổ của An có thể ghép hoàn thành bức tranh.
	\item {\sf Input.} Vào từ file {\tt picture.inp} gồm:
	\begin{itemize}
		\item Dòng 1 chứa số bộ test $t\in\mathbb{N}^\star$.
		\item Mỗi bộ test gồm 1 dòng chứa $n,k\in\mathbb{N}^\star$.
	\end{itemize}
	\item {\sf Output.} Với mỗi bộ test, ghi ra file {\tt picture.out} 1 số nguyên duy nhất là số cách ghép tranh khác nhau.
	\begin{table}[H]
		\centering
		\begin{tabular}{|l|l|}
			\hline
			{\tt picture.inp} & {\tt picture.out} \\
			\hline
			1 & 7 \\
			4 3 & \\
			\hline
		\end{tabular}
	\end{table}
\end{baitoan}

\begin{baitoan}[\cite{Trung_THCS_Tin}, Vĩnh Phúc HSG9 2022--2023, 2: lật ký tự, p. 210]
	Cho xâu $S$ gồm $n\in\mathbb{N}^\star$ ký tự \verb|. #| được đánh số từ $1$ đến $n$. Thao tác lật ký tự của xâu được định nghĩa như sau:
	\begin{itemize}
		\item Chọn $i\in[n]$.
		\item Nếu ký tự thứ $i$ của xâu $S$ là \verb|.| thì nó sẽ được thay thế bằng \verb|#|. Ngược lại, nếu là \verb|#| thì sẽ được thay thế bằng \verb|.|.
	\end{itemize}
	\item {\sf Requirement.} Lập trình tính xem cần thực hiện ít nhất bao nhiêu thao tác để trong xâu không có ký tự \verb|.| nào ở ngay bên phải ký tự \verb|#|.
	\item {\sf Input.} Vào từ file {\tt pchar.inp}: Dòng 1 chứa số bộ test $t\in\mathbb{N}^\star$. Với mỗi bộ test:
	\begin{itemize}
		\item Dòng 1 ghi $n\in[200000]$ là số lượng ký tự xâu trong $S$.
		\item Dòng 2 ghi xâu $S$ gồm $n$ ký tự \verb|. #|.
	\end{itemize}
	\item {\sf Subtask.}
	\begin{itemize}
		\item Subtask 1: Ràng buộc $1\le n\le2000$
		\item Subtask 2: Không có ràng buộc bổ sung.
	\end{itemize}
	\item {\sf Output.} Ghi ra file {\tt pchar.out} 1 số nguyên duy nhất là số thao tác ít nhất cần thực hiện để xâu $S$ không có bất kỳ thứ tự \verb|.| nào ở ngay bên phải ký tự \verb|#|.
	\item {\sf Sample.}
	\begin{table}[H]
		\centering
		\begin{tabular}{|l|l|}
			\hline
			{\tt pchar.inp} & {\tt pchar.out} \\
			\hline
			3 & 1 \\
			3 & 2 \\
			\verb|#.#| & 0 \\
			5 & \\
			\verb|#.##.| & \\
			9 & \\
			\verb|.........| & \\
			\hline
		\end{tabular}
	\end{table}
\end{baitoan}

\begin{dinhnghia}[Dãy phần tử cực trị của 1 dãy số thực cho trước]
	Cho dãy số thực $\{a_i\}_{i=1}^n$. {\rm Dãy phần tử lớn nhất} của dãy số thực $\{a_i\}_{i=1}^n$ được định nghĩa bởi:
	\begin{equation*}
		a_i^{\max}\coloneqq\max_{i\le j\le n} a_j.
	\end{equation*}
	Tương tự, {\rm dãy phần tử nhỏ nhất} của dãy số thực $\{a_i\}_{i=1}^n$ được định nghĩa bởi:
	\begin{equation*}
		a_i^{\min}\coloneqq\min_{i\le j\le n} a_j.
	\end{equation*}
\end{dinhnghia}

\begin{baitoan}[\cite{Trung_HSG_THPT_Tin}, HSG12 Nam Định 2020--2021, 4: work, pp. 21--2]
	Trong 1 dây chuyền làm việc của công ty có $n$ công nhân làm $n$ việc. Người ta đánh số cho công nhân từ $1$ đến $n$ theo thứ tự đứng trong dây chuyền. Thời gian hoàn thành 1 công việc của người thứ $i$ là $t_i$ phút. Mỗi người cần làm xong công việc của mình nhưng được quyền làm tối đa $2$ việc. Vì thế họ có thể phối hợp với người đứng ngay trước mình cùng làm, nếu người thứ $i$ \& người thứ $i + 1$ phối hợp thì thời gian làm xong việc cho $2$ người là $p_i$.
\end{baitoan}

\begin{baitoan}[\cite{Thu_Phuong_Tien_Triet_Phuong_KTLT}, p. 446, tính $C_n^k$]
	(a) Viết chương trình {\sf C{\tt/}C++, Python} để tính $C_n^k$ -- số tổ hợp chập $k$ của $n$ phần tử bằng công thức truy hồi nhờ đẳng thức Pascal:
	\begin{equation*}
		C_n^k = \left\{\begin{split}
			&1&&\mbox{if } k = 0\lor k = n,\\
			&C_{n-1}^k + C_{n-1}^{k-1}&&\mbox{if } 0 < k < n,
		\end{split}\right.
	\end{equation*}
	với $n,k\in\mathbb{N}$, $0\le k\le n$, được nhập vào. (b) Có thể sử dụng mảng 1 chiều để lưu lại dòng trước đó mà không cần phải lưu lại tất cả.
\end{baitoan}

\begin{baitoan}[\cite{Thu_Phuong_Tien_Triet_Phuong_KTLT}, II.1, p. 447, tìm dãy con đơn điệu tăng dài nhất]
	Cho dãy số nguyên $a = \{a_i\}_{i=1}^n = a_1,\ldots,a_n$ gồm $n\in\mathbb{N}^\star$ phần tử. 1 dãy con của $a$ là 1 cách chọn trong $a$ 1 số phần tử giữ nguyên thứ tự (có tất cả $2^n$ dãy con của 1 dãy có $n$ phần tử). Tìm dãy con đơn điệu tăng (resp., giảm, không tăng, không giảm) của $a$ có độ dài lớn nhất.
\end{baitoan}

\begin{proof}[CS solution]
	Giả sử dãy ban đầu gồm $a[1],a[2],\ldots,a[n]$. Bổ sung vào $a$ 2 phần tử $a[0] = -\infty$ \& $a[n + 1] = \infty$ (khi viết chương trình $\pm\infty$ sẽ được cài đặt các giá trị thích hợp). Khi đó dãy con đơn điệu tăng dài nhất sẽ bắt đầu từ $a[0]$ \& kết thúc ở $a[n + 1]$. Đặt $l(i)$ là độ dài dãy con đơn điệu tăng dài nhất bắt đầu từ $a[i]$, $\forall i = 0,1,\ldots,n + 1$. Cần tính tất cả $l(i)$ này. Đáp số bài toán sẽ là dãy ứng với $l(i_0)$ có GTLN.
	
	{\sf Cơ sở quy hoạch động (bài toán nhỏ nhất).} Trường hợp đặc biệt, $l(n + 1)$ là độ dài dãy con đơn điệu tăng dài nhất bắt đầu tại $a[n + 1] = \infty$. Do dãy con này chỉ gồm 1 phần tử $\infty$ nên $l(n + 1) = 1$.
	
	{\sf Công thức truy hồi.} Cần tính $l(i)$ với $i = n,\ldots,1,0$. Giá trị $l(i)$ sẽ được tính trong điều kiện $l(i + 1),\ldots,l(n + 1)$ đã biết. Dãy con đơn điệu tăng dài nhất bắt đầu từ $a[i]$ sẽ được thành lập bằng cách lấy $a[i]$ ghép vào đầu 1 trong số các dãy con đơn điệu tăng dài nhất bắt đầu từ vị trí $a[j] > a[i]$ (để đảm bảo tính tăng) nào đó đứng sau $a[i]$ \& chọn dãy dài nhất trong số đó để ghép $a[i]$ vào đầu để đảm bảo tính dài nhất. Nên $l(i)$ được tính bằng cách xét tất cả các chỉ số $j = i + 1,\ldots,n + 1$ mà $a[j] > a[i]$, chọn ra chỉ số $j_{\max}$ có $l(j_{\max})$ lớn nhất:
	\begin{equation*}
		l(i) = l(j_{\max}) + 1 = \max\{l(j);i < j \le n + 1,\ a[i] < a[j]\} + 1.
	\end{equation*}
\end{proof}

\begin{baitoan}[\cite{Thu_Phuong_Tien_Triet_Phuong_KTLT}, II.1.4.1., p. 451, bố trí phòng họp]
	Có $n\in\mathbb{N}^\star$ cuộc họp, cuộc họp thứ $i$ bắt đầu vào thời điểm $s_i$ (start) \& kết thúc vào thời điểm $f_i$ (final). Do chỉ có 1 phòng hội thảo nên 2 cuộc họp bất kỳ sẽ được cùng bố trí phục vụ nếu khoảng thời gian làm việc của chúng chỉ giao nhau tại 1 đầu mút. Bố trí phòng họp để phục vụ được nhiều cuộc họp nhất.
\end{baitoan}

\begin{baitoan}[\cite{Thu_Phuong_Tien_Triet_Phuong_KTLT}, II.1.4.2., p. 451, cho thuê máy]
	Trung tâm tính toán hiệu năng cao nhận được đơn đặt hàng của $n\in\mathbb{N}^\star$ khách hàng. Khách hàng $i$ muốn sử dụng máy trong khoảng thời gian từ $s_i$ (start) đến $f_i$ (final) \& thả tiền thuê là $c_i$. Bố trí lịch thuê máy để tổng số tiền thu được là lớn nhất mà thời gian sử dụng máy của 2 khách bất kỳ được phục vụ đều không giao nhau.
\end{baitoan}

\begin{baitoan}[\cite{Thu_Phuong_Tien_Triet_Phuong_KTLT}, II.1.4.3., p. 451, dãy tam giác bao nhau]
	Cho $n\in\mathbb{N}^\star$ tam giác trên mặt phẳng. Tam giác $i$ bao tam giác $j$ nếu 3 đỉnh của tam giác $j$ đều nằm trong tam giác $i$ (có thể nằm trên cạnh). Tìm dãy tam giác bao nhau có nhiều tam giác nhất.
\end{baitoan}

\begin{problem}[\href{https://cses.fi/problemset/task/1633}{CSES Problem Set{\tt/}dice combinations}]
	Count the number of ways to construct sum $n\in\mathbb{N}^\star$ by throwing a dice 1 or more times. Each throw produces an outcome between $1$ \& $6$. E.g., if $n = 3$, there are $4$ ways: $3 = 1 + 1 + 1 = 1 + 2 = 2 + 1 = 3$.
	\item {\sf Input.} The only input line has an integer $n\in\mathbb{N}^\star$.
	\item {\sf Output.} Print the number of ways module $10^9 + 7$.
	\item {\sf Constraints.} $n\in[10^6]$.
\end{problem}

\begin{problem}[\href{https://cses.fi/problemset/task/1634}{CSES Problem Set{\tt/}minimizing coins}]
	Consider a money system consisting of $n$ coins. Each coin has a positive integer value. Your task is to produce a sum of money $x$ using the available coins in such a way that the number of coins is minimal. E.g., if the coins are $\{1,5,7\}$ \& the desired sum is $11$, an optimal solution is $5 + 5 + 1$ which requires $3$ coins.
	\item {\sf Input.} The 1st input line has 2 integer $n,x\in\mathbb{N}^\star$: the number of coins \& the desired sum of money. The 2nd line has $n$ distinct integers $\{c_i\}_{i=1}^n = c_1,c_2,\ldots,c_n$: the value of each coin.
	\item {\sf Output.} Print 1 integer: the minimum number of coins. If it is not possible to produce the desired sum, print {\tt-1}.
	\item {\sf Constraints.} $n\in[100],x\in[10^6],c_i\in[10^6]$.
	\item {\sf Sample.}
	\begin{table}[H]
		\centering
		\begin{tabular}{|l|l|}
			\hline
			\verb|minimizing_coin.inp| & \verb|minimizing_coin.out| \\
			\hline
			3 11 & 3 \\
			1 5 7 & \\
			\hline
		\end{tabular}
	\end{table}
\end{problem}

\begin{problem}[\href{https://cses.fi/problemset/task/1635}{CSES Problem Set{\tt/}coin combinations I}]
	Consider a money system consisting of $n$ coins. Each coin has a positive integer value. Calculate the number of distinct ways you can produce a money sum $x$ using the available coins. E.g., if the coins are $\{2,3,5\}$ \& the desired sum is $9$, there are $8$ ways: $8 = 2 + 2 + 5 = 2 + 5 + 2 = 5 + 2 + 2 = 3 + 3 + 3 = 2 + 2 + 2 + 3 = 2 + 2 + 3 + 2 = 2 + 3 + 2 + 2 = 3 + 2 + 2 + 2$.
	\item {\sf Input.} The 1st input line has 2 integers $n,x\in\mathbb{N}^\star$: the number of coins \& the desired sum of money. The 2nd line has $n$ distinct integers $\{c_i\}_{i=1}^n = c_1,c_2,\ldots,c_n$: the value of each coin.
	\item {\sf Output.} Print 1 integer: the number of ways module $10^9 + 7$.
	\item {\sf Constraints.} $n\in[100],x\in[10^6],c_i\in[10^6]$.
	\item {\sf Sample.}
	\begin{table}[H]
		\centering
		\begin{tabular}{|l|l|}
			\hline
			\verb|coin_combination_I.inp| & \verb|coin_combination_I.out| \\
			\hline
			3 9 & 8 \\
			2 3 5 & \\
			\hline
		\end{tabular}
	\end{table}
\end{problem}

\begin{problem}[\href{https://cses.fi/problemset/task/1636}{CSES Problem Set{\tt/}coin combinations II}]
	Consider a money system consisting of $n$ coins. Each coin has a positive integer value. Calculate the number of distinct ordered ways you can produce a money sum $x$ using the available coins. E.g., if the coins are $\{2,3,5\}$ \& the desired sum is $9$, there are $3$ ways: $8 = 2 + 2 + 5 = 3 + 3 + 3 = 2 + 2 + 2 + 3$.
	\item {\sf Input.} The 1st input line has 2 integers $n,x\in\mathbb{N}^\star$: the number of coins \& the desired sum of money. The 2nd line has $n$ distinct integers $\{c_i\}_{i=1}^n = c_1,c_2,\ldots,c_n$: the value of each coin.
	\item {\sf Output.} Print 1 integer: the number of ways module $10^9 + 7$.
	\item {\sf Constraints.} $n\in[100],x\in[10^6],c_i\in[10^6]$.
	\item {\sf Sample.}
	\begin{table}[H]
		\centering
		\begin{tabular}{|l|l|}
			\hline
			\verb|coin_combination_II.inp| & \verb|coin_combination_II.out| \\
			\hline
			3 9 & 3 \\
			2 3 5 & \\
			\hline
		\end{tabular}
	\end{table}
\end{problem}

\begin{problem}[\href{https://cses.fi/problemset/task/1637}{CSES Problem Set{\tt/}removing digits}]
	You are given an integer $n\in\mathbb{N}^\star$. On each step, you may subtract 1 of the digits from the number. How many steps are required to make the number equal to $0$?
	\item {\sf Input.} The only input line has an integer $n\in\mathbb{N}^\star$.
	\item {\sf Output.} Print 1 integer: the minimum number of steps.
	\item {\sf Constraints.} $n\in[10^6]$.
	\item {\sf Sample.}
	\begin{table}[H]
		\centering
		\begin{tabular}{|l|l|}
			\hline
			\verb|removing_digit.inp| & \verb|removing_digit.out| \\
			\hline
			27 & 5 \\
			\hline
		\end{tabular}
	\end{table}
	Explanation: An optimal solution is $27\to20\to18\to10\to9\to0$.
\end{problem}

\begin{problem}[\href{https://cses.fi/problemset/task/1638}{CSES Problem Set{\tt/}grid paths I}]
	Consider an $n\times n$ grid whose squares may have traps. It is not allowed to move to a square with a trap. Calculate the number of paths from the upper-left square to the lower-right square. You can only move right or down.
	\item {\sf Input.} The 1st input line has an integer $n\in\mathbb{N}^\star$: the size of the grid. After this, there are $n$ lines that describe the grid. Each line has $n$ character: {\tt.} denotes an empty cell, \& {\tt*} denotes a trap.
	\item {\sf Output.} Print the number of paths modulo $10^9 + 7$.
	\item {\sf Constraints.} $n\in[10^3]$.
	\item {\sf Sample.}
	\begin{table}[H]
		\centering
		\begin{tabular}{|l|l|}
			\hline
			\verb|grid_path_I.inp| & \verb|grid_path_I.out| \\
			\hline
			4 & 3 \\
			.... & \\
			.*.. & \\
			...* & \\
			*... & \\
			\hline
		\end{tabular}
	\end{table}
\end{problem}

\begin{problem}[\href{https://cses.fi/problemset/task/1158}{CSES Problem Set{\tt/}book shop}]
	You are in a book shop which sells $n\in\mathbb{N}^\star$ different books. You know the price \& number of pages of each book. You have decided that the total price of your purchases will be at most $x$. What is the maximum number of pages you can buy? You can buy each book at most once.
	\item {\sf Input.} The 1st input line contains 2 integers $n,x\in\mathbb{N}^\star$: the number of books \& the maximum total price. The next line contains $n$ integers $h_1,h_2,\ldots,h_n$: the price of each book. The last line contains $n$ integers $s_1,s_2,\ldots,s_n$: the number of pages of each book.
	\item {\sf Output.} Print 1 integer: the maximum number of pages.
	\item {\sf Constraints.} $n\in[10^3],x\in[10^5],h_i,s_i\in[10^3]$, $\forall i\in[n]$.
	\item {\sf Sample.}
	\begin{table}[H]
		\centering
		\begin{tabular}{|l|l|}
			\hline
			\verb|book_shop.inp| & \verb|book_shop.out| \\
			\hline
			4 10 & 13 \\
			4 8 5 3 & \\
			5 12 8 1 & \\
			\hline
		\end{tabular}
	\end{table}
\end{problem}

\begin{problem}[\href{https://cses.fi/problemset/task/1746}{CSES Problem Set{\tt/}array description}]
	You know that an array has $n\in\mathbb{N}^\star$ integers in $[m]$, \& the absolute difference between 2 adjacent values is at most $1$. Given a description of the array where some values may be unknown, count the number of arrays that match the description.
	\item {\sf Input.} The 1st input line has integers $n,m\in\mathbb{N}^\star$: the array size \& the upper bound for each value. The next line has $n$ integers $x_1,x_2,\ldots,x_n\in\mathbb{N}^\star$: the contents of the array. Value $0$ denotes an unknown value.
	\item {\sf Output.} Print 1 integer: the number of arrays module $10^9 + 7$.
	\item {\sf Constraints.} $n\in[10^5],m\in[100],x_i\in\{0,1,\ldots,m\}$, $\forall i\in[n]$.
	\item {\sf Sample.}
	\begin{table}[H]
		\centering
		\begin{tabular}{|l|l|}
			\hline
			\verb|array_description.inp| & \verb|array_description.out| \\
			\hline
			3 5 & 3 \\
			2 0 2 & \\
			\hline
		\end{tabular}
	\end{table}
	Explanation: The arrays $[2,1,2],[2,2,2],[2,3,2]$ match the description.
\end{problem}

\begin{problem}[\href{https://cses.fi/problemset/task/2413}{CSES Problem Set{\tt/}counting towers}]
	Build a tower whose width is $2$ \& height is $n$. You have an unlimited supply of blocks whose width \& height are integers. Given $n$, how many different towers can you build? Mirrored \& rotated towers are counted separately if they look different.
	\item {\sf Input.} The 1st input line has integers $t\in\mathbb{N}^\star$: the number of tests. After this, there are $t$ lines, \& each line contains an integer $n\in\mathbb{N}^\star$: the height of the tower.
	\item {\sf Output.} For each test, print the number of towers module $10^9 + 7$.
	\item {\sf Constraints.} $t\in[100],n\in[10^6]$.
	\item {\sf Sample.}
	\begin{table}[H]
		\centering
		\begin{tabular}{|l|l|}
			\hline
			\verb|counting_tower.inp| & \verb|counting_tower.out| \\
			\hline
			3 & 8 \\
			2 & 2864 \\
			6 & 640403945 \\
			1337 & \\
			\hline
		\end{tabular}
	\end{table}
\end{problem}

\begin{problem}[\href{https://cses.fi/problemset/task/1639}{CSES Problem Set{\tt/}edit distance}]
	The {\rm edit distance} between 2 strings is the minimum number of operations required to transform 1 string into the other. The allowed operations are:
	\begin{itemize}
		\item Add 1 character to the string.
		\item Remove 1 character from the string.
		\item Replace 1 character in the string.
	\end{itemize}
	E.g., the edit distance between {\tt LOVE} \& {\tt MOVIE} is $2$, because you can 1st replace {\tt L} with {\tt M}, \& then add {\tt I}. Calculate the edit distance between 2 strings.
	\item {\sf Input.} The 1st input line has a string that contains $n\in\mathbb{N}^\star$ characters between {\tt A--Z}. The 2nd input line has a string that contains $m\in\mathbb{N}^\star$ characters between {\tt A--Z}.
	\item {\sf Output.} Print 1 integer: the edit distance between the strings.
	\item {\sf Constraints.} $m,n\in[5000]$.
	\item {\sf Sample.}
	\begin{table}[H]
		\centering
		\begin{tabular}{|l|l|}
			\hline
			\verb|edit_distance.inp| & \verb|edit_distance.out| \\
			\hline
			LOVE & 2 \\
			MOVIE & \\
			\hline
		\end{tabular}
	\end{table}
\end{problem}

\begin{problem}[\href{https://cses.fi/problemset/task/3403}{CSES Problem Set{\tt/}longest common subsequence}]
	Given 2 arrays of integers, find their longest common subsequence. A subsequence is a sequence of array elements from left to right that can contain gaps. A common subsequence is a subsequence that appears in both arrays.
	\item {\sf Input.} The 1st input line has 2 integers $n,m\in\mathbb{N}^\star$: the size of the arrays. The 2nd line has $n$ integers $a_1,a_2,\ldots,a_n$: the contents of the 1st array. The 3rd line has $m$ integers $b_1,b_2,\ldots,b_m$: the contents of the 2nd array.
	\item {\sf Output.} 1st print the length of the longest common subsequence. After that, print an example of such a sequence. If there are several solutions, you can print any of them.
	\item {\sf Constraints.} $m,n\in[10^3],a_i,b_i\in[10^9]$, $\forall i\in[n]$.
	\item {\sf Sample.}
	\begin{table}[H]
		\centering
		\begin{tabular}{|l|l|}
			\hline
			\verb|longest_common_subsequence.inp| & \verb|longest_common_subsequence.out| \\
			\hline
			8 6 & 3 \\
			3 1 3 2 7 4 8 2 & 1 2 4 \\
			6 5 1 2 3 4 & \\
			\hline
		\end{tabular}
	\end{table}
\end{problem}

\begin{problem}[\href{https://cses.fi/problemset/task/1744}{CSES Problem Set{\tt/}rectangle cutting}]
	Given an $a\times b$ rectangle, cut it into squares. On each move, you can select a rectangle \& cut it into 2 rectangles in such a way that all side lengths remain integers. What is the minimum possible number of moves?
	\item {\sf Input.} The only input line has 2 integers $a,b\in\mathbb{N}^\star$: 
	\item {\sf Output.} Print 1 integer: the minimum number of moves.
	\item {\sf Constraints.} $a,b\in[500]$.
	\item {\sf Sample.}
	\begin{table}[H]
		\centering
		\begin{tabular}{|l|l|}
			\hline
			\verb|rectangle_cutting.inp| & \verb|rectangle_cutting.out| \\
			\hline
			3 5 & 3 \\
			\hline
		\end{tabular}
	\end{table}
\end{problem}

\begin{problem}[\href{https://cses.fi/problemset/task/3359}{CSES Problem Set{\tt/}minimal grid path}]
	You are given an $n\times n$ grid whose each square contains a letter. You should move from the upper-left square to the lower-right square. You can only move right or down. What is the lexicographically minimal string you can construct?
	\item {\sf Input.} The 1st input line has integers $n,m\in\mathbb{N}^\star$: the size of the grid. After this, there are $n$ lines that describe the grid. Each line has $n$ letters betweeen {\tt A--Z}.
	\item {\sf Output.} Print the lexicographically minimal string.
	\item {\sf Constraints.} $n\in[3000]$.
	\item {\sf Sample.}
	\begin{table}[H]
		\centering
		\begin{tabular}{|l|l|}
			\hline
			\verb|minimal_grid_path.inp| & \verb|minimal_grid_path.out| \\
			\hline
			4 & AAABACA \\
			AACA & \\
			BABC & \\
			ABDA & \\
			AACA & \\
			\hline
		\end{tabular}
	\end{table}
\end{problem}

\begin{problem}[\href{https://cses.fi/problemset/task/1745}{CSES Problem Set{\tt/}money sums}]
	You have $n$ coins with certain values. Find all money sums you can create using these coins.
	\item {\sf Input.} The 1st input line has integers $n,m\in\mathbb{N}^\star$: the number of coins. The next line has $n$ integers $x_1,x_2,\ldots,x_n$: the values of the coins.
	\item {\sf Output.} 1st print an integer $k\in\mathbb{N}^\star$: the number of distinct money sums. After this, print all possible sums in increasing order.
	\item {\sf Constraints.} $n\in[100],x_i\in[10^3]$, $\forall i\in[n]$.
	\item {\sf Sample.}
	\begin{table}[H]
		\centering
		\begin{tabular}{|l|l|}
			\hline
			\verb|money_sum.inp| & \verb|money_sum.out| \\
			\hline
			4 & 9 \\
			4 2 5 2 & 2 4 5 6 7 8 9 11 13 \\
			\hline
		\end{tabular}
	\end{table}
\end{problem}

\begin{problem}[\href{https://cses.fi/problemset/task/1097}{CSES Problem Set{\tt/}removal game}]
	There is a list of $n\in\mathbb{N}^\star$ numbers \& 2 players who move alternately. On each move, a player removes either the 1st or last number from the list, \& their score increases by that number. Both players try to maximize their scores. What is the maximum possible score for the 1st player when both players play optimally?
	\item {\sf Input.} The 1st input line has integers $n,m\in\mathbb{N}^\star$: the size of the list. The next line has $n$ integers $x_1,x_2,\ldots,x_n$: the contents of the list.
	\item {\sf Output.} Print the maximum possible score for the 1st player.
	\item {\sf Constraints.} $n\in[5000],m\in[100],x_i\in\overline{-10^9,10^9}$, $\forall i\in[n]$.
	\item {\sf Sample.}
	\begin{table}[H]
		\centering
		\begin{tabular}{|l|l|}
			\hline
			\verb|removal_game.inp| & \verb|removal_game.out| \\
			\hline
			4 & 8 \\
			4 5 1 3 & \\
			\hline
		\end{tabular}
	\end{table}
\end{problem}

\begin{problem}[\href{https://cses.fi/problemset/task/1093}{CSES Problem Set{\tt/}2 sets II}]
	Count the number of ways numbers $[n]$ can be divided into 2 sets of equal sum. E.g., if $n = 7$, there are $4$ solutions: $\{1,3,4,6\}$ \& $\{2,5,7\}$, $\{1,2,5,6\}$ \& $\{3,4,7\}$, $\{1,2,4,7\}$ \& $\{3,5,6\}$, $\{1,6,7\}$ \& $\{2,3,4,4\}$.
	\item {\sf Input.} The only input line contains an integer $n\in\mathbb{N}^\star$: 
	\item {\sf Output.} Print the answer modulo $10^9 + 7$.
	\item {\sf Constraints.} $n\in[500]$.
	\item {\sf Sample.}
	\begin{table}[H]
		\centering
		\begin{tabular}{|l|l|}
			\hline
			\verb|two_sets_II.inp| & \verb|two_sets_II.out| \\
			\hline
			7 & 4 \\
			\hline
		\end{tabular}
	\end{table}
\end{problem}

\begin{problem}[\href{https://cses.fi/problemset/task/3314}{CSES Problem Set{\tt/}mountain range}]
	There are $n\in\mathbb{N}^\star$ mountains in a row, each with a specific height. You begin your hang gliding route from some mountain. You can glide from mountain $a$ to mountain $b$ if mountain $a$ is taller than mountain $b$ \& all mountains between $a$ \& $b$. What is the maximum number of mountains you can visit on your route?
	\item {\sf Input.} The 1st input line has an integer $n\in\mathbb{N}^\star$: the number of mountains. The next line has $n$ integers $h_1,h_2,\ldots,h_n$: the heights of the mountains.
	\item {\sf Output.} Print 1 integer: the maximum number of mountains.
	\item {\sf Constraints.} $n\in[2\cdot10^5],h_i\in[10^9]$, $\forall i\in[n]$.
	\item {\sf Sample.}
	\begin{table}[H]
		\centering
		\begin{tabular}{|l|l|}
			\hline
			\verb|.inp| & \verb|.out| \\
			\hline
			10 & 5 \\
			20 15 17 35 25 40 12 19 13 12 & \\
			\hline
		\end{tabular}
	\end{table}
\end{problem}

\begin{problem}[\href{https://cses.fi/problemset/task/1145}{CSES Problem Set{\tt/}increasing subsequence}]
	You are given an array containing $n\in\mathbb{N}^\star$ integers. Determine the longest increasing subsequence in the array, i.e., the longest subsequence where every element is larger than the previous one. A subsequence is a sequence that can be derived from the array by deleting some elements without changing the order of the remaining elements.
	\item {\sf Input.} The 1st input line has an integer $n\in\mathbb{N}^\star$: the size of the array. After this there are $n$ integers $x_1,x_2,\ldots,x_n$: the contents of the array.
	\item {\sf Output.} Print the length of the longest increasing subsequence.
	\item {\sf Constraints.} $n\in[2\cdot10^5],x_i\in[10^9]$, $\forall i\in[n]$.
	\item {\sf Sample.}
	\begin{table}[H]
		\centering
		\begin{tabular}{|l|l|}
			\hline
			\verb|increasing_subsequence.inp| & \verb|increasing_subsequence.out| \\
			\hline
			8 & 4 \\
			7 3 5 3 6 2 9 8 & \\
			\hline
		\end{tabular}
	\end{table}
\end{problem}

\begin{problem}[\href{https://cses.fi/problemset/task/1140}{CSES Problem Set{\tt/}projects}]
	There are $n\in\mathbb{N}^\star$ you can attend. For each project, you know its starting \& ending days \& the amount of money you would get as reward. You can only attend 1 project during a day. What is the maximum amount of money you can earn?
	\item {\sf Input.} The 1st input line has an integer $n\in\mathbb{N}^\star$: the number of projects. After this, there are $n$ lines. Each such line has 3 integers $a_i,b_i,p_i\in\mathbb{N}^\star$: the starting day, the ending day, \& the reward.
	\item {\sf Output.} Print 1 integer: the maximum amount of money you can earn.
	\item {\sf Constraints.} $n\in[2\cdot10^5],a_i,b_i,p_i[10^9]$, $a_i\le b_i$, $\forall i\in[n]$.
	\item {\sf Sample.}
	\begin{table}[H]
		\centering
		\begin{tabular}{|l|l|}
			\hline
			\verb|project.inp| & \verb|project.out| \\
			\hline
			4 & 7 \\
			2 4 4 & \\
			3 6 6 & \\
			6 8 2 & \\
			5 7 3 & \\
			\hline
		\end{tabular}
	\end{table}
\end{problem}

\begin{problem}[\href{https://cses.fi/problemset/task/1653}{CSES Problem Set{\tt/}elevator rides}]
	There are $n\in\mathbb{N}^\star$ people who want to get to the top of a building which has only 1 elevator. You know the weight of each person \& the maximum allowed weight in the elevator. What is the minimum number of elevator rides?
	\item {\sf Input.} The 1st input line has 2 integers $n,x\in\mathbb{N}^\star$: the number of people \& the maximum allowed weight in the elevator. The 2nd line has $n$ integers $w_1,w_2,\ldots,w_n$: the weight of each person.
	\item {\sf Output.} Print 1 integer: the minimum number of rides.
	\item {\sf Constraints.} $n\in[20],x\in[10^9],w_i\in[x]$.
	\item {\sf Sample.}
	\begin{table}[H]
		\centering
		\begin{tabular}{|l|l|}
			\hline
			\verb|elevator_ride.inp| & \verb|elevator_ride.out| \\
			\hline
			4 10 & 2 \\
			4 8 6 1 & \\
			\hline
		\end{tabular}
	\end{table}
\end{problem}

\begin{problem}[\href{https://cses.fi/problemset/task/2181}{CSES Problem Set{\tt/}counting tilings}]
	Count the number of ways you can fill an $n\times m$ grid using $1\times2$ \& $2\times1$ tiles.
	\item {\sf Input.} The 1st input line has 2 integers $n,m\in\mathbb{N}^\star$: 
	\item {\sf Output.} Print 1 integer: the number of ways module $10^9 + 7$.
	\item {\sf Constraints.} $n\in[10],m\in[1000]$.
	\item {\sf Sample.}
	\begin{table}[H]
		\centering
		\begin{tabular}{|l|l|}
			\hline
			\verb|counting_tiling.inp| & \verb|counting_tiling.out| \\
			\hline
			4 7 & 781 \\
			\hline
		\end{tabular}
	\end{table}
\end{problem}

\begin{problem}[\href{https://cses.fi/problemset/task/2220}{CSES Problem Set{\tt/}counting numbers}]
	Count the number of integers between $a,b\in\mathbb{N}\star$ where no 2 adjacent digits are the same.
	\item {\sf Input.} The 1st input line has 2 integers $a,b\in\mathbb{N}^\star$: 
	\item {\sf Output.} Print $1$ integer: the answer to the problem.
	\item {\sf Constraints.} $0\le a\le b\le10^{18}$.
	\item {\sf Sample.}
	\begin{table}[H]
		\centering
		\begin{tabular}{|l|l|}
			\hline
			\verb|counting_number.inp| & \verb|counting_number.out| \\
			\hline
			123 & 321 \\
			\hline
		\end{tabular}
	\end{table}
\end{problem}

\begin{problem}[\href{https://cses.fi/problemset/task/1748}{CSES Problem Set{\tt/}increasing subsequence II}]
	Given an array of $n\in\mathbb{N}^\star$ integers, calculate the number of increasing subsequences it contains. If 2 subsequences have the same values but in different positions in the array, they are counted separately.
	\item {\sf Input.} The 1st input line has an integer $n\in\mathbb{N}^\star$: the size of the array. The 2nd line has $n$ integers $x_1,x_2,\ldots,x_n$: the contents of the array.
	\item {\sf Output.} Print $1$ integer: the number of increasing subsequences module $10^9 + 7$.
	\item {\sf Constraints.} $n\in[2\cdot10^5],x_i\in[10^9]$.
	\item {\sf Sample.}
	\begin{table}[H]
		\centering
		\begin{tabular}{|l|l|}
			\hline
			\verb|increasing_subsequence_II.inp| & \verb|increasing_subsequence_II.out| \\
			\hline
			3 & 5 \\
			2 1 3 & \\
			\hline
		\end{tabular}
	\end{table}
	Explanation: The increasing subsequences are $[2],[1],[3],[2,3],[1,3]$.
\end{problem}

\begin{baitoan}[\href{https://oj.vnoi.info/problem/vncut}{VNOI{\tt/}cắt hình chữ nhật}]
    Người ta dùng máy cắt để cắt 1 hình chữ nhật kích thước $m\times n$, với $m,n\in[5000]$, thành 1 số ít nhất các hình vuông có kích thước nguyên dương \& có các cạnh song song với cạnh hình chữ nhật ban đầu. Máy cắt khi cắt luôn cắt theo phương song song với 1 trong 2 cạnh của hình chữ nhật \& chia hình chữ nhạt thành 2 phần.
    \item {\sf Input.} Gồm 2 số là kích thước $m,n$ cách nhau bởi dấu cách.
    \item {\sf Output.} Ghi số $k$ là số hình vuông nhỏ nhất được tạo ra.
    \item {\sf Sample.}
    \begin{table}[H]
        \centering
        \begin{tabular}{|l|l|}
            \hline
            \verb|cut_rectangle.inp| & \verb|cut_rectangle.out| \\
            \hline
            5 6 & 5 \\
            \hline
        \end{tabular}
    \end{table}
\end{baitoan}

%------------------------------------------------------------------------------%

\section{Graph Algorithms -- Thuật Toán Đồ Thị}

\begin{problem}[\href{https://cses.fi/problemset/task/1192}{CSES Problem Set{\tt/}counting rooms}]
	You are given a map of a building, \& your task is to count the number of its rooms. The size of the map is $n\times m$ squares, \& each square is either floor or wall. You can walk left, right, up, \& down through the floor squares.
	\item {\sf Input.} The 1st input lines has 2 integers $n,m$: the height \& width of the map. Then there are $n$ lines of $m$ characters describing the map. Each character is either . (floor) or {\tt\#} (wall).
	\item {\it Output.} Print 1 integer: the number of rooms.
	\item {\sf Constraints.} $1\le n,m\le10^3$.
	\item {\sf Sample.} Input:
	\begin{verbatim}
		5 8
		########
		#..#...#
		####.#.#
		#..#...#
		########		
	\end{verbatim}
	Output: {\tt3}.
\end{problem}

\begin{problem}[\href{https://cses.fi/problemset/task/1193}{CSES Problem Set{\tt/}labyrinth}]
	You are given a map of a labyrinth, task: find a path from start to end. You can walk left, right, up, \& down.
	\item {\sf Input.} The 1st input line has 2 integers $n,m$: the height \& width of the map. Then there are $n$ lines $m$ characters describing the labyrinth. Each character is . (floor), {\tt\#} (wall), {\tt A} (start), or {\tt B} (end). There is exactly 1 {\tt A} \& 1 {\tt B} in the input.
	\item {\sf Output.} 1st print {\tt YES}, if there is a path, \& {\tt No} otherwise. If there is a path, print the length of the shortest such path \& its description as a string consisting of characters {\tt L} (left), {\tt R} (right), {\tt U} (up), \& {\tt D} (down). You can print any valid solution.
	\item {\sf Constraints.} $1\le n,m\le1000$.
	\item {\sf Sample.} Input:
	\begin{verbatim}
		5 8
		########
		#.A#...#
		#.##.#B#
		#......#
		########
	\end{verbatim}
	Output:
	\begin{verbatim}
		YES
		9
		LDDRRRRRU
	\end{verbatim}
\end{problem}

\begin{problem}[\href{https://cses.fi/problemset/task/1666}{CSES Problem Set{\tt/}building roads}]
	Byteland has $n$ cities, \& $m$ roads between them. Goal: construct new roads so that there is a route between any 2 cities. Task: find out the minimum number of roads required, \& also determine which roads should be built.
	\item {\sf Input.} The 1st input line has 2 integers $n,m$: the number of cities \& roads. The cities are numbered $1,2,\ldots,n$. After that, there are $m$ lines describing the roads. Each lines has 2 integers $a,b$: there is a road between those cities. A road always connects 2 different cities, \& there is at most 1 road between any 2 cities.
	\item {\sf Output.} 1st print an integer $k$: the number of required roads. Then, print $k$ lines that describe the new roads. You can print any valid solution.
	\item {\sf Constraints.} $1\le n\le10^5,1\le m\le2\cdot10^5,1\le a,b\le n$.
	\item {\sf Sample.}
	\begin{table}[H]
		\centering
		\begin{tabular}{|l|l|}
			\hline
			\verb|build_road.inp| & \verb|build_road.out| \\
			\hline
			4 2 & 1 \\
			1 2 & 2 3 \\
			3 4 &  \\
			\hline
		\end{tabular}
	\end{table}
\end{problem}

\begin{problem}[{https://cses.fi/problemset/task/1667}{CSES Problem Set{\tt/}message route}]
	Syrjälä's network has $n$ computers \& $m$ connections. Task: find out if Uolevi can send a message to Maija, \& if it is possible, what is the minimum number of computers on such a route.
	\item {\sf Input.} The 1st input line has 2 integers $n,m$: the number of computers \& connections. The computers are numbered $1,2,\ldots,n$. Uolevi's computer is $1$ \& Maija's computer is $n$. Then, there are $m$ lines describing the connections. Each line has 2 integers $a,b$: there is a connection between those computers. Every connection is between 2 different computers, \& there is at most 1 connection between any 2 computers.
	\item {\sf Output.} If it is possible to send a message, 1st print $k$: the minimum number of computers on a valid route. After this, print an example of such a route. You can print any valid solution. If there are no routes, print {\tt IMPOSSIBLE}.
	\item {\sf Constraints.} $2\le n\le10^5,1\le m\le2\cdot10^5,1\le a,b\le n$.
	\item {\sf Sample.}
	\begin{table}[H]
		\centering
		\begin{tabular}{|l|l|}
			\hline
			\verb|message_route.inp| & \verb|message_route.out| \\
			\hline
			5 5 & 3 \\
			1 2 & 1 4 5 \\
			1 3 & \\
			1 4 & \\
			2 3 & \\
			5 4 & \\
			\hline
		\end{tabular}
	\end{table}
\end{problem}

\begin{problem}[\href{https://cses.fi/problemset/task/1668}{CSES Problem Set{\tt/}building team}]
	There are $n$ pupils in Uolevi's class, \& $m$ friendships between them. Task: divide pupils into 2 teams in such a way that no 2 pupils in a team are friends. You can freely choose the sizes of the teams.
	\item {\sf Input.} The 1st input line has 2 integers $n,m$: the number of pupils \& friendships. The pupils are numbered $1,2,\ldots,n$. Then, there are $m$ lines describing the friendships. Each line has 2 integers $a,b$: pupils $a,b$ are friends. Every friendship is between 2 different pupils. You can assume that there is at most 1 friendship between any 2 pupils.
	\item {\sf Output.} Print an example of how to build the teams. For each pupil, print {\tt1} or {\tt2} depending on to which team the pupil will be assigned. You can print any valid team. If there are no solutions, print {\tt IMPOSSIBLE}.
	\item {\sf Constraints.} $1\le n\le10^5,1\le m\le2\cdot10^5,1\le a,b\le n$.
	\item {\sf Sample.}
	\begin{table}[H]
		\centering
		\begin{tabular}{|l|l|}
			\hline
			\verb|build_team.inp| & \verb|build_team.out| \\
			\hline
			5 3 & 1 2 2 1 2 \\
			1 2 & \\
			1 3 & \\
			4 5 & \\
			\hline
		\end{tabular}
	\end{table}
\end{problem}

\begin{problem}[\href{https://cses.fi/problemset/task/1669}{CSES Problem Set{\tt/}round trip}]
	Byteland has $n$ cities \& $m$ roads between them. Task: design a round trip that begins in a city, goes through 2 or more other cities, \& finally returns to starting city. Every intermediate city on the route has to be distinct.
	\item {\sf Input.} The 1st input line has 2 integers $n,m$: the number of cities \& roads. The cities are numbered $1,2,\ldots,n$. Then, there are $m$ lines describing the roads. Each line has 2 integers $a,b$: there is a road between those cities. Every road is between 2 different cities, \& there is at most 1 road between any 2 cities.
	\item {\sf Output.} 1st print an integer $k$: the number of cities on the route. Then print $k$ cities in order they will be visited. You can print any valid solution. If there are no solutions, print {\tt IMPOSSIBLE}.
	\item {\sf Constraints.} $1\le n\le10^5,1\le m\le2\cdot10^5,1\le a,b\le n$.
	\item {\sf Sample.}
	\begin{table}[H]
		\centering
		\begin{tabular}{|l|l|}
			\hline
			\verb|build_team.inp| & \verb|build_team.out| \\
			\hline
			5 6 & 4 \\
			1 3 & 3 5 1 3 \\
			1 2 & \\
			5 3 & \\
			1 5 & \\
			2 4 & \\
			4 5 & \\
			\hline
		\end{tabular}
	\end{table}
\end{problem}

\begin{problem}[\href{{https://cses.fi/problemset/task/1194}}{CSES Problem Set{\tt/}monsters}]
	You \& some monsters are in a labyrinth. When taking a step to some direction in the labyrinth, each monster may simultaneously take 1 as well. Goal: reach 1 of the boundary squares without ever sharing a square with a monster. Task: find out if your goal is possible, \& if it is, print a path that you can follow. Your plan has to work in any situation; even if the monsters know your path beforehand.
	\item {\sf Input.} The 1st input line has 2 integers $n,m$: the height \& width of the map. After this there are $n$ lines of $m$ characters describing the map. Each character is . (floor), {\tt\#} (wall), {\tt A} (start), or {\tt M} (monster). There is exactly 1 {\tt A} in the input.
	\item {\sf Output.} 1st print {\tt YES} if your goal is possible, \& {\tt NO} otherwise. If your goal is possible, also print an example of a valid path (the length of the path \& its description using characters {\tt D, U, L, R}). You can print any path, as long as its length is at most $mn$ steps.
	\item {\sf Constraints.} $1\le m,n\le10^3$.
	\item {\sf Sample.} Input:
	\begin{verbatim}
		5 8
		########
		#M..A..#
		#.#.M#.#
		#M#..#..
		#.######
	\end{verbatim}
	Output:
	\begin{verbatim}
		YES
		5
		RRDDR
	\end{verbatim}
\end{problem}

\begin{problem}[\href{https://cses.fi/problemset/task/1671}{CSES Problem Set{\tt/}shortest routes I}]
	There are $n$ cities \& $m$ flight connections between them. Task: determine the length of the shortest route from Syrjälä to every city.
	\item {\sf Input.} The 1st input line has 2 integers $n,m$: the number of cities \& flight connections. The cities are numbered $1,2,\ldots,n$, \& city $1$ is Syrjälä. After that, there are $m$ lines describing the flight connections. Each line has 3 integers $a,b,c$: a flight begins at city $a$, ends at city $b$, \& its length is $c$. Each flight is a 1-way flight. You can assume that it is possible to travel from Syrjälä to all other cities.
	\item {\sf Output.} Print $n$ integers: the shortest route lengths from Syrjälä to cities $1,2,\ldots,n$.
	\item {\sf Constraints.} $1\le n\le10^5,1\le m\le2\cdot10^5,1\le a,b\le n,1\le c\le10^9$.
	\item {\sf Sample.}
	\begin{table}[H]
		\centering
		\begin{tabular}{|l|l|}
			\hline
			\verb|shortest_route_I.inp| & \verb|shortest_route_I.out| \\
			\hline
			3 4 & 0 5 2 \\
			1 2 6 & \\
			1 3 2 & \\
			3 2 3 & \\
			1 3 4 & \\
			\hline
		\end{tabular}
	\end{table}
\end{problem}

\begin{problem}[\href{https://cses.fi/problemset/task/1672}{CSES Problem Set{\tt/}shortest routes II}]
	There are $n$ cities \& $m$ roads between them. Task: process $q$ queries where you have to determine the length of the shortest route between 2 given cities.
	\item {\sf Input.} The 1st input line has 3 integers $n,m,q$: the number of cities, roads, \& queries. Then, there are $m$ lines describing the roads. Each line has 3 integers $a,b,c$: there is a road between cities $a$ \& $b$ whose length is $c$. All roads are 2-way roads. Finally, there are $q$ lines describing the queries. Each line has 2 integers $a,b$: determine the length of the shortest route between cities $a$ \& $b$.
	\item {\sf Output.} Print the length of the shortest route for each query. If there is no route, print {\tt-1} instead.
	\item {\sf Constraints.} $1\le n\le500,1\le m\le n^2,1\le q\le10^5,1\le a,b\le n,1\le c\le10^9$.
	\item {\sf Sample.}
	\begin{table}[H]
		\centering
		\begin{tabular}{|l|l|}
			\hline
			\verb|shortest_route_II.inp| & \verb|shortest_route_II.out| \\
			\hline
			4 3 5 & 5 \\
			1 2 5 & 5 \\
			1 3 9 & 8 \\
			2 3 3 & -1 \\
			1 2 & 3 \\
			2 1 & \\
			1 3 & \\
			1 4 & \\
			3 2 & \\
			\hline
		\end{tabular}
	\end{table}
\end{problem}

\begin{problem}[\href{https://cses.fi/problemset/task/1673}{CSES Problem Set{\tt/}high score}]
	You play a game consisting of $n$ rooms \& $m$ tunnels. Your initial score is $0$, \& each tunnel increases your score by $x$ where $x$ may be both positive or negative. You may go through a tunnel several times. Task: walk from room $1$ to room $n$. What is the maximum score you can get?
	\item {\sf Input.} The 1st input line has 2 integers $n,m$: the number of rooms \& tunnels. The rooms are numbered $1,2,\ldots,n$. Then, there are $m$ lines describing the tunnels. Each line has 3 integers $a,b,x$: the tunnel starts at room $a$, ends at room $b$, \& it increases your score by $x$. All tunnels are 1-way tunnels. You can assume that it is possible to get from room $1$ to room $n$.
	\item {\sf Output.} Print 1 integer: the maximum score you can get. However, if you can get an arbitrarily large score, print {\tt-1}.
	\item {\sf Constraints.} $1\le n\le2500,1\le m\le5000,1\le a,b\le n,-10^9\le x\le10^9$.
	\item {\sf Sample.}
	\begin{table}[H]
		\centering
		\begin{tabular}{|l|l|}
			\hline
			\verb|high_score.inp| & \verb|high_score.out| \\
			\hline
			4 5 & 5 \\
			1 2 3 & \\
			2 4 -1 & \\
			1 3 -2 & \\
			3 4 7 & \\
			1 4 4 & \\
			\hline
		\end{tabular}
	\end{table}
\end{problem}

\begin{problem}[\href{https://cses.fi/problemset/task/1195}{CSES Problem Set{\tt/}flight discount}]
	Task: find a minimum-price flight route from Syrjälä to Metsälä. You have 1 discount coupon, using which you can halve the price of any single flight during the route. However, you can only use the coupon once. When you use the discount coupon for a flight whose price is $x$, its price becomes $\lfloor\frac{x}{2}\rfloor$.
	\item {\sf Input.} The 1st input line has 2 integers $n,m$: the number of cities \& flight connections. The cities are numbered $1,2,\ldots,n$. City 1 is Syrjälä, \& city $n$ is Metsälä. After this there are $m$ lines describing the flights. Each line has 3 integers $a,b,c$: a flight begins at city $a$, ends at city $b$, \& its price is $c$. Each flight is unidirectional. You can assume that it is always possible to get from Syrjälä to Metsälä.
	\item {\sf Output.} Print 1 integer: the price of the cheapest route from Syrjälä to Metsälä.
	\item {\sf Constraints.} $1\le n\le10^5,1\le m\le2\cdot10^5,1\le a,b\le n,1\le c\le10^9$.
	\item {\sf Sample.}
	\begin{table}[H]
		\centering
		\begin{tabular}{|l|l|}
			\hline
			\verb|flight_discount.inp| & \verb|flight_discount.out| \\
			\hline
			3 4 & 2 \\
			1 2 3 & \\
			2 3 1 & \\
			1 3 7 & \\
			2 1 5 & \\
			\hline
		\end{tabular}
	\end{table}
\end{problem}

\begin{problem}[\href{https://cses.fi/problemset/task/1197}{CSES Problem Set{\tt/}cycle finding}]
	You are given a directed graph, \& task: find out if it contains a negative cycle, \& also give an example of such a cycle.
	\item {\sf Input.} The 1st input line has 2 integers $n,m$: the number of nodes \& edges. The nodes are numbered $1,2,\ldots,n$. After this, the input has $m$ lines describing the edges. Each line has 3 integers $a,b,c$: there is an edge from node $a$ to node $b$ whose length is $c$.
	\item {\sf Output.} If the graph contains a negative cycle, print 1st {\tt YES}, \& then the nodes in the cycle in their correct order. If there are several negative cycles, you can print any of them. If there are no negative cycles, print {\tt NO}.
	\item {\sf Constraints.} $1\le n\le2500,1\le m\le5000,1\le a,b\le n,-10^9\le c\le10^9$.
	\item {\sf Sample.}
	\begin{table}[H]
		\centering
		\begin{tabular}{|l|l|}
			\hline
			\verb|cycle_finding.inp| & \verb|cycle_finding.out| \\
			\hline
			4 5 & YES \\
			1 2 1 & 1 2 4 1 \\
			2 4 1 & \\
			3 1 1 & \\
			4 1 -3 & \\
			4 3 -2 & \\
			\hline
		\end{tabular}
	\end{table}
\end{problem}

%------------------------------------------------------------------------------%

\section{Number Theory -- Lý Thuyết Số}

%------------------------------------------------------------------------------%

\subsection{Primorial}
In mathematics, \& more particular in number theory, {\it primorial}, denoted by $p_n\#:\mathbb{N}\to\mathbb{N}$ similar to the \href{https://en.wikipedia.org/wiki/Factorial}{factorial} function, but rather than successively multiplying positive integers, the function only multiplies prime numbers. The name ``primorial'', coined by \href{https://en.wikipedia.org/wiki/Harvey_Dubner}{\sc Harvey Dubner}, draws an analogy to {\it primes} similar to the way the name ``factorial'' relates to {\it factors}.

-- Trong toán học, \& cụ thể hơn trong lý thuyết số, {\it primorial}, được ký hiệu là $p_n\#:\mathbb{N}\to\mathbb{N}$ tương tự như hàm giai thừa, nhưng thay vì nhân liên tiếp các số nguyên dương, hàm này chỉ nhân các số nguyên tố. Tên ``primorial'', do {\sc Harvey Dubner} đặt ra, có sự tương tự với {\it primes} tương tự như cách tên ``factorial'' liên quan đến các thừa số.

\begin{definition}[Primorial for prime numbers]
	For the $n$th prime number $p_n$, the primorial $p_n\#$ is defined as the product of the 1st $n$ primes: $p_n\# = \prod_{i=1}^n p_i$, where $p_i$ is the $i$th prime number.
\end{definition}
The 1st few primorials $p_n\#$ are $1, 2, 6, 30, 210, 2310, 30030, 510510, 9699690,\ldots$ (sequence A002110 in the OEIS). Asymptotically, primorials $p_n\#$ grow according to $p_n\# = e^{(1 + o(1))n\log n}$.

\begin{definition}
	The primorial $n\#$ for $n\in\mathbb{N}^\star$ is the product of the primes that are not greater than $n$, i.e.,
	\begin{equation*}
		n\# = \prod_{p\le n,\,p\ {\rm prime}} p = \prod_{i=1}^{\pi(n)} p_i = p_{\pi(n)}\#,
	\end{equation*}
	where $\pi(n)$ is the \href{https://en.wikipedia.org/wiki/Prime-counting_function}{prime-counting function} (sequence A000720 in the OEIS), which gives the number of primes $\le n$. This is equivalent to:
	\begin{equation*}
		n\# = \left\{\begin{split}
			&1&&\mbox{if } n = 0,1,\\
			&(n - 1)\#\times n&&\mbox{if } n\mbox{ is prime},\\
			&(n - 1)\#&&\mbox{if } n\mbox{ is composite}.
		\end{split}\right.
	\end{equation*}
\end{definition}

\begin{example}
	$12\# = 2\cdot3\cdot5\cdot7\cdot11 = 2310$. Since $\pi(12) = 5$, this can be calculated as $12\# = p_{\pi(12)}\# = p_5\# = 2310$.
\end{example}
Consider the 1st 12 values of $n\#$: $1, 2, 6, 6, 30, 30, 210, 210, 210, 210, 2310, 2310$. We see that for composite $n$ every term $n\#$ simply duplicates the preceding term $(n - 1)\#$, as given in the definition. Since $12$ is a composite number, $12\# = p_5\# = 11\#$.

Primorials are related to the 1st \href{https://en.wikipedia.org/wiki/Chebyshev_function}{Chebyshev function}, written $\vartheta(n)$ or $\theta(n)$ according to $\ln(n\#) = \vartheta(n)$. Since $\vartheta(n)$ asymptotically approaches $n$ for large values of $n$, primorials therefore grow according to $n\# = e^{(1 + o(1))n}$. The idea of multiplying all known primes occur in some proofs of the \href{https://en.wikipedia.org/wiki/Infinitude_of_the_prime_numbers}{infinitude of the prime numbers}, where it is used to derive the existence of another prime.

\begin{theorem}[Characteristics of primorials]
	(a) Let $p,q$ be 2 adjacent prime numbers. Given any $n\in\mathbb{N}$, where $p\le n < q$, $n\# = p\#$.
	\item(b) The fact that the binomial coefficient $\binom{2n}{n}$ is divisible by every prime between $n + 1$ \& $2n$, together with the inequality $\binom{2n}{n}\le2^n$, allows to derive the upper bound $n\#\le4^n$.
\end{theorem}

%------------------------------------------------------------------------------%

\subsection{Divisor function -- Hàm ước số}
\textbf{\textsf{Resources -- Tài nguyên.}}
\begin{enumerate}
	\item \href{https://en.wikipedia.org/wiki/Divisor_function}{Wikipedia{\tt/}divisor function}.
\end{enumerate}
In mathematics, \& specifically in \href{https://en.wikipedia.org/wiki/Number_theory}{number theory}, a {\it divisor function} is an \href{https://en.wikipedia.org/wiki/Arithmetic_function}{arithmetic function} related to the \href{https://en.wikipedia.org/wiki/Divisor}{divisors} of an integer. When referred to as {\it the} divisor function, it counts the {\it number of divisors of an integer} (including 1 \& the number itself). It appears in a number of remarkable identities, including relationships on the \href{https://en.wikipedia.org/wiki/Riemann_zeta_function}{Riemann zeta function} \& the \href{https://en.wikipedia.org/wiki/Eisenstein_series}{Eisenstein series} of \href{https://en.wikipedia.org/wiki/Modular_form}{modular forms}. Divisor functions were studied by \href{https://en.wikipedia.org/wiki/Ramanujan}{\sc Ramanujan}, who gave a number of important \href{https://en.wikipedia.org/wiki/Modular_arithmetic}{congruences} \& \href{https://en.wikipedia.org/wiki/Identity_(mathematics)}{identities}; these are treated separately in \href{https://en.wikipedia.org/wiki/Ramanujan%27s_sum}{Wikipedia{\tt/}Ramanujan's sum}.

\begin{definition}
	The {\rm sum of positive divisors function} $\sigma_z(n)$, for a real or complex number $z$, is defined as the sum of the $z$th powers of the positive divisors of $n$. It can be expressed in \href{https://en.wikipedia.org/wiki/Summation#Capital-sigma_notation}{sigma notation} as
	\begin{equation*}
		\sigma_z(n) = \sum_{d|n} d^z,\ \forall n\in\mathbb{N}^\star.
	\end{equation*}
\end{definition}
The notation $d(n),\nu(n),\tau(n)$ (for the German {\it Teiler} $=$ divisors) are also used to denote $\sigma_0(n)$, or the {\it number-of-divisors function} (OEIS: \href{https://oeis.org/A000005}{A000005}). When $z = 1$, the function is called the {\it sigma function} or {\it sum-of-divisors function}, \& the subscript is often omitted, so $\sigma(n)$ is the same as $\sigma_1(n)$ (OEIS: \href{https://oeis.org/A000203}{A000203}).

The \href{https://en.wikipedia.org/wiki/Aliquot_sum}{aliquot sum} $s(n)$ of $n$ is the sum of the \href{https://en.wikipedia.org/wiki/Proper_divisor}{proper divisors} (i.e., the divisors excluding $n$ itself, OEIS: \href{https://oeis.org/A001065}{A001065}), \& equals $\sigma_1(n) - n$; the \href{https://en.wikipedia.org/wiki/Aliquot_sequence}{aliquot sequence} of $n$ is formed by repeatedly applying the aliquot sum function.

\begin{example}
	The number of divisors of 12 is $\sigma_0(12) = 6$, the sum of all the divisors of $12$ is $\sigma_1(12) = 1 + 2 + 3 + 4 + 6 + 12 = 28$, \& the aliquot sum $s(12) = 1 + 2 + 3 + 4 + 6 = 16$. $\sigma_{-1}(n)$ is sometimes called the \href{https://en.wikipedia.org/wiki/Abundancy_index}{abundancy index} of $n$, \& we have $\sigma_{-1}
	(12) = 1^{-1} + 2^{-1} + 3^{-1} + 4^{-1} + 6^{-1} + 12^{-1} = \frac{1}{1} + \frac{1}{2} + \frac{1}{4} + \frac{1}{6} + \frac{1}{12} = \frac{12 + 6 + 4 + 3 + 2 + 1}{12} = \frac{28}{12} = \frac{7}{3} = \frac{\sigma_1(12)}{12}$. 
\end{example}

\begin{theorem}
	For any prime number $p$: (a) $\sigma_0(p) = 2,\sigma_1(p) = p + 1$. (b) $\sigma_0(p^n) = n + 1,,\sigma_1(p^n) = \sum_{i=0}^n p^i = \dfrac{p^{n+1} - 1}{p - 1}$, $\forall n\in\mathbb{N}$. (c) $\sigma_0(p_n\#) = 2^n$ where $p_n\#$ denotes the \href{https://en.wikipedia.org/wiki/Primorial}{primorial}
\end{theorem}

%------------------------------------------------------------------------------%

\section{Combinatorial Optimization -- Tối Ưu Tổ Hợp}

\begin{baitoan}[\cite{Thanh_computational_complexity_theory}, p. 25, xếp balô, MAX-KNAPSACK]
	Cho 1 lô hàng hóa gồm các gói hàng, mỗi gói đều có khối lượng cùng với giá trị cụ thể, \& cho 1 chiếc balô. Chọn từ lô này vài gói hàng nào đó \& xếp đầy vào balô, nhưng không được quá, sao cho thu được 1 {\rm GTLN} có thể.
\end{baitoan}
Đây là 1 bài toán tối ưu tổ hợp quen thuộc, được ký hiệu là MAX-KNAPSACK \& được phát biểu bằng ngôn ngữ Toán học dưới dạng tổng quát:
\begin{itemize}
	\item {\sf Input.} Cho 2 dãy số nguyên dương $\{s_i\}_{i=1}^N\cup\{S\} = s_1,\ldots,s_n,S\in\mathbb{N}^\star$ \& $\{\nu_i\}_{i=1}^n = \nu_1,\ldots,\nu_n$.
	\item {\sf Task.} Tìm 1 tập con $I\subset[n]$ thỏa
	\begin{equation*}
		\sum_{i\in I} s_i\le S,\ \sum_{i\in I} \nu_i\to\max.
	\end{equation*}
	\item {\sf Formulation of an optimization problem.}
	\begin{equation*}
		\max_{I\subset[n]}\sum_{i\in I} \nu_i\mbox{ subject to }\sum_{i\in I} s_i\le S.
	\end{equation*}
\end{itemize}

%------------------------------------------------------------------------------%

\section{CSES Problem Set}
Link: \url{https://cses.fi/problemset/}.

\subsection{Introductory Problems}

\begin{problem}[CSES]
	There are $n$ concert tickets available, each with a certain price. Then, $m$ customers arrive, one after another. Each customer announces the maximum price they are willing to pay for a ticket, \& after this, they will get a ticket with nearest possible price such that it does not exceed the maximum price.
	\item {\sf Input.} 1st input line contains $n,m\in\mathbb{N}$: number of tickets \& number of customers. The next line contains $n$ integers $h_1,h_2,\ldots,h_n$: the price of each ticket. The last line contains $m$ integers $t_1,t_2,\ldots,t_m$: the maximum price for each customer in the order they arrive.
	\item {\sf Output.} Print, for each customer, the price that they will pay for their ticket. After this, ticket cannot be purchased again. If a customer cannot get any ticket, print $-1$.
	\item {\sf Constraints.} $1\le m,n\le 2\cdot10^5$, $1\le h_i,t_i\le10^9$.
\end{problem}

\subsection{Graph Algorithms}

\subsection{Range Queries}

\subsection{Mathematics}

\begin{problem}[CSES{\tt/}Josephus Queries, \url{https://cses.fi/problemset/task/2164}]
	Consider a game where there are $n\in\mathbb{N}^\star$ children, numbered $1,2,\ldots,n$, in a circle. During the game, every 2nd child is removed from circle, until there are no children left. Task: process $q$ queries of the form: ``when there are $n$ children, who is the $k$th child that will be removed?''
	\begin{itemize}
		\item {\sf Input.} The 1st input line has an integer $q$: the number of queries. After this, there are $q$ lines that describe the queries. Each line has 2 integers $n,k$: the number of children \& the position of the child.
		\item {\sf Output.} Print $q$ integers: the answer for each query.
	\end{itemize}
\end{problem}
It seems to me that {\sf Jack97} (nickname: \verb|abortion_grandmaster|) proposed this problem.

Codes:
\begin{itemize}
	\item C++: \url{https://github.com/NQBH/advanced_STEM_beyond/blob/main/OLP_ICPC/C++/gcd_Pascal_triangle.cpp}.
\end{itemize}

\begin{problem}[CSES{\tt/}Dice Probability, \url{https://cses.fi/problemset/task/1725}]
	Throw a dice $n\in\mathbb{N}^\star$ times, \& every throw produces an outcome between $1$ \& $6$. What is the probability that the sum of outcomes is between $a,b\in\mathbb{Z}$?
	\begin{itemize}
		\item {\sf Input.} The only input line contains 3 integers $n,a,b\in\mathbb{N}^\star$.
		\item {\sf Output.} Print probability rounded to 6 decimal places (rounding half to even).
		\item {\sf Constraints.} $1\le n\le100,1\le a\le b\le6n$.
		\item {\sf Example.} Input: {\tt2 9 10}. Output: {\tt0.194444}.
	\end{itemize}
\end{problem}
{\it Phân tích.} Gọi $n$ outcomes là $a_1,\ldots,a_n\in\{1,\ldots,6\}$. Sum of outcomes: $S\coloneqq\sum_{i=1}^n a_i\in\{n,\ldots,6n\}$.

\subsection{String Algorithms}

%------------------------------------------------------------------------------%

\section{Geometry -- Hình Học}

\begin{problem}[\href{https://cses.fi/problemset/task/2189}{CSES Problem Set{\tt/}point location test}]
    There is a line that goes through the points $p_1 = (x_1,y_2),p_2(x_2,y_2)$. There is also a point $p_3 = (x_3,y_3)$. Determine whether $p_3$ is located on the left or right side of the line or if it touches the line when we are looking from $p_1$ to $p_2$.
    \item {\sf Input.} The 1st input line has an integer $t\in\mathbb{N}^\star$: the number of tests. After this, there are $t$ lines that describe the tests. Each line has $6$ integers $x_1,y_1,x_2,y_2,x_3,y_3\in\mathbb{Z}$.
    \item {\sf Output.} For each test, print {\tt LEFT, RIGHT} or {\tt TOUCH}.
    \item {\sf Constraints.} $t\in[10^5],x_i,y_i\in\overline{-10^9,10^9}$, $\forall i\in[3]$, $x_1\ne x_2$ or $y_1\ne y_2$ (i.e., $(x_1,x_1)\ne(x_2,y_2)$).
    \item {\sf Sample.}
    \begin{table}[H]
        \centering
        \begin{tabular}{|l|l|}
            \hline
            \verb|point_location_test.inp| & \verb|point_location_test.out| \\
            \hline
            3 & LEFT \\
            1 1 5 3 2 3 0 2 & RIGHT \\
            1 1 5 3 4 1 & TOUCH \\
            1 1 5 3 3 2 & \\
            \hline
        \end{tabular}
    \end{table}
\end{problem}

\begin{problem}[\href{https://cses.fi/problemset/task/2190}{CSES Problem Set{\tt/}line segment intersection}]
    There are $2$ line segments: the 1st goes through the points $(x_1,y_1),(x_2,y_2)$, \& the 2nd goes through the points $(x_3,y_3),(x_4,y_4)$. Determine if the line segments intersect, i.e., they have at least $1$ common point.
    \item {\sf Input.} The 1st input line has an integer $t\in\mathbb{N}^\star$: the number of tests. After this, there are $t$ lines that describe the tests. Each line has $8$ integers $x_1,y_1,x_2,y_2,x_3,y_3,x_4,y_4\in\mathbb{Z}$.
    \item {\sf Output.} For each test, print {\tt TEST} if the line segments intersect \& {\tt NO} otherwise.
    \item {\sf Constraints.} $t\in[10^5],x_i,y_i\in\overline{-10^9,10^9}$, $\forall i\in[4]$, $(x_1,x_1)\ne(x_2,y_2)$, $(x_3,x_3)\ne(x_4,y_4)$.
    \item {\sf Sample.}
    \begin{table}[H]
        \centering
        \begin{tabular}{|l|l|}
            \hline
            \verb|line_segment_intersection.inp| & \verb|line_segment_intersection.out| \\
            \hline
            5 & NO \\
            1 1 5 3 1 2 4 3 & YES \\
            1 1 5 3 1 1 4 3 & YES \\
            1 1 5 3 2 3 4 1 & YES \\
            1 1 5 3 2 4 4 1 & YES \\
            1 1 5 3 3 2 7 4 & \\
            \hline
        \end{tabular}
    \end{table}
\end{problem}

\begin{problem}[\href{https://cses.fi/problemset/task/2191}{CSES Problem Set{\tt/}polygon area}]
    Calculate the area of a given polygon. The polygon consists of $n$ vertices $(x_i,y_i)$, $\forall i\in[n]$. The vertices $(x_i,y_i)$ \& $(x_{i+1},y_{i+1})$ are adjacent for $i\in[n - 1]$, \& the vertices $(x_1,y_1)$ \& $(x_n,y_n)$ are also adjacent.
    \item {\sf Input.} The 1st input line has an integer $n\in\mathbb{N}^\star$: the number of vertices. After this, there are $n$ lines that describe the vertices. The $i$th such line has $2$ integers $x_i,y_i\in\mathbb{Z}$. You may assume that the polygon is simple, i.e., it does not intersect itself.
    \item {\sf Output.} Print $1$ integer: $2a\in\mathbb{Z}$ where the area of the polygon is $a$ (is ensures that the result is an integer).
    \item {\sf Constraints.} $n\in\overline{3,1000},x_i,y_i\in\overline{-10^9,10^9}$, $\forall i\in[n]$.
    \item {\sf Sample.}
    \begin{table}[H]
        \centering
        \begin{tabular}{|l|l|}
            \hline
            \verb|polygon_area.inp| & \verb|polygon_area.out| \\
            \hline
            4 & 16 \\
            1 1 & \\
            4 2 & \\
            3 5 & \\
            1 4 & \\
            \hline
        \end{tabular}
    \end{table}
\end{problem}

\begin{problem}[\href{https://cses.fi/problemset/task/2192}{CSES Problem Set{\tt/}point in polygon}]
    You are given a polygon of $n\in\mathbb{N}^\star$ vertices \& a list of $m\in\mathbb{N}^\star$ points. Determine for each point if it is inside, outside or on the boundary of the polygon. The polygon consists of $n$ vertices $(x_i,y_i)$, $\forall i\in[n]$. The vertices $(x_i,y_i)$ \& $(x_{i+1},y_{i+1})$ are adjacent for $i\in[n - 1]$, \& the vertices $(x_1,y_1)$ \& $(x_n,y_n)$ are also adjacent.
    \item {\sf Input.} The 1st input line has $2$ integer $n,m\in\mathbb{N}^\star$: the number of vertices in the polygon \& the number of points. After this, there are $n$ lines that describe the polygon. The $i$th such line has $2$ integers $x_i,y_i$. You may assume that the polygon is simple, i.e., it does not intersect itself. Finally, there are $m$ lines that describe the points. Each line has $2$ integers $x,y\in\mathbb{Z}$.
    \item {\sf Output.} For each point, print {\tt INSIDE, OUTSIDE} or {\tt BOUNDARY}.
    \item {\sf Constraints.} $M,n\in\overline{3,1000},m\in[1000],x,y,x_i,y_i\in\overline{-10^9,10^9}$, $\forall i\in[n]$.
    \item {\sf Sample.}
    \begin{table}[H]
        \centering
        \begin{tabular}{|l|l|}
            \hline
            \verb|point_in_polygon.inp| & \verb|point_in_polygon.out| \\
            \hline
            4 3 & INSIDE \\
            1 1 & OUTSIDE \\
            4 2 & BOUNDARY \\
            3 5 & \\
            1 4 & \\
            2 3 & \\
            3 1 & \\
            1 3 & \\
            \hline
        \end{tabular}
    \end{table}
\end{problem}

\begin{problem}[\href{https://cses.fi/problemset/task/2193}{CSES Problem Set{\tt/}polygon lattice points}]
    Given a polygon , calculate the number of lattice points inside the polygon \& on its boundary. A {\rm lattice point} is a point whose coordinates are integers. The polygon consists of $n\in\mathbb{N}^\star$ vertices $(x_i,y_i)$, $\forall i\in[n]$. The vertices $(x_i,y_i)$ \& $(x_{i+1},y_{i+1})$ are adjacent for $i\in[n - 1]$, \& the vertices $(x_1,y_1)$ \& $(x_n,y_n)$ are also adjacent.
    \item {\sf Input.} The 1st input line has an integer $n\in\mathbb{N}^\star$:  the number of vertices. After this, there are $n$ lines that describe the vertices. The $i$th such line has $2$ integers $x_i,y_i$. You may assume that the polygon is simple, i.e., it does not intersect itself.
    \item {\sf Output.} Print $2$ integers: the number of lattice points inside the polygon \& on its boundary.
    \item {\sf Constraints.} $n\in\overline{3,10^5},x_i,y_i\in\overline{-10^9,10^9}$, $\forall i\in[n]$.
    \item {\sf Sample.}
    \begin{table}[H]
        \centering
        \begin{tabular}{|l|l|}
            \hline
            \verb|polygon_lattice_point.inp| & \verb|polygon_lattice_point.out| \\
            \hline
            4 & 6 8 \\
            1 1 & \\
            5 3 & \\
            3 5 & \\
            1 4 & \\
            \hline
        \end{tabular}
    \end{table}
\end{problem}

\begin{problem}[\href{https://cses.fi/problemset/task/2194}{CSES Problem Set{\tt/}minimum Euclidean distance}]
    Given a set of points in 2D plane, find the minimum Euclidean distance between $2$ distinct points. The Euclidean distance of points $(x_1,y_1),(x_2,y_2)$ is $d((x_1,y_1),(x_2,y_2)) = \sqrt{(x_1 - x_2)^2 + (y_1 - y_2)^2}$.
    \item {\sf Input.} The 1st input line has an integer $n\in\mathbb{N}^\star$: the number of points. After this, there are $n$ lines that describe the points. Each line has $2$ integers $x,y\in\mathbb{Z}$: the coordinates of a point. You may assume that each point is distinct.
    \item {\sf Output.} Print $1$ integer: $d^2$ where $d$ is the minimum Euclidean distance (this ensures that the result is an integer).
    \item {\sf Constraints.} $n\in\overline{2,2\cdot10^5},x,y\in\overline{-10^9,10^9}$.
    \item {\sf Sample.}
    \begin{table}[H]
        \centering
        \begin{tabular}{|l|l|}
            \hline
            \verb|minimum_Euclidean_distance.inp| & \verb|minimum_Euclidean_distance.out| \\
            \hline
            4 & 2 \\
            2 1 & \\
            4 4 & \\
            1 2 & \\
            6 3 & \\
            \hline
        \end{tabular}
    \end{table}
\end{problem}

\begin{problem}[\href{https://cses.fi/problemset/task/2195}{CSES Problem Set{\tt/}convex hull}]
    Given a set of $n\in\mathbb{N}^\star$ points in the 2D plane, determine the convex hull of the points.
    \item {\sf Input.} The 1st input line has an integer $n\in\mathbb{N}^\star$: the number of points.  After this, there are $n$ lines that describe the points. Each line has $2$ integers $x,y\in\mathbb{Z}$: the coordinates of a point. You may assume that each point is distinct, \& the area of the hull is positive.
    \item {\sf Output.} 1st print an integer $k\in\mathbb{N}$: the number of points in the convex hull. After this, print $k$ lines that describe the points. You can print the points in any order. Print all points that lie on the convex hull.
    \item {\sf Constraints.} $n\in\overline{3,2\cdot10^5},x,y\in\overline{-10^9,10^9}$.
    \item {\sf Sample.}
    \begin{table}[H]
        \centering
        \begin{tabular}{|l|l|}
            \hline
            \verb|.inp| & \verb|.out| \\
            \hline
            6 & 4 \\
            2 1 & 2 1 \\
            2 5 & 2 5 \\
            3 3 & 4 4 \\
            4 3 & 6 3 \\
            4 4 & \\
            6 3 & \\
            \hline
        \end{tabular}
    \end{table}
\end{problem}

\begin{problem}[\href{https://cses.fi/problemset/task/3410}{CSES Problem Set{\tt/}maximum Manhattan distance}]
    A set is initially empty \& $n\in\mathbb{N}^\star$ points are added to it. Calculate the maximum Manhattan distance of $2$ points after each addition.
    \item {\sf Input.} The 1st input line has an integer $n\in\mathbb{N}^\star$: the number of points. The following $n$ lines describe the points. Each line has $2$ integers $x,y\in\mathbb{Z}$. You can assume that each point is distinct.
    \item {\sf Output.} After each addition, print the maximum distance.
    \item {\sf Constraints.} $n\in\overline{2,2\cdot10^5},x,y\in\overline{-10^9,10^9}$.
    \item {\sf Sample.}
    \begin{table}[H]
        \centering
        \begin{tabular}{|l|l|}
            \hline
            \verb|maximum_Manhattan_distance.inp| & \verb|maximum_Manhattan_distance.out| \\
            \hline
            5 & 0 \\
            1 1 & 3 \\
            3 2 & 4 \\
            2 4 & 4 \\
            2 1 & 7 \\
            4 5 & \\
            \hline
        \end{tabular}
    \end{table}
\end{problem}

\begin{problem}[\href{https://cses.fi/problemset/task/3411}{CSES Problem Set{\tt/}all Manhattan distances}]
    Given a set of points, calculate the sum of all Manhattan distances between $2$ point pairs.
    \item {\sf Input.} The 1st input line has an integer $n\in\mathbb{N}^\star$: the number of points. The following $n$ lines describe the points. Each line has $2$ integers $x,y\in\mathbb{Z}$. You can assume that each point is distinct.
    \item {\sf Output.} Print the sum of all Manhattan distances.
    \item {\sf Constraints.} $n\in\overline{2,2\cdot10^5},x,y\in\overline{-10^9,10^9}$.
    \item {\sf Sample.}
    \begin{table}[H]
        \centering
        \begin{tabular}{|l|l|}
            \hline
            \verb|all_Manhattan_distance.inp| & \verb|all_Manhattan_distance.out| \\
            \hline
            5 & 36 \\
            1 1 & \\
            3 2 & \\
            2 4 & \\
            2 1 & \\
            4 5 & \\
            \hline
        \end{tabular}
    \end{table}
\end{problem}

\begin{problem}[\href{https://cses.fi/problemset/task/1740}{CSES Problem Set{\tt/}intersection points}]
    Given $n\in\mathbb{N}^\star$ horizontal \& vertical line segments, calculate the number of their intersection points. You can assume that no parallel line segments intersect, \& no endpoint of a line segment is an intersection point.
    \item {\sf Input.} The 1st input line has an integer $n\in\mathbb{N}^\star$: the number of line segments. Then there are $n$ lines describing the line segments. Each line has $4$ integers: $x_1,y_1,x_2,y_2$: a line segment begins at point $(x_1,y_1)$ \& ends at point $(x_2,y_2)$.
    \item {\sf Output.} Print the number of intersection points.
    \item {\sf Constraints.} $n\in[10^5],x_1,x_2,y_1,y_2\in\overline{-10^6,10^6}$, $(x_1,y_1)\ne(x_2,y_2)$.
    \item {\sf Sample.}
    \begin{table}[H]
        \centering
        \begin{tabular}{|l|l|}
            \hline
            \verb|intersection_point.inp| & \verb|intersection_point.out| \\
            \hline
            3 & 2 \\
            2 3 7 3 & \\
            3 1 3 5 & \\
            6 2 6 6 & \\
            \hline
        \end{tabular}
    \end{table}
\end{problem}

\begin{problem}[\href{https://cses.fi/problemset/task/3427}{CSES Problem Set{\tt/}line segments trace I}]
    There are $n\in\mathbb{N}^\star$ line segments whose endpoints have integer coordinates. The left $x$-coordinate of each segment is $0$ \& the right $x$-coordinate is $m\in\mathbb{N}^\star$. The slope of each segment is an integer. For each $x$-coordinate $0,1,\ldots,m$, find the maximum point in any line segment.
    \item {\sf Input.} The 1st input line has $2$ integer $n,m\in\mathbb{N}^\star$: the number of line segments \& the maximum $x$-coordinate. The next $n$ lines describe the line segments. Each line has $2$ integers $y_1,y_2$: there is a line segment between points $(0,y_1)$ \& $(m,y_2)$.
    \item {\sf Output.} Print $m + 1$ integers: the maximum points for $x = 0,1,\ldots,m$.
    \item {\sf Constraints.} $m,n\in[10^5],m\in[100],y_1,y_2\in\overline{0,10^9}$.
    \item {\sf Sample.}
    \begin{table}[H]
        \centering
        \begin{tabular}{|l|l|}
            \hline
            \verb|line_segment_trace_I.inp| & \verb|line_segment_trace_I.out| \\
            \hline
            4 5 & 10 8 6 5 5 6 \\
            1 6 & \\
            7 2 & \\
            5 5 & \\
            10 0 & \\
            \hline
        \end{tabular}
    \end{table}
\end{problem}

\begin{problem}[\href{https://cses.fi/problemset/task/3428}{CSES Problem Set{\tt/}line segments trace II}]
    There are $n\in\mathbb{N}^\star$ line segments whose endpoints have integer coordinates. Each $x$-coordinate is between $0$ \& $m$. The slope of each segment is an integer. For each $x$-coordinate $0,1,\ldots,m$, find the maximum point in any line segment. If there is no segment at some point, the maximum is $-1$.
    \item {\sf Input.} The 1st input line has $2$ integers $n,m\in\mathbb{N}^\star$: the number  of line segments \& the maximum $x$-coordinate. The next $n$ lines describe the line segments. Each line has $4$ integers $x_1,y_1,x_2,y_2$: there is a line segment between points $(x_1,y_1),(x_2,y_2)$.
    \item {\sf Output.} Print $m + 1$ integers: the maximum points for $x = 0,1,\ldots,m$.
    \item {\sf Constraints.} $m,n\in[10^5],0\le x_1 < x_2\le m,y_1,y_2\in\overline{0,10^9}$.
    \item {\sf Sample.}
    \begin{table}[H]
        \centering
        \begin{tabular}{|l|l|}
            \hline
            \verb|line_segment_trace_II.inp| & \verb|line_segment_trace_II.out| \\
            \hline
            4 5 & -1 2 8 6 6 7 \\
            1 1 3 3 & \\
            1 2 4 2 & \\
            2 4 5 7 & \\
            2 8 5 2 & \\
            \hline
        \end{tabular}
    \end{table}
\end{problem}

\begin{problem}[\href{https://cses.fi/problemset/task/3429}{CSES Problem Set{\tt/}lines \& queries I}]
    Efficiently process the following types of queries:
    \begin{enumerate}
        \item Add a line $ax + b$.
        \item Find the maximum point in any line at position $x$.
    \end{enumerate}
    \item {\sf Input.} The 1st input line has an integer $n\in\mathbb{N}^\star$: the number of queries. The following $n$ lines describe the queries. The format of each line is either {\tt1 a b} or {\tt2 x}. You may assume that the 1st query is of type 1.
    \item {\sf Output.} Print the answer for each query of type 2.
    \item {\sf Constraints.} $n\in[2\cdot10^5],a,b\in\overline{-10^9,10^9},x\in\overline{0,10^5}$.
    \item {\sf Sample.}
    \begin{table}[H]
        \centering
        \begin{tabular}{|l|l|}
            \hline
            \verb|line_query_I.inp| & \verb|line_query_I.out| \\
            \hline
            6 & 3 \\
            1 1 2 & 5 \\
            2 1 & 4 \\
            2 3 & 5 \\
            1 0 4 & \\
            2 1 & \\
            2 3 & \\
            \hline
        \end{tabular}
    \end{table}
\end{problem}

\begin{problem}[\href{https://cses.fi/problemset/task/3430}{CSES Problem Set{\tt/}lines \& queries II}]
    Efficiently process the following types of queries:
    \begin{enumerate}
        \item Add a line $ax + b$ that is active in range $[l,r]$.
        \item Find the maximum point in any active line at position $x$.
    \end{enumerate}
    \item {\sf Input.} The 1st input line has an integer $n\in\mathbb{N}^\star$: the number of queries. The following $n$ lines describe the queries. The format of each line is either {\tt1 a b l r} or {\tt2 x}.
    \item {\sf Output.} Print the answer for each query of type 2. If no line is active, print {\tt NO}.
    \item {\sf Constraints.} $n\in[2\cdot10^5],a,b\in\overline{-10^9,10^9},x,l,r\in\overline{0,10^5}$.
    \item {\sf Sample.}
    \begin{table}[H]
        \centering
        \begin{tabular}{|l|l|}
            \hline
            \verb|line_query_II.inp| & \verb|line_query_II.out| \\
            \hline
            6 & 5 \\
            1 1 2 1 3 & NO \\
            2 3 & 5 \\
            2 4 & 4 \\
            1 0 4 1 5 & \\
            2 3 & \\
            2 4 & \\
            \hline
        \end{tabular}
    \end{table}
\end{problem}

\begin{problem}[\href{https://cses.fi/problemset/task/1741}{CSES Problem Set{\tt/}area of rectangles}]
    Given $n\in\mathbb{N}^\star$, determine the total area of their union.
    \item {\sf Input.} The 1st input line has an integer $n\in\mathbb{N}^\star$: the number of rectangles. After that, there are $n$ lines describing the rectangles. Each line has $4$ integers $x_1,y_1,x_2,y_2$: a rectangle begins at point $(x_1,y_1)$ \& ends at point $(x_2,y_2)$.
    \item {\sf Output.} Print the total area covered by the rectangles.
    \item {\sf Constraints.} $n\in[10^5],m\in[100],x_1,x_2,y_1,y_2\in\overline{-10^6,10^6}$.
    \item {\sf Sample.}
    \begin{table}[H]
        \centering
        \begin{tabular}{|l|l|}
            \hline
            \verb|area_rectangle.inp| & \verb|area_rectangle.out| \\
            \hline
            3 & 24 \\
            1 3 4 5 & \\
            3 1 7 4 & \\
            5 3 8 6 & \\
            \hline
        \end{tabular}
    \end{table}
\end{problem}

\begin{problem}[\href{https://cses.fi/problemset/task/1742}{CSES Problem Set{\tt/}robot path}]
    You are given a description of a robot's path. The robot begins at point $(0,0)$ \& performs $n\in\mathbb{N}^\star$ commands. Each command moves the robot some distance up, down, left, or right. The robot will stop when it has performed all commands, or immediately when it returns to a point that it has already visited. Calculate the total distance the robot moves.
    \item {\sf Input.} The 1st input line has an integer $n\in\mathbb{N}^\star$: the number of commands. After that, there are $n$ lines describing the commands. each line has a character $d$ \& an integer $x\in\mathbb{N}^\star$: the robot moves the distance $x$ to the direction $d$. Each direction is {\tt U, D, L, R} (up, down, left, right).
    \item {\sf Output.} Print the total distance the robot moves.
    \item {\sf Constraints.} $n\in[10^5],x\in[10^6]$.
    \item {\sf Sample.}
    \begin{table}[H]
        \centering
        \begin{tabular}{|l|l|}
            \hline
            \verb|robot_path.inp| & \verb|robot_path.out| \\
            \hline
            5 & 9 \\
            U 2 & \\
            R 3 & \\
            D 1 & \\
            L 5 & \\
            U 2 & \\
            \hline
        \end{tabular}
    \end{table}
\end{problem}

%------------------------------------------------------------------------------%

\section{Advanced Techniques -- Các Kỹ Thuật Nâng Cao}

\begin{problem}[\href{https://cses.fi/problemset/task/1628}{CSES Problem Set{\tt/}meet in the middle}]
    You are given an array of $n\in\mathbb{N}^\star$ numbers. In how many ways can you choose a subset of the numbers with sum $x\in\mathbb{N}^\star$?
    \item {\sf Input.} The 1st input line has 2 integers $n,x\in\mathbb{N}^\star$: the array size \& the required sum. The 2nd line has $n$ integers $t_1,\ldots,t_n$: the numbers in the array.
    \item {\sf Output.} Print the number of ways you can create the sum $x$.
    \item {\sf Constraints.} $n\in[40],x\in[10^9],t_i\in[10^9]$, $\forall i\in[n]$.
    \item {\sf Sample.}
    \begin{table}[H]
        \centering
        \begin{tabular}{|l|l|}
            \hline
            \verb|meet_middle.inp| & \verb|meet_middle.out| \\
            \hline
            4 5 & 3 \\
            1 2 3 2 & \\
            \hline
        \end{tabular}
    \end{table}
\end{problem}

\begin{problem}[\href{https://cses.fi/problemset/task/2136}{CSES Problem Set{\tt/}Hamming distance}]
    The Hamming distance between 2 strings $a,b$ of equal length is the number of positions where the strings differ. You are given $n\in\mathbb{N}^\star$ bit strings, each of length $k\in\mathbb{N}^\star$ \& your task is to calculate the minimum Hamming distance between 2 strings.
    \item {\sf Input.} The 1st input line has 2  integer $n\in\mathbb{N}^\star$: the number of bit strings \& their length. Then there are $n$ lines each consisting of 1 bit string of length $k$.
    \item {\sf Output.} Print the minimum Hamming distance between 2 strings.
    \item {\sf Constraints.} $2\le n\le2\cdot10^4,k\in[30]$.
    \item {\sf Sample.}
    \begin{table}[H]
        \centering
        \begin{tabular}{|l|l|}
            \hline
            \verb|Hamming_distance.inp| & \verb|Hamming_distance.out| \\
            \hline
            3 5 & 3 \\
            2 0 2 & \\
            \hline
        \end{tabular}
    \end{table}
\end{problem}
C++:
\begin{enumerate}
    \item NHT's C++: Hamming distance.
    \begin{Verbatim}[numbers=left,xleftmargin=5mm]
#include <bits/stdc++.h>
using namespace std;

const int maxN = 2e4;
int N, ans, b[maxN];

int scanBinary() {
    char c;
    int res = 0;
    while ((c = getchar()) != '\n') {
        res <<= 1;
        res += (c - '0') & 1;
    }
    return res;
}

int main() {
    scanf("%d %d ", &N, &ans);
    for (int i = 0; i < N; i++)
    b[i] = scanBinary();
    
    for (int i = 0; i < N; i++)
    for (int j = i + 1; j < N; j++)
    ans = min(ans, __builtin_popcount(b[i] ^ b[j]));
    
    printf("%d\n", ans);
}
    \end{Verbatim}
\end{enumerate}

\begin{problem}[\href{https://cses.fi/problemset/task/3360}{CSES Problem Set{\tt/}corner subgrid check}]
    You are given a grid of letters. Find subgrids whose height \& width is at least $2$ \& all the corners have the same letter. For each letter, check if there is a valid subgrid whose corners have that letter.
    \item {\sf Input.} The 1st input line has $2$ integers $n,k\in\mathbb{N}^\star$: the size of the grid \& the number of letters. The letters are the 1st $k$ uppercase letters. After this, there are $n$ lines that describe the grid. Each line has $n$ letters.
    \item {\sf Output.} Print $k$ lines: for each letter, {\tt YES} if there is a valid subgrid \& {\tt NO} otherwise.
    \item {\sf Constraints.} $n\in[3000],k\in[26]$.
    \item {\sf Sample.}
    \begin{table}[H]
        \centering
        \begin{tabular}{|l|l|}
            \hline
            \verb|corner_subgrid_check.inp| & \verb|corner_subgrid_check.out| \\
            \hline
            4 5 & YES \\
            AAAA & YES \\
            CBBC & NO \\
            CBBE & NO \\
            AAAA & NO \\            
            \hline
        \end{tabular}
    \end{table}
\end{problem}

\begin{problem}[\href{https://cses.fi/problemset/task/2137}{CSES Problem Set{\tt/}corner subgrid count}]
    You are given an $n\times n$ grid whose each square is either black or white. A subgrid is called {\rm beautiful} if its height \& width is at least $2$ \& all of its corners are black. How many beautiful subgrids are there within the given grid?
    \item {\sf Input.} The 1st input line has an integer $n\in\mathbb{N}^\star$: the size of the grid. Then there are $n$ lines describing the grid: $1$ means that a square is black \& $0$ means it is white.
    \item {\sf Output.} Print the number of beautiful subgrids.
    \item {\sf Constraints.} $n\in[3000]$.
    \item {\sf Sample.}
    \begin{table}[H]
        \centering
        \begin{tabular}{|l|l|}
            \hline
            \verb|corner_subgrid_count.inp| & \verb|corner_subgrid_count.out| \\
            \hline
            5 & 4 \\
            00010 & \\
            11111 & \\
            00110 & \\
            11001 & \\
            00010 & \\
            \hline
        \end{tabular}
    \end{table}
\end{problem}

\begin{problem}[\href{https://cses.fi/problemset/task/2138}{CSES Problem Set{\tt/}reachable nodes}]
    A directed acyclic graph consists of $n\in\mathbb{N}^\star$ nodes \& $m\in\mathbb{N}^\star$ edges. The nodes are numbered $1,2,\ldots,n$. Calculate for each node the number of nodes you can reach from that node (including the node itself).
    \item {\sf Input.} The 1st input line has $2$ integers $n,m\in\mathbb{N}^\star$: the number of nodes \& edges. Then there are $m$ lines describing the edges. Each line has $2$ distinct integers $a,b\in\mathbb{N}^\star$: there is an edge from node $a$ to node $b$.
    \item {\sf Output.} Print $n$ integers: for each node the number of reachable nodes.
    \item {\sf Constraints.} $n\in[5\cdot10^4],m\in[10^5]$.
    \item {\sf Sample.}
    \begin{table}[H]
        \centering
        \begin{tabular}{|l|l|}
            \hline
            \verb|reachable_node.inp| & \verb|reachable_node.out| \\
            \hline
            5 6 & 5 3 2 2 1 \\
            1 2 & \\
            1 3 & \\
            1 4 & \\
            2 3 & \\
            3 5 & \\
            4 5 & \\
            \hline
        \end{tabular}
    \end{table}
\end{problem}

\begin{problem}[\href{https://cses.fi/problemset/task/2143}{CSES Problem Set{\tt/}reachability queries}]
    A directed graph consists of $n\in\mathbb{N}^\star$ nodes \& $m\in\mathbb{N}^\star$ edges. The edges are numbered $1,2,\ldots,n$. Answer $q$ queries of the form ``can you reach node $b$ from node $a$?''
    \item {\sf Input.} The 1st input line has $3$ integers $n,m,q\in\mathbb{N}^\star$: the number of nodes, edges, \& queries. Then there are $m$ lines describing the edges. Each line has $2$ distinct integers $a,b$: there is an edge from node $a$ to node $b$. Finally there are $q$ lines describing the queries. Each line consists of $2$ integers $a,b$: ``can you reach node $b$ from node $a$?''
    \item {\sf Output.} Print the answer for each query: either {\tt YES} or {\tt NO}.
    \item {\sf Constraints.} $n\in[5\cdot10^4],m\in[100],m,q\in[10^5]$.
    \item {\sf Sample.}
    \begin{table}[H]
        \centering
        \begin{tabular}{|l|l|}
            \hline
            \verb|reachability_query.inp| & \verb|reachability_query.out| \\
            \hline
            4 4 3 & YES \\
            1 2 & NO \\
            2 3 & YES \\
            3 1 & \\
            4 3 & \\
            1 3 & \\
            1 4 & \\
            4 1 & \\
            \hline
        \end{tabular}
    \end{table}
\end{problem}

\begin{problem}[\href{https://cses.fi/problemset/task/2072}{CSES Problem Set{\tt/}cut \& paste}]
    Given a string, process operations where you cut a substring \& paste it to the end of the string. What is the final string after all the operations?
    \item {\sf Input.} The 1st input line has $2$ integers $n.m\in\mathbb{N}^\star$: the length of the string \& the number of operations. The characters of the string are numbered $1,2,\ldots,n$. The next line has a string of length $n$ that consists of characters {\tt A--Z}. Finally, there are $m$ lines that describe the operations. Each line has $2$ integers $a,b\in\mathbb{N}^\star$: you cut a substring from position $a$ to position $b$.
    \item {\sf Output.} Print the final string after all the operations.
    \item {\sf Constraints.} $m,n\in[2\cdot10^5],1\le a\le b\le n$.
    \item {\sf Sample.}
    \begin{table}[H]
        \centering
        \begin{tabular}{|l|l|}
            \hline
            \verb|cut_paste.inp| & \verb|cut_paste.out| \\
            \hline
            7 2 & AYABTUB \\
            AYBABTU & \\
            3 5 & \\
            3 5 & \\
            \hline
        \end{tabular}
    \end{table}
\end{problem}

\begin{problem}[\href{https://cses.fi/problemset/task/2073}{CSES Problem Set{\tt/}substring reversals}]
    Given a string, process operations where you reverse a substring of the string. What is the final string after all the operations?
    \item {\sf Input.} The 1st input line has $2$ integers $n,m\in\mathbb{N}^\star$: the length of the string \& the number of operations. The characters of the string are numbered $1,2,\ldots,n$. The next line has a string of length $n$ that consists of characters {\tt A--Z}. Finally, there are $m$ lines that describe the operations. Each line has $2$ integers $a,b$: you reverse a substring from position $a$ to position $b$.
    \item {\sf Output.} Print the final string after all the operations.
    \item {\sf Constraints.} $m,n\in[2\cdot10^5],1\le a\le b\le n$.
    \item {\sf Sample.}
    \begin{table}[H]
        \centering
        \begin{tabular}{|l|l|}
            \hline
            \verb|substring_reversal.inp| & \verb|substring_reversal.out| \\
            \hline
            7 2 & AYAUTBB \\
            AYBABTU & \\
            3 4 & \\
            4 7 & \\
            \hline
        \end{tabular}
    \end{table}
\end{problem}

\begin{problem}[\href{https://cses.fi/problemset/task/2074}{CSES Problem Set{\tt/}reversals \& sums}]
    Given an array of $n\in\mathbb{N}^\star$ integers, you have to process following operations:
    \begin{enumerate}
        \item Reverse a subarray
        \item Calculate the sum of values in a subarray.
    \end{enumerate}
    \item {\sf Input.} The 1st input line has $2$ integer $n,m\in\mathbb{N}^\star$: the size of the array \& the number of operations. The array elements are numbered $1,2,\ldots,n$. The next line as $n$ integers $x_1,x_2,\ldots,x_n$: the contents of the array. Finally, there are $m$ lines that describe the operations. Each line has $3$ integers $t,a,b$. If $t = 1$, you should reverse a subarray from $a$ to $b$. If $t = 2$, you should calculate the sum of values from $a$ to $b$.
    \item {\sf Output.} Print the answer to each operation where $t = 2$.
    \item {\sf Constraints.} $n\in[2\cdot10^5],m\in[10^5],x_i\in\overline{0,10^9},1\le a\le b\le n$.
    \item {\sf Sample.}
    \begin{table}[H]
        \centering
        \begin{tabular}{|l|l|}
            \hline
            \verb|reversal_sum.inp| & \verb|reversal_sum.out| \\
            \hline
            8 3 & 8 \\
            2 1 3 4 5 3 4 4 & 9 \\
            2 2 4 & \\
            1 3 6 & \\
            2 2 4 & \\
            \hline
        \end{tabular}
    \end{table}
\end{problem}

\begin{problem}[\href{https://cses.fi/problemset/task/2076}{CSES Problem Set{\tt/}necessary roads}]
    There are $n\in\mathbb{N}^\star$ cities \& $m\in\mathbb{N}^\star$ roads between them. There is a route between any $2$ cities. A road is called {\rm necessary} if there is no route between some $2$ cities after removing that road. Find all necessary roads.
    \item {\sf Input.} The 1st input line has $2$ integers $n,m\in\mathbb{N}^\star$: the number of cities \& roads. The cities are numbered $1,2,\ldots,n$. After this, there are $m$ lines that describe the roads. Each line has $2$ integers $a,b\in\mathbb{N}^\star$: there is a road between cities $a$ \& $b$. There is at most $1$ road between $2$ cities, \& every road connects $2$ distinct cities.
    \item {\sf Output.} 1st print an integer: the number of necessary roads. After that, print $k$ lines that describe the roads. You may print the roads in any order.
    \item {\sf Constraints.} $n\in\overline{2,10^5},m\in[2\cdot10^5],a,b\in[n]$.
    \item {\sf Sample.}
    \begin{table}[H]
        \centering
        \begin{tabular}{|l|l|}
            \hline
            \verb|necessary_road.inp| & \verb|necessary_road.out| \\
            \hline
            5 5 & 2 \\
            1 2 & 3 5 \\
            1 4 & 4 5 \\
            2 4 & \\
            3 5 & \\
            4 5 & \\
            \hline
        \end{tabular}
    \end{table}
\end{problem}

\begin{problem}[\href{https://cses.fi/problemset/task/2077}{CSES Problem Set{\tt/}necessary cities}]
    There are $n\in\mathbb{N}^\star$ cities \& $m\in\mathbb{N}^\star$ roads between them. There is a route between any $2$ cities. A city is called {\rm necessary} if there is no route between some other $2$ cities after removing that city (\& adjacent roads). Find all necessary cities.
    \item {\sf Input.} The 1st input line has $2$ integers $n,m\in\mathbb{N}^\star$: the number of cities \& roads. The cities are numbered $1,2,\ldots,n$. After this, there are $m$ lines that describe the roads. Each line has $2$ integers $a,b\in\mathbb{N}^\star$: there is a road between cities $a$ \& $b$. There is at most $1$ road between $2$ cities, \& every road connects $2$ distinct cities.
    \item {\sf Output.} 1st print an integer: the number of necessary cities. After that, print a list of $k$ cities. You may print the cities in any order.
    \item {\sf Constraints.} $n\in\overline{2,10^5},m\in[2\cdot10^5],a,b\in[n]$.
    \item {\sf Sample.}
    \begin{table}[H]
        \centering
        \begin{tabular}{|l|l|}
            \hline
            \verb|necessary_city.inp| & \verb|necessary_city.out| \\
            \hline
            5 5 & 2 \\
            1 2 & 4 5 \\
            1 4 & \\
            2 4 & \\
            3 5 & \\
            4 5 & \\
            \hline
        \end{tabular}
    \end{table}
\end{problem}

\begin{problem}[\href{https://cses.fi/problemset/task/2078}{CSES Problem Set{\tt/}Eulerian subgraphs}]
    You are given an undirected graph that has $n\in\mathbb{N}^\star$ nodes \& $m\in\mathbb{N}^\star$ edges. We consider subgraphs that have all nodes of the original graph \& some of its edges. A subgraph is called {\it Eulerian} if each node has even degree. Count the number of Eulerian subgraphs modulo $10^9 + 7$.
    \item {\sf Input.} The 1st input line has $2$ integers $n,m\in\mathbb{N}^\star$: the number of nodes \& edges. The nodes are numbered $1,2,\ldots,n$. After this, there are $m$ lines that describe the edges. Each line has $2$ integers $a,b\in\mathbb{N}^\star$: there is an edge between nodes $a$ \& $b$. There is at most $1$ edge between $2$ nodes, \& each edge connects $2$ distinct nodes.
    \item {\sf Output.} Print the  number of Eulerian subgraphs modulo $10^9 + 7$.
    \item {\sf Constraints.} $n\in[10^5],m\in[100],x_i\in\{0,1,\ldots,m\}$.
    \item {\sf Sample.}
    \begin{table}[H]
        \centering
        \begin{tabular}{|l|l|}
            \hline
            \verb|Eulerian_subgraph.inp| & \verb|Eulerian_subgraph.out| \\
            \hline
            4 3 & 2 \\
            1 2 & \\
            1 3 & \\
            2 3 & \\
            \hline
        \end{tabular}
    \end{table}
    \item {\sf Explanation.} You can either keep or remove all edges, so there are $2$ possible Eulerian subgraphs.
\end{problem}

\begin{problem}[\href{https://cses.fi/problemset/task/2084}{CSES Problem Set{\tt/}monster game I}]
    You are playing a game that consists of $n\in\mathbb{N}^\star$ levels. Each level has a monster. On levels $1,2,\ldots,n - 1$, you can either kill or escape the monster. However, on level $n$ you must kill the final monster to win the game. Killing a monster takes $sf$ time where $s$ is the monster's strength \& $f$ is your skill factor (lower skill factor is better). After killing a monster, you get a new skill factor. What is the minimum total time in which you can win the game?
    \item {\sf Input.} The 1st input line has $2$ integers $n,x\in\mathbb{N}^\star$: the number of levels \& your initial skill factor. The 2nd line has $n$ integers $s_1,s_2,\ldots,s_n\in\mathbb{N}^\star$: each monster's strength. The 3rd line has $n$ integers $f_1,f_2,\ldots,f_nn\in\mathbb{N}^\star$: your new skill factor after killing a monster.
    \item {\sf Output.} Print $1$ integer: the minimum total time to win the game.
    \item {\sf Constraints.} $n\in[2\cdot10^5],x\in[10^6],1\le s_1\le s_2\le\cdots\le s_n\le10^6,x\ge f_1\ge f_2\ge\cdots\ge f_n\ge1$.
    \item {\sf Sample.}
    \begin{table}[H]
        \centering
        \begin{tabular}{|l|l|}
            \hline
            \verb|monster_game_I.inp| & \verb|monster_game_I.out| \\
            \hline
            5 100 & 4800 \\
            20 30 30 50 90 & \\
            90 60 20 20 10 & \\
            \hline
        \end{tabular}
    \end{table}
    \item {\sf Explanation.} The best way to play is to kill the 3rd \& 5th monsters.
\end{problem}

\begin{problem}[CSES Problem Set{\tt/}monster game II]
    You are playing a game that consists of $n\in\mathbb{N}^\star$ levels. Each level has a monster. On levels $1,2,\ldots,n - 1$, you can either kill or escape the monster. However, on level $n$ you must kill the final monster to win the game. Killing a monster takes $sf$ time where $s$ is the monster's strength \& $f$ is your skill facto. After killing a monster, you get a new skill factor (lower skill factor is better). What is the minimum total time in which you can win the game?
    \item {\sf Input.} The 1st input line has $2$ integers $n,x\in\mathbb{N}^\star$: the number of levels \& your initial skill factor. The 2nd line has $n$ integers $s_1,s_2,\ldots,s_n\in\mathbb{N}^\star$: each monster's strength. The 3rd line has $n$ integers $f_1,f_2,\ldots,f_nn\in\mathbb{N}^\star$: your new skill factor after killing a monster.
    \item {\sf Output.} Print $1$ integer: the minimum total time to win the game.
    \item {\sf Constraints.} $n\in[2\cdot10^5],x\in[10^6],x\in[10^6],s_i,f_i\in[10^6]$.
    \item {\sf Sample.}
    \begin{table}[H]
        \centering
        \begin{tabular}{|l|l|}
            \hline
            \verb|monster_game_II.inp| & \verb|monster_game_II.out| \\
            \hline
            5 100 & 2600 \\
            50 20 30 90 30 & \\
            60 20 20 10 90 & \\
            \hline
        \end{tabular}
    \end{table}
    \item {\sf Explanation.} The best way to play is to kill the 2nd \& 5th monsters.
\end{problem}

\begin{problem}[\href{https://cses.fi/problemset/task/2086}{CSES Problem Set{\tt/}subarray squares}]
    Given an array of $n\in\mathbb{N}^\star$ elements, divide into $k\in\mathbb{N}^\star$ subarrays. The cost of each subarray is the square of the sum of the values in the subarray. What is the minimum total cost if you act optimally?
    \item {\sf Input.} The 1st input line has $2$ integers $n,k\in\mathbb{N}^\star$: the array elements \& the number of subarrays. The array elements are numbered $1,2,\ldots,n$. The 2nd line has $n$ integers $x_1,x_2,\ldots,x_n$: the contents of the array.
    \item {\sf Output.} Print $1$ integer: the minimum total cost.
    \item {\sf Constraints.} $1\le k\le n\le3000,x_i\in[1o^5]$, $\forall i\in[n]$.
    \item {\sf Sample.}
    \begin{table}[H]
        \centering
        \begin{tabular}{|l|l|}
            \hline
            \verb|subarray_square.inp| & \verb|subarray_square.out| \\
            \hline
            8 3 & 110 \\
            2 3 1 2 2 3 4 1 & \\
            \hline
        \end{tabular}
    \end{table}
    \item {\sf Explanation.} An optimal solution is $[2,3,1],[2,2,3],[4,1]$, whose cost is $(2 + 3 + 1)^2 + (2 + 2 + 3)^2 + (4 + 1)^2 = 110$.
\end{problem}

\begin{problem}[\href{https://cses.fi/problemset/task/2087}{CSES Problem Set{\tt/}houses \& schools}]
    There are $n\in\mathbb{N}^\star$ houses on a street, numbered $1,2,\ldots,n$. The distance of houses $a,b$ is $|a - b|$. You know the number of children in each house. Establish $k$ schools in such a way that each school is in some house. Then, each child goes to the nearest school. What is the minimum total walking distance of the children if you act optimally?
    \item {\sf Input.} The 1st input line has $2$ integers $n,k\in\mathbb{N}^\star$: the number of houses \& the number of schools. The houses are numbered $1,2,\ldots,n$. After this, there are $n$ integers $c_1,c_2,\ldots,c_n$: the number of children in each house.
    \item {\sf Output.} Print the minimum total distance.
    \item {\sf Constraints.} $1\le k\le n\le3000,c_i\in[10^9]$, $\forall i\in[n]$.
    \item {\sf Sample.}
    \begin{table}[H]
        \centering
        \begin{tabular}{|l|l|}
            \hline
            \verb|house_school.inp| & \verb|house_school.out| \\
            \hline
            6 2 & 11 \\
            2 7 1 4 6 4 & \\
            \hline
        \end{tabular}
    \end{table}
    \item {\sf Explanation.} Houses $2,5$ will have schools.
\end{problem}

\begin{problem}[\href{https://cses.fi/problemset/task/2088}{CSES Problem Set{\tt/}Knuth division}]
    Given an array of $n$ numbers, divide it into $n$ subarrays, each of which has a single element. On each move, you may choose any subarray \& split it into $2$ subarrays. The cost of such a move is the sum of values in the chosen subarray. What is the minium total cost if you act optimally?
    \item {\sf Input.} The 1st input line has an integers $n\in\mathbb{N}^\star$: the array size. The array elements are numbered $1,2,\ldots,n$. The 2nd line has $n$ integers $x_1,x_2,\ldots,x_n$: the contents of the array.
    \item {\sf Output.} Print $1$ integer: the minimum total cost.
    \item {\sf Constraints.} $n\in[5000],x_i\in[10^9]$.
    \item {\sf Sample.}
    \begin{table}[H]
        \centering
        \begin{tabular}{|l|l|}
            \hline
            \verb|Knuth_division.inp| & \verb|Knuth_division.out| \\
            \hline
            5 & 43 \\
            2 7 3 2 5 & \\
            \hline
        \end{tabular}
    \end{table}
\end{problem}

\begin{problem}[CSES Problem Set{\tt/}apples \& bananas]
    There are $n\in\mathbb{N}^\star$ apples \& $m\in\mathbb{N}^\star$ bananas, \& each of them has an integer weight between $1,\ldots,k$. Calculate, for each weight $w$ between $2,\ldots,2k$, the number of ways we can choose an apple \& a banana whose combined weight is $w$.
    \item {\sf Input.} The 1st input line contains $3$ integers $k,n,m\in\mathbb{N}^\star$: the number $K$, the number of apples \& the number of bananas. The next line contains $n$ integers $a_1,a_2,\ldots,a_n$: weight of each apple. The last line contains $m$ integers $b_1,b_2,\ldots,b_m$: weight of each banana.
    \item {\sf Output.} For each integer $w$ between $2,\ldots,2k$, print the number of ways to choose an apple \& a banana whose combined weight is $w$.
    \item {\sf Constraints.} $m,n,k\in[2\cdot10^5],a_i,b_j\in[k]$, $\forall i\in[n],\forall j\in[m]$.
    \item {\sf Sample.}
    \begin{table}[H]
        \centering
        \begin{tabular}{|l|l|}
            \hline
            \verb|apple_banana.inp| & \verb|apple_banana.out| \\
            \hline
            5 3 4 & 0 0 1 2 1 2 4 2 0 \\
            5 2 5 & \\
            4 3 2 3 & \\
            \hline
        \end{tabular}
    \end{table}
    \item {\sf Explanation.} E.g., for $w = 8$ there are $4$ different ways: we can pick an apple of weight $5$ into $2$ different ways \& a banana of weight $3$ in $2$ different ways.
\end{problem}

\begin{problem}[\href{https://cses.fi/problemset/task/2112}{CSES Problem Set{\tt/}1 bit positions}]
    You are given a binary string of length $n$. Calculate, for every $k$ between $1,\ldots,n - 1$, the number of ways we can choose $2$ positions $i,j$ s.t. $i - j = k$ \& there is a $1$-bit at both positions.
    \item {\sf Input.} The only input line has a string that consists only of characters $0,1$.
    \item {\sf Output.} For every distance $k$ between $1,\ldots,n - 1$ print the number of ways we can choose $2$ such positions.
    \item {\sf Constraints.} $n\in\overline{2,2\cdot10^5}$.
    \item {\sf Sample.}
    \begin{table}[H]
        \centering
        \begin{tabular}{|l|l|}
            \hline
            \verb|one_bit_position.inp| & \verb|one_bit_position.out| \\
            \hline
            1001011010 & 1 2 3 0 2 1 0 1 0 \\
            \hline
        \end{tabular}
    \end{table}
\end{problem}

\begin{problem}[\href{https://cses.fi/problemset/task/2113}{CSES Problem Set{\tt/}signal processing}]
    You are given $2$ integer sequences: a signal \& a mask. Process the signal by moving the mask through the signal from left to right. At each mask position calculate the sum of products of aligned signal \& mask values in the part where the signal \& the mask overlap.
    \item {\sf Input.} The 1st input line consists of $2$ integers $n,m\in\mathbb{N}^\star$: the length of the signal \& the length of the mask. The next line consists of $n$ integers $a_1,a_2,\ldots,a_n$ defining the signal. The last line consists of $m$ integers $b_1,b_2,\ldots,b_m$ defining the mask.
    \item {\sf Output.} Print $n + m - 1$ integer: the sum of products of aligned values at each mask position from left to right.
    \item {\sf Constraints.} $m,n\in[2\cdot10^5],a_i,b_i\in[100]$.
    \item {\sf Sample.}
    \begin{table}[H]
        \centering
        \begin{tabular}{|l|l|}
            \hline
            \verb|signal_processing.inp| & \verb|signal_processing.out| \\
            \hline
            5 3 & 3 11 13 10 16 9 4 \\
            1 3 2 1 4 & \\
            1 2 3 & \\
            \hline
        \end{tabular}
    \end{table}
    \item {\sf Explanation.} E.g., at the 2nd mask position the sum of aligned products is $2\cdot1 + 3\cdot3 = 11$.
\end{problem}

\begin{problem}[\href{https://cses.fi/problemset/task/2101}{CSES Problem Set{\tt/}new roads queries}]
    There are $n\in\mathbb{N}^\star$ cities in Byteland but no roads between them. However, each day, a new road will be built. There will be a total of $m\in\mathbb{N}^\star$ roads. Process $q\in\mathbb{N}^\star$ queries of the form: ``after how many days can we travel from city $a$ to city $b$ for the 1st time?''
    \item {\sf Input.} The 1st input line has $3$ integers $n,m,q\in\mathbb{N}^\star$: the number of cities, roads, \& queries. The cities are numbered $1,2,\ldots,n$. After this, there are $m$ lines that describe the roads in the order they are built. Each line has $2$ integers $a,b$: there will be a road between cities $a$ \& $b$. Finally, there are $q$ lines that describe the queries. Each line has $2$ integers $a,b$: we want to travel from city $a$ to city $b$.
    \item {\sf Output.} For each query, print the number of days, or $-1$ if it is never possible.
    \item {\sf Constraints.} $m,n,q\in[2\cdot10^5],a,b\in[n]$.
    \item {\sf Sample.}
    \begin{table}[H]
        \centering
        \begin{tabular}{|l|l|}
            \hline
            \verb|new_roads_query.inp| & \verb|new_roads_query.out| \\
            \hline
            5 4 3 & 2 \\
            1 2 & $-1$ \\
            2 3 & 4 \\
            1 3 & \\
            2 5 & \\
            1 3 & \\
            3 4 & \\
            3 5 & \\
            \hline
        \end{tabular}
    \end{table}
\end{problem}

\begin{problem}[\href{https://cses.fi/problemset/task/2133}{CSES Problem Set{\tt/}dynamic connectivity}]
    Consider an undirected graph that consists of $n\in\mathbb{N}^\star$ nodes \& $m\in\mathbb{N}^\star$ edges. There are $2$ types of events that can happen:
    \begin{enumerate}
        \item A new edge is created between nodes $a,b$.
        \item An existing edge between nodes $a,b$ is removed.
    \end{enumerate}
    Report the number of components after every event.
    \item {\sf Input.} The 1st input line has $3$ integers $n,m,k\in\mathbb{N}^\star$: the number of nodes, edges, \& events. After this there are $m$ lines describing the edges. Each line has $2$ integers $a,b$: there is an edge between nodes $a,b$. There is at most $1$ edge between any pair of nodes. Then there are $k$ lines describing the events. Each line has the form {\tt t a b} where $t = 1$ (create a new edge) or $t = 2$ (remove an edge). A new edge is always created between $2$ nodes that do not already have an edge between them, \& only existing edges can get removed.
    \item {\sf Output.} Print $k + 1$ integers: 1st the number of components before the 1st event, \& after this the new number of components after each event.
    \item {\sf Constraints.} $n\in\overline{2,10^5},m,k\in[10^5],a,b\in[n]$.
    \item {\sf Sample.}
    \begin{table}[H]
        \centering
        \begin{tabular}{|l|l|}
            \hline
            \verb|dynamic_connectivity.inp| & \verb|dynamic_connectivity.out| \\
            \hline
            5 3 3 & 2 2 2 1 \\
            1 4 & \\
            2 3 & \\
            3 5 & \\
            1 2 5 & \\
            2 3 5 & \\
            1 1 2 & \\
            \hline
        \end{tabular}
    \end{table}
\end{problem}

\begin{problem}[\href{https://cses.fi/problemset/task/2121}{CSES Problem Set{\tt/}parcel delivery}]
    There are $n\in\mathbb{N}^\star$ cities \& $m\in\mathbb{N}^\star$ routes through which parcels can be carried from $1$ city to another city. For each route, you know the maximum number of parcels \& the cost of a single parcel. You want to send $k$ parcels from Syrjälä to Lehmälä. What is the cheapest way to do that?
    \item {\sf Input.} The 1st input line has $3$ integers $n,m,k\in\mathbb{N}^\star$: the number of cities, routes, \& parcels. The cities are numbered $1,2,\ldots,n$. City $1$ is Syrjälä \& city $n$ is Lehmälä. After this, there are $m$ lines that describe the routes. Each line has $4$ integers $a,b,r,c$: there is a route from city $a$ to city $b$, at most $r$ parcels can be carried through the route, \& the cost of each parcel is $c$.
    \item {\sf Output.} Print $1$ integer: the minimum total cost or $-1$ if there are no solutions.
    \item {\sf Constraints.} $2\le n\le500,m\in[1000],k\in[100],a,b\in[n],r,c\in[1000]$.
    \item {\sf Sample.}
    \begin{table}[H]
        \centering
        \begin{tabular}{|l|l|}
            \hline
            \verb|parcel_delivery.inp| & \verb|parcel_delivery.out| \\
            \hline
            4 5 3 & 750 \\
            1 2 5 100 & \\
            1 3 10 50 & \\
            1 4 7 500 & \\
            2 4 8 350 & \\
            3 4 2 100 & \\
            \hline
        \end{tabular}
    \end{table}
    \item {\sf Explanation.} 1 parcel is delivered through route $1\to2\to4$ (cost $1\cdot450 = 450$) \& $2$ parcels are delivered through route $1\to3\to4$ (cost $2\cdot150 = 300$).
\end{problem}

\begin{problem}[\href{https://cses.fi/problemset/task/2129}{CSES Problem Set{\tt/}task assignement}]
    A company has $n\in\mathbb{N}^\star$ employees \& there are $n$ tasks that need to be done. We know for each employee the cost of carrying out each task. Every employee should be assigned to exactly $1$ task. What is the minium total cost if we assign the task's optimally \& how could they be assigned?
    \item {\sf Input.} The 1st input line has $1$ integer $n\in\mathbb{N}^\star$: the number of employees \& the number of tasks that need to be done. After this, there are $n$ lines each consisting of $n$ integers. The $i$th line consists of integers $c_{i1},c_{i2},\ldots,c_{in}$: the cost of each task when it is assigned to the $i$th employee.
    \item {\sf Output.} 1st print the minimum total cost. Then print $n$ lines each consisting of $2$ integers $a,b$: you assign the $b$th task to the $a$th employee. If there are multiple solutions you can print any of them.
    \item {\sf Constraints.} $n\in[200],c_{ij}\in[1000]$.
    \item {\sf Sample.}
    \begin{table}[H]
        \centering
        \begin{tabular}{|l|l|}
            \hline
            \verb|task_assignement.inp| & \verb|task_assignement.out| \\
            \hline
            4 & 33 \\
            17 8 16 9 & 1 4 \\
            7 15 12 19 & 2 1 \\
            6 9 10 11 & 3 3 \\
            14 7 13 10 & 4 2 \\
            \hline
        \end{tabular}
    \end{table}
    \item {\sf Explanation.} The minium total cost is $33$. We can reach this by assigning employee 1 task 4, employee 2 task 1, employee 3 task 3, \& employee 4 task 2. This will cost $9 + 7 + 10 + 7 = 33$.
\end{problem}

\begin{problem}[\href{https://cses.fi/problemset/task/2130}{CSES Problem Set{\tt/}distinct routes II}]
    A game consists of $n\in\mathbb{N}^\star$ rooms \& $m\in\mathbb{N}^\star$ teleporters. At the beginning of each day, you start in room $1$ \& you have to reach room $n$. You can use each teleporter at most once during the game. You want to play the game for exactly $k$ days. Every time you use any teleporter you have to pay $1$ coin. What is the minimum number of coins you have to pay during $k$ days if you play optimally?
    \item {\sf Input.} The 1st input line has $3$ integers $n,m,k\in\mathbb{N}^\star$: the number of rooms, the number of teleporters \& the number of days you play the game. The rooms are numbered $1,2,\ldots,n$. After this, there are $m$ lines describing the teleporters. Each line has $2$ integers $a,b$: there is a teleporter from room $a$ to room $b$. There are no $2$ teleporters whose starting \& ending room are the same.
    \item {\sf Output.} 1st print $1$ integer: the minimum number of coins you have to pay if you pay optimally. Then, print $k$ route descriptions according to the example. You can print any valid solution. If it is not possible to play the game for $k$ days, print only $-1$.
    \item {\sf Constraints.} $2\le n\le500,m\in[1000],1\le k\le n - 1,1\le a,b\le n$.
    \item {\sf Sample.}
    \begin{table}[H]
        \centering
        \begin{tabular}{|l|l|}
            \hline
            \verb|distinct_routes_II.inp| & \verb|distinct_routes_II.out| \\
            \hline
            8 10 2 & 6 \\
            1 2 & 4 \\
            1 3 & 1 2 4 8 \\
            2 5 & 4 \\
            2 4 & 1 3 5 8 \\
            3 5 & \\
            3 6 & \\
            4 8 & \\
            5 8 & \\
            6 7  & \\
            7 8 & \\
            \hline
        \end{tabular}
    \end{table}
\end{problem}

%------------------------------------------------------------------------------%

\subsection{Additional Problems}

%------------------------------------------------------------------------------%

\section{Miscellaneous}

\subsection{Contributors}

\begin{enumerate}
	\item {\sc Võ Ngọc Trâm Anh}. C++ codes.
	\item {\sc Đặng Phúc An Khang}. C++ codes.
	\begin{itemize}
		\item {\sc Đặng Phúc An Khang}. {\it Combinatorics \& Number Theory in Competitive Programming -- Tổ Hợp \& Lý Thuyết Số trong Lập Trình Thi Đấu}.
		\item {\sc Đặng Phúc An Khang}. {\it Hướng Đến Kỳ Thi Olympic Tin học Sinh Viên Toàn Quốc \& ICPC 2025}.
		
		{\sc url}: \url{https://github.com/GrootTheDeveloper/OLP-ICPC/blob/master/2025/COMPETITIVE_REPORT.pdf}.
	\end{itemize}
	\item {\sc Nguyễn Lê Anh Khoa}. C++ codes.
	\item {\sc Phan Vĩnh Tiến}. C++ codes.
\end{enumerate}

\subsection{Donate or Buy Me Coffee}
Donate (but do not donut) or buy me some coffee via NQBH's bank account information at \url{https://github.com/NQBH/publication/blob/master/bank/NQBH_bank_account_information}.

\subsection{See also}

\begin{enumerate}
	\item {\it Vietnamese Mathematical Olympiad for High School- \& College Students (VMC) -- Olympic Toán Học Học Sinh \& Sinh Viên Toàn Quốc}.
	
	PDF: {\sc url}: \url{https://github.com/NQBH/advanced_STEM_beyond/blob/main/VMC/NQBH_VMC.pdf}.
	
	\TeX: {\sc url}: \url{https://github.com/NQBH/advanced_STEM_beyond/blob/main/VMC/NQBH_VMC.tex}.
	\begin{itemize}
		\item Codes:
		\begin{itemize}
			\item C++ code: \url{https://github.com/NQBH/advanced_STEM_beyond/tree/main/VMC/C++}.
			\item Python code: \url{https://github.com/NQBH/advanced_STEM_beyond/tree/main/VMC/Python}.
		\end{itemize}
		\item Resource: \url{https://github.com/NQBH/advanced_STEM_beyond/tree/main/VMC/resource}.
		\item Figures: \url{https://github.com/NQBH/advanced_STEM_beyond/tree/main/VMC/figure}.
	\end{itemize}
\end{enumerate}

%------------------------------------------------------------------------------%

\printbibliography[heading=bibintoc]
	
\end{document}