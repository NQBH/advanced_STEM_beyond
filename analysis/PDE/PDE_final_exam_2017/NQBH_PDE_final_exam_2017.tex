\documentclass[11pt,a4paper]{article}
\usepackage{longtable,float,hyperref,color,amsmath,amsxtra,amssymb,latexsym,amscd,amsthm,amsfonts,graphicx}
\numberwithin{equation}{section}
\textwidth=16 cm
\textheight=22 cm
\topmargin= -1 cm
\oddsidemargin=0 cm
\evensidemargin=1 cm
\parindent=0.6 cm
\parskip=1.5 mm
\allowdisplaybreaks
\usepackage{fancyhdr}
\pagestyle{fancy}
\fancyhf{}
\fancyhead[RE,LO]{\footnotesize \textsc \leftmark}
\cfoot{\thepage}
\renewcommand{\headrulewidth}{0.5pt}
\setcounter{tocdepth}{3}
\setcounter{secnumdepth}{3}
\usepackage{imakeidx}
\makeindex[columns=2, title=Alphabetical Index, 
           options= -s index.ist]

\usepackage[utf8]{inputenc} 
\usepackage[english]{babel} 
\usepackage{color}
\usepackage{eufrak,makeidx}
% For insert figure
\usepackage{graphicx,epstopdf,subfigure}
\numberwithin{equation}{section}
% No page break in Bibliography

\addto{\captionsenglish}{%
  \renewcommand{\bibname}{References}
}

\newtheorem{lemma}{Lemma}[section]
\newtheorem{corollary}{Corollary}[section]
\newtheorem{definition}{Definition}[section]
\newtheorem{prop}{Proposition}[section]
\newtheorem{theorem}{Theorem}[section]
\newtheorem{problem}{Problem}[section]
\newtheorem{notation}{Notation}[section]
\newtheorem{remark}{Remark}[section]
\newtheorem{example}{Example}[section]
\newtheorem{ques}{Question}[section]
\newtheorem{sol}{Solution}[section]
\renewcommand{\thenotation}{}
\renewcommand{\thesection}{\arabic{section}}
\renewcommand{\thesubsection}
{\arabic{section}.\arabic{subsection}}


%\newtheorem{theorem}{Theorem}[section]
%\newtheorem{corollary}[theorem]{Corollary}
%\newtheorem{lemma}[theorem]{Lemma}
%\newtheorem{proposition}[theorem]{Proposition}

%\theoremstyle{definition}
%\newtheorem{definition}[theorem]{Definition}
%\newtheorem{remark}{Remark}
%\newtheorem{properties}{Properties}
%\newtheorem*{notation}{Notation}
%\newtheorem{counter}{Counter-example}
%\newtheorem{open}{Open problem}
%\newtheorem{conjecture}{Conjecture}


%% FONT commands
\newcommand{\txt}[1]{\;\text{ #1 }\;}%% Used in math only
\newcommand{\tbf}{\textbf}%% Bold face. Usage: \tbf{...}
\newcommand{\tit}{\textit}%% Italic
\newcommand{\tsc}{\textsc}%% Small caps
\newcommand{\trm}{\textrm}
\newcommand{\mbf}{\mathbf}%% Math bold
\newcommand{\mrm}{\mathrm}%% Math Roman
\newcommand{\bsym}{\boldsymbol}%% Bold math symbol
%%Macros for changing font size in math.
\newcommand{\scs}{\scriptstyle}%% as in subscript
\newcommand{\sss}{\scriptscriptstyle}%% as in sub-subscript
\newcommand{\txts}{\textstyle}
\newcommand{\dsps}{\displaystyle}
%%Macros for changing font size in text.
\newcommand{\fnz}{\footnotesize}
\newcommand{\scz}{\scriptsize}
%%\tiny<\scz<\fsz<\small<\large<\Large<\huge<\Huge
%%%%%%%%%%%%
%%%%%%%%%%%%
%% EQUATION commands
\newcommand{\be}{\begin{equation}}
\newcommand{\bel}[1]{\begin{equation}\label{#1}}
\newcommand{\ee}{\end{equation}}
%% This macro does not work with amstex.
\newcommand{\eqnl}[2]{\begin{equation}\label{#1}{#2}\end{equation}}
%%use not advisable; confusing
\newcommand{\barr}{\begin{eqnarray}}
\newcommand{\earr}{\end{eqnarray}}
\newcommand{\bars}{\begin{eqnarray*}}
\newcommand{\ears}{\end{eqnarray*}}
\newcommand{\nnu}{\nonumber \\}
%%%%%%%%%%%%%%%
%% Unnumbered THEOREM env.
%% New env. to be used for unnumbered theorem, lemma etc.
%%(but with specified name)

\newtheorem{subn}{\name}
\renewcommand{\thesubn}{}
\newcommand{\bsn}[1]{\def\name{#1}\begin{subn}}
\newcommand{\esn}{\end{subn}}
%%%%%%%%%%%%%%
%% NUMBERED THEOREM env.
%% Environments: theorem, lemma, corollary defintion and
%%related commands,
%% designed to provide consecutive numbering of these forms.


\newtheorem{sub}{\name}[section]
\newcommand{\dn}[1]{\def\name{#1}}

%%used in conjuction with sub or subn.

\newcommand{\bs}{\begin{sub}}
\newcommand{\es}{\end{sub}}
\newcommand{\bsl}[1]{\begin{sub}\label{#1}}
	
%% the above must be preceeded by \dn (name definition),
%% however this is superceded by the list of commands bth etc. below.
%%%%%%%%%%%%
%% NUMBERED THEOREM env. (cont.)
%% List of commands derived from 'sub' env. for theorem, lemma etc.
%% designed to provide consecutive numbering of these forms.
\newcommand{\bth}[1]{\def\name{Theorem}\begin{sub}\label{t:#1}}
\newcommand{\blemma}[1]{\def\name{Lemma}\begin{sub}\label{l:#1}}
\newcommand{\bcor}[1]{\def\name{Corollary}\begin{sub}\label{c:#1}}	
\newcommand{\bdef}[1]{\def\name{Definition}\begin{sub}\label{d:#1}}
\newcommand{\bprop}[1]{\def\name{Proposition}\begin{sub}\label{p:#1}}	
%% ARRAY commands.%%%%%%%%%%%%%%%%%%%%%%%%%%%%%%%%%%
%% RERERENCE commands.
%% \newcommand{\R}[1]{(\ref{#1})}

\newcommand{\R}{\eqref}
\newcommand{\re}{\eqref}
\newcommand{\rth}[1]{Theorem~\ref{t:#1}}
\newcommand{\rlemma}[1]{Lemma~\ref{l:#1}}
\newcommand{\rcor}[1]{Corollary~\ref{c:#1}}
\newcommand{\rdef}[1]{Definition~\ref{d:#1}}
\newcommand{\rprop}[1]{Proposition~\ref{p:#1}}
%%%%%%%%%%%
\newcommand{\BA}{\begin{array}}
\newcommand{\EA}{\end{array}}
\newcommand{\BAN}{\renewcommand{\arraystretch}{1.2}
\setlength{\arraycolsep}{2pt}\begin{array}}
\newcommand{\BAV}[2]{\renewcommand{\arraystretch}{#1}
\setlength{\arraycolsep}{#2}\begin{array}}
%Note: The first variable gives the amount of stretching:
%(#1) x default.
%For instance #1=1.2 means a 20% stretching.
%The second variable should be
%written for instance in the form  4pt ; here the default is 5pt
%\newcommand{\EAN}{\end{array}\setlength{\arraycolsep}{5pt}}
\newcommand{\BSA}{\begin{subarray}}
\newcommand{\ESA}{\end{subarray}}	
%Note: These are used in subscripts as well as superscripts.
%They work essentially like 'array'.

\newcommand{\BAL}{\begin{aligned}}
	\newcommand{\EAL}{\end{aligned}}
\newcommand{\BALG}{\begin{alignat}}
	\newcommand{\EALG}{\end{alignat}}
%% the abbrev. does not work with latex2e
\newcommand{\BALGN}{\begin{alignat*}}
	\newcommand{\EALGN}{\end{alignat*}}
%% the abbrev. does not work with latex2e
%% The 'aligned' environment must be placed inside an 'equation' env.
%% in the same way as the array.
%% One could use also the 'align' env. or the 'alignat' env.
%% However in this case each line is numbered, unless '\notag' is used.
%% The 'alignat'
%% has a slightly different format (the number of columns must be %%specified in advance)
%% but it has the advantage that the distance between columns
%%is at our disposition.
%% (The default would be zero distance.) Using 'alignat*' we can have %%the advantages
%% of alignat plus the situation where separate lines are not numbered.
%% However in this case there is no numbering at all
%%(unless we provide a tag).
%%%%%%%%%%
%% PROOF, REMARK etc.
\newcommand{\note}[1]{\noindent\textit{#1.}\hspace{2mm}}
\newcommand{\Proof}{\note{Proof}}
%\newcommand{\qed}{\hspace{10mm}\hfill $\square$}
%\newcommand{\qed}{\\${}$ \hfill $\square$}
\newcommand{\Remark}{\note{Remark}}
%%%%%%%% Style command.
\newcommand{\modin}{$\,$\\[-4mm] \indent}
%% To be used after \section in order to start new line with \indent.
%%%%%%%%%%%%
%% MATHEMATICAL symbols
\newcommand{\forevery}{\quad \forall}
\newcommand{\set}[1]{\{#1\}}
\newcommand{\setdef}[2]{\{\,#1:\,#2\,\}}
\newcommand{\setm}[2]{\{\,#1\mid #2\,\}}
%% Arrows
\newcommand{\mt}{\mapsto}
\newcommand{\lra}{\longrightarrow}
\newcommand{\lla}{\longleftarrow}
\newcommand{\llra}{\longleftrightarrow}
\newcommand{\Lra}{\Longrightarrow}
\newcommand{\Lla}{\Longleftarrow}
\newcommand{\Llra}{\Longleftrightarrow}
\newcommand{\warrow}{\rightharpoonup}

%% Brackets, delimiters
\newcommand{\paran}[1]{\left (#1 \right )}
%% adjustable parantheses
\newcommand{\sqbr}[1]{\left [#1 \right ]}
%% adjustable square brackets
\newcommand{\curlybr}[1]{\left \{#1 \right \}}
%% adjustable curly brackets
\newcommand{\abs}[1]{\left |#1\right |}

%% adjustable vertical delimiters
\newcommand{\norm}[1]{\left \|#1\right \|}

%% adjustable norm
\newcommand{\paranb}[1]{\big (#1 \big )}

%% non-adjustable parantheses (big)
\newcommand{\lsqbrb}[1]{\big [#1 \big ]}

%% non-adjustable square brackets (big)
\newcommand{\lcurlybrb}[1]{\big \{#1 \big \}}

%% non-adjustable curly brackets(big)
\newcommand{\absb}[1]{\big |#1\big |}

%% non-adjustable vertical delimiters(big)
\newcommand{\normb}[1]{\big \|#1\big \|}

%% non-adjustable norm (big)
\newcommand{	\paranB}[1]{\Big (#1 \Big )}

%% non-adjustable parantheses (Big)
\newcommand{\absB}[1]{\Big |#1\Big |}

%% non-adjustable vertical delimiters(Big)
\newcommand{\normB}[1]{\Big \|#1\Big \|}%% non-adjustable norm (Big)
\newcommand{\produal}[1]{\langle #1 \rangle}%% the pairing of X' and X
%%%%%%%%%%%%%%%%%
%% Adjustable parantheses etc. in a different DEFINITION format.
%\def\adp(#1){\left (#1 \right )}%% adjustable parantheses
%\def\adsb(#1){\left [#1\right ]}%% adjustable square brackets
%\def\adcb(#1){\left \{#1\right \}}%% adjustable curly brackets
%\def\abs|#1|{\left |#1\right |}%% adjustable vertical delimiters
%%%%%%%%%%%%%%%%
%% More mathematical symbols
\newcommand{\thkl}{\rule[-.5mm]{.3mm}{3mm}}
\newcommand{\thknorm}[1]{\thkl #1 \thkl\,}
\newcommand{\trinorm}[1]{|\!|\!| #1 |\!|\!|\,}
\newcommand{\bang}[1]{\langle #1 \rangle}%% angle bracket
\def\angb<#1>{\langle #1 \rangle}%% angle bracket
%% The two last lines yield the same result.
%% The second is used as follows: \angb<a,b>
\newcommand{\vstrut}[1]{\rule{0mm}{#1}}
\newcommand{\rec}[1]{\frac{1}{#1}}
%% OPERATOR names.
%% OPERATOR names.
\newcommand{\opname}[1]{\mbox{\rm #1}\,}
\newcommand{\supp}{\opname{supp}}
\newcommand{\dist}{\opname{dist}}
\newcommand{\myfrac}[2]{{\displaystyle \frac{#1}{#2} }}
\newcommand{\myint}[2]{{\displaystyle \int_{#1}^{#2}}}
\newcommand{\mysum}[2]{{\displaystyle \sum_{#1}^{#2}}}
\newcommand {\dint}{{\displaystyle \myint\!\!\myint}}%%%%%%%%%%
%%%%%%% SPACE commands
\newcommand{\q}{\quad}
\newcommand{\qq}{\qquad}
\newcommand{\hsp}[1]{\hspace{#1mm}}
\newcommand{\vsp}[1]{\vspace{#1mm}}
%%%%%%%%%%%
%% ABREVIATIONS
\newcommand{\ity}{\infty}
\newcommand{\prt}{\partial}
\newcommand{\sms}{\setminus}
\newcommand{\ems}{\emptyset}
\newcommand{\ti}{\times}
\newcommand{\pr}{^\prime}
\newcommand{\ppr}{^{\prime\prime}}
\newcommand{\tl}{\tilde}
\newcommand{\sbs}{\subset}
\newcommand{\sbeq}{\subseteq}
\newcommand{\nind}{\noindent}
\newcommand{\ind}{\indent}
\newcommand{\ovl}{\overline}
\newcommand{\unl}{\underline}
\newcommand{\nin}{\not\in}
\newcommand{\pfrac}[2]{\genfrac{(}{)}{}{}{#1}{#2}}

%% frac with parantheses.
%%%%%%%%%%%
%%%%%%%%%%%%%

%%Macros for Greek letters.
\def\ga{\alpha}     \def\gb{\beta}       \def\gg{\gamma}
\def\gc{\chi}       \def\gd{\delta}      \def\gep{\epsilon}
\def\gth{\theta}                         \def\vge{\varepsilon}
\def\gf{\phi}       \def\vgf{\varphi}    \def\gh{\eta}
\def\gi{\iota}      \def\gk{\kappa}      \def\gl{\lambda}
\def\gm{\mu}        \def\gn{\nu}         \def\gp{\pi}
\def\vgp{\varpi}    \def\gr{\gd}        \def\vgr{\varrho}
\def\gs{\sigma}     \def\vgs{\varsigma}  \def\gt{\tau}
\def\gu{\upsilon}   \def\gv{\vartheta}   \def\gw{\omega}
\def\gx{\xi}        \def\gy{\psi}        \def\gz{\zeta}
\def\Gg{\Gamma}     \def\Gd{\Delta}      \def\Gf{\Phi}
\def\Gth{\Theta}
\def\Gl{\Lambda}    \def\Gs{\Sigma}      \def\Gp{\Pi}
\def\Gw{\Omega}     \def\Gx{\Xi}         \def\Gy{\Psi}

%%Macros for calligraphic letters.
\def\CS{{\mathcal S}}   \def\CM{{\mathcal M}}   \def\CN{{\mathcal N}}
\def\CR{{\mathcal R}}   \def\CO{{\mathcal O}}   \def\CP{{\mathcal P}}
\def\CA{{\mathcal A}}   \def\CB{{\mathcal B}}   \def\CC{{\mathcal C}}
\def\CD{{\mathcal D}}   \def\CE{{\mathcal E}}   \def\CF{{\mathcal F}}
\def\CG{{\mathcal G}}   \def\CH{{\mathcal H}}   \def\CI{{\mathcal I}}
\def\CJ{{\mathcal J}}   \def\CK{{\mathcal K}}   \def\CL{{\mathcal L}}
\def\CT{{\mathcal T}}   \def\CU{{\mathcal U}}   \def\CV{{\mathcal V}}
\def\CZ{{\mathcal Z}}   \def\CX{{\mathcal X}}   \def\CY{{\mathcal Y}}
\def\CW{{\mathcal W}} \def\CQ{{\mathcal Q}}
%%%%%
%%Macros for 'blackboard' letters (See (27) for display.)
\def\BBA {\mathbb A}   \def\BBb {\mathbb B}    \def\BBC {\mathbb C}
\def\BBD {\mathbb D}   \def\BBE {\mathbb E}    \def\BBF {\mathbb F}
\def\BBG {\mathbb G}   \def\BBH {\mathbb H}    \def\BBI {\mathbb I}
\def\BBJ {\mathbb J}   \def\BBK {\mathbb K}    \def\BBL {\mathbb L}
\def\BBM {\mathbb M}   \def\BBN {\mathbb N}    \def\BBO {\mathbb O}
\def\BBP {\mathbb P}   \def\BBR {\mathbb R}    \def\BBS {\mathbb S}
\def\BBT {\mathbb T}   \def\BBU {\mathbb U}    \def\BBV {\mathbb V}
\def\BBW {\mathbb W}   \def\BBX {\mathbb X}    \def\BBY {\mathbb Y}
\def\BBZ {\mathbb Z}

%%Macros for Ghotic (Fraktur) letters.
\def\GTA {\mathfrak A}   \def\GTB {\mathfrak B}    \def\GTC {\mathfrak C}
\def\GTD {\mathfrak D}   \def\GTE {\mathfrak E}    \def\GTF {\mathfrak F}
\def\GTG {\mathfrak G}   \def\GTH {\mathfrak H}    \def\GTI {\mathfrak I}
\def\GTJ {\mathfrak J}   \def\GTK {\mathfrak K}    \def\GTL {\mathfrak L}
\def\GTM {\mathfrak M}   \def\GTN {\mathfrak N}    \def\GTO {\mathfrak O}
\def\GTP {\mathfrak P}   \def\GTR {\mathfrak R}    \def\GTS {\mathfrak S}
\def\GTT {\mathfrak T}   \def\GTU {\mathfrak U}    \def\GTV {\mathfrak V}
\def\GTW {\mathfrak W}   \def\GTX {\mathfrak X}    \def\GTY {\mathfrak Y}
\def\GTZ {\mathfrak Z}   \def\GTQ {\mathfrak Q}
\def\sign{\mathrm{sign\,}}
\def\bdw{\prt\Gw\xspace}
\def\nabu{|\nabla u|}
\def\tr{\mathrm{tr\,}}
\def\gap{{\ga_+}}
\def\gan{{\ga_-}}

\def\N{\mathbb{N}}
\def\Z{\mathbb{Z}}
\def\Q{\mathbb{Q}}
\def\R{\mathbb{R}}


\def\Proof.{{\bf{Proof. }}}
\def\End{\hspace{1cm} $\Box$\\}


\renewcommand{\baselinestretch}{1.1}

\let\e=\varepsilon
\let\vp=\varphi
\let\t=\tilde
\let\ol=\overline
\let\ul=\underline
\let\.=\cdot
\let\0=\emptyset
\let\mc=\mathcal
\def\ex{\exists\;}
\def\fa{\forall\;}
\def\se{\ \Leftarrow\ }
\def\solose{\ \Rightarrow\ }
\def\sse{\ \Leftrightarrow\ }
\def\meno{\,\backslash\,}
\def\pp{,\dots,}
\def\D{\mc{D}}
\def\O{\Omega}


\def\loc{\text{\rm loc}}
\def\diam{\text{\rm diam}}
\def\dist{\text{\rm dist}}
\def\dv{\text{\rm div}}
\def\sign{\text{\rm sign}}
\def\supp{\text{\rm supp}}
\def\tr{\text{\rm Tr}}
\def\vec{\text{\rm vec}}
\def\inter{\text{\rm int\,}}
\def\norma#1{\|#1\|_\infty}

\newcommand{\esssup}{\mathop{\rm ess{\,}sup}}
\newcommand{\essinf}{\mathop{\rm ess{\,}inf}}
\newcommand{\su}[2]{\genfrac{}{}{0pt}{}{#1}{#2}}

\def\eq#1{{\rm(\ref{eq:#1})}}
\def\thm#1{Theorem \ref{thm:#1}}
\def\seq#1{(#1_n)_{n\in\N}}
\def\limn{\lim_{n\to\infty}}


\def\PP{\mc{P}}
\def\pe{principal eigenvalue}
\def\MP{maximum principle}
\def\SMP{strong maximum principle}
\def\l{\lambda_1}

\def\bq{\begin{equation}}
\def\eq{\end{equation}}

\def\l{\label}

\newenvironment{formula}[1]{\begin{equation}\label{eq:#1}}	{\end{equation}\noindent}


\title{PDE Final Exam 2017}
\author{\textsc{Nguyen Quan Ba Hong}\\
{\small Students at Faculty of Math and Computer Science,}\\ 
{\small Ho Chi Minh University of Science, Vietnam} \\
{\small \texttt{email. nguyenquanbahong@gmail.com}}\\
{\small \texttt{blog. \url{www.nguyenquanbahong.com}} 
\footnote{Copyright \copyright\ 2016-2018 by Nguyen Quan Ba Hong, Student at Ho Chi Minh University of Science, Vietnam. This document may be copied freely for the purposes of education and non-commercial research. Visit my site \texttt{\url{www.nguyenquanbahong.com}} to get more.}}}
\begin{document}
\maketitle
\begin{abstract}
This context aims at solving the problems given in the PDE Final Exam 2017, posed by Prof. Dang Duc Trong.
\end{abstract}
\newpage
\tableofcontents
\newpage
\section{Problems}
\begin{problem}\label{problem1.1}
Let $a>0$, $b>0$, $\Omega =\left(0,a\right)\times \left(0,b\right)$, ${S_1} = \left[ {0,a} \right] \times \left\{ 0 \right\}$, ${S_2} = \partial \Omega \backslash {S_1}$, $f\in L^2\left(\Omega\right)$. Consider the following equation
\begin{align}
\label{1.1}
Lu =f \mbox{ in } \Omega ,
\end{align}
where 
\begin{align}
Lu \equiv  - \frac{\partial }{{\partial x}}\left( {\left( {1 + {x^2}} \right){u_x}} \right) - \frac{\partial }{{\partial y}}\left( {\left( {1 + {y^2}} \right){u_y}} \right),
\end{align}
with the boundary conditions ${\left. u \right|_{{S_2}}} = 0$, ${\left. {{u_y}} \right|_{{S_1}}} = 0$.
\begin{enumerate}
\item Find the weak formulation of this boundary value problem on the solution space $V$ needing determining.
\item Use the equality 
\begin{align}
{u^2}\left( {x,y} \right) = 2\int_0^x {{u_x}\left( {s,y} \right)u\left( {s,y} \right)ds},
\end{align}
to prove that there exists $C>0$ such that
\begin{align}
\label{1.4}
{\left\| {{u_x}} \right\|_2^2} + {\left\| {{u_y}} \right\|_2^2} \ge C\left\| u \right\|_{{H^1}\left( \Omega  \right)}^2 \mbox{ for all } u \in V.
\end{align}
\item Prove that \eqref{1.1} has a weak solution in $V$ by using Lax-Milgram theorem.
\item Suppose that this weak solution, say $\bar u$, satisfies $\bar u \in {H^2} \cap V$, prove that this solution $\bar u$ satisfies the given problem.
\item Write the functional $J: V\to \mathbb{R}$ such that $u = \arg {\min _{w \in V}}J\left( w \right)$. Which boundary value problem does the minimum of $J$ satisfy if $V$ is replaced by $H^1$?
\end{enumerate}
\end{problem} 
\noindent 
\textsc{Solution.}
\begin{enumerate}
\item The complementary arcs $S_1$ and $S_2$ can be written as 
\begin{align}
{S_1} &= \left\{ {\left( {x,y} \right) \in \bar \Omega ;y = 0} \right\},\\
{S_2} &= \left\{ {\left( {x,y} \right) \in \bar \Omega ;x = 0 \mbox{ or } x = a \mbox{ or } y = a} \right\}\backslash \left\{ {\left( {0,0} \right) \cup \left( {a,0} \right)} \right\} .
\end{align}
For arbitrary $u$ and $v$ in $C^2\left(\Omega\right)$, the integration by parts formula gives us
\begin{align}
\left\langle {Lu,v} \right\rangle  =&  - \int_\Omega  {\frac{\partial }{{\partial x}}\left( {\left( {1 + {x^2}} \right){u_x}} \right)vd\Omega }  - \int_\Omega  {\frac{\partial }{{\partial y}}\left( {\left( {1 + {y^2}} \right){u_y}} \right)vd\Omega } \\
 =&  - \int_0^b {\int_0^a {\frac{\partial }{{\partial x}}\left( {\left( {1 + {x^2}} \right){u_x}} \right)vdxdy} }  - \int_0^a {\int_0^b {\frac{\partial }{{\partial y}}\left( {\left( {1 + {y^2}} \right){u_y}} \right)vdxdy} } \\
 =&  - \int_0^b {\left[ {\left. {\left( {1 + {x^2}} \right){u_x}v} \right|_{x = 0}^{x = a} - \int_0^a {\left( {1 + {x^2}} \right){u_x}{v_x}dx} } \right]dy} \\
& - \int_0^a {\left[ {\left. {\left( {1 + {y^2}} \right){u_y}v} \right|_{y = 0}^{y = b} - \int_0^b {\left( {1 + {y^2}} \right){u_y}{v_y}dy} } \right]dx} \\
 =&\ \int_\Omega  {\left[ {\left( {1 + {x^2}} \right){u_x}{v_x} + \left( {1 + {y^2}} \right){u_y}{v_y}} \right]d\Omega } \\
& - \int_0^b {\left. {\left( {1 + {x^2}} \right){u_x}v} \right|_{x = 0}^{x = a}dy}  - \int_0^a {\left. {\left( {1 + {y^2}} \right){u_y}v} \right|_{y = 0}^{y = b}dx} .
\end{align}
Now define 
\begin{align}
V: = \left\{ {v \in {H^1}\left( \Omega  \right);v = 0 \mbox{ on } {S_2}} \right\}.
\end{align}
and note that for $u \in V$, $v\in V$, 
\begin{align}
\int_0^b {\left. {\left( {1 + {x^2}} \right){u_x}v} \right|_{x = 0}^{x = a}dy}  + \int_0^a {\left. {\left( {1 + {y^2}} \right){u_y}v} \right|_{y = 0}^{y = b}dx}  = \int_0^a {\left( {1 + {y^2}} \right){u_y}\left( {x,0} \right)v\left( {x,0} \right)dx} .
\end{align}
Moreover, if $u \in V$ satisfies ${\left. {{u_y}} \right|_{{S_1}}} = 0$, then 
\begin{align}
\int_0^a {\left( {1 + {y^2}} \right){u_y}\left( {x,0} \right)v\left( {x,0} \right)dx}  = 0.
\end{align}
Thus, if $u$ is a classical solution of the given boundary value problem, $u$ must satisfy its weak formulation given by
\begin{align}
K\left[ {u,v} \right] = F\left[ v \right] \mbox{ for all } v \in V,
\end{align}
where, for $u\in V$ and $v\in V$,
\begin{align}
K\left[ {u,v} \right]: &= \int_\Omega  {\left[ {\left( {1 + {x^2}} \right){u_x}{v_x} + \left( {1 + {y^2}} \right){u_y}{v_y}} \right]d\Omega } ,\\
F\left[ v \right]: &= \int_\Omega  {fvd\Omega } .
\end{align}
\item Here are two solutions.

\textit{Solution 1.} Recall that $\left\| u \right\|_{{H^1}}^2: = \left\| u \right\|_2^2 + \left\| {{u_x}} \right\|_2^2 + \left\| {{u_y}} \right\|_2^2$, let $u\in V$ be given. Since $u\left(0,y\right)=0$ for all $y\in \left(0,b\right]$, we have
\begin{align}
2\int_0^x {{u_x}\left( {s,y} \right)u\left( {s,y} \right)ds}  &= 2\int_{u\left( {0,y} \right)}^{u\left( {x,y} \right)} {sds} \\
& = {u^2}\left( {x,y} \right) - {u^2}\left( {0,y} \right)\\
& = {u^2}\left( {x,y} \right) \mbox{ for all } y \in \left( {0,b} \right].
\end{align}
Then using successively Cauchy-Schwarz inequality for integrals and Cauchy inequality $2ab\le a^2+b^2$ yields
\begin{align}
{u^2}\left( {x,y} \right) &= 2\int_0^x {{u_x}\left( {s,y} \right)u\left( {s,y} \right)ds} \\
& \le 2{\left( {\int_0^x {\frac{{u_x^2\left( {s,y} \right)}}{M}ds} } \right)^{\frac{1}{2}}}{\left( {\int_0^x {M{u^2}\left( {s,y} \right)ds} } \right)^{\frac{1}{2}}} \\
& \le \int_0^x {\frac{{u_x^2\left( {s,y} \right)}}{M}ds}  + \int_0^x {M{u^2}\left( {s,y} \right)ds} ,
\end{align}
for all $\left(x,y\right)\in \left[0,a\right]\times \left(0,b\right]$, where $M$ is a positive constant depending only on $a$, $b$. Thus,
\begin{align}
\left\| u \right\|_2^2 &= \int_0^a {\left( {\int_0^b {{u^2}\left( {x,y} \right)dy} } \right)dx} \\
& \le \int_0^a {\left( {\int_0^b {\int_0^x {\frac{{u_x^2\left( {s,y} \right)}}{M}ds} dy}  + \int_0^b {\int_0^x {M{u^2}\left( {s,y} \right)ds} dy} } \right)dx} \\
& \le \int_0^a {\left( {\int_0^b {\int_0^a {\frac{{u_x^2\left( {s,y} \right)}}{M}ds} dy}  + \int_0^b {\int_0^a {M{u^2}\left( {s,y} \right)ds} dy} } \right)dx} \\
& = \int_0^a {\left( {\frac{1}{M}\left\| {{u_x}} \right\|_2^2 + M\left\| u \right\|_2^2} \right)dx} \\
& = \frac{a}{M}\left\| {{u_x}} \right\|_2^2 + aM\left\| u \right\|_2^2.
\end{align}
Now we choose $M$ such that $aM<1$, for instance, $M=\frac{1}{2a}$. Then the last estimate implies that 
\begin{align}
\label{1.30}
\left\| u \right\|_2^2 \le 4{a^2}\left\| {{u_x}} \right\|_2^2.
\end{align}
Similarly, since $u\left(x,b\right)=0$ for all $x\in \left[0,a\right]$, we have
\begin{align}
2\int_y^b {{u_y}\left( {x,r} \right)u\left( {x,r} \right)dr} & = 2\int_{u\left( {x,y} \right)}^{u\left( {x,b} \right)} {rdr} \\
& = {u^2}\left( {x,b} \right) - {u^2}\left( {x,y} \right)\\
& =  - {u^2}\left( {x,y} \right) \mbox{ for all } x \in \left[ {0,a} \right].
\end{align}
Then
\begin{align}
{u^2}\left( {x,y} \right) &=  - 2\int_y^b {{u_y}\left( {x,r} \right)u\left( {x,r} \right)dr} \\
& \le 2{\left( {\int_y^b {\frac{{u_y^2\left( {x,r} \right)}}{N}dr} } \right)^{\frac{1}{2}}}{\left( {\int_y^b {N{u^2}\left( {x,r} \right)dr} } \right)^{\frac{1}{2}}}\\
& \le \int_y^b {\frac{{u_y^2\left( {x,r} \right)}}{N}dr}  + \int_y^b {N{u^2}\left( {x,r} \right)dr} ,
\end{align}
for all $\left(x,y\right) \in \overline \Omega$, where $N$ is a positive constant depending only on $a$, $b$, and thus
\begin{align}
\left\| u \right\|_2^2 &= \int_0^b {\left( {\int_0^a {{u^2}\left( {x,y} \right)dx} } \right)dy} \\
& \le \int_0^b {\left( {\int_0^a {\int_y^b {\frac{{u_y^2\left( {x,r} \right)}}{N}dr} dx}  + \int_0^a {\int_y^b {N{u^2}\left( {x,r} \right)dr} dx} } \right)dy} \\
& \le \int_0^b {\left( {\int_0^a {\int_0^b {\frac{{u_y^2\left( {x,r} \right)}}{N}dr} dx}  + \int_0^a {\int_0^b {N{u^2}\left( {x,r} \right)dr} dx} } \right)dy} \\
& = \int_0^b {\left( {\frac{1}{N}\left\| {{u_y}} \right\|_2^2 + N\left\| u \right\|_2^2} \right)dy} \\
& = \frac{b}{N}\left\| {{u_y}} \right\|_2^2 + bN\left\| u \right\|_2^2.
\end{align}
Now we choose $N$ such that $bN<1$, for instance, $N=\frac{1}{2b}$. Then the last estimate implies that 
\begin{align}
\label{1.42}
\left\| u \right\|_2^2 \le 4{b^2}\left\| {{u_y}} \right\|_2^2.
\end{align}
Combining \eqref{1.30} and \eqref{1.42} yields
\begin{align}
\left\| {{u_x}} \right\|_2^2 + \left\| {{u_y}} \right\|_2^2 \ge \frac{1}{4}\left( {\frac{1}{{{a^2}}} + \frac{1}{{{b^2}}}} \right)\left\| u \right\|_2^2.
\end{align}
We now choose $C>0$ such that 
\begin{align}
\frac{C}{{1 - C}} = \frac{1}{4}\left( {\frac{1}{{{a^2}}} + \frac{1}{{{b^2}}}} \right) ,
\end{align}
i.e., 
\begin{align}
C = \frac{{{a^2} + {b^2}}}{{{a^2} + {b^2} + 4{a^2}{b^2}}},
\end{align}
then \eqref{1.4} holds.

\textit{Solution 2.} We only need the inequality $\left\| u \right\|_2^2 \le 4{a^2}\left\| {{u_x}} \right\|_2^2$ in the previous solution, 
\begin{align}
\left\| {{u_x}} \right\|_2^2 + \left\| {{u_y}} \right\|_2^2 &\ge \frac{1}{2}\left\| {{u_x}} \right\|_2^2 + \frac{1}{{8{a^2}}}\left\| u \right\|_2^2 + \left\| {{u_y}} \right\|_2^2\\
& \ge \min \left\{ {\frac{1}{2},\frac{1}{{8{a^2}}}} \right\}\left( {\left\| {{u_x}} \right\|_2^2 + \left\| u \right\|_2^2 + \left\| {{u_y}} \right\|_2^2} \right)\\
& = \min \left\{ {\frac{1}{2},\frac{1}{{8{a^2}}}} \right\}\left\| u \right\|_{{H^1}}^2.
\end{align}
\item Consider the defined bilinear form $K\left[u,v\right]$, it is continuous since
\begin{align}
K\left[ {u,v} \right] &= \int_\Omega  {\left[ {\left( {1 + {x^2}} \right){u_x}{v_x} + \left( {1 + {y^2}} \right){u_y}{v_y}} \right]d\Omega } \\
& \le \max \left\{ {1 + {a^2},1 + {b^2}} \right\}\int_\Omega  {\left| {{u_x}{v_x} + {u_y}{v_y}} \right|d\Omega } \\
& \le \left( {1 + \max {{\left\{ {a,b} \right\}}^2}} \right)\int_\Omega  {{{\left( {u_x^2 + u_y^2} \right)}^{\frac{1}{2}}}{{\left( {v_x^2 + v_y^2} \right)}^{\frac{1}{2}}}d\Omega } \\
& \le \left( {1 + \max {{\left\{ {a,b} \right\}}^2}} \right){\left( {\int_\Omega  {\left( {u_x^2 + u_y^2} \right)d\Omega } } \right)^{\frac{1}{2}}}{\left( {\int_\Omega  {\left( {v_x^2 + v_y^2} \right)d\Omega } } \right)^{\frac{1}{2}}}\\
& \le \left( {1 + \max {{\left\{ {a,b} \right\}}^2}} \right){\left( {\int_\Omega  {\left( {{u^2} + u_x^2 + u_y^2} \right)d\Omega } } \right)^{\frac{1}{2}}}{\left( {\int_\Omega  {\left( {{v^2} + v_x^2 + v_y^2} \right)d\Omega } } \right)^{\frac{1}{2}}}\\
& = \left( {1 + \max {{\left\{ {a,b} \right\}}^2}} \right){\left\| u \right\|_{{H^1}}}{\left\| v \right\|_{{H^1}}},\hspace{0.2cm}\forall u,v \in V,
\end{align}
and it is coercive since 
\begin{align}
K\left[ {u,u} \right] &= \int_\Omega  {\left[ {\left( {1 + {x^2}} \right)u_x^2 + \left( {1 + {y^2}} \right)u_y^2} \right]d\Omega } \\
& \ge \int_\Omega  {\left( {u_x^2 + u_y^2} \right)d\Omega } \\
& = \left\| {{u_x}} \right\|_2^2 + \left\| {{u_y}} \right\|_2^2\\
& \ge C \left\| u \right\|_{{H_1}}^2, \hspace{0.2cm}\forall u \in V ,
\end{align}
where $C$ is the constant given in Solution 1 or Solution 2 of the previous result. 

It is easy to prove that $F \in V^\star$\footnote{The notation $V^\star$ denotes the the \textit{dual space} of $V$, that is, the space of all \textit{continuous linear functionals} on $V$, see, e.g., \cite[p. 3]{Haim}.}. Indeed, $F$ is linear since
\begin{align}
F\left[ {\alpha u + \beta v} \right] = \int_\Omega  {f\left( {\alpha u + \beta v} \right)d\Omega }  = \alpha F\left[ u \right] + \beta F\left[ v \right],
\end{align}
for all $\alpha \in \mathbb{R}$, $\beta \in \mathbb{R}$, and $u\in V$, $v\in V$. It is also continuous because
\begin{align}
F\left[ v \right] = \int_\Omega  {fvd\Omega }  \le {\left\| f \right\|_2}{\left\| v \right\|_2} \le {\left\| f \right\|_2}{\left\| v \right\|_{{H^1}}},\hspace{0.2cm}\forall v \in V.
\end{align}
Now applying Lax-Milgram theorem (see, e.g., \cite[Corollary 5.8, p. 140]{Haim}) yields that there exists a unique element $\bar u\in V$ such that 
\begin{align}
\label{1.61}
K\left[ {\bar u,v} \right] = F\left[ v \right], \hspace{0.2cm}\forall v \in V.
\end{align}
\item Suppose that $\bar u$ satisfies \eqref{1.61} and $\bar u \in {H^2} \cap V$, integrating by parts \eqref{1.61} again gives us
\begin{align}
 - \int_\Omega  {\left[ {\frac{\partial }{{\partial x}}\left( {\left( {1 + {x^2}} \right){{\bar u}_x}} \right) + \frac{\partial }{{\partial y}}\left( {\left( {1 + {y^2}} \right){{\bar u}_y}} \right)} \right]vd\Omega }  - \int_0^a {{{\bar u}_y}\left( {x,0} \right)v\left( {x,0} \right)dx}  = \int_\Omega  {fvd\Omega } ,
\end{align}
for all $v\in V$, or equivalently,
\begin{align}
\label{1.63}
\left\langle {L\bar u,v} \right\rangle  - \int_0^a {{{\bar u}_y}\left( {x,0} \right)v\left( {x,0} \right)dx}  = F\left[ v \right],\hspace{0.2cm} \forall v \in V.
\end{align}
We now consider the function $\bar v: = {{\bar u}_y}{\chi _{{S_1}}}$, i.e.,
\begin{equation}
\bar v\left( {x,y} \right): = \left\{ \begin{split}
& 0, &\mbox{ if } \left( {x,y} \right) &\in \overline \Omega \backslash {S_1},\\
& {{\bar u}_y}\left( {x,0} \right), &\mbox{ if } \left(x,y\right) &\in S_1,
\end{split} \right.
\end{equation}
Since $\bar v =0$ a.e. in $\Omega$ and ${\left. {\bar v} \right|_{{S_2}}} = 0$, we have $\bar v \in V$. Plugging $v=\bar v$ into \eqref{1.63} yields 
\begin{align}
\int_0^a {\bar u_y^2\left( {x,0} \right)dx}  = 0,
\end{align}
which implies ${\left. {{\bar u_y}} \right|_{{S_1}}} = 0$ and thus $\bar u$ satisfies the given boundary conditions. Substituting $\bar u_y \left(x,0\right) =0$ for all $x\in \left[0,a\right]$ back to \eqref{1.63} yields 
\begin{align}
\left\langle {L\bar u,v} \right\rangle  = F\left[ v \right],\hspace{0.2cm}\forall v \in V ,
\end{align}
In particular,
\begin{align}
\left\langle {L\bar u - f,v} \right\rangle  = 0,\hspace{0.2cm}\forall v \in C_c^\infty \left( \Omega  \right).
\end{align}
It follows (see \cite[Corollary 4.24, p. 110]{Haim}) that $L\bar u = f$ a.e. on $\Omega$. Therefore, $\bar u$ satisfies the given boundary value problem almost everywhere.\footnote{If the stronger assumption $\bar u \in {C^2}\left( \Omega  \right) \cap V$ is active, then the validity of the equality $L\bar u=f$ can be passed from ``almost everywhere'' to ``everywhere'' in $\Omega$ by the smoothness of $\bar u$.}
\item Since $K\left[ { \cdot , \cdot } \right]$ is also symmetric, the later statement of Lax-Milgram theorem \ref{theorem1.1} gives us $\bar u = \arg {\min _{w \in V}}J\left( w \right)$, where the functional $J\left(\cdot\right)$ is defined by
\begin{align}
J\left( w \right): &= \frac{1}{2}K\left[ {w,w} \right] - F\left[ w \right]\\
& = \frac{1}{2}\int_\Omega  {\left[ {\left( {1 + {x^2}} \right)w_x^2 + \left( {1 + {y^2}} \right)w_y^2 - 2fw} \right]d\Omega } .
\end{align}
If $V$ is replaced by $H^1$, we denote $\widetilde u: = \arg {\min _{w \in {H^1}}}J\left( w \right)$. Both the domain $\mathcal{A}$ of the functional $J$ and a set $\mathcal{M}$ of \textit{comparison functions} are set as $H^1$, i.e., $\mathcal{A}=\mathcal{M}=H^1\left(\Omega\right)$ (see, e.g., \cite[p. 189]{Schaum}). Notice that $H^1\left(\Omega\right)$ is dense in $L^2\left(\Omega\right)$. For all $u \in H^1\left(\Omega\right)$, $v\in H^1\left(\Omega\right)$, we have
\begin{align}
& \frac{{J\left[ {u + \varepsilon v} \right] - J\left[ u \right]}}{\varepsilon } \\
&= \frac{1}{{2\varepsilon }}\left[ \begin{array}{l}
\int_\Omega  {\left[ {\left( {1 + {x^2}} \right){{\left( {{u_x} + \varepsilon {v_x}} \right)}^2} + \left( {1 + {y^2}} \right){{\left( {{u_y} + \varepsilon {v_y}} \right)}^2} - 2f\left( {u + \varepsilon v} \right)} \right]d\Omega } \\
 - \int_\Omega  {\left[ {\left( {1 + {x^2}} \right)u_x^2 + \left( {1 + {y^2}} \right)u_y^2 - 2fu} \right]d\Omega } 
\end{array} \right]\\
& = \frac{1}{{2\varepsilon }}\left[ {\int_\Omega  {\left[ {\left( {1 + {x^2}} \right)\left( {2\varepsilon {u_x}{v_x} + {\varepsilon ^2}v_x^2} \right) + \left( {1 + {y^2}} \right)\left( {2\varepsilon {u_y}{v_y} + {\varepsilon ^2}v_y^2} \right) - 2\varepsilon fv} \right]d\Omega } } \right]\\
& = \frac{1}{2}\left[ {\int_\Omega  {\left[ {\left( {1 + {x^2}} \right)\left( {2{u_x}{v_x} + \varepsilon v_x^2} \right) + \left( {1 + {y^2}} \right)\left( {2{u_y}{v_y} + \varepsilon v_y^2} \right) - 2fv} \right]d\Omega } } \right],
\end{align}
and thus the variation of $J$ at $u$ in the direction $v$ is calculated by
\begin{align}
\delta J\left[ {u;v} \right]: &= \mathop {\lim }\limits_{\varepsilon  \to 0} \frac{{J\left[ {u + \varepsilon v} \right] - F\left[ u \right]}}{\varepsilon }\\
& = \frac{1}{2}\mathop {\lim }\limits_{\varepsilon  \to 0} \int_\Omega  {\left[ {\left( {1 + {x^2}} \right)\left( {2{u_x}{v_x} + \varepsilon v_x^2} \right) + \left( {1 + {y^2}} \right)\left( {2{u_y}{v_y} + \varepsilon v_y^2} \right) - 2fv} \right]d\Omega } \\
& = \int_\Omega  {\left[ {\left( {1 + {x^2}} \right){u_x}{v_x} + \left( {1 + {y^2}} \right){u_y}{v_y} - fv} \right]d\Omega } .
\end{align}
Integrating by parts, as above, the last integral yields
\begin{align}
\delta J\left[ {u;v} \right] = \left\langle {Lu - f,v} \right\rangle  + \int_0^b {\left. {\left( {1 + {x^2}} \right){u_x}v} \right|_{x = 0}^{x = a}dy}  + \int_0^a {\left. {\left( {1 + {y^2}} \right){u_y}v} \right|_{y = 0}^{y = b}dx} ,
\end{align}
for all $v\in H^1\left(\Omega\right)$. 

Now applying Theorem \ref{theorem12.3} to $J$ and its minimizer $\widetilde u \in H^1$ yields
\begin{align}
\delta J\left[ {\widetilde u;v} \right] = 0, \hspace{0.2cm}\forall v \in {H^1}\left(\Omega\right),
\end{align}
i.e.,
\begin{align}
\label{1.79}
\left\langle {L\widetilde u - f,v} \right\rangle  + \int_0^b {\left. {\left( {1 + {x^2}} \right){{\widetilde u}_x}v} \right|_{x = 0}^{x = a}dy}  + \int_0^a {\left. {\left( {1 + {y^2}} \right){{\widetilde u}_y}v} \right|_{y = 0}^{y = b}dx}=0 , \hspace{0.2cm}\forall v \in {H^1}\left(\Omega\right).
\end{align}
Use the same trick as in the proof of previous statement, plugging successively $v = {{\widetilde u}_x}{\chi _{\left\{ 0 \right\} \times \left[ {0,b} \right]}}$, $v = {{\widetilde u}_x}{\chi _{\left\{ a \right\} \times \left[ {0,b} \right]}}$, $v = {{\widetilde u}_y}{\chi _{\left[ {0,a} \right] \times \left\{ 0 \right\}}}$, and $v = {{\widetilde u}_y}{\chi _{\left[ {0,a} \right] \times \left\{ b \right\}}}$ into \eqref{1.79} gives us ${\left. {{{\widetilde u}_x}} \right|_{\left\{ {0,a} \right\} \times \left[ {0,b} \right]}} = {\left. {{{\widetilde u}_y}} \right|_{\left[ {0,a} \right] \times \left\{ {0,b} \right\}}} = 0$ a.e. in $\bar \Omega$. This is equivalent to the Neumann boundary condition
\begin{align}
\label{1.80}
\frac{{\partial \widetilde u}}{{\partial \overrightarrow {\bf{n}} }} = 0,\mbox{ on }\partial \Omega ,
\end{align}
where $\overrightarrow {\bf{n}}$ denotes the exterior normal to the boundary $\partial \Omega$. Substituting \eqref{1.80} back to \eqref{1.79} yields 
\begin{align}
\left\langle {L\widetilde u - f,v} \right\rangle  = 0,\hspace{0.2cm}\forall v \in {H^1}\left(\Omega\right),
\end{align}
Use the same argument before, we deduce that $\widetilde u$ satisfies the following Neumann boundary value problem
\begin{align}
Lu &= f,\mbox{ in } \Omega ,\\
\frac{{\partial u}}{{\partial \overrightarrow {\bf{n}}}} &= 0,\mbox{ on } \partial \Omega .
\end{align}
This completes our solution. \hfill $\square$
\end{enumerate}
\begin{problem}
Let $L$ be the operator given in Problem \ref{problem1.1}. Consider the following problem
\begin{equation}
\label{1.84}
\left\{ \begin{split}
& {u_t} + Lu = 0,  & \mbox{ in } \Omega  \times \left( {0, + \infty } \right),\\
& u\left( {{\bf{x}},0} \right) = g\left( {\bf{x}} \right), & \mbox{ in } \Omega ,\\
& u \left({\bf{x}},t\right)= 0, & \mbox{ on } \partial \Omega \times \left[0,+\infty\right).
\end{split} \right.
\end{equation}
\begin{enumerate}
\item Determine $D\left( L \right) \subset {L^2}\left(\Omega\right)$.
\item Do not use Poincar\'e inequality, prove that there exists $C>0$ such that
\begin{align}
\label{1.85}
\left\| {{u_x}} \right\|_2^2 + \left\| {{u_y}} \right\|_2^2 \ge C\left\| u \right\|_{{H^1}}^2,\hspace{0.2cm} \forall u \in H_0^1 \left(\Omega\right).
\end{align}
\item Prove that the operator $L$ is maximal monotone.
\item Prove that $L$ is symmetric, then deduce that $L$ is self-adjoint.
\item Use Hille-Yosida theorem, prove that for $g\in L^2\left(\Omega\right)$ the problem \eqref{1.84} has a solution
\begin{align}
u \in C\left( {\left[ {0, + \infty } \right];{L^2}\left(\Omega\right)} \right) \cap C\left( {\left( {0, + \infty } \right);D\left( L \right)} \right) \cap {C^1}\left( {\left( {0, + \infty } \right);{L^2}\left(\Omega\right)} \right).
\end{align}
\end{enumerate}
\end{problem}
\begin{proof}
\begin{enumerate}
\item In order that $Lv$ makes sense for all $v\in D\left(L\right)$, it is required that $D\left( L \right) \subset {H^2}\left( \Omega  \right)$. For $u\in H^2 \left(\Omega\right)$, $v\in {H^2}\left( \Omega  \right)$, the integration by parts formula gives us
\begin{align}
\left\langle {Lu,v} \right\rangle  = & - \int_\Omega  {\left[ {\frac{\partial }{{\partial x}}\left( {\left( {1 + {x^2}} \right){u_x}} \right) + \frac{\partial }{{\partial y}}\left( {\left( {1 + {y^2}} \right){u_y}} \right)} \right]vd\Omega } \\
 =&\ \int_\Omega  {\left[ {\left( {1 + {x^2}} \right){u_x}{v_x} + \left( {1 + {y^2}} \right){u_y}{v_y}} \right]d\Omega } \\
 &- \int_0^b {\left. {\left( {1 + {x^2}} \right){u_x}v} \right|_{x = 0}^{x = a}dy}  - \int_0^a {\left. {\left( {1 + {y^2}} \right){u_y}v} \right|_{y = 0}^{y = b}dx} .
\end{align}
Notice that ${H^2}\left( \Omega  \right) \cap H_0^1\left( \Omega  \right)$ is a linear subspace of $L^2\left(\Omega\right)$, we choose the domain of the operator $L$ as $D\left( L \right): = {H^2}\left( \Omega  \right) \cap H_0^1\left( \Omega  \right)$. Then 
\begin{align}
\left\langle {Lu,v} \right\rangle  = \int_\Omega  {\left[ {\left( {1 + {x^2}} \right){u_x}{v_x} + \left( {1 + {y^2}} \right){u_y}{v_y}} \right]d\Omega } ,\hspace{0.2cm} \forall u,v \in D\left( L \right).
\end{align}
\item Similar to the proof of the second statement of Problem \ref{problem1.1}, let $u\in H_0^1\left(\Omega\right)$ be given. Since $u \left(x,y\right) =0$ for $\left(x,y\right) \in \partial \Omega$ we can modify the argument presented above to prove \eqref{1.85}.
\item The given unbounded linear operator $L: D\left(L\right) \subset L^2\left(\Omega\right) \to L^2\left(\Omega\right)$ is monotone since
\begin{align}
\left\langle {Lv,v} \right\rangle  = \int_\Omega  {\left[ {\left( {1 + {x^2}} \right)v_x^2 + \left( {1 + {y^2}} \right)v_y^2} \right]d\Omega }  \ge 0,\hspace{0.2cm}\forall v \in D\left( L \right) .
\end{align}
To prove that $L$ is maximal monotone, it suffices to verify that $R\left(I+L\right)= L^2\left(\Omega\right)$, i.e., 
\begin{align}
\label{1.92}
\forall f \in {L^2}\left( \Omega  \right),\hspace{0.2cm}\exists u \in D\left( L \right)\mbox{ such that } u + Lu = f.
\end{align}
Given $f\in L^2\left(\Omega\right)$, the main idea is to modify the arguments given in the proof of Problem \ref{problem1.1} for the operator $S:=I+L$, instead of $L$, as follows.

\textsc{Step 1.} \textit{Find the weak formulation of the problem}
\begin{equation}
\label{1.93}
\left\{ \begin{split}
& Su = f, & \mbox{ in } \Omega ,\\
& u\left( {x,y} \right) = 0, & \mbox{ on } \partial \Omega ,
\end{split} \right.
\end{equation}
\textit{on the solution space $H_0^1\left(\Omega\right)$}: For any $u\in H_0^1\left(\Omega\right)$ and $v\in H_0^1\left(\Omega\right)$, 
\begin{align}
\left\langle {Su,v} \right\rangle  &= \left\langle {u,v} \right\rangle  + \left\langle {Lu,v} \right\rangle \\
 &= \int_\Omega  {\left[ {u - \frac{\partial }{{\partial x}}\left( {\left( {1 + {x^2}} \right){u_x}} \right) - \frac{\partial }{{\partial y}}\left( {\left( {1 + {y^2}} \right){u_y}} \right)} \right]vd\Omega } \\
& = \int_\Omega  {\left[ {uv + \left( {1 + {x^2}} \right){u_x}{v_x} + \left( {1 + {y^2}} \right){u_y}{v_y}} \right]d\Omega } .
\end{align}
The weak formulation of \eqref{1.93} is then given by
\begin{align}
R\left[ {u,v} \right] = F\left[ v \right], \hspace{0.2cm}\forall v \in H_0^1\left( \Omega  \right),
\end{align}
where, for $u\in H_0^1\left(\Omega\right)$ and $v\in H_0^1\left(\Omega\right)$,
\begin{align}
R\left[ {u,v} \right]: = \int_\Omega  {\left[ {uv + \left( {1 + {x^2}} \right){u_x}{v_x} + \left( {1 + {y^2}} \right){u_y}{v_y}} \right]d\Omega } .
\end{align}
\textsc{Step 2.} \textit{Prove that \eqref{1.93} has a weak solution in $H_0^1\left(\Omega\right)$ by using Lax-Milgram theorem}: Consider the defined bilinear form $R\left[u,v\right]$, it is continuous since
\begin{align}
R\left[ {u,v} \right] &= \int_\Omega  {\left[ {uv + \left( {1 + {x^2}} \right){u_x}{v_x} + \left( {1 + {y^2}} \right){u_y}{v_y}} \right]d\Omega } \\
& \le \max \left\{ {1 + {a^2},1 + {b^2}} \right\}\int_\Omega  {\left| {uv + {u_x}{v_x} + {u_y}{v_y}} \right|d\Omega } \\
& \le \left( {1 + \max {{\left\{ {a,b} \right\}}^2}} \right)\int_\Omega  {{{\left( {{u^2} + u_x^2 + u_y^2} \right)}^{\frac{1}{2}}}{{\left( {{v^2} + v_x^2 + v_y^2} \right)}^{\frac{1}{2}}}d\Omega } \\
& \le \left( {1 + \max {{\left\{ {a,b} \right\}}^2}} \right){\left( {\int_\Omega  {\left( {{u^2} + u_x^2 + u_y^2} \right)d\Omega } } \right)^{\frac{1}{2}}}{\left( {\int_\Omega  {\left( {{v^2} + v_x^2 + v_y^2} \right)d\Omega } } \right)^{\frac{1}{2}}}\\
& = \left( {1 + \max {{\left\{ {a,b} \right\}}^2}} \right){\left\| u \right\|_{{H^1}\left( \Omega  \right)}}{\left\| v \right\|_{{H^1}\left( \Omega  \right)}},\hspace{0.2cm} \forall u,v \in H_0^1\left( \Omega  \right) ,
\end{align}
and it is coercive since
\begin{align}
R\left[ {u,u} \right] &= \int_\Omega  {\left[ {{u^2} + \left( {1 + {x^2}} \right)u_x^2 + \left( {1 + {y^2}} \right)u_y^2} \right]d\Omega } \\
 &\ge \int_\Omega  {\left( {{u^2} + u_x^2 + u_y^2} \right)d\Omega } \\
 &= \left\| u \right\|_{{H^1}\left( \Omega  \right)}^2,\hspace{0.2cm}\forall u \in H_0^1\left( \Omega  \right).
\end{align}
It is easy to prove $F \in {H^{ - 1}}\left( \Omega  \right)$\footnote{The dual space of $H_0^1\left(\Omega\right)$ is denoted by $H^{-1}\left(\Omega\right)$, see \cite[p. 291]{Haim}.}. Now applying Lax-Milgram theorem yields that there exists a unique element $\hat u \in H_0^1\left( \Omega  \right)$ such that
\begin{align}
\label{1.107}
R\left[ {\hat u,v} \right] = F\left[ v \right],\hspace{0.2cm}\forall v \in H_0^1\left( \Omega  \right).
\end{align}
\textsc{Step 3.} \textit{Suppose that this weak solution satisfies $\hat u \in {H^2}\left( \Omega  \right) \cap H_0^1\left( \Omega  \right)$, i.e., $\hat u \in D\left( L \right)$, prove that this weak solution $\hat u$ satisfies \eqref{1.93}}: Since $\hat u \in H_0^1\left(\Omega\right)$, $\hat u$ satisfies the homogeneous Dirichlet boundary conditions. It then suffices to prove that $\hat u$ satisfies the PDE in \eqref{1.93}. To this end, integrating by parts the left-hand side of \eqref{1.107} gives us 
\begin{align}
\left\langle {S\hat u,v} \right\rangle  = F\left[ v \right],\hspace{0.2cm}\forall v \in H_0^1\left( \Omega  \right),
\end{align}
In particular, this implies
\begin{align}
\left\langle {\left( {I + L} \right)\hat u - f,v} \right\rangle  = 0,\hspace{0.2cm}\forall v \in C_c^\infty \left( \Omega  \right).
\end{align}
It follows (see \cite[Corollary 4.24, p. 110]{Haim}) that $\left( {I + L} \right)\hat u = f$ a.e. on $\Omega$, i.e., \eqref{1.92} holds. 

Therefore, $L$ is a maximal monotone operator.
\item Note that $\overline {D\left( L \right)}  = \overline {{H^2}\left( \Omega  \right) \cap H_0^1\left( \Omega  \right)}  = {L^2}\left( \Omega  \right)$\footnote{See \cite[Corollary 4.23, p.109]{Haim}.}. The operator $L$ is symmetric since 
\begin{align}
\left\langle {u,Lv} \right\rangle  &=  - \int_\Omega  {u\left[ {\frac{\partial }{{\partial x}}\left( {\left( {1 + {x^2}} \right){v_x}} \right) + \frac{\partial }{{\partial y}}\left( {\left( {1 + {y^2}} \right){v_y}} \right)} \right]d\Omega } \\
& = \int_\Omega  {\left[ {\left( {1 + {x^2}} \right){u_x}{v_x} + \left( {1 + {y^2}} \right){u_y}{v_y}} \right]d\Omega } \\
& = \left\langle {Lu,v} \right\rangle ,\hspace{0.2cm}\forall u,v \in D\left( L \right).
\end{align}
Thus, $L$ is a maximal monotone symmetric operator. Applying \cite[Proposition 7.6, p. 193]{Haim} to $L$ yields that $L$ is self-adjoint.
\item Since $L$ is a self-adjoint maximal monotone operator, applying Hille-Yosida theorem \cite[Theorem 7.7, p. 194]{Haim} yields that for every $g\in L^2\left(\Omega\right)$ there exists a unique function
\begin{align}
u \in C\left( {\left[ {0, + \infty } \right];{L^2}\left( \Omega  \right)} \right) \cap {C^1}\left( {\left( {0, + \infty } \right);{L^2}\left( \Omega  \right)} \right) \cap C\left( {\left( {0, + \infty } \right);D\left( L \right)} \right)
\end{align}
such that
\begin{equation}
\left\{ \begin{split}
& u_t + Lu = 0, & \mbox{ in } \Omega \times \left( {0, + \infty } \right),\\
& u\left( x,y,0 \right) = g\left(x,y\right), & \mbox{ in } \Omega .
\end{split} \right.
\end{equation}
Moreover (bonus), we have
\begin{align}
{\left\| {u\left( { \cdot , \cdot ,t} \right)} \right\|_2} &\le {\left\| g \right\|_2},\\
{\left\| {{u_t}\left( { \cdot , \cdot ,t} \right)} \right\|_2} &= {\left\| {Au\left( { \cdot , \cdot ,t} \right)} \right\|_2} \le \frac{1}{t}{\left\| g \right\|_2},\hspace{0.2cm}\forall t > 0,\\
u &\in {C^k}\left( {\left( {0, + \infty } \right);D\left( {{L^l}} \right)} \right),\hspace{0.2cm}\forall k,l \mbox{ integers}.
\end{align}
\end{enumerate}
This completes our proof.
\end{proof}
\section{Appendices}
\begin{theorem}[Lax-Milgram]\label{theorem1.1}
Assume that $a\left(u,v\right)$ is a continuous coercive bilinear from on $H$. Then, given any $\varphi \in H^\star$, there exists a unique element $u\in H$ such that
\begin{align}
a\left( {u,v} \right) = \left\langle {\varphi ,v} \right\rangle ,\hspace{0.2cm} \forall v \in H.
\end{align}
Moreover, if $a$ is symmetric, then $u$ is characterized by the property
\begin{align}
u \in H\mbox{ and } \frac{1}{2}a\left( {u,u} \right) - \left\langle {\varphi ,u} \right\rangle  = \mathop {\min }\limits_{v \in H} \left\{ {\frac{1}{2}a\left( {v,v} \right) - \left\langle {\varphi ,v} \right\rangle } \right\}.
\end{align}
\end{theorem}

\begin{theorem}[\cite{Schaum}, Theorem 12.3, p. 189]\label{theorem12.3}
Let $J$ denote a functional on domain $\mathcal{A}$ with associated set of comparison functions $\mathcal{M}$, and suppose that $u_0\in \mathcal{A}$ is a local extreme point for $J$. If $J$ has a variation at $u_0$, it must vanish; i.e.,
\begin{align}
\delta J\left[ {{u_0};v} \right] = 0 \mbox{ for all } v \in M.
\end{align}
\end{theorem}

\begin{theorem}[\cite{Schaum}, Theorem 12.4, p. 190]
If the subspace $\mathcal{M}$ of comparison functions is dense in $L^2\left(\Omega\right)$ and if $u_0\in \mathcal{A}$ is a local extreme point for $J$, then $u_0$ necessarily belongs to $\mathcal{D}$ and 
\begin{align}
\nabla J\left[ {{u_0}} \right] = 0,
\end{align}
(the Euler equation for $J$).
\end{theorem}

\begin{theorem}[\cite{Haim}, Corollary 4.23, p. 109]
Let $\Omega \subset \mathbb{R}^N$ be an open set. Then $C_c^\infty \left(\Omega\right)$ is dense in $L^p\left(\Omega\right)$ for any $1\le p<\infty$.
\end{theorem}

\begin{theorem}[\cite{Haim}, Corollary 4.24, p. 110]
Let $\Omega\in \mathbb{R}^N$ be an open set and let $u\in L_{\rm{loc}}^1\left(\Omega\right)$ be such that
\begin{align}
\int {uf}  = 0, \hspace{0.2cm} \forall f \in C_c^\infty \left( \Omega  \right).
\end{align}
Then $u=0$ a.e. on $\Omega$.
\end{theorem}

\begin{theorem}[\cite{Haim}, Proposition 7.6, p. 193]
Let $A$ be a maximal monotone symmetric operator. Then $A$ is self-adjoint.
\end{theorem}

\begin{theorem}[Hill-Yosida, \cite{Haim}, Theorem 7.4, p. 185] 
Let $A$ be a maximal monotone operator. Then, given any $u_0\in D\left(A\right)$ there exists a unique function
\begin{align}
u \in {C^1}\left( {\left[ {0, + \infty } \right);H} \right) \cap C\left( {\left[ {0, + \infty } \right);D\left( A \right)} \right)
\end{align}
satisfying
\begin{equation}
\left\{ \begin{split}
& \frac{{du}}{{dt}} + Au = 0 \mbox{ on } \left[ {0, + \infty } \right),\\
& u\left( 0 \right) = {u_0}.
\end{split} \right.
\end{equation}
Moreover, 
\begin{align}
\left| {u\left( t \right)} \right| \le \left| {{u_0}} \right| \mbox{ and } \left| {\frac{{du}}{{dt}}\left( t \right)} \right| = \left| {Au\left( t \right)} \right| \le \left| {A{u_0}} \right|,\forall t \ge 0.
\end{align}
\end{theorem}

\begin{theorem}[\cite{Haim}, Theorem 7.7, p. 194]
Let $A$ be a self-adjoint maximal monotone operator. Then for every $u_0\in H$ there exists a unique function
\begin{align}
u \in C\left( {\left[ {0, + \infty } \right];H} \right) \cap {C^1}\left( {\left( {0, + \infty } \right);H} \right) \cap C\left( {\left( {0, + \infty } \right);D\left( A \right)} \right)
\end{align}
such that
\begin{equation}
\left\{ \begin{split}
& \frac{{du}}{{dt}} + Au = 0 \mbox{ on } \left[ {0, + \infty } \right),\\
& u\left( 0 \right) = {u_0}.
\end{split} \right.
\end{equation}
Moreover, we have
\begin{align}
\left| {u\left( t \right)} \right| &\le \left| {{u_0}} \right| \mbox{ and } \left| {\frac{{du}}{{dt}}\left( t \right)} \right| = \left| {Au\left( t \right)} \right| \le \frac{1}{t}\left| {{u_0}} \right|,\forall t \ge 0,\\
& u \in {C^k}\left( {\left( {0, + \infty } \right);D\left( {{A^l}} \right)} \right),\hspace{0.2cm}\forall k,l \mbox{ integers}.
\end{align}
\end{theorem}
\vspace{1cm}
\begin{center}
\textsc{The End}
\end{center}
\newpage
\begin{thebibliography}{99}
\bibitem {Haim} Haim Brezis, \textit{Funcitonal Analysis, Sobolev Spaces and Partial Differential Equations}, Springer.
\bibitem {Schaum} Paul DuChateau, David W. Zachmann, \textit{Theory and Problems of Partial Differential Equations}, Schaum's outline series, McGraw-Hill.
\end{thebibliography}

%%%%%%%%%%%%%%%%%%%%%%%%%%%%%%%%%%%%%%%%%%%%%%%%%%%%%%%%%%%%%%%%
%%%%%%%%%%%%%%%%%%%%%%%%%%%%%%%%%%%%%%%%%%%%%%%%%%%%%%%%%%%%%%%%%%%%%%%%%%
%\Addresses
\end{document} 
