\documentclass{article}
\usepackage[backend=biber,natbib=true,style=alphabetic,maxbibnames=50]{biblatex}
\addbibresource{/home/nqbh/reference/bib.bib}
\usepackage[utf8]{vietnam}
\usepackage{tocloft}
\renewcommand{\cftsecleader}{\cftdotfill{\cftdotsep}}
\usepackage[colorlinks=true,linkcolor=blue,urlcolor=red,citecolor=magenta]{hyperref}
\usepackage{amsmath,amssymb,amsthm,enumitem,float,graphicx,mathtools,tikz}
\usetikzlibrary{angles,calc,intersections,matrix,patterns,quotes,shadings}
\allowdisplaybreaks
\newtheorem{assumption}{Assumption}
\newtheorem{baitoan}{Bài toán}
\newtheorem{cauhoi}{Câu hỏi}
\newtheorem{conjecture}{Conjecture}
\newtheorem{corollary}{Corollary}
\newtheorem{dangtoan}{Dạng toán}
\newtheorem{definition}{Definition}
\newtheorem{dinhly}{Định lý}
\newtheorem{dinhnghia}{Định nghĩa}
\newtheorem{example}{Example}
\newtheorem{ghichu}{Ghi chú}
\newtheorem{hequa}{Hệ quả}
\newtheorem{hypothesis}{Hypothesis}
\newtheorem{lemma}{Lemma}
\newtheorem{luuy}{Lưu ý}
\newtheorem{nhanxet}{Nhận xét}
\newtheorem{notation}{Notation}
\newtheorem{note}{Note}
\newtheorem{principle}{Principle}
\newtheorem{problem}{Problem}
\newtheorem{proposition}{Proposition}
\newtheorem{question}{Question}
\newtheorem{remark}{Remark}
\newtheorem{theorem}{Theorem}
\newtheorem{vidu}{Ví dụ}
\usepackage[left=1cm,right=1cm,top=5mm,bottom=5mm,footskip=4mm]{geometry}
\def\labelitemii{$\circ$}
\DeclareRobustCommand{\divby}{%
	\mathrel{\vbox{\baselineskip.65ex\lineskiplimit0pt\hbox{.}\hbox{.}\hbox{.}}}%
}
\setlist[itemize]{leftmargin=*}
\setlist[enumerate]{leftmargin=*}

\title{Lecture Note: Mathematical Analysis -- Bài Giảng: Giải Tích Toán Học}
\author{Nguyễn Quản Bá Hồng\footnote{A Scientist {\it\&} Creative Artist Wannabe. E-mail: {\tt nguyenquanbahong@gmail.com, hong.nguyenquanba@umt.edu.vn}. Bến Tre City, Việt Nam.}}
\date{\today}

\begin{document}
\maketitle
\begin{abstract}
	This text is a part of the series {\it Some Topics in Advanced STEM \& Beyond}:
	
	{\sc url}: \url{https://nqbh.github.io/advanced_STEM/}.
	
	Latest version:
	\begin{itemize}
		\item {\it Lecture Note: Mathematical Analysis -- Bài Giảng: Giải Tích Toán Học}.
		
		PDF: {\sc url}: \url{https://github.com/NQBH/advanced_STEM_beyond/blob/main/analysis/lecture/NQBH_mathematical_analysis_lecture.pdf}.
		
		\TeX: {\sc url}: \url{https://github.com/NQBH/advanced_STEM_beyond/blob/main/analysis/lecture/NQBH_mathematical_analysis_lecture.tex}.
		\item {\it Slide: Mathematical Analysis -- Slide: Giải Tích Toán Học}.
		
		PDF: {\sc url}: \url{https://github.com/NQBH/advanced_STEM_beyond/blob/main/analysis/slide/NQBH_mathematical_analysis_slide.pdf}.
		
		\TeX: {\sc url}: \url{https://github.com/NQBH/advanced_STEM_beyond/blob/main/analysis/slide/NQBH_mathematical_analysis_slide.tex}.
		\item Codes:
		\begin{itemize}
			\item C++: \url{https://github.com/NQBH/advanced_STEM_beyond/tree/main/analysis/C++}.
			\item Python: \url{https://github.com/NQBH/advanced_STEM_beyond/tree/main/analysis/Python}.
		\end{itemize}
	\end{itemize}
\end{abstract}
\tableofcontents

%------------------------------------------------------------------------------%

\section{Basic Mathematical Analysis -- Giải Tích Toán Học Cơ Bản}
\textbf{\textsf{Resources -- Tài nguyên.}}
\begin{enumerate}
	\item {\sc Đặng Đình Áng}. {\it Nhập Môn Giải Tích}.
	\item \cite{Rudin1976}. {\sc Walter Rudin}. {\it Principles of Mathematical Analysis}.
	
	\item \cite{Tao_analysis_1}. {\sc Terence Tao}. {\it Analysis I}.
	
	\item \cite{Tao_analysis_2}. {\sc Terence Tao}. {\it Analysis II}.
\end{enumerate}

\begin{question}[Definition of mathematical analysis]
	What is mathematical analysis? Cf. mathematical analysis with other types of analysis.
\end{question}
For answers, see, e.g., \cite[Chap. 1, Sect. 1.1: {\it What Is Analysis?}, pp. 1--2]{Tao_analysis_1}, \href{https://en.wikipedia.org/wiki/Mathematical_analysis}{Wikipedia{\tt/}mathematical analysis}. For other types of analysis, see, e.g., \href{https://en.wikipedia.org/wiki/Analysis}{Wikipedia{\tt/}analysis}.

\begin{question}[Motivation of mathematical analysis]
	Why do mathematical analysis?
\end{question}
For answers, see, e.g., \cite[Chap. 1, Sect. 1.2: {\it Why Do Analysis?}, pp. 2--10]{Tao_analysis_1}

\begin{example}[Division by zero \& infinity]
	The cancellation law for multiplication $ac = bc\Rightarrow a = b$ does not work when $c = 0$ \& $c = \pm\infty$. The cancellation law for addition $a + c = b + c\Rightarrow a = b$.
\end{example}

\begin{example}[Cancellation properties]
	
\end{example}
See, e.g., \href{https://en.wikipedia.org/wiki/Cancellation_property}{Wikipedia{\tt/}cancellation property}.

\begin{example}[Geometric series -- Chuỗi hình học]
	When does the geometric series $G(a)\coloneqq\sum_{i=0}^\infty \frac{1}{a^i}$ converge? When does $G(a)$ diverge? 
\end{example}

%------------------------------------------------------------------------------%

\subsection{Notations \& conventions -- Ký hiệu \& quy ước}
Đặt tập hợp các đa thức (polynomial) 1 biến với hệ số nguyên, hệ số hữu tỷ, hệ số thực, hệ số phức lần lượt cho bởi:
\begin{align*}
	\mathbb{Z}[x]&\coloneqq\left\{\sum_{i=0}^n a_ix^i;n\in\mathbb{N},\ a_i\in\mathbb{Z},\ \forall i = 0,\ldots,n,\ a_n\ne0\right\},\\
	\mathbb{Q}[x]&\coloneqq\left\{\sum_{i=0}^n a_ix^i;n\in\mathbb{N},\ a_i\in\mathbb{Q},\ \forall i = 0,\ldots,n,\ a_n\ne0\right\},\\
	\mathbb{R}[x]&\coloneqq\left\{\sum_{i=0}^n a_ix^i;n\in\mathbb{N},\ a_i\in\mathbb{R},\ \forall i = 0,\ldots,n,\ a_n\ne0\right\},\\
	\mathbb{C}[x]&\coloneqq\left\{\sum_{i=0}^n a_ix^i;n\in\mathbb{N},\ a_i\in\mathbb{C},\ \forall i = 0,\ldots,n,\ a_n\ne0\right\}.
\end{align*}
Ta có quan hệ hiển nhiên $\mathbb{N}[x]\subset\mathbb{Z}[x]\subset\mathbb{Q}[x]\subset\mathbb{R}[x]\subset\mathbb{C}[x]$. Tổng quát, với $\mathbb{F}$ là 1 trường bất kỳ, tập hợp các đa thức 1 biến với hệ số thuộc trường $\mathbb{F}$ (e.g., $\mathbb{Z},\mathbb{Z}_p,\mathbb{Q},\mathbb{R},\mathbb{C}$) cho bởi:
\begin{equation*}
	\mathbb{F}[x]\coloneqq\left\{\sum_{i=0}^n a_ix^i;n\in\mathbb{N},\ a_i\in\mathbb{F},\ \forall i = 0,\ldots,n,\ a_n\ne0\right\}.
\end{equation*}
Tập xác định của đa thức có thể là toàn bộ trường số thực $\mathbb{R}$ hoặc trường số phức $\mathbb{C}$, i.e., $D_P = {\rm dom}(P) = \mathbb{R}$ or $D_P = {\rm dom}(P) = \mathbb{C}$, tùy vào trường $\mathbb{F}$ của các hệ số \& mục đích sử dụng đa thức.

\begin{problem}[Cf: Calculus vs. Mathematical Analysis]
	Distinguish \& compare Calculus vs. Mathematical Analysis.
\end{problem}
Analysis is more pure mathematics. Calculus is more applied mathematics.

\begin{problem}[Examples \& counterexamples in mathematical analysis -- Ví dụ \& phản ví dụ trong phân tích toán học]
	Find, from simple to advanced, examples \& counterexamples to each mathematical concepts \& mathematical results, including lemmas, propositions, theorems, \& consequences.
	
	-- Tìm các ví dụ \& phản ví dụ từ đơn giản đến nâng cao cho mỗi khái niệm toán học \& kết quả toán học, bao gồm các bổ đề, mệnh đề, định lý, \& hệ quả.
\end{problem}

\begin{problem}[Python {\tt SymPy}]
	Study {\tt SymPy} to support calculus \& mathematical analysis.
\end{problem}

%------------------------------------------------------------------------------%

\section{Continuity -- Sự Liên Tục}

\begin{definition}[\cite{Tao_analysis_1}, Def. 6.1.1, p. 109: distance between 2 reals]
	Given $x,y\in\mathbb{R}$, their distance $d(x,y)$ is defined to be $d(x,y)\coloneqq|x - y|\in[0,\infty)$.
\end{definition}

\begin{definition}[\cite{Tao_analysis_1}, Def. 6.1.2, p. 109: $\varepsilon$-close reals]
	Let $\varepsilon > 0$ be a real number. $x,y\in\mathbb{R}$ is said to be {\rm$\varepsilon$-close} iff $d(x,y)\le\varepsilon$.
\end{definition}

%------------------------------------------------------------------------------%

\section{Derivative -- Đạo Hàm}

%------------------------------------------------------------------------------%

\section{Integral -- Tích Phân}

\subsection{SymPy{\tt/integrals} module}
See \url{https://docs.sympy.org/latest/modules/integrals/integrals.html}. The {\tt integrals} module in {\tt SymPy} implements methods to calculate definite \& indefinite integrals of expressions. Principal method in this module is {\tt integrate()}:
\begin{itemize}
	\item {\tt integrate(f, x)} returns the indefinite integral $\int f\,{\rm d}x$
	\item {\tt integrate(f, (x, a, v))} returns the definite integral $\int_a^b f\,{\rm d}x$.
\end{itemize}

\begin{problem}[Integration of elementary functions]
	Use {\tt SymPy} to compute definite- \& indefinite integrals of elementary functions as many as possible.
\end{problem}

\begin{problem}[Integration of nonelementary functions]
	Use {\tt SymPy} to compute definite- \& indefinite integrals of nonelementary functions as many as possible.
\end{problem}

\begin{example}[Integral of error function]
	The indefinite integral of the nonelementary function $e^{-x^2}{\rm erf}(x)$, where ${\rm erf}(x)$ is the {\rm error function}, is given by
	\begin{equation*}
		\int e^{-x^2}{\rm erf}(x)\,{\rm d}x = \frac{\sqrt{\pi}}{4}{\rm erf}(x).
	\end{equation*}
\end{example}
Run the following Python code:
\begin{verbatim}
	from sympy import *
	x = Symbol('x')
	print(integrate(exp(-x**2)*erf(x), x))
\end{verbatim}
to obtain the following output:
\begin{verbatim}
	sqrt(pi)*erf(x)**2/4
\end{verbatim}
For more information about the error function, see, e.g., \href{https://en.wikipedia.org/wiki/Error_function}{Wikipedia{\tt/}error function}.

%------------------------------------------------------------------------------%

\section{Miscellaneous}

\subsection{See also}

\begin{enumerate}
	\item \cite{Strogatz_infinite_power}. {\sc Steven Strogatz}. {\it Infinite Powers: How Calculus Reveals the Secrets of the Universe}.
	
	\item \cite{Strogatz_infinite_power_VN}. {\sc Steven Strogatz}. {\it Infinite Powers: How Calculus Reveals the Secrets of the Universe -- Sức Mạnh Vô Hạn: Giải Tích Toán Khám Phá Bí Mật Của Vũ Trụ Như Thế Nào?}.
	
	{\sf Nhận xét.} 1 quyển sách hay về thường thức về lịch sử phát triển của Giải tích Toán học \& các ý tưởng cơ bản nhất của Giải tích. Khuyến khích đọc thử, cũng như các tác phẩm thường thức Khoa học Tự nhiên nói chung \& Toán học nói riêng khác của tác giả {\sc Steven Strogatz}.
	\item {\it Vietnamese Mathematical Olympiad for High School- \& College Students (VMC) -- Olympic Toán Học Học Sinh \& Sinh Viên Toàn Quốc}.
	
	PDF: {\sc url}: \url{https://github.com/NQBH/advanced_STEM_beyond/blob/main/VMC/NQBH_VMC.pdf}.
	
	\TeX: {\sc url}: \url{https://github.com/NQBH/advanced_STEM_beyond/blob/main/VMC/NQBH_VMC.tex}.
	\begin{itemize}
		\item Codes:
		\begin{itemize}
			\item C++ code: \url{https://github.com/NQBH/advanced_STEM_beyond/tree/main/VMC/C++}.
			\item Python code: \url{https://github.com/NQBH/advanced_STEM_beyond/tree/main/VMC/Python}.
		\end{itemize}
		\item Resource: \url{https://github.com/NQBH/advanced_STEM_beyond/tree/main/VMC/resource}.
		\item Figures: \url{https://github.com/NQBH/advanced_STEM_beyond/tree/main/VMC/figure}.
	\end{itemize}
\end{enumerate}

%------------------------------------------------------------------------------%

\printbibliography[heading=bibintoc]
	
\end{document}