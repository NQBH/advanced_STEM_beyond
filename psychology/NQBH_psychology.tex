\documentclass{article}
\usepackage[backend=biber,natbib=true,style=alphabetic,maxbibnames=50]{biblatex}
\addbibresource{/home/nqbh/reference/bib.bib}
\usepackage[utf8]{vietnam}
\usepackage{tocloft}
\renewcommand{\cftsecleader}{\cftdotfill{\cftdotsep}}
\usepackage[colorlinks=true,linkcolor=blue,urlcolor=red,citecolor=magenta]{hyperref}
\usepackage{amsmath,amssymb,amsthm,enumitem,float,graphicx,mathtools,tikz}
\usetikzlibrary{angles,calc,intersections,matrix,patterns,quotes,shadings}
\allowdisplaybreaks
\newtheorem{assumption}{Assumption}
\newtheorem{baitoan}{}
\newtheorem{cauhoi}{Câu hỏi}
\newtheorem{conjecture}{Conjecture}
\newtheorem{corollary}{Corollary}
\newtheorem{dangtoan}{Dạng toán}
\newtheorem{definition}{Definition}
\newtheorem{dinhly}{Định lý}
\newtheorem{dinhnghia}{Định nghĩa}
\newtheorem{example}{Example}
\newtheorem{ghichu}{Ghi chú}
\newtheorem{hequa}{Hệ quả}
\newtheorem{hypothesis}{Hypothesis}
\newtheorem{lemma}{Lemma}
\newtheorem{luuy}{Lưu ý}
\newtheorem{nhanxet}{Nhận xét}
\newtheorem{notation}{Notation}
\newtheorem{note}{Note}
\newtheorem{principle}{Principle}
\newtheorem{problem}{Problem}
\newtheorem{proposition}{Proposition}
\newtheorem{question}{Question}
\newtheorem{remark}{Remark}
\newtheorem{theorem}{Theorem}
\newtheorem{vidu}{Ví dụ}
\usepackage[left=1cm,right=1cm,top=5mm,bottom=5mm,footskip=4mm]{geometry}
\def\labelitemii{$\circ$}
\DeclareRobustCommand{\divby}{%
	\mathrel{\vbox{\baselineskip.65ex\lineskiplimit0pt\hbox{.}\hbox{.}\hbox{.}}}%
}
\setlist[itemize]{leftmargin=*}
\setlist[enumerate]{leftmargin=*}

\title{Psychology $\Psi$ -- Tâm Lý Học $\Psi$}
\author{Nguyễn Quản Bá Hồng\footnote{A Scientist {\it\&} Creative Artist Wannabe. E-mail: {\tt nguyenquanbahong@gmail.com}. Bến Tre City, Việt Nam.}}
\date{\today}

\begin{document}
\maketitle
\begin{abstract}
	This text is a part of the series {\it Some Topics in Advanced STEM \& Beyond}:
	
	{\sc url}: \url{https://nqbh.github.io/advanced_STEM/}.
	
	Latest version:
	\begin{itemize}
		\item {\it Psychology $\Psi$ -- Tâm Lý Học $\Psi$}.
		
		PDF: {\sc url}: \url{https://github.com/NQBH/advanced_STEM_beyond/blob/main/psychology/NQBH_psychology.pdf}.
		
		\TeX: {\sc url}: \url{https://github.com/NQBH/advanced_STEM_beyond/blob/main/psychology/NQBH_psychology.tex}.
	\end{itemize}
\end{abstract}
\tableofcontents

%------------------------------------------------------------------------------%

\section{Afred Adler. The Science of Living}
\textbf{\textsf{Resources -- Tài nguyên.}}
\begin{enumerate}
	\item \cite{Adler_science_living}. {\sc Alfred Adler}. {\it The Science of Living}.
\end{enumerate}
\textbf{\textsf{Amazon description.}} ``2011 Reprint of 1930. Full facsimile of the original edition, not reproduced with Optical Recognition Software. Adler left behind many theories \& practices that very much influenced the world of psychiatry. Today these concepts are known as \textit{Adlerian psychology}. His theories focused on the feelings of inferiority, \& how each person tries to overcome such feelings by overcompensating (trying too hard to make up for what is lacking). Adler claimed that an individual's lifestyle becomes established by the age of 4 or 5, \& he stressed the importance of social forces, or the child's environment, on the development of behavior. He believed that each person is born with the ability to relate to other people \& realize the importance of society as a whole. As a therapist, Adler was a teacher who focused on a patient's mental health, not sickness. Adler encouraged self-improvement by pinpointing the error in patients' lives \& correcting it. He thought of himself as an enabler, one who guides the patient through ``self-determination,'' so that the patients themselves can make changes \& improve their state. Adler was a pioneer in that he was 1 of the 1st psychiatrists to use therapy in social work, the education of children, \& in the treatment of criminals. \textit{The Science of Living} is an intended to help the reader realize his potential.''

\textbf{\textsf{The Science of Living.}} ``Originally published in 1930 \textit{The Science of Living} looks at Individual Psychology as a science. Adler discusses the various elements of Individual Psychology \& its application to everyday life: including the inferiority complex, the superiority complex \& other social aspects, such as, love \& marriage, sex \& sexuality, children \& their education. This is an important book in the history of psychoanalysis \& Alderian therapy.''

\section*{A Note on the Author \& His Work}
``\textsc{Dr. Alfred Adler}'s work in psychology, while it is scientific \& general in method, is essentially the study of the separate personalities we are, \& is therefore called Individual Psychology. Concrete, particular, unique human beings are the subjects of this psychology, \& it can only be truly learned from the men, women \& children we meet.

The supreme importance of this contribution to modern psychology is due to the manner in which it reveals how all the activities of the soul are drawn together into the service of the individual, how all his faculties \& strivings are related to 1 end. We are enabled by this to enter into the ideals, the difficulties, the efforts \& discouragements of our fellow-men, in such a way that we may obtain a whole \& living picture of each as a personality. In this coordinating idea, something like finality is achieved, though we must understand it as finality of foundation. There has never before been a method so rigorous \& yet adaptable for following the fluctuations of that most fluid, variable \& elusive of all realities, the individual human soul.

Since Adler regards not only science but even intelligence itself as the result of the communal efforts of humanity, we shall find his consciousness of his own unique contribution more than usually tempered by recognition of his collaborators, both past \& contemporary. It will therefore be useful to consider Adler's relation to the movement called Psycho-analysis, \& 1st of all to recall, however briefly, the philosophic impulses which inspired the psycho-analytic movement as a whole.

The conception of the Unconscious as vital memory -- biological memory -- is a common to modern psychology as a whole. But Freud, from the 1st a specialist in hysteria, took the memories of success or failure in the sexual life, as of the 1st -- \& almost the only -- importance. Jung, a psychiatrist of genius, has tried to widen this distressingly narrow view, by seeking to reveal the super-individual or racial memories which, he believes, have as much power as the sexual \& a higher kind of value for life.

It was left to Alfred Adler, a physician of wide \& general experience, to unite the conception of the Unconscious more firmly with biological reality. A man of the original school of psycho-analysts, he had done much work by that method of analyzing memories out of their coagulated emotional state into clearness \& objectivity. But he showed that the whole scheme of memory is different in every individual. Individuals do not form their unconscious memories all around the same central motive -- not all around sexuality, e.g., . In every individual we find an individual way of selecting its experiences from all possible experience. What is the principle of that selectivity? Adler has answered that it is, fundamentally, the organic consciousness of a \textit{need}, of some specific inferiority which has to be compensated. It is as though every soul had consciousness of its whole physical reality, \& were concentrated, with sleepless insistence, upon achieving compensation for the defects in it.

Thus the whole life of the small man, e.g., , would be interpretable as a struggle to achieve immediate greatness in some way, \& that of a deaf man to obtain a compensation for not hearing. It is not so simple as that, of course, for a system of defects may give rise to a constellation of guiding ideas, \& also in human life we have to deal with imaginary inferiorities \& fantastic strivings, but even here the principle is the same.

The sexual life, far from controlling all activities, fits perfectly into the frame of those more important strivings, for it is pre-eminently under the control of emotion, \& emotion is moulded by the entire vital history. Thus a Freudian analysis gives a true account of the sexual \textit{consequences} of a given life-line, but it is a true \textit{diagnosis} only in that sense.

Psychology becomes now for the 1st time rooted in biology. The tendencies of the soul, \& the mind's development, are seen to be controlled from the 1st by the effort to compensate for organic defects or for positions of inferiority. Everything that is exceptional or individual in the disposition of an organic being originates in this way. The principle is common to man \& animal, probably even to the vegetable kingdom also; \& the special endowments of species are to be taken as arising from experience of defects \& inferiorities in relation to their environment, which has been successfully compensated by activity, growth \& structure.

There is nothing new in the idea of compensation as a biological principle, for it has been long known that the body will over-develop certain parts in compensation for the injury of others. If 1 kidney ceases to function, e.g., the other develops abnormally until it does the work of both; if the heart springs a leak in a valve, the whole organ grows larger to allow for its loss of efficiency, \& when nervous tissue is destroyed, adjacent tissue of another kind endeavors to take on the nerve-function. The compensatory developments of the whole organism to meet the exigencies of any special work or exertion are too numerous \& well known to need illustration. But it is Dr. Adler who has 1st transferred this principle bodily to psychology as a fundamental idea, \& demonstrated the part it plays in the soul \& intellect.

Adler recommends the study of Individual Psychology not only to doctors, but generally to laymen \& especially to teachers. Culture in psychology has become a general necessity, \& must be firmly advocated in the teeth of popular opposition to it, which is founded upon the notion that modern psychology requires an unhealthy concentration of the mind upon cases of disease \& misery. It is true that the literature of psychoanalysis has revealed the most central \& the most universal evils in modern society. But it is not now a question of contemplating our errors, it is necessary that we should learn by them. We have been trying to live as though the soul of man were not a reality, as though we could build up a civilized life in defiance of psychic truths. What Adler proposes is not the universal study of psycho-pathology, but the practical reform of society \& culture in accordance with a positive \& scientific psychology to which he has contributed the 1st principles. But this is impossible if we are too much afraid of the truth. The clearer consciousness of right aims in life, which is indispensable to us, cannot be gained without a deeper understanding also of the mistakes in which we are involved. We may not desire to know ugly facts, but the more truly we are aware of life, the more clearly we perceive the real errors which frustrate it, much as the concentration of a light gives definition to the shadows.

A positive psychology, useful for human life, cannot be derived from the psychic phenomena alone, still less from pathological manifestations. It requires also a regulative principle, \& Adler has not shrunk from this necessity, by recognizing, as if it were of absolute metaphysical validity, the logic of our communal life in the world.

Recognizing this principle, we must proceed to estimate the psychology of the individual in relation too it. The way in which an individual's inner life is related to the communal being is distinguishable in 3 ``life-attitudes,'' as they are called -- his general reactions to society, to work \& to love.

By their feeling towards society as a whole -- to any other \& to tell others -- man \& women may know how much social courage they possess. The feeling of inferiority is always manifested in a sense of fear or uncertainty in the presence of society, whether its outward expression is 1 of timidity or defiance, reserve or over-anxiety. All feelings of innate suspicion or hostility, of an undefined caution \& desire for some concealment, when such feelings affect the individual in social relations generally, evince the same tendency to withdraw from reality, which inhibits self-affirmation. The ideal, or rather normal, attitude to society is an unstrained \& unconsidered assumption of human equality unchanged by any inequalities of position. Social courage depends upon this feeling of secure membership of the human family, a feeling which depends upon the harmony of one's own life. By the tone of his feeling towards his neighbors, his township \& nation \& to other nationalities, \& even by his reactions when he reads of all these things in his newspaper, a man may infer how securely his own soul is grounded in itself.

The attitude towards work is closely dependent upon this self-security in society. In the occupation by which a man earns his share in social goods \& privileges, he has to face the logic of social needs. If he has too great a sense of weakness or division from society, it will make him unable to believe that his worth will ever be recognized, \& he will not even work for recognition: instead, he will play for safety, \& work for money or advantage only, suppressing his own valuation of what is the truest service he can render. He will always be afraid to supply or demand the best, for fear it may not pay. Or he may be always seeking for some quiet backwater of the economic life, where he can do something just as he likes himself, without proper consideration of either usefulness or profit. In both cases it is not only society that suffers by not getting the best service: the individual who has not attained his proper social significance is also deeply dissatisfied. The modern world is full of men, both successful \& unsuccessful in a worldly sense, who are in open conflict with their occupation. They do not believe in it, \& they blame social \& economic conditions with some real justice; but it is also a fact that they have often had too little courage to fight for the best value in their economic function. They were afraid to claim the right to give what they genuinely believed in, or else they felt disdainful of the service society really needed of them. Hence they pursued their gain in an individualistic or even furtive spirit. We must, of course, recognize that so much is wrong in the organization of society, that, besides the possibility of making mistakes of judgment, the individual who is determined to render real social service has often to face heavy opposition. But it is precisely that sense of struggle to give his best which the individual needs no less than society benefits by it. One cannot love a vocation which does not afford some experience of victory over difficulties, \& not merely of compromise with them.

It is the 3rd of these life-attitudes -- the attitude to love -- which determines the course of the erotic life. Where the 2 previous life-attitudes, to society \& to work, have been rightly adjusted, this last comes right by itself. Where it is distorted \& wrong it cannot be improved by itself apart from the others. Although we can think how to improve the social relations \& the occupations, a concentration of thought upon the individual sex-problem is almost sure to make it worse. For this is far more the sphere of results than of course. A soul that is defeated in ordinary social life, or thwarted in its occupation, acts in the sex-life as though it were trying to obtain compensation for the kinds of expression of which it fails in their proper spheres. This is actually the best way in which we can understand all sexual vagaries, whether they isolate the individual, degrade the sexual partner or in any way distort the instinct. The friendships of an individual also are integral with the love-life as a whole; not, as the 1st psycho-analysts imagined, because friendship is a sublimation of sexual attraction, but the other way about. Sexual compulsion -- sex as an insubordinate psychic factor -- is an abnormal substitute for the vitalizing intimacy of useful friendships, \& homosexuality is always the consequence of incapability for love.

The meaning \& value which we give to sensations are also united closely with the erotic life, as many good poets have testified. The quality of our feeling for Nature, our response to the beauty of sea \& land, \& to significances of form \& sound \& color, as well as our confidence in scenes of storm \& gloom, are all involved with our integrity as lovers. The aesthetic life, with all it means to art \& culture, is thus ultimately derived, through individuals, from social courage \& intelligent usefulness.

We ought not to regard the communal feeling as something to be created with difficulty. It is as natural \& inherent as egoism itself, \& indeed as a principle of life it has priority. We have not to create, but only to liberate, it where it is repressed. It is the saving principle of life as we experience it. If anyone thinks that the services of `busmen, railwaymen \& milkmen would be rendered as well as they are without the presence of very much instinctive communal feeling he must be suspected of a highly neurotic scheme of apperception. What inhibits it is, to speak bluntly, the enormous vanity of the human soul, which is, moreover, so subtle that no professional psychologist before Adler had been able to demonstrate it, though a few artists had divined its omnipresence. All unsuspected as it often is, the ambition of many a minor journalist or shop-assistant, to say nothing of the great ones of the world, would be enough to bring about the fall of an archangel. Every feeling of inferiority that has embittered his contact with life has fed the imagination of greatness with another god-like assumption until, in many cases, the fantasy has become so inflated as to demand not even supremacy in this world for its appeasement, but the creation of a new world altogether, \& to be the god of it. This revelation of the depth of human nature is verified, not so strikingly from the study of cases of practical ambition, however Napoleonic, as from those of passive resistance, procrastination, \& malingering, for it is these which show most clearly that an individual who feels painfully unable to dominate the real world will refuse to co-operate with it, at whatever disadvantage to himself, partly in order to tyrannize over a narrower sphere, \& partly even from an irrational feeling that the real world, without his divine assistance, will some day crumble \& shrink to his own diminished measure.\footnote{In case this should seem an exaggeration, we may recall the fact that nearly all the narrowest kind of sects, religious or secular, have a belief in world-catastrophe: the world from which they have withdrawn, \& which they despair of converting, is to be brought to destruction, \& only a remnant will survive, who will be of their own persuasion.}

The question is thus raised, how should we act, knowing this tendency to inordinate vanity in the human soul, \& that we dare not merely add to that vanity by assuming ourselves to be miraculous exceptions? Adler's reply is that we should preserve a certain attitude to all our experience, which he calls the attitude of ``half-\&-half.'' Our conception of normal behavior should be to allow the world or society, or the person with whom we are confronted, to be somehow in the right equally with ourselves. We should not depreciate either ourselves or our environment; but, assuming that each is 1-half in the right, affirm the reality of ourselves \& others equally. This applies not only to contacts with other souls, but to our mental reactions towards rainy weather, holidays or comforts that we cannot afford, \& even to the omnibus we have just missed.

Rightly understood, this is not an ideal of difficult \& distasteful humility. It is in reality a tremendous assumption of worth, to claim exactly equal reality \& omnipotence with the whole of the rest of creation, in whatever particular manifestation we may be meeting with it. To claim less than this is a false humility, for what results from any contact we make does in fact depend for half its reality upon the way in which we make it. The individual should affirm his part in everything which occurs to him, as his own half of it.

This is often a particularly difficult counsel to keep in relation to the occupation. In their business, people face more naked realities than are usually allowed to appear in social life; \& it is often almost impossible to allow equal validity to one's own aims \& to the conditions of a disorganized world. To do so, means the admission that conditions, just such as they are, are one's real problem -- \&, indeed, one's proper sphere of action. The division of labor, logical \& useful as it is in itself, has given opportunity for human megalomania to create entirely false inequalities, distinctions \& injustices, so that we live in an economic disorder which will hardly hold together. To such crazy conditions, the best of men often find it difficult to oppose themselves with perseverance, equally grating its reality \& working for its reform. They are tempted to acquiesce in disorder by some inner subterfuge, or to devote themselves to superficial remedies which evade the real problem; \& sometimes they treat their work-life as an unavoidable contamination by things inherently squalid, quite unaware that such an attitude makes them conceited, haughty \&, in a profound sense, unscrupulous. It occurs to very few that the right way would be to make alliance on human grounds with others in the same predicament \& profession, to assert its proper dignity as a social service \& improve it; but this is the only way in which the individual can really be reconciled with his economic function. Many of those who complain most about the conditions prevailing in their work are doing nothing whatever to reorganize it as a function of human life, \& never think of attacking the anarchic individualism which is its ruin. We derive it from Individual Psychology, as a categorical imperative, that every man's duty is to work to make his profession, whatever it may be, into a brotherhood, a friendship, a social unity with a powerful morale of co-operation, \& that if a man does not want to do this his own psychological state is precarious. It is true that now, in many professions, the task that this presents is terribly difficult. It is all the more essential that the effort should be made towards integration. For a man's work will never liberate the forces of his psyche unless he is striving, in a large sense, to make it the expression of his whole being, \& his idea of his profession must be not only an executive in which he has independence of action, but also a legislative in which he has some authority of direction. In a man's business life that half-\&-half valuation leads equally to recognition of reality \& to struggle with it by the only realistic method, which is necessarily co-operative.

The pedagogic principles of Individual Psychology, infallible as far as they go, are useless without this practical work of social organization. What has been written above of an individual's duty in his occupation applies in a large sense to his entire social function. A person's function includes active membership of his nation \& of humanity, to say nothing of his family. There is a certain parliament which rises for no vacation, \& to whose decisions all elected assemblies must in the end defer. It meets in schools, markets, \& everywhere on sea \& land, for it is the Parliament of Man, in which every word or look exchanged, whether of courtesy or recrimination, of wisdom or folly, has its measure of importance in the affairs of the race. It is everyone's interest to make this wide assembly more united \& its discussion more intelligible, for none of us has any real human existence except by reflection from it. When its conclaves are peaceful, all our lives are heightened in tone, health \& wealth accrues \& arts \& education flourish; when its conversation is reversed \& suspicious, work fails, men starve \& children languish. In the heat of its dissensions we perish by the million. All its decrees, by which we live or die, \& grow or decay, are rooted in our individual attitudes towards man, woman \& child in every relation of life.

When we face, objectively, this fact of the relation of all souls \& their mutual responsibility, what are we to think of the inner confusion of the neurotic? Is it not simply a narrowing of the sphere of interest, an over-concentration upon certain personal or subjective gains? The neurotic soul is the result of treating the rest of humanity as though its life \& aims were altogether of less importance than one's own, \& thus losing interest in any larger life. Paradoxically, it often happens that a neurotic has very large schemes of saving himself \& others. He is intelligent enough to try to compensate his real sense of isolation \& impotence in the human assembly, by a fantasy of exaggerated importance \& beneficent activity. He may want to reform education, to abolish war, to establish universal brotherhood or create a new culture, \& even plans or joins societies with these aims. He is defeated in such aims, of course, by the unreality of his contact with others \& with life as a whole. It is as though he had taken a standpoint outside of life altogether \& were trying to direct it by some unexplained magic.

Modern city life especially, with its intellectualism, gives unlimited scope for the neurotic thus to compensate his real unsociability with imaginary messianism, \& the result is the disintegration of a people full of saviors who are not on speaking terms.

What is needed, of course, is something very different. It is not that the individual should renounce messianism; for it is a fact that a share of responsibility for the whole future of the race is his alone. It is only necessary that he should take a reasonable view of his power to save society, correctly viewed from his own standpoint: he must become able to regard his immediate personal relations \& his occupations as if \textit{they} were of world-importance, for in fact they are so, being the only world-meaning an individual has. When they are chaotic or wrong, it is because we do not, in day-to-day experience, treat them as things of universal meaning. We sometimes treat them as important, no doubt, but generally in a personal sense only.

This tendency of the modern soul, to narrow the sphere of interest, both practically \& ideally, is most difficult to subdue, because it is reinforced by the scheme of apperception. For that reason an individual alone cannot do it, excepting only in rare cases. He needs conference with other minds, \& an entirely new kind of conference. A resolution to treat one's immediate surroundings \& daily activities as if they were the supreme significance of life brings an individual immediately into conflict with internal resistances of his own, \& often with external difficulties also, which he cannot at once understand \& which no others could rightly estimate unless they were making the same experiment. Hence, the practice of Individual Psychology demands that its students should submit themselves to mutual scrutiny, each one to be estimated by the others as a whole personality. This practice, striking at the root of the false individualism which is the basis of all neurosis, is naturally very difficult to initiate. Upon its success, however, depends the whole future of psycho-analysis as an influence in life at large, outside of clinics \& consulting rooms.

In Vienna the work of such groups has already made itself felt in education. The co-operation it has established between teachers \& medical practitioners has revolutionized the work of certain schools, \& established an equality between teachers \& pupils \& between pupils themselves, which has cured many children of criminal tendencies, dullness \& laziness. Abolition of competition \& the cultivation of encouragement have been found to liberate the energy of both pupils \& teachers. These changes are already affecting the surrounding family life, which comes into question immediately the child is psychologically considered. Education, though naturally the 1st, is not the only sphere of life which out to be invaded by the activity of such groups. Business \& political circles, which experience the deadlock of modern life most acutely, need to be vitalized with knowledge of human nature, which they have forgotten how to recognize.

It is for this work of releasing a new energy for daily life \& its reformation, that Alfred Adler has founded the International Society for Individual Psychology. The culture of human behavior which this work has begun already to propagate might well be mistaken for an almost platitudinous ethics, but for 2 things -- its practical results, \& the background of scientific method out of which it is appearing. In his realistic grasp of the social nature of the individual's problem \& his inexorable demonstration of the unity of health \& harmonious behavior, Adler resembles no one so much as the great Chinese thinkers. If the occidental world is not too far gone to make use of his service, he may well come to be known as the Confucius of the West. -- \textsc{Phillipe Mairet}'' -- \cite[pp. 9--30]{Adler_science_living}

%------------------------------------------------------------------------------%

\section{The Science of Living}
``Only a science which is directly related to life, said the great philosopher William James, is really a science. It might also be said that in a science which is directly related to life theory \& practice become almost inseparable. The science of life, precisely because it models itself directly on the movement of life, becomes a science of living. These considerations apply with special force to the science of Individual Psychology. Individual Psychology tries to see individual lives as a whole \& regards each single reaction, each movement \& impulse as an articulated part of an individual attitude towards life. Such a science is of necessity oriented in a practical sense, for which the aid of knowledge we can correct \& alter our attitudes. Individual Psychology is thus \textit{prophetic} in a double sense: not only does it predict what will happen, but, like the prophet Jonah, it predicts what \textit{will} happen in order that it should \textit{not} happen.

The science of Individual Psychology developed out of the effort to understand that mysterious creative power of life -- that power which expresses itself in the desire to develop, to strive \& to achieve -- \& even to compensate for defeats in 1 direction by striving for success in another. This power is \textit{teleological} -- it expresses itself in the striving after a goal, \& in this striving every bodily \& psychic movement is made to co-operate. It is thus absurd to study bodily movements \& mental conditions abstractly without relation to an individual whole. It is absurd, e.g., that in criminal psychology we should pay so much more attention to the crime than to the criminal. It is the criminal, not the crime that counts, \& no matter how much we contemplate the criminal act we shall never understand its criminality unless we see it as an episode in the life of a particular individual. The same outward act may be criminal in 1 case \& not criminal in another. The important thing is to understand the individual context -- the goal of an individual's life which marks the line of direction for all his acts \& movements. This goal enables us to understand the hidden meaning behind the various separate acts -- we see them as parts of a whole. Vice versa when we study the parts -- provided we study them as parts of a whole -- we get a better sense of the whole.

In the author's own case the interest in psychology developed out of the practice of medicine. The practice of medicine provided the teleological or purposive viewpoint which is necessary for the understanding of psychological facts. In medicine we see all organs striving to develop towards definite goals. They have definite forms which they achieve upon maturity. Moreover, in cases where there are organic defects we always find nature making special efforts to overcome the deficiency, or else to compensate for it by developing another organ to take over the functions of the defective one. Life always seeks to continue, \& the life force never yields to external obstacles without a struggle.

Now the movement of the psyche is analogous to the movement of organic life. In each mind there is the conception of a goal or ideal to get beyond the present state, \& to overcome the present deficiencies \& difficulties by postulating a concrete aim for the future. By means of this concrete aim or goal the individual can think \& feel himself superior to the difficulties of the present because he has in mind his success of the future. Without the sense of a goal individual activity would cease to have anything meaning.

All evidence points to the fact that the fixing of this goal -- giving it a concrete form -- must take place early in life, during the formative period of childhood. A kind of prototype or model of a matured personality begins to develop at this time. We can imagine how the process takes place. A child, being weak, feels inferior \& finds itself in a situation which it cannot bear. Hence it strives to develop, \& it strives to develop along a line of direction fixed by the goal which it chooses for itself. The material used for development at this stage is less important than the goal which decides the line of direction. How this goal is fixed it is difficult to say, but it is obvious that such a goal exists \& that it dominates the child's every movement. Little is indeed understood about powers, impulses, reasons, abilities or disabilities at this early period. As yet there is really no key, for the direction is definitely established only after the child has fixed its goal. Only when we see the direction in which a life is tending can we guess what steps will be taken in the future.

When the prototype -- that early personality which embodies the goal -- is formed, the line of direction is established \& the individual becomes definitely oriented. It is this fact which enables us to predict what will happen later in life. The individual's apperceptions are from then on bound to fall into a groove established by the line of direction. The child will not perceive given situations as they actually exist, but according to a personal scheme of apperception -- that is to say, he will perceive situations under the prejudice of his own interests.

An interesting fact that has been discovered in this connection is that children with organic defects connect all their experiences with the function of the defective organ. E.g., a child having stomach trouble shows an abnormal interest in eating, while one with defective eyesight is more preoccupied with things visible. This preoccupation is in keeping with the private scheme of apperception which we have said characterizes all persons. It might be suggested, therefore, that in order to find out where a child's interest lies we need only to ascertain which organ is defective. But things do not work out quite so simply. The child does not experience the fact of organ inferiority in the way that an external observer sees it, but as modified by its own scheme of apperception. Hence while the fact of organ inferiority counts as an element in the child's scheme of appreciation, the external observation of the inferiority does not necessarily give the cue to the scheme of apperception.

The child is steeped in a scheme of relativity, \& in this he is indeed like the rest of us -- none of us is blessed with the knowledge of the absolute truth. Even our science is not blessed with absolute truth. It is based on common sense, which is to say that it is ever changing \& that it is content gradually to replace big mistakes by smaller ones. We all make mistakes, but the important thing is that we can correct them. Such correction is easier at the time of the formation of the prototype. \& when we do not correct them at that time, we may correct the mistakes later on by recalling the whole situation of that period. Thus if we are confronted with the task of treating a neurotic patient, our problem is to discover, not the ordinary mistakes he makes in later life, but the very fundamental mistakes made early in his life in the course of the constitution of his prototype. If we discover these mistakes, it is possible to correct them by appropriate treatment.

In the light of Individual Psychology the problem of inheritance thus decreases in importance. It is not what one has inherited that is important, but what one does with his inheritance in the early years -- that is to say, the prototype that is built up in the childhood environment. Heredity is of course responsible for inherited organic defects, but our problem there is simply to relieve the particular difficulty \& place the child in a favorable situation. As a matter of fact we have even a great advantage here, inasmuch as when we see the defect we know how to act accordingly. Oftentimes a healthy child without any inherited defects may fare worse through malnutrition or through any of the many errors in upbringing.

In the case of children born with imperfect organs it is the psychological situation which is all-important. Because these children are placed in a more difficult situation they show marked indications of an exaggerated feeling of inferiority. At the time the prototype is being formed they are already more interested in themselves than in others, \& they tend to continue that way later on in life. Organic inferiority is not the only cause of mistakes in the prototype: other situations may also cause the same mistakes -- the situations of pampered \& hated children, e.g. We shall have occasion later on to describe these situations more in detail \& to present actual case histories illustrating the 3 situations which are particularly unfavorable, that of children with imperfect organs, that of petted children, \& that of hated children. For the present it is sufficient to note that these children grow up handicapped \& that they constantly fear attacks inasmuch as they have grown up in an environment in which they never learned independence.

It is necessary to understand the social interest from the very upset since it is the most important part of our education, of our treatment \& of our cure. Only such persons as are courageous, self-confident \& at home in the world can benefit both by the difficulties \& by the advantages of life. They are never afraid. They know that there are difficulties, but they also know that they can overcome them. They are prepared for all the problems of life, which are invariably social problems. From a human standpoint it is necessary to be prepared for social behavior. The 3 types of children we have mentioned develop a prototype with a lesser degree of social interest. They have not the mental attitude which is conducive to the accomplishment of what is necessary in life or to the solution of its difficulties. Feeling defeated, the prototype has a mistaken attitude towards the problems of life \& tends to develop the personality on the useless side of life. On the other hand our task in treating such patients is to develop behavior on the useful side \& to establish in general a useful attitude towards life \& society.

Lack of social interest is equivalent to being oriented towards the useless side of life. The individuals who lack social interest are those who make up the groups of problem children, criminals, insane persons, \& drunkards. Our problem in their case is to find means to influence them to go back to the useful side of life \& to make them interested in others. In this way it may be said that our so-called Individual Psychology is actually a social psychology.

After the social interest, our next task is to find out the difficulties that confront the individual in his development. This task is somewhat more confusing at 1st glance, but it is in reality not very complicated. We know that every petted child becomes a hated child. Our civilization is such that neither society nor the family wishes to continue the pampering process indefinitely. A pampered child is very soon confronted with life's problems. In school he finds himself in a new social institution, with a new social problem. He does not want to write or play with his fellows, for his experience has not prepared him for the communal life of the school. In fact his experiences as lived through at the prototype stage make him afraid of such situations \& make him look for more pampering. Now the characteristics of such an individual are not inherited -- far from it -- for we can deduce them from a knowledge of the nature of his prototype \& his goal. Because he has the particular characteristics conducive to his moving in the direction of his goal, it is not possible for him to have characteristics that would tend in any other direction.

The next step in the science of living lies in the study of the feelings. Not only does the axis line, the line of direction posited by the goal, affect individual characteristics, physical movements, expressions \& general outward symptoms, but it dominates the life of the feelings as well. It is a remarkable thing that individuals always try to justify their attitudes by feelings. Thus if a man wants to do good work, we will find this idea magnified \& dominating his whole emotional life. We can conclude that the feelings always agree with the individual's viewpoint of his task: they strengthen the individual in his bent for activity. We always do that which we would do even without the feelings, \& the feelings are simply an accompaniment to our acts.

We can see this fact quite clearly in dreams, the discovery of whose purpose was perhaps 1 of the latest achievements of Individual Psychology. Every dream has of course a purpose, although this was never clearly understood until now. The purpose of a dream -- expressed in general \& not specific terms -- is to create a certain movement of feeling or emotion, which movement of emotion in turn furthers the movement of the dream. It is an interesting commentary on the old idea that a dream is always a deception. We dream in the way that we would like to behave. Dreams are an emotional rehearsal of plans \& attitudes for waking behavior -- a rehearsal, however, in which the actual play may never come off. In this sense dreams are deceptive -- the emotional imagination gives us the thrill of action without the action.

This characteristic of dreams is also found in our waking life. We always have a strong inclination to deceive ourselves emotionally -- we always want to persuade ourselves to go the way of our prototypes as they were formed in the 4th or 5th year of life.

The analysis of the prototype is next in order in our scheme of science. As we have said, at 4 or 5 the prototype is already built up, \& so we have to look for impressions made n the child before or at that time. These impressions can be quite varied, far more varied than we imagine from a normal adult's point of view. 1 of the most common influences on a child's mind is the feeling of suppression brought about by a father's or mother's excessive punishment or abuse. This influence makes the child strive for release, \& sometimes this is expressed in an attitude of psychological exclusion. Thus we find that some girls having high-tempered fathers have prototypes that exclude men because they are high-tempered. Or boys suppressed by severe mothers may exclude women. This excluding attitude may of course be variously expressed: e.g., the child may become bashful, or on the other hand, he may become perverted sexually (which is simply another way of excluding women). Such perversions are not inherited, but arise from the environment surrounding the child in these years.

The early mistakes of the child are costly. \& despite this fact the child receives little guidance. Parents do not know or will not confess to the child the results of their experiences, \& the child must thus follow his own line.

Curiously enough we will find that no 2 children, even those born in the same family, grow up in the same situation. Even within the same family the atmosphere that surrounds each individual child is quite particular. Thus the 1st child has notoriously a different set of a circumstances from the other children. The 1st child is at 1st alone \& is thus the center of attention. Once the 2nd child is born, he finds himself dethroned \& he does not like the change of situation. In fact it is quite a tragedy in his life that he has been in power \& is so no longer. This sense of tragedy goes into the formation of his prototype \& will crop out in his adult characteristics. As a matter of fact case histories show that such children always suffer downfall.

Another intra-family difference of environment is to be found in the different treatments accorded to boys \& to girls. The usual case is for boys to be overvalued \& the girls to be treated as if they could not accomplish anything. These girls will grow up always hesitating \& in doubt. Throughout life they will hesitate too much, always remaining under the impression that only men are really able to accomplish anything.

The position of the 2nd child is also characteristic \& individual. He is in an entirely different position from that of the 1st child, inasmuch as for him there is always a pace-maker, moving along parallel with him. Usually the 2nd child overcomes his pace-maker, \& if we look for the cause we shall find simply that the older child was annoyed by having such a competitor \& that the annoyance in the end affected his position in the family. The older child becomes frightened by the competition \& does not do so well. He sinks more \& more in the estimation of his parents, who begin to appreciate the 2nd child. On the other hand the 2nd child is always confronted by the pace-maker, \& he is thus always in a race. All his characteristics will reflect this peculiar position in the family constellation. He shows rebellion \& does not recognize power or authority.

History \& legend recount numerous incidents of powerful youngest children. Joseph is a case in point: he wanted to overcome all the others. The fact that a younger brother was born into the family unknown to him years after he left home obviously does not alter the situation. His position was that of the youngest. We find also the same description in all the fairy tales, in which the youngest child plays the leading role. We can see how these characteristics actually originate in early childhood \& cannot be changed until the insight of the individual has increased. In order to readjust a child you must make him understand what happened in his 1st childhood. He must be made to understand that his prototype is erroneously influencing all the situations in his life.

A valuable tool for understanding the prototype \& hence the nature of the individual is the study of old remembrances. All our knowledge \& observation force us to the conclusion that our remembrances belong to the prototype. An illustration will make our point clear. Consider a child of the 1st type, one with imperfect organs -- with a weak stomach, let us say. If he remembers having seen something or heard something it will probably in some way concern eatables. Or take a child that is left-handed: his left-handedness will likewise affect his viewpoint. A person may tell you about his mother who pampered him, or about the birth of a younger child. He may tell you how he was beaten, if he had a high-tempered father, or how he was attacked if he was a hated child at school. All such indications are very valuable provided we learn the art of reading their significance.

The art of understanding old remembrances involves a very high power of sympathy, a power to identify oneself with the child in his childhood situation. It is only by such power of sympathy that we are able to understand the intimate significance in a child's life of the advent of a younger child in the family, or the impression made on a child's mind by the abuse of a high-tempered father.

\& while we are on the subject it cannot be overemphasized that nothing is gained by punishing, admonishing \& preaching. Nothing is accomplished when neither the child nor the adult knows on which point the change has to be made. When the child does not understand, he becomes slyer \& more cowardly. His prototype, however, cannot be changed by such punishment \& preaching. It cannot be changed by mere experience of life, for the experience of life is already in accordance with the individual's personal scheme of apperception. It is only when we get at the basic personality that we accomplish any changes.

If we observe a family with badly developed children, we shall see that though they all seem to be intelligent (in the sense that if you ask a question they give the right answer), yet when we look for symptoms \& expressions, they have a great feeling of inferiority. Intelligence of course is not necessarily common sense. The children have an entirely personal -- what we might term, a private -- mental attitude of the sort that one finds among neurotic persons. In a compulsion neurosis, e.g., the patient realizes the futility of always counting windows but cannot stop. One interested in useful things would never act this way. Private understanding \& language are also characteristic of the insane. The insane never speak in the language of common sense, which represents the height of social interest.

If we contrast the judgment of common sense with private judgment, we shall find that the judgment of common sense is usually nearly right. By common sense we distinguish between good \& bad, \& while in a complicated situation we usually make mistakes, the mistakes tend to correct themselves through the very movement of common sense. But those who are always looking out for their own private interests cannot distinguish between right \& wrong as readily as others. In fact they rather betray their inability, inasmuch as all their movements are transparent to the observer.

Consider e.g. the commission of crimes. If we inquire about the intelligence, the understanding \& the motive of a criminal, we shall find that the criminal always looks upon his crimes as both clever \& heroic. He believes that he has achieved a goal of superiority -- namely, that he has become more clever than the police \& is able to overcome others. He is thus a hero in his own mind, \& does not see that his actions indicate something quite different, something very far from heroic. His lack of social interest, which puts his activity on the useless side of life, is connected with a lack of courage, with cowardice, but he does not know this. Those who turn to the useless side of things are often afraid of darkness \& isolation; they wish to be with others. This is cowardice \& should be labeled as such. Indeed the best way to stop crime would be to convince everybody that crime is nothing but an expression of cowardice.

It is well known that some criminals when they approach the age of 30 will take a job, marry \& become good citizens in later life. What happens? Consider a burglar. How can a 30-year old burglar compete with a 20-year old burglar? The latter is cleverer \& more powerful. Moreover, at the age of 30 the criminal is forced to live differently from the way he lived before. As a result the profession of crime no longer pays the criminal \& he finds it convenient to retire.

Another fact to be borne in mind in connection with criminals is that if we increase the punishments, so far from frightening the individual criminal, we merely help to increase his belief that he is a hero. We must not forget that the criminal lives in a self-centered world, a world in which one will never find true courage, self-confidence, communal sense, or understanding of common values. It is not possible for such persons to join a society. Neurotics seldom start a club, \& it is an impossible feat for persons suffering from agoraphobia or for insane persons. Problem children or persons who commit suicide never make friends -- a fact for which the reason is never given. There is a reason, however: they never make friends because the early life took a self-centered direction. Their prototypes were oriented towards false goals \& followed lines of direction on the useless side of life.

Let us now consider the program which Individual Psychology offers for the education \& training of neurotic persons -- neurotic children, criminals, \& persons who are drunkards \& wish to escape by such means from the useful side of life.

In order to understand easily \& quickly what is wrong, we begin by asking at what time the trouble originated. Usually the blame is laid on some new situation. But this is a mistake, for before this actual occurrence, our patient -- so we shall find upon investigation -- had not been well prepared for the situation. So long as he was in a favorable situation the mistakes of his prototype were not apparent, for each new situation is in the nature of an experiment to which he reacts according to the scheme of apperception created by his prototype. His responses are not mere reactions, they are creative \& consistent with his goal, which is dominant throughout his life. Experience taught us early in our studies of Individual Psychology that we might exclude the importance of inheritance, as well as the importance of an isolated part. We see that the prototype answers experiences in accordance with its own scheme of apperception. \& it is this scheme of apperception that we must work upon in order to produce any results.

This sums up the approach of Individual Psychology which has been developed in the last 25 years. As one may see, Individual Psychology has traveled a long way in a new direction. There are many psychologies \& psychiatries in existence. 1 psychologist takes 1 direction, another another direction, \& no one believes that the others are right. Perhaps the reader, too, should not rely on belief \& faith. Let him compare. He will see that we cannot agree with what is called ``drive'' psychology (McDougall represents this tendency best in America), because in their ``drives'' too big a place is set aside for inherited tendencies. Similarly we cannot agree with the ``conditioning'' \& ``reactions'' of Behaviorism. It is useless to construct the fate \& character of an individual out of ``drives'' \& ``reactions'' unless we understand the goal to which such movements are directed. Neither of these psychologies thinks in terms of individual goals.

It is true that when the word ``goal'' is mentioned, the reader is likely to have a hazy impression. The idea needs to be concretized. Now in the last analysis to have a goal is to aspire to be like God. But to be like God is of course the ultimate goal -- the goal of goals, if we may use the term. Educators should be cautious in attempting to educate themselves \& their children to be like God. As a matter of fact we find that the child in his development substitutes a more concrete \& immediate goal. Children look for the strongest person in their environment \& make him their model or their goal. It may be the father, or perhaps the mother, for we find that even a boy may be influenced to imitate his mother if she seems the strongest person. Later on they want to be coachmen because they believe the coachman is the strongest person.

When children 1st conceive such a goal they behave, feel \& dress like the coachman \& take on all the characteristics consistent with the goal. But let the policeman lift a finger, \& the coachman becomes nothing $\ldots$ Later on the ideal may become the doctor or the teacher. For the teacher can punish the child \& thus he arouses his respect as a strong person.

The child has a choice of concrete symbols in selecting his goal, \& we find that the goal he chooses is really an index of his social interests. A boy, asked what he wanted to be in later life, said, ``I want to be a hangman.'' This displays a lack of social interest. The boy wished to be the master of life \& death -- a role which belongs to God. He wished to be more powerful than society, \& he was thus headed for the useless life. The goal of being a doctor is also fashioned around the God-like desire of being master of life \& death, but here the goal is realized through social service.'' -- \cite[pp. 31--55]{Adler_science_living}

%------------------------------------------------------------------------------%

\section{The Inferiority Complex}
``The use of the terms ``consciousness'' \& ``unconsciousness'' to designate distinctive factors is incorrect in the practice of Individual Psychology. Consciousness \& unconsciousness move together in the same direction \& are not contradictions, as is so often believed. What is more, there is no definite line of demarcation between them. It is merely a question of discovering the purpose of their joint movement. It is impossible to decide on what is conscious \& what is not until the whole connection has been obtained. This connection is revealed in the prototype, that pattern of life which we analyzed in the last chapter.

A case history will serve to illustrate the intimate connection between conscious \& unconscious life. A married man, 40 years old, suffered from 1 anxiety -- a desire to jump out of the window. He was always struggling against this desire, but aside from this he was quite well. He had friends, a good position, \& lived with his wife happily. His case is inexplicable except in terms of the collaboration of consciousness \& unconsciousness. Consciously he had the feeling that he must jump out of a window. Nonetheless he lived on, \& in fact he never even attempted to jump out of a window. The reason for this is that there was another side to his life, a side in which a struggle against his desire to commit suicide played an important part. As a result of the collaboration of this unconscious side of his being with his consciousness, he came out victorious. In fact in his ``style of life'' -- to use a term about which we shall have more to say in a later chapter -- he was a conqueror who had attained the goal of superiority. The reader might ask how could this man feel superior when he had this conscious tendency \textit{to commit suicide?} The answer is that there was something in his being that was fighting his battle against his suicidal tendency. It is his success in this battle that made him a conqueror \& a superior being. Objectively his struggle for superiority was conditioned by his own weakness, as is very often the case with persons who in 1 way or another feel inferior. But the important thing is that in his own private battle his striving for superiority, his striving to live \& to conquer, came out ahead of his sense of inferiority \& desire to die -- \& this despite the fact that the latter was expressed in his conscious life \& the former in his unconscious life.

Let us see if the development of this man's prototype bears out our theory. Let us analyze his childhood remembrances. At an early age, we learn, he had trouble at school. He did not like other boys \& wanted to run away from them. Nonetheless he collected all his powers to stay \& face them. In other words we can already perceive an effort on his part to overcome his weakness. He faced his problem \& conquered.

If we analyze our patient's character, we shall see that his 1 aim in life was to overcome fear \& anxiety. In this aim his conscious ideas cooperated with his unconscious ones \& formed a unity. Now a person who does not see the human being as a unity might believe that this patient was not superior \& was not successful. He might think him to be only an ambitious person, one who wanted to struggle \& fight but who was at bottom a coward. Such a view would be erroneous, however, since it would not take into consideration all the facts in the case \& interpret them with reference to the unity of a human life. Our whole psychology, our whole understanding or striving to understand individuals would be futile \& useless if we could not be sure that the human being is a unity. If we presupposed 2 sides without relation to one another it would be impossible to see life as a complete entity.

In addition to regarding an individual's life as a unity, we must also take it together with its context of social relations. Thus children when 1st born are weak, \& their weakness makes it necessary for other persons to care for them. Now the style or the pattern of a child's life cannot be understood without reference to the persons who look after him \& who make up for his inferiority. The child has interlocking relations with the mother \& family which could never be understood if we confined our analysis to the periphery of the child's physical being in space. The individuality of the child cuts across his physical individuality, it involves a whole context of social relations.

What applies to the child applies also, to a certain extent, to men as a whole. The weakness which is responsible for the child's living in a family group is paralleled by the weakness which drives men to live in society. All persons feel inadequate in certain situations. They feel overwhelmed by the difficulties of life \& are incapable of meeting them single-handed. Hence 1 of the strongest tendencies in man has been to form groups in order that he may live as a member of a society \& not as an isolated individual. This social life has without doubt been a great help to him in overcoming his feeling of inadequacy \& inferiority. We know that this is the case with animals, where the weaker species always live in groups in order that their combined powers might help to meet their individual needs. Thus a herd of buffaloes can defend themselves against wolves. 1 buffalo alone would find this impossible, but in a group they stick their heads together \& fight with their feet until they are saved. On the other hand, gorillas, lions \& tigers can live isolated because nature has given them the means of self-protection. A human being has not their great strength, their claws, nor their teeth, \& so cannot live apart. Thus we find that the beginning of social life lies in the weakness of the individual.

Because of this fact we cannot expect to find that the abilities \& faculties of all human begins in society are equal. But a society that is rightly adjusted will not be behindhand in supporting the abilities of the individuals who compose it. This is an important point to grasp, since otherwise we would be led to suppose that individuals have to be judged entirely on their inherited abilities. As a matter of fact an individual who might be deficient in certain faculties if he lived in an isolated condition could well compensate for his lacks in a rightly organized society.

Let us suppose that our individual insufficiencies are inherited. It then becomes the aim of psychology to train people to live well with others, in order to help decrease the effect of their natural disabilities. The history of social progress tells the story of how men co-operated in order to overcome deficiencies \& lacks. Everybody knows that language is  a social invention, but few people realize that individual deficiency was the mother o that invention. This truth, however, is illustrated in the early behavior of children. When their desires are not being satisfied, they want to gain attention \& they try to do so by some sort of language. But if a child should not need to gain attention, he would not try to speak at all. This is the case in the 1st few months, when the child's mother supplies everything that the child wishes before it speaks. There are cases on record of children who did not speak until 6 years of age because it was never necessary for them to do so. The same truth is illustrated in the case of a particular child of deaf \& dumb parents. When he fell \& hurt himself he cried, but he cried without noise. He knew that noise would be useless as his parents could not hear him. Therefore he made the appearance of crying in order to gain the attention of his parents, but it was noiseless.

We see therefore that we must always look at the whole social context of the facts we study. We must look at the social environment in order to understand the particular ``goal of superiority'' an individual chooses. We must look at the social situation, too, in order to understand a particular maladjustment. Thus many persons are mal-adjusted because they find it impossible to make the normal contact with others by means of language. The stammerer is a case in point. If we examine the stammerer we shall see that since the beginning of his life he was never socially well adjusted. He did not want to join in activities, \& he did not want friends or comrades. His language development needed association with others, but he did not want to associate. Therefore his stammering continued. There are really 2 tendencies in stammerers -- one to associate with others, \& another that makes them seek isolation for themselves.

Later in life, among adult persons not living a social life, we find that they cannot speak in public \& have a tendency to stage fright. This is because they regard their audiences as enemies. They have a feeling of inferiority when confronted by a seemingly hostile \& dominating audience. The fact is that only when a person trusts himself \& his audience can he speak well, \& only then will he not have stage fright.

The feeling of inferiority \& the problem of social training are thus intimately connected. Just as the feeling of inferiority arises from a social maladjustment, so social training is the basic method by which we can all overcome our feelings of inferiority.

There is a direct connection between social training \& common sense. When we say that people solve their difficulties by common sense, we have in mind the pooled intelligence of the social group. On the other hand, as we indicated in the last chapter, persons who act with a private language \& a private understanding manifest an abnormality. The insane, the neurotics \& the criminals are of this type. We find that certain things are not interesting to them -- people, institutions, the social norms make no appeal to them. \& yet it is through these things that the road to their salvation lies.

In working with such persons our task is to make social facts appeal to them. Nervous persons always feel justified if they show good will. But more than good will is needed. We must teach them that it is what they actually accomplish, what they actually give, that matters in society.

While the feeling of inferiority \& the striving for superiority are universal, it would be a mistake to regard this fact as indicating that all men are equal. Inferiority \& superiority are the general conditions which govern the behavior of men, but besides these conditions there are differences in bodily strength, in health, \& in environment. For that reason different mistakes are made by individuals in the same given conditions. If we examine children we shall see that there is no one absolutely fixed \& right manner for them to respond. They respond in their individual ways. They strive towards a better style of life, but they all strive in their own way, making their own mistakes \& their own type of approximations to success.

Let us analyze some of the variations \& peculiarities of individuals. Let us take, e.g., left-handed children. There are children who may never known that they are left-handed because they have been so carefully trained in the use of the right hand. At 1st they are clumsy \& imperfect with the right hand, \& they are scolded, criticized \& derided. It is an error to deride, but both hands should be trained. A left-handed child can be recognized in the cradle because his left hand moves more than his right. In later life he may feel that he is burdened because of the imperfection of his right hand. On the other hand, he often develops a greatest interest in his right hand \& arm, which interest is manifested, e.g., in drawing, writing, etc. In fact it is not surprising to find that later in life such a child is better trained than a normal child. Because he has had to get interested, he has gotten up earlier, so to speak, \& thus his imperfection has led him to more careful training. This is often a great advantage in developing artistic talent \& ability. A child in such a position is usually ambitious \& fights to overcome his limitations. Sometimes, however, if the struggle is a serious one, he may become envious or jealous of others \& thus develop a greater feeling of inferiority which is more difficult to overcome than in normal cases. Through constant struggling a child may become a fighting child or a fighting adult, always striving with the fixed idea in mind that he ought not to be clumsy \& deficient. Such an individual is more burdened than others.

Children strive, make mistakes, \& develop in various ways according to the prototypes they formed in the 1st 4 or 5 years of life. The goal of each is different. 1 child may want to be a painter, while another may wish himself out of this world where he is a misfit. We may know how he can overcome his imperfection, but he does not know it, \& all too often the facts are not explained to him in the right way.

Many children have imperfect eyes, ears, lungs or stomachs, \& we find their interest stimulated in the direction of the imperfection. A curious instance of this is revealed in the case of a man who suffered from attacks of asthma only when he came home at night from the office. He was a man of 45, married, \& with a good position. He was asked why the attacks always occurred after he came home from the office. He explained, ``You see, my wife is very materialistic \& I am idealistic, hence we do not agree. When I come home I would like to be quiet, to enjoy myself at home, but my wife wants to go into society \& so she complains about remaining at home. Whereupon I get into a bad temper \& start to suffocate.''

Why did this man suffocate: why did he not vomit? The fact is he was only being true to his prototype. It seems that as a child he had to be bandaged for some weakness \& this tight binding affected his breathing \& made him very uncomfortable. He had a maid servant, however, who liked him \& would sit beside him \& console him. All her interest was in him \& not in herself. She thus gave him the impression that he would always be amused \& consoled. When he was 4 years old the nurse went away to a wedding \& he accompanied her to the station crying very bitterly. After the nurse had left he said to his mother, ``The world has no more interest for me now that my nurse has gone away.''

Hence we see him in manhood as in the years of his prototype, looking for an ideal person who would always amuse him \& console him \& be interested in him alone. The trouble was not too little air but the fact that he was not being amused \& consoled at all times. Naturally, to find a person who will always amuse you is not easy. He always wanted to rule the whole situation \& to a certain degree it helped him when he succeeded. Thus when he took to suffocating, his wife stopped wanting to go to the theater or into society. He had then obtained his ``goal of superiority.''

Consciously this man was always right \& proper, but in his mind he has the desire to be the conqueror. He wanted to make his wife what he called idealistic instead of materialistic. We should suspect such a man of motives different from those on the surface $\ldots$

We often see children with imperfect eyes take more of an interest in visual things. They develop a keen faculty in this way. We see Gustav Freitag, a great poet who had poor, astigmatic eyes, accomplishing much. Poets  \& painter often have trouble with their eyes. But this in itself often creates greater interest. Freitag said about himself: ``Because my eyes were different from those of other people, it seems that I was compelled to use \& train my fantasy. I do not know that this has helped me to be a great writer, but in any case as a result of my eyesight it has come about that I can see better in fantasy than others in reality.''

If we examine the personalities of geniuses we shall often find poor eyes or some other deficiency. In the history of all ages even the gods have had some deficiency such as blindness in 1 or both eyes. The fact that there are geniuses who though nearly blind are yet able to understand better than others the differences in lines, shadows \& colors shows what can be done with afflicted children if their problems are properly understood.

Some people are more interested in eatables than others. Because of this they are always discussing what they can \& what they cannot eat. Usually such persons have had a hard time at the beginning of life in the matter of eating \& so have developed more interest in it than others. They had probably been told constantly by a watchful mother what they could \& could not eat. Such persons have to train to overcome the imperfections of their stomachs, \& they become vitally interested in what they will have for lunch, dinner or breakfast. As a result of their constant thought about eating they sometimes develop the art of cookery or become experts on questions of diet.

At times, however, a weakness of the stomach or the intestines causes people to look for a substitute for eating. Sometimes this substitute is money, \& such persons become miserly or great money-making bankers. They often strive extremely hard to collect money, training themselves for this purpose day \& night. They never stop thinking of their business, -- a fact which may sometimes give them a great advantage over others in similar walks of life. \& it is interesting to note that we often hear of rich men suffering from stomach trouble.

Let us remind ourselves at this point of the connection frequently made between body \& mind. A given defect does not always lead to the same result. There is no necessary cause \& effect relation between a physical imperfection \& a bad style of life. For the physical imperfection we can often give good treatment in the form of right nutrition \& thereby partly obviate the physical situation. But it is not the physical defect which causes the bad results: it is the patient's attitude which is responsible. That is why for the individual psychologist mere physical defects or exclusive physical causality does not exist, but only mistaken attitudes towards physical situations. Also that is why the individual psychologist seeks to foster a striving against the feeling of inferiority during the development of the prototype.

Sometimes we see a person impatient because he cannot wait to overcome difficulties. Whenever we see persons constantly in motion, with strong tempers \& passions, we can always conclude that they are persons with a great feeling of inferiority. A person who knows he can overcome his difficulties will not be impatient. On the other hand he may not always accomplish what is necessary. Arrogant, impertinent, fighting children also indicate a great feeling of inferiority. It is our task in their case to look for the reasons -- for the difficulties they have -- in order to prescribe the treatment. We should never criticize or \textit{punish} mistakes in the style of life of the prototype.

We can recognize these prototype traits among children in very peculiar ways -- in their unusual interests, in their scheming \& striving to surpass others, \& in building toward the goal of superiority. There is a type that does not trust himself in movement \& expression. He prefers to exclude others as far as possible. He prefers not to go where he is confronted with new situations but to stay in the little circle in which he feels sure. In school, in life, in society, in marriage he does the same. He is always hoping to accomplish much in his little place in order to arrive at a goal of superiority. We find this trait among many human beings. They all forget that to accomplish results, one must be prepared to meet all situations. Everything must be faced. If one eliminates certain situations \& certain persons, one has only private intelligence to justify oneself, \& this is not enough. One needs all the renovating winds of social contact \& common sense.

If a philosopher wants to accomplish his work, he cannot always go to lunch or dinner with others, for he needs to be alone for long periods of time in order to collect his ideas \& use the right method. But later on he must grow through contact with society. This contact is an important part of his development. \& so when we meet with such a person we must remember his 2 requirements. We must remember, too, that he can be useful or useless \& should therefore look carefully for the difference between useful \& useless behavior.

The key to the entire social process is to be found in the fact that persons are always striving to find a situation in which they excel. Thus children who have a great feeling of inferiority want to exclude stronger children \& play with weaker children whom they can rule \& domineer. This is an abnormal \& pathological expression of the feeling of inferiority, for it is important to realize that it is not the sense of inferiority which matters but the degree \& character of it.

The abnormal feeling of inferiority has acquired the name of ``inferiority complex.'' But complex is not the correct word for this feeling of inferiority that permeates the whole personality. It is more than a complex, it is almost a disease whose ravages vary under different circumstances. Thus we sometimes do not notice the feeling of inferiority when a person is on his job because he feels sure of his work. On the other hand he may not be sure of himself in society or in his relations with the opposite sex, \& in  this way we are able to discover his true psychological situation.

We notice mistakes in a greater degree in a tense or difficult situation. It is in the difficult or new situation that the prototype appears rightly, \& in fact the difficult situation is nearly always the new one. That is why, as we said in the 1st chapter, the expression of the degree of social interest appears in a new social situation.

If we put a child to school we may observe his social interest there just as in general social life. We can see whether he mixes with his fellows or avoids them. If we see hyperactive, sly, clever children, we must look into their minds to find the reasons. \& if we see some go forward only conditionally or hesitatingly, we must be on the lookout for the same characteristics to be revealed later on in society, life \& marriage.

We always meet persons who say, ``I would do this in this way,'' ``I would take that job,'' ``I would fight that man, $\ldots$ but $\ldots$!'' All such statements are a sign of a great feeling of inferiority, \& in fact if we read them this way we get a new light on certain emotions, such as doubt. We recognize that a person in doubt usually remains in doubt \& accomplishes nothing. However, when a person says ``I won't,'' he will probably act accordingly.

The psychologist, if he looks closely can often see contradictions in men. Such contradictions may be considered as a sign of a feeling of inferiority. But we must also observe the movements of a person who constitutes our problem on hand. Thus, his approach, his way of meeting people, may be poor, \& we must observe if he comes towards persons with a hesitating step \& bodily attitude. This hesitation will often be expressed in other situations of life. There are many persons who take 1 step forward \& 1 backward -- a sign of a great feeling of inferiority.

Our whole task is to train such persons away from their hesitating attitude. The proper treatment for such persons is to encourage them -- never to discourage them. We must make them understand that they are capable of facing difficulties \& solving the problems of life. This is the only way to build self-confidence, \& this is the only way the feeling of inferiority should be treated.'' -- \cite[pp. 56--77]{Adler_science_living}

%------------------------------------------------------------------------------%

\section{The Superiority Complex}
``In the last chapter we discussed the inferiority complex \& its relation to the general feeling of inferiority which all of us share \& struggle against. Now we have to turn to the inverse topic, the superiority complex.

We have seen how every symptom of an individual's life is expressed in a movement -- in a progress. Thus the symptom may be said to have a past \& a future. Now the future is tied up with our striving \& with our goal, while the past represents the state of inferiority or inadequacy which we are trying to overcome. That is why in an inferiority complex we are interested in the beginning, while in a superiority complex we are more interested in the continuity, in the progression of the movement itself. Moreover, the 2 complexes are naturally related. We should not be astonished if in the cases where we see an inferiority complex we find a superiority complex more or less hidden. On the other hand, if we inquire into a superiority complex \& study its continuity, we can always find a more or less hidden inferiority complex.

We muse bear in mind of course that the word complex as attached to inferiority \& superiority merely represents an exaggerated condition of the sense of inferiority \& the striving for superiority. If we look at things this way it takes away the apparent paradox of 2 contradictory tendencies, the inferiority complex \& the superiority complex, existing in the same individual. For it is obvious that as normal sentiments the striving for superiority \& the feeling of inferiority are naturally complementary. We should not strive to be superior \& to succeed if we did not feel a certain lack in our present condition. Now inasmuch as the so-called complexes develop out of the natural sentiments, there is no more contradiction in them than in the sentiments.

The striving for superiority never ceases. It constitutes in fact the mind, the psyche of the individual. As we have said, life is the attainment of a goal or form, \& it is the striving for superiority which sets the attainment of form into motion. It is like a stream which drags along all the material it can find. If we look at lazy children \& see their lack of activity, their lack of interest in anything, we should say that they do not seem to be moving. But nonetheless we find in them a desire to be superior, a desire which makes them say, ``If I were not so lazy, I could be president.'' They are moving \& striving conditionally, so to speak. They hold a high opinion of themselves \& take the view that they could accomplish much on the useful side of life, if $\ldots$! This is lying, of course -- it's fiction, but as we all know, mankind is very often satisfied with fiction. \& this is especially true of persons who lack courage. They content themselves quite well with fiction. They do not feel very strong \& so they always make detours -- they always want to escape difficulties. Through this escape, through this avoiding of battle they get a feeling of being much stronger \& cleverer than they really are.

We see children who start stealing suffering from the feeling of superiority. They believe they are deceiving others; that others do not know they are stealing. Thus they are richer with little effort. This same feeling is very pronounced among criminals who have the idea that they are superior heroes.

We have already spoken of this trait from another aspect as a manifestation of private intelligence. It is not common or social sense. If a murderer thinks himself a hero, it is a private idea. He is lacking in courage since he wants to arrange matters so that he escapes the solution of the problems of life. Criminality is thus the result of a superiority complex \& not the expression of fundamental \& original viciousness.

We see similar symptoms appearing among neurotic persons. E.g., they suffer from sleeplessness \& so are not strong enough next day to comply with the demands of their occupations. Because of sleeplessness they feel that they cannot be required to work because they are not equal to doing what they could accomplish. They lament, ``What could I not do if I could only get my sleep!''

We see this also among depressed persons suffering from anxiety. Their anxiety makes them tyrants over others. In fact they use their anxiety to rule others, for they must always have people with them, they must be accompanied wherever they go, etc. The companions are made to live their lives in accordance with the demands of the depressed person.

Melancholy \& insane persons are always the center of attention in the family. In them we see the power wielded by the inferiority complex. They complain that they feel weak \& are losing weight, etc., but nonetheless they are the strongest of all. They dominate healthy persons. This fact should not surprise us, for in our culture weakness can be quite strong \& powerful. (In fact if we were to ask ourselves who is the strongest person in our culture, the logical answer would be, the baby. The baby rules \& cannot be dominated.)

Let us study the connection between the superiority complex \& inferiority. Let us take e.g. a problem child with a superiority complex -- a child that is impertinent, arrogant \& pugnacious. We shall find that he always wants to appear greater than he really is. We all know how children with temper tantrums want to control others by getting a sudden attack. Why are they so impatient? Because they are not sure they are strong enough to attain their goal. They feel inferior. We will always discover in fighting, aggressive children an inferiority complex \& a desire to overcome it. It is as if they were trying to lift themselves on their toes in order to appear greater \& to gain by this easy method success, pride \& superiority.

We have to find methods of treatment for such children. They act that way because they do not see the coherence of life. They do not see the natural order of things. We should not censure them because they do not want to see it, for if we confront them with the question, they will always insist that they do not feel inferior but superior. We must therefore in a friendly manner explain to them our point of view \& get them gradually to understand.

If a person is a show-off it is only because he feels inferior, because he does not feel strong enough to compete with others on the useful side of life. That is why he stays on the useless ride. He is not in harmony with society. He is not socially adjusted, \& he does not know how to solve the social problems of life. \& so we always find a struggle between him \& his parents \& teachers during his childhood. In such cases the situation must be understood \& also made understandable to the children.

We see the same combination of inferiority \& superiority complexes in neurotic illnesses. The neurotic frequently expresses his superiority complex but does not see his inferiority complex. The case history of a compulsion neurotic is very illuminating in this regard. There was a young girl in close association with an elder sister who was very charming \& much esteemed. This fact is significant at the outset, for if 1 person out of a family is more outstanding than the others, the latter will suffer. This is always so, whether the favored individual be the father, 1 of the children, or the mother. A very difficult situation is created for the other members of the family, \& sometimes they feel they cannot bear it.

Now we will find among these other children that they all have an inferiority complex \& are striving toward a superiority complex. So long as they are interested not only in themselves but in others, they will solve their problems of life satisfactorily. But if their inferiority complex is clearly marked, they find themselves living, as it were, in an enemy country -- always looking out for their own interests rather than for those of others, \& thus not having the right amount of communal sense. They approach the social questions of life with a feeling that is not conducive to their solution. \& so, seeking relief, they go over to the useless side of life. We know that  this is not really relief, but it seems like relief not to solve questions but to be supported by others. They are like beggars, who are being supported by others \& who feel comfortable neurotically exploiting their weakness.

It seems to be a trait of human nature that when individuals -- both children \& adults -- feel weak, they cease to be interested socially but strive for superiority. They want to solve the problems of life in such a way as to obtain personal superiority without any admixture of social interest. As long as a person strives for superiority \& tempers it with social interest, he is on the useful side of life \& can accomplish good. But if he lacks social interest, he is not really prepared for the solution of the problems of life. In this category should be put, as we have already said, the problem children, the insane, the criminals, those who commit suicide, etc.

Now this girl of whom we started to speak grew up outside of a favorable circle \& felt herself restricted. If she had been socially interested, \& had understood what we understand, she could have developed along another line. She began to study to be a musician, but she was always at such tension, due to the inferiority complex caused by always thinking of her preferred sister, that she was blocked here too. When she was 20 her sister married \& so she began to look for marriage in order to compete with her sister. In this way she was getting in deeper, \& drifting more \& more from the healthy, useful side of life. She developed the idea that she was a bad, bad girl, \& possessed magic power which could send a person to hell.

We see this magic power as a superiority complex but she on the other hand complained, just as we sometimes hear rich men complain of how bad their fate is to be rich men. Not only did she feel that she had the god-like power of sending people to hell, but at times she got the impression that she could \& ought to save these people. Of course both of these claims were ridiculous, but by means of this system of fiction she assured herself of possessing a power that was higher than her preferred sister's. She could overcome her sister only by this game. \& so she complained that she had this power, for the more she complained about it the more plausible it was that she actually possessed it. If she had laughed about it, the claim of power would have been questionable. Only by complaining could she feel happy with her lot. We see here how a superiority complex may sometimes be hidden, not recognized as present, yet existing in fact as a compensation for the inferiority complex.

The older sister -- of whom we shall now speak -- was very much favored, for at 1 time she was the only child, much pampered, \& the center of attention in the family. 3 years later there arrived a younger sister, which fact changed the whole situation for the older girl. Formerly she had always been alone, the center of attention. Now she was suddenly thrown out of this position. As a result she became a fighting child. But there can be fighting only where there are weaker companions. A fighting child is not really courageous -- he fights only against weaker persons. If the environment is strong, then instead of becoming pugnacious, a child becomes peevish, or depressed, \& is likely to be less appreciated in the home circle for this reason.

In such cases the older child feels she is not as dearly loved as before, \& she sees the manifestations of the changed attitude as a confirmation of her view. She considers her mother the most guilty inasmuch as it is she who has brought this other girl into the home. Thus we can understand her directing attacks against her mother.

The baby, on the other hand, has to be watched, observed, pampered as all babies are, \& is thus in a favorable position. Therefore she does not need to exert herself, does not need to fight. She develops as a very sweet, very soft \& very much beloved creature -- the center of the family. Sometimes virtue in the form of obedience may conquer!

Now let us examine \& see if this sweetness, softness \& kindness was on the useful side of life or not. We may presuppose that she was so amenable \& tractable only because she was so pampered. But our civilization does not regard pampered children with favor. Sometimes the father realizes this \& wants to end this state of affairs. Sometimes the school comes into the situation. The position of such a child is always in danger \& for this reason the pampered child feels inferior. We do not notice this feeling of inferiority among pampered children so long as they are in a favorable situation, but the moment an unfavorable situation arises we see these children either breaking down \& becoming depressed or developing a superiority complex.

The superiority complex \& inferiority complex agree on 1 point, namely, that they are always on the useless side. We can never find an arrogant, impertinent child, one with a superiority complex, on the useful side of life.

When these pampered children go to school, they are no longer in a favorable situation. From that moment on we see them adopting a hesitating attitude in life \& never finishing anything. So it was with the younger sister of whom we 1st spoke. She began to learn to sew, to play the piano, etc., but after a short time she stopped. At the same time she lost interest in society, did not like to go out any more \& felt depressed. She felt herself overshadowed by her sister with her more agreeable characteristics. Her hesitating attitude made her weaker \& caused a deterioration of her character.

Later in life she hesitated in the matter of occupations \& never finished anything. She also hesitated in love \& marriage, despite her desire to compete with her sister. When she reached 30 she looked around \& found a man who was suffering from tuberculosis. Of course we can readily see that this selection would be opposed by her parents. In this case it was not necessary for her to stop action, for her parents stopped the action, \& the marriage did not take place. A year later she married a man 35 years her senior. Now as such a man is not thought to be a man any more, this marriage which was not a marriage seemed useless. We often find an expression of an inferiority complex in the selection of a much older person for marriage or in the selection of a person who cannot be married; e.g., a married man or woman. There is always a suspicion of cowardice when there are hindrances. Because this girl did not justify her feeling of superiority in marriage, she found another way of acquiring a superiority complex.

She insisted that the most important thing in this world is duty. She had to wash herself all the time. If anybody or anything touched her, she had to wash again. In this way she became wholly isolated. As a matter of fact her hands were as dirty as they could be. The reason was obvious: because of her continual washing she acquired a very rough skin that collected dirt in great quantities.

Now all this looks like an inferiority complex, but she felt herself to be the only pure person in the world \& was continually criticizing \& accusing others because they did not have her washing mania. So she played her role as in a pantomime. She had always wanted to be superior \& now in a fictitious way she was. She was the purest person in the world. So we see that her inferiority complex had become a superiority complex, very distinctly expressed.

We see the same phenomenon in megalomaniacs who believe themselves to be Jesus Christ or an emperor. Such a person is on the useless side of life \& plays his role almost as if it were true. He is isolated in life, \& we shall find, if we go back to his past, that he felt inferior \& that, in a worthless way, he developed a superiority complex.

There is the case of a boy of 15 who entered an asylum for the insane because of his hallucinations. At that time, which was before the war, he fancied that the emperor of Austria was dead. This was not true, but he claimed that the emperor had appeared to him in a dream demanding that he lead the Austrian army against the enemy. \& he is a little undersized boy! He would not be convinced when he was shown the newspapers, which reported that the emperor was stopping at his castle or that he had been out driving in his car. He insisted that the emperor was dead \& had appeared to him in a dream.

At that time Individual Psychology was trying to find out the importance of positions in sleep in indicating a person's feeling of superiority or inferiority. One can see that such information might prove useful. Some persons lie in bed in a curved line like a hedgehog, covering their heads with the covers. This expresses an inferiority complex. Can we believe such a person to be courageous? Or if we see a person stretched out straight, can we believe him weak or bent in life? Both in a literal \& metaphorical way he will appear great, as he does in sleep. It has been observed that persons who sleep on their stomachs are stubborn \& pugnacious.

This boy was examined in an attempt to find correlations between his walking behavior \& his positions in sleep. It was found that he slept with arms crossed on his breast, like Napoleon. As we all know the pictures show Napoleon with his arms in such a position. Next day the boy was asked, ``Do you know somebody of whom this position reminds you?'' He answered, ``Yes, my teacher.'' The discovery was a little disturbing until it was suggested that the teacher might be like Napoleon. This proved to be the case. Moreover, the boy had loved this teacher \& wanted to be a teacher like him. But for lack of funds with which to assure him an education, his family had to put him to work in a restaurant where he patrons had all derided him because he was undersized. He could not bear this \& wanted to escape from this feeling of humiliation. But he escaped to the useless side of life.

We are able to understand what happened in the case of this boy. In the beginning he had an inferiority complex because he was undersized \& hence derided by the guests in the restaurant. But he was constantly striving for superiority. He wanted to be a teacher. But because he was blocked in attaining this occupation, he found another goal of superiority by making a detour to the useless side of life. He became superior in sleep \& dreams.

Thus we see that the goal of superiority may be on the useless or useful side of life. If a person is benevolent, e..g, it may mean either of 2 things -- it may mean that he is socially adjusted \& wants to help, or else it may mean simply that he wants to boast. The psychologist meets with many whose main goal is to boast. There is the case of a boy who was not very accomplished in school; in fact he was so bad that he became a truant \& stole things, but he was always boastful. He did these things because of his inferiority complex. He wanted to accomplish results in some line -- be it only the line of cheap vanity. Thus he stole money \& presented prostitutes with flowers \& other gifts. 1 day he drove a car far away to a little town \& there he demanded a carriage \& 6 horses. He rode all through the town in state until he was arrested. In all his behavior his great striving was to appear greater than others -- \& greater than he really was.

A similar tendency may be remarked in the behavior of criminals -- the tendency to claim easy success, which we have already discussed in another connection. The New York newspapers some time ago reported how a burglar broke into the home of some schoolteachers \& had a discussion with them. The burglar told the women they did not know how much trouble there was in ordinary honest occupations. It was much easier to be a burglar than to work. This man had escaped to the useless side of life. But by taking this road he had developed a certain superiority complex. He felt stronger than the women, particularly since he was armed \& they were not. But did he realize that he was a coward? We know he is because we see him as a person who had escaped his inferiority complex by going over to the useless side of life. He thought himself a hero, however, \& not a coward.

Some types turn to suicide \& desire in this way to throw off the whole world with its difficulties. They seem not to care for life \& so feel superior, although they are really cowards. We see that a superiority complex is a 2nd phase. It is a compensation for the inferiority complex. We must always try to find the organic connection -- the connection which may seem to be a contradiction but which is quite in the course of human nature, as we have already shown. Once this connection is found we are in a position to treat both the inferiority \& superiority complexes.

We should not conclude the general subject of inferiority \& superiority complexes without saying a few words as to the relation of these complexes to normal persons. Everyone, as we have said, has a feeling of inferiority. But the feeling of inferiority is not a disease, it is rather a stimulant to healthy normal striving \& development. It becomes a pathological condition only when the sense of inadequacy overwhelms the individual, \& so far from stimulating him to useful activity, makes him depressed \& incapable of development. Now the superiority complex is 1 of the ways which a person with an inferiority complex may use as a method of escape from his difficulties. He assumes that he is superior when he is not, \& this false success compensates him for the state of inferiority which he cannot bear. The normal person does not have a superiority complex, he does not even have a sense of superiority. He has the striving to be superior in the sense that we all have ambition to be successful, but so long as this striving is expressed in work it does not lead to false valuations, which is at the root of mental disease.'' -- \cite[pp. 78--97]{Adler_science_living}

%------------------------------------------------------------------------------%

\section{The Style of Life}
``If we look at a pine tree growing in the valley we will notice that it grows differently from one on top of a mountain. It is the same kind of a tree, a pine, but there are 2 distinct styles of life. Its style on top of the mountain is different from its style when growing in the valley. The style of life of a tree is the individuality of a tree expressing itself \& moulding itself in an environment. We recognize a style when we see it against a background of an environment different from what we expect, for then we realize that every tree has a life pattern \& is not merely a mechanical reaction to the environment.

It is much the same way with human beings. We see the style of life under certain conditions of environment \& it is our task to analyze its exact relation to the existing circumstances, inasmuch as mind changes with alteration of the environment. As long as a person is in a favorable situation we cannot see his style of life clearly. In new situations, however, where he is confronted with difficulties, the style of life appears clearly \& distinctly. A trained psychologist could perhaps understand a style of life of a human being even in a favorable situation, but it becomes apparent to everybody when the human subject is put into unfavorable or difficult situations.

Now life, being something more than a game, does not lack difficulties. There are always situations in which human beings find themselves confronted with difficulties. It is while the subject is confronted with these difficulties that we must study him \& find out his different movements \& characteristic distinguishing marks. As we have previously said, the style of life is a unity because it has grown out of the difficulties of early life \& out of the striving for a goal.

But we are interested not so much in the past as in the future. \& in order to understand a person's future we must understand his style of life. Even if we understand instincts, stimuli, drive, etc., we cannot predict what must happen. Some psychologists indeed try to reach conclusions by noting certain instincts, impressions or traumas, but on closer examination it will be found that all these elements presuppose a consistent style of life. Thus whatever stimulates, stimulates only to \textit{save} \& \textit{fix} a style of life.

How does the notion of the style of life tie up with what we have discussed in previous chapters? We have seen how human beings with weak organs, because they face difficulties \& feel insecure, suffer from a feeling or complex of inferiority. But as human beings cannot endure this for long, the inferiority feeling stimulates them, as we have seem, to movement \& action. This results in a person having a goal. Now Individual Psychology has long called the consistent movement toward this goal a plan of life. But because this name has sometimes led to mistakes among students, it is now called a style of life.

Because an individual has a style of life, it is possible to predict his future sometimes just on the basis of talking to him \& having him answer questions. It is like looking at the 5th act of a drama, where all the mysteries are solved. We can make predictions in this way because we know the phases, the difficulties \& the questions of life. Thus from experience \& knowledge of a few facts we can tell what will happen to children who always separate themselves from others, who are looking for support, who are pampered \& who hesitate in approaching situations. What happens in the case of a person whose goal it is to be supported by others? Hesitating, he stops or escapes the solution of the questions of life. We know how he can hesitate, stop, or escape, because we have seen the same thing happen a thousand times. We know that he does not want to proceed alone but wants to be pampered. He wants to stay far away from the great problems of life, \& he occupies himself with useless things rather than struggle with the useful ones. He lacks social interests, \& as a result he may develop into a problem child, a neurotic, a criminal or a suicide -- that final escape. All these things are now better understood than formerly.

We realize, e.g., that in looking for the style of a life of a human being we may use the normal style of life as a basis for measurement. We use the socially adjusted human being as a stand, \& we can measure the variations from the normal.

At this point perhaps it would be helpful to show how we determine the normal style of life \& how on the basis of it we understand mistakes \& peculiarities. But before we discuss this we ought to mention that we do not count types in such studies. We do not consider human beings types because every human being has an individual style of life. Just as one cannot find 2 leaves of a tree absolutely identical, so one cannot find 2 human beings absolutely alike. Nature is so rich \& the possibilities of stimuli, instincts \& mistakes are so numerous, that it is not possible for 2 persons to be exactly identical. If we speak of types, therefore, it is only as an intellectual device to make more understandable the similarities of individuals. We can judge better if we postulate an intellectual classification like a type \& study its special peculiarities. However, in doing so we do not commit ourselves to using the same classification at all times; we use the classification which is most useful for bringing out a particular similarity. People who take types \& classifications seriously, once they put a person in a pigeonhole, do not see how he can be put into any other classification.

An illustration will make our point clear. E.g., when we speak of a type of individual not socially adjusted, we refer to one who leads a barren life without any social interests. This is 1 way of classifying individuals, \& perhaps it is the most important way. But consider the individual, whose interest, however limited, is centered on visual things. Such a person differs entirely from one whose interests are largely concentrated on things oral, but both of them may be socially mal-adjusted \& find it difficult to establish contact with their fellow-men. The classification by types can thus be a source of confusion if we do not realize that types are merely convenient abstractions.

Let us return now to the normal man, who is our standard for measuring variations. The normal man is an individual who lives in society \& whose mode of life is so adapted that whether he wants it or not society derives a certain advantage from his work. Also from a psychological point of view he has enough energy \& courage to meet the problems \& difficulties as they come along. Both of these qualities are missing in the case of psychopathic persons: they are neither socially adjusted nor are they psychologically adjusted to the daily tasks of life. As an illustration we may take the care of a certain individual, a man of 30 who was always at the last moment escaping the solution of his problems. He had a friend but was very suspicious of him, \& as a result this friendship never prospered. Friendship cannot grow under such conditions because the other partner feels the tension in the relation. We can readily see how this man really had no friends despite the fact that he was on speaking terms with a large number of persons. He was not sufficiently interested nor adjusted socially to make friends. In fact he did not like society, \& was always silent in company. He explained this on the ground that in company he never had any ideas \& therefore he had nothing to say.

Moreover, the man was bashful. He had a pink skin which flushed from time to time when he talked. When he could overcome this bashfulness he would speak quite well. What he really needed was to be helped in this direction without criticism. Of course when he was in this state he did not present a nice picture \& was not very much liked by his neighbors. He felt this, \& as a result his dislike for speech increased. One might say that his style of life was such that if he approached other persons in society he called attention to himself.

Next to social life \& the art of getting along with friends, is the question of occupation. Now our patient always had the fear that he might fail in his occupation, \& so he studied day \& night. He overworked \& overstrained himself. \& because he overstrained himself he put himself out of commission for solving the question of occupation.

If we compare our patient's approach to the 1st \& 2nd questions in his life, we see that he was always in too great a tension. This is a sign that he had a great feeling of inferiority. He undervalued himself \& looked on others \& on new situations as things that were unfriendly to him. He acted as though he was in an enemy country.

We have now enough data to picture the style of life of this man. We can see that he wants to go on but at the same time he is blocked because he fears defeat. It is as if he stood before an abyss, straining \& always at a tension. He manages to go forward but only conditionally, \& he would prefer to stay at home \& not mingle with others.

The 3rd question with which this man was confronted -- \& it is a question on which most persons are not very well prepared -- is the question of love. He hesitated to approach the other sex. He found that he wanted to love \& to get married, but on account of his great feeling of inferiority he was too frightened to face the prospect. He could not accomplish what he wanted \&, so we see his whole behavior \& attitude summed up in the words, ``Yes $\ldots$ but!'' We see him in love with 1 girl \& then in love with another. This is of course a frequent occurrence with neurotic persons because in a sense 2 girls are less than one. This truth sometimes accounts for a tendency towards polygamy.

\& now let us take up the reasons for this style of life. Individual Psychology undertakes to analyze the causes for a style of life. This man established his style of life during the 1st 4 or 5 years. At that time some tragedy happened which moulded \& formed him, \& so we have to look for the tragedy. We can see that something made him lose his normal interest in others \& gave him the impression that life is simply 1 great difficulty \& that it is better not to go on at all than to be always confronting difficult situations. Therefore he became cautious, hesitant, \& a seeker of ways of escape.

We must mention the fact that he was a 1st child. We have already spoken about the great significance of this position. We have shown how the chief problem in the case of a 1st child arises from the fact that he is for years the center of attention, only to be displaced from his glory \& another preferred. In a great many cases where a person is bashful \& afraid to go on we find the reason to be that another person has been preferred. Hence in this case it is not difficult to find out where the trouble lies.

In many cases we need only ask a patient. Are you the 1st, 2nd, or 3rd child? Then we have all we need. We can also use an entirely different method: we can ask for old remembrances, which we shall discuss at some length in the next chapter. This method is worthwhile because these remembrances or 1st pictures are a part of the building up of the early style of life which we have called the prototype. One comes upon an actual part of the prototype when a person tells of his early remembrances. Looking back, everybody remembers certain important things, \& indeed what is fixed in memory is always important. There are schools of psychology which act on the opposite assumption. They believe that what a person has forgotten is the most important point, but there is really no great difference between the 2 ideas. Perhaps a person can tell us his conscious remembrances, but he does not know what they mean. He does not see their connection with his actions. Hence the result is the same, whether we emphasize the hidden or forgotten significance of conscious memories or the importance of forgotten memories.

Little descriptions of old remembrances are highly illuminating. Thus a man might tell you that when he was small, his mother took him \& his younger brother to market. That is enough. We can then discover his style of life. He pictures himself \& a younger brother. Therefore we see it must have been important to him to have had a younger brother. Lead him further \& you may find a situation similar to a certain one in which a man recalled that it began to rain that day. His mother took him in her arms, but when she saw the younger brother she put him down to carry the little one. Thus we can picture his style of life. He always hsa the expectation that another person will be preferred. \& so we can understand why he cannot speak in society for he is always looking around to see if another will not be preferred. The same is true with friendship. He is always thinking that another is more preferred by his friend, \& as a result he can never have a true friend. He is constantly suspicious, looking out for little things that disturb friendship.

We can also see how the tragedy he has experienced has hindered the development of his social interest. He recalls that his mother took the younger brother in her arms \& we see that he feels that this baby took more of his mother's attention than he did. He feels that the younger brother is preferred \& is looking constantly for confirmation of this idea. He is wholly convinced he is right, \& so he is always under strain -- always under the great difficulty of trying to accomplish things when someone else is preferred.

Now the only solution for such a suspicious person is complete isolation, so that he would not have to compete at all with others \& would be, so to speak, the only human being on this earth's crust. Sometimes indeed it appears in fancy to such a child that the whole world has broken down, that he is the only person left \& that hence no one else can be preferred. We see how he taps all the possibilities to save himself. But he does not go along the lines of logic, common sense, or truth -- rather along the lines of suspicion. He lives in a limited world, \& he has a private idea of escape. He has absolutely no connection with others \& no interest in others. But he is not to be blamed for we know that he is not really normal.

It is our task to give such a person the social interest demanded of a well-adjusted human being. How is this to be done? The great difficulty with persons trained in this way is that they are overstrained \& are always looking for a confirmation of their fixed ideas. It thus becomes impossible to change their ideas unless somehow we penetrate into their personality in a manner that will disarm their preconceptions. To accomplish this it is necessary to use a certain art \& a certain tact. \& it is best if the adviser is not closely related or interested in the patient. For if one is directly interested in the case, one will find that one is acting for one's own interest \& not for the interest of the patient. The patient will not fail to notice this \& will become suspicious.

The important thing is to decrease the patient's feeling of inferiority. It cannot be extirpated altogether, \& in fact we do not want to extirpate it because a feeling of inferiority can serve as a useful foundation on which to build. What we have to do is to change the goal. We have seen that his goal has been 1 of escape just because someone else is preferred, \& it is around this complex of ideas that we must work. We must decrease his feeling of inferiority by showing him that he really undervalues himself. We can show him the trouble with his movements \& explain to him his tendency to be over-tense, as if standing before a great abyss or as if living in an enemy country \& always in danger. We can indicate to him how his fear that others may be preferred, is standing in the way of his doing his best work \& making the best spontaneous impression.

If such a person could act as a host in society, making his friends have a good time \& being friendly with them \& thinking of their interests, he would improve tremendously. But in ordinary social life we see that he does not enjoy himself, does not have ideas \& as a result says: ``Stupid persons -- they cannot enjoy me, they cannot interest me.''

The trouble with such persons is that they do not understand the situation because of their private intelligence \& their lack of common sense. As we have said, it is as if they were always confronted by enemies \& were leading the life of a lone wolf. In the human situation such a life is a tragic abnormality.

Let us now look at another specific case -- the case of a man afflicted with melancholia. This is a very common illness, but it can be cured. Such persons are distinguishable very early in life. In fact we notice many children who in their approach to a new situation show signs of suffering from melancholia. This melancholy man of whom we are speaking had about 10 attacks, \& these always occurred when he took a new position. As long as he was in his old position he was nearly normal. But he did not want to go out into society \& he wanted to rule others. Consequently he had no friends \& at 50 he had not married.

Let us look at his childhood in order to study his style of life. He had been very sensitive \& quarrelsome, always ruling his older brothers \& sisters by emphasizing his pains \& weaknesses. When playing on a couch 1 day, he pushed them all off. When his aunt reproached him for this, he said, ``Now my whole life is ruined because you have blamed me!'' \& at that time he was only 4 or 5 years old.

Such was his style of life -- always trying to rule others, always complaining of his weakness \& of how he suffered. This trait led in his later life to melancholy, which in itself is simply an expression of weakness. Every patient with melancholia uses almost the same words: ``My whole life is ruined. I have lost everything.'' Frequently such a person has been pampered \& is so no longer, \& this influences his style of life.

Human beings in their reactions to situations are much like the different species of animals. A hare reacts differently to the same situation from a wolf or a tiger. So it is with human individuals. The experiment was once made of taking 3 different types of boys to a lion's cage in order to see how they would behave on seeing this terrible animal for the 1st time. The 1st boy turned \& said, ``Let's go home.'' The 2nd boy said, ``How nice!'' He wanted to appear courageous but he was trembling when he said it. He was a coward. The 3rd boy said, ``May I spit at him?'' Here then we see 3 different reactions, 3 different ways of experiencing the same situation. We see also that for the most part human beings have a tendency to be afraid.

This timidity, when expressed in a social situation, is 1 of the most frequent causes of maladjustment. There was a man of high-born family who never wanted to exert himself but always wished to be supported. He appeared weak, \& of course he could not find a position. Now when the situation at home changed for the worse, his brothers went after him, saying, ``You are so stupid that you cannot find a position. You do not understand anything.'' So this man began to drink. After some months he was a confirmed drunkard \& was put in an asylum for 2 years. It helped him but it did not benefit him permanently, for he was put back into society without preparation. He could find no work except as a laborer, although he was a scion of this well-known family. Soon he began to have hallucinations. He thought a man appeared to tease him so that he could not work. 1st he could not work because he was a drunkard \& then because he had hallucinations. \& so we see that it is not the right treatment merely to make a drunkard sober; we must find \& correct his style of life.

We discover on investigation that this man was a pampered child, always wanting to be helped. He was not prepared to work alone \& we see the results. We must make all children independent, \& this can be done only if we get them to understand the mistakes in their style of life. This child should have been trained to do something, \& then he would not have had to be ashamed in the presence of his brothers \& sisters.'' -- \cite[pp. 98--116]{Adler_science_living}

%------------------------------------------------------------------------------%

\section{Old Remembrances}
``Having analyzed the significance of an individual's style of life, we turn now to the topic of old remembrances, which are perhaps the most important means for getting at a style of life. By looking back through childhood memories we are able to uncover the prototype -- the core of the style of life -- better than by any other method.

If we want to find out the style of life of a person -- child or adult -- we should, after we have heard a little about his complaints, ask him for old remembrances \& then compare them with the other facts he has given. For the most part the style of life never changes. There is always the same person with the same personality, the same unity. A style of life, as we have shown, is built up through the striving for a particular goal of superiority, \& so we must except every word, act \& feeling to be an organic part of the whole ``action line.'' Now at some points this ``action line'' is more clearly expressed. This happens particularly in old remembrances.

We should not, however, distinguish too sharply between old \& new remembrances, for in new remembrances also the action line is involved. It is easier \& more illuminating to find the action line in the beginning, for then we discover the theme \& are able to understand how the style of life of a person does not really change. In the style of life formed at the age of 4 or 5 we find the connection between remembrances of the past \& actions of the present. \& so after many observations of this kind we can hold fast to the theory that in these old remembrances we can always find a real part of the patient's prototype.

When a patient looks back into his past we can be sure that anything his memory will turn up will be of emotional interest to him, \& thus we will find a clue to his personality. It is not to be denied that the forgotten experiences are also important for the style of life \& for the prototype, but many times it is more difficult to find out the forgotten remembrances, or, as they are called, the unconscious remembrances. Both conscious \& unconscious remembrances have the common quality of running towards the same goal of superiority. They are both a part of the complete prototype. It is well, therefore, to find both the conscious \& unconscious remembrances if possible. Both conscious \& unconscious remembrances are in the end about equally important, \& the individual himself generally understands neither. It is for the outsider to understand \& interpret both of them.

Let us begin with conscious remembrances. Some persons, when they are asked for old remembrances, answer, ``I do not know any.'' We must ask such persons to concentrate \& try to remember. After some effort we will find that they will recall something. But this hesitation may be considered as a sign that they do not want to look far back into their childhood \& we may then come to the conclusion that their childhood has not been pleasant. We have to lead such people. We must give them hints in order to find out what we want. They always remember something in the end.

Some persons claim that they can remember back to their 1st year. This is scarcely possible, \& the truth is probably that these are fancied memories, not true remembrances. But it does not matter whether they are fancies or true since they are parts of one's personality. Some persons insist they are not sure whether they remember a thing or whether their parents have told them about it. This, too, is not really important because even if their parents did tell them they have fixed it in their minds \& therefore it helps to tell us where their interest lies.

As we have explained in the last chapter it is convenient for certain purposes to classify individuals into types. Now old remembrances go according to types \& reveal what is to be expected of the behavior of a particular type. E.g., let us take the case of a person who remembers that he saw a marvelous Christmas tree, filled with lights, presents \& holiday cakes. What is the most interesting thing in this story? \textit{That he saw}. Why does he tell us that he has seen? Because he is always interested in visual things. He has struggled against some difficulties in sight, \&, having been trained, has always been interested \& attentive to seeing. Perhaps this is not the most important element of his style of life, but it is an interesting \& important part. It indicates that if we are to give him an occupation it should be one in which he will use his eyes.

In school the education of children too often disregards this principle of types. We may find a child interested in sight who will not listen because he always wants to be looking at something. In the case of such a child we ought to be patient in trying to educate him to hear. Many children at school are taught only in 1 way because they enjoy with 1 sense. They may be only good at listening or good at seeing. Some always like to be moving \& to be working. we cannot expect the same results for the 3 types of children, especially if the teacher prefers 1 method, as, e.g., the method for listening children. When such a method is used the lookers \& the doers will suffer \& will he hindered in their development.

Consider the case of a young man, 24 years old, who suffered from fainting spells. When asked for his remembrances, he recalled that when he was 4 years old he fainted when he heard an engine whistle. In other words, he was a man \textit{who had heard}, \& was therefore interested in hearing. It is not necessary to explain here how this young man later developed fainting spells, but it is sufficient to note that from his childhood he was very sensitive to sounds. He was very musical, for he could not bear noises, disharmonies or strident tones. We are not surprised, therefore, that he should have been so affected by the sound of a whistle. There are often things in which children or adults are interested because they have suffered through them. The reader will remember the case of the man with asthma mentioned in a previous chapter. He had been bound tightly about his lungs in childhood for some trouble, \& as a result had developed an extraordinary interest in ways to breathe.

One meets person whose whole interest seems to lie in things to eat. Their early remembrances have to do with eating. It seems the most important thing in the world for them -- how to eat, what to eat, \& what not to eat. We will often find that difficulties connected with eating in early life have enhanced the importance of eating for such an individual.

We turn now to a case of remembrance that has to do with movement \& walking. We have seen how many children cannot move very well at the beginning of life because they are weak or suffer from rickets. They become abnormally interested in movement \& always want  to hurry. The case is an illustration of this fact. A man of 50 came to a doctor complaining that whenever he accompanied a person across the street he suffered from a terrible fear that they would both be run over. When alone he was never bothered with this fear, \& in fact was very composed in crossing the street. It was only when another was with him that he wanted to save this person. He would then grasp his companion's arm, push him now right \& now left, \& generally annoy him. We meet with such persons occasionally, though not frequently. Let us analyze the reasons for his stupid actions.

Asked for his old remembrances, he explained that when he was 3 years old he could not move very well \& was suffering from rickets. He was twice run over when crossing a street. \& so, now that he was a man it was important for him to prove that he had overcome this weakness. He wanted to show, so to speak, that he was the only man who could cross a street. He was always looking for an opportunity to prove it whenever he was with a companion. Of course to be able to cross a street safety is not something that most people would take pride in or compete with others. But with such persons as our patient, the desire to move \& to show off about the ability to move can be quite lively.

We turn now to another case -- the case of a boy who was on the way to becoming a criminal. He stole, played ``hookey'' from school, etc. until his parents were in despair. His early remembrances were of how he had always wanted to move around \& to hurry. He was now working with his father \& was sitting still all day. From the nature of the case part of the treatment prescribed was that he be made a salesman -- a traveler for his father's business.

1 of the most significant types of old remembrances is the memory of a death during the period of childhood. When children see a person die suddenly \& abruptly, the effect on their minds is very marked. Sometimes such children become morbid. Sometimes, without becoming morbid, they devote their whole lives to the problem of death \& are always occupied in struggling against illness \& death in some form. We may find many of these children interested in medicine later in life, \& they may become physicians or chemists. Such a goal of course is on the useful side of life. They not only struggle against death but help others to do so. Sometimes, however, the prototype develops a very egotistical point of view. A child who was very much affected by the death of an older sister was asked what he wanted to be. The answer expected was that he would be a physician; instead he replied: ``A grave-digger.'' He was asked why he wanted to follow this occupation, \& he answered, ``Because I want to be the one to bury the others \& not the one buried.'' This goal, we see, is on the useless side of life, for the boy is interested only in himself.

Let us turn now to consider old remembrances of people who were pampered children. The old remembrances mirror the characteristics of this class very clearly. A child of this type often mentions his mother. Now perhaps this is natural but it is a sign that he has had to struggle for a favorable situation. Sometimes the old remembrances seem to be quite innocuous, but they repay analysis. E.g., a man tells you, ``I was sitting in my room \& my mother stood by the cabinet.'' This appears unimportant, but his mentioning his mother is a sign that this has been a matter of interest to him. Sometimes the mother is more hidden \& the study more complicated. We have to guess about the mother. Thus the man in question may tell you, ``I remember I made a trip.'' If you ask who accompanied him, you will discover it was his mother. Or, if children tell us, ``I remember I was in the country at a certain place 1 summer,'' we can presuppose that the father was in the city working \& the mother was with the children. We can ask, ``Who was with you?'' In this way we often see the hidden influence of the mother.

From a study of these remembrances we can see a struggle for preferment. We can see how a child in the course of his development begins to value the pampering his mother gives him. This is important for our understanding because if children or adults tell us about such remembrances, we may be sure that such persons always feels that they are in danger or that another will be preferred to them. We see the tension becoming increased \& more \& more obvious \& we see that their minds are sharply focused on this idea. Such a fact is important: it indicates that in later life such persons will be jealous.

Sometimes persons express interest on 1 point above all others. E.g., a child may say, ``I had to watch my little sister 1 day \& I wanted to protect her very well. I put her at the table but the cover caught \& my little sister fell down.'' This child was only 4 years old. It is of course an early age at which to permit an older child to watch a younger girl. We can see what a tragedy it is in the life of the older child who was doing everything possible to protect the younger one. This particular older girl grew up \& married a kind -- we might almost say, obedient -- husband. But she was always jealous \& critical, always afraid that her husband would prefer another. We can easily understand how the husband tired of her \& turned to the children.

Sometimes tension is more clearly expressed \& people remember that they actually wanted to hurt other members of their family, in fact to kill them. Such persons are people who are interested in their own affairs exclusively. They do not like other people. They feel a certain rivalry towards them. This feeling already exists in the prototype.

We have here the type of person who can never finish anything because he fears someone else will be preferred in friendship \& comradeship, or because he is suspicious of people always trying to surpass him. He can never really becomes a part of society because of the idea that another might outshine him \& be preferred. In every occupation he is extremely tense. This attitude appears specially in connection with love \& marriage.

Even if we cannot completely cure such persons, we can, with a certain art in the study of old remembrances, see that they improve.

1 of the subjects for our methods of treatment was the boy whom we described in another chapter as having gone to market with his mother \& younger brother 1 day. When it started to rain the mother took him up in her arms, but, on noticing the younger brother, she set him down \& took up the younger child. Hence he felt that the younger brother was preferred.

If we can obtain such old remembrances we can predict, as we have said, what will happen later in the life of our patients. However, it must be remembered that old remembrances are not reasons, they are hints. They are signs of what happened \& how development took place. They indicate the movement toward a goal \& what obstacles had to be overcome. They show how a person becomes more interested in 1 side of life than another. We see that he may have what we call a trauma, along the lines of sex, e.g.; i.e., he may be more interested in such matters than in others. We cannot be surprised if, when we ask for old remembrances, we hear some sex experiences. Some persons are interested in sex features more than in others at an early age. It is part of the usual human behavior to be interested in sex but, as I have said before, there are many varieties \& degrees of interest. We often find that in a case where a person tells us about sex remembrances, he later develops in this direction. The resulting life is not harmonious because his 1 side of human life is over-valued. There are persons who insist that everything has a sex bias. On the other hand, there are others who insist that the stomach is the most important organ \& we will find that old remembrances parallel later characteristics in such instances also.

There was a boy whose getting into high school was always a riddle. He wanted to be constantly moving, \& would never settle down to study. He was always thinking about something else, frequenting coffee houses \& visiting at friends' houses -- all when he should have been studying. It was therefore interesting to examine his old remembrances. He said, ``I can remember lying in my cradle \& looking at the wall. I noticed the paper on the wall, with all its flowers, figures, etc.'' This person was prepared only for lying in a cradle, not for taking examinations. He could not concentrate on his studies because he was always thinking of other things \& trying to go after 2 hares at once, which cannot be done. We can see that this man was a pampered child \& could not work alone.

We come now to the hated child. This type is rare \& represents extreme cases. If a child is really hated from the beginning of life, he cannot live. Such a child would perish. Usually children have parents or a nurse who \textit{pampers} them to some extent \& satisfies their desires. We find the hated children among illegitimate, criminal \& not wanted children, \& we often see these children becoming depressed. Frequently we find in their remembrances this feeling of being hated. E.g., there was the case of a man who said, ``I remember I was spanked; my mother scolded me, criticized me until I ran away.'' While running away he came very nearly being drowned.

This man came to a psychologist because he could not leave his home. We see from his old remembrances that he went out once \& met with great danger. This stuck in his memory \& he constantly looked for danger when he went out. He was a bright child but always feared that he might not make 1st place in examinations. So he hesitated \& could not go on. When he at last got to the university he feared that he could not compete in the prescribed way. We see how all this may be traced back to his old remembrances of danger.

Another case which may be taken as an illustration is that of an orphan whose parents died when he was only about a year old. He had rickets, \& being in an asylum, he was not cared for properly. Nobody looked after him, \& in later life it was very difficult for him to make friends or comrades. Looking back to his remembrances we see that he always felt that others were preferred. This feeling played an important part in his development. He always felt hated \& this hindered his approach to all problems. He was excluded from all questions \& situations of life, such as love, marriage, friendship, business -- all these situations which required contact with his fellows -- on account of his feeling of inferiority.

Another interesting case is that of a middle-aged man who was always complaining of sleeplessness. He was 46 or 48 years old, married, \& had children. He was very critical of everybody, \& was always trying to tyrannize, particularly over the members of his family. His actions made everyone feel miserable.

When asked for his old remembrances he explained that he had grown up in a home with quarrelsome parents, who were always fighting \& threatening each other, so that he was afraid of them both. He went to school dirty \& uncared for. 1 day his usual teacher was absent \& a substitute took her place. This substitute woman was interested in her task \& its possibilities. She saw that it was a good \& noble work. She saw possibilities in this ill-kept boy \& went out to encourage him. This was the 1st time in his life he had had any such treatment. From that time one he began to develop, but it was always as if he were pushed from behind. He did not really believe he was able to be superior, \& so he worked all day \& half the night. In this way he grew up trained to use half the night for his work or else not to sleep at all but to spend the time thinking of what he had to do. As a result he grew to think that it was necessary to be awake almost all night in order to accomplish results.

We see later his desire to be superior expressed in his attitude towards his family \& in his behavior towards others. His family being weaker than he, he could appear in the role of a conqueror before them. His wife \& children suffered through this type of behavior, as was inevitable.

Summing up the character of this man as a whole, we may say that he had a goal of superiority \& that it was the goal of a person with a great feeling of inferiority. This we often find among over-strained persons. Their tenseness is a sign of their doubt of their own success, \& their doubt in turn is covered up by a superiority complex which is really a superiority pose. A study of old remembrances reveals the situation in its true light.'' -- \cite[pp. 117--134]{Adler_science_living}

%------------------------------------------------------------------------------%

\section{Attitudes \& Movements}
``In the last chapter we endeavored to describe the manner in which old remembrances \& fancies may be used to illuminate the hidden style of life of an individual. Now the study of old remembrances is only 1 device of a whole class of devices for the study of personality. They all depend on the principle of using isolated parts for an interpretation of the whole. Besides old remembrances we can observe movements \& attitudes. The movements themselves are \textit{expressed} or imbedded in attitudes, \& the attitudes are an expression of that whole attitude to life which constitutes what we call the style of life.

Let us 1st speak about the movements of the body. Everybody knows that we judge a person by his manner of standing, walking, moving, expressing himself, etc. We do not always consciously judge, but there is always a feeling of sympathy or antipathy created by these impressions.

Let us consider attitudes in standing, e.g. We notice promptly whether a child for adult stands upright or whether he is crooked or bent. This is not very difficult. We have to watch specially for exaggerations. A person who stands too straight, in a stretched position, causes us to suspect that he is using too much power to assume this posture. We can suppose that this person feels much less great than he wants to appear. In this little point we can see how he mirrors what we have called the superiority complex. He wants to appear more courageous -- he wants to express himself more as he would be if he were not so tense.

On the other hand we see persons with just the opposite posture -- persons who appear bent \& who are always stooping. Such a posture implies to a certain extent that they are cowards. But it is a rule of our art \& science that we should always be cautious, looking for other points \& never judging solely by 1 consideration. Sometimes we feel that we are almost sure of being correct, but we still want to verify our judgment by other points. We ask, ``Are we right in insisting that persons who stoop are always cowards? What can we expect of them in a difficult situation?''

To look at another point in this connection, we will notice how such a person always tries to rest upon something, to lean on a table or chair e.g. He does not trust his own power but wants to be supported. This reflects the same attitude of mind as when standing crooked, \& so when we find both types of action present our judgment is somewhat confirmed.

We will find that children who want always to be supported have not the same posture as independent children. We can tell the degree of independence by how a child stands, how he approaches other persons. In such cases we need not be in doubt, for we have many possibilities of confirming our conclusion. \& once we have confirmed our conclusion, we can take steps to remedy the situation \& put the child on the right path.

Thus we may experiment with such a child who wants to be supported. Sit his mother on a chair \& then let the child come into the room. We will find that he does not look at any other person but goes directly towards his mother \& leans on the chair or against his mother. This confirms what we expect -- that the child wants to be supported.

It is interesting also to note that child's approach, for it shows the degree of social interest \& adjustment. It expresses the confidence of the child in others. We will find that a person who does not want to approach others \& who always stands far away is also reserved in other respects. We will find that he does not speak enough \& is unusually silent.

We can see how all these things point the same way because every human being is a unity \& reacts as such towards the questions of life. As an illustration let us take the case of a woman who came to a doctor for treatment. The doctor expected that she would take a seat near him, but when she was offered a chair she looked around \& took a seat far away. It could only be concluded that this was a person who wanted to be connected with only 1 person. She said that she was married, \& from this the whole story could be guessed. It could be guessed that she wanted to be connected only with her husband. It could also be guessed that she wanted to be pampered, that she is the sort of person who would demand that her husband be very exact \& always on time in coming home. If she was alone she would suffer great anxiety, \& she would never want to go out of her house alone \& would not enjoy meeting other people. In short from her 1 physical movement we could guess the whole story. But we have also ways of confirming our theory.

She may tell us: ``I am suffering from anxiety.'' Now nobody would understand what this meant unless he knew that anxiety can be used as a weapon to rule another person. If a person or adult suffers from anxiety we can guess that there is another person who supports this child or adult.

There was once a couple who insisted that they were free thinkers. Such people believe that everybody can do what he wants in marriage, so long as each one tells the other what happens. The consequence was that the husband had some love affairs \& told all of them to his wife. She seemed perfectly content. But later on she began to suffer from anxiety. She would not go out alone. Her husband must always go with her. We can see then how this free thinking became modified by anxiety or phobia.

Some persons will always stay near a wall of a house \& lean on it. This is a sign that they are not courageous enough, not independent enough. Let us analyze the prototype of such a timid \& hesitating person. There was a boy who came to school appearing very shy. This is an important sign that he does not want to be connected with others. He had no friends \& was always waiting for school to close. He moved very slowly, \& would go down the stairs close to the wall, look down the street \& rush for his house. He was not a good pupil in school, \& in fact was very poor in his school work since he did not feel happy inside of school walls. He always wanted to go home to his mother, a widow who was weak \& pampered him very much.

In order to understand more about the case the doctor went to talk with his mother. He asked her, ``Does he want to go to bed?'' She said, ``Yes.'' ``Does he cry out at night?'' ``No.'' ``Does he wet the bed?'' ``No.''

The doctor thought that either he had made a mistake or that the boy had made a mistake. Then he concluded that the boy must sleep in bed with his mother. How was this conclusion arrived at? Well, to cry out at night is to demand attention of the mother. If he slept in her bed, this would not be necessary. Similarly to wet the bed is also to demand the mother's attention. The doctor's conclusion was verified: the boy slept in bed with his mother.

If we look carefully we will see that all the little things to which the psychologist pays attention form part of a consistent plan of life. Hence when we can see the goal -- in the child's case, to be always tied up with his mother -- we can conclude a great many things. We can conclude by this means whether a child is feeble-minded or not. A feeble-minded child would not be able to establish such an intelligent plan of life.

Now let us turn to the mental attitudes distinguishable in persons. Some persons are more or less pugnacious. Some on the other hand want to give up the ship. However, we never see a person who really gives up. It is not possible, for it is beyond human nature. The normal being cannot give up. If he seems to do so, it indicates even more of a struggle to carry on than otherwise.

There is a type of child who always wants to give up. He is usually the center of attention in a family. Everybody has to care for him, push him forward \& admonish him. He must be supported in life \& is always a burden to others. This is his goal of superiority -- he expresses his desire to dominate others in this fashion. Such a goal of superiority is of course the result of an inferiority complex, as we have already shown. If he had not been doubtful of his own powers, he would not take this easy way out for attaining success.

There was a boy of 17 who illustrated this trait. He was the oldest in the family. We have already seen how the oldest child usually experiences a tragedy when the coming of another child dethrones him from his place in the center of family affections. This was the case with this boy. He was very depressed \& peevish \& had no occupation. 1 day he tried to commit suicide. Soon after that he came to a doctor \& explained that he had had a dream before his attempt at suicide. He dreamt he had shot his father. We see how such a person -- depressed, lazy \& not moving -- has all the time the possibility of movement present in his mind. We also see how all these children who are indolent in school, \& all these indolent adults who seem incapable of doing anything may be on the brink of danger. Oftentimes this indolence is only on the surface. Then something happens, \& we have an attempt at suicide, or else a neurotic condition or insanity may appear. To ascertain the mental attitude of such persons is sometimes a difficult scientific task.

Shyness in a child is another thing that is full of danger. A shy child must be carefully treated. The shyness must be corrected or it will ruin his whole life. He will always have great difficulties unless his shyness is corrected, for in our culture things are so established that only courageous persons get good results \& the advantages of life. If a person is courageous \& suffers defeat he is not hurt so much, but a shy person makes his escape to the useless side of life as soon as he sees difficulties ahead. Such children will become neurotics or insane in later life.

We see such persons going about with a hangdog air, \& when they are with others they stammer \& will not speak or they will avoid people altogether.

The characteristics that we have been describing are mental attitudes. They are not inborn or inherited, but are simply reactions toward a situation. A given characteristic is the answer that my style of life gives to my apperception of a problem that confronts me. Of course it is not always the logical answer that the philosopher would expect. It is the answer that my childhood experiences \& mistakes have trained me to make.

We can see the functioning of these attitudes as well as the way in which they have been built up in children or in abnormal persons better than we can in the case of normal adults. The prototype stage of the style of life, as we have seen, is much clearer \& simpler than the later style. In fact one may compare the functioning of the prototype to an unripe fruit that will assimilate everything that comes along -- manure, water, food, air. All these things will be taken up in its development. The difference between a prototype \& the style of life is like the difference between an unripe \& a ripe fruit. The unripe fruit stage in human beings is much easier to open up \& examine, but what it reveals is to a large extent valid for the ripe fruit stage.

We can see, e.g., how a child who is a coward at the beginning of life expresses this cowardice in all his attitudes. A world of differences separate the cowardly child from the aggressive, fighting child. The fighting child always has a certain degree of courage which is the natural outgrowth of what we have called common sense. Sometimes, however, a very cowardly child may appear like a hero in a certain situation. This happens whenever he is deliberately trying to attain 1st place. This is clearly illustrated in the case of a boy who did not know how to swim. 1 day he went swimming with other boys who had asked him to join them. The water was very deep, \& the boy, who could not swim, nearly drowned. This of course is not real courage, \& is all on the useless side of life. The boy merely did what he did because he wanted to be admired. He ignored the danger he was in, \& hoped that the others would save him.

The question of courage \& timidity is psychologically closely related to the belief in predestination. The belief in predestination affects our capacity for useful action. There are persons who have such a feeling of superiority that they feel they can accomplish anything. They know everything \& do not want to learn anything. We all know the result of such ideas. Children who feel this way in school usually get poor marks. There are other people who always want to try the most dangerous things: they feel that nothing can happen to them, that they cannot suffer defeat. Very often the result is a bad one.

We find this feeling of predestination among people whenever something terrible has happened in their lives \& they have remained unhurt. E.g., they may have been present in a serious accident \& were not killed. As a result they feel that they are destined for higher purposes. There was once a man who had such a feeling but after going through an experience which resulted differently from his expectation he lost courage \& became depressed \& melancholy. His most important support had fallen away.

When asked for his early remembrances he related a very significant experience. He said he was once about to go to a theater in Vienna, but had to attend to something 1st. When he finally arrived at the theater it had burned down. Everything was over, but he was saved. One can well understand how such a person felt himself destined for higher things. All went well until he suffered defeat in his relations with his wife. Then he broke down.

Much could be said \& written about the significance of the belief in fatalism. It affects whole peoples \& civilizations as well as individuals, but for our part we desire to point out only its connection with the sprints of psychological activity \& the style of life. The belief in predestination is in many ways a cowardly escape from the task of striving \& building up activity along the useful line. For that reason it will prove a false support.

1 of the basic attitudes of mind that affects our relations with our fellow-men is the attitude of envy. Now to be envious is a sign of inferiority. True, we all have a certain amount of envy in our make-up. A small amount does no harm \& is quite common. We must, however, demand that envy be useful. It must result in work, in a going in, \& in a facing of problems. In such cases it is not useless. For that reason we should pardon the bit of envy which is found in all of us.

On the other hand jealousy is a much more difficult \& dangerous mental attitude, because it cannot be made useful. There is no single way in which a jealous person can be useful.

Moreover, we see in jealousy the result of a great \& deep feeling of inferiority. A jealous person is afraid of his inability to hold his or her partner. \& so at the very moment when he wants to influence his partner in some manner, he betrays his weakness by his expressions of jealousy. If we look in the prototype of such a person we shall see a sense of curtailment. In fact whenever we meet with jealous persons it is well to look back into their past \& see whether we have not to do with a dethroned person who expects that he will be dethroned again.

From the general problem of envy \& jealousy we may pass to the consideration of a very peculiar type of envy -- the envy on the part of the female sex of the superior social position of the male sex. We find many women \& girls who want to be boys. This attitude is quite understandable, for if we look at things impartially we can see that in our culture the men are always in the lead; they are always more appreciated, valued \& esteemed than women. Morally this is not right \& ought to be corrected. Now girls see that in the family the men \& boys are much more comfortable \& do not have to bother with little things. They see that they are freer in many ways, \& this superior freedom of the male sex makes them dissatisfied with their own role. They therefore try to act like boys. This imitation of boys may appear in various ways. We see them, e.g., trying to dress like boys, \& in this they are sometimes supported by their parents since boys' clothes are admittedly more comfortable. Now a number of these acts are useful \& need not be discouraged. But there are some useless attitudes, as when a girl wants to be called by a boy's name \& not by the name of a girl. Such girls get very angry if others do not call them by the boy's name which they have chosen. This attitude is very dangerous if it reflects something below the surface \& is not a mere prank. In such a case it may appear later in life as a dissatisfaction with the sex role \& a distaste for marriage -- or, when married, a distaste for the sex role of woman.

One should not find fault with women for wearing short clothes, because it is an advantage. It is also fitting for them to develop like men in many ways, \& to have a job like men. But it is dangerous for them to be dissatisfied with their feminine role \& try to adopt the vices of men.

This dangerous tendency makes its appearance in the adolescent period, for it is then that the prototype becomes poisoned. The immature minds of the girls become jealous of the privileges of the boys. It reacts in the desire to imitate boys. Now this is a superiority complex -- it is an escape from proper development.

As we have said, this can lead to a great disinclination for love \& marriage. This is not to say that girls who have this disinclination do not want to be married, for in our culture not to be married is taken as a sign of defeat. Even the girls who are not interested in marriage want to get married.

One who believes in regulating the basis of the relations of the sexes on the principle of equality should not encourage this ``masculine protest'' of women. The equality of the sexes must be fitted into the natural scheme of things, while the masculine protest is a blind revolt against reality \& is thus a superiority complex. As a matter of fact through this masculine protest all the sex functions can be disturbed \& affected. Many serious symptoms can be produced, \& if we trace them back we shall see that the conditions started in childhood.

Not so frequently as in the case of girls who want to be boys, we also meet with the boy who wants to be like a girl. He wants to imitate not the ordinary girl, but the type of girl who flirts in an exaggerated manner. Such boys use face powder, they wear flowers, \& try to act in the manner of a frivolous girl. This is also a form of superiority complex.

We find in fact that in many such cases the boy had grown up in an environment in which a woman was at the head. Thus the boy grew up to imitate the traits of the mother, not of the father.

There was a boy who came for consultation because o certain sex troubles. He related how he was always with his mother. The father was almost a nonentity in the home. Now his mother had been a dressmaker before she was married \& continued something of her occupation after her marriage. The boy being always near her got to be interested in the things she made. He began to sew \& draw pictures of dresses for women, etc. One can judge how interested he was in his mother from the fact  that at 4 years he had learned to tell time because his mother always went out at 4 \& came back at 5 o'clock. Impelled by his pleasure on seeing her return, he learned to read the clock.

Later in life, when he went to school, he acted like a girl. He took no part in sports or games. The boys made fun of him, \& at times they even kissed him, as they frequently do in such cases. 1 day they had to give a theatrical play, \& as we can imagine this boy had the part of a girl. He acted it so well that many in the audience actually thought he was a girl. 1 man in the audience even fell in love with him. In this way this boy got to see that even if he could not be much appreciated as a man he could be greatly appreciated as a woman. This was the genesis of his later sexual troubles.'' -- \cite[pp. 135--153]{Adler_science_living}

%------------------------------------------------------------------------------%

\section{Dreams \& Their Interpretation}
``For Individual Psychology consciousness \& unconsciousness form a single unity, as we have already explained in a number of contexts. In the last 2 chapters we have been interpreting conscious parts -- remembrances, attitudes, movements -- in terms of the individual whole. We shall now apply the same method of interpretation to our unconscious or semiconscious life -- the life of our dreams. The justification for this method is that our dream life is just as much a part of the whole, as our waking life -- no more \& no less. Followers of other schools of psychology are constantly trying to find new views concerning dreams, but our understanding of dreams has been developed along the same line as our understanding of all the integral parts manifested in the expressions \& movements of the psyche.

Now just as our waking life, we have seen, is determined by the goal of superiority, so we may see that dreams are determined by the individual goal of superiority. A dream is always a part of the style of life \& we always find the prototype involved in it. In fact it is only when you see how the prototype is bound up to a particular dream that you can be sure that you have really understood the dream. Also, if you know a person well, you can pretty nearly guess the character of his dreams.

Take, e.g., our knowledge that mankind as a whole is really cowardly. From this general fact we can presuppose that the largest number of dreams will be dreams of fear, danger, or anxiety. \& so if we know a person \& see that his goal is to escape the solution of life's problems, we can guess that he often dreams that he falls down. Such a dream is like a warning to him: ``Do not go on -- you will be defeated.'' He expresses his view of the future in this way -- by falling. The large majority of men have these dreams of falling.

A specific case is a student on the eve of an examination -- a student whom we know to be a quitter. We can guess what will happen with him. He is worried the whole day, cannot concentrate, \& finally says to himself, ``The time is too short.'' He wants to postpone the examination. His dream will be 1 of falling down. \& this expresses his style of life, for to attain his goal, he must dream in such a way.

Take another student who makes progress in his studies is courageous \& not afraid, \& never uses subterfuges. We can also guess his dreams. Before an examination he will dream that he climbs a high mountain, is enchanted with the view from the mountain top, \& in this way awakes. This is an expression of his current of life, \& we can see how it reflects his goal of accomplishment.

Then there is the person who is limited -- the person who can proceed only up to a certain point. Such a person dreams about limits, \& about being unable to escape persons \& difficulties. He often has dreams of being chased \& hunted.

Before we go on to the next type of dream it may be well to remark that the psychologist is never discouraged if somebody says to him, ``I will not tell you any dreams for I cannot remember them. But I will make up some dreams.'' The psychologist knows that his fancy cannot create anything other than that which his style of life commands. His made-up dreams are just as good as his genuinely remembered dreams, for his imagination \& fancy will also be an expression of his style of life.

Fancy need not literally copy a man's real movements in order to be an expression of his style of life. We find, e.g., the type of person who lives more in fancies than in reality. He is the type that is very cowardly in the daytime but quite courageous in dreams. But we will always find some manifestations which indicate that he does not want to finish his work. Such manifestations will be quite evident even in his courageous dreams.

It is always the purpose of a dream to pave the way towards the goal of superiority -- that is to say, the individual's private goal of superiority. All the symptoms, movements \& dreams of a person are a form of training to enable one to find this dominating goal -- to be goal 1 of being the center of attention, of domineering, or of escape.

The purpose of a dream is neither logically or truthfully expressed. It exists in order to create a certain feeling, mood or emotion, \& it is impossible fully to unravel its obscurities. But in this it differs from waking life \& the movements of waking life only in degree, not in kind. We have seen that the answers of the psyche to life's problems are relative to the individual scheme of life: they do not fit into a pre-established frame of logic, although it is our aim, for purposes of social intercourse, to make them do so more \& more. Now once we give up the absolute point of view for waking life, dream life loses its mystery. It becomes a further expression of the same relativity \& the same mixture of fact \& emotion that we find in waking life.

Historically dreams have always appeared very mysterious to primitive peoples, \& they have generally resorted to the prophetic interpretation. Dreams were regarded as prophecies of events to come. In this there was a half-truth. It is true that a dream is a bridge that connects the problem which confronts the dreamer with his goal of attainment. In this way a dream will often come true, because the dreamer will be training his part during the dream \& will be thus preparing for it to come true.

Another way of saying the same thing is that there is the same interconnectedness revealed in dreams as in our waking life. If a person is keen \& intelligent he can foresee the future whether he analyzes his waking life or his dream life. What he does is to diagnose. E.g. if somebody dreams that an acquaintance has died \& the person does die, this might be no more than what a physician or a close relative could foresee. What a dreamer does is to think in his sleep rather than in waking life.

The prophetic view of dreams, precisely because it contains a certain half-truth, is a superstition. It is generally clung to by persons who believe in other superstitions. Or else it is championed by men who seek importance by giving the impression that they are prophets.

To dispel the prophetic superstition \& the mystery that surrounds dreams we have to explain of course why most people do not understand of their own dreams. The explanation is to be found in the fact that few people know themselves even in waking life. Few persons have the power of reflective self-analysis which permits them to see whither they are headed, \& the analysis of dreams is, as we have said, a more complicated \& obscure affair than the analysis of waking behavior. It is thus no wonder that the analysis of dreams should be beyond the scope of most persons -- \& it is also no wonder that in their ignorance of what is involved they should turn to charlatans.

It will help us to understand the logic of dreams if we compare it, not directly with the movements of normal waking life, but with the type of phenomena which we have described in previous chapters as a manifestation of private intelligence. The reader will remember how we described the attitudes of criminals, problem children \& neurotics -- how they create a certain feeling, temper or mood in order to convince themselves of a given fact. Thus the murderer justifies himself by saying, ``Life has no place for this man; therefore I must kill him.'' By emphasizing in his own mind the view that there is not sufficient place on earth he creates a certain feeling which prepares him for the murder.

Such a person may also reason that so-\&-so has nice trousers \& he has not. He puts such value on this circumstance that he becomes envious. His goal of superiority becomes to have nice trousers, \& so we may find him dreaming a dream which creates a certain emotion which will lead to the accomplishment of that goal. We see this illustrated, in fact, in well-known dreams. There are, e.g., the dreams of Joseph in the Bible. He dreamt that all the others bent before him. Now we can see how this dream fitted in with the whole episode of the coat of many colors -- \& with his banishment by his brothers.

Another well-known dream is that of the Greek poet Simonides, who was invited to go to Asia Minor to lecture. He hesitated \& continually postponed the trip in spite of the fact that the ship was in the harbor waiting for him. His friends tried to make him go, but to no avail. Then he had a dream. He dreamt that a dead man whom he had once found in a forest appeared to him \& said, ``Because you were so pious \& cared for me in the forest, I now warn you not to go to Asia Minor.'' Simonides arose \& said, ``I will not go.'' But he had already been inclined not to go before he ever had the dream. He had simply created a certain feeling or emotion to back up a conclusion that he had already reached, although he did not understand his own dream.

If one understands it is clear that one creates a certain fantasy for purposes of self-deception, which results in a desired feeling or emotion. Frequently this is all that is remembered of the dream.

In considering this dream of Simonides we come to another point. What should be the procedure in interpreting dreams. 1stly, we must bear in mind that a dream is part of a person's creative power. Simonides, dreaming, used his fancy \& built up a sequence. He selected the incident of the dead man. Why should this poet pick the experience of the dead man from out of all his experiences? Obviously because he was very much concerned with ideas of death, due to the fact that he was terrified at the thought of sailing on a ship. In those days a sea voyage presented real danger, \& so he hesitated. It is a sign that he was probably not only afraid of seasickness but also that he feared the ship might sink. As a result of this preoccupation with the thought of death, his dream selected the episode of the dead man.

If we consider dreams in this manner, the task of interpretation does not become too difficult. We should remember that the selection of pictures, remembrances \& fancies is an indication of the direction in which the mind is moving. It shows you the dreamer's tendency, \& eventually we can see the goal at which he wants to arrive.

Let us consider, e.g., the dream of a certain married man. He was not content with his family life. He had 2 children, but was always worried, thinking that his wife did not take care of them \& was too much interested in other things. He was always criticizing his wife about these things \& tried to reform her. 1 night he dreamt that he had a 3rd child. This child got lost \& was not to be found. He reproached his wife because she had not taken care of him.

Here we see his tendency: he had in mind the thought that 1 of his 2 children might get lost, but he was not courageous enough to make it 1 of them in his dream. \& so he invented a 3rd child \& made him get lost.

Another point to be observed is that he liked his children \& did not want them to get lost. Also that he felt that his wife was overburdened with 2 children \& could not care for 3. This 3rd child would perish. Hence we find another aspect of the dream, which, when interpreted, reads: ``Should I have a 3rd child or not?''

The real result of the dream was that he had created an emotion against his wife. No child really got lost, but he got up in the morning criticizing \& feeling antagonistic towards her. Thus people frequently get up in the morning -- argumentative \& critical as a result of an emotion created by the night's dream. It is like a state of intoxication \& not unlike what one finds in melancholia, where the patient intoxicates himself with ideas of defeat, of death \& of all being lost.

We may also see that this man selected things in which he was sure to be superior, as, e.g., the feeling, ``I am careful of the children, but my wife is not \& therefore one got lost.'' Thus his tendency to dominate is revealed in his dream.

The modern interpretation of dreams is about 25 years old. Dreams were 1st regarded by Freud as the fulfillment of infantile sex desires. We cannot agree with this, inasmuch as if dreams are such a fulfillment then everything can be expressed in terms of a fulfillment. Every idea behaves in this way -- going from the depths of the subconscious up into consciousness. The formula of sex-fulfillment thus explains nothing in particular.

Later Freud suggested that the desire for death was involved. But it is certain that this last dream could not be explained very well in this way, for we cannot say that the father wanted the child to get lost \& die.

The truth is that there is no specific formula which will explain dreams, except the general postulates which we have discussed about the unity of psychical life \& about the special affective character of dream life. This affective character, \& its accompaniment of self-deception is a theme with many variations. Thus it is expressed in the preoccupation with comparisons \& metaphors. The use of comparisons is 1 of the best means of deceiving oneself \& others. For we may be sure that if a person uses comparisons he does not feel sure that he can convince you with reality \& logic. He always wants to influence you by means of useless \& far-fetched comparisons.

Even poets deceive, but pleasantly, \& we enjoy being entertained by their metaphors \& poetic comparisons. We may be sure, however, that they are meant to influence us more than we would be influenced by usual words. If Homer, e.g., speaks of an army of Greek soldiers overrunning a field like lions, the metaphor will not deceive us when we think sharply but it will certainly intoxicate us when we are in a poetic mood. The author makes us believe he has marvelous power. He could not do this if he were merely to describe the clothes the soldiers wore \& the arms they carried, etc.

We see the same thing in the case of a person who is in difficulty about explaining things: if he sees he cannot convince you, he will use comparisons. This use of comparisons, as we have said, is self-deceptive, \& this is the reason it is so prominently manifested in dreams in the selection of pictures, images, etc. This is an artistic way of intoxicating oneself.

The fact the dreams are emotionally intoxicating offers, curiously enough, a method for preventing dreams. If a person understands what he has been dreaming about \& realizes that he has been intoxicating himself, he will stop dreaming. To dream will have no more purpose for him. At least this is the case with the present writer, who stopped dreaming as soon as he realized what dreaming meant.

Incidentally it may be said that this realization, to be effective, must have the aspects of a thorough-going emotional conversion. This was brought about, in the case of the writer, by his last dream. The dream occurred during war time. In connection with his duties he was making a great effort to keep a certain man from being sent to the front in a place of danger. In the dream the idea came to him that he had murdered someone, but he did not know whom. He got himself into a bad state wondering, ``Whom have I murdered?'' The fact is he was simply intoxicated with the idea of making the greatest possible effort to put the soldier in the most favorable position for avoiding death. The dream emotion was meant to be conducive to this idea, but when he understood the subterfuge of the dream, he gave up dreaming altogether, since he did not need to deceive himself in order to do the things that for reasons of logic he might want either to do or to leave undone.

What we have said may be taken as an answer to the question that is frequently asked, ``Why do some persons never dream?'' These are persons who do not want to deceive themselves. They are too much tied up with movement \& logic, \& want to face problems. Persons of this sort, if they dream, often forget their dreams very soon. They forget so quickly that they believe they have not dreamed.

This brings up the theory that we always dream \& that we forget most of our dreams. If we accepted such a theory it would put a different construction on the fact that some persons never dream: they would then become persons who dream but who always forget their dreams. The present writer does not accept this theory. He rather believes that there are persons who never dream \& that there are also dreamers who sometimes forget their dreams. In the nature of the case such a theory is hard to refute, but perhaps the burden of proof should be put on the propounders of the theory.

Why do we have the same dream repeatedly? This is a curious fact for which no definite explanation can be given. However, in such repeated dreams we are able to find the style of life expressed with much more clarity. Such a repeated dream gives us a definite \& unmistakable indication where the individual goal of superiority lies.

In the case of long \& extended dreams we must believe that the dreamer is not fully ready. He is looking for the bridge from the problem to the attainment of the goal. For this reason the dreams which can be best understood are short dreams. Sometimes a dream consists of only 1 picture, a few words, \& it shows how the dreamer is really trying to find a short way to deceive himself.

We may close our discussion with the question of sleep. A great many persons put to themselves needless questions about sleep. They imagine that sleep is the contradiction of being awake, \& that it is the ``brother of death.'' But such views are erroneous. Sleep is not a contradiction of being awake, but is rather a degree of being awake. We are not separated from life in sleep. On the contrary we are thinking, \& hearing in sleep. The same tendencies are generally expressed in sleep as in waking life. Thus there are mothers who cannot be awakened by any of the street noises, but if the children move in the least bit they immediately jump up. We see how their interest is really awake. Also from the fact that we do not fall out of bed we can see that we realize limits in sleep.

The whole personality is expressed by night \& by day. This explains the phenomena of hypnotism. What superstition has made to appear as a magic power is for the most nothing more than a variety of sleep. But it is a variety in which 1 person wants to obey another \& knows that the 2nd person wants to make him sleep. A simple form of the same thing is when parents say, ``It is enough -- now sleep!'' \& the children obey. In hypnotism, too, the results take place because the person is obedient. \& in proportion to his obedience is the ease with which he may become hypnotized.

In hypnotism we have an opportunity of making a person create pictures, ideas, remembrances which he would not do with his waking inhibitions. The only requirement is obedience. By this method we can find some solutions -- some old remembrances -- which may have been forgotten before.

As a method of treatment \& cure, hypnotism has its dangers, however. The present writer does not like hypnotism \& uses it only when a patient trusts no other method. One will find that hypnotized persons are rather revengeful. In the beginning they overcome their difficulties, but they do not really change their style of life. It is like a drug or a mechanical means: the person's true nature has not been touched. What we have to do is to give a person courage, self-confidence \& better understanding of his mistakes, if we are really to help him. Hypnotism does not do this, \& should not be used except in rare cases.'' -- \cite[pp. 154--172]{Adler_science_living}

%------------------------------------------------------------------------------%

\section{Problem Children \& Their Education}
``How shall we educate our children? This is perhaps the most important question in our present social life. It is a question to which Individual Psychology has a great deal to contribute. Education, whether carried on in the home or at school, in an attempt to bring out \& direct the personalities of individuals. Psychological science is thus a necessary basis for the proper educational technique, or if we will, we may look upon all education as a branch of that vast psychological art of living.

Let us begin with certain preliminaries. The most general principle of education is that it must be consistent with the later life which the individuals will be called upon to face. This means that it must be consistent with the ideas of the nation. If we do not educate children with the ideals of the nation in view, then these children are likely to encounter difficulties later in life. They will not fit in as members of society.

To be sure the ideals of a nation may change -- they may change suddenly, as after a revolution, or gradually, in the process of evolution. But this simply means that the educator should keep in mind a very broad ideal. It should be an ideal which will always have its place, \& which will teach the individual to adjust himself properly to changing circumstances.

The connection of schools with social ideals is of course due to their connection with the government. It is the influence of the government which causes national ideas to be reflected in the school system. The government does not readily reach the parents or the family, but it watches the schools in its own behalf.

Historically, the schools have reflected different ideas at different periods. In Europe schools were originally established for aristocratic families. The schools were aristocratic in spirit, \& only aristocrats were taught in them. Later on, the schools were taken over by the churches, \& they appeared as religious schools. Only priests were teachers. Then the demands of the nation for more knowledge began to increase. More subjects were sought \& a greater number of teachers was needed than the church could supply. In this way others besides priests \& clergymen entered the profession.

Until quite modern times the teachers were never exclusively teachers. They followed many other trades, such as shoemaking, tailoring, etc. It is obvious that they knew how to teach only by using the rod. Their schools were not the sort in which the psychological problems of the children could be solved.

The beginning of the modern spirit in education was made in Europe in Pestalozzi's time. Pestalozzi was the 1st teacher to find other teaching methods besides the rod \& punishment.

Pestalozzi is valuable for us because he showed the great importance of methods in the schools. With correct methods, every child -- unless he is feeble-minded -- can learn to read, to write, to sing, \& to do arithmetic. We cannot say that we have already discovered the best methods; they are in the process of development all the time. As is right \& proper, we are always searching for new \& better methods.

To return to the history of European schools, it is to be noted that just after pedagogical technique had developed to some extent, there appeared a great need for workmen who could read, write, count, \& be generally independent without needing constant guidance. At this time there appeared the slogan, ``a school for every child.'' At present every child is forced to go to school. This development is due to the conditions of our economic life \& to the ideals which reflect these conditions.

Formerly in Europe only aristocrats were influential, \& there was a demand only for officials \& for laborers. Those who had to be prepared for higher stations went to higher schools; the rest did not go to school at all. The educational system reflected the national ideals of the time. Today the school system corresponds to a different set of national ideals. We no longer have schools in which children must sit quietly, hands folded in their laps, \& not allowed to move. We now have schools in which the children are the teacher's friends. They are no longer compelled by authority, no longer compelled merely to obey, but are allowed to develop more independently. Naturally there are many such schools in democratic United States, since the schools always develop with the ideals of a country as crystallized in government regulations.

The connection of the school system with national \& social ideals is organic -- due to their origin \& organization, as we have seen -- but from a psychological point of view it gives them a great advantage as an educational agency. From a psychological point of view the principal aim of education is social adjustment. Now the school can guide the current of sociability in the individual child more easily than the family because it is much nearer to the demands of the nation \& more independent of the criticism of the children. It does not pamper the children, \& in general it has a much more detached attitude.

On the other hand the family is not always permeated with the social idea. Too often we find traditional ideas dominating there. Only when the parents are themselves socially adjusted \& understand that the aim of education must be social, can progress be made. Wherever parents know \& understand these things we will find children rightly educated \& prepared for school, just as in school they are rightly prepared for their special place in life. This should be the ideal development of the child at home \& in school, with the school standing midway between the family \& the nation.

We have gathered from previous discussions that the style of life of a child in a family is fixed after it is 4 or 5 years old \& cannot directly be changed. This indicates the way in which the modern school has to go. It must not criticize or punish, but try to mould, educate \& develop the social interest of children. The modern school cannot work on the principle of suppression \& censorship, but rather on the idea of trying to understand \& solve the personal problems of the child.

On the other hand, parents \& children being so closely united in the family, it is often difficult for the former to educate the latter for society. They prefer to educate the children for their own sakes, \& thereby they create a tendency which will conflict with the situation of the child in later life. Such children are bound to face great difficulties. They are already confronted with them the moment they enter school, \& the problems become still more difficult in life after school.

To remedy this situation it is of course necessary to educate the parents. Often this is not easy, for we cannot always lay our hands on the elders as we do on the children. \& even when we get to the parents, we may find that they are not very much interested in the ideals of the nation. They are so set in tradition they they do not want to understand.

Not being able to do much with the parents, we simply have to content ourselves with spreading more understanding everywhere. The best point of attack is our schools. This is true 1st because the large numbers of children are gathered there; 2ndly, because mistakes in the style of life appear better there than in the family; \&, 3rdly, because the teacher is supposedly a person who understands the problems of children.

Normal children, if there be such, do not concern us. We would not touch them. If we see children who are fully developed \& socially adjusted, the best thing is not to suppress them. They should go their own way, because such children can be depended upon to look for a goal on the useful side in order to develop the sense of superiority. Their superiority feeling, precisely because it is on the useful side, is not a superiority complex.

On the other hand both the feeling of superiority \& the feeling of inferiority exist on the useless side among problem children, neurotics, criminals, etc. Such persons express a superiority complex as a compensation for their inferiority complex. The feeling of inferiority, as we have shown, exists in every human being, but this feeling becomes a complex only when it discourages him to the point of stimulating training on the useless side of life.

All these problems of inferiority \& superiority have their root in family life during the period before the child enters school. It is during this period that he has built up his style of life, which in contrast with the adult style of life we have designated as a prototype. This prototype is the unripe fruit, \& like an unripe fruit, if there is some trouble with it, if there is a worm, the more it develops \& ripens the larger the worm grows.

As we have seen, the worm or difficulty develops from problems over imperfect organs. It is the difficulty with imperfect organs that is the usual root of the feeling of inferiority, \& here again we must remember that it is not the organic inferiority that causes the problem but the social maladjustments which it brings in its wake. It is this that provides the educational opportunity. Train a person to adjust himself socially \& the organic inferiorities, so far from being liabilities, may become assets. For as we have seen, an organic inferiority may be the origin of a very striking interest, developed through training, which may rule the individual's whole life, \& provided this interest runs in a useful channel, it may mean a great deal to the individual.

It all depends on the way the organic difficulty fits in with the social adjustment. Thus in the case of a child who wants only to see, or only to hear, it is up to the teacher to develop his interest in the use of all his sense organs. Otherwise he will be out of line with the rest of the pupils.

We are all familiar with the case of the left-handed child who grows up clumsy. As a rule no one realizes that this child is left-handed \& that this accounts for his clumsiness. Because of his left-handedness he is constantly at odds with the family. We find that such children either become fighting or aggressive children -- which is advantage -- or else they become depressed \& peevish. When such a child goes to school with his problems, we shall find him either combative, or else downhearted, irritable \& lacking in courage.

Besides the children with imperfect organs, a problem is presented by the great number of pampered children who come to school. Now the way schools are organized, it is physically impossible for a single child always to remain the center of attention. It may indeed happen occasionally that a teacher is so kind \& soft-hearted that she plays favorites, but as the child moves from grade to grade it falls out of its position of favor. Later in life it is even worse, for it is not considered proper in our civilization for 1 person always to be the center of attention, without doing anything to merit it.

All such problem children have certain defined characteristics. They are not well fitted for the problems of life; they are very ambitious, \& want to rule personally, not in behalf of society. In addition they are always quarrelsome \& at enmity with others. They are usually cowards, since they lack interest in all the problems of life. A pampered childhood has not prepared them for life's problems.

Other characteristics which we discover among such children is that they are cautious \& continually hesitating. They postpone the solution of the problems that life presents to them. Or else they come to a stop altogether before problems, going off on distractions \& never finishing anything.

These characteristics come to light more clearly in school than in the family. School is like an experiment or acid test, for there it becomes apparent whether or not a child is adjusted to society \& its problems. A mistaken style of life often escapes unrecognized at home, but it comes out in school.

Both the pampered-child \& the organ-inferiority type of children always want to ``exclude'' the difficulties of life because of their great feeling of inferiority which robs them of strength to cope with them. However, we may control the difficulties at school, \& thus gradually put them in a position to solve problems. The school thus becomes a place where we really educate, \& not merely give instruction.

Besides these 2 types, we have to consider the hated child. The hated child is usually ugly, mistaken, crippled, \& in no way prepared for social life. He has, perhaps, the greatest difficulty of all 3 types upon entering school.

We see, then, that whether or not teachers \& officials like it, an understanding of all these problems \& of the best methods for handling them must be developed as part of the school administration.

Besides these specifically problem children, there are also the children who are believed to be prodigies -- the exceptionally bright children. Sometimes because they are ahead in some subjects it is easy for them to appear brilliant in others. They are sensitive, ambitious, \& not usually very well liked by their comrades. Children immediately seem to feel whether 1 of their number is socially adjusted or not. Such prodigies are admired but not beloved.

We can understand how many of these prodigies pass through school satisfactorily. But when they enter social life they have no adequate plan of life. When they approach the 3 great problems of life -- society, occupation, \& love \& marriage -- their difficulties come out. What happened in their prototype years becomes apparent, \& we see the effect of their not being well adjusted in the family. There they continually found themselves in favorable situations, which did not bring out the mistakes in their style of life. But the moment that a new situation comes their way, the mistakes appear.

It is interesting to note that poets have seen the connection between these things. A great many poets \& dramatists have described, in their dramas \& romances, they very complicated current of life seen in such persons. There is e.g., Shakespeare's character, Northumberland. Shakespeare, who was a master of psychology, portrays Northumberland as quite loyal to his king until real danger came. Then he betrayed him. Shakespeare understood the fact that the true style of life of a person becomes apparent under very difficult circumstances. But it is not the difficult circumstances that produce the style -- it has been built up before.

The solution that Individual Psychology offers for the problems of prodigies is the same as that for other problem children. The individual psychologist says, ``Everybody can accomplish everything.'' This is a democratic maxim which takes the edge off prodigies, who are always burdened with expectations, are always pushed forward \& become too much interested in their own persons. Persons who adopt this maxim can have very brilliant children, \& these children do not have to become conceited or too ambitious. They understand that what they accomplished was the result of training \& good fortune. If their good training is continued they can accomplish whatever others can accomplish. But other children, who are less favorably influenced \& not as well trained \& educated, may also accomplish good things if their teacher can make them understand the method.

These later children may have lost courage. They must therefore be protected against their marked feeling of inferiority, a feeling that none of us can suffer for long. Originally such children were not confronted with as many difficulties as they now meet at school. One can understand their being overwhelmed by these difficulties \& wanting to play truant or else not go to school at all. They believe that there is no hope for them at school, \& if this belief were true we should have to agree that they are acting consistently \& rationally. But Individual Psychology does not accept the belief that their case is hopeless at school. It believes that everybody can accomplish useful works. There are always mistakes, but these can be corrected \& the child can go on.

In the usual circumstances, however, the situation is not handled properly. At the every time when the child is overwhelmed by the new difficulties at school, the mother takes on a watching \& anxious attitude. The school reports, the criticisms \& scoldings that the child gets at school are magnified by the repercussions at home. Very often a child who has been a good child at home, because he has been pampered, becomes very bad in school because his latent inferiority complex shows up the moment he loses contact with the family. It is then that the pampering mother will be hated by such a child because he feels that she has deceived him. She does not appear in the same light as she did before. All her old behavior \& pampering is forgotten in the anxiety of the new situation.

We find very often that a child, who is a fighting child at home, is quiet, calm, \& even suppressed at school. Sometimes the mother comes to schools \& says, ``This child occupies me the entire day. He is always fighting.'' The teacher says, ``He sits quietly all day \& does not move.'' \& sometimes we have the reverse. I.e., the mother comes \& says, ``This child is very quiet \& sweet at home,'' while the teacher says, ``He corrupts my whole class.'' We can easily understand the last situation. The child is the center of attention at home \& for that reason is quiet \& unassuming. In school he is not the center of attention, \& so he fights. Or it may be the other way around.

There is the case, e.g., of a girl 8 years old, who was very well liked by her schoolmates \& was head of her class. Her father came to the doctor saying ``This child is very sadistic -- a veritable tyrant. We can no longer bear her.'' What was the reason? She was a 1st child in a weak family. Only a weak family could be so tortured by a child. When another child was born this girl felt herself in danger, \& still wanting to be the center of attention as before, she began to fight. At school she was quite appreciated, \& not having any reason to right she developed well.

Some children have difficulty both at home \& in school. Both family \& school complain, \& the result is that the children's mistakes increase. Some are untidy at home \& in school. Now if the behavior is the same both within the family \& at school, we must look for the cause in things that have gone before. In any case we must always consider both the actions in the family \& in school in order to form a judgment on a child's problems. Every part becomes important for us if we are correctly to understand his style of life \& the direction in which he is trying.

It sometimes happens that a fairly well-adjusted child, when he encounters the new situations in school, when he encounters the new situations in school, may not seemed adjusted. This usually happens when a child comes to a school where the teacher \& the pupils are very much against him. To take an example from European experience, a child not an aristocrat, comes to an aristocratic school, being sent there because his parents are very rich \& conceited. Since he is not of an aristocratic family, his comrades are all against him. Here is a child, previously pampered or at least comfortably adjusted, who suddenly finds himself in a very hostile atmosphere. Sometimes the cruelty of such comrades can reach such a point that it is really astonishing for a child to be able to stand it. In most cases the child never speaks a word about it at home because he feels ashamed. He suffers his terrible ordeal in silence.

Often such children when they come to the age of 16 or 18 years -- the age when they have to behave towards society like adults \& face life's problems squarely -- stop short because they have lost courage \& hope. \& along with their social handicaps goes their handicap in love \& marriage because they cannot go on.

What are we to do with such cases? They have no outlet for their energies. They are separated, or feel separated from the whole world. The type of person who wants to hurt himself for the sake of hurting others may commit suicide. On the other hand there is the type who wants to disappear. He disappears in an asylum. He loses even the few social abilities he had before. He does not speak in the common way, does not approach people, \& is always antagonistic towards the whole world. This state we call dementia praecox, insanity. If we are to help any of these we must find a way to rebuild their courage. They are very difficult cases, but they can be cured.

Inasmuch as the treatment \& cure of children's educational problems depend primarily upon the diagnosis of their style of life, it is well to review here the methods that Individual Psychology has developed for this diagnosis. The diagnosis of the style of life is of course useful for many other things besides education, but it is quite essential in education practice.

Besides direct study of a child during his formative years, Individual Psychology uses the methods of asking for old remembrances \& fancies concerning future occupations, the observation of posture \& bodily movements, \& certain inferences from the order of the child in the family. We have discussed all these methods before, but it is perhaps necessary to emphasize again the position of the child in the family, as this is more closely connected with educational development than the other methods.

The important thing about the order of children in the family is, as we have seen, that a 1st child is for a while in a position of an only child \& is later dethroned from that position. He thus enjoys great power for a while, only to lose it. On the other hand the psychology of the other children is fixed \& determined by the fact that they are not 1st children.

Among oldest children we often find a conservative view prevailing. They have the feeling that those in power should remain in power. It is only an accident that they have lost their power \& they have great admiration for it.

The 2nd child is in an entirely different situation. He goes along, not as the center of attention, but with a pace-maker running before him. He always wants to equal him. He does not recognize power, but wants power to change hands. He feels a forward urge as in a race. All his movements show that he is looking at a point ahead in order to catch up to it. He is always trying to change the laws of science \& nature. He is really revolutionary -- not so much in politics, but in social life \& in his attitude toward his fellows. We have a good example in the biblical story of Jacob \& Esau.

In a case where there are several children who are nearly grown up before another is born, the latest child finds himself in a situation similar to that of a 1st child.

The position of the youngest in the family is of remarkable interest from a psychological viewpoint. By youngest we mean of course the child that is always the youngest \& never has any successors. Such a child is in an advantageous position since he can never be dethroned. The 2nd child may be dethroned, \& sometimes he experiences the tragedy of the 1st child, but this can never happen in the life of the youngest child. He is therefore the most favorably situated, \& other circumstances being equal, we find that the youngest child gets the best development. He resembles the 2nd child in that he is very energetic \& tries to overcome others. He, too, has pace-makers to outdistance. But in general he takes an entirely different way from the rest of the family. If the family be 1 of scientists, the youngest will probably be a musician or a merchant. If the family be 1 of merchants, the youngest may be a poet. He must always be different. For it is easier not to have to compete in the same field but to work in another one, \& for that reason he likes to follow a different line from the rest. Obviously, this is a sign that he is somewhat lacking in courage, for where such a child courageous, he would compete in the same field.

It is worthy of note that our predictions based on the position of children are expressed in the form of tendencies; there is no necessity about them. \& in fact if a 1st child is bright, he may not at all be conquered by the 2nd, \& thus will not suffer any tragedy. Such a child is socially well-adjusted, \& his mother is likely to have spread his interest toward others, including the newborn baby. On the other hand if this 1st child cannot really be conquered, then it is a greater difficulty for the 2nd, \& this 2nd child may become a problem. Such 2nd children result in the worst types, because they often lose courage \& hope. We know that children in a race must always have the hope of winning; \& when this hope is gone, all is lost.

The only child also has his tragedy, for he has been the center of attention in the family throughout his childhood, \& his goal in life is always to be the center. He does not reason along the lines of logic, but along the lines of his own style of life.

The position of an only boy among a family of girls is also difficult \& presents a problem. It is commonly supposed that such a boy behaves in a girlish manner, but this view is rather exaggerated. After all, we are all educated by women. However, there is a certain amount of difficulty, inasmuch as the whole family in such a case is established for women. One can immediately tell upon entering a house, whether the family has more boys or girls. The furniture is different, there is more or less noise, \& the order is different. There are more broken things where there are more boys, \& everything is much cleaner where there are more girls in the family.

A boy in such an environment may strive to appear more of a man \& exaggerate this feature of his character; or else he may indeed grow girlish like the rest of the household. In short we will find that such a boy is either soft \& mild or else very wild. In the latter eventually it would seem that he is always trying to prove \& emphasize the fact that he is a man.

The only girl among boys is in an equally difficult situation. Either she is very quiet \& develops very femininely, or else she wants to do everything that the boys do \& to develop like them. A feeling of inferiority is quite apparent in such a case, since she is the only girl in a situation where boys are superior. The inferiority complex lies in the feeling that she is \textit{only} a girl. In this word ``only'' the whole inferiority complex is expressed. We see the development of a compensating superiority complex when she tries to dress like the boys \& when later in life she wants to have the sexual relations that she understands men have.

We may conclude our discussion of the position of a child in a family with the peculiar case where the 1st child is a boy \& the 2nd a girl. Here there is always a fierce competition between the two. The girl is pushed forward not only because she is the 2nd child but also because she is a girl. She trains more, \& thus becomes a very marked type of 2nd child. She is very energetic \& very independent, \& the boy notices how she always approaches nearer \& nearer to him in the race. As we know it is a fact that girls develop more rapidly physically \& mentally than boys -- a girl of 12, e.g., is much more developed than a boy of the same age. The boy sees this \& cannot explain it. Hence he feels inferior \& has a longing to give up. He does not progress any more. Instead he starts looking for escapes. Sometimes he develops ways of escape in the direction of art. At other times he becomes neurotic, criminal, or insane. He does not feel strong enough to go on with the race.

This type of situation is a difficult one to solve even with the viewpoint that ``Everybody can accomplish everything.'' The main thing we can do is to show the boy that if the girl seems to be ahead it is only because she practices more \& by practicing finds better methods for development. We can also seek to direct the girl \& the boy into non-competitive fields, as far as possible, so as to diminish the atmosphere of running a race.'' -- \cite[pp. 173--198]{Adler_science_living}

%------------------------------------------------------------------------------%

\section{Social Problems \& Social Adjustment}
``The goal of Individual Psychology is \textit{social} adjustment. This may seem a paradox, but if it is a paradox, it is so only verbally. The fact is that it is only when we pay attention to the concrete psychological life of the individual do we come to realize how all-important is the social element. The individual becomes an individual only in a social context. Other systems of psychology make a distinction between what they call individual psychology \& social psychology, but for us there is no such distinction. Our discussion hitherto have attempted to analyze the individual style of life, but the analysis has always been with a social point of view \& for a social application.

We now continue our analysis with more emphasis on the problems of social adjustment. The realities to be discussed are the same, but instead of concentrating our attention on diagnosing styles of life, we shall discuss the styles of life in action \& the methods for furthering proper action.

The analysis of social problems continues directly on our analysis of the problems of educational upbringing, which was the theme of our last chapter. The school \& nursery are miniature social institutions, \& we can study there the problems of social maladjustment in a simplified form.

Take the behavior problems of a boy of 5. A mother came to the doctor complaining that her boy was restless, hyperactive, \& very troublesome. She was always occupied with him \& at the end of the day was exhausted. She said she could not stand the boy any more \& was willing to have him removed from the house if such a treatment was advisable.

From these behavior details we can readily ``identify'' with the boy -- we can readily put ourselves in his place. If we hear that a child of 5 is hyperactive, we can easily imagine what his line of conduct would be. What would anyone do if he were that age \& hyperactive? He would climb on the table with his heavy shoes. He would always like to go about dirty. \& if the mother wanted to read, he would play with the lights \& turn them on \& off. Or again if the mother \& father wanted to play the piano or wanted to sing together -- what would such a boy do! He would yell. Or else hold his ears \& insist that he did not like such a noise. He would always have temper tantrums if he did not get what he wanted -- \& he would always want something.

If we note such behavior in the nursery school, we may be sure that such a boy wants to fight \& that everything he does is done in order to induce a fight. He is restless day \& night, while his father \& mother are always tired. The boy is never tired because unlike his parents he does not have to do what he does not want. He simply wants to be restless \& occupy the others.

A particular incident well illustrates how his boy fought for the center of attention. 1 day he was taken to a concert at which his mother \& father played \& sang. In the middle of a song the boy called out, ``Hello daddy!'' \& walked all around the hall. One could have predicted this, but the mother \& father did not understand \& the reason for such behavior. They regarded him as a normal child, in spite of the fact that he did not behave normally.

To this extent he was, however, normal: he had an intelligent plan of life. What he did was rightly done, in accordance with his plan. \& if we see the plan we can guess the actions that result. Hence we may conclude that he is not feeble-minded, for a feeble-minded person never has an intelligent plan of life.

When his mother had visitors \& wanted to enjoy the party, he would push the visitors off the chairs \& always wanted the particular chair upon which one was about to sit. We see how this, too, is consistent with his goal \& with his prototype. His goal is to be superior \& to rule others, \& always to occupy the attention of his father \& mother.

We can judge that he used to be a pampered child, \& that were he to be pampered again he would not fight. In other words, it is a child who has lost his favorable situation.

How did he lose his favorable situation? The answer is, he must have acquired a younger brother or sister. He is thus a 5-year-old in a new situation, feeling dethroned \& fighting to hold his important central position which he believes to have lost. For that reason he keeps his father \& mother always occupied with him. Also there is another reason. One can see that the boy has not been prepared for the new situation \& that in his position of pampered child he never developed any communal feeling. He is thus not socially adjusted. He is interested only in himself \& occupied only with his own welfare.

When his mother was asked how the boy behaved towards the younger brother, she insisted that he liked him, but that whenever he played with him he always knocked him down. We might be pardoned for presuming that such actions do not indicate marked affection.

To understand fully the significance of this behavior we should compare it to the cases we frequently meet of fighting children who do not fight continuously. The children are too intelligent to fight continuously, for they know that the father \& mother would put an end to their fighting. Hence such children from time to time stop their fighting \& go on their good behavior. But the old movement reappears, as it does in this case when, in the course of his playing with the younger brother, he knocks him down. His goal in playing is in fact to knock him down.

Now what is the boy's behavior towards his mother? If she tries to spank him, he either laughs \& insists that the spanking does not hurt him; or else, if she beats him a little harder, he becomes quiet for a while, only to begin his fighting a little later. One should notice how all the boy's behavior is conditioned by his goal \& how everything he does is rightly directed towards it -- so much so that we can predict his actions. We could not predict them if the prototype were not a unity, or if we did not know the goal of the prototype's movement.

Imagine this boy starting out in life. He goes to the nursery school, \& we can predict what will happen there. We could have predicted what would happen if the boy were to be taken to a concert, as he actually was. In general he will rule in a weak environment, or, in a more difficult one, he will fight to rule. \& so his stay at the nursery school is likely to be shortened if the teacher is severe. In that case the boy might try to find subterfuges. He would be in a constant tension, \& this tension might make him suffer from headache, restlessness, etc. The symptoms would appear as the 1st indications of a neurosis.

On the other hand if the environment were soft \& pleasant, he might feel that he was the center of attention. Under such circumstances he might even become the leader of the school -- the complete champion.

The nursery school, as we can see, is a social institution with social problems. An individual must be prepared for such problems because he has to follow the laws of the community. The child must be able to make himself useful to that little community, \& he cannot be useful unless he is more interested in others than in himself.

In public school the same situation is repeated, \& we can imagine what would happen to a boy of this sort. Things might be a little easier in a private school, since in such a school there are generally fewer pupils, to whom more attention can be paid. Perhaps in such an environment no one could notice that he was a problem child. Perhaps they might even say instead, ``This is our most brilliant boy, our best pupil.'' It is also possible that if he were the head of the class, his behavior at home might change. He might be satisfied to be superior in 1 way only.

In cases where a child's behavior improves after he goes to school, one may take it for granted that he has a favorable situation in his class \& feels superior there. Usually, however, the contrary is true. Children who are very much loved \& very obedient at home often corrupt the class at school.

We have spoken in the last chapter of the school as standing midway between the home \& life in society. If we apply that formula we can understand what happens to a boy of our type when he goes out into life. Life will not offer him the favorable situation which he may sometimes find in school. People are often surprised \& cannot understand how children who are brilliant at home \& brilliant in school should turn out in later life worthless. We have here problem adults with a neurosis which may later turn into insanity. No one understands such cases because the prototype has been covered over by favorable situations until the age of adult life.

On this account we must learn to understand the mistaken prototype in the favorable situation, or at least to realize that it may exist, since it is very difficult to recognize it there. There are a few signs which may be taken as definite indications of a mistaken prototype. A child who wants to attract attention \& who is lacking in social interest will often be untidy. By being untidy he occupies other people's time. He will also not want to go to sleep, \& will cry out at night or wet the bed. He plays at anxiety because he has noticed that anxiety is a weapon by which he may force others to obey. All these signs appear in favorable situations, \& by looking for them one is likely to reach a correct conclusion.

Let us look at this boy with the mistaken prototype later in life, when he is on the verge of maturity -- say at 17 or 18. There is a great \textit{hinterland} of life behind him -- a hinterland which is not so easily evaluated because it is not very distinct. It is not easy to see the goal \& the style of life. But as he faces life he has to meet what we have called the 3 great questions of life -- the social question, the question of occupation, \& that of love \& marriage. These questions arise out of the relationships bound up in our very existence. The social question involves our behavior towards other people, our attitude to mankind \& the future of mankind. This question involves the preservation \& the salvation of man. For human life is so limited that we can carry on only if we pull together.

As regards occupation, we can judge from what we have seen of the boy's behavior at school. We can be sure that if the boy seeks an occupation with the idea of being superior, he will have difficulty in obtaining such a position. It is difficult to find a position where one will not be subordinate, or where one will not have to work with others. But as this boy is interested only in his own welfare, he will never get along well in a subordinate position. Moreover, such a person does not prove very trustworthy in business. He can never subordinate his own interest to the interest of the firm.

In general we may say that success in an occupation is dependent on social adjustment. It is a great advantage in business to be able to understand the needs of neighbors \& customers, to see with their eyes, hear with their ears, \& feel as they feel. Such persons will move ahead, but this boy that we are studying cannot do so, for he is always looking out for his own interests. He can develop only a part of what is necessary for progress. Hence he will often be a failure in his occupation.

In most cases one will find that such persons never finish their preparation for an occupation, or at least, are late in taking up an occupation. They are perhaps 30 years old, \& do not know what they intend to do in life. They frequently change from 1 course of study to another, or from 1 type of position to another. This is a sign that they cannot be suited in any way.

Sometimes we find a youth of 17 or 18 who is striving, but does not know what to do. It is important to be able to understand such a person \& to advise him regarding the choice of a vocation. He can still get interested in something from the beginning \& train properly.

On the other hand it is rather disturbing to find a boy of this age not knowing what he wants to do in later life. He is too often the type that does not accomplish much. Both at home \& at school efforts ought to be made to interest boys' thoughts in their future occupations before they reach this age. In school this might be done by giving composition assignments on such topics as ``What I want to be later in life.'' If they are asked to write on such a theme, they are definitely confronted with the question, which otherwise they might never face until it is too late.

The last question that our youth has to face is that of love \& marriage. Inasmuch as we are living as 2 separate sexes, this is an all-important question. It would be very different if we were all 1 sex. As it is, we have to train in ways of behaving towards the other sex. We shall discuss the question of love \& marriage at length in a succeeding chapter: here it is only necessary to show its connection with the problems of social adjustment. The same lack of social interest which is responsible for social \& occupational maladjustments is also responsible for the common inabilities to meet the other sex properly. A person who is exclusively self-centered has not the proper preparation for a \textit{manage a deux}. Indeed it would seem that 1 of the chief purposes of the sex instinct is to pull the individual out of his narrow shell \& to prepare him for social life. But psychologically we have to meet the sex instinct half-way -- it cannot accomplish its function properly unless we are already predisposed to forget our own self \& merge it in a larger life.

We may now draw some conclusions about this boy we have been studying. We have seen him stand before the 3 great questions of life, despairing \& afraid of defeat. We have seen him with his personal goal of superiority excluding as far as possible all the questions of life. What then is left for him? He will not join in society, he is antagonistic to others, he is very suspicious \& seclusive. \& being no longer interested in others, he does not care how he appears before them, \& so he will often be ragged \& dirty -- with all the appearance of an insane person. Language we know is a social necessity, but our subject does not wish to use it. He does not speak at all -- a trait that is to be seen in dementia praecox.

Blocked by a self-imposed blockage from all the questions of life, the way of our subject leads straight to the asylum. His goal of superiority brings about a hermetical isolation from others, \& it transforms his sex drives so that he is no longer a normal person. We find him sometimes trying to fly to heaven, or thinking himself to be Jesus Christ or the Emperor of China. In this way he manages to express his goal of superiority.

As we have frequently said, all the problems of life are at bottom social problems. We see social problems expressed in the nursery school, the public school, in friendship, in politics, in economic life, etc. It is evident that all our abilities are socially focused \& directed for the use of mankind.

We know that a lack in social adjustment begins in the prototype. The question is how to correct this lack before it is too late. if the parents could be told not only how to prevent the great mistakes but also how to diagnose the little expressions of the mistakes in the prototype \& how to correct them, it would be a great advantage. But the truth is it is not possible to accomplish much in this way. Few parents are inclined to learn \& to avoid mistakes. They are not interested in questions of psychology \& education. Either they pamper the children \& are antagonistic to anyone who does not regard their children as perfect jewels, or else they are not interested at all. Not much can thus be accomplished through them. Also it is impossible to give them a good understanding in a short time. It would take a great deal of time to tell parents \& advise them of what they should know. It is much better, therefore, to call in a physician or psychologist.

Outside of the individual work of the physician \& psychologist, the best results can be accomplished only through schools \& education. Mistakes in the prototype often do not appear until a child enters school. A teacher cognizant of the methods of Individual Psychology will notice a mistaken prototype in a short time. She can see whether a child joins the others, or wants to be the center of attention by pushing himself forward. She also sees which children have courage \& which lack it. A well-taught teacher could understand the mistakes of a prototype in the 1st week.

Teachers, by the very nature of their social function, are better equipped, to correct the mistakes of children. Mankind started schools because the family was not able to educate children adequately for the social demands of life. The school is the prolonged hand of the family, \& it is there that the character of a child is formed to a great extent, \& that he is taught to face the problems of life.

All that is necessary is that the schools \& teachers should be equipped with psychological insight which will enable them to perform their task properly. In the future schools will surely be run more along the lines of Individual Psychology, for the true purpose of a school is to build character.'' -- \cite[pp. 199--214]{Adler_science_living}

%------------------------------------------------------------------------------%

\section{Social Feeling, Common Sense \& the Inferiority Complex}
``We have seen that social maladjustments are caused by the social consequences of the sense of inferiority \& the striving for superiority. The terms inferiority complex \& superiority complex already express the result after a maladjustment has taken place. These complexes are not in the germplasm, they are not in the blood-stream: they simply happen in the course of the interaction of the individual \& his social environment. Why don't they happen to all individuals? All individuals have a sense of inferiority \& a striving for success \& superiority which makes up the very life of the psyche. The reason all individuals do not have complexes is that their sense of inferiority \& superiority is harnessed by a psychological mechanism into socially useful channels. The springs of this mechanism are social interest, courage, \& social-mindedness, or the logic of common sense.

Let us study both the functioning \& non-functioning of this mechanism. As long as the feeling of inferiority is not too great, we know that a child will always strive to be worth while \& on the useful side of life. Such a child, in order to gain his end, is interested in others. Social feeling \& social adjustment are the right \& normal compensations, \& in a sense it is almost impossible to find anybody -- child or adult -- in whom the striving for superiority has not resulted in such development. We can never find anyone who could say truly, ``I am not interested in others.'' He may act this way -- he may act as if he were not interested in the world -- but he cannot justify himself. Rather does he claim to be interested in others, in order to hide his lack of social adjustment. This is mute testimony to the universality of the social feeling.

Nonetheless maladjustments do take place. We can study their genesis by considering marginal cases -- cases where an inferiority complex exists but is not openly expressed on account of a favorable environment. The complex is then hidden, or at least a tendency to hide it is shown. Thus if a person is not confronted with a difficulty, he may look wholly satisfied. But if we look closely we shall see how he really expresses -- if not in words or opinions, at least in attitudes -- the fact that he feels inferior. This is an inferiority complex \& is the result of an exaggerated feeling of inferiority. People who are suffering from such a complex are always looking for relief from the burdens which they have imposed upon themselves through their self-centeredness.

It is rather interesting to observe how some persons hide their inferiority complex, while others confess, ``I am suffering from an inferiority complex.'' The confessors are always elated at their confession. They feel greater than others because they have confessed while others cannot. They say to themselves, ``I am honest. I do not lie about the cause of my suffering.'' But at the very moment that they confess their inferiority complex, they hint at some difficulties in their lives or other circumstances which are responsible for their situation. They may speak of their parents or family, of not being well educated, or of some accident, curtailment, suppression, or other things.

Often the inferiority complex may be hidden by a superiority complex, which serves as a compensation. Such persons are arrogant, impertinent, conceited \& snobbish. They lay more weight on appearances than on actions.

In the early strivings of a man of this type one may find a certain stage fright, which is thereafter used to excuse all the person's failures. He says, ``If I did not suffer from stage fright, what could I not do!'' These sentences with ``if'' generally hide an inferiority complex.

An inferiority complex may also be indicated by such characteristics as slyness, cautiousness, pedantry, the exclusion of the greater problems of life, \& the search for a narrow field of action which is limited by many principles \& rules. It is also a sign of an inferiority complex if a person always leans on a stick. Such a person does not trust himself, \& we will find that he develops queer interests. He is always occupied with little things, such as collecting newspapers or advertisements. They waste their time this way \& always excuse themselves. They train too much on the useless side, \& this training when long continued leads to a compulsion neurosis.

An inferiority complex is usually hidden in all problem children no matter what type of problem the children present on the surface. Thus to be lazy is in reality to exclude the important tasks of life \& is a sign of a complex. To steal is to take advantage of the insecurity or absence of another; to lie is not to have the courage to tell the truth. All these manifestations in children have an inferiority complex as their core.

A neurosis is a developed form of inferiority complex. What can a person not accomplish when he is suffering from an anxiety neurosis! He may be constantly striving to have someone accompany him; if so, he succeeds in his purpose. He is supported by others \& gets others to be occupied with him. Here we see the transition from an inferiority to a superiority complex. Other people must serve! In getting other people to serve, the neurotic becomes superior. A similar evolution is manifested in the case of insane persons. After having been forced into difficulties because of the policy of exclusion engendered by an inferiority complex, they become successful in an imaginary way by regarding themselves as great persons.

In all these cases where complexes develop, the failure to function in social \& useful channels is due to a lack of courage on the part of the individuals. It is this lack of courage which prevents him from following the social course. Side by side with the lack of courage are the intellectual accompaniments of a failure to understand the necessity \& utility of the social course.

All this is most clearly illustrated in the behavior of criminals -- who are really cases of inferiority complexes par excellence. Criminals are cowardly \& stupid; their cowardice \& social stupidity go together as 2 parts of the same tendency.

Drinking may be analyzed on similar lines. The drunkard seeks relief from his problems, \& is cowardly enough to be satisfied with the relief that comes from the useless side of life.

The ideology \& intellectual outlook of such persons differentiate themselves sharply from the social common sense which accompanies the courageous attitudes of normal persons. Criminals, e.g., always make excuses or accuse others. They mention unprofitable conditions of labor. They speak of the cruelty of society is not supporting them. Or they say the stomach commands \& cannot be ruled. When sentenced, they always find such excuses as that of the child-murderer Hickman, who said, ``It was done by a command from above.'' Another murderer, upon being sentenced, said, ``What is the use of such a boy as I have killed? There are a million other boys.'' Then there is the ``philosopher,'' who claims that it is not bad to kill an old woman with a lot of money, when so many worth-while people are starving.

The logic of such arguments strikes us as quite frail, \& it is frail. The whole outlook is conditioned by their socially useless goal, just as the selection of that goal is conditioned by their lack of courage. They always have to justify themselves, whereas a goal on the useful side of life goes without saying \& does not need any excuses in its favor.

Let us take some actual clinical cases which illustrate how social attitudes \& goals are transformed into anti-social ones. Our 1st case is of a girl who was nearly 14. She grew up in an honest family. The father, a hardworking man, had supported the family as long as he was able to work, but he had taken sick. The mother was a good \& honest woman \& was very much interested in her children, who were 6 in all. The 1st child was a brilliant girl, who died at the age of 12. The 2nd girl was sick, but later recovered \& took a position by means of which she helped support the family. Then comes the girl of our story. This girl was always very healthy. Her mother had always been very much occupied with the 2 sick girls \& with her husband, \& did not have much time for this girl -- whom we shall call Anne. There was a younger boy, also brilliant \& sick, \& as a result Anne found herself crushed, so to speak, between 2 very beloved children. She was a good child, but felt that she was not as much liked as the others. She complained of being slighted \& of feeling suppressed.

In school, however, Anne did good work. She was the best pupil. On account of her excellence in her studies the teacher recommended that she continue her education, \& when she was only 13 \& a half she went to high school. Here she found a new teacher, who did not like her. Perhaps in the beginning she was not a good pupil, but any way with the lack of appreciation she grew worse. She had not been a problem child as long as she was appreciated by her old teacher. She had had good reports \& had been well liked by her comrades. An individual psychologist could have seen even that something was wrong by looking at her friendships. She was always criticizing her friends \& always wanted to rule them. She wanted to be the center of attention \& to be flattered, but never to be criticized.

Anne's goal was to be appreciated, to be favored \& to be looked after. She found she could accomplish this only in school -- not at home. But at the new school she found appreciation blocked there, too. The teacher scolded her, insisted that she was not prepared, gave her bad reports, so that at last she became a truant \& stayed away altogether for a few days. When she came back, things were worse than ever, \& in the end the teacher proposed that she be expelled from school.

Now expulsion from school accomplishes nothing. It is a confession on the part of the school \& teacher that they cannot solve the problem. But if they cannot solve the problem they should call in others, who perhaps might be able to do something. Perhaps after talking with her parents arrangements might have been made to try another school. Perhaps there might have been another teacher who would have understood Anne better. But her teacher did not reason that way; she reasoned, ``If a child plays truant \& is backward, she must be expelled.'' Such reasoning is a manifestation of private intelligence, not of common sense, \& common sense is specially to be expected in a teacher.

We can guess what happened next. The girl lost her last hold in life \& felt that everything was failing her. On account of being expelled from school, she lost even her slight modicum of appreciation at home. \& so she ran away from both home \& school. She disappeared for some days \& nights. Finally it develops that she was had a love affair with a soldier.

We can easily understand her action. Her goal was to be appreciated, \& up to this time she had been trained toward the useful side, but now she began her training on the useless side. This soldier in the beginning appreciated her \& liked her. Later, however, the family received letters from her saying that she was pregnant \& that she wanted to take poison.

This action of writing to her folks is in line with her character. She is always turning in the direction in which she expects to find appreciation \& she keeps turning until she comes back home. She knows that her mother is in despair \& that she will not therefore be scolded. Her family, she feels, will be only too glad to find her again.

In treating a case of this sort, identification -- the ability to place oneself sympathetically in the situation of a person -- is all-important. Here is an individual who wants to be appreciated \& is pushing forward towards this 1 goal. If one were to identify oneself with such a person, one would ask himself, ``What should I do?'' Sex \& age must be considered. We should always try to encourage such a person -- but encourage her towards the useful side. We should try to get her to the point where she would say, ``Perhaps I should change my school, but I am not backward. Perhaps I have not trained enough -- perhaps I have not observed rightly -- perhaps I showed too much private intelligence at school \& did not understand the teacher.'' If it is possible to lend courage, then a person will learn to train on the useful side. It is the lack of courage connected with an inferiority complex that ruins a person.

Let us put another person in the girl's place. A boy, e.g., at her age might become a criminal. Such cases are often met with. If a boy loses courage in school, he is likely to drift away \& become a member of a gang. Such behavior is easily understood. When he loses hope \& courage, he will begin to be tardy, forge signatures to excuses, not do his homework, \& look for places where he can play truant. In such places he find companions who have gone the same way before, \& now he becomes a member of a gang. He loses all interest in school, \& he develops more \& more a private understanding.

The inferiority complex is often connected with the idea that a person has no special abilities. The opinion is that some persons are gifted \& others not. Such a view is itself an expression of an inferiority complex. According to Individual Psychology, ``Everybody can accomplish everything,'' \& it is a sign of an inferiority complex when a boy or girl despairs of following this maxim \& feels unable to accomplish his goal on the useful side of life.

As part of the inferiority complex we have the belief in inherited characteristics. If this belief were really true -- if success were completely dependent upon innate ability -- then the psychologist could accomplish nothing. Actually, however, success is dependent on courage, \& the task of the psychologist is to transform the feeling of despair into a feeling of hopefulness which rallies energies for the performance of useful work.

We see sometimes a youth of 16 expelled from school \& committing suicide out of despair. The suicide is a sort of revenge -- an accusation against society. It is the youth's way of affirming himself, via private intelligence, instead of via common sense. All that is necessary in such a situation is to win over the boy \& to give him courage to take the useful path.

We can take many other examples. Consider a case of a girl, 11 years old, who was not liked at home. All the other children were preferred, \& she felt she was not wanted. She became peevish, pugnacious \& disobedient. It is a case that we can analyze quite simply. The girl felt she was not appreciated. In the beginning she tried to struggle, but then she lost hope. 1 day she began to steal. For the individual psychologist when a child steals, it is not so much a crime as a case of a movement of the child to enrich herself. It is not possible to enrich oneself unless one feels deprived. Her stealing was thus the result of her lack of affection at home \& of her feeling of hopelessness. We will always notice that children begin to steal when they feel deprived. Such a feeling may not express the truth, but it is nonetheless the psychological cause for their action.

Another case is that of an 8-year-old boy -- an illegitimate, ugly child who was living with foster parents. These foster parents did not take good care of him \& did not restrain him. Sometimes the mother gave him candy, \& this was a bright spot in his life. When candy was scarce the poor boy suffered terribly. His mother had married an old man \& had 1 child by him, \& this child was the old man's only pleasure. He pampered her continually. The only reason the pair kept the boy was in order not to have to pay any money for his maintenance outside. When the old man came home he would bring candy for the little girl, but none for the boy. As a result the boy began to steal candy. He stole because he felt deprived, \& wanted to enrich himself. The father beat him for stealing, but he kept on. One might think that the boy showed courage in that he kept on in spite of the beatings, but this is not true. He always had the hope of escaping detection.

This is a case of a hated child never experiencing what it means to be a fellow-man. We have to win him. We must give him the opportunity of living as a fellow-man. When he learns to identify \& to place himself in others' positions he will understand how his stepfather feels when he sees him stealing \& how the little girl feels when she discovers her candy gone. We see here again how lack of social feeling, lack of understanding \& lack of courage go together to form an inferiority complex -- in this case the inferiority complex of a hated child.'' -- \cite[pp. 215--230]{Adler_science_living}

%------------------------------------------------------------------------------%

\section{Love \& Marriage}
``The right preparation for love \& marriage is 1st of all to be a fellow-man \& to be socially adjusted. Along with this general preparation should be put a certain training of the instinct of sex from early childhood down to adult maturity -- a training that has in view the normal satisfaction of the instinct in marriage \& a family. All the abilities, disabilities \& inclinations for love \& marriage can be found in the prototype formed in the 1st years of life. By observing the traits in the prototype we are able to put our finger on the difficulties that appear later in adult life.

The problems that we meet in love \& marriage are of the same character as the general social problems. There are the same difficulties \& the same tasks, \& it is a mistake to regard love \& marriage as a paradise in which all things happen according to one's desires. There are tasks throughout, \& these tasks must be accomplished with the interests of the other person always in mind.

More than the ordinary problems of social adjustment, the love \& marriage situation requires an exceptional sympathy, an exceptional ability to identify oneself with the other person. If few persons are properly prepared nowadays for family life it is that they have never learned to see with the eyes, hear with the ears, \& feel with the heart of another.

Much of our discussion in the previous chapters has centered on the type of child who grows up interested only in himself \& not in others. Such a type cannot be expected to change his character overnight with the maturing of the physical sex instinct. He will be unprepared for love \& marriage in the same way that he is unprepared for social life.

Social interest is a slow growth. Only those persons who are really trained in the direction of social interest from their 1st childhood \& who are always striving on the useful side of life will actually have social feeling. For this reason it is not particularly difficult to recognize whether a person is really well prepared for life with the other sex or not.

We need only to remember what we have observed with regard to the useful side of life. A person on that side of life is courageous \& has confidence in himself. He faces the problems of life \& goes on to find solutions. He has comrades, friends \& gets along with his neighbors. A person who does not have these traits is not to be trusted \& is not to be regarded as prepared for love \& marriage. On the other hand we may conclude that a person is probably ready for marriage if he has an occupation \& is progressing in it. We judge by a small sign, but the sign is very significant in that it indicates whether or not a person has social interest.

An understanding of the nature of social interest shows us that the problems of love \& marriage can be solved satisfactorily only on the basis of entire equality. This fundamental give-\&-take is the important thing; whether 1 partner esteems the other is not very significant. Love by itself does not settle things, for there are all kinds of love. It is only when there is a proper foundation of equality that love will take the right course \& make marriage a success.

If either the man or the woman wishes to be a conqueror after marriage, the results are likely to be fatal. Looking forward to marriage with such a view in mind is not the right preparation, \& the events after marriage are likely to prove it. It is not possible to be a conqueror in a situation in which there is no place for a conqueror. The marriage situation calls for an interest in the other person \& an ability to put oneself in the other's place.

We turn now to the special preparation necessary for marriage. This involves, as we have seen, the training of the social feeling in connection with the instinct of sexual attraction. As a matter of fact we know that everyone creates in his own mind from childhood days on an ideal of a person of the other sex. In the case of a boy it is very probable that the mother plays the role of the ideal, \& that the boy will always look for a similar type of woman to marry. Sometimes there may be a state of unhappy tension between the boy \& his mother, in which case he will probably look for the opposite type of girl. So close is the correspondence between the child's relation with his mother \& the type of woman he afterwards marries that we can observe it in such little details as eyes, figures, color of hair, etc.

We know, too, that if the mother is domineering \& suppresses the boy, he will not want to go on courageously when the time comes for love \& marriage. For in such a case his sexual ideal is likely to be a weak, obedient type of girl. Or, if he is a pugnacious type, he will also fight with his wife after marriage \& want to dominate her.

We can see how all the indications manifested in childhood are emphasized \& increased when a person faces the problem of love. We can imagine how a person suffering from an inferiority complex will behave in sexual matters. Perhaps because he feels weak \& inferior he will express the feeling by always wanting to be supported by other persons. Often such a type has an ideal which is motherly in character. Or sometimes, by way of compensation for his inferiority, he may take the opposite direction in love \& become arrogant, impudent \& aggressive. Then, too, if he has not very much courage he will feel restricted in his choice. He may possibly select a pugnacious girl, finding it more honorable to be the conqueror in a severe battle. 

Neither sex can act successfully in this way. It seems silly \& ridiculous to have the sexual relationship exploited for the satisfaction of an inferiority or superiority complex, \& yet this happens very frequently. If we look closely we see that the mate that many a person seeks is really a victim. Such persons do not understand that the sex relationship cannot be exploited for such an end. For if 1 person seeks to be a conqueror, the other will want to be a conqueror also. As a result life in common becomes impossible. The idea of satisfying one's complexes illuminates certain peculiarities in the choice of a partner which are otherwise difficult to understand. It tells us why some persons choose weak, sick, or old persons: they choose them because they believe things will be easier for them. Sometimes they look for a married person: here it is a case of never wanting to reach a solution of a problem. Sometimes we find people falling in love with 2 men or 2 women at the same time, because, as we have already explained, ``2 girls are less than 1.''

We have seen how a person who suffers from an inferiority complex changes his occupation, refuses to face problems, \& never finishes things. When confronted with the problem of love he acts in a similar fashion. Falling in love with a married person or with 2 persons at once is a way of satisfying his habitual tendency. There are other ways, too, as for instance, overlong engagements, or even perpetual courtships, which are never consummated into marriage.

Spoiled children show up in marriage true to type. They want to be pampered by their marital partners. Such a state of affairs may exist without danger in the early stages of courtship or in the 1st years of marriage, but later it will bring about a complicated situation. We can imagine what happens when 2 such pampered persons marry. Both want to be pampered, \& neither of them wants to play the pamperer. It is as if they stood before one another expecting something which neither gives. Both have the feeling that they are not understood.

We can understand what happens when a person feels himself misunderstood \& his activity curtailed. He feels inferior \& wants to escape. Such feelings are especially bad in marriage, particularly if a sense of extreme hopelessness arises. When this happens revenge begins to creep in. 1 person wants to disturb the life of the other person. The most common way to do this is to be unfaithful. Infidelity is always a revenge. True, persons who are unfaithful always justify themselves by speaking of love \& sentiments, but we know the value of feelings \& sentiments. Feelings always agree with the goal of superiority, \& should not be regarded as arguments.

As an illustration we may take the case of a certain pampered woman. She married a man who had always felt himself curtailed by his other brother. We can see how this man was attracted by the mildness \& sweetness of this only girl, who in turn expected always to be appreciated \& preferred. The marriage was quite happy until a child came. \& then we can predict what happened. The wife wanted to be the center of attention but was afraid that the child might step into that position. \& so she was not very happy to have given birth to the child. The husband, on the other hand, also wanted to be preferred \& was afraid that the child would usurp his place. As a result both husband \& wife became suspicious. They did not perhaps neglect the child \& were quite good parents, but they were always expecting that their love for each other would decrease. Such a suspicion is dangerous, because if one starts to measure every word, every action, movement \& expression, it is easy to find, or to appear to find, a decrease in affection. Both parties found it. As it happened the husband went on a holiday, traveled to Paris \& enjoyed himself while his wife recuperated from childbirth \& looked after the infant. The husband wrote gay letters from Paris, telling his wife what good a time he was having, how he met all sorts of people, etc. The wife began to feel herself forgotten. \& so she was not as happy as before, but became quite depressed \& soon began to suffer from agorophobia. She could not go out alone any more. When her husband returned he always had to accompany her. On the surface at least, it would seem that she had attained her goal, \& that she was now the center of attention. But nonetheless it was not the right satisfaction, for she had the feeling that if her agorophobia disappeared, her husband would disappear, too. Hence she continued to have agorophobia.

During this illness she found a doctor who gave her much attention. While under his care she felt much better. What feelings of friendship she had were all directed towards him. But when the doctor saw that the patient was better, he left her. She wrote him a nice letter thanking him for all he had done for her, but he did not answer. From this time on the illness became worse.

At this time she began to have ideas \& fancies about liaisons with other men in order to revenge herself against her husband. However, she was protected by her agorophobia, inasmuch as she could not go out alone but had always to be accompanied by her husband. \& thus she could not succeed in being unfaithful.

We see so many mistakes made in marriage that the question inevitably arises, ``Is all this necessary?'' We know that these mistakes were begun in childhood, \& we know, too, that it is possible to change mistaken styles of life by recognizing \& discovering the prototype traits. One wonders, therefore, whether it would not be possible to establish advisory councils which could untangle the mistakes of matrimony by the methods of Individual Psychology. Such councils would be composed of trained persons who would understand how all the events in individuals' lives cohere \& hang together, \& who would have the power to sympathetic identification with the persons seeking advice.

Such councils would not say: ``You cannot agree -- you quarrel continuously -- you should get a divorce.'' For what use is a divorce? What happens after a divorce? As a rule the divorced persons want to marry again \& continue the same style of life as before. We sometimes see persons who have been divorced time \& again, \& still they remarry. They simply repeat their mistakes. Such persons might ask this advisory council whether their proposed marriage or love relation had in it any possibility of success. Or they might consult it before obtaining a divorce.

There are a number of little mistakes which begin in childhood \& which do not seem important until marriage. Thus some persons always think they will be disappointed. There are children who are never happy \& who are constantly in fear of being disappointed. These children either feel that they have been displaced in affection \& that another will be preferred, or else an early difficulty which they experienced has made them superstitiously afraid that the tragedy may recur again. We can easily see that this fear of disappointment will create jealousies \& suspicions in married life.

Among women there is a particularly difficulty in that they feel they are only toys for men to play with, \& that men are always unfaithful. With such an idea it is easy to see that marriage will not be happy. Happiness is impossible if 1 party has the fixed idea that the other is likely to be unfaithful.

From the way in which people always seek advice on love \& marriage, one would judge that it is generally regarded as the most important question of life. From the point of view of Individual Psychology, however, it is not the most important question, although its importance is not to be undervalued. For Individual Psychology no one question in life is more important than another. If persons accentuate the question of love \& marriage \& give it a paramount importance they lose the harmony of life.

Perhaps the reason why this question is given such undue importance in the minds of people is that, unlike other questions, this is a topic on which we do not get any regular instruction. Recollect what we have said about the 3 great questions of life. Now as regards the 1st, the social question, which involves our behavior with others, we are taught from the 1st day of our life how to act in the company of others. We learn these things quite early. We likewise have a regular course of training for our occupations. We have masters to instruct us in these arts, \& we have also books which tell us what to do. But where are the books that tell us how to prepare for love \& marriage? To be sure there are a great many which deal with love \& marriage. All literature deals with love stories, but we will find few books which deal with happy marriages. Because our culture is so tied up with literature everybody has his attention fixed on portraits of men \& women who are always in difficulties. No wonder people feel cautious \& overcautious about marriage.

This has been the practice of mankind from the beginning. If we look at the Bible we will find there the story that woman began all the trouble, \& that ever since the men \& women have always experienced great dangers in their love life. Our education is certainly too strict in the direction it follows. Instead of preparing boys \& girls as if for sin, it would be much wiser to train the girls better in the feminine \& the boys in the masculine role of marriage -- but train them both in such way that they would feel equal.

The fact that women now feel inferior proves that, in this particular, our culture has failed. If the reader is not convinced by this, let him look at the strivings of women. He will find that they usually want to overcome others \& that often they develop \& train more than is necessary. They are also more self-centered than men. In the future women must be taught to develop more social interest \& not always to seek benefit for themselves without regard for others. But in order to do this, we must 1st banish the superstition regarding the privileges of men.

Let us take an instance to show how poorly prepared for marriage some people are. A young man was dancing at a ball with a pretty young girl whom he was engaged to marry. He happened t drop his glasses, \& to the utter amazement of the spectators he almost knocked the young lady down in order to pick them up. When a friend asked him, ``What were you doing?'' he replied ``I could not let her break my glasses.'' We can see that this young man was not prepared for marriage. \& indeed the girl did not marry him.

Later in life he came to the doctor \& said he was suffering from melancholia, as is often the case with persons who are too much interested in themselves.

There are a thousand signs by which one can understand whether or not a person is prepared for marriage. Thus one should not trust a person in love who comes late for an appointment without an adequate excuse. Such action shows hesitating attitude. It is a sign of lack of preparation for the problems of life.

It is also a sign of lack of preparation if 1 member of a couple always wants to educate the other or always wants to criticize. Also to be sensitive is a bad sign, since it is an indication of an inferiority complex. The person who has no friends \& does not mix well in society is not well prepared for marital life. Delay in choosing an occupation is also not a good sign. A pessimistic person is ill-fitted, doubtless because pessimism betrays a lack of courage to face situations.

Despite this life of undesirables it should not be so difficult to choose the right person, or rather to choose a person along the right lines. We cannot expect to find the ideal person. \& indeed if we see someone looking for an ideal person for marriage \& never finding him or her, we may be sure that such a person is suffering from a hesitating attitude. Such a person does not want to go on at all.

There is an old German method for finding out whether a couple is prepared for marriage. It is the custom in rural districts to give the couple a double-handled saw, each person to hold 1 end, \& then have them saw the trunk of a tree while all the relatives stand around \& watch. No sawing a tree is a task for 2 persons. Each one has to be interested in what the other is doing \& harmonize his strokes with his. This method is thus considered a good test of fitness for marriage.

In conclusion we reiterate our statement that the questions of love \& marriage can be solved only by socially adjusted persons. The mistakes in the majority of cases are due to lack of social interest, \& these mistakes can be obviated only if the persons change. Marriage is a task for 2 persons. Now it is a fact that we are educated either for tasks that can be performed by 1 person alone or else by 20 persons -- never for a task for 2 persons. But despite our lack of education the marriage task can be handled properly if the 2 persons recognize the mistakes in their character \& approach things in a spirit of equality.

It is almost needless to add that the highest form of marriage is monogamy. There are many persons who claim on pseudo-scientific grounds that polygamy is better adapted to the nature of human beings. This conclusion cannot be accepted, \& the reason it cannot be accepted is that in our culture love \& marriage are social tasks. We do not marry for our private good only, but indirectly for the social good. In the last analysis marriage is for the sake of the race.'' -- \cite[pp. 231--248]{Adler_science_living}

%------------------------------------------------------------------------------%

\section{Sexuality \& Sex Problems}
``We discussed in the preceding chapter the normal problems of love \& marriage. We turn now to a more specific phase of the same general question -- the problems of sexuality \& their bearing upon real or fancied abnormalities. We have already seen that in the question of love life, most persons are less well prepared \& less well trained than for the other questions of life. This conclusion applies with even greater force to the topic of sex. In questions of sexuality it is extraordinary how many superstition must be wiped out.

The most common superstition is that of inherited characteristics -- the belief that there are degrees of sexuality which are inherited \& which cannot be changed. We know how easily questions of inheritance can be used as excuses or subterfuges, \& how these subterfuges can hinder improvement. It is necessary therefore to clarify some of the opinions that are advanced on behalf of science. These views are taken too seriously by the average layman, who do not realize that these authors give only the results \& do not discuss either the degree of inhibition possible or the artificial stimulation of the sex instinct which is possible for the results.

Sexuality exists very early in life. Every nurse or parent who observes carefully can find in the 1st days after the birth of a child that there are certain sexual irritations \& movements. However, this display of sexuality is much more dependent upon environment than one might expect. \& so when a child begins to express himself in this way, the parents should find ways to distract him. Often they use means which do not produce the right type of distraction, \& sometimes the right means are not available.

If a child does not at an early stage find the correct functions, he may naturally develop a greater desire for sexual movements. These things happen, we have seen, as regards other organs of the body, \& the sex organs are no exception. But if one begins early it is possible to train the child correctly.

In general it may be said that some sex expression in childhood is quite normal, \& we should not therefore be terrified by the sight of sexual movements in a child. After all, the goal of each sex is to be eventually joined to the other. Our policy should therefore be 1 of watchful waiting. We must stand by \& see that sexual expression does not develop in the wrong direction.

There is a tendency to attribute to inherited deficiency what is really the result of self-training during childhood. Sometimes this very act of training is regarded as an inherited characteristic. Thus if a child happens to be more interested in his own sex than in the opposite sex, this is considered an inherited disability. But we know that this disability is something which he develops from day to day. Sometimes a child or adult shows signs of sex perversion; \& here again many persons believe the perversion to be inherited. But if this is the case, why does such a person train himself? Why does he dream \& rehearse his actions?

Certain persons stop this training at a certain time, \& this fact can be explained along the lines of Individual Psychology. There are, e.g., those who fear defeat. They have an inferiority complex. Or they may train so far that the result is a superiority complex, \& in a case of this kind we will note an exaggerated movement which looks like overstressed sexuality. Such persons may possess greater sexual power.

This type of striving may be specially stimulated by the environment. We know how pictures, books, movies, or certain social contacts tend to over-stress this sex drive. In our time one may say that everything tends to develop an exaggerated interest in sex. One need not depreciate the great importance of these organic drives \& of the part they play in love, marriage \& the procreation of mankind, in order to assert that sex is over-emphasized at the present time.

It is the exaggeration of sex tendencies that is most to be guarded against by the parents who watch their children. Thus too often a mother pays too much attention to the 1st sexual movements in childhood \& thereby tends to make the child overvalue their significance. She is perhaps terrified \& is always occupied with such a child, always talking to him about these matters \& punishing him. Now we know that many children like to be the center of attention, \& hence it is frequently the case that a child continues in his habits precisely because he is scolded for them. It is better not to over-value the subject with a child, but to treat the matter as of 1 of the ordinary difficulties. If one does not show children that one is impressed by these matters, one will have a much easier time.

Sometimes there are traditions back of the child which incline him in a certain direction. It may be that the mother is not only affectionate, but expresses her affection in kisses, embraces, etc. Such things should not be overdone, although many mothers insist that they cannot resist doing them. Such actions, however, are not an example of motherly love. It is treating the child like an enemy rather than like a mother's child. A pampered child does not develop well sexually.

In this connection it may be pointed out that many doctors \& psychologists believe that the development of sexuality is the basis for the development of the whole mind \& psyche, as well as for all the physical movements. In the view of the present writer this is not true, inasmuch as the whole form \& development of sexuality is dependent upon the personality -- the style of life \& the prototype.

Thus, e.g., if we know that a child expresses his sexuality in a certain way, or that another child suppresses it, we may guess what will happen to both of them in their adult life. If we know that the child always wants to be the center of attention \& to conquer, then he will also develop his sexuality so as to conquer \& be the center of attention.

Many persons believe that they are superior \& dominant when they express their sex instinct polygamously. They therefore have sex relations with many, \& it is easy to see that they deliberately overstress their sexual desires \& attitudes for psychological reasons. They think that thereby they will be conquerors. This is an illusion, of course, but it serves as a compensation for an inferiority complex.

It is the inferiority complex which is the core of sexual abnormalities. A person who suffers from an inferiority complex is always looking for the easiest way out. Sometimes he finds this easiest way by excluding most of life \& exaggerating his sexual life.

In children we find this tendency very often. Generally we find it among children who want to occupy others. They occupy their parents \& teachers by creating difficulties \& thus following out their striving on the useless side of life. Later in life they will occupy others with their tendencies \& want to be superior in that way. Such children grow up confusing their sexual desire with the desire for conquest \& superiority. Sometimes, in the course of their exclusion of part of the possibilities \& problems of life, they may exclude the whole of the other sex \& train homosexuality. It is significant that among perverted persons an over-stressed sexuality is often to be found. They in fact exaggerate their tendency to be perverted as an insurance against having to face the problem of normal sex life which they wish to avoid.

We can understand all this only if we understand their style of life. We have here persons who want to have much attention paid to them \& who yet believe themselves incapable of interesting the other sex sufficiently. They have an inferiority complex in regard to the other sex which may be traced back to childhood. E.g., if they found the behavior of the girls in the family \& the behavior of the mother more attractive than their own, they got the feeling that they will never have the power to interest women. They may admire the opposite sex so much that they begin to imitate the members of that sex. Thus we see men who appear like girls, \& likewise girls who appear like men.

There is the case of a man accused of sadism \& of actions against children which well illustrates the formation of the tendencies we have discussed. Inquiring into his development, we learn that he had a ruling \& dominating mother who was always criticizing him. Despite this he developed into a good \& intelligent pupil at school. But his mother was never satisfied with his success. For this reason he wanted to exclude his mother from his family affections. He was not interested in her, but he occupied himself with his father \& was greatly attached to him.

We can see how such a child might get the idea that women are severe \& hypercritical \& that contact with them is not to be made with pleasure but only in case of great necessity. In this way he came to exclude the other sex. This person, moreover, was of a familiar type who is always sexually irritated when afraid. Suffering from anxiety \& being thus irritated, this type constantly looks for situations where he will not be afraid. Later in life he might like to punish or torture himself, or see a child tortured, or even fancy himself or another tortured. \& because he is of the type described, he will have sexual irritation \& satisfactions in the course of his real or imaginary tortures.

The case of this man indicates the result of wrong training. The man never understood the interconnection of his habits, or if he did, he only saw it when it was too late. Of course it is very difficult to start to train a person properly at the age of 25 or 30. The right time is early childhood.

But in childhood matters are complicated by the psychological relations with the parents. It is curious to see how bad sexual training results as an incident in the psychological conflict of child \& parent. A fighting child, especially in the period of adolescence, may abuse sexuality with the deliberate intention of hurting the parents. Boys \& girls have been known to have sex relations just after a fight with their parents. Children take these means of revenging themselves on their parents particularly when they see that they are sensitive in this regard. A fighting child will almost invariably take this point of attack.

The only way such tactics can be avoided is to make each child responsible for himself, so that he should not believe that it is the parents' interest alone which is at stake, but his own as well.

Besides the influence of childhood environment as reflected in the style of life, political, \& economic conditions in a country have their influence on sexuality. These conditions give rise to a social style which is very contagious.  After the Russo-Japanese war \& the collapse of the 1st revolution in Russia, when all the people had lost their hope \& confidence, there ws a great movement of sexuality known as Saninism. All the adults \& adolescents were caught up in this movement. One finds a similar exaggeration of sexuality during revolution, \& it is of course notorious that in war time there is a great recourse to sexual sensitivity because life seems worthless.

It is curious to note that the police understand this use of sexuality as a psychological release. In Europe at least, whenever any crime is committed, the police look usually in the houses of prostitutes. There they find the murderer or other criminal that they are looking for. The criminal is there because after committing a crime he feels over-strained \& looks for himself. He wants to convince himself of his strength \& to prove that he is still a powerful being \& not a lost soul.

A certain Frenchman has remarked that man is the only animal that eats when he is not hungry, drinks when he is not thirsty \& has sex relations at all times. The over-indulgence of the sex instinct is really quite on a par with the over-indulgence of other appetites. Now when any appetite is over-indulged \& any interest is overdeveloped, the harmony of life is interfered with. Psychological annals are full of cases of persons who develop interests or appetites to the point where they become a compulsion with them. The cases of misers who overstress the importance of money are familiar to the common man. But there are also the cases of persons who think cleanliness all important. They put washing ahead of all other activities \& at times they wash the whole day \& half the night. Then there are persons who insist on the paramount importance of eating. They eat all day long, are interested only in eatables, \& talk about nothing but eating.

The cases of sexual excess are precisely similar. They lead to an unbalancing of the entire harmony of activity. Inevitably they drag the whole style of life to the useless side.

In the proper training of the sex instinct the sexual drives should be harnessed to a useful goal in which the whole of our activities are expressed. If the goal is properly chosen neither sexuality nor any other expression of life will be overstressed.

On the other hand while all appetites \& interests have to be controlled \& harmonized, there is danger in complete suppression. Just as in the matter of food, when a person diets to the extreme, his mind \& body suffer, so, too, in the matter of sex complete abstinence is undesirable.

What this statement implies is that in a normal style of life sex will find its proper expression. It does not mean that we can overcome neuroses, which are the marks of an unbalanced style of life, merely by free sex expression. The belief, so much propagated, that a suppressed libido is the cause of a neurosis is untrue. Rather it is the other way around: neurotic persons do not find their proper sex expression.

One meets persons who have been advised to give more free expression to their sex instincts \& who have followed that advice, only to make their condition worse. The reason things work out that way is that such persons fail to harness their sexual life with a socially useful goal, which alone can change their neurotic condition. The expressions of sex instinct by itself does not cure the neurosis, for the neurosis is a disease in the style of life, if we may use the term, \& it can be cured only by ministering to the style of life.

For the individual psychologist all this is so clear that he does not hesitate to fall back on happy marriage as the only satisfactory solution for sex troubles. A neurotic does not look with favor on such a solution, because a neurotic is always a coward \& not well prepared for social life. Similarly all persons who overstress sexuality, talk of polygamy, \& companionate or trial marriage are trying to escape the social solution of the sex problem. They have no patience for solving the problem of social adjustment on the basis of mutual interest between husband \& wife \& dreams of escape through some new formula. The most difficult road, however, is sometimes the most direct.'' -- \cite[pp. 249--262]{Adler_science_living}

%------------------------------------------------------------------------------%

\section{Conclusion}
``It is time now to conclude the results of our survey. The method of Individual Psychology -- we have no hesitation in confessing it -- begins \& ends with the problem of inferiority.

Inferiority, we have seen, is the basis for human striving \& success. On the other hand the sense of inferiority is the basis for all our problems of psychological maladjustment. When the individual does not find a proper concrete goal of superiority, an inferiority complex results. The inferiority complex leads to a desire for escape \& this desire for escape is expressed in a superiority complex, which is nothing more than a goal on the useless \& vain side of life offering the satisfaction of false success.

This is the dynamic mechanism of psychological life. More concretely, we know that the mistakes in the functioning of the psyche are more harmful at certain times than at others. We know that the style of life is crystallized in tendencies formed in childhood -- in the prototype that develops at the age of 4 or 5. \& this being so, the whole burden of the guidance of our psychological life rests on proper childhood guidance.

As regards childhood guidance we have shown that the principal aim should be the cultivation of proper social interests in terms of which useful \& healthy goals can be crystallized. It is only by training children to fit in with the social scheme that the universal sense of inferiority is harnessed properly \& is prevented from engendering either an inferiority or superiority complex.

Social adjustment is the observe face of the problem of inferiority.

It is because the individual man is inferior \& weak that we find human beings living in society. Social interest \& social cooperation are therefore the salvation of the individual.'' -- \cite[pp. 263--264]{Adler_science_living}

%------------------------------------------------------------------------------%

%------------------------------------------------------------------------------%

\section{Afred Adler. Understanding of Human Nature}
\textbf{\textsf{Resources -- Tài nguyên.}}
\begin{enumerate}
	\item \cite{Adler_human_nature}. {\sc Alfred Adler}. {\it Understanding Human Nature}.
	
	Với bản dịch tiếng Việt:
	\item \cite{Adler_human_nature_VN}. {\sc Alfred Adler}. {\it Understanding Human Nature -- Hiểu Về Bản Chất Con Người}.
\end{enumerate}

\subsection{Introduction}

\begin{quote}
	{\it``The destiny of man lies in his soul.''} -- {\sc Herodotus}
\end{quote}
``The science of human nature may not be approached with too much presumption \& pride. On the contrary, its understanding stamps those who practice it with a certain modesty. The problem of human nature is one which presents an enormous task, whose solution has been the goal of our culture since time immemorial. It is a science that cannot be pursued with the sole purpose of developing occasional experts. Only the understanding of human nature by every human being can be its proper goal. This is a sore point with academic investigators who consider their researches the exclusive property of a scientific group.

Owing to our isolated life none of us knows very much about human nature. In former times it was impossible for human beings to live such isolated lives as they live today. We have from the earliest days of our childhood few connections with humanity. The family isolates us. Our whole way of living inhibits that necessary intimate contact with our fellow men, which is essential for the development of the science \& art of knowing human nature. Since we do not find sufficient contact with our fellow men, we become their enemies. Our behavior towards them is often mistaken, \& our judgments frequently false, simply because we do not adequately understand human nature. It is an oft-repeated truism that human beings walk past, \& talk past, each other, fail to make contacts, because they approach each other as strangers, not only in society, but also in the very narrow circle of the family. There is no more frequent complaint than the complaint of parents that they cannot understand their children, \& that of children that they are misunderstood by their parents. Our whole attitude toward our fellow man is dependent upon our understanding of him; an implicit necessity for understanding him therefore is a fundamental of the social relationship. Human beings would live together more easily if their knowledge of human nature were more satisfactory. Disturbing social relationships could then be obviated, for we know that unfortunate adjustments are possible only when we do not understand one another \& are therefore exposed to the danger of being deceived by superficial dissimulations.

It is now our purpose to explain why an attempt is made to approach the problem from the standpoint of the medical sciences, with the objective of laying the foundations of an exact science in this enormous field; \& to determine what the premises of this science of human nature must be, what problems it must solve, \& what results might be expected from it.

In the 1st place, psychiatry is already a science which demands a tremendous knowledge of human nature. The psychiatrist must obtain insight into the soul of the neurotic patient as quickly \& as accurately as possible. In this particular field of medicine one can judge, treat, \& prescribe effectively only when one is quite sure of what is going on in the soul of the patient. Superficiality has no place here. Error is followed quickly by punishment, \& the correct understanding of the ailment is crowned by success in the treatment. In other words, a very effective test of our knowledge of human nature occurs. In ordinary life, an error in the judgment of another human being need not be followed by dramatic consequences, for these may occur so long after the mistake has been made that the connection is not obvious. Often we find ourselves astonished to see what great misfortunes follow decades after a misinterpretation of a fellow man. Such dismal occurrences teach us the necessity \& duty of every man to acquire a working knowledge of human nature.

Our examinations of nervous diseases prove that the psychic anomalies, complexes, mistakes, which are found in nervous diseases are fundamentally not different in structure from the activity of normal individuals. The same elements, the same premises, the same movements, are under consideration. The sole difference is that in the nervous patient they appear more marked, \& are more easily recognized. The advantage of this discovery is that we can learn from the abnormal cases, \& sharpen our eye for the discovery of related movements \& characteristics in the normal psychic life. It is solely a question of that training, ardor, \& patience which are required by any profession.

The 1st great discovery was this: the most important determinants of the structure of the soul life are generated in the earliest days of childhood. In itself this was not such an audacious discovery; similar findings has been made by the great students of all times. The novelty lay in the fact that we were able to join the childhood experiences, impressions, \& attitudes, so far as we were capable of determining them, with the later phenomena of the soul life, in 1 incontrovertible \& continuous pattern. In this way we were able to compare the experiences \& attitudes of the earliest childhood days with the experiences \& attitudes of the mature individual later on in life; \& in this connection the important discovery was made that the single manifestations of the psychic life must never be regarded as entities sufficient unto themselves. It was learned that we could gain an understanding of these single manifestations only when we considered them as partial aspects of an indivisible whole, \& that these single manifestations could be valued only when we could determine their place in the general stream of activity, in the general behavior pattern -- only when we could discover the individual's whole style of life, \& make perfectly clear that the secret goal of his childhood attitude was identical with his attitude in maturity. In short, it was proven with astonishing clarity that, from the standpoint of psychic movements, no change had taken place. The outer form, the concretization, the verbalization of certain psychic phenomena might change, but the fundamentals, the goal, the dynamics, everything which directed the psychic life towards its final objective, remained constant. A mature patient who has an anxious character, whose mind is constantly filled with doubts \& mistrust, whose every effort is directed toward isolating himself from society, shows the identical character traits \& psychic movements in his 3rd \& 4th year of life, though in their childish simplicity they are more transparently interpreted. We made it a rule therefore to direct the greater part of our investigation to the childhood of all patients; \& thus we developed the art of being able, often, to reveal characteristics of a mature person whose childhood  we knew, before we were told of them. What we observe in him as an adult we consider the direct projection of that which he has experienced in childhood.

When we hear the most vivid recollections of a patient's childhood, \& know how to interpret these recollections correctly, we can reconstruct with great accuracy the pattern of his present character. In doing this we make use of the fact that an individual can deviate from the behavior into which he has grown in childhood only with great difficulty. Very few individuals have ever been able to change the behavior pattern of their childhood, though in adult life they have found themselves in entirely different situations. A change of attitude in adult life need not necessarily signify a change of behavior pattern. The psychic life does not change its foundation; the individual retains the same line of activity both in childhood \& in maturity, leading us to deduce that his goal in life is also unaltered. There is another reason for concentrating our attention upon childhood experiences if we wish to change the behavior pattern. It makes little difference whether we alter the countless experiences \& impressions of an individual in maturity; what is necessary is to discover the fundamental behavior pattern of our patient. Once this is understood we can learn his essential character \& the correct interpretation of his illness.

The examination of the soul life of the child thus became the fulcrum of our science, \& a great many researches were dedicated to the study of the 1st years of life. There is so much material in this field which has never been touched nor probed that everyone is in a position to discover new \& valuable data which would prove of immense use in the study of human nature.

A method of preventing bad character traits was simultaneously developed, since our studies do not exist for their own sake but for the benefit of mankind. Quite without previous thought, our researches led into the field of pedagogy, to which we have contributed for years. Pedagogy is a veritable treasure-trove for anyone who wishes to experiment in it, \& apply to it what he has found valuable in the study of human nature, because pedagogy, like the science of human nature, is not to be got out of books, but must be acquired in the practical school of life.

We must identify ourselves with every manifestation of the soul life, live ourselves into it, accompany human beings through their joys \& their sorrows, in much the same way that a good painter paints into a portrait those characteristics which he has felt in the person of his subject. The science of human nature is to be thought of as an art which has many instruments at its disposal, an art closely related to all other arts, \& useful to them. In literature \& poetry, particularly, it is of exceptional import. Its 1st object must be to enlarge our knowledge of human beings, i.e., it must enable us all to acquire the possibility of fashioning for ourselves a better \& a riper psychic development.

1 of our great difficulties is that we very frequently find people extraordinarily sensitive on just the point of their understanding of human nature. There are very few human beings who do not consider themselves masters in this science even though they have had very few studies preparatory to their degree; \& there are even fewer such who would not feel offended if one would demand that they put their knowledge of mankind to the test. Those who really wish to know human nature are only those who have experienced the worth \& value of people through their own empathy, i.e., through the fact that they also have lived through psychic crises, or have been able to fully recognize them in others.

From this circumstance arises the problem \& the necessity of finding a precise tactic \& strategy, \& a technique in the application of our knowledge. For nothing is more hateful, \& nothing will be met with a more critical glance, than that we should brusquely throw into the face of an individual the stark facts which we have discovered in the exploration of his soul. It might be well to advise anyone who does not want to be hated that he be careful in this very connection. An excellent way to acquire a bad reputation is carelessly to make use of facts gained through a knowledge of human nature, \& misuse them, as for instance in the desire to show how much one knows or has guessed concerning the character of one's neighbor at a dinner. It is also dangerous to cite merely the basic truths of this science as finished products, for the edification of someone who does not understand the science as a whole. Even those who do understand the science would feel themselves insulted through such a procedure. We must repeat what we have already said: the science of human nature compels us to modesty. We may not announce the results of our experiments unnecessarily \& hastily. This would be germane ($=$ relevant) only to a little child who was anxious to parade himself \& show off all the things that he can do. It is hardly to be considered as an appropriate action for an adult.

We should advise the knower of the human soul 1st to test himself. He should never cast the results of his experiments which he has won in the service of mankind, into the face of an unwilling victim. He would only be making fresh difficulties for a still-growing science, \& actually defeat his purpose! We should then have to bear the onus of mistakes which had arisen from the unthinking enthusiasm of young explorers. It is better to remain careful \& mindful of the fact that we must have a complete whole in view before we can draw any conclusions about its parts. Such conclusions, furthermore, should be published only when we are quite certain that they are to someone's advantage. One can accomplish a great deal of mischief by asserting in a bad way, or at an improper moment, a correct conclusion concerning character.

We must now, before going on with our considerations, meet a certain objection which has already suggested itself to many readers. The foregoing assertion, that the style of life of the individual remains unchanged, will be incomprehensible to many, because an individual has so many experiences in life which change his attitude toward it. We must remember that any experience may have many interpretations. We will find that there are no 2 people who will draw the same conclusion from a similar experience. This accounts for the fact that our experiences do not always make us any cleverer. One learns to avoid some difficulties, it is true, \& acquires a philosophical attitude towards others, but the pattern along which one acts does not change as a result of this. We will see in the course of our further considerations that a human being always employs his experiences to the same end. Closer examination reveals that all his experiences to the same end. Closer examination reveals that all his experiences must fit into his style of life, into the mosaic of his life's pattern. It is proverbial that we fashion our own experiences. Everyone determines how \& what he will experience. In our daily life we observe people drawing whatever conclusions they desire from their experiences. There is the man who constantly makes a certain mistake. If you succeed in convincing him of his mistake, his reactions will be varied. He may conclude that, as a matter of fact, it was high time to avoid this mistake. This is a very rare conclusion. More probably he will object that he has been making this mistake so long that he is now no longer able to rid himself of the habit. Or he will blame his parents, or his education, for his mistake; he may complain that he has never had anyone who ever cared for him, or that he was very much petted, or that he was brutally treated, \& excuse his error with an alibi. Whatever excuse he makes, he betrays 1 thing, \& that is that he wishes to be excused of further responsibility. In this manner he has an apparent justification \& avoids all criticism of himself. He himself is never to blame. The reason he has never accomplished what he desired to do is always someone else's fault. What such individuals overlook is the fact that they themselves have made very few efforts to obviate their mistakes. They are far more anxious to remain in error, blaming their bad education with a certain fervor\footnote{very strong feelings about something, $=$  enthusiasm.}, for their faults. This is an effective alibi so long as they wish to have it so. The many possible interpretations of an experience \& the possibility of drawing various conclusions from any single one, enables us to understand why a person does not change his behavior pattern, but turns \& twists \& distorts his experiences until they fit it. The hardest thing for human beings to do is to know themselves \& to change themselves.

Any one who is not a master in the theory \& technique of the science of human nature would experience great difficulty in attempting to educate human beings to be better men. He would be operating entirely on the surface, \& would be drawn into the error of believing that because the external aspect of things had changed, he had accomplished something significant. Practical cases show us how little such technique will change an individual, \& how all the seeming changes are only apparent changes, valueless so long as the behavior pattern itself has not been modified.

The business of transforming a human being is not a simple process. It demands a certain optimism \& patience, \& above all the exclusion of all personal vanity, since the individual to be transformed is not in duty bound to be an object of another's vanity. The process of transformation, moreover, must be conducted in such a way that it seems justified for the one changed. We can easily understand that someone will refuse a dish which would otherwise be very tasty to him if it is not prepared \& offered to him in an appropriate manner.

The science of human nature has yet another aspect, which we may call its social aspect. Human beings would doubtless get along with each other better, \& would approach each other more closely, were they able to understand one another better. Under such circumstances it would be impossible for them to disappoint \& deceive each other. An enormous danger to society lies in this possibility of deception. This danger must be demonstrated to our fellow-workers, whom we are introducing to this study. They must be capable of making those upon whom they are practicing their science understand the value of the unknown \& unconscious forces working within us; in order to help them they must be cognizant of all the veiled, distorted, disguised tricks \& legerdemain\footnote{movements of your hand that are done with skill so that other people cannot see them, $=$ sleight of hand.}, of human behavior. To this end we must learn the science of human nature \& practice it consciously with its social purpose in view.

Who is best fitted to collect the material of this science, \& to practice it? We have already noted that it is impossible to practice this science only theoretically. It is not enough simply to know all the rules \& data. It is necessary to transmute our studies into practice, \& correlate them so that our eyes will acquire a sharper \& deeper view than has been previously possible. This is the real purpose of the theoretical side of the science of human nature. But we can make this science living only when we step out into life itself \& test \& utilize the theories which we have gained. There is an important reason for our question. In the course of our education we acquire too little knowledge of human nature -- \& much of what we learn is incorrect, because contemporary education is still unsuited to give us a valid knowledge of the human soul. Every child is left entirely to himself to evaluate his experiences properly, \& to develop himself beyond his classroom work. There is no tradition for the acquisition of a true knowledge of the human soul. The science of human nature finds itself today in the condition that chemistry occupied in the days of alchemy.

We have found that those who have not been torn out of their social relationships by the complicated muddle of our educational system are best adapted to pursue these researches in human nature. We are dealing with men \& women who are, in the last analysis, either optimists, or fighting pessimists who have not been driven to resignation by their pessimism. But contact with humanity, alone, is not enough. There must be experience as well. A real appreciation for human nature, in the face of our inadequate education today, will be gained only by 1 class of human beings. These are the contrite sinners, either those who have been in th whirlpool of psychic life, entangled in all its mistakes \& errors, \& saved themselves out of it, or those who have been close to it \& felt its currents touching them. Others naturally can learn it, especially when they have the gift of identification, the gift of empathy. The best knower of the human soul will be the one who has lived through passions himself. The contrite sinner seems as valuable a type in our day \& age as he was in the days when the great religions developed. He stands much higher than 1000 righteous ones. How does this happen? An individual who has lifted himself above the difficulties of life, extricated himself from the swamp of living, found power to profit by bad experiences, \& elevate himself as a result of them, understands the good \& the bad sides of life. No one can compare with him in this understanding, certainly not the righteous one.

When we find an individual whose behavior pattern has rendered him incapable of a happy life, there arises out of our knowledge of human nature the implicit duty to aid him in readjusting the false perspectives with which he wanders through his life. We must give him better perspectives, perspectives which are adapted to the community, which are more appropriate for the achievement of happiness in this existence. We must give him a new system of thought, indicate another pattern for him in which the social feeling \& the communal consciousness play a more important role. We do not purpose to make an ideal structure of his psychic life. A new viewpoint in itself is of great value to the perplexed, since from this he learns where he has gone astray in making his mistakes. According to our view the strict determinists who consider all human activity as the sequence of cause \& effect ae not far from wrong. Causality becomes a different causality, \& the results of experience acquire entirely new values, when the power of self-knowledge \& self-criticism is still alive, \& remains a living motif. The ability to know one's self becomes greater when one can determine the wellsprings\footnote{(literary) a supply or source of a particular quality, especially one that never ends.} of his activity \& the dynamics of his soul. Once he has understood this, he has become a different man \& can no longer escape the inevitable consequences of his knowledge.'' -- \cite[pp. 3--14]{Adler_human_nature}

\subsection{Book I: Human Behavior}

\subsubsection{The Soul}

\paragraph*{The Concept \& Premise of the Psychic Life.} ``We attribute a soul only to moving, living organisms. The soul stands in innate relationship to free motion. Those organisms which are strongly rooted have no necessity for a soul. How supernatural it would be to attribute emotions \& thoughts to a deeply rooted plant! To hold that a plant could, perhaps, accept pain which it could in no way escape, or that it could have a presentiment of that which it could not later avoid! To attribute reason \& free will to it at the same time that we considered it a foregone conclusion that the plant could not make any use of its will! Under such conditions the will \& the reason of the plant would of necessity remain sterile.

There is a strict corollary between movement \& psychic life. This constitutes the difference between plant \& animal. In the evolution of the psychic life, therefore, we must consider everything which is connected with movement. All the difficulties that are connected with change of place demand of the soul that it foresee, gather experiences, develop a memory, in order that the organism be better fitted for the business of life. We can ascertain then in the very beginning that the development of the psychic life is connected with movement, \& that the evolution \& progress of all those things which are accomplished by the soul are conditioned by the free movability of the organism. This movability stimulates, promotes, \& requires an always greater intensification of the psychic life. Imagine an individual to whom we have predicated every movement, \& we can conceive of his psychic life as at a standstill. ``Liberty alone breeds giants. Compulsion only kills \& destroys.'''' -- \cite[pp. 17--18]{Adler_human_nature}

\paragraph*{The Function of the Psychic Organ.} If we regard the function of the psychic organ from this standpoint, we will become aware that we are considering the evolution of a hereditary capability, an organ for offense \& defense with which the living organism responds according to the situation in which it find itself. The psychic life is a complex of aggressive \& security-finding activities whose final purpose is to guarantee the continued existence on this earth of the human organism, \& to enable him to securely accomplish his development. If we grant this premise, then further considerations grow out of it, which we deem necessary for a true conception of the soul. {\it We cannot imagine a psychic life which is isolated}. We can only imagine a psychic life bound up with its environment, which receives stimuli from the outside \& somehow answers them, which disposes of capabilities \& powers which are not fitted to secure the organism against the ravages of the outer world, or somehow bind it to these forces, in order to guarantee its life.

The relationships which suggest themselves from this are many. They have to do with the organism itself, the peculiarities of human beings, their physical nature, their assets \& their defects. These are entirely relative concepts, since it is entirely a relative matter whether a power or an organ shall be interpreted as asset or liability. These values can be given only by the situations in which the individual finds himself. It is very well-known that the foot of man is, in a sense, a degenerate hand. In an animal which had to climb this would be of decided disadvantage, but for a man, who must walk on the flat ground, it is of such advantage that no one would prefer a ``normal'' hand to a ``degenerate'' foot. As a matter of fact, in our personal lives, as in the lives of all peoples, inferiorities are not to be considered as the source of all evil. Only the situation can determine whether they are assets or liabilities. When we recall how variegated the relationships are between the cosmos, with its day \& night, its dominance of the sun, its movability of atoms, \& the psychic life of man, we realize how much these influenced affect our psychic life.'' -- \cite[pp. 18--19]{Adler_human_nature}

\paragraph{Purposiveness (Teleology) in the Psychic Life.} The 1st thing we can discover in the psychic trends is that the movements are directed toward a goal. We cannot, therefore, imagine the human soul as a sort of static whole. We can imagine it only as a complex of moving powers which are, however, the result of a unit cause, \& which strive for the consummation of a single goal. This teleology, this striving for a goal, is innate in the concept of adaptation. We can only imagine a psychic life with a goal towards which the movements which exist in the psychic life, are directed.

{\it The psychic life of man is determined by his goal}. No human being can think, feel, will, dream, without all these activities being determined, continued, modified, \& directed, toward an ever-present objective. This results, of itself, from the necessity of the organism to adapt itself \& respond to the environment. The bodily \& psychic phenomena of human life are based upon those fundamentals which we have demonstrated. We cannot conceive of a psychic evolution except within the pattern of an ever-present objective, which is determined in itself by the dynamics of life. The goal itself we may conceive as changing or as static.

On this basic all phenomena of the soul life may be conceived as preparations for some future situation. It seems hardly possible to recognize in the psychic organ, the soul, anything but a force acting toward a goal, \& Individual Psychology considers all the manifestations of the human soul as though they were directed toward a goal.

Knowing the goal of an individual, \& knowing, also, something of the world, we must understand what the movements \& expressions of his life mean, \& what their value is as a preparation for his goal. We must know also what type of movements this individual must make to reach his goal, just as we know what path a stone must take if we let it drop to earth, although the soul knows no natural law, for the ever-present goal is always in flux. If, however, one has an ever-present goal, then every psychic tendency must follow with a certain compulsion, as though there were a natural law which it obeyed. A law governing the psychic life exists, to be sure; but it is a man-made law. If anyone feels that the evidence is sufficient to warrant speaking of a psychic law he has been deceived by appearances, for when he believes that he has demonstrated the unchangeable nature \& determination of circumstance, he has stacked the cards. If a painter desires to paint a picture, one attributes to him all the attitudes which are germane to an individual who has that goal before his eyes. He will make all the necessary movements with inevitable consequence, just as though there were a natural law at work. But is he under any necessity to paint the picture?

There is a difference between movements in nature \& those in the human soul life. All the questions about free will hinge upon this important point. Nowadays it is believed that human will is not free. It is true that human will becomes bound as soon as it entangles itself or binds itself to a certain goal. \& since circumstances in the cosmic, animal, \& social relationships of man frequently determine this goal, it is not strange that the psychic life should often appear to us as though it were under the regency of unchangeable laws. But if a man, e.g., denies his relationships to society \& fights them, or refuses to adapt himself to the facts of life, then all these seeming laws are abrogated \& a new law steps in which is determined by the new goal. In the same manner, the law of communal life does not bind an individual who has become perplexed\footnote{confused and anxious because you are unable to understand something; showing this.} at life \& attempts to extirpate\footnote{{\it extirpate something} to destroy or get rid of something that is bad or not wanted.} his feeling for his fellowmen. \& so we must assert that a movement in the psychic life must arise of {\it necessity} only when an appropriate goal has been posited.

On the other hand, it is quite possible to deduce what the goal of an individual must be from his present activities. This is of the greater importance because so few people know exactly what their goal is. In actual practice it is the procedure which we must follow in order to gain a knowledge of mankind. Since movements may have many meanings this is not always so simple. We can however take many movements of an individual, compare them, \& graphically represent them; in this way we arrive at an understanding of a human being by connecting 2 points wherein a definite attitude of the psychic life was expressed, in which the difference in time is noted by a curve. This mechanism is utilized to obtain a unified impression of a whole life. An example will serve to illustrate how we may rediscover a childhood pattern in an adult (see also \cite{Giang_after_childhood}), in all its astonishing similarity.

A certain 30-year-old man of extraordinarily aggressive character, who achieved success \& honor despite difficulties in his development, comes to the physician in circumstances of greatest depression, \& complains that he has no desire to work or to live. He explains that he is about to be engaged, but that he looks at the future with great mistrust. he is plagued by a strong jealousy, \& there is great danger that his engagement will be broken. The facts in the case, which he brings up to prove his point, are not very convincing. Since no one can reproach the young lady, the obvious distrust which he shows lays him open to suspicion. He is 1 of those many men who approach another individual, feel themselves attracted, but immediately assume an aggressive attitude, which destroys the very contact which they seek to establish.

Now let us plot the graph of this man's style of life as we have indicated above, by taking out 1 event in his life \& seeking to join it up with his present attitude. According to our experience, we usually demand the 1st childhood remembrance, even though we know that it is not always possible to test the value of this remembrance objectively. This was his 1st childhood remembrance: he was at the market place with his mother \& his younger brother. Because of the turmoil \& crowding, his mother took him, the elder brother, on her arm. As she noticed her error, she put him down again \& took the younger child up, leaving our patient to run around crushed by the crowd, very much perplexed. At that time he was 4 years old. In the recital of this remembrance, we hear the identical notes that we surmised\footnote{to guess or suppose something using the evidence you have, without definitely knowing, $=$ conjecture.} in a description of his present complaint. He is not certain of being the favored one, \& he cannot bear to think that another might be favored. Once the connection is made clear to him, our patient, very much astonished, sees the relationship immediately.

The goal toward which every human being's actions are directed, is determined by those influences \& those impressions which the environments gives to the child. The ideal state, i.e., the goal, of each human being, is probably formed in the 1st months of his life. Even at this time certain sensations play a role which evoke a response of joy or discomfort in the child. Here the 1st traces of a philosophy of life come to the surface, although expressed in the most primitive fashion. The fundamental factors which influence the soul life are fixed at the time when the child is still an infant. Upon these foundations a superstructure is built, which may be modified, influenced, transformed. A multiplicity of influences soon forces the child into a definite attitude towards life, \& conditions his particular type of response to the problems which life gives.

Investigators who believe the characteristics of an adult are noticeable in his infancy are not far wrong; this accounts for the fact that character is often considered hereditary. But the concept that character \& personality are inherited from one's parents is universally harmful because it hinders the educator in his task \& cramps his confidence. The real reason for assuming that character is inherited lies elsewhere. This evasion enables anyone who has the task of education to escape his responsibilities by the simple gesture of blaming heredity for the pupil's failures. This, of course, is quite contrary to the purpose of education.

Our civilization makes important contributions to the determination of the goal. It sets boundaries against which a child batters himself until he finds a way to the fulfillment of his wishes which promises both security \& adaptation to life. How much security the child demands in relation to the actualities of our culture may be learned early in his life. By security we do not consider only security from danger; we refer to that further coefficient of safety which guarantees the continued existence of the human organism under optimum circumstances, in very much the same way that we speak of the ``coefficient of safety'' in the operation of a well-planned machine. A child acquires this coefficient of safety by demanding a ``plus'' factor of safety greater than is necessary merely for the satisfaction of his given instincts, greater than would be necessary for a quiet development. Thus arises a new movement in his soul life. This new movement is, very plainly, a tendency toward domination \& superiority. Like the grownup, a child wants to out-distance all his rivals. He strains for a superiority which will vouchsafe him that security \& adaptation which are synonymous with the goal he has previously set for himself. There thus wells up a certain unrest in his psychic life which becomes markedly accentuated as time goes on. Suppose now that the world requires a more intensive response. If in this time of need the child does not believe in his own ability to overcome his difficulties we will notice his strenuous\footnote{1. needing great effort and energy, $=$ {\it arduous}; 2. determined and showing great energy.} evasions \& complicated alibis, which serve only to make the underlying thirst for glory the more evident.

In these circumstances the immediate goal frequently becomes the evasion of all greater difficulties. This type recoils from difficulties or wriggles out of them in order temporarily to evade the demands of life. We must understand that the reactions of the human soul are not final \& absolute: every response is but a partial response, valid temporarily, but in no way to be considered a final solution of a problem. In the development of the child-soul especially, are we reminded that we are dealing with temporary crystallizations of the goal idea. We cannot apply the same criteria to the child soul that we use to measure the adult psyche. In the case of the child we must look farther \& question the objective to which the energies \& activities working themselves out in his life, would eventually lead him. Could we translate ourselves into his soul, we could understand how each expression of his power was appropriate to the ideal which he had created for himself as the crystallization of a final adaptation of life. We must assume the child's point of view if we want to know why he acts as he does. The feeling-tone connected with his point of view directs the child in various ways. There is the way of optimism, in which the child is confident of easily solving the problems which he meets. Under these circumstances he will grow up with the characteristics of an individual who considers the tasks of life eminently within his power. In his case we see the development of courage, openness, frankness, responsibility, industry, \& the like. The opposite of this is the development of pessimism. Imagine the goal of the child who is not confident of being able to solve his problems! How dismal the world must appear to such a child! Here we find timidity, introspectiveness, distrust, \& all those other characteristics \& traits with which the weakling seeks to defend himself. His goal will lie beyond the boundaries of the attainable, but far behind the fighting front of life.'' -- \cite[pp. 19--25]{Adler_human_nature}

\subsubsection{Social Aspects of The Psychic Life}
``In order to know how a man thinks, we have to examine his relationship to his fellowmen. The relation of man to man is determined on the 1 hand by the very nature of the cosmos, \& is thus subject to change. On the other hand, it is determined by fixed institutions such as political traditions in the community or nation. We cannot comprehend the psychic activities without at the same time understanding these social relationships.'' -- \cite[p. 26]{Adler_human_nature}

\paragraph{The Absolute Truth.} ``Man's soul cannot act as a free agent because the necessity of solving the problems which constantly arise, determines the line of its activity. These problems are indivisibly bound up with the logic of man's communal life; the essential conditions of this group-existence influence the individual, yet the facts of the communal life seldom allow themselves to be influenced by the individual, \& then only to a certain degree. The existing conditions of our communal life however cannot yet be considered final; they are too numerous, \& are subject to much change \& transformation. We are hardly in a position to completely illuminate the dark recesses of the problem of the psychic life, \& understand it thoroughly, since we cannot escape from the meshes of our own relationships.

Our sole recourse\footnote{the fact of having to, or being able to, use something that can provide help in a difficult situation.} in this quandary is to assume the logic of our group life as it exists on this planet as though it were an ultimate absolute truth which we could approach step by step after the conquest of mistakes \& errors arising from our incomplete organization \& our limited capabilities as human beings.

An important aspect of our considerations lies in the materialistic stratification of society which {\sc Marx} \& {\sc Engels} have described. According to their teaching, the economic basis, the technical form in which a people lives, determines the ``ideal, logical superstructure,'' the thinking \& behavior of individuals. our conception of the ``logic of human communal life,'' of the ``absolute truth,'' is in part an agreement with those concepts. History, \& our insight into the life of the individual (i.e., our Individual Psychology), have taught us however that it is occasionally expedient for the individual to make a mistaken response to the demands of an economic situation. In attempting to evade the economic situation, he may become inextricably entangled in the meshes of his own mistaken reactions. Our way to the absolute truth will lead over countless errors of this kind.'' -- \cite[pp. 26--27]{Adler_human_nature}

\paragraph{The Need for Communal Life.} ``The rules of communal life are really just as self-explanatory as the laws of climate, which compel certain measures for the protection against cold, for the building of houses, \& the like. The compulsion toward the community \& communal life exists in institutions whose forms we need not entirely understand, as in religion, where the sanctification of communal formulae serves as a bond between members of the community. If the conditions of our life are determined in the 1st place by cosmic influences, they are also further conditioned by the social \& communal life of human beings, \& by the laws \& regulations which arise spontaneously from the communal life. The communal need regulates all relationships between men. The communal life of man antedates the individual life of man. In the history of human civilization no form of life whose foundations were not laid communally can be found. No human being ever appeared except in a community of human beings. This is very easily explained. The whole animal kingdom demonstrates the fundamental law that species whose members are incapable of facing the battle for self-preservation, gather new strength through herd life.

The herd instinct has served humanity to this end: the most notable instrument which it has developed against the rigors of the environment is the soul, whose very essence is permeated with the necessity of communal life. Darwin long ago drew attention to the fact that one never found weak animals living alone; we are forced to consider man among these weak animals, because he likewise is not strong enough to live alone. He can offer only little resistance to nature. He must supplement his feeble body with many artificial machines in order to continue his existence upon this planet. Imagine a man alone, \& without an instrument of culture, in a primitive forest! He would be more inadequate than any other living organism. He has not the speed nor the power of other animals. He has not the teeth of the carnivore, nor the sense of hearing, nor the sharp eyes, which are necessary in the battle for existence. Man needs an extensive apparatus to guarantee his existence. His nutrition, his characteristics, \& his style of life, demand an intensive program of protection.

Now we can understand why a human being can maintain his existence only when he has placed himself under particularly favorable conditions. These favorable conditions have been offered him by the social life. Social life became a necessity, because through the community \& the division of labor in which every individual subordinated himself to the group, the species was enabled to continue its existence. Division of labor (which means essentially, civilization) alone is capable of making available to mankind those instruments of offense \& defense which are responsible for all its possessions. Only after he learned the division of labor did man learn how to assert himself. Consider the difficulties of childbirth \& the extraordinary precautions which are necessary for keeping a child alive during its 1st days! This care \& precaution could be exercised only where there was such a division of labor. Think of the number of sicknesses \& infirmities to which the human flesh is heir, particularly in its infancy, \& you have some conception of the unusual amount of care which human life demands, some comprehension of the necessity of a social life! The community is the best guarantee of the continued existence of human beings!'' -- \cite[pp. 27--29]{Adler_human_nature}

\paragraph{Security \& Adaption.} ``From the previous expositions we conclude that man, seen from the standpoint of nature, is an inferior organism. This feeling of his inferiority \& insecurity is constantly present in his consciousness. It acts as an ever-present stimulus to the discovery of a better way \& a finer technique in adapting himself to nature. This stimulus forces him to seek situations in which the disadvantages of the human status in the scheme of life will be obviated \& minimized. At this point arises the necessity for a psychic organ which can effect the processes of adaptation \& security. It would have been much harder to have made an organism out of the primitive \& original man-animal, which would be capable of fighting nature to a standstill, by the addition of anatomic defenses such as horns, claws, or teeth. The psychic organ alone could render 1st-aid quickly, \& compensate for the organic deficiencies of man. The very stimulation growing from an uninterrupted feeling of inadequacy, developed foresight \& precaution in man, \& caused his soul to develop to its present state, an organ of thinking, feeling, \& acting. Since society has played an essential role in the process of adaptation, the psychic organ must reckon from the very beginning with the conditions of communal life. All its faculties are developed upon an identic basis: the logic of communal life.

In the origin of logic with its innate necessity for universal applicability we should doubtless find the next step in the development of man's soul. Only that which is universally useful is logical. Another instrument of the communal life is to be found in articulate speech, that miracle which distinguishes man from all other animals. The phenomenon of speech, whose forms clearly indicate its social origins, cannot be divorced from this same concept of universal usefulness. Speech would be absolutely unnecessary to an individual organism living alone. Speech is justified only in a community; it is a product of communal life, a bond between the individuals of the community. Proof for the correctness of this assumption is to be found in those individuals who have grown up under circumstances which have made contact with other human beings difficult or impossible. Some of these individuals have often evaded all connections with society for personal reasons, others are the victims of circumstance. In each case, they suffer from speech defects or difficulties \& never acquire the talent for learning foreign languages. It is as though this bond can be fashioned \& retained only when the contact with humanity is secure.

Speech has an enormously important value in the development of the human soul. Logical thinking is possible only with the premise of speech, which gives us the possibility of building up concepts \& of understanding differences in values; the fashioning of concepts is not a private matter, but concerns all society. Our very thoughts \& emotions are conceivable only when we premise their universal utility; our joy in the beautiful is based on the fact that the recognition, understanding, \& feeling for the beautiful are universal. It follows that thoughts \& concepts, like reason, understanding, logic, ethics, \& aesthetics, have their origin in the social life of man; they are at the same time bonds between individuals whose purpose is to prevent the distintegration of civilization.

Desire \& will may also be understood as aspects of man's situation as an individual. Will is but a tendency in the service of the feeling of inadequacy, an instrument for the attainment of the feeling of a satisfactory adaptation. To ``will'' means to feel this tendency, \& to enter into its movement. Every voluntary act begins with a feeling of inadequacy, whose resolution proceeds toward a condition of satisfaction, of repose, \& totality.'' -- \cite[pp. 29--31]{Adler_human_nature}

\paragraph{The Social Feeling.} \footnote{The word ``Gemeinschaftsgef\"uhl'' for which no adequate English equivalent exists, has been rendered as ``social feeling'' through the book. ``Gemeinschaftsgef\"uhl'' however connotes the sense of human solidarity, the connectedness of man to man in a cosmic relationship. Whenever the brief phrase ``social feeling'' has been used therefore, the wider connotation of a ``sense of fellowship in the human community'' should be borne in mind.}``We may now understand that any rules that serve to secure the existence of mankind, such as legal codes, totem \& taboo, superstition, or education, must be governed by the concept of the community \& be appropriate to it. We have already examined this idea in the case of religion, \& we find adaptation to the community is the most important function of the psychic organ, in the case of the individual, as in the case of society. What we call justice \& righteousness, \& consider most valuable in the human character, is essentially nothing more than the fulfillment of the conditions which arise in the social needs of mankind. These conditions give shape to the soul \& direct its activity; responsibility, loyalty, frankness, love of truth, \& the like are virtues which have been set up \& retained only by the universally valid principle of communal life. We can judge a character as bad or good only from the standpoint of society. Character, just as any achievement in science, politics or art, becomes noteworthy only when it has proven its universal value. The criteria by which we can measure an individual are determined by his value to mankind in general. We compare an individual with the ideal picture of a fellowman, a man who overcomes the tasks \& difficulties which lie before him, in a way which is useful to society in general, a man who has developed his feeling to a high degree. According to the expression of Furtm\"uller, he is one ``who plays the game of life according to the laws of society.'' In the course of our demonstrations it will become increasingly evident that no adequate man can grow up without cultivating a deep sense of his fellowship in humanity \& practicing the art of being a human being.'' -- \cite[pp. 31--32]{Adler_human_nature}

\subsubsection{Child \& Society}
``Society exacts certain obligation of us which influence the norms \& forms of our life, as well as the development of our mind. Society has an organic basis. The point of tangency between the individual \& society may be found in the fact of man's bisexuality. Not in the isolation of man \& woman, but in the community of man \& wife, does he satisfy the impulse of life \& achieve security \& guarantee his happiness. When we observe the slow development of a child, we may be certain no evolution of human life is possible without the presence of a protecting community. The various obligations of life carry in themselves the necessity for a division of labor which not only does not separate human beings, but strengthens their bonds.

Everyone must help his neighbor. Everyone must feel himself bound to his fellow man. The vital relationships of man to man have originated thus. We must now discuss in more detail some of these relationships which greet a child upon his birth.'' -- \cite[p. 33]{Adler_human_nature}

\paragraph{The Situation of the Infant.} ``Every child, dependent as he is on the help of the community, finds himself face to face with a world that gives \& takes \cite{Grant_give_take,Grant_give_take_VN}, that expects adaptation \& satisfies life. His instincts are baffled in their fulfillment by obstacles whose conquest gives him pain. He realizes at an early age that there are other human beings who are able to satisfy their urges more completely, \& are better prepared to live. His soul is born, one might say, in those situations of childhood which demand an organ of integration, whose function is to make a normal life possible. This the soul accomplishes by evaluating each situation \& directing the organism to the next one, with the maximum satisfaction of instincts \& the least possible friction. In this way he learns to over-value the size \& stature which enable one to open a door, or the ability to move heavy objects, or the right of others to give commands \& claim obedience to them. A desire to grow, to become as strong or even stronger than all others, arises in his soul. To dominate those who are gathered about him because his chief purpose in life, since his elders, though they act as if he were inferior, are obligated to him because of his very weakness. 2 possibilities of action lie open to him. On the 1 hand, to continue activities \& methods which he realizes the adults use, \& on the other hand to demonstrate his weakness, which is felt by these same adults as an inexorable demand for their help. We shall continually find this branching of psychic tendencies in children.

The formation of types begins at this early period. Whereas some children develop in the direction of the acquisition of power \& the selection of a courageous technique which results in their recognition, others seem to speculate on their own weaknesses, \& attempt to demonstrate it in the most varied forms. One need but recall the attitude, the expression, \& the bearing of individual children, to find individuals who fit into 1 group or the other. Every type has a meaning only as we understand its relationship to the environment. Reflections of environment are usually to be found in the behavior of any child.

The basis of educability lies in the striving of the child to compensate for his weaknesses. 1000 talents \& capabilities arise from the stimulus of inadequacy. Now the situations of individual children are extraordinarily different. In the 1 case we are dealing with an environment which is hostile to the child \& which gives him the impression that the whole world is an enemy country. The incomplete perspectives of child thought-processes explain this impression. If his education does not forestall this fallacy, the soul of such a child may develop so that in later years he will act always {\it as if} the world really were an enemy country. His impression of hostility will become accentuated as soon as he meets with greater difficulties in life. This occurs frequently in the case of children with inferior organ-systems. Such children greet their environment with an attitude entirely different from those who come into the world with relatively normal organs. Organic inferiority can express itself in difficulties of motion, in inadequacies of single organs, or in weakened resistance of the entire organism, which results in frequent sickness.

Difficulties in facing the world are not necessarily caused only be deficiencies of the childish organism. The unreasonable demands made on a child by a foolish environment (or the unfortunate manner in which these demands are presented to him) are comparable to actual difficulties in the environment. A child who desires to adapt himself to his environment suddenly finds difficulties lying in his way, especially where he grows up in an environment which has itself lost its courage \& is imbued with a pessimism only too quickly transferred to the child.'' -- \cite[pp. 33--35]{Adler_human_nature}

\paragraph{The Influence of Difficulties.} ``In view of the obstacles which approach every child from countless angles, it is not to be wondered at that his response is not always adequate. His psychic habits have but a short time to develop, \& the child finds himself under the necessity of orientating himself to the unchangeable conditions of actuality, while his technique of adjustment is still immature. Whenever we consider any number of mistaken responses to the environment we find ourselves dealing with constant developmental attempts on the part of the soul to make a correct response \& to progress throughout life as in a continual experiment. The thing which we particularly see in the expression of the child's behavior pattern is the type of response which an adolescent gives in a definite situation, in the course of his maturation. His response attitude gives us an insight into his soul. We must at the same time take cognizance of the fact that the responses of any individual, just as those of society, are not to be judged according to a pattern.

The obstacles a child meets with in the development of his soul usually result in the stunting or distortion of his social feeling. They may be divided into those which arise out of defects in his physical environment; such as originate in abnormal relationships in his economic, social, racial, or family circumstances; \& further, into those which arise out of defect in his bodily organs. Our civilization is a culture which is based upon the health \& adequacy of fully-developed organs. Therefore, a child whose important organs suffer defects is at a disadvantage in solving the problems of life. Children who learn to walk late, or who have difficulties of any kind in locomotion, or those who learn to speak late, who are clumsy for a long time because the development of their cerebral activity takes longer than in the case of the usual child, belong in this class. We all know how such children are constantly bumping themselves, are clumsy \& slow, \& carry with them a burden of bodily \& psychic sorrows. They are obviously not tenderly touched by a world which was not appropriately fashioned for them. Difficulties which arise out of some such inadequate development are many. Of course there is always the possibility that in the course of time a compensation is established automatically without a scar remaining, if the bitterness of the psychic need has not, in the meantime, developed in the child an attitude of despair which is felt in his later life; such a state of affairs may be complicated, in addition, by economic helplessness. It is easy to understand that the fixed laws of human society are but poorly comprehended by defectively equipped children. They look with suspicion \& mistrust at the opportunities which they see developing around them, \& have the tendency to isolate themselves \& evade their tasks. They have a peculiarly sharp sense of life's hostility, \& they unconsciously exaggerate it. Their interest in the bitterness of life is much greater than in its brighter side. For the most part, they overrate both, so that theirs is a lifelong attitude of belligerency\footnote{unfriendly and aggressive, $=$ {\it hostile}.}. They demand that an extraordinary amount of attention be paid to them, \& of course they think far more of themselves than of others. They conceive of the necessary obligations of life more as difficulties than as stimuli. Soon a gulf, which is continually widened because of their hostility to their fellows, builds itself between them \& their environment. Now they approach every experience with an exaggerated cautiousness, removing themselves farther \& farther from the truth \& actuality with every contact, \& succeed only in continually making fresh difficulties for themselves.

Similar difficulties may arise when the normal tenderness of parents toward their children is not manifested to a proper degree. Whenever this occurs serious consequences for the development of the child ensue. The child's attitude becomes so fixed that he cannot recognize love nor make the proper use of it, because his instincts for tenderness have never been developed. It will be difficult to mobilize a child who has grown up in a family where there has never been a proper development of the feeling of tenderness, to the expression of any kind of tenderness. His whole attitude in life will be a gesture of escape, an evasion of all love \& all tenderness. The identical effect may be produced by unthinking parents, educators, or other adults, who teach children that love \& tenderness are improper, ridiculous, or unmanly, by impression some pernicious\footnote{having a very harmful effect on somebody/something, especially in a way that is not easily noticed.} motto upon them. It is not so seldom that we find that a child is taught that tenderness is ridiculous. This is especially the case among those children who have often been ridiculed. Such children are veritably afraid of showing emotions or feelings because they feel their tendency to show love toward others is ridiculous \& unmanly. They fight against normal tenderness as though it were an instrument to enslave or degrade them. Thus boundaries to the love life may be set in early childhood. After a brutal education in which all tenderness is damped up \& repressed, a child withdraws from the circle of his environment, \& loses, little by little, contacts which are of utmost important to his soul. Sometimes a single person in the environment offers an opportunity of concord; when this happens the child joins himself to his friend in a very deep relation. This accounts for the individuals who grow up with social relationships directed to but a single person, whose social tendencies can never be stretched to include more than 1 other human being. The example of the boy who felt himself neglected when he noticed that his mother was tender only to his younger brother, \& therefore wandered up \& down through life trying to find the warmth \& affection which he had missed from earliest childhood, is a case in point, which demonstrates the difficulties such a person may find in life. It goes without saying that the education of such individuals proceeds only under pressure.

Education accompanied by too much tenderness is as pernicious as education which proceeds without it. A pampered child, as much as a hated one, labors under great difficulties. Where it is instituted, a desire for tenderness arises which grows beyond all boundaries; the result is that a petted child binds himself to 1 or more persons \& refuses to allow himself to be detached. The value of tenderness becomes so accentuated by various mistaken experiences that the child concludes that his own love enforces certain implicit responsibilities on his grown-ups. This is easily accomplished: the child says to his parents, ``Because I love you, you must do this or that.'' It is this type of social dogma which frequently grows up within the circle of the family. No sooner does the child recognize a tendency like this on the part of others than he increases his own tenderness in order to make them more dependent upon him. The flaming up of such tenderness to 1 particular person in the family must always be kept in mind. There is no doubt that the future of a child is influenced injuriously by such training. His life becomes involves in the struggle to hold the tenderness of others by fair means or foul. To accomplish this he dares to use every means which lies at hand; he may attempt the subjugation of his rival, a brother or sister, or occupy himself with tale-bearing against them. Such a child will actually incite his brothers to misdeeds in order that he may be able to sun himself in the love of his parents in relative glory \& righteousness. He applies a definite social pressure to his parents in order to fix their attention on himself. To do this he will leave no stone unturned until he occupies the limelight\footnote{the centre of public attention.} \& has achieved more importance than anyone else. He is lazy, or bad, for the sole purpose of giving his parents the task of busying themselves more with him; he becomes a model child, because he considers the attention of others a sort of reward.

After the discussion of these mechanisms we may conclude that anything may become a means to an end, once the pattern of psychic activity is fixed. The child may develop himself in an evil direction, in order to arrive at his goal, or he may become a model child, with the same goal in view. One can often observe how 1 of several children seeks the limelight through particular unruliness, while another, being shrewder, attains the same goal through particular virtue.

With the petted children we may also group those who have had every difficulty removed from their path, whose capabilities have been belittled in a friendly way. They have never had an opportunity to meet responsibilities. Such children have all been denied every opportunity to make those preparations which are so necessary for future life. They are not prepared to make contacts with anyone who is willing to join with them, \& are certainly not capable of making contacts with others, who, as a result of difficulties \& errors in their own childhood, through obstacles in the way of human contacts. Such children are utterly unprepared for life, because they have never had an opportunity to practice the conquest of difficulties. As soon as they step out of the hothouse atmosphere of the tiny kingdom of their home, they suffer defeats almost of necessity, for the reason that they cannot find any human being willing to assume the duties \& responsibilities which they expect at the hands of their petting educators, nor in the degree to which they are accustomed.

All the phenomena of this type have 1 thing in common: they tend to the greater or lesser isolation of the child. Children whose gastro-intestinal\footnote{{\it  gastrointestinal0} [a]: of or related to the stomach and intestines.} tracts are defective assume a special attitude towards nutrition, \& as a result go through an entirely different developmental process from children who are normal in this respect. Children with defective organs have a peculiar style of life which may eventually drive them into isolation. There are other children who do not clearly understand their connection with the environment, \& actually try to avoid it. They cannot find a comrade, hold themselves distinct from the games of their companions, \& either envious of their fellows, or, despising the play of children of the same age, busy themselves in a shut-in preoccupation with their own private games. Isolation also threatens children who grow up under the pressure of an education marked by great strictness. Life will not appear in a favorable light to them, because they are expecting bad impressions on every hand. Either they have the impression that they must be tolerant of all difficulties \& take up their sorrows in a humble way, or they feel like champions, ready to take up the battle with the environment they have always found hostile. Such children feel that life \& its tasks are inordinately difficult; it is not hard to understand how such a child will be busied for the most part with the defense of his personal boundary lines, lest he suffer some defeat of his personality. We may expect him constantly to retain before his eyes an unfriendly picture of the outer world. Burdened by an exaggerated cautiousness, he develops a tendency to evade all greater difficulties, rather than to lay himself open to the dangers of a possible defeat.

A further common characteristic of these pampered children, which is a sign of their inadequately developed social feeling, is the fact that they think more of themselves than of others. In this trait one sees clearly their whole development toward a pessimistic philosophy of the world. It is impossible for them to be happy unless they find a solution for their false behavior pattern.'' -- \cite[pp. 36--42]{Adler_human_nature}

\paragraph{Man as a Social Being.} ``We have been at some length to show how we can understand the personality of the individual only when we see him in his context, \& judge him in his particular situation in the world. By situation we mean his place in the cosmos, \& his attitude toward his environment \& the problems of life, such as the challenges of occupation, contact, \& union with his fellow men, which are inherent in his being. In this way we have been able to determine that the impressions which storm in upon every individual from the earliest days of his infancy influence his attitude throughout his whole life. One can determine how a child stands in relation to life a few months after his birth. It is impossible to confuse the behavior of 2 infants after these months because they have already demonstrated a well-defined pattern which becomes the clearer as they develop. Variations from the pattern do not occur. The child's psychic activity becomes increasingly permeated by his social relationships. The 1st evidence of the inborn social feeling unfolds in his early search for tenderness, which leads him to seek the proximity\footnote{the state of being near somebody/something in distance or time.} of adults. The child's love life is always directed towards others, not, as Freud would say, upon his own body. According to the person, these erotic strivings vary in their intensity \& manifestation. In children who are more than 2 years old these differences may be demonstrated in their speech. Only under the stress of the most severe psychopathological degeneration does the social feeling which has become firmly based in the soul of every child at this time, forsake him. This social feeling remains throughout life, changed, colored, circumscribed in some cases, enlarged \& broadened in others until it touches not only the members of his own family, but also his clan, his nation, \& finally, the whole of humanity. It is possible that it may extend beyond these boundaries \& express itself towards animals, plants, lifeless objects, or finally towards the whole cosmos. An understanding of the necessity for dealing with man as a social being is the essential conclusion of our studies. Once we have grasped this, we have gained an important adjunct to the understanding of man's behavior.'' -- \cite[pp. 42--43]{Adler_human_nature}

\subsubsection{The World We Live In}

\paragraph{The Structure of our Cosmos.} ``Owning to the fact that every human being must make an adjustment to his environment, his psychic mechanism has the faculty of taking up impressions from the outer world. In addition, the psychic mechanism pursues a definite aim according to a definite interpretation of the world, \& along the lines of an ideal behavior pattern which dates from early childhood. Although we cannot express this cosmic interpretation \& this goal in a definite \& exact term, we can nevertheless describe it as an ever present aura, \& as always in contradistinction to the feeling of inadequacy. Psychic movements can occur only when they have an innate goal. The construction of a goal, as we know, premises the capacity for change, \& a certain freedom of movement. The spiritual enrichment which results from freedom of movement is not to be undervalued. A child who raises himself from the ground for the 1st time comes into an entirely new world, \& in that second he somehow senses a hostile atmosphere. In his 1st attempt at movement, \& particularly in rising to his feet \& learning to walk, he experiences various degrees of difficulty, which may either strengthen or destroy his hope for the future. Impressions which grown-ups might consider unimportant or commonplace, may have an enormous influence on the child's soul \& entirely shape his impression of the world in which he lives. In this way children who have had difficulties in locomotion construct an ideal for themselves which is permeated with violent \& hasty movements; we can discover this ideal by asking them what their favorite games are, or what they would like to do when they are grown. Usually such children answer that they desire to be automobile drivers, locomotive engineers, or the like -- thus signifying clearly their desire to overcome every difficulty which hinders their freedom of movement. The goal of their life is to attain a point at which their feeling of inferiority \& their sense of handicap is entirely removed by perfect freedom of motion. It is readily understood that such a sense of handicap can originate easily in the soul of a child who has developed slowly, or has encountered much sickness in his life. Similarly, children who have come into the world with defects in their eyes attempt to translate the entire world into more intensive visual concepts. Children who have auditory defects show an intense interest for certain tones which seem to sound more pleasant to them; in short, they become ``musical.''

Of all the organs with which a child attempts the conquest of the world the sense organs are the most important in the determination of the essential relationships to the world in which he lives. It is through the sense organs that one constructs one's cosmic picture. Above all, it is the eye which approaches the environment, it is the visual world predominantly which forces itself upon the attention of every human being \& gives him the main data in his experiences. The visual picture of the world in which we live has an incomparable significance in that it deals with unchanging, lasting bases, in contrast to the other sense organs, the ear, the nose, the tongue, \& the skin, which are sensitive solely to temporary stimuli. There are, however, individuals in whom the ear is the predominant organ. Here a psychic fund of information based more particularly upon acoustic values is created. In this case the soul might be said to have a predominantly auditory constellation. Less frequently we find individuals in whom motor activity is predominant. A predominance of interest for olfactory or gustatory stimuli determines another type, \& of these, the 1st type, which is more sensitive to smell, is under a relative disadvantage in our civilization. Then there are a number of children in whom the musculature\footnote{the system of muscles in the body or part of the body.} plays the leading role. This group comes into the world characterized with a greater restlessness, which forces them to constant movement in childhood, \& to greater activity in maturity. Such individuals are interested only in such activities in which the functioning muscles play the chief role. They exhibit their activity even during sleep, as anyone can prove for himself by observing them restlessly tossing about in their beds. We must class those ``fidgety'' children whose restlessness is often considered a vice, in this category. In general we can say that a child who does not approach the world with heightened interest in someone organ or organ group, whether these be his sense organs or his locomotive apparatus, hardly exists. From the impressions which his more sensitive organ gathers from the world each child constructs a picture of the world in which he lives. We can therefore understand a human being only when we know with what sense organs or organ-systems he approaches the world, because all his relationships are colored by this fact; his actions \& reactions gain their value from our knowledge of the influence which his organic defects have had upon the constellation of his cosmic picture in his childhood, \& thus upon his later development.'' -- \cite[pp. 44--46]{Adler_human_nature}

\paragraph{Elements in the Development of the Cosmic Picture.} ``The ever present goal which determines all our activity influences also the choice, intensity, \& activity of those particular psychic faculties which serve to give shape \& meaning to the cosmic picture. This explains the fact that each of us experiences a very specific segment of life, or of a particular event, or, indeed of the entire world in which we live. Each of us values only that which is appropriate to his goal. A real understanding of the behavior of any human being is impossible without a clear comprehension of the secret goal which he is pursuing; nor can we evaluate every aspect of his behavior until we know that his whole activity has been influenced by this goal.

\subparagraph{A. Perception.} The impressions \& stimuli which arise in the outer world are transmitted by means of the sense organs to the brain, where certain traces of them may be retained. On these vestiges are built the world of imagination \& the world of memory. But a perception is never to be compared with a photographic image because something of the peculiar \& individual quality of the person who perceives is inextricably bound up with it. One does not perceive everything that one sees. No 2 human beings react in quite the same way to the identical picture; if we ask them what they have perceived they will give very diverse answers. A child perceives only that in his environment which fits into a behavior pattern previously determined by a variety of causes. The perceptions of children whose visual desire is especially well developed have a predominantly visual character. The majority of mankind is probably visual-minded. Others fill in the mosaic\footnote{a picture or pattern made by placing together small pieces of glass, stone, etc. of different color.} picture of the world which they have created for themselves with predominantly auditory perceptions. These perceptions need not be strictly identical with actuality. Everyone is capable of reconfiguring \& rearranging his contacts with the outer world to fit his life pattern. The individuality \& uniqueness of a human being consists in {\it what} he perceives \& {\it how} he perceives. Perception is more than a simple physical phenomenon; it is a psychic function from which we may draw the most far going conclusions concerning the inner life.

\subparagraph{B. Memory.} The development of the soul is intimately related to the necessity for activity, upon the basis of the facts of perception. The soul is innately related to the motility of the human organism, \& its activities are determined by the goal \& purpose of this motility. It is necessary for man to collect \& arrange his stimuli \& relationships to the world in which he lives, \& his soul, as an organ of adaptation, must develop all those faculties which play a role in his defense \& are otherwise active in maintaining his existence.

It is clear now that the individual response of the soul to the problems of life leaves traces in the structure of the soul. The functions of memory \& evaluation are dominated by the necessity for adaptation. Without memories it would be impossible to exercise any precaution for the future. We may deduce that all recollections have an unconscious purpose within themselves. They are not fortuitous phenomena, but speak clearly the language of encouragement or of warning. There are no indifferent or nonsensical recollections. One can evaluate a recollection only when one is certain about the goal \& purpose which it subserves. It is not important to know {\it why} one remembers certain things \& forgets others. We remember those events whose recollection is important for a specific psychic tendency, because these recollections further an important underlying movement. We forget likewise all those events which detract from the fulfillment of a plan. We find thus that memory, too, is subordinated to the business of purposive adaptation, \& that every memory is dominated by the goal idea which directs the personality-as-a-whole. A lasting recollection, {\it even though it is a false one}, as is often the case in childhood, where memories are frequently surcharged with a 1-sided prejudice, may be transposed out of the realm of the conscious, \& appear as an attitude, or as an emotional tone, or even as a philosophic point of view, if this be necessary for the attainment of the desired goal.

\subparagraph{C. Imagination.} Nowhere does the uniqueness of an individual show more clearly than in the products of his fantasy \& of his imagination. By imagination we mean the reproduction of a perception without the presence of the object itself which gave rise to it. In other words imagination is reproduced perception: -- another evidence of the creative faculty of the soul. The product of imagination is not only the repetition of a perception (which in itself is a product of the creative power of the soul), but is an entirely new \& unique product built upon the basis of the perception, just as the perception was created on the basis of physical sensations.

Now there are fantasies which far exceed the customary imagination in sharpness of focus. Such visions are so sharply outlined that they have a value not of imaginary products, but influence the behavior of the individual as though the absent stimulating object were actually present. We speak of {\it hallucinations}, when fantasies appear as though they were the result of an actually present stimulus. The conditions for the appearance of hallucinations are in no wise different from those which determine fantastic day dreams. Every hallucination is an artistic creation of the soul, shaped, \& constellated according to the goals \& purposes of the particular individual in which it appears. Let us make this clear with an example.

An intelligent young woman married against the advice of her parents. Her parents were so angry at her mismarriage that they broke off all relations with her. In the course of time the young woman became convinced that her parents had not treated her well, but many attempts at reconciliation failed because of the pride \& obstinacy of both parties. As a result of her marriage this young woman, who belonged to an honored \& wealthy family, had fallen into rather impoverished circumstances. Yet externally no one could observe any signs of unhappiness in her connubial\footnote{(literary) relating to marriage, or the relationship between people who are married.} relations. One could have been quite reconciled to the fact that she had made a good adjustment were it not for the appearance of a very peculiar phenomenon in her life.

This girl had grown up as the favorite child of her father. So intimate had been their relationship that their present breach was the more remarkable. The occasion of her marriage, however, caused her father to treat her very badly, \& their rupture was very deep. Even when her child was born, her parents could not be moved to visit their daughter or to see the child; the young woman took the harsh treatment of her parents the more to heart, because, actuated by a great ambition, she was touched to the quick by their attitude toward her in a situation in which she might well have been treated with consideration.

We must remember that the mood of this young woman was completely dominated by her ambition. It is this character trait which gives us an insight into the reasons why the breach with her parents affected her so deeply. Her mother was a stern, righteous person who had many good qualities, although she had treated her daughter with a heavy hand. She knew how to submit to her husband, at least so far as outer appearances were concerned, without really relinquishing her own rank. Indeed she drew attention to her submission with a certain pride, \& considered it an honor. Now in this family there was also a son who was considered a chip of the old block \& the future heir of the family name. The fact that he was considered somewhat more valuable than our young woman served only to spur her ambition. The difficulties \& poverty which this young woman, educated in a comparatively sheltered atmosphere all her life, was experiencing in her marriage, now caused her to think constantly \& with ever increasing displeasure about the mistreatment she had received at the hands of her parents.

1 night before she had fallen asleep it happened that a door opened \& the Virgin Mary stepped to her bed \& said: ``Because I love you so well, I must tell you that you will die in the middle of December. I do not want you to be unprepared.''

The young woman was not frightened by this apparition, but she wakened her husband \& told him everything. On the next day she went to the physician \& told him about it. It was a hallucination. The young woman maintained that she had seen \& heard everything quite clearly. At 1st glance this seems impossible, yet when we apply the key of our knowledge we can understand it quite well. Here is the situation: a young woman who is very ambitious, \&, as the examination shows, has the tendency to dominate everyone else, breaks with her parents \& finds herself in poverty. It is quite understandable that a human being, in an effort to conquer everything in the physical sphere in which he lives, should approach God \& converse with Him. If the Virgin Mary had remained only an imaginary figure (as is the case in prayer) no one would have found anything particularly noteworthy in this occurrence, but this young woman needed stronger arguments.

The phenomenon loses all its mystery when we understand what tricks the soul is capable of producing. Is not every human being who dreams in a similar position? The difference really is only this: that this young woman can dream while she is awake. We must add, too, that her feeling of depression has placed her ambition under greater tension. Now we become aware of the fact that actually another mother is coming to her, indeed, that Mother who in popular conception is the greatest Mother of all. These 2 mothers must stand in certain contrast to one another. The Mother of God appeared because her own mother did {\it not} come. The apparition\footnote{a ghost or a ghost-like image of a person who is dead.} is an accusation against her own mother \& her insufficient love for her child.

The young woman is now trying to find some way of proving that her parents are wrong. The middle of December is not an insignificant time. It is that time o the year in which people are more apt to consider their deeper relationships, when most human beings approach each other with a greater warmth, give presents, \& the like. It is at this time too that the possibility of reconciliation comes closer, so that we can understand that this particular time stands in a close relationship to the quandary in which the young woman finds herself.

The only strange thing in this hallucination seems to be that the friendly approach of this Mother of God is accompanied by the sad news of the young woman's approaching death. The fact that she told her husband of this vision with an almost happy tone of voice is also not without significance. This prophecy quickly spread beyond the narrow circle of her family \& the physician had learned of it on the following day: \& it was thus very simple to bring it about that her mother actually visited her.

A few days later the Virgin Mary appeared for the 2nd time \& spoke the same words. When the young woman was asked how her meeting with her own mother had turned out, she answered that her mother could not admit that she had done wrong. We see therefore the old theme cropping up again. Her desire to dominate her mother had not yet been fulfilled.

At this time it was attempted to make the parents understand what was actually going on in the life of their daughter, \& as a result, a very satisfactory meeting between the young woman \& her father obtained. A touching scene occurred, but the young woman was not yet satisfied, because she said there was something theatrical in her father's behavior. She complained that he had let her wait too long! Even in triumph she could not rid herself of the tendency to prove everyone else wrong \& herself to appear in the light of a triumphant victor.

We may conclude from our previous discussion that hallucinations appear in that moment when the psychic tension is at its greatest \& in circumstances in which one fears that the attainment of one's goal is impossible. There is no question that hallucinations won a considerable influence in districts where the population has been somewhat backward in its development, in days gone by.

Descriptions of hallucinations in the writings of travelers are well known. The mirages which are seen by wanderers in the desert, who have lost their way \& are suffering from hunger, thirst, \& fatigue, are excellent examples. We can understand that the tension which arises when life is in danger, compels the imagination of the sufferer to create a clear \& refreshing situation for himself, in order that he may escape from the unpleasant oppression of his environment. The mirage represents a new situation which can encourage the fatigued, refresh the failing powers of the irresolute, make the traveler stronger, or more sensitive: or on the other hand, it may act as a balsam, or narcotic, which can rob his misery of its horrors.

Hallucination is nothing new for us since we have already seen similar phenomena in perception, in the mechanism of memory, \& in imagination. We shall find these same processes again when we consider dreams. By accentuating the imagination, \& excluding the critique of the higher centers, it is easy to produce the phenomena of hallucination. Under circumstances of necessity or danger, \& under the pressure of a situation in which one's power is threatened, one strives to obviate the feeling of weakness \& overcome it, by this mechanism. The greater the tension, the less the consideration which will be paid to the critical faculties. Under such circumstances, with the motto ``Help yourself as you can!'' anyone with the aid of every ounce of his psychic energy, can force his imagination to project itself into the hallucination.

Illusion is closely related to hallucination, the only difference being that some point of external contact remains, but is misinterpreted, as is the case in the story of Goethe's {\it Erlk\"onig}. The underlying situation, the feeling of psychic peril, is the same.

Another example will show how the creative power of the soul can produce either an illusion or a hallucination, as the need arises. A man of excellent family who had never amounted to anything because of a bad education, held an unimportant clerkship. He had given up all hoope of ever amounting to anything. His hopelessness weighed heavily upon him, \& in addition his psychic tension was increased by the reproaches of his friends. In these circumstances he took to drink, which gave him at once a sweet forgetfulness, \& an excuse for his failure. After some time he was brought to the hospital in delirium tremens. Delirium is closely related to hallucination, \& in the delirium of alcoholic intoxication, small animals, such as mice, or insects, or snakes, frequently appear. Other hallucinations which are related to the patient's occupation may also occur.

Our patient came into the hands of physicians who were strongly opposed to the use of alcohol. They put him through a strict course of treatment \& he was completely freed of his alcoholism, left the hospital cured, \& did not touch alcohol for 3 years. At this time he returned to the hospital with a new complaint. He stated that he constantly saw a leering, grinning man who watched him at his work. He was now a day-laborer. Once when he was particularly angry because this man was laughing at him he took his pick \& threw it at him to see whether he was a real man or only an apparition. The apparition dodged the missile, but thereupon attacked him \& beat him badly.

In this case we can no longer speak of a ghost because the hallucination had very real fists. The explanation is not hard to find. It was his {\it custom} to hallucinate, but he made his {\it test} upon a real man. This shows us clearly that although he had been freed of his desire to drink, he had in reality sunk further since his discharge from the hospital. He had lost his job, had been put out of his house, \& now had to earn his living as a day-laborer, which he as well as his friends considered the lowest form of work. The psychic tension in which he had lived had not become less. Although he had been freed from alcohol he had actually become poorer by a consolation, despite the great advantage of this cure. He could do his 1st job with the help of drink, for when he was reproached too loudly at home for not accomplishing anything, the excuse that he was a drunkard seemed less shameful to him than his incapability of holding a job. After his cure he was again face to face with reality \& in a situation which was in no wise less oppressive than his former one. Should he now fail he had nothing to console himself with \& nothing to blame, not even alcohol.

In this situation of psychic peril the hallucinations reappear. He identifies himself with the previous situation \& looks at the world {\it as if} he were still a drunkard, \& says very clearly with this gesture, that he has ruined his whole life with his drinking \& nothing can be done about it now. By being sick he hoped to be freed from his little honored, \& therefore, for him, very unpleasant occupation as a ditch digger, without having to make a decision about it himself. The above-mentioned hallucination lasted for a long time until he finally was forced to the hospital again. Now he could console himself with the thought that he could have accomplished a great deal more had not the misfortune of drink ruined his life. This mechanism enabled him to maintain his personal evaluation at a high level. It was more important for him not to allow his personal evaluation to sink than it was for him to work. All his efforts were directed at maintaining the conviction that he might have accomplished great things had he not been visited by misfortune. This was the proof which maintained him in his power relationship \& enabled him to feel that other men were not better than he but that an impassable obstacle lay in his way. The mood in which he attempted to find a consoling alibi produced the apparition of the leering man; the ghost was the savior of his self-esteem.'' -- \cite[pp. 47--57]{Adler_human_nature}

\paragraph{Fantasy.} ``Fantasy is but another creative faculty of the soul. Traces of this activity may be found in the various phenomena which we have already described. Just as the projection of certain memories into the sharp focus of consciousness, or the erection of the bizarre superstructures of the imagination, fantasy \& day-dreaming are to be considered part of the creative activity of the soul. The prevision \& prejudgment which are an essential faculty in any mobile organism constitute an important factor in fantasy. Fantasy is bound up with the mobility of the human organism \& is indeed nothing more than a method of prevision \& prescience. The fantasies of children \& grown-ups, sometimes called {\it day-dreams}, are always concerned with the future, the ``castles in the air'' are the goal of their activity, built up in fictional form as models for real activity. Examinations of childhood fantasies show clearly that the striving for power plays the predominant role. Children deal with the goal of their ambition in their day-dreams. Most of their fantasies begin with the words ``when I am grown up,'' \& the like. There are many adults who live as though they still had to grow up. The clear emphasis on the striving for power indicates to us again that the soul life can develop only when a certain goal has been set. In our civilization this goal is the goal of social recognition \& significance. An individual never remains long at any neutral goal, for the communal life of mankind is accompanied by constant self-measurement which gives rise to the desire for superiority, \& the hope of success in competition. The forms of prevision which are so evident in the fantasies of children are almost entirely situations in which the child's power is expressed.

We must not generalize here because it is impossible to lay down rules for the degree of fantasy or the extent of imagination. What we have said before is valid for a number of cases, but may not be applicable to some. Those children who approach life with belligerent eyes will develop their fantastic powers to greater lengths because their precaution is stimulated to a greater tension as a result of their attitude. Weak children for whom life is not always pleasant develop greater powers of fantasy, \& have the tendency to occupy themselves particularly with this type of activity. At a certain stage in their development their ability to imagine may become a mechanism whereby the realities of life are evaded. Fantasy may be misused as a condemnation of reality. In such cases it becomes a kind of power-intoxication in an individual who raises himself above the meanness of living, by the fictional lever of his imagination.

The social feeling, together with the striving for power, also plays a great role in the fantasy life. In childhood fantasies, it is only seldom that power strivings appear without some application of this power to social ends. This trait we see clearly in those fantasies in which the content concerns itself with being a savior or a good knight, a victor over evil forces, devils, \& the like. The fantasy that the child does not belong to his own family frequently occurs. Many children believe that they actually originated from a different family, \& that some day their real father, some important personage, will come \& fetch them. This happens most frequently where children with a deep feeling of inferiority, hounded by the deprivations they have suffered, are forced into the background, or become dissatisfied with the love \& tenderness they receive in their family circle. Ideas of grandeur betray themselves in the external attitude of the child who acts as though he were already grown up. Sometimes one finds almost pathological\footnote{1. not reasonable or sensible; impossible to control; 2. caused by, or connected with, disease or illness.} expressions of this fantasy, as e.g., in children who will wear only stiff hats, or go about picking up cigar butts in order to appear men; or in the case of young girls who decide to become men, \& bear themselves \& dress themselves in a manner more appropriate to boys.

There are children who are said to have no imagination. This is surely an error. Either such children do not express themselves, or there are other reasons which compel them to take up battle against the appearance of fantasies. A child may contrive to feel a certain sense of power by suppressing his imagination. In a cramped striving to adjust to reality, these children believe that fantasy is unmanly or childlike, \& refuse to partake in it; \& there are cases in which his disinclination goes so far that their imagination seems totally lacking.'' -- \cite[pp. 57--59]{Adler_human_nature}

\paragraph{Dreams: General Considerations.} ``In addition to the previously described day-dreams we must deal with that important \& significant activity which occurs during our sleep, the night dream. In general it may be said that the night dreaming is a repetition of the same process which goes on in day-dreams. Old experienced psychologists have pointed to the fact that the character of a human being may easily be read from his dreams. Actually dreams have occupied the thinking of mankind to an enormous degree since the dawn of history. In the sleep dream, as in the day-dream, we are concerned with the activity of an organism which is attempting to map, plan, \& direct its future life toward a goal of security. The most apparent difference is that day-dreams are comparatively easily understood, whereas sleep dreams are but seldom comprehended. That dreams are not understandable is not surprising \& we might easily be tempted to find this an indication that they are superfluous \& insignificant. For the time being let it be said that the striving for power of an individual who is seeking to overcome difficulties \& maintain his position in the future, is reechoed in his dreams. Dreams offer us important grips on the problems of the psychic life.'' -- \cite[pp. 59--60]{Adler_human_nature}

\paragraph{Empathy \& Identification.} ``The soul has the faculty not only of perceiving what actually exists in reality, but also of feeling; of guessing, what will occur in the future. This is an important contribution to the function of pre-vision necessary to any easily mobile organism since such an organism is constantly faced with the problem of making adjustments. We call this faculty identification, or empathy. It is extraordinarily well developed in human beings. Its extent is so great that one finds it in every corner of the psychic life. The necessity for pre-vision is the prime condition of its existence. If we are forced to preview, to prejudge, to presume how we should act if a certain situation were to occur, we must learn how to gain a sound judgment of a situation which has not yet occurred, through correlation of our thinking, feeling, \& perception. It is essential to win a point of view so that we may either approach the new situation with more strenuous efforts, or avoid it with greater caution.

Empathy occurs in the moment 1 human being speaks with another. It is impossible to understand another individual if it is impossible at the same time to identify oneself with him. Drama is the artistic expression of empathy. Other examples of empathy are those cases in which someone has a strange feeling of uneasiness when he notices another in danger. This empathy may be so strong that one makes involuntary defense movements, even though there is no danger to oneself. We all know the well known gesture which is made when someone has dropped his glass! At a bowling alley one may see certain players following the course of the ball with movements of their body as though they wanted to influence its course by this movement! Similarly during football games whole sections of people in the grand stand will push in the direction of their favorite team, or make resistive pressure when the opponent team has the ball. A common expression is the involuntary application of imaginary brakes by the occupants of a motor car whenever they feel that they are in danger. Few people can pass a tall building in which someone is washing a window without experiencing certain contractions \& defense movements. When a speaker loses his presence of mind \& cannot proceed, people in the audience are oppressed \& uneasy. In the theater particularly we can hardly avoid identifying ourselves with the players, or prevent ourselves from acting the most varied roles within ourselves. Our entire life is very much dependent upon the faculty of identification. If we seek for the origin of this ability to act \& feel as if we were someone else, we can find it in the existence of an inborn social feeling. This is, as a matter of fact, a cosmic feeling \& a reflection of the connectedness of the whole cosmos which lives in us; it is an inescapable characteristic of being a human being. It gives us the faculty of identifying ourselves with things which are quite outside our own body.

Just as there are various degrees of the social feeling so there are various degrees of empathy. These may be observed even in childhood. There are children who occupy themselves with dolls just as though they were human beings, whereas others are more interested in seeing what is inside of them. By projecting the communal relationships from human beings to less valuable or lifeless objects, the development of an individual may be entirely stopped. Cases of cruelty to animals which we see in childhood were impossible without an almost total absence of the social feeling, \& the ability to identify with other living beings. The consequences of this defect lead children to develop interest in things which are of very little value or significance for their development into fellow human beings. They think only of themselves, \& lose all interest for the joys or woes of others. These are manifestations which are intimately related to a deficient degree of empathy. The inability to identify oneself with another may lead so far that an individual refuses entirely to cooperate with his fellow men.'' -- \cite[pp. 60--62]{Adler_human_nature}

\paragraph{Hypnosis \& Suggestion.} ``Individual Psychology answers the question ``How is it possible for 1 individual to influence the behavior of another?'' by saying that this phenomenon is 1 of the accompanying manifestations of our psychic life. Our whole communal life were impossible unless 1 individual could influence another. This mutual influence becomes markedly accentuated in some cases, as e.g., in the relationship between teacher \& pupil, between parents \& children, between husband \& wife. Under the influence of the social feeling there exists a certain degree of willingness to be influenced by one's environment. The degree of this readiness to be influenced is dependent upon the degree to which the rights of the one on whom the influence is exerted are considered by that one who exercises the influence. It is impossible to have a lasting influence upon an individual whom one is harming. One can influence another individual best when he is in the mood in which he feels his own rights guaranteed. This is a very important point in pedagogy. Perhaps it is possible to conceive, or even to carry out some other form of education, but a system of education which takes this point into consideration will be adequate for the reason that it is connected with the most primitive instinct in man, the feeling of his relatedness to man \& the cosmos.

It will fail only when it is dealing with a human being who has purposely withdrawn himself from the influence of society. Such a withdrawal does not occur by accident. A lasting war must have occurred, during the course of which his connections with the world about him have been dissolved, little by little, so that now he stands in open opposition to the social feeling. Every type of influence upon his behavior is now made more difficult or impossible. One views the dramatic spectacle of a human being who responds to every attempt to influence him with a counter attack of opposition.

We may expect that children who feel themselves oppressed by their environment will show a deficient amenity to the influence of their educators. Cases occur, however, in which the external pressure is so strong that it removes all obstacles with the result that the authoritative influence is retained \& obeyed. It is easy to prove that this obedience is sterile of all social good. It sometimes manifests itself in such a grotesque fashion that it renders the obedient individual unfit for life. By dint of their servile obedience such individuals are incapable of any action or thought without an appropriate command from someone else. The great danger which this far reaching submission carries in itself is to be measured by the fact that there are children who, when they develop into adults, obey anyone's commands, even to the commission of crimes.

Interesting examples are to be found in gangs. Those who carry out the gang's commands belong to this class, whereas the leader of the gang usually holds himself far from the scene of action. In almost every important criminal case dealing with a gang crime, some such servile man has been the cat's paw. This far-reaching blind obedience attains such unbelievable depths that we can occasionally find people who are actually proud of their servility, \& find it a way to the satisfaction of their ambition.

If we limit ourselves to normal cases of mutual influence, we find that those people are most capable of being influenced, who are most amenable to reason \& logic, those whose social feeling has been least distorted. On the contrary, those who thirst for superiority \& desire domination are very difficult to influence. Observation teaches us this fact every day.

When parents complain about a child it is only very rarely that they do so because of his blind obedience. The most common complaint arises because of his disobedience. Examination shows that such children are caught in a current which would make them superior to their environment; they are striving to batter down the cramping walls of their little life. They have been made unapproachable for educational influence by virtue of a mistaken treatment at home.

The intensive striving for power is inversely proportional to the degree to which one can be educated. Despite this fact, our family education is concerned, for the most part, in spurring on the ambition of the child, \& awakening ideas of grandeur in his mind. This does not occur because of thoughtlessness, but because our whole culture is permeated with similar grandiose delusions. In the family, as in our civilization, the greatest emphasis is placed upon that individual who is greater, \& better, \& more glorious, than all the others in his environment. In the chapter on vanity we shall have occasion to show how maladapted this method of education towards ambition is to the communal life, \& how the development of the mind can be stunted by the difficulties which ambition places in its way.

Every medium is in a position similar to individuals who are influenced by every turn of their environment as a consequence of their unconditional obedience. Imagine obeying every whim that anyone voices, for a short time! Hypnosis is based upon a similar preparation. Anyone may say, or believe, that he has the will to be hypnotized, but the psychic readiness to submit may be wanting. A second individual may consciously resist, \& still be innately desirous to submit. In hypnosis the psychic attitude of the medium alone determines his behavior. What he says, or what he believes, is of no consequence. Confusion over this fact has allowed much misinformation to grow up concerning hypnosis. In hypnosis one is usually occupied with individuals who {\it seem} to be striving against the hypnosis, but are essentially desirous of submitting to the demands of the hypnotizer. This readiness may have various boundaries so that the results of hypnosis differ in every individual. In no case does the degree of readiness to be hypnotized depend upon the will of the hypnotizer. It is conditioned entirely by the psychic attitude of the medium.

In its essence, hypnosis somewhat resembles sleep. It is mysterious only because this sleep may be produced at the command of another. The command is effective solely when it is given to someone who is willing to submit to it. The determining factors are, as usual, the nature \& character of the medium or subject. Only that man who is willing to accede to the demands of another without the exercise of his critical faculties, is capable of producing a hypnotic sleep; the hypnosis is more than an ordinary sleep in that it excludes the faculty of movement to such a degree that even the motor centers are mobilized at the command of the hypnotizer. A certain twilight slumber is all that is left of normal sleep in this state, in which the subject can remember only those things which the hypnotizer allows him to remember. The most important fact in hypnosis is that our critical faculties, those finest products of the soul, are completely paralyzed during the hypnotic trance. The hypnotized subject becomes, so to speak, the elongated hand of the hypnotizer, an organ functioning at his command.

Most people who have the power of influencing the behavior of others ascribe this faculty to some mysterious power which is peculiar to them. This leads to an enormous amount of mischief, especially in the pernicious activities of the telepaths \& hypnotizers. These gentlemen commit such arrant crimes against mankind that they are quite capable of utilizing any instrument appropriate to their nefarious\footnote{criminal, extremely bad.} purposes. This does not say that all the manifestations which they produce are based upon a swindle\footnote{to cheat somebody in order to get something, especially money, from them.}. The human animal, unfortunately, is capable of such submission that it falls victim to anyone who poses as the possessor of special powers. Only too many human beings have acquired the habit of recognizing an authority without testing it. The public wants to be fooled. It wants to swallow every bluff without subjecting it to rational examination. Such activity will never bring any order into the communal life of mankind but will lead only, again \& again, to the revolt of those who have been imposed upon. No telepath nor hypnotizer has had luck with his experiments for any great length of time. Very frequently they have come in contact with someone, some so-called medium, who has fooled them for all he was worth. This has sometimes been the experience of important scientists who have attempted to show their powers on mediums.

There are other cases in which there is a curious admixture of truth \& falsehood: the medium is, so to speak, a deceived deceiver, one who fools the hypnotizer in part, but also subordinates himself to his will. The power which apparently is at work here is never the power of the hypnotizer, but always the readiness of the medium to subordinate himself \& submit. There is no magic power which influences the medium unless it be the ability of the hypnotizer to bluff. Any man who is accustomed to living rationally, who makes his own decisions, who does not swallow anyone's word uncritically, is naturally not to be hypnotized, \& will, therefore, never be able to show any telepathic powers. Hypnosis \& telepathy are only the manifestations of servile obedience.

At this point we must also consider suggestion. Suggestion can be best understood when one includes it in the category of impressions \& stimuli. It is self-understood that no human being is stimulated only occasionally. All of us are constantly under the influence of innumerable impressions arising in the outer world. The mere perception of a stimulus never occurs. Once an impression is felt, it continues to exercise its effect. When these impressions take the form of the demands \& the entreaties of another human being, his attempts at conviction or his arguments, we speak of suggestion. It is a case either of the transformation, or of the reinforcement, of a point of view which is already present in the person to whom the suggestions are made. The more difficult problem really begins with the fact that every human being reacts variously to stimuli coming from the external world. The degree to which he can be influenced is intimately connected with his independence. There are 2 types of human beings which we must bear in mind. 1 type always overvalues the other fellow's opinion \& therefore values its own opinions only lightly, whether they are right or wrong. They are given to over-rating the importance of others, \& \& to adapting themselves gladly to their opinions. These individuals are exceptionally susceptible to suggestion, or hypnosis. A second type considers every stimulus or suggestion as an insult. Here are the individuals who consider that only their own opinion is right, \& are really not concerned as to its actual correctness, or incorrectness. They disregard anything originating in another human being. Both types carry with them a sense of weakness. The 2nd type expresses this weakness by not being able to receive anything from another human being. Members of this category are usually very belligerent persons, although they may pride themselves upon being open to suggestions. They talk about this openness \& reasonableness, however, only in order to reinforce their isolated position. Actually they cannot be approached, \& it is very difficult to do anything with them.'' -- \cite[pp. 62--68]{Adler_human_nature}

\subsubsection{The Feeling of Inferiority \& The Striving for Recognition}

\paragraph{The Situation in Early Childhood.} ``We are now certainly prepared to recognize the fact that children who have been treated as step-children by Nature have an entirely different attitude toward life \& toward their fellow human beings than those to whom the joys of existence were vouchsafed at an early age. One can state as a fundamental law that children who come into the world with organ inferiorities become involved at an early age in a bitter struggle for existence which results only too often in the strangulation of their social feelings. Instead of interesting themselves in an adjustment to their fellows, they are continually preoccupied with themselves, \& with the impression which they make on others. What holds good for an organic inferiority is as valid for any social or economic burden which might manifest itself as an additional load, capable of producing a hostile attitude toward the world. The deciding trend becomes determined at an early age. Such children frequently have a sentiment as early as their 2nd year of life, that they are somehow not as adequately equipped for the struggle as their playmates; they sense that they dare not trust themselves to the common games \& pastimes. As a result of past privations they have acquired a feeling of being neglected, which is expressed in their attitude of anxious expectation. One must remember that every child occupies an inferior position in life; were it not for a certain quantum of social feeling on the part of his family he would be incapable of independent existence. One realizes that the beginning of every life is fraught with a more or less deep feeling of inferiority when one sees the weakness \& helplessness of every child. Sooner or later every child becomes conscious of his inability to cope single-handed with the challenges of existence. This feeling of inferiority is the driving force, the starting point from which every childish striving originates. It determines how this individual child acquires peace \& security in life, it determines the very goal of his existence, \& prepares the path along which this goal may be reached.

The basis of a child's educability lies in this peculiar situation which is so closely bound up with his organic potentialities. Educability may be shattered by 2 factors. 1 of these factors is an exaggerated, intensified, unresolved feeling of inferiority, \& the other is a goal which demands not only security \& peace \& social equilibrium, but a striving to express power over the environment, a goal of dominance over one's fellows. Children who have such a goal are always easily recognized. They become ``problem'' children because they interpret every experience as a defeat, \& because they interpret every experience as a defeat, \& because they consider themselves always neglected \& discriminated against both by nature \& by man. One need but consider all these factors to see with what compulsive necessity a crooked, inadequate, error-ridden development may occur in the life of a child. Every child runs the danger of a mistaken development. Every child finds itself in a situation which is precarious, at some time or another.

Since every child must grow up in an environment of adults he is predisposed to consider himself weak, small, incapable of living alone; he does not trust himself to do those simple tasks that one thinks him capable of doing, without mistakes, errors, or clumsinesses. Most of our errors in education begin at this point. In demanding more than the child can do, the idea of his own helplessness is thrown into his face. Some children are even consciously made to feel their smallness \& helplessness. Other children are regarded as toys, as animated dolls; others, again, are treated as valuable property that must be carefully watched, while others still are made to feel they are so much useless human freight. A combination of these attitudes on the part of the parents \& adults often leads a child to believe that there are but 2 things in his power, the pleasure or displeasure of his elders. The type of inferiority feeling produced by the parents may be further intensified by certain peculiar characteristics of our civilization. The habit of not talking children seriously belongs in this category. A child gets the impression that he is a nobody, without rights; that he is to be seen, not heard, that he must be courteous, quiet, \& the like.

Numerous children grow up in the constant dread of being laughed at. Ridicule of children is well-nigh criminal. It retains its effect upon the soul of the child, \& is transferred into the habits \& actions of his adulthood. An adult who was continually laughed at as a child may be easily recognized; he cannot rid himself of the fear of being made ridiculous again. Another aspect of this master of not taking children seriously in the custom of telling children palpable lies, with the result that the child begins to doubt not only his immediate environment but also to question the seriousness \& reality of life.

Cases have been recorded of children who laughed continually at school, seemingly without reason, who when questioned, admitted that they thought school was 1 of their parents' jokes \& not worth taking seriously!'' -- \cite[pp. 69--71]{Adler_human_nature}

\paragraph{Compensating for the Feeling of Inferiority; the Striving for Recognition \& Superiority.} ``It is the feeling of inferiority, inadequacy, insecurity, which determines the goal of an individual's existence. The tendency to push into the limelight, to compel the attention of parents, makes itself felt in the 1st days of life. Here are found the 1st indications of the awakening desire for recognition developing itself under the concomitant influence of the sense of inferiority, with its purpose the attainment of a goal in which the individual is seemingly superior to his environment.

The degree \& quality of the social feeling helps to determine the goal of dominance. We cannot judge any individual, whether it is a child or adult, without drawing a comparison between his goal of personal dominance \& the quantum of his social feeling. His goal is so constructed that its achievement promises the possibility either of a sentiment of superiority, or an elevation of the personality to such a degree that life seems worth living. It is this goal which gives value to our sensations, which links \& co-ordinates our sentiments, which shapes our imagination \& directs our creative powers, determines what we shall remember \& what we must forget. We can realize how relative are the values of sensations, sentiments, affects, \& imagination, when not even these are absolute quantities; these elements of our psychic activity are influenced by the striving for a definite goal, our very perceptions are prejudiced by it, \& are chosen, so to speak, with a secret hint at the final goal toward which the personality is striving.

We orient ourselves according to a fixed point which we have artificially created, which does not in reality exist, a fiction. This assumption is necessary because of the inadequacy of our psychic life. It is very similar to other fictions which are used in other sciences, such as the division of the earth by non-existent, but highly useful meridians. In the case of all psychic fictions we have to do with the following: we assume a fixed point even though closer observation forces us to admit that it does not exist. The purpose of this assumption is simply to orient ourselves in the chaos of existence, so that we can arrive at some apperception of relative values. The advantage is that we can categorize every sensation \& every sentiment according to this fixed point, once we have assumed it.

Individual Psychology, therefore, creates for itself a heuristic system \& method: to regard human behavior \& understand it as though a final constellation of relationships were produced under the influence of the striving for a definite goal upon the basic inherited potentialities of the organism. Our experience, however, has shown us that the assumption of a striving for a goal is more than simply a convenient fiction. It has shown itself to be largely coincident with the actual facts in its fundamentals, whether these facts are to be found in the conscious or unconscious life. The striving for a goal, the purposiveness of the psychic life is not only a philosophic assumption, but actually a fundamental fact.

When we question how we can most advantageously oppose the development of the striving for power, this most prominent evil of our civilization, we are faced with a difficulty, for this striving begins when the child cannot be easily approached. One can begin to make attempts at improvement \& clarification only much later in life. But {\it living} with the child at this time does offer an opportunity to so develop his social feeling that the striving for personal power becomes a negligible factor.

A further difficulty lies in the fact that children do not express their striving for power openly, but hide it under the guise of charity \& tenderness, \& carry out their work behind a veil. Modestly, they expect to escape disclosure in this way. An uninhibited striving for power is capable of producing degenerations in the psychic development of the child, an exaggerated drive for security \& might, may change courage to impudence, obedience into cowardice, tenderness into a subtle treachery for dominating the world. Every natural feeling or expression finally carries with it a hypocritical afterthought whose final purpose is the subjugation of the environment.

Education affects the child by virtue of its conscious or unconscious desire to compensate him for his insecurity, by schooling him in the technique of life, by giving him an educated understanding, \& by furnishing him with a social feeling for his fellows. All these measures, whatever their source, are means to help the growing child rid himself of his insecurity \& his feeling of inferiority. What happens in the soul of the child during this process we must judge by the character traits he develops since these are the mirror of the activity in his soul. The actual inferiority of a child, important as it is for his psychic economy, is no criterion of the weight of his feeling of insecurity \& inferiority, since these depend largely upon his interpretation of them.

One cannot expect a child to have a correct estimation of himself in any particular situation; one does not expect it of adults! It is precisely here that difficulties grow apace. 1 child will grow up in a situation so complicated that errors concerning the degree of his inferiority are absolutely unavoidable. Another child will be able better to interpret his situation. But taken by \& large the interpretation which the child has of his feeling of inferiority varies from day to day until it becomes consolidated, finally, \& is expressed as a definite self-estimation; this becomes a ``constant'' of self-evaluation which the child retains, in all his conduct. According to this crystallized norm or ``constant of self-estimation'' the compensation trends which the child creates to guide him out of his inferiority will be directed toward this, or the other, goal.

The mechanism of the striving for compensation with which the soul attempts to neutralize the tortured feeling of inferiority has its analogy in the organic world. It is a well known fact that those organs of our body which are essential for life, produce an overgrowth, \& overfunction when their productivity is lessened through damage to their normal state. Thus in difficulties of circulation, the heart, seeming to draw its new strength from the whole body, may enlarge until it is more powerful than a normal heart. Similarly, the soul, under pressure of the feeling of inferiority, or the torturing thought that the individual is small \& helpless, attempts with all its might to become master over this ``inferiority complex.''

When the feeling of inferiority is intensified to the degree that the child fears he will never be able to compensate for his weakness, the danger arises that in his striving for compensation he will be satisfied not with a simple restoration of the balance of power; he will demand an over-compensation, will seek an overbalance of the scales!

The striving for power \& dominance may become so exaggerated \& intensified that it must be called pathological. When this occurs the ordinary relationships of life will never be satisfactory. The movements in these cases are apt to have a certain grandiose gesture about them. They are well adapted to their goal. Where we are dealing with a pathological power-drive we find individuals who seek to secure their position in life with extraordinary efforts, with greater haste \& impatience, with more violent impulses, \& without consideration of anyone else. These are the children whose actions become more noticeable because of their exaggerated movements towards their exaggerated goal of dominance; their attacks on the lives of others necessitate that they defend their own lives. They are against the world, \& the world is against them.

This need not necessarily occur in the worst sense of the word. There are children who express the striving of power in a manner not calculated to bring them into immediate conflict with society, \& their ambition may be considered as no abnormal characteristic. Yet when we carefully investigate their activity \& achievements we find that society at large does not benefit from their triumphs, because their ambition is an asocial one. Their ambition will always put them in the path of other human beings as disturbing elements. Little by little, too, other characteristics will appear which, if we consider total human relationships, will assume an increasingly antisocial color.

In the forefront of these manifestations are pride, vanity, \& the desire to conquer everyone at any price. the latter may be subtly accomplished by the relative elevation of the individual, by his deprecation of all those with whom he comes in contact. In the latter case the important thing is the ``distance'' which separates him from his fellows. His attitude is not only uncomfortable for the environment, but for the individual who practices it, because it continually brings him into contact with the dark side of life \& prevents him from experiencing any joy in living.

The exaggerated drive for power with which some children wish to assure their prestige over their environment, soon forces them into an attitude of resistance against the ordinary tasks \& duties of everyday life. Compare such a power-hungry individual with the ideal social being, \& one can, after some little experience, specify, so to speak, his social index, i.e., the degree to which he has removed himself from his fellow-man. A keen judge of human nature, keeping his eyes open to the value of physical defects \& inferiorities, knows nevertheless that such character traits were impossible without antecedent difficulties in the evolution of his soul.

When we have gained a true knowledge of human nature, built upon a recognition of the value of the difficulties which may occur in the proper development of the soul, it can never be an instrument of harm so long as we have ourselves thoroughly developed our social feeling. We can but help our fellow-men with it. We must not blame the bearer of a physical defect, nor a disagreeable character trait, for his indignation. He is not responsible for it. We must indeed admit his right to be indignant to the last limits, \& we must be conscious that we bear a part of the common blame for his situation. The blame belongs to us because we too have taken part in the inadequate precautions against the social misery which has produced it. If we stick to this standpoint we can eventually ameliorate the situation.

We approach such an individual not as a degraded, worthless outcast, but as a fellow human being; we give him an atmosphere in which he will find that there are possibilities for feeling himself the equal of every other human being in his environment. Think how unpleasant the sight of an individual's appearance, whose organ or bodily inferiorities are externally visible, may be to you! It is a good index of the amount of education you yourself need in order to come to an absolutely just sense of social values, \& put yourself into complete harmony with the truth of the social feeling. \& we can judge then, too, how much our civilization owes to such an individual.

It is self-understood that those who come into the world with organ inferiorities feel an added burden of existence from their earliest days, \&, as a result, find themselves pessimistic as regards the whole matter of existence. Children in whom the feeling of inferiority has become intensified through some cause or other, although their organ inferiorities are not nearly so noticeable, find themselves in a similar situation. The feeling of inferiority may be so intensified artificially that the result is exactly the same as though the child came into the world greatly crippled. A very severe education during the critical period, e.g., may effect such an unfortunate result. The thorn which has been stuck into their side in the early days of their existence is never removed, \& the coldness which they have experienced prevents them from approaching other human beings in their environment. They believe themselves thus in a world devoid of love \& affection, with which they have no common point of contact.

An example: A patient, noticeable because he is continually telling us about his great sense of duty, \& the importance of all his actions, lives with his wife in the worst possible relationship. Here are 2 individuals who measure the value of any event as a means toward the subjugation of their mate, to the thickness of a hair. Wrangling, reproaches, insults, in the course of which the 2 become entirely estranged from 1 another, are the inevitable result. What little social feeling for his fellow men the husband retains, at least so far as his wife \& friends are concerned, is choked by his thirst for superiority.

We learn the following facts from the story of his life: -- he was practically undeveloped physically until his 17th year. His voice was the voice of a young boy, he had no body or face hair, \& he was among the smallest boys in his school. Today he is 36. Nothing which is not entirely masculine is noticeable about his outer appearance, \& Nature seemingly has caught up with herself \& completed everything which she had hardly begun to fashion when he was 17. But for 8 years he suffered from this failure of development, \& at that time he had no guarantee that Nature would ever compensate for his anomalies. During this entire period he was tortured with the thought that he must always remain a ``child.''

At that early age the beginnings of his present character traits could be noted. He acted as though he were very important, \& as if his every action had the utmost weight. Every movement served the purpose of bringing him into the center of attention. In the course of time he acquired those characteristics which we see in him today. After he married he was continually occupied with impressing his wife with the fact that he was really bigger \& more important than she thought, while she was continually busied with showing him that his assertions concerning his value were untrue! Under these circumstances their marriage, which showed signs of disruption even during their engagement, could hardly develop favorably, \& ended finally in a social cataclysm\footnote{a sudden disaster or a violent event that causes change, for example a flood or a war.}. The patient came to the physician at this time -- since the break-up of his marriage served only to accentuate\footnote{{\it accentuate something} to emphasize something or make it easier to notice.} the dilapidation\footnote{the state in buildings and furniture of being old and in very bad condition.} of his already battered self-esteem. To be cured, he had to learn 1st from the physician how to know human nature, he had to learn how to appreciate the error he had made in life. \& this error, this wrong evaluation of his inferiority, had colored his entire life up to the time of his treatment.'' -- \cite[pp. 72--80]{Adler_human_nature}

\paragraph{The Graph of Life \& the Cosmic Picture.} ``When we demonstrate cases like these it is frequently convenient to show relationships between the childhood impressions \& the actual complaint, as presented by the patient; this is best done by a graph, similar to a mathematical formula. A line connecting 2 points represents such an equation. We will succeed in many cases in being able to plot this graph of life, the spiritual curve along which the entire movement of an individual has taken place. The equation of the curve is the behavior pattern which this individual has followed since earliest childhood. Perhaps there will be some readers who have the impression that we are attempting to belittle human fate by over-simplifying it, or that we have a tendency to deny that every human being is the master of his life, \& that we are denying free will \& judgment. So far as free will is concerned this accusation is true. Actually we see this behavior pattern, whose final configuration is subject to some few changes, but whose essential content, whose energy \& meaning, remain unchanged from earliest childhood, is the determining factor, even though the relations to the adult environment which follow the childhood situation may tend to modify it in some instances. In our examination we must ferret out\footnote{(informal) to discover information or to find somebody/something by searching carefully and completely, asking a lot of questions, etc.} the history of the earliest childhood days, because the impressions of early infancy indicate the direction in which a child has developed, as well as the direction in which he will respond in the future to the challenge of existence. In his response to the challenge of existence a child will utilize all the developed mental possibilities he has brought with him into life; the particular pressure he has felt in the days of earliest infancy will color his attitude toward life \& determine in a primitive fashion his world-view, his cosmic philosophy.

It should not surprise us to learn that people do not change their attitude toward life after their infancy, though its expressions in later life are quite different from those of their earliest days. It is important therefore to put an infant into relationships in which it will be difficult for him to assume a false concept of life. The strength \& resistance of his body is an important factor in this process. His social position, \& the characteristics of those who educate him, are almost equally important. Even though the response to life is automatic \& reflex in the beginning, type reactions become modified according to a certain purposiveness, in later life. In the beginning the factors of personal necessity condition his pain \& his happiness, but later he acquires the ability to evade \& circumvent the pressure of these primitive needs. This phenomenon occurs in the time of self discovery, approximately during the time that a child begins to speak of himself as ``I.'' It is during this time also that the child is already conscious that he stands in a fixed relationship to his environment. This relationship is by no means neutral, since it forces the child to assume a different attitude \& to adjust his relationships according to the demands which his world-view, \& his conception of happiness \& completeness, give him.

If we reaffirm what has been said concerning the teleology of the psychic life of mankind it will become increasingly clear to us that an indestructible unity must be a special token of this behavior pattern. The necessity for treating with a human being only as unit personality becomes increasingly clear in those cases where seemingly contrasting expressions of psychic trends are to be found. There are children whose behavior at school \& at home are diametrically opposite to 1 another, just as there are adults whose character traits appear so contradictory that we are deceived concerning their true character. In the same way the movements \& expressions of 2 human beings may be outwardly identical, yet when examined for their underlying behavior patterns, prove themselves to be entirely different. When 2 individuals seem to be doing the same thing, each one is really doing something distinct \& different, yet when 2 individuals are doing seemingly different things they may actually be doing the same thing!

Because of this possibility of many meanings we can never judge the expressions of the psychic life as single isolated phenomena; on the contrary, we must evaluate them according to the unit goal toward which they are directed. The essential meaning can be learned only when we know what value a phenomenon has in the entire context of a person's life. We can have an understanding of his psychic life only when we have reaffirmed the law that every expression of a man's life is an aspect of his unit behavior pattern.

When we have finally comprehended that all human behavior is based upon the striving for a goal, that it is conditioned by its end as well as by its beginning, then we can comprehend also wherein lies the possibility of the greatest mistakes. The source of these errors lies in the fact that every 1 of us utilizes his triumphs \& psychic assets according to his particular pattern, \& in the sense of a reinforcement of his individual life pattern. This is possible only because we do not test anything, but receive, transform, \& assimilate all perceptions in the shadow of our own conscious, or the depths of our unconscious. Science alone can illuminate the process \& make it comprehensible; science alone is finally able to modify it. We shall conclude our exposition at this point with an example in which we will analyze \& explain every phenomenon by those Individual Psychological concepts which we have already learned.

A young woman comes as a patient \& complains of her unconquerable dissatisfaction with life, which arises, as she believes, in the fact that her whole day is taken up with a great number of duties of all kinds. Externally we can see in her a hasty being, with restless eyes, who complains of the great unrest which seizes her whenever she must do some simple duty. From her family \& friends we learn that she takes everything seriously \& seems to be breaking under the burden of her work. The general impression that we get is that of a person who takes everything very seriously, a characteristic which is common to many people. 1 member of her family gives us the clue in saying, ``She is always making a big fuss over everything!''

Let us test this tendency to consider every simple task a particularly hard \& important one, by attempting to imagine what kind of an impression this behavior would make upon a group of people, or in the marriage relationship. We cannot help feeling that such a tendency simulates an appeal to the environment not to force any further work upon her since she can no longer do the most elementary tasks.

Our knowledge of this woman's personality is not yet adequate. We must stimulate her to further expressions of herself. One must proceed by innuendo \& with the proper delicacy in such examinations. There must be no attempt to dominate the patient as this would only serve to make her belligerent\footnote{aggressive \& unfriendly, $=$ {\it hostile}.}. Once her confidence is won \& the possibility of conversation is given, we come to the conclusion that her whole being is concerned with but a single goal. Her behavior shows that she is attempting to demonstrate to someone, probably her husband, that she cannot bear any further obligations or responsibilities, that she must be treated carefully \& with tenderness. We can further suspect \& imagine that all this must have begun definitely at some time in the past, \& that such demands must have been made of her. We succeed in stimulating her to the affirmation that many years ago she had to live through a period in which nothing was more wanting than tenderness. Now we can understand her behavior better; it is a reinforcement of her desire for consideration, \& an attempt to prevent the recurrence of a situation in which her hunger for warmth \& affection might somehow remain unsatisfied.

Our findings are clinched by a further explanation on her part. She tells of a friend who is in many ways her opposite, who is living in an unhappy marriage from which she desires to escape. Once she met her friend at a moment when she was standing, book in hand, telling her husband in a bored voice that she really did not know whether she would be able to prepare dinner that day. This irritated her husband so that he criticized her whole personality in harsh terms. To this occurrence our patient added: ``When I think of this occurrence I think that my method is much better. No one can reproach me in this way because I am overburdened with work from morning until night. If a luncheon is not prepared on time at my house no one can say anything to me because my time is full of haste \& constant excitement. Should I give up this method now?''

One can understand what is going on in this soul. In a relatively innocuous way she attempts to attain a certain superiority, but remain at the same time beyond every reproach by pleading constantly for tender treatment. Since this mechanism is successful it seems hardly reasonable to ask her to forego it, but there is more to her behavior than just this. Her appeal for tenderness (which at the same time is an attempt to dominate others) can never be made drastic enough. Contradictions of all kinds occur in this connection. Should anything be lost in the house there is a consequent ``much ado about nothing''; subsequently she has so much business that she is constantly suffering from headaches, \& she can never sleep quietly because she is under the necessity of putting her activities in the right light. An invitation which she may get is in itself an important occasion. Enormous preparations are necessary for its acceptance. Since the least activity appears to her inordinately large, paying a call is a difficult labor which demands hours \& days to complete. We can predict with some certainty that she will either send her regrets or, at the very least, come late. The social feeling in the life of such a person can never go beyond certain limits.

In married life there are a number of relationships which assume a peculiar significance through this appeal for tenderness. It is conceivable for instance that a husband must be absent because of his business, or that he must make visits by himself, or that he must appear at meetings of societies to which he belongs. If he left his wife at home at these times would this not be a breach of tenderness \& consideration? At 1st we might say, \& very often this is the case, marriage justifies keeping a husband at home as much as possible. Pleasant as this obligation might seem in part, in actuality it signifies insupportable difficulties for any man who has a profession. Dis-harmony would appear unavoidable in such cases \& it occurred quickly in this one. The husband attempted occasionally to come to his bed late at night without disturbing his wife only to be surprised to find her still awake, greeting him with reproachful glances.

We need not picture here all the well known situations of this kind. Nor should we overlooked the fact that it is not alone the petty vices of women which we are discussing, for there are as many men whose attitude is similar. We are simply concerned with showing that the demand for especially consideration may occasionally take a different course. In our case the following procedure would occur: If on some occasion the husband has to spend an evening out, his wife tells him that since he goes into society so seldom, he should not come home too early. Although she says this in a jocular\footnote{1. humorous; 2. (of a person) enjoying making people laugh, $=$ {\it jolly}.} tone her words have very serious meaning. It seems to negate the previous impression but when we observe more closely we can see the connection. The wife is clever enough not to act too strictly. Outwardly she is utterly charming. There is no blemish upon her character, \& she interests us only in a psychological way. The real significance of her words to her husband lies in the fact that it is the {\it wife} who has given an ultimatum. Now since {\it she} has permitted it, he may stay out late whereas she would be dreadfully hurt \& slighted if he had remained away for reasons of his own. Her words drape a veil over the whole situation. She has become the directing partner; \& her husband, even though he is only fulfilling his social obligations, is made dependent upon the wish \& will of his wife.

Now let us connect this hunger for particular tenderness with our newly won concept that this woman can bear a situation only when she herself commands. We suddenly become aware that throughout her whole life she has been actuated by an impulse never to play the 2nd fiddle, always to maintain her dominancy, never to be thrust from her secure position by any reproach, \& always to remain in the center of her little environment. We will find this movement in every situation in which we find her; e.g., when she has to get a new maid, she becomes highly excited. Clearly she is concerned to know whether she will be able to maintain the same dominance over the new servant that she was able to hold over the old. In like manner, when she is about to leave the house for a walk, she leaves a sphere where her dominance is unconditionally secure, \& goes out into the world, on the street where suddenly nothing is under the shadow of her dominance, where she has to dodge every automobile, indeed where she plays a very subordinate role. The cause \& meaning of her tension becomes perfectly clear when one can understand what tyranny she exercises at home.

These characteristics may often appear in such a pleasant pattern that at the 1st glance one would never think that the person was suffering. On the other hand this suffering can reach a very high degree. Just imagine this tension exaggerated \& enlarged. There are human beings who are afraid of using a street car because in a street car they are not the masters of their own will \& this may go so far that they finally do not leave their homes at all.

A further development in our case is an instructive example of the influence which childhood impressions exercise in the life of an individual. We cannot deny the fact that this woman, from her standpoint, is perfectly right; if one's attitude, \& one's whole life are directed with unheard of intensity toward the acquisition of warmth, respect, honor, \& tenderness, then to act as though one were constantly overloaded \& constantly exhausted is not a bad means to this end. No other way will always serve to keep off criticism, \& simultaneously force the environment to be gentle, \& avoid everything likely to disturb a wavering psychic equilibrium.

If we go back a considerable period in the life of our patient we learn that even in school, whenever she could not do her homework, she became extraordinarily excited \& forced her teacher in this way to be very gentle with her. To this she adds, she was the oldest of 3 children \& was followed by a boy \& then by a sister. She was constantly at war with her brother. He always appeared as the preferred one. She angered herself especially because people paid more attention to his school work, whereas her work, (\& she had originally been a good pupil) was met with a certain indifference as to her accomplishments. Finally she could hardly bear it any longer \& was forever nagging to know why her accomplishments were not judged of equal value.

Thus we can understand that this young girl was striving for equality, \& that from earliest childhood she had had a feeling of inferiority which she was attempting to overcome. Her compensation in school was made in such a manner that she became a bad pupil. She attempted to outdo her brother by means of bad school reports! These are no high ethics, but in her childish interpretation she acted rationally, for the attention of her parents would be more often directed to her in this way. Some of her tricks must have been conscious because she declared quite clearly that she {\it wanted} to be a bad pupil!

Her parents, however, did not trouble themselves in the least about her failures in school. \& now something interesting happened. She suddenly showed marked success in her studies, for now her younger sister entered upon the scene in a new role! This younger sister also had failed in school but her mother troubled herself almost as much about her failure as she had about the brother, \& for the peculiar reason that whereas our patient had had bad reports only in her studies her sister got bad reports in conduct \& behavior. She was thus able to gain her mother's attention more easily since bad reports in conduct have an entirely different social effect than merely bad reports in studies. They were bound up with peculiar emergencies that forced the parents to occupy themselves more with their child.

The battle for equality was temporarily lost. Now the loss of a battle for equality never leads to a permanent peace. No human being can bear such a situation. Hence we shall constantly find new tendencies \& activities contributing toward the formation of her character. We can now understand the meaning of her great to-do, her constant haste, her desire to show herself under pressure, somewhat better. It was meant originally for her mother \& was intended to compel her parents' attention to her as well as to her brother \& sister; at the same time it was a reproach to her parents that they treated her worse than the others. The fundamental attitude created at that time has remained until today.

We can go back even farther in her life. She remembers as a particularly vivid occurrence of her childhood, that she wanted to hit her brother who had just been born, with a piece of wood, \& that only the care of her mother had prevented her from doing great damage. At this time she was 3 years old. This little girl had discovered (even at that time) the cause for her neglect \& lesser evaluation was that she was only a girl. She remembers quite vividly that the wish to be a boy was expressed countless times. The arrival of her brother not only forced her out of the warmth of her nest but she was particularly insulted because as a boy he was treated much better than she had ever been. In her striving to compensate for this defect she happened upon the method of appearing always overloaded with work.

Let us interpret a dream now to show how deeply this behavior pattern is anchored in the soul. This woman dreamt that she was at home conversing with her husband, but her husband did not look like a man but appeared as a woman. This detail is symbolic of the pattern with which she approaches all her experiences \& all her relationships. The dream means that she has found equality with her husband. He is no longer the dominant male as her brother once was, he is already like a woman. There is no difference in elevation between them. In her dream she has achieved that which she has always wished since her childhood.

In this way we have succeeded in joining 2 points in the soul life of a human being. We have discovered her style of life, her life curve, her behavior pattern, \& from this we can acquire a unified picture which we might sum up as follows: we are dealing here with a human being who strives to play the dominant role by amiable means.'' -- \cite[pp. 80--90]{Adler_human_nature}

\subsubsection{The Preparation For Life}
``1 of the fundamental tenets of Individual Psychology is that all the psychic phenomena can be considered as preparations for a definite goal. In the configuration of the soul life which we have previously described we can see a constant preparation for the future in which the wishes of the individual appear fulfilled. This is a general human experience \& all of us must go through this process. All the myths, legends \& sagas which speak of an ideal future state concern themselves with it. The convictions of all peoples that there was once a paradise, \& the further echo of this process in the desire of humanity for a future in which all difficulties have been overcome, may be found in all religions. The dogma of the immortality of the soul, or its re-incarnation, is a definite evidence of the belief that the soul can arrive at a new configuration. Every fairy tale is a witness of the fact that the hope of a happy future has never failed in mankind.'' -- \cite[p. 91]{Adler_human_nature}

\paragraph{Play.} ``There is in the child life an important phenomenon which shows very clearly the process of preparation for the future. It is play. Games are not to be considered as haphazard ideas of parents or educators, but they are to be considered as educational aids \& as stimuli for the spirit, for the fantasy, \& for the life-technique of the child. The preparation for the future can be seen in every game. The manner in which a child approaches a game, his choice, \& the importance which he places upon it, indicate his attitude \& relationship to his environment \& how he is related to his fellow men. Whether he is hostile or whether he is friendly, \& particularly whether he has the tendency to be a ruler, is evident in his play; \& in observing a child in his play we can see his whole attitude toward life. Play is of utmost importance to every child. The discovery of these facts which teach us that the play of children is to be considered as a preparation for the future is due to Gross, a professor of pedagogy, who discovered the same tendencies in the play of animals.

But we have not exhausted all the view points as to the nature of play, with the concept of preparation. Above all else games are communal exercises, they enable the child to satisfy \& fulfill his social feeling. Children who evade games \& play are always open to the suspicion that they have made a bad adjustment to life. These children gladly withdraw themselves from all games, or when they are put on the playground with other children usually spoil the pleasure of the others. Pride, deficient self-esteem \& the consequent fear of playing one's role badly are the chief reasons for this behavior. In general by watching a child at play we shall be able to determine with great certainty the quantum of his social feeling.

The goal of superiority, another factor obvious in play, betrays itself in the child's tendency to be the commander \& the ruler. We can discover this tendency by watching how the child pushes himself forward \& to what degree he prefers those games which give him an opportunity to satisfy his desire to play the leading role. There are very few games which do not have at least 1 of these factors, preparation for life, social feeling, or the striving for domination, incorporated in them.

There is, however, 1 other factor which is present in play. It is the possibility that the child can express himself in a game. The child is more or less placed upon his own in play, \& his performance is stimulated by his connection with the other children. There are a number of games which especially emphasize this creative bent. In the preparation for a future profession those plays which carry in themselves the possibility for the exercise of the creative spirit of the child are especially important. In the life histories of many people, it has happened that they have made dresses for dolls in their childhood, \& later made dresses for adults.

Play is indivisibly connected with the soul. It is, so to speak, a kind of profession, \& must be considered as such. Therefore, it is not an insignificant matter to disturb a child in his play. Play should never be considered as a method of killing time. In regard to the goal of preparing for the future, every child has in him something of the adult he will be at sometime. Thus in the appraisal of an individual we can draw our conclusions more easily when we have a knowledge of his childhood.'' -- \cite[pp. 91--93]{Adler_human_nature}

\paragraph{Attention \& Distraction.} ``Attention is 1 of the characteristics of the soul which is in the very forefront of human accomplishments. When we bring our sense organs to the consideration of some particular event outside or inside our person, we have a feeling of particular tension, which does not spread over our entire body, but is limited to a single sense organ, as for instance, the eye. We have the feeling that something is being prepared. In the case of the eye the direction of the ocular axis gives us this particular feeling of tension.

If attention calls forth a particular tension in any part of the soul or in our motor organism, then other tensions are at the same time excluded. Thus as soon as we wish to be attentive to anyone thing we desire to exclude all other disturbances. Attention, so far as the soul is concerned, means an attitude of willingness to make a special bridge between ourselves \& a definite fact, a preparation for offense, which grows out of our necessity, or out of an unusual situation which demands that our whole power be directed toward a particular purpose.

Every human being, if we exclude sickness \& feeble-mindedness, possesses the ability to pay attention, but inattentive persons are frequently found. There are a number of reasons for this. In the 1st place, fatigue or sickness are factors which influence the ability to pay attention. Further, there are other individuals whose deficient attention is due to the fact that they do no want to pay attention, because the object to which they should be attentive does not fit into their behavior pattern; on the other hand their attention immediately awakens when they are considering some matter which is germane to their style of life. A further reason for deficient attention is to be found in the tendency toward opposition. Children are very easily given to opposition, \& it often happens that such children answer ``No'' to every stimulus which is offered them. It is necessary for their opposition to become open. It is the duty of the educator \& of educational tact to reconcile such a child by relating the study which he must learn to his behavior pattern, \& making it germane to his style of life.

Some see \& hear \& perceive every change. Some approach life entirely with their eyes; others entirely with their auditory apparatus. Some see nothing, take notice of nothing, \& are not to be interested in visual things. We may find an individual inattentive when his situation would warrant his utmost interest because his more sensitive receptors are not stimulated.

The most important factor in the awakening of attention is a really deep rooted interest in the world. Interest lies in a much deeper psychic stratum than attention. If we have interest, then it is self-understood that we should also pay attention; \& where interest exists, an educator need not concern himself with attention. It becomes a simple instrument with which one conquers a field of knowledge for a definite purpose. No one has ever developed without making mistakes in the process. It follows that the attention is likewise involved when some such mistaken attitude has become fixed in an individual, \& it thus happens that attention is directed toward things which are not important in the preparation for life. When the {\it interest} is directed towards one's own body, or towards one's own power, one is {\it attentive} wherever these interests become involved, wherever there is something to be won, or wherever one's power is threatened. Attention can never be linked with something extraneous so long as some new interest is not substituted in place of the power interest. One can observe how children become immediately attentive when their recognition \& significance are in question. Their attention on the other hand is easily extinguished when they have the feeling there is ``nothing in it'' for them.

A defective attention actually means nothing more than that a person prefers to withdraw from a situation, to which he is supposed to pay attention. It is incorrect, therefore, to say that someone cannot concentrate himself. It can easily be proved that he concentrates very well, but always on something else. Lack of will power \& lack of energy are similar to the inability to concentrate. We usually find an obdurate will \& an indominate energy expressed in a different direction in these cases. Treatment is not simple. It can be attempted solely by changing the entire style of life of the individual. In every case we can be sure that we are dealing with a defect only because another goal is being pursued.

Not infrequently inattention becomes a permanent characteristic. We often meet individuals who have been given a definite task which they have declined to do, which they have only partially accomplished, or have fully evaded, with the result that they are always a burden to someone else. Their constant inattention is a fixed character trait, which appears as soon as they are under the necessity of doing something which is demanded of them.'' -- \cite[pp. 93--96]{Adler_human_nature}

\paragraph{Criminal Negligence \& Forgetfulness.} ``We usually speak of criminal negligence when the safety or health of an individual is threatened by neglect in the application of necessary precaution. Criminal neglect is a phenomenon which demonstrates the utmost degree of inattention. Such defective attention is based on a defective interest for one's fellow men. One can determine whether children think only of themselves, or whether they take into consideration rights of others, by watching for traits of negligence in their games. Such phenomena are definite standards of the communal consciousness \& the social feeling of a human being. When the social feeling has been insufficiently developed, one acquires sufficient interest for his fellows only with the greatest difficulty, even under threat of punishment; whereas in the presence of a well-developed community consciousness, this interest is self-evident.

Criminal neglect, therefore, amounts to a defective social feeling, yet we must not be too intolerant lest we forget to investigate why an individual does not possesses the interest in his fellow-men which we might expect of him.

We can produce forgetfulness by setting limits to our attention, just as we can arrange the loss of valuables. Despite the presence of the possibility for greater tension -- i.e., interest -- this interest may be so dampened by displeasure, that a loss or memory lapse is initiated, or at least facilitated thereby. Such is the case, e.g., when children lose their school books. It is always easy to prove that they have not yet become accustomed to their school surroundings. Housewives who are constantly losing or misplacing their keys are usually women who have never become friendly with the profession as housewife. Forgetful people usually prefer not to revolt openly, yet a certain lack of interest in their tasks is betrayed by their forgetfulness.'' -- \cite[pp. 96--97]{Adler_human_nature}

\paragraph{The Unconscious.} ``Our descriptions have often shown individuals who are not conscious of the meaning of the phenomena of their psychic life. Seldom will an attentive man be able to tell you why he sees everything at once. Certain psychic faculties are not to be sought in the realm of consciousness; although we can consciously force our attention to a certain degree, the stimulus to that attention lies not in consciousness, but in our interests, \& these, again, lie for the most part in the sphere of the unconscious. Taken in its largest scope, this is at once an aspect \& an important factor in the soul life. We may seek \& find the behavior pattern of a man in the unconscious. In his conscious life we have but a reflection, a negative, to deal with. A vain woman usually has no knowledge of her vanity in most of the instances in which she exhibits it; quite to the contrary, she will behave so that only her modesty will be apparent to everyone. It is not necessary to know that one is vain to be vain. Indeed for the purposes of this woman, it would be quite futile for her to know that she is vain, since if she knew she were vain, she could not continue to be vain. We can acquire a certain dramatic security is not seeing anything of our own vanity solely by directing our attention to something extraneous or irrelevant. The whole process takes place in the dark. Attempt to talk to a vain man about his vanity \& you will find it very difficult to achieve a conversation on the subject. He may show a tendency to evade the matter, to circumlocute, lest he be disturbed; this can but make us more certain of our opinion. He wants to play his little game, \& immediately assumes a defensive attitude when someone inadvertently attempts to lift the veil from his little trick.

Human beings may be differentiated into 2 types; those who know more concerning their unconscious life than the average, \& those who know less; i.e., according to the extent of their sphere of consciousness. In a great many cases, we will find coincidently that an individual of the 2nd type concentrates upon a small sphere of activity, whereas the individuals of the 1st type are connected with a many-sided sphere, \& have large interests in men, things, events, \& ideas. Those individuals who feel themselves pushed to the wall will naturally satisfy themselves with a small section of life, since they are foreign to life, \& cannot see its problems with as much clarity as those who are playing the game according to the rules. They make bad teammates. They will not be so capable of understanding the finer things in life. Because of their very limited interest in living, they perceive but an insignificant segment of its problems for the reason that they fear a broader view would be synonymous with a loss of personal power. As to individual occurrences in life, we can often discover that an individual knows nothing of his capabilities of living, because he undervalues himself. We will find also that he is not sufficiently oriented concerning his short-comings; he will consider himself a good man, whereas in reality, he does everything out of egoism; or vice-versa, he will consider himself an egoist in instances in which a closer analysis shows him to be a very good person indeed. It really does not matter what you think of yourself, or what other people think of you. The important thing is the general attitude toward human society, since this determines every wish \& every interest \& every activity of each individual.

We are dealing again with 2 types of human beings. In the 1st class are those who live a more conscious life, who approach the problems of life without blinders on their eyes, in an objective manner. The 2nd class approaches life with a prejudiced attitude, \& sees only a small part of it. The behavior \& speech of individuals of this type are always directed in an unconscious manner. 2 human beings living with 1 another may find difficulties in life because 1 of them is constantly in opposition. This is not an uncommon occurrence. It is perhaps even less uncommon that both parties are in opposition. Each party knows nothing about his opposition, believes himself right, \& give arguments to show that he is the champion of peace \& harmony. The facts nevertheless belie his words. In actuality it is impossible for him to say a single word without attacking his partner on the flank with an opposing remark, albeit his attack is externally unnoticeable. On closer inspection we find that he has given himself up to a hostile \& belligerent attitude throughout his life.

Human beings develop powers in themselves which are constantly at work, though they know nothing of them. These faculties lie hidden in the unconscious, influence their lives \& occasionally lead to bitter consequences when they are not discovered. Dostoyevsky described such a case so beautifully in his novel {\it``The Idiot''} that it has been the marvel of psychologists ever since: during a social gathering a lady cautions the duke who is the hero of the novel, not to upset an expensive Chinese vase which stands near him, in a taunting tone. The duke assures her that he will take care, but a few minutes later the vase lies on the ground, shattered into pieces. No one in the group saw a mere accident in this occurrence; everyone felt it was a very consequent action, quite in keeping with the whole character of this man who felt himself insulted by the lady's words.

In judging a human being we must not be guided solely by his conscious actions \& expressions. Often little details in his thinking \& behavior of which he is not conscious will give us a better clue to his real nature.

People e.g. who practice such unpleasant activities as nail-biting or nose-boring do not know that they betray the fact that they are stubborn human beings in doing so, since they do not understand the relationships which have led them to these traits. Yet it is perfectly clear to us that a child must have been scolded repeatedly because of these habits; if, then, he does not give them up, despite the scoldings, he must be a stubborn human being! Were we more expert in our observation, we would have to draw very far reaching conclusions concerning any human being, by watching for such insignificant details, in which his whole being is mirrored.

These 2 following cases will show how important it is to the psychic economy that events which are unconscious, remain in the unconscious. The human soul has the capability of directing the consciousness, i.e., of making conscious of directing the consciousness, i.e., of making conscious that which is necessary from the standpoint of some psychic movement, \& vice-versa, to allow something to remain in the unconscious or make it unconscious, whenever this would seem advisable for the maintenance of the individual's behavior pattern.

The 1st case is that of a young man, a firstborn son, who grew up with a younger sister; his mother died when he was 10 years old, \& from that time his father, who was a very intelligent, well-meaning, ethical man, had to be the educator. The father spent most of his efforts developing his son's ambition \& spurring him on to greater activity. The boy tried to be the leader in his classes, developed himself extraordinarily well, \& so far as his ethical \& scientific qualities were concerned, always took 1st place, much to the joy of his father, who expected him to play an important role in life, from the very 1st.

In the course of time this young man developed certain traits which caused his father sorrow, \& these he tried to change. The boy's sister grew up to be his obdurate\footnote{refusing to change your mind or your actions in any way, $=$ {\it stubborn}.} rival. She also developed very well, although she was satisfied with utilizing the weapons of weakness for her triumphs, while she enlarged her significance at the cost of her brother. She had acquired a considerable facility in household tasks, which made competition difficult for her brother. As a boy, he found it very difficult to achieve, in domesticity, that recognition \& significance which he had so easily won in other fields of endeavor. The father soon noticed that his son was acquiring a peculiar social life, which became the more evident as his puberty approached. As a matter of fact he had no social life. He was hostile to all new acquaintanceships \& where these acquaintanceships concerned girls, he simply ran away. At 1st his father saw nothing extraordinary in this, but as time went on the boy's social reactions acquired such dimensions that he hardly went out of the house, \& even little walks, except in the late twilight, were unpleasant to him. He became so shut in that he refused, finally, to greet even his old acquaintances, although his attitude in school \& towards his father remained beyond criticism.

When it had gone so far that no one could bring him anywhere, the father brought this boy to the physician. A few consultations sufficed to discover the cause of the difficulty. This boy believed that his ears were small \& that therefore everyone considered him very ugly. As a matter of fact this was not the case. When his objection was overruled \& he was told that his ears were in nowise different from those of other boys, \& it was shown him that he was using this as an excuse to withdraw from the company of human beings, he added further that his teeth \& his hair also were ugly. This also was not the case.

On the other hand it was easily discovered that he was inordinately ambitious. He was well aware of his ambition \& believed that his father, who had constantly stimulated him to greater \& greater activity so that he might achieve a high position in life, had produced this trait in him. His plans for the future came to their climax in his desire to play the role of a hero of science. This would not be so remarkable were it not that with it was combined a tendency to avoid all the obligations of humanity \& fellowship. {\it Why did this boy make use of such very childish arguments?} Had these arguments been right they might have justified him in approaching life with a certain caution \& anxiety, because it is undoubtedly true that an ugly man encounters many difficulties in our civilization.

Further examination showed that this boy followed a particular goal with his great ambition. Formerly he had always been the 1st one in his class \& he wanted to remain the 1st one. To achieve such a goal one has certain instruments such as concentration, industry \& the like, at hand. These were not enough for him. He attempted to exclude everything which seemed unnecessary, out of his life. He might have expressed himself somewhat like this: ``Since I am going to become famous \& since I am going to dedicate myself entirely to my scientific labors, I must exclude all social relationships as unnecessary.''

But he neither said nor thought this. On the contrary he took the unessential element of his allege ugliness \& utilized it for the attainment of his purpose. The elevation of this insignificant fact acquired value in his scheme of things in that it justified him in doing what in reality he wanted to do. All he needed to do now was to have courage to argue falsely, to exaggerate his ugliness, in order to pursue his secret purpose. Had he said that he wished to live like an ascetic hermit in order to attain his goal of being the 1st, his argument would have been transparent to everyone. Although unconsciously he was dedicated to the idea of playing the heroic role, he was consciously unaware of his purpose.

That he wished to hazard everything else in life \& gain this 1 point had never entered his head. If he had taken this into his consciousness \& decided openly to stake everything in life in order to become a scientific hero, he could not have been as sure of himself as if he were to accomplish his purpose by saying that he was an ugly man \& dared not go into society; in addition anyone who would say openly that he wanted forever to be 1st \& the greatest, \& was willing to sacrifice all human relationships for the sake of his goal, would make himself ridiculous in the eyes of his fellows. It would be too fearful a thought, a thought which one dared not think. There are certain ideas which one cannot hold too openly, both for the sake of others \& for the sake of oneself. For this reason the guiding idea of this boy's life had to remain in his unconscious.

If now we make obvious to such an individual the mainsprings of his life, \& demonstrate to him tendencies which he dared not look at in himself lest he lose his behavior pattern, we naturally disturb his entire psychic mechanism. What this individual has been trying at all costs to prevent, now happens! His unconscious thought processes become clear \& transparent! Thoughts which were not to be thought, ideas which one dared not retain, tendencies which, if conscious, would disturb our entire behavior, are laid bare. It is a universal \& human phenomenon that everyone seizes upon those thoughts which justify him in his attitude \& rejects every idea which might prevent him from going on. Human beings dare only those things which in their interpretation of the world are valuable to them. That which is helpful we are conscious of; whatever can disturb our arguments we push into the unconscious.

The 2nd case is the history of a very capable young boy whose father, a teacher, constantly spurred his son on to be the 1st in his class. In this cases, too, the early days of this child were a series of victories. Wherever he appeared he was always the conqueror. He was 1 of the most charming members of his society \& he had several close friends.

A great change occurred in his 18th year. He lost all his pleasure in life, was depressed, distracted, \& went to great lengths to withdraw from the world. No sooner would he make a friendship than he broke it. Everyone found a stumbling block in his behavior. His father however hoped that his shut-in life would enable him to dedicate himself more intensely to his studies.

During the treatment of this boy he complained constantly that his father had robbed him of all joy in life, that he could find no self-confidence nor courage to go on with life, \& that there was nothing left for him to do but to sorrow through the rest of his days in solitude. His progress in his studies had already become slower \& he was failing in College. He explained that the change in his life had begun on the occasion during a social gathering in which his ignorance of modern literature had made him the object of ridicule among his friends. The repetition of similar experiences caused him to begin his isolation \& gave him occasion to assume a position outside society. He was ruled by the idea that his father was to blame for his misfortune, \& their relationship became worse day by day.

These 2 cases are similar to each other in many respects. In the 1st case our patient was shipwrecked on the resistance of his sister, whereas in the 2nd it was the belligerent attitude toward a father who was at fault. Both patients were led on by an idea which we have been accustomed to call the heroic ideal. Both of them had become so intoxicated with their heroic ideal that they had lost all contact with life, had become discouraged \& would have liked nothing better than to withdraw entirely from the struggle. But we cannot believe that our 2nd boy would ever have said to himself: ``Since I cannot continue this heroic existence I shall withdraw from life \& embitter the rest of my days!''

To be sure, his father was wrong \& his education was bad. It was quite evident that he had eyes for nothing but his bad education, of which he constantly complained, since he wanted to justify himself in his withdrawal, by assuming that his education had been so bad that withdrawal from society alone remained a solution of his problem. In this way he achieved a situation in which he suffered no more defeats, \& he was able to credit his father with the total blame for his misfortune. Only in this way was he able to save a fraction of his self-esteem for himself \& satisfy his striving for significance. He had a glorious past \& his future triumphs had been stopped only by the fatal fact that his father, because of his bad pedagogy, had hindered him from developing to even more brilliant accomplishments.

In a way we might say that something like this train of thought remained unconsciously in his mind: ``Since I now stand closer to the firing front of life, \& since I realize that it will no longer be so easy always to be the 1st, I shall make every effort to withdraw entirely from life.'' Yet such an idea is clearly unthinkable. No one could say such a thing, but an individual can act {\it as if} he had taken this thought to heart. This is accomplished by making use of still further arguments; by busying himself entirely with the educational mistakes of his father he succeeds in evading society, \& avoids all necessary decisions in life. Had this train of thought become conscious to him his secret behavior would, of necessity, have been disturbed. Therefore it remained unconscious. {\it How could anyone say that he was an untalented human being when he had such a glorious past?} To be sure none could blame him now if he succeeded to no new triumphs! The pernicious\footnote{having a very harmful effect on somebody/something, especially in a way that is not easily noticed.} influence of his father's educational efforts was not to be laid aside. The son was judge, claimant, \& defendant all in his own person. {\it Should he now give up such a favorable position?} He knew too well that his father was to blame only so long as he, the son, wanted it, so long as he plied the lever which he held between his hands.'' -- \cite[pp. 97--107]{Adler_human_nature}

\paragraph{Dreams.} ``It has long been maintained that one could draw conclusions concerning the personality-as-a-whole from the dream of an individual. Lichtenberg, a contemporary of Goethe, once said that one could guess the character \& essence of a human being better from his dreams than from his actions \& words. This is saying a little too much. We have the standpoint that one must utilize {\it single} phenomena of the psychic life with the greatest care \& only in connection with other phenomena. We therefore will draw conclusions concerning his character from the dreams of an individual only when we can find additional supporting evidence in other characteristics, to substantiate our interpretation of the dream.

The interpretation of dreams dates from prehistoric times. The research of various epochs\footnote{1. (formal or literary) a period of time in history, especially one during which important events or changes happen, $=$ {\it era}; 2. (geology) a length of time that is a division of a period.} in the developmental history of culture, especially as evidenced in myths \& sagas, leads us to the conclusion that in times by-gone people were far more concerned with the interpretation of dreams than we are today. We also find a much better understanding of dreams on the part of the average man of those days than is the case today. One need but recall the enormous role of dreams in the life of the ancient Greeks, or the fact that Cicero wrote a book about them, or remind oneself of the many dreams told in the Bible, to prove this point. \& more. The Bible dreams are either cleverly interpreted, or they are related as though it were self-understood that everyone would then interpret them correctly \& understand them. This is the case in the instance of Joseph's dream of the sheaves which he told his brothers. In the Nibelungen sagas, which originated in an entirely different culture, furthermore, we can conclude that dreams were used as evidence.

If we busy ourselves with dreams as a means of approaching \& learning something of the human soul, we shall hardly view the problem from the standpoint of those investigators who seek in the dream \& in dream interpretation fantastic \& supernatural influences. We shall depend upon the evidence of dreams only when we can be justified \& strengthened in our assertions by other far reaching observations.

The tendency to believe that dreams have a particular meaning for the future, persists even today. There are idealists who go so far as to allow themselves to be influenced by their dreams. In this way 1 of our patients tricked himself into avoiding every honorable occupation \& devoted himself to gambling on the Exchange. He always gambled according to dreams which he had. He had collected historical evidence to prove he had always had misfortune whenever he did not follow 1 of his dreams. To be sure, he would dream of nothing except that which was the object of his constant waking attention. In this way he patted himself on the back, so to speak, in his dream, \& was enabled for a considerable period of time to say that he had won very much under the influence of his dream. Some time later he explained that he placed no value whatever upon his dreams. It seems that he had lost all his money. Since this happens frequently to stock market operators even without dreams we see no miracle at work here. An individual who is intensely interested in some particular task is pursued by the necessity of solving this problem even at night. Some people do not sleep at all \& constantly follow their problem while awake, others sleep but busy themselves with their plans in their dreams.

This peculiar phenomenon which occupies our thoughts during our sleep, is nothing more than the bridge from yesterday to tomorrow. If we know what attitude an individual takes towards life in general, how he bridges from the ``now'' into the ``then,'' as a rule we will be able to understand also the peculiarities of his bridge structure in his dreams \& be able to make valid conclusions from it. In other words it is the general attitude toward life which is at the basis of all dreams.

A young woman has the following dream: she dreams that her husband has forgotten her wedding anniversary \& she reproaches him for it. This dream may mean several things. If such a problem can occur at all it immediately shows us that this marriage is marked by certain difficulties; the wife feels herself neglected. She explains however that she also forgot about the wedding anniversary but it was {\it she} who finally remembered it whereas her husband had to be reminded of it by her. She is the ``better half.'' To a further question she said that actually nothing like this has ever happened \& that her husband has always remembered the wedding anniversary. Therefore in the dream we see her tendency to be anxious for the future: something like this {\it might} happen. We can further conclude that she is given to making reproaches, to using arguments which are intangible\footnote{something that does not exist as a physical thing but is still valuable to a company.}, \& to nagging her husband for things which {\it might} occur.

Still we could not be sure of our interpretation if we did not have other evidence at hand which would reinforce our conclusions. Asked about her earliest childhood remembrance, this woman narrated an event which had always remained in her memory. When she was a 3 year old child her aunt presented her with a carved wooden spoon of which she was very proud; but once as she was playing with it, it fell in a brook \& floated away. She sorrowed about this event for many days in such a way that everyone in her environment was concerned with it.

The dream might lead us to assume that she was now again thinking of the possibility that her marriage also might float away from her. What if her husband {\it should} forget about her wedding anniversary?

Another time she dreamt that her husband led her up into a high building; the stairs grow more \& more steep. Thinking that she has perhaps climbed too high she becomes terribly dizzy, has an attack of anxiety, \& faiths. One may experience a similar sensation during the waking life, especially if one suffers from dizziness in high places in which the fear is less that of the height than of the depth. By connecting the 2nd dream with the 1st one \& melting them together the thought, feeling, \& content of these dreams give a clear impression that this is a woman who is anxious about falling, who is afraid of mischief or calamity\footnote{an event that causes great damage to people's lives, property, etc., $=$ {\it disaster}.}. We can imagine that the warning affection of her husband, or something similar, would be such a calamity. What would happen if her husband in some way were not compatible? What if their married life were disturbed? Scenes might occur, fights take place, which might end with the wife's fainting as though lifeless. This actually occurred once during a family argument!

Now we come nearer to the meaning of the dream. It is quite a matter of indifference in which material the thought \& emotion content of the dreams expresses itself, or what instruments are used for this expression, so long as the material is in any way useful \& {\it some} expression is assured. In the dream the life problem of an individual is expressed in a simile\footnote{a word or phrase that compares something to something else, using the words like or as, for example a face like a mask or as white as snow; the use of such words and phrases.}. It is as though she said, ``Do not climb too high so that you will not fall too far!'' It may be well to recall the reproduction of a dream in the ``Marriage Song'' of Goethe. A knight comes home from the country \& finds his castle deserted. Tired out he falls into his bed \& in his dream he sees little figures coming out from under his bed \& notices a marriage ceremony among these dwarfs. He is agreeably pleased by his dream. It is as though he wanted to corroborate in his thoughts the need for finding a woman. What he saw here in miniature occurred later in reality as he celebrated his own marriage.

We find many well known elements in this dream. In the 1st place the preoccupation of the poet with his own marriage is hidden behind it. We can see further how the dreamer, in his utter need strikes an attitude toward his contemporary situation in life. This situation demands marriage. He occupies himself in his dream with the problem of marriage \& on the following day decides that it would be better if he too, were to get married.

Now let us consider a dream of a 28-year-old man. The movement of the dream, changing from ascent to descent like the temperature curve of a fever, indicates very clearly the psychic movements with which the life of this man is filled. The feeling of inferiority from which arise the tendencies \& strivings for power \& for dominance are easily recognized. He relates: ``I am making an excursion with a large group of people. We must get out at a way-station because the ship on which we are to make this excursion is too small, \& we must stay in this town over night. During the night the report comes that the ship is sinking, \& all participants in the excursion are called to man the pumps in order to prevent it. I remember that I have some valuables in my baggage \& rush to the ship where everyone else is already working at the pumps. I seek to escape this work \& look for the baggage room. I succeed in fishing my knapsack through the window \& at the same time I see a penknife which I like very much next to my knapsack. I put it in the knapsack. With an acquaintance I jump off as the ship sinks deeper \& deeper. We jump off into the sea \& land on the ground. Since the pier is too high we wander further along \& come to a precipitous cliff down which I must go. I slide down. I have not seen my companion since leaving the ship. I go faster \& faster \& fear that I will be killed. Finally I land at the bottom \& fall just in front of an acquaintance. It is an otherwise unknown young man who had been in a strike \& had worked very quietly among the strikers, who was agreeable to me. He greets me with reproachful words, just as though he knew that I had left the others on the ship in the lurch. `What are {\it you} doing here?' he asks. I seek to escape from this abyss which is surrounded on all sides by precipitous cliffs from which ropes hang down. I do not dare use them because they are too thin. With every attempt to climb out of the abyss I always slide back again. Finally I am on top, but I don't know how I got there. It seems to me that I purposely did not want to dream this part of the dream, as if I wanted to skip over it impatiently. On the edge of the abyss, on top, there was a road which was protected on the side of the abyss by a fence. People were going by here, \& greeted me in a friendly fashion.''

If we go back into the life of this dreamer the 1st thing that we hear is that he constantly suffered from severe illness up to the 5th year of his life, \& that after this time he was often ill. As a result of his weak health he was carefully \& anxiously guarded by his parents. His contact with other children was very slight. When he wanted to make contact with grown-ups he was always told by his parents that children should be seen \& not heard, \& that children do not belong with adults. He thus failed at a very early age to find those points of contact which are necessary for social life, \& remained in connection solely with his parents. The further outcome of this was that he remained considerably behind his companions of the same age, with whom he could not keep up. We are not to be astonished to hear that he was also considered stupid among them, \& soon became the butt of all their jokes. This circumstance, again, prevented him from finding friends.

An extraordinary feeling of inferiority was accentuated to the highest degree by these circumstances. His education was directed entirely by his well-wishing, but very irascible, military father, \& by his weak, uncomprehending, very domineering mother. Although his parents were constantly reiterating their good will, his education must have been a very strict one. His discouragement played a considerable role in this process. A very significant event retained in his earliest childhood remembrances was that when he was but 3 years old his mother made him kneel on peas for half an hour. The reason for this was a disobedience whose cause his mother knew very well, as the child had told her. He had become frightened of a horseman \& had therefore refused to run an errand for his mother. As a matter of fact he was spanked very seldom, but when this did occur, he was always beaten with a many-tailed dog whip, \& never without being under the necessity of afterwards begging for forgiveness \& relating the causes for which he had been beaten. ``The child should know,'' said the father, ``how he has misbehaved.'' Once he was beaten unjustly \& as he could not say afterwards why he was beaten, was beaten again, indeed was beaten until he confessed to some misdeed or other.

A belligerent feeling towards his parents was present from his earliest days. His feeling of inferiority had acquired such dimensions that he could not even conceive of being superior. His life at school as well as at home was an almost unbroken chain of greater or lesser defeats. The smallest victory, in his opinion, was denied him. At school, up to the time that he was 18 years old, he was always the one who was laughed at. Once he was laughed at even by his teacher, who read a bad theme aloud to the class \& accompanied the reading with derisive remarks.

Everyone of these occurrences forced him further \& further into isolation, \& sooner or later he began to withdraw from the world, of his own accord. In his battle with his parents he happened upon a very effective although costly method of attack. He refused to speak, \& with this gesture, he relinquished the most important grappling hook with which one fastens himself to the outer world. Since he was unable to speak with anyone, he became entirely solitary. Misunderstood by all, he spoke to none, particularly not with his parents; \& finally no one addressed him. Every attempt to bring him into society came to grief, as every attempt to establish love relationships later in this life also failed, much to his sorrow. This is the course of his life until his 28 year. The deep inferiority complex which had permeated his whole spirit had as a consequence given rise to an ambition beyond all reason, an unreined striving for significance \& superiority which ceaselessly distorted his feeling of human fellowship. The less he spoke, the more was his psychic life filled, by day \& by night, with dreams of triumphs \& victories of every sort.

\& thus he dreamt 1 night the dream which we have related above, in which we see clearly the movement \& the pattern according to which his psychic life developed. In conclusion let us recall a dream which Cicero has related, 1 of the most famous prophetic dreams in literature.

The poet Simonides, who at 1 time had found the corpse of an unidentified man lying on the street \& had cared for his descent interment, was warned by the ghost of this dead man, as he was about to attempt a sea journey, that if he should take the journey he would be shipwrecked. Simonides did not go \& all those who did, died.\footnote{Cf. Enne Nielson, ``The Unexplained, In Its Course Through The Centuries.'' Published by the Langewies che-Brandt. Ebenhausen near Munich.}

According to the report this event in connection with the dream is supposed to have had an unusually deep impression on all people for hundreds of years.

If we want to interpret this occurrence we must maintain that in that time ships were wrecked very frequently, \& also that because of this reason many people who were on the eve of a sea journey, dreamt of shipwrecks, \& that among these many dreams this particular dream demonstrated a particular coincidence between dream \& reality which was so remarkable that it remained for posterity. It is quite conceivable that those who hare a tendency to ferret out mysterious relationships have an especial weakness for just such stories, whereas we very calmly \& soberly interpret the dream as follows: our poet probably never showed any great desire to take this trip because of his considerable care for his bodily well-being; as the hour of decision neared he was hard put to it to find a justification for his hesitating attitude. For this reason he allowed the corpse who was under the necessity of proving himself grateful for his decent burial, to appear in a prophetic role. That he now did not take the trip is self-understood. If the ship had not gone under the world would never have learned anything about the dream nor the story, in all probability. For we experience only those things which set our brain into unrest, which demonstrate to us that there is more wisdom hidden between heaven \& earth than we allow ourselves to dream of. We can understand the prophetic nature of dreams in so far as we know that both dream \& reality contain the same attitude toward life which an individual shows.

Another thing which we must consider is the fact that all dreams are not so easily understood; as a matter of fact only a very few are. We forget the dream immediately after it has left its peculiar impression \& do not understand what is behind it unless we have been versed in the interpretation of dreams. Yet these dreams, too, are but a symbolic \& metaphoric reflection of the activity \& behavior pattern of an individual. The main meaning of a simile or comparison is that it affords us access to a situation in which we are anxious to find ourselves. If we are occupied with the solution of a problem \& if our personality points a specific direction of approach, then we need but seek for an animating push to propel us into it. The dream is extraordinarily well suited to intensify an emotion, or produce the verse which is necessary to the solution of a particular situation. Nothing is altered by the fact that the dreamer does not understand the connection. It suffices that he finds the material \& the boost in some fashion; the dream itself will give evidence of the manner in which the thought processes of the dreamer express themselves, as it will indicate the behavior pattern of the dreamer. The dream is like a column of smoke which shows that a fire is burning somewhere. The experienced woodsman can observe the smoke \& tell what kind of wood is burning, just as the psychiatrist can draw conclusions concerning the nature of an individual by interpreting his dream.

Summing up, we can say that a dream shows not only that the dreamer is occupied in the solution of 1 of his life's problems, but also how he approaches these problems. In particular, those 2 factors which influence the dreamer in his relationship with the world \& reality, the social feeling \& the striving for power, will make themselves evident in his dream.'' -- \cite[pp. 107--117]{Adler_human_nature}

\paragraph{Talent.} ``Among those psychic phenomena which enable us to judge an individual we have left out of consideration one which is concerned with his intellectual powers. We have placed little value upon what an individual says or thinks of himself. We are convinced that each of us can somehow go astray \& that each of us feels himself under the necessity of retouching his psychic image for his fellow man, through various of the complicated egoistic, moral, or other tricks. 1 thing we are, however, permitted to do, \& that is to draw certain conclusions from specific thought processes \& their expression in speech, even though this is possible only to a limited degree. We cannot exclude thought \& speech from our examination if we wish to judge the individual correctly.

What we are pleased to call talent, that is the special ability to make judgments, has been the subject of numerous observations, analyses, \& tests, among which the tests of intelligence, in children \& adults, are well known. These are the so-called tests for talent. Up to the present time these tests have been unsuccessful. Whenever a number of pupils are tested the results usually show what the teacher could easily have determined without tests. In the beginning the experimental psychologists were very proud of this although it must have been evident at the same time that the tests were, to a certain degree, superfluous. Another objection to intelligence tests is the fact that the thought \& judgment processes \& abilities of children do not develop regularly, so that many children who show poor results on the tests, suddenly show an extraordinarily good development \& talent after a few years. Another element which must be considered is that children in large cities, \& those from certain social circles, are better prepared for the tests by virtue of their broader life. Their seemingly greater intelligence is deceptive \& places other children who have not such a fund of preparation, relatively in the shadow. It is well known that 8--10-year-old children of well-to-do families, are much more quick witted than poor children of the same age. This does not mean that the children of the wealthy are more talented but that the cause for this difference lies entirely in the circumstances of their previous life.

Up to the present time we have not gone very far with tests of talent, as is very evident when we view the sorry results which have been shown in Berlin \& Hamburg where those children who evinced the greatest talent in the tests failed in conspicuous\footnote{easy to see or notice; likely to attract attention.} percentages later on in their education. This phenomenon would seem to prove that we have no certain guarantee for the future healthy development of the child in the results of his mental test. The experiments of Individual Psychology, quote on the contrary, have stood the test far better, because they have not been directed toward the determination of a particular degree of development, but rather have been designed to further the understanding of the positive factors underlying this development. These same observations have, when necessary, pressed the proper instruments of correction into the hands of the child. It has been the principle of Individual Psychology never to dissolve the thought \& judgment powers of a child out of the structure of his soul life, but to view them solely in connection with his other psychic processes.'' -- \cite[pp. 117--119]{Adler_human_nature}

\subsubsection{Sex}

\paragraph{Bisexuality \& the Division of Labor.} ``From our previous considerations we have learned that 2 great tendencies dominate all psychic phenomena. These 2 tendencies, the social feeling, \& the individual striving for power \& domination, influence every human activity \& color the attitude of every individual in his striving for security, in his fulfillment of the 3 great challenges of life: love, work, \& society. We shall have to accustom ourselves, in judging psychic phenomena, to investigate the quantitative \& the qualitative relationships of these 2 factors if we want to understand the human soul. The relationship of these factors to 1 another conditions the degree to which anyone is capable of comprehending the logic of communal life, \& therefore, the degree to which he is capable of subordinating himself to the division of labor which grows out of the necessity of that communal life.

Division of labor is a factor in the maintenance of human society which must not be overlooked. Everyone at some time, or at some place, must contribute his quota. That man who does not deliver his quota, who denies the value of communal life, becomes an anti-social being, \& resigns his fellowship in humanity. In simple cases of this sort we speak of egotism, of mischievousness, of self-centeredness, of nuisance. In the more complicated cases, we see the eccentrics, the hobos, \& the criminals. Public condemnation of these traits \& characteristics grows out of an appreciation of their origins, an intuition of their incompatibility with the demands of social life. Any man's value, therefore, is determined by his attitude toward his fellow men, \& by the degree in which he partakes of the division of labor which communal life demands. His affirmation of this communal life makes him important to other human beings, makes him a link in the great chain which binds society, the chain which we cannot in any way disturb without also disturbing human society. A man's capabilities determine his place in the total production of human society. Much confusion has clouded this simple truth, because the striving for power \& the lust for dominance have introduced false values into the normal division of labor. This striving for dominance has disturbed \& thwarted the total production, \& has given us a false basis for the judgment of human values.

Individuals have disturbed this division of labor by refusing to adapt themselves to the place that they must fill. Further, difficulties have arisen out of the false ambition \& power wishes of individuals who have blocked communal life \& the communal work for their own egoistic interests. Similarly, entanglements have been caused by class differences in our society. Personal power or economic interests have influenced the division of the field of labor by reserving all the better positions for individuals of certain classes, i.e., those affording the greater power, while other individuals, of other classes, have been excluded from them. The recognition of these numerous factors in the structure of society enables us to understand why the division of labor has never proceeded smoothly. Forces continually disturbing this division of labor have created privilege for one, \& slavery for another.

The bisexuality of the human race conditions another division of labor. Woman, by virtue of her physical constitution, is excluded from some certain activities, while on the other hand, there are certain labors which are not given to man, because man could better be employed at other tasks. This division of labor should have been instituted according to an entirely unprejudiced standard, \& all the movements for the emancipation of women in so far as they have not overstepped logical points in the heat of conflict, have taken up the logic of this point of view. A division of labor is far from robbing woman of her femininity, or disturbing the natural relationships between man \& woman. Each acquires those opportunities of labor which are best fitted for him. In the course of human development this division of labor has so configured itself that woman has taken over a certain part of the world's work (which might otherwise occupy a man too), in return for which man is in the position to use his powers to greater effect. We cannot call this division of labor senseless so long as the powers for work are not misused, \& so long as physical \& mental powers are not deflected to a bad end.'' -- \cite[pp. 120--122]{Adler_human_nature}

\paragraph{The Dominance of the Male in the Culture of Today.} ``As a consequence of the development of culture in the direction of personal power, especially through the efforts of certain individuals \& certain classes of society, who wish to secure privileges for themselves, this division of labor has fallen into characteristic channels which have colored our entire civilization. The importance of the male in the culture of today is greatly emphasized as a result. The division of labor is such that the privileged group, men, are guaranteed certain advantages, \& this as a result of their domination over women in the division of labor. Thus the dominant male assumes advantages \& directs the activity of women to the end that the more agreeable forms of life shall appertain always to the males, whereas those activities are allowed women which men can advantageously avoid.

As things stand now there is a constant striving on the part of men to dominate women, \& an appropriate dissatisfaction with masculine domination on the part of women. Since the 2 sexes are so narrowly connected it is easily conceivable that this constant tension leads to psychic dissonances \& to far reaching physical disturbances which must of necessity be extraordinarily painful to both sexes.

All our institutions, our traditional attitudes, our laws, our morals, our customs, give evidence of the fact that they are determined \& maintained by privileged males for the glory of male domination. These institutions reach out into the very nurseries \& have a great influence upon the child's soul. A child's understanding of these relationships need not be very great, but we must admit that his emotional life is immensely affected by them. These attitudes may well be investigated when e.g. we see a young boy responding to the request to put on girls' clothes, with a terrific temper tantrum. Once let a boy's craving for power reach a certain degree, \& you will surely find him showing a preference for the privileges of being a man which, he recognizes, guarantee his superiority everywhere. We have already mentioned the fact that the education in our families nowadays is only too wll designed to overvalue the striving for power. The consequent tendency to maintain \& exaggerate the masculine privilege follows naturally, for it is usually the father who stands as the family symbol of power. His mysterious comings \& goings arouse the interest of the child much more than the constant presence of a mother. The child quickly recognizes the prominent role his father plays, \& notes how he sets the pace, makes all arrangements, \& appears everywhere as the leader. He sees how all obey his commands \& how his mother asks him for his advice. From every angle, his father seems to be the one who is strong \& powerful. There are children for whom the father is so much a standard that they believe that everything he says must be holy; they attest to the rightness of their views simply by saying that their father once said so. Even in those cases in which the fatherly influence does not seem to be so well marked, children will get the idea of the domination of the father because the whole load of the family seems to rest upon him, whereas, as a matter of fact, it is only the division of labor which enables the father in the family to use his powers to better advantage.

So far as the history of the origin of masculine dominance is concerned, we must call attention to the fact that this is a phenomenon which does not occur as a natural thing. This is indicated by the numerous laws which are necessary legally to guarantee this domination to men. It is also an indication that previous to the legal enforcement of masculine domination there must have been other epochs in which the masculine privilege was not nearly to certain. History proves that such epochs actually existed in the days of the matriarchate, the age in which it was the mother, the woman, who played the important role in life, particularly so far as the child was concerned. At that time each man in the clan was in duty bound to respect the honored position of the mother. Certain customs \& usages are still colored by this ancient institution, as e.g., the introduction of all strange men to a child with the title of ``uncle'' or ``cousin.'' A terrific battle must have preceded the transition from matriarchate to masculine domination. Men who like to believe that their privileges \& prerogatives are determined by nature will be surprised to learn that men did not possess these prerogatives from the beginning, but had to fight for them.\footnote{A very good description of this development can be found in August Bebel's ``Woman \& Socialism'' \& in Matthias \& Mathilde Vaerting's ``The Dominant Sex.''} The triumph of man was simultaneous with the subjugation of women, \& it is especially the evidence in the development of the law which bears witness to this long process of subjugation.

Masculine dominance is not a natural thing. There is evidence to prove that it occurred chiefly as a result of constant battles between primitive peoples, during the course of which man assumed the more prominent role as warrior, \& finally used his newly won superiority in order to retain the leadership for himself \& for his own ends. Hand in hand with this development was a development of property rights \& inheritance rights which became a basis of masculine domination, in so far as man usually was the acquirer \& owner of property.

A growing child need not however read books on this theme. Despite the fact that he knows nothing of these archaelogical data he senses the fact that the male is the privileged member of the family. This occurs even when fathers \& mothers with considerable insight are disposed to overlook those privileges which we have inherited from ancient days, in favor of a greater equality. It is very difficult to make it clear to a child that a mother who is engaged in household duties is as valuable as a father.

Think what it means to a young boy who sees the prevailing privilege of manhood before his eyes from his earliest days. From the day of his birth he is received with greater acclamation than a girl child. It is a well known \& all too frequent occurrence that parents prefer to have boys as children. A boy senses at every step that, as a chip of the old block, he has certain privileges \& a greater social value. Casual words directed toward him or taken up by him occasionally are constantly calling to his attention the fact of the greater importance of the masculine role.

The domination of the male also appears to him in the institution of female servants about the house who are used for menial tasks, \& finally he is reinforced in his sentiments by the fact that the women in his environment are not at all convinced of their equality with men. That most important question which all women should ask their prospective husbands before marriage: ``What is your attitude toward masculine domination, particularly in family life?'' is usually never answered. In 1 case we find an expression of the striving for equality \& in another case any of the various degrees of resignation. In contrast we see the father convinced from boyhood that as a man he has a more important role to play. He interprets his conviction as an implicit duty, \& concerns himself solely with responding to the challenges of life \& society in favor of masculine privilege.

Every situation which arises out of this relationship is experienced by the child. What he gets out of it is a number of pictures concerning the nature of woman, in which for the most part the woman plays a sorry figure. In this way the development of the boy has a distinct masculine color. What he believes to be the worth-while goals in his striving for power are exclusively masculine qualities \& masculine attitudes. A typical masculine virtue grows out of these power relationships, which patently indicates its origins to us. Certain character traits count as masculine, others as feminine, albeit there is no basis to justify these valuations. If we compare the psychic state of boys \& girls \& seemingly find evidence in support of this classification, we do not deal with natural phenomena, but are describing the expressions of individuals who have been directed into a very specific channel, whose style of life \& behavior pattern have been narrowed down by specific channel, whose style of life \& behavior pattern have been narrowed down by specific conceptions of power. These conceptions of power have indicated to them with compelling force the place where they must seek to develop. There is no justification for the differentiation of ``manly'' \& ``womanly'' character traits. We shall see how both these traits are capable of being used to fulfill the striving for power. In other words, that one can express power with the so-called ``feminine'' traits, such as obedience \& submission. The advantages which an obedient child enjoys can sometimes bring it much more into the lime-light than a disobedient child, though the striving for power is present in both cases. Our insight into psychic life is often made more difficult by the fact that striving for power expresses itself in the most complex fashion.

As a boy grows older his masculinity becomes a significant duty, his ambition, his desire for power \& superiority is indisputably connected \& identified with the duty to be masculine. For many children who desire power it is not sufficient to be simply aware of their masculinity; they must show a proof that they are men, \& therefore they must have privileges. They accomplish this, on the 1 hand, by efforts to excel, thereby measuring their masculine traits; on the other hand they may succeed by tyrannizing their feminine environment in every possible way. According to the degree of resistance which they meet, these boys utilize either stubbornness \& wild insurgency, or craft \& cunning, to gain their ends.

Since every human being is measured according to the standard of the privileged male, it is no wonder that one always holds this standard before a boy. Finally he measures himself according to it, observing \& asking whether his activities are sufficiently ``masculine,'' whether he is ``fully a man.'' What we consider ``masculine'' nowadays is common knowledge. Above all it is something purely egoistic, something which satisfies self-love, gives a feeling of superiority \& domination over others, all with the aid of seemingly ``active'' characteristics such as courage, strength, duty, the winning of all manner of victories, especially those over women, the acquisition of positions, honors, titles, \& the desire to harden himself against so-called ``feminine'' tendencies, \& the like. There is a constant battle for personal superiority because it counts as a ``masculine'' virtue to be dominant.

In this manner every boy assumes characteristics which he sees in adult men, especially his father. We can trace the ramifications of this artificially nourished delusion of grandeur in the most diverse manifestations of our society. At an early age a boy is urged to secure for himself a reserve of power \& privileges. This is what is called ``manliness.'' In bad cases it degenerates into the well-known expressions of rudeness \& brutality.

The advantages of being a man are, under such conditions, very alluring. We must not be astonished therefore when we see many girls who maintain a masculine ideal either as an unfulfilable desire, or as a standard for the judgment of their behavior; this ideal may evince itself as a pattern for action \& appearance. It would seem that in our culture every woman wanted to be a man! In this class we find those girls particularly who have an uncontrollable desire to distinguish themselves in games \& activities which are more appropriate to boys by virtue of their different physique. They climb up every tree, play rather with boys than with girls, \& avoid every ``womanly'' activity as a shameful thing. Their satisfaction lies only in masculine activities. The preference for manliness makes all these phenomena understandable when we understand how the striving for superiority is more concerned with the symbols of things than with the activities of life.'' -- \cite[pp. 122--129]{Adler_human_nature}

\paragraph{The Alleged Inferiority of Women.} ``Man has been wont to justify his domination not only by maintaining that his position is natural, but also that his dominance results from the inferiority of woman. This conception of the inferiority of woman is so widespread that it appears as the common property of all races. Linked with this prejudice is a certain unrest on the part of men which may well have originated in the time of the war against the matriarchate, when woman was a source of actual anxiety. We come upon indications of this constantly in literature \& history. A Latin author writes ``Mulier est hominis confusio,'' ``Woman is the confusion of man.'' In the theological consilia the question was often argued whether a woman had a soul, \& learned theses were written concerning the question whether woman was actually a human being. The century-long period of witch-persecution \& witch-burning is a sorry witness of the errors, the tremendous uncertainty \& confusion of that happily forgotten age, concerning this question.

Woman was often held up as the source of all evil, as in the Biblical conception of the original sin, or as in the {\it Iliad} of Homer. The story of Helen demonstrated how 1 woman was capable of throwing whole peoples into misfortune. Legends \& fairy tales of all times contain indices of the moral inferiority of woman, of her wickedness, of her falsity, of her treachery \& of her fickleness. ``Womanly folly'' has even been used as an argument in legal cases. Coincident with these prejudices is the degradation of woman's capability, industry, \& ability. Figures of speech, anecdotes, mottoes, \& jokes, in all literatures \& among all peoples, are full of degrading critiques of woman. Woman is reproached with her spitefulness, her pettiness, her stupidity, \& the like.

An extraordinary acuity is sometimes developed in order to bear witness to the inferiority of woman. The number of men like Strindberg, Moebius, Schopenhauer, \& Weininger, who have upheld this thesis, has been enlarged by a not inconsiderable number of women whose resignation has caused them to subscribe to a belief in the inferiority of woman. They are the champions of woman's role of submission. The degradation of woman \& womanly labor is further indicated by the fact that woman are paid less than men, regardless of whether their work is of equal value.

In the comparison of the results of intelligence \& talent tests it was actually found that for particular subjects, as e.g., mathematics, boys showed more talent, whereas girls showed more talent for other subjects, e.g. languages. Boys actually do show greater talent than girls for studies which are capable of preparing them for their masculine occupation but this is only a seemingly greater talent. If we investigate the situation of the girls more closely we learn that the story of the lesser capability of woman is a palpable fable.

A girl is daily subjected to the argument that girls are less capable than boys \& are suitable only for unessential activities. It is not surprising then that a girl is firmly convinced of the unchangeable \& bitter fate of a woman \& sooner or later because of her lack of training in childhood, actually believes in her own incapability. Discouraged in this manner, a girl approaches ``masculine'' occupations if the opportunity to approach them ever presents, with a foregone conclusion that she will not have the necessary interest for them. Should she possess such interest, she soon loses it, \& thus she is denied both an outer \& an inner preparation.

Under such circumstances proof of the incapability of woman seems valid. There are 2 causes for this. In the 1st place the error is accentuated by the fact that the value of a human being is frequently judged from purely business standpoints, or on 1-sided \& purely egoistic grounds. With such prejudices we can hardly be expected to understand how far performance \& capability are coincident with psychic development. \& this leads us to the 2nd main factor to which the fallacy of the lesser capability of woman may thank its existence. It is a frequently overlooked fact that a girl comes into the world with a prejudice sounding in her ears which is designed only to rob her of her belief in her own value, to shatter her self-confidence \& destroy her hope of ever doing anything worth while. If this prejudice is constantly being strengthened, if a girl sees again \& again how women are given servile roles to play, it is not hard to understand how she loses courage, fails to face her obligations, \& sinks back from the solution of her life's problems. Then indeed she is useless \& incapable! Yet if we approach a human being, undermine his self-respect so far as his relationship to society is concerned, cause him to abandon all hope of ever accomplishing anything, ruin his courage, \& then find that he actually never amounts to anything, then we dare not maintain that we were right, for we must admit that it is {\it we} who have caused all his sorrow!

It is easy enough for a girl to lose her courage \& her self-confidence in our civilization, yet, as a matter of fact, certain intelligence tests proved the interesting fact that in a certain group of girls, aged from 14--18, greater talent \& capability were evinced than was shown by all other groups, boys included. Further researches show that these were all girls from families in which the mother was either the sole bread winner, or at least contributed largely to the family support. What this means is that these girls were in a situation at home in which the prejudice of the lesser capability of woman was either not present or existed only to a slight extent. They could see with their own eyes how their mothers' industry had its rewards, \& as a result they developed themselves much more freely \& much more independently, entirely uninfluenced by those inhibitions which are inevitably associated with the belief in the lesser powers of a woman.

A further argument against this prejudice is the not inconsiderable number of women who have accomplished results in the most varied fields, particularly in literature, art, crafts, \& medicine, of such remarkable value that they are quite capable of standing any comparison with the results of men in these fields. There are so many men furthermore who not only do not show any achievements but are possessed of such a high grade of incapability that we could easily find an equal number of proofs (of course falsely) that men were the inferior sex.

1 of the bitter consequences of the prejudice concerning the inferiority of women is the sharp division \& pigeon-holing of concepts according to a scheme: thus ``masculine'' signifies worth-while, powerful, victorious, capable, whereas ``feminine'' becomes identical with obedient, servile, subordinate. This type of thinking has become so deeply anchored in human thought processes that in our civilization everything laudable has a ``masculine'' color whereas everything less valuable or actually derogatory is designated ``feminine.'' We all know men who could not be more insulted than if we told them that they were feminine, whereas if we say to a girl that she is masculine it need signify no insult. The accent always falls so that everything which is reminiscent of woman appears inferior.

Character traits which would seem to prove this fallacious contention of the inferiority of woman prove themselves on closer observation nothing more than the manifestation of an inhibited psychic development. We do not maintain that we can make what is called a ``talented'' individual out of every child, but we can always make an ``untalented'' adult out of him. We have never done this fortunately. Others, however, we know have succeeded only too well. That such a fate overtakes girls more frequently than boys, in our day \& age, is easily understood. We have often had the opportunity of seeing these ``untalented'' children suddenly become so talented that one might have spoken of a miracle!'' -- \cite[pp. 129--133]{Adler_human_nature}

\paragraph{Desertion from Womanhood.} ``

'' -- \cite[pp. 133--107]{Adler_human_nature}

\subsubsection{The Family Constellation}

\subsection{Book II: The Science of Character}

\subsubsection{General Considerations}

\subsubsection{Aggressive Character Traits}

\subsubsection{Non-aggressive Character Traits}

\subsubsection{Other Expressions of Character}

\subsubsection{Affects \& Emotions}

\subsubsection{Appendix}

%------------------------------------------------------------------------------%

\section{Miscellaneous}

%------------------------------------------------------------------------------%

\printbibliography[heading=bibintoc]
	
\end{document}